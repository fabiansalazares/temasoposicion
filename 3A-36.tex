\documentclass{nuevotema}

\tema{3A-36}
\titulo{La política monetaria (II): los mecanismos de transmisión de la política monetaria. Los efectos de la política monetaria.}

\begin{document}

\ideaclave

Leer Pfister y Sahuc (2020) para efectos de políticas monetarias no convencionales.

\seccion{Preguntas clave}

\begin{itemize}
	\item ¿Cómo afectan las decisiones de política monetaria a variables nominales y reales?
	\item ¿Cómo se transmiten las decisiones de política monetaria?
	\item ¿Qué canales de transmisión son más relevantes?
	\item ¿Qué efectos tienen las decisiones de política monetaria?
	\item ¿Qué predicen los distintos modelos?
	\item ¿Qué evidencia empírica existe al respecto?
\end{itemize}

\esquemacorto

\begin{esquema}[enumerate]
	\1[] \marcar{Introducción}
		\2 Contextualización
			\3 Política monetaria
			\3 Evolución de la política monetaria
			\3 Mecanismos y efectos
		\2 Objeto
			\3 ¿Qué efectos tiene la política monetaria?
			\3 ¿Qué variables se ven afectadas?
			\3 ¿Qué modelos teóricos modelizan los efectos de la PM?
			\3 ¿Qué evidencia empírica existe al respecto?
			\3 ¿Cómo se transmiten las decisiones de PM a otras variables?
			\3 ¿Qué canales de transmisión existen?
		\2 Estructura
			\3 Mecanismos de transmisión de la PM
			\3 Efectos de la PM
	\1 \marcar{Efectos de la política monetaria}
		\2 Idea clave
			\3 Regularidades empíricas
			\3 Neutralidad del dinero
			\3 Activación de la demanda
			\3 Papel de las expectativas
		\2 Modelos teóricos
			\3 Preclásicos
			\3 Teoría monetaria clásica
			\3 Keynes
			\3 Síntesis neoclásica
			\3 Monetarismo
			\3 Nueva Macroeconomía Clásica
			\3 Nueva Economía Keynesiana
		\2 Evidencia empírica
			\3 Idea clave
			\3 Estimación de hechos empíricos
			\3 Largo plazo
			\3 Corto plazo
			\3 Políticas no convencionales
	\1 \marcar{Mecanismos de transmisión de la política monetaria}
		\2 Idea clave
			\3 Balance del banco central
			\3 Papel del Banco Central
			\3 Instrumentos de política monetaria
			\3 Efectos de diferentes instrumentos
		\2 Interés keynesiano o efecto indirecto
			\3 Esquema
			\3 Explicación
		\2 Monetarista o efecto directo
			\3 Esquema
			\3 Explicación
		\2 Activos financieros: q de Tobin
			\3 Esquema
			\3 Explicación
		\2 Activos financieros: renta permanente
			\3 Esquema
			\3 Explicación
		\2 Activos financieros: expectativa de crisis
			\3 Esquema
			\3 Explicación
		\2 Activos inmobiliarios: consumo e inversión
			\3 Esquema
			\3 Explicación
		\2 Crédito bancario
			\3 Esquema
			\3 Explicación
		\2 Balances: información asimétrica
			\3 Esquema
			\3 Explicación
		\2 Balances: flujo de caja
			\3 Esquema
			\3 Explicación
		\2 Expectativas
			\3 Idea clave
			\3 Modelos DSGE de NEK
		\2 Tipo de cambio
			\3 Esquema
			\3 Explicación
		\2 Efectos redistributivos
			\3 Idea clave
		\2 Efectos de segunda ronda
			\3 Idea clave
			\3 Ejemplos
	\1[] \marcar{Conclusión}
		\2 Recapitulación
			\3 Efectos de la PM
			\3 Mecanismos de transmisión de la PM
		\2 Idea final
			\3 Diseño e implementación de la PM
			\3 Spillovers internacionales de la PM
			\3 Incertidumbre y falta de información

\end{esquema}

\esquemalargo

\begin{esquemal}
	\1[] \marcar{Introducción}
		\2 Contextualización
			\3 Política monetaria
				\4 Decisiones tomadas por BCentral o AMonetaria
				\4[] Respecto a:
				\4[] $\to$ Cantidad de dinero en circulación
				\4[] $\to$ Condiciones de financiación
				\4 Para alcanzar determinados objetivos macro
				\4[] $\to$ Crecimiento sostenido de producto real
				\4[] $\to$ Tasa de paro
				\4[] $\to$ Estabilidad de precios
			\3 Evolución de la política monetaria
				\4 Ligada a evolución de tres factores:
				\4[] $\to$ Avances tecnológicos
				\4[] $\to$ Entorno macroeconómico
				\4[] $\to$ Contexto político
				\4 Primera manifestación de PMonetaria
				\4[] Descuento de deuda pública y privada
				\4[] $\to$ Banco de Inglaterra
				\4[] Fomentar mercado de deuda pública
				\4[] Prestamista de prestamistas
				\4 Segunda fase de política monetaria
				\4[] Suavizar fluctuaciones de tipos
				\4[] $\to$ Estacionales y cíclicas
				\4[] Mantener precio y reservas de oro
				\4 Política monetaria moderna
				\4[] Tres instrumentos convencionales
				\4[] $\to$ Tipo de descuento
				\4[] $\to$ Operaciones de mercado abierto
				\4[] $\to$ Requisitos de reservas bancarias
				\4[] Instrumentos no convencionales
				\4[] $\to$ Forward guidance
				\4[] $\to$ Quantitative Easing
			\3 Mecanismos y efectos
				\4 Para diseñar PM óptima dados objetivos
				\4[] Necesario comprender procesos subyacentes
				\4 Necesario conocer mecanismos de transmisión
				\4[] $\to$ Vías por las que afecta a vars. reales
				\4 Necesario conocer efectos
				\4[] $\to$ Descripción cuantitativa y cualitativa
				\4[] $\to$ Modelos teóricos y evidencia empírica
		\2 Objeto
			\3 ¿Qué efectos tiene la política monetaria?
			\3 ¿Qué variables se ven afectadas?
			\3 ¿Qué modelos teóricos modelizan los efectos de la PM?
			\3 ¿Qué evidencia empírica existe al respecto?
			\3 ¿Cómo se transmiten las decisiones de PM a otras variables?
			\3 ¿Qué canales de transmisión existen?
		\2 Estructura
			\3 Mecanismos de transmisión de la PM
			\3 Efectos de la PM
	\1 \marcar{Efectos de la política monetaria}
		\2 Idea clave
			\3 Regularidades empíricas
				\4 Desde primeros análisis macroeconómicos
				\4 Corto plazo
				\4[] PM afecta output real
				\4 Largo plazo
				\4[] PM afecta sólo variables nominales
			\3 Neutralidad del dinero
				\4 Objeto de controversia
				\4 Hasta Keynes, cierto consenso
				\4[] Dinero neutral en el l/p
				\4 Keynes
				\4[] L/p no es periodo de análisis relevante
				\4[] L/p son muchos c/p sucesivos
			\3 Activación de la demanda
				\4 Si no hay mecanismos de estabilidad
				\4[] Economía puede mantenerse en eqs. subóptimos
			\3 Papel de las expectativas
				\4 Introducción en modelos formales
				\4[] $\to$ Monetarismo
				\4 NMC
				\4[] $\to$ Expectativas son racionales
				\4[] $\to$ Ya no son fenómeno ad-hoc
		\2 Modelos teóricos
			\3 Preclásicos
				\4 Martín de Azpilcueta, Copérnico
				\4[] Dinero no tiene efectos reales
				\4[] Precios se ajustan a cantidad de dinero
				\4[] $\to$ Teoría Cuantitativa del Dinero
				\4 Hume (1752)
				\4[] On Money and Interest
				\4[] Origen de teoría monetaria
				\4[] Dinero no tiene efectos reales en l/p
				\4[] $\to$ Sí admite efectos en c/p hasta ajuste
			\3 Teoría monetaria clásica
				\4 Dinero no tiene efectos reales a l/p
				\4 En general, ``dicotomía clásica'' consolidado
				\4[] Variables nominales y reales
				\4[] $\to$ Se determinan por separado
				\4 Aparece debate currency vs banking schools
				\4[] Currency school
				\4[] $\to$ M puede controlarse
				\4[] $\to$ Crecimiento rápido de M causa inflación
				\4[] $\then$ Política monetaria es importante
				\4[] Banking school
				\4[] $\to$ M se ajusta endógenamente a economía
				\4[] $\to$ Descuento de letras reales no causa inflación
				\4[] $\to$ M se ajusta a necesidades del comercio
				\4[] $\then$ Política monetaria no es importante
			\3 Keynes
				\4 Política monetaria tiene efectos reales
				\4[] $\to$ Porque salarios y precios son $+/-$ rígidos
				\4[] $\to$ Porque interés se determina en MMonetario
				\4 Expansión monetaria
				\4[] Reduce tipos de interés
				\4[] Causa débil de inflación
				\4 Reducción de tipos de interés
				\4[] Aumenta demanda de inversión
				\4[] $\to$ Canal keynesiano
				\4[] Reduce coste de déficit público
				\4 Aumento de precios
				\4[] Reduce salario real
				\4[] $\to$ Aumenta demanda de trabajo
				\4[] $\then$ Reduce desempleo
				\4[] Debilita demanda agregada
				\4[] $\to$ Necesario estímulo fiscal
				\4 Trampa de liquidez
				\4[] Política monetaria puede no tener efectos
				\4[] Motivos para demandar dinero
				\4[] $\to$ Precaución y transacción
				\4[] $\to$ Especulación
				\4[] Si tipo de interés nominal muy bajo
				\4[] $\to$ Tipo de interés sólo puede subir
				\4[] $\to$ Especulación es motivo muy fuerte
				\4[] $\to$ Demanda de dinero absorbe toda oferta
				\4[] $\then$ Tipos de interés no varían
				\4[] $\then$ Política monetaria inefectiva
				\4[$\then$] PM tiene efectos reales
				\4[$\then$] Pero sujeta a alteraciones no previsibles
				\4[$\then$] Necesario complementar con política fiscal
				\4[$\then$] Efectividad limitada
			\3 Síntesis neoclásica
				\4 Dicotomía corto y largo plazo
				\4[] Efectos keynesianos a c/p
				\4[] Modelo clásico en l/p
				\4 Corto plazo
				\4[] Política monetaria tiene efectos reales
				\4[] Efectos keynesianos
				\4[] Representado en marco IS-LM
				\4[] \grafica{islmkeynesianopoliticamonetaria}
				\4 Largo plazo
				\4[] Política monetaria no tiene efectos reales
				\4[] $\to$ PIB real resultado de factores reales
				\4[] $\then$ PM afecta a precios e interés nominal
				\4[] Basado en modelo neoclásico
				\4[] \grafica{islmneoclasicopoliticamonetaria}
			\3 Monetarismo
				\4 Política monetaria tiene efectos reales
				\4[] Y son los más importantes
				\4 Corto plazo
				\4[] Posible activar demanda vía PM
				\4[] Pero activación sujeta a:
				\4[] $\to$ Lags o efectos retardados de duración imprevisible
				\4[] $\to$ Difícil conocer distribución entre efectos reales y nominales
				\4[] $\to$ Cada vez mayores necesidades de activación
				\4 Largo plazo
				\4[] Inflación es fenómeno puramente monetario
			\3 Nueva Macroeconomía Clásica
				\4 Tres programas de investigación
				\4[] i. Información imperfecta
				\4[] ii. Consistencia de la política monetaria
				\4[] iii. Modelo del Ciclo Real
				\4 i. Información imperfecta
				\4[] Lucas (1972) y siguientes
				\4[] PM sí tiene efectos reales
				\4[] $\to$ Sólo shocks imprevistos
				\4[] $\to$ Consecuencia de información imperfecta
				\4[] Estímulo monetario aumenta output
				\4[] Agentes tienen información imperfecta sobre precios
				\4[] $\to$ No conocen nivel de precios
				\4[] $\to$ No conocen valor real de oferta de trabajo
				\4[] Posible hacer creer que salario real $\uparrow$
				\4[] $\to$ Realmente, $\uparrow$ M y $\uparrow$ P
				\4[] Conocen modelo completo
				\4[] $\then$ PM no tiene efectos sistemáticos
				\4[] $\then$ PM sistemática pierde efectividad
				\4 ii. Consistencia de la política monetaria
				\4[] Kydland y Prescott (1977)
				\4[] Barro y Gordon (1983)
				\4[] PM óptimas pueden ser inconsistentes
				\4[] Senda de PM óptima desde hoy
				\4[] $\to$ Deja de serlo mañana
				\4[] $\then$ Cuando el pasado esté dado
				\4[] Agentes comprenden los incentivos de AMonetaria
				\4[] $\to$ Saben que tiene incentivos a intentar ``sorprender''
				\4[] $\then$ No se sorprenden
				\4[] Esperan inflación más alta en el futuro
				\4[] $\to$ Prevén que la sorpresa tendrá lugar
				\4[] $\then$ Sorpresa ya no tiene efecto
				\4[] $\then$ PM sin efectos relevantes
				\4 iii. Modelo del Ciclo Real
				\4[] Kydland y Prescott (1982)
				\4[] Long y Plosser (1983)
				\4[] Política monetaria es perfectamente irrelevante
				\4[] Sólo shocks reales tienen efectos reales
				\4[] $\to$ Shocks monetarios no afectan output
				\4[] $\then$ PM afecta precios e interés nominal
			\3 Nueva Economía Keynesiana
				\4 PM es instrumento básico de PEconómica
				\4[] Tiene efectos determinantes sobre economía
				\4 Modelo básico con cuatro ecuaciones
				\4[DIS] IS dinámica
				\4[] \fbox{$\tilde{y}_t = \textrm{E}_t \left\lbrace \tilde{y}_{t+1} \right\rbrace - \frac{1}{\sigma} \left( \underbrace{i_t - \textrm{E}_t \left\lbrace \pi_{t+1} \right\rbrace}_{r_t} - r^n_t \right) $}
				\4[NKPC] Curva de Phillips Neo-Keynesiana
				\4[] \fbox{$\pi_t = \text{E}_t \left\lbrace \pi_{t+1} \right\rbrace + \textsc{k} \tilde{y}_t $}
				\4[TR] Regla de Taylor simple
				\4[] \fbox{$i_t = \rho + \phi_\pi \pi_t + \phi_y \tilde{y}_t + v_t $}
				\4[MP] Mercado de dinero
				\4[] \fbox{$m_t - p_t = y_t - \eta i_t$}
				\4 Rigidez nominal y real
				\4[] Diferentes justificaciones microeconómicas
				\4[] Precios à la Calvo
				\4[] Salarios de eficiencia
				\4[] ...
				\4 Mark-ups y fijación de precios
				\4[] Precios no son flexibles
				\4[] $\to$ Empresas no pueden fijar mark-up deseado
				\4[] $\then$ Empresas ajustan en cantidades
				\4 Efectos reales de PM
				\4[] Dos vías
				\4[] i. Reglas de política monetaria
				\4[] $\to$ Expresión sistemática de política de BCentral
				\4[] ii. Shocks de política monetaria
				\4[] $\to$ Acciones inesperadas
				\4[] $\to$ Fallos en transmisión de decisión
				\4 Shock monetario con regla de interés
				\4[] (modelo NEK básico sin rigidez real)
				\4[] Aumento exógeno de tipo de interés
				\4[] $\to$ Output natural: CONSTANTE
				\4[] $\to$ Output gap: CAE
				\4[] $\to$ Output: CAE
				\4[] $\to$ Inflación: CAE
				\4[] $\to$ Empleo: CAE
				\4[] $\to$ Salario real: CAE
				\4[] $\to$ Nivel de precios: CAE y se mantiene bajo
				\4[] $\to$ Interés nominal: SUBE
				\4[] $\to$ Interés real: SUBE más que interés nominal
				\4[] $\to$ Oferta de dinero: CAE
				\4 Shock monetario con regla de oferta monetaria
				\4[] (modelo NEK básico de tipo de interés)
				\4[] Aumento exógeno de oferta monetaria
				\4[] Resumen de efectos
				\4[] $\to$ Output natural: CONSTANTE
				\4[] $\to$ Output gap: SUBE
				\4[] $\to$ Output: SUBE
				\4[] $\to$ Inflación: SUBE
				\4[] $\to$ Empleo: SUBE
				\4[] $\to$ Salario real: SUBE
				\4[] $\to$ Nivel de precios: SUBE
				\4[] $\to$ Interés nominal: SUBE
				\4[] $\to$ Interés real: CAE\footnote{Como hay rigidez nominal de los precios, el efecto liquidez predomina sobre el efecto Fisher a priori.}
				\4[] $\to$ Oferta de dinero: SUBE
				\4 Efectos normativos de PM
				\4[] Depende del modelo concreto
				\4[] Sin rigidez real:
				\4[] $\to$ Gap constante entre $y$ natural y eficiente
				\4[] $\to$ Optimalidad si mínima inflación
				\4[] $\then$ Divina coincidencia
				\4[] Con rigidez real:
				\4[] $\to$ Gap variable entre $y$ natural y eficiente
				\4[] $\to$ Mínima inflación no es óptima
				\4[] $\then$ Sin divina coincidencia
		\2 Evidencia empírica
			\3 Idea clave
				\4 Objetivo
				\4[] Describir relaciones entre:
				\4[] $\to$ Precios
				\4[] $\to$ Agregados monetarios
				\4[] $\to$ Interés nominal y real
				\4[] $\to$ Output y empleo
				\4[] Predecir reacciones futuras
				\4[$\to$] Contrastar veracidad de modelos teóricos
				\4 Múltiples herramientas empíricas
				\4[] Regresiones sección cruzada
				\4[] Datos de panel
				\4[] Análisis de series temporales
				\4[] Simulaciones de modelos DSGE
				\4[] Análisis narrativo de comunicados de BCentrales
			\3 Estimación de hechos empíricos
				\4 Métodos VAR
				\4 Métodos narrativos
				\4 Cuasi-experimentos
			\3 Largo plazo
				\4 Dos resultados fundamentales
				\4[I] Correlación ~1 entre precios y $\frac{\dot{M}}{M}$
				\4[] MacCandless y Weber (1995)
				\4[] $\to$ 30 años y 110 países
				\4[] Berentsen, Menzio, Wright (2011)
				\4[] $\to$ Correlación fuerte tras filtrar $\uparrow$ $\downarrow$ de c/p
				\4[] Correlación \underline{no} implica causalidad:
				\4[] $\to$ Si $M$ y $\pi$ son vars. endógenas a output
				\4[] $\to$ Si $Y$ y $M$/$\pi$ reciben perturbaciones comunes
				\4[] Relación compleja tras desagregar
				\4[] $\to$ No es tan simple como a priori
				\4[] Países con inflación baja
				\4[] $\to$ Correlación mucho más débil
				\4[] $\to$ Conclusión no muy robusta a subconjuntos
				\4[] Períodos históricos más pequeños
				\4[] $\to$ Cambian conclusiones
				\4[] Gran Moderación: 1985--2007
				\4[] $\to$ Relación $\pi$--M muy débil o negativa
				\4[] Interés nominal y $\pi$/M
				\4[] $\to$ Correlación clara en largo plazo
				\4[] $\to$ Bastante robusta a periodos y subsets
				\4[] Interés nominal en Gran Moderación
				\4[] $\to$ Interés reacciona más que inflación
				\4[] $\to$ Política de respuesta de Bancos Centrales
				\4[] $\then$ Difícil establecer causalidad
				\4[] $\then$ ¿Interés nominal causa $\pi$/M o al revés?
				\4[] $\then$ Caracterizar regla de PM es objetivo empírico
				\4[II] Sin correlación entre $\pi$/$\Delta M$ y $\Delta Y$
				\4[] Datos muestran todo tipo de casos
				\4[] $\to$ Baja y alto $\pi$ con alto y bajo $\Delta Y$
				\4[] Subsets muestran alguna correlación entre $\Delta M$--Y
				\4[] $\to$ Pero pequeña y no robustas a subsets
				\4[] Shocks permanentes de inflación
				\4[] $\to$ Sin efectos de largo plazo
				\4[] Oferta monetaria y desempleo
				\4[] $\to$ Muy claramente, ausencia de correlación l/p
			\3 Corto plazo
				\4 Objeto principal de análisis empírico de PM
				\4[] $\to$ ¿Fluctuaciones de $M$ afectan vars. reales a c/p?
				\4 Friedman y Schwartz (1963)
				\4[] Primer gran estudio empírico
				\4[] Fija debate y herramientas posteriores
				\4[] Cambios en $M$ preceden $\Delta Y$
				\4[] $\to$ Relación temporal muy robusta hasta 60s
				\4[] $\to$ Interpretan causalmente
				\4 Críticas a interpretación causal de $\Delta M$--$\Delta Y$
				\4[] Cambios en M no son exógenos
				\4[] $\to$ M es de hecho variable pasiva
				\4[] Correlación temporal no es causalidad
				\4[] $\to$ Tendencia subyacente previa
				\4[] M (cualquier agregado) puede no ser var. operativa
				\4[] $\uparrow$ Y puede inducir $\uparrow$ M
				\4[] $\to$ Sin cambio en política monetaria
				\4 Causalidad inversa
				\4[] Tobin (1970), King y Plosser (1984)
				\4[] Aportan evidencia de que:
				\4[] $\to$ M1 y M2 reaccionan a cambios en Y
				\4[] $\to$ M0 sí es exógena
				\4 Problema de endogeneidad
				\4[] Interés es instrumento operativo
				\4[] $\to$ Agrava el problema
				\4 Ecuaciones de Saint Louis
				\4[] Regresiones econométricas
				\4[] Var. dependiente
				\4[] $\to$ Output o renta
				\4[] Vars. independientes
				\4[] $\to$ Gasto autónomo (PF)
				\4[] $\to$ Oferta monetaria (PM)
				\4[] $\to$ Otros factores
				\4[] $\to$ Con lags
				\4[] Muestran relación entre Y, $\Delta$ M
				\4[] $\to$ Sujeto a lags
				\4[] Apoyo a conclusiones monetaristas
				\4[] No desagregan efecto sobre Y y $\pi$
				\4 Agregados monetarios y relevancia
				\4[] ¿Qué agregado monetario es relevante?
				\4[] $\to$ ¿Cuál induce efectos reales?
				\4[] Innovaciones financieras
				\4[] $\to$ Volatilizan definiciones de M
				\4[] $\to$ Afectan a demanda de dinero
				\4[] Construcción de índices Divisia
				\4[] $\to$ Ponderando por ``moneyness''
				\4[] Belongia y Ireland (2016)
				\4[] $\to$ Series actualizadas
				\4[] $\to$ Resultados similares a Friedman y Schwartz
				\4 Causalidad de Granger
				\4[] Herramienta econométrica
				\4[] $\to$ Establecer causalidad temporal
				\4[] X granger-causa Y si:
				\4[] $X_{-t}$ predice Y presente
				\4[] Intento de utilizar para problema $\Delta M$-Y
				\4[] Resultados poco robustos
				\4[] $\to$ A introducción de otras variables
				\4[] $\to$ Sobre todo, interés
				\4 Regresiones PIB real frente a M
				\4[] Caracterizar efecto de shocks imprevistos
				\4[] Lucas (1972) y (1973) inician técnica
				\4[] $\to$ Efecto de $\sigma M$ sobre $\sigma Y$
				\4[] $\then$ ¿Más volatilidad de sorpresas reduce efectos?
				\4[] Barro, Mishkin encuentran shocks imprevistos de M
				\4[] $\to$ Afectan Y
				\4[] $\to$ Efectos asimétricos si contracción/expansión
				\4 Análisis VAR
				\4[] Vector Auto-Regression
				\4[] Regresiones de series temporales respecto:
				\4[] $\to$ Valores retardados de sí mismas
				\4[] $\to$ Valores retardados de otras endógenas
				\4[] $\to$ Shocks exógenos
				\4[] Sims (1992) y muchos otros
				\4[] Conclusiones habituales
				\4[] $\to$ Output reacciona a cambios en interés
				\4[] $\to$ Aumento gradual y desaparición posterior
				\4[] Problemas de identificación
				\4[] $\to$ Procedimientos operativos importan
				\4[] $\to$ Decisiones de PM son endógenas también
				\4[] $\to$ Perturbaciones comunes a PM y output
			\3 Políticas no convencionales\footnote{Mejorable con Kuttner (2018).}
				\4 Análisis de comunicados de bancos centrales
				\4[] Cuantificación de discursos
				\4[] $\to$ Construcción de índices
				\4[] $\to$ Más dove o más hawk
				\4[] Valoración de forward-guidance
				\4[] $\to$ Efecto sobre precios de activos
				\4 Effective Lower Bound vs Zero-Lower Bound
				\4[] Hechos empíricos han desmentido ZLB
				\4[] $\to$ BCentrales han bajado por debajo de 0
				\4[] $\then$ 0 no es el límite inferior
				\4[] $\then$ Pero sigue existiendo límite $\to$ ELB
				\4 Tipos de interés constantes durante años post-crisis
				\4[] Necesario otros indicador de PM
				\4[] Habitual construir ``shadow states''
				\4[] $\to$ Representan posición de PM
				\4 Quantitative Easing
				\4[] Dos preguntas relevantes
				\4[] i. ¿Afecta a precios de activos y tipos l/p?
				\4[] ii. ¿Afecta a variables macroeconómicas?
				\4[] Efectos sobre interés de l/p
				\4[] $\to$ Claramente sí afecta en activos comprados
				\4[] $\to$ Menos claro sobre no comprados
				\4[] $\to$ Evidencia de fragmentación de mercado
				\4[] Efecto sobre deuda privada
				\4[] $\to$ Reducción de PRiesgo en sector privado
				\4[] $\to$ Poca reducción en sector financiero
				\4[] Efectos de QE sobre macroeconomía
				\4[] Wu y Xia (2016)
				\4[] $\to$ Muy difícil construcción de contrafactuales
				\4[] $\to$ Difícil postular no-linealidades
				\4[] $\to$ Simulaciones con modelos DSGE
				\4[] $\to$ Modelos apuntan efectos positivos moderados
				\4[] $\to$ Programas QE muy grande son necesarios
				\4[] Variabilidad de efectos
				\4[] $\to$ No todos los programas han afectado igual
				\4[] $\to$ Primeros QE tuvieron mayor efecto
				\4 Efectos no deseados de políticas no convencionales
				\4[] Inflación
				\4[] $\to$ Temores sobre posible aparición
				\4[] $\to$ No ha sucedido (de momento?)
				\4[] Aumento de la asunción de riesgos
				\4[] $\to$ Alguna evidencia a favor
				\4[] $\to$ Hasta ahora, efectos no parecen negativos
	\1 \marcar{Mecanismos de transmisión de la política monetaria}
		\2 Idea clave\footnote{De \textit{``monetary transmission mechanism''} en Palgrave. Basado indirectamente en Mishkin (1995).}
			\3 Balance del banco central
				\4 Pilar central de la política monetaria
				\4 Activo
				\4[] Activos netos frente a exterior
				\4[] Activos frente a sector público
				\4[] Activos frente a sector privado
				\4 Pasivo
				\4[] Base monetaria
				\4[] $\to$ Efectivo en manos del público
				\4[] $\to$ Reservas de entidades financieras
				\4[] Patrimonio neto del banco
			\3 Papel del Banco Central
				\4 Afectar variables operativas
				\4[] Sobre las que puede ejercer un control efectivo
				\4[] $\to$ Grados variables de control
				\4 Base monetaria
				\4[] Ampliando/reduciendo balance
				\4[] Comprando y vendiendo activos financieros
				\4 Oferta monetaria
				\4[] Exigiendo reservas mínimas a bancos
				\4 Interés nominal de corto plazo
				\4[] Fijando tipos de depósito y préstamo c/p
				\4[] Operaciones de mercado abierto
				\4[] $\to$ Compraventa de deuda
				\4[] $\to$ Subasta de liquidez
			\3 Instrumentos de política monetaria
				\4 Requisitos de reservas mínimas
				\4[] Exigir reservas mínimas depositadas en BC
				\4[] $\to$ En relación a depósitos
				\4[] Afectar necesidades estructurales de liquidez
				\4 Operaciones de mercado abierto
				\4[] $\to$ Compra/venta de deuda pública
				\4 Tipo oficial o de descuento
				\4[] $\to$ Reservas depositadas en banco central
				\4[] $\to$ Préstamo en ventanillas de descuento
				\4[] $\to$ Descuento de activos de corto plazo
			\3 Efectos de diferentes instrumentos
				\4 Repercuten en variables reales
				\4[] $\to$ Output
				\4[] $\to$ Empleo
				\4[] $\to$ Saldos reales
				\4[] $\to$ Riqueza de las familias
				\4[] ...
				\4 Diferentes vías
		\2 Interés keynesiano o efecto indirecto
			\3 Esquema
				\4[] $\downarrow$ M $\to$ $\uparrow$ i $\to$ $\downarrow$ I $\to$ $\downarrow$ Y
			\3 Explicación
				\4 $\uparrow$ interés nominal de c/p
				\4[$\then$] $\uparrow$ interés nominal de l/p
				\4[] Porque agentes prefieren liquidez
				\4[] Arbitran cambio en precios de activos
				\4[] $\to$ Venden l/p para comprar c/p
				\4[] $\then$ $\Delta$ se transmite a lo largo de yield curve
				\4 Inversores valoran interés de l/p
				\4[] Para decidir invertir
				\4[] Comparan tipo de interés (real) con EMK\footnote{Eficiencia marginal del capital (\textit{marginal efficiency of capital}). En terminología keynesiana, equivale a la TIR de la inversión. }
				\4[] Si EMK > interés l/p
				\4[] $\to$ Inversión se lleva a cabo
				\4 Caída de la inversión
				\4[] Porque menos proyectos cumplen regla
				\4[] $\then$ Caída de la demanda agregada
				\4 Animal spirits
				\4[] Afectan exógenamente a EMK
				\4[] $\to$ Flujos esperados de caja
				\4[] Interaccionan con canal de interés
		\2 Monetarista o efecto directo
			\3 Esquema
				\4[] $\uparrow$ $M_S$ $\to$ $\downarrow M_D - M_S$ $\to$ $\uparrow (B_D - B_S), \uparrow Y_D - Y_S$
			\3 Explicación
				\4 Efecto directo
				\4[] Friedman entiende como efecto más importante
				\4[] Partiendo de Ley de Walras
				\4[] En contexto Keynesiano:
				\4[] $M^D -  M^S + B^D - B^S = 0$
				\4[] $\then$ $M^S > M^D$ se cubre con demanda de bonos
				\4[] Friedman afirma que:
				\4[] \fbox{$(M^D - M^S) + (B^D - B^S) + (Y^D - Y^S) = 0$}
				\4[] $\then$ $M^S > M^D$ se pueden cubrir con demanda de bienes duraderos
				\4[] $\then$ $\Delta M$ no tiene por qué reducir tipo de interés
				\4[] Aumento de Y aumenta demanda de saldos reales
				\4[] $\then$ Mercado de dinero se equilibra
				\4[] $\then$ Aumento de M tiene efecto directo sobre Y
		\2 Activos financieros: q de Tobin
			\3 Esquema
				\4[] $\downarrow$ M $\to$ $\downarrow$ $P_E$ $\to$ $\downarrow$ $q$ $\to$ $\downarrow$ $I$ $\to$ $\downarrow$ $Y$, $\downarrow$ L
			\3 Explicación
				\4 Caída de oferta monetaria
				\4[] Reduce liquidez para comprar acciones
				\4[] $\to$ Precio del equity cae
				\4 Caída del precio del equity
				\4[] $\to$ Reduce $q$ de Tobin $\left( \dfrac{P_E}{\text{Coste de reposición}} \right)$
				\4 Caída de la $q$ de Tobin
				\4[] Reduce inversión
				\4[] $\to$ Caída del output y empleo
		\2 Activos financieros: renta permanente
			\3 Esquema
				\4[] $\downarrow$ M $\to$ $\downarrow$ $P_\text{E}$ $\to$ $\downarrow$ $Y_P$ $\to$ $\downarrow$ C $\to$ $\downarrow$ $Y$, $\downarrow$ L
			\3 Explicación
				\4 Caída de la oferta monetaria
				\4[] Reduce precios de activos financieros
				\4 Caída de riqueza financiera
				\4[] Reduce renta permanente de hogares
				\4 Caída de renta permanente
				\4[] Reduce demanda de consumo
				\4[] $\to$ Caída de output y empleo
		\2 Activos financieros: expectativa de crisis
			\3 Esquema
				\4[] $\downarrow$ M $\to$ $\downarrow$ $P_E$ $\to$ $\uparrow$ Prob. crisis $\to$ $\downarrow$ $C_\text{durad.}$, $\downarrow$ I $\to$ $\downarrow$ Y, $\downarrow$ L
			\3 Explicación
				\4 Caída de precios de activos financieros
				\4[] Aumenta expectativas de crisis
				\4 Expectativa de crisis
				\4[] Reduce consumo bienes duraderos e inversión
				\4[] $\to$ Caída de output y empleo
		\2 Activos inmobiliarios: consumo e inversión
			\3 Esquema
				\4[] $\downarrow$ M $\to$ $\downarrow$ $P_\text{Im.}$ $\to$ $\downarrow$ $Y_P$ $\to$ $\downarrow$ C $\to$ $\downarrow$ $Y$, $\downarrow$ L
			\3 Explicación
				\4 Proceso similar a anteriores
				\4 Destacado por monetaristas
				\4[] Especialmente Meltzer
		\2 Crédito bancario
			\3 Esquema
				\4[] $\downarrow$ M $\to$ $\downarrow$ Depósitos $\to$ $\downarrow$ Crédito $\to$ $\downarrow$ I, C $\to$ $\downarrow$ Y, $\downarrow$ L
			\3 Explicación
				\4 Bancos tienen papel central en asignación
				\4[] Amortiguan asimetrías de información
				\4[] Canalizan crédito a agentes pequeños
				\4[] $\to$ Sin acceso directo a mercados de capital
				\4 Reducción de oferta monetaria
				\4[] Reduce oferta de reservas
				\4 Escasez de reservas bancarias
				\4[] Reduce créditos ofertados
				\4[] $\to$ Caída del crédito
				\4 Escasez de crédito
				\4[] Reduce inversión y consumo
				\4[] $\to$ Caída de empleo y output
		\2 Balances: información asimétrica
			\3 Esquema
				\4[] $\downarrow$ M $\to$ $\downarrow$ $P_E$ $\to$ $\uparrow$ SA, RM $\to$ $\downarrow$ Crédito $\to$ $\downarrow$ I $\to$ $\downarrow$ Y, $\downarrow$ L
			\3 Explicación
				\4 Caída en precio de equity
				\4[] Reduce valor del colateral
				\4[] Agudiza problemas de información
				\4 Selección adversa
				\4[] Inversores con garantías salen de mercado
				\4[] $\to$ Aumento del riesgo de las inversiones
				\4[] $\then$ Reducción del crédito
				\4 Riesgo moral
				\4[] Caída del precio del equity
				\4[] $\to$ Accionistas tienen menos que perder
				\4[] $\to$ Aumenta riesgo de proyectos
				\4[] $\then$ Reducción del crédito
		\2 Balances: flujo de caja
			\3 Esquema
				\4[] $\downarrow$ M $\to$ $\uparrow$ i $\to$ $\downarrow$ $\text{FCF}_\text{EQUITY}$ $\to$ $\uparrow$ SA y RM $\to$ $\downarrow$ Crédito $\to$ $\downarrow$ I $\to$ $\downarrow$ Y, $\downarrow$ L
			\3 Explicación
				\4 Aumento del interés de deuda
				\4[] Aumenta salida de flujos a deuda
				\4[] Reduce FCF al equity
				\4 Reducción de FCF a equity
				\4[] Aparecen problemas de agencia
				\4[] $\to$ Crédito se contrae
				\4[] $\then$ Reducción de la inversión
		\2 Expectativas
			\3 Idea clave
				\4 Modelos con HER
				\4 Agentes estiman variables nominales futuras
				\4[] Inflación
				\4[] Interés nominal
				\4[] $\to$ Deciden en base a estimación
				\4 Información sobre políticas futuras
				\4[] Afecta expectativas de agentes
				\4[] $\to$ Altera comportamiento racional
				\4[] $\then$ Afecta variables distintas de operativas
			\3 Modelos DSGE de NEK
				\4 IS dinámica
				\4[] $\tilde{y}_t = \textrm{E}_t \left\lbrace \tilde{y}_{t+1} \right\rbrace - \frac{1}{\sigma} \left( \underbrace{i_t - \textrm{E}_t \left\lbrace \pi_{t+1} \right\rbrace}_{r_t} - r^n_t \right) $
				\4 Fijando expectativas de agentes sobre:
				\4[] Output gap esperado futuro $E_t(\tilde{y}_{t+1})$
				\4[] $\to$ Vía interés nominal futuro
				\4[] Inflación futura $E_t{\pi_{t+1}}$
				\4[$\then$] Posible afectar output gap presente $\tilde{y}_t$
		\2 Tipo de cambio
			\3 Esquema
				\4[] $\downarrow$ M $\to$ $\downarrow$ E $\to$ $\downarrow$ NX $\to$ $\downarrow$ Y, $\downarrow$ L
			\3 Explicación
				\4 $\Delta$ tipo nominal doméstica
				\4[] Entrada de capitales
				\4[] $\to$ Apreciación de moneda
				\4 Pérdida de competitividad exterior
				\4[] Caída de exportaciones netas
				\4[] $\to$ Caída de output y empleo
		\2 Efectos redistributivos
			\3 Idea clave
				\4 PM afecta distribución de renta
				\4 Ejemplo: caída del tipo de interés
				\4[] Cae rendimiento de deuda
				\4[] Cae riqueza de acreedores
				\4[] Aumenta riqueza de familias endeudadas
				\4[] $\then$ Redistribución de acreedores a deudores
		\2 Efectos de segunda ronda
			\3 Idea clave
				\4 Variables afectadas afectan a otras variables
				\4[] Incluida la política monetaria
				\4[] $\to$ Que puede tener componente endógeno
				\4[] $\then$ Que a su vez afectan vars. afectadas inicialmente
			\3 Ejemplos
				\4 PM expansiva aumenta PIB
				\4 Demanda agregada aumenta
				\4[] Efectos multiplicador
				\4 Economía alcanza capacidad
				\4[] $\then$ Aumento de la inflación
				\4[] $\then$ Caída del tipo real
				\4[] $\then$ Reacción de PM y aumento de tipos
	\1[] \marcar{Conclusión}
		\2 Recapitulación
			\3 Efectos de la PM
			\3 Mecanismos de transmisión de la PM
		\2 Idea final
			\3 Diseño e implementación de la PM
				\4 No analizada en este tema
				\4 Objetivo último de análisis de efectos
				\4 Necesario definir objetivos
				\4[] Modelos con agentes optimizadores
				\4[] $\to$ Derivar resultados normativos
			\3 Spillovers internacionales de la PM
				\4 PM tienen efectos más allá de fronteras
				\4 Coordinación de política monetaria
				\4[] Importante factor no analizado
			\3 Incertidumbre y falta de información
				\4 Constante en política monetaria
				\4 Sistema financiero enorme complejidad
				\4 Difícil caracterizar todos efectos
				\4 No linealidades
				\4[$\to$] Blinder: PM es arte y ciencia
\end{esquemal}

\graficas

\begin{axis}{4}{Efectos de política monetaria expansiva en un modelo IS-LM con supuestos keynesianos.}{Y}{r}{islmkeynesianopoliticamonetaria}
	% LM
	\draw[-] (0.3, 0.5) -- (1,0.5) to [out=0, in=260](4,4);
	\node[left] at (4,4){$\text{LM}_0$};
	
	\draw[-] (1,0.5) -- (1.7,0.5) to [out=0, in=260](4.7,4);
	\node[right] at (4.7,4){$\text{LM}_1$};
	
	\draw[-{Latex}] (4.05,3.7) -- (4.6,3.7);
	\draw[-{Latex}] (3.8,3) -- (4.4,3);
	
	% IS
	\draw[-] (0.2,4) -- (3,0.2);
	\node[right] at (3,0.2){IS};
	
	% Equilibrios
	\node[circle,fill=black,inner sep=0pt,minimum size=4pt] (a) at (2.44,0.98) {};
	\node[above] at (2.44, 0.98){\tiny 0};
	
	\node[circle,fill=black,inner sep=0pt,minimum size=4pt] (a) at (2.64,0.71) {};
	\node[below] at (2.64,0.67){\tiny 1};
\end{axis}

\begin{axis}{4}{Efectos de una expansión de la oferta monetaria en un modelo IS-LM con supuestos neoclásicos}{Y}{r}{islmneoclasicopoliticamonetaria}
	% IS
	\draw[-] (0.2,4) -- (4,0.2);
	\node[right] at (4,0.2){IS};
	
	% LM
	\draw[-] (0.2,0) -- (4,4);
	\node[right] at (4,4){$\text{LM}_0$, $\text{LM}_1$};
	
	\draw[-] (1.5,0) -- (4,2.6325);
	\node[right] at (4,2.6235){LM'};
	
	\draw[-{Latex}] (1,0.6) -- (1.5,0.2);
	\draw[{Latex}-] (2.7,2.3) -- (3.2,1.9);
	
	% Y fijo
	\draw[dotted] (2.15,0) -- (2.15,4);
	\node[below] at (2.15,0){$\bar{Y}$};
	
	% Y tras expansión monetaria
	\draw[dashed] (2.81,0) -- (2.81,1.35);
	\node[below] at (2.81,0){$Y'$};
	
	% Equilibrios
	\node[circle,fill=black,inner sep=0pt,minimum size=4pt] (a) at (2.16,2.08) {};
	\node[left] at (2.10,2.08){\tiny 0, 1};
	
	\node[circle,fill=black,inner sep=0pt,minimum size=4pt] (a) at (2.8,1.4) {};
	\node[left] at (2.8,1.4){\tiny 0'};
\end{axis}

\preguntas
\seccion{13 de marzo de 2017}
\begin{itemize}
    \item Comente los “lags” internos y externos de la política monetaria.
    \item En economías abiertas, el tipo de cambio es una variable fundamental. Comente su papel como canal de transmisión de política monetaria.
    \item ¿Cuáles son las limitaciones de la efectividad de la política monetaria más allá de los retardos?
    Ha dicho usted que con la aparición de la Teoría General de 1936, Keynes es pionero en reconocer la efectividad de la política monetaria a la hora de generar efectos reales. Pero también ha explicado usted la trampa de liquidez. ¿Cómo compatibilizar estos dos aspectos?
\end{itemize}

\seccion{22 de marzo de 2017}
\begin{itemize}
    \item Le ha dado poca importancia a la NEK. ¿Podría aportar algo más sobre sus aportaciones en PM?
    \item Según la Tª Cuantitativa del dinero, todos los aumentos en la masa monetaria se trasladan a precios. ¿Cómo explica que la masa monetaria en EEUU se haya multiplicado por 4 y no los precios?
    \item Ha dicho que la PM es más rápida y discrecional que la PF. ¿Por qué?
    \item Imagine un país con exceso de liquidez. ¿Qué instrumentos de PM y PF utilizaría?
\end{itemize}


\seccion{Test 2006}

\textbf{22.} Indique cuál de estas afirmaciones es FALSA:

\begin{itemize}
	\item[a] La Reserva Federal de EEUU instrumenta su política monetaria mediante operaciones de mercado abierto (``open market'').
	\item[b] La Reserva Federal tiene objetivos explícitos no cuantificados de estabilidad de precios y de pleno empleo.
	\item[c] El Banco Central Europeo instrumenta su política monetaria a través de operaciones de mercado abierto (``open market'').
	\item[d] El Banco Central Europeo tiene un objetivo explícito cuantificado de estabilidad de precios y no tiene objetivo explícito de pleno empleo.
\end{itemize}


\notas


\textbf{2006} \textbf{22.} ANULADA

\bibliografia

Mirar en Palgrave:
\begin{itemize}
	\item capital, credit and money markets *
	\item central banking *
	\item central bank independence
	\item financial intermediaries *
	\item financial intermediation *
	\item high-powered money and the monetary base *
	\item inflation
	\item inflation dynamics
	\item inflation targeting
	\item international monetary policy
	\item \textbf{liquidity trap} *
	\item monetary aggregation
	\item monetary and fiscal policy overview
	\item monetary business cycle models (sticky prices and wages)
	\item monetary business cycle (imperfect information)
	\item monetary economics, history of *
	\item monetary equilibrium
	\item monetary overhang
	\item \textbf{monetary policy} *
	\item \textbf{monetary policy, history of} *
	\item \textbf{monetary transmission mechanism} *
	\item money
	\item money and general equilibrium
	\item money illusion
	\item money in economic activity
	\item money supply
	\item money, classical theory of
	\item neutrality of money
	\item optimum quantity of money
	\item Phillips curve (new views)
	\item \textbf{quantitative easing by the major western central banks during the global financial crisis}
	\item quantity equations: early history
	\item quantity theory of money
	\item regulatory responses to the financial crisis: an interim assessment
	\item search-and-matching models of monetary exchange
	\item sound money
	\item Taylor rules
	\item term structure of interest rates
	\item tight money
	\item Tobin's q
\end{itemize}

Arrow, K. et al \textit{Handbook of Monetary Economics} Ch. 24 Implementation of Monetary Policy:How Do Central Banks Set Interest Rates? -- En carpeta de macroeconomía

Ball, L.; Mankiw, G.  \textit{The NAIRU in theory and practice.} (2002) Journal of Economic Perspectives

Bernanke, B. S.; Gertler, M. \textit{Inside the Black Box: The Credit Channel of Monetary Policy Transmission} (1995) Journal of Economic Perspectives: fall 1995 -- En carpeta del tema

Boivin, J.; Kiley, M. T.; Mishkin, F. S. \textit{How Has the Monetary Transmission Mechanism Evolved Over Time?} Ch. 8 Handbook of Monetary Economics

Cecchetti, S.; Schoenholtz, K. \textit{Money, Banking, and Financial Markets} (2014) Fourth Edition -- En carpeta Finanzas

De Grauwe, P.; Polan, M. \textit{Is Inflation Always and Everywhere a Monetary Phenomenon?} (2005) Scandinavian Journal of Economics -- En carpeta del tema

Dell'Ariccia, G.; Rabanal, P.; Sandri, D. \textit{Unconventional Monetary Policies in the Euro Area, Japan, and the United Kingdom} (2018) Journal of Economic Perspectives, Fall -- En carpeta del tema

Metzler, A. H. \textit{Monetary, Credit and (Other) Transmission Processes: A Monetarist Perspective} (1995) Journal of Economic Perspectives: fall 1995 -- En carpeta del tema

MacCandless, G. T.; Weber, W. \textit{Some Monetary Facts} (1995) Federal Reserve of Minneapolis Quarterly Review -- En carpeta del tema

Mishkin, F. S. \textit{Symposium on the Monetary Transmission Mechanism} (1995) Journal of Economic Perspectives -- En carpeta del tema

Obstfeld, M.; Rogoff, M. \textit{The Mirage of Fixed Exchange Rates} (1995) Journal of Economic Perspectives: fall 1995 -- En carpeta del tema

Pfister, C.; Sahuc, J-G. (2020) \textit{ Unconventional Monetary Policies: A Stock-Taking Exercise} Banque de France. Working Paper. \href{https://publications.banque-france.fr/sites/default/files/medias/documents/wp761.pdf}{Disponible aquí} -- En carpeta del tema.

Poole, W. \textit{Optimal Choice of Monetary Policy Instruments in a Simple Stochastic Macro Model} (1970) The Quarterly Journal of Economics -- En carpeta del tema

Ramey, V. A. \textit{Macroeconomic shocks and their propagation} (2016) NBER Working Paper Series -- En carpeta del tema

Taylor, J. B. \textit{The Monetary Transmission Mechanism: An Empirical Framework} (1995) Journal of Economic Perspectives: fall 1995 -- En carpeta del tema

Varios Autores. \textit{Symposium: the Monetary Transmission Mechanism} (1995) Journal of Economic Perspectives: Fall 1995 -- \url{https://www.aeaweb.org/issues/224}

Veronesi, P. \textit{Handbook of Fixed Income Securities} (2016) Ch. 5 Bond Markets and Monetary Policy. Ch. 6 Bond Markets and Unconventional Monetary Policy

Walsh, C. E. \textit{Monetary Theory and Policy} 3rd edition. Ch. 11 Monetary Policy Operating Procedures.

\end{document}
