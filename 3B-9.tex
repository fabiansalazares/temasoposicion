\documentclass{nuevotema}

\tema{3B-9}
\titulo{Comercio internacional y crecimiento económico. Especial referencia a los efectos del comercio sobre el crecimiento.}

\begin{document}

\ideaclave

\seccion{Preguntas clave}

\begin{itemize}
	\item ¿Cómo afecta el crecimiento económico al comercio internacional?
	\item ¿Cómo afecta el comercio internacional al crecimiento?
	\item ¿Qué modelos teóricos muestran la relación entre ambos fenómenos?
	\item ¿Qué evidencia empírica existe al respecto?
\end{itemize}

\esquemacorto

\begin{esquema}[enumerate]
	\1[] \marcar{Introducción}
		\2 Contextualización
			\3 Evolución histórica de la renta per cápita
			\3 Evolución del comercio internacional
			\3 Interacción entre comercio y crecimiento
		\2 Objeto
			\3 ¿Cómo afecta el crecimiento económico al comercio internacional?
			\3 ¿Cómo afecta el comercio internacional al crecimiento?
			\3 ¿Qué modelos teóricos muestran la relación entre ambos fenómenos?
			\3 ¿Qué evidencia empírica existe al respecto?
		\2 Estructura
			\3 Efectos del crecimiento sobre el comercio int.
			\3 Efectos del comercio int. sobre el crecimiento
	\1 \marcar{Efectos del crecimiento sobre el comercio internacional}
		\2 Idea clave
			\3 Contexto
			\3 Objetivos
		\2 Dotación de factores
			\3 Contexto
			\3 Objetivo
			\3 Resultados
		\2 Progreso técnico
			\3 Concepto
			\3 Progreso neutral en un sector
			\3 Progreso ahorrador de f.p. intensivo de exportado
			\3 Progreso ahorrador de f.p. no intensivo de exportado
			\3 Progreso ahorrador de f.p. intensivo en importado
			\3 Progreso ahorrador de f.p. no intensivo en importado
			\3 Progreso ahorrador de X en sector intensivo en X
			\3 Progreso ahorrador de X en sector intensivo en Y
		\2 Renta y dependencia exterior
			\3 Idea clave
			\3 Efectos sobre el consumo
			\3 Efectos sobre la producción
			\3 Efectos globales
			\3 Implicaciones
		\2 Crecimiento empobrecedor
			\3 Idea clave
			\3 Implicaciones
	\1 \marcar{Efectos del comercio internacional sobre el crecimiento}
		\2 Idea clave
			\3 Política comercial
			\3 Efectos del comercio sobre el crecimiento
		\2 Promoción de exportaciones vs. sust. de importaciones
			\3 Idea clave
			\3 ISI -- Sustitución de importaciones
			\3 EOI -- Industrialización orientada a la exportación
			\3 Valoración
		\2 Crecimiento exógeno
			\3 Idea clave
			\3 Ricardo
			\3 Efecto de CI en modelos neoclásicos
		\2 Crecimiento endógeno
			\3 Idea clave
			\3 Smith
			\3 Learning-by-doing y home-market effects
			\3 Spillovers tecnológicos
			\3 Calidad y destrucción creativa
		\2 Instituciones
			\3 Idea clave
			\3 Implicaciones
			\3 Coordinación y cambio político
		\2 Nueva Economía Geográfica
			\3 Idea clave
			\3 Formulación
			\3 Implicaciones
			\3 Valoración
		\2 Economía política del comercio
			\3 Idea clave
			\3 Formulación
			\3 Implicaciones
			\3 Valoración
		\2 Evidencia empírica
			\3 Dirección de la causalidad
			\3 Efectos escala
			\3 Spillovers tecnológicos
			\3 Convergencia
	\1[] \marcar{Conclusión}
		\2 Recapitulación
			\3 Efectos del crecimiento sobre el comercio int.
			\3 Efectos del comercio int. sobre el crecimiento
		\2 Idea final
			\3 Globalización
			\3 Desarrollo económico
			\3 Liberalización comercial y financiera
			\3 Inversión extranjera directa

\end{esquema}

\esquemalargo


















\begin{esquemal}
	\1[] \marcar{Introducción}
		\2 Contextualización
			\3 Evolución histórica de la renta per cápita
				\4 A lo largo de historia humana
				\4[] PIBpc prácticamente estable
				\4[] Muy similar en todo el mundo
				\4 Divergencia global
				\4[] A partir del año 1000 d.C
				\4[] $\to$ Según algunos autores
				\4[] A partir de 1800 d.C.
				\4[] $\to$ Según toda la literatura
				\4[] Europa occidental + satélites
				\4[] $\to$ Comienzan a divergir
				\4[] $\then$ Crecimiento económico sostenido
				\4[] $\then$ Diferencias de renta actuales
			\3 Evolución del comercio internacional
				\4 Explosión en últimos siglos
				\4[] $\to$ Y más aún desde post 2GM
				\4 Avance tecnológico:
				\4[] $\downarrow$ de costes de transporte
				\4[] $\downarrow$ de costees informacionales
				\4 Sujeto de estudio relativamente antiguo:
				\4[] $\to$ Smith, Ricardo, Mill
				\4[] Ligado a la evolución de:
				\4[] $\to$ teoría económica
				\4[] $\to$ hallazgos empíricos
			\3 Interacción entre comercio y crecimiento
				\4 Efectos recíprocos y dinámicos
				\4 Controversias en policy-making
				\4[] Efectos positivos y negativos de CI sobre comercio
				\4[] $\to$ Cuánto comerciar con el exterior
				\4[] $\to$ Qué bienes comerciar
				\4[] $\to$ Qué sectores crecen con comercio
				\4 Evolución histórica del análisis
				\4[] Como habitualmente, sujeto a:
				\4[] $\to$
				\4[] Adam Smith:
				\4[] $\to$ CI aumenta especialización y crecimiento
				\4[] Ricardo:
				\4[] $\to$ CI para escapar de estado estacionario
				\4[] Mill y Marshall
				\4[] $\to$ Demandas relativas son importantes
				\4[] S. XX hasta 60s
				\4[] $\to$ Análisis en marco neoclásico
				\4[] $\to$ Estática comparativa
				\4[] $\to$ Crecimiento dado exógenamente
				\4[] A partir de años 60
				\4[] $\to$ Crecimiento neoclásico con CI
				\4[] $\to$ CI como motor de crecimiento endógeno
				\4[] $\then$ Análisis dinámico
		\2 Objeto
			\3 ¿Cómo afecta el crecimiento económico al comercio internacional?
			\3 ¿Cómo afecta el comercio internacional al crecimiento?
			\3 ¿Qué modelos teóricos muestran la relación entre ambos fenómenos?
			\3 ¿Qué evidencia empírica existe al respecto?
		\2 Estructura
			\3 Efectos del crecimiento sobre el comercio int.
				\4 Teoría
				\4 Evidencia empírica
			\3 Efectos del comercio int. sobre el crecimiento
				\4 Teoría
				\4 Evidencia empírica
	\1 \marcar{Efectos del crecimiento sobre el comercio internacional}
		\2 Idea clave
			\3 Contexto
				\4 Crecimiento presente aunque no haya comercio
				\4[] $\then$ Posible tomar como dado
				\4 Múltiples modelos de análisis
				\4[] Generalmente estáticos
				\4[] $\to$ Análisis de equilibrios de estática comparativa
			\3 Objetivos
				\4 Cómo cambia patrón de comercio
				\4[] Ante crecimiento en variables
				\4[] $\to$ Renta
				\4[] $\to$ Dotación de factores
				\4[] $\to$ Productividad de factores
		\2 Dotación de factores
			\3 Contexto
				\4[] Modelo neoclásico/H-O-S
				\4[] Teorema de Rybczynski
			\3 Objetivo
				\4[] Qué efecto tiene crecimiento exógeno de ff.pp.
				\4[] $\to$ Sobre patrón de comercio
			\3 Resultados
				\4[] $\uparrow$ de un factor
				\4[] $\to$ $\uparrow$ producción de bien intensivo en factor
				\4[] $\to$ $\uparrow$
				\4[] $\then$ Tendencia a especialización en bien intensivo en f.p. que $\uparrow$
				\4[] Representación gráfica en diagrama de Lerner-Pearce
				\4[] \grafica{rybczynski}
		\2 Progreso técnico
			\3 Concepto
				\4[] Crecimiento de la prod. de uno o varios f.p.
				\4[] $\to$ En un sector de la economía
				\4 Neutral en sentido de Hicks
				\4[] Productividades relativas de ff.pp
				\4[] $\to$ Se mantienen constantes si ratio L/K constante
				\4 Cambio tecnológico ahorrador de X/sesgado a X
				\4[] Implica aumento de productividad de X
				\4[] $\to$ Superior al de Y
				\4[] $\then$ Para misma cantidad de output, se utiliza menos X
				\4[] $\then$ Cae ratio X/Y si precios de ff.pp. constantes
				\4 Contexto Heckscher-Ohlin
			\3 Progreso neutral en un sector
				\4 Aumenta producción del sector
				\4[] Aumenta especialización en el sector
				\4 Si el sector produce un exportable
				\4[] $\to$ RRI se moverá contra el país
				\4[] $\then$ País perderá bienestar
				\4 Si el sector produce un importable
				\4[] $\to$ RRI se moverá a favor del país
				\4[] $\then$ País ganará bienestar
			\3 Progreso ahorrador de f.p. intensivo de exportado
				\4 Factor intensivo de exportado
				\4[] Disponible ahora en mayor cantidad
				\4[] $\to$ Menor especialización
				\4[] $\to$ Mayor uso de bien intensivo de importado
				\4[] $\then$ Menor producción de importado
				\4[] $\then$ Mayor cantidad de ff.pp. disponibles
				\4[$\then$] Posible empeoramiento de RRI
			\3 Progreso ahorrador de f.p. no intensivo de exportado
				\4 Factor no intensivo de bien exportado
				\4[] Disponible ahora en mayor cantidad
				\4[] $\to$ Para producir bien importado
				\4 Bien importado puede producirse más
				\4[] Reducción de importación
				\4[$\then$] Mejora de RRI
			\3 Progreso ahorrador de f.p. intensivo en importado\footnote{Es decir, si el bien Y es el bien importado, y el bien Y es intensivo en K a diferencia del bien X exportado que es intensivo en L, el progreso técnico permite un ahorro en bien K en la producción de Y.}
				\4 Factor intensivo de bien importado
				\4[] Disponible ahora en mayor cantidad
				\4 Más barato producir bien importado
				\4[] Utilizando factor relativamente abundante
				\4[$\then$] Mejora de RRI
			\3 Progreso ahorrador de f.p. no intensivo en importado
				\4 Factor no intensivo en importado
				\4[] Liberado para utilizarse en bien exportado
				\4 Abaratamiento de bien exportado
				\4[$\then$] Puede empeorar RRI
			\3 Progreso ahorrador de X en sector intensivo en X
				\4 Aumenta productividad de factor Y
				\4[] Sustitución de X por Y
				\4[] $\to$ Ahorro de X
				\4[] $\then$ Sector menos intensivo en X
				\4 Si intensivo en X era bien importado
				\4[] País relativamente poco dotado en X
				\4[] $\to$ Ahora puede producir más barato
				\4[] $\then$ Menor importación
				\4[] $\then$ Más producción doméstica
				\4[] $\then$ Mejora de RRI
				\4[] Si era bien exportado
				\4[] $\to$ Menos recursos dedicados a exportado
				\4[] $\to$ Más recursos dedicables a importado
				\4[] $\to$ Más producción de exportado por menor coste
				\4[] $\then$ Efecto ambiguo sobre RRI
			\3 Progreso ahorrador de X en sector intensivo en Y
				\4 Aumenta productividad de factor Y
				\4[] Sustitución de X por Y
				\4[] $\to$ Ahorro de X
				\4[] $\then$ Sector menos intensivo en X
				\4[] $\then$ Sector más intensivo en Y
				\4 Mayor especialización exterior
				\4[] Si era bien importado
				\4[] $\to$ Aumenta beneficio para socio del que se importa
				\4[] Si era bien exportado
				\4[] $\to$ Mayor especialización nacional
				\4[] $\to$ Efecto ambiguo sobre RRI
				\4[] $\then$ Aumenta producción de bien exportado
				\4[] $\then$ Abaratamiento de bien exportado
				\4[] $\then$ Elasticidades relativas determinan efecto
		\2 Renta y dependencia exterior
			\3 Idea clave
				\4[] Taxonomía de efectos de aumento de renta
				\4[] $\to$ Sobre dependencia exterior
				\4[] $\then$ Sobre diferencia entre bien producido e importado
				\4[] Asumiendo:
				\4[] $\to$ Un bien importable (EDemanda positivo)
				\4[] $\to$ Especialización incompleta
				\4[] Cómo afecta un aumento de la renta:
				\4[] $\to$ Al \% de renta dedicada a importación
				\4[] Resultado de dos efectos separados
				\4[] $\to$ Demanda de consumo de bien importable
				\4[] $\to$ Oferta de bien importable
			\3 Efectos sobre el consumo
				\4 Elasticidad demanda importable-renta
				\4[] $\eta_m = \frac{d \, \ln \text{Dda. importable}} {d \, \ln \text{Renta}}$
				\4 Ultra-pro-comercial
				\4[] $\eta_m > 1$
				\4[] $\to$ $\uparrow$ Demanda importable $>$ $\uparrow$ renta en términos absolutos
				\4[] $\then$ Aumenta \% de importable sobre renta
				\4[] $\then$ Consumo de exportable cae
				\4 Pro-comercial
				\4[] $\eta_m > 1$
				\4[] $\to$ $\uparrow$ Demanda importable $>$ $\uparrow$ renta en términos relativos
				\4[] $\then$ Aumenta \% de importable sobre renta
				\4[] $\then$ Consumo de resto de bienes también aumenta pero menos
				\4 Neutral
				\4[] $\eta_m = 1$
				\4[] $\to$ $\uparrow$ demanda importable proporcional a $\uparrow$ renta
				\4[] $\then$ \% de importable sobre renta se mantiene
				\4 Anti-comercial
				\4[] $0 < \eta_m < 1$
				\4[] $\to$ $\uparrow$ Demanda importable $<$ $\uparrow$ renta en términos absolutos
				\4[] $\then$ Cae \% de importable sobre renta
				\4 Ultra-anti-comercial
				\4[] $\eta_m < 0$
				\4[] $\to$ Demanda importable cae en términos absolutos
				\4[] $\then$ Cae \% de importable sobre renta
				\4[] $\then$ Consumo de exportable aumenta más que renta
				\4[] \grafica{efectosconsumo}
			\3 Efectos sobre la producción
				\4 Referidos a producción de bien importable
				\4 Invertidos respecto a efectos de consumo
				\4 Elasticidad producción de importado-renta
				\4[] $\epsilon_m = \frac{d \, \ln \text{Prod. importable}}{d \, \ln \text{Renta}}$
				\4 Ultra-pro-comercial
				\4[] Aumento de producción de exportado
				\4[] Caída de producción de importado
				\4[] $\then$ Cae \% producción de importado
				\4[] $\then$ Cae producción de importado
				\4 Pro-comercial
				\4[] Aumenta producción de exportado
				\4[] Aumenta producción de importado
				\4[] $\to$ Producción de exportado aumenta más
				\4[] $\then$ Cae \% de producción de importado
				\4 Neutral
				\4[] Proporciones de producción
				\4[] $\to$ Exportado e importado
				\4[] $\then$ Se mantienen
				\4 Anti-comercial
				\4[] Aumento de producción de exportado
				\4[] Aumento de producción de importado
				\4[] $\to$ Mayor aumento de importado
				\4[] $\then$ Aumenta \% de importado sobre producción
				\4 Ultra-anti-comercial
				\4[] Aumento de producción de importado
				\4[] Caída de producción de exportado
				\4[] $\then$ Aumenta \% de
				\4[] \grafica{efectosproduccion}
			\3 Efectos globales
				\4 Combinación de ambos efectos
				\4 Cuando ambos son pro-comercio o anti-comercio
				\4[] $\to$ Combinación tiene mismo efecto
				\4 En caso contrario
				\4[] $\to$ Resultados ambiguos
			\3 Implicaciones
				\4[] Crecimientos de renta alteran dependencia exterior
				\4 Dos canales
				\4[] $\to$ Consumo
				\4[] $\to$ Producción
		\2 Crecimiento empobrecedor
			\3 Idea clave
				\4 Bhagwati (1958)
				\4 Crecimiento altera patrón de comercio
				\4[] Patrón de comercio altera RRI
				\4[] $\to$ Importaciones se encarecen respecto exportaciones
				\4[] $\then$ Crecimiento puede empobrecer al país vía comercio
			\3 Implicaciones
				\4 Tiene lugar con:
				\4[] UP, P, N, A en consumo
				\4 No tiene lugar con
				\4[] UA
				\4[$\then$] Crecimiento de renta puede perjudicar bienestar
				\4[] \grafica{empobrecedor}
				\4 Si RRI empeora para el país:
				\4[] $\to$ Bienestar puede empeorar, pero no necesariamente
				\4[] $\then$ Condición necesaria, no suficiente
				\4 Depende de:
				\4[] $\to$ Cuanto empeore RRI
				\4[] $\to$ Cuánto aumente la renta
				\4[] $\to$ Preferencias del consumidor representativo
	\1 \marcar{Efectos del comercio internacional sobre el crecimiento}
		\2 Idea clave
			\3 Política comercial
				\4 Efecto de comercio sobre crecimiento
				\4[] Problema fundamental de policy-making
				\4[] $\to$ Especialmente en contexto de globalización
				\4 Apertura comercial
				\4[] Valorar intervención en comercio
				\4[] $\to$ ¿Deseable?
				\4[] $\to$ ¿Qué intervención induce crecimiento?
			\3 Efectos del comercio sobre el crecimiento
				\4 Muy complejos y sobre muchos ámbitos
				\4[] Mayor variedad de inputs
				\4[] Incentivos a mejorar calidad
				\4[] Competencia y rentas
				\4[] Transferencia tecnológica
				\4[] Alteran patrones de especialización regional
				\4[] Efectos diferentes sobre distintos grupos sociales
				\4 Problemas empíricos
				\4[] Posibles factores comunes a crecimiento y comercio
				\4[] Cuantificación de innovación y similares
				\4[] Causalidad inversa crecimiento $\to$ comercio
		\2 Promoción de exportaciones vs. sust. de importaciones
			\3 Idea clave
				\4 Debate de países en desarrollo
				\4[] ¿Cómo aumentar renta?
				\4 Hipótesis de Prebisch-Singer
				\4[] Estructuralismo
				\4[] $\to$ Economía mundial: periferia agrícola--centro industrial
				\4[] Manufacturas tienen demanda elástica
				\4[] $\to$ Materias primas, más inelástica
				\4[] Deterioro de largo plazo en RRI
				\4[] $\to$ A favor de industrializados
				\4[] $\to$ Contra agrícolas y productores de materias primas
				\4[] $\then$ Industrializarse es esencial para PEDs
				\4 Dos políticas comerciales enfrentadas
				\4[] ISI -- Import-Substitution Industralization
				\4[] EOI -- Export-oriented Industrialization
				\4[] $\then$ Tratar de favorecer crecimiento de largo plazo
			\3 ISI -- Sustitución de importaciones
				\4 Énfasis en sustituir bienes importados
				\4[] $\to$ Por variedades producidas localmente
				\4[] $\then$ En consumo doméstico
				\4 Herramientas de política comercial
				\4[] Aranceles, cuotas, contigentes, subsidios
				\4 Objetivos
				\4[] Economías de escala en sector manufacturero
				\4[] $\to$ Hasta que sean capaces de competir
			\3 EOI -- Industrialización orientada a la exportación
				\4 Énfasis en exportadores nacionales
				\4[] $\to$ Aumentar competitividad en mercados internacionales
				\4[] $\then$ Capturar cuota de mercado internacional
				\4[] $\then$ Aprovechar ventajas de especialización
				\4 Herramientas de política comercial
				\4[] Subvención de exportaciones
				\4[] Manipulación de tipo de cambio
				\4[] Especialización en sectores con VComparativa
			\3 Valoración
				\4 EOI en sudeste asiático
				\4 ISI en Argentina, India, Latam...
				\4 Generalmente, EOI considerado exitoso
				\4[] Sujeto también a críticas y matizaciones
				\4[] $\to$ Política deliberada o racionalización a posteriori?
				\4[] $\to$ Elevada protección en Sudeste Asiático
				\4[] $\to$ Corea primero ISI y luego EOI
				\4[] $\to$ ¿Cómo sabe gobierno en qué especializar?
				\4[] $\to$ Muy difícil prever sectores competitivos futuros
				\4[] $\to$ Especialización intraindustrial también importante
		\2 Crecimiento exógeno
			\3 Idea clave
				\4 Modelos teóricos de crecimiento económico
				\4[] Dos grandes familias
				\4[] -- Crecimiento exógeno/neoclásico
				\4[] $\to$ Acumulación de ff.pp. tiende a EE
				\4[] $\to$ Crecimiento en EE es exógeno a modelo
				\4[] -- Crecimiento endógeno
				\4[] $\to$ Crecimiento de l/p endógeno a modelo
				\4[] $\to$ Múltiples variantes de modelización
				\4 Acumulación de factores
				\4[] Afectada por CI
				\4[] $\to$ CI determina EE
			\3 Ricardo
				\4 Dados:
				\4[] Rendimientos decrecientes de K-y-L
				\4[] Oferta inelástica de tierra fértil
				\4[] Salarios constantes
				\4 Estado estacionario
				\4[] Beneficios nulos
				\4[] Acumulación de capital se detiene
				\4 Comercio internacional
				\4[] Permite escapar a EE
				\4[] Intercambiar manufacturas por grano
				\4[] $\to$ Tierra no fértil para otros usos más rentables
				\4[] $\to$ Cae renta
				\4[] $\to$ Más beneficios durante más tiempo
				\4[] $\then$ Incentivos a acumulación de capital
				\4[] $\then$ Economía sigue creciendo gracias a CI
			\3 Efecto de CI en modelos neoclásicos
				\4 Baldwin (1989) y otros
				\4 Efecto ricardiano del CI en modelo neoclásico
				\4 CI aumenta PMg de K y H
				\4[] $\to$ Más incentivos a acumulación
				\4[] $\to$ Posibles externalidades (y crec. endógeno)
				\4 Formalización matemática en contexto dinámico
		\2 Crecimiento endógeno
			\3 Idea clave
				\4 Adam Smith
				\4[] Especialización internacionalización
				\4[] $\to$ División del trabajo a nivel internacional
				\4[] $\to$ COMECON: especialización nacional
				\4 Modelos teóricos de crecimiento económico
				\4[] Dos grandes familias
				\4[] -- Crecimiento exógeno/neoclásico
				\4[] $\to$ Acumulación de ff.pp. tiende a EE
				\4[] $\to$ Crecimiento en EE es exógeno a modelo
				\4[] -- Crecimiento endógeno
				\4[] $\to$ Crecimiento de l/p endógeno a modelo
				\4[] $\to$ Múltiples variantes de modelización
				\4[] $\then$ Aprendizaje a medida que se produce
				\4[] $\then$ Transferencia de tecnología entre países
				\4[] $\then$ Diferenciación vertical
				\4 CI y crecimiento endógeno
				\4[] Presencia de CI
				\4[] $\to$ Altera dinámica de crec. endógeno
			\3 Smith
				\4 División del trabajo
				\4[] Fuente de progreso tecnológico
				\4[] $\to$ Trabajadores cada vez trabajan mejor
				\4[] $\to$ Especialización aumenta productividad
				\4 Especialización internacional
				\4[] Aumento de productividad
			\3 Learning-by-doing y home-market effects
				\4 PTF depende de producción acumulada
				\4[] CI aumenta demanda y producción
				\4[] $\to$ Aumento de productividad
				\4 Demanda nacional grande en relación a socios
				\4[] Aparecen economías de escala
				\4 Apertura al comercio
				\4[] Más demanda
				\4[] $\to$ Aumenta efecto de economías de escala
				\4[] $\to$ Aumenta productividad total de los factores
				\4[] $\then$ Comercio favorece crecimiento
				\4[] $\then$ Apertura al comercio favorece aumento de cuota de mercado
				\4 Posibles efectos negativos de comercio
				\4[] Países que inicialmente tienen:
				\4[] $\to$ Menor tamaño de mercado
				\4[] $\to$ Menores niveles de capital
				\4[] Competidores capturan mercados tras apertura
				\4[] $\to$ Evitan acumulación de capital
				\4[] $\to$ Dificultan learning-by-doing
				\4[] $\to$ Reducen economías de escala
				\4[] $\then$ CI puede perjudicar países y/o industrias
			\3 Spillovers tecnológicos
				\4 Grossman y Helpman (1991)
				\4[] Trade, knowledge spillovers and growth
				\4 Variedades disponibles de bienes intermedios
				\4[] Determinan utilidad de consumo y output
				\4[] $\then$ Más variedades, más crecimiento
				\4 Entrada de nuevas variedades
				\4[] Depende de coste fijo
				\4[] Coste fijo constante
				\4[] $\to$ Equilibrio estacionario sin empresas nuevas
				\4[] Coste fijo variable dependiente de variedades
				\4[] $\to$ Crecimiento endógeno
				\4 Spillovers tecnológicos
				\4[] Número de variedades en socios comerciales
				\4[] $\to$ Aumentan utilidad de consumo
				\4[] $\to$ Reducen coste de nuevas variedades en doméstico
				\4[] $\then$ Aumento de tasa de crecimiento
				\4[] $\then$ Comercio tiene efecto positivo sobre crecimiento
				\4[] $\then$ Efecto escala
				\4 Sin spill-overs tecnológicos
				\4[] Variedades de socios comerciales
				\4[] $\to$ Aumentan utilidad de consumo y output
				\4[] $\to$ No reducen coste de nuevas variedades en doméstico
				\4[] $\then$ Comercio desfavorable a crecimiento
			\3 Calidad y destrucción creativa
				\4 Grossman y Helpman (1991)
				\4[] Quality Ladders in the Theory of Growth
				\4 Crecimiento como mejora de calidad
				\4 CI como catalizador de crecimiento
				\4[] Permite mayores rentas por más calidad
				\4[] Aprendizaje de calidades superiores
		\2 Instituciones\footnote{Ver North (1991).}
			\3 Idea clave
				\4 Concepto de institución
				\4[] Conjunto de restricciones formales y informales
				\4[] $\to$ Que reducen incertidumbre del intercambio
				\4 Explosión reciente de la literatura
				\4[] En últimas décadas
				\4[] Nuevo institucionalismo
				\4[] Esfuerzo por cuantificar impacto de instituciones
			\3 Implicaciones
				\4 Efecto de CI sobre crecimiento
				\4[] Posible gracias a instituciones
				\4[] Permiten mitigar problemas:
				\4[] $\to$ Problemas de agencia
				\4[] $\to$ Selección adversa
				\4[] $\to$ Violencia
				\4[] $\to$ Problemas de comunicación
				\4[] $\then$ Especialmente problemáticos en CI
				\4 Reparto de beneficios
				\4[] Determinado institucionalmente
			\3 Coordinación y cambio político
				\4 Frontera de la investigación
				\4 Acemoglu (2011) y otros
				\4[] Sobre primavera árabe
				\4 Otros autores sobre revueltas en Canadá
				\4[] Encuentran asociación robusta:
				\4[] $\to$ Exposición a comercio internacional
				\4[] $\to$ Probabilidad de revuelta frente Imperio Británico
				\4 Comercio mejora coordinación
				\4 Coordinación reduce coste de cambio institucional
				\4[] Vía rebeliones, grupos políticos, etc..
		\2 Nueva Economía Geográfica
			\3 Idea clave
				\4 Contexto
				\4[] Von Thünen
				\4[] $\to$ Análisis de localización óptima en ciudades
				\4[] $\to$ Pionero en economía espacial
				\4[] Marshall
				\4[] $\to$ Economías de escala tecnológicas con concentración
				\4[] i. Más facilidad para encontrar mano de obra
				\4[] ii. Spill-overs de información
				\4[] iii. Producción de inputs intermedios no comerciables
				\4[] $\then$ Difícil formalización
				\4[] $\then$ Análisis canónico hasta NEG
				\4[] Teoría clásica del CI
				\4[] $\to$ Economías son puntos sin dimensión espacial
				\4[] $\to$ Factores inmóviles entre países
				\4[] Hotelling
				\4[] $\to$ Análisis pionero
				\4[] $\to$ Formalización de decisión de localización empresas
				\4[] Salop (1979)
				\4[] $\to$ Entrada de empresas en contexto espacial
				\4[] Dixit y Stiglitz (1977)
				\4[] $\to$ Formalización de competencia monopolística
				\4[] $\to$ Análisis formal de equilibrio general
				\4[] $\then$ Agentes prefieren variedad
				\4[] $\then$ Incentivos a entrada de nuevas empresas/variedades
				\4[] Evidencia empírica
				\4[] $\to$ Aparición de núcleos y cinturones industriales
				\4[] $\to$ Concentración de población donde hay desarrollo industrial
				\4[] $\to$ Trabajo industrial móvil con patrones persistentes
				\4 Objetivos
				\4[] Explicar dinámicas de aglomeración
				\4[] $\to$ En contexto de desarrollo industrial
				\4[] $\to$ En contexto de bajada de precios de transporte
				\4[] Explicar patrón de comercio
				\4[] $\to$ En contexto de movilidad del trabajo
				\4[] $\to$ En contexto de costes de transporte y H-M-Effect
				\4 Resultados
				\4[] Dinámicas de aglomeración-dispersión
				\4[] $\to$ Dependen de parámetros clave
				\4[] Patrón de comercio endógeno
				\4[] $\to$ Depende de evolución de aglomeración-dispersión
				\4[] Integración comercial
				\4[] $\to$ Puede inducir aglomeración
				\4[] Movimientos de trabajadores
				\4[] $\to$ Endógenos
				\4[] $\to$ Posible concentración geográfica
			\3 Formulación
				\4 Dos bienes consumidos
				\4[] Agrícola homogéneo $C_A$
				\4[] Manufacturado compuesto $C_M = \left( C_i^{\frac{\epsilon-1}{\epsilon}} \right)^{\frac{\epsilon}{\epsilon-1}}$
				\4 Dos factores de producción
				\4[] Campesinos inmóviles entre países
				\4[] $\to$ Distribuidos entre los dos países de manera exógena
				\4[] Obreros móviles entre países
				\4 Dos países/regiones A y B
				\4[] Campesinos repartidos equitativamente entre países
				\4[] Obreros con distribución inicial arbitraria
				\4[] $\to$ Sujeto a variación endógena
				\4 Demanda de bienes
				\4[] Obreros y campesinos iguales demandas
				\4[] $\to$ Se distribuye entre agrícola y manufacturero
				\4 Coste de transporte
				\4[] Agrícola sin coste de transporte
				\4[] Manufacturero con costes tipo iceberg
				\4[] $\to$ Para que llegue 1 hace falta enviar $\tau > 1$
				\4 Decisión de localización de obreros
				\4[] Donde haya mayor salario
				\4[] Dos efectos contrapuestos afectan localización
				\4[] $\to$ Efecto competencia
				\4[] $\to$ Efecto demanda
				\4 Efecto demanda
				\4[] Localización cerca de la demanda
				\4[] $\to$ Permite superar costes de transporte
				\4[] Si coste fijo superior a costes de transporte
				\4[] $\to$ Preferible concentrar producción
				\4[] Cuanta mayor población obrera
				\4[] $\to$ Más se retroalimenta el efecto demanda
				\4[] Producción de variedades manufactureras concentradas
				\4[] $\to$ Aumenta salario real de obreros en aglomeración
				\4[] $\then$ Tendencia hacia concentración donde ya se produce manufact.
				\4 Efecto competencia
				\4[] Costes de transporte
				\4[] $\to$ Reducen competencia con variedades en otro país
				\4[] $\then$ Permiten aumentar precios
				\4[] Cuanta más población campesina sobre total
				\4[] $\to$ Mayor es la demanda que no se mueve
				\4[] $\then$ Más incentivos a localizarse donde no se producen variedades
				\4 Parámetros iniciales determinan resultado
				\4[] Preferencia por la variedad $\epsilon$
				\4[] $\then$ Aumenta importancia de tener más variedades
				\4[] Costes de transporte
				\4[] $\to$ Reduce competencia con variedades en otro país
				\4[] $\to$ Actúa a favor de la aglomeración
				\4[] Peso del sector manufacturero en población
				\4[] $\to$ Aumenta efecto de movimiento de L sobre demanda
				\4[] $\then$ Actúa a favor de la aglomeración
				\4 Dinámica del movimiento de obreros y comercio
				\4[] Asumiendo
				\4[] $\to$ Preferencia suficiente por la variedad
				\4[] $\to$ Suficiente peso del sector manufacturero
				\4[] Dispersión en equilibrio
				\4[] $\to$ Costes de transporte elevados +  pob. obrera reducida
				\4[] $\to$ Elevado efecto competencia
				\4[] $\to$ Poco efecto demanda
				\4[] $\then$ Tendencia a dispersión
				\4[] $\then$ \grafica{krugman91dispersion}
				\4[] Múltiples equilibrios
				\4[] $\to$ Costes de transporte intermedios + pob. obrera moderada
				\4[] $\to$ Si población dispersa, tendencia a dispersión
				\4[] $\to$ Si población inicialmente aglomerada, tendencia aglom.
				\4[] $\then$ Equilibrio depende de shock/condición inicial
				\4[] $\then$ Múltiples equilibrios dispersos y aglomerados
				\4[] $\then$ \grafica{krugman91multiplesequilibrios}
				\4[] Aglomeración en equilibrio
				\4[] $\to$ Costes de transporte reducidos + elevada pob. obrera
				\4[] $\to$ Poco efecto competencia
				\4[] $\to$ Efecto demanda elevado
				\4[] $\then$ Tendencia a aglomeración
				\4[] $\then$ \grafica{krugman91aglomeracion}
				\4[] Sin costes de transporte y con costes de congestión
				\4[] $\to$ Posible dispersión de nuevo
				\4[] $\to$ Sin home-market effect
				\4[] $\to$ Costoso concentrarse
				\4[] $\to$ Sin costes de exportar
				\4[] $\then$ Dispersión máxima
			\3 Implicaciones
				\4 Núcleo y periferia
				\4[] Núcleo
				\4[] $\to$ Concentración de obreros
				\4[] $\to$ Concentración de variedades industriales
				\4[] $\to$ Salarios elevados
				\4[] $\to$ Exportación de producto manufacturado
				\4[] $\to$ Importación de productos agrícolas
				\4[] Periferia
				\4[] $\to$ Sin obreros
				\4[] $\to$ Sin variedades industriales
				\4[] $\to$ Salarios reducidos
				\4[] $\to$ Importación de producto manufacturado
				\4[] $\to$ Exportación de productos agrícolas
				\4 Integración comercial induce aglomeración
				\4[] Posible aumento desigualdades regionales
				\4[] Posibles tensiones de economía política
				\4 Crecimiento internacional produce crecimiento asimétrico
				\4[] Posible aglomeración en determinadas áreas
				\4[] Posible salida de población en otras
				\4[] Dinámicas dependen de parámetros concretos
			\3 Valoración
				\4 Premio Nobel a Krugman en 2008
				\4[] Culmina programa de comp. monop. y EEscala en CI
				\4 Abre programa de investigación
				\4[] Geografía económica basada en
				\4[] $\to$ Externalidades pecunarias
				\4 De manera paradójica, mundo se vuelve más clásico\footnote{Ver conclusión de Krugman (2008) Nobel Prize Lecture.}
				\4[] En últimas décadas
				\4[] Aumenta comercio basado en VComparativa
				\4[] $\to$ Cadenas de valor global
				\4[] $\to$ Especialización
				\4[] $\to$ IDE vertical frente a horizontal
		\2 Economía política del comercio
			\3 Idea clave
				\4 Contexto
				\4[] Economía política
				\4[] $\to$ Análisis de efectos de política económica
				\4[] $\then$ Sobre intereses de diferentes grupos sociales
				\4[] $\then$ Como resultado de intereses de diferentes grupos
				\4[] Efectos de política comercial
				\4[] $\to$ Afectan distinto a diferentes sectores
				\4[] Comercio positivo para el crecimiento
				\4[] $\to$ Entonces, factores de economía política deben gestionarse
				\4 Objetivo
				\4[] Caracterizar efectos sobre diferentes sectores
				\4[] Entender impacto de estructura política sobre pol. arancelaria
				\4 Resultados
				\4[] Efectos de aranceles sobre diferentes grupos sociales
				\4[] $\to$ Beneficios y perjuicios
				\4[] $\to$ Diferentes grados de concentración
				\4[] $\to$ Diferente capacidad de respuesta
				\4[] Economía política puede afectar crecimiento
				\4[] $\to$ En la medida en que comercio afecta crecimiento
			\3 Formulación
				\4 Stolper-Samuelson
				\4[] En contexto Heckscher-Ohlin
				\4[] Tras apertura comercial
				\4[] $\to$ Factor intensivo de sector de especialización
				\4[] $\then$ Aumenta pago al factor
				\4[] $\to$ Factor intensivo de sector que pierde producción
				\4[] $\then$ Coste de factores cae
				\4[] Sector de factor intensivo en bien de especialización
				\4[] $\then$ Presión hacia reducción de aranceles
				\4[] Sector de factor intensivo en bien que pierde producción
				\4[] $\then$ Presión hacia mantenimiento de aranceles
				\4[] Países ricos
				\4 Redistribución de beneficios del comercio
				\4[] Permite a perdedores aceptar reducción de aranceles
				\4[] Pero costes de redistribución
				\4[] $\to$ Negociación entre sectores
				\4[] $\to$ Votaciones
				\4[] $\to$ Adquisición de información
				\4[] $\then$ Posible no sea rentable redistribuir
				\4 Modelo de factores específicos
				\4[] Dos factores de capital inmóviles
				\4[] Desarme arancelario mutuo
				\4[] $\to$ Aumenta beneficios que pasan a exportar
				\4[] $\to$ Reduce beneficio en sectores que ahora importan
				\4[] $\then$ Flujo de trabajo de un sector a otro
				\4[] $\then$ Caída de PMgK en sector perjudicado
				\4[] Diferentes intereses dentro de un mismo factor
				\4[] $\to$ Capital vs trabajo no siempre oposicion homogénea
				\4 Aversión a la pérdida
				\4[] Behavioral economics
				\4[] Empíricamente, aversión a pérdida mayor que ganancia
				\4[] Apertura arancelaria
				\4[] $\to$ Induce beneficio en un sector
				\4[] $\to$ Aumenta pérdidas en otro
				\4[] Si aversión a pérdida mayor que ganancia por beneficio
				\4[] $\then$ Oposición más fuerte
				\4 Aversión a incertidumbre
				\4[] Apertura aumenta incertidumbre
				\4[] $\to$ ¿Efectos de equilibrio general serán positivos?
				\4 Aversión a desigualdad
				\4[] Apertura al comercio puede aumentar desigualdad
				\4[] $\to$ Sector de especialización más rico
				\4[] $\to$ Sector que reduce producción más pobre
				\4[] Seres humanos muestran cierta aversión a la desigualdad
				\4[] $\to$ Factor de oposición a apertura
				\4 Concentración de intereses
				\4[] Efectos de reducción arancelaria
				\4[] $\to$ Difusos sobre consumidores
				\4[] $\to$ Muy concentrados sobre industria desprotegida
				\4[] Perjuicio concentrado
				\4[] $\to$ Facilita coordinación entre perjudicados
				\4[] $\then$ Facilita oposición política a apertura
			\3 Implicaciones
				\4 Oposición a apertura más fuerte que presión apertura
				\4 Instituciones multilaterales pueden catalizar
				\4[] Commitment liberalizador
				\4[] Aumenta poder de negociación de liberalizadores
				\4 Redistribución puede ser necesaria
				\4[] Mejora aceptación de apertura
				\4[] También es costosa
			\3 Valoración
				\4 Programa de investigación con muchas vertientes
				\4 Interacciones con sociología, ciencia política, demografía..
				\4 Ciencia económica no siempre ha examinado
				\4[] Supuestos demasiado fuertes
				\4[] $\to$ ¿Planificador social?
				\4[] $\to$ ¿Funciones de bienestar social?
				\4[] $\then$ ¿Realmente existen?
				\4[] $\then$ ¿Realmente consideradas en decisiones de PComercial?
		\2 Evidencia empírica
			\3 Dirección de la causalidad
				\4 Frankel y Romer (1999)
				\4 Dirección de la causalidad difícil de distinguir
				\4[] Comercio causa crecimiento
				\4[] $\to$ Por alguna de las vías mencionadas anteriormente
				\4[] Crecimiento causa comercio
				\4[] $\to$ Porque comercio tiene = determinantes que crecimiento
				\4 Ejemplo:
				\4[] Países que liberalizan comercio interno
				\4[] $\to$ Liberalizan también comercio exterior
				\4[] Liberalización de comercio interior y exterior
				\4[] $\to$ Afecta crecimiento y comercio a la vez
				\4[] $\then$ aparece correlación comercio-crecimiento
				\4 Necesario estimar instrumento alternativo
				\4[] Determinante de comercio
				\4[] $\to$ Que no dependa de decisiones de PE
				\4[] $\then$ Relacionar instrumento con crecimiento
				\4 Modelos de gravedad
				\4[] Explicar comercio como resultado de:
				\4[] $\to$ Tamaño relativo
				\4[] $\to$ Distancia
				\4 Regresión
				\4[] Crecimiento contra instrumento de comercio
				\4[] $\to$ Estimado mediante modelo de gravedad
				\4 Resultados
				\4[] Comercio aumenta crecimiento
				\4[] $\to$ No por causas comunes de crecimiento y comercio
				\4[] Comercio interno aumenta crecimiento
				\4[] Resultados robustos a cambios en formulación
			\3 Efectos escala
				\4 Modelos de crecimiento endógeno
				\4[] A menudo predicen relación entre
				\4[] $\to$ Tamaño de la economía
				\4[] $\then$ Comercio ``integra'' economías
				\4[] $\to$ Tasa de crecimiento
				\4 Muy largo plazo
				\4[] Indicios favorables
				\4[] Kremer (1993)
				\4[] $\to$ Un millón de años hasta hoy
				\4 Corto plazo
				\4[] Pocos indicios favorables
			\3 Spillovers tecnológicos
				\4 Instrumento de estimación
				\4[] Estimar medidas de gasto en I+D
				\4[] $\to$ De importadores y exportadores
				\4[] Ponderar medidas de gasto en I+D
				\4[] $\to$ Por volumen de importaciones y exportaciones
				\4 Objetivo
				\4[] Relacionar crecimiento de TFP con I+D ponderando:
				\4[] $\to$ Volumen de comercio sobre total
				\4[] $\to$ Cercanía geográfica
				\4 Ponderando por volumen de comercio
				\4[] Relación $\Delta$ TFP con i+D de importación
				\4[] $\to$ Relación pequeña o poco significativa
				\4[] Relación $\Delta$ TFP con i+D de exportación
				\4[] $\to$ Relación significativa
				\4[] $\then$ Exportadores aprenden de sus clientes
				\4[] $\then$ Clientes no aprenden mucho de sus proveedores
				\4 Ponderando por distancia geográfica
				\4[] Relación significativa
				\4[] $\to$ Debilita conclusión respecto volumen de comercio
				\4[] $\then$ ¿Comercian más porque están más cerca?
				\4 Dirección de los spillovers
				\4[] ¿Son simétricos entre PEDs y desarrollados?
				\4[] Evidencia apunta a asimetría
				\4[] $\to$ De países más avanzados hacia menos
				\4[] IDE también juega papel importante
				\4 Transferencia vía exportaciones e importaciones
				\4[] Proveedores aprenden de clientes
				\4[] $\to$ Demandas de clientes transfieren tecnología
				\4[] $\then$ Evidencia favorable
				\4[] Importaciones de productos con tec. más avanzada
				\4[] $\to$ Poca evidencia de que aumenten crecimiento
			\3 Convergencia
				\4 Apertura al comercio influye en convergencia
				\4[] Evidencia favorable
				\4[] Países que comercian entre sí
				\4[] $\to$ Más velocidad de convergencia entre sí
				\4 Países cerrados al comercio
				\4[] Evidencia contraria a convergencia
	\1[] \marcar{Conclusión}
		\2 Recapitulación
			\3 Efectos del crecimiento sobre el comercio int.
				\4 Aumento de dotaciones factoriales
				\4 Progreso tecnológico
				\4 Renta y dependencia exterior
				\4 Trampas de pobreza
			\3 Efectos del comercio int. sobre el crecimiento
				\4 ISI vs EOI
				\4 Crecimiento exógeno
				\4 Crecimiento endógeno
				\4 NEG
				\4 Economía política
				\4 Instituciones
				\4 Evidencia empírica
		\2 Idea final
			\3 Globalización
				\4 Tendencia generalizada a nivel mundial
				\4 Concepto
				\4[] Definición compleja
				\4[] $\to$ Incremento de transacciones internacionales
				\4[] $\to$ Integración de mercados mundiales
				\4[] $\to$ Homogeneización cultural y de preferencias
				\4 Comercio es pilar fundamental de globalización
				\4[] Fuerte impacto sobre comercio
				\4[] $\to$ Policy-making puede alterar efectos y distribución
			\3 Desarrollo económico
				\4 Íntimamente conectado con CI y crecimiento
				\4 En general, CI es motor de desarrollo
				\4[] Pero existen dinámicas complejas
				\4[] $\to$ RRI
				\4[] $\to$ Economía política
				\4[] $\then$ Comercio puede perjudicar desarrollo
			\3 Liberalización comercial y financiera
				\4 Ambas han permitido aumento de CI en últimas décadas
				\4 Catalizadores de crecimiento desde años 80
				\4 Problemas:
				\4[] ¿Cuál es la secuencia óptima de liberalización?
				\4[] ¿Qué interacción con democracia?
			\3 Inversión extranjera directa
				\4 Complemento y sustitutivo de CI
				\4[] Complemento
				\4[] $\to$ Cadenas de valor y externalización
				\4[] Sustitutivo
				\4[] $\to$ Evitar protección de mercado nacional
				\4 Papel en crecimiento
				\4[] Transferencia tecnológica
				\4[] Aumento de variedades
				\4[] Especialización e industrialización
\end{esquemal}

\graficas

\begin{axis}{4}{Posibles efectos sobre el consumo de bien importable tras un aumento de la renta.}{Exportable}{Importable}{efectosconsumo}
	% Renta inicial
	\draw[-] (0,2.5) -- (2.5,0);
	\node[left] at (0,2.5){{\scriptsize H}};
	\node[below] at (2.5,0){{\scriptsize G}};
	
	% Renta final
	\draw[-] (0,3.5) -- (3.5,0);
	\node[left] at (0,3.5){{\scriptsize H'}};
	\node[below] at (3.5,0){{\scriptsize G'}};
	
	% Consumo inicial
	\node[circle, fill=black, inner sep=0pt, minimum size=3pt] (a) at (1.25,1.25) {};
	
	% Expansión de la renta
	\draw[-] (0,0) -- (1.75,1.75);
	
	% Líneas de separación de áreas
	\draw[dashed] (2.25,1.25) -- (1.25,1.25) -- (1.25,2.25);
	
	% Área UP
	\draw[decoration={brace,raise=3pt},decorate] (0,3.5) -- node[above=9pt, right=3pt] {{\scriptsize $\text{UP}$}} (1.25,2.25);
	
	% Área P
	\draw[decoration={brace,raise=3pt},decorate] (1.25,2.25) -- node[above=9pt, right=3pt] {{\scriptsize $\text{P}$}} (1.73,1.77);
	
	% Neutral
	\node[below] at (1.75,1.75){\scriptsize C};
	\node[circle, fill=black, inner sep=0pt, minimum size=3pt] (a) at (1.75,1.75) {};
	
	% Área A
	\draw[decoration={brace,raise=3pt},decorate] (1.77,1.73) -- node[above=9pt, right=3pt] {{\scriptsize $\text{A}$}} (2.25,1.25);

	% Área UA
	\draw[decoration={brace,raise=3pt},decorate] (2.25,1.25) -- node[above=9pt, right=3pt] {{\scriptsize $\text{UA}$}} (3.5,0);
	
\end{axis}

\begin{axis}{4}{Posibles efectos sobre la producción de bien importable tras un aumento de la renta.}{Importable}{Exportable}{efectosproduccion}
	% Renta inicial
	\draw[-] (0,2.5) -- (2.5,0);
	\node[left] at (0,2.5){{\scriptsize H}};
	\node[below] at (2.5,0){{\scriptsize G}};
	
	% Renta final
	\draw[-] (0,3.5) -- (3.5,0);
	\node[left] at (0,3.5){{\scriptsize H'}};
	\node[below] at (3.5,0){{\scriptsize G'}};
	
	% Consumo inicial
	\node[circle, fill=black, inner sep=0pt, minimum size=3pt] (a) at (1.25,1.25) {};
	
	% Expansión de la renta
	\draw[-] (0,0) -- (1.75,1.75);
	
	% Líneas de separación de áreas
	\draw[dashed] (2.25,1.25) -- (1.25,1.25) -- (1.25,2.25);
	
	% Área UP
	\draw[decoration={brace,raise=3pt},decorate] (0,3.5) -- node[above=9pt, right=3pt] {{\scriptsize $\text{UA}$}} (1.25,2.25);
	
	% Área P
	\draw[decoration={brace,raise=3pt},decorate] (1.25,2.25) -- node[above=9pt, right=3pt] {{\scriptsize $\text{A}$}} (1.73,1.77);
	
	% Neutral
	\node[below] at (1.75,1.75){\scriptsize C};
	\node[circle, fill=black, inner sep=0pt, minimum size=3pt] (a) at (1.75,1.75) {};
	
	% Área A
	\draw[decoration={brace,raise=3pt},decorate] (1.77,1.73) -- node[above=9pt, right=3pt] {{\scriptsize $\text{P}$}} (2.25,1.25);
	
	% Área UA
	\draw[decoration={brace,raise=3pt},decorate] (2.25,1.25) -- node[above=9pt, right=3pt] {{\scriptsize $\text{UP}$}} (3.5,0);
	
\end{axis}


\begin{axis}{4}{Efectos del crecimiento sobre la RRI}{X}{Y}{empobrecedor}
	% Oferta recíproca de extranjero -- Importador de X, exportador de Y
	\draw[thick] (0,0) to [out=80,in=185](4,2.5);
	\node[above] at (4,2.5){B};
	
	% Oferta recíproca de doméstico -- Exportador de X, importador de Y
	\draw[thick] (0,0) to [out=5,in=260](2.4,4);
	\node[right] at (2.4,4){A};
	
	% UAC
	\draw[dashed] (0,0) to [out=5,in=260](2,4);
	
	% AC
	\draw[dashed] (0,0) to [out=5,in=260](3,4);

	% PC
	\draw[dashed] (0,0) to [out=5,in=260](3.5,4);
	
	% UPC
	\draw[dashed] (0,0) to [out=5,in=260](4,4);
	
	% RRI inicial
	\draw[-] (0,0) -- (3.8,4);
	
	% RRI final
	\draw[-] (0,0) -- (4,2.81);
	
\end{axis}

La economía A experimenta un crecimiento de la renta con diferentes efectos respecto a la proporción de la renta dedicada a la importación. Así, si el crecimiento tiene efecto ultra-anti-comercial, la curva de oferta recíproca se desplaza a la izquierda y la RRI experimenta una variación favorable para el país. En el resto de los casos, las curvas de oferta recíproca se desplazan a la derecha (porque aumenta la demanda de importaciones) y la relación relativa de intercambio empeora --es decir, aumenta el precio de las importaciones en términos de unidades exportadas.


\begin{axis}{4}{Teorema de Rybczynski: aumento de la dotación de un factor de producción y subsecuente aumento de la producción del bien cuya producción es intensiva en ese factor y disminución de la producción del otro bien.}{}{}{lernerrybczynski}
	% extender eje de abscisas
	\draw[-] (4,0) -- (6,0);
	\node[below] at (5,0){L};
	
	% extender eje de ordenadas
	\draw[-] (0,4) -- (0,6);
	\node[left] at (0,6){K};
	
	% Y: AUTARQUÍA y LIBRE COMERCIO - isocuanta de tecnología intensiva en K
	\draw[-] (1,5) to [out=271, in=179](3,3);
	\node[above] at (1,5){\tiny $Y=1$};
	
	% X: AUTARQUÍA - isocuanta de tecnología intensiva en L de autarquía
	\draw[-] (3,3.5) to [out=271, in=179](5,1.5);
	\node[right] at (5,1.5){\tiny $X=1/P_x$};
	
	% isocoste de autarquía
	\draw[-] (0,4.75) -- (6.33,0);
	\node[right] at (6.05,0.3){\tiny $-\frac{w}{r}$};
	% punto de tangencia con bien Y
	\node[circle, fill=black, inner sep=0pt, minimum size=3pt] (a) at (1.8,3.4) {};
	% punto de tangencia con bien X
	\node[circle, fill=black, inner sep=0pt, minimum size=3pt] (a) at (3.8,1.9) {};
	
	% cono de diversificación de autarquía
	% frontera con bien X
	\draw[dashed] (0,0) -- (8,4);
	% con bien y
	\draw[dashed] (0,0) -- (3.2,6);
	
	% DOTACIÓN INICIAL de capital y trabajo
	\node[circle, fill=black, inner sep=0pt, minimum size=3pt] (a) at (5.5,4.4) {};
	\node[left] at (5.5,4.6){$(\bar{L}, \bar{K})$};
	% uso de dotación desde situación de autarquía
	\draw[-{Latex},dashed] (4.3,2.15) -- (5.5,4.4););
	
	
	% DOTACIÓN POST-AUMENTO de trabajo manteniendo capital constante
	\node[circle, fill=black, inner sep=0pt, minimum size=3pt] (a) at (7.5,4.4) {};
	\node[right] at (7.5,4.6){$(\bar{L'}, \bar{K'})$};
	% transición de una dotación a otra (para mostrar que es línea recta horizontal y sólo aumenta el trabajo)
	\draw[-{Latex}] (5.5,4.4) -- (7.5,4.4);
	\draw[-{Latex},dashed] (7.03,3.51) -- (7.5,4.4);
	
\end{axis}


\begin{axis}{4}{Modelo de Krugman (1991) del núcleo y la periferia. Diagrama de fase de la dinámica con costes de transporte elevados que resultan en un equilibrio con la población de obreros dispersada.}{}{$\dot{\lambda}$}{krugman91dispersion}
	% Extensión del eje de ordenadas hacia abajo y de abscisas hacia la derecha
	\draw[-] (0,0) -- (0,-4);
	\draw[-] (4,0) -- (8,0);
	\node[below] at (8.4,0){$\lambda$};	
	% Límite con lambda = 1
	\node[below] at (8,-0.2){$1$};
	\draw[-] (8,0.2) -- (8,-0.2);
	% Centro con lambda = 0.5
	\node[below] at (4,-0.3){0,5};	
	\draw[-] (4,0.2) -- (4,-0.2);

	% Dinámica de dot{lambda}
	\draw[-] (0,3) to [out=10, in=100](4,0) to [out=-80, in=200](8,-3);

	% Flechas de tendencia
	% Hacia la derecha desde origen
	\draw[-{Latex}] (0.5,0.5) -- (3.5,0.5);

	% Hacia la izquierda desde límite derecho
	\draw[-{Latex}] (7.5,-0.5) -- (4.5,-0.5);

\end{axis}

La variable $\lambda$ representa la concentración de la población de trabajadores obreros móviles en un uno de los países en cuestión.


\begin{axis}{4}{Modelo de Krugman (1991) del núcleo y la periferia. Diagrama de fase de la dinámica con costes de transporte intermedios que resultan en múltiples equilibrios posibles.}{}{$\dot{\lambda}$}{krugman91multiplesequilibrios}
	% Extensión del eje de ordenadas hacia abajo y de abscisas hacia la derecha
	\draw[-] (0,0) -- (0,-4);
	\draw[-] (4,0) -- (8,0);
	\node[right] at (8.4,0){$\lambda$};	
	% Límite con lambda = 1
	\node[below] at (8,-0.2){$1$};
	\draw[-] (8,0.2) -- (8,-0.2);
	% Centro con lambda = 0.5
	\node[below] at (4,-0.3){0,5};	
	\draw[-] (4,0.2) -- (4,-0.2);

	% Dinámica de dot{lambda}
	\draw[-] (0,-4) to [out=80,in=180](3,1) to [out=0, in=180](5,-1) to [out=0, in=260](8,4);

	% Flechas de tendencia
	% Hacia límite izquierdo, aglomeración
	\draw[-{Latex}] (1.4,0.5) -- (0.2,0.5);
	% Hacia centro desde izquierda, dispersión
	\draw[-{Latex}] (1.8,0.5) -- (3.6,0.5);
	% Hacia centro desde derecha, dispersión
	\draw[-{Latex}] (6.8,0.5) -- (4.4,0.5);
	% Hacia derecha, aglomeración	
	\draw[-{Latex}] (7.2,0.5) -- (8,0.5);

\end{axis}


\begin{axis}{4}{Modelo de Krugman (1991) del núcleo y la periferia. Diagrama de fase de la dinámica con costes de transporte reducidos pero presentes: el equilibrio tiende a la aglomeración.}{}{$\dot{\lambda}$}{krugman91aglomeracion}
	% Extensión del eje de ordenadas hacia abajo y de abscisas hacia la derecha
	\draw[-] (0,0) -- (0,-4);
	\draw[-] (4,0) -- (8,0);
	\node[right] at (8.4,0){$\lambda$};	
	% Límite con lambda = 1
	\node[below] at (8,-0.2){$1$};
	\draw[-] (8,0.2) -- (8,-0.2);
	% Centro con lambda = 0.5
	\node[below] at (4,-0.3){0,5};	
	\draw[-] (4,0.2) -- (4,-0.2);

	
	% Dinámica de dot{lambda}
	\draw[-] (0,-4) to [out=80, in=260](8,4) ;

	% Flechas de tendencia
	\draw[-{Latex}] (3.7,0.5) -- (0.3,0.5);
	\draw[-{Latex}] (4.3, -0.5) -- (7.7,-0.5);

\end{axis}


\preguntas

\seccion{Test 2018}

\textbf{27.} ¿Cuál de las siguientes respuestas es \textbf{\underline{CORRECTA}}?

\begin{itemize}
	\item[a] El aumento de la dotación del factor abundante en un país pequeño, ceteris paribus, podría dar lugar a una situación de crecimiento empobrecedor.
	\item[b] El crecimiento empobrecedor puede afectar únicamente a economías grandes.
	\item[c] El crecimiento económico siempre tiene un efecto positivo sobre el bienestar, y por tanto nunca puede ser empobrecedor.
	\item[d] Todas las opciones anteriores son falsas.
\end{itemize}


\seccion{Test 2014}

\textbf{31.} El modelo de Grossman y Helpman (1991) de crecimiento endógeno analiza el efecto del comercio sobre el crecimiento concluyendo que:

\begin{itemize}
	\item[a] El comercio internacional perjudica el crecimiento de los países.
	\item[b] No se cumpliría el Teorema de Stolper-Samuelson.
	\item[c] No existiría un arancel óptimo.
	\item[d] La protección arancelaria de los sectores intensivos en I+D supone una mejora del crecimiento a través de un mayor incentivo a la inversión en I+D.
\end{itemize}


\seccion{Test 2013}

\textbf{34.} Señale cuál de las afirmaciones siguientes es falsa:

\begin{itemize}
	\item[a] El progreso técnico neutral en la industria que produce el bien de exportación de un país grande (un país con capacidad para influir en los precios internacionales) puede producir un fenómeno de empobrecimiento.
	\item[b] Cuando crece un factor de producción en el que un país es abundante, sus términos de comercio pueden tender a deteriorarse, para el caso de un país grande.
	\item[c] El progreso técnico neutral se caracteriza porque las productividades marginales de los dos factores de producción no varían.
	\item[d] El progreso técnico neutral experimentado por un bien se representa con el mismo mapa de isocuantas que tenía antes de experimentar el progreso técnico, pero desplazado hacia el origen.
\end{itemize}

\notas

\textbf{2018:} \textbf{27.} ANULADA

\textbf{2014:} \textbf{31.} D

\textbf{2013:} \textbf{34.} C

\bibliografia


Mirar en Palgrave:
\begin{itemize}
	\item balanced growth
	\item economic growth, empirical regularities in 
	\item economic growth in the very long run
	\item economic growth non-linearities
	\item endogenous growth theory *
	\item globalization *
	\item globalization and labour
	\item growth accounting
	\item growth and cycles
	\item growth and institutions
	\item \textbf{growth and international trade} *
	\item growth and learning-by-doing *
	\item Heckscher-Ohlin trade theory *
	\item import substitution and export-led growth
	\item inequality between nations
	\item infant-industry protection
	\item neoclassical growth theory
	\item neoclassical growth theory (new perspectives)
	\item political economy, legacy of institutions from
	\item political economy of institutional change
	\item political economy, economic approaches to
	\item schumpeterian growth and growth policy design
	\item skill-biased technical change
	\item size of nations, economics approach to the *
	\item total factor productivity 
	\item trade cycle *
	\item uneven development
\end{itemize}

Alesina, A.; Spolaore, E.; Wacziarg, R. \textit{Handbook of Economic Growth 1B. Ch. 23 Trade, Growth and the Size of Countries} 1B. -- En carpeta crecimiento económico

Baldwin, R. \textit{Measurable Dynamic Gains from Trade} (1989) NBER Working Paper Series -- En carpeta del tema

Deardoff, A. V. \textit{Introduction to the Lerner Diagram} (2002) \url{http://www-personal.umich.edu/~alandear/writings/Lerner.pdf} -- En carpeta del tema

Deardoff, A. V. \textit{Lerner Diagram. Glossary of International Economics} (2002) \url{http://www-personal.umich.edu/~alandear/glossary/figs/Lerner/ld.html#} -- Muy buena explicación interactiva del diagrama y su estática comparativa.

Feenstra, R. C. \textit{Advanced International Trade: Theory and Evidence} (2004) -- En carpeta economía internacional

Frankel, J.; Romer, D. \textit{Does Trade Cause Growth?} (1999) American Economic Review -- En carpeta del tema

Gandolfo, G. \textit{International Trade and Policy} (2014) 2ed -- En carpeta \textit{Economía Internacional}

Grossman, G.; Helpman, E. \textit{Quality Ladders and Product Cycles} (1991) Quarterly Journal of Economics -- En carpeta del tema

Grossman, G.; Helpman, E. \textit{Trade, knowledge spillovers and growth} (1991) European Economic Review -- En carpeta del tema

Grossman, G.; Helpman, E. \textit{Quality Ladders in the Theory of Growth} (1991) The Review of Economic Studies -- En carpeta del tema

Krugman, P. R. (1991) \textit{Increasing Returns and Economic Geography} Journal of Political Economy, Vol. 99, No. 3 -- En carpeta del tema

North, D. C. \textit{Institutions} (1991) Journal of Economic Perspectives -- En carpeta del tema

\end{document}
