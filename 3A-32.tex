\documentclass{nuevotema}

\tema{3A-32}
\titulo{Teorías de la demanda de consumo. Implicaciones de política económica.}

\begin{document}

\ideaclave

En el contexto macroeconómico, se denomina como consumo final o simplemente consumo al componente del gasto que no tiene como destino la producción de otros bienes o servicios. Paralelamente, el consumo representa en el plano microeconómico la variable que determina el nivel de utilidad de los agentes. Por otra parte, la diferencia entre el producto y la parte de éste que se destina al consumo final constituye el ahorro. A su vez, el ahorro determina el volumen de inversión, lo que convierte al consumo en determinante indirecto de la producción futura y le confiere un papel fundamental en el análisis dinámico de una economía. Además, el consumo representa un porcentaje muy elevado del producto interior bruto (el consumo privado representó casi el 60\% del PIB en 2017 y el consumo total incluyendo a las AAPP ascendió al 80\%). Por todo ello, el análisis de la demanda de consumo y la modelización del efecto que tienen otras variables sobre éste es fundamental para predecir el comportamiento futuro de una economía.

Este tema trata de dar respuesta a una \textbf{pregunta fundamental}: ¿de qué depende la demanda de consumo? De ella se derivan otras preguntas más específicas tales como: ¿qué modelos son relevantes para contestar a la primera pregunta?, ¿cómo evoluciona la demanda de consumo en el tiempo?, ¿cómo afecta cada factor?, ¿cómo ha evolucionado la modelización de la demanda de consumo?, ¿qué aportan los modelos más recientes?

El análisis empírico de la demanda de consumo como agregado macroeconómico comienza con la aparición de la \textbf{contabilidad nacional}. Simon Kuznets -uno de sus pioneros y Premio Nobel-- encuentra que los datos empíricos arrojan una \textbf{paradoja} respecto a la función de consumo que \textbf{Keynes} había propuesto. Así, los datos empíricos corroboraban la función de demanda de consumo de Keynes según la cual la propensión marginal al consumo debía ser inferior a la propensión media\footnote{La función de consumo tomaba así la forma: $C_t (C_0, Y_t) = C_0 + c Y_t$, siendo $0<c<1$.} entre diferentes individuos en un momento determinado y en el corto plazo. En el largo plazo, los datos arrojaban una conclusión distinta: las regresiones estimaban una proporción de consumo respecto a la renta constante (la función de consumo de Keynes predecía una proporción decreciente a medida que aumentaba la renta). Para tratar de fundamentar esta paradoja y más generalmente, los datos de contabilidad nacional, aparecen una serie de teorías que especifican diferentes funciones de consumo que van más allá de la función keynesiana de consumo.

Duesenberry planteó el modelo de demanda consumo del \textbf{ingreso relativo}. Aunque las predicciones de su modelo eran consistentes con los datos empíricos, resulta muy difícil de formular en términos de teoría microeconómica estándar. Este hecho, junto al desarrollo de explicaciones más tratables y más fácilmente contrastables, acabo por relegar el modelo a la historia del pensamiento económico. Su idea clave es que los agentes deciden cuanto consumo demandar teniendo en cuenta cuanto consumen sus semejantes y cuanto consumían ellos mismos en el pasado. Cuando otros agentes similares consumen un nivel dado, aunque su renta sea inferior un agente dado tratará de aproximarse al nivel de los demás. De forma similar, cuando la renta de un agente cae, el agente comparará con su consumo anterior y por ello no disminuirá tanto su consumo como lo hizo su renta. Sin embargo, cuando la renta crece los agentes no consideran el consumo anterior más bajo y aumentan su consumo atendiendo sólo a ese crecimiento de la renta. Dado que a largo plazo la renta crece generalmente, la paradoja de Kuznets estaría justificada por esta función de consumo.

Franco Modigliani, Ando y Brumberg, y por otro lado Milton Friedman plantearon respectivamente las hipótesis del ciclo vital y de la \textbf{renta permanente}. Se trata de dos modelos fundamentalmente idénticos, aunque sus formulaciones específicas inciden sobre diferentes aspectos. El modelo del \textbf{ciclo vital} parte del hecho de que los agentes tratan de suavizar su patrón de consumo a lo largo de su vida, de forma que se endeudan cuando son jóvenes, ahorran durante la vida adulta y desinvierten cuando alcanzan la jubilación. El modelo de la \textbf{renta permanente de Friedman} incide en el proceso de optimización intertemporal por parte de agentes con un horizonte final infinito. Como resultado de éste, los agentes tratan de consumir de acuerdo con una senda de consumo óptimo limitada por el ingreso total que obtienen y que no es sino la suma de los ingresos de cada periodo, trasladables interemporalmente. Dado que el ingreso es perfectamente transladable, el consumo de cada periodo puede determinarse en relación a un hipótetico ingreso constante por periodo que no es sino una fracción del ingreso total.

Además de explicar la paradoja de Kuznets, la hipótesis de la renta permanente pone de manifiesto los diferentes efectos de \textbf{cambios temporales del ingreso y cambios permanentes}. A la hora de decidir cuanto consumir, el agente no tiene en cuenta el ingreso en un periodo sino en el total de periodos, por lo que un aumento temporal del ingreso será \comillas{repartido} entre todos los periodos y por tanto el consumo en cada periodo apenas aumentará. Un aumento permanente, sin embargo, tendrá un efecto sobre todos los ingresos permanentes equivalentes, y por ello, sí aumentará el consumo de forma efectiva. Se trata de una predicción que concuerda con numerosos experimentos naturales de política económica.

La revolución de la \textbf{Hipótesis de las Expectativas Racionales} le dio una vuelta de tuerca adicional al modelo la renta permanente. El modelo de \textbf{Hall} introduce este concepto en la modelización de la demanda de consumo. Si los agentes estiman las rentas futuras sin cometer errores sistemáticos y deciden sobre la senda de consumo en relación a esas estimaciones, la senda de consumo será en la práctica un \textbf{paseo aleatorio}. Esto es así porque cada elemento de información relevante para estimar el ingreso futuro será tomado en consideración de forma inmediata y sin cometer errores sistemáticos, por lo que toda variación en el consumo habrá de ser fruto de nueva información, que por definición es imposible de prever a priori (si fuese posible preverla, los agentes ya lo hubiesen hecho y además lo hubiesen tenido en consideración para decidir el consumo).

Dado que los agentes trasladan ingreso de unos periodos a otros, la tasa a la que pueden realizar ese translado, el \textbf{tipo de interés} es muy relevante en el análisis de la demanda de consumo. Por ello, es necesario detenerse y explicar sus efectos en función de su signo, magnitud y contexto. El contexto o más bien, la situación crediticia del agente tienen una especial relevancia. Un agente que ha ahorrado dinero se verá beneficiado por una subida del tipo de interés, ya que pasará a recibir una mayor renta en el futuro. Una mayor renta futura tiene efectos en el presente si asumimos la hipótesis de la renta permanente: el agente estima un mayor ingreso total y por ello, tiende a aumentar su consumo presente. Hablamos en esta situación de efecto renta del aumento del tipo de interés. Por otro lado, el aumento del tipo de interés aumenta el precio del consumo presente en relación al futuro, y actúa en sentido contrario al efecto renta anterior. Estos dos efectos actúan sin embargo en el mismo sentido cuando el agente está endeudado, es decir, se encuentra en un contexto de ahorro negativo. En esta situación, el aumento del tipo de interés aumenta la cantidad que deberá pagar en el futuro para devolver lo que tomó prestado, y por ende reduce su ingreso total y su consumo presente. 

Teniendo en cuenta el efecto de las variaciones en el tipo de interés y la incertidumbre respecto al ingreso futuro, se plantea la necesidad de analizar la relación entre el retorno de un activo financiero y la decisión de consumo. Dado que el retorno de un activo financiero no es sino un tipo de interés, los agentes tendrán en cuenta éste a la hora de utilizar uno u otro método de traslado intertemporal de renta (es decir, uno u otro activo arriesgado). A este respecto aparece el modelo \textbf{CCAPM}, o CAPM del consumo, que relaciona el rendimiento exigido a un activo determinado dado con la correlación entre el rendimiento y la inversa de la utilidad marginal del consumo. Si la correlación es alta, los agentes exigirán un rendimiento más elevado dado que cuando la inversa de la utilidad marginal es elevada, mayores consumos adicionales les proporcionan menos utilidad que cuando sucede al contrario. Un ejemplo: supongamos un trabajador cuyo sueldo depende la marcha de la empresa (por ejemplo, vía comisiones, o vía la posibilidad de ser despedido). Este trabajador debería invertir (si se comportase racionalmente) en acciones de empresas cuya marcha esté incorrelada o negativamente correlacionada con la de la empresa en la que trabaja. De esta forma podría reducir la varianza del ingreso o equivalentemente, aumentar el grado de certidumbre con el que podrá llevar a cabo un consumo determinado: cuando su empresa vaya mal y su sueldo se reduzca, recibirá un mayor retorno de los activos arriesgados en los que ha invertido y que están negativamente correlacionados.

Por último, el análisis de la demanda de consumo va más allá de la hipótesis de la renta permanente. Fenómenos como las restricciones de liquidez, la no separabilidad del consumo o la ausencia de optimización deben también ser tenidos en cuenta a la hora de entender qué determina el nivel de consumo demandado por una economía. En cierta medida, el examen de estas cuestiones surge como resultado de otra anomalía empírica: el elevado grado de relación entre el consumo y la renta en un periodo determinado, de forma contraria a lo que implica la hipótesis de la renta permanente.

Dada la importancia del consumo como determinante del PIB presente y futuro y del bienestar, su magnitud y sus variaciones son objetivos de las \textbf{políticas económicas}. No hay que olvidar a lo largo de la exposición que el análisis de la demanda de consumo debe resultar en último término en la extracción de una serie de conclusiones respecto a la política económica óptima.

\seccion{Preguntas clave}
\begin{itemize}
    \item ¿Qué es la demanda de consumo?
    \item ¿De qué depende?
    \item ¿Cómo evoluciona en el tiempo?
    \item ¿Por qué depende de determinados factores?
    \item ¿Qué modelos explican la demanda de consumo?
\end{itemize}

\esquemacorto

\begin{esquema}[enumerate]
	\1[] \marcar{Introducción} 2'-2'
		\2 Contextualización
			\3 Economía
			\3 Consumo en micro y macroeconomía
			\3 Demanda de consumo
			\3 Macroeconomía
			\3 Importancia en policy-making
		\2 Objeto
			\3 De qué depende la demanda agregada de consumo
			\3 Cómo se distribuye a lo largo del tiempo
			\3 Qué modelos tratan de explicarla y predecirla
			\3 Qué efectos derivados del tipo de interés
			\3 Cómo determinan los agentes económicos qué demandar
		\2 Estructura
			\3 Modelos estáticos
			\3 Modelos dinámicos
			\3 Impacto del tipo de interés
			\3 Otros desarrollos
	\1 \marcar{Predecesores}
		\2 Malthus
			\3 Idea clave
			\3 Implicaciones
		\2 Veblen (1899)
			\3 Idea clave
			\3 Implicaciones
		\2 Irving Fisher
			\3 Idea clave
			\3 Implicaciones
		\2 Pigou
			\3 Idea clave
			\3 Implicaciones
	\1 \marcar{Teorías keynesianas del consumo}
		\2 Demanda de consumo keynesiana
			\3 Idea clave
			\3 Formulación
			\3 Implicaciones
			\3 Valoración
		\2 Paradoja de Kuznets
			\3 Idea clave
			\3 Formulación
			\3 Implicaciones
		\2 Teoría de la renta relativa de Duesenberry
			\3 Idea clave
			\3 Formulación
			\3 Implicaciones
			\3 Valoración
	\1 \marcar{Modelos dinámicos microfundamentados}
		\2 Idea clave
			\3 Contexto
			\3 Objetivos
			\3 Resultados
		\2 Teoría del ciclo vital de Modigliani y Ando
			\3 Idea clave
			\3 Formulación
			\3 Implicaciones
			\3 Valoración
		\2 Teoría de la renta permanente de Friedman
			\3 Idea clave
			\3 Formulación
			\3 Implicaciones
			\3 Valoración
		\2 Modelo de Hall (1978)
			\3 Idea clave
			\3 Formulación
			\3 Implicaciones
			\3 Valoración
	\1 \marcar{Impacto del tipo de interés}
		\2 Elasticidad intertemporal de sustitución
			\3 CRRA
			\3 Euler
			\3 Parámetro $\theta$
			\3 Elasticidad intertemporal de sustitución
		\2 Implicaciones
			\3 Efectos renta (aumento de r)
			\3 Efecto sustitución (aumento de r)
			\3 $\varDelta r$ sin ahorro ni deuda
			\3 $\varDelta r$ con deuda
			\3 $\varDelta r$ con ahorro
		\2 Empíricamente
			\3 Difícil contrastación
			\3 Horizontes largos
	\1 \marcar{Otros desarrollos} 5'-28'
		\2 Activos arriesgados: CCAPM
			\3 Idea clave
			\3 Formulación
			\3 Implicaciones
			\3 Valoración
		\2 Anomalía: correlación entre crecimiento de ingreso y consumo
			\3 HRP predice
			\3 Estudios muestran
		\2 Restricciones de liquidez
			\3 Práctica
			\3 Límite de crédito es vinculante
			\3 Límite de crédito podría ser vinculante
		\2 Optimización incompleta
			\3 Inconsistencia temporal
			\3 Reglas de oro
		\2 Ocio-consumo
			\3 Complementariedad bienes-ocio
			\3 Impuestos
		\2 Equivalencia ricardiana
	\1[] \marcar{Conclusión} 2'-30'
		\2 Recapitulación
			\3 Modelos estáticos
			\3 Modelos dinámicos
			\3 Impacto del tipo de interés
			\3 Otros desarrollos
		\2 Idea final
			\3 Renta permanente
			\3 Anomalías
			\3 Tributación

\end{esquema}

\esquemalargo

\begin{esquemal}
	\1[] \marcar{Introducción} 2'-2'
		\2 Contextualización
			\3 Economía
				\4 Definición de Robbins
				\4 Consumo como resultado de decisión
				\4[] Sobre destino de recursos finitos
			\3 Consumo en micro y macroeconomía
				\4 Determinante directo de utilidad
				\4[] Determinante de bienestar
				\4 Importancia sobre demanda agregada
				\4[] Componente de PIB: elevado peso
				\4[] Determinante indirecta de inversión
				\4[] Vía ahorro
			\3 Demanda de consumo
				\4 No economistas
				\4[] Pregunta absurda
				\4[] Demandan infinito
				\4 Análisis económico
				\4[] Como distribuir renta
				\4[] $\to$ Consumo presente
				\4[] $\to$ Inversión y consumo futuro
			\3 Macroeconomía
				\4 Componente de la demanda agregada
				\4[] Más estable
				\4[] Principal componente
				\4 Determinante de ahorro nacional
				\4[] Otra cara de la moneda
				\4 Sujeto a variación por diferentes factores
				\4[] Diferentes teorías plantean diferentes impactos
			\3 Importancia en policy-making
				\4 Efecto de políticas macroeconómicas
				\4[] $\to$ Políticas fiscales
				\4[] $\then$ Impuestos al consumo
				\4[] $\then$ Cuotas seguridad social
				\4[] $\then$ Impuestos sobre la renta
				\4[] $\then$ Inversión pública
				\4[] $\to$ Política monetaria
				\4[] $\then$ Tipos de interés nominal
				\4[] $\then$ Acceso a crédito
				\4 Provisión de bienes públicos
				\4[] Función aseguradora del estado
				\4[] $\to$ Enfermedad
				\4[] $\to$ Catástrofes
				\4[] $\to$ ...
				\4 Regulación
				\4[] Sistemas de pensiones
		\2 Objeto
			\3 De qué depende la demanda agregada de consumo
			\3 Cómo se distribuye a lo largo del tiempo
			\3 Qué modelos tratan de explicarla y predecirla
			\3 Qué efectos derivados del tipo de interés
			\3 Cómo determinan los agentes económicos qué demandar
		\2 Estructura
			\3 Modelos estáticos
			\3 Modelos dinámicos
			\3 Impacto del tipo de interés
			\3 Otros desarrollos
	\1 \marcar{Predecesores}
		\2 Malthus
			\3 Idea clave
				\4 Excesos agregados de oferta son posibles
				\4 Clases pasivas permiten cubrir
				\4[] Consumen bienes de lujo y similares
				\4[] No producen bienes y servicios
				\4[] Empujan demanda agregada
			\3 Implicaciones
				\4 Consumo insuficiente es posible
				\4[] Inferior a producción de economía
				\4 Consumo puede estabilizar economía
				\4[] Pero depende de distribución de la renta
				\4 Entendido por Marx como defensa de rentismo
				
		\2 Veblen (1899)
			\3 Idea clave
				\4 ``The Theory of the Leisure Class''
				\4 Análisis del consumo en términos psicológicos
				\4 Consumo para mostrar estatus frente a otros
				\4 Visión negativa de consumo excesivo
				\4[] Induce desperdicio de trabajo y rentas
				\4 Introduce idea de análisis psicológico
			\3 Implicaciones
				\4 Aspectos conductuales del consumo
				\4[] Más allá de utilidad por consumo de bienes
				\4 Consumo individual depende de consumo de otros
		\2 Irving Fisher
			\3 Idea clave
				\4 Modelo microeconómico de consumo intertemporal
				\4 Dos periodos
				\4 Activo financiero permite trasladar rentas
				\4 Inversión en capital permite renta futuro
			\3 Implicaciones
				\4 Teorema de la separación
				\4[] Decisión de inversión independiente de consumo
				\4 Decisión de consumo tiene dimensión dinámica
				\4[] Depende de flujo temporal de rentas
				\4 Consumo depende de valor actual de la renta
		\2 Pigou
			\3 Idea clave
				\4 Efecto Pigou
				\4[] Consumo depende de riqueza
				\4[] Riqueza depende de saldos reales
				\4[] Saldos reales dependen negativamente de precio
				\4[$\then$] Deflación puede aumentar consumo
			\3 Implicaciones
				\4 Consumo es estabilizador de economía
				\4[] Permite:
				\4[] $\to$ Aumentar DA cuando por debajo de capacidad
				\4[] $\to$ Reducir DA cuando por encima de capacidad
	\1 \marcar{Teorías keynesianas del consumo}
		\2 Demanda de consumo keynesiana\footnote{Keynes (1936) Book III: The Propensity to Consume.}
			\3 Idea clave
				\4 Contexto
				\4[] Teoría general de la macroeconomía
				\4[] Debate sobre estabilidad de economía
				\4[] $\to$ Qué papel juega consumo
				\4[] Influencia de autores clásicos que matizan estabilidad
				\4[] $\to$ Malthus: necesarios rentistas
				\4[] $\to$ Hobson: posible desempleo masivo
				\4[] Microfundamentación formal con opt. racional
				\4[] $\to$ Aún no aparece
				\4 Objetivo
				\4[] Caracterizar demanda agregada de consumo
				\4[] $\to$ Mediante regla simple que aproxime bien
				\4[] Extraer conclusiones sobre impacto en demanda agregada
				\4 Resultados
				\4[] Supuesto ad-hoc sobre comportamiento
				\4[] $\to$ Demanda de consumo crece menos que renta
				\4[] \% de renta dedicada a consumo
				\4[] $\to$ Cae con aumento de la renta
				\4[] $\then$ Posibles excesos de capacidad
				\4[] Consumo total
				\4[] $\to$ Aumenta con aumento de la renta
			\3 Formulación
				\4 ``Ley psicológica fundamental''\footnote{Keynes (1936): \textit{The fundamental psychological law, upon which we are entitled to depend with great confidence both a priori from our knowledge of human nature and from the detailed facts of experience, is that men are disposed, as a rule and on the average, to increase their consumption as their income increases, but not by as much as the increase in their income. That is to say, if $C_w$ is the amount of consumption and $Y_w$ is income (both measured in wage-units), $\Delta C_w$ has the same sign as $\Delta Y_w$ but is smaller in amount, i.e. $d C_w /d Y_w$ is positive and less than unity.}}
				\4 $C_t = C_0 + c Y_t$
				\4[] $ c < 1$
				\4 Proporción marginal al consumo
				\4[] Aumento de $C_t$ dado aumento en $Y_t$
				\4[] $\to$ $\text{PMgC} = c$
				\4 Proporción media al consumo
				\4[] $\frac{C_t}{Y_t} = \frac{C_0 +c Y_t}{Y_t} = \frac{C_0}{Y_t} +c$
				\4 Otros factores que determinan consumo
				\4[] i. Cambios en salario nominal
				\4[] ii. Cambios en renta neta\footnote{Concepto keynesiano aproximadamente similar al de renta nacional bruta.}
				\4[] iii. Cambios en valor del capital/riqueza
				\4[] iv. Cambios en tasa de descuento del futuro
				\4[] $\to$ Aunque diferente a interés, asume iguales s.p.g
				\4[] $\to$ Efectos a c/p sólo si cambios muy bruscos
				\4[] $\to$ Efectos a l/p sí pueden ser relevantes
				\4[] v. Cambios en expectativas sobre ingreso presente y futuro
				\4[] $\to$ No le otorga gran importancia
				\4[] $\to$ Shocks de expectativas promedian 0
			\3 Implicaciones
				\4 Política fiscal
				\4[] $\to$ Políticas más redistributivas aumentan propensión
				\4[] $\then$ Reducen renta de agentes con menores PMe consumir
				\4[] $\to$ Aumento brusco de impuestos puede deprimir el consumo
				\4 Consumo aumenta con renta
				\4[] $\frac{d C_t}{d Y_t} = c > 0$
				\4 Proporción de consumo sobre renta cae con renta
				\4[] $\frac{d (C_t / Y_t)}{d Y_t} = < 0$
				\4 Renta no destinada a consumo aumenta en el tiempo
				\4[] Asumiendo crecimiento de la renta
				\4[] $\to$ Cada vez menor proporción consumida
				\4 Tasa de ahorro creciente
				\4[] A medida que $C_t$ crece menos que $Y_t$
				\4[] $\to$ Mayor proporción de renta ahorrada
				\4 Insuficiencias de demanda agregada crecientes
				\4[] A medida que aumenta la renta
				\4[] $\to$ Menos proporción se dedica a consumo
				\4[] $\then$ Aumenta tasa de ahorro
				\4[] $\then$ Economía por debajo de potencial
				\4 Estímulos exógenos a la demanda agregada
				\4[] Efecto sobre renta multiplicado por $\frac{1}{s}$
				\4[] $\then$ Teoría de consumo induce multiplicador keynesiano
				\4[] Necesarios para llevar economía a capacidad
				\4 Caídas de renta reducen menos el consumo
				\4[] Demanda de consumo puede de hecho exceder renta
				\4[] Factor de estabilización de economías
				\4[] $\to$ Si Y y C cayesen igual
				\4[] $\then$ Posible trayectoria inestable de la renta
				\4 Multiplicador del consumo:
				\4[] Aumenta renta más que aumenta consumo
				\4[] $\to$ Multiplicador superior a 1
				\4[] $\then$ Deseable estímulo exógeno de demanda agregada
				\4[] Efecto limitado
				\4[] $\to$ Converge siempre que $c<1$
			\3 Valoración
				\4 Elemento fundamental de teoría macro keynesiana
				\4 Consumo como resultado de primitivas psicológicas
				\4[] Enfoque adelanta formulación behavioral
				\4 Confirmación empírica parcial
				\4[] Basado en confirmación del multiplicador
		\2 Paradoja de Kuznets
			\3 Idea clave
				\4 Contexto
				\4[] Teoría keynesiana del consumo
				\4[] $\to$ Bloque central de modelo de DAgregada
				\4[] Predicción
				\4[] $\to$ Ahorro debería aumentar con aumento renta
				\4[] Desarrollo de contabilidad nacional
				\4[] $\to$ Por el el propio Kuznets, entre otros
				\4[] Hechos estilizados de la macroeconomía
				\4[] $\to$ Evidencia empírica macro consistente y repetida
				\4 Objetivos
				\4[] Caracterizar hechos estilizados del consumo
				\4[] Contrastar empíricamente predicción keynesiana
				\4 Resultados
				\4[] Divergencia empírica con dda. keynesiana
				\4[] En corto plazo sí es compatible
				\4[] $\to$ Consumo más estable que renta
				\4[] $\then$ Cae menos ante caídas de renta
				\4[] $\then$ Aumenta menos ante aumentos de renta
				\4[] En largo plazo no
				\4[] $\to$ Tasa de ahorro no sigue tendencia clara
			\3 Formulación
				\4 Estimación econométrica
				\4[] $C_t = C_0 + c Y_t$
				\4 Estimaciones de corto plazo consumo-renta
				\4[] En series de corto plazo
				\4[] En sección cruzada ricos y pobres
				\4[] $C_0 >0$
				\4[] $c\sim 0.75$
				\4[] $\text{PMeC} > \text{PMgC}$
				\4[$\then$] Se cumple predicción keynesiana
				\4 Largo plazo
				\4[] $\text{PMeC} = \text{PMgC}$
				\4[] $\then$ Consumo = proporción cte. de Y
				\4[$\then$] No se cumple predicción keynesiana
				\4 Hechos empíricos fundamentales del consumo
				\4[] i. Ricos ahorran más que pobres
				\4[] ii. Ahorro no sigue una tendencia clara en el tiempo
				\4[] iii. El consumo es estable en el corto plazo
				\4[] \grafica{kuznets}
			\3 Implicaciones
				\4 Contradicción entre c/p y l/p
				\4[] $\text{PMeC} > \text{PMgC} \then C_0 > 0$
				\4[] $\text{PMeC} = \text{PMgC} \then C_0 = 0$
				\4 Demanda keynesiana
				\4[] ¿Sólo se cumple en corto plazo?
				\4[] ¿En l/p no hay consumo autónomo?
				\4[] ¿En l/p aumenta PMgC?
		\2 Teoría de la renta relativa de Duesenberry\footnote{Ver \href{https://www.nytimes.com/2005/06/09/business/the-mysterious-disappearance-of-james-duesenberry.html}{Frank (2005) en NYTimes sobre Duesenberry.}}
			\3 Idea clave
				\4 Contexto
				\4[] Teoría keynesiana del consumo
				\4[] $\to$ $\Delta$ \% de renta a consumo < renta
				\4[] Supuestos de carácter behavioral frecuentes
				\4[] $\to$ Comportamiento sigue reglas heurísticas
				\4[] HER de Muth aún no generalizado
				\4[] $\to$ No se asume optimización intertemporal
				\4[] Thorstein Veblen a principios de s. XX
				\4[] $\to$ Consumimos para emular e impresionar a otros
				\4[] Duesenberry (1949)
				\4[] Evidencia empírica de consumo por emulación
				\4[] $\to$ Familias rentas
				\4[] $\then$ ``Keeping up with the Joneses''
				\4 Objetivos
				\4[] Explicar paradoja de Kuznets
				\4[] $\to$ Por qué $\frac{C}{Y}$ no cae en l/p
				\4[] Mantener hechos estilizados del consumo
				\4[] $\to$ Ricos ahorran más que pobres
				\4[] $\to$ Tasas de ahorro sin tendencia clara
				\4[] $\to$ Consumo nacional estable en el corto plazo
				\4 Resultado
				\4[] Consumo es relativo a consumo de otros
				\4[] $\to$ De sí mismo en pasado
				\4[] $\to$ De otros en presente
				\4[] Fuerte influencia en años 50 y 60
				\4[] $\to$ Hasta emergencia renta permanente y microfund.
			\3 Formulación
				\4 Dos variantes al mismo tiempo
				\4[] Explicar incoherencia corto y largo plazo
				\4[] $\to$ A corto plazo PMeC mayor que PMeC
				\4[] $\to$ A largo plazo, PMeC constante igual a PMeC
				\4 Sección cruzada
				\4[] Si consumo más bajo:
				\4[] $\to$ Que otros agente o en otro periodo
				\4[] $\then$ Trata de igualarse
				\4[] Ej.: $c_{t-1}$ alto
				\4[] $\then$ $c_{t} \downarrow$ menos que $y$
				\4[] Si renta más alta
				\4[] Tiene menos con quién compararse
				\4[] Consume proporción constante de nueva renta
				\4[] $\to$ Sin tendencia clara en el ahorro
				\4[] Evidencia empírica:
				\4[] $\to$ Familias blancas y negras con igual ingreso
				\4[] $\then$ Blancos consumen más que negros
				\4[] $\then$ Blancas gastan más para emular vecinos ricos
				\4[] $\then$ Ahorro depende de diferencias en ingreso relativo
				\4[] Explicación
				\4[] $\to$ Blancos se comparan con vecinos
				\4[] $\to$ ``Keep up with the joneses
				\4[] $\then$ Blancos tienen mayor PMeC
				\4[] $\then$ PMeC mayor que PMeC
				\4 Series temporales
				\4[] Consumidores se comparan consigo mismos
				\4[] Cuando economía crece para todos
				\4[] $\to$ Todos aumentan consumo proporcionalmente
				\4[] Cuando economía cae
				\4[] $\to$ Se comparan consigo mismos antes de recesión
				\4[] $\then$ Reducen consumo menos que renta
				\4[] $\then$ Ahorran menos
				\4[] $\then$ Consumo más estable que renta
				\4[] Largo plazo
				\4[] $\to$ Renta crece para todos
				\4[] $\then$ no compara
				\4[] $\then$ $c_t:$ proporción cte.
				\4 Representación gráfica
				\4[] Rectas con menor pendiente
				\4[] $\to$ Representan distintos grupos sociales
				\4[] \grafica{duesenberry}
			\3 Implicaciones
				\4 Pobres tienen menor tasa de ahorro
				\4[] Porque tratan de emular a ricos
				\4 Nuevos pobres ahorran menos
				\4[] Porque se comparan consigo mismos antes
				\4[] $\then$ Estabilidad del consumo ante recesión
				\4 Ricos tienen mayor tasa de ahorro
				\4[] Porque no tienen con quién comparar
			\3 Valoración
				\4 Explicación basada en heurística psicológica
				\4 Difícil incorporación en marco de optimización
				\4[] Intentos de representación con teoría de juegos
				\4[] $\to$ Dinámicas muy complejas
	\1 \marcar{Modelos dinámicos microfundamentados}
		\2 Idea clave
			\3 Contexto
				\4 Síntesis neoclásica
				\4[] Consumo es pieza central
				\4[] Permite:
				\4[] $\to$ En c/p, exceso de capacidad
				\4 Revolución de las expectativas racionales
				\4 NMC
			\3 Objetivos
				\4 Modelo tratable de paradoja de Kuznets
				\4 Demanda de consumo resistente a crítica de Lucas
			\3 Resultados
				\4 Microfundamentación de bloque central de IS-LM
				\4 Consolidación de dinámica de dda. de consumo
		\2 Teoría del ciclo vital de Modigliani y Ando
			\3 Idea clave
				\4 Contexto
				\4[] SNC
				\4[] $\to$ Función de consumo bloque fundamental
				\4[] Paradoja de Kuznets
				\4[] $\to$ Propensión media al consumo decreciente en c/p
				\4[] $\to$ Ahorro sin tendencia clara en l/p
				\4 Objetivos
				\4[] Explicar transferencia de renta a lo largo de vida
				\4[] $\to$ Efectos sobre perfil de consumo
				\4 Resultados
				\4[] Consumo depende de renta total
				\4[] Periodos con renta alta
				\4[] $\to$ Aumento del ahorro
				\4[] Periodos con renta baja
				\4[] $\to$ Caída del ahorro
			\3 Formulación
				\4 RPresupuestaria depende de renta a lo largo de vida
				\4[] Riqueza total $= A_0 + \sum_{t=0}^T \frac{Y_t}{(1+r)^t}$
				\4 Fase de acumulación de capital humano
				\4[] Agentes consumen más que renta
				\4 Fase de acumulación de ahorro
				\4[] Tras formación hasta jubilación
				\4[] Ingresos superiores a consumo
				\4 Fase de desinversión
				\4[] Jubilación hasta muerte
			\3 Implicaciones
				\4 Renta total aumenta consumo en periodo
				\4 Aumento de renta en un periodo tiende a ahorrarse
				\4 Tipo de interés
				\4[] Afecta consumo
				\4[] $\to$ Reduce si fase de endeudamiento
				\4[] $\to$ Aumenta si fase de desinversión
			\3 Valoración
				\4 Paralelismo con modelo de Diamond OLG
				\4[] Horizontes finitos de decisión
				\4 Difícil estimación empírica
				\4[] Muy difícil estimar renta en ciclo vital
				\4[] Flujos de renta presente sí son estimables
				\4[] Muy difícil estimar stocks de riqueza
				\4 Problemas de obtención de datos
				\4[] Muy difícil conocer valor total de riquezas
				\4 Capital humano
				\4[] Mayor proporción de riqueza en mayoría de agentes
				\4[] Muy difícil medición
		\2 Teoría de la renta permanente de Friedman
			\3 Idea clave
				\4 Contexto
				\4[] Keynesianismo y SNC
				\4[] $\to$ Insuficiencias de demanda agregada persistentes
				\4[] Kuznets
				\4[] $\to$ Difícilmente compatibles con Keynes en l/p
				\4[] Duesenberry
				\4[] $\to$ Teoría basada en sesgos psicológicos
				\4[] $\to$ Explicación behavioral de hechos de Kuznets
				\4 Objetivos
				\4[] Explicar hechos estilizados de Kuznets
				\4[] Microfundamentar decisión de consumo
				\4[] $\to$ Con horizonte infinito
				\4[] Efectos de ciclo económico
				\4[] Efectos shocks transitorios y permanentes
				\4 Resultados
				\4[] Agentes deciden consumo respecto renta permanente
				\4[] $\to$ Entre otros factores
				\4[] $\to$ Renta en el periodo es factor poco relevante
				\4[] Estímulos de renta de corto plazo
				\4[] $\to$ No tienen efecto
				\4[] $\then$ Reduce justificación de estímulos a DA
				\4[] Explicación de hechos estilizados
				\4[] i. Ricos ahorran más que pobres
				\4[] $\to$ Porque ingresan más que su RPermanente
				\4[] ii. Ahorro nacional sin tendencia respecto renta
				\4[] $\to$ Cuestiones demográficas relevantes
				\4[] iii. Consumo más estable que renta
				\4[] $\to$ Porque consumo depende de RPermanente
			\3 Formulación
				\4 Ingreso permanente
				\4[] Concepto central del modelo
				\4[] Valor presente de flujo irregular de ingresos
				\4[] $\sum_{t=0}^\infty \frac{Y^P}{(1+r)^t} = A_0 + \sum_{t=0}^\infty \frac{Y_t}{(1+r)^t}$
				\4 Estimación del ingreso permanente
				\4[] Inicialmente, expectativas adaptativas
				\4[] $\to$ Depende de consumo pasado y desviación
				\4[] HER dará lugar a nueva familia de modelos
				\4 Consumo dividido en dos componentes
				\4[] Consumo permanente
				\4[] $\to$ Planificado
				\4[] $\to$ Estable
				\4[] $\then$ Depende de ingreso permanente
				\4[] Consumo transitorio
				\4[] $\to$ Inesperado
				\4[] $\to$ Variable
				\4[] $\then$ Término de error en regresión de consumo
				\4 Optimización del consumo permanente
				\4[] $\underset{\left\lbrace C_t \right\rbrace}{\max}\quad U(\left\lbrace C_t \right\rbrace) = \sum_{t=0}^n \beta^t u(c_t)$
				\4[] $\text{s.a:}$
				\4[] $\quad \sum_{t=0}^n C_t \frac{1}{(1+r)^t} = \sum_{t=0}^\infty \frac{Y^P}{(1+r)^t} = A_0 + \sum_{t=0}^\infty \frac{Y_t}{(1+r)^t}$
				\4[] $\to$ $\text{s.a:} \quad \sum_{t=0}^\infty \frac{C^P}{(1+r)^t} = \sum_{t=0}^\infty \frac{Y^P}{(1+r)^t} = A_0 + \sum_{t=0}^\infty \frac{Y_t}{(1+r)^t}$
				\4[] Utilidad separable, aditiva y cóncava
				\4[] Óptimo:
				\4[] $u'(C_t) = \beta (1+r) u'(C_{t+1}) \quad \forall t$
				\4[] $\to$ Asumiendo $\beta=1$ y $r=0$ s.p.g.:
				\4[$\then$] $C_t = C_{t+1} = a \Omega = \sum_{t=0}^\infty \frac{Y^P}{(1+r)^t} \cdot \frac{1}{n}$
				\4[$\then$] Perfil de consumo $\neq$ perfil de renta
				\4[$\then$] Consumo = $f(\textrm{renta permanente})$
				\4 Consumo permanente total
				\4[] Igualando $r=0$ s.p.g.
				\4[] $\sum_{t=1}^T C_t \leq A_0 + \sum_{t=1}^T Y_t$
				\4[] $ \then C_t = \frac{1}{T} \left(\sum_{t=1}^T Y^P \right) = \frac{1}{T} \Omega$
				\4[] $ \then C_t = a \Omega $
				\4[] $\then$ En cada periodo, parte de la renta permanente
			\3 Implicaciones
				\4 Aumentos transitorios:
				\4[] Muy poco efecto
				\4[] $\to$ Sólo en la medida en que afectan ingreso permanente
				\4[] $\then$ Se divide a lo largo de senda
				\4[] i.e.: $\varDelta$ temporales de impuestos
				\4[] Estados Unidos en 1968\footnote{Ante el calentamiento de la economía con motivo de la Guerra de Vietnam, la síntesis neoclásica recomendaba una subida de impuestos para reducir la demanda agregada. Se anunció expresamente como temporal un recargo del 10\%. Apenas tuvo un efecto perceptible sobre output e inflación.}
				\4 Aumentos permanentes
				\4[] Efectos grande sobre consumo permanente
				\4 Consumo permanente y transitorio incorrelados
				\4[] Transitorio depende de shocks aleatorios
				\4[] Permanente depende de renta permanente
			\3 Valoración
				\4 Enorme impacto
				\4[] Origen de microfundamentación de consumo en macro
				\4 Debilita SNC
				\4[] Intervención sobre DA agregada transitoria
				\4[] $\to$ Menos justificable
				\4 Guerra de Vietnam y aumento de impuestos
				\4[] Finales de los 60
				\4[] Demanda agregada muy fuerte
				\4[] $\to$ Gasto militar
				\4[] $\then$ Aumento de la inflación
				\4[] Gobierno intenta reducir $\uparrow$ impuesto renta temporal
				\4[] $\to$ Explícitamente anunciado como temporal
				\4[] Modelos macroeconométricos keynesianos
				\4[] $\to$ Estimaban respuesta con función keynesiana
				\4[] $\then$ Esperaban bajada fuerte de demanda agregada
				\4[] Realmente
				\4[] $\to$ Consumo apenas afectado
				\4[] $\then$ Ataque al paradigma keynesiano
				\4 Multitud de variantes
				\4 Difícil contrastación
				\4[] Contrastación implica muchos supuestos sobre f. de u
				\4[] $\to$ Contrastación implica contrastar muchos supuestos
				\4[] $\then$ Difícil distinguir exactamente hipótesis RP
		\2 Modelo de Hall (1978)
			\3 Idea clave
				\4 Elementos centrales
				\4[] $\to$ Hipótesis de la renta permanente
				\4[] $\to$ Incertidumbre
				\4[] $\to$ HER
				\4 Rentas futuras:
				\4[] Sufren $\varDelta$ estocásticos
				\4[] Expectativas HER sobre $\varDelta$ futuras.
				\4 Consumo fracción rentas esperadas
				\4 Sólo shocks estocásticos $\uparrow \downarrow$ $C_t$
				\4 Consumo sigue martingala
				\4[] Paseo aleatorio es caso particular
				\4[] $\to$ Martingala con shocks i.i.d.
				\4[] Necesaria utilidad cuadrática
				\4[] Bajo supuestos más generales sobre f. de u.
				\4[] $\to$ Desviaciones de consumo medio son impredecibles
				\4[] $\then$ Conclusión más débil que paseo aleatorio
                %\footnote{Pero es habitual afirmar que el modelo predice que el consumo será un paseo aleatorio. Un paseo aleatorio es una martingala en la que los shocks son i.i.d., es decir, un caso especial de martingala. Es importante notar que el consumo sigue una martingala sólo si se asume una función de utilidad cuadrática. Bajo supuestos más generales se llega a la conclusión menos fuerte de que las desviaciones del consumo medio son impredecibles.}
			\3 Formulación
				\4 Utilidad del periodo cuadrática:
				\4[] $u(c_t) = c_t - \frac{a}{2} c_t^2$
				\4 Utilidad total
				\4[] $E(U) = E \left[ \sum_{t=0}^T \left( c_t - \frac{a}{2} c_t^2 \right) \right]$
				\4 Problema de maximización
				\4[] $\underset{\left\lbrace c_t \right\rbrace^T_0}{\max} \quad E(U) = E \left[ \sum_{t=0}^T u(c_t) \right] = E \left[ \sum_{t=0}^T \left( c_t - \frac{a}{2} c_t^2 \right) \right] $
				\4[] $\text{s.a:} \quad \quad \sum_{t=0}^T E_0 \left[ C_t \right] = \sum_{t=0}^T E_0 \left[ Y_t \right]$
				\4 Óptimo
				\4[] $ u'(c_t) = E\left( u'(c_{t+1}) \right)$
				\4[] $c_t = E(c_{t+1}) \then c_{t+1} = c_t + e_t$
			\3 Implicaciones
				\4 Consumo es fracción de RPermanente esperada
				\4[] $c_t =  \frac{1}{T} \left( E_t \sum_{t}^T  \left[ Y_t \right] \right)$
				\4 Shocks $e_t$
				\4[] Son cambios en expectativas de rentas
				\4[] $c_{t+1} = c_t + \frac{1}{T-1} \left( E_{t+1} \sum_{s=t+1}^T [Y_s] - E_t \sum_{t+1}^T [Y_s] \right)$
				\4 Variación del consumo en cada periodo
				\4[] Diferencia entre estimaciones de renta futura
				\4[] $\then$ Sólo shocks imprevistos afectan a consumo
				\4[] $\then$ Shocks de renta ya previstos no tienen efecto
			\3 Valoración
				\4 Resultados mixtos
				\4 Difícil contrastación
				\4 Campbell y Mankiw (1989)
				\4[] Rechazan Hall (1979)
				\4[] $\varDelta y$ anticipados predicen $\varDelta C$
				\4[] $\to$ Pero sólo shocks inesperados deberían afectar
				\4 Otros estudios:
				\4[] Si $\varDelta y$ son grandes y predecibles
				\4[] $\then$ Hipótesis RP describe bien
				\4 ¿Papel de restricciones de liquidez?
				\4 ¿Contradicción datos micro-macro?
	\1 \marcar{Impacto del tipo de interés}\footnote{Hasta ahora habíamos considerado un tipo de interés y una tasa de descuento nulas, o equivalentemente, que ambos eran iguales. En esta sección se introducen ambos.} 3'-23'
		\2 Elasticidad intertemporal de sustitución
			\3 CRRA
				\4 $u(c_t) = \frac{c_t^{1-\theta}}{1-\theta}$
			\3 Euler\footnote{Derivado de: $\frac{1}{(1+\rho)^t} u'(c_t) = (1+r) \frac{1}{(1+\rho)^{t+1}} u'(c_{t+1})$.}
				\4 $\frac{C_{t+1}}{C_t} = \left( \frac{1+r}{1+\rho} \right)^{1/\theta}$
			\3 Parámetro $\theta$
				\4 Mide aversión relativa al riesgo
			\3 Elasticidad intertemporal de sustitución
				\4 Sustitución de consumo presente por futuro
				\4[] Ante variaciones del precio del consumo presente
				\4[] \fbox{$\sigma = \frac{1}{\theta} = \frac{d \ln \left( \frac{c_{t+1}}{c_t} \right)}{d r}$}
				\4 Cuanto más elevado $\sigma$
				\4[] Mayor efecto de interés sobre consumo
				\4[] $\then$ Menor consumo presente si aumenta interés
		\2 Implicaciones\footnote{Dibujar gráficas}
			\3 Efectos renta (aumento de r)
				\4 Mayor renta en siguiente periodo
				\4 Consumo es bien normal
				\4[] Respecto a renta permanente
				\4[$\then$] Aumenta demanda de consumo presente
			\3 Efecto sustitución (aumento de r)
				\4 Consumo presente más caro
				\4[$\then$] Cae demanda de consumo
			\3 $\varDelta r$ sin ahorro ni deuda
				\4 Sin ER
				\4 ES negativo
				\4 Baja consumo en t
				\4 Aumento consumo en $t+1$
			\3 $\varDelta r$ con deuda
				\4 ER negativo
				\4 ES negativo
				\4 Consumo presente baja
			\3 $\varDelta r$ con ahorro
				\4 ER positivo
				\4 ES negativo
				\4 Resultado ambiguo sobre consumo presente
		\2 Empíricamente
			\3 Difícil contrastación
				\4 Cambios en tipo de interés junto a otros cambios
			\3 Horizontes largos
				\4 Efecto de interés se acumula
				\4 Ahorro puede ser muy sensible a $\varDelta r$
				\4[] A pesar de $1/\theta$ muy baja
	\1 \marcar{Otros desarrollos} 5'-28'
		\2 Activos arriesgados: CCAPM
			\3 Idea clave
				\4 Contexto
				\4[] Hall (1979)
				\4[] $\to$ Consumo agregado depende de shocks aleatorios
				\4[] CAPM
				\4[] $\to$ Modelo de valoración de activos en equilibrio
				\4[] $\to$ Sharpe, Litner, Treynor...
				\4[] Demanda de consumo
				\4[] $\to$ Activos financieros para suavizar temporalmente
				\4 Objetivo
				\4[] Valorar activos financieros
				\4[] $\to$ En función de demanda de consumo
				\4[] $\to$ En función relación con rentas inciertas
				\4 Resultados
				\4[] Beta de consumo
				\4[] Sensibilidad de precio a senda de mercado
				\4[] $\to$ Determina rendimiento de activos
				\4[] Si $P_{t+1}$ y $C_{t+1}$ correlado positivamente
				\4[] $\to$ Agentes exigen más rentabilidad
				\4[] Porque si $c_t$ alto, menor $u'(c_t)$
				\4[] Sustitución de cartera de mercado
				\4[] Por consumo agregado
			\3 Formulación
				\4 Función de utilidad cuadrática
				\4[] Permite análisis media varianza
				\4 Activos financieros
				\4[] Permiten suavizar senda de consumo
				\4[] $\to$ Vender en malos tiempos
				\4[] $\to$ Comprar en buenos tiempos
				\4 Agente recibe rentas inciertas
				\4[] De forma similar a Hall
				\4 Precio de activos financieros
				\4[] En función de correlación con rentas recibidas
				\4 Elevada correlación con rentas
				\4[] Aumenta prima de riesgo exigida
				\4[] $\to$ Cae precio del activo
				\4 Baja correlación con rentas recibidas
				\4 Ejemplo
				\4[] Demanda de consumo a suavizar
				\4[] Trabajador en fábrica de automóviles
				\4[] $\to$ Renta salarial sujeta a venta automóviles
				\4[] Decisión de inversión:
				\4[] $\to$ Acciones de fabricantes automóviles
				\4[] $\then$ Alta correlación con shocks a salario
				\4[] $\to$ Acciones de empresas tecnológicas
				\4[] $\then$ Baja correlación con shocks a salario
				\4[] Pagará prima por acciones tecnológicas
				\4[] Exigirá prima por acciones tecnológicas
				\4[] Activos financieros le permiten suavizar senda
				\4[] $\to$ Reducir impacto de shocks a salario
				\4[] $\then$ Demanda de consumo determina precio activos
			\3 Implicaciones
				\4 Precio de activos financieros
				\4[] Depende de demanda de consumo
				\4 Diferencias con CAPM
				\4[] Múltiples periodos
				\4[] Sustitución de cartera de mercado
				\4[] $\to$ Por demanda de consumo agregado
			\3 Valoración
				\4 Supuestos similares CAPM
				\4 Relevancia a nivel teórica
				\4 [] poca relevancia práctica
		\2 Anomalía: correlación entre crecimiento de ingreso y consumo
			\3 HRP predice
				\4 $\varDelta C$ es función de $r$ y $\rho$
				\4[] No patrón de ingreso
			\3 Estudios muestran
				\4 Correlación entre ingresos y consumo
				\4 Carroll and Summers (1991)
				\4 Nivel agregado y micro
				\4 Hogares tienen poco ahorro
				\4 Generalmente, ahorro precautorio
				\4[] Evitar pérdida de acceso a mercados financieros
		\2 Restricciones de liquidez
			\3 Práctica
				\4 Límites de crédito
				\4 Diferencia interés deuda y ahorro
			\3 Límite de crédito es vinculante\footnote{Es decir, cuando el consumo óptimo es tal que el agente desearía pedir prestada una cantidad superior a la que puede acceder.}
				\4 Individuos consumen menos que desarían
			\3 Límite de crédito podría ser vinculante\footnote{Es decir, los agentes estiman que la restricción puede vincular en el futuro.}
				\4 Restricción podría vincular en un futuro
				\4[] Agentes estiman mayores necesidades en el futuro
				\4 Agentes ahorran para no sufrir restricción
				\4[$\then$] Ahorro precautorio
		\2 Optimización incompleta
			\3 Inconsistencia temporal
				\4 Planes a largo plazo
				\4 No se llevan a cabo cuando l/p $\rt$ c/p
			\3 Reglas de oro
				\4 Posible respuesta racional a costes de cálculo
		\2 Ocio-consumo
			\3 Complementariedad bienes-ocio
				\4 Demanda de ciertos bienes $\uparrow $ si $\uparrow $ ocio
			\3 Impuestos
				\4 Debate de tributación óptima
				\4 Gravar complementarios de ocio
				\4[] Aumenta oferta de trabajo?
				\4[] Reduce oferta de trabajo?
				\4 Generalmente: gravar complementarios a ocio
		\2 Equivalencia ricardiana
	\1[] \marcar{Conclusión} 2'-30'
		\2 Recapitulación
			\3 Modelos estáticos
			\3 Modelos dinámicos
			\3 Impacto del tipo de interés
			\3 Otros desarrollos
		\2 Idea final
			\3 Renta permanente
				\4 Consistente con teoría de demanda
				\4 Microfundamentada
				\4 Construye marco de análisis
			\3 Anomalías
				\4 No se explica por un sólo factor
				\4 Mezcla de tres puntos
			\3 Tributación
				\4 Enorme relevancia demanda de consumo
				\4 Diferentes efectos de impuestos sobre PIB
\end{esquemal}






































%\end{esquema}
%\end{multicols}

\graficas

\begin{axis}{4}{Representación gráfica de la paradoja de Kuznets.}{Renta}{Consumo}{kuznets}
	% Consumo de largo plazo
	\draw[-] (0,0) -- (4,4);
	
	% Consumo de corto plazo 1
	\draw[-] (0.5,1) -- (3.5,2);
	\node[right] at (3.5,2){Consumo en 1};
	
	% Consumo de corto plazo 2
	\draw[-] (1,1.5) -- (4,2.5);
	\node[right] at (4,2.5){Consumo en 2};
	
\end{axis}

\conceptos

\concepto{Hipótesis de la renta relativa}:

\begin{axis}{4}{Consumo y renta en el modelo de Duesenberry}{Y}{C}{duesenberry}
	
	% Largo plazo
    \draw[-] (0,0) -- (4,4);
    \node[right] at (4,4){\small l/p};
    
    % Curva de corto plazo con renta baja
    \draw[-] (0.5,1) -- (4, 1.5);
    \node[right] at (4,1.5){\small c/p tras aumento de renta};
    
    % Curva de largo plazo con renta alta
    \draw[-] (0.5, 2.5) -- (4,3);
    \node[right] at (4,3){\small c/p};
\end{axis}

\preguntas

\seccion{Test 2018}

\textbf{17.} ¿Cuál de las siguientes afirmaciones es \underline{\textbf{INCORRECTA}}?

\begin{itemize}
	\item[a] Según la teoría de la demanda de consumo de Keynes la propensión media al consumo es creciente con la renta.
	\item[b] La Teoría de la Renta Relativa de Duesenberry señala que el consumo de los hogares depende tanto de su nivel de renta actual como de los niveles de renta obtenidos en el pasado.
	\item[c] La Teoría del Ciclo Vital de Modigliani asume que los consumidores conocen con certeza la trayectoria de su renta futura.
	\item[d] La Hipótesis del Paseo Aleatorio de Hall sugiere que el mejor predictor del consumo futuro de un individuo es su consumo presente.
\end{itemize}

\seccion{Test 2009}

\textbf{16.} Algunos estudiosos han evaluado que alguna medida del componente cíclico del consumo privado de la economía española fluctúa tanto o más que el componente cíclico del PIB.

\begin{enumerate}
	\item[a] Esta observación es consistente con la hipótesis de la renta permanente.
	\item[b] Esta observación es una anomalía respecto a la hipótesis de la renta permanente que podría explicarse por la existencia de restricciones de liquidez.
	\item[c] Esta observación es consistente con el proceso de convergencia de las instituciones financieras españolas con las de la Unión Europea.
	\item[d] Esta observación tiene por estar por fuerza equivocada.
\end{enumerate}

\seccion{Test 2004}
\textbf{21}. Los modelos de decisión intertemporal bajo expectativas racionales resaltan:
\begin{enumerate}
	\item[a] Los mecanismos de sustitución intertemporal del consumo y del ocio.
	\item[b] La contradicción entre las decisiones de consumo y la hipótesis de la renta permanente.
	\item[c] La ausencia de respuesta del consumo presente a los aumentos de la renta futura.
	\item[d] Los efectos de las políticas actuales sobre las decisiones de consumo y ahorro contemporáneas.
\end{enumerate}

\seccion{9 de marzo de 2017}
\begin{itemize}
    \item Papel del crédito al consumo. Elabora sobre el papel de la financiación en el consumo.
    \item ¿Cree usted que el envejecimiento esta descontado en el consumo? ¿Habría que hacer más evidente que el sistema de pensiones no es sostenible para que los consumidores incorporen este patrón a su consumo?
    \item Cuales son las principales críticas de la HER según la teoría de behavioral economics.
\end{itemize}

\notas


Corregido hasta antes de CCAPM, mirar en Romer la contradicción micro-macro, restricciones de liquidez

\textbf{2018}: \textbf{17.} A

\textbf{2009}: \textbf{16.} B

\textbf{2004}: \textbf{21}. A

\bibliografia

Mirar en Palgrave:
\begin{itemize}

    \item consumption-based asset pricing models (empirical performance)
    \item consumption-based asset pricing models (theory)
	\item consumption function
    \item Euler equations
    \item Friedman, Milton
    \item liquidity constraints
    \item Modigliani, Franco
    \item permanent-income hypothesis
    \item precautionary saving and precautionary wealth
\end{itemize}

Mirar en Palgrave Money \& Finance:
\begin{itemize}
	\item consumption function
\end{itemize}

Romer, D. \textit{Advanced Macroeconomics}. Chapter 8.

Parker, J. \textit{Economics 314 Coursebook, 2010}. Chapter 16.

\end{document}
