\documentclass{nuevotema}

\tema{3A-23}
\titulo{Economía del bienestar (II). La optimalidad de la competencia perfecta y las imperfecciones del mercado. Las externalidades y los bienes públicos. Los fallos del sector público.}

\begin{document}

\ideaclave

Lionel Robbins y posteriormente Samuelson definieron la ciencia económica como el estudio de las decisiones que tratan de gestionar una serie de recursos finitos con usos alternativos con el fin de satisfacer una serie de necesidades humanas. La economía del bienestar es un campo dentro de la ciencia económica que examina la medida y la maximización del bienestar social. El examen y la comparación de diferentes estados sociales que resultan de diferentes configuraciones institucionales y sociales son así los principales objetos de estudio de la economía del bienestar, pero también de la economía en general. Como afirmó Atkinson, el gran teórico de la desigualdad del siglo XX, la ciencia económica no existe sólo para describir el comportamiento humano y satisfacer la curiosidad y la vanidad de los economistas, sino para emitir recomendaciones y diseñar y valorar políticas que contribuyan a mejorar la vida de los ciudadanos. Por ello, la economía del bienestar forma parte del ``corazón'' de la ciencia económica y como tal, debe ocupar un lugar preeminente en la formación de un economista y más aún de un policy-maker. Así, si la valoración de diferentes estados sociales de acuerdo a su deseabilidad social y la caracterización de los medios para lograrlos es el objetivo de la economía del bienestar, la primera línea de actuación concierne la caracterización del óptimo económico. Este concepto define la capacidad de una asignación de recursos para inducir un bienestar tal que el bienestar individual de ningún agente puede mejorar sin empeorar el de otro. En estos términos abstractos, el óptimo económico se abstrae de las instituciones sociales que permiten el intercambio y la producción de bienes y servicios entre agentes económicos. La competencia perfecta con información perfecta y ausencia de externalidades es el marco institucional que sirve como benchmark de las demás configuraciones de mercado y su optimalidad es un resultado especialmente relevamente. La introducción en el marco de representación de fenómenos comunes en la práctica tales como las externalidades y los bienes públicos afectan a la optimalidad de los resultados. Las soluciones a la suboptimalidad de determinadas configuraciones del intercambio económico pueden requerir de la intervención del gobierno. Sin embargo, esta intervención está también sujeta a problemas que es necesario tener en cuenta. Así, el \textbf{objeto} de la exposición es dar respuesta preguntas tales como: ¿bajo qué condiciones los equilibrios de competencia perfecta son óptimos de Pareto? ¿cualquier óptimo de Pareto puede ser un equilibrio competitivo? ¿qué es una externalidad? ¿cómo puede modelizarse en términos microeconómicos? ¿qué mecanismos pueden corregir el efecto de las externalidades para alcanzar óptimos de Pareto? ¿qué es un bien público y cómo se modeliza? ¿qué son los fallos del sector público y como pueden mitigarse sus efectos? La \textbf{estructura} de la exposición se divide en cuatro partes. En la primera examinamos la optimalidad de la competencia perfecta, atendiendo a los requisitos necesarios y planteando su demostración formal. A continuación analizamos las externalidades como fenómeno a representar en términos formales y las posibles soluciones a la ineficiencia. Tratamos seguidamente el concepto de bien público y contemplamos también las posibles soluciones a la ineficiencia. Por último, ilustramos el problema de los fallos del sector público planteando sus características generales y una serie de ejemplos donde aparece, así como algunas posibles propuestas de solución o al menos mitigación de la ineficiencia. 

El modelo de \marcar{competencia perfecta} es una representación altamente idealizada de una economía en la que todos los agentes se comportan como si no tuviesen incentivos a aceptar un precio diferente al del mercado. Este modelo ocupa un lugar central en la ciencia económica por su capacidad para servir de referencia a la hora de comparar la eficiencia de los equilibrios que resultan de las infinitas diferentes configuraciones institucionales posibles. El análisis del modelo de competencia perfecta en relación a la eficiencia asignativa consiste en mostrar las condiciones bajo las cuales una asignación de competencia perfecta puede efectivamente dar lugar a óptimos de Pareto. En su obra La Riqueza de las Naciones (1776), Adam Smith afirmó que era posible que la búsqueda individual del máximo bienestar resultase en la maximización del bienestar común, dadas algunas condiciones. La microeconomía moderna formalizó esta apreciación de Smith en términos matemáticos, aplicando herramientas como el cálculo diferencial, el óptimo de Pareto y el equilibrio competitivo o walrasiano. Un equilibrio walrasiano es una asignación de bienes de consumo a un conjunto de agentes y un vector de precios relativos que cumplen dos condiciones: i) las asignaciones de bienes son soluciones a los problemas individuales de maximización de las preferencias y ii) la suma de los bienes de las asignaciones es igual a la suma de las dotaciones iniciales de bienes de que disponen los agentes. Un óptimo de Pareto es una asignación de recursos tal que no es posible reasignarlos para mejorar a al menos uno de los agentes sin empeorar a al menos uno de los demás agentes. Así, el objetivo de esta parte de la exposición es definir claramente los requisitos necesarios para que un equilibrio competitivo sea un óptimo de Pareto y demostrarlo a través del llamado Primer Teorema Fundamental del Bienestar (PTFB). Por otro lado, el Segundo Teorema Fundamental del Bienestar (STFB) muestra bajo qué condiciones todos los óptimos de Pareto pueden alcanzarse como equilibrios competitivos. 

El \textbf{Primer Teorema Fundamental del Bienestar} puede ser demostrado por distintas vías y en numerosas configuraciones institucionales concretas. Elegimos una demostración basada en el cálculo diferencial y un contexto de equilibrio general 2x2x2x2. Esto es, dos consumidores que consumen dos bienes de consumo producidos por dos empresas a partir de dos factores de producción disponibles en cantidades fijas. Asumimos también que no hay externalidades al margen de externalidades pecuniarias y que las preferencias no están saturadas. La estrategia de demostración consiste en caracterizar primero las \underline{condiciones de óptimo del equilibrio competitivo} en el que los agentes resuelven individualmente un problema de maximización, y después comparar con las condiciones de óptimo de Pareto. Los consumidores demandan bienes hasta igualar las relacióń marginal de sustitución con el cociente de precios de cada bien. Dado que todos enfrentan los mismo precios y éstos son iguales a las RMS, las RMS habrán de ser iguales entre sí en el equilibrio competitivo. En lo que respecta a las empresas, en el equilibrio competitivo demandan cantidades de factores de producción tales que el cociente de precios de los factores de producción habrá de igualarse a la relación marginal de sustitución técnica. A partir de esta igualdad es posible derivar el coste marginal de cada bien, que habrá de ser igual al precio de una unidad de factor de producción dividido entre la productividad marginal de ese factor de producción. La relación marginal técnica es habitualmente definida como la cantidad de un producto que se puede producir a cambio de reducir la cantidad producida de otro bien, dadas unas cantidades fijas de input. Dados unos precios fijos de los inputs, la relación marginal técnica equivale al cociente de costes marginales de producción. Además, dado el contexto de competencia perfecta, el precio al que las empresas ofrecen su producto es igual al coste marginal de producción, por lo que la relación marginal de sustitución es igual al cociente de precios de los bienes, que a su vez es igual a las relaciones marginales de sustitución y así quedan definidas las condición de primer orden de equilibrio competitivo. Asumiendo que se satisfacen algunas condiciones de carácter técnico respecto de la convexidad de los conjuntos de producción y las preferencias, las condiciones de primer orden son necesarias y también suficientes para caracterizar el equilibrio walrasiano. Paralelamente, para completar la demostración, es necesario\underline{caracterizar el óptimo de Pareto global} y compararlo con las condiciones de primer orden del equilibrio competitivo. Para ello, basta con plantear un problema de maximización de la utilidad de un agente respecto de los bienes que consume y la asignación de factores de producción, dadas una serie de restricciones sobre la utilidad mínima del otro agente, sobre la factibilidad de la asignación de bienes de consumo en relación a la cantidad producida resultado de la asignación de factores de producción a los procesos productivos respectivos de cada bien, y sobre la cantidad total de factores de producción utilizada. La condición de primer orden de este problema de maximización es tal que las relaciones marginales de sustitución de cada agente han de igualarse entre sí y son además iguales a la relación marginal de transformación de los bienes consumidos. Es decir, las condiciones de primer orden de maximización del bienestar total son exactamente iguales a las del equilibrio competitivo, y de ello se deduce que todos los equilibrios competitivos son óptimos de Pareto. Es preciso tener en cuenta que este resultado no implica que a partir de unas dotaciones iniciales se alcance necesariamente al equilibrio competitivo, sino simplemente que todo equilibrio competitivo alcanzable dadas las dotaciones iniciales y la configuración institucional, será un óptimo de Pareto global. Analizar la cuestión de si efectivamente se alcanza o no concierne el ajuste hacia el equilibrio y la estabilidad del sistema, algo que no es el propósito de esta exposición. También es preciso notar que esta demostración es un arma de doble filo en lo que respecta a la defensa de la libre competencia como medio para maximizar el bienestar social. Así, ha sido utilizado como argumento de los partidarios del laissez faire para sostener que la competencia perfecta da lugar a resultados óptimos. Por otro lado, los muy restrictivos supuestos necesarios para el cumplimiento del PTFB han sido utilizados como argumento contrario al laissez faire, aduciendo que las situaciones en las que es posible tal maximización de la eficiencia por vía de la libre competencia y el mercado no se dan en la práctica.

El \textbf{Segundo Teorema Fundamental del Bienestar} señala, por su parte, que cualquier óptimo de Pareto puede alcanzarse como un equilibrio competitivo a través de una reasignación de las dotaciones iniciales. Su demostración es más compleja pero puede presentarse su intuición a partir de una caja de Edgeworth mostrando como para cualquier punto dentro de la curva de contrato puede encontrarse una dotación inicial generadora de una ``lente'' de curvas de indiferencia que comprende el punto en el óptimo. Para demostrar este teorema, los supuestos de convexidad de las preferencias y del conjunto de producción no son meramente deseables por razones técnicas sino que además son necesarios. Las implicaciones del STFB son incluso más relevantes que el primero en el ámbito de la economía del bienestar, ya que implican la posibilidad de separar la búsqueda de la eficiencia y equidad si la introducción de impuestos de suma fija es posible. Así, es posible en teoría separar la búsqueda de los estados sociales Pareto-eficientes de la elección de un estado social concreto que satisfaga un criterio dado basado en un conjunto de juicios de valor explícitamente definido.

Las \marcar{externalidades} hacen referencia a los efectos que produce la decisión de un agente sobre un tercero. Las externalidades pecuniarias son aquellas que actúan a través del sistema de precios. Así, por ejemplo, cuando un agente compra una cierta cantidad de botellas de whisky y el precio de equilibrio aumenta, el resto de agentes ven reducida su capacidad de compra de whisky dados niveles fijos de renta. Las externalidades tecnológicas son aquellas que actúan de forma directa a través de las funciones de utilidad o de los conjuntos de producción. Una externalidad tecnológica se produce, por ejemplo cuando una empresa produce un bien y ello resulta en contaminación en un manantial de tal manera que los pescadores río abajo pescan menos peces: su función de producción se ve afectada negativamente. Otro ejemplo es el efecto de una vacuna en un tercer agente que no puede infectarse ya a partir del agente que ha sido inmunizado: su utilidad aumenta en la medida en que la probabilidad de infectarse se reduce. Las externalidades pecuniarias son el fundamento último de los mercados como institución reguladora de la escasez y por ello, no son las externalidades a las que se hace referencia cuando se señala que el PTFB requiere de la ausencia de externalidades. Las externalidades tecnológicas pueden ser negativas o positivas, en la medida en que su efecto sobre terceros agentes se considere deseable o no. Así, el término externalidad hace referencia en generalidad a las externalidades tecnológicas. Su presencia implica el incumplimiento del Primer Teorema Fundamental del Bienestar. Su formalización y optimalidad se analiza a continuación.

El punto de partida del análisis microeconómico de las externalidades es mostrar como su presencia en una economía aleja a los equilibrios competitivos del óptimo Paretiano. Existen múltiples variantes de esta demostración, pero adoptamos aquí un modelo de equilibrio parcial simplificado al máximo y basado en Mas-Colell, Whinston y Green (1995) que ilustra los rasgos fundamentales de una externalidad negativa. En el modelo hay dos y agentes y un sólo bien, que tiene efectos externos. El bien con efectos externos tiene un efecto positivo sobre el agente A que decrece a medida que aumenta la cantidad consumida. Así, si la utilidad marginal del agente A en el bien es positiva pero de menor cuantía a medida que aumenta el consumo (primera derivada positiva y segunda derivada negativa), al agente B le sucede lo contrario. El efecto marginal de una unidad adicional de bien consumido por A tiene un efecto negativo y de mayor cuantía en valor absoluto a medida que aumenta el consumo (primera y segunda derivadas negativas). La presencia de la externalidad se concreta en el hecho de que la utilidad del agente que sufre la externalidad se ve determinada por el comportamiento de otro agente. El agente B no tiene capacidad de decisión alguna sobre la cantidad de bien que consume A, y el agente A optimiza sus preferencias teniendo en cuenta exclusivamente su bienestar. Así, el equilibrio competitivo tiene como condición de primer orden la anulación de la utilidad marginal que le aporta al agente A una unidad adicional de bien. Para extraer conclusiones acerca de la optimalidad de este equilibrio competitivo, es necesario caracterizar el óptimo social en términos de su condición de primer orden. Un programa de maximización de la suma de las utilidades de A y B arroja una condición de primer orden tal que la suma de las utilidades marginales ha de ser igual a cero. O equivalentemente, que la utilidad marginal de A ha de ser igual a la utilidad marginal de B con signo negativo. Así, salvo que la utilidad marginal de B sea siempre 0 --y esto es algo que si se produjese, no existirá externalidad negativa y no tendría sentido el análisis-, las condiciones de primer orden del equilibrio competitivo y del óptimo de Pareto no son iguales y por tanto, el equilibrio competitivo no es socialmente eficiente. Más concretamente, se consume demasiado bien que provoca la externalidad negativa, y es posible mejorar el óptimo reduciendo la producción del bien. La modelización de una externalidad positiva es muy similar, pero en este caso la utilidad marginal del agente B que no decide sobre la producción es marginal. Ello resulta en que la producción de equilibrio competitivo es inferior a la del óptimo de Pareto y por ello, puede mejorarse aumentando la producción.

Las \textbf{soluciones a la ineficiencia} que inducen las externalidades se basan fundamentalmente en implementar medidas coactivas, instituciones y mecanismos de mercado, en grados variables. La \underline{imposición de cuotas} es la medida más sencilla. Las cuotas son medidas coactivas que restringen la producción de un agente de tal manera que se reduce o aumenta hasta que la autoridad pública considera que se ha soluciona la ineficiencia causada por la externalidad. Los \underline{impuestos pigouvianos} son aquellos que gravan una actividad con el fin de reducirla hasta el punto en el que el equilibrio de mercado se corresponde con el de equilibrio eficiente. En términos del modelo anterior de externalidades negativas, un impuesto pigouviano que indujese un óptimo de Pareto sería tal que el agente causante de la externalidad pagase un impuesto lineal por cada unidad consumida. La cuantía de este impuesto sería tal que la condición de primer orden del equilibrio competitivo se equiparase con la de óptimo de Pareto. Para ello, el impuesto por unidad consumida habría de ser igual a la desutilidad marginal que genera el consumo en el agente B que sufre la externalidad. El \underline{Teorema de Coase} muestra como es posible alcanzar el óptimo de Pareto en un contexto de externalidades negativas a partir de la creación de un mercado de derechos sobre la desutilidad provocada por las externalidades. En el contexto del modelo anterior, el gobierno podría asignar al agente B el derecho a no sufrir la externalidad. Posteriormente, B podría vender a A parte de este derecho, de tal manera que A debería pagar por consumir el bien con externalidad. La cantidad que B exigiría por esos derechos sería suficientemente baja para que A prefiriese consumir pero suficientemente alta como para que B alcanzase un bienestar mayor al que tendría respecto al equilibrio competitivo inicial. El Teorema de Coase muestra de forma general que se pueden alcanzar óptimos de Pareto a través de la asignación de este tipo de derechos a no sufrir externalidad o a provocarla. Por supuesto, los óptimos alcanzados son diferentes en función de a quién se asigne el derecho. Aunque la aplicación práctica del Teorema de Coase es hasta la fecha limitada (se ha tratado de implementar con éxito variable en los mercados de emisiones), es un resultado de gran relevancia en el plano teórico porque caracteriza el fenómeno de las externalidades como un problema de mercados incompletos. Otra solución más expeditiva pero a menudo más factible y también basada en los derechos de propiedad consiste en fusionar el proceso de decisión de los agentes causantes y víctimas o beneficiarios de la externalidad, internalizando la externalidad de manera que el problema a resolver por los agentes sea la maximización del bienestar social y no el beneficio individual del causante de la externalidad.

Los \textbf{bienes públicos} son especialmente relevantes en el diseño de políticas públicas. Un bien público se puede definir de forma simplificada como un bien sujeto a externalidades positivas en la producción. Una definición más ajustada requiere definir dos conceptos previos: excluibilidad y rivalidad. Un bien es excluible cuando es posible restringir su consumo de forma arbitraria a uno o varios agentes. En sentido contrario, un bien no es excluible cuando cualquier agente puede tener acceso al bien, independientemente del deseo del productor o de que el consumidor efectivamente abone o no un precio. Cuando un bien no es excluible, los agentes no asumen necesariamente el coste de su consumo y pueden consumir cantidades arbitrarias independientes de lo que hayan pagado. Aparece así el problema del free-riding. La rivalidad hace referencia al hecho de que el consumo de un bien por un agente precluya o no su consumo por otros agentes. Así, el consumo de una cantidad dada sobre el total de bien disponible implica que sólo podrá consumirse la cantidad restante del total. O de otra forma, que cuando un agente consume una unidad más de un bien rival, otro agente tendrá que consumir una unidad menos, y viceversa. La clasificación de Ostrom (1977) combina las propiedades de excluibilidad y rivalidad para definir cuatro categorías de bienes. Los bienes privados son excluibles y rivales. Un ejemplo paradigmático es cualquier tipo de comida. Los bienes de club son excluibles pero no rivales, de tal manera que el vendedor puede decidir arbitrariamente quien lo consume, pero una unidad más de consumo no precluye físicamente el consumo de otro agente. Un ejemplo clásico es la televisión por cable. Los bienes comunes o públicos impuros son rivales pero no excluibles. Por ejemplo, los peces de un lago en el que cualquiera puede pescar, o la madera de un bosque de libre acceso y tala. Este tipo de bienes son susceptibles de provocar la llamada \textit{tragedia de los comunes} descrita por Hardin (1968). El objeto de esta penúltima parte de la exposición, los bienes públicos, se caracterizan por no ser ni rivales ni excluibles. Cualquiera puede consumir el bien sin afectar al resto de potenciales consumidores, y no es posible establecer mecanismos efectivos para que limitar el acceso al bien a determinados consumidores. La defensa nacional, la educación ciudadana o el oxígeno atmosférico son ejemplos habituales. Desde el punto de vista de la optimalidad, el problema de la provisión de bienes públicos en un contexto de equilibrio competitivo reside en el hecho de que los agentes consideran el beneficio individual a la hora de tomar decisiones, y no el impacto sobre el beneficio social de la provisión del bien público. Así, contribuyen a la financiación del bien público hasta que el precio de una unidad adicional iguala el beneficio privado que obtienen, tomando como dadas las aportaciones del resto de agentes. Ello resulta en unas compras inferiores al nivel óptimo que maximizaría el beneficio social.

La \textbf{formulación} en términos matemáticos del problema de los bienes públicos se puede plantear en términos de equilibrio general o en términos de equilibrio parcial. En términos de equilibrio general, la condición de óptimo de Pareto es tal que la relación marginal de transformación entre el bien público y el bien privado debe igualarse con la suma de las relaciones marginales de sustitución entre bien público y privado. Esta condición se denomina habitualmente condición de Samuelson o en ocasiones, de Bowel-Lindhal-Samuelson. Para caracterizar de forma más sencilla el problema, Mas-Colell, Whinston y Green plantea un modelo de equilibrio parcial y gran simplicidad de la siguiente forma. La utilidad de un número finito de agentes depende positivamente del bien público disponible y negativamente de forma lineal de la cantidad aportada para financiar el bien público. La primera derivada de la función de utilidad respecto del bien público es positiva y la segunda es negativa. El bien público disponible corresponde a la suma de las cantidades que se pueden financiar con la aportación de los agentes. Asumiendo fijas la cantidad financiada por los demás agentes, el agente maximiza la cantidad de financiación que aporta él mismo. La condición de primer orden de equilibrio competitivo es así tal que se iguala el precio de una unidad adicional de bien público y la utilidad marginal que aporta una unidad más a un agente. Sin embargo, la condición de óptimo de Pareto es distinta. El óptimo de Pareto se puede caracterizar como el resultado de la suma de las utilidades individuales menos el coste total del bien público. La condición de primer orden de éste segundo programa de maximización es tal que la suma de las utilidades marginales debe ser igual al coste marginal de provisión del bien público. Se puede apreciar que es idéntica a la condición anteriormente señalada, teniendo en cuenta que en un contexto de utilidad cuasilineal la relación marginal de sustitución es simplemente la utilidad marginal del bien público, y la relación marginal de transformación es simplemente el coste marginal del bien público. 

Las principales \textbf{soluciones} al problema de la provisión de los bienes públicos son numerosas y han dado lugar a una amplia literatura. La \underline{provisión pública} y su financiación mediante impuestos es la solución a priori más sencilla. La autoridad pública detrae renta de los individuos y la destina a la provisión de la cantidad de bien público considerada óptima. Pero surgen varios problemas. ¿Cuánto bien público se debe proveer?, ¿cuánta renta se debe detraer a cada agente?, ¿qué distorsiones provoca la introducción de impuestos? La elección de un impuesto que minimice la distorsión corresponde a la teoría de la tributación óptima. La elección de una cantidad de bien público a proveer y la distribución entre agentes de la carga tributaria es objeto de estudio de la teoría de la elección social. Los \textbf{precios de Lindahl} son una solución teórica que intenta financiar la provisión de bien público sin medidas coactivas, a través de la creación de mercados individuales en los que la autoridad pública ofrece el bien público a cada gente a un precio individualizado. Cada precio individualizado se corresponde con la utilidad marginal que aportaría el bien público en el óptimo a cada agente considerado. La suma de estas utilidades marginales en el óptimo se correspondería con el coste marginal de producción en el nivel óptimo. Los agentes se verían así incentivados a comprar tanto bien como requiere el óptimo, y la suma de los precios que pagan por el bien público se correspondería con el coste marginal de provisión del bien público en el óptimo. Este método adolece sin embargo de un problema grave. Si se preguntase a los agentes por la utilidad marginal que les proporciona el bien público, tendrían incentivos a mentir y reportar cantidades inferiores, ya que de ello dependería el precio que enfrentan. Así, ¿cómo es posible conocer las utilidades marginales verdaderas de los agentes? Si no es posible, ¿cómo es posible que los agentes revelen sus preferencias verdaderas? A este efecto aparecen las soluciones basadas en \underline{mecanismos de revelación}. La teoría del diseño de mecanismos ha contribuido enormemente en los últimos cuarenta años al desarrollo de esta teoría. Los mecanismos de revelación tratan de superar el problema de conocer las condiciones de óptimo cuando se ignoran los ingresos marginales, los costes marginales y las utilidades marginales: ¿qué cantidad de bien público es óptima en este contexto? Para ello, el diseño de mecanismos propone configuraciones institucionales que inducen a los agentes a revelar su información, de tal manera que se compatibilice la revelación de la información necesaria con los incentivos a no revelar. Los \underline{mecanismos de Vickrey-Groves-Clark} son un una familia de mecanismos de revelación de gran relevancia teórica y de forma creciente, práctica. El objetivo es romper el vínculo entre reportar más utilidad y contribuir con más cantidad. Un ejemplo sencillo de mecanismo de VGC puede formularse en términos de la provisión de un bien público en cantidad fija, con un coste también fijo y exógenamente dado. Suponiendo que los agentes están plenamente informados del proceso a seguir, el primer paso consistiría en preguntar individualmente la valoración del bien público. A continuación se suman las cantidades reportadas y si se cubre el coste total del bien público, se compra la cantidad necesaria. Si no se cubre el coste, el bien público no se provee y el proceso finaliza. Por último, si el bien público se ha decidido comprar, se detrae a cada agente una cantidad igual al coste total del bien menos la suma de las valoraciones reportadas del resto de agentes. Así, el hecho de reportar su verdadera valoración no les penaliza, porque el pago al que deben hacer frente depende negativamente de lo reportado por otros agentes y no por ellos mismos. Aunque en la práctica este tipo de mecanismos son aún difíciles de implementar por los requisitos informacionales y las dificultades de diseño que implican, han abierto un muy fértil área de investigación con interesantes perspectivas de futuro. \underline{Otras líneas de investigación} al respecto beben de la teoría de juegos y el análisis del comportamiento estratégico de agentes a la hora de decidir cuanto bien proveer, los mecanismos de votación para decidir los niveles de bien público totales, la consideración de la localización de los bienes públicos como elemento relevante para su provisión y financiación (especialmente relevante en contexto de financiación de haciendas locales y teoría del federalismo fiscal) o el análisis de bienes públicos globales en los que la localización no es --por el contrario relevante- y no existe la posibilidad de financiar coactivamente.

Hasta ahora hemos examinado la propiedades normativas de la competencia perfecta y las consecuencias de los fallos de mercado. A la hora de proponer soluciones, hemos asumido implícitamente la existencia de un sector público que actúa de forma de acuerdo con sus fines y no está sujeto él mismo a ineficiencias y desviaciones del comportamiento esperado. En la práctica, sin embargo, los \marcar{fallos del sector público} son un fenómeno recurrente que impide que la intervención pública tenga los resultados esperados y deseados. Los \textbf{ejemplos} son múltiples. 

Las distorsiones generadas por los \underline{impuestos} son un ejemplo omnipresente. Aunque el sector público podría en teoría implementar impuestos de suma fija que bajo determinados marcos de modelización no distorsionarían el comportamiento, en la práctica los impuestos implementados acaban por distorsionar los precios relativos e introducen pérdidas de eficiencia que pueden compensar negativamente las mejoras que por otra parte induzca la intervención pública. El fenómeno del \underline{rent-seeking} es otro fenómeno habitual en relación al comportamiento del sector público. Grupos de interés dentro del sector público tienen su propias preferencias privadas que maximizar que no se corresponden necesariamente con la maximización del bienestar social. Este comportamiento desvía recursos a intereses particulares y aleja el equilibrio del óptimo social. La \underline{captura del regulador} hace referencia a la creación de vínculos personales de amistad, lealtad o tratos de favor entre reguladores y regulados que influyen sobre la práctica regulatoria. Así, pueden aparecer sesgos inconscientes a favor del regulado, intercambios de favores y decisiones regulatorias que se desvían del objetivo de maximización del bienestar social. La teoría de la burocracia y el presupuesto público de \underline{Niskanen} caracteriza a los burócratas gubernamentales como maximizadores de su propia utilidad diferente del interés general, a través de variables como la remuneración, el poder o el prestigio personal. Cuando la remuneración es fija como aproximadamente sucede en el sector público, la maximización del poder y el prestigio personal a través del aumento del presupuesto inducen resultados ineficientes. La \underline{aversión al riesgo} de los policy-makers en relación a las consecuencias personales del resultado de sus políticas puede también inducir resultados subóptimos. La \underline{ilusión fiscal} es también un fenómeno constatado que consiste en la infravaloración sistemática por los ciudadanos de la presión fiscal a la que hacen frente, de tal manera que perciben una detracción de la renta inferior a la que realmente sufren. Los gobiernos aplican técnicas que acrecientan este fenómeno tales como retenciones fiscales, imposición indirecta, sistemas tributarios opacos o preferencia por el déficit antes que subidas de impuestos para financiar aumentos del gasto público.

Las \textbf{propuestas de solución} de los fallos del sector público son un campo de estudio en si mismo, del que es interesante señalar algunas aportaciones concretas. \underline{Ligar retribución a productividad} es una solución habitualmente propuesta que tiene por objetivo alinear los incentivos privados y públicos. Sin embargo, adolece del problema de medición de la productividad y de conocer cuáles son las verdaderas preferencias de los burócratas. Los \underline{mecanismos de supervisión y control} tratan de supervisar las desviaciones respecto de la maximización del bienestar social. El problema de estos sistemas es que aumentan los costes administrativos y los supervisores tienen también una agenda propia que hay que supervisar. La \underline{individualización de responsabilidades} puede ser útil para evitar dilución de la responsabilidad y atribución a terceros de desviaciones respecto de la optimalidad. La \underline{separación entre diseño y ejecución} de políticas trata de hacer imposible la maximización de la utilidad privada. Un ejemplo de este tipo de mecanismo es la aplicación de análisis coste-beneficio a posteriori por agencias independientes. Los \underline{presupuestos de base cero} tratan de forzar un replanteamiento de cada partida de gasto de tal manera que el gestor público se pregunte si realmente contribuyen al bienestar social o son el resultado de un fallo del sector público. La \underline{privatización} de actividades públicas puede ser útil cuando el sector público provee un servicio ineficiente o innecesario resultado de un fallo del sector público como una excesiva búsqueda de poder y control de la burocracia gubernamental. La \textit{limitación legal del tamaño del sector público} trata de establecer una serie de principios básicos de muy dificil modificación que restrinjan el margen de aparición de fallos del sector público por vía de su crecimiento excesivo. 

A lo largo de la exposición hemos examinado el problema de la optimalidad de la competencia perfecta, las externalidades y los bienes públicos, y los fallos del sector público cuando trata de implementar soluciones que maximicen el bienestar social. Se trata de cuestiones que constituyen el núcleo del debate social en lo que respecta a la intervención del estado en la economía. ¿Debe el estado intervenir? ¿Cómo debe hacerlo? ¿Qué problemas genera su intervención? La exposición ha presentado los rasgos principales del análisis teórico del problema que hace la economía del bienestar. En la práctica, más allá de este marco teórico de referencia, el policy-maker debe tener en cuenta que el principal problema es la información: ¿cuáles son las verdaderas preferencias de los agentes? ¿cómo es posible llegar a conocerlas? ¿cómo incentivar a los agentes a que las revelen? ¿cómo entender correctamente la sociedad que maximiza el bienestar? Las aplicaciones concretas de este marco teórico son múltiples y de importancia capital: diseño de sistemas tributarios, diseño de mercados de emisiones, sistemas públicos de salud y educación, financiación de haciendas territoriales, cambio climático...

\seccion{Preguntas clave}
\begin{itemize}
	\item ¿Bajo qué condiciones competencia perfecta implica optimalidad de Pareto?
	\item ¿Qué teoremas se demuestran la relación entre competencia perfecta y optimalidad?
	\item ¿Puede alcanzarse cualquier equilibrio competitivo óptimo de Pareto? ¿Bajo qué condiciones?
	\item ¿Qué es una externalidad?
	\item ¿Cómo se modelizan las externalidades?
	\item ¿Qué mecanismos inducen óptimos en presencia de externalidades?
	\item ¿Qué es un bien público?
	\item ¿Qué mecanismos inducen óptimos en presencia de bienes públicos?
	\item ¿Qué son los fallos del sector público?
\end{itemize}

\esquemacorto

\begin{esquema}[enumerate]
	\1[] \marcar{Introducción}
		\2 Contextualización
			\3 Economía
			\3 Economía del bienestar
			\3 Instituciones de mercado
			\3 Competencia perfecta
		\2 Objeto
			\3 ¿Bajo qué condiciones la CP induce óptimos de Pareto?
			\3 ¿Puede alcanzarse cualquier óptimo de Pareto dadas condiciones iniciales?
			\3 ¿Qué sucede cuando no se cumplen las condiciones?
			\3 ¿Qué mecanismos pueden inducir óptimos de Pareto?
		\2 Estructura
			\3 Optimalidad de la competencia perfecta
			\3 Externalidades
			\3 Bienes públicos
			\3 Fallos del sector público
	\1 \marcar{Optimalidad de la competencia perfecta}
		\2 Idea clave
			\3 Objetivo del modelo
			\3 Equilibrio competitivo o walrasiano
			\3 Óptimo de Pareto
			\3 Optimalidad de la competencia perfecta
		\2 Formulación
			\3 Idea clave
			\3 Consumidores
			\3 Empresas
			\3 Equilibrio competitivo
		\2 Primer Teorema del Bienestar (PTB)
			\3 Idea clave
			\3 Óptimos de Pareto
			\3 Implicaciones
		\2 Segundo Teorema del Bienestar (STB)
			\3 Idea clave
			\3 Formulación
			\3 Implicaciones
			\3 Problema del cálculo socialista
	\1 \marcar{Externalidades}
		\2 Idea clave
			\3 Concepto de externalidad
			\3 Externalidades tecnológicas y pecuniarias
			\3 Externalidades positivas y negativas
			\3 Optimalidad
		\2 Formulación (consumo)
			\3 2x2
			\3 MWG
		\2 Soluciones a la ineficiencia
			\3 Idea clave
			\3 Cuotas
			\3 Impuesto pigouviano
			\3 Derechos de Propiedad: Teorema de Coase
			\3 Derechos de Propiedad: internalización de coste
	\1 \marcar{Bienes públicos}
		\2 Idea clave
			\3 Excluibilidad
			\3 Rivalidad
			\3 Clasificación de Ostrom (1977)
			\3 Optimalidad
		\2 Formulación
			\3 Equilibrio general
			\3 MWG
		\2 Soluciones a la ineficiencia
			\3 Provisión pública vía impuestos
			\3 Provisión óptima vía mercado: precios de Lindahl
			\3 Otras líneas de investigación
			\3 Leyes de Wagner: crecimiento de dda. de bien público
		\2 Valoración de bienes públicos
			\3 Experimentos
			\3 Valoración contingente
			\3 Precios hedónicos
			\3 Vida humana y tiempo*
			\3 Mecanismos de revelación
		\2 Dimensión espacial de la provisión de bienes públicos
			\3 Idea clave
			\3 Hipótesis de Tiebout (1956)
			\3 Análisis empírico de Oates (1972)
			\3 Teorema de Oates (1972)
			\3 Tamaño óptimo de zona de provisión
	\1 \marcar{Fallos del sector público}
		\2 Idea clave
			\3 Optimización del bienestar social
		\2 Ejemplos
			\3 Distorsiones que generan ineficiencias
			\3 Rent seeking
			\3 Captura del regulador
			\3 Burocracia (Niskanen)
			\3 Aversión al riesgo
			\3 Ilusión fiscal
		\2 Propuestas de solución
			\3 Retribución ligada a productividad
			\3 Mecanismos de supervisión y control
			\3 Individualización de responsabilidades
			\3 Separación entre diseño y ejecución
			\3 Presupuestos de base cero
			\3 Privatización
			\3 Limitación legal del tamaño del sector público
	\1[] \marcar{Conclusión}
		\2 Recapitulación
			\3 Optimalidad de la competencia perfecta
			\3 Externalidades
			\3 Bienes públicos
			\3 Fallos del sector público
		\2 Idea final
			\3 Debate social fundamental
			\3 Desarrollo teórico del problema
			\3 Múltiples aplicaciones

\end{esquema}

\esquemalargo













\begin{esquemal}
	\1[] \marcar{Introducción}
		\2 Contextualización
			\3 Economía
				\4 Definición de Robbins
				\4[] Economía es estudio de comportamiento humano
				\4[] $\to$ Gestionando recursos escasos con usos alternativos
				\4[] $\to$ Para satisfacer una serie de necesidades humanas
				\4 Microeconomía
				\4[] Entender y predecir
				\4[] Comportamiento de agentes individuales
				\4[] Agrupados como
				\4[] $\to$ Consumidores
				\4[] $\to$ Empresas
			\3 Economía del bienestar
				\4 Valorar diferentes estados sociales
				\4[] $\to$ Definir herramientas de valoración
				\4[] $\to$ Explicitar supuestos de comparación
				\4[] $\to$ ¿Cuándo sociedad x es preferible a y?
				\4 Corazón de la economía (Atkinson)
				\4[] Ciencia económica no existe sólo para
				\4[] $\to$ Describir comportamiento humano
				\4[] $\to$ Satisfacer curiosidad humana
				\4[] Existe sobre todo para
				\4[] $\to$ Emitir recomendaciones
				\4[] $\to$ Diseñar políticas
				\4[] $\Rightarrow$ Para mejorar sociedad
			\3 Instituciones de mercado
				\4 Regulan la producción y el intercambio voluntario
				\4 En términos de optimalidad:
				\4[] ¿a qué resultados dan lugar unas y otras?
			\3 Competencia perfecta
				\4 Marco básico de modelización microeconómica
				\4[] Caracteriza condiciones bajo las cuales:
				\4[] Agentes obtienen máximo bienestar
				\4[] $\to$ Buscando maximizar su propio interés
				\4[] $\Rightarrow$ ¿Cuándo CP induce sociedades óptimas?
				\4[] $\Rightarrow$ ¿Cuándo no induce resultados óptimos?
		\2 Objeto
			\3 ¿Bajo qué condiciones la CP induce óptimos de Pareto?
				\4 ¿Qué teoremas demuestran los resultados?
			\3 ¿Puede alcanzarse cualquier óptimo de Pareto dadas condiciones iniciales?
				\4 ¿Qué teoremas lo demuestran?
			\3 ¿Qué sucede cuando no se cumplen las condiciones?
				\4 ¿Qué es una externalidad?
				\4 ¿Qué es un bien público?
			\3 ¿Qué mecanismos pueden inducir óptimos de Pareto?
				\4 ¿Qué mecanismos de mercado pueden introducirse?
				\4 ¿Qué papel puede jugar el sector público?
				\4 ¿Qué problemas tiene la utilización del sector público?
		\2 Estructura
			\3 Optimalidad de la competencia perfecta
			\3 Externalidades
			\3 Bienes públicos
			\3 Fallos del sector público
	\1 \marcar{Optimalidad de la competencia perfecta}
		\2 Idea clave
			\3 Objetivo del modelo
				\4 Mostrar condiciones bajo las cuales
				\4[] la competencia perfecta induce a óptimos
				\4 Competencia perfecta entendida como
				\4[] Situación en la que agentes
				\4[] $\to$ Son precio aceptantes
				\4[] $\to$ Se comportan como si no pudiesen afectar a precios
				\4 Doble filo como defensa del libre mercado
				\4[] $\to$ Demostrar que competencia induce óptimos
				\4[] $\to$ Explicita supuestos restrictivos necesarios
				\4 Adam Smith:
				\4[] Esfuerzo individual contribuye a máximo beneficio social
				\4[] Individuos en general no lo pretenden
				\4[] $\to$ Mano invisible induce fin no pretendido
				\4 Términos técnicos:
				\4[] Mostrar condiciones bajo las cuales:
				\4[] $\to$ Equilibrios walrasianos son óptimos de Pareto
			\3 Equilibrio competitivo o walrasiano
				\4 Dado:
				\4[] Una serie de bienes de consumo
				\4[] Un conjunto de agentes con unas dot. de bienes
				\4 Un Equilibrio Walrasiano es:
				\4[] Una asignación de consumo
				\4[] $\to$ Cuánto de cada bien consume cada consumidor
				\4[] Un vector de precios relativos de los bienes
				\4 Que cumplen:
				\4[] i) Asignaciones solucionan problema de máx.
				\4[] ii) Suma de asignaciones = suma de dotaciones
			\3 Óptimo de Pareto
				\4 Óptimo débil de Pareto (ODP)
				\4[] Un asignación es ODP
				\4[] $\to$ No existe asignación que mejore a todos
				\4 Óptimo fuerte de Pareto (OFP)
				\4[] Una asignación es OFP
				\4[] $\to$ No existe asignación que mejore al menos a uno
				\4[] $\to$ Manteniendo al resto indiferente
				\4 OFP $\then$ ODP
				\4 ODP $\nRightarrow$ OFP
			\3 Optimalidad de la competencia perfecta
				\4 Competencia perfecta
				\4[] ¿Induce óptimos de Pareto?
				\4[] ¿Pueden alcanzarse todos los óptimos de Pareto?
				\4 Necesario demostrar
				\4[] $\to$ Explicitar supuestos necesarios
				\4[] $\to$ Caracterizar marco de análisis formal
				\4 NO implica deseabilidad del equilibrio
				\4[] Sólo optimalidad paretiana
				\4[] Óptimos Paretianos no siempre deseables
				\4[] $\to$ P.ej.: todo para un agente, nada para resto
		\2 Formulación
			\3 Idea clave
				\4 2x2x2x2
				\4[] $\to$ 2 consumidores
				\4[] $\to$ 2 productores
				\4[] $\to$ 2 bienes de consumo
				\4[] $\to$ 2 factores de producción
				\4 Generalizable a $n$
				\4 Caracterizar condiciones de eq. competitivo
				\4[$\to$] Comparar con condiciones de óptimo de Pareto
				\4[$\then$] Si son iguales, CP induce OPareto
			\3 Consumidores
				\4 2 consumidores: $A$, $B$
				\4[] $U_A = u_a(x_a, y_a)$
				\4[] $U_B = u_b(x_b, y_b)$
				\4 Restricciones presupuestarias
				\4[] $p_x \cdot x_a + p_y \cdot y_a \leq w_A$
				\4[] $p_x \cdot x_b + p_y \cdot y_b \leq w_B$
				\4 Problema de maximización
				\4[] $\underset{x_i, y_i}{\max} \quad u_i(x_i, y_i)$
				\4[] \quad $\text{s.a}: \quad p_x \cdot x_i + p_y \cdot y_i \leq w_i$
				\4 Solución
				\4[] $\mathcal{L} = u_i(x_i, y_i - \lambda (p_x \cdot x_i + p_y \cdot y_i - w_i)$
				\4[] $\mathcal{L}_{x_i} = \pdv{u_i}{x_i} - \lambda p_x = 0$
				\4[] $\mathcal{L}_{y_i} = \pdv{u_i}{x_i} - \lambda p_y = 0$
				\4[] $\then$ \fbox{$\text{CPO:} \quad \frac{p_x}{p_y} = \frac{\partial u_i / \partial x_i}{\partial u_i / \partial y_i} = |\text{RMS}_{xy}^i| = |\text{RMS}^j_{xy}|$}
			\3 Empresas
				\4 2 empresas: $X$, $Y$
				\4[] $X = x(L_X, K_Y)$
				\4[] $Y = y(L_Y, K_Y)$
				\4 Maximización del beneficio
				\4[] $\underset{L_x, K_x}{\max} \quad \pi_x = p_x \cdot X(L_X, K_X) - w L_x - r K_x$
				\4[] $\underset{L_y, K_y}{\max} \quad \pi_y = p_y \cdot Y(L_Y, K_Y) - w L_y - r K_y$
				\4 Solución del problema de maximización
				\4[] CPO:
				\4[] $p_x \pdv{X}{L_X} - w = 0$
				\4[] $p_x \pdv{X}{K_x} - r = 0$
				\4[] $p_y \pdv{Y}{L_Y} - w = 0$
				\4[] $p_Y \pdv{Y}{K_Y} - r = 0$
				\4[] $\then \frac{w}{r} = \frac{\text{PMgL}_X}{\text{PMgK}_X} = \frac{\text{PMgL}_Y}{\text{PMgK}_Y}$
				\4[] $\left| \text{RMST}_{LK}^X \right| = \left| \text{RMST}_{LK}^Y \right| = \frac{w}{r}$
				\4[] $\then \text{CMgX} = \frac{r}{\text{PMgK}_X} = \frac{w}{\text{PMgL}_x}$
				\4[] $\then \text{CMgY} = \frac{r}{\text{PMgK}_Y} = \frac{w}{\text{PMgL}_Y}$
				\4[] $\then$ \fbox{$ | \text{RMT}_{xy} | = \frac{-d \, Y}{d \, X} = \frac{ \text{CMgX} }{\text{CMgY} } = \frac{p_x}{p_y}$}
			\3 Equilibrio competitivo
				\4 $(x_A, y_A, x_B, y_B), (p_x, p_y) (L_X, K_X, L_Y, K_X), (w,r)$:
				\4[] \fbox{$\left| \text{RMS}_{xy}^A \right| = \left| \text{RMS}_{xy}^B \right| = \left| \text{RMT}_{XY} \right| $}
		\2 Primer Teorema del Bienestar (PTB)
			\3 Idea clave
				\4 Todos los ECompetitivos son óptimos de Pareto
				\4[] Cuando se cumplen los supuestos:
				\4[] $\to$ Ausencia de externalidades
				\4[] $\to$ Preferencias no saturadas
				\4[] $\to$ Preferencias convexas\footnote{No es necesario para el cumplimiento del Primer TFB. Pero es conveniente por razones técnicas para que las condiciones de primer orden sean también suficientes. V. pág. 326 de MWG.}
				\4[] $\to$ Función de producción convexa\footnote{Ídem.}
				\4 Demostración
				\4[] Diferentes estrategias
				\4[] Demostración basada en cálculo diferencial
				\4[] $\to$ Basado en contexto anterior
			\3 Óptimos de Pareto
				\4 Conjunto de asignaciones óptimo de Pareto
				\4[] Maximizando utilidad de A
				\4[] $\to$ Dada utilidad mínima de B
				\4[] $\underset{x_A,y_A, L_X, K_X, L_Y, K_Y}{\max} \quad u_A(x_A, y_A)$
				\4[] $\text{s.a:} \quad u_B(x_B, y_B) \geq u_0$
				\4[] $\quad \quad x_A + x_B \leq X(L_X, K_X)$
				\4[] $\quad \quad y_A + y_B \leq Y(L_Y, K_Y)$
				\4[] $\quad \quad L_X + L_Y = \bar{L}$
				\4[] $\quad \quad K_X + K_Y = \bar{K}$
				\4 Asignaciones óptimas cumplen:
				\4[] $\Rightarrow$ \fbox{$ \left| \text{RMS}_{xy}^A \right| = \left| \text{RMS}_{xy}^B \right| = \left| \text{RMT}_{XY} \right| $}
				\4[$\then$] Mismas condiciones de óptimo que ECompetitivo
				\4[$\then$] ECompetitivos son óptimo de Pareto
			\3 Implicaciones
				\4 PTB no implica que se alcance óptimo
				\4[] Sólo que los EC que existen
				\4[] $\to$ Son óptimos
				\4 Alcanzar óptimos
				\4[] Depende de estabilidad del sistema
				\4[] Demostración diferente
				\4 Libre mercado y competencia
				\4[] PTB es arma de doble filo
				\4[] Utilizado a favor y en contra
				\4[] $\to$ Competencia puede inducir eficiencia
				\4[] $\to$ Necesarios supuestos muy restrictivos para eficiencia
		\2 Segundo Teorema del Bienestar (STB)
			\3 Idea clave
				\4 Dadas:
				\4[] Preferencias convexas y no saturadas
				\4[] Conjunto de producción convexo
				\4 Es posible alcanzar cualquier Óptimo de Pareto
				\4[] $\to$ Como un equilibrio competitivo
				\4[] $\then$ Mediante un impuesto de suma fija a cada agente
				\4 Todos los óptimos de Pareto
				\4[] Pueden alcanzarse con transferencias adecuadas
				\4[] \grafica{segundoteoremadelbienestar}
			\3 Formulación
				\4 Supuestos:
				\4[i] Preferencias y conjunto de prod. convexos
				\4[ii] Preferencias no saturadas
				\4[iii] Mercados completos con precios conocidos
				\4[iv] Agentes son precio aceptantes
				\4 Teorema
				\4[] Cumpliéndose i-iii
				\4[] Todos los óptimos de Pareto pueden alcanzarse
				\4[] $\to$ En contexto de equilibrio competitivo
				\4[] $\to$ Dada una reasignación de las dotaciones
				\4[] $\to$ En forma de impuesto de suma fija
			\3 Implicaciones
				\4 Posible separar eficiencia y equidad
				\4[] Interpretación simplificada habitual
				\4[] $\to$ Reasignar dotaciones
				\4[] $\then$ Dejar a mercado trabajar
				\4[] Dictador benévolo que no quiera comunismo
				\4[] $\to$ Reasigna dotaciones
				\4[] $\then$ Deja que se alcancen
				\4[] Realmente interpretación errónea
				\4[] $\to$ STFB no implica eq. deseado se alcance
				\4 No implica alcanzar equilibrio deseado\footnote{Ver Bryant (1994).}
				\4[] Implica que \textbf{puede} alcanzarse
				\4[] $\to$ Pero no que se alcance necesariamente
				\4[] ¿Por qué no?
				\4[] $\to$ Una dotación inicial puede dar lugar
				\4[] $\then$ A diferentes equilibrios
				\4[] $\then$ Posibles factores más allá de modelo
				\4 Identificar óptimo preferido
				\4[] Puede separarse de problema de eficiencia
				\4[] $\then$ Teoría de la decisión colectiva
			\3 Problema del cálculo socialista
				\4 Hayek, Von Mises y otros
				\4[] Precede a formulación de STFB
				\4 Crítica de planificación económica
				\4[] No hay información suficiente ni la habrá
				\4[] $\to$ Para conocer asignación óptima
				\4[] $\to$ Para conocer resultado de reasignación
				\4 Interpretación socialista de STFB
				\4[] Posible conocer dotaciones y eq. final
				\4[] $\to$ Posible planificar resultado
				\4[] $\to$ Aprovechando mercados
				\4 Crítica a interpretación socialista
				\4[] No hay información suficiente
				\4[] Es simplemente imposible conocer
				\4 Se añaden otros problemas
				\4[] Crítica de Lucas
				\4[] $\to$ ¿Estamos teniendo en cuenta reacción de agentes?
	\1 \marcar{Externalidades}
		\2 Idea clave
			\3 Concepto de externalidad
				\4 Efecto del comportamiento de un agente
				\4[] sobre comportamiento de un tercero
			\3 Externalidades tecnológicas y pecuniarias
				\4 Externalidades pecuniarias actúan
				\4[] a través de sistema de precios
				\4[] P.ej:
				\4[] $\to$ Agente compra whiskey
				\4[] $\to$ Sube el precio del whiskey
				\4[] $\to$ Otro agente debe consumir menos whiskey
				\4 Externalidades tecnológicas
				\4[] Actúan directamente sobre
				\4[] $\to$ Funciones de utilidad
				\4[] $\to$ Conjuntos de producción
				\4[] $\to$ Dotaciones
				\4[] P.ej:
				\4[] $\to$ Empresa contamina manantial de río
				\4[] $\then$ Pescadores río abajo pescan menos peces
				\4[] P.ej:
				\4[] $\to$ Agente se vacuna y no puede transmitir virus
				\4[] $\then$ Otro agente reduce probabilidad de infectarse
				\4[] Formalización de las externalidades
				\4[] Variables en funciones de utilidad o producción
				\4[] $\to$ Un agente optimiza respecto de variable
				\4[] $\to$ Otro sufre/beneficia sin poder maximizar
			\3 Externalidades positivas y negativas
				\4 Positivos
				\4[] $\to$ Más utilidad dada cesta
				\4[] $\to$ Mayor producción dados inputs
				\4 Negativos
				\4[] $\to$ Menor utilidad dada cesta
				\4[] $\to$ Menor producción dados inputs
			\3 Optimalidad
				\4 PTB y STB requieren ausencia de externalidades
				\4[] ECompetitivos con externalidades
				\4[] $\to$ No son óptimos de Pareto
				\4[] Agentes ecualizan:
				\4[] $\to$ Costes y beneficios privados
				\4[] Con externalidades es necesario:
				\4[] $\to$ Ecualizar costes y beneficios sociales
				\4 Objetivo:
				\4[] $\to$ Caracterizar óptimos de Pareto con externalidades
				\4[] $\to$ Mostrar que no coinciden
				\4[] $\to$ Exponer mecanismos que inducen óptimos
		\2 Formulación (consumo)
			\3 2x2
			\3 MWG\footnote{Sección 11.B, pág. 351.}
				\4 $h$: bien con efectos externos
				\4 Asumimos precios demás bienes no varían
				\4 Agente A decide cuánto bien h
				\4 Utilidad de agente $i$:
				\4[] $u_i(w_i, h) = \phi_i(h) + w_i$
				\4[] $\to$ Sin efecto renta
				\4 Agente A decide cuanto bien $h$
				\4[] $\to$ Más $h$ le genera efecto positivo
				\4[] $\to$ $\phi_A'(h) > 0$, $\phi_A''(h) < 0$
				\4[] $\to$ $\exists \; h \; / \; \phi_A'(h) = 0$
				\4 Agente B no decide cantidad de $h$
				\4[] $\to$ Pero sufre efecto negativo
				\4[] $\to$ $\phi_B'(h) < 0$ \quad $\phi_B''(h) < 0$
				\4[] $\to$ $\phi_B(0) = 0$
				\4 Equilibrio competitivo:
				\4[] $\to$ máx. de la utilidad de A
				\4[] $\underset{h}{\max} \quad \phi_A(h)+w_i$
				\4[] \fbox{ $\text{CPO:} \quad \phi_A'(h^*) = 0$}
				\4 Óptimo de Pareto: máx. del bienestar social
				\4[] $\underset{h}{\max} \quad \phi_A(h) + \phi_B(h)$
				\4[] \fbox{ $\text{CPO:} \quad \phi_A'(h) + \phi_B'(h) = 0$ }
				\4[] $\Rightarrow \quad \quad  \phi_A'(h^0) = - \phi_B'(h^0)$
				\4 Distintas CPO de ECompetitivo y óptimo de Pareto
				\4[] Salvo que:
				\4[] $\phi_A'(h) = -\phi_B'(h) = 0$
				\4[] No se produce por $\phi_B(0)=0$ y $\phi_B'(h) < 0$
				\4[] $\Rightarrow$ $h^* > h^0$
				\4[] $\Rightarrow$ Se consume demasiado bien con externalidad
				\4[] $\Rightarrow$ Posible mejorar óptimo reduciendo externalidad
				\4[] \grafica{externalidadnegativa}
				\4 Externalidades positivas
				\4[] Misma representación
				\4[] Pero $\phi_B'(h) > 0$
				\4[] $\Rightarrow$ $h^* < h^0$
		\2 Soluciones a la ineficiencia
			\3 Idea clave
				\4 Implementar:
				\4[] $\to$ Medidas coactivas
				\4[] $\to$ Instituciones
				\4[] $\to$ Mecanismos de mercado
				\4[] Que induzcan nivel de bien $h$
				\4[] $\to$ Que satisfaga óptimo de Pareto
			\3 Cuotas
				\4 Gobierno limita producción directamente
				\4[] Produce directamente bien con externalidad
				\4[] $\then$ Internaliza problema de máx. bienestar social
			\3 Impuesto pigouviano
				\4 Transformar condición de equilibrio competitivo
				\4[] $\to$ En condición de óptimo social
				\4 Imponer impuesto en problema de máx.
				\4[] $\to$ Qué iguale condiciones de primer orden
				\4[] $\then$ Agente A internaliza max. bienestar social
				\4 Formulación (MWG)
				\4[] Imponer impuesto T a máx. de A tal que:
				\4[] $\underset{h}{\max} \quad u_A = \phi_A(h) - t_h \cdot h $
				\4[] \fbox{$\text{CPO:} \quad \quad \phi_A'(h) = t_h$}
				\4[] Elegir $t_h$ de óptimo tal que:
				\4[] $\to$ Se igualen CPOs competitiva + impuesto y óptimo
				\4[] $\to$ $t_h = \phi_B'(h^0)$
				\4[] $\Rightarrow$ Impuesto igual a UMg de B en óptimo social
				\4[] \grafica{impuestopigouviano}
			\3 Derechos de Propiedad: Teorema de Coase
				\4 Idea clave
				\4[] Asignación de derechos sobre externalidad
				\4[] + creación de un mercado de derechos
				\4[] + ausencia de restricciones
				\4[] $\to$ ÓPareto para cualquier reparto de derechos
				\4 Ejemplo (MWG)
				\4[] Agente B tiene derecho a no sufrir externalidad
				\4[] A debe pagar a B si quiere consumir $h$
				\4[] B exige cantidad T a A por consumir cantidad $h$
				\4[] $\to$ B resuelve problema de maximización:
				\4[] $ \underset{h,T \geq T}{\max} \quad \phi_B(h) + T$
				\4[] $\text{s.a:} \quad \phi_A(h) - T \geq \phi_A(0)$
				\4[] La restricción es vinculante: $\phi_A(h) -T = \phi_A(0)$
				\4[] $\to$ Problema se convierte en:
				\4[] $\underset{h \geq 0}{\max} \quad \phi_B(h) + \phi_A(h) - \phi_A(0)$
				\4[] $\text{CPO:} \quad \phi_A'(h) = - \phi_B'(h)$
				\4[] $\Rightarrow$ Misma condición de óptimo de Pareto
				\4[] Diferentes asignaciones de derechos de propiedad
				\4[] $\to$ Inducen diferentes óptimos
				\4[] $\to$ Inducen diferentes puntos en FPU
				\4[] $\Rightarrow$ Pero siguen siendo óptimos de Pareto
				\4[] $\Rightarrow$ Impuestos de suma fija pueden modificar
				\4 Implicaciones
				\4[] Externalidad que no da lugar a optimalidad
				\4[] Puede entenderse como resultado de:
				\4[] $\to$ Inexistencia de un mercado
				\4[] $\Rightarrow$ Optimalidad requiere mercados completos
			\3 Derechos de Propiedad: internalización de coste
				\4 Agentes que producen y experimentan externalidad
				\4[] Se integran en una sola unidad de decisión
				\4[] $\to$ Decisión es maximización de bienestar social
	\1 \marcar{Bienes públicos}
		\2 Idea clave
			\3 Excluibilidad
				\4 Posibilidad de restringir consumo a determinados agentes
				\4[] Característica fundamental de los bienes privados
				\4 Qué sucede cuando bien no es excluible?
				\4[] $\Rightarrow$ Agentes no asumen coste de su consumo
				\4[] $\Rightarrow$ Agentes consumen más de lo que pagan
				\4[] $\Rightarrow$ Posible el free-riding
			\3 Rivalidad
				\4 Consumo de un bien por un agente
				\4[] ¿precluye el consumo por parte de otro agente?
				\4 ¿Qué sucede cuando un bien no es rival?
				\4[$\Rightarrow$] Consumo por un agente no afecta a otros
			\3 Clasificación de Ostrom (1977)
				\4[] \grafica{clasificaciondeostrom}
				\4 Bienes privados
				\4[] $\to$ Excluibles
				\4[] $\to$ Rivales
				\4[] Ejemplo:
				\4[] $\to$ Comida
				\4 Bienes de club
				\4[] $\to$ Excluibles
				\4[] $\to$ No rivales
				\4[] Ejemplo:
				\4[] $\to$ Televisión por cable
				\4 Bienes comunes o públicos impuros
				\4[] No excluibles
				\4[] Rivales
				\4[] Ejemplo:
				\4[] $\to$ Peces en un lago
				\4[] $\to$ Madera de un bosque
				\4[] $\Rightarrow$ Tragedia de los comunes
				\4 Bienes públicos
				\4[] No excluibles
				\4[] No rivales
				\4[] $\to$ Defensa nacional
			\3 Optimalidad
				\4 Consumidores consideran beneficio individual
				\4[] $\to$ No beneficio social
				\4[] Compran bien público
				\4[] $\to$ hasta que precio iguala beneficio individual
				\4[] $\to$ Tomando como dadas compras de otros agentes
				\4[] $\then$ Agentes no tienen en cuenta beneficio social
				\4[] $\then$ Infraprovisión del bien público
				\4[] $\then$ Compra total de bien público inferior a óptimo
				\4 Compra de bien público de óptimo social
				\4[] Maximización tiene en cuenta beneficio total
				\4[] $\to$ Del conjunto de agentes
				\4[] $\to$ No sólo de un agente
		\2 Formulación
			\3 Equilibrio general
				\4[] $\to$ Condición de Samuelson\footnote{En ocasiones denominada condición de Bowen o incluso condición de Bowen-Lindahl-Samuelson.}
				\4 \fbox{$ \text{CPO:} \quad \left| \text{RMT}_{x,G} \right| = \sum_i \left| \text{RMS}^i_{x,G} \right|$}
				\4[] \grafica{condiciondesamuelson}
			\3 MWG\footnote{Sección 11.C Pág. 363.}
				\4 Bien $q$
				\4 Utilidades de A y B dependen de mismo bien $x$
				\4[] $u_i(x) = \phi_i(x)$
				\4[] $\to$ $\phi_i'(x) > 0$ \quad $\phi_i''(x) < 0$
				\4 Equilibrio competitivo: máx. utilidad individual
				\4[] Cada agente decide cuanto $x_i$ comprar
				\4[] $\to$ Aunque utilidad dependa de $x$ total
				\4[] $\to$ Tomando como dadas resto de $x_{-i}$
				\4[] $\underset{x_i\geq 0}{\max} \quad \phi_i \left( x_i + \sum_{k\neq i} x_k^* \right) - p x_i$
				\4[] $\then$ \fbox{$\text{CPO:} \quad \phi_i'\left( x_i^* + \sum_{k\neq i} x_k^* \right) = p$}
				\4[] $\to$ Utilidad marginal individual se iguala a precio
				\4[] $\to$ Precio depende positivamente de cantidad
				\4 Óptimo de Pareto: máx. del bienestar social
				\4[] $\underset{x}{\max} \quad \sum_{i=0}^I \phi_i (x) - c(x)$
				\4[] \fbox{$\text{CPO:} \quad \sum_{i=0}^I \phi_i'(x) = c'(x)$}
				\4[] $\to$ \underline{Condición de Samuelson}
				\4[]\grafica{bienespublicos}
		\2 Soluciones a la ineficiencia
			\3 Provisión pública vía impuestos
				\4 Autoridad pública extrae renta
				\4[] $\to$ La dedica a producir bien público óptimo
				\4 Problema:
				\4[] ¿Cuánto bien público proveer?
				\4[] ¿Cuánta renta detraer a cada agente?
				\4[] ¿Qué distorsiones aparecen?
				\4[] $\Rightarrow$ Problema de la elección social
			\3 Provisión óptima vía mercado: precios de Lindahl
				\4 Alcanzar equilibrio sin medidas coactivas
				\4[] $\to$ Sin detraer renta vía impuestos
				\4 Crear mercado para cada consumidor
				\4[] Ofrecer bien público a precio tal que:
				\4[] $\to$ Precio iguale utilidad marginal en óptimo
				\4[] $p_i^* = \phi_i'(q^0)$
				\4[] $\Rightarrow$ Suma de bien público demandado es cantidad óptima
				\4 Problema:
				\4[] ¿Cómo conocer utilidades marginales individuales?
				\4[] ¿Cómo hacer que agentes revelen sus preferencias?
				\4[] $\to$ Agentes tienen incentivos a mentir sobre UMg
			\3 Otras líneas de investigación
				\4 Teoría de juegos
				\4[] Comportamiento estratégico entre agentes
				\4[] $\to$ Cuando deciden cuanto contribuir
				\4[] Especialmente relevante con n. reducido de consumidores
				\4 Decisión mediante votación
				\4[] Si agentes votan cuanto bien proveer
				\4[] $\to$ ¿El equilibrio es óptimo?
				\4[] $\to$ ¿Cómo influyen los diferentes procesos políticos?
				\4 Localización de los bienes públicos
				\4[] Formulación de Samuelson
				\4[] $\to$ Localización de agentes no es relevante
				\4[] $\Rightarrow$ BPúblico disponible para todos
				\4[] Hipótesis de Tiebout (1956):
				\4[] $\to$ los bienes públicos pueden ser locales
				\4[] $\Rightarrow$ Sólo se disfrutan en determinada localización
				\4[] $\Rightarrow$ Movilidad/migración revela pref. por BPúblico
				\4 Bienes públicos globales
				\4[] Localización no es relevante
				\4[] $\to$ Todo el planeta se beneficia de forma similar
				\4[] ¿Cómo financiar sin coacción?
				\4[] ¿El mundo está en la frontera de posibilidades de producción?
				\4[] Si existe aversión a la desigualdad
				\4[] $\to$ ¿Todos los países deben contribuir igual?
			\3 Leyes de Wagner: crecimiento de dda. de bien público
				\4 Idea clave
				\4[] Desarrollo económico aumenta demanda de BPúblico
				\4 Formulación
				\4[] Primera ley de Wagner
				\4[] $\to$ Industrialización aflora fallos de mercado
				\4[] $\then$ Industrialización aumenta dda. de bien público
				\4[] $\then$ Mayor complejidad aumenta dda. de seguridad
				\4[] Segunda ley de Wagner
				\4[] $\to$ BPúblicos son elásticos a la renta
				\4[] $\to$ BPúblicos como bienes de lujo
				\4[] $\then$ Industrialización aumenta renta y dda. de BP
				\4 Valoración
				\4[] Ajustan razonable de evidencia empírica
				\4[] No explica consolidaciones fiscales\footnote{Vaya morcillaca de Sahuquillo.}
				\4[] Asume provisión pública necesaria
				\4[] No fundamenta preferencia por bienes públicos
				\4[] Correlación PIB-Bien público no tiene por qué ser causal
		\2 Valoración de bienes públicos
			\3 Experimentos
				\4 Vernon Smith y otros
				\4 Diseñar esquema de remuneración
				\4 Exponer a incertidumbre
				\4 Observar decisiones
			\3 Valoración contingente
				\4 ¿Qué disposición a...
				\4[] ...pagar para llevar a cabo el proyecto?
				\4[] ...recibir a cambio de aceptar el proyecto?
				\4 Marco teórico básico
				\4[] Teoría de la dualidad y preferencia revelada
				\4[] Variación equivalente y compensatoria
				\4 Encuestas
				\4[] Instrumento de revelación
				\4[] $\to$ ¿Incentivos a revelar?
				\4[] $\to$ ¿Qué van a querer revelar?
				\4 Variación compensatoria
				\4[] Cuánta renta detraer para inducir situación inicial
				\4[] $\to$ Tras haberse producido el cambio
				\4 Variación equivalente
				\4[] Cuánta renta transferir para inducir situación final
				\4[] $\to$ Sin que se produzca el cambio
				\4 Revelación de información
				\4[] Diseño de mecanismos
				\4[] Mecanismo de Vickrey-Groves-Clark
			\3 Precios hedónicos
				\4 Estimación de características
				\4[] A partir de precios y datos sobre características
				\4 Marco teórico
				\4[] Demanda de características de Lancaster
				\4 Marco empírico
				\4[] Regresiones de sección cruzada
			\3 Vida humana y tiempo*
				\4 Enfoque contable
				\4[] $V_\text{vida} = \sum_{t=0}^T \frac{A_t}{(1+d)^t}$
				\4[] $A_t$: aportación social de la persona
				\4[] $T$: años hasta muerte
				\4 Enfoque político
				\4[] Schelling (1968)
				\4[] A partir de valor implícito de decisiones políticas
				\4[] Ejemplo:
				\4[] $\to$ A es 100 millones más caro que B
				\4[] $\to$ A salva 5 vidas más que B
				\4[] $\then$ Valor implícito es 20 millones
				\4[] Grado importante de arbitrariedad
				\4[] $\to$ ¿Cuánto está dispuesto a pagar la sociedad por una vida?
				\4 Enfoque actuarial
				\4[] Basado en primas de seguros
				\4[] $\to$ ¿Valor implícito de reducir probabilidad de muerte?
				\4[] Ejemplo:
				\4[] $\to$ Seguro salud implica $1,5\%$ de probabilidad de muerte
				\4[] $\to$ Cuesta 1000 euros
				\4[] $\then$ $\frac{1000 €}{1.5}\cdot 100=150.000 €$ por muerte
			\3 Mecanismos de revelación
				\4 Conocemos condiciones que dan lugar a óptimos
				\4[] Pero a menudo ignoramos información necesaria
				\4[] $\to$ Ingresos marginales
				\4[] $\to$ Costes marginales
				\4[] $\to$ Utilidades marginales
				\4[] $\Rightarrow$ ¿qué cantidad óptima de bien público es óptima?
				\4 ¿Cómo inducir agentes a revelar información?
				\4[] Diseñar mecanismos que compatibilicen
				\4[] $\to$ Revelación de información necesaria
				\4[] $\to$ Incentivos a no revelar de los agentes
				\4 Mecanismos de Vickrey-Groves-Clark\footnote{Ver Spiegel.}
				\4[] Familia de mecanismos de revelación
				\4[] Romper vínculo entre:
				\4[] $\to$ Reportar más utilidad
				\4[] $\to$ Contribuir con más cantidad
				\4[] Ejemplo:
				\4[] Necesario financiar compra de bien público discreto
				\4[] $\to$ Coste es cantidad $c$
				\4[] 1. Preguntar a agentes valoraciones $b_i$
				\4[] 2. Sumar cantidades reportadas
				\4[] $\to$ Si cubren coste, comprar bien público
				\4[] $\to$ Si no cubren coste, no se compra bien público
				\4[] 3. Detraer a agente $i$: $c - \sum_{j\neq i} b_j$
				\4[] $\to$ I.e.: coste menos suma de valoraciones ajenas
				\4[] $\Rightarrow$ Estrat. dominante: reportar beneficio verdadero
				\4[] Reportar verdadero no les penaliza
				\4[] $\to$ Pago depende de lo que reporten otros
				\4[] Reportar menos les penaliza si no se cubre coste
				\4[] $\to$ No disfrutan del bien público
				\4[] Valoración:
				\4[] $\to$ Abre campo de investigación teórico
				\4[] $\to$ Muy costoso a nivel administrativo
				\4[] $\Rightarrow$ Generalmente inviable en la práctica
		\2 Dimensión espacial de la provisión de bienes públicos
			\3 Idea clave
				\4 Bienes públicos tienen dimensión espacial
				\4[] Disfrutan agentes en determinados localizaciones
				\4 Provisión pública de bienes públicos
				\4[] Vía impuestos y gasto público
				\4[] $\to$ Importante porcentaje de BP provistos
				\4 Federalismo fiscal
				\4[] Teoría de la estructura vertical del estado
				\4[] $\to$ ¿A qué nivel proveer qué?
				\4 Provisión de BP homogéneo en todo el estado
				\4[] Posible ineficiencia
				\4[] $\to$ Diferentes preferencias respecto BPúblico
				\4[] $\to$ Bien público provisto homogéneamente
				\4[] $\then$ Ineficiencia en la provisión
				\4 Descentralización fiscal
				\4[] Cada jurisdicción puede decidir cuanto proveer
				\4 Objetivos
				\4[] Caracterizar solución a problemas de:
				\4[] $\to$ Revelación de prefs. en contexto espacial
				\4[] $\to$ Agregación en contexto espacial
			\3 Hipótesis de Tiebout (1956)
				\4 Propuesta de solución a revelación y agregación
				\4 Bajo supuestos realistas
				\4[] Provisión local puede ser óptima
				\4[] $\to$ Preferible a provisión homogénea por nivel superior
				\4 Supuestos
				\4[] i. Agentes perfectamente móviles entre áreas
				\4[] ii. Diferentes áreas proveen diferentes niveles
				\4 Desplazamiento de agentes
				\4[] Donde BP se provee a nivel deseado
				\4[] $\to$ Agentes votan con los pies
				\4 Equilibrio
				\4[] Todos los votantes se desplazan
				\4[] $\to$ A niveles preferidos
				\4[] Todos los agentes son votante mediano en su juris.
				\4[] $\to$ Comunidades homogéneas
				\4[] $\then$ Provisión óptima de bien público
				\4 Argumentos en contra
				\4[] Costes de movilidad
				\4[] Congestión tras movilidad
				\4[] Otros factores que determinan localización
				\4[] $\to$ Pueden hacer ineficiente movimiento
				\4[] Economías de escala en provisión
				\4[] $\to$ Dispersión puede reducir
			\3 Análisis empírico de Oates (1972)
			\3 Teorema de Oates (1972)
			\3 Tamaño óptimo de zona de provisión
	\1 \marcar{Fallos del sector público}
		\2 Idea clave
			\3 Optimización del bienestar social
				\4 En modelizaciones de resultados de mercado:
				\4[] Empresas y consumidores maximizan su bienestar
				\4[] Objetivo es maximizar bienestar social
				\4[] ¿Coinciden...
				\4[] $\to$ Óptimo social
				\4[] $\to$ y resultado de optimización individual?
				\4 Soluciones propuestas asumen:
				\4[] Policy-maker busca maximizar óptimo social
				\4[] $\to$ Pero no necesariamente
				\4[] $\to$ Policy-maker puede maximizar propio beneficio
				\4[] $\Rightarrow$ No coincidir con óptimo
				\4[] Solución efectivamente maximiza bienestar social
				\4[] $\to$ Pero en la práctica existen distorsiones
		\2 Ejemplos
			\3 Distorsiones que generan ineficiencias
				\4 Impuestos distorsionantes
				\4[] Alteran precios relativos
				\4 Ganancias de eficiencia por intervención pública
				\4[] $\to$ Pueden compensarse negativamente
			\3 Rent seeking
				\4 Grupos de interés
				\4[] $\to$ influyen en decisión de poder público
				\4[] Objetivo:
				\4[] $\to$ Favorecer sus interes privados
				\4 Crean ineficiencias por dos motivos:
				\4[] $\to$ Recursos se desvían a intereses particulares
				\4[] $\to$ Decisiones no aumentan bienestar social
			\3 Captura del regulador
				\4 Contacto personal entre regulador y regulado
				\4[] Aparecen vínculos de:
				\4[] $\to$ Amistad
				\4[] $\to$ Lealtad
				\4[] $\to$ Trato de favor
				\4[] Afectan a decisiones del regulador:
				\4[] $\to$ Sesgo inconsciente en favor de regulado
				\4[] $\to$ Intercambio de favores
				\4[] $\Rightarrow$ Decisiones no óptimas
			\3 Burocracia (Niskanen)
				\4 Burócratas gubernamentales
				\4[] $\to$ Maximizan su propia función de utilidad
				\4 No tiene por qué coincidir con interés general
				\4 Utilidad de burócratas depende de:
				\4[] Remuneración
				\4[] Poder
				\4[] Prestigio
				\4 Si remuneración es fija
				\4[] $\to$ Maximizan poder y prestigio
				\4[] $\Rightarrow$ Compiten por aumentar presupuesto a su favor
			\3 Aversión al riesgo
				\4 Los policy-makers son aversos al riesgo
				\4[] $\to$ Proyectos implican posibilidad de fracaso
				\4[] $\to$ Fracaso aumenta posibilidad de despido
				\4 Proyectos socialmente óptimos no se llevan a cabo
			\3 Ilusión fiscal
				\4 Los ciudadanos infravaloran presión fiscal
				\4[] Perciben menos detracción de renta que real
				\4 Gobiernos aplican técnicas ilusorias
				\4[] $\to$ Retenciones fiscales
				\4[] $\to$ Imposición indirecta
				\4[] $\to$ Sistemas tributarios muy opacos
				\4[] $\to$ Déficit antes que impuestos
		\2 Propuestas de solución
			\3 Retribución ligada a productividad
				\4 Objetivo
				\4[] $\to$ Alinear incentivos privados y públicos
				\4 Problema
				\4[] $\to$ Cómo medir productividad?
				\4[] $\to$ Qué preferencias tienen los burócratas
			\3 Mecanismos de supervisión y control
				\4 Supervisar desviaciones máx. social vs privados
				\4 Problema
				\4[] Aumenta costes administrativos
				\4[] Controlador también tiene agenda propia
			\3 Individualización de responsabilidades
				\4 Evitar:
				\4[] Dilución de responsabilidad
				\4[] Atribución a terceros de desviaciones
			\3 Separación entre diseño y ejecución
				\4 Objetivo
				\4[] $\to$ Hacer imposible optimización individual
				\4 Ejemplo:
				\4[] Análisis coste-beneficio por agencia independiente
			\3 Presupuestos de base cero
				\4 Objetivo:
				\4[] Replantearse cada partida de gasto
				\4[] $\to$ Contribuyen a bienestar social?
				\4[] $\to$ Son resultado de un fallo del sector público?
			\3 Privatización
				\4 Objetivo:
				\4[] Evitar provisión pública de un bien o servicio
				\4[] $\to$ Si se estima resultado de fallo de sector público
				\4[] $\to$ O susceptible de más coste que beneficio
			\3 Limitación legal del tamaño del sector público
				\4 Establecer principios básicos
				\4[] $\to$ Difícilmente modificables o derogables
				\4[] $\to$ Restrictivos del tamaño del setor público
				\4[] $\to$ Restrictivos de prácticas perjudiciales
	\1[] \marcar{Conclusión}
		\2 Recapitulación
			\3 Optimalidad de la competencia perfecta
			\3 Externalidades
			\3 Bienes públicos
			\3 Fallos del sector público
		\2 Idea final
			\3 Debate social fundamental
				\4 ¿El estado debe o no intervenir?
				\4 ¿Si interviene, como intervenir?
				\4 ¿Qué problemas genera su intervención?
			\3 Desarrollo teórico del problema
				\4 Exposición presenta rasgos principales
				\4[] $\to$ Caracterizar óptimo
				\4[] $\to$ Caracterizar equilibrio competitivo
				\4[] $\Rightarrow$ Comparar y valorar
				\4 En la práctica:
				\4[] Información es problema principal
				\4[] $\to$ Conocer verdaderas preferencias
				\4[] $\to$ Incentivar revelación de información
				\4[] $\to$ Maximizar bienestar social y utilidad individual
			\3 Múltiples aplicaciones
				\4 Diseño de sistemas tributarios
				\4 Diseño de mercados de emisiones
				\4 Sistemas públicos de salud y educación
				\4 Financiación de haciendas territoriales
				\4 Cambio climático
				\4 ...
\end{esquemal}

























\graficas

\begin{dibujo}{4}{Representación gráfica del Segundo Teorema del Bienestar}{x}{y}{segundoteoremadelbienestar}
	% ejes que forman un cuadrado
	
	% eje al derecho, del agente A
	\draw[-{Latex}] (0,0) -- (0,4);
	\draw[-{Latex}] (0,0) -- (6,0);
	
	\node[below] at (6,-0.1){$x_A$};
	\node[left] at (0,3.9){$y_a$};
	
	\node[left] at (0,-0.3){$O_A$};
	
	% eje al revés, del agente B
	\draw[-{Latex}] (6,4) -- (0,4);
	\draw[-{Latex}] (6,4) -- (6,0);	
	
	\node[above] at (0.1,4.1){$x_B$};
	\node[right] at (6,0){$y_B$};
	
	\node[right] at (6,4.3){$O_B$};
	
	% dotación inicial
	
	\node[circle, fill=black, inner sep=0pt, minimum size=3pt] (a) at (1.7,3.5) {}; 
	
	\node[left] at (1.63,3.53){ \tiny $\bar{e}$};
	
	% Curva de indiferencia de agente A que pasa por dotación inicial
	
	\draw[dashed] (1.7,3.5) to [out=280, in=170](4.6,1.2);
	
	% Curva de indiferencia de agente B que pasa por dotación inicial
	
	\draw[dashed] (1.7,3.5) to [out=350, in=100](4.6,1.2);
	
	% Curvas de indiferencia de óptimo
	%
	% Curva de A que pasa por EGW
	\draw[-] (2.2,4) to [out=280, in=170](5.1,1.7);
	\node[right] at (4.95,1.55){\tiny $u^*_A$};
	%
	% Curva de B que pasa por EGW
	\draw[-] (1.32,3.12) to [out=350, in=100](4.22,0.82);
	\node[right] at (4.02,0.62){\tiny $u^*_B$};
	
	% Recta presupuestaria dados precios de equilibrio
	\draw[-] (1.65,3.55) -- (5.15,1);
	
	% Óptimo - punto de tangencia entre curvas de indiferencia
	\node[circle, fill=black, inner sep=0pt, minimum size=3pt] (a) at (3.2,2.4) {}; 
%	\node[above] at (3.03,2.21){\tiny $x^*$};

	% Óptimos de Pareto / curva de contrato
	\draw[thick] (0.5,0.7) to [out=70,in=190](3.2,2.4); 
	\draw[thick] (3.2,2.4) to [out=10, in=260](5.6,3.6);
	\node[right] at (4.05,3.4){\tiny Curva de contrato};
	
\end{dibujo}

Donde $\bar{e}$ es la dotación inicial y $x^*$ es el óptimo. Variando la posición de $\bar{e}$ en la caja de Edgeworth, es posible obtener cualquiera de los puntos de la línea gruesa.

\begin{axis}{4}{Representación del consumo de bien con externalidad en equilibrio competitivo y de óptimo de Pareto, en el modelo simplificado de MWG (pág. 354)}{$h$}{$\phi_i(h)$}{externalidadnegativa}
	% utilidad marginal de agente B que sufre externalidad
	\draw[-] (0,3.5) -- (4,-1);
	\node[right] at (4,3.5){$-\phi_B'(h)$};
	
	% utilidad marginal de agente A que induce externalidad
	\draw[-] (0,0) to [out=20, in=260](4,4);
	\node[above] at (1.2,2.8){$\phi_A'(h)$};
	
	% equilibrio óptimo
	\draw[dashed] (2.14,0) -- (2.14,1.1);
	\node[below] at (2.14,0){$h^0$};
	
	
	% equilibrio competitivo
	\node[below] at (2.95,-0.05){$h^*$};
\end{axis}

El gráfico muestra como la cantidad de bien con externalidad óptima ($h^0$) es aquella para la que se iguala la utilidad marginal para el agente que consume la externalidad $\phi_A'(h)$, y la desutilidad marginal para el que la sufre $-\phi'_B(h)$.

\begin{axis}{4}{Efecto de la introducción de un impuesto pigouviano para corregir el efecto de una externalidad negativa.}{$h$}{$\phi_i(h)$}{impuestopigouviano}
	% utilidad marginal de agente A que induce externalidad
	\draw[-] (0,3.5) -- (4,-1);
	\node[right] at (4,3.5){$-\phi_B'(h)$};
	
	% utilidad marginal de agente B que sufre externalidad
	\draw[dashed] (0,0) to [out=20, in=260](4,4);
	\node[above] at (1.2,2.8){$\phi_A'(h)$};
	
	% equilibrio óptimo
	\draw[dashed] (2.15,0) -- (2.15,1.08);
	\node[below] at (2.14,0){$h^0$};
	
	% impuesto pigouviano
	\draw[-] (0,1.08) -- (4,1.08);
	\node[left] at (0,1.08){$t_h$};
\end{axis}

El gráfico muestra como la introducción del impuesto específico permite inducir al agente que genera la externalidad a consumir el nivel óptimo $h^0$. En este punto, el impuesto reduce la utilidad de A tanto como una unidad de $h$ la aumenta y por ello el agente prefiere no consumir más.

\begin{tabla}{Clasificación de de los bienes de Ostrom (1977) de acuerdo con su excluibilidad y rivalidad}{clasificaciondeostrom}
	\begin{tabular}{l || l l}
		& \textbf{Excluibles} & \textbf{No excluibles} \\ \hline \hline
		\textbf{Rivales}    & Bienes privados     & Bien común \\
		\textbf{No rivales} & Bienes de club      & Bien público 
	\end{tabular}
\end{tabla}

\begin{dibujo}{4}{Representación gráfica de la condición de Samuelson como igualdad entre la relación marginal de transformación y la suma de las relaciones marginales de sustitución.}{}{}{condiciondesamuelson}
	% GRAFICA SUPERIOR
	
	% ejes
	\draw[-] (0,4) -- (0,0) -- (4.5,0);
	\node[left] at (0,4){X};
	\node[below] at (4.5,0){G};
	
	% FPP
	\draw[-] (0,3.5) to [out=-10, in=120](3.9,0);
	\node[left] at (3.4,2.3){FPP};
	
	% Utilidad de B
	\draw[-] (0.4,4) to [out=280, in=170](4,0.4);
	\node[right] at (0.4,4){$u_B$};
	
	% consumo óptimo de bien público en gráfica de B
	\node[left] at (2.87,-0.2){$G^*$};
	
	% consumo de bien privado de B
	\draw[dashed] (2.87, 0.75) -- (0,0.75);
	\node[left] at (0,0.75){$X_B^*$};
	
	% LÍNEAS DISCONTINUAS QUE CONECTAN SUPERIOR E INFERIOR
	
	% consumo alto de bien privado y bajo de público
	\draw[dashed] (0.54,3.4) -- (0.5,-5);
	
	% consumo bajo de bien privado y alto de público
	\draw[dashed] (3.63, 0.46) -- (3.6, -5);
	
	% consumo óptimo de bien público
	\draw[dashed] (2.85,-5) -- (2.85,0.75);

	
	% GRAFICA INFERIOR
	
	% ejes
	\draw[-] (0,-1) -- (0,-5) -- (4.5,-5);
	\node[left] at (0,-1){X};
	\node[below] at (4.5,-5){G};
	
	% Utilidad de A máxima dada FPC a partir de utilidad de B
	\draw[-] (0.4,-0.78) to [out=280, in=170](4,-4.38);
	\node[right] at (0.8,-2){$u_B$};
	
	% FPC
	\draw[-] (0.5,-5) to [out=86, in=180](2.1,-3.9);
	\draw[-] (2.1,-3.9) to [out=0, in=94](3.6,-5);
	\node[above] at (1.3,-4){$\text{FPC}_A$};
	
	% Consumo óptimo de bien público en gráfica de A
	\node[] at (2.87,-5.2){$G^*$};
	
	% Consumo de bien privado de A
	\draw[dashed] (2.87, -4.05) -- (0,-4.05);
	\node[left] at (0,-4.05){$X_A^*$};

\end{dibujo}

Se aprecia como la pendiente de la FPP en el óptimo es igual a la suma de la pendiente de la curva de indiferencia de B y la curva de indiferencia de A. El consumidor A consume la misma cantidad de bien público que B, pero sólo consume tanta cantidad de bien privado como resulte de restar el consumo de B al total de X producido.


\begin{axis}{4}{Equilibrio competitivo y de óptimo de Pareto en presencia de un bien público $q$.}{}{}{bienespublicos}
	% extensión del eje de abscisas
	\draw[-] (4,0) -- (6,0);
	\node[below] at (6,0){$q$};
	
	% Coste marginal
	\draw[-] (0,0.5) to [out=15, in=240](6,4);
	\node[right] at (6,4){$c'(q)$};
	
	% Beneficio social marginal
	\draw[-, thick] (0,3.5) -- (6,2);
	\node[right] at (6,2){$\sum_{i=1}^I \phi_i'(q)$};
	
	% Utilidad marginal del individuo I
	\draw[-] (0,2.5) to [out=280,in=160](3,0.5);
	\node[right] at (3,0.5){$\phi_I'(q)$};
	
	% Equilibrio competitivo
	\draw[dashed] (1.63,0) -- (1.63,1);
	\node[below] at (1.63,0){$q^*$};
	
	% Equilibrio de óptimo de Pareto
	\draw[dashed] (4.61,0) -- (4.61,2.36);
	\node[below] at (4.61,0){$q^0$};
\end{axis}

La línea fina ($\phi_I'(q)$) muestra la utilidad marginal de una unidad adicional de bien público para un consumidor I dado. El consumidor I estará dispuesto a pagar por una unidad adicional hasta que la utilidad marginal iguale el coste marginal. La línea gruesa muestra la utilidad social marginal de una unidad adicional de bien público.

\conceptos

\concepto{Equilibrio Walrasiano}

Dados N bienes e I consumidores que reciben dotaciones $\vec{e^i}$ y cuyas preferencias están representadas por funciones de utilidad $u_i(\vec{x^i})$, un Equilibrio Walrasiano o competitivo es un vector de precios $\vec{p}^*$ y un conjunto de asignaciones $(\vec{x^0}, ..., \vec{x^I})$ para los que se cumple que:

\begin{itemize}
	\item[i)] Para cada consumidor $i$, $\vec{x^i}$ es la solución del problema de optimización:
	
	\begin{align*}
		\underset{\vec{x^i}}{\max} \quad u_i (\vec{x^i}) \\
		\textrm{s.a:} \quad \vec{p} \cdot \vec{x^i} \leq \vec{p} \cdot \vec{e^i} 
	\end{align*}
	
	\item[ii)] El conjunto de asignaciones es factible, de modo que la suma de los asignaciones es igual a la suma de las dotaciones iniciales:
	
	\begin{equation*}
		\sum_i \vec{x^i} \leq \sum_i \vec{e^i}
	\end{equation*}
	
\end{itemize}

\preguntas

\seccion{Test 2017}

\textbf{12.} ¿Por qué puede un impuesto \textit{pigouviano} lograr que una empresa que fabrica cemento reduzca las externalidades negativas que causa con sus emisiones contaminantes a la atmósfera?

\begin{itemize}
	\item[a] Porque la empresa tendrá que asumir el impuesto como un coste adicional de la producción y, por tanto, internalizará los efectos externos que causa a los ciudadanos y aproximará su producción al óptimo social.
	\item[b] Porque el gobierno puede dedicar la recaudación del impuesto a mejorar el sistema sanitario y paliar las enfermedades causadas por la externalidad.
	\item[c] Porque los ciudadanos afectados por la externalidad pueden recibir pagos compensatorios por los daños.
	\item[d] No es posible corregir una externalidad con la aplicación de un impuesto \textit{pigouviano} y siempre es más eficiente aplicar límites administrativos a las emisiones contaminantes.
\end{itemize}

\seccion{Test 2016}

\textbf{9}. Un gobierno que se está planteando el modo de financiación de la provisión de un bien público baraja varias opciones:

\begin{itemize}
	\item[a] Con el mecanismo de Gloves-Clarke no queda garantizado que los pagos de los consumidores cubran el coste de provisión.
	\item[b] Una posible vía de financiación con la que se cubriría íntegramente el coste de provisión sería utilizar el mecanismo de Groves-Clarke, que evita el problema de revelación de preferencias distintas a las reales que se da en los precios de Lindahl.
	\item[c] Una posible forma de financiación con la que se cubriría íntegramente el coste de provisión sería utilizar los precios de Lindahl. Pero este mecanismo puede conducir a una situación de provisión excesiva del bien público, dado el incentivo que tienen los individuos a no revelar sus preferencias reales.
	\item[d] Los usuarios del bien público no deberían contribuir a su provisión dado que, por definición, son bienes no rivales en el consumo y no existen vías de excluir a los agentes de su disfrute. 
\end{itemize}

\seccion{Test 2015}
\textbf{13.} Señale la respuesta verdadera con respecto al Teorema de Coase partiendo de los siguientes supuestos:

\begin{itemize}
	\item El mercado de competencia perfecta con 2 empresas (misma estructura de costes).
	\item La empresa A contamina el agua y repercute negativamente en la producción de la empresa B.
	\item La curva de daños marginales de la empresa B (DMgB) está por debajo de la de costes marginales y su pendiente es positiva y constante.
	\item $Y_A^*$: nivel de producción de la empresa A eficiente desde un punto de vista social.
	\item IMgA: ingreso marginal de la empresa A.
	\item CMgA: coste marginal de la empresa A.
\end{itemize}

\begin{itemize}
	\item[a] Si se asignan los derechos de propiedad sobre el agua a la empresa A se llegará a un óptimo social eficiente. Esto no pasaría si los derechos se asignasen a la empresa B.
	\item[b] Para niveles de producción $Y_A > Y_A^*$ la disposición a pagar de la empresa B por disminuir la producción de la empresa A es menor que la compensación que está dispuesta a recibir la empresa A por la disminución de su producción.
	\item[c] Para niveles de producción $Y_A < Y_A^*$, la compensación que está dispuesta a recibir la empresa A por disminuir su producción es la diferencia entre el $\text{IMgA}$ y el $\text{CMgA}$.
	\item[d] La solución óptima desde un punto de vista social implica una producción $Y_A^*$ en $\text{IMgA} = \text{CMgA} - \text{DMgB}$.
\end{itemize}

\seccion{Test 2013}

\textbf{2.} El modelo de Pigou sobre externalidades:
\begin{itemize}
	\item[a] Es refutado por el teorema de Coase.
	\item[b] Es independiente de los resultados del teorema de Coase.
	\item[c] Ofrece una guía para la política económica en situaciones en las que el teorema de Coase no es aplicable.
	\item[d] No es mencionado por Coase en su artículo sobre el coste social.
\end{itemize}

\seccion{Test 2008}
\textbf{13.} En relación con las soluciones disponibles para resolver la prestación de un bien público, es falso que:

\begin{itemize}
	\item[a] La solución de precios de Lindahl tiene el problema de la revelación de preferencias.
	\item[b] El Sector Público puede prestar el bien público desligando los beneficios para los consumidores del pago que les correspondería.
	\item[c] Una solución es establecer un precio individual para cada consumidor igual a la valoración media que perciben de dicho bien.
	\item[d] La votación del nivel de consumo por parte de los consumidores, en general no permite encontrar el nivel de producción eficiente.
\end{itemize}

\seccion{Test 2005}

\textbf{4.} Suponga una economía con un consumidor, dos bienes y un recurso (trabajo), donde el bien X causa una externalidad negativa en la producción del bien Y. Entonces, si definimos la relación marginal de transformación como $\left| \text{RMT}_{y,x} \right| = - \left. \frac{d Y}{d X} \right|_{FPP}$ y la relación marginal de sustitución como $\left| \text{RMS}_{y,x} \right| = \left. - \frac{d Y}{d X} \right|_{U}$, entonces:

\begin{itemize}
	\item[a] El equilibrio general competitivo supone una producción del bien X inferior a la que sería socialmente óptima.
	\item[b] El equilibrio general competitivo no será un óptimo de Pareto porque el impacto de la externalidad hace que: $\left| \text{RMS}_{y,x} \right| \neq \frac{P_x}{P_y}$.
	\item[c] Una manera de conseguir que el mercado asigne eficientemente sería colocar un impuesto de $t=\frac{P_x}{P_y}$ euros por unidad producida del bien $Y$.
	\item[d] El equilibrio general competitivo verifica $\left| \text{RMT}_{y,x} \right| > \left| \text{RMS}_{y,x} \right|$ y, por tanto, no será óptimo de Pareto.
\end{itemize}

\seccion{Test 2004}

\textbf{11.} Señale la respuesta \textbf{CORRECTA} referida a los precios de Lindahl:
\begin{itemize}
	\item[a] Permiten resolver la ineficiencia causada por una externalidad negativa en la producción.
	\item[b] Son los precios que hacen que los consumidores elijan la cantidad eficiente de un bien público.
	\item[c] Permiten resolver la ineficiencia causada por una externalidad negativa en el consumo.
	\item[d] Son la suma de las Relaciones Marginales de Sustitución entre el bien público y el privado de los consumidores.
\end{itemize}

\notas

\textbf{2017:} \textbf{12.} A

\textbf{2016:} \textbf{9.} A

\textbf{2015:} \textbf{13.} C

\textbf{2013:} \textbf{2.} C La condición de primer orden del óptimo social en este tipo de problemas es $\text{IMg}_A = \text{CMg}_A + \text{DMg}_B$. El óptimo competitivo para la empresa A es, por supuesto, $\text{IMg}_A = \text{CMg}_A$. El hecho de sumar $\text{DMg}_B$, que siempre es positivo e inferior a $\text{CMg}_A$ implicará que la igualdad con $\text{IMg}_A$ se alcanzará produciendo un numero de unidades menor a las del óptimo competitivo. 

\textbf{2008:} \textbf{13.} C

\textbf{2005:} \textbf{4.} D

\textbf{2004:} \textbf{11.} B


Este tema puede ser uno de los peores temas de la línea Tiana--Sahuquillo y seguramente de CECO también. 


\bibliografia

Bhattacharya, S. \textit{Groves-Clarke Mechanism Example}. \url{http://web.uvic.ca/~sukanta/Econ313/gcex.pdf}

Blaug, M. \textit{Fundamental Theorems of Modern Welfare Economics, Historically Contemplated} (2007) History of Political Economy -- En carpeta del tema

Bryant, W. D. A. (1994) \textit{Misinterpretations of the Second Fundamental Theorem of Welfare Economics: Barriers to Better Economic Education} The Journal of Economic Education, Vol. 25, No. 1 -- En carpeta del tema


Niskanen, W. A. (1968) \textit{The Peculiar Economics of Bureaucracy} American Economic Review -- En carpeta del tema

Spiegel, Y. \textit{Clarke-Groves mechanisms for optimal provision of public goods.} \url{https://www.tau.ac.il/~spiegel/teaching/inter-micro/clarke-groves.pdf} -- En carpeta del tema.

Mirar en Palgrave:
\begin{itemize} 
	\item clubs
	\item Coase theorem
	\item external economies
	\item externalities
	\item general equilibrium
	\item incentive compatibility
	\item Lindahl equilibrium
	\item public choice
	\item public goods
	\item welfare economics
\end{itemize}

\end{document}
