\documentclass{nuevotema}

\tema{3A-5}
\titulo{Críticas al modelo de la síntesis neoclásica: los modelos neokeynesianos de desequilibrio y la crítica monetarista.}

\begin{document}

\ideaclave

La aparición de la Teoría General del Empleo, el Interés y el Dinero en 1936 supuso el inicio de la llamada revolución keynesiana. La metodología y el objeto de estudio de los fenómenos económicos de gran escala se transformaron hasta el punto de que es habitual considerar la obra de Keynes como el germen de la macroeconomía como subdisciplina de la ciencia económica. En las dos décadas posteriores a la Teoría General, las ideas de Keynes fueron la inspiración de un conjunto de modelos que construían un paradigma de gran coherencia formado por el modelo IS-LM, una serie de extensiones a éste, un conjunto de modelos macroeconométricos estructurales inspirados en el IS-LM y desde finales de los años 50, la curva de Phillips como caracterización de la relación entre oferta agregada e inflación. Este paradigma derivado de una interpretación de Keynes tomó el nombre de síntesis neoclásica a partir de los años 60. El mensaje central de la síntesis neoclásica es que la intervención pública puede y debe utilizarse para corregir la tendencia de las economías a producir por debajo de su potencial. El origen de esta tendencia hacia la producción subóptima está en las rigideces que afectan al ajuste de variables nominales y reales. La política fiscal debe utilizarse para estimular la demanda hasta el pleno empleo, mientras que la política monetaria debe mantener los tipos de interés en niveles que acomoden la expansión de la deuda pública. En el largo plazo, por el contrario, los niveles de producción se asumen exógenos a la intervención estatal y determinados por las condiciones de la oferta. Se introduce así una dicotomía entre corto y largo plazo a la hora de estudiar el crecimiento. A partir de los años 60 aparecen una serie de críticas y anomalías que debilitan la posición predominante de la síntesis neoclásica. La inflación y el paro crecientes de los años 60, la metodología basada en supuestos ad-hoc, la revisión de las causas de la Gran Depresión y la vuelta al primer plano de Walras con la traducción de 1954 abonan el terreno para la aparición de una serie de críticas estructuradas de la síntesis neoclásica que acabarían determinando los programas de investigación de las décadas posteriores hasta la actualidad. El \textbf{objeto} de la exposición consiste en explicar los modelos y los programas de investigación que surgen como respuesta a una serie de críticas a las que la síntesis neoclásica no ofrece respuestas, las conclusiones de política económica que se derivan de estos nuevos programas, y la influencia que tienen sobre la economía posterior. La presentación de los neokeynesianos del desequilibrio ocupa la primera parte de la exposición. La segunda parte examina el monetarismo liderado por Milton Friedman.

La adscripción a Keynes de los \marcar{neokeynesianos del desequilibrio} es explícita. Sin embargo, critican la interpretación dominante de las ideas de Keynes: critican la capacidad de la síntesis neoclásica y especialmente el modelo IS-LM a la hora de representar fielmente el mensaje original. El objetivo de los autores enmarcados en esta corriente es extraer el mensaje de la Teoría General evitando las incoherencias y los supuestos ad-hoc que atribuyen al modelo IS-LM. Para ello, utilizarán un conjunto de herramientas que está lejos de ser homogéneo entre autores pero que se caracteriza en conjunto por una fuerte influencia walrasiana. La publicación en 1956 de Dinero, Interés y Precios de \textbf{Don Patinkin} es el primer intento por interpretar el modelo keynesiano y más concretamente el modelo IS-LM con un enfoque de equilibrio general. El objetivo central de la obra de Patinkin es fundamentar las interrelaciones entre variables reales y nominales que Walras había asumido independientes. En ese objetivo, afirma que el modelo de Keynes es la primera aplicación de las herramientas walrasianas. Los tres mercados que Keynes describe son presentados como una aplicación del concepto de equilibrio general walrasiano simplificado. En este contexto, se interroga acerca de las causas del desempleo o equivalentemente, del racionamiento de la oferta de trabajo. Aunque acepta la idea de los salarios rígidos como supuesto apropiado para simplificar el análisis, lo rechaza como una explicación satisfactoria en general. Patinkin acaba afirmando que el desempleo no es sino el resultado del ajuste lento de las variables nominales. Así, el racionamiento de la oferta de trabajo no es un resultado de equilibrio sino de transición, un estado temporal (aunque de duración potencialmente indefinida) que tiende a dejar de existir. \textbf{Leijonhufvud} publica en 1968 un libro que rechaza la utilización del modelo IS-LM por distorsionar el mensaje keynesiano. Leijonhufvud admite la lentitud del ajuste propuesta por Patinkin pero no lo considera un factor suficiente para explicar el desempleo. Según Leijonhufvud, Keynes pretende transmitir la idea de que en una economía descentralizada, los problemas de información y señalización resultan en una descoordinación sistemática de las transacciones entre agentes que impide el ajuste de la economía hacia sus producción de pleno potencial. Además, argumenta que los precios de los activos tienden a mantenerse demasiado bajos durante demasiado tiempo, manteniendo altos tipos de interés de largo plazo e induciendo una inversión insuficiente. Por último, argumenta el carácter puramente marshalliano de Keynes frente a la compatibilidad con Walras que Patinkin le había atribuido. Así, según Leijonhufvud, Keynes invierte el modelo marshalliano de corto plazo: para las empresas es más fácil ajustar cantidades que precios y por ello, el ajuste se produce en cantidades antes que en precios.

En 1965, \textbf{Clower} publica un artículo verdaderamente pionero y que abriría el camino a los elementos centrales del neokeynesianismo del desequilibrio. Clower trata de enmarcar la teoría de precios tradicional derivada de Walras y la microeconomía neoclásica como un caso particular de la teoría general alumbrada por Keynes, planteando la posibilidad de que el ajuste de precios hacia equilibrio walrasiano sea sólo un caso particular dentro de una teoría más general. Cuando existe un exceso de demanda en un mercado y de oferta en otro, el mecanismo de ajuste de los precios habitualmente asumido señala que el precio aumentará allí donde hay exceso de demanda y se reducirá donde haya exceso de oferta. Clower redefine idiosincráticamente los conceptos de demanda efectiva y nocional para atacar esta idea. Demanda nocional es aquello que los agentes desearían consumir dados unos precios. Demanda efectiva es aquella cantidad que los agentes pueden efectivamente consumir, teniendo en cuenta que ante excesos de demanda la cantidad consumida es racionada. Si los precios varían en función de excesos de demanda calculados a partir de la demanda nocional, efectivamente el mercado tenderá a eliminar excesos de demanda. Pero si el mecanismo de ajuste toma como referencia la demanda efectiva, el exceso de demanda nocional se mantendrá y será factible un equilibrio con desempleo. Otro concepto central del artículo de Clower es la hipótesis de la decisión dual. Supongamos dos agentes, una consultora y un productor de champán. La consultora quiere consumir champán y ofrece sus servicios de consultoría. El productor ofrece champán y desea consumir servicios de consultoría. Bajo supuestos walrasianos, ambos agentes toman decisiones de oferta y demanda al mismo tiempo, de tal manera que intercambian sus ofertas hasta alcanzar un óptimo. Bajo la hipótesis de decisión dual, las decisiones de demanda y oferta no son simultáneas. La consultora no puede demandar champán hasta que no haya vendido servicios de consultoría. El productor de champán no puede demandar servicios de consultoría hasta que no haya vendido el champán. En este contexto de decisiones duales no simultáneas, la economía se sitúa en un equilibrio con desempleo y demanda insuficiente. Es preciso señalar que algunos años después de la publicación del artículo, Clower repudiaría la hipótesis de la decisión dual. Denominó al conjunto de modelos inspirados en su artículo como ``monstruos'' y emprendió un viraje hacia las tesis de Leijonhufvud, rechazando la idea de compatibilizar Walras y Keynes y afirmando el carácter puramente marshalliano de la Teoría General.

El modelo de \textbf{Barro y Grossman} de 1971 es el punto álgido del neokeynesianismo del desequilibrio. Los autores se inspiran en Patinkin y Clower. Del primero, toman la idea de spill-over de un mercado a otro para fundamentar la relación entre paro y exceso de oferta en los mercados de bienes. De Clower toman la definición de demandas efectivas y nocionales para construir un modelo de equilibrio general en el que los precios no se ajustan hacia el equilibrio walrasiano, sino que se mantienen en un equilibrio con excesos de demanda y oferta. Es decir, fundamentan las implicaciones de Patinkin con las herramientas de Clower acerca del ajuste de precios. El gran mensaje final del trabajo es la caracterización del desempleo en función de los precios de equilibrio walrasiano y no walrasiano. Cuando el precio y el salario en los mercados de bienes y trabajo, respectivamente, son mayores que los de equilibrio walrasiano, el equilibrio corresponde con el desempleo keynesiano. Cuando el precio de los bienes es inferior al de equilibrio walrasiano pero el salario es superior, la economía se encuentra en un situación de paro clásico. Cuando precios y salarios son ambos inferiores a los de equilibrio walrasiano, estamos ante una situación de inflación reprimida. 

El artículo de Barro y Grossman de 1971 dio lugar a una fértil literatura en la década de los 70 entre las que destacan los trabajos de \textbf{Benassy}, refinando la microfundamentación del desequilibrio con desempleo, y la labor de divulgación de \textbf{Malinvaud}, entre otros. Éste último publica en 1977 un libro titulado Teoría del Desempleo Reconsiderada con el objetivo de simplificar el lenguaje de los neokeynesianos inspirados por Walras. Se trata de una obra orientada a los policy makers, con abundantes ejemplos y aplicaciones prácticas de los modelos teóricos. El libro incluye el conocido cuadro que clasifica las situaciones de paro keynesiano, clásico, inflación reprimida y subempleo en función de la presencia de exceso de oferta o demanda en los mercados de bienes y trabajo. 

El programa de investigación de los neokeynesianos del desequilibrios acabó entrando en un punto muerto a finales de los años 70. Las recomendaciones de política económica de corte claramente keynesiano y favorable a los estímulos de demanda dejaron de encontrar una acogida favorable ni en ámbitos políticos ni en círculos académicos. El monetarismo que se expone a continuación había conseguido imponerse postulando la importancia de la política monetaria y la estabilidad inherente de las macroeconomías. La aparición de la nueva macroeconomía clásica de la mano de Lucas, Kydland, Prescott, Sargent y otros autores termina por enterrar el enfoque del desequilibrio. Pero sería erróneo considerar que el neokeynesianismo del desequilibrio no tuvo impacto sobre la macroeconomía. La introducción de la microfundamentación de fenómenos macroeconómicos, la conceptualización del desempleo como un fenómeno racional o la importancia de los procesos de ajuste hacia el desequilibrio son elementos centrales de corrientes posteriores de la macroeconomía como la Nueva Macroeconomía Clásica y la Nueva Economía Keynesiana que perduran hasta la actualidad.

Si el neokeynesianismo del desequilibrio trata de recuperar el mensaje keynesiano aplicando herramientas diferentes de las que predominan en la síntesis neoclásica, el \marcar{monetarismo} es lo contrario: una crítica a las ideas keynesianas que no ataca frontalmente los métodos de la síntesis neoclásica. Así, el monetarismo se caracteriza por aceptar el marco teórico general y el método keynesiano de análisis basado en variables agregadas y el enfoque marshalliano de ecuaciones simples que caracterizan el funcionamiento de mecanismos y relaciones concretas entre pequeños conjuntos de variables. Milton Friedman fue el líder de la corriente y principal autor. Él mismo empezó rechazando la utilización del modelo IS-LM pero acabó por aceptarla como un instrumento útil para divulgar su modelo.

Las \textbf{ideas centrales} del programa monetarista son la teoría cuantitativa del dinero como modelo de la demanda de dinero, la importancia de la política monetaria a la hora de causar fluctuaciones cíclicas y estimular o deprimir la economía, la estabilidad de los mercados privados y su tendencia al equilibrio walrasiano con plena utilización de los recursos disponibles, la necesidad de fijar reglas estables de política monetaria y el rechazo a la existencia de trade-offs sistemáticos y estables entre desempleo e inflación. El autor principal de la escuela monetarista es Milton Friedman. Fuertemente influenciado por Marshall, Fisher, Keynes, Knight y Wicksell, sus principales obras se publican en los años 60. \textit{Una historia monetaria de los Estados Unidos} (1963) y el discurso presidencial de la Asociación Americana de Economía en 1967 titulado ```\textit{El rol de la política monetaria}'', definen los pilares del programa de investigación  monetarista. En general, los autores relevantes del monetarismo han estado fuertemente asociados a la Universidad de Chicago y entre ellos se cuentan --además de Friedman- Laidler, Frenkel, Schwartz o Brunner.

La \underline{demanda de dinero} es el punto de partida del análisis monetarista de la \textbf{política monetaria}. Para ello, Friedman se basa en la \underline{teoría cuantitativa del dinero} cuya relevancia había sido puesta en duda por el keynesianismo. En el marco keynesiano y las formulaciones habituales de la síntesis neoclásica, la dependencia de la demanda de dinero respecto del tipo de interés le confiere un grado de inestabilidad que reduce la efectividad de la política monetaria como instrumento para alcanzar el pleno empleo. Por el contrario, Milton Friedman afirma que la demanda de dinero es lo suficientemente estable como para poder establecer una relación entre variaciones de la oferta de dinero y del ingreso nominal. La función de demanda monetarista depende de un conjunto más amplio de variables que la demanda de dinero keynesiana habitual en modelos de la síntesis. El modelo monetarista incorpora la incipiente teoría de carteras de tal manera que la demanda no depende de un tipo de interés de los bonos como categoría general, sino del conjunto de activos financieros disponibles para los agentes y que pueden representarse con un nivel de desagregación variable. La riqueza total del agente también es un factor relevante en la demanda de dinero y enlaza con la teoría de la renta permanente propuesta por Friedman. Según ésta, los agentes no toman decisiones de consumo en base a su renta presente sino teniendo en cuenta toda su renta a lo largo del tiempo. La inflación es también un factor que modula la demanda de dinero. Partiendo de esta demanda de dinero aumentada, cabe preguntarse ¿\underline{qué mecanismos trasladan} variaciones de la oferta de dinero en cambios de la producción, precios y los tipos de interés? El monetarismo acepta la existencia de varios mecanismos de transmisión, incluyendo dos que actúan directamente a partir de la demanda de dinero. El efecto directo es resultado de introducir en la Ley de Walras la demanda de bienes y servicios. Si en el contexto keynesiano los agentes distribuían su ahorro entre dinero y bonos, bajo los supuestos monetaristas los agentes distribuyen su ahorro entre dinero, bonos y bienes de inversión. Cuando se produce un exceso de oferta de dinero, la Ley de Walras implica un exceso de demanda en alguno de los otros dos mercados. Así, una expansión monetaria induce aumentos de demanda de bonos, dinero y bienes duraderos. De ello se deriva que expansiones monetarias no tienen necesariamente que reducir el tipo de interés y que pueden tener un efecto directo sobre la demanda agregada que aumente la producción. El segundo mecanismo considerado es el efecto indirecto o de Keynes. Si la expansión monetaria reduce el tipo de interés, la demanda de inversión aumentará y la demanda agregada aumentará como resultado. Los monetaristas tienden a asumir una demanda de inversión más sensible al tipo de interés que los keynesianos y por tanto, entienden que el efecto indirecto es más fuerte. El efecto riqueza o de Pigou no guarda relación directa con la demanda de dinero sino con la demanda de consumo, pero también se acepta en los modelos de corte monetarista y constituye una crítica a la trampa de liquidez. Si estos mecanismos tratan de explicar las vías por las que la política monetaria afecta al ingreso nominal, el monetarismo no aportó ningún modelo teórico que explicase la distribución relativa del aumento del output nominal entre precios y producción. Los autores monetaristas tendieron a considerar que la respuesta a esta cuestión debía permanecer en el plano empírico dada la existencia de lags y factores de incertidumbre que hacen muy difícil una identificación teórica general. La ``ecuación perdida del monetarismo'' hace referencia a este hecho. Los efectos de primera ronda se refieren al efecto que tienen diferentes modos de aumentar la oferta monetaria. Friedman también prefirió dar respuesta a esta pregunta con métodos empíricos.

Junto con la demanda de dinero y la rehabilitación de la teoría cuantitativa del dinero, la \textbf{crítica a la curva de Phillips} de la síntesis neoclásica es el segundo pilar del monetarismo. La \underline{síntesis neoclásica} entendía la curva de Phillips como una relación relativamente estable en la que la inflación salarial y general resultaba de cuestiones institucionales y de excesos de demanda. Aunque Samuelson y Solow (1960) no afirmaron realmente que la curva de Phillips sea estable en el largo plazo e introdujeron la idea de histéresis, sí plantearon la relación precios y desempleo como un menú de política económica a disposición de los gobiernos en el corto plazo. En el discurso presidencial de la AEA, \underline{Friedman rechaza de plano la idea}. La curva de Phillips es una relación de corto plazo que los gobiernos pueden explotar mediante políticas monetarias expansivas cada vez más costosas en términos de inflación. A medida que intentan reducir el nivel de desempleo a cambio de inflación, cada vez necesitarán mayores aumentos de los precios para una misma reducción del desempleo. Si dejan de aplicar estímulos monetarios, el desempleo tenderá hacia la \underline{tasa natural de desempleo}. El concepto se inspira en la idea de tipo de interés natural de Wicksell. Friedman define el desempleo natural como el nivel resultante del sistema walrasiano de ecuaciones. La definición, ciertamente difusa, puede entenderse como el salario real que equilibra oferta y demanda. Así, son factores institucionales y de oferta los que definen esta tasa, y por ello la intervención pública debe destinarse a generar las condiciones para que ésta se reduzca, flexibilizando la contratación y evitando distorsiones. Pero, ¿qué mecanismo hace posible reducir el desempleo a corto plazo vía estímulos monetarios? Basándose en la teoría cuantitativa, Friedman asume que políticas monetarias expansivas aumentan los precios. A iguales salarios, la inflación general reduce el salario real, por lo que las empresas demandan más trabajo. Los trabajadores observan su salario nominal pero estiman el nivel general de precios mediante expectativas adaptativas, que dependen de precios pasados. De esta forma, estiman un salario real superior al que recibirán realmente, y por ello ofertan más trabajo. El desempleo aumenta en tanto que los trabajadores mantengan esas expectativas, pero vuelve a aumentar cuando recalculan sus expectativas de inflación. El desempleo vuelve entonces a su tasa natural, salvo que el gobierno vuelva a aplicar un estímulo monetario que acelere de nuevo la inflación. En casos extremos, la inflación es tan alta que el sistema de precios se vuelve inoperante y la curva de Phillips se torna creciente.

Las \textbf{implicaciones de política económica} del monetarismo son numerosas y rompen con el consenso keynesiano y de la síntesis neoclásica. Inicialmente, Friedman y otros autores monetaristas rechazan el marco IS-LM para plantearlas, pero a partir de los 70 Friedman acepta el modelo IS-LM como herramienta de divulgación y debate, y lo utiliza para resaltar las diferencias entre su modelo y el keynesiano. Si la síntesis neoclásica entendía la \underline{política monetaria} como un instrumento poco efectivo para modular el producto, el monetarismo lo entiende como la herramienta de mayor potencia aunque también, de consecuencias más peligrosas. Partiendo de los supuestos de oferta monetaria exógenamente determinada y demanda de dinero estable, la política monetaria tiene la capacidad de modular el ingreso nominal mediante los mecanismos estimados anteriormente. El monetarismo atribuye a la política monetaria el grueso de las fluctuaciones cíclicas. Si bien en un primer momento Friedman defendió la utilización de la política monetaria como instrumento estabilizador frente a la política fiscal, posteriormente rechazó su utilización por la dificultad para conocer a priori sus efectos. Los problemas de información señalados anteriormente son la causa principal y resultan políticas monetarias que acaban agravando los problemas que intentan solucionar. Por ello, Friedman acabó proponiendo que la expansión de la oferta monetaria se fijase a una tasa fija en la constitución y que el Banco Central estuviese sujeto al poder legislativo y no al ejecutivo. Para facilitar el control de la oferta monetaria, propuso también un coeficiente de reservas del 100\% para los depósitos a la vista. En cuanto a la política fiscal, Friedman también defiende implicaciones de \underline{política fiscal} radicalmente distintas a las de la síntesis. Postula que la inversión es muy sensible al tipo de interés y que dada la verticalidad de la curva de oferta agregada, expansiones fiscales sólo provocarán crowding-out de la inversión privada con mayor precios y tipos de interés más elevados. 

Una \textbf{valoración} global del impacto del monetarismo sobre el diseño de políticas y el pensamiento económico debe partir de la consideración de que el monetarismo tuvo una \underline{orientación eminente empírica}. Friedman y Schwartz transformaron el consenso general acerca de las causas de la Gran Depresión. Si la síntesis neoclásica atribuía la culpa a una insuficiencia generalizada de la demanda, el monetarismo apuntaba a la dramática caída de la oferta de dinero y una insuficiente actuación por parte de la Reserva Federal. En 2002, Bernanke aceptó la responsabilidad de la Reserva Federal, testimoniando el éxito del monetarismo a la hora de señalar la importancia de los factores monetarios en las fluctuaciones cíclicas. El monetarismo dio lugar también a numerosos estudios empíricos que trataban de demostrar la relación entre fluctuaciones de la oferta monetaria y la producción. La ecuación de San Luis, Sims (1972) y Brunner y Meltzer (1976) son algunos de sus trabajos más significativos. La \underline{influencia} del monetarismo sobre la economía actual sigue muy presente: la Nueva Macroeconomía Clásica y la Nueva Economía Keynesiana atribuyen a las expectativas un papel central en el ciclo y centran muchos de sus esfuerzos de investigación sobre el papel del dinero en las fluctuaciones cíclicas. Además, el monetarismo recuperó la idea de la estabilidad de la economía y su fluctuación en torno a un valor natural o de equilibrio, frente al enfoque keynesiano tendente a asumir que la economía tiende a producir por debajo de su potencial. El monetarismo derivó también en crítica general a la intervención gubernamental, la defensa del libre mercado y las políticas basadas en reglas frente a la intervención puntual.

El monetarismo también sufrió duras \underline{críticas} y se le atribuyeron algunos fracasos por parte de sus oponentes. La hipótesis de expectativas adaptativas fue remplazada por las expectativas racionales en la mayoría de modelos macroeconómicos. El concepto de tasa natural de desempleo pasó a denominarse NAIRU (non-accelerating inflation rate of unemployment) para eliminar la connotación positiva y de estabilidad natural hacia un ajuste. Si los principales bancos centrales del mundo capitalista adoptaron un cierto monetarismo al poner el énfasis sobre la evolución de los agregados monetarios desde finales de los 70 y en el inicio de los 80, el experimento fue abandonado rápidamente. Las innovaciones financieras y otros factores desestabilizaron la demanda de dinero y los bancos centrales tomaron como objetivo otras variables como el tipo de interés o el tipo de cambio. La ausencia de un modelo explícito de política monetaria que fuese más allá de relaciones empíricas de equilibrio parcial frenó la continuidad del monetarismo como tal y cedió frente a los incipientes modelos de equilibrio general dinámico y estocástico. 

La exposición ha versado sobre las dos principales corrientes críticas de la síntesis neoclásica. Ambas escuelas tienen en común no haber sobrevivido el paso del tiempo en los términos en los que sus autores principales las formularon. Tampoco sus recomendaciones de política económica fueron implementadas en su mayoría. Sin embargo, su impronta perdura en la actualidad. Los programas de investigación cuyo desarrollo inspiraron y los avances metodológicos han permitido la formulación de modelos que aportan una mejor comprensión del funcionamiento de la economía y predicciones más acertadas sobre el impacto de las política económica.

\seccion{Preguntas clave}

\begin{itemize}
	\item ¿Qué críticas recibió la síntesis neoclásica?
	\item ¿A qué modelos o programas de investigación dió lugar?
	\item ¿Qué conclusiones de política económica se derivaron?
	\item ¿Qué influencia tuvieron sobre la macroeconomía posterior?
\end{itemize}

\esquemacorto

\begin{esquema}[enumerate]
	\1[] \marcar{Introducción}
		\2 Contextualización
			\3 Evolución de la ciencia económica
			\3 Historia del pensamiento económico
			\3 Macroeconomía
			\3 Revolución keynesiana
			\3 Síntesis neoclásica
			\3 Anomalías y críticas
			\3 Contexto de aparición de las críticas
		\2 Objeto
			\3 Qué críticas recibió la síntesis neoclásica
			\3 Qué programas de investigación surgen
			\3 Qué conclusiones de política económica se derivaron
			\3 Qué influencia tuvieron sobre la macroeconomía posterior
		\2 Estructura
			\3 Neokeynesianos del desequilibrio
			\3 Monetarismo
	\1 \marcar{Neokeynesianos del desequilibrio}
		\2 Visión general
			\3 Contexto
			\3 Objetivos
			\3 Metodología
		\2 Autores
			\3 Patinkin
			\3 Clower
			\3 Leijonhufvud
			\3 Barro y Grossman
			\3 Benassy
			\3 Malinvaud
			\3 Diamond
		\2 Valoración
			\3 Impacto sobre política económica
			\3 Influencia sobre ciencia económica
	\1 \marcar{Monetarismo}
		\2 Visión general
			\3 Contexto
			\3 Metodología
			\3 Ideas centrales
		\2 Curva de Phillips
			\3 Consenso pre-monetarista en policy-making
			\3 Idea clave
			\3 Tasa natural de desempleo (TND)
			\3 Crítica de Friedman
			\3 Reducción permanente del desempleo
			\3 Phelps
		\2 Política monetaria
			\3 Teoría cuantitativa del dinero
			\3 Demanda de dinero monetarista
			\3 Mecanismos de transmisión dinero-ingreso nominal
		\2 Política fiscal
			\3 Crowding-out de la inversión
			\3 Expansión fiscal contractiva
			\3 Financiación vía deuda vs monetaria
		\2 Implicaciones de política económica
			\3 Marco IS-LM
			\3 Política monetaria
			\3 Política fiscal
		\2 Valoración
			\3 Estudios empíricos
			\3 Críticas
			\3 Influencia
	\1[] \marcar{Conclusión}
		\2 Recapitulación
			\3 Neokeynesianos del desequilibrio
			\3 Monetarismo
		\2 Idea final
			\3 Robert Solow sobre modelos macro y economistas
			\3 Recomendaciones de escuelas macroeconómicas
			\3 Aportación de escuelas macroeconómicas

\end{esquema}

\esquemalargo
















\begin{esquemal} 
	\1[] \marcar{Introducción}
		\2 Contextualización
			\3 Evolución de la ciencia económica
				\4 Conjunción de múltiples factores
				\4 Contexto económico
				\4 Contexto teórico previo en economía
				\4 Avances en otras discIplinas
				\4[] Filosofía
				\4[] Matemáticas
				\4[] Biología
			\3 Historia del pensamiento económico
				\4 Permite entender origen de pensamiento actual
				\4 Permite entender problemas históricos
				\4 Permite valorar programas de investigación
			\3 Macroeconomía
				\4 Estudio de fenómenos económicos de gran escala
				\4[] Interacción de economías con millones de agentes
				\4[] Emergencia de fenómenos ajenos a microeconomía
				\4 Énfasis sobre variables agregadas
				\4[] Especialmente renta, precios, empleo, interés
				\4[] Requiere de nuevas herramientas de análisis
				\4[] Nuevos supuestos robustos a agregación
			\3 Revolución keynesiana
				\4 Salto metodológico
				\4 Nuevos objetos de estudio
				\4 Nuevas conclusiones
			\3 Síntesis neoclásica
				\4 Corriente predominante tras Keynes
				\4 Metodología
				\4[] Modelo IS-LM y extensiones
				\4[] Modelos macroeconométricos estructurales
				\4[] Escasa microfundamentación
				\4 Implicaciones de política económica
				\4[] Política económica puede reducir desempleo
				\4[] Posible estabilizar el ciclo vía fine-tuning
			\3 Anomalías y críticas
				\4 Inflación y paro crecientes en años 60
				\4 Metodología basada en supuestos ad-hoc
				\4 Críticas a interpretación de Gran Depresión
				\4 Atención de nuevo sobre cuestiones monetarias
				\4 Marco Walras vuelve a primer plano
				\4[$\then$] Aparecen críticas estructuradas de SNC
			\3 Contexto de aparición de las críticas
				\4 Estabilidad de precios y output posguerra
				\4 Bretton Woods
				\4[] Sistema de cambios fijos pero ajustables
				\4[] Relativa estabilidad cambiaria
				\4[] Control de cambios
				\4 Políticas inflacionarias en Estados Unidos en los 60
				\4[] Desestabilizan sistema de Bretton Woods
				\4[] Output cae a principios de los 70
				\4 Crisis del petróleo
		\2 Objeto
			\3 Qué críticas recibió la síntesis neoclásica
			\3 Qué programas de investigación surgen
			\3 Qué conclusiones de política económica se derivaron
			\3 Qué influencia tuvieron sobre la macroeconomía posterior
		\2 Estructura
			\3 Neokeynesianos del desequilibrio
				\4 Idea clave
				\4 Autores destacados
				\4 Valoración
			\3 Monetarismo
				\4 Idea clave
				\4 Demanda de dinero
				\4 Curva de Phillips
				\4 Implicaciones de política económica
				\4 Valoración
	\1 \marcar{Neokeynesianos del desequilibrio}
		\2 Visión general
			\3 Contexto
				\4 Inspiración keynesiana
				\4 No critican Keynes
				\4[] $\to$ Crítica de interpretaciones de Keynes
				\4[] $\to$ Mejorar fundamentación
				\4 Predominancia de paradigma keynesiano
				\4[] Recomendaciones de pol. econ. basadas en SNC
				\4[] $\to$ Predominan en policy making
				\4 Recuperar programa original keynesiano
			\3 Objetivos
				\4 Recuperar mensaje de Keynes
				\4[] Extraer verdaderas innovaciones de Teoría General
				\4[] Superando carencias de IS-LM
				\4[] Incorporando otras herramientas metodológicas
				\4 ¿Por qué se produce el desempleo?
				\4[] Fundamentar problemas de ajuste
				\4[] Argumentar causas de desequilibrio duradero
				\4[$\then$] Mejorar base teórica de Keynesianismo
				\4[$\then$] Permitir evolución del pensamiento keynesiano
			\3 Metodología
				\4 Microeconomía neoclásica
				\4[] Teoría de la demanda
				\4[] Fuerte influencia walrasiana
				\4[] $\to$ Arrow, Debreu, Hicks
				\4[] $\to$ Elementos de Economía Política Pura en inglés (1954)
				\4[] Aplicando ideas de Marshall
				\4[] Criticando ajuste walrasiano
				\4[] $\to$ Idea del ``subastador''
				\4 Marco estático
				\4[] Aunque énfasis sobre periodo entre equilibrios
				\4 Mecanismos de ajuste
				\4[] Escepticismo sobre subastador walrasiano
				\4[] Equilibrios no necesariamente estables
				\4[] $\to$ Proponer razones por las que sucede
		\2 Autores
			\3 Patinkin
				\4 Dinero, Interés y Precios (1956)
				\4[] Análisis conjunto de vars. reales y nominales
				\4[] Traducir Keynes a lenguaje walrasiano
				\4[] $\to$ Keynes es modelo walrasiano simplificado
				\4 Keynes en marco walrasiano
				\4[] 3 mercados
				\4[] $\to$ Trabajo
				\4[] $\to$ Bienes
				\4[] $\to$ Financiero
				\4 Definición de desempleo involuntario
				\4[] Situación en la que trabajo es inferior a oferta
				\4 Rechaza desempleo de equilibrio
				\4[] Realmente, afirma son estados de desequilibrio
				\4 ¿Por qué la oferta de L está racionada en Keynes?
				\4[] Salarios rígidos no son explicación satisfactoria
				\4 Ajuste lento de variables nominales
				\4[] Explicar desempleo como ajuste lento y prolongado
				\4 Saldos monetarios reales en función de utilidad
				\4[] Precede a Sidrauski
				\4[] Sin modelo dinámico microfundamentado
				\4 Efecto de saldos reales empuja economía hacia equilibrio
				\4[] Pero es un ajuste lento
				\4[] Queda debilitado por otros factores
				\4 Crítica a ilusión monetaria
				\4[] Si saldos reales determinan utilidad
				\4[] $\to$ Agentes ya no son indiferentes a $\Delta$ Precios
				\4[$\then$] Predecesor de neokeynesianos del desequilibrio
				\4[] Al intentar compatibilizar Keynes y Walras
				\4[] Aunque es referente de SNC
			\3 Clower
				\4 La contrarrevolución keynesiana: un examen teórico (1965)
				\4[] Relación entre Keynes y teoría de precios tradicional
				\4[] Son compatibles?
				\4[] $\then$ Teoría de precios es caso particular de Keynes
				\4[] $\then$ Ley de Walras es caso concreto
				\4 Hipótesis de la decisión dual
				\4[] En marco walrasiano:
				\4[] Decisiones de compra y venta simultáneas
				\4[] En realidad:
				\4[] Decisiones separadas en el tiempo
				\4[] $\to$ Problema de coordinación puede evitar equilibrio
				\4 Ejemplo:
				\4[] Dos agentes:
				\4[] Productor de champán y consultora
				\4[] Consultora demanda champán y vende servicios
				\4[] Productor vende champán y compra servicios
				\4[] Consultora no compra champán porque no vende servicios
				\4[] Productor no compra servicios porque no vende champán
				\4 Demanda nocional
				\4[] Cantidad deseada
				\4 Demanda efectiva
				\4[] Cantidad a la que puede acceder
				\4[] $\to$ Dada restricción presupuestaria presente
				\4 Ajuste de precios
				\4[] Contexto walrasiano:
				\4[] $\to$ EDemanda en base a dda. nocional
				\4[] $\then$ $\text{ED} > 0$ $\then$ $\uparrow P$ $\then$ $\downarrow \text{ED} < 0$
				\4[] Pero no necesariamente:
				\4[] Mecanismo alternativo: ajuste en base a dda. efectiva
				\4[] $\to$ EDda. efectiva no induce $\Delta$ de precios
				\4[] $\then$ No se alcanza equilibrio
				\4[] \grafica{coordinacionclower}
			\3 Leijonhufvud
				\4 Economía keynesiana y la economía de Keynes (1968)
				\4 Rechazo de IS-LM
				\4[] IS-LM malinterpreta mensaje central de Keynes
				\4 Economía descentralizada
				\4[] $\to$ Problemas de información
				\4[] $\to$ Problemas de señalización
				\4[] $\then$ Coordinación subóptima entre agentes
				\4 Contraste con Patinkin
				\4[] Ajuste lento de precios no basta para desempleo
				\4[] Keynes no es walrasiano
				\4[] $\to$ Keynes es énfasis sobre proceso de ajuste
				\4[] $\to$ No sobre existencia del equilibrio
				\4 Causas adicionales del desequilibrio
				\4[] Tipo de interés a l/p: ajuste más lento de todos
				\4[] $\to$ Precios de los activos demasiado bajos
				\4[] $\to$ Posibles intereses excesivos durante décadas
				\4[] $\to$ Inversión insuficiente
				\4[] Keynes invierte modelo marshalliano de c/p
				\4[] $\to$ Empresas ajustan antes Q que P
				\4[] $\then$ Porque ajuste de precios es muy complejo
			\3 Barro y Grossman
				\4 Barro y Grossman (1971)
				\4[] ``Un modelo de desequilibrio general de renta y empleo''
				\4 Influido por:
				\4[] Patinkin
				\4[] $\to$ Spill-over de un mercado a otro
				\4[] Clower:
				\4[] $\to$ Demandas efectivas y nocionales
				\4 Desempleo no es sólo periodo entre equilibrios
				\4[] Modelizar desempleo como equilibrio
				\4 Tres mercados
				\4[] Trabajo
				\4[] Bien de consumo
				\4[] Bien no producido $\to$ P.ej.: oro
				\4[] $\to$ Analizan dos mercados por Ley de Walras
				\4 Spill-over de un mercado a otro
				\4[] Fundamentado como Clower
				\4[] Para llegar a implicaciones de Patinkin
				\4[$\then$] Posibles resultados
				\4[] Equilibrio walrasiano: $p^*$, $w^*$
				\4[] 1. $p > p^*$, $w > w^*$ $\to$ Régimen keynesiano
				\4[] 2. $p < p^*$, $w > w^*$ $\to$ Régimen clásico
				\4[] 3. $p < p^*$, $w < w^*$ $\to$ Inflación reprimida
				\4[] 4. $p> p^*$, $w < w^*$ $\to$ Subempleo\footnote{Necesario mirar si estaba en el original.}
			\3 Benassy
				\4 Benassy (1975), (1976),(1978)
				\4 Culmina programa neokeynesiano
				\4 Acuña nombre neokeynesianismo
				\4 Recopilación de ideas anteriores
				\4[] Idea de la demanda efectiva de Clower
				\4[] Rechazo del equilibrio walrasiano de Leijonhufvud
				\4[] Modelización microfundamentada de Barro y Grossman
				\4[] Equilibrio con desempleo de Barro y Grossman
				\4 Desarrolla enfoque de eq. general no walrasiano
				\4[] Analiza existencia de equilibrio
				\4[] $\to$ En contexto de equilibrio no walrasiano
				\4[] $\to$ Ajuste en base a demanda
				\4 Extrae esencia de neokeynesianismo
				\4[] Agentes son precio aceptantes
				\4[] Pero precios no se ajustan para eliminar EDemanda
				\4[] $\to$ Son cantidades las que se ajustan
			\3 Malinvaud
				\4 Teoría del desempleo reconsiderada (1977)
				\4 Inspirado en Barro-Grossman
				\4[] $\to$ Simplificar lenguaje neo-walrasiano
				\4 Diagrama de excesos de demanda oferta
				\4[] \grafica{desequilibrios}
				\4 Orientado a policy-makers
				\4[] Ejemplos y aplicaciones prácticas abundantes
				\4 Labor de divulgación de desequilibrio neokeynesiano
			\3 Diamond
				\4 Diamond (1982), Robert (1987), Howitt
				\4[] Inspirado en Leijonhufvud, Clower, Patinkin..
				\4 Modelo de los cocos de Diamond
				\4[] Metáfora con cocos à la islas de Phelps
				\4[] Agentes viven en economía cerrada
				\4[] Pueden producir bienes recogiendo cocos de arboles
				\4[] Tabú impide consumir cocos que uno mismo recoge
				\4[] $\to$ Debe intercambiar con otro agente
				\4 Para que un agente recoja cocos
				\4[] $\to$ Debe tener expectativa de que otros también
				\4[] $\then$ Debe creer que podrá intercambiarlos con otro
				\4 Sin expectativa de intercambiar
				\4[] $\to$ Nadie tendrá incentivo a producir
				\4[] $\then$ Posibles múltiples equilibrios
				\4[] $\then$ Posible capacidad sin utilizar
				\4 Cooper y John (1988)
				\4[] Abandonan idea de rigideces nominales
				\4[] $\to$ Inicialmente entendido como alternativa a NEK 1aGEN
				\4[] $\then$ Posteriormente integrada con Ball y Romer (1991)
				\4[] Complementos estratégicos pueden determinar eq. macro
				\4[] Estrategia óptima de un agente
				\4[] $\to$ Depende positivamente de estrategias de otros
				\4[] $\then$ ``Si nadie produce/vende/baja precios, yo tampoco''
				\4[] Economías pueden quedarse atrapadas en desempleo
				\4[] $\to$ Aunque exista un equilibrio mejor
		\2 Valoración
			\3 Impacto sobre política económica
				\4 No aporta diferencias globales con Keynes
				\4[] Trata de formular en nuevo lenguaje
				\4[] $\to$ Acercándose más a ``verdadero'' Keynes
				\4 Continúa línea de políticas de demanda
				\4[] Ajuste en cantidades y no en precios
				\4[] Demandas efectivas prevalecen sobre nocionales
				\4[] $\to$ Para ajustar precios
				\4[] Necesario crear demanda efectiva
				\4[] $\to$ Políticas fiscales expansivas
				\4[] $\to$ Estímulos a la demanda agregada
				\4[] $\then$ En la línea con Keynes
			\3 Influencia sobre ciencia económica
				\4 Neokeynesianismo como tal no tiene continuidad
				\4[] Aparece NMC
				\4[] $\to$ Inspiración metodológica pero conclusiones muy distintas
				\4[] $\to$ Desempleo es equilibrio con vaciado de mercados
				\4[] $\to$ Análisis dinámico más sencillo
				\4[] $\to$ RBC introduce fuerte componente aplicado
				\4 Conceptos y enfoque sí
				\4[] $\to$ Importancia de procesos de ajuste
				\4[] $\to$ Microfundamentación de rigideces nominales
				\4[] $\to$ Análisis micro del desempleo
	\1 \marcar{Monetarismo}
		\2 Visión general
			\3 Contexto
				\4 Crítica del keynesianismo dominante
				\4[] Aceptación del marco teórico y método keynesiano
				\4[] Rechazo de prescripciones de política económica
				\4[] Tradición del depart. de economía de Chicago
				\4[] $\to$ Knight, Stigler y otros
				\4[] $\to$ El dinero es importante
				\4[] $\to$ Influencia: Fisher, Hume, Mill, Currency School
				\4 Autores destacados
				\4[] \underline{Milton Friedman}
				\4[] Vida
				\4[] $\to$ 1912-2006
				\4[] Influenciado por
				\4[] $\to$ Alfred Marshall
				\4[] $\to$ Irving Fisher
				\4[] $\to$ Keynes
				\4[] $\to$ Knight
				\4[] $\to$ Wicksell
				\4[] Influenció a
				\4[] $\to$ Nueva Macroeconomía Clásica
				\4[] $\to$ Bancos centrales
				\4[] $\to$ Teoría de la demanda consumo
				\4[] $\to$ Economía monetaria
				\4[] Obras
				\4[] -- Un programa para la estabilidad monetaria (1960)
				\4[] -- Capitalismo y libertad (1962)
				\4[] -- Una historia monetaria de los Estados Unidos (1963)
				\4[] -- El rol de la política monetaria (1967)
				\4[] Premio Nobel de 1976
				\4[] $\to$ Análisis de la demanda de consumo
				\4[] $\to$ Historia y teoría monetaria
				\4[] $\to$ Problemas de las políticas de estabilización
				\4[] Otros
				\4[] $\to$ Laidler
				\4[] $\to$ Frenkel
				\4[] $\to$ Anna Schwartz
				\4[] $\to$ Brunner
				\4[] $\to$ Meltzer
			\3 Metodología
				\4 Énfasis macroeconómico
				\4[] Aunque microfundamentación de algunos aspectos
				\4 Enfoque marshalliano de modelización
				\4[] Friedman señala a Marshall como referente
				\4[] $\to$ Énfasis sobre mecanismos concretos de ajuste
				\4[] $\to$ Algunas referencias a Walras también
				\4 Relación con la síntesis neoclásica
				\4[] Acepta y utiliza el marco IS-LM
				\4[] Monetarismo formulable en términos de IS-LM
				\4[] Introduciendo cambios en demandas y oferta
			\3 Ideas centrales
				\4 Ecuación cuantitativa del dinero
				\4[] En último término, describe demanda de dinero
				\4 Inflación como fenómeno monetario
				\4[] En último término, toda inflación resulta de $\Delta M$
				\4 La política monetaria es importante
				\4[] A largo plazo, variables nominales y reales son independientes
				\4[] Pero PM tiene efectos sobre output
				\4 Autoridades no tienen suficiente información
				\4[] Conectado con von Hayek
				\4[] Estabilización a c/p acaba perjudicando
				\4[] Imposible conocer reacción y lags
				\4[] ¿Cómo $\Delta M$ se transmiten a $\Delta P$ y $\Delta Y$?
				\4[] $\to$ Ecuación perdida del monetarismo
				\4[] $\then$ Pequeñas fluctuaciones son inevitables
				\4[] $\then$ Rechazo del \textit{fine-tuning}
				\4 Política monetaria basada en reglas estables
				\4[] Crecimiento sostenido de la oferta monetaria
				\4 Los mercados privados son estables
				\4[] Tienden a la utilización plena de recursos
				\4[] Sin necesidad de intervención
				\4[] $\to$ Economía tiende a pleno empleo
				\4 Curva de Phillips es vertical en el largo plazo
				\4[] $\nexists$ trade-off permanente empleo-inflación
				\4[] $\nexists$ una relación explotable entre empleo e inflación
				\4[] $\exists$ sólo un trade-off temporal
				\4[] $\then$ Rechazo de CPhillips como menú de políticas
		\2 Curva de Phillips
			\3 Consenso pre-monetarista en policy-making
				\4 Relación precios y M no es relevante
				\4[] Dependen de:
				\4[] -- Excesos de demanda
				\4[] $\to$ demand-pull
				\4[] -- Shocks de oferta
				\4[] $\to$ cost-push
				\4[$\then$] No es necesariamente un fenómeno monetario
				\4[$\then$] Dda. reduce desempleo por inflación
				\4[$\then$] Posible reducir desempleo a cambio de inflación
				\4[$\then$] Disponible menú de política económica
			\3 Idea clave
				\4 La curva de Phillips es un fenómeno de corto plazo
				\4[] $\to$ En el largo plazo, es vertical
				\4 Desempleo por debajo de natural tiene un coste
				\4[] Estímulo monetario cada vez debe ser mayor
				\4[] $\then$ Inflación cada vez más elevada
				\4 Inflación como fenómeno monetario
				\4[] En el largo plazo, producción no depende de $\Delta M$
				\4[] Variaciones de precios se deben a variación de M
			\3 Tasa natural de desempleo (TND)
				\4 Concepto relativamente difuso
				\4 Influencia de Wicksell
				\4[] ``tipo de interés natural''
				\4 Introducido en discurso presidencial AEA (1968)
				\4[] ``nivel resultante del sist. walrasiano de ecuaciones''
				\4 Desempleo que corresponde a eq. de salarios reales
				\4[] Demanda y oferta depende de salario real
				\4[] $\to$ TND es aquella que iguala oferta y demanda
				\4[] $\then$ Salario real estable
				\4 Factores institucionales y de estructura del mercado
				\4[] Determinan tasa natural
				\4[] P. ej.: facilidad para ``encontrarse''
				\4[] Poder de mercado de sindicatos
			\3 Crítica de Friedman
				\4 Partiendo de ecuación cuantitativa del dinero
				\4[] $\to$ Inflación es fenómeno monetario
				\4 Oferta y dda. de trabajo dependen de salario real
				\4[] $\to$ No de salario nominal
				\4[] $\to$ Pero salario real sólo puede estimarse
				\4[$\then$] ¿Cómo afectan estímulos monetarios al desempleo?
				\4 Oferta de trabajo
				\4[] Hipótesis de expectativas adaptativas (HEA):
				\4[] \fbox{$E_{t}(P_{t+1}) = E_{t-1} (P_t) + \lambda \left( P_t - E_{t-1} (P_t) \right)$}
				\4[] Trabajadores estiman mediante HEA
				\4[] $\to$ \underline{No} observan aumento de precios presente
				\4[] $\to$ \underline{Sí} observan aumento de salario nominal
				\4[] $\then$ Salario real estimado aumenta
				\4[] $\then$ Aumentan oferta de trabajo
				\4 Demanda de trabajo
				\4[] 1. Estímulo monetario aumenta precios
				\4[] 2. Empresas observan aumento de precio
				\4[] $\to$ Aumentan demanda de trabajo
				\4[] 3. Aumenta W pero menos que P
				\4[] $\to$ Porque trabajadores estiman con HEA
				\4[] $\then$ Salario real baja
				\4[] $\then$ Aumentan demanda de trabajo
				\4 Periodo inmediato
				\4[] Aumenta inflación
				\4[] Aumenta trabajo utilizado
				\4 Reajuste de expectativas de inflación
				\4[] Estiman de nuevo inflación
				\4[] Tienen en cuenta ``sorpresa anterior''
				\4[] $\to$ Estiman menor salario real
				\4[] $\then$ Baja oferta de trabajo
				\4[$\then$] Vuelta a desempleo natural
				\4 Curva de Phillips aumentada por las expectativas
				\4[] Resume lo anterior
				\4[] \fbox{$\pi_t  = f(\bar{u} - u) + \pi^e_t$}
				\4[] $\bar{u}$: tasa de paro natural
				\4[] $f'(\cdot) > 0$
				\4[] $f(0) = 0$
				\4[] $u< \bar{u} \then f(\bar{u} - u) > 0 \then \pi_t > \pi^e_t$
				\4[] CPhillips como fenómeno de corto plazo
				\4[] $\to$ En el l/p, es una recta vertical
			\3 Reducción permanente del desempleo
				\4 Requiere $\uparrow$ constantes de la tasa de inflación
				\4[] \grafica{curvadephillipsmonetarista}
				\4 Cada vez más costoso reducir desempleo
				\4 A l/p, la curva de Phillips es vertical
				\4 Inflación muy elevada:
				\4[] $\to$ Sistema de precios es inoperante
				\4[] $\then$ Curva de Phillips puede volverse creciente
			\3 Phelps
				\4 Microfundamentación de la curva de Phillips
				\4[] $\to$ Dos modelos: 1968 y 1970
				\4[] Explicar CPhillips en base a decisiones de agentes
				\4[] Establecer CPhillips de c/p y l/p
				\4 1968: Predecesor de modelos de búsqueda
				\4[] Desempleo es resultado de problema de emparejamiento
				\4[] Estímulo aumenta vacantes y empleados dimiten
				\4[] Para evitarlo, empresas aumentan salario
				\4[] Aumento de salarios reduce vacantes
				\4[] $\to$ Equilibrio con desempleo ``natural''
				\4 1970: Inspirador de modelo de Lucas
				\4 Encontrar empleados es costoso
				\4[] $\to$ Empresas mantienen salarios altos
		\2 Política monetaria
			\3 Teoría cuantitativa del dinero
				\4 Elemento central del monetarismo
				\4 Diferentes formulaciones
				\4[] Transacciones: $MV = PT$
				\4[] Ingreso: $MV = PY$
				\4[] Saldos nominales (Cambridge): $M=kPY$
				\4[] Otras
				\4 Posible formulación tautológica o como identidad
				\4[] $k$ o $V$ se adaptan pasivamente
				\4[] $\to$ Denominada ``ecuación de intercambio''
				\4 Si $k$ o $v$ no se adaptan pasivamente
				\4[] $\to$ Demanda de dinero es estable
				\4[] $\to$ Teoría cuantitativa del dinero propiamente
				\4[] $\to$ Modelo de relación entre dinero y gasto nominal
				\4[] $\to$ Posible caracterizar ciclos
				\4 Múltiples interpretaciones
				\4[] Friedman define su teoría como basada en la TCT
				\4[] $\to$ En el que el dinero es importante
				\4[] $\to$ La masa monetaria es exógena, determinada por BC
				\4[] Causalidad de MV $\to$ PY
%			\3 Demanda de dinero Keynesiana / síntesis neoclásica
%				\4 Precios dependen de factores institucionales
%				\4 Demanda de dinero depende de interés y renta presente
%				\4[] $\to$ $\frac{M^D}{P} = L(r,Y) = L_{\text{transacción}}(Y) + L_{\text{especulacion}}(r)$
%				\4 Motivo de transacción
%				\4[] $\to$ Tenencia de dinero reduce costes de transacción
%				\4[] $\to$ Cuanta más renta, más dinero se demanda
%				\4 Motivo de especulación
%				\4[] $\to$ $r$ porque tenencia de dinero tiene coste de oportunidad
%				\4[] $\to$ A mayor interés, menos rentable tener dinero
%				\4[] $\to$ Tipos bajarán, bonos aumentarán precio y beneficio a extraer
%				\4 Trampa de liquidez
%				\4[] $\to$ Elasticidad demanda de dinero-interés tiende a infinito
%				\4[] $\then$ $k$ se adapta pasivamente a $\Delta M$
%				\4[] $\then$ $\Delta M$ es poco efectivo para estabilizar economía
			\3 Demanda de dinero monetarista
				\4 Demanda de dinero es estable
				\4[] $\to$ Estable como misma forma funcional
				\4[] $\to$ No en sentido "siempre misma cantidad''
				\4 Depende de estructura de tipos interés
				\4[] $\to$ Bonos c/p y l/p
				\4[] $\to$ Acciones
				\4[] $\to$ ...
				\4[] $\then$ Dinero entendido como activo en una cartera
				\4[] $\then$ Dinero y bonos tienen otros sustitutivos
				\4[] $\then$ Trampa de liquidez imposible y/o irrelevante
				\4 Depende de riqueza total no-capital humano
				\4[] $\to$ Ingreso permanente utilizable como proxy
				\4 Depende de inflación
				\4[$\then$] $M^D = P\cdot  L(y, w, r_b, r_e, \pi^e, u, ...)$
			\3 Mecanismos de transmisión dinero-ingreso nominal
				\4 ¿Cómo afectan $\Delta M$ a precios, producción e interés?
				\4[] $\to$ existen varios mecanismos
				\4[] $\to$ Datos empíricos favorecen efecto directo
				\4 Efecto directo
				\4[] Friedman entiende como efecto más importante
				\4[] Partiendo de Ley de Walras
				\4[] En contexto Keynesiano:
				\4[] $M^D -  M^S + B^D - B^S = 0$
				\4[] $\then$ $M^S > M^D$ se cubre con demanda de bonos
				\4[] Friedman afirma que:
				\4[] \fbox{$(M^D - M^S) + (B^D - B^S) + (Y^D - Y^S) = 0$}
				\4[] $\then$ $M^S > M^D$ se pueden cubrir con demanda de bienes duraderos
				\4[] $\then$ $\Delta M$ no tiene por qué reducir tipo de interés
				\4[] Aumento de Y aumenta demanda de saldos reales
				\4[] $\then$ Mercado de dinero se equilibra
				\4[] $\then$ Aumento de M tiene efecto directo sobre Y
				\4 Efecto indirecto o de Keynes
				\4[] Monetaristas dan más importancia que propio Keynes
				\4[] $\to$ Inversión más sensible a interés
				\4[] $\to$ Animal spirits no son tan importantes
				\4[] $\uparrow \frac{M^S}{P} \to \uparrow L(Y,r) \to \uparrow Y, \uparrow r$
				\4[] Alternativamente:
				\4[] $M^D - M^S + B^D - B^S + Y^D - Y^S = 0$ dado $Y^D - Y^S = 0$
				\4[] $\then$ $\Delta M^S >0  \to \Delta r <0 \to \Delta I(r) >0$
				\4[] En términos gráficos, LM se desplaza a la derecha
				\4 Efecto riqueza/de Pigou
				\4[] Crítica a trampa de liquidez
				\4[] Deflación aumenta valor real de saldo monetario
				\4[] $\downarrow$ P $\to$ $\uparrow \frac{M}{P} \to \uparrow L \to \uparrow Y, \downarrow i$
				\4[] $\then$ Aumenta poder adquisitivo de hogares
				\4[] Aumento de poder adquisitivo
				\4[] $\then$ Aumenta demanda de consumo
				\4[] Empíricamente poco relevante
				\4 ¿Cómo se reparten $\Delta M$ en $\Delta P$ y $\Delta Y$?
				\4[] Cuestión empírica
				\4[] Monetarismo no aporta respuesta teórica
				\4[] $\then$ ``Ecuación perdida del monetarismo''
				\4[] Monetarismo afirma existen
				\4[] $\to$ Lags
				\4[] $\to$ Factores de incertidumbre
				\4[] $\then$ Imposible conocer transmisión exacta
				\4 Efectos de primera ronda\footnote{De \textit{quantity theory of money} (Palgrave), apartado \textit{f) first-round effects}.}
				\4[] ¿Es importante la manera en que $\Delta M$?
				\4[] $\to$ Compra de bonos
				\4[] $\to$ Compra directa de bienes
				\4[] Friedman cree que es una cuestión empírica
				\4[] Citando estudios de Cagan
				\4[] $\to$ Efectos de primera ronda no son relevantes
		\2 Política fiscal
			\3 Crowding-out de la inversión
				\4 Demanda de dinero muy poco sensible a interés
				\4[] Para equilibrar mercado monetario
				\4[] $\to$ Aumentos de output...
				\4[] $\then$ ...requieren fuertes aumentos del interés
				\4 Demanda de inversión
				\4[] Muy sensible al tipo de interés
				\4[] Déficits públicos financiados con deuda
				\4[] $\to$ Aumento del interés de la deuda pública
				\4[] $\to$ Demanda de dinero poco sensible a interés
				\4[] $\to$ Necesario fuerte $\uparrow r$ de interés para eq. mercado monetario
				\4[] $\then$ Contracción por canal indirecto/keynes
				\4[] $\then$ Crowding-out de la inversión privada
			\3 Expansión fiscal contractiva
				\4 Efecto riqueza
				\4[] Déficit aumenta DPública en manos de SPrivado
				\4[] Deuda pública es outside-money para SPrivado
				\4[] $\to$ Aumento de riqueza de sector privado
				\4[] Aumento de riqueza aumenta demanda de dinero
				\4[] $\to$ Desplaza curva LM hacia dentro
				\4[] $\then$ Efecto contractivo del aumento de riqueza
				\4[] $\then$ Reduce multiplicador del gasto vía déficit
				\4[] Aumento de riqueza aumenta demanda de consumo
				\4[] $\to$ Desplazamiento ulterior de IS hacia afuera
				\4[] $\then$ Efecto expansivo del aumento de riqueza
				\4[] $\then$ Cuestión empírica si es menor o mayor que el otro
				\4[] Monetaristas afirman tiene efecto total contractivo
			\3 Financiación vía deuda vs monetaria
				\4 Déficit público financiado con deuda
				\4[] $\to$ Aumenta interés y crowding-out
				\4[] $\to$ Induce efecto riqueza
				\4[] $\then$ Muy poco efecto sobre output
				\4 Déficit público financiado con emisión de dinero
				\4[] $\to$ Mantiene interés bajo
				\4[] $\to$ Aumenta oferta monetaria
				\4[] $\then$ Efecto sobre output porque aumenta dinero
		\2 Implicaciones de política económica
			\3 Marco IS-LM
				\4 Criticado por Friedman, Brunner, Meltzer
				\4 A partir de 70s, Friedman acepta
				\4[] $\to$ Como herramienta para divulgar y debatir
				\4 Útil para exponer diferencias
				\4[] Mantener presente que monetarismo en IS-LM
				\4[] $\to$ Es una simplificación
			\3 Política monetaria
				\4 Síntesis neoclásica
				\4[] Si trampa de liquidez:
				\4[] $\to$ efectos nulos
				\4[] \grafica{keynestrampadeliquidez}
				\4[] Si no trampa de liquidez:
				\4[] $\to$ Efectos pequeños de PM
				\4 Monetarismo
				\4[] La oferta de dinero es exógena
				\4[] Controlable por el banco central
				\4[] Mecanismos de transmisión implican:
				\4[] $\Delta M^S \to \Delta PY$
				\4[] $V$/$k$ son estables
				\4[] $\to$ No sujetos a cambios impredecibles
				\4[] En el corto plazo, $\Delta M$ afecta producción
				\4[] En el largo plazo, Y está determinado por factores reales
				\4[] $\then$ Inflación es un fenómeno monetario
				\4[] \grafica{politicamonetariamonetarista}
				\4 Ciclos tienen origen monetario
				\4[] Fluctuaciones de $M$ afectan a Y
				\4[] Con lags variables, shocks reales aleatorios, errores de estimación
				\4[] $\then$ PM no puede estabilizar ciclo
				\4[] $\then$ PM activa provoca ciclos
				\4 Política monetaria contracíclica
				\4[] Inicialmente, Friedman recomendó PM contracic.
				\4[] $\to$ Apoya en 1948 el Plan de Chicago de Simons-Mints
				\4 Regla de política monetaria
				\4[] Programa de Estabilidad Monetaria (1959)
				\4[] Basado en trabajos empíricos
				\4[] $\to$ muestran lags variables
				\4[] Propone regla de $\Delta \%$ fijo de la oferta monetaria
				\4[] Banco Central sujeto a poder legislativo
				\4[] $\to$ no ejecutivo
				\4[] Contrario a la independencia de los BC
				\4[] Propone coeficiente de reservas del 100\%
			\3 Política fiscal
				\4 Síntesis neoclásica
				\4[] PF utilizable para estabilizar ciclo
				\4 Monetarismo
				\4[] Macroeconomía es estable
				\4[] Curva de oferta agregada es casi vertical
				\4[] Inversión es sensible a tipo de interés
				\4[] $\to$ PF es inútil para estabilizar economía
				\4[] $\to$ Tipos altos provocan crowding-out de inv. privada
		\2 Valoración
			\3 Estudios empíricos
				\4 {Gran Depresión}
				\4[] Reserva Federal agravó la caída del producto
				\4[] Tipos de interés bajos no indicaban liquidez
				\4[] M cayó fuertemente
				\4[] $\to$ Cayó demanda de bienes
				\4[] $\to$ Cayeron precios
				\4[] $\to$ Subieron salarios reales
				\4[] $\then$ Contracción enorme pero evitable
				\4 {Friedman y Schwartz -- Historia Monetaria de los Estados Unidos (1963)}
				\4[] Fluctuaciones monetarias preceden expansiones/contracciones
				\4[] Bancos centrales son responsables de ciclos
				\4 {Ecuación de St. Louis}
				\4 Regresión de serie temporal
				\4[] $\Delta Y_t = a  + \sum_{i=0}^4 m_i \Delta M_{t-i} + \sum_{t=0}^4 e_i \Delta E_{t-i}$
				\4 Valorar efecto de política monetaria y fiscal
				\4[] Sobre producto nominal
				\4 Parámetros estimados
				\4[] $\to$ Política monetaria > política fiscal
				\4 Sims (1972)
				\4[] Técnicas econométricas avanzadas
				\4[] $\to$ Causalidad de Granger
				\4[] Para diferenciar sentido de la causalidad
				\4[] $\Delta M \to Y$ o $\Delta Y \to M$
				\4[] Encuentran que $\Delta M$ Granger-causa $Y$
				\4 Brunner y Meltzer (1976)
				\4[] PF financiada mediante $\Delta M$
				\4[] $\to$ Más efecto que PF financiada con bonos
				\4 Curva de Phillips
				\4[] En los años 70, la relación u-$\pi$ se rompe
				\4[] Justo a tiempo para dar fuerza a monetarismo
			\3 Críticas
				\4 Expectativas adaptativas
				\4[] Es razonable suponer errores sistemáticos?
				\4[] Los trabajadores se equivocan sistemáticamente
				\4[] No aprenden del error?
				\4[] Están siempre dispuestos a trabajar más por menos?
				\4 Tasa natural de desempleo
				\4[] Monetarismo identifica natural con óptimo
				\4[] $\to$ Pero no necesariamente
				\4 NAIRU vs Tasa Natural de Desempleo
				\4[] \textit{Non-Accelerating Inflation Rate of Unemployment}
				\4[] $\to$ Eliminar connotación positiva de ``natural''
				\4[] $\to$ Crítica a la idea de estabilización automática
				\4 No es constante y sólo se puede estimar
				\4[] Friedman reconoce estas críticas
				\4 Oferta monetaria
				\4[] Críticos niegan que banco central pueda controlar
				\4[] Base monetaria puede ser controlada
				\4[] Pero bancos privados piden prestadas reservas
				\4[] $\to$ ¿Debe limitar provisión de liquidez de emergencia?
				\4[] Si no limita provisión de liquidez:
				\4[] $\to$ Riesgo moral
				\4[] $\to$ Credibilidad de la regla dañada
				\4[] Si limita:
				\4[] $\to$ Riesgo de aumentar recesión $\to$ Depresión
				\4 Demanda de dinero
				\4[] Críticos afirman que es inestable
				\4[] Si interés bajo, muy elástica a tipo de interés
				\4[] Innovaciones financieras son impredecibles
				\4[] $\to$ Alteraciones impredecibles de la demanda
				\4[] Controlar oferta monetaria + fluctuaciones demanda
				\4[] $\to$ Inestabilidad de inflación y output
				\4 Política monetaria
				\4[] Modelo explícito ausente
				\4[] $\to$ Dificulta lectura actual de Friedman
				\4[] Experimento monetarista
				\4[] $\to$ Finales 70 a finales 80
				\4[] $\to$ Bancos Centrales priorizan agregados monetarios
				\4[] Grandes fluctuaciones de output
				\4[] $\to$ A pesar de reducciones pequeñas de inflación
				\4[] Bancos centrales abandonan a mediados de los 80
				\4[] $\to$ Agregados monetarios dejan de ser prioridad
				\4[] Friedman niega implementación de sus recomendaciones
				\4[] ECB siguió considerando agregados monetarios
				\4 Impulsores del monetarismo en PM
				\4[] $\to$ Friedman, Schwartz, Brunner
			\3 Influencia
				\4 Causas de la Gran Depresión
				\4[] Consenso general: Reserva Federal culpable
				\4[] $\to$ Dinero causa crisis
				\4[] Discurso de Bernanke (2002)
				\4[] $\to$ ``We're sorry, we did it''
				\4 Relevancia de la política monetaria
				\4[] Instrumento fundamental de política económica
				\4[] $\to$ Macroeconomía monetaria
				\4[] Papel central en macroeconomía actual
				\4[] Consenso sobre importancia de factores monetarios en inflación
				\4[] Reglas vs discrecionalidad
				\4[] Aporta argumentos a favor de independencia bancos centrales
				\4[] $\to$ Debate aún actual
				\4 Nueva Macroeconomía Clásica y NEK
				\4[] Papel de las expectativas
				\4[] Ciclos monetarios
				\4[] $\to$ NMC inicial, también influenciada por Phelps
				\4[] $\to$ Posteriormente, RBC obvia ciclos nominales
				\4[] Fluctuaciones alrededor de tendencia/tasa natural
				\4[] $\to$ No sistemáticamente por debajo de potencial
				\4 Libre mercado y libertades individuales
				\4[] Ataque a uso abusivo de PM por Gobiernos
				\4[] Defensa de libre mercado
				\4[] Política basada en reglas
				\4[] $\to$ No en intervenciones puntuales
	\1[] \marcar{Conclusión}
		\2 Recapitulación
			\3 Neokeynesianos del desequilibrio
			\3 Monetarismo
		\2 Idea final
			\3 Robert Solow sobre modelos macro y economistas
				\4 Existen dos tipos de macroeconomistas
				\4 Macroeconomistas que formulan modelo canónico
				\4[] Y tratan de resolver todas las preguntas con el
				\4[] $\to$ Aplicando ligeros cambios
				\4 Macroeconomistas que utilizan un conjunto de modelos
				\4[] Cada uno para resolver diferentes cuestiones
				\4 Neokeynesianos del desequilibrio
				\4[] Enfoque heterogéneo
				\4[] Reinterpretación de un modelo general
				\4[] Incorporar Keynes en lengua
				\4 Monetarismo
				\4[] Sin modelo general explícito
				\4[] Énfasis en mecanismos concretos
				\4[] $\to$ Dinero
				\4[] $\to$ Consumo
				\4[] $\to$ ...
			\3 Recomendaciones de escuelas macroeconómicas
				\4 Tienden a perder relevancia
				\4[] Cambio en contexto económico y político
				\4 Nuevos argumentos a menudo desacreditan
			\3 Aportación de escuelas macroeconómicas
				\4 No tanto recomendaciones de pol. económica...
				\4[] ...como otros factores
				\4 Apertura de nuevos programas de investigación
				\4 Avance de la metodología
				\4[] Que permitan mejor:
				\4[] $\to$ Comprensión del funcionamiento de la economía
				\4[] $\to$ Predicciones respecto de impacto de políticas
\end{esquemal}































\graficas


\begin{axis}{4}{Desempleo involuntario como resultado de un problema de coordinación en Clower (1965) }{L}{c}{coordinacionclower}
	% Producción
	\draw[-] (0.7,1) to [out=60, in=185](4,2.8);
	\node[right] at (4,2.8){\tiny $Y$};
	
	% curva de indiferencia de equilibrio
	\draw[-] (0.99,2.17) to [out=10, in=250](3.19,3.29);
	
	% salario de equilibrio
	\draw[-] (0.7,1.73) -- (4,3.17);
	\node[right] at (4,3.17){\tiny $L^*$};
	
	% curva de indiferencia de desequilibrio
	\draw[-] (0.73,2.47) to [out=10, in=250](2.93,3.53);
	
	% salarios de desequilibrio
	\draw[-] (0.7, 1.22) -- (3.7,4.05);
	\node[right] at (3.7,4.05){\tiny $L' > L^*$};
	
	% demanda y oferta de trabajo de equilibrio walrasiano
	\draw[dashed] (2.45,2.5) -- (2.45,0);
	\node[below] at (2.40,0){\tiny $l^*$};
	
	% demanda de trabajo de desequilibrio
	\draw[dashed] (1.3,1.78) -- (1.3,0);
	\node[below] at (1.3,0){\tiny $l_\text{\tiny D}$};
	
	% oferta de trabajo de desequilibrio
	\draw[dashed] (2.6,3) -- (2.6,0);
	\node[below] at (2.7,0){\tiny $l_S$};
	
	% demanda y oferta de bienes de equilibrio walrasiano
	\draw[dashed] (2.45,2.5) -- (0,2.5);
	\node[left] at (0,2.5){\tiny $c^*$};
	
	% demanda de bienes de desequilibrio
	\draw[dashed] (2.6,3) -- (0,3);
	\node[left] at (0,3){\tiny $c_D$};
	
	% oferta de bienes de desequilibrio
	\draw[dashed] (1.3,1.78) -- (0,1.78);
	\node[left] at (0,1.78){\tiny $c_S$}; 
\end{axis}

La curva cóncava Y muestra la frontera de posibilidades de producción. Las dos curvas convexas paralelas son curvas de indiferencia de un consumidor representativo. La recta $L^*$ representa el salario de equilibrio walrasiano. La recta $L'$ cuya pendiente es mayor que $L^*$ representa un salario mayor al de equilibrio walrasiano que induce un exceso de oferta de trabajo ($l_S > l_D$) y un exceso de demanda de consumo. Definiendo la demanda nocional como aquella cantidad que un agente desearía consumir, y demanda efectiva como aquella cantidad a la que efectivamente accede, es posible argumentar que una situación de desequilibrio puede ser estable. En el marco walrasiano, los excesos de demanda que inducen variaciones de precios se derivan de diferencias entre demanda nocional y oferta. En el contexto del gráfico, un exceso de demanda derivado de la demanda nocional induciría un aumento del precio del consumo, reduciendo el salario real y por ello, la pendiente de la recta hasta alcanzar el equilibrio walrasiano. Sin embargo, si los excesos de demanda no fuese calculados a partir de la demanda nocional sino de la demanda efectiva, estos no serían tales y el sistema no tendería hacia el ajuste. La conclusión que Clower extrae es que cuando el ingreso aparece como una variable independiente en la función de exceso de demanda, la teoría de precios tradicional no permite derivar conclusiones sobre la estabilidad de una economía y su tendencia hacia el equilibrio.

\begin{tabla}{Posibles estados de desequilibrio recogidos por Malinvaud a partir del modelo de Barro-Grossman}{desequilibrios}
	\begin{tabular}{l l c c}
	& & \multicolumn{2}{c}{\textbf{Mercado de bienes}}\\ \cline{3-4}
	& & \textit{ES} & \textit{ED} \\ \hline
	\multirow{2}{*}{\textbf{Mercado de trabajo}} & \textit{ES} & Paro keynesiano & Paro clásico \\
	& \textit{ED} & Subempleo & Inflación reprimida \\ \hline
	\end{tabular}
\end{tabla}

\begin{axis}{4}{Curva de Phillips propuesta por Friedman que asume la hipótesis de expectativas adaptativas.}{$u$}{$\pi$}{curvadephillipsmonetarista}
	\draw[-] (0.1,3.5) to [out=275,in=175](4,-1);
	
	\draw[-] (0.7,4.2) to [out=275,in=175](4.7,-0.3);
	
	\draw[-] (1.4,4.9) to [out=275,in=175](5.4,0.4);
	
	% paro natural
	\draw[dashed] (1.69,0) -- (1.69,4);
	
	\node[below] at (1.69,0){$u^*$};
	
	% primer aumento del empleo
	\draw[dashed,-{Latex}] (0.64,1.42) -- (1.69,1.42);
	
	\draw[-{Latex}] (1.69,0) to [out=138, in=300](0.64,1.42);
	
	% segundo aumento del empleo
	\draw[dashed,-{Latex}] (0.81,3.47) -- (1.69,3.47);
	
	\draw[-{Latex}] (1.69,1.42) to [out=124, in=282](0.81,3.47);
	
	% paro menor que tasa natural con curva de Phillips inicial
	%\draw[dashed] (0.64,1.42) -- (0.64,0);
	
	% paro menor que tasa natural con curva de Phillips con curva de Phillips ajustándose a equilibrio
	%\draw[dashed] (0.81, 3.47) -- (0.81,0);
\end{axis}

\begin{axis}{4}{En un contexto de supuestos keynesianos, efectos de la política monetaria en una situación de trampa de liquidez}{y}{r}{keynestrampadeliquidez}
	% IS
	\draw[-] (0.5,4) -- (1.9,0);
	\node[above] at (0.5,4){IS};
	
	% LM antes del aumento de la oferta monetaria
	\draw[-] (0,1) -- (1.5,1) to [out=0, in=270](4,4);
	\node[above] at (4,4){$\text{LM}_0$};
	
	% LM después del aumento de la oferta monetaria
	\draw[-] (0,1) -- (1.5,1) to [out=0, in=270](5.5,4);
	\node[above] at (5.5,4){$\text{LM}_1$};
\end{axis}

\begin{axis}{4}{En un contexto de supuestos monetaristas, efectos de una política monetaria expansiva en el corto plazo}{y}{r}{politicamonetariamonetarista}
	% IS monetarista con elasticidad al tipo de interés alta
	\draw[-] (0.5,3) -- (4,1.5);
	\node[above] at (0.5,3){IS};
	
	% LM antes de expansión monetaria
	\draw[-] (0.5,0.5) -- (1.5,4);
	\node[above] at (1.5,4){$\text{LM}_0$};
	
	% LM post expansión monetaria
	\draw[-] (2.5,0.5) -- (3.5,4);
	\node[above] at (3.5,4){$\text{LM}_1$};
\end{axis}

\conceptos

\concepto{Ley de Goodhart}

La Ley de Goodhart fue postulada por Charles Goodhart en 1975 para criticar la política monetaria que Margaret Thatcher comenzaba a implementar y que se basaba en la fijación de objetivos relativos a la base y la oferta monetaria. La Ley de Goodhart afirma que un indicador deja de ser útil como indicador cuando se convierte en objetivo. Aplicado a los planteamientos monetaristas, constituye una crítica a la validez de la oferta monetaria como determinante del producto nominal. 

\preguntas

\seccion{Test 2018}
\textbf{2.} Entre las críticas realizadas a los modelos neokeynesianos del desequilibrio puede citarse:

\begin{itemize}
	\item[a] La premisa de que en el largo plazo se alcanza una situación de pleno empleo.
	\item[b] El no conceder importancia a los fallos de coordinación.
	\item[c] El no establecer los fundamentos microeconómicos de las rigideces de precios y salarios.
	\item[d] El considerar sólo un tipo de desempleo involuntario.
\end{itemize}

\seccion{Test 2011}
\textbf{2.} Según Malinvaud, la situación de subconsumo surge como consecuencia de la combinación de:
\begin{itemize}
	\item[a] Un exceso de demanda en el mercado de bienes y un exceso de oferta en el mercado de trabajo.
	\item[b] Un exceso de demanda en el mercado de bienes y un exceso de demanda en el mercado de trabajo.
	\item[c] Un exceso de oferta en el mercado de bienes y un exceso de demanda en el mercado de trabajo.
	\item[d] Un exceso de oferta en el mercado de bienes y un exceso de oferta en el mercado de trabajo.
\end{itemize}

\seccion{Test 2007}
\textbf{1.} La tesis de Milton Friedman de que el Gobierno debe abstenerse de realizar políticas de estabilización puede apoyarse en el hecho de que lleva un tiempo:
\begin{itemize}
	\item[a] Conocer cual es el verdadero estado de la economía.
	\item[b] Diseñar y llevar adelante cualquier política.
	\item[c] Hasta que las políticas inciden sobre la economía.
	\item[d] Todas las anteriores.
\end{itemize}

\textbf{2.} Señale cuál, de entre las siguientes afirmaciones relativas a las políticas diseñadas para reducir el paro de una economía, es INCORRECTA:
\begin{itemize}
	\item[a] Una reducción de los impuestos indirectos es eficaz en situaciones de paro clásico.
	\item[b] Un aumento del gasto público no conseguirá reducir el paro si éste es de tipo clásico.
	\item[c] Las políticas de demanda son más eficaces que las de oferta en situaciones de paro keynesiano.
	\item[d] Con independencia del tipo de paro que haya en la economía, las políticas de oferta son siempre más eficaces que las de demanda para reducir el paro.
\end{itemize}

\seccion{Test 2004}
\textbf{2.} Suponga un modelo macroeconómico tradicional keynesiano, donde explicamos el ciclo económico en función del comportamiento de los mercados de bienes y trabajo. Señale cuál de las siguientes afirmaciones es FALSA:
\begin{itemize}
	\item[a] Si existe rigidez nominal en el precio de los bienes, los salarios son flexibles y el mercado de trabajo es competitivo, al igual que ocurre en el modelo planteado por Keynes (1936) en la Teoría General, las fluctuaciones en la demanda agregada pueden causar desempleo involuntario.
	\item[b] Los modelos de la Macroeconomía del Desequilibrio parten de la rigidez nominal en el precio de los bienes y en los salarios, planteando el ajuste de los mercados vía cantidades. Todos estos modelos son típicamente walrasianos, excepto por la ausencia de un subastador.
	\item[c] Si existe rigidez nominal en el precio de los bienes, los salarios son flexibles y el mercado de trabajo presenta características no walrasianas, entonces el comportamiento de los salarios reales se corresponde con lo predicho por la teoría de los salarios de eficiencia. En este caso, los movimientos en la demanda agregada también pueden generar desempleo involuntario.
	\item[d] Los modelos de la nueva macroeconomía keynesiana, en general, no pueden caracterizarse por el comportamiento anticíclico o procíclioco de los salarios reales.
\end{itemize}

\notas

\textbf{2018:} \textbf{2.} C

\textbf{2011:} \textbf{2.} C

\textbf{2007:} \textbf{1.} D \textbf{2.} D

\textbf{2004:} \textbf{2.} A

\bibliografia

Mirar en Palgrave:
\begin{itemize}
	\item adaptive expectations
	\item dynamic models with non-clearing markets
	\item fixprice models
	\item Friedman, Milton
	\item Kalecki, Michal
	\item Keynesianism
	\item monetarism
	\item monetary economics, history of
	\item monetary policy, history of
	\item monetary and fiscal policy overview
	\item neoclassical synthesis
	\item non-clearing markets in general equilibrium
	\item permanent-income hypothesis
	\item Phelps, Edmund
	\item Phillips curve
	\item Phillips curve (new views)
	\item Post Keynesian economics
	\item real balances
	\item quantity theory of money
	\item sticky wages and staggered wage setting
	\item underemployment equilibria
\end{itemize}

\textbf{{\large Neokeynesianos del desequilibrio}}


FALTAN CLOWER, LEIJ, PATINKIN, BARRO Y GROSSMAN, MALINVAUD

Benassy, J. P. (1975) \textit{Neo-Keynesian Disequilibrium Theory in a Monetary Economy} The Review of Economic Studies, Vol. 42, No. 4

Leijonhufvud, A. \textit{On Keynesian Economics And The Economics Of Keynes: A Study In Monetary Theory} (1968) \url{https://ia600804.us.archive.org/31/items/EconomicsOnKeynesianEconomicsAndTheEconomicsOfKeynesAStudyInMonetaryTheoryAxelLe/Economics-\%20On\%20Keynesian\%20Economics\%20and\%20the\%20Economics\%20of\%20Keynes\%2C\%20A\%20Study\%20in\%20Monetary\%20Theory\%2C\%20Axel\%20Leijonhufvud\%2C\%20Oxford\%20University\%20Theory\%20\%281968\%29\%204.pdf}



\textbf{{\large Monetarismo}}

Blanchard, O. \textit{Should We Reject the Natural Rate Hypothesis?} (2018, Winter) Journal of Economic Perspectives -- En carpeta del tema

Blaug, M. \textit{Economic Theory in Retrospect} (1997) 5th edition - En carpeta \textit{Historia del Pensamiento Económico}

Bradford, J. B. \textit{The Triumph of Monetarism} (2000) Journal of Economics Perspectives -- En carpeta del tema

De Vroey, M. \textit{A History of Macroeconomics. From Keynes to Lucas and Beyond}

Friedman, B. \textit{What remains of the Volcker Experiment} (2005) Federal Reserve Bank of St. Louis Review -- En carpeta del tema

Friedman, B. \textit{Lessons on Monetary Policy from the 1980s} (1988) Journal of Economic Perspectives -- En carpeta del tema

Friedman, M. (1970)\textit{A Theoretical Framework for Monetary Analysis}  Journal of Political Economy -- En carpeta del tema

Friedman, M. (1976) \textit{Inflation and Unemployment} Nobel Memorial Lecture, December 13, 1976 -- En carpeta del tema

Friedman, M. (1967) \textit{The Role of Monetary Policy}  Presidential Address at the Annual Meeting of the American Economic Association -- En carpeta del tema

Friedman, M.; Taylor, J. (2001) \textit{An Interview With Milton Friedman}  Macroeconomic Dynamics -- En carpeta del tema

Forder, J. \text{Nine Views of the Phillips Curve: Eight Authentic and One Inauthentic} \url{https://papers.ssrn.com/sol3/papers.cfm?abstract_id=2502145}

Hall, S. G; Swamy, P.; Tavlas, G. \textit{Milton Friedman, the Demand for Money, and the ECB’s Monetary Policy Strategy} (2012) -- En carpeta del tema

Hall, R. E.; Sargent, T. J. \textit{Short-Run and Long-Run Effects of
Milton Friedman’s Presidential Address} (2018, Winter) Journal of Economic Perspectives - En carpeta del tema

Hetzel, R. L. \textit{What remains of Milton Friedman's monetarism?} (2017) Federal Reserve Bank of Richmond -- En carpeta del tema

Journal of Economic Perspectives. \textit{SYMPOSIUM: FRIEDMAN'S NATURAL RATE HYPOTHESIS AFTER 50 YEARS} (2018) Winter -- \url{https://www.aeaweb.org/issues/496#10.1257/jep.32.1.81}

Levacic, R.; Rebmann, A. \textit{Macroeconomics. An Introduction to Keynesian-Neoclassical Controversies} (1982) 2nd Edition -- En carpeta Macroeconomía

Mankiw, G.; Reis, R. \textit{Friedman’s Presidential Address in the Evolution of Macroeconomic Thought} (2018, Winter) Journal of Economic Perspectives -- En carpeta del tema

McCallum, B.; Nelson, E. \textit{Money and Inflation: Some Critical Issues} (2011) Ch. 5 of the Handbook of Monetary Economics -- En carpeta Macroeconomía

Meltzer, A. H. \textit{Monetarism} Concise Encyclopedia of Economics -- \url{http://www.econlib.org/library/Enc1/Monetarism.html}

Meltzer, A. H. \textit{Monetarist, Keynesian and Quantity Theories} (1977) Kredit und Kapital -- En carpeta del tema

Yamagata, H. \textit{Monetarism} \url{https://cruel.org/econthought/essays/monetarism/monetarcont.html}

Modigliani, F. \textit{The Monetarist Controversy. Or, Should We Forsake Stabilization Policies?} (1977) Presidential Address at the Annual Meeting o the American Economic Association -- En carpeta del tema

Phelps, E. \textit{Phillips Curves, Inflation, Expectations and Optimal Employment overt Time} (1967)

Phelps, E. \textit{Money-wage Dynamics and Labor Market Equilibrium} (1968)

Poole, W. \textit{Monetary Policy Lessons of Recent Inflation and Disinflation} (1988) Journal of Economic Perspectives -- En carpeta del tema

Rozborilová, D. \textit{Don Israel Patinkin} BIATEC (2003) Profiles of World Economists -- En carpeta del tema

Screpanti, E; Zamagni, S. \textit{An Outline of the History of Economic Thought} (2005) -- En carpeta \textit{Historia del Pensamiento Económico}

Snowdon, B.; Vane, H. R. \textit{Modern Macroeconomics. Its Origins, Development and Current State} (2005) Edward Elgar Publishing --  En carpeta Macroeconomía

\end{document}
