\documentclass{nuevotema}

\tema{3B-38}
\titulo{Las Finanzas de la Unión Europea y el presupuesto comunitario. Las perspectivas financieras actuales.}

\begin{document}

\ideaclave

IMPORTANTE Leer \href{https://www.bde.es/f/webbde/SES/Secciones/Publicaciones/PublicacionesSeriadas/DocumentosOcasionales/20/Fich/do2021.pdf}{Banco de España (2020): Endeudamiento y necesidades de financiación en la Unión Europea}.

\href{https://www.ansa.it/documents/1595314851582_210720-euco-final-conclusions-en.pdf}{Consejo Europeo: Acuerdo de 21 de julio de 2020 sobre MFP 2021-2027 y FRR}

\href{https://twitter.com/SMerler/status/1285538863776178176}{Ver hilo de Twitter de Silvia Merler sobre Consejo Europeo de Julio de 2020 y NextGenerationEU} para nuevo MFP 2014-2020 y valoración

Ver \href{https://ec.europa.eu/info/business-economy-euro/economic-and-fiscal-policy-coordination/eu-financial-assistance/loan-programmes_en}{para actualizar sobre nuevos instrumentos en marco de crisis Covid}

Ver \url{http://blognewdeal.com/andrea-lucai/una-funcion-de-estabilizacion-fiscal-para-la-ue/} sobre función de estabilización fiscal a nivel europeo. 

Crear sección sobre nuevo ORD -- Decisión de recursos propios. Ver \href{https://www.europarl.europa.eu/legislative-train/theme-new-boost-for-jobs-growth-and-investment/file-mff-post-2020-own-resources}{Parlamento Europeo (2020}


\seccion{Preguntas clave}
\begin{itemize}
	\item ¿Cómo se financia la UE?
	\item ¿En qué gasta sus recursos la UE?
	\item ¿Qué procedimientos articulan el presupuesto de la UE?
	\item ¿Cuál es la situación actual de las finanzas comunitarias?
\end{itemize}

\esquemacorto

\begin{esquema}[enumerate]
	\1[] \marcar{Introducción}
		\2 Contextualización
			\3 Unión Europea
			\3 Competencias de la UE
			\3 Actividades financieras
			\3 Actividad presupuestaria y extrapresupuestaria
			\3 Impacto económico de actividad
			\3 Sistema presupuestario y financiero propio
		\2 Objeto
			\3 ¿Cómo se financia la UE?
			\3 ¿Cómo gasta sus recursos?
			\3 ¿Qué procedimiento presupuestario?
			\3 ¿Qué situación actual?
		\2 Estructura
			\3 Actividades presupuestarias
			\3 Act. Extrapresupuestarias
	\1 \marcar{Actividad presupuestaria}
		\2 Marco legal
			\3 Disposiciones generales
			\3 Principios presupuestarios
			\3 Marco Financiero Plurianual
			\3 Decisión de Recursos Propios
		\2 Fases del ciclo presupuestario
			\3 Elaboración
			\3 Ejecución (Comisión)
			\3 Control
		\2 Evolución presupuestaria de la UE
			\3 Desarrollo de un sistema financiero original (1951-1975)
			\3 Crisis de las finanzas comunitarias (1975-1987)
			\3 Reforma de las finanzas comunitarias: Paquete Delors I (88-92)
			\3 Paquete Delors II (93-99)
			\3 Agenda 2000-2006
			\3 Marco Financiero Plurianual (2007-2013)
			\3 Marco Financiero Plurianual (2014-2020)
			\3 Presupuesto año 2020
		\2 MFP 2014-2020
			\3 Gasto total
			\3 Ingresos: recursos propios: TFUE.311
			\3 Gasto (porcentajes sobre el total)
		\2 Propuesta de MFP 2021-2027
			\3 Propuesta de Comisión
			\3 Gasto total
			\3 Grandes rasgos de propuesta de CE 2018
			\3 \underline{Acuerdo de Consejo Europeo de julio de 2020}
			\3[1] Mercado Único, Innovación y Digital - 12,4\%
			\3[2] Cohesión, resiliencia y valores -- 35\%
			\3[3] Recursos naturales y medioambiente -- 33,5\%
			\3[4] Migraciones y fronteras -- 2\%
			\3[5] Seguridad y defensa -- 1,2\%
			\3[6] Política de vecindad y global -- 9,3\%
			\3[7] Administración Pública Europea -- 6,9\%
			\3[] Instrumentos especiales
			\3 Next Generation EU
	\1 \marcar{Act. extrapres.: préstamo, endeudamiento y estabilización}
		\2 Justificación
			\3 Principio de equilibrio
			\3 Acceso a mercados financieros
			\3 UE como intermediario de créditos
		\2 Instrumentos sectoriales
			\3 Préstamos de la CECA: préstamos aún por amortizar
			\3 Préstamos EURATOM
		\2 Instrumentos macroeconómicos
			\3 Instrumentos de asistencia a EEMM con garantía presupuestaria
			\3 Instrumentos sin garantía presupuestaria
			\3 Instrumentos de asistencia macro a terceros países
			\3 Asistencia especial en Covid-19
		\2 Préstamos del Banco Europeo de Inversiones (BEI)
			\3 Dimensiones
			\3 Independencia presupuesto de la UE
			\3 Proyectos acordes a objetivos de la UE
			\3 EFSI -- Plan de Inversión Para Europa
			\3 Programa Invest EU
			\3 Ampliación de capital tras Brexit
		\2 Fondo Europeo de Desarrollo
			\3 Estatus jurídico
			\3 Objetivo
			\3 Propuesta de MFP 2021-2027
	\1[] \marcar{Conclusión}
		\2 Recapitulación
			\3 Actividad presupuestaria
			\3 Actividad extrapresupuestaria
		\2 Idea final
			\3 Enorme contribución UE
			\3 Altibajos \4 Crisis
			\3 Situación actual

\end{esquema}

\esquemalargo


















\begin{esquemal}
	\1[] \marcar{Introducción}
		\2 Contextualización
			\3 Unión Europea
				\4 Institución supranacional ad-hoc
				\4[] Diferente de otras instituciones internacionales
				\4[] Medio camino entre:
				\4[] $\to$ Federación
				\4[] $\to$ Confederación
				\4[] $\to$ Alianza de estados-nación
				\4 Origen de la UE
				\4[] Tras dos guerras mundiales en tres décadas
				\4[] $\to$ Cientos de millones de muertos
				\4[] $\to$ Destrucción económica
				\4[] Marco de integración entre naciones y pueblos
				\4[] $\to$ Evitar nuevas guerras
				\4[] $\to$ Maximizar prosperidad económica
				\4[] $\to$ Frenar expansión soviética
				\4 Objetivos de la UE
				\4[] TUE -- Tratado de la Unión Europea
				\4[] $\to$ Primera versión: Maastricht 91 $\to$ 93
				\4[] $\to$ Última reforma: Lisboa 2007 $\to$ 2009
				\4[] Artículo 3
				\4[] $\to$ Promover la paz y el bienestar
				\4[] $\to$ Área de seguridad, paz y justicia s/ fronteras internas
				\4[] $\to$ Mercado interior
				\4[] $\to$ Crecimiento económico y estabilidad de precios
				\4[] $\to$ Economía social de mercado
				\4[] $\to$ Pleno empleo
				\4[] $\to$ Protección del medio ambiente
				\4[] $\to$ Diversidad cultural y lingüistica
				\4[] $\to$ Unión Económica y Monetaria con €
				\4[] $\to$ Promoción de valores europeos
				\4[$\to$] Objetivos de la UE
				\4[] Paz y bienestar a pueblos de Europa
			\3 Competencias de la UE
				\4 Tratado de la Unión Europea
				\4[] Atribución
				\4[] $\to$ Sólo las que estén atribuidas a la UE
				\4[] Subsidiariedad
				\4[] $\to$ Si no puede hacerse mejor por EEMM y regiones
				\4[] Proporcionalidad
				\4[] $\to$ Sólo en la medida de lo necesario para objetivos
				\4 Exclusivas
				\4[] i. Política comercial común
				\4[] ii. Política monetaria de la UEM
				\4[] iii. Unión Aduanera
				\4[] iv. Competencia para el mercado interior
				\4[] v. Conservación recursos biológicos en PPC
				\4 Compartidas
				\4[] i. Mercado interior
				\4[] ii. Política social
				\4[] iii. Cohesión económica, social y territorial
				\4[] iv. Agricultura y pesca \footnote{Salvo en lo relativo a la conservación de recursos biológicos marinos, que se trata de una competencia exclusiva de la UE}
				\4[] v. Medio ambiente
				\4[] vi. Protección del consumidor
				\4[] vii. Transporte
				\4[] viii. Redes Trans-Europeas
				\4[] ix. Energía
				\4[] x. Área de libertad, seguridad y justicia
				\4[] xi. Salud pública común en lo definido en TFUE
				\4 De apoyo
				\4[] Protección y mejora de la salud humana
				\4[] Industria
				\4[] Cultura
				\4[] Turismo
				\4[] Educación, formación profesional y juventud
				\4[] Protección civil
				\4[] Cooperación administrativa
				\4 Coordinación de políticas y competencias
				\4[] Política económica
				\4[] Políticas de empleo
				\4[] Política social
			\3 Actividades financieras
				\4 Políticas de UE
				\4[] Conjunto muy amplio de políticas
				\4 Necesario financiar
				\4 Necesario establecer marco coherente de:
				\4[] De ingresos
				\4[] De gastos
			\3 Actividad presupuestaria y extrapresupuestaria
				\4 Presupuestaria
				\4[] Propiamente en el marco de UE
				\4[] Marco presupuestario multianual y anual
				\4[] UE lleva a cabo como pers. jurídica
				\4 Extrapresupuestaria
				\4[] Más allá de UE
				\4[] EEMM en su propia capacidad
				\4[] Coordinado con políticas de UE
				\4[] No existiría sin UE
				\4[] Potencialmente UE también participante
			\3 Impacto económico de actividad
				\4 Pequeños efectos estabilizadores
				\4 Efectos sobre crecimiento de l/p e inversión
				\4 Potencial utilización estabilizadora futura
			\3 Sistema presupuestario y financiero propio
				\4 Desarrollado durante décadas
				\4 Progresiva construcción de procedimiento presupuestario
				\4 Relativa estabilización últimas dos décadas
		\2 Objeto
			\3 ¿Cómo se financia la UE?
			\3 ¿Cómo gasta sus recursos?
			\3 ¿Qué procedimiento presupuestario?
			\3 ¿Qué situación actual?
		\2 Estructura
			\3 Actividades presupuestarias
				\4 Marco legal
				\4 Fases del ciclo presupuestario
				\4 Evolución presupuestaria de la UE
				\4 Presupuesto actual
				\4 Propuesta de MFP 2021-2027
			\3 Act. Extrapresupuestarias
				\4 Justificación
				\4 Instrumentos sectoriales
				\4 Instrumentos macroeconómicos
				\4 Préstamos del Banco Europeo de Inversiones
				\4 Fondo Europeo del Desarrollo
	\1 \marcar{Actividad presupuestaria}
		\2 Marco legal
			\3 Disposiciones generales
				\4 TUE
				\4[] 14. y 16
				\4[] $\to$ Atribuciones CdUE y PE en leg. presupuestaria
				\4[] 17.
				\4[] $\to$ CE ejecuta el presupuesto
				\4 TFUE
				\4[] 312 y ss.
				\4[] $\to$ MFP
				\4[] $\to$ Presupuesto anual
				\4 Reglamento financiero 2018
				\4[] Procedimiento de elaboración de presupuesto
				\4[] Normas de control de finanzas europeas
				\4[] Normas para acceso a fondos comunitarios
				\4 Reglamento MFP 14-20
				\4 Acuerdo interinstitucional diciembre de 2013
				\4[] Entre PE, Consejo y Comisión
				\4[] Líneas generales de ejecución de MFP2014
				\4[] Sobre:
				\4[] $\to$ Disciplina presupuestaria
				\4[] $\to$ Cooperación presupuestaria
				\4[] $\to$ Buena gestión financiera
			\3 Principios presupuestarios
				\4[U] Unidad y veracidad:
				\4[] Todos los ingresos y gastos
				\4[] $\to$ Deben consignarse en el presupuesto
				\4[U] Universalidad:
				\4[] Todos ingresos cubren todos los pagos
				\4[] Sin compensación entre sí
				\4[] No hay ingresos adscritos a gastos determinados
				\4[U] Unidad de cuenta
				\4[] Establecimiento, ejecución, rendición de cuentas
				\4[] $\to$ En euros
				\4[E] Equilibrio
				\4[] Igualdad entre créditos de ingreso y pago
				\4[E] Especialidad
				\4[] Gastos asignados a objetivo concreto
				\4[] 1. Títulos
				\4[] 2. Capítulos
				\4[] 3. Artículos
				\4[] 4. Partidas
				\4[A] Anualidad:
				\4[] Créditos consignados se autorizan por un año
				\4[] MFP a menudo necesitan gastos de más de un año
				\4[] Créditos no disociados son específicos para un año
				\4[] $\to$ Si no gastados al final de periodo se anulan
				\4[] Créditos disociados permiten programas multianuales
				\4[] $\to$ Dan lugar a créditos de compromiso y pago
				\4[] Créditos de compromiso
				\4[] $\to$ Coste total de compromisos para acciones de >1 año
				\4[] $\to$ Recogen gasto máximo
				\4[] Créditos de pago
				\4[] $\to$ Cubren gastos de ejecución en el periodo en cuestión
			\3 Marco Financiero Plurianual\footnote{Ver \url{https://www.europarl.europa.eu/factsheets/en/sheet/29/multiannual-financial-framework}.}
				\4 Categorías de gastos previsibles para cada periodo
				\4 Doble tope:
				\4[] $\to$ por categoría de gastos
				\4[] $\to$ límite total anual
				\4 Aprobación por
				\4[] $\to$ Reglamento
				\4[] $\to$ Acuerdo institucional de acompañamiento
				\4 Procedimiento de aprobación
				\4[] Procedimiento legislativo
				\4[] PE debe otorgar consentimiento mayoritario
				\4[] CdUE debe adoptar por unanimidad
				\4 Mínimo 5 años en vigor
				\4 Debate periódico a largo plazo
				\4 Compatibilidad operaciones y marco financiero establecido
				\4 Fuertemente politizado
				\4[] Países intentan mejorar saldo neto
				\4[] Controversia sobre
				\4[] $\to$ Cheque británico
				\4[] $\to$ Otras reducciones
			\3 Decisión de Recursos Propios\footnote{Own Resources Decision -- ORD.}
				\4 Aprobada previamente a MFP
				\4[] Unanimidad del Consejo
				\4 Última en 2014
				\4[] Sin límite temporal
				\4 Categorías de recursos propios
				\4 Techo a los ingresos por recursos propios
				\4[] \% suma RNB
				\4 Techo a los gastos
				\4[] Créditos de pago anuales no pueden superar $1,23\%$ RNB
				\4[] Créditos de compromiso no pueden superar $1,29\%$ RNB
				\4 Recurso del IVA
				\4[] $0,30\%$ de base imponible de IVA para UE
				\4[] $\to$ Base imponible no puede exceder 50\% de RNB
				\4[] En MFP actual, $12,8\%$ de ingresos de UE
				\4 Cheque británico
				\4[] A derogar tras Brexit
				\4 Sin fecha de caducidad
				\4[] Aunque normalmente, previo a MFP
				\4 Propuesta de ORD para 2020
				\4[] Ver \href{https://ec.europa.eu/info/sites/info/files/about_the_european_commission/eu_budget/com_2020_445_en_act_v8.pdf}{Comisión Europea (2020): propuesta}
				\4[] Ver \href{https://www.europarl.europa.eu/legislative-train/theme-new-boost-for-jobs-growth-and-investment/file-mff-post-2020-own-resources}{Parlamento Europeo (2020): resumen de propuesta de Comisión}
				\4[] Reducción de 10\% de recursos tradicionales para EEMM
				\4[] Nuevos recursos propios:
				\4[] $\to$ CCCTB ligada a mercado único
				\4[] $\to$ Parte de ingresos por ETS
				\4[] $\to$ Contribución nacional basada en residuos plásticos
				\4[] $\then$ Deberían ser el 12\% de ingresos MFP 2021-2027
		\2 Fases del ciclo presupuestario
			\3 Elaboración
				\4 Marco legislativo
				\4[] TFUE
				\4[] Reglamento del MFP vigente
				\4[] Reglamento financiero de 2018
				\4[] Acuerdo Interinstitucional\footnote{A partir del MFP de 2007-2013, el MFP pasa a ser un acto jurídicamente vinculante en forma de Reglamento y está acompañado por un AI que trata cuestiones relativas a disciplina presupuestaria, cooperación presupuestaria y buena gestión financiera.}
				\4 Duración del proceso
				\4[] 1 de julio a 31 de diciembre
				\4[1.] Proyecto de presupuesto (Comisión):
				\4[] tras previsiones gasto instituciones
				\4[] 1 de septiembre como tarde
				\4[2.] Consejo toma posición y envía a PE
				\4[] 1 de octubre como tarde
				\4[3.] Parlamento aprueba/enmienda
				\4[] Si aprueba, presupuesto aprobado
				\4[] Si no se pronuncia, aprobado
				\4[] Si enmienda, a Conciliación
				\4[4.] Conciliación (Comité de Conciliación)
				\4[] Compuesto por miembros de PE y CdUE
				\4[] 21 días para acordar texto conjunto
				\4[] Si no hay acuerdo
				\4[] $\to$ Comisión debe proponer nuevo proyecto
				\4[] Si hay acuerdo:
				\4[] $\to$ Sometido a aprobación en CdUE y PE
				\4[] $\to$ 14 días para aprobar
				\4[5.] Aprobación o extensión de 1/12 de anterior
				\4[] CdUE
				\4[] $\to$ Aprueba o rechaza
				\4[] Si CdUE rechaza PE puede aprobar con:
				\4[] $\to$ Mayoría de los miembros de PE
				\4[] $\to$ 3/5 favorables de votos emitidos
			\3 Ejecución (Comisión)
				\4 Operaciones básicas:
				\4[] compromiso y ordenación del pago
				\4 Gestión directa:
				\4[] 22\%
				\4[] Comisión gestiona de forma directa
				\4 Gestión compartida:
				\4[] 76\%
				\4[] Comisión gestiona junto a EEMM
				\4[] $\to$ PAC
				\4[] $\to$ cohesión
				\4[] $\to$ política interior EEMM
				\4 Gestión indirecta:
				\4[] 2\%
				\4[] EEMM gestionan directamente
			\3 Control
				\4 Tribunal de Cuentas:
				\4[] control externo
				\4 Parlamento Europeo:
				\4[] control político a través de la descarga
				\4 Oficina Europea de Lucha contra el Fraude (OLAF)
		\2 Evolución presupuestaria de la UE\footnote{Ver \url{https://www.europarl.europa.eu/news/es/headlines/eu-affairs/20130723STO17551/pasado-presente-y-futuro-del-sistema-europeo-de-recursos-propios} y \url{https://eur-lex.europa.eu/legal-content/ES/TXT/?uri=LEGISSUM:l34011}}
			\3 Desarrollo de un sistema financiero original (1951-1975)
				\4 Predominio del Consejo de la UE
				\4 Basado en aportaciones de los EEMM
				\4[] Combinación de:
				\4[] $\to$ Impuestos (tasas agrícolas)
				\4[] $\to$ Contribuciones de los 6 EEMM
				\4 Similar a instituciones internacionales
				\4 Tratado de Luxemburgo (1970)
				\4[] Instauración de recursos propios:
				\4[] $\to$ Aduanas
				\4[] $\to$ IVA \footnote{Realmente comienza a aplicarse desde 1980 por problemas con la armonización de las bases.}
				\4[] $\to$ Exacciones agrícolas
			\3 Crisis de las finanzas comunitarias (1975-1987)
				\4 Armonización del IVA
				\4[] Creado en 1970
				\4[] $\to$ No aplicado efectivamente hasta 1980
				\4 Gasto excesivo en PAC
				\4[] Precios mínimos
				\4[] Subsidios a exportación
				\4[] Sin cuotas de producción
				\4[] $\to$ Crecimiento acelerado de gasto PAC
				\4 Reclamaciones británicas: cheque británico (1984)
				\4[] Consejo de Fontainebleau de 1984
				\4[] Compensación al RU de 66\% de saldo neto negativo
				\4[] $\to$ En función de renta nacional
				\4[] $\to$ GER, NED, SWE y AUS sólo 25\% de lo que les correspondería
				\4 Ampliación GRE, ESP, POR
				\4 Conflictos entre Parlamento y Consejo $\to$ parálisis
			\3 Reforma de las finanzas comunitarias: Paquete Delors I (88-92)
				\4 Primer Acuerdo Interinstitucional
				\4 Límite máximo de gasto en función de RNB
				\4 Introducción de recurso de cierre basado en RNB
				\4 Recurso del IVA
				\4[] Base correspondiente a 55\% del PNB
				\4[] $\to$ Hasta 1994
				\4[] 1.4\% de base IVA armonizada  desde 86
				\4[] $\to$ Hasta 1994
			\3 Paquete Delors II (93-99)
				\4 Segundo Acuerdo Interinstitucional
				\4 Límite de recursos: 1,27\% RNB
				\4 Establecimiento Fondos Estructurales
				\4 Desde 1995:
				\4[] Base no puede superar el 50\% PNB con PNBpc<90\% media UE
				\4[] $\to$ Restricción aplicada posteriormente a todos los Estados
				\4[] Tipo aplicable a base IVA armonizada reducido al 1\%
				\4[] $\to$ Desde 1994
			\3 Agenda 2000-2006
				\4 Reforma de la PAC
				\4 Marco para 2000-2006
				\4 Ampliación a Europa del Este
				\4 Mantiene techo de gasto
				\4 Reduce tipo exigible a recurso de IVA
				\4[] $0.5\%$ de base IVA armonizada
				\4 Ajusta carga cheque británico
			\3 Marco Financiero Plurianual (2007-2013)
				\4 MFP deja de ser Acuerdo Institucional
				\4[] Acto jurídicamente vinculante
				\4[] $\to$ En forma de reglamento
				\4[] $\to$ Duración mínima de 5 años
			\3 Marco Financiero Plurianual (2014-2020)
				\4 Límite máximo recursos propios 1,23\%
				\4[] $\to$ Créditos de pago
				\4 Límite del 1\% RNB de créditos de compromiso
				\4[] Finalmente, 1,04\%
				\4 Contexto de crisis y presión para reducir gasto
				\4 Revisión intermedia a mitad de periodo
			\3 Presupuesto año 2020
		\2 MFP 2014-2020
			\3 Gasto total
				\4 Créditos de pago
				\4 Cifra cercana a los 960.000 millones €
			\3 Ingresos: recursos propios: TFUE.311
				\4 Decisión de recursos propios de 2014
				\4 Recursos propios: 99\%
				\4[] Recursos propios tradicionales (aduanas, exacciones): 12\% \footnote{De los cuales los EEMM se quedan con un 20\% en concepto de costes de gestión. Antes fue del 25\%.}
				\4[] Recursos IVA: 13\%
				\4[] Recurso basado en la RNB: 74\%
				\4[] Cheque británico: reembolso 2/3 contribución
				\4 Otros recursos: 1\% (actividades habituales)
				\4[] Multas
				\4[] Compensaciones
				\4[] Otras contribuciones
				\4 Techo a los gastos
				\4[] Créditos de pago anuales no pueden superar $1,23\%$ RNB
				\4[] Créditos de compromiso no pueden superar $1,29\%$ RNB
				\4 Recurso del IVA
				\4[] $0,30\%$ de base imponible de IVA para UE
				\4[] $\to$ Base imponible no puede exceder 50\% de RNB
				\4[] En MFP actual, $12,8\%$ de ingresos de UE
			\3 Gasto (porcentajes sobre el total)\footnote{Ver \href{https://www.europarl.europa.eu/RegData/etudes/BRIE/2019/635545/EPRS_BRI(2019)635545_EN.pdf}{EPRS: Future financing of EU policies}, pág. 2.}
				\4 Créditos de compromiso totales
				\4[] 1.087.000 M de €
				\4[1] \underline{Crecimiento Inteligente e integrador: 47\%}
				\4[] 513.000 M de €
				\4[] 1a. Competitividad, crecimiento y empleo 13\% (Erasmus, Galileo, Mecanismo Interconectar...)
				\4[] $142$ mil M de €
				\4[] 1b. Cohesión económica, social y territorial (objetivo convergencia): 34\%
				\4[] $371$ mil M de €
				\4[] $\to$ FEDER
				\4[] $\to$ Fondo de Cohesión
				\4[] $\to$ Fondo Social Europeo
				\4[] $\to$ RUP -- Regiones Ultra Periféricas
				\4[] $\to$ Objetivo de cooperación territorial
				\4[2] \underline{Crecimiento sostenible: recursos naturales. 39\%}
				\4[] $420$ mil M de €
				\4[] PAC Pilar I: apoyo directo (28\%)
				\4[] $\to$ FEAGA
				\4[] PAC Pilar II: gasto para desarrollo rural (9\%)
				\4[] $\to$ FEADER
				\4[] Pesca (2\%)
				\4[3] \underline{Seguridad y ciudadanía 1.6\%}
				\4[] $18$ mil M de €
				\4[] Asilo, migraciones
				\4[] Seguridad interior
				\4[] Control de fronteras
				\4[4] \underline{Una Europa global 6\%}
				\4[] $66$ mil M de €
				\4[] Ayuda humanitaria
				\4[] Ayuda Oficial al Desarrollo
				\4[] (Fondo Europeo de Desarrollo no incluido)
				\4[5] \underline{Administración: 6,3\%}
				\4[] $70$ mil M de €
				\4[] Ligera reducción frente a MFP previo
				\4[6] \underline{Compensaciones: 0,1\%}
				\4 Instrumentos especiales
				\4[] Reserva para ayuda de emergencia
				\4[] Fondo de Europeo de Solidaridad
				\4[] Instrumento de flexibilidad
				\4[] $\to$ Financiar gastos excepcionales
				\4[] $\to$ Partidas necesarias no cubiertas por CCompromiso
				\4[] FEAG -- Fondo Europeo de Ajuste a la Globalización
				\4[] Margen de contingencia -- $0.03\%$ del MFP
		\2 Propuesta de MFP 2021-2027\footnote{Leer Brehon (2018), EPRS (2018), especialmente tabla 1 de página 4. Para propuesta final presentada a Parlamento Europeo, ver \href{https://www.consilium.europa.eu/media/45109/210720-euco-final-conclusions-en.pdf}{Conclusiones del Consejo Europeo de 17 a 21 de julio de 2020}.}
			\3 Propuesta de Comisión
				\4 No adoptado
				\4 VER BULLETIN DE LA BANQUE DE FRANCE
				\4 Mayo de 2018
				\4 Cambio total por crisis de Covid
			\3 Gasto total
				\4 Propuesta inicial
				\4[] 1.135.000 M de euros
				\4[] $\to$ 1.11\% de RNB en créditos de compromiso
				\4 Aumenta un 18\% en términos nominales
				\4 Aumenta un 5\% en términos reales
				\4 Cae como porcentaje de la RNB
				\4[] De 1,16\% al 1,11\%
				\4 \underline{Acuerdo de Consejo Europeo}
				\4[] Precios de 2018
				\4[] Reducción respecto a propuesta de Comisión
				\4[] $\to$ Aumento vía NGEU
				\4[] 1.074.300 M de € en créditos de compromiso
				\4[] 1.061.058 M de € en créditos de pago
			\3 Grandes rasgos de propuesta de CE 2018\footnote{Ver \url{https://www.europarl.europa.eu/factsheets/en/sheet/29/multiannual-financial-framework}}
				\4 Reducción en PAC del 15\%
				\4[] Más reducción en desarrollo rural
				\4[] Reducción más ligera en pagos directos
				\4[] Propone aumentar cofinanciación
				\4[] Cubre $55\%$ de aumento en otras áreas
				\4 Aumento del programa de acción climática
				\4 Reducción de Fondo de Cohesión
				\4[] A casi la mitad
				\4 Aumento Erasmus+
				\4[] Casi al doble
				\4 Aumento ligero de gasto administrativo
				\4[] Comisión justifica en aumento responsabilidades
				\4 Reducción margen entre créditos de pago y compromiso
				\4 Aumento de techo de recursos de RNB
				\4[] Del 1.20\% al 1.29\% de la RNB
				\4 Incorpora Fondo Europeo de Desarrollo
				\4 Eliminación gradual de correcciones de contrib.
				\4[] Actualmente muy complejas
				\4[] Muchas excepciones para muchos países
				\4 InvestEU
				\4[] Agrupa EFSI (Plan Juncker) + 13 instrumentos
				\4[] Movilizar 650.000 M de € en 2021-2027
				\4 Introduce nuevos instrumentos presupuestarios especiales
				\4[] Reserva de Ayuda de Emergencia
				\4[] $\to$ 600 M de € al año
				\4[] Fondo Europeo de Solidaridad
				\4[] $\to$ 600 M de € al año
				\4[] Fondo Europeo de Ajuste a la Globalización
				\4[] $\to$ 200 M de € al año
				\4[] Fondo Europeo de Paz
				\4[] Función Europea de Estabilización
				\4[] $\to$ 30.000 M de préstamos entre 2021-2027
				\4 FED incluido en MFP
				\4 Nuevas categorías de recursos propios
				\4[] Ingresos por sistema de emisiones
				\4[] Pago por residuos plásticos de EEMM
				\4[] Parte por base imponible consolidada de IS
				\4 \underline{Propuesta inicial de comisión Europea}
				\4[1] Cohesión y valores (34,5\%) $\sim$ $35\%$
				\4[2] Recursos naturales y medio ambiente (29.7\%) $\sim$ $30\%$
				\4[3] Mercado único, innovación y digital (14,6\%) $\sim$ $15\%$
				\4[4] Vecindad y el mundo (9,6\%) $\sim$ $10\%$
				\4[5] Administración pública europea (6,7\%) $\sim$ $7\%$
				\4[6] Migraciones y gestión de fronteras (4,9\%) $\sim$ $5\%$
				\4[$\then$] Caída en PAC
				\4[$\then$] Aumento migraciones y política de seguridad
			\3 \underline{Acuerdo de Consejo Europeo de julio de 2020}
				\4 Sin revisión de MFP a medio camino
				\4 Sin aumento de gasto en MFP
				\4 Aumento extraordinario vía extrapresupuestaria
				\4[] Next Generation EU
				\4 Mínimo de gasto total a objetivos climáticos
				\4[] 30\% del total
				\4 Creación de:
				\4[] Mecanismo de Transición Justa
				\4[] Fondo de Transición Justa
				\4 Régimen de condicionalidad será introducido
				\4[] Comisión Europea propondrá medidas si incumplimiento
				\4[] $\to$ A aprobar por CdUE por QMV
				\4 Recursos propios
				\4[] Límite máximo de 1,40\% RNB en pagos
				\4[] Límite máximo de 1,46\% RNB en compromisos
				\4 Nuevos recursos propios
				\4[] Impuesto a plástico no reciclado
				\4[] $\to$ 0,8 € por kilo
				\4[] Ajuste fiscal en frontera al carbono
				\4[] Impuesto digital
				\4[] ETS
				\4[] Posible impuesto a transacciones financieras
				\4[] Utilizados para financiar préstamos NGEU
				\4 Correcciones y ajustes para algunos EEMM
				\4[] DIN, GER, NED, AUS, SWE
				\4[] Reducción de contribuciones respecto RNB
				\4[] \underline{Conclusiones Consejo Europeo de julio 2020}
				\4[] Aumento de gasto total respecto propuesta
				\4[] $\to$ En términos corrientes
				\4[] Créditos de pago totales:
				\4[] $\to$ 1.195.000 M de €
				\4[] Créditos de compromiso totales:
				\4[] $\to$ 1.210.000 M de €
				\4[] En precios de 2018
				\4[] $\to$ Apenas 1.074.000 créditos de compromiso
				\4[] $\to$ Similar a MFP 2014-2020
			\3[1] Mercado Único, Innovación y Digital - 12,4\%
				\4[] 130.000 M de €
				\4[] 12,4\% de total
				\4[] Horizon Europe: 76.000 M de €
				\4[] Proyectos científicos de gran escala
				\4[] ITER
				\4[] Digitalización
				\4[] Competitividad de PYMES
				\4[] Invest UE
				\4[] $\to$ Reemplazar otros instrumentos financieros
				\4[] $\to$ 2.800 más remanentes
				\4[] Connecting Europe Facility: 30.000 M
				\4[] $\to$ Transporte: 21.000 M de €
				\4[] $\then$ De los cuales 10.000 M de € del FCohesión
				\4[] $\to$ Energía: 5.000 M de €
				\4[] $\to$ Digital: 1.800 M de €
				\4[] Europa Digital: 7.000 M de €
			\3[2] Cohesión, resiliencia y valores -- 35\%
				\4[] 380.000 M de €
				\4[] i. Cohesión
				\4[] $\to$ 330.000 M de €
				\4[] $\to$ 200.000 a regiones menos desarrolladas
				\4[] $\to$ 50.000 a regiones de transición
				\4[] $\to$ 40.000 EEMM que reciben FCohesión
				\4[] $\to$ 2.000 a periféricos
				\4[] $\to$ Similares clasificaciones de renta
				\4[] $\then$ FEDER
				\4[] $\then$ ESF+
				\4[] $\then$ Fondo de Cohesión
				\4[] ii. Resiliencia y valores
				\4[] $\to$ RescEU
				\4[] $\to$ ESF+
				\4[] $\to$ Programa Erasmus y similares
				\4[] $\to$ 50.000 M de €
			\3[3] Recursos naturales y medioambiente -- 33,5\%
				\4[] 360.000 M de €
				\4[] Ligera reducción respecto a MFP 14-20
				\4[] Más flexibilidad para EEMM
				\4[] Simplificación
				\4[] 40\% debe dedicarse a acción climática
				\4[] Convergencia de pagos por hectárea entre EEMM
				\4[] Fondos FEAGA y FEADER
				\4[] $\to$ Similar a 14-20
				\4[] i. Pilar I: Pagos directos e intervención de mercado
				\4[] $\to$ 260.000 M de €
				\4[] ii. Pilar II
				\4[] $\to$ 80.000 M de €
			\3[4] Migraciones y fronteras -- 2\%
				\4[] 22.000 M de €
				\4[] Coordinación de fronteras
				\4[] Salvaguardar libre movimiento personas y bienes
				\4[] Fondo de Asilo y Migración
				\4[] $\to$ 9.000 M de €
			\3[5] Seguridad y defensa -- 1,2\%
				\4[] 13.000 M de €
				\4[] Fondo de Seguridad Interna
				\4[] Fondo Europeo de Defensa
			\3[6] Política de vecindad y global -- 9,3\%
				\4[] 100.000 M de €
				\4[] Cooperación al desarrollo
				\4[] Política Europea de Vecindad
				\4[] Integración del FED
				\4[] Agenda 2030
				\4[] Acción Exterior
				\4[] Fusión de Instrumentos de Acción Exterior
				\4[] $\to$ Instrumento Único con 70.000 M de €
				\4[] $\then$ Vecindad, Desarrollo y Cooperación Internacional
				\4[] Instrumento Europeo de Paz fuera del presupuesto
				\4[] $\to$ 5.000 M de €
			\3[7] Administración Pública Europea -- 6,9\%
				\4[] 73.000 M de €
			\3[] Instrumentos especiales
				\4 Fuera de límites anteriores
				\4[] Instrumentos temáticos
				\4[] $\to$ 20.000 M de €
				\4[] $\to$ Fondo de Ajuste para el Brexit
				\4[] $\to$ Fondo Europeo de Ajuste a la Globalización
				\4[] $\to$ Reserva de Emergencia y Solidaridad: 1.200 M anuales
				\4 Instrumentos no temáticos
				\4[] $\to$ Instrumento Único de Margen
				\4[] $\then$ Créditos disponibles de MFP anterior
				\4[] $\then$ Créditos futuros en caso excepcional
			\3 Next Generation EU
				\4 Extrapresupuestario
				\4 Aprobación conjunta a MFP 21-27
				\4 750.000 M de € precios de 2018
				\4 Préstamos hasta 360.000 M de €
				\4 Transferencias hasta 390.000 M de €
				\4 Epígrafes de gasto:
				\4[] MRR -- Mecanismo de Recuperación y Resiliencia:
				\4[] $\to$ 670.000 M de €
				\4[] $\then$ Transferencias: 312.000 M
				\4[] $\then$ Préstamos: 360.000 M de €
				\4[] ReactEU:
				\4[] $\to$ 47.500 M de €
				\4[] $\then$ Complemento a fondos de cohesión
				\4[] $\then$ Sin preasignar a regiones
				\4[] $\then$ Destinadas a regiones más afectadas por crisis\footnote{Ver \href{https://ec.europa.eu/regional_policy/sources/docgener/factsheet/2020_mff_reacteu_en.pdf}{EC (2020): Factsheet sobre ReactEU}.}
				\4[] Horizon Europe:
				\4[] $\to$ 5.000 M de €
				\4[] InvestEU:
				\4[] $\to$ 5.600 M de €
				\4[] Desarrollo Rural:
				\4[] $\to$ 7.500 M de €
				\4[] Transición Justa:
				\4[] $\to$ 10.000 M de €
				\4[] RescEU:
				\4[] $\to$ 2.000 M de €
				\4[] Total:
				\4[] $\to$ 750.000 M de €
				\4 Planes de recuperación y resiliencia nacional
				\4[] Analizados por CE
				\4[] Recomendaciones específicas
				\4[] Valoración del cumplimiento
				\4[] Necesaria aprobación por el Consejo
				\4 Endeudamiento de Comisión
				\4[] Hasta 750.000 M de €
				\4[] Endeudamiento neto aumenta sólo hasta 2026
				\4[] A pagar hasta 31 de diciembre de 2058
				\4[] Aumento temporal de techo de recursos propios
				\4[] $\to$ En +0,6\% para pagar endeudamiento
	\1 \marcar{Act. extrapres.: préstamo, endeudamiento y estabilización}
		\2 Justificación
			\3 Principio de equilibrio
				\4 Impide a UE incurrir en déficit
			\3 Acceso a mercados financieros
				\4 Sin embargo, UE puede financiarse barato
				\4[] Condiciones muy ventajosas respecto a PEDs y EEMM
			\3 UE como intermediario de créditos
				\4 Toma prestado en mercados a condiciones ventajosas
				\4 Presta a determinados receptores objetivo
		\2 Instrumentos sectoriales
			\3 Préstamos de la CECA: préstamos aún por amortizar
				\4 Inversiones carbón y acero
				\4 Reconversión de sistemas industriales
			\3 Préstamos EURATOM
				\4 Tratado de 1957 autorizaba deuda proyectos energía nuclear
		\2 Instrumentos macroeconómicos
			\3 Instrumentos de asistencia a EEMM con garantía presupuestaria
				\4 Facilidad de la Balanza de Pagos
				\4[] Creado en 1975
				\4[] Capacidad de préstamo\footnote{Límite ampliado en 2009 a esta cifra de 50.000 millones de €.}: 50.000 millones de €
				\4[] Tras 2008, HUN, LET, ROM han recibido préstamos
				\4 Mecanismo Europeo de Estabilización Financiera (EFSM/MEEF)
				\4[] Creado en 2010 para miembros en problemas
				\4[] Ayuda a IRL, POR, GRE
			\3 Instrumentos sin garantía presupuestaria
				\4 EFSF/FEEF
				\4[] Facilidad Europea de Estabilidad Financiera
				\4[] GRE, IRL, POR
				\4[] Financiado mediante emisión de deuda
				\4[] Carácter temporal, no presta ayuda desde 2013
				\4 ESM/MEDE
				\4[] Mecanismo Europeo de Estabilidad
				\4[] Sustituye FEEF y MEEF (2012)
				\4[] Capital autorizado: 700 MM €
				\4[] Capital desembolsado: 80 MM €
				\4[] Capacidad de préstamo: 500 MM €
				\4[] Prestado a CHI, GRE, ESP.
				\4[] Propuesta de reconversión en FME
				\4[] $\to$ Fondo Monetario Europeo
			\3 Instrumentos de asistencia macro a terceros países
				\4 Países de Centroeuropa
				\4 Mediterráneos (ISR, ARG)
				\4 Ex-repúblicas soviéticas (UCR, MOL, GEO)
			\3 Asistencia especial en Covid-19
				\4 Pandemic Crisis Support
				\4[] Hasta diciembre de 2021
				\4[] Marco de ECCL
				\4[] $\to$ Enhanced Condition Credit Line de MEDE
				\4[] Gastos asociados a la pandemia
				\4[] $\to$ Directos e indirectos
				\4[] $\to$ Sin condicionalidad
				\4[] $\to$ 2\% de PIB máximo
				\4[] $\to$ 10 años
				\4[] $\to$ 240.000 M de € totales
				\4[] $\to$ Coste de fin. de MEDE + 10pb + apertura
				\4 SURE
				\4[] 100.000 M de €
				\4[] Exposición a tres países con más acceso
				\4[] Inferior a 60\% total
				\4[] Garantizado por compromiso voluntario de EEMM
				\4[] $\to$ En proporción a su participación
				\4[] $\then$ No es realmente mutualización
		\2 Préstamos del Banco Europeo de Inversiones (BEI)
			\3 Dimensiones\footnote{\href{https://www.eib.org/en/about/key_figures/data.htm}{EIB Group: key statutory figures.}}
				\4 Tamaño total del balance
				\4[] 550.000 M de €
				\4 Capital suscrito desembolsado
				\4[] 22.000 M de €
				\4 Préstamos desembolsados
				\4[] 450.000 M de €
				\4 Prestamos a desembolsar
				\4[] 112.000 M de €
			\3 Independencia presupuesto de la UE
				\4 Creado por Tratado de Roma (1957)
				\4 Accionistas: 27 EEMM
				\4[] RU sale de accionariado tras Brexit
			\3 Proyectos acordes a objetivos de la UE
				\4 90\% en UE
				\4 Obtención de fondos en mercados de capital
				\4 Garantía presupuesto UE (27 MM €)
				\4 Puede prestar a:
				\4[] $\to$ Organismos públicos nacionales y regionales
				\4[] $\to$ Entidades privadas
				\4[] $\to$ Empresas públicas
				\4 Rango geográfico
				\4[] UE es prioridad
				\4[] Fuera de UE, apoyo a países socios
				\4[] $\to$ Potenciales ampliaciones
				\4[] $\to$ ACP
				\4[] $\to$ Mediterráneo
				\4[] $\to$ Asia y América Latina
			\3 EFSI -- Plan de Inversión Para Europa
				\4 European Fund for Strategic Investments
				\4[] Fondo Europeo para Inversiones Estratégicas
				\4[] ``Plan Juncker''
				\4 Idea clave
				\4[] UE aporta garantías a BEI
				\4[] BEI toma prestado en mercados financieros
				\4[] BEI invierte en proyectos públicos y privados
				\4[] Inversiones de BEI movilizan inversión privada
				\4 Movilizar inversión público privada vía BEI
				\4[] Garantías aportadas de 21.000 M de €
				\4[] A través de:
				\4[] $\to$ Banco Europeo de Inversiones
				\4[] $\then$ 55.000 M de €
				\4[] $\to$ Fondo Europeo de Inversiones
				\4[] $\then$ Parte del grupo BEI
				\4[] $\then$ 20.000 M de €
				\4 Objetivo inicial
				\4[] Movilizar 375.000 M de €
				\4 Inversiones ejecutadas hasta 2019
				\4[] 75.000 M de € inversiones directas
				\4[] 410.000 M de € totales
				\4[] $\to$ Directa e indirectamente
			\3 Programa Invest EU
				\4 Propuesta para MFP 2021-2027
				\4 Agrupar multitud de instrumentos financieros de inversión
				\4 Continuar modelo exitoso de Plan Juncker EFSI
			\3 Ampliación de capital tras Brexit
				\4 Salida de Reino Unido
				\4[] Reduce capital disponible
				\4 Prevista ampliación de capital
				\4[] Mantener ratio de apalancamiento 2.5
				\4[] $\to$ Sin reducir proyectos de inversión
		\2 Fondo Europeo de Desarrollo
			\3 Estatus jurídico
				\4 Sin personalidad jurídica
				\4 Contribuciones directas de EEMM
				\4[] $\sim 30.5$ mil M de €
				\4 Gestionado por Comisión
				\4 Sin integrar en Presupuesto
				\4[] Pero negociado paralelamente
				\4[] Marco Financiero para 2014-2020 anexo a Acuerdo de Cotonú
				\4[] Integración propuesta en MFP 2021-2027
			\3 Objetivo
				\4 Cooperación con Países ACP y OCT\footnote{Overseas Countries and Territories.}
			\3 Propuesta de MFP 2021-2027
				\4 FED se incluye en MFP
	\1[] \marcar{Conclusión}
		\2 Recapitulación
			\3 Actividad presupuestaria
			\3 Actividad extrapresupuestaria
		\2 Idea final
			\3 Enorme contribución UE
				\4 Prosperidad
				\4 Estabilidad
				\4 Desarrollo regiones menos prósperas
			\3 Altibajos \4 Crisis
				\4 Ampliaciones
				\4 Ausencia de efecto estabilizador crisis
				\4[] Tamaño muy reducido
			\3 Situación actual
				\4 MFP 2014-2020
				\4 Revisión mitad de periodo
				\4 Calidad del gasto
				\4 Eficiencia
\end{esquemal}



































\conceptos

\concepto{Comité de Conciliación del Presupuesto}

If the Council does not accept the Parliament's amendments, a Conciliation Committee is set up, composed of the members of the Council or their representatives and an equal number of members representing the European Parliament. The Conciliation Committee is assigned to come up with a joint text within 21 days. If the conciliatory procedure fails, the Commission has to come up with a new draft budget.

Once a joint text is agreed upon by the Conciliation Committee in early November, the Council and the Parliament have 14 days to approve or reject it. The Parliament may adopt the budget even if the Council rejects the joint text, but with a specific majority (majority of component members and three-fifths of the votes cast). In case the Council and the Parliament both reject the joint draft or fail to decide, the budget is rejected and the Commission has to submit a new draft budget.

If, at the beginning of a financial year, the budget has not yet been definitively adopted, a sum equivalent to not more than 1/12 of the budget appropriations for the preceding financial year may be spent each month (system of provisional twelves).

\preguntas

\seccion{Test 2018}

\textbf{41.} Señale la respuesta \textbf{\underline{CORRECTA}} en relación a las finanzas de la UE:

\begin{itemize}
	\item[a] Jurídicamente, el Marco Financiero Plurianual de la UE se establece mediante Reglamento.
	\item[b] Algunas actividades financieras de la UE se encuentran al margen del presupuesto, como las desarrolladas por el Banco Europeo de Inversiones (BEI).
	\item[c] Algunas actividades financieras de la UE se encuentran al margen del presupuesto, como las desarrolladas por el Mecanismo Europeo de Estabilidad (MEDE).
	\item[d] Todas las opciones anteriores son correctas.
\end{itemize}

\seccion{Test 2017}

\textbf{40.} En el sistema de recursos propios/ingresos de la Unión Europea:

\begin{itemize}
	\item[a] Los recursos propios derivados de la Renta Nacional Bruta (RNB) se calculan como un porcentaje fijo determinado con ocasión de la aprobación del Marco Financiero Plurianual.
	\item[b] Una parte de los recursos propios tradicionales se los quedan los Estados en concepto de gastos de recaudación.
	\item[c] Los recursos propios basados en el Impuesto sobre el Valor Añadido (IVA) se recaudan actualmente a un tipo uniforme del 1,3\% sobre la base imponible armonizada del IVA de cada Estado miembro.
	\item[d] Los recursos propios tradicionales suponen más de la mitad de los ingresos/recursos propios totales del presupuesto de la Unión.
\end{itemize}

\seccion{Test 2016}
\textbf{44.} El presupuesto de la Unión Europea:
\begin{enumerate}
	\item[a] Asciende a unos 250.000 millones de euros.
	\item[b] La línea presupuestaria que absorbe más recursos en 2016 es \comillas{crecimiento sostenible}.
	\item[c] El Comité de Conciliación resuelve los conflictos originados en el proceso de aprobación del presupuesto comunitario y es un órgano paritario integrado por representantes de la Comisión, del Consejo y de Parlamento.
	\item[d] Está sujeto a límites derivados de un marco actualmente a siete años, cuya aprobación requiere unanimidad del Consejo previa aprobación de la mayoría de miembros del Parlamento.
\end{enumerate}

\seccion{Test 2015}
\textbf{43.} Señale la respuesta verdadera relativa al Marco Financiero Plurianual de la UE (o perspectivas financieras):
\begin{enumerate}
	\item[a] Actualmente, la rúbrica 2 de gasto se denomina \comillas{desarrollo sostenible y lucha contra el cambio climático}.
	\item[b] Tras la reforma del procedimiento de aprobación, el Parlamento Europeo fue el órgano que aprobó formalmente las perspectivas financieras 2014-2020.
	\item[c] Las transferencias nacionales basadas en la renta nacional bruta de los estados miembros son consideradas recursos propios de la UE.
	\item[d] Establecen únicamente un límite máximo anual para los créditos de compromiso.
\end{enumerate}

\seccion{Test 2013}
\textbf{39.} Son ingresos del presupuesto de la UE:
\begin{enumerate}
	\item[a] Un porcentaje del IRPF.
	\item[b] Un porcentaje del IVA.
	\item[c] Un porcentaje de los ingresos por contigentes.
	\item[d] Ninguna de las anteriores.
\end{enumerate}

\seccion{Test 2009}
\textbf{41.} El recurso propio de la Unión Europea basado en la Renta Nacional Bruta (RNB):
\begin{enumerate}
	\item[a] Es una cantidad fija establecida por la Comisión de la Unión Europea.
	\item[b] Se deriva de la aplicación del arancel aduanero común al valor en aduana de los productos importados de terceros países.
	\item[c] Pretende suplir las necesidades de gasto y equilibrar el presupuesto puesto que es un recurso complementario que cubrirá las necesidades de gasto no satisfechas por el resto de recursos.
	\item[d] Procede de los derechos de importación que gravan las importaciones de productos agrarios provenientes de terceros países.
\end{enumerate}

\seccion{Test 2007}
\textbf{42.} El Fondo Europeo de Desarrollo (FED):
\begin{enumerate}
	\item[a] Forma parte del presupuesto comunitario y se dirige a los países de América Latina.
	\item[b] No forma parte del presupuesto comunitario y se dirige a los países de América Latina.
	\item[c] Forma parte del presupuesto comunitario y se dirige a los países ACP.
	\item[d] No forma parte del presupuesto comunitario y se dirige a los países ACP.
\end{enumerate}

\seccion{Test 2006}
\textbf{43.} Constituyen los ingresos del Presupuesto de la Unión Europea:
\begin{enumerate}
	\item[a] Los recursos propios tradicionales, los recursos procedentes del impuesto sobre el valor añadido, los recursos basados en la renta nacional bruta, los otros ingresos y el cheque británico.
	\item[b] Los recursos procedentes del impuesto sobre el valor añadido y los recursos basados en la renta nacional bruta.
	\item[c] Los recursos propios tradicionales, los recursos procedentes del impuesto sobre el valor añadido, los recursos basados en la renta nacional bruta y los otros ingresos.
	\item[d] Los recursos propios tradicionales y las transferencias de los Estados miembros basados en la renta nacional bruta.
\end{enumerate}

\seccion{Test 2005}
\textbf{39.} Indique cuál de las siguientes afirmaciones es \textbf{CORRECTA}:
\begin{enumerate}
	\item[a] El Banco Europeo de Inversiones (BEI) no puede prestar a empresas privadas.
	\item[b] El BEI sólo puede financiar operaciones en la UE o países candidatos.
	\item[c] El BEI presta tanto a organismos públicos de la UE como de fuera de la UE así como a empresas privadas.
	\item[d] Cuando financia operaciones fuera de la UE el BEI exige siempre la garantía soberana de los Estados beneficiarios.
\end{enumerate}

\textbf{43.} En relación con el presupuesto de la Unión Europea, señale cuál de las siguientes afirmaciones es la única \textbf{FALSA}:
\begin{enumerate}
	\item[a] El Acuerdo Interinstitucional de 1999 sobre la disciplina presupuestaria y la mejora del procedimiento presupuestario incluye las perspectivas financieras para el periodo 2000-2006.
	\item[b] El Acuerdo Institucional actualmente en vigor incorpora un procedimiento de revisión de los límites máximos que permita hacer frente a situaciones imprevistas. Estas revisiones se adoptan por decisión común del Consejo y del Parlamento, a propuesta de la Comisión.
	\item[c] De las ocho rúbricas principales incluidas en las perspectivas, la de acciones estructurales y la de acciones exteriores representan la parte más importante de los gastos previstos.
	\item[d] Las reservas incluidas en la rúbrica 6 son las siguientes: la reserva monetaria (suprimida desde 2003), la reserva para garantías de préstamos a terceros países y la reserva para ayudas de emergencia.
\end{enumerate}

\notas

\textbf{2018}: \textbf{41}. D

\textbf{2017}: \textbf{40}. B

\textbf{2016}: \textbf{44}. D

\textbf{2015}: \textbf{43}. C

\textbf{2013}: \textbf{39}. B

\textbf{2009}: \textbf{41}. C

\textbf{2007}: \textbf{42}. D

\textbf{2006}: \textbf{43}. C

\textbf{2005}: \textbf{39}. C \textbf{43}. C

\bibliografia

Mirar en Palgrave:
\begin{itemize}
	\item European Central Bank and Monetary Policy in the Euro Area
	\item European Monetary Integration
	\item European Monetary Union
	\item European Union Budget
	\item Euro Zone Crisis 2010
\end{itemize}

Banque de France (2019) \textit{Le cadre financier pluriannuel de l'Union européenne pour 2021-2027: nouveaux équilibres, nouveaux enjeux} Bulletin de la Banque de France - 226/3 Novembre-Décembre 2019 -- En carpeta del tema

Brehon, N.J. \textit{European Union Budget: which possible compromise is there between France and Germany?} (2018) European Issues nº 476. Fondation Robert Schuman Policy Paper -- En carpeta del tema

Delgado-Téllez, M.; Kataryniuk, I.; López-Vicente, F.; Pérez, J. J. (2020) \textit{Endeudamiento supranacional y necesidades de financiación en la Unión Europea} Banco de España. Documentos ocasionales. Nº 2021 -- En carpeta del tema.

El Agraa, A. \textit{The European Union}. Ch. 19 The General Budget

European Court of Auditors. \textit{The Commission's Proposal for the 2021-2027 Multiannual Financial Framework} Briefing Paper - July 2018 -- En carpeta del tema

European Parliament. \textit{Multiannual Financial Framework 2021-2027: Commission Proposal. Initial comparison with the current MFF} (2018) Alina Dobreva -- European Parliamentary Research Service -- En carpeta del tema

EPRS (European Parliamentary Research Service) (2018) \textit{2021-2027 multiannual financial framework and new own resources. Analysis of the Commission's proposal} European Parliament -- En carpeta del tema



\end{document}
