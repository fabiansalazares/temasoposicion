\documentclass{nuevotema}

\tema{3B-12}
\titulo{Balanza de pagos: concepto, medición e interpretación}

\begin{document}

\ideaclave

\seccion{Preguntas clave}
\begin{itemize}
	\item ¿Qué es la balanza de pagos?
	\item ¿Cómo se estructura?
	\item ¿Cómo se construye?
	\item ¿Cómo se interpreta?
	\item ¿Qué relación tiene con otros conceptos?
	\item ¿Qué es la Posición de Inversión Internacional?
\end{itemize}

\esquemacorto

\begin{esquema}[enumerate]
	\1[] \marcar{Introducción}
		\2 Contextualización
			\3 Flujos econ. y fin. internacionales
			\3 Impacto económico de la globalización
			\3 Información sobre transacciones internacionales
		\2 Objeto
			\3 ¿Qué es la balanza de pagos?
			\3 ¿Para qué sirve?
			\3 ¿Cómo se construye?
			\3 Cómo se estructura?
			\3 ¿Cómo se interpreta?
			\3 ¿Qué relación con otros conceptos?
			\3 ¿Qué es la posición internacional de inversión?
		\2 Estructura
			\3 Concepto y medición
			\3 Panorama general de las cuentas internacionales
			\3 Interpretación
	\1 \marcar{Concepto y medición}
		\2 Conceptos
			\3 Cuentas internacionales
			\3 Posición de Inversión Internacional (PII)
			\3 Balanza de pagos
			\3 Cuenta de otras variaciones de activos y pasivos
		\2 Medición y registro
			\3 Principios contables
			\3 Criterio de residencia
			\3 Valoración
			\3 Fuentes de información
	\1 \marcar{Cuenta corriente}
		\2 Cuenta de bienes
			\3 Idea clave
			\3 Hidrocarburos y extractivos
			\3 Manufacturas
		\2 Cuenta de servicios
			\3 Idea clave
			\3 Viajes
			\3 Servicios a empresas
			\3 Transporte
			\3 Comunicaciones
			\3 Servicios financieros y seguros
			\3 Propiedad intelectual
			\3 Servicios manufactureros
			\3 Construcción
			\3 Mantenimiento y reparación
		\2 Rentas primarias
			\3 Idea clave
			\3 Remuneración de empleados
			\3 Dividendos
			\3 Beneficios reinvertidos
			\3 Intereses de deuda
			\3 Ingresos de inversión por pólizas de seguros, fondos de pensiones y garantías
			\3 Rentas de recursos naturales
			\3 Impuestos y subvenciones sobre productos y producción
		\2 Rentas secundarias
			\3 Idea clave
			\3 Transferencias personales
			\3 Impuestos corrientes sobre ingreso, patrimonio, etc...
			\3 Contribuciones sociales
			\3 Prestaciones sociales
			\3 Primas netas de seguros no vida y garantías
			\3 Indemnizaciones de seguros no de vida y ejecuciones
			\3 Cooperación internacional corriente
			\3 Transferencias corrientes diversas
			\3 Ajuste por cambio de derechos jubilatorios
	\1 \marcar{Cuenta de capital}
		\2 Transferencias de capital
			\3 Condonación de deudas
			\3 Otras transferencias de capital
		\2 Adq./dis. activos no financieros no producidos
			\3 Recursos naturales
			\3 Contratos, arrendamientos y licencias
			\3 Activos de comercio y fondos de comercio
	\1 \marcar{Cuenta financiera}
		\2 Idea clave
			\3 Transacciones con activos y pasivos financieros
			\3 Categorías funcionales
			\3 Saldo global
		\2 Inversión directa
			\3 Inversión que implica control sobre gestión
			\3 Incluye también:
			\3 Inversión que implica control sobre gestión
			\3 Tipos de IDE
			\3 Vertical vs horizontal
			\3 Incluye también:
		\2 Inversión de cartera
			\3 Inversión en títulos de deuda o acciones
		\2 Derivados financieros y OCAE
			\3 Derivados financieros distintos de reservas
			\3 No devengan ingresos primarios
			\3 Opciones de compra de acciones para empleados
		\2 Otra inversión
			\3 Categoría residual
			\3 Otras participaciones de capital
			\3 Moneda y depósitos
			\3 Préstamos
			\3 Reservas técnicas de seguros
			\3 Créditos y anticipos comerciales
			\3 Cuentas por cobrar/pagar
			\3 Asignaciones de DEG
		\2 Reservas
			\3 Activos externos disponibles de inmediato
			\3 Usos
			\3 Denominados en moneda extranjera
			\3 Deben estar realmente constituidos
			\3 Activos de reserva
		\2 Movimientos clandestinos
	\1 \marcar{Interpretación de la balanza de pagos}
		\2 Balanza de bienes/balanza comercial
			\3 Idea clave
			\3 Saldos
		\2 Balanza de servicios
			\3 Idea clave
			\3 Saldos
		\2 Balanza de bienes y servicios
			\3 Idea clave
			\3 Saldos
		\2 Balanza corriente
			\3 Idea clave
			\3 Saldo
		\2 Balanza de rentas
			\3 Idea clave
			\3 Saldos
		\2 Balanza general (overall balance)
			\3 Idea clave
			\3 Saldos
		\2 Análisis desde la óptica ahorro e inversión
			\3 Idea clave
			\3 Formulación
			\3 Implicaciones
		\2 Cap./Nec. de financiación y cuenta financiera
			\3 Idea clave
			\3 Implicaciones
		\2 Errores y omisiones
			\3 Idea clave
			\3 Saldos
	\1 \marcar{Posición de inversión internacional}
		\2 Idea clave
			\3 Cuenta estadística
			\3 Fuentes de variación
			\3 Importancia creciente
		\2 Estructura
			\3 Idea clave
			\3 Categoría funcional
			\3 Instrumentos
			\3 Sector institucional de la parte residente
		\2 Categorías funcionales
			\3 Inversión directa
			\3 Inversión de cartera
			\3 Derivados financieros y OCAE
			\3 Otra inversión
			\3 Reservas
			\3 Valoración
		\2 Interpretación de la PII
			\3 Vulnerabilidades
			\3 Estática comparativa
			\3 Sostenibilidad
			\3 Desequilibrios globales
	\1 \marcar{Crisis de balanza de pagos}
		\2 Idea clave
			\3 Contexto
			\3 Objetivos
			\3 Resultados
		\2 Formulación
			\3 Identidades del ahorro, la inversión y la entrada de capital
			\3 Sudden stops y reversiones de flujos de capital
			\3 Factores de riesgo de sudden stops
		\2 Implicaciones
			\3 Flujos de capital pueden ser desestabilizantes
			\3 Liberalización de CF puede tener inconvenientes
			\3 Sistema financiero doméstico es factor clave
			\3 Uniones monetarias requieren mecanismos emergencia
		\2 Valoraciones
			\3 Suceso recurrente
			\3 Papel clave del FMI
	\1 \marcar{Conclusión}
		\2 Recapitulación
			\3 Conceptos y medición
			\3 Balanza de pagos
			\3 Posición de inversión internacional
		\2 Idea final
			\3 Concepto contable
			\3 Aplicaciones de la balanza de pagos y PII
			\3 Importancia creciente de la PII

\end{esquema}

\esquemalargo




















\begin{esquemal}
	\1[] \marcar{Introducción}
		\2 Contextualización
			\3 Flujos econ. y fin. internacionales
				\4 Cantidades crecientes
				\4 Diferentes características
				\4 Comercio de bienes y servicios
				\4 Rentas primarias
				\4 Transferencias corrientes, permantes..
				\4 Flujos financieros
				\4[] Innovaciones financieras
				\4[] Reducción del papel del oro
			\3 Impacto económico de la globalización
				\4 Cadenas de producción globales
				\4 Obligaciones financieras
				\4 Determinación de renta y precios
			\3 Información sobre transacciones internacionales
				\4 Necesario conocer estructura de:
				\4[] $\to$ transacciones internacionales
				\4[] $\to$ derechos y obligaciones frente al exterior
				\4[] Con el objetivo de:
				\4[] -- Entender funcionamiento de economía
				\4[] -- Comparar con otras economías
				\4[] -- Comparar con otra información macroeconómica
				\4[] -- Diagnosticar desequilibrios
				\4[] -- Diseñar políticas que reduzcan
				\4[$\then$] Necesario marco sistemático y estándarizado
				\4[$\then$] BP: transacciones con no residentes en periodo
				\4[$\then$] PII: valor neto frente al exterior
		\2 Objeto
			\3 ¿Qué es la balanza de pagos?
			\3 ¿Para qué sirve?
			\3 ¿Cómo se construye?
			\3 Cómo se estructura?
			\3 ¿Cómo se interpreta?
			\3 ¿Qué relación con otros conceptos?
			\3 ¿Qué es la posición internacional de inversión?
		\2 Estructura
			\3 Concepto y medición
			\3 Panorama general de las cuentas internacionales
			\3 Interpretación
	\1 \marcar{Concepto y medición}
		\2 Conceptos
			\3 Cuentas internacionales
				\4 Resumen relaciones económicas entre países
				\4[] Derechos y obligaciones con no residentes
				\4[] $\to$ En qué estado se encuentran
				\4[] $\to$ Cómo varían en un periodo dado
				\4 Tres elementos:
				\4[] I. Posición de Inversión Internacional (PII)
				\4[] II. Balanza de pagos
				\4[] III. Otras variaciones de activos y pasivos
			\3 Posición de Inversión Internacional (PII)
				\4 Estado estadístico que recoge el valor de:
				\4[] $\to$ Activos financieros y lingotes de oro
				\4[] $\to$ Pasivos financieros
				\4 Frente al exterior
				\4[] Activos:
				\4[] $\to$ pasivos de no residentes en manos de residentes
				\4[] Pasivos:
				\4[] $\to$ pasivos de residentes en manos de no residentes
				\4 En un momento temporal determinado
				\4 Clasificando en términos de:
				\4[] $\to$ Sectores institucionales
				\4[] $\to$ Función de los activos
				\4 Posición de Inversión Internacional Neta:
				\4[] Diferencia entre activos y oro y pasivos
			\3 Balanza de pagos
				\4 Estado estadístico que resume transacciones
				\4[] Reales y de carácter financiero
				\4 Entre residentes y no residentes
				\4[] Desde el punto de vista de un país determinado
				\4 En un periodo temporal
				\4[] La balanza de pagos mide flujos
				\4 Expresada en términos de cuentas
				\4[] Sujeto a métodos contables
				\4 Historia
				\4[] Primera edición del manual en 1948
				\4[] Sexta y última edición: 2008
				\4[] $\to$ Revisiones menores respecto a 5a edición
			\3 Cuenta de otras variaciones de activos y pasivos
				\4 Variaciones de posiciones financieras
				\4[] Por motivos distintos de transacciones con RM
				\4 A veces denominados ``otros flujos''
				\4 Revisten importancia por sí solos
				\4[] No son un componente residual
				\4[] Eventos económicos de gran importancia
				\4[] $\to$ Fluctuaciones de tipo de cambio
				\4[] $\to$ Impagos que reducen valor bruscamente
				\4[] $\to$ ...
				\4 Dividida en dos partes
				\4[] 1. Variaciones de volumen
				\4[] \quad Condonaciones y cancelaciones contables
				\4[] \quad Apariciones y desapariciones de activos
				\4[] \quad Reclasificaciones
				\4[] \quad Cambios de residencia de entidades
				\4[] 2. Revalorizaciones
				\4[] \quad Debidas a tipo de cambio
				\4[] \quad Debidas a otras variaciones de precios
		\2 Medición y registro
			\3 Principios contables
				\4 Partida doble y cuádruple
				\4[] Toda transacción se registra como crédito y débito\footnote{Ver pág. 10 del 6MBP en español. La clave para entender es esto es tener en cuenta que la balanza de pagos mide una parte de las variaciones del valor neto de la economía, que no es sino un balance. Por eso, si aumentan los activos es un débito, si bajan un crédito, y si aumentan los pasivos un crédito y si bajan un débito. }
				\4[] Suma de créditos igual a suma de débitos
				\4[] Débitos:
				\4[] $\to$ Importaciones de ByS
				\4[] $\to$ Ingresos a pagar
				\4[] $\to$ $\uparrow$ de activos
				\4[] $\to$ $\downarrow$ de pasivos
				\4[] Créditos
				\4[] $\to$ Exportaciones de ByS
				\4[] $\to$ Ingresos a cobrar
				\4[] $\to$ $\uparrow$ de pasivos
				\4[] $\to$ $\downarrow$ de activos
				\4 Devengo
				\4[] En general, criterio temporal relevante
				\4[] Excepcionalmente, criterio de caja aplicable
				\4[] $\to$ Si devengo muy costoso o complejo
				\4[] Reglas aplicables a bienes, ANFNP y AFinancieros
				\4[] $\to$ Momento de transferencia de propiedad económica
				\4[] Reglas aplicables a servicios:
				\4[] $\to$ Momento de la prestación
				\4[] Reglas aplicables a rentas primaria y secundaria:
				\4[] $\to$ Cuando surge de derecho de cobro/obligación de pago
				\4 Partida cuádruple
				\4[] Toda transacción es débito y crédito
				\4[] $\to$ En la contabilidad de un agente
				\4[] $\to$ En la contabilidad de dos agentes
				\4[] Todo asiento tiene contrapartida en:
				\4[] $\to$ Cuentas de residentes
				\4[] $\to$ Cuentas de no residentes
				\4 Ejemplo:
				\4[] Español vende barco a marroquí por 1000€
				\4[] Española:
				\4[] $\to$ Crédito: exportación 1000€
				\4[] $\to$ Débito: aumento activos 1000€
				\4[] $\Rightarrow$ Igualdad débito=crédito
				\4[] Marroquí:
				\4[] $\to$ Débito: importación 1000€
				\4[] $\to$ Crédito: aumento pasivos 1000€
				\4[] $\Rightarrow$ Igualdad D=C en cada país
				\4[] $\Rightarrow$ Igualdad D=C entre ambos países
			\3 Criterio de residencia
				\4 Centro de interés económico predominante
				\4[] Lugar de producción
				\4[] Vivienda principal
				\4[] Domicilio habitual
				\4[] Duración de la estancia en un territorio
				\4 Excepciones
				\4[] Turistas, estudiantes, funcionarios, tripulaciones...
				\4 Criterio en empresas
				\4[] Lugar donde se produce cantidad significativa
				\4[] Criterios complementarios
				\4[] $\to$ Jurisdicción de legislación aplicable
				\4[] $\to$ Base de operaciones
				\4[] Filiales
				\4[] $\to$ Residentes donde prestan servicios
				\4[] Sucursales:
				\4[] $\to$ Sólo consideradas residentes excepcionalmente
				\4 Administraciones Públicas
				\4[] Todos los organismos de un país determinado
				\4[] $\to$ Son residentes
				\4[] Aunque estén físicamente en el exterior:
				\4[] $\to$ Embajadas, consulados, bases militares
				\4[] Organismos internacionales no son residentes
			\3 Valoración
				\4 En moneda local
				\4 Transacciones:
				\4[] A precio vigente en momento de operación
				\4 Posiciones en activos y pasivos
				\4[] Precio a fecha de cierre
				\4[] Activos no negociados: criterios alternativos
				\4[] $\to$ Valor contable
				\4[] $\to$ Valor nominal
			\3 Fuentes de información
				\4 Proveedores de Servicios de Pagos
				\4[] Canalizan flujos monetarios
				\4[] Informan a organismo compilador de BP y PII
				\4[] Cobros/pagos recibidos/ordenados
				\4[] Entre residentes y no residentes
				\4[] Superen umbral monetario mínimo
				\4[] $\then$ Aplicables a financieras y no financieras
				\4[] $\then$ Útiles para comprobar coherencia
				\4 Estadísticas de Comercio Internacional de Mercancías (ECIM)
				\4[] Departamentos de aduanas respectivos
				\4[] $\to$ AEAT: Departamento de Aduanas e Impuestos Especiales\footnote{DataComex en España (consultable): \url{http://datacomex.comercio.es/principal\_comex\_es.aspx}}
				\4[] Comercio intracomunitario:
				\4[] $\to$ Declaración Instrastat
				\4[] comercio extracomunitario
				\4[] $\to$ Declaraciones de despacho de aduanas\footnote{Denominados Documento Único Administrativo.}
				\4[] $\then$ Utilizada en bienes
				\4 Información sobre servicios
				\4[] MECIS -- Manual de Estadísticas del Comercio Int. de Servicios
				\4[] $\to$ Definiciones coherentes con GATS
				\4[] $\to$ Mismo marco conceptual del SCN 2008
				\4[] Encuestas a personas físicas o empresas
				\4[] Encuesta de gasto turístico
				\4[] Encuesta sobre pasajeros de vuelos internacionales
				\4[] Encuestas a empresas
				\4[] Información fiscal
				\4[] Información de comercio electrónico
				\4[] Pagos con tarjetas bancarias
				\4 Registros de transferencias
				\4[] Registros de instituciones supranacionales y AD
				\4[] Remesas: padrón de nacionales en extranjero
				\4[$\then$] Transferencias corrientes y de capital del gobierno
				\4 Cuenta financiera y PII
				\4[] Estadísticas del banco central respectivo
				\4[] Requerimientos de información a Inst. Monetarias
				\4[] Declaraciones de trans. financieras con exterior
				\4[] Declaraciones de bienes en extranjero
				\4[] Registros de IDE e inversión de cartera
	\1 \marcar{Cuenta corriente}
		\2 Cuenta de bienes\footnote{Ver Capítulo 10 de 6MBP ``Cuenta de bienes y servicios''.}
			\3 Idea clave
				\4 Créditos
				\4[] Exportaciones
				\4 Débitos
				\4[] Importaciones de bienes
				\4 Valoración en ECIM
				\4[] X a FOB
				\4[] M a CIF
				\4 En balanza de pagos
				\4[] Transporte y seguros
				\4[] $\to$ A cuenta de servicios
				\4[] $\then$ No incluido en cuenta de mercancías
				\4[] $\then$ Importaciones no pueden valorarse a CIF
				\4[] $\then$ Importaciones deben deagregarse
				\4 Necesario estimar FOB a partir de CIF
				\4 Bienes en tránsito no son X ni M
				\4 Desplazamientos temporales no van a CMercancías
				\4[] Criterio es cambio de propiedad
				\4[] $\to$ No cambio de localización física
			\3 Hidrocarburos y extractivos
			\3 Manufacturas
				\4 Maquinaria y transportes
				\4 Químicos
				\4 Acero y hierro
				\4 Prendas de vestir
				\4 Textil
				\4 Otras manufacturas
		\2 Cuenta de servicios\footnote{Ver Capítulo 10 de 6MBP ``Cuenta de bienes y servicios''.}
			\3 Idea clave
				\4 También denominado ``comercio invisible''
				\4 Créditos
				\4[] Exportaciones de servicios
				\4 Débitos
				\4[] Importaciones de servicios
			\3 Viajes
			\3 Servicios a empresas
			\3 Transporte
			\3 Comunicaciones
			\3 Servicios financieros y seguros
			\3 Propiedad intelectual
			\3 Servicios manufactureros
			\3 Construcción
			\3 Mantenimiento y reparación
		\2 Rentas primarias
			\3 Idea clave
				\4[] Ingresos y pagos de factores de producción
				\4[] $\to$ Trabajo, capital, recursos naturales
				\4[] Saldo es diferencia entre RNB y PIB
				\4[] $\to$ Rentas a cobrar de no residentes
				\4[] $\to$ Rentas a pagar a no residentes
				\4[] Derivado de c. de Asignación de la Renta Primaria
			\3 Remuneración de empleados
				\4[] Pago a cambio del insumo de la mano de obra
				\4[] Registrado cuando empleador y empleado
				\4[] $\to$ Residentes en diferentes economía
				\4[] Incluye:
				\4[] $\to$ Sueldos en especie y efectivo
				\4[] $\to$ Contribuciones sociales
			\3 Dividendos
				\4[] Beneficios repartidos a titulares de acciones
			\3 Beneficios reinvertidos
				\4[] Difícil cálculo
				\4[] Derivado mediante extrapolación, relaciones y modelos
			\3 Intereses de deuda
				\4[] Ingresos por tenencia de activos financieros
				\4[] $\to$ Depósitos
				\4[] $\to$ Títulos de deuda
				\4[] $\to$ Préstamos
				\4[] $\to$ Otras cuentas por cobrar
				\4[] $\to$ Tenencias de DEG
			\3 Ingresos de inversión por pólizas de seguros, fondos de pensiones y garantías
				\4[] Rendimiento generado a favor de los titulares
				\4[] Por tenencia de reservas técnicas e ingresos por cobrar
			\3 Rentas de recursos naturales
				\4[] Derivadas de arrendamiento de recursos naturales
			\3 Impuestos y subvenciones sobre productos y producción
				\4 Clasificación funcional alternativa\footnote{Relacionada con clasificación funcional de activos y pasivos.}
				\4[] Ingreso de inversión
				\4[] $\to$ Inversión directa
				\4[] $\to$ Inversión de cartera
				\4[] $\to$ Otra inversión
				\4[] Otro ingreso primario
				\4[] $\to$ Renta
				\4[] $\to$ Impuestos a producción e importaciones\footnote{Los pagos e ingresos a la UE en concepto de ``recursos propios tradicionales'' se registran en la cuenta de ingreso primario.}
				\4[] $\to$ Subsidios
				\4[] $\to$ FEAGA de PAC
		\2 Rentas secundarias
			\3 Idea clave
				\4[] Transferencias corrientes entre
				\4[] $\to$ Residentes y no residentes
				\4[] $\to$ Efectivo o en especie
				\4[] ¿Qué es una transferencia?
				\4[] Suministro De bien, servicio o activo fin. o no producido
				\4[] $\to$ Sin obtener a cambio valor económico
				\4[] Impuestos no productos y producción
				\4[] $\to$ Son transferencias
				\4[] $\to$ Dan lugar a contrapartidas inciertas
				\4[] Transferencias entre entidades comerciales son raras
				\4[] Diferencia entre TRK y TRC
				\4[] $\to$ TRK traspaso prop. de un activo no efectivo ni existencias
				\4[] $\to$ O acreedor condona una obligación
				\4[] $\to$ U obligan a adquirir o disponer de un activo
				\4[] $\then$ Transferencias de capital no incluidas
			\3 Transferencias personales
				\4[] Efectivo o especie de hogares res. a no hogares no rs.
				\4[] Fundamentalmente remesas y juegos de azar
			\3 Impuestos corrientes sobre ingreso, patrimonio, etc...
				\4[] Aplicados a no residentes
			\3 Contribuciones sociales
				\4[] Contribuciones a sistemas de seguros sociales
			\3 Prestaciones sociales
				\4[] Pagaderas por sistemas de SS y planes de pensiones
			\3 Primas netas de seguros no vida y garantías
			\3 Indemnizaciones de seguros no de vida y ejecuciones
			\3 Cooperación internacional corriente
				\4[] Entre gobiernos y organismos internacionales
				\4[] No comprende las destinadas a formación de capital
				\4[] FSE -- Fondo Social Europeo
				\4[] FED -- Fondo Europeo de Desarrollo
			\3 Transferencias corrientes diversas
				\4[] Todas las no descritas hasta ahora
				\4[] Incluye multas y sanciones, pagos por daños
				\4[] Pagos a autoridades supranacionales
				\4[] $\to$ No son impuestos pero son obligatorios
				\4[] $\to$ Recurso IVA de UE
				\4[] $\to$ Recurso RNB de UE
			\3 Ajuste por cambio de derechos jubilatorios
				\4[] Realmente no incluido en rentas secundarias
				\4[] Cuenta junto a ByS, RPrimaria, RSecundaria
	\1 \marcar{Cuenta de capital}
		\2 Transferencias de capital
			\3 Condonación de deudas
			\3 Otras transferencias de capital
				\4[] Compensaciones y pagos importantes no periódicos
				\4[] Donaciones y herencias
				\4[] $\to$ Incluyendo ONG y gobiernos para pagar deuda
				\4[] Rescates financieros
				\4[] $\to$ Diferencia entre precio pagado y precio de mercado
				\4[] FEADER
				\4[] FEDER
				\4[] Fondo de Cohesión
		\2 Adq./dis. activos no financieros no producidos
			\3 Recursos naturales
				\4[] Tierras, minerales, pesca, espectro...
			\3 Contratos, arrendamientos y licencias
			\3 Activos de comercio y fondos de comercio
				\4[] Marcas comerciales, cabeceras, logotipos...
				\4[] $\to$ Cuando se venden por separado
	\1 \marcar{Cuenta financiera}

		\2 Idea clave
			\3 Transacciones con activos y pasivos financieros
				\4[] Entre residentes y no residentes
			\3 Categorías funcionales
				\4[] ¿Para qué sirven los activos y pasivos?
				\4[] Saldos de categorías funcionales
				\4[] $\to$ Especialmente relevantes para análisis
			\3 Saldo global
				\4[] Préstamo neto/capacidad de financiación
				\4[] $\to$ Si positivo
				\4[] $\then$ Economía financia al resto del mundo
				\4[] Endeudamiento neto/necesidad de financiación
				\4[] $\to$ Si negativo
				\4[] $\then$ Economía necesita financiación del RM
				\4[] En teoría igual a CC + CK
				\4[] $\to$ En la práctica, discrepancias estadísticas
				\4[] $\then$ Errores y omisiones para equilibrar
		\2 Inversión directa
			\3 Inversión que implica control sobre gestión
				\4[] O grado significativo de influencia
				\4[] $\to$ >50\% control
				\4[] $\to$ >10\% grado significativo
			\3 Incluye también:
				\4[] Reinversión de beneficios/pérdidas
				\4[] Inversión directa en especie
				\4[] Fusiones y adquisiciones
				\4[] Reestructuración de sociedades
				\4[] Dividendos de liquidación
				\4[] Retiradas de capital
			\3 Inversión que implica control sobre gestión
				\4[] O grado significativo de influencia
			\3 Tipos de IDE
				\4[] Brownfield
				\4[] $\to$ Adquisición de planta ya existente
				\4[] $\to$ Planta existente continúa su actividad anterior
				\4[] Greenfield
				\4[] $\to$ Construcción de nueva planta
				\4[] $\to$ Planta antigua que cambia actividad tras inversión
			\3 Vertical vs horizontal
				\4[] Vertical
				\4[] $\to$ División del proceso productivo en fases
				\4[] $\to$ Plantas extranjeras llevan a cabo sólo una fase
				\4[] $\to$ Producción destinada a reexportación
				\4[] $\then$ Aprovechar ventajas competitivas en determinado segmento
				\4[] Horizontal
				\4[] $\to$ Replicación de proceso productivo en otro país
				\4[] $\to$ Evitar costes de transporte/aranceles
			\3 Incluye también:
				\4[] Reinversión de beneficios/pérdidas
				\4[] Inversión directa en especie
				\4[] Fusiones y adquisiciones
				\4[] Reestructuración de sociedades
				\4[] Dividendos de liquidación
				\4[] Retiradas de capital
		\2 Inversión de cartera
			\3 Inversión en títulos de deuda o acciones
				\4[] Diferente de inversión directa o reservas
				\4[] No implica gestión
				\4[] Necesario que sean títulos valor
				\4[] $\to$ Si no lo son, IDE u otra inversión
				\4[] $\to$ No se limita a títulos cotizados
		\2 Derivados financieros y OCAE\footnote{Opciones de Compra de Acciones por parte de Empleados.}
			\3 Derivados financieros distintos de reservas
				\4[] Sólo si no están relacionados con gestión de reservas
			\3 No devengan ingresos primarios
				\4[] Montos devengados son revalorizaciones
				\4[] $\to$ Cuenta de otras variaciones de activos y pasivos
			\3 Opciones de compra de acciones para empleados
				\4[] Contrapartida de
				\4[] $\to$ remuneración de empleados
				\4[] $\to$ inversión directa
				\4[] Se registra diferencia entre:
				\4[] $\to$ Valor de mercado
				\4[] $\to$ Precio pagado por comprador de acción
		\2 Otra inversión
			\3 Categoría residual
				\4[] Activos no incluidos en el resto
			\3 Otras participaciones de capital
			\3 Moneda y depósitos
			\3 Préstamos
				\4[] Incluidos crédito del FMI
			\3 Reservas técnicas de seguros
			\3 Créditos y anticipos comerciales
			\3 Cuentas por cobrar/pagar
			\3 Asignaciones de DEG
				\4[] No la tenencia, que va a reservas
		\2 Reservas
			\3 Activos externos disponibles de inmediato
				\4[] Bajo control de autoridades monetarias
				\4[] Disponibles incondicionalmente
				\4[] Deben ser líquidos
			\3 Usos
				\4[] Satisfacer necesidad de financiación de BP
				\4[] Intervenir mercados para $\uparrow \downarrow$ tipo de cambio
				\4[] Otros fines conexos
			\3 Denominados en moneda extranjera
			\3 Deben estar realmente constituidos
				\4[] Excluidos activos potenciales
			\3 Activos de reserva
				\4[] Oro monetario
				\4[] DEG
				\4[] Posición en FMI
				\4[] Otros activos
				\4[] $\to$ Moneda y depósitos
				\4[] $\to$ Títulos
		\2 Movimientos clandestinos
	\1 \marcar{Interpretación de la balanza de pagos}

		\2 Balanza de bienes/balanza comercial
			\3 Idea clave
				\4 Diferencia entre valor de X y M de bienes
				\4 Fortaleza exportadora de mercancías
			\3 Saldos
				\4 Saldo positivo
				\4[] El valor de las mercancías exportadas
				\4[] $\to$ Supera a las importadas
		\2 Balanza de servicios
			\3 Idea clave
			\3 Saldos
				\4 Diferencia entre valor de X y M de servicios
		\2 Balanza de bienes y servicios
			\3 Idea clave
			\3 Saldos
				\4 Diferencia de valor de X y M de ByS
		\2 Balanza corriente
			\3 Idea clave
			\3 Saldo
				\4 Saldo de la cuenta corriente
				\4 Diferencia de valor entre:
				\4[] Exportaciones, RP y RS cobradas
				\4[] -- Importaciones, RP y RS pagadas
				\4 Saldo positivo
				\4[] Economía no necesita financiación
				\4[] $\to$ Para cubrir gastos corrientes
				\4[] Economía puede financiar al exterior
				\4[] $\to$ Aumenta stock de activos fin. extranjeros
				\4[] A pesar de ello, puede aún necesitar financiarse
				\4[] $\to$ Si realiza transferencias de K elevadas
		\2 Balanza de rentas
			\3 Idea clave
			\3 Saldos
				\4 Suma de saldo de RPrimarias y RSecundarias
		\2 Balanza general (overall balance)
			\3 Idea clave
			\3 Saldos
				\4 Variación de las reservas
				\4[] Resultado de todas las demás transacciones
				\4[] $\to$ ¿Cuánto aumenta el stock de reservas?
				\4[] $\to$ ¿Cuántas reservas hacen falta para financiar pagos netos?
				\4 Cuenta financiera dividida en dos
				\4[] $\text{CF} = \text{CF}^* + \Delta \text{R}$
				\4[] $\to$ Activos financieros netos de reservas -- $\text{CF}^*$
				\4[] $\to$ Transacciones de reservas
				\4 Saldo de la balanza general
				\4[] $\text{CC} + \text{CK} - \text{CF}^* = \Delta {R}$
		\2 Análisis desde la óptica ahorro e inversión
			\3 Idea clave
			\3 Formulación
				\4 Ingreso nacional total:
				\4[] $\text{RNBD} = \text{Y} + \text{RP} + \text{RS} = \text{C}+\text{G}+\text{I} + \text{X} - \text{M} + \text{RP}+\text{RS}$
				\4 Ahorro (S):
				\4[] Renta nacional no destinada a consumo corriente:
				\4[] $\text{S} = \text{RNBD} - \text{C} - \text{G}$
				\4 Ahorro se utiliza para:
				\4[] $\to$ Financiar inversión nacional
				\4[] $\to$ $\uparrow$ valor neto de activos financieros con RM
				\4[] $\to$ Realizar transferencias netas de K al RM
				\4[] $\text{S}=\text{I} + \underbrace{(\text{TRK}_{\to RM} - \text{TRK}_{\to N})}_{\text{-CK}} + \text{CNF}$
				\4[] $\underbrace{\text{S}-\text{I}}_{\text{CC}}+ \text{CK} = \text{CF}$
				\4[] $\Rightarrow$ \fbox{$\text{CC} = \text{S} - \text{I}$}
				\4 Ahorro público y privado
				\4[] Ahorro nacional como suma de púb. y privado
				\4[] $\text{S} = \text{S}_\text{P} + \text{S}_\text{G}$
				\4[] $\Rightarrow$ $\text{CC} = (\text{S}_P - \text{I}_P) + (\text{S}_G - \text{I}_G)$
				\4[] $\Rightarrow$ Déf. público no compensado por ahorro privado $\to$ Hipótesis de los déficit gemelos
			\3 Implicaciones
		\2 Cap./Nec. de financiación y cuenta financiera
			\3 Idea clave
			\3 Implicaciones
				\4 Variación de la PIIN derivada de transacciones
				\4 Dos formas de calcular
				\4[] Suma de saldos de CC y CK
				\4[] $\to$ Renta que no se gasta ni se transfiere
				\4[] $\Rightarrow$ Se dedica a financiar/financiarse del RM
				\4[] Saldo de CF
				\4[] $\to$ Diferencia entre variaciones netas de A y P
				\4 Interpretación de la Cap./Nec.
				\4[] ¿Cuánto puede dedicar a financiar el RM?
				\4[] ¿Cuánto financia el RM al país?
		\2 Errores y omisiones
			\3 Idea clave
				\4 En teoría CC + CK = Cap.(+)/Nec.(-) = CF
				\4[] $\text{Cr} - \text{D} + \text{Cr} - \text{D} = \text{D} - \text{Cr}$
				\4[] $\to$ Dos cálculos de CNF deberían coincidir
				\4 En la práctica:
				\4[] Cap./Nec. no coinciden
				\4[] CC + CK arrojan una Cap/Nec diferente a CF
				\4[$\Rightarrow$] Discrepancias estadísticas en la medición
				\4 Errores y omisiones
				\4[] Elemento añadido a CC + CK:
				\4[] $\text{CC} + \text{CK} + \text{EyO} = \text{CF}$
			\3 Saldos
				\4 Si positivo:
				\4[] Créditos de CC y CK demasiado bajos
				\4[] Débitos de CC y CK demasiado altos
				\4[] $\to$ Exportaciones o $\text{TRC}_{\to N}$ infravaloradas
				\4[] $\to$ Importaciones o $\text{TRC}_{\to RM}$ sobrevalorados
				\4[] $\then$ Débitos de CF demasiado altos
				\4[] $\then$ Créditos de CF demasiado bajos
				\4[] $\then$ VNActivos sobrevalorada
				\4[] $\then$ VNPasivos infravalorada
				\4[] Signo consistente en el tiempo
				\4[] $\to$ Sesgo de estimación permanente
				\4 Si negativo
				\4[] Créditos de CC y CK demasiado altos
				\4[] $\to$ Exportaciones o ingresos sobrevalorados
				\4[] $\to$ Importaciones o pagos infravalorados
				\4[] Adquisición de activos no registrada
				\4[] Redención de pasivos no registrada
				\4[] Nuevos pasivos sobrevalorados
	\1 \marcar{Posición de inversión internacional}
		\2 Idea clave
			\3 Cuenta estadística
				\4 Subconjunto del balance nacional
				\4 Valor y composición
				\4[1] Activos financieros frente a NR y oro en reservas
				\4[2] Pasivos de residentes frente a NR
				\4 En momento determinado:
				\4[] $\to$ PII de apertura
				\4[] $\to$ PII de cierre
				\4 PII Neta
				\4[] Diferencia entre Activos y oro, y pasivos frente a NR
			\3 Fuentes de variación
				\4 Saldo de la cuenta financiera (CF)
				\4 Otras variaciones de los activos y pasivos
				\4[] Revalorizaciones (EV)
				\4[] Otros cambios de volumen (OCV)
				\4 $\text{PII}_{t+1} - \text{PII}_{t} = \text{CF}_t + \text{EV}_t + \text{OCV}_t$
			\3 Importancia creciente
				\4 En últimas décadas y ediciones de MBP
				\4 Permite:
				\4[] Análisis de sostenibilidad y vulnerabilidad
				\4[] Riesgo de tipo de cambio respecto a crisis
				\4[] Efecto de estructura sobre liquidez
				\4[] Implicaciones de la composición de la deuda
				\4[] Medición de tasas de rentabilidad
		\2 Estructura
			\3 Idea clave
				\4 Múltiples dimensiones posibles
				\4 Análisis de dimensiones especialmente útil
				\4[] $\to$ Análisis de vulnerabilidad
				\4[] $\to$ Exposición a tipo de cambio
			\3 Categoría funcional
				\4 Inversión directa
				\4 Inversión de cartera
				\4 Derivados financieros y OCE
				\4 Reservas
			\3 Instrumentos
				\4 Participaciones de capital
				\4 Participaciones en fondos de inversión
				\4 Instrumentos de deuda
				\4 Otros activos y pasivos financieros
			\3 Sector institucional de la parte residente
				\4 Habitual dividir en:
				\4[] $\to$ Economía general excepto Banco Central
				\4[] $\to$ Banco Central
				\4 Otras divisiones posibles
				\4[] Operaciones entre países zona euro\footnote{Ver Banco de España  (2015).}
				\4[] BC nacionales ya no tienen reservas frente a ZEuro
				\4[] Activos netos del BCN frente al Eurosistema
				\4[] $\to$ Sustituto en el balance
				\4[] Liquidadas vía TARGET
				\4[] Vencimiento
				\4[] Moneda
				\4[] $\to$ Nacional
				\4[] $\to$ Extranjera
				\4[] Estructura de las tasas de interés (deuda)
				\4[] $\to$ Variable
				\4[] $\to$ Fija
		\2 Categorías funcionales
			\3 Inversión directa
				\4 Inversión que implica control sobre gestión
				\4[] O grado significativo de influencia
				\4 Tipos de IDE
				\4[] Brownfield
				\4[] $\to$ Adquisición de planta ya existente
				\4[] $\to$ Planta existente continúa su actividad anterior
				\4[] Greenfield
				\4[] $\to$ Construcción de nueva planta
				\4[] $\to$ Planta antigua que cambia actividad tras inversión
				\4 Vertical vs horizontal
				\4[] Vertical
				\4[] $\to$ División del proceso productivo en fases
				\4[] $\to$ Plantas extranjeras llevan a cabo sólo una fase
				\4[] $\to$ Producción destinada a reexportación
				\4[] $\then$ Aprovechar ventajas competitivas en determinado segmento
				\4[] Horizontal
				\4[] $\to$ Replicación de proceso productivo en otro país
				\4[] $\to$ Evitar costes de transporte/aranceles
				\4 Incluye también:
				\4[] Reinversión de beneficios/pérdidas
				\4[] Inversión directa en especie
				\4[] Fusiones y adquisiciones
				\4[] Reestructuración de sociedades
				\4[] Dividendos de liquidación
				\4[] Retiradas de capital
			\3 Inversión de cartera
				\4 Inversión en títulos de deuda o acciones
				\4[] Diferente de inversión directa o reservas
				\4[] No implica gestión
				\4[] Necesario que sean títulos valor
				\4[] $\to$ Si no lo son, IDE u otra inversión
				\4[] $\to$ No se limita a títulos cotizados
			\3 Derivados financieros y OCAE\footnote{Opciones de Compra de Acciones por parte de Empleados.}
				\4 Derivados financieros distintos de reservas
				\4[] No relacionados con gestión de activos de reserva
				\4 No devengan ingresos primarios
				\4[] Montos devengados son revalorizaciones
				\4[] $\to$ Cuenta de otras variaciones de activos y pasivos
				\4 Opciones de compra de acciones para empleados
				\4[] Contrapartida de
				\4[] $\to$ remuneración de empleados
				\4[] $\to$ inversión directa
				\4[] Se registra diferencia entre:
				\4[] $\to$ Valor de mercado
				\4[] $\to$ Precio pagado por comprador de acción
			\3 Otra inversión
				\4 Categoría residual
				\4[] Activos no incluidos en el resto
				\4 Otras participaciones de capital
				\4 Moneda y depósitos
				\4 Préstamos
				\4[] Incluidos crédito del FMI
				\4 Reservas técnicas de seguros
				\4 Créditos y anticipos comerciales
				\4 Cuentas por cobrar/pagar
				\4 Asignaciones de DEG
				\4[] No la tenencia, que va a reservas
			\3 Reservas
				\4 Activos externos disponibles de inmediato
				\4[] Bajo control de autoridades monetarias
				\4[] Disponibles incondicionalmente
				\4[] Deben ser líquidos
				\4 Usos
				\4[] Satisfacer necesidad de financiación de BP
				\4[] Intervenir mercados para $\uparrow \downarrow$ tipo de cambio
				\4[] Otros fines conexos
				\4 Denominados en moneda extranjera
				\4 Deben estar realmente constituidos
				\4[] Excluidos activos potenciales
				\4 Activos de reserva
				\4[] Oro monetario
				\4[] DEG
				\4[] Posición en FMI
				\4[] Otros activos
				\4[] $\to$ Moneda y depósitos
				\4[] $\to$ Títulos
			\3 Valoración
				\4 Dependiente del activo
				\4 Precio de mercado si posible
		\2 Interpretación de la PII
			\3 Vulnerabilidades
				\4 Mismatch de vencimientos
				\4[] Elevado nivel de deuda a corto plazo
				\4[] Activos exterior a largo plazo
				\4 Mismatch de liquidez
				\4[] Activos exteriores son IDE ilíquida
				\4[] Pasivos son activos líquidos de cartera
				\4[] $\to$ Salida de capitales es factible
				\4 Mismatch de divisas
				\4[] Activos en divisa nacional
				\4[] Pasivos en divisa extranjera
				\4[] Acreedores dejan de confiar en acceso a FX
				\4[] $\to$ Captación de fondos imposible
				\4[] $\then$ Crisis financiera
				\4 Problemas de estructura financiera
				\4[] Uso excesivo de deuda frente a acciones
				\4[] Problemas de generación de caja
				\4[] $\then$ Impagos y quiebras
				\4[] $\then$ Crisis financieras agravadas
				\4 Dependencia excesiva
				\4[] Economía socia concreta financia casi totalmente
				\4[] Aumenta posibilidad de:
				\4[] $\to$ Contagio
				\4[] $\to$ Vulnerabilidad
			\3 Estática comparativa
				\4 Apreciación de la moneda nacional
				\4[] Aumenta el valor de:
				\4[] $\to$ Activos en moneda nacional
				\4[] $\to$ Pasivos en moneda nacional
				\4[] Decrece el valor de:
				\4[] $\to$ Activos en moneda extranjera
				\4[] $\to$ Pasivos en moneda extranjera
				\4[] Efecto conjunto
				\4[] $\to$ Depende de composición de activos y pasivos
				\4[] Países emisores de activos de reserva
				\4[] $\to$ EEUU (principal), UE
				\4[] $\to$ Emiten pasivos denominados en moneda nacional
				\4[] $\to$ Poseen activos extranjeros en moneda extranjera
				\4[] $\then$ Depreciación de moneda nacional aumenta PIIN
				\4 Aumento del valor de pasivos nacionales
				\4[] Acciones de empresas residentes
				\4[] Títulos de deuda emitidos por residentes
				\4[] $\then$ Reducción del valor neto de la economía
				\4[] Paradoja: aumento en la confianza en la economía
				\4[] $\to$ Deteriora PIIN
				\4[] $\then$ Necesario desagregar para valorar sostenibilidad vía PIIN
			\3 Sostenibilidad
				\4 ¿La senda de la PIIN es sostenible?
				\4[] En relación a PIIN sobre PIB total
				\4[] ¿Converge a una cantidad determinada?
				\4[] ¿Tiende a crecer al infinito?
				\4 Necesario análisis de sostenibilidad
				\4[] Para inversores
				\4[] Para gestores de deuda
				\4[] Si diverge, devolución será imposible
				\4[] $\then$ Crisis de deuda
				\4 Variación de la PII depende:
				\4[] -- Efectos valoración y volumen (OC)
				\4[] -- Cap./Nec. de financiación (CNF)
				\4[] -- Crecimiento del PIB
				\4 Capacidad de financiación depende:
				\4[] -- Factores autónomos
				\4[] $\quad$ $\to$ Bienes y servicios
				\4[] $\quad$ $\to$ Rentas secundarias
				\4[] $\quad$ $\to$ Cuenta de capital
				\4[] $\then$ $\text{CNF}^*$
				\4[] -- Factores endógenos a PII
				\4[] $\to$ Rentas primarias
				\4[] $\to$ Otros cambios (valor y volumen)
				\4[] $\Rightarrow$ RPI
				\4 Modelización de la convergencia
				\4[] $\Delta \text{PII} = \text{PII}_t - \text{PII}_{t-1} = \text{CNF}^*_t + \text{RPI}_t + OC_t$
				\4[] $\to$ $\text{RPI} =i_t^A \cdot A_{t-1} - i_t^P\cdot P_{t-1}$
				\4[] $\to$ $i_t^A = i_t^P \Rightarrow i \cdot A_{t-1} - i\cdot P_{t-1} = i \cdot \text{PII}_t$
				\4[] $\text{PII}_t - \text{PII}_{t-1} = \text{CNF}^*_t + i_t^A \cdot A_{t-1} - i_t^P\cdot P_{t-1} + OC_t $
				\4[] $\text{PII}_t = \text{CNF}^*_t + (1+i) \text{PII}_{t-1} + OC_t$
				\4[] $\to$ Dividiendo entre $y_t = y_{t-1} \cdot (1+g)$
				\4[] $\text{pii}_t = \text{cnf}^*_t + \frac{1+i}{1+g}\text{pii}_{t-1} + \text{oc}_t$
				\4[] $\to$ Estado estacionario: $\text{pii}_t = \text{pii}_{t+1} = \text{pii}_{t-1}$
				\4[] \fbox{$\text{cnf}^* = \frac{g-i}{1+g}\text{pii} + \text{oc}_t$}
				\4 Implicaciones
				\4[] Asumiendo:
				\4[] $\to$ Sin efectos de valoración y volumen ($\text{oc}_t=0$)
				\4[] $\to$ sin rentas secundarias ni transferencias de K
				\4[] $\then$ $\text{CNF}^*$ es balanza comercial
				\4[] \fbox{$g>i$}: crecimiento del PIBn mayor a interés
				\4[] $\text{CNF}^*$ puede ser < 0
				\4[] $\to$ Hasta igualar $\text{pii} \cdot \frac{g-i}{1+g}$
				\4[] $\to$ Deuda sostenible aun con déficit comercial
				\4[] $\to$ Interpretable como I más rentable que deuda
				\4[] \fbox{$g<i$}: crecimiento del PIB menor que interés
				\4[] $\text{CNF}^*$ <0 sólo sostenible si:
				\4[] $\to$ $\text{pii}_t > 0$
				\4[] $\then$ Déficit sólo sostenible si economía acreedora
				\4[] $\then$ Déficit sostenible hasta límite $\text{pii}\cdot \frac{g-i}{1+g}$
				\4[] $\to$ Interpretable como I menos rentable que deuda
				\4[$\then$] Deseable $\text{CNF}^*$ > umbral para evitar ajustes
			\3 Desequilibrios globales
				\4 PII muy negativas pueden implicar riesgo sistémico
				\4[] Crisis soberanas de la zona euro
				\4[] Crisis de los 90
				\4 Periodos post-crisis
				\4[] Reajustes de CNF
				\4[] $\to$ Ajuste brusco de balanza de pagos
				\4[] $\to$ Financiación internacional + programas de ajuste
	\1 \marcar{Crisis de balanza de pagos}\footnote{Ver \href{https://voxeu.org/content/sudden-stops-primer-balance-payments-crises}{Cecchetti y Schoenholtz en VOXEU (2018)}.}
		\2 Idea clave
			\3 Contexto
				\4 Cita de Dornbusch
				\4[] ``It is not speed that kills, but the sudden stop''
				\4[] Viejo dicho de mercados financieros
				\4 Ajustes bruscos son costosos
				\4 A lo largo de la exposición
				\4[] Componentes de la balanza de pagos
				\4[] Vía de ajuste para igualar sumas de saldos CC, CK, CF
				\4 Ahorro e inversión
				\4[] Exceso de ahorro sobre inversión
				\4[] $\to$ Superávit por cuenta corriente
				\4[] $\to$ Salida de capital
				\4[] $\then$ Necesario encontrar activos de inversión
				\4[] $\then$ Activos aumentan más que pasivos
				\4[] Exceso de inversión sobre ahorro
				\4[] $\to$ Déficit por cuenta corriente
				\4[] $\to$ Entrada de capital
				\4[] $\then$ Necesario encontrar contrapartes para pasivo nacional
				\4[] $\then$ Pasivos aumentan más que activos
				\4 Problemas de balanza de pagos
				\4[] Desequilibrios graves en cuenta corriente
				\4[] Coste de financiación elevado
				\4[] Cambios bruscos en flujos de capital
				\4[] $\to$ De entrantes a salientes rápidamentemente
				\4[] Reacciones excesivas a shocks de información
				\4[] $\to$ Pánicos de inversores
				\4[] $\to$ Wake-up calls
				\4[] $\to$ Contagio
				\4[] $\to$ ...
			\3 Objetivos
				\4 ¿Qué sucede cuando no es posible financiar déficit?
				\4 ¿Cómo se ajusta la balanza de pagos?
				\4 ¿Qué implicaciones de política económica?
			\3 Resultados
				\4 Ajustes bruscos son mucho más costosos que graduales
				\4 Programas de ajuste del FMI para suavizar ajuste
				\4 Reducción de factores de riesgo importante
				\4 Liberalizaciones cuenta financiera deben evitar $\uparrow$ riesgos
		\2 Formulación
			\3 Identidades del ahorro, la inversión y la entrada de capital
				\4 Renta Nacional Bruta Disponible
				\4[] $\text{RNBD} = C+G+I+\text{NX}+ \text{RP}+\text{RS}$
				\4 Ahorro Nacional
				\4[] $S= \text{RNBD} - C -G = I + \text{NX} + \text{RP}+ \text{RS}$
				\4 Exceso de ahorro nacional, CCorriente y CFinanciera
				\4[] $S-I+\text{CK} = \underbrace{\text{NX}+\text{RP} + \text{RS}}_{\text{CC}} +\text{CK} = \text{VNA} - \text{VNP}$
				\4 Ahorro insuficiente para cubrir inversión
				\4[] Necesario aumentar pasivos netos
				\4[] $\to$ ¿Quién los acepta?
				\4[] $\to$ ¿Quién provee el capital?
				\4[] $\to$ ¿A qué coste?
				\4[] $\to$ ¿Es posible en todas circunstancias encontrar financiación?
			\3 Sudden stops y reversiones de flujos de capital
				\4 Ocurren relativamente frecuentemente
				\4 Especialmente en países en desarrollo/emergentes
				\4 Persisten al menos un año, generalmente
				\4 Sudden stop y flow reversal al tiempo
				\4 Inducen depreciación del tipo de de cambio
				\4[] No quedan otras herramientas de ajuste disponibles
				\4 Inducen caídas fuertes del PIB via $\downarrow$ I
			\3 Factores de riesgo de sudden stops
				\4 Libre movimiento de capital
				\4[] Sin coste para
				\4 Préstamos de corto plazo
				\4[] Prestamistas pueden inducir sudden-stop
				\4[] $\to$ Simplemente evitando renovación de préstamos
				\4 Endeudamiento en moneda extranjera
				\4[] Banco central
				\4[] $\to$ No puede proveer liquidez
				\4[] $\to$ No puede monetizar deuda
				\4 Pequeño sector exportador
				\4[] Si flujos de capital se revierten
				\4[] $\to$ Necesario aumentar exportaciones
				\4[] Si sector exportador es pequeño
				\4[] $\to$ Necesario reorganizar producción
				\4[] $\then$ Muy costoso
				\4 Aumento de percepciones globales del riesgo
				\4[] Capital se desplaza hacia activos percibidos como seguros
				\4 TCN fijo + libre movimiento de K
				\4[] Vulnerabilidad clásica
				\4[] Incentiva ataques especulativos de primera generación
				\4 Stock de reservas pequeño
				\4[] Asiáticos aprenden lección tras crisis de 90s
		\2 Implicaciones
			\3 Flujos de capital pueden ser desestabilizantes
				\4 Pueden alimentar inversión excesiva
				\4 Presiones especulativas sobre tipo de cambio fijo
			\3 Liberalización de CF puede tener inconvenientes
				\4 Exceso de inversión
				\4 Apreciación del tipo de cambio
				\4 Sudden stops y reversiones del flujo de K
				\4 Crisis financieras
			\3 Sistema financiero doméstico es factor clave
				\4 Relaciones con proveedores de capital extranjeros
				\4 Estructura de incentivos de bancos nacionales
				\4[] Determina dependencia de flujos de capital extranjeros
				\4[] $\to$
			\3 Uniones monetarias requieren mecanismos emergencia
				\4 Target 2 en eurozona
				\4[] Ante fuga de capitales
				\4[] $\to$ De bancos privados en un país en crisis
				\4[] $\to$ Hacia países centrales
				\4[] BCNs incurren saldos acreedores con BCE
				\4[] $\to$ BCE provee liquidez automáticamente
		\2 Valoraciones
			\3 Suceso recurrente
				\4 Crisis latinoamericanas de los 80
				\4 Crisis asiática de los 90
				\4 Crisis de la eurozona de 2010s
			\3 Papel clave del FMI
				\4 Programas de asistencia financiera
				\4[]
				\4 Programas de asistencia concesional
				\4[] Especial importancia en crisis actual
				\4 Programas de reforma
				\4[]
				\4 Fracaso en los 90
				\4[] Propone aumento fuerte de tipos de interés
				\4[] $\to$ Incentivar entrada de capital
				\4[] Acabo provocando recesión
				\4[] $\to$ Agudizó crisis
	\1 \marcar{Conclusión}
		\2 Recapitulación
			\3 Conceptos y medición
			\3 Balanza de pagos
			\3 Posición de inversión internacional
		\2 Idea final
			\3 Concepto contable
				\4 Subjetividad en la medición
				\4[] Estimaciones necesarias para ciertas transacciones
				\4[] Criterios contables sujetos a discreccionalidad
				\4 Papel del FMI, instituciones
				\4[] Homogeneizar criterios
				\4[] Proveer marco estable de presentación
			\3 Aplicaciones de la balanza de pagos y PII
				\4 Modelizar crecimiento
				\4 Prever crisis
				\4[] Evitar ajustes costosos
				\4 Entender desequilibrios externos
				\4 Imponer disciplina de mercado
			\3 Importancia creciente de la PII
				\4 Teorías macro recientes
				\4[] Enfoque intertemporal de la balanza de pagos
				\4[] NMC y NEK
				\4[] $\to$ Importancia de la decisión intertemporal
				\4 PII como vector de la decisión intertemporal
				\4[] $\Rightarrow$ Aplicacion empírica de la teoría
\end{esquemal}







































\conceptos

\concepto{Rentas, servicios e ingresos de la inversión}
<<1 El hecho de permitir que otra entidad utilice activos producidos da origen a un servicio (véase el párrafo 10.153). En cambio, el hecho de permitir que otra entidad utilice activos no financieros no producidos da origen a una renta (párrafo 11.86) y el hecho de permitir que otra entidad utilice activos financieros da origen a un ingreso de la inversión, como intereses, dividendos y utilidades no distribuidas.>> (Sexto Manual FMI - pág. 9))

\preguntas


\seccion{Test 2018}

\textbf{29.} En relación a la elaboración de la Balanza de Pagos (de acuerdo con la nomenclatura de la sexta edición del Manual de Balanza de Pagos y Posición de Inversión Internacional), señale la respuesta \textbf{\underline{CORRECTA}}:

\begin{itemize}
	\item[a] Las operaciones de balanza de pagos correspondientes a remesas de emigrantes se anotan en la cuenta de rentas primarias.
	\item[b] En el caso de los países de la UE, las transferencias de capital procedentes de los fondos estructurales FEDER se registran en la cuenta de capital.
	\item[c] si las empresas que trasportan las mercancías importadas pertenecen al mismo país que elabora la balanza, el valor de dicho transporte debe anotarse en la cuenta de servicios.
	\item[d] Las respuestas a) y b) son correctas.
\end{itemize}


\seccion{Test 2016}

\textbf{28}. La Balanza Financiera presenta déficit debido a una fuerte salida neta de capital en el país X. Suponiendo nula la variación de reservas internacionales, a partir de las equivalencias entre los saldos de las Cuentas de la Balanza de Pagos y la Capacidad o Necesidad de financiación del país, indique cuál de las afirmaciones siguientes es correcta:

\begin{enumerate}
	\item[a] Hay déficit por cuenta corriente.
	\item[b] La Variación Neta de Pasivos > Variación Neta de Activos
	\item[c] El Ahorro Nacional es mayor que la Inversión Nacional
	\item[d] Los pagos superan a los ingresos en la Balanza por Cuenta Corriente y de Capital.
\end{enumerate}

\seccion{Test 2009}

\textbf{29}. Dado que en la balanza de pagos, los saldos de la balanza por cuenta corriente, de la cuenta financiera y de la cuenta de capital deben sumar cero; entonces, la variación total de la posición acreedora neta de un país debe igual a:

\begin{enumerate}
    \item[a] El saldo de la cuenta de capital precedido de signo positivo.
    \item[b] El saldo de la cuenta de capital precedido de signo negativo.
    \item[c] El saldo de la cuenta financiera precedido de signo positivo.
    \item[d] El saldo de la cuenta financiera precedido de signo negativo.
\end{enumerate}

\textbf{31}. En el análisis de la balanza de pagos de un país perteneciente a la eurozona, un saldo positivo de la Rúbrica de Activos netos frente al eurosistema significa:

\begin{enumerate}
    \item[a] Un aumento de la posición deudora neta del país frente al exterior.
    \item[b] Un aumento de la posición deudora de su Banco Central frente al eurosistema.
    \item[c] Que el país tiene necesidad de financiación
    \item[d] Que el país tiene superávit por cuenta corriente.
\end{enumerate}

\notas

\textbf{2018}: \textbf{29}. B

\textbf{2016}: \textbf{28}. C

\textbf{2009}: \textbf{29}. D. Hay que tener en cuenta que la pregunta puede estar referida al 5º manual, con una convención de signo diferente. En el 5º manual, la identidad fundamental es: $\text{CC} + \text{CK} = - \text{CF} = I - P = - (\varDelta P - \varDelta A)$. Puede tener que ver con el diferente convenio de signos entre 5º y 6º manual. Si \comillas{rúbrica de activos netos} se calcula restando variación de pasivos a variación de activos, un saldo positivo efectivamente indica un aumento de la posición deudora con el exterior. Se puede comprobar comparando el boletín estadístico de 2014 (sistema anterior) y cualquier boletín posterior, en el apartado 2.6 del documento (Balanza de pagos y posición inversión internacional). \textbf{30}. B.

\bibliografia

Mirar en Palgrave:
\begin{itemize}
	\item
\end{itemize}

Banco de España. \textit{Boletines Estadísticos}. http://www.bde.es/bde/es/secciones/informes/boletines/Boletin\_Estadist/

Banco de España. \textit{Los activos del Banco de España frente al Eurosistema y el tratamiento de los billetes en euros en la Balanza de Pagos y la Posición de Inversión Internacional}. (2015) -- En carpeta del tema

Banco de España. (2019) \textit{The Balance of Payments and the International Investment Position: Methodological Note} -- En carpeta del tema

Bank of International Settlements. \textit{Interpreting TARGET 2 Balances}

Cecchetti, S. Schoenholt, K. (2018) \textit{Sudden stops: A primer on balance-of-payments crises} Voxeu.org \href{https://voxeu.org/content/sudden-stops-primer-balance-payments-crises}{Enlace}

Eichengreen, B.; Gupta, P. (2016) \textit{Managing Sudden Stops} World Bank Group. Policy Research Working Paper -- En carpeta del tema


Gandolfo, G. \textit{International Finance and Open-Economy Macroeconomics}. 2016. Ch. 5: The Balance of Payments.

FMI. \text{Sexto Manual de la Balanza de Pagos}. 2009

FMI. \textit{FAQs on conversion from BPM5 to BPM6}

Pastor Escribano, A. \textit{CECO. Temario 2016}. Tema 12.

Reinhart, C.; Calvo, G. (2000) \textit{When Capital Inflows Come to a Sudden Stop: Consequences and Policy Option} Reforming the International Monetary and Financial System: IMF -- En carpeta del tema 



\end{document}
