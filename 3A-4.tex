\documentclass{nuevotema}

\tema{3A-4}
\titulo{El pensamiento económico de Keynes. Formalización y comparación con el modelo neoclásico. La síntesis neoclásica.}

\begin{document}

\ideaclave

La macroeconomía se define habitualmente como el estudio de fenómenos económicos de gran escala en los que interaccionan cantidades muy elevadas de agentes. En este contexto, la enorme complejidad de las interacciones y la emergencia de fenómenos que no aparecen a nivel macroeconómico hacen necesaria la utilización de de variables agregadas como objeto final del análisis tales como renta nacional, nivel de precios, tasas de empleo o tipos medios de interés. La obra de John Maynard Keynes supuso la formulación de un nuevo paradigma de análisis que ha sido calificado como una revolución científica por su capacidad para transformar el conjunto de herramientas teóricas utilizadas para entender el funcionamiento y predecir la evolución de la macroeconomía. Además, del valor que tiene en sí misma, la obra de Keynes ha dado lugar a una amplísima literatura y numerosos contraargumentos, síntesis y reformulaciones. La llamada \textit{síntesis neoclásica} se convirtió en la expresión predominante de las aportaciones de Keynes y constituyó un conjunto de modelos y teorías accesorias que gravitaban en torno al modelo IS-LM y permitían caracterizar una interpretación del modelo de Keynes pero también el modelo neoclásico y otras teorías variando ligeramente los supuestos. La curva de Phillips supuso un complemento empírico a la síntesis neoclásica y determinó el debate posterior acerca de la efectividad de la política económica. Por ello, el primer \textbf{objeto} de este tema consiste en exponer la figura de Keynes como economista, sus aportaciones a la ciencia económica y las diferencias con el modelo neoclásico. El segundo objeto de la exposición es examinar la influencia que las ideas de Keynes han tenido sobre el pensamiento económico posterior y más específicamente, presentar las características principales de la síntesis neoclásica. La exposición se divide así en dos partes: una primera centrada en Keynes y su obra, y una segunda en la que se explica el modelo IS-LM, sus extensiones e implicaciones y la Curva de Phillips.

John Maynard \marcar{Keynes} nace en 1883 y muere en 1946. Las obras de Malthus, Fisher, Marshall y Pigou influyen fuertemente toda su pensamiento. En el contexto histórico de su obra, son relevantes también las aportaciones y correcciones de Harrod y en general de sus contemporáneos en el departamento de economía de Cambridge. El trabajo de Simon Kuznets sobre el ingreso nacional contribuyó también al clima intelectual en el que Keynes alumbró su teoría. La influencia de Keynes se extiende a la práctica totalidad del pensamiento económico posterior. En el marco de la síntesis neoclásica que trataremos posteriormente, Hicks, Hansen, Samuelson, Solow, Tobin y Patinkin son los nombres principales de autores dentro de esta corriente que dedicaron gran parte de sus obra a discutir las implicaciones de las ideas de Keynes. Los neokeynesianos del desequilibrio llevaron a cabo un programa de investigación que trataba de reconstituir las implicaciones de la obra de Keynes en un marco diferente al de la síntesis neoclásica, al que consideraban alejado del verdadero mensaje de Keynes. Entre ellos se cuentan nombres como Benassy, Clower, Leijonhufvud o Malinvaud. El monetarismo liderado por Friedman es gran medida una reacción a las ideas de Keynes y a pesar de criticarlas en aspectos centrales, se basa en ellas y en su metodología para articular sus propuestas de política económica. La Nueva Economía Keynesiana predominante en la modelización macroeconómica actual trata de fundamentar microeconómicamente algunos postulados keynesianos tales como las rigideces reales y nominales y su influencia sobre el ciclo económico. Las obras principales de Keynes son cuatro: Las Consecuencias Económicas de la Paz, Tratado sobre la Reforma Monetaria, Tratado sobre el Dinero y su obra principal, La Teoría General del Empleo, el Interés y el Dinero publicada en 1936. El contexto económico en el que Keynes publica sus trabajos es relevante para entenderla. Tras la Primera Guerra Mundial, la economía mundial experimenta una expansión generalizada al tiempo que Alemania paga reparaciones de guerra a los vencedores de la guerra. El Reino Unido restablece el patrón oro e incurre en un creciente déficit comercial fruto de una deflación previa de precios y salarios insuficiente para restaurar su competitividad exterior. El crack bursátil de 1929 desencadena una contracción generalizada del crédito que resulta a su vez en devaluaciones competitivas y un aumento espectacular del desempleo. En 1931, el Reino Unido pone fin a la convertibilidad de la libra esterlina en oro y Roosevelt es elegido en Estados Unidos. Se pone en marcha el New Deal y comienza el rearme en en Europa previo a la Segunda Guerra Mundial, estimulando la demanda de inversión y sacando progresivamente a las economías de la Gran Depresión. El pensamiento económico está dominado en esta etapa por el llamado enfoque neoclásico predominante en círculos académicos y entre policy makers. Se caracteriza a grandes rasgos por postular que las economías tienden de forma natural a la plena utilización de sus capacidades productivas gracias a una serie de mecanismos de ajuste, entendidas las desviaciones de esa plena utilización como fenómenos de corto plazo tendentes a corregirse de forma automática de forma acorde con la acepción general de Ley de Say.

La metodología aplicada por Keynes se caracteriza por ser fundamentalmente verbal. En la Teoría General, el uso de ecuaciones y gráficas es muy escaso en comparación con la macroeconomía actual o con numerosas obras contemporáneas o anteriores y contrasta con la notable formación matemática de Keynes. Para exponer sus planteamientos, crea la figura de la ``economía clásica'' en un sentido diferente al utilizado por Marx y basado en la obra de Pigou y el llamado \textit{Treasure View}. Asocia a ese paradigma ``clásico'' la teoría cuantitativa del dinero, la tendencia natural a la estabilidad y la utilización plena de recursos, y las desviaciones respecto al pleno empleo como resultado de procesos de ajuste incompletos. La modelización keynesiana de la macroeconomía se caracteriza por el tratamiento directo de variables agregadas y una microfundamentación superficial y en gran medida ad-hoc de la forma funcional implícita que Keynes asigna a las diferentes funciones de demanda. El modelo planteado en la Teoría General puede entenderse como un modelo de equilibrio general de la economía. La adscripción marshalliana o walrasiana del modelo suscita sin embargo fuertes debates, y es dudoso que Keynes conociese en profundidad o apenas la obra de Walras. La postura de Keynes respecto a la economía cuantitativa es relativamente escéptica. Critica especialmente los métodos cuantitativos aplicados a la macroeconomía, a pesar de mencionar en ocasiones el trabajo de Kuznets. Especialmente contrario a los métodos cuantitativos aplicados por Tinbergen, que critica por sus problemas para extraer conclusiones acerca de la causalidad.

La Teoría General del Empleo, el Interés y el Dinero es la obra fundamental de Keynes y por ello es necesario detenerse en especial en ella. El libro se divide en cinco secciones: una introducción, una sección sobre definiciones e ideas, un análisis de los factores que determinan la inversión, una sección sobre los salarios nominales y los precios, y una sección adicional a modo de apéndice con una serie de conclusiones en forma de notas breves sobre la teoría general expuesta. Para sintetizar la obra, cabe resumir sus ideas clave en una serie de conceptos diferenciados que guardan cierta consonancia con las secciones del libro. La \underline{propensión al consumo} caracteriza la relación entre una renta dada y la cantidad de esa renta que se destina al consumo. Keynes no explicita una función de consumo pero postula que ante aumentos de la renta, el consumo adicional demandado es una fracción del aumento. Es decir, que la proporción marginal al consumo es inferior a la unidad. Aunque justifica este hecho en múltiples factores, los resume como una ``ley psicológica fundamental'' generalmente estable y empíricamente corroborada. Un corolario de esta demanda de consumo es que parte de la renta se ahorra. Si la demanda de inversión no es tan grande como ese ahorro, aparecerá un exceso de oferta de bienes y servicios. El \underline{multiplicador del gasto y su relación con la inversión} es otro corolario de la demanda de consumo postulada. Los aumentos de la renta provocan aumentos del consumo que a su vez provocan aumentos de la renta, dando lugar a un proceso dinámico estable dada la propensión marginal al consumo postulada inferior a la unidad. Teniendo en cuenta ambos factores, Keynes llega a la conclusión de que dados los límites a la demanda de consumo, el aumento de la inversión pública autónoma (en el sentido de que no dependa del interés) puede llevar a la economía al pleno empleo a través de su efecto directo y el efecto a través del multiplicador. El \underline{principio de la demanda efectiva} de Keynes afirma que no toda producción o renta se convierte en gasto \textit{ex-ante}, en base a los supuestos respecto a la demanda de consumo y la inversión. En contraposición a los clásicos, a los que atribuye la creencia en la Ley de Say entendida como igualdad ex-ante entre gasto y renta, Keynes plantea la posibilidad de excesos de oferta en el mercado de bienes con carácter persistente y entendibles como equilibrios estables. El famoso \textit{aspa keynesiana} es la expresión gráfica de este concepto y representa un modelo keynesiano de un sólo sector en el que la economía no se ajusta automáticamente hasta utilizar plenamente sus capacidades productivas. El análisis de Keynes se extiende también al \underline{sector monetario y financiero}. Atribuye a los clásicos la creencia en la determinación del tipo de interés como resultado del equilibrio entre oferta y demanda en el mercado de fondos prestables, es decir, como resultado del equilibrio entre ahorro e inversión. Por su parte, Keynes afirma que el tipo de interés se determina como resultado del equilibrio del mercado monetario y el mercado de bonos. La demanda de dinero que postula Keynes es el elemento central. Los clásicos entendían la demanda de dinero como resultado de la demanda de un medio de pago con el que reducir los costes de transacción. La demanda de dinero se caracteriza por medio de la ecuación de Cambridge o su equivalente ecuación cuantitativa del dinero. Keynes añade la demanda de dinero por razones de especulación: los agentes prefieren mantener mayores saldos monetarios cuanto menor sea el tipo de interés. Un tipo de interés menor implica un menor coste de oportunidad por los retornos de los bonos e implica una mayor posibilidad de subida de tipos en el futuro. En este contexto, y asumiendo una oferta monetaria exógena, el tipo de interés se ajusta para equilibrar la demanda y la oferta de dinero sin atender exclusivamente al equilibrio entre ahorro e inversión y haciendo posibles tasas de interés que no ajusten ahorro e inversión para mantener la plena utilización de la capacidad productiva. La política monetaria se convierte así en una herramienta tanto menos efectiva cuanto mayor sea la demanda de dinero. Dado que la demanda de dinero será infinitamente elástica cuando el tipo de interés es lo suficientemente bajo, existirá un grado de liquidez en el sector financiero en el que la política monetaria será totalmente inútil. Se trata del concepto de trampa de liquidez tan relevante en nuestros días. La \underline{demanda de inversión} postulada por Keynes completa el modelo. Según Keynes, las empresas llevan a cabo los proyectos cuya eficiencia marginal del capital supera el tipo de interés que pagan por el capital. La eficiencia marginal del capital no es sino la tasa de interés que iguala coste del proyecto y valor de los flujos descontados. Así, la demanda de inversión depende efectivamente del tipo de interés. Pero los flujos descontados que las empresas esperan recibir son, según Keynes, fuertemente volátiles y modulados por los ``animal spirits''. Por ello la demanda de inversión es muy volátil y el efecto de variaciones del tipo de interés sobre la demanda de inversión es reducido. El sector público puede efectivamente contribuir a estabilizar esa demanda de inversión, desencadenando un ciclo virtuoso de feedback entre el estímulo a la inversión, la renta y el consumo. Por último, Keynes examina el rol de los precios y los salarios como mecanismos potenciales de ajuste hacia el pleno empleo. Empieza por asumir rigidez de salarios nominales a la baja y ajuste relativamente lento de los precios, lo que redunda en una efectividad muy reducida de los variaciones en en variables nominales como mecanismo de ajuste. Además, aunque los mecanismos de ajuste vía precios existiesen realmente, son muy lentos y provocarían desempleo, pobreza y potencial no aprovechado durante largos periodos de tiempo. Además, Keynes plantea en el capítulo 19 del libro una serie de mecanismos que contrarrestan el ajuste hacia el pleno empleo para el caso hipotético de que los salarios fuesen perfectamente flexibles. En general, su origen estaría en la depresión de la demanda que resultaría de reducciones de salarios y por tanto, de la renta disponible de los hogares. Las \textit{implicaciones globales} de la obra de Keynes son básicamente tres: La insuficiencia de la demanda como situación habitual y estable que causa desempleo, el potencial de la intervención pública para aumentar el empleo por vía de la inversión pública, y la relativa inefectividad de la política monetaria como instrumento de estabilización. 

La obra de Keynes no se caracteriza por explicitar el modelo formal subyacente. Por ello, su aparición dio lugar a una serie de interpretaciones que formulaban el modelo keynesiano con métodos matemáticos relativamente simples. Este conjunto de interpretaciones más o menos coherentes y similares conformaron la llamada \marcar{síntesis neoclásica}. Se trata de un paradigma macroeconómico integrado que reconcilia el modelo keynesiano y el modelo clásico. La síntesis neoclásica dominó el estudio de la macroeconomía y las políticas económicas aplicadas. Más allá del modelo IS-LM que forma el núcleo de la teoría, fue central la Curva de Phillips en su fase tardía, así como los modelos macroeconométricos. Las ideas centrales de la síntesis neoclásica son la confianza en la estabilización del ciclo y el empleo mediante la intervención pública, la efectividad del \textit{fine-tuning}, la asunción del ajuste lento de precios y la rigidez nominal de los salarios y el estudio diferenciado del corto y el largo plazo aplicando diferentes conjuntos de supuestos aunque un marco general similar.

El \textit{modelo IS-LM} es en esencia un sistema de dos ecuaciones y tres incógnitas que representa el equilibrio en el mercado de bienes y de dinero de forma explícita, y en el de bonos de forma implícita. Su capacidad para fundamentar de forma simple diferentes implicaciones de política económica introduciendo variaciones en determinados supuestos clave le convirtió en el eje central de la síntesis neoclásica. Aparece en su primera versión en 1937 de la mano de Hicks y posteriormente Hansen, Modigliani y Patinkin --entre otros- lo perfeccionan y mejoran su presentación. Las dos ecuaciones del modelo representan respectivamente el equilibrio en el mercado de bienes y en el de dinero. La primera ecuación se denomina IS y resulta de igualar la renta y la demanda agregada en función de renta y tipo de interés. La ecuación LM representa el equilibrio en el mercado monetario resultante de igualar el valor real de oferta y demanda de dinero en función de renta e interés. Dado el mayor número de incógnitas que de ecuaciones, una solución determinada requiere fijar el valor de una de las incógnitas. El modelo IS-LM permite representar el modelo neoclásico o de largo plazo fijando un valor exógeno de la renta y asumiendo unos precios plenamente flexibles. Además, es habitual asumir una demanda de inversión sensible al tipo de interés y opcionalmente, la ecuación cuantitativa del dinero como demanda de saldos monetarios. En este contexto, las variables endógenas son precio y tipos de interés. Cuando la política fiscal se utiliza para aumentar la demanda agregada, se produce un exceso de demanda en el mercado de bienes. Este desequilibrio se corrige por vía de un aumento de los precios que reduce la oferta de saldos monetarios reales, que a su vez aumenta el tipo de interés de equilibrio y reduce la demanda. Las políticas monetarias expansivas tampoco son efectivas en este contexto de supuestos neoclásicos. La curva LM se desplazará hacia la derecha implicando un nuevo exceso de demanda en el mercado de bienes. Los precios aumentarán como resultado de este exceso de demanda, y la curva LM volverá a su posición inicial. El nivel de precios habrá subido y el tipo de interés se mantendrá igual al inicial. La introducción de supuestos keynesianos arroja resultados muy diferentes. En este caso, el precio es la variable cuyo valor se fija exógenamente, y la renta y el interés las variables endógenas. Cuando se aplica un estímulo de política fiscal en este contexto aumentando la demanda, la curva IS se desplaza hacia la derecha. El nuevo equilibrio implica un desplazamiento a lo largo de la curva LM con mayor renta y mayor tipo de interés. En situaciones de trampa de liquidez, todo la expansión fiscal se convertirá en aumento de la renta por la ausencia de crowding-out por subida del tipo de interés. La representación en un modelo IS-LM con supuestos keynesianos de un estímulo monetario arroja un resultado menos favorable. En condiciones normales la expansión monetaria desplaza la curva LM a la derecha e induce un nuevo equilibrio sobre la misma curva IS con más renta y menos interés. Sin embargo, cuanto más cerca se encuentre la economía de la trampa de liquidez, menos efectivo será el estímulo monetario. En el caso de encontrarse en la trampa de liquidez. Todo aumento de liquidez será absorbido por la demanda de dinero de los agentes de tal manera que el efecto estimulante sobre la renta y el interés será nulo. Las \textit{extensiones del modelo IS-LM} han tenido un impacto en ocasiones mayor que el propio modelo original. El modelo de oferta agregada-demanda agregada introduce un mercado adicional: el mercado de trabajo. Si la cantidad de trabajo utilizada depende del equilibrio en el mercado de trabajo, y éste depende del nivel de precios que determina el salario real, es posible establecer una relación entre nivel de precios y producción que convierte al modelo en un sistema de tres ecuaciones y tres incógnitas. En este contexto, es posible caracterizar todo tipo de modelos de la macroeconomía. Por su claridad expositiva y cualidades didácticas, este modelo ha servido y sirve como vehículo de enseñanza de la economía y de presentación de propuestas de política económica. El modelo de Mundell-Fleming incorpora al modelo IS-LM un modelo de determinación de la oferta monetaria en función de las transacciones exteriores y permite analizar los efectos de diferentes regímenes de tipo de cambio, restricciones de la cuenta financiera y políticas monetarias.

La \textit{curva de Phillips} es una relación empírica entre inflación y desempleo cuya primera estimación matemática explícita se debe a Phillips (1958). El autor encontró una relación entre inflación salarial y desempleo presente a lo largo de diferentes periodos temporales en el Reino Unido y estimó una regresión lineal que ajustaba esa relación. Posteriormente Lipsey, Phelps y Friedman trataron de fundamentar esta relación con diferentes modelos alternativos, y la inflación salarial tendió a extrapolarse a inflación general. En un artículo de gran impacto publicado en 1960, Samuelson y Solow plantean que en el corto plazo los gobiernos pueden elegir entre un menú de combinaciones entre inflación y desempleo. Aunque los autores no afirman que esa relación sea sostenible en el largo plazo y de hecho introducen la idea de histéresis en el mercado de trabajo, la idea de menú de políticas se convierte en el principal objeto de debate en los últimos años 60. En el contexto de la síntesis neoclásica en el que se enmarca el artículo de Solow y Samuelson, la curva de Phillips se plantea como una justificación empírica de la intervención pública y el fine-tuning. Con el paso del tiempo, la curva de Phillips evolucionó en una mera expresión de una relación entre dos variables que ni tiene por qué ser estable, ni tiene por qué ser susceptible de ajuste discrecional por parte de las autoridades. Los trabajos de Friedman y otros autores posteriores contribuyeron a acabar con la idea de la Curva de Phillips como un menú de política económica a disposición del gobierno.

La obra de Keynes y el keynesianismo posterior personificado en la síntesis neoclásica supuso una auténtica revolución del pensamiento económico. En el ámbito de la política económica, se tradujo en una confianza generalizada en la posibilidad de reducir las fluctuaciones cíclicas que provocaban grandes pérdidas de bienestar. La posibilidad de entender la macroeconomía como un sistema que no tiende de forma rápida hacia la plena utilización de su potencial en equilibrio transformó el debate en las décadas posteriores e inspiró nuevas teorías y modelos que ayudaron a convertir la macroeconomía en una disciplina capaz de aportar mejores y más acertadas recomendaciones de política económica.

\seccion{Preguntas clave}
\begin{itemize}
	\item ¿Quién fue Keynes?
	\item ¿Qué aportaciones realizó?
	\item ¿Qué diferenciaba el pensamiento de Keynes del modelo clásico?
	\item ¿Qué implicaciones ha tenido la obra de Keynes?
	\item ¿Qué formalizaciones del pensamiento de Keynes se han consolidado?
	\item ¿Qué es la síntesis neoclásica?
\end{itemize}

\esquemacorto

\begin{esquema}[enumerate]
	\1[] \marcar{Introducción}
		\2 Contextualización
			\3 Evolución de la ciencia económica
			\3 Historia del pensamiento económico
			\3 Macroeconomía
			\3 Keynesianismo
		\2 Objeto
			\3 ¿Quién fue Keynes?
			\3 ¿Qué aportó a la ciencia económica?
			\3 ¿En qué se diferencia su modelo del neoclásico?
			\3 ¿Qué influencia posterior han tenido Keynes?
			\3 ¿Qué es la síntesis neoclásica?
		\2 Estructura
			\3 Keynes
			\3 Síntesis neoclásica
	\1 \marcar{Keynes}
		\2 Trayectoria
			\3 Vida
			\3 Influenciado por
			\3 Influenció a
			\3 Obras
		\2 Contexto
			\3 Económico
			\3 Teoría económica
		\2 Metodología
			\3 Método de exposición
			\3 Modelización
			\3 Econometría y economía cuantitativa
		\2 Ideas clave de la Teoría General
			\3 La propensión al consumo
			\3 El multiplicador del gasto
			\3 Principio de la demanda efectiva
			\3 Oferta monetaria, preferencia por liquidez e interés
			\3 Inversión y el tipo de interés
			\3 Precios y salarios
		\2 Implicaciones globales
			\3 Mercado de trabajo
			\3 Insuficiencias de demanda agregada
			\3 Intervención pública
			\3 Política monetaria puede no ser efectiva
	\1 \marcar{La síntesis neoclásica}
		\2 Idea clave
			\3 Contexto
			\3 Objetivos
			\3 Resultados
		\2 IS-LM/AD-AS con supuestos clásicos
			\3 Idea clave
			\3 Oferta: AS
			\3 Demanda: IS-LM $\to$ AD
			\3 Representación gráfica
			\3 Política fiscal
			\3 Política monetaria
			\3 Valoración
		\2 IS-LM/AD-AS con supuestos keynesianos
			\3 Idea clave
			\3 Oferta
			\3 Demanda
			\3 Política fiscal con supuestos keynesianos
			\3 Política monetaria con supuestos keynesianos
			\3 Valoración
			\3 Modelo de Mundell-Fleming
			\3 IS-LM Dinámico
		\2 Curva de Phillips
			\3 Idea clave
			\3 Phillips (1958)
			\3 Lipsey (1960)
			\3 Samuelson y Solow (1960)
			\3 Valoración
		\2 Otros programas de investigación
			\3 Demanda de dinero
			\3 Demanda de consumo
			\3 Demanda de inversión
			\3 Modelos estructurales de la macroeconomía
		\2 Implicaciones
			\3 Influencia política económica
			\3 Influencia teórica
	\1[] \marcar{Conclusión}
		\2 Recapitulación
			\3 Keynes
			\3 Síntesis neoclásica
		\2 Idea final
			\3 Robert Solow sobre modelos macro y economistas
			\3 Política económica
			\3 Confianza en gestión del ciclo
			\3 Impacto global del keynesianismo

\end{esquema}

\esquemalargo












\begin{esquemal}
	\1[] \marcar{Introducción}
		\2 Contextualización
			\3 Evolución de la ciencia económica
				\4 Conjunción de múltiples factores
				\4 Contexto económico
				\4 Contexto teórico previo en economía
				\4 Avances en otras discIplinas
				\4[] Filosofía
				\4[] Matemáticas
				\4[] Biología
			\3 Historia del pensamiento económico
				\4 Permite entender origen de pensamiento actual
				\4 Permite entender problemas históricos
				\4 Permite valorar programas de investigación
			\3 Macroeconomía
				\4 Estudio de fenómenos económicos de gran escala
				\4[] Interacción de economías con millones de agentes
				\4[] Emergencia de fenómenos ajenos a microeconomía
				\4 Énfasis sobre variables agregadas
				\4[] Especialmente renta, precios, empleo, interés
				\4[] Requiere de nuevas herramientas de análisis
				\4[] Nuevos supuestos robustos a agregación
			\3 Keynesianismo
				\4 Primer enfoque propiamente macroeconómico
				\4 Revolución científica
				\4[] Nuevo marco de análisis
				\4[] $\to$ Conjunto de nuevas herramientas
				\4[] $\to$ Explicar anomalías de modelo anterior
				\4[] $\then$ Paradigma en sentido de Kuhn
				\4 Keynesianismo trasciende obra de Keynes
				\4[] Da lugar a síntesis neoclásica
				\4[] Condiciona programas de investigación posteriores
				\4 Implicaciones de política económica
				\4[] Cataliza cambio general en política económica
				\4[] Sintetiza y fundamenta políticas ya aplicadas
		\2 Objeto
			\3 ¿Quién fue Keynes?
			\3 ¿Qué aportó a la ciencia económica?
			\3 ¿En qué se diferencia su modelo del neoclásico?
			\3 ¿Qué influencia posterior han tenido Keynes?
			\3 ¿Qué es la síntesis neoclásica?
		\2 Estructura
			\3 Keynes
				\4 Trayectoria
				\4 Contexto
				\4 Metodología
				\4 Ideas clave de la Teoría General
				\4 Implicaciones
			\3 Síntesis neoclásica
				\4 IS-LM
				\4 Curva de Phillips
	\1 \marcar{Keynes}
		\2 Trayectoria
			\3 Vida
				\4 1883-1946
			\3 Influenciado por
				\4 Ricardo
				\4 Malthus
				\4 Marshall
				\4 Pigou
				\4 Irving Fisher
				\4 Kuznets -- National Income 1929-1932 (1934)
				\4 Thornton, Jevons (animal spirits)
				\4 Harrod
			\3 Influenció a
				\4 Macroeconomía posterior
				\4 Síntesis neoclásica
				\4[] Hicks
				\4[] Samuelson
				\4[] Patinkin
				\4 Neokeynesianos del desequilibrio
				\4[] Benassy
				\4[] Clower
				\4[] Leijonhufvud
				\4[] Malinvaud
				\4 Monetaristas
				\4[] Friedman
				\4 Nueva Economía Keynesiana
				\4 Contabilidad nacional
			\3 Obras
				\4 Consecuencias económicas de la paz (1919)
				\4 Tratado sobre la reforma monetaria (1923)
				\4 Tratado sobre el dinero -- I y II (1930)
				\4 La Teoría General del Empleo, el Interés y el Dinero (1936)
				\4[I] Introducción
				\4[II] Definiciones e ideas
				\4[III] La propensión al consumo
				\4[IV] La inducción a la inversión
				\4[V] Salarios nominales y precios
				\4[VI] Notas breves inspiradas por la Teoría General
		\2 Contexto
			\3 Económico
				\4 Post Primera Guerra Mundial
				\4 Años 20:
				\4[] Expansión mundial generalizada
				\4[] Reparaciones de guerra Alemania $\to$ vencedores
				\4[] Restablecimiento del patrón oro en Reino Unido
				\4[] $\to$ Deflación de precios y salarios insuficiente
				\4[] $\then$ Déficit comercial con Francia y EEUU
				\4 Crack del 29
				\4[] Contracción generalizada del crédito
				\4[] Aumento de tipos de interés
				\4[] Devaluaciones competitivas
				\4[] Aumento espectacular del desempleo
				\4[] Fin de la convertibilidad a oro en 1931
				\4[] Elección de Roosevelt
				\4[] Rearme en Europa a partir de años 30
			\3 Teoría económica
				\4 Dominada por enfoque neoclásico
				\4 Economías tienden a plena utilización de capacidad
				\4[] Tendencia a dejar de lado procesos de ajuste
				\4[] Tendencia a poner énfasis en largo plazo
				\4 Ley de Say como elemento central
				\4[] Excesos de oferta generales son temporales
				\4[] Economía se ajusta hacia igualdad S=D
				\4[] Desajuste es sólo posible temporalmente
		\2 Metodología
			\3 Método de exposición
				\4 Fundamentalmente verbal
				\4 Teoría General
				\4[] Uso moderado de ecuaciones y gráficas
				\4[] $\to$ Muy escaso si comparado con macro actual
				\4 Crea concepto de ``clásicos'' diferente al de Marx
				\4[] Basado en el \textit{Treasure view} y obra de Pigou
				\4[] $\to$ Teoría cuantitativa del dinero
				\4[] $\to$ Economía es estable a nivel agregado
				\4[] $\to$ Desviaciones de pleno empleo son transitorias
			\3 Modelización
				\4 Hereda métodos de economistas neoclásicos
				\4 Agregación general de variables
				\4[] Microfundamentación superficial
				\4[] Enfoque macroeconómico
				\4[] $\to$ Frisch es primero en escribir ``macroeconomía''
				\4 Reminiscencia de equilibrio general
				\4[] Aunque debate sobre si walrasiano o no
				\4[] $\to$ ¿Conocía obra de Walras?
				\4[] $\to$ ¿Todos los mercados relevantes conectados?
				\4[] $\to$ Literatura posterior: ¿ajuste walrasiano o no?
				\4[] Representa relaciones entre mercados relevantes
			\3 Econometría y economía cuantitativa
				\4 Escéptico en general
				\4[] A pesar de formación matemática
				\4 Crítica a métodos cuantitativos
				\4[] Incipientes a finales de años 30
				\4[] Critica métodos cuantitativos de Tinbergen
				\4[] Señala problema de factores omitidos
				\4[] $\to$ Causalidad muy difícilmente distinguible
		\2 Ideas clave de la Teoría General
			\3 La propensión al consumo
				\4 Ddas. de inversión y consumo son distintas
				\4[] Dependen de diferentes factores
				\4[] Pero tienen en común
				\4[] $\to$ No toda renta se convierte en gasto
				\4[] Neoclásicos
				\4[] $\to$ Toda renta se convierte en gasto ex-ante
				\4 Función de consumo
				\4[] Keynes no explicita forma de función de consumo
				\4[] Depende de muchos factores
				\4[] Pero resumibles en ``ley psicológica fundamental''
				\4[] $\to$ Se consume menos de lo que se ingresa
				\4[] $\to$ Habitualmente como $C(Y) = C_0 + cY$
				\4 Estabilidad de la función de consumo
				\4[] Implica $c<1$
				\4[] $\to$ Consumo converge
				\4[] $\to$ Empíricamente corroborada
				\4[$\then$] No todo el ingreso se transforma en demanda
				\4[$\then$] Para alcanzar pleno empleo necesario $\uparrow$ inversión
			\3 El multiplicador del gasto
				\4 Aumentos en demanda
				\4[] $\to$ Inducen aumentos mayores en producción
				\4[] Porque en contexto de exceso de capacidad:
				\4[] $\Delta I \to \Delta Y \to \Delta C \to \Delta Y \to \Delta C \to \ldots $
				\4 Supuesto de propensión marginal < 1
				\4[] $\to$ Implica convergencia del proceso dinámico
				\4[] $\to$ Demanda agregada no crece de forma explosiva
				\4 Formulación
				\4[] $Y = C_0 + cY + I$
				\4[] $Y = \frac{1}{1-c}\cdot (C_0 + I)$
				\4[] $\to$ Multiplicador = $\frac{1}{1-c}$
				\4[] \grafica{multiplicador}
				\4 Dados límites a la demanda de consumo
				\4[] $\to$ $\Delta I$ puede llevar a pleno empleo
			\3 Principio de la demanda efectiva
				\4 Economistas clásicos asumen Ley de Say
				\4[] No hay excesos agregados de oferta sostenibles
				\4[] Producción iguala gasto planeado ex-ante
				\4 Keynes niega Ley de Say
				\4[] Excesos agregados de oferta pueden ser permanentes
				\4[] Prod./renta no siempre igual a gasto ex-ante
				\4[] $\then$ EDemanda de dinero, EOferta bienes
				\4 Cuando existen excesos de capacidad
				\4[] $\to$ Demanda agregada determina producción
				\4 Modelo keynesiano de un sector
				\4[] Producción de pleno empleo $\bar{Y}$
				\4[] $Y = \text{DA}$
				\4[] $\text{DA} \equiv C_0 + cY + I$
				\4[] \grafica{aspakeynesiana}
				\4[] Dados $I$ y $C_0$
				\4[] $\to$ Ningún mecanismo hace que $Y=\bar{Y}$
				\4[] $\then$ Economía no ajusta a pleno empleo
			\3 Oferta monetaria, preferencia por liquidez e interés
				\4 Tipo de interés
				\4[] Determinado en mercado monetario
				\4[] Neoclásicos:
				\4[] $\to$ Determinado en mercado de fondos prestables
				\4[] $\to$ equilibrio ahorro-inversión
				\4 Demanda de dinero
				\4[] ¿Por qué los agentes prefieren liquidez?
				\4[] Motivo de transacción:
				\4[] $\to$ depende positivamente de renta
				\4[] Motivo de especulación:
				\4[] $\to$ depende negativamente de interés
				\4[] Cuando tipos de interés son muy bajos:
				\4[] $\to$ Agentes prevén $\uparrow$ de tipos/$\downarrow$ de precios
				\4[] $\then$ Demanda cualquier cantidad adicional de M
				\4[] $\then$ Elasticidad infinita $M^D$ a interés
				\4[] Neoclásicos:
				\4[] $\to$ Sólo motivo de transacción
				\4[] $\to$ Ecuación cuantitativa del dinero
				\4 Equilibrio del mercado monetario
				\4[] Asumiendo oferta monetaria exógena
				\4[] Tipo de interés se ajusta para $M^S = M^D$
			\3 Inversión y el tipo de interés
				\4 Demanda de inversión
				\4[] Empresas ordenan proyectos según:
				\4[] Eficiencia marginal del capital (EMK)
				\4[] EMK = interés que iguala:
				\4[] -- precio\footnote{Según Keynes, el precio no es el precio de mercado sino el ``supply price'', que corresponde al precio que induce a productor a producir una nueva unidad. Sólo en un contexto de competencia se igualan supply price y precio de mercado.}
				\4[] -- flujos de caja descontados
				\4[] $\to$ Equivalente a TIR
				\4[] Si EMK > tipo de interés
				\4[] $\to$ Proyecto se lleva a cabo
				\4 Formulación
				\4[] Generalmente, caracterizada como $I = I_0 + I(i)$
				\4[] $\to$ $\frac{d \, I(i)}{d \, i} < 0$
				\4[] Pero $I(i)$ es inestable
				\4 ¿Por qué inestable?
				\4[] ¿Cómo estiman inversores los flujos esperados?
				\4[] Estimaciones dependen de ``animal spirits''
				\4[] Animal spirits: volatilidad de flujos esperados
				\4[] $\to$ Demanda de inversión es inestable
				\4[] $\to$ Efecto limitado de $\Delta i$ sobre I
				\4[$\then$] Sector público puede estabilizar dda. de inversión
				\4[$\then$] Estímulo público estimula también expectativas
				\4[$\then$] Feedback positivo estímulo-inversión-renta
			\3 Precios y salarios
				\4 Existen mecanismos de ajuste hacia pleno empleo
				\4[] -- Variación de precios
				\4[] $\to$ Pero exceso de oferta de bienes
				\4[] $\then$ De hecho, tendencia a deflación
				\4[] -- Bajada de salarios
				\4[] $\to$ Pero rigidez nominal a la baja
				\4[] $\to$ Si bajan, la demanda cae
				\4 Existen también mecanismos contra-ajuste
				\4[] Incluso aunque salarios sean perf. flexibles
				\4[] Reducción de salarios puede deprimir demanda
				\4[] $\to$ Círculo vicioso que deprime demanda
				\4 Hasta ajuste:
				\4[] $\to$ Desempleo
				\4[] $\to$ Pobreza
				\4[] $\to$ Potencial no aprovechado
				\4[$\then$] Paro persistente
				\4[$\then$] Exceso de capacidad
				\4 Solución a problema de DAgregada deficiente
				\4[] Inflación + estímulo A para reducir desempleo
				\4[] Salario real por encima de equilibrio
				\4[] Empresas igualan PMgL con salario real
				\4[] $\to$ Demandan poco trabajo
				\4[] $\to$ Trabajadores ofertan mucho trabajo
				\4[] $\then$ Oferta racionada
				\4[] $\then$ Desempleo
				\4[] Aumento de los precios
				\4[] $\to$ Empresas no-competitivas aumentan output
				\4[] $\to$ Salario real cae
				\4[] $\to$ Aumenta demanda de trabajo
				\4[] $\to$ Cae oferta de trabajo
				\4[] $\then$ Aumento del trabajo contratado
				\4[] $\then$ Aumento del output
				\4[] $\then$ Reducción del desempleo
				\4[] $\then$ Salario real contracíclico
				\4 Representación gráfica
				\4[] \grafica{rigideznominal}
		\2 Implicaciones globales
			\3 Mercado de trabajo
				\4 No es un mercado neoclásico
				\4 Salario real no se ajusta hasta equilibrio
				\4 Desempleo involuntario es posible
				\4[] Trabajadores dispuestos a trabajar
				\4[] $\to$ A salario que trabajan otros
				\4[] No pueden encontrar trabajo a ese salario
				\4[] $\to$ Racionamiento de la oferta
			\3 Insuficiencias de demanda agregada
				\4 Economía puede sufrir insuficiencias de dda.
				\4 Insuficiencias de dda. causan paro
				\4[$\to$] En la práctica, lo habitual es dda. insuficiente
			\3 Intervención pública
				\4 Puede aumentar demanda
				\4[] $\to$ Reducir el desempleo
				\4[] $\to$ Llevar economía a potencial
				\4 Puede reducir tipo de interés
				\4[] Por vía de estímulo monetario
				\4[] Estimular demanda de inversión
				\4 Inversión pública
				\4[] Herramienta de pol. económica
			\3 Política monetaria puede no ser efectiva
				\4 En determinados contextos
				\4[] Trampa de liquidez
				\4 Tipo de interés se determina en mercados monetarios
				\4[] No iguala ahorro e inversión
				\4 En general, herramienta pasiva
				\4[] $\to$ Acomodar medidas de demanda
				\4[] $\to$ Mantener tipos de interés bajos
	\1 \marcar{La síntesis neoclásica}
		\2 Idea clave
			\3 Contexto
				\4 Revolución keynesiana
				\4[] Economías no tienden a plena capacidad
				\4[] Desempleo es un hecho empírico evidente
				\4 Análisis agregado de la economía
				\4[] Variables agregadas
				\4[] Relaciones ad-hoc entre agregados
				\4 Estadísticas agregadas
				\4[] Incipientes en años 30
				\4[] Desarrollo de cuentas nacionales desde 40s
				\4[] Mejoras en recogidas de datos y computación
				\4 Evidencia empírica
				\4[] Relación consistente entre:
				\4[] $\to$ Empleo/output
				\4[] $\to$ Inflación salarial/inflación general
				\4[] $\then$ Curva de Phillips
				\4[] $\then$ Múltiples explicaciones
				\4 Principales autores
				\4[] Hicks
				\4[] Hansen
				\4[] Modigliani
				\4[] Solow
				\4[] Tobin
				\4[] Samuelson
				\4[] Patinkin
			\3 Objetivos
				\4 Reconciliar Keynes y neoclasicismo
				\4 Explicar corto y largo plazo
				\4 Predecir efectos de PM y PF
			\3 Resultados
				\4 Consenso dominante en macroeconomía
				\4$\to$  Desde años 40 hasta años 60
				\4 Paradigma macroeconómico coherente
				\4[] Reconcilia keynesianismo + neoclasicismo
				\4[] Más algunas aportaciones originales
				\4[] Complementado con mod. macroeconométricos
				\4[] Ejemplo principal: MPS-FMP de Modigliani et al.
				\4[] Núcleo formado por IS-LM y Curva de Phillips
				\4 Modelos y técnicas accesorias
				\4[] Teorías de la demanda de dinero
				\4[] Teorías de la demanda de consumo
				\4[] Modelos de crecimiento económico
				\4[] Modelos de economía abierta
				\4[] Modelos macroeconométricos con formas reducidas
				\4 IS-LM Modelo matemático macroeconómico de referencia
				\4[] Número reducido de variables
				\4[] Variables agregadas
				\4[] Apenas microfundamentado
				\4[] Síntesis neoclasicismo-keynesianismo
				\4[] $\to$ Variando ligeramente supuestos
				\4[] $\then$ Puede caracterizar diferentes escuelas
				\4[] $\then$ Conclusiones keynesianas o neoclásicas
				\4 Estudio diferenciado de c/p y l/p
				\4[] Mismo marco pero diferentes supuestos
				\4[] Estudio del ciclo vs estudio del crecimiento
				\4 Implicaciones fundamentales
				\4[] Ciclo puede ser domado
				\4[] $\to$ \textit{Fine-tuning} de la economía es efectivo
				\4[] Ajuste de precios y salarios es lento
				\4[] $\to$ Gobierno puede acelerar ajuste
				\4[] Intervención pública puede mejorar resultados
				\4 Origen de nuevos programas de investigación
				\4[] Microfundamentación de supuestos ad-hoc
				\4[] Acaban generando relación esquizofrénica con micro
		\2 IS-LM/AD-AS con supuestos clásicos
			\3 Idea clave
				\4 Contexto
				\4[] Concepción previa a Keynes de la economía
				\4[] Keynes critica visión predominante en Gran Depresión
				\4[] $\to$ Inversiones inviables deben liquidarse
				\4[] $\to$ Salarios reales deben caer
				\4[] $\to$ Tendencia inherente de la economía a equilibrio
				\4[] $\to$ Intervención inefectiva
				\4 Objetivo
				\4[] Caracterizar economía con supuestos clásicos
				\4[] $\to$ Explicitar supuestos
				\4[] Representar macroeconomía en  largo plazo
				\4[] Contrastar con modelo keynesiano
				\4 Resultados
				\4[] Output determinado en lado de la oferta
				\4[] $\to$ Mercado de trabajo en equilibrio
				\4[] $\to$ Sin desempleo
				\4[] Políticas monetaria y fiscal
				\4[] $\to$ No afectan al output
				\4[] $\to$ Sí afectan a precios y a interés
			\3 Oferta: AS
				\4 Mercado de trabajo neoclásico
				\4[] Equilibrio walrasiano
				\4[] $\to$ Salario real
				\4[] $\then$ Iguala oferta y demanda de trabajo
				\4[] $\then$ con empresas maximizando beneficios respecto trabajo
				\4[] $\then$ Con trabajadores optimizando preferencias ocio-consumo
				\4 Demanda de trabajo
				\4[] Empresas maximizan beneficio
				\4[] Condición de óptimo
				\4[] $\to$ Empresas igualan IMg y CMg
				\4[] $\then$ $Y'(L) \cdot (1 - \frac{1}{| \epsilon_{y-P}|}) = \frac{W}{P}$
				\4[] Mercado de bienes asumido competitivo
				\4[] $\to$ $| \epsilon_{y-P} | \to -\infty$
				\4[] $\then$ $Y'(L) = \frac{W}{P}$
				\4 Oferta de trabajo
				\4[] Optimización ocio-consumo
				\4[] Condición de óptimo:
				\4[] $\then$ $\frac{W}{P} = \frac{u_l}{u_c}$
				\4 Equilibrio
				\4[] Con PMgL decreciente y demanda decreciente
				\4[] $\to$ Existe equilibrio
				\4[] $\then$ $\left( \frac{W}{P}\right)^*$, $L^*$
			\3 Demanda: IS-LM $\to$ AD
				\4[IS] Mercado de bienes: ahorro e inversión
				\4[] Demanda de bienes:
				\4[] $\to$ $C(\bar{Y}) + G + I(r)$
				\4[] Oferta de bienes:
				\4[] $\bar{Y}$
				\4[] $\to$ Dado en bloque de oferta
				\4[] Equilibrio oferta y demanda de bienes
				\4[] $\to$ $\underbrace{\bar{Y} - C(\bar{Y})}_{S} -G = I(r)$
				\4[] $\then$ $S = I(r)$
				\4[] Tipo de interés $r$ se ajusta para igualar
				\4[] $\to$ Mercado de fondos prestables implícito
				\4[] Posible también efecto Pigou\footnote{Ver Metzler (1951).}
				\4[] $\to$ Ahorro depende negativamente de riqueza
				\4[] $\to$ Riqueza depende de saldos monetarios reales
				\4[] $\then$ Mecanismo adicional hacia estabilidad
				\4[] $\then$ Interés nominal también relevante, no sólo real\footnote{Sin efecto Pigou, el equilibrio en el mercado de fondos prestables depende sólo del tipo de interés real. Sin embargo, al aparecer los precios como determinantes del ahorro, el interés nominal pasa también a ser relevante en la medida en que junto a la inflación, es un elemento determinante del interés real.}
				\4[LM] Mercado de dinero
				\4[] Teoría cuantitativa del dinero
				\4[] Por definición, exceso demanda nulo
				\4[] $\to$ $M = k P \bar{Y}$
				\4[] $\then$ $\frac{d \, Y}{d \, M} = 0$
				\4[] $\then$ $\frac{d \, P}{d \, M} = k \cdot \bar{Y}$
			\3 Representación gráfica
				\4 \grafica{islmadasclasico}
			\3 Política fiscal
				\4 Impulso
				\4[] Aumento de G
				\4 Mercado de bienes y fondos prestables
				\4[] Ahorro menor que inversión
				\4[] $\to$ Aumento de interés
				\4[] $\then$ Caída de inversión
				\4 Efecto
				\4[] Mismo output
				\4[] $\to$ Determinado en mercado de trabajo
				\4[] Más interés
				\4[] $\to$ Crowding-out de inversión
				\4[] Precios iguales
				\4[] $\to$ Oferta monetaria y output constantes
			\3 Política monetaria
				\4 Impulso
				\4[] Aumento de M
				\4 Mercado monetario
				\4[] Por definición, en equilibrio constante
				\4[] $\to$ Dinero es token
				\4[] $\to$ Sólo sirve como medio de intercambio
				\4[] $\then$ Dicotomía clásica
				\4[] $\then$ Transmisión perfecta a precios
				\4 Efecto
				\4[] Mismo output
				\4[] $\to$ Determinado en mercado de trabajo
				\4[] Precios más altos
				\4[] $\to$ Mayor oferta monetaria
			\3 Valoración
				\4 Aplicado al largo plazo
				\4 Políticas públicas no aportan beneficios
				\4 Dicotomía clásica tiende a cumplirse en l/p
		\2 IS-LM/AD-AS con supuestos keynesianos
			\3 Idea clave
				\4 Contexto
				\4[] Historia
				\4[] $\to$ Tras publicación de la Teoría General
				\4[] $\to$ Caracterizar sistema subyacentes de ec.
				\4[] Hicks (1937)
				\4[] $\to$ Primera presentación con dos curvas
				\4[] $\to$ SI-LL
				\4[] Modigliani (1944)
				\4[] $\to$ Aclara papel de salarios nominales rígidos
				\4[] $\then$ Imposibilitan ajuste de salario real
				\4[] Popularizado por Hansen (1949, 1953)
				\4[] Metzler (1951)
				\4[] $\to$ Análisis de efectos riqueza
				\4[] Extendido por Patinkin y otros
				\4[] Keynes compara modelo clásico
				\4[] $\to$ Afirma es caso particular de su Tª General
				\4 Objetivos
				\4[] Modelo formal de Keynes (1936)
				\4[] Expresar ideas de Keynes en modelo matemático
				\4[] Marco de análisis para comparar
				\4[] $\to$ Modelo clásico en sentido de Keynes
				\4[] $\to$ Modelo keynesiano
				\4 Resultados
				\4[] Supuestos keynesianos
				\4[] $\to$ Aplicables a corto plazo
				\4[] Supuestos neoclásicos
				\4[] $\to$ Aplicables a largo plazo
				\4[] Enorme impacto
				\4[] $\to$ Propuestas de PM/PF basadas en IS-LM keynesiano
				\4[] $\to$ Herramienta de enseñanza de macroeconomía
				\4[] $\to$ Modelo a batir por otras corrientes
				\4[] Inicio de otros programas de investigación
				\4[] $\to$ Demanda de dinero
				\4[] $\to$ Demanda de consumo
				\4[] $\to$ Demanda de inversión
				\4[] $\to$ Modelos macroeconométricos
			\3 Oferta
				\4 Keynes no es explícito sobre competencia en bienes
				\4[] Posibles:
				\4[] $\to$ Competencia monopolística
				\4[] $\then$ Empresas deciden trabajo a contratar
				\4[] $\to$ Competencia perfecta
				\4 Salarios nominales rígidos a la baja
				\4[] Imposible ajuste hacia salario real de equilibrio
				\4 Salario real se mantiene por encima de eq.
				\4 Diferentes explicaciones
				\4[] Empresas en competencia monopolística
				\4[] $\to$ Deciden producción óptima en mercado de bienes
				\4[] $\then$ Ajustan demanda de trabajo a producción óptima
				\4[] $\then$ Demanda de bienes determina
				\4[] Competencia perfecta en bienes
				\4[] $\to$ Rigidez nominal asumida ad-hoc
				\4[] $\to$ Posibles shocks anteriores
				\4[] $\to$ Caída de precios y salarios reales
				\4[] $\then$ Mantiene insuficiencia de demanda agregada
				\4[] $\then$ Imposibilita ajuste hacia equilibrio
				\4 Ciclicidad del salario real
				\4[] Posible modelizar procíciclico o anticíclico
				\4 Salario real contracíclico
				\4[] Conclusión básica de bloque de oferta walrasiano
				\4[] Para alcanzar pleno empleo
				\4[] $\to$ Necesario aumentar precios más que salarios
				\4[] $\then$ Salario real necesariamente contracíclico
				\4 Salario real procíclico
				\4[] Keynes tardío
				\4[] $\to$ Acepta salario real contracíclico poco realista
				\4[] SNC
				\4[] $\to$ Ajuste de salarios más rápido que precios
				\4[] $\to$ Evidencia empírica favorable
				\4[] Competencia monopolística en bienes
				\4[] $\to$ Aumento de demanda aumenta costes
				\4[] $\to$ Precios son rígidos en el corto plazo
				\4[] $\then$ Mark-up cae con trabajo y output
				\4[] $\then$ Demanda de trabajo creciente
				\4[] $\then$ Salario real procíclico
				\4[] $\then$ Supuesto habitual en modelos modernos
			\3 Demanda
				\4[IS] Mercado de bienes: ahorro e inversión
				\4[] Demanda de bienes:
				\4[] $\to$ $C(Y) + G + I(i)$
				\4[] Consumo $C(Y)$
				\4[] $\to$ $C(Y) = C_0 + cY$
				\4[] $\then$ Ley psicológica fundamental de Keynes
				\4[] $\then$ Demanda de consumo crece con renta
				\4[] $\then$ Dinámica estable hasta equilibrio
				\4[] Inversión $I(i)$
				\4[] $\to$ Modelo keynesiano de la inversión
				\4[] $\then$ Inversión cae con $i$
				\4[] $\then$ Sujeto a animal spirits
				\4[] Gasto público autónomo G
				\4[] $\to$ Decisión discrecional del gobierno
				\4[] $\to$ Financiado mediante déficit
				\4[] $\to$ Posible monetización del déficit
				\4[LM] Mercado de dinero
				\4[] Función keynesiana de la demanda de dinero
				\4[] No sólo demanda por motivo de transacción
				\4[] $\to$ También por motivo de especulación
				\4[] $\then$ Tipos de interés de bonos es relevante
				\4[] $M = P \cdot L(i, Y)$
				\4[] $\to$ Demanda cae con $i$
				\4[] $\then$ Más coste de oportunidad de demandar dinero
				\4[] $\then$ Demanda infinita si $i$ = 0
				\4[] $\then$ Posible trampa de liquidez\footnote{Aunque en el contexto de la Teoría General, aparece como una posibilidad teórica de escasa relevancia práctica.}
				\4[] $\to$ Demanda L aumenta con Y
				\4[] $\then$ Más dinero a demandar
			\3 Política fiscal con supuestos keynesianos
				\4 Estímulo a demanda agregada
				\4[] Aumenta demanda de trabajo
				\4[] $\then$ Reducción directa del desempleo
				\4 Variación de componentes autónomos
				\4[] $\uparrow C_0$, $\uparrow I_0$ o ambos $\to$ $\downarrow S(Y)$
				\4[1] Curva IS se desplaza hacia derecha
				\4[2] Desplazamiento a lo largo de LM
				\4[3] Nueva intersección de IS y LM
				\4[$\Rightarrow$] $\downarrow S(Y) \to \uparrow Y, \uparrow r$
				\4[] \grafica{islmkeynesianopoliticafiscal}
				\4 Si equilibrio inicial en trampa de liquidez
				\4[] Política fiscal muy efectiva
				\4 Si equilibrio inicial cerca de Y máximo
				\4[] Política fiscal inefectiva
				\4[] $\uparrow$ demanda sube tipos $\Rightarrow$ crowding-out
			\3 Política monetaria con supuestos keynesianos
				\4 Estímulo indirecto a demanda agregada
				\4[] Vía reducción del tipo de interés en MMonetario
				\4[] $\to$ Canal indirecto o keynesiano de la política monetaria
				\4 Variación de $M^S$
				\4 Suponemos $\Delta M^S > 0$
				\4[1] Curva LM se desplaza hacia la derecha
				\4[2] Desplazamiento a lo largo de IS
				\4[3] Nueva intersección de IS y LM
				\4[] \grafica{islmkeynesianopoliticamonetaria}
				\4 Trampa de liquidez
				\4[] Tipos de interés cercanos a 0
				\4[] $\to$ Elasticidad\footnote{Fácilmente justificable con un modelo s-S de tipo Tobin. La demanda de $m$ de dinero óptima es aquella que minimiza una función de coste $c(m) = \frac{T}{m} k + \frac{m}{2} i$. La condición de primer orden implica que la demanda de dinero óptima es $m^* = \sqrt{\frac{Tk}{2i}}$. Así, a medida que $i$ se acerca a 0, la demanda óptima de dinero se acerca a infinito.} DDinero-interés $\to \infty$ 0
				\4[] Todo aumento de oferta monetaria
				\4[] $\to$ Se absorbe por la oferta
				\4[] $\then$ Sin impacto alguno
				\4[] Caso considerado poco probable
				\4[] $\to$ Interés nominal alejado de cero
			\3 Valoración
				\4 Aplicado en corto plazo
				\4 Rigideces de precios y salarios relajadas con el tiempo
				\4[] Inicialmente, muy rígidos
				\4[] Posteriormente, se asume ajuste lento
				\4 Justificación de intervención pública
				\4[] Porque precios y salarios no se ajustan rápido
			\3 Modelo de Mundell-Fleming
				\4 Idea clave
				\4[] IS-LM en contexto de economía abierta
				\4[] Demanda agregada también depende de:
				\4[] $\to$ Demanda de importaciones
				\4[] $\to$ Demanda de exportaciones
				\4[] $\to$ Tipo de cambio
				\4[] Tipo de interés depende de:
				\4[] $\to$ Entrada y salida neta de capitales
				\4 Formulación
				\4[] IS: $Y = C(Y) + I(r) + NX(Y, S)$
				\4[] LM: $M_S = P \cdot L(Y,r)$
				\4[] BP: $\Delta R = NX(Y,S) - CF(r, r^*)$
				\4 Implicaciones
				\4[] PM efectiva aumentando Y si:
				\4[] $\to$ Mov. de K + TCFlexible
				\4[] $\to$ Sin mov. de K + TCFlexible
				\4[] PF efectiva aumentando Y si:
				\4[] $\to$ Mov. de K + TCFijo
				\4[] $\to$ Sin mov. de K + TCFlexible
				\4 Valoración
				\4[] Referencia de políticas macro en economía abierta
				\4[] Predicciones generalmente consistentes con evidencia
			\3 IS-LM Dinámico
				\4 Dornbusch (1976)
				\4 Blanchard y Kiyotaki (1986)
		\2 Curva de Phillips\footnote{Ver \href{http://economics.weinberg.northwestern.edu/robert-gordon/files/RescPapers/HistoryPhillipsCurve.pdf}{Gordon, R. J (2011) \textit{The History of the Phillips Curve: Consensus and Bifurcation}} en Economica. Probablemente la mejor historia del concepto y de la macroeconomía en términos resumidos. Esta sección sólo contiene algunas ideas tomadas de ese artículo. El resto, Palgrave, de Vroey y artículos originales.}
			\3 Idea clave
				\4 Contexto
				\4[] Hume y TCD
				\4[] $\to$ Admiten pequeño efecto $M$ $\to$ $Y$ en C/P
				\4[] $\then$ Muy ligera correlación $P$ $\to$ $Y$
				\4[] Fisher (1926)
				\4[] $\to$ Trabajo poco conocido
				\4[] $\to$ Redescubierto décadas después
				\4[] $\to$ Postula relación EOferta y $\downarrow$ salarios
				\4[] $\to$ Relación más fuerte cuanto mayor EOferta
				\4[] Modelo clásico en sentido de Keynes
				\4[] $\to$ Output determinado independiente de P
				\4[] $\to$ $\then$ Dicotomía clásica
				\4[] Keynes
				\4[] $\to$ Exceso de capacidad en economía
				\4[] $\to$ Exceso de oferta en mercado de trabajo
				\4[] $\to$ Estímulo a demanda agregada
				\4[] $\then$ Aumenta empleo y output
				\4[] $\then$ Aumentan precios
				\4[] $\then$ Relación decreciente precios-paro
				\4[] Síntesis neoclásica
				\4[] $\to$ Modelo keynesiano en corto plazo
				\4[] $\to$ Modelo neoclásico en largo plazo
				\4[] $\to$ Salarios nominales sí se ajusta pero lentos
				\4 Objetivos
				\4[] Estimación empírica de relación desempleo-salarios
				\4[] Estimación empírica relación desempleo-precios
				\4[] Estimación empírica relación output-precios
				\4[] Definir relación entre corto y largo plazo
				\4 Resultados
				\4[] Relaciones empíricas observable:
				\4[] -- Precios y empleo
				\4[] $\to$ Inflación salarial-desempleo
				\4[] $\to$ Inflación general-desempleo
				\4[] $\then$ Relativamente consistente en el tiempo
				\4[] -- Inflación a pesar de paro
				\4[] $\to$ En rango de tasas de paro, inflación persiste
				\4[] $\to$ También con paro relativamente elevado
				\4[] Doble papel en la literatura
				\4[] $\to$ Confirmación de modelos teóricos
				\4[] $\to$ Hecho estilizado que justificar a nivel teórico
				\4[] Inicialmente:
				\4[] $\to$ Muestra relación decreciente inflación-paro
				\4[] $\then$ Entendido como confirmación de SNC
				\4[] $\then$ Curva de Phillips como menú de políticas
				\4[] Posteriormente, campo de batalla de diferentes teorías
				\4[] $\to$ ¿Posible aumentar empleo con inflación?
				\4[] $\to$ ¿Posible hacerlo sin acelerar inflación?
				\4[] $\to$ ¿Pueden los gobiernos explotar esa relación?
			\3 Phillips (1958)
				\4 Idea clave
				\4[] Estimar empíricamente
				\4[] $\to$ Relación salarios y empleo
				\4[] Diferentes fases históricas
				\4[] $\to$ Valorar robustez de relación
				\4[] Proponer explicación de relación
				\4[] $\to$ ¿Exceso de demanda?
				\4[] $\to$ ¿Presión de costes sobre salario?
				\4 Formulación
				\4[] Dos posibles explicaciones de $\Delta$ Salarios
				\4[] $\to$ Inflación de demanda: \textit{demand pull}
				\4[] $\then$ Competencia de empresas por factor trabajo
				\4[] $\then$ Relación positiva $\Delta$ salarios-trabajo
				\4[] $\to$ Inflación de costes: \textit{cost-push}
				\4[] $\then$ Trabajadores exigen mantener poder adquisitivo
				\4[] $\then$ Relación negativa salarios-trabajo
				\4[] Conjuntos de datos de dos fases:
				\4[] $\to$ 1861-1913
				\4[] $\to$ 1913-1948
				\4[] Estimación de curvas de aproximación
				\4[] Estimación de regresiones $\Delta$ salarios-paro
				\4 Implicaciones
				\4[] Relación decreciente
				\4[] $\to$ Desempleo
				\4[] $\to$ Aumento de los salarios
				\4[] Relación asimétrica $\Delta$ salarios-paro
				\4[] $\to$ Menos paro aumenta más el $\Delta$ salarios
				\4[] $\to$ Más paro reduce más el $\Delta$ salarios
				\4[] Tasa de variación de salarios
				\4[] $\to$ Puede explicarse con desempleo
				\4[] $\then$ En mayoría de casos
				\4[] $\then$ Salvo que haya inflación de costes
				\4[] Inflación de demanda
				\4[] $\to$ Empresas compiten por factor trabajo
				\4[] Inflación de costes
				\4[] $\to$ Aumento de precios de bienes de importación
				\4[] $\then$ Aumento de salarios sin efecto sobre paro
				\4[] $\then$ Suceso relativamente raro
				\4 Valoración
				\4[] Artículo seminal
				\4[] Abre programa de investigación
				\4 Phillips (1958) encuentra relación empírica
				\4[] Diferentes periodos desde 1861 a 1957
				\4[] Regresión $\dot{w}$ contra $u_t$
				\4[] \fbox{$\ln (\delta w_t) = \beta_0 + \beta_1 \ln u_t$}
				\4[] $\to$ Coeficiente $\beta_1$ negativo
				\4[] $\to$ Inflación salarial > 0 para desempleo positivo
				\4[$\Rightarrow$] Postula rel. negativa entre $u$ y $\dot{w}$
				\4[$\Rightarrow$] Inflación salarial aún en presencia de desempleo
				\4[] \grafica{curvadephillips}
			\3 Lipsey (1960)
				\4 Idea clave
				\4[] Curva de Phillips a la derecha de ordenadas
				\4[] $\to$ Siempre hay paro
				\4[] $\to$ Imposible alcanzar paro 0 aun con inflación
				\4[] Incompatible con modelo walrasiano de mercado de trabajo
				\4 Paro friccional como explicación
				\4[] Desempleados no encuentran vacantes
				\4[] Vacantes no encuentran desempleados
				\4[] $\to$ Paro friccional
				\4 Pleno empleo
				\4[] No implica desempleo 0
				\4[] Entendido como ausencia de desempleo involuntario
				\4[] $\to$ Puede implicar paro positivo
				\4 Implicaciones
				\4[] Racionaliza relación observada por Phillips
				\4[] Exceso de demanda de trabajo empuja precios
				\4[] $\to$ Demand-pull
				\4[] Extrapola relación a inflación-desempleo
				\4[] $\to$ Inflación salarial como proxy de inflación
			\3 Samuelson y Solow (1960)
				\4 Idea clave
				\4[] Extraer conclusiones de PEconómica de Phillips (1960)
				\4[] Primera mención a ``curva de Phillips''
				\4 Formulación
				\4[] Proponen curva de Phillips como menú a c/p
				\4[] $\to$ ¿Cuánto paro quieren los gobiernos?
				\4[] $\to$ ¿Cuánta inflación están dispuestos a pagar?
				\4[] Gobiernos pueden elegir a c/p combinación $u$--$\pi$
				\4[] Decisión de gobiernos puede tener efectos a l/p
				\4[] $\to$ Puede desplazarse hacia abajo si histéresis
				\4[] $\to$ Evita mencionar desplazamiento hacia arriba
				\4[] $\then$ Insinúa CPhillips con expectativas
				\4[] $\then$ Adelanta idea de histéresis
				\4[] Debate CPhillips como herramienta para distinguir
				\4[] inflaciones demand-pull vs cost-push
				\4 Friedman y otros
				\4[] Descartan curva de Phillips sea estable a largo plazo
				\4[] $\to$ No es utilizable como menú de pol. económica
				\4[] $\to$ En el largo plazo, la curva de Phillips es vertical
				\4[] $\to$ Fundamentan en HEA y TCD
			\3 Valoración
				\4 Ley de movimiento de macroeconomía en SNC
				\4[] Componente adicional a modelo IS-LM
				\4[] Relación más o menos ad-hoc que explica evolución
				\4[] $\to$ Precios se ajustan al alza
				\4[] $\to$ Salarios se ajustan hacia equilibrio
				\4 Germen de cambios de paradigma posteriores
				\4[] Cuando curva de Phillips cambia comportamiento
				\4[] $\to$ Debilitamiento de paradigma de SNC
				\4 Concepto central de macroeconomía moderna
				\4[] Caracteriza idea central de predicción de un modelo
				\4[] $\to$ ¿Qué efecto tienen estímulos monetarios y fiscales?
				\4 Elemento central de todas escuelas posteriores
				\4[] Monetarismo
				\4[] Nueva Macroeconomía Clásica
				\4[] Nueva Economía Keynesiana
		\2 Otros programas de investigación
			\3 Demanda de dinero
				\4 Modelos s-S
				\4[] Base conceptual
				\4[] Mantener stocks implica trade-off
				\4[] $\to$ Coste de oportunidad por tener
				\4[] $\to$ Ahorro por no tener que reponer
				\4 Modelos de inventario aplicados a dinero
				\4[] Más dinero cuanto más renta
				\4[] $\to$ Porque será necesario para más gasto
				\4[] Menos dinero cuanto mayor interés
				\4[] $\to$ Porque aumenta coste de oportunidad
				\4 Permite explicar demanda de dinero en LM Keynes
				\4[] Demanda aumenta con renta
				\4[] Demanda cae con tipo de interés
			\3 Demanda de consumo
				\4 Paradoja de Kuznets
				\4[] En largo plazo
				\4[] $\to$ Proporción de consumo/ahorro aprox. constante
				\4[] Incompatible con función keynesiana de Consumo
				\4 Duesenberry
				\4[] Consumidores comparan para decidir su consumo
				\4[] $\to$ Con otros consumidores
				\4[] $\to$ Con su consumo en el pasado
				\4[] Tratan de mantenerse similar a comparaciones
				\4[] $\to$ PMgC aumenta si aumenta renta de otros
				\4[] $\to$ PMgC aumenta si cae su propia renta
				\4 Friedman y Modigliani
				\4[] Consumidores deciden en función de renta total
				\4[] $\to$ A lo largo de ciclo vital (Modigliani)
				\4[] $\to$ A lo largo de horizonte de decisión (Friedman)
				\4[] Germen de microfundamentación del consumo
				\4[] $\to$ Junto con modelo de Fisher
				\4[] $\then$ Modelo de Hall (1979)
				\4[] $\then$ Modelos de la NMC
				\4[] $\then$ Modelos DSGE NNS
			\3 Demanda de inversión
				\4 Inversión es componente fundamental de la DA
				\4[] Caracterizar volatilidad relativamente elevada
				\4 Modelo neoclásico de Jorgenson
				\4[] Desligar parte de inversión de Fisher
				\4[] Caracterizar efecto de interés sobre inversión
				\4[] Problema:
				\4[] $\to$ Posibles cambios discretos de inversión
				\4 Modelo de la Q de Tobin
				\4[] Incorporar costes de ajuste
				\4[] Efectos de incertidumbre sobre inversión
			\3 Modelos estructurales de la macroeconomía
				\4 MPS de Modigliani\footnote{Ver \href{https://fraser.stlouisfed.org/files/docs/publications/FRB/pages/1985-1989/32204_1985-1989.pdf}{St. Louis Fed (1987)}}
				\4[] Modelo culminante de programa de SNC
				\4[] >300 ecuaciones
				\4[] Reserva federal usa hasta finales de 80
				\4[] Corto plazo
				\4[] Largo plazo
				\4[] $\to$ Similar a modelo de crecimiento exógeno
				\4[] $\to$ Convergencia a golden-rule no garantizada
				\4[] $\to$ Política fiscal determina equilibrio
		\2 Implicaciones
			\3 Influencia política económica
				\4 Intervención pública para $\uparrow$ Demanda agregada
				\4[] Objetivo principal de sector público
				\4 Curva de Phillips entendida como menú
				\4[] Samuelson y Tobin
				\4 Política monetaria
				\4[] Se reconoce capacidad para aumentar DA
				\4[] Generalmente, rol pasivo respecto PF
				\4[] $\to$ Mantener tipos de interés bajos
				\4 Aceleración de la inflación en los 60
				\4[] Intentos por bajar desempleo estimulando DA
				\4[] Inflación se acelera más que cae desempleo
				\4[] Guerra de Vietnam intensifica proceso
				\4 Paradigma keynesiano a partir de los 70
				\4[] Fuertemente atacado
				\4[] Curva de Phillips como demand-pull se rompe
				\4[] Cost-push y expectativas
				\4[] $\to$ Curva de Phillips aceleracionista
				\4[] $\then$ Paradigma keynesiano sustituido por NMC, monetarismo
				\4 Problemas de intervención pública
				\4[] Relativamente conocidos y aceptados
				\4[] Considerados de segundo orden respecto insuficiencias
			\3 Influencia teórica
				\4 Shocks de oferta
				\4[] Planteados por PHilllips
				\4[] $\to$ Valoración de cost-push por importaciones
				\4[] Pero SNC no presta gran atención
				\4[] $\to$ Hasta crisis del petróleo
				\4 Monetarismo
				\4[] Mercado de trabajo:
				\4[] $\to$ es competitivo
				\4[] $\to$ Se negocia respecto a salario real
				\4[] Pero trabajadores no conocen inflación
				\4[] $\to$ Estiman a partir de inflación pasada conocida
				\4[] $\then$ Posible reducir salario real en el corto plazo
				\4[] $\then$ Posible aumentar output en corto plazo
				\4[] Tendencia aceleracionista de la inflación
				\4[] $\to$ Si gobiernos tratan de explotar curva de Phillips
				\4[] $\then$ Trabajadores se acostumbran a mayor inflación
				\4[] $\then$ Cada vez necesario mayor inflación
				\4[] Demanda de dinero
				\4[] $\to$ No sólo depende de interés y renta
				\4[] $\to$ Muchos otros factores
				\4[] $\then$ Más estable
				\4[] $\then$ Teoría cuantitativa del dinero se cumple
				\4[] Demanda de consumo
				\4[] $\to$ Ciclo vital
				\4[] $\to$ Renta permanente
				\4 Neokeynesianos del desequilibrio
				\4[] Rechazo de marco de SNC
				\4[] Afirmación de prescripciones keynesianas
				\4[] $\to$ Insuficiencias de demanda agregada
				\4[] $\to$ Mercados no tienden a equilibrio
				\4[] Desequilibrios son posibles en mercados
				\4[] $\to$ Problemas de búsqueda y coordinación
				\4[] $\to$ Hipótesis de decisión dual
				\4[] $\to$ Microfundamentación de mercados no walrasianos
				\4 Nueva Macroeconomía Clásica
				\4[] Incapacidad para predecir inflación de 70s
				\4[] $\to$ No es sólo porque es un nuevo shock
				\4[] $\then$ Problema más profundo
				\4[] Parámetros de modelos no son inmunes a políticas
				\4[] $\to$ Agentes reaccionan a políticas
				\4[] $\then$ Parámetros cambian
				\4[] $\then$ Formas reducidas estimadas cada vez más erróneas
				\4[] Curva de Phillips
				\4[] $\to$ Explicable en términos de información imperfecta
				\4[] $\to$ Agentes predicen eficientemente senda de precios futuros
				\4[] $\then$ Imposible explotar de manera sistemática
				\4[] Mercados en equilibrio walrasiano
				\4[] $\to$ No hay sendas de ajuste hacia equilibrio
				\4[] $\to$ Siempre en equilibrio
				\4[] DGSE
				\4[] $\to$ Bases metodológicas para NEK
				\4 Nueva Economía Keynesiana / Nueva Síntesis Neoclásica
				\4[] Marco DSGE heredado de NMC y RBC
				\4[] Mercados en equilibrio
				\4[] Microfundamentación frente a crítica de Lucas
				\4[] Microfundamentación de rigideces nominales y reales
				\4[] $\to$ Explicaciones de desempleo
				\4[] $\to$ Variedad de modelos
				\4[] Modelo canónico de Nueva Síntesis Neoclásica
				\4[] $\to$ Ecuación similar a IS
				\4[] $\to$ Ecuación similar CPhillips con HER
	\1[] \marcar{Conclusión}
		\2 Recapitulación
			\3 Keynes
			\3 Síntesis neoclásica
		\2 Idea final
			\3 Robert Solow sobre modelos macro y economistas
				\4 Existen dos tipos de macroeconomistas
				\4 Macroeconomistas que formulan modelo canónico
				\4[] Y tratan de resolver todas las preguntas con el
				\4[] $\to$ Aplicando ligeros cambios
				\4 Macroeconomistas que utilizan un conjunto de modelos
				\4[] Cada uno para resolver diferentes cuestiones
				\4 Keynes
				\4[] ``La Teoría General''
				\4 Síntesis Neoclásica
				\4[] Núcleo central que integra muchos modelos
				\4[] $\to$ Clásico
				\4[] $\to$ Keynesiano
				\4[] $\to$ Componentes del modelo: dinero, consumo, inversión...
			\3 Política económica
			\3 Confianza en gestión del ciclo
				\4[]  Es posible reducir fluctuaciones cíclicas
				\4[]  Intervención pública posible y necesaria
				\4[]  Posible ajustar macromagnitudes de forma sistemática
				\4 Ejecución de intervención
				\4[] Políticas keynesianas preceden a Keynes
				\4[] Pero obra de Keynes construye sistema teórico que:
				\4[] $\to$ Justificaciones de intervención
				\4[] $\to$ Métodos de intervención
			\3 Impacto global del keynesianismo
				\4 Transformación del debate
				\4[]  Macroeconomía no necesariamente estable
				\4[]  Diferente marco de exposición
				\4[]  Programas de investigación siguen estela de Keynes
				\4 Estímulo de nuevas teorías
				\4[]  Habitualmente contrarias
				\4[]  También recuperadoras de keynesianismo original
\end{esquemal}























\graficas

\begin{axis}{4}{Concepto de demanda efectiva como intersección de oferta y demanda agregada}{Y}{Demanda}{aspakeynesiana}
	% Equilibrio
	\draw[-] (0,0) -- (4,4);
	\node[right] at (4,4){$Y=DA$};
	
	% Demanda agregada 0
	\draw[-] (0,1) -- (4,3);
	\node[right] at (4,3){$\text{DA}_0$};
	
	% Demanda agregada 1
	\draw[-] (0,1.5) -- (4,3.5);
	\node[right] at (4,3.5){$\text{DA}_1$};
	
	% Demanda efectiva 0
	\draw[dashed] (2,2) -- (2,0);
	\node[below] at (2,0){$y_0$};
	
	% Demanda efectiva 1
	\draw[dashed] (3,3) -- (3,0);
	\node[below] at (3,0){$y_1$};
	
	% Pleno empleo
	\draw[thick] (3.5,0) -- (3.5,4);
	\node[below] at (3.5,0){$\bar{y}$};
	
\end{axis}

Siendo $y_0$ y $y_1$ las demandas efectivas respectivas para dos demandas agregadas de diferente cuantía, y $\bar{y}$ la producción de pleno empleo.

\begin{axis}{4}{Ajuste dinámico de demanda y renta tras un aumento de la demanda de inversión}{Y}{Demanda}{multiplicador}
	% equilibrio
	\draw[-] (0,0) -- (4,4);
	\node[right] at (4,4){Y=DA};
	
	% demanda agregada 0
	\draw[-] (0,1) -- (4,2.5);
	\node[right] at (4,2.5){$\text{DA}_0$};
	
	% demanda agregada 1
	\draw[-] (0,2) -- (4,3.5);
	\node[right] at (4,3.5){$\text{DA}_1$};
	
	% senda de ajuste
	\draw[-{Latex}] (1.6,1.6) -- (1.6,2.6);
	
	\draw[decorate,decoration={brace,amplitude=3pt},xshift=-2pt,yshift=0pt] (1.6,1.63) -- (1.6,2.57) node[black,midway,xshift=-0.4cm] {\footnotesize $\Delta I$};
	
	\draw[-{latex}] (1.6,2.6) -- (2.6,2.6);
	\draw[-{Latex}] (2.6,2.6) -- (2.6,2.97);
	\draw[-{Latex}] (2.6,2.97) -- (2.97, 2.97);
	\draw[-{Latex}] (2.97,2.97) -- (2.97,3.12);
	
	% Output de equilibrio inicial
	\draw[dashed] (1.6,1.6) -- (1.6,0);
	\node[below] at (1.6,0){\small $Y_0$};
	
	% Output de equilibrio final
	\draw[dashed] (3.2,3.2) -- (3.2,0);
	\node[below] at (3.2,0){\small $Y_1$};
\end{axis}

\begin{axis}{4}{Representación gráfica de un ajuste del salario real vía inflación en el contexto del modelo keynesiano del mercado de trabajo.}{$L$}{$\frac{W}{P}$}{rigideznominal}
	% Demanda de trabajo
	\draw[-] (0.5,3.5) -- (3.5,0.5);
	\node[right] at (3.5,0.5){\small D};
	
	
	% Oferta de trabajo
	\draw[-] (0.5,0.5) -- (3.5,3.5);
	\node[right] at (3.5,3.5){\small S};
	
	% Salario inicial
	\draw[dashed] (0,3) -- (3,3);
	\node[left] at (0,3){\small $(W/P)_0$};
	
	% Salario tras inflación
	\draw[dashed] (0,2.5) -- (2.5,2.5);
	\node[left] at (0,2.5){\small $(W/P)_1$};
	
	% Variación del salario real
	\draw[-,-{Latex}] (0.5,3) -- (0.5,2.5);
	\draw[-,-{Latex}] (2,3) -- (2,2.5);
	
	% Salario de equilibrio
	\draw[dotted] (0,2) -- (2,2);
	\node[left] at (0,2){\small $(W/P)^*$};
\end{axis}


\begin{dibujo}{4}{Representación gráfica del modelo neoclásico en el bloque de oferta (mercado de trabajo), el de demanda (IS-LM) y el equilibrio (AS-AD).}{x}{y}{islmadasclasico}
	%%%%%%%%%%%%%%%%%%% TRABAJO
	
	% Ejes
	\draw[-] (-8,4) -- (-8,0) -- (-4,0);
	\node[left] at (-8,4){W/P};
	\node[below] at (-4,0){L};
	
	
	% Demanda de trabajo
	\draw[-] (-7.5,3.5) -- (-4.5,0.5);
	\node[right] at (-4.5,0.5){$Y'(L) = \frac{W}{P}$};
	
	% Oferta de trabajo
	\draw[-] (-7.5,0.5) -- (-4.5,3.5);
	\node[right] at (-4.5,3.5){$\frac{u_l}{u_c} = \frac{W}{P}$};

	% Equilibrio
	\draw[dashed] (-8,2) -- (-6,2) -- (-6,0);
	\node[left] at (-8,2){$(W/P)^*$};
	\node[below] at (-6,0){$L^*$};
	
	%%%%%%%%%%%%%%%%%%% IS-LM
	
	% Ejes
	\draw[-] (-2,4) -- (-2,0) -- (2,0);
	\node[left] at (-2,4){r};
	\node[below] at (2,0){Y};
	
	% IS 
	\draw[-] (-1.5,3.5) -- (1.5,0.5);
	\node[right] at (1.5,0.5){IS};
	
	% LM
	\draw[-] (0,0) -- (0,4);
	\node[right] at (0,3.5){LM};
	\node[below] at  (0,0){$Y(L^*)$};
	
	% Equilibrio
	
	%%%%%%%%%%%%%%%%%%% AS-AD
	
	% Ejes
	\draw[-] (4,4) -- (4,0) -- (8,0);
	\node[left] at (4,4){P};
	\node[below] at (8,0){Y};
	
	% AD
	\draw[-] (4.5,3.5) -- (7.5,0.5);
	\node[right] at (7.5,0.5){AD};
	
	% AS
	\draw[-] (6,0) -- (6,4);
	\node[right] at (6,3.5){AS};
	\node[below] at  (6,0){$Y(L^*)$};
	
	% Equilibrio
	
\end{dibujo}

\begin{dibujo}{4}{Representación gráfica del modelo keynesiano en el bloque de oferta (mercado de trabajo), el de demanda (IS-LM) y el equilibrio (AS-AD).}{x}{y}{islmadaskeynesiano}
	%%%%%%%%%%%%%%%%%%% TRABAJO
	
	% Ejes
	\draw[-] (-8,4) -- (-8,0) -- (-4,0);
	\node[left] at (-8,4){W/P};
	\node[below] at (-4,0){L};
	
	
	% Demanda de trabajo
	\draw[-] (-7.5,3.5) -- (-4.5,0.5);
	\node[right] at (-4.5,0.5){$Y'(L) = \frac{W}{P}$};
	
	% Oferta de trabajo
	\draw[-] (-7.5,0.5) -- (-4.5,3.5);
	\node[right] at (-4.5,3.5){$\frac{u_l}{u_c} = \frac{W}{P}$};
	
	% Equilibrio
	\draw[dashed] (-8,3) -- (-5,3);
	\draw[-] (-7,3) -- (-7,0);
	\node[below] at (-7,0) {$L^*$};
	
	% Desempleo
	\draw[decoration={brace,raise=9pt},decorate]
	(-7,3.05) -- node[above=12pt] {$\text{ED}<0$} (-5,3.05);	
	
	%%%%%%%%%%%%%%%%%%% IS-LM
	
	% Ejes
	\draw[-] (-2,4) -- (-2,0) -- (2,0);
	\node[left] at (-2,4){r};
	\node[below] at (2,0){Y};
	
	% LM
	\draw[-] (-1.3, 0.5) -- (0,0.5) to [out=0, in=260](2,4);
	\node[right] at (2,4){LM};
	
	% IS
	\draw[-] (-1.8,4) -- (1,0.2);
	\node[right] at (1,0.2){IS};
	
	% Equilibrio
	
	%%%%%%%%%%%%%%%%%%% AS-AD
	
	% Ejes
	\draw[-] (4,4) -- (4,0) -- (8,0);
	\node[left] at (4,4){P};
	\node[below] at (8,0){Y};
	
	% AD
	\draw[-] (4.5,3.5) -- (6.5,0.5);
	\node[right] at (7.5,0.5){AD};
	
	% AS
	%\draw[-] (4.5,0.5) -- (7.5,3.5);
	\draw[-] (4.5,0.5) to [out=15,in=260](7,1.5) -- (7,4);
	\node[right] at (7,4){AS};
	
	% Equilibrio
	
\end{dibujo}


\begin{axis}{4}{Modelo IS-LM con precios rígidos y trampa de liquidez.}{Y}{r}{islm}
	% LM
	\draw[-] (0.3, 0.5) -- (1,0.5) to [out=0, in=260](4,4);
	\node[right] at (4,4){LM};
	
	% IS
	\draw[-] (0.2,4) -- (3,0.2);
	\node[right] at (3,0.2){IS};
\end{axis}

\begin{axis}{4}{Modelo IS-LM con supuestos neoclásicos: el precio es flexible pero el nivel de producción es fijo.}{Y}{r}{islmneoclasico}
	% IS
	\draw[-] (0.2,4) -- (4,0.2);
	\node[right] at (4,0.2){IS};
	
	% LM
	\draw[-] (0.2,0) -- (4,4);
	\node[right] at (4,4){LM};
	
	% Y fijo
	\draw[dotted] (2.15,0) -- (2.15,4);
	\node[below] at (2.15,0){$\bar{Y}$};
\end{axis}

\begin{axis}{4}{Efectos de una expansión de la demanda en un modelo IS-LM con supuestos neoclásicos}{Y}{r}{islmneoclasicopoliticafiscal}
	% IS
	\draw[-] (0.2,4) -- (4,0.2);
	\node[right] at (4,0.2){$\text{IS}_0$};
	
	\draw[-] (1.5,4) -- (5.3,0.2);
	\node[right] at (5.3,0.2){$\text{IS}_1$};
	
	% flechas desplazamiento de IS
	\draw[-{Latex}] (1.3,3.1) -- (1.8,3.6);
	\draw[-{Latex}] (3.3,1.1) -- (3.8, 1.6);
	
	% LM
	\draw[-] (0.2,0) -- (4,4);
	\node[right] at (4,4){$\text{LM}_0$};
	
	\draw[-] (2.74,4) -- (0,1.12);
	\node[right] at (2.74,4){$\text{LM}_1$};
	
	% flechas desplazamiento de LM
	\draw[{Latex}-] (0.5,1.4) -- (1,1);
	\draw[{Latex}-] (1.3,2.2) -- (1.8,1.8);
	
	% Y fijo
	\draw[dotted] (2.15,0) -- (2.15,4);
	\node[below] at (2.15,0){$\bar{Y}$};
	
	% Y tras expansión monetaria
	\draw[dashed] (2.79,0) -- (2.79,2.73);
	\node[below] at (2.81,0){$Y'$};
	
	% Equilibrios
	\node[circle,fill=black,inner sep=0pt,minimum size=4pt] (a) at (2.16,2.08) {};
	\node[left] at (2.16,2.06){\tiny 0};
	
	\node[circle,fill=black,inner sep=0pt,minimum size=4pt] (a) at (2.15,3.35) {};
	\node[right] at (2.16,3.34){\tiny 1};
	
	\node[circle,fill=black,inner sep=0pt,minimum size=4pt] (a) at (2.79,2.71) {};
	\node[left] at (2.79,2.71){\tiny 0'};
	
\end{axis}

\begin{axis}{4}{Efectos de una expansión de la oferta monetaria en un modelo IS-LM con supuestos neoclásicos}{Y}{r}{islmneoclasicopoliticamonetaria}
	% IS
	\draw[-] (0.2,4) -- (4,0.2);
	\node[right] at (4,0.2){IS};
	
	% LM
	\draw[-] (0.2,0) -- (4,4);
	\node[right] at (4,4){$\text{LM}_0$, $\text{LM}_1$};
	
	\draw[-] (1.5,0) -- (4,2.6325);
	\node[right] at (4,2.6235){LM'};
	
	\draw[-{Latex}] (1,0.6) -- (1.5,0.2);
	\draw[{Latex}-] (2.7,2.3) -- (3.2,1.9);
	
	% Y fijo
	\draw[dotted] (2.15,0) -- (2.15,4);
	\node[below] at (2.15,0){$\bar{Y}$};
	
	% Y tras expansión monetaria
	\draw[dashed] (2.81,0) -- (2.81,1.35);
	\node[below] at (2.81,0){$Y'$};
	
	% Equilibrios
	\node[circle,fill=black,inner sep=0pt,minimum size=4pt] (a) at (2.16,2.08) {};
	\node[left] at (2.10,2.08){\tiny 0, 1};
	
	\node[circle,fill=black,inner sep=0pt,minimum size=4pt] (a) at (2.8,1.4) {};
	\node[left] at (2.8,1.4){\tiny 0'};
\end{axis}

Cuando la oferta monetaria se expande, la curva $\text{LM}_0$ se desplaza hasta $\text{LM}'$. El exceso de demanda generado aumenta los precios y desplaza, de nuevo, la curva LM pero a la izquierda en este caso hasta volver a alcanzar el equilibrio inicial con la producción determinada exógenamente.

\begin{axis}{4}{Modelo IS-LM con equilibrio en trampa de liquidez}{x}{y}{islmtrampadeliquidez}
	% LM
	\draw[-] (0.3, 0.5) -- (3,0.5) to [out=0, in=260](6,4);
	\node[right] at (6,4){LM};
	
	% IS
	\draw[-] (0.2,4) -- (3,0.2);
	\node[right] at (3,0.2){IS};
\end{axis}

\begin{axis}{4}{Efectos de política fiscal expansiva en un modelo IS-LM con supuestos keynesianos.}{Y}{r}{islmkeynesianopoliticafiscal}
	% LM
	\draw[-] (0.3, 0.5) -- (1,0.5) to [out=0, in=260](4,4);
	\node[right] at (4,4){LM};
	
	% IS
	\draw[-] (0.2,4) -- (3,0.2);
	\node[right] at (3,0.2){$\text{IS}_0$};
	
	\draw[-] (1.2,4) -- (4,0.2);
	\node[right] at (4,0.2){$\text{IS}_1$};
	
	\draw[-{Latex}] (1.25,2.75) -- (1.7,3.2);
	\draw[-{Latex}] (2.1,1.5) -- (2.55,1.95);
	
	% Equilibrios
	\node[circle,fill=black,inner sep=0pt,minimum size=4pt] (a) at (2.44,0.98) {};
	\node[above] at (2.44, 0.98){\tiny 0};
	
	\node[circle,fill=black,inner sep=0pt,minimum size=4pt] (a) at (3,1.57) {};
	\node[above] at (3, 1.57){\tiny 1};
\end{axis}

\begin{axis}{4}{Efectos de política monetaria expansiva en un modelo IS-LM con supuestos keynesianos.}{Y}{r}{islmkeynesianopoliticamonetaria}
	% LM
	\draw[-] (0.3, 0.5) -- (1,0.5) to [out=0, in=260](4,4);
	\node[left] at (4,4){$\text{LM}_0$};
	
	\draw[-] (1,0.5) -- (1.7,0.5) to [out=0, in=260](4.7,4);
	\node[right] at (4.7,4){$\text{LM}_1$};
	
	\draw[-{Latex}] (4.05,3.7) -- (4.6,3.7);
	\draw[-{Latex}] (3.8,3) -- (4.4,3);
	
	% IS
	\draw[-] (0.2,4) -- (3,0.2);
	\node[right] at (3,0.2){IS};
	
	% Equilibrios
	
	\node[circle,fill=black,inner sep=0pt,minimum size=4pt] (a) at (2.44,0.98) {};
	\node[above] at (2.44, 0.98){\tiny 0};
	
	\node[circle,fill=black,inner sep=0pt,minimum size=4pt] (a) at (2.64,0.71) {};
	\node[below] at (2.64,0.67){\tiny 1};
\end{axis}


\begin{axis}{4}{Modelo de Oferta Agregada-Demanda Agregada (AS-AD) con supuestos keynesianos: curva de oferta agregada creciente}{Y}{P}{asadkeynes}
	% AD
	\draw[-] (0.2,4) -- (4,0.3);
	\node[right] at (4,0.3){AD};
	
	% AS
	\draw[-] (0.2,0.3) to [out=10, in=260] (4,4);
	\node[right] at (4,4){AS};
\end{axis}

\begin{axis}{4}{Modelo de Oferta Agregada-Demanda Agregada (AS-AD) con supuestos neoclásicos: curva de oferta agregada vertical.}{Y}{P}{asadneoclasico}
	% AD
	\draw[-] (0.2,4) -- (4,0.3);
	\node[right] at (4,0.3){AD};
	
	% AS
	\draw[-] (3,0.3) -- (3,4);
	\node[above] at (3,4){AS};
\end{axis}

\begin{axis}{4}{Ejemplo de curva de Phillips similar a la estimada por Phillips en 1958.}{}{$\dot{w}$}{curvadephillips}
	\node[below] at (6,0){$u$};
	\draw[-] (0,0) -- (-2,0);
	\draw[-] (0,0) -- (0,-2);
	\draw[-] (4,0) -- (6,0);
	
	\draw[-] (1,4) to [out=280, in= 175](6,-1);
\end{axis}

\conceptos

\concepto{Punto de nivelación}

En el modelo de la demanda efectiva, el punto de nivelación caracteriza el equilibrio entre renta y demanda agregada. Así, en un aspa keynesiana, el punto de nivelación corresponde con el nivel de output en el que interseccionan las dos rectas.

\concepto{Racionamiento}

Cuando en un mercado dado existe un exceso de demanda o de oferta, es habitual afirmar que existe \textit{racionamiento}. Gráficamente y en el espacio cantidad-precio, el concepto de racionamiento puede ilustrarse trazando una línea horizontal desde un punto arbitrario del eje de ordenadas (que representa el precio). La intersección de esta línea con las curvas de oferta y demanda determina quién sufre el racionamiento. Si la recta intersecciona con la curva de demanda antes que con la curva de oferta, la demanda se encuentra \textit{racionada} o es el \textit{lado corto} y existe un exceso de oferta para ese precio. En caso contrario, la oferta se encuentra racionada o es el lado corto y existe un exceso de demanda.


\preguntas

\seccion{Test 2018}

\textbf{1.} Entre los autores más destacados de la Síntesis Neoclásica puede citarse a:

\begin{itemize}
	\item[a] Robert Lucas
	\item[b] Neil Wallace
	\item[c] George Akerlof
	\item[d] Paul Samuelson
\end{itemize}

\seccion{Test 2016}

\textbf{22.} Suponga una economía cerrada descrita por el modelo IS-LM en la que, a partir de una situación inicial de equilibrio, el Gobierno desea reducir el consumo y aumentar la inversión sin que varíe la renta de equilibrio. Señale la combinación adecuada de políticas:

\begin{enumerate}
	\item[a] Reducción del coeficiente de reserva y aumento del tipo impositivo.
	\item[b] Compra de bonos en el mercado abierto y reducción del gasto público.
	\item[c] Venta de bonos en el mercado abierto y aumento del gasto público.
	\item[d] Incremento del coeficiente de reserva y reducción del tipo impositivo.
\end{enumerate}

\textbf{23.} En una economía cerrada definida por el modelo de Oferta y Demanda agregadas y partiendo de un equilibrio inicial a corto plazo con sobre-empleo $Y > Y_n$, el Banco Central decide realizar una política monetaria:

\textit{Consideraciones y nomenclatura: salvo nota en contrario se asume que:}

\begin{itemize}
	\item La economía es cerrada
	\item Los impuestos son fijos
	\item Las curvas IS-LM, DA-OA tienen las pendientes habituales según la síntesis neoclásica. No existe trampa de liquidez.
	\item Existe una cierta flexibilidad de los salarios nominales ante el desempleo.
	\item $Y_t$: producción en periodo t.
	\item $Y_n$: producción natural.
	\item $u_n$: desempleo natural.
	\item $P^e$: precios esperados.
	\item $P_t$: precios efectivos en t.
	\item Los salarios no están totalmente indiciados a los precios y la revisión de los salarios se produce anualmente mediante negociación colectiva entre sindicatos y patronal, con miopía en la formación de expectativas de precios $P^e=P_{t-1}$.
\end{itemize}

\begin{enumerate}
	\item[a] Deberá realizar una política monetaria expansiva para acelerar el proceso de ajuste automático a medio plazo.
	\item[b] Al cambiar la oferta de dinero, la Oferta Agregada se verá afectada, alcanzándose el nivel de renta natural en un sólo periodo.
	\item[c] Debería realizarse una política monetaria restrictiva para frenar la actividad económica y evitar que suban más los precios durante el proceso de ajuste automático.
	\item[d] No tiene sentido utilizar la política monetaria porque es ineficaz.
\end{enumerate}

\seccion{Test 2014}
\textbf{16.} A corto plazo, el modelo de Keynes

\begin{enumerate}
	\item[a] Tiene la oferta agregada vertical.
	\item[b] Supone salarios reales rígidos.
	\item[c] Provoca salarios reales anti-cíclicos, a pesar de la evidencia acumulada.
	\item[d] Genera desempleo involuntario cuando el salario real es menor al de equilibrio en el mercado de trabajo.
\end{enumerate}

\textbf{17.} En el modelo keynesiano si la curva de oferta de capital no es perfectamente elástica, una variación discreta que dé lugar a un aumento del stock deseado de capital:
\begin{enumerate}
	\item[a] Provocará el descenso de la tasa de reposición de la inversión.
	\item[b] Elevará el precio de los bienes de capital.
	\item[c] Reducirá el precio de los bienes de capital.
	\item[d] Reducirá la relación capital-producto.
\end{enumerate}

\textbf{18.} En un modelo keynesiano, a la derecha del punto de nivelación de una economía:

\begin{enumerate}
	\item[a] La economía está en una situación de desahorro.
	\item[b] Las familias ahorran más de lo que las empresas invierten.
	\item[c] El nivel de consumo es superior a la renta.
	\item[d] Se producen tensiones inflacionistas.
\end{enumerate}

\textbf{19.} El efecto Fisher muestra que el cambio en la inflación que tiene su origen en un cambio en el crecimiento de la cantidad de dinero se traduce en:

\begin{enumerate}
	\item[a] Una variación del tipo de interés nominal en igual proporción.
	\item[b] Una variación del tipo de interés real en igual proporción.
	\item[c] Una variación del tipo de interés nominal en mayor proporción.
	\item[d] Una variación del tipo de interés real en menor proporción.
\end{enumerate}

\seccion{Test 2009}

\textbf{1.} Indica cual de las siguientes afirmaciones acerca de los postulados keynesianos es \textbf{CORRECTA}:

\begin{enumerate}
	\item[a] La reducción de los salarios reales constituye la única medida para mantener el empleo.
	\item[b] En presencia de una tasa de paro elevada conviene estimular una moderada inflación mediante un aumento del dinero en circulación.
	\item[c] Las autoridades monetarias han de mantener los tipos de interés en niveles superiores a la eficiencia marginal del capital.
	\item[d] Todas las anteriores.
\end{enumerate}

\seccion{Test 2008}

\textbf{14.} Es \underline{falso} que, en la teoría Keynesiana original, y descartando cualquier posible efecto riqueza, las condiciones que justifican un equilibrio con desempleo persistente son:

\begin{enumerate}
	\item[a] \comillas{La trampa de la liquidez}.
	\item[b] La baja elasticidad de la inversión con respecto al tipo de interés.
	\item[c] Los salarios monetarios rígidos.
	\item[d] La demanda especulativa de dinero.
\end{enumerate}

\seccion{Test 2005}

\textbf{1.} ¿Qué sucede con el tipo de interés en el modelo de Keynes cuando aumento la oferta monetaria?

\begin{enumerate}
	\item[a] El tipo de interés continúa a la baja, siempre que aumenta la cantidad de dinero.
	\item[b] El tipo de interés se reduce hasta un determinado nivel y ya no baja más.
	\item[c] Los tipos de interés a corto y largo siguen sendas divergentes hasta que el incremento en la oferta monetaria dobla, aproximadamente, la cantidad de dinero existente al inicio.
	\item[d] El modelo keynesiano no contempla esta hipótesis, propia del monetarismo. 
\end{enumerate}

\textbf{14.} Suponga una economía cerrada. Un gobierno desea estimular la producción aumentando la cantidad de dinero en circulación. Logra su objetivo: 

\begin{enumerate}
	\item[a] Si todos los precios son flexibles.
	\item[b] Si el salario real es rígido y constante.
	\item[c] Si el salario nominal es rígido. 
	\item[d] En ningún caso.
\end{enumerate}

\seccion{4 de abril de 2017}
\begin{itemize}
    \item ¿Quiénes son los clásicos a los que critica Keynes?
    \item ¿Qué otros libros escribió Keynes?
    \item ¿Quién defendía otra visión de la Gran Depresión?
    \item ¿Qué decían los neokeynesianos del desequilibrio?
    \item ¿Quién resucita la equivalencia ricardiana? ¿por que?
    \item ¿Podría escribir los multiplicadores?
    \item ¿Hay alguna corriente de pensamiento económica que en PF hable de estabilizador automático?
\end{itemize}

\notas

\textbf{2018:} \textbf{1.} D

\textbf{2016:} \textbf{22.} A. Reducción del coeficiente de reserva para aumentar la oferta monetaria y aumento de los impuestos para reducir el consumo. \textbf{23.} C

\textbf{2014}: \textbf{16.} C \textbf{17.} B \textbf{18.} B \textbf{19.} A

\textbf{2009}: \textbf{1.} B

\textbf{2008}: \textbf{14.} D

\textbf{2005}: \textbf{1.} B \textbf{14.} C

\bibliografia

Mirar en Palgrave
\begin{itemize}
	\item cost-push inflation
	\item demand-pull inflation
	\item effective demand
	\item functional finance
	\item IS-LM
	\item Keynes, John Maynard *
	\item Keynes, John Maynard (new perspectives) *
	\item Keynesian revolution
	\item keynesianism
	\item liquidity preference
	\item liquidity trap
	\item macroeconomics, origins and history of
	\item neoclassical synthesis
	\item Patinkin, Don
	\item Phillips curve
	\item Phillips curve (new views)
	\item Samuelson, Paul Anthony
	\item Tobin, James
	\item underemployment equilibria
	\item unemployment
	\item wage curve
	\item Walras' law
\end{itemize}



Blaug, M. \textit{Economic Theory in Retrospect} (1997) 5th edition - En carpeta \textit{Historia del Pensamiento Económico}

De Vroey, M. \textit{A History of Macroeconomics. From Keynes to Lucas and Beyond}

Dutt, A. K. \textit{Aggregate Demand–Aggregate Supply Analysis: A History} (2002) History of Political Economy -- En carpeta del tema

Forder, J. \text{Nine Views of the Phillips Curve: Eight Authentic and One Inauthentic} \url{https://papers.ssrn.com/sol3/papers.cfm?abstract_id=2502145}

Gordon, R. J. (2008) \textit{The History of the Phillips Curve: Consensus and Bifurcation} Economica -- En carpeta del tema

Hicks, J. R. \textit{Mr. Keynes and the ``Classics'': A suggested interpretation} (1937) Econometrica -- En carpeta del tema

History of Economic Thought Encyclopedia. \textit{A. William Phillips, 1914-1975}. \url{http://www.hetwebsite.net/het/profiles/phillips.htm}

Levacic, R.; Rebmann, A. \textit{Macroeconomics. An Introduction to Keynesian-Neoclassical Controversies} (1982) 2nd Edition -- En carpeta Macroeconomía

Rojas, R. \textit{The Keynesian Model in the General Theory: A Tutorial} (2012) \url{https://arxiv.org/pdf/1708.07509.pdf} -- En carpeta del tema

Samuelson, P.; Solow, R. \textit{Analytical Aspects of Anti-Inflation Policy} (1960) The American Economic Review -- En carpeta del tema

Yamagata, H. \textit{The Neo-Keynesian World} \url{https://cruel.org/econthought/essays/keynes/islmcont.html}

Screpanti, E; Zamagni, S. \textit{An Outline of the History of Economic Thought} (2005) -- En carpeta \textit{Historia del Pensamiento Económico}

Snowdon, B.; Vane, H. R. \textit{Modern Macroeconomics. Its Origins, Development and Current State} (2005) Edward Elgar Publishing --  En carpeta Macroeconomía

Wells, P. (1990) \textit{Keynes's General Theory Critique of the Neoclassical Theories of Employment and Aggregate Demand} Faculty Working Paper No. 90-1680 -- En carpeta del tema

\end{document}
