\documentclass{nuevotema}

\tema{3A-10}
\titulo{Teorías de los costes. Análisis de dualidad.}

\begin{document}

\ideaclave

VER WANG Y YANG (2001) sobre costes fijos y hundidos, en carpeta del tema.


Si la ciencia economía pretende entender y predecir el comportamiento humano que implica gestión de recursos escasos, la microeconomía es una rama de aquella que trata de entender y predecir las decisiones de agentes individuales. Para ello, la microeconomía formula modelos abstractos que simplifican la extraordinaria complejidad del comportamiento humano, reduciéndola a los aspectos imprescindibles para explicar un fenómeno dado. En el tema que nos ocupa, el agente económico cuyo comportamiento se pretende describir son las empresas, entendidas como entidades con unidad de decisión en relación a una transformación de unos bienes en otros. 

Para formular esas simplificaciones conocidas como modelos, la microeconomía utiliza una serie de herramientas matemáticas tales como el cálculo diferencial, el análisis real, la optimización convexa o los métodos numéricos de aproximación. La aplicabilidad de estas herramientas no se extiende sólo a la modelización puramente teórica del comportamiento de las empresas, sino también al análisis empírico y a la contrastación de las predicciones de los modelos teóricos. La presente exposición tiene por \textbf{objeto} presentar el análisis de costes en tanto que modelización en sí misma del comportamiento y los incentivos de las empresas, pero también en tanto que herramienta con la que superar los supuestos necesarios para analizar el comportamiento de la empresa directamente a partir de una función de producción. El nexo entre el estudio del comportamiento por medio de un problema de maximización de beneficios y el estudio a partir de un problema de minimización de costes, viene dado por la teoría de dualidad. La dualidad es un resultado relativo a determinados problemas de optimización consistente en la existencia de otro problema de optimización cuya solución guarda una relación determinada con la solución del primero. Cuando un problema de optimización dado (llamémosle problema primal) cumple una serie de supuestos, su solución es idéntica a la de otro problema (denominado problema dual) que comparte algunos parámetros con el otro problema primal. Esta identidad permite derivar conclusiones sobre un problema determinado sin tener que explicitar determinados supuestos que en ocasiones se revelan prohibitivos, concretamente la especificación de las funciones de producción.

La exposición comienza con el análisis de costes habitual en el \marcar{análisis neoclásico}. Los pilares básicos son la consideración de la empresa como una caja negra cuya única decisión consiste en transformar inputs en outputs sin atender a cómo se lleva a cabo tal proceso, la ausencia de ilusiones monetarias y las herramientas matemáticas habituales en microeconomía. Partiendo de esos tres puntos, es posible \textbf{formular} los problemas primal y dual. El primal consistente en la maximización de beneficio es el verdadero supuesto articulador del comportamiento de las empresas. El problema dual es el problema de minimización del coste de un nivel dado de producción cuya solución es idéntica --bajo determinados supuestos generalmente asumidos en microeconomía- a la del problema primal. De éstos dos problemas puede derivarse un tercero: el problema de maximización de beneficios directamente en función de cantidades dadas de producto, introduciendo la función de coste mínimo dados output y precios de inputs derivado del problema de minimización. A su vez, de estos problemas de optimización se derivan una serie de importantes resultados que forman el núcleo del análisis microeconómico de la empresa. 

La enumeración de estas \textbf{implicaciones} comienza por la función (o correspondencia) de demanda de factores, que refleja la cantidad de factores que demanda una empresa cuando maximiza el beneficio. La función (o correspondencia) de demanda condicionada de factores captura la demanda de inputs de una empresa que minimiza el coste de producir una cantidad dada de output. La senda de expansión está íntimamente relacionada con la demanda de factores: se trata del segmento que une las demandas óptimas de input para una serie de niveles de output. La función de beneficio y la función de costes reflejan el óptimo que alcanza la función objetivo de los problemas primal y dual, respectivamente, cuando varían los parámetros (precios de output, costes de inputs y cantidad de output en el caso del problema dual). Las funciones de coste medio y coste marginal se derivan de la función de costes y son, respectivamente, el coste medio por unidad de output y el coste asociado a la producción de una unidad adicional de output. La relación entre el valor del coste marginal y el coste medio es tal que la cantidad que minimiza el coste medio iguala coste medio con coste marginal. El concepto de economía de escala captura la relación entre el coste medio y la escala de la producción. Así, existen economías de escala cuando un aumento del output induce un aumento menor que proporcional del coste de producción, y existen deseconomías de escala cuando sucede al contrario: aumentos de la producción inducen aumentos proporcionalmente mayores del coste. La relación entre las economías de escala y los rendimientos a escala es relevante pero en ocasiones confusa, y se explicitan las implicaciones recíprocas de uno y otro fenómeno. Por último, se presenta en este apartado la función de oferta, posiblemente el resultado más importante de la teoría de costes. La función de oferta relaciona el precio que prevalece en el mercado del output y la producción que la empresa decide llevar a cabo. Esta relación se deriva de forma relativamente sencilla del problema de maximización, y su forma concreta se ve determinada fundamentalmente por la forma de la función de producción o equivalentemente, por la función de costes. El análisis de costes del modelo neoclásico presentado tiene por objetivo ultimo insertarse en modelos más complejos del comportamiento empresarial, y para ello se enumeran algunas de las aplicaciones teóricas y empíricas de éste.

Si bien el análisis neoclásico de los costes presentado en la primera parte de la exposición tiene importantes aplicaciones en sí mismo, algunas extensiones y modificaciones permiten aumentar su alcance. En la segunda parte de la exposición se presentan brevemente tres de estas \marcar{extensiones}: el análisis de corto y largo plazo, el análisis dinámico de costes, y el análisis de decisiones de producción entre múltiples outputs. El \textbf{primero} se caracteriza matemáticamente por la introducción de restricciones adicionales sobre algunos variables al problema de minimización cuando se trata de caracterizar el corto plazo. Estas restricciones representan el hecho de que en el corto plazo, algunos factores de producción tienen ofertas o demandas inelásticas y por ello, su cantidad se considera fija por la empresa maximizadora. El análisis de corto y largo plazo se fundamenta en las dificultades que existen en la práctica para sustituir determinados inputs por otros, ya sea por restricciones tecnológicas o por restricciones del mercado en forma de ofertas de inputs inelásticas. El enfoque dinámico de los costes examina las decisiones de la empresa cuando se trata de caracterizar la evolución de su comportamiento a lo largo de un periodo de tiempo determinado, ya sea de carácter discreto o continuo. La existencia de los llamados costes de ajuste en los que las empresas incurren cuando modifican sus demandas de input introduce un grado de complejidad adicional a la modelización que se menciona brevemente y en líneas muy generales, dada la necesidad de introducir técnicas matemáticas adicionales de optimización dinámica. Por último, se menciona la posibilidad de extender el análisis de costes a contextos multiproducto, en los que las empresas deben decidir qué cantidades de input demandar y cómo distribuirlas entre la producción de diferentes cantidades de output. El fenómeno de las economías de alcance se presenta en contextos multiproducto cuando la producción de varios outputs diferentes disminuye el coste medio de cada output, en contraposición con un contexto en el que la producción de varios outputs incrementa el coste medio --y en cuyo caso, hablaríamos de deseconomías de alcance.

Por último, se examinan dos anomalías notables que la realidad contrapone al análisis neoclásico de los costes y sus supuestos habituales. La teoría de los costes formulada por Stigler en 1939 trata de caracterizar el fenómeno de los excesos de capacidad que las empresas tienden a instalar para cubrirse ante variaciones de la demanda y posibles inelasticidades de oferta de inputs fijos. El concepto de ineficiencia X introducido por Leibenstein recoge y trata de explicar los motivos por los que las empresas no minimizan tanto como podrían los costes de su producción. Se trata sin embargo de un concepto que en cierto modo viola el marco neoclásico de la modelización de la empresa, porque se sumerge en el análisis de las particularidades del proceso de transformación de inputs en outputs.

Para concluir, la exposición se detiene muy brevemente en la situación actual de la teoría de costes y el efecto que tiene sobre su aplicabilidad la aparición de tecnologías que permiten recogidas masivas de datos de costes, precios y decisiones de producción. Estos ingentes volúmenes de datos que empiezan no sólo a ser recogidos, sino también a ser tratados por medio de nuevas herramientas matemáticas, empujan a la ciencia económica no sólo a reformular algunos de los supuestos en que basa sus modelos teóricos, sino también a contrastarlos, mejorarlos y rechazarlos o confirmarlos.


\seccion{Preguntas clave}
\begin{itemize}
	\item ¿Cómo se modelizan los costes en microeconomía?
	\item ¿Qué relación con otras áreas de la microeconomía?
	\item ¿Para qué sirve la modelización de los costes?
	\item ¿Qué anomalías sufre la modelización microeconómica tradicional?
	\item ¿Qué teorías alternativas?
\end{itemize}

\esquemacorto

\begin{esquema}[enumerate]
	\1[] \marcar{Introducción}
		\2 Contextualización
			\3 Microeconomía
			\3 Decisión de producción
			\3 Maximización de beneficios
			\3 Teoría de los costes y dualidad
		\2 Objeto
			\3 Modelizar decisiones de empresas
			\3 Cómo se modelizan los costes
			\3 Qué anomalías respecto a la teoría micro tradicional
		\2 Estructura
			\3 Teoría de costes del modelo neoclásico
			\3 Extensiones
			\3 Anomalías y alternativas
	\1 \marcar{Teoría de costes del modelo neoclásico}
		\2 Idea clave
			\3 Contexto
			\3 Objetivo
			\3 Resultados
		\2 Formulación
			\3[Primal] Problema de maximización de beneficios
			\3[Dual] Problema de minimización de costes
			\3 Maximizar beneficios respecto a output
		\2 Implicaciones
			\3 \underline{Demanda incondicional de factores}
			\3 \underline{Demanda condicionada de factores}
			\3 Senda de expansión
			\3 \underline{Función de beneficio}
			\3 \underline{Función de costes}
			\3 \underline{Función de oferta}
			\3 Coste medio
			\3 Coste marginal
			\3 Relación coste medio--coste marginal
			\3 Economías y deseconomías de escala
			\3 Elasticidad de sustitución
			\3 Integrabilidad
			\3 Leyes de la demanda derivada de Hicks-Marshall
		\2 Aplicaciones
			\3 Modelización teórica
			\3 Análisis empírica
	\1 \marcar{Extensiones del modelo neoclásico}
		\2 Corto y largo plazo
			\3 Idea clave
			\3 Costes fijos y hundidos
			\3 Corto plazo
			\3 Largo plazo
			\3 Representación gráfica
		\2 Enfoque dinámico
			\3 Idea clave
			\3 Sin costes de ajuste
			\3 Costes de ajuste
		\2 Múltiples outputs
			\3 Idea clave
			\3 Economías de alcance
	\1 \marcar{Anomalías y críticas}
		\2 Teoría de los costes de Stigler
			\3 Idea clave
		\2 Ineficiencia X de Leibenstein
			\3 Idea clave
	\1[] \marcar{Conclusión}
		\2 Recapitulación
			\3 Teoría neoclásica de costes
			\3 Extensiones
			\3 Anomalías y críticas
		\2 Idea final
			\3 Análisis empírico
			\3 Avances en recogida de datos

\end{esquema}

\esquemalargo














\begin{esquemal}
	\1[] \marcar{Introducción}
		\2 Contextualización
			\3 Microeconomía
				\4 Explicar y predecir comportamiento de agentes
				\4 Objetivo de las empresas:
				\4[$\to$] Transformar bienes i.e. producción
				\4[$\to$] Maximizando beneficios
			\3 Decisión de producción
				\4 Factores tecnológicos
				\4[] $\to$ Qué se puede producir
				\4 Factores económicos:
				\4[] Dependen de fact. tecnológicos + precios
				\4[] $\to$ Qué ingreso se puede obtener
				\4[] $\to$ Qué coste tiene producir
				\4[] $\to$ \textit{Qué beneficio se puede obtener}
			\3 Maximización de beneficios
				\4 Piedra angular del comportamiento de las empresas
				\4 Inputs y outputs tal que:
				\4[] Máxima diferencia entre ingresos y costes
				\4[$\to$] Necesario conocer de f. producción
			\3 Teoría de los costes y dualidad
				\4 Dados supuestos razonables
				\4 Max. beneficios respecto a inputs
				\4[] implica minimizar costes dado output
				\4[] $\to$ Primal y dual tienen misma solución
				\4 Problema dual $\to$ función de costes
				\4 Función de producción
				\4[] $\to$ no es directamente observable
				\4[$\to$] F. de costes sí
				\4 Función de costes permite:
				\4[$\to$] Estudiar comportamiento a partir de observables
				\4[] Precios de outputs, costes de inputs
				\4[$\to$] Mejorar tratabilidad análisis teórico
				\4[$\to$] Propone contrastes empíricos
		\2 Objeto
			\3 Modelizar decisiones de empresas
				\4 A partir de costes observables
			\3 Cómo se modelizan los costes
				\4 Para qué sirve su modelización
			\3 Qué anomalías respecto a la teoría micro tradicional
				\4 Qué teorías alternativas
		\2 Estructura
			\3 Teoría de costes del modelo neoclásico
			\3 Extensiones
			\3 Anomalías y alternativas
	\1 \marcar{Teoría de costes del modelo neoclásico}
		\2 Idea clave
			\3 Contexto
				\4 Economía neoclásica
				\4[] Individualismo metodológico
				\4[] Decisión racional

				\4 Comportamiento empresarial en marco neoclásico
				\4[] Empresas como cajas negras
				\4[] Convierten input en output
				\4 Maximización de beneficios
				\4[] Piedra angular de comportamiento empresarial
				\4[] Obtener máximos beneficios dados
				\4[] $\to$ Precio de output
				\4[] $\to$ Coste de input
				\4[] Supuesto teórico del comportamiento empresarial
				\4[] $\to$ Realmente, imposible conocer función de producción
				\4 Estimación empírica de comportamiento empresarial
				\4[] Función de producción es concepto teórico
				\4[] Otras variables sí son conocibles
				\4[] $\to$ Cuánto produce una empresa
				\4[] $\to$ Cuánto le cuestan sus inputs
				\4[] $\to$ A qué precio los vende
				\4 Herramientas matemáticas
				\4[] Optimización convexa
				\4[] Teorema Dualidad
				\4[] Establece condiciones bajo las cuales
				\4[] $\to$ Dual y primal tienen misma solución
				\4[] $\then$ No se plantea explícitamente
				\4 Problema dual
				\4[] Mismo óptimo que problema primal
				\4[] Diferente función objetivo y restricción
				\4[] Aplicación a comportamiento empresarial
				\4[] $\to$ Óptimo de problema de maximización de bfcios.
				\4[] $\to$ = óptimo que problema de minimización de costes
			\3 Objetivo
				\4 Entender y predecir decisión de empresas
				\4 Cuánto producen
				\4 Qué inputs utilizan
				\4 A partir de información disponible
				\4[] Precios de inputs
				\4[] Cantidad de output producido
				\4 Estimar información no disponible
				\4[] Función de producción
			\3 Resultados
				\4 Conceptos teóricos que caracterizan decisión
				\4[] Función de demanda
				\4[] $\to$ Incondicional
				\4[] $\to$ Condicional
				\4[] Función de beneficios
				\4[] Función de costes
				\4 Condiciones de integrabilidad
				\4[] Cuándo es posible estimar f. de prod.
				\4[] $\to$ A partir de función de costes
		\2 Formulación
			\3[Primal] Problema de maximización de beneficios
				\4[] $\underset{y, \vec{x}}{\max} \quad \pi(y, \vec{x}) = py - \vec{w} \vec{x} $
				\4[] $s.a: \quad f(\vec{x}) \geq y$
				\4 \underline{Existencia y unicidad}
				\4 Rendimientos crecientes a escala
				\4[] No existe máximo
				\4[] Siempre posible aumentar inputs y outputs
				\4[] $\to$ Y alcanzar isobeneficio superior
				\4[] \grafica{maximizacionrcreescala}
				\4 Rendimientos constantes a escala
				\4[] Posible:
				\4[] -- Existen infinitos máximos
				\4[] $\to$ Si precio mayor a coste de input
				\4[] $\then$ Siempre posible alcanzar mayor isobeneficio
				\4[] -- Existe un sólo máximo
				\4[] $\to$ Si precio menor a coste de input
				\4[] $\then$ Isobeneficio con input y output = 0
				\4[] -- No existe máximo
				\4[] $\to$ Si precio y coste de input iguales
				\4[] \grafica{maximizacionrconescala}
				\4 Rendimientos decrecientes a escala
				\4[] \grafica{maximizacionrdecescala}
			\3[Dual] Problema de minimización de costes
				\4[] $\underset{\vec{x}}{\min} \quad \vec{w} \vec{x}$
				\4[] $s.a: \quad f(\vec{x}) \geq y$
				\4 \underline{Existencia y unicidad}
				\4 $\to$ $V(y) = \left\lbrace \vec{x}: f(\vec{x}) \geq y \right\rbrace$ es compacto
				\4[] $\Rightarrow$ Siempre existe mínimo de función (Weierstrass)
				\4[] Mínimo único si $f(\vec{x})$ estrictamente cuasicóncava
				\4[] \grafica{minimizacion}
%				\4[] \hyperref[fig:minimizacion]{Gráfica}
			\3 Maximizar beneficios respecto a output
				\4 Dado que máx. beneficios implica min. costes
				\4[] $py - \vec{w} \vec{x} = py - \vec{w} \vec{z} (\vec{w}, y) = py - c(\vec{w}, y)$
				\4[$\to$] Posible reformular problema en términos de output
				\4[] $\underset{y}{\max} \quad p y - c(y)$
				\4[] \grafica{maximizarconcostes}
				\4[] \textit{Rendimientos crecientes a escala}
				\4[] $\to$ No existe máximo
				\4[] \textit{Rendimientos constantes a escala}
				\4[] $\to$ No tiene por qué existir máximo:
				\4[] $\to$ Si $p > \text{CMe}$ no existe máximo
				\4[] $\to$ Si $p = \text{CMe}$ existen infinitos máximos
				\4[] $\to$ Si $p < \text{CMe}$ existe 1 máximo: $y=0$
				\4[] \textit{Rendimientos decrecientes a escala}
				\4[] $\to$ Existe máximo y -es único
				\4[] $\to$ Si $P<\text{CMe}(y_{EC})$, máximo en $y=0$
				\4[] $\to$ Si $P=\text{CME}(y_{EC})$, máximo en $y=y_EC$ y $y=0$
				\4[] $\to$ Si $P>\text{CME}(y_{EC})$, máximo en $y: P=\text{CMg}(y)$
		\2 Implicaciones
			\3 \underline{Demanda incondicional de factores}
				\4 Del problema de maximización
				\4 $x(p, \vec{w}) = \left\lbrace \arg \underset{\vec{x}}{\max} \; \; py - \vec{w} \vec{x} \quad s.a.: f(\vec{x}) \geq y \right\rbrace$
				\4 Necesario que exista máximo en problema de maximización
				\4 Asumiendo que existe, siguientes propiedades
				\4 {Propiedades}
				\4[(i)] \textit{Continua en $p$, $\vec{w}$}
				\4[(ii)] No decreciente en $p$
				\4[(iii)] No creciente en $\vec{w}$
				\4[(iv)] Homogénea de grado 0 en $(p, \vec{w})$
			\3 \underline{Demanda condicionada de factores}
				\4 Del problema de minimización
				\4 $ \vec{z}(\vec{w}, y) = \left\lbrace \arg \underset{\vec{x}}{\min} \; \; \vec{w} \vec{x} \quad s.a: f(\vec{x}) \geq y \right\rbrace$
				\4 {Propiedades}
				\4[(i)] \textit{Convexidad de $\vec{z} (\vec{w}, y)$}
				\4[] Si f. prod. es cuasicóncava
				\4[(i')] F. prod. estrictamente cuasicóncava
				\4[] $\Rightarrow$ $z(\vec{w}, y)$ es una función.\footnote{Y no una correspondencia. Ver pág. 141 MWG.}
				\4[(ii)] \textit{Homogeneidad de grado 0 en $\vec{w}$} de $z(\vec{w},y)$
				\4[(iii)] Creciente en $y$
				\4[(iv)] Decreciente en $\vec{w}$
			\3 Senda de expansión
				\4 Conjunto de inputs óptimos en función de output producido
				\4[] Dados precios relativos de inputs
				\4 F. homotéticas: senda es línea recta desde origen
				\4[] \grafica{(sendadeexpansion)}
			\3 \underline{Función de beneficio}
				\4 $ \pi(p, \vec{w}) = \underset{\vec{x}}{\max} \; \; py - \vec{w} \vec{x}$
				\4 {Propiedades}
				\4[(i)] \textit{Homogénea de grado 1 en $(p, \vec{w})$}
				\4[(ii)] \textit{Convexa en $p$, $w_i$ $\forall$ $w_i$}
				\4[] $\then$ Matriz hessiana de $\pi(p, \vec{w})$ es semidef. positiva
				\4[(iii)] \textit{No decreciente en precio de output $p$}
				\4[(iv)] \textit{No creciente en precio de input $\vec{w}$}
				\4[(v)] \textit{Lema de Hotelling}\footnote{También se cumple $x_i (p,w) = \frac{\partial \pi (p,\vec{w})}{\partial w_i}$, dado que $y(p, \vec{w})$ no es sino una ``demanda'' negativa dado su carácter de output. Es decir, el lema de Hotelling no es sino una relación entre la f. de beneficio y las demandas/ofertas de inputs/outputs.}
				\4[] \fbox{$x_i (p,\vec{w}) = \pdv{\pi (p,\vec{w})}{w_i} $, $y (p,\vec{w}) = \pdv{\pi(p,\vec{w})}{p}$}
			\3 \underline{Función de costes}
				\4 $c(y, \vec{w}) = \left\lbrace \underset{\vec{x}}{\min} \, \vec{w} \vec{x} \right\rbrace = \vec{w} \vec{z}(\vec{w},y) $
				\4 {Propiedades}\footnote{Ver \href{https://economics.ubc.ca/files/2014/05/pdf_course_erwin-diewert-ECON594Ch3.pdf}{Diewert (2014).}}
				\4[(i)] \textit{Continua en $y$ y $\vec{w}$}
				\4[(ii)] \textit{No negativa}
				\4[(iii)] \textit{Homogénea de grado 1 en $(\vec{w})$}
				\4[(iv)] \textit{No decreciente en $\vec{w}$}
				\4[(v)] \textit{Creciente en $y$}
				\4[(vi)] \textit{Cóncava\footnote{Nótese \textit{cóncava}, no estrictamente cóncava. Si los factores son sustitutivos perfectos, los costes son constantes ante un aumento de uno de los factores, luego no se cumpliría la propiedad de concavidad estricta en este caso.} en $w_i$ $\forall w_i$}
				\4[] $\Rightarrow$ Hessiana de $c(\vec{w}, y)$ es semidef. negativa
				\4[(vii)] \text{Lema de Shephard}\footnote{Necesario que $z(\vec{w},y)$ sea una función, y que la función de costes sea derivable.}:
				\4[] \fbox{$\pdv{c(y, \vec{w})}{w_i} = z_i (\vec{w},y)$}
			\3 \underline{Función de oferta}
				\4 $y(p, \vec{w}) = f(\vec{x}(p, \vec{w}))$
				\4 Habitual derivar de máx. respecto de output:
				\4[] CPO: $p-\text{CMg}(y^*) \leq 0$, igualdad si $y^* > 0$.
				\4[$\Rightarrow$] \fbox{$y^* = 0$ si $p<\text{CMe}(y)$ }
				\4[] Porque $p < \text{CMe}(y) \Rightarrow \pi < 0$, $\pi (y) =0$
				\4[$\Rightarrow$] \fbox{$y^* = y: p=\text{CMg(y)}$ si  $p \geq \text{CMe}(y)$}
				\4[] Porque $p \geq \text{CMe}(y) \Rightarrow \pi \geq 0$ si $y>0$
				\4[] \grafica{funciondeoferta}
			\3 Coste medio
				\4 \fbox{$\text{CMe}(y, \vec{w}) = \frac{c(y, \vec{w})}{q}$}
				\4 Escala eficiente: output $y_E$ que minimiza CMe
			\3 Coste marginal
				\4 $ \text{CMg}(y) = \pdv{c(y, \vec{w})}{y}$
			\3 Relación coste medio--coste marginal
				\4 Escala eficiente
				\4[] Cantidad producida $y_{EF}$ que minimiza $\text{CMe}$
				\4 Coste marginal y medio se igualan
				\4[] $\then \text{CMe} (y_{EF}) = \text{CMg} (y_{EF}) $
				\4 Demostración
				\4[] $ \frac{d \, \text{CMe} (y)}{d \, y} = \frac{c'(y) \cdot y - c(y)}{y^2}=0 $
				\4[] $ c' \cdot y_E = c \Rightarrow \frac{c}{y_E} = c' \Rightarrow \text{CMe}(y_E) = \text{CMg}(y_E) $
			\3 Economías y deseconomías de escala
				\4 Relación entre costes y escala de producción.
				\4[] Economía de escala $\iff c(\lambda y) < \lambda c(y)$
				\4[] Deseconomía de escala $\iff c(\lambda y) > \lambda c(y)$
				\4 \underline{Economías de escala y rdtos. a escala}\footnote{(Bell, 1988).}
				\4[] $R \uparrow E$ $\then$ Economías de escala
				\4[] F. homotética $\then \left( R \uparrow E \iff \text{Economías de escala} \right) $ \footnote{Luego si no homotética: economía de escala $\nRightarrow$ Rdtos. crecientes a escala.}
				\4[] Demostración gráfica:
				\4[] \grafica{eenoimplicarce}
				\4 \underline{Deseconomías de escala y rdtos. a escala}
				\4[] Deseconomías de escala $\then R \downarrow E$
				\4[] F. no homotética: $ R \downarrow E \nRightarrow \text{Deseconomías de escala}$
%			\3 Geometría de las funciones de coste
%				\4 \textit{Cobb-Douglas}
%				\4 \textit{Función de prod. de Leontieff}
			\3 Elasticidad de sustitución
				\4 Propiedad muy importante en estática comparativa
				\4[] $\to$ ¿Cómo responde demanda de inputs ante cambios en precios?
				\4 $\varDelta \%$ de la proporción entre dos inputs
				\4[] dado $\varDelta \%$ de la RMST o de precios relativos
				\4[] $\to$ Medida de la curvatura de las isocuantas
				\4[] $\then$ \fbox{$\sigma = \frac{d \ln (x_2/x_1)}{d \ln p_1 / p_2} = \frac{d \ln (x_2/x_1)}{d \ln \text{RMST}_{12}} > 0$}\footnote{ La condición de primer orden de un programa de optimización de los beneficios tal que: $\left\lbrace \max \quad  \vec{p} \vec{x} \quad s.a. \quad F(\vec{x}) = 0 \right\rbrace$
				corresponde a $\text{RMST}_{ji} \equiv \frac{d \, j}{d \, i} = \frac{\partial F / \partial y_i}{\partial F / \partial y_j} = \frac{p_i}{p_j}$ }
				\4 Propiedad muy importante en estática comparativa
				\4[] $\to$ ¿Cómo $\varDelta$ demanda de inputs ante $\varDelta$ en precios?
			\3 Integrabilidad\footnote{Ver pág. 136 de \href{http://people.tamu.edu/~gtian/micro1.pdf}{Tian (2013)}.}
				\4 Postular función de producción
				\4[] Tarea prohibitiva en términos de supuestos
				\4[] $\to$ Requiere supuestos muy fuertes y arbitrarios
				\4[] $\then$ ¿Por qué una función de producción y no otra?
				\4[] $\then$ ¿Cómo conocer todos los outputs dado inputs?
				\4 Teorema de la dualidad permite caracterizar f. de prod
				\4[] A partir de costes que sí son observables
				\4[] Ejemplo:
				\4[] $\to$ Observar $\Delta$ de costes dado $\uparrow$ precio de f.p.
				\4[] $\then$ Perfectamente posible
				\4 Integrabilidad de las funciones de costes
				\4[] Caracterizar requisitos que deben cumplir f. de costes
				\4[] $\to$ Para que exista conjunto de producción
				\4[] $\then$ Que dé lugar a comportamiento observado
				\4[] $\then$ Que se pueda recuperar
				\4 Requisitos de la función de costes
				\4[] i. Homogénea de grado 1 en $\vec{w}$
				\4[] ii. No negativa para todo $\vec{w} > 0$
				\4[] iii. No decreciente en $\vec{w}$
				\4[] iv. Cóncava en $\vec{w}$
				\4[] $\then$ Hessiana de $C(\vec{w},y)$ semidef. negativa y simétrica
				\4[$\then$] Existirá tecnología $V^*(y)$ que induce costes observados
			\3 Leyes de la demanda derivada de Hicks-Marshall
				\4 Marshall y Hicks postulan por separado
				\4 Tratan de aproximar evidencia empírica
				\4[] No necesariamente derivadas de modelos
				\4[] La elast. de dda. de un factor es mayor cuanto más:
				\4[1] ...$\uparrow$ la elasticidad de demanda del output
				\4[2] ...$\uparrow$ la elasticidad de sustitución del factor
				\4[3] ...$\uparrow$ la elasticidad de oferta de otros factores
				\4[4] ...$\uparrow$ el peso del factor en el coste total\footnote{Dado que la elasticidad de demanda de un factor a una variación del coste del factor es negativa, esto implica que cuanto más peso tenga en el coste total, menos negativa será la elasticidad a su propio coste, y de ahí la $\uparrow$}
				%\4[1] ...más elástica sea la dda. del output
				%\4[2] ...más alta sea la elast. de sustitución del factor
				%\4[3] ...más elástica sea la elast. de oferta de otros factores
				%\4[4] ...mayor sea el peso del factor en el coste total\footnote{Cuando la elasticidad del output sea mayor que la de la demanda de factor.}
		\2 Aplicaciones
			\3 Modelización teórica
				\4 Organización industrial
				\4 Comercio internacional
				\4 Competencia perfecta
				\4 Equilibrio general
				\4 Economía laboral
				\4 Macroeconomía
			\3 Análisis empírica
				\4 Preguntas clave
				\4[] ¿Comportamiento empresarial maximizador de beneficios?
				\4[] ¿Responden empresas a cambios en costes de inputs?
				\4[] ¿Cuánto produce una empresa en un mercado determinado?
				\4 Función de producción no conocida
				\4[] $\to$ Difícilmente estimable de forma directa
				\4 Teorema de dualidad establece condiciones
				\4[] Para estimar función de producción
				\4[] $\to$ A partir de costes que sí son conocibles
				\4[] $\then$ F. de costes sí son estimables
				\4 Análisis econométrico de producción
				\4[] Entender factores que aumentan prod.
				\4[] Entender efectos precio sobre ff.pp.
				\4[] ...
	\1 \marcar{Extensiones del modelo neoclásico}
		\2 Corto y largo plazo
			\3 Idea clave
				\4 Distintos horizontes temporales
				\4[$\to$] Diferentes ofertas de inputs
				\4 A menor horizonte temporal
				\4[$\to$] Oferta más inelástica o limitada de algunos inputs
				\4[$\to$] Ej.: maquinaria pesada, bienes inmobiliarios
				\4 Costes fijos
				\4[] Independientes de cantidad producida
				\4[$\to$] Ligados a inputs no sustituibles
				\4 Costes a c/p nunca menores que a l/p
				\4[$\to$] A c/p existen restricciones adicionales
				\4[$\to$] En el problema de minimización de costes
			\3 Costes fijos y hundidos
				\4 Idea clave
				\4[] Distinción a menudo confusa
				\4[] Especialmente relevante en dicotomía c/p vs l/p
				\4 Costes hundidos
				\4[] También denominados
				\4[] $\to$ Costes fijos inevitables
				\4[] Costes incurridos de forma:
				\4[] $\to$ Irrevocable
				\4[] $\to$ Irrecuperable
				\4[] Sin supuestos sobre cuantía de la producción
				\4[] $\to$ Posible relacionados con cantidad producida
				\4[] $\to$ Posible relacionados sin relación con cantidad
				\4[] En corto plazo
				\4[] $\to$ Siempre hay costes hundidos en cierta medida
				\4[] En largo plazo
				\4[] $\to$ No hay costes hundidos
				\4 Costes fijos
				\4[] También denominados
				\4[] $\to$ Costes fijos evitables
				\4[] Independientes de cantidad producida
				\4[] Dependientes de producción o inactividad
				\4[] $\to$ Sólo aparecen sin hay producción > 0
				\4[] En corto plazo
				\4[] $\to$ Puede haber o no costes fijos
				\4[] En largo plazo
				\4[] $\to$ Puede haber o no costes fijos
				\4[] $\then$ Si existen, inducen CMe en forma de U
			\3 Corto plazo
				\4 {Idea clave}
				\4[] Diferentes ff.pp. tienen diferentes elasticidades
				\4[] Reorganización de planes de producción
				\4[] $\to$ Requiere cambios en combinaciones de inputs
				\4[] $\then$ Algunos inputs no pueden variar
				\4[] Algunos costes ya han sido incurridos irrevocablemente
				\4[] $\to$ No pueden recuperarse
				\4[] $\then$ Costes hundidos/fijos inevitables
				\4[] Algunos inputs pueden ser necesarios para iniciar producción
				\4[] $\to$ Pueden incurrirse o no
				\4[] $\to$ Se incurren si producción es rentable
				\4[] $\then$ Costes fijos/fijos evitables
				\4[] Empresas maximizan beneficios que pueden ser <0
				\4[] $\to$ No producen si $P<\text{CVMe}$
				\4 {Formulación}
				\4[] $\underset{\vec{x_v}}{\min} \quad \vec{w}_v \vec{x}_v + \vec{w}_f \vec{x}_f + \vec{w}_h \vec{x}_h$
				\4[] $s.a: \quad f(\vec{x}_v, \vec{x}_f) = y$
				\4 {Coste Variable:}
				\4[] $\vec{w}_v \vec{z}^v(y, \vec{w}, \vec{x_f})$
				\4[] Coste Variable Medio:
				\4[] $\text{CVMe} = \frac{\text{CV}}{y}$
				\4[] Empíricamente plausible que:
				\4[] $\to$ $\pdv{\text{CVMe}}{y} > 0$ a partir de cierto $y$
				\4[] \grafica{cvmedio}
				\4 {Coste hundido/coste fijo inevitable}
				\4[] $\text{CF}=\vec{w}_f \vec{x}_f$
				\4[] Coste Fijo Medio: $\text{CFMe} = \frac{\text{CF}}{y}$
				\4[] $\to \text{CFMe} = \pdv{\text{CF}}{y} < 0$
				\4[] \grafica{cfmedio}
				\4 Optimización en corto plazo
				\4[] Empresa produce cantidad positiva si:
				\4[] $\to$ $\text{CVMe} \leq \text{P}$
				\4[] $\then$ Siempre que cubra costes variables medios
				\4[] $\then$ Punto muerto: $\frac{\text{CF}}{P - \text{CVMe}}$
				\4[] Empresa sale de mercado si:
				\4[] $\to$ Beneficio variable no cubre coste fijo
				\4[] $\then$ Si cantidad vendida no supera el punto muerto
			\3 Largo plazo
				\4 {Idea clave}
				\4[] Todos factores tienen ofertas perfectamente elásticas
				\4[] $\to$ Todos costes son variables
				\4[] $\to$ Y dependen de output
				\4 No hay coste variable ni fijo
				\4[] Hay coste marginal y coste medio
				\4[] $\to$ Objetivo es superar coste medio
				\4 Optimización en largo plazo
				\4[] Empresa produce cantidad positiva si:
				\4[] $\to$ $\text{CMe} < P$
				\4[] $\then$ Siempre que cubra costes medios
				\4 Principio de LeChatelier Samuelson
				\4[] Elasticidad compensada de demanda de factores
				\4[] $\to$ Mayor en largo plazo que en corto
				\4[] Ejemplo
				\4[] $\to$ Caen salarios
				\4[] $\then$ En corto plazo, imposible sustituir K por L
				\4[] $\then$ L aumenta sólo por efecto demanda
				\4[] $\then$ En largo plazo, dda. de L aumenta por sust. con K
			\3 Representación gráfica
				\4 CMe a l/p es envolvente de curvas de costes a c/p
				\4 \grafica{cmedioslpcp}
		\2 Enfoque dinámico
			\3 Idea clave
				\4 Costes resultado de:
				\4[] Tantas decisiones de producción como periodos
				\4 Producción óptima:
				\4 secuencia completa de inputs que máx. beneficio
			\3 Sin costes de ajuste
				\4 Problema de maximización en cada periodo
				\4[] considerando sólo precios del periodo
			\3 Costes de ajuste
				\4 Cambios en planes de producción entre periodos
				\4[] inducen $\uparrow$ de costes.
				\4 Necesario considerar secuencia completa en cada periodo
				\4[$\to$] Herramientas matemáticas optimización dinámica
				\4[] Cálculo de variaciones, programación dinámica, control óptimo
		\2 Múltiples outputs
			\3 Idea clave
				\4 Elección de dos o más outputs a producir
				\4[$\to$] Distribuir inputs entre varios outputs
				\4 Mismas herramientas matemáticas
			\3 Economías de alcance
				\4 Fenómeno habitual en numerosos sectores
				\4 Producción de varios outputs por una empresa
				\4[$\to$] Más barato que una empresa por cada output
				\4 Economias de alcance $\Rightarrow$ Subaditividad
	\1 \marcar{Anomalías y críticas}
		\2 Teoría de los costes de Stigler\footnote{Stigler (1939).}
			\3 Idea clave
				\4 Las empresas instalan exceso de capacidad
				\4[] para responder a incertidumbre en demanda
				\4[$\to$]Curvas de CMe y CMg coinciden en un intervalo
				\4 \grafica{stiglercp}
				\4 A l/p, la curva de CMe tiene forma de L
				\4[$\to$] Desec. escala administrativas se compensan con eco. escala técnicas
		\2 Ineficiencia X de Leibenstein\footnote{Leibenstein, Harvey (1966).}
			\3 Idea clave
				\4 Las empresas se sitúan dentro del conjunto de producción
				\4[$\to$] No en la frontera
				\4 Pueden elegir planes de producción más eficientes
				\4[$\to$] Pero no lo hacen, por diferentes razones
				\4[i] Ausencia de presión competitiva
				\4[ii] Inercias organizativas
				\4[iii] Contratos incompletos
				\4[iv] Minimización del esfuerzo
	\1[] \marcar{Conclusión}
		\2 Recapitulación
			\3 Teoría neoclásica de costes
				\4 Idea clave
				\4 Formulación
				\4 Implicaciones
				\4 Aplicaciones
			\3 Extensiones
				\4 Corto y largo plazo
				\4 Enfoque dinámico
				\4 Múltiples outputs
			\3 Anomalías y críticas
				\4 Stigler
				\4 Ineficiencia X
		\2 Idea final
			\3 Análisis empírico
				\4 Teoría de costes delimita problema:
				\4[] ¿qué necesitamos para predecir y explicar comportamiento de empresas?
				\4[$\to$] Datos sobre costes, producción y precios
			\3 Avances en recogida de datos
				\4 TI permiten ahora masivas recogidas de microdatos
				\4 Cada vez más cerca de predecir comp. empresas
				\4[] En tiempo real y de forma general
\end{esquemal}



























\conceptos

\concepto{Aditividad y economías de escala y alcance} véase Tema 3A-15 Monopolio

\concepto{Costes fijos y hundidos: diferencias en caracterización de conjuntos de producción}

La diferencia general entre los conceptos de coste fijo y coste hundido no es inequívoca, y ha dado lugar a encendidos debates. ¿Hasta qué punto la compra de un input es irreversible? ¿Hasta qué punto son irrevocables los contratos de compra o alquiler de inputs que se extienden en el tiempo? ¿Hasta cuándo se extiende en el tiempo esa irrevocabilidad? 

A pesar de la dificultad para dar respuestas categóricas a estas preguntas, la caracterización de los conjuntos de producción con costes fijos y hundidos sí que es sencilla y claramente distinta en el contexto de la teoría microeconómica neoclásica y el análisis puramente estático. Así, tenemos que los costes hundidos --entendidos como aquellos en los que se incurre independientemente del nivel de producción- se pueden representar eliminando la propiedad de inacción del conjunto de producción. 

El fenómeno de los costes fijos --entendidos como aquellos en los que es obligado incurrir para alcanzar un nivel de producción positivo- es representable restringiendo el conjunto de producción de tal manera que niveles positivos de output impliquen un nivel mínimo de un input determinado, pero manteniendo la posibilidad de inacción (esto es, el plan de producción representado en el origen de coordenadas).

\begin{axis}{4}{Conjunto de producción con costes fijos}{$x$}{$y$}{maximizacion}
	\draw[-] (0,0) -- (-4,0); % extensión de ejes
	\node[below] at (-4,0){$-x$};
	\draw[-] (0,0) -- (0,-1);
	
	\draw[line width=2pt] (-4,4) -- (-1.5,0) -- (0,0) -- (0,-1);
		
	\draw [blue, fill=yellow, opacity=0.2] (-4,4) -- (-1.5,0) -- (0,0) -- (0,-1) -- (-4,-1);
\end{axis}

\begin{axis}{4}{Conjunto de producción con costes hundidos. }{$x$}{$y$}{maximizacion}
	\draw[-] (0,0) -- (-4,0); % extensión de ejes
	\node[below] at (-4,0){$-x$};
	\draw[-] (0,0) -- (0,-1);
	
	\draw[line width=2pt] (-4,4) -- (-1.5,0) -- (-1.5,-1);
	
	\draw [blue, fill=yellow, opacity=0.2] (-4,4) -- (-1.5,0) -- (-1.5,-1) -- (-4,-1);
\end{axis}

\pagebreak 

\concepto{Economías de escala internas y externas}

Las economías de escala interna son aquellas disminuciones en el coste medio a medida que aumenta la producción que son resultado de la forma de la función de producción y del aumento de los factores de producción aplicados al proceso. Las economías de escala externa se refieren a fenómenos de reducción de coste medio relacionados con aumentos de la producción de otras empresas en la misma industria. Así, las economías de escala externas no son sino externalidades positivas en la producción.

\concepto{Problema de maximización del beneficio}
\begin{axis}{4}{Problema de maximización del beneficio con un output y un input }{$x$}{$y$}{maximizacion}
	\draw[-] (0,0) -- (-4,0); % extensión de ejes
	\node[below] at (-4,0){$-x$};
	\draw[-] (0,0) -- (0,-1);	
	
	\draw[-] (0,0) to [out=100, in=340](-4,3);	
	\draw[-] (-4,3.28) -- (0,1.28);
	
	\draw[dotted] (-4,2.28) -- (0,0.28);	
	\draw[dotted] (-4,1.28) -- (0,-0.72);	
	\node[circle,fill=black,inner sep=0pt,minimum size=4pt] (a) at (-1.7,2.12) {};
	
	
	\draw [blue, fill=yellow, opacity=0.2] (0,0) to [out=100, in=340](-4,3) -- (-4,-1) -- (0,-1);
\end{axis}

\graficas

\begin{axis}{4}{Maximización de beneficios dada una función de producción con rdtos. crecientes a escala.}{$x$}{$y$}{maximizacionrcreescala}
	\draw[-] (0,0) -- (-4,0);
	
	% función de producción
	
	\draw[-] (0,0) to [out=170, in=290](-3,4);
	
	\node[left] at (-1.7,0.5){$y=f(x)$};
	
	\draw [white, fill=yellow, opacity=0.2] (0,0) to [out=170, in=290](-3,4) -- (-4,4) -- (-4,0);
	
	
	
	% curvas isobeneficio

	\draw[dashed] (-3.5,4) -- (0.5,0);
	\draw[dashed] (-2.5,4) -- (1.5,0);
	\draw[dashed] (-4.5,4) -- (-0.5,0);

	\node[above] at (-3.4,4){$\pi_1 \quad < \; \pi_2 \quad < \; \pi_3$};
	
	\draw[-{Latex}] (-2.3,1.7) -- (-0.8,2.9);
\end{axis}

\begin{axis}{4}{Maximizacion de beneficios dada una funcion de producción con rendimientos constantes a escala y precio del output menor que coste del input.}{$x$}{$y$}{maximizacionrconescala}
	\draw[-] (0,0) -- (-4,0);
	
	% función de producción
	
	\draw[-] (0,0) -- (-4,3);
	
	\node[left] at (-1.7,0.5){$y=f(x)$};
	
	
	\draw [white, fill=yellow, opacity=0.2] (0,0) -- (-4,3) -- (-4,0) -- (0,0);
	
	% curvas isobeneficio
	
	\draw[dotted] (-3.5,4) -- (0.5,0);
	\draw[dotted] (-2.5,4) -- (1.5,0);
	\draw[dashed] (-4,4) -- (-0,0);
	
	\node[above] at (-3.4,4){$\pi_1 < \; \pi_2 \; \; < \;  \; \pi_3$};
	
	\draw[-{Latex}] (-2.3,1.7) -- (-0.8,2.9);
	
	% óptimo
	
	\node[circle, fill=black, inner sep=0pt, minimum size=6pt] (a) at (0,0) {};
	\node[below] at (0,0){Óptimo};
	
\end{axis}

\begin{axis}{4}{Maximización de beneficios dada una función de producción con rendimientos decrecientes a escala.}{$x$}{$y$}{maximizacionrdecescala}
	\draw[-] (0,0) -- (-4,0);
	
	% función de producción
	
	\draw[-] (0,0) to [out=110, in=-20](-3.04,4);
	
	\node[left] at (-1.7,0.5){$y=f(x)$};
	
	\draw [white, fill=yellow, opacity=0.2] (0,0) to [out=110, in=-20](-3.04,4) -- (-4,4) -- (-4,0);
	
	% curvas isobeneficio
	
	\draw[dotted] (-3.5,4) -- (0.5,0);
	\draw[dashed] (-2.5,4) -- (1.5,0);
	\draw[dotted] (-4.5,4) -- (-0.5,0);
	
	\node[above] at (-3.4,4){$\pi_1 \quad < \; \pi_2 \quad < \; \pi_3$};
	
	\draw[-{Latex}] (-2.3,1.7) -- (-0.8,2.9);
	
	% óptimo
	
	\node[circle, fill=black, inner sep=0pt, minimum size=3pt] (a) at (-1.61,3.09) {};
	\node[right] at (-1.61,3.15){Óptimo};
\end{axis}

\begin{axis}{4}{Problema de minimización de costes.}{$x_1$}{$x_2$}{minimizacion}
	% isocuanta
	\draw[-] (.5,4) to [out=290, in= 160](4,.5);
	
	\draw [white, fill=yellow, opacity=0.2] (.5,4) to [out=290, in=160](4,.5) -- (4,4);
	
	\node[right] at (4,.5){$f(x_1, x_2) = y$};
	
	
	% isocoste
	\draw[dotted] (.5,3.82) -- (3.5,.82);
	
	\draw[dashed] (.5,3.12) -- (3.5,.12);
	
	\draw[dotted] (.5,2.5) -- (3.5,-0.5);
	
	
	% óptimo
	\node[circle, fill=black, inner sep=0pt, minimum size=3pt] (a) at (1.78,1.85) {};
	
	\draw[dotted, color=gray] (1.78,0) -- (1.78, 1.85);
	\node[below] at (1.78,0){$x_1^*$};
	
	\draw[dotted, color=gray] (1.78,1.85) -- (0,1.85);
	\node[left] at (0,1.85){$x_2^*$};
	
\end{axis}

\begin{axis}{4}{Maximización de beneficios incorporando función de costes mínimos dado output.}{$y$}{$\pi(y)$}{maximizarconcostes}
	\draw[-] (0,0) to [out=80, in=180](2,3) to [out=0, in=100](4,0);
	
	\draw[dashed] (0,3) -- (4,3);
	
	\node[left] at (0,3){$\pi (y^*)$};
	
	\draw[dashed] (2,0) -- (2,3);
	\node[below] at (2,0){$y^*$};
\end{axis}

En este caso, existe un máximo local (y posiblemente global) dada la concavidad de la función de beneficios. Una función de beneficios $\pi = p y -  c(y)$ será cóncava siempre que la función de costes mínimos $c(y)$ sea convexa. La convexidad de la función de costes equivale a costes medios crecientes e implica rendimientos decrecientes a escala.

\begin{axis}{4}{Senda de expansión de una función de producción homogénea.}{$x_1$}{$x_2$}{sendadeexpansion}
	
	% curva interior
	\draw[-] (.5,4) to [out=290, in= 160](4,.5);
	% isocoste tangente a curva interior
	\draw[dashed] (.5,3.12) -- (3.5,.12);
	% óptimo con primera curva
	\node[circle, fill=black, inner sep=0pt, minimum size=5pt] (a) at (1.78,1.85) {};
	
	% curva intermedia
	\draw[-] (1,4.5) to [out=290, in= 160](4.5,1);
	% isocoste tangente a curva interior
	\draw[dashed] (1,3.62) -- (4,.62);
	% óptimo con primera curva
	\node[circle, fill=black, inner sep=0pt, minimum size=5pt] (a) at (2.28,2.35) {};
	
	%curva exterior
	\draw[-] (1.5,5) to [out=290, in= 160](5,1.5);
	%isocoste tangente a curva interior
	\draw[dashed] (1.5,4.12) -- (4.5,1.12);
	% óptimo con primera curva
	\node[circle, fill=black, inner sep=0pt, minimum size=5pt] (a) at (2.78,2.85) {};
	
	% senda de expansión
	\draw[-] (0,0) -- (4, 4.15);
\end{axis}

\begin{axis}{4}{Función de oferta en función de precios y su relación con costes marginales y medios.}{$y$}{CMe(y) \\ CMg(y) \\ P(y)}{funciondeoferta}
	% costes medios
 	\draw[-] (0.9,4) to [out=280, in=180](2.65,2) to [out=0, in=260](4.4,4);
	\node[right] at (4.5,4){CMe};
	
	% costes marginales
	\draw[-] (0,0) to [out=15, in=235](2.65,2) to [out=55,in=260](3.4,4);
	\node[above] at (3.5,4){CMg};	

	% oferta
	\draw[thick,color=red] (2.63,2) to [out=55,in=260](3.38,4);
	\draw[thick, color=red] (0,0) -- (0,2);
\end{axis}

En rojo se muestra la función inversa de oferta. Se puede apreciar que cuando el coste medio es superior al coste marginal, la oferta de la empresa es igual a 0. Esto es así porque si la empresa produciese una cantidad de output que igualase coste marginal y precio, obtendría beneficios negativos y en el contexto del ejemplo, siempre es posible obtener beneficios nulos a partir de una producción nula.


\begin{axis}{4}{Economías de escala no son condición suficiente para la existencia de rendimientos crecientes a escala: demostración gráfica basada en Bell (1988)}{x}{y}{eenoimplicarce}
	% Isocuanta de producción q
	\draw[-] (0.1,3) to [out=280, in=176](3,0.1);
	\node[above] at (2.4,0.1){\tiny $q$};
	
	% Isocuanta de producción aq
	\draw[-] (0.2,3.4) to [out=290,in=170](4,1.2);
	\node[right] at (4,1.2){\tiny $aq$};
	
	% Isocoste tangente a isocuanta de producción q
	\draw[-] (0,1.7) -- (2.33,0);
	
	% Punto de tangencia entre isocoste e isocuanta de producción q
	\node[circle, fill=black, inner sep=0pt, minimum size=3pt] (a) at (1.47,0.63) {};
	\node[above] at (1.47,0.63){\tiny A};
	
	% Isocoste tangente a isocuanta de producción aq
	%\draw[-] (0,)
	\draw[-] (0,3.09) -- (3.62,0);
	
	% Punto de tangencia entre isocoste e isocuanta de producción aq
	\node[circle, fill=black, inner sep=0pt, minimum size=3pt] (a) at (1,2.25) {};
	\node[below] at (1,2.25){\tiny B};

	% Isocoste de factores multiplicados por a
	\draw[-] (0,3.43) -- (4,0);	

	% Rayo vector que corta a punto de tangencia entre isocuanta de q e isocoste tangente
	\draw[-] (0,0) -- (4,1.7);
	
	% Inputs para producir q
	\draw[dashed] (0,0.63) -- (1.47,0.63) -- (1.47,0);
	\node[left] at (0,0.63){\tiny $q_y$};
	\node[below] at (1.47,0){\tiny $q_x$};
	
	% Inputs para producir q multiplicados por a
	\draw[dashed] (0,1.14) -- (2.67,1.14) -- (2.67,0);
	\node[left] at (0,1.14){\tiny $a q_y$};
	\node[below] at (2.67,0){\tiny $a q_x$};
\end{axis}

La gráfica muestra cómo una función de costes puede mostrar economías de escala sin que la función de producción muestre rendimientos crecientes a escala. Para producir una cantidad $q$ de output, se requieren cantidades $q_x$ y $q_y$ de $x$ y de $y$, respectivamente. Dados los precios de los factores fijos, para producir una cantidad $aq$ son necesarios los factores señalados en el punto B, correspondiente a la tangencia entre la isocuanta de $aq$ y la recta isocoste correspondiente. Sin embargo, los si multiplicamos los factores $q_x$ y $q_y$ por el mismo factor $a$, la isocoste correspondiente se encuentra más a la derecha y comprende la isocoste correspondiente al punto B, por lo que el coste de haber producido $aq$ unidades es menor al que resulta que multiplicar los factores $q_x$ y $q_y$ por $a$. Así, los costes muestran economías de escala al tiempo que la función de producción muestra rendimientos decrecientes a escala, y queda demostrado que la presencia de economías de escala no es condición suficiente para la existencia de rendimientos crecientes a escala.


\begin{axis}{4}{Costes variables medios con rendimientos decrecientes a escala.}{$y$}{$\text{CVMe}$}{cvmedio}
	\draw[-] (0,0) to [out=10, in=260](4,4);
\end{axis}

\begin{axis}{4}{Costes fijos medios.}{$y$}{$\text{CFMe}$}{cfmedio}
	\draw[-] (0.5, 4) to [out=280, in=175](4,1);
	
	\draw[dashed] (0,2.5) -- (1.14,2.5);
	\draw[dashed] (1.14, 0) -- (1.14,2.5);
	\node[below] at (1.14,0) {\small 1};
	\node[left] at (0,2.5){\small F};
\end{axis}

\begin{axis}{4}{Curva de costes medios a largo plazo como envolvente de las curvas de costes medios a corto plazo. }{$y$}{$\text{CMe}$}{cmedioslpcp}
	\draw[-] (4,0)  -- (8,0);
	
	% largo plazo
	\draw[-] (0.4, 4) to [out=280, in=180](5.2,.8);
	\draw[-] (5.2,.8) to [out=0,in=260](8,4);
	
	\node[right] at (8.05,4){CMeLP};
	
	% curvas de corto plazo
	
	% primera curva
	\draw[-] (1.13, 4) to [out=271, in=180](2.03, 1.8);
	\draw[-] (2.03, 1.8)  to [out=0, in=269](2.93, 4);
	\node[above] at (2.93,4){\scriptsize $\text{CMeCP}_1$};
	
	% segunda curva
	\draw[-] (2.13, 3.40) to [out=271, in=180](3.03, 1.2);
	\draw[-] (3.03, 1.2)  to [out=0, in=269](3.93, 3.4);
	\node[above] at (3.97,3.4){\scriptsize $\text{CMeCP}_2$};
	
	% tercer curva
	\draw[-] (3.84, 3.02) to [out=271, in=180](4.74, 0.82);
	\draw[-] (4.74, 0.82)  to [out=0, in=269](5.64, 3.02);
	\node[right] at (5.64,3.02){\scriptsize $\text{CMeCP}_3$};
	
	% cuarta curva
	\draw[-] (5.34, 3.40) to [out=271, in=180](6.24, 1.2);
	\draw[-] (6.24, 1.2)  to [out=0, in=269](7.14, 3.4);
	\node[above] at (7.14,3.4){\scriptsize $\text{CMeCP}_4$};
	
\end{axis}

\begin{axis}{4}{Curva de costes medios según Stigler (1939).}{$y$}{$\text{CMe}$ \\ CMg}{stiglercp}
	
	% costes medios
	\draw[thick] (.5,4) to [out=280, in=160](2,2);
	\draw[thick] (2,2) -- (3,2);
	\draw[thick,] (3,2) to [out=10, in=260](4.5,3.1);
	\node[right] at (4.5,3.1){CMe};
	

	
	% costes marginales
	\draw[thick, color=red] (0,0) to [out=30, in=260](2,2);
	\draw[thick, color=red] (2,1.98) -- (3,1.98);
	\draw[thick, color=red] (3,2) to [out=20, in= 260](4.5,4);
	\node[right] at (4.5,4){CMg};
\end{axis}

\preguntas

\seccion{Test 2019}

\textbf{6.} La función de costes de una empresa competitiva que produce el bien Y utilizando dos factores productivos $z_1$ y $z_2$ es: \textbf{$C(\omega_1, \omega_2, Y) = 2 \sqrt{\omega_1 \omega_2} Y$}, donde $\omega_1$ y $\omega_2$ son respectivamente los precios de los factores $z_1$ y $z_2$. Encuentre las funciones de demanda de factores $z_1$ y $z_2$, y señale la respuesta correcta:

\begin{itemize}
	\item[a] Si $Y=10$, $\omega_1 = 4$ y $\omega_2 = 16$, la demanda óptima de factor 1 es $z_1 = 2.5$.
	\item[b] Si $Y=10$, y $\omega_1 = 2$ y $\omega_2 = 8$, la demanda óptima de factor 1 es $z_1 = 40$.
	\item[c] Si $Y=100$, $\omega_1 = 10$ y $\omega_2 = 90$, la demanda óptima de factor 1 es $z_1 = 33.33$.
	\item[d] Ninguna de las anteriores afirmaciones es correcta.
\end{itemize}


\seccion{Test 2017}

\textbf{6.} Una empresa fabrica un único bien Y utilizando un conjunto de factores $\textbf{x} = (x_1, x_2, ..., x_n)$, de acuerdo con la función de producción $Y=f(\textbf{x})$. La empresa tiene el objetivo de maximizar el beneficio y adquiere los factores productivos en mercados competitivos con precios $\textbf{w} = (w_1, w_2, ..., w_n)$. Si $f$ es una función cóncava, entonces la función de costes de la empresa $C(\textbf{w}, Y)$ debe verificar la siguiente propiedad:

\begin{itemize}
	\item[a] Homogeneidad de grado uno en $\textbf{w}$, es decir $C(\lambda \cdot \textbf{w}, Y) = \lambda \cdot C(\textbf{w}, Y)$
	\item[b] Homogeneidad de grado cero en $\textbf{w}$, es decir, $C(\lambda \cdot \textbf{w}, Y) = C(\textbf{w}, Y)$.
	\item[c] Homogeneidad de grado cero en $\textbf{w}$ e Y, es decir, $C (\lambda \cdot \textbf{w}, \lambda \cdot Y) = C(\textbf{w}, Y)$.
	\item[d] Homogeneidad de grado uno en $\textbf{w}$ e Y, es decir, $C(\lambda \cdot \textbf{w}, \lambda \cdot Y) = \lambda \cdot C(\textbf{w}, Y)$.
\end{itemize}

\textbf{7.} Disponemos de la función de costes $C(\textbf{w}, Y)$ de una empresa. ¿Qué información podemos extraer de ella a partir del resultado conocido como ``Lema de Shephard''?

\begin{itemize}
	\item[a] Se puede calcular la demanda marshalliana $Y_i (p_Y, m_i)$ de los n consumidores, $i = 1, ..., n$ del bien Y que fabrica la empresa.
	\item[b] Se puede obtener la curva de demanda de mercado a la que se enfrenta la empresa.
	\item[c] Se pueden obtener las funciones de demanda de la empresa de los $n$ factores productivos que adquiere, como $x_i = \partial C / \partial w_i$, $i=1,...,n$.
	\item[d] Se pueden calcular las funciones de coste de las empresas rivales que oferten producto Y en el mismo periodo.
\end{itemize}

\seccion{Test 2016}

\textbf{14}. Una empresa competitiva tiene una función de producción $y=x_1 x_2$. Cuando $w_1=w_2 = 1$, el mínimo coste de producción es igual a 4.

\begin{enumerate}
	\item[a] La producción es igual a 0.
	\item[b] La producción es igual a 2.
	\item[c] La producción es cualquier valor de y.
	\item[d] Si $p=1$, el beneficio es nulo.
\end{enumerate}

\textbf{15}. La función de producción de una empresa competitiva con un solo factor de producción es $f(x) = 20x - x^2$, $w$ es el precio del factor $x$ y $p$, el precio del bien $x$, es igual a 1.

\begin{enumerate}
	\item[a] El valor de $w$ para que la cantidad óptima de $x$ sea 0 es 20.
	\item[b] El valor de $w$ para que la cantidad óptima de $x$ sea 10 es 1.
	\item[c] El máximo beneficio no depende del valor de la cantidad producida.
	\item[d] El máximo beneficio no depende del valor de $w$.
\end{enumerate}

\seccion{Test 2014}
\textbf{6}. Si existen dos tecnologías para producir un bien cuyas funciones de costes totales son $\text{CA}(q) = 3q^2 + 12q +3$ y $\text{CB}(q)=5q^2 + 20$, respectivamente, el precio de equilibrio competitivo a largo plazo (pL) es:

\begin{enumerate}
	\item[a] pL = 18
	\item[b] pL = 9
	\item[c] pL = 10
	\item[d] pL = 20
\end{enumerate}

\seccion{Test 2013}

\textbf{5}. Sea Y la producción de leche de vacas que comen avena (A) y heno (H) y que sigue la siguiente función de producción $Y = \sqrt{A+2H}$. Esta producción de leche tendrá:

\begin{enumerate}
	\item[a] deseconomías de escala.
	\item[b] costes marginales crecientes.
	\item[c] rendimientos a escala crecientes.
	\item[d] costes medios constantes.
\end{enumerate}

\seccion{Test 2011}

\textbf{3}. Para un nivel de producción dado, el coste medio de una empresa competitiva es decreciente si:

\begin{enumerate}
	\item[a] La empresa tiene un coste fijo.
	\item[b] La empresa tiene rendimientos decrecientes de escala.
	\item[c] El coste marginal es menor que el coste medio.
	\item[d] La empresa tiene poder de mercado en la compra de factores.
\end{enumerate}

\seccion{Test 2009}

(En el excel, esta pregunta está anotada al tema 3A-14 sobre Competencia Perfecta)

\textbf{6}. Una empresa que persigue maximizar sus beneficios tiene una función de producción $y=f(x)$, donde $x=(x_1, x_2, ..., x_n)$, es el vector que representa las cantidades de factores utilizados por la empresa, cuyos precios de mercado son $w=(w_1, w_2, ..., w_n)$. Si $f$ es una función cóncava, entonces la función de costes de la empresa C(w, y) debe verificar la siguiente propiedad:

\begin{enumerate}
	\item[a] Homogeneidad de grado cero en $w$, es decir, $C(\lambda w, y) = C(w,y)$.
	\item[b] Homogeneidad de grado cero en $w$ e $y$, es decir, $C(\lambda w, \lambda y)= C(w, y)$.
	\item[c] Homogeneidad de grado uno en $w$, es decir, $C(\lambda w, y) = \lambda C(w, y)$.
	\item[d] Homogeneidad de grado uno en $w$ e $y$, es decir, $C(\lambda w, \lambda y) = \lambda C(w, y)$.
\end{enumerate}

\seccion{Test 2007}

\textbf{9}. Una empresa que produce el bien x en un mercado de competencia perfecta, con unos costes $C(x)$, es \textbf{FALSO} que:

\begin{enumerate}
	\item[a] Si la empresa está en equilibrio producirá donde $P=\text{CMg}_x$, independientemente de que está operando en el corto o en el largo plazo.
	\item[b] Si la empresa está en equilibrio a corto plazo producirá una cantidad sobre la rama creciente de los costes marginales.
	\item[c] Si la empresa está en equilibrio a largo plazo producirá una cantidad sobre la rama creciente de los costes marginales.
	\item[d] Si la empresa está en equilibrio a largo plazo producirá una cantidad igual al tamaño óptimo de planta o escala mínima eficiente.
\end{enumerate}

\seccion{Test 2006}

\textbf{10}. Si se compara la minimización de costes a corto y largo plazo de una empresa competitiva con función de producción $x=f(y_1, y_2)$, se verifica que: 

\begin{enumerate}
	\item[a] Los costes a largo plazo son superiores a los de corto plazo para todos los niveles de producción.
	\item[b] La curva de oferta a largo plazo es más inelástica que la curva de oferta a corto plazo.
	\item[c] La minimización de costes en ambos casos siempre se produce cuando la relación marginal de sustitución técnica es igual al cociente de los precios de los dos factores.
	\item[d] Cuando la cantidad demandada a largo plazo del factor fijo a corto plazo coincide con dicho factor a corto plazo, los costes marginales a corto y largo plazo coinciden.
\end{enumerate}

\seccion{Test 2005}

\textbf{9}. Sea una empresa precio-aceptante con función de costes a corto plazo dada por la expresión: $C^C(x) = x^2 +x+4$. Siendo P el precio de equilibrio en el mercado del producto:

\begin{enumerate}
	\item[a] La empresa solo producirá una cantidad positiva si P>5.
	\item[b] Si P=3, la empresa estará cubriendo sus costes variables y parte de sus costes fijos.
	\item[c] Si P=1,5 la empresa no querrá producir porque produciendo obtendría mayores pérdidas que cerrando.
	\item[d] Esta empresa nunca producirá porque siempre tendrá pérdidas.
\end{enumerate}

\textbf{10}. Considere un mercado compuesto por una única empresa precio aceptante cuyas funciones de costes a largo plazo $C^L(x)$ y a corto plazo $C^C(x)$ vienen dadas por: $C^L(x) = 4x$ y $C^C(x,K)=4\sqrt{K}x^2 + K$. Suponga que a corto plazo la cantidad de factor fijo empleado por la empresa es $K=1$. Entonces es \textbf{FALSO} que:

\begin{enumerate}
	\item[a] Si la empresa decidde producir $x=\frac{1}{2}$ unidades del bien, entonces el stock de factor fijo a corto plazo es óptimo en el largo plazo.
	\item[b] Siempre que $x \neq \frac{1}{2}$, el coste a corto plazo será mayor que el coste a largo plazo.
	\item[c] Para un stock de factor fijo $K \neq 1$, el coste de producir $x=\frac{1}{2}$ es idéntico a corto y a largo plazo.
	\item[d] Para cualquier nivel de producción siempre se verifica que el coste de corto plazo es mayor o igual que el coste de largo plazo.
\end{enumerate}

\notas

\textbf{2019}: \textbf{6.} D

\textbf{2017}: \textbf{6.} A \textbf{7.} C

\textbf{2016}: \textbf{14}. D \textbf{15}. A

\textbf{2014}: \textbf{6}. A

\textbf{2013}: \textbf{5}. A

\textbf{2011}: \textbf{3}. C

\textbf{2008}: \textbf{6}. C

\textbf{2007}: \textbf{9}. D En la matriz de respuestas está registrada efectivamente la 'de' como respuesta correcta. Pero en todos los libros de microeconomía de referencia, en el equilibrio de largo plazo con libre entrada se cumple que las empresas no obtienen beneficios porque entran nuevas empresas hasta que el beneficio se reduce a cero. Lo cual se produce cuando el precio se iguala al coste medio mínimo: es decir, al correspondiente a la escala mínima eficiente. El de la academia intentaba, por lo visto, justificar esta respuesta con algún supuesto adicional que no concuerda con ningún manual. Ver Gravelle págs. 185 y 186, Nicholson págs. 426 y 427, MWG págs. 336 y 337.

\textbf{2006}: \textbf{10}. D

\textbf{2005}: \textbf{9}. B \textbf{10}. C


\bibliografia

Mirar en Palgrave:
\begin{itemize}
	\item cost functions
	\item duality
	\item production functions
	\item X-efficiency
\end{itemize}


Bell, Christopher R. \textit{Economies of, versus Returns to, Scale: A Clarification}. Journal of Economic Education, 1998

Chiang, A; Wainwright, K. \textit{Fundamental Methods of Mathematical Economics.} Ch. 12 Optimization with Equality Constraints.

Hallam, Arne. \textit{Cost functions}. \url{http://www2.econ.iastate.edu/classes/econ501/Hallam/documents/CostFunctions.pdf}

Mas-Colell, A; Whinston, M. D.; Green, J. R. \text{Microeconomic Theory}. (1995) Ch. 5 Production

Rosas, F.; Lence, S. H. \textit{Duality theory econometrics: How reliable is it with real-world data?} (2015) Selected Papers for the 2015 Agricultural \& Applied Economics Association and Western Agricultural Economics Association Anual Meeting, San Francisco -- En carpeta del tema


Stigler, G. \textit{Production and Distribution in the Short Run}. 1939

Tian, G. \textit{Lecture notes: Microeconomic Theory} (2013) \url{http://people.tamu.edu/~gtian/micro1.pdf} -- En carpeta Microeconomía

Tirole, J. \textit{The Theory of Industrial Organization}

Varian, H. \textit{Microeconomic analysis} Ch. 2-6

Wang, H.; Yang, B. Z. (2001) \textit{Fixed and Sunk Costs Revisited} The Journal of Economic Education, Vol. 32, No. 2 -- En carpeta del tema
\end{document}
