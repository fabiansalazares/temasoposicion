\documentclass{nuevotema}

\tema{3B-14}
\titulo{Mercado de divisas. Operaciones e instrumentos.}

\begin{document}

\ideaclave

Mejorar currency swaps

Ver \href{https://www.dbresearch.com/servlet/reweb2.ReWEB?rwsite=RPS_EN-PROD&rwobj=ReDisplay.Start.class&document=PROD0000000000502442}{Deutsche Bank (2019) sobre evolución del mercado de divisas y perspectivas futuras de centralización y compensación.}

\seccion{Preguntas clave}

\begin{itemize}
	\item ¿Qué es el mercado de divisas?
	\item ¿Qué es el tipo de cambio?
	\item ¿Quiénes actúan en el mercado de divisas?
	\item ¿Qué instrumentos se negocian?
	\item ¿Qué operaciones se llevan a cabo?
	\item ¿Es un mercado eficiente?
\end{itemize}

\esquemacorto

\begin{esquema}[enumerate]
	\1[] \marcar{Introducción}
		\2 Contextualización
			\3 Globalización e internacionalización
			\3 Actualidad
			\3 Evolución de las características
		\2 Objeto
			\3 ¿Qué es el mercado de divisas?
			\3 ¿Qué es el tipo de cambio?
			\3 ¿Quiénes actúan en el mercado de divisas?
			\3 ¿Qué instrumentos se negocian?
			\3 ¿Qué operaciones se llevan a cabo?
			\3 ¿Es un mercado eficiente?
		\2 Estructura
			\3 Características, convenciones, agentes y eficiencia
			\3 Instrumentos
			\3 Operaciones
	\1 \marcar{Características, convenciones, agentes y eficiencia}
		\2 Características del mercado de divisas
			\3 Divisas:
			\3 Concepto de mercado de divisas
			\3 Convertibilidad de las divisas
			\3 Monedas de Libre Uso
			\3 Volumen de transacción
			\3 Principales divisas
			\3 Tiempo de apertura
			\3 Concentración geográfica
			\3 Over-the-Counter
		\2 Convenciones
			\3 Códigos de divisas
			\3 Pares
			\3 Moneda base y moneda de cuenta
			\3 Convención sobre orden de base y cuenta
			\3 Apodos de pares de divisas
			\3 Cotización directa e indirecta
			\3 Tipos de cambio
			\3 Bid, ask y spread
			\3 Pips y ticks
		\2 Agentes
			\3 Idea clave
			\3 Proveedores de liquidez
			\3 Demandantes de liquidez
	\1 \marcar{Instrumentos}
		\2 Spot
			\3 Idea clave
			\3 Características
			\3 Aplicaciones
		\2 Forwards
			\3 Idea clave
			\3 Características
			\3 Aplicaciones
		\2 Swaps de tipo de cambio/FX swaps
			\3 Idea clave
			\3 Características
			\3 Aplicaciones
		\2 Swaps de divisas/currency swaps
			\3 Idea clave
			\3 Características
			\3 Aplicaciones
		\2 Opciones
			\3 Idea clave
		\2 Otros productos
			\3 Differential swaps
			\3 Otros instrumentos ``exóticos''
	\1 \marcar{Operaciones}
		\2 Operaciones de cuenta corriente
			\3 Idea clave
			\3 Oferta de divisas
			\3 Demanda de divisas
		\2 Operaciones de cobertura
			\3 Idea clave
			\3 Forwards y futuros
			\3 Cuentas en divisas
			\3 Opciones sobre tipo de cambio
		\2 Operaciones de arbitraje
			\3 Idea clave
			\3 Arbitraje espacial
			\3 Arbitraje temporal
			\3 Arbitraje por latencia
		\2 Operaciones especulativas
			\3 Idea clave
			\3 Segmento spot: paridad descubierta
			\3 Segmento forward: paridad cubierta
			\3 Carry trade
		\2 Intervención
			\3 Idea clave
			\3 Instrumentos
			\3 Factores de éxito de la intervención
			\3 Valoración
		\2 Regulación
			\3 Idea clave
			\3 Instrumentos
			\3 Valoración
	\1 \marcar{Eficiencia del mercado de divisas}
		\2 Eficiencia del mercado de divisas
			\3 Factores determinantes de eficiencia
			\3 Hipótesis de eficiencia
			\3 Tests de eficiencia
			\3 Fuerte impacto de digitalización
		\2 Eficiencia del mercado de divisas
	\1[] \marcar{Conclusión}
		\2 Recapitulación
			\3 Características, agentes y convenciones
			\3 Instrumentos y operaciones
		\2 Idea final
			\3 Importancia del mercado de divisas
			\3 Determinación del tipo de cambio
			\3 Regímenes cambiarios
			\3 Cita de Edgeworth (1905)
			\3 Cita de Dornbusch (1983)

\end{esquema}

\esquemalargo



















\begin{esquemal}
	\1[] \marcar{Introducción}
		\2 Contextualización
			\3 Globalización e internacionalización
				\4 Régimen de Bretton Woods
				\4[] Progresiva liberalización de intercambios comerciales
				\4[] Progresiva apertura de las cuentas financieras
				\4[] $\Rightarrow$ Aumento de los intercambios de divisas
				\4[] $\Rightarrow$ Mercado de divisas: pilar de economía mundial
			\3 Actualidad
				\4 Mercado de divisas mayor mercado financiero
				\4 Media diaria (BIS 2019)
				\4[] $\sim$ 6.6 billones (españoles) de euros
				\4 Paul Volcker:
				\4[] Tipo de cambio es precio más importante
				\4 Tipo de cambio se determina en mercado de divisas
				\4[] Efectos sobre:
				\4[] $\to$ Exportaciones e importaciones
				\4[] $\to$ Inflación
				\4[] $\to$ Valor de las obligaciones con el exterior
				\4[] $\Rightarrow$ Efectos macroeconómicos
			\3 Evolución de las características
				\4 Mercado sujeto a continua transformación
				\4 Factores:
				\4[] Avances tecnológicos
				\4[] Contexto legal y político
				\4[] Intervención de bancos centrales
				\4[] Evolución de las economías
				\4[] Papel de los agentes participantes
				\4 Aumento de la transparencia y velocidad
				\4[] Redes globales permiten conectar oferta y demanda
		\2 Objeto
			\3 ¿Qué es el mercado de divisas?
			\3 ¿Qué es el tipo de cambio?
			\3 ¿Quiénes actúan en el mercado de divisas?
			\3 ¿Qué instrumentos se negocian?
			\3 ¿Qué operaciones se llevan a cabo?
			\3 ¿Es un mercado eficiente?
		\2 Estructura
			\3 Características, convenciones, agentes y eficiencia
			\3 Instrumentos
			\3 Operaciones
	\1 \marcar{Características, convenciones, agentes y eficiencia}
		\2 Características del mercado de divisas
			\3 Divisas:
				\4 Medios de pago en moneda extranjera
				\4 Mantenidos por residentes del país
				\4 Incluye:
				\4[] $\to$ Billetes y monedas extranjeros
				\4[] $\to$ Depósitos bancarios en moneda extranjera
				\4[] $\to$ Derechos a disponer sin restricción (cheques, talones, tarjetas de crédito)
				\4[] No son divisas:
				\4[] $\to$ Acciones
				\4[] $\to$ Obligaciones
				\4[] $\to$ Otros activos financieros en moneda extranjera
			\3 Concepto de mercado de divisas
				\4 Marco organizativo e instituciones que:
				\4[] Negocian e intercambian divisas extranjeras
				\4 Incluye:
				\4[] $\to$ Infraestructura física y humana
				\4[] $\to$ Conjunto de reglas, convenciones, instituciones
			\3 Convertibilidad de las divisas
				\4 Concepto de convertibilidad
				\4[] Intercambiable por otras sin restricciones
				\4[] No sujeta a controles cambiarios
				\4[] Sin restricciones para realizar pagos
				\4 Convertibilidad externa
				\4[] únicamente para no residentes
				\4 Convertibilidad interna
				\4[] Sólo para residentes en país de emisión
			\3 Monedas de Libre Uso
				\4 Definido por el FMI
				\4[] Monedas que:
				\4[] $\to$ Ampliamente utilizadas en trans. internacionales
				\4[] $\to$ Negociadas extensamente
				\4 Libre Uso es un requisito para cesta de DEG
				\4[] $\to$ Junto con valor alto de exportaciones
				\4 Libre uso no implica convertibilidad
				\4[] $\to$ LUso y restricciones de CFinanciera son posibles
				\4[] $\to$ Plenamente convertible pero no LUso posible
			\3 Volumen de transacción
				\4 Mayor mercado financiero del mundo
				\4[] En volumen de transacciones diarias
				\4 Media diaria (BIS 2019)
				\4[] 6.6 billones (españoles) de euros
				\4 Crecimiento muy fuerte desde 2003
				\4 Crisis financiera redujo volumen
				\4[] $\to$ Recuperación recuperó volúmenes
			\3 Principales divisas
				\4 Dólar, euro, yen, libra esterlina, franco suizo
				\4[] $\to$ Según Informe Trianual BIS 2019\footnote{En carpeta del tema.}
				\4 Ranking de monedas
				\4[] USD $\to$ 88\% de los pares intercambiados
				\4[] EUR $\to$ 32\%
				\4[] JPY $\to$ 17\%
				\4[] GBP $\to$ 12\%
				\4[] AUD $\to$ 6,8\%
				\4 Crisis de deuda soberana de zona euro
				\4[] Afectó volumenes del euro
				\4[] Progresiva recuperación
			\3 Tiempo de apertura
				\4 24 horas al día
				\4 6 días a la semana
				\4 Siempre hay algún centro financiero abierto
				\4[] Tokio, Singapur, HK,  Frankfurt, Londres, NY, Sydney
				\4 Volatilidad elevada en aperturas
				\4[] Sobre todo en aperturas europeas
				\4[] 30 mins pre-apertura y dos horas post-apertura
			\3 Concentración geográfica
				\4 Principalmente en 5 centros financieros
				\4[] Londres: 41\%
				\4[] Estados Unidos: 19\%
				\4[] Singapur: 5,7\%
				\4[] Japón: 5,6\%
				\4[] Hong Kong: 4,1\%
				\4[] $\then$ Más del 75\%
			\3 Over-the-Counter
				\4 Generalmente, contratos bilaterales
				\4 Sin mercados centralizados
				\4 Compensación incipiente
				\4[] Generalmente sin compensar
				\4[] Cámaras de compensación relativamente recientes
				\4[] CLS Bank
				\4[] $\to$ Creado en 2002
				\4[] $\to$ Mayores entidades del mercado de divisas
				\4[] $\to$ Volúmenes crecientes
		\2 Convenciones
			\3 Códigos de divisas
				\4 Tres letras
				\4[] Primeras dos letras:
				\4[] $\to$ País
				\4[] Última letra
				\4[] $\to$ Nombre de la divisa
				\4[] Ejemplo:
				\4[] JPY $\to$ JaPanese Yen
			\3 Pares
				\4 Pares mayores
				\4[] Emparejamientos entre USD y divisas principales
				\4 Pares cruzados
				\4[] $\to$ entre dos divisas principales sin USD
				\4[] Generalmente, dólar como intermediario
				\4 Nuevos desarrollos
				\4[] ¿Papel futuro del dólar tras aranceles?
				\4[] ¿Sancionados por USA dejarán de usar USD?
			\3 Moneda base y moneda de cuenta
				\4 Pares suelen expresarse como:
				\4[] EUR/USD, JPY/USD, EUR/JPY
				\4 Primer código de divisa:
				\4[] Moneda base/base currency
				\4 Segundo código de divisa:
				\4[] Moneda cuenta/quote currency
				\4 Precio de par XXX / YYY:
				\4[] Cantidad de YYY para comprar 1 de XXX
				\4[] Ej.: 1.1 EUR/USD
				\4[] $\to$ 1.1 USD por EUR
				\4 También denominadasforeign exchange market efficiency
				\4[] Base: CCY1/Currency 1
				\4[] Cuenta: CCY2/Currency 2
			\3 Convención sobre orden de base y cuenta
				\4 En mercados internacionales
				\4 Convenciones habituales sobre pares de divisas
				\4[] $\to$ Qué moneda debe ser cuenta
				\4[] $\to$ Qué moneda debe ser base
				\4 Moneda base es primera de lista
				\4[] 1. EUR\footnote{El BCE estipuló en 1999 que el EUR debía ser representado como moneda base en las cotizaciones con el resto de monedas.}
				\4[] 2. GBP
				\4[] 3. AUD
				\4[] 4. NZD
				\4[] 5. USD
				\4[] 6. CHF
				\4[] 7. JPY
				\4 Generalmente, moneda con más valor es base
				\4[] Más unidades de cuenta si es base
				\4[] $\to$ No siempre
				\4 En mercados locales, cambios de convención
				\4[] En algunos mercados locales
				\4[] $\to$ Agentes prefieren local sea cuenta
			\3 Apodos de pares de divisas
				\4 GBPUSD: cable
				\4 EURUSD: fiber
				\4 EURGBP: chunnel
				\4 EURJPY: yuppy
			\3 Cotización directa e indirecta
				\4 Implica definir extranjero y nacional
				\4 Cotización directa
				\4[] Precio de extranjera en términos de nacional
				\4[] En pares XXX/YYY
				\4[] $\to$ Moneda cuenta es moneda local
				\4[] Ej.: desde punto de vista de España
				\4[] $\to$ 1.12 euros/libra
				\4 Cotización indirecta
				\4[] Precio de nacional en términos de extranjera
				\4[] En pares XXX/YYY
				\4[] $\to$ Moneda base es moneda local
			\3 Tipos de cambio
				\4 Tipo de cambio nominal (E)
				\4[] Precio de una divisa en términos de otra
				\4 Tipo de cambio real (e)
				\4[] Precio relativo de los bienes de dos países
				\4[] $e = E \cdot \frac{P^*}{P}$
				\4 Tipo de cambio efectivo nominal (TCE)
				\4[] Ponderación del tipo nominal
				\4[] $\to$ En función del volumen de comercio/otros criterios
				\4[] $\text{TCEN} = \sum_i^n E_i \cdot \alpha_i$
				\4 Tipo de cambio efectivo real (TCER)
				\4[] Ponderación de tipos reales
				\4[] $\to$ Criterios similares a TCE
				\4[] $\text{TCER} = \sum_i^n e_i \cdot \alpha_i$
				\4[] Habitualmente, TCR en cotización indirecta
				\4[] $\then$ TCER más alto indica apreciación de moneda local
			\3 Bid, ask y spread
				\4 Dos precios diferentes para cada par
				\4[] Bid y ask
				\4[] Spread $\to$ Diferencia entre ambos
				\4 Bid:
				\4[] Precio al que me compran una divisa
				\4 Ask:
				\4[] Precio al que me venden una divisa
				\4 Spread
				\4[] $\text{Ask}-\text{Bid} = \text{Spread}$ $\then$ $\text{Spread} > 0$
				\4[] depende negativamente de liquidez
				\4[] constituye comisión del broker
			\3 Pips y ticks
				\4 Pip:
				\4[] Variación mínima cuantificable
				\4[] Milésima de euro en euros
				\4 Tick
				\4[] $\Delta$ producida en un intervalo de tiempo
		\2 Agentes
			\3 Idea clave
				\4 Instituciones o individuos que intercambian divisas
				\4 Causantes últimos de formación de precios
				\4 Importancia relativa depende de
				\4[] Agente genera o demanda liquidez
				\4[] Información de que dispone el agente
				\4 Generación de liquidez
				\4[] Estar dispuesto a absorber demanda de liquidez
				\4[] $\to$ En cualquier momento
				\4[] $\to$ Volumen flexible
				\4 Información
				\4[] Conjunto de datos que permiten a agentes
				\4[] $\to$ Postular distribución de prob. sobre precios
				\4[] $\Rightarrow$ Maximizar beneficios
				\4 Evolución histórica
				\4[] En primer momento, comerciantes de bienes y servicios
				\4[] $\to$ Necesario disponer de divisas para importar
				\4[] $\to$ Necesario colocar divisas recibidas por exportar
				\4[] Aparición de intermediarios
				\4[] $\to$ Emparejan vendedores y compradores
				\4[] $\to$ Proveen liquidez
				\4[] $\to$ Reducen costes de transacción y búsqueda
				\4[] $\to$ Intercambio mutuo con fines de especulación
				\4[] Actualidad
				\4[] $\to$ Plataformas electrónicas emparejan automáticamente
				\4[] $\to$ Redes globales mejoran emparejamiento
			\3 Proveedores de liquidez
				\4 Dealers/market makers
				\4[] Siempre preparados para comprar/vender
				\4[] $\to$ Por prestigio, siempre ofrecen bid/ask
				\4[] $\to$ Horquillas reducidas para grandes operaciones
				\4[] $\then$ Principales proveedores de liquidez
				\4[] Operaciones especulativas de gran volumen
				\4[] $\to$ En el mercado interdealer (39\% de total)
				\4[] Fundamentalmente, grandes bancos globales
				\4[] Agentes mejor informados del mercado
				\4[] $\to$ Conocen demanda/oferta de muchos clientes
				\4[] Algunos bancos centrales actúan como dealers
				\4 Bancos custodios
				\4[] Gestionan carteras de activos de inversores
				\4[] Negocian divisas en nombre de inversores
				\4[] $\to$ Proveen liquidez a clientes
				\4[] $\to$ Cargan mark-up sobre precio pagado a dealers
				\4[] $\Rightarrow$ Papel similar a brokerage
				\4 Plataformas de operadores minoristas
				\4[] Intermedian entre pequeños inversores
				\4[] Emparejan oferta y demanda de clientes
				\4[] Operan en la web
				\4[] Algunos también crean mercado
			\3 Demandantes de liquidez
				\4 Instituciones financieras
				\4[] Todas aquellas que no actúan como dealers
				\4[] $\to$ Bancos comerciales regionales o locales
				\4[] $\to$ Fondos de inversión y hedge funds
				\4[] $\to$ Gestoras de carteras de valores
				\4[] $\to$ Compañías de leasing
				\4[] $\to$ Aseguradoras
				\4[] $\to$ Bancos centrales (no siempre)
				\4[] Más informadas que no financieras
				\4[] Operan con fines:
				\4[] $\to$ Especulativos
				\4[] $\to$ Cobertura
				\4[] $\to$ Inversión a largo plazo
				\4 Sociedades no financieras
				\4[] Casi todas transacciones de ByS
				\4[] $\to$ Requieren transacciones en divisas
				\4[] Exportadores e importadores clientes de dealers
				\4[] Compraventa de divisas y cobertura
				\4 Inversores minoristas
				\4[] Pequeños volumenes
				\4[] Progresivo aumento
				\4[] Contacto con brokers y dealers a través de internet
				\4 Operadores de alta frecuencia
				\4[] Programas informáticos que ejecutan estrategias de:
				\4[] $\to$ Arbitraje
				\4[] $\to$ Especulación
				\4[] Ventaja competitiva en periodos muy cortos
				\4[] $\to$ High-frequency trading
	\1 \marcar{Instrumentos}
		\2 Spot
			\3 Idea clave
				\4 Transacciones directas de divisas
				\4[] Dos cantidades de una moneda por otra
			\3 Características
				\4 Fecha de liquidación:
				\4[] Máximo dos días después de cierre de contrato
				\4 Swaps al contado:
				\4[] Las que se liquidan de un día para otro
				\4 Volumen de transacción
				\4[] Se ha elevado en últimos años
				\4[] Más de 2 billones de dólares
			\3 Aplicaciones
				\4 Segundo instrumento más importante
				\4 Comercio
				\4 Intervención
		\2 Forwards
			\3 Idea clave
				\4 Transacciones directas de divisas
				\4[] Liquidadas más allá de dos días después
			\3 Características
				\4 Venta/compra de un par en momento futuro
				\4 Prima forward
				\4[] Diferencia positiva entre:
				\4[] $\to$ Precio de divisa en fecha futura
				\4[] $\to$ Precio de divisa en spot
				\4 CIP -- Covered interest parity
				\4[] Arbitrajistas eliminan posibilidad de:
				\4[] $\to$ Pedir prestado en divisa con interés más alto
				\4[] $\to$ Cambiar spot por divisa con interés más alto
				\4[] $\to$ Vender divisa comprada forward
				\4[] $\to$ Prestar divisa
				\4[] $\to$ Obtener interés en divisa
				\4[] $\then$ Obtener beneficio sin riesgo
				\4[] $\then$ Necesario $\frac{F_0}{S_0} = \frac{1+i}{1+i^*}$
				\4[] En mercados suficientemente líquidos
				\4[] $\to$ Se cumple condición
			\3 Aplicaciones
				\4 Cobertura
				\4 Segundo instrumento más importante
		\2 Swaps de tipo de cambio/FX swaps
			\3 Idea clave
				\4 Intercambio de divisas:
				\4[] $\to$ En diferentes momentos temporales
				\4[] $\to$ A precios definidos para cada intercambio
				\4 Compra y venta de divisas
				\4[] Acordada simultáneamente
				\4[] $\to$ Cada una en diferente momento
			\3 Características
				\4 Dividido en dos partes:
				\4[] ``\textit{Spot leg}''
				\4[] $\to$ A precio spot
				\4[] $\to$ Cantidad E de divisa XXX por $E^*$ de YYY
				\4[] ``\textit{Forward leg}''
				\4[] $\to$ A precio forward
				\4[] $\to$ Cantidad Divisa YYY por XXX
				\4 También posible \textit{forward}--\textit{forward}
				\4[] Dos patas de la operación son forward
			\3 Aplicaciones
				\4 Instrumento más utilizado
				\4[] $>2.5$ billones de volumen medio diario
				\4[] $\to$ Contabilizada sólo forward leg
				\4 Ejemplo:
				\4[] Spot: X USD por Y EUR
				\4[] Forward: Z EUR por W USD
				\4 Ejemplo de uso
				\4[] Empresa europea:
				\4[] $\to$ recibe pago USD en $t=0$
				\4[] $\to$ debe pagar en USD en $t+2$
				\4[] $\then$ Necesita EUR hasta $t+2$
				\4[] $\then$ Necesitará USD en $t+2$
				\4[] $\then$ No necesita USD hasta $t+2$
				\4[] Opciones:
				\4[] -- Vender USD spot en $t$ y comprar spot en $t+2$
				\4[] -- Contratar FX Swap
				\4[] FX Swap a contratar
				\4[] $\to$ Venta spot de USD por EUR en $t=0$
				\4[] $\to$ Compra forward de USD en $t+2$
				\4[] $\then$ Elimina riesgo de tipo de cambio hasta $t+2$
				\4[] $\then$ Cubre necesidades de USD
				\4[] $\then$ Dispone de EUR para actividades corrientes
		\2 Swaps de divisas/currency swaps
			\3 Idea clave
				\4 Intercambio de principal y cupones fijos
				\4[] En distintas divisas
				\4[] A lo largo de todo el periodo de vida
				\4[] $\to$ Desde obtención préstamo inicial hasta rendención
			\3 Características
			\3 Aplicaciones
				\4 Ejemplo: inversión en país extranjero
				\4 Inditex en Europa
				\4[] $\to$ Quiere llevar a cabo proyecto en Japón
				\4[] $\then$ Necesita JPY
				\4 Toyota en Japón
				\4[] $\to$ Quiere llevar a cabo proyecto en Europa
				\4[] $\then$ Necesita EUR
				\4 Inditex se financia más barato en EUR que Toyota
				\4[] Toyota se financia más baro en JPY que Inditex
				\4 Currency swap:
				\4[1.] Emisión de bonos en mercados respectivos
				\4[] $\to$ Inditex emite bonos EUR en Europa
				\4[] $\to$ Toyota emite bonos JPY en Japón
				\4[2.] Intercambio de ingresos por emisión de bonos
				\4[] $\to$ Inditex Transfiere EUR recibidos a Toyota
				\4[] $\to$ Toyota Transfiere JPY recibidos a Inditex
				\4[3.] Intercambio de intereses
				\4[] $\to$ Inditex paga interés de bonos JPY a Toyota
				\4[] $\to$ Toyota paga interés de bonos EUR a Inditex
				\4[4.] Pago de principales
				\4[] $\to$ Inditex paga principal de bono JPY a Toyota
				\4[] $\to$ Toyota paga principal de bono EUR a Inditex
				\4[5.] Resultado neto
				\4[] $\to$ Inditex se ha endeudado en JPY a coste de Toyota
				\4[] $\to$ Toyota se ha endeudado en EUR a coste de Inditex
		\2 Opciones
			\3 Idea clave
				\4 Derecho a comprar/vender una divisa
				\4[] $\to$ A tipo de cambio determinado (strike)
				\4[] $\to$ En una fecha concreta (vencimiento)
				\4 Currency warrants
				\4[] Opciones a muy largo plazo (> 1 año)
		\2 Otros productos
			\3 Differential swaps
				\4[] Swaps con:
				\4[] $\to$ Principal en una divisa
				\4[] $\to$ Flujos de interés en otra
			\3 Otros instrumentos ``exóticos''
	\1 \marcar{Operaciones}
		\2 Operaciones de cuenta corriente
			\3 Idea clave
				\4 MdDivisas hace posible comercio de ByS
				\4 Segmento spot principal
				\4 Sobre todo, sociedades no financieras
			\3 Oferta de divisas
				\4 Exportaciones de ByS
				\4 Entradas de capitales
				\4 Creciente en TCN directo
				\4[] TCN directo más alto
				\4[] $\to$ Exportaciones más baratas en divisa
				\4[] $\then$ Aumentan exportaciones
				\4[] $\then$ Compra de moneda local por divisa
			\3 Demanda de divisas
				\4 Importaciones de ByS
				\4 Salidas de capitales
				\4 Decreciente en TCN directo
				\4[] TCN directo más alto
				\4[] $\to$ Importaciones más baratas en moneda local
				\4[] $\then$ Aumentan importaciones
				\4[] $\then$ Venta de moneda local por divisa
		\2 Operaciones de cobertura
			\3 Idea clave
				\4 Cubrir riesgo de tipo de cambio
				\4 Tipo de cambio futuro puede fluctuar
				\4[] Agente opera en moneda nacional
				\4[] Realiza cobros/pagos en divisas
				\4[] $\to$ Necesita eliminar riesgo sobre cuantía del pago/cobro
				\4 Cobertura total vs parcial
				\4[] Habitualmente, cobertura parcial
				\4[] $\to$ Minimiza riesgo de pérdida
				\4[] $\to$ Mantiene cierto riesgo favorable
			\3 Forwards y futuros
				\4 Aseguran tipo de cambio futuro
				\4[] $\to$ Eliminan riesgo de tipo de cambio
				\4 Ampliamente utilizados en:
				\4[] Import/export
				\4[] Reducción de riesgos de tesorería
				\4 Cobertura de exportación
				\4[] Si importadores pagan en su divisa
				\4[] $\to$ Ponerse corto en forward de la divisa
			\3 Cuentas en divisas
				\4 Cuentas bancarias denominadas en divisas
				\4 Utilizadas por importadores y exportadores
				\4 Permiten armonizar fluctuaciones de tesorería
				\4 Junto a forwards, flexibilizan pagos/cobros
			\3 Opciones sobre tipo de cambio
				\4 Permiten eliminar riesgo de tipo de cambio
				\4 Múltiples variedades y estrategias
				\4 Derecho a comprar/vender una determinada divisa
				\4 Menor coste que forward
		\2 Operaciones de arbitraje
			\3 Idea clave
				\4 Comprar y vender simultáneamente
				\4[] Una divisa en dos mercados distintos
				\4[] $\to$ En los que prevalecen distintos precios
				\4[] $\then$ Para obtener beneficio sin riesgo
			\3 Arbitraje espacial
				\4 Diferente cotización en diferentes plazas
				\4 Mismo segmento temporal e instrumento
			\3 Arbitraje temporal
				\4 Incumplimientos de la Covered Interest Parity
				\4 Prima de cotización del forward:
				\4[] Margen forward: $\frac{F-S}{S}$
				\4[] Asumiendo tipo directo
				\4[] Si margen > 0 $\then$ moneda local se deprecia
				\4 Diferencial de tipos de interés:
				\4[] $i-i^*$ donde $i^*$ es interés de la divisa
				\4[] Si $i-i^* > 0$, moneda local más rentable
				\4 Paridad cubierta de tipos se cumple si
				\4[] $i-i^* \approx \frac{F-S}{S}$
				\4[] $\to$ = retorno con interés o forwards
				\4 Si no se cumple:
				\4[] $\then$ Oportunidad de arbitraje
			\3 Arbitraje por latencia
				\4 Retardo en transmisión de precios y quotes
				\4[] $\to$ Aprovechables por quien accede a info
				\4[] $\then$ Arbitraje
		\2 Operaciones especulativas
			\3 Idea clave
				\4 Inversores asumen posiciones arriesgadas
				\4[] Esperando crear valor
				\4[] $\to$ Utilizando información pública y privada
			\3 Segmento spot: paridad descubierta
				\4 Especulador estima tipo spot futuro
				\4[] Cree en desviación de Uncovered Interest Parity
				\4[] $\to$ Compran/venden divisa para aprovechar
			\3 Segmento forward: paridad cubierta
				\4 Especulador estima tipo spot futuro
				\4[] Compara con tipo forward
				\4[] $\to$ Compra/vende spot y forward para aprovechar
				\4 Ejemplo:
				\4[] GBP/EUR spot esperado: 1.5
				\4[] GBP/EUR forward: 1.4
				\4[] $\to$ Estima GBP estará más caro que forwards
				\4[] $\then$ Venderá GBP spot y comprará forward
			\3 Carry trade
				\4 Aprovechar diferencial de tipos de interés
				\4 Estrategia
				\4[] Endeudarse en divisa de interés bajo
				\4[] Convertir a divisa de interés alto
				\4[] Prestar en divisa de interés alto
				\4 Influido por políticas de QE
				\4 Si carry trade es rentable
				\4[] $\to$ UIP no se está cumpliendo
		\2 Intervención
			\3 Idea clave
				\4 BCentrales son agentes en mercados financieros
				\4[] En mercados cambiarios
				\4[] $\to$ Compran y venden divisas y moneda local
				\4[] En mercado abierto
				\4[] $\to$ Compran y venden deuda nacional
				\4 Actuaciones en mercados financieros internacionales
				\4[] Afecta TCN que prevalece
				\4[] $\to$ Intervención afecta regímenes cambiarios
				\4[] $\then$ Herramienta para implementar régimen cambiario
				\4[] $\then$ Existen también otras razones para intervenir
			\3 Instrumentos
				\4 Compraventa de divisas
				\4[] Aut. monetaria compra/vende directamente
				\4[] Ejemplo:
				\4[] $\to$ BoJ en septiembre de 2010
				\4[] $\to$ Compra 24 billones de dólares en un día
				\4 Tipo de interés
				\4[] Variar interés doméstico
				\4[] $\to$ Para atraer capital
				\4[] $\then$ Sostener TCN
				\4[] Ejemplo: Turquía en verano de 2018
				\4 Acumulación de reservas
				\4[] Compra de activos líquidos en divisas
				\4[] $\to$ A cambio de moneda nacional
				\4[] $\then$ Presión hacia depreciación moneda nacional
				\4[] Desde principios años 2000
				\4[] $\to$ Enorme aumento de reservas en emergentes
				\4[] ¿Por qué aumentar reservas?
				\4[] $\to$ Sostener TCN si presión a la baja
				\4[] $\to$ Mantener acceso a financiación de CC
				\4[] $\to$ Mantener gasto público en recesión
				\4 Esterilización
				\4[] Ante compraventa de divisas
				\4[] $\to$ ¿Qué sucede con balance de BCentral?
				\4[] $\to$ ¿Crece? ¿Decrece?
				\4[] Esterilización es mantener tamaño del balance
				\4[] $\to$ Aunque varíe la cantidad de reservas
				\4[] Cómo esterilizar
				\4[] $\to$ $\Delta$ de crédito doméstico igual a $\Delta$ de activos exteriores netos
				\4[] Debate sobre efectividad de interv. esterilizada
				\4 Reputación del banco central
				\4[] Percepción de agentes del mercado
				\4[] $\to$ Capacidad para intervenir con éxito
				\4[] Trayectoria de disciplina monetaria
				\4[] $\to$ Aumenta reputación
				\4[] $\then$ Aumenta poder para afectar expectativas
				\4 Coordinación de política monetaria
				\4[] Intervención coordinada con otros bancos centrales
				\4 Expansión cuantitativa
			\3 Factores de éxito de la intervención
				\4[i] Percepción fuerte o débil del mercado
				\4[] Mercados con opiniones ``fuertes'' sobre TCN de equilibrio
				\4[] $\to$ Muy dificil afectar vía intervención
				\4[] Mercados tienen opinión débil sobre evolución futura
				\4[] $\to$ Intervención sirve como ``nudge''
				\4[] $\then$ Permite fijar opinión de agentes
				\4[ii] Factor sorpresa de la intervención
				\4[] Intervención esperada puede haberse descontado ya
				\4[] Intervenciones inesperadas tienen más efecto
				\4[] $\to$ Incluso, overshooting
				\4[iii] Operaciones concertadas con otros bancos centrales
				\4[] Obviamente, BCNs interviniendo en distinta dirección
				\4[] $\to$ Tienen menor efecto
				\4[iv] Expectativas de los agentes respecto futuro
				\4[] Factor principal de éxito de intervención
				\4[] Agentes estiman intervención será exitosa
				\4[] $\to$ Ellos mismos actúan en la dirección de intervención
				\4[] $\then$ Intervención puede de hecho no ser necesaria
				\4[v] Cambios en fundamentales
				\4[] Intervención que no afecta fundamentales
				\4[] $\to$ Menor probabilidad de éxito a m/p
			\3 Valoración
				\4 ¿La intervención funciona?
				\4[] Debate académico y policy-making
				\4 Canales de actuación de la intervención
				\4[] Canal del balance
				\4[] $\to$ Variación en ofertas relativas
				\4[] $\to$ Ratio divisas/domésticos cambia aun con esterilización
				\4[] $\to$ No son sustitutivos perfectos
				\4[] $\then$ Agentes buscan equilibrar sus carteras
				\4[] $\then$ Efectos sobre TCN
				\4[] Canal de las expectativas
				\4[] $\to$ Qué TCN esperan los agentes dados anuncios
				\4[] $\to$ Señalización de intervención futura
				\4[] $\to$ Evitan actuar en contra de intervención
				\4[] $\to$ Pueden actuar a favor
				\4[] Canal de la coordinación
				\4[] $\to$ Intervención es señal para desencadenar flujos
				\4[] $\to$ Agentes privados y domésticos
				\4 Efectividad
				\4[] Generalmente, no esterilizada es efectiva
				\4[] Esterilizada puede ser efectiva también
				\4[] $\to$ Aunque resultados ambiguos
				\4[] $\to$ En el l/p, con algunos países no lo es
				\4 Efectividad de intervención esterilizada
				\4[] Debate de largo alcance
				\4[] Contrarios a efectividad
				\4[] $\to$ Mercados corrigen intervención rápidamente
				\4[] $\to$ Como mucho, efecto a corto plazo
				\4[] Favorables a efectividad
				\4[] $\to$ Mercados segmentados hacen efecto sea real
				\4[] $\to$ Efectos de c/p pueden inducir efectos l/p
		\2 Regulación
			\3 Idea clave
				\4 Mercado de divisas determina TCN
				\4 Gobierno puede ejercer coerción
				\4[] Imponer límites o prohibiciones:
				\4[] $\to$ Qué transacciones en divisas
				\4[] $\to$ A qué tasa pueden venderse divisas
				\4[] $\to$ Cuánto se puede comprar/vender
			\3 Instrumentos
				\4 Controles de capital
				\4[] Restricciones en cuenta financiera
				\4[] $\to$ Qué activos financieros pueden intercambiarse
				\4[] $\to$ Quién puede obligarse con no residentes
				\4[] $\to$ Quién puede comprar activos financieros en divisas
				\4[] $\to$ Cuánto capital puede circular
				\4[] $\to$ Impuestos a entradas de capital
				\4[] Permitidos en BW y FMI
				\4[] Años 80 y hasta crisis de 90s
				\4[] $\to$ Liberalización de mov. de K es necesaria
				\4[] Actualidad
				\4[] $\to$ Algunas experiencias positivas con control de K
				\4[] $\to$ Aumentan costes de financiación de empresas
				\4[] $\to$ Reducen volatilidad de TC
				\4[] $\to$ Pueden reducir coste de ajuste en crisis monetarias
				\4 Control de cambio
				\4[] Restricciones en mercado de divisas
				\4[] $\to$ Caso particular de control de K
				\4[] Muy habituales en el pasado
				\4[] Actualidad:
				\4[] $\to$ Persisten en formas muy débiles
				\4[] $\to$ Notificación a BC
				\4[] $\to$ Otros
				\4 Tipos de cambio múltiples
				\4[] Distintos TC según uso
				\4[] $\to$ Importación
				\4[] $\to$ Turismo
				\4[] $\to$ ...
				\4[] Objetivos habituales
				\4[] $\to$ Subvencionar sector exportador
				\4[] $\to$ Limitar salida de divisas
			\3 Valoración
				\4 Efectividad de la regulación
				\4[] Depende de:
				\4[] $\to$ Desarrollo de sistema financiero
				\4[] $\to$ Rigidez de regulación
				\4[] $\to$ Desajuste fundamental
				\4[] $\to$ Voluntad política y economía política
				\4[] Largo plazo
				\4[] $\to$ Aumenta dificultad para regular flujos
				\4[] $\to$ Aparición de nuevas formas de mov. K
				\4[] $\to$ Ejemplo: bitcoin en Venezuela
				\4 Aplicaciones efectivas
				\4[] China
				\4[] $\to$ Aislamiento de crisis asiática
				\4[] Malasia
				\4[] $\to$ Salida rápida tras crisis
				\4[] Chipre
				\4[] $\to$ Controles de corto plazo tras crisis bancaria
				\4 Aplicaciones fallidas
				\4[] Aumentan incentivos a no reformar
				\4[] Reducen disciplina fiscal
				\4[] Tentación de políticas monetarias expansivas
				\4[] $\to$ Que acaban disminuyendo efectividad de regulación
	\1 \marcar{Eficiencia del mercado de divisas}
		\2 Eficiencia del mercado de divisas\footnote{Ver \href{https://web.stanford.edu/class/msande247s/2008/fourth\%20week\%20posting/2008chap07a\%200716\%202008.pdf} y \href{https://www.athensjournals.gr/business/2018-1-X-Y-Kallianiotis.pdf}{Kallianiotis: How efficient is the foreign exchange market.}.}
			\3 Factores determinantes de eficiencia
				\4 Amplitud
				\4[] Mayor gama de valores negociados
				\4[] $\to$ Mayor posibilidad de cubrir estados de naturaleza
				\4 Profundidad
				\4[] Depende de volumen de negociación
				\4[] Capacidad de absorción de órdenes de compra/venta
				\4[] Volumen que puede ser ejecutado
				\4[] $\to$ Sin afectar significativamente al precio
				\4 Libertad
				\4[] Entrada y salida libre
				\4[] Coste reducido de transacción
				\4 Transparencia
				\4[] Obtención de información no es costosa
				\4[] Agentes conocen precios de oferta y demanda
				\4[] Intermediarios tienen poca información privada que aprovechar
				\4 Liquidez
				\4[] Emparejamiento de comprador y vendedor poco costosa
				\4[] Fácil encontrar contrapartida de operaciones
				\4[$\then$] Implican que el mercado se acerca a eficiencia
			\3 Hipótesis de eficiencia
				\4 Concepto relativamente antiguo
				\4[] Bachelier (1900), Samuelson (1965),
				\4[] Fama, Black...
				\4 ¿El tipo de cambio refleja info. disponible?
				\4[] ¿Qué es reflejar?
				\4[] $\to$ Elimina beneficio econ. ajustando por riesgo
				\4[] ¿Qué información está disponible?
				\4[] $\to$ Privada, semipública, pública
				\4 Difícil respuesta
				\4[] Muy difícil test empírico
				\4[] Testar eficiencia implica
				\4[] $\to$ Test de varios supuestos accesorios
				\4[] $\then$ Imposible saber qué se cumple realmente
			\3 Tests de eficiencia
				\4 Forex es uno de los que más se acerca
				\4 Covered Interest Parity
				\4[] Se cumple generalmente
				\4[] $\to$ Especialmente, pares más frecuentes
				\4[] Oportunidades de arbitraje eliminadas
				\4 Uncovered Interest Parity
				\4[] En ocasiones, no se cumple
				\4[] $\to$ De hecho, diferencial de interés
				\4[] $\then$ Predictor de apreciación
				\4[] Posibles explicaciones
				\4[] $\to$ Mercado sí se acerca a eficiencia
				\4[] $\then$ Asigna prima de riesgo al con más interés
				\4 Comparación con paseo aleatorio
				\4[] Paseo aleatorio suele ser mejor predictor
				\4[] $\to$ Indicativo de eficiencia
				\4[] Especialmente, EURUSD
				\4 Test de eficiencia básico
				\4[] Muy numerosos tests, con otros supuestos
				\4[] Testar uncovered interest parity
				\4[] $\to$ Comparar margen forward y $\Delta$ de spots
				\4[] Estimar: $S_{t+n} - S_t = \beta_0 + \beta_1 (F_t^{t+n} - S_t) + u_{t+n}$
				\4[] $\to$ $\beta_1 \neq 0$ $\then$ Margen forward predice
				\4[] Resultados empíricos habituales: $\beta_1 < 0$
				\4[] $\then$ Paradoja del margen forward
				\4[] $\then$ Forward es estimador sesgado
				\4 Explicaciones a paradoja
				\4[] Inversores son aversos al riesgo
				\4[] $\then$ Aplican prima de riesgo
				\4[] Expectativas formadas ineficientemente
				\4[] ...
			\3 Fuerte impacto de digitalización
				\4 Aumento de transparencia
				\4[] $\to$ Más fácil y barato descubrimiento de precios
				\4 Reducción de coste de transacción
				\4[] Más fácil encontrar oferta y demanda
				\4 Mayor número de agentes y mejor matching
				\4[] Reducciones de costes operativos
				\4[] Negociación y ejecución vía electrónica
				\4[] $\then$ Reducción de spreads
				\4[] $\then$ Reducción spreads relativos dealer--minoristas
				\4 Concentración de dealers
				\4[] Mayores dealers aumentan cuota de mercado
				\4[] Bancos regionales especializados en divisas pequeñas
				\4 Menor importancia del mercado interdealer
				\4[] Acceso más fácil de minoristas
				\4[] Plataformas de operadores minoristas
				\4[] Agregadores de liquidez
				\4[] $\to$ Recopilan información sobre bid/ask
				\4[] $\to$ Ofrecen mejores precios a todas plataformas
				\4 Big data
				\4[] Grandes plataformas analizan datos de transacciones
				\4[] $\to$ Customer profiling
				\4 Trading automatizado
				\4[] Trading algorítimico
				\4[] High frequency trading
				\4[] $\to$ \% elevado de mercados spot
				\4[] $\to$ Aumentan liquidez
				\4[] $\to$ Debate sobre si aumentan volatilidad
		\2 Eficiencia del mercado de divisas
	\1[] \marcar{Conclusión}
		\2 Recapitulación
			\3 Características, agentes y convenciones
			\3 Instrumentos y operaciones
		\2 Idea final
			\3 Importancia del mercado de divisas
				\4 Importancia capital en economías abiertas
				\4 Determina:
				\4[] Precio de exportación/importación
				\4 Canal de transmisión de política monetaria
				\4[] Mecanismo de intervención de BC
				\4 Prestigio internacional
				\4 Crisis financieras internacionales
				\4[] FX es pieza fundamental
				\4[] Factor de crisis y contagio
			\3 Determinación del tipo de cambio
				\4 Gran problema teórico y empírico
				\4[] Más importante tras fin de BWoods
				\4[] ¿De qué depende el tipo de cambio?
				\4[] ¿Qué tipo de cambio mañana?
				\4[] ¿Qué conclusiones de política económica?
			\3 Regímenes cambiarios
				\4 ¿Cómo deben intervenir las autoridades?
				\4 ¿Tipo de cambio debe ser fijo o libre?
				\4 ¿Cómo mantener un tipo de cambio fijo?
				\4[$\then$] Todo estado debe preguntarse
				\4[] Conocimiento de mercado de divisas
				\4[] $\to$ Respuestas acertadas
			\3 Cita de Edgeworth (1905)
				\4 $\Delta$ de CC es como manillas de reloj
				\4[] Resultado de muchos engranajes ocultos
				\4[] $\to$ Cuyo funcionamiento no observamos a priori
				\4[] $\then$ TC resulta de esos mecanismos ocultos
			\3 Cita de Dornbusch (1983)
				\4 Los modelos de TCN son visiones parciales
				\4[] Cada uno explica aspecto importante
				\4[] $\to$ En un episodio histórico determinado
\end{esquemal}			




































			
%	\1 Tipo de cambio
%		\2 Idea clave
%		\2 Tipo de cambio nominal
%		\2 Tipo de cambio real
%		\2 Tipo de cambio efectivo
%	\1 Mercados de divisas
%		\2 Definición
%	\1 Agentes
%		\2 Institucional\footnote{Extraido de Handbook of Foreign Exchange Rates Ch. 1 - Who needs and who provides liquidity?}
%			\3 Dealers
%			\3 Bancos custodios
%			\3 Plataformas de operadores minoristas\footnote{Conocidos como \textit{retail aggregators}.}
%			\3 Instituciones financieras
%			\3 Sociedades no financieras
%			\3 Inversores no minoristas
%			\3 Operadores de alta frecuencia
%		\2 Funcional\footnote{Extraido de Gandolfo - Ch. 2.}
%			\3 Especuladores
%			\3 No especuladores
%			\3 Instituciones financieras
%	\1 Instrumentos
%		\2 Spot
%		\2 Forward
%		\2 Swaps de tipo de cambio
%		\2 Currency swaps
%		\2 Opciones
%		\2 otros
%	\1 Operaciones
%	\1 Eficiencia
%	\1[] Conclusión
%		\2 Recapitulación
%		\2 Idea final
%\end{esquema}
%\end{multicols}

\preguntas

\seccion{Test 2013}

\textbf{32.} Un especulador de divisas espera que el tipo de cambio al contado del euro respecto al dólar dentro de tres meses (desde hoy) sea menor que el tipo de cambio a futuros del dólar con entrega en tres meses (desde hoy). El especulador:
\begin{enumerate}
	\item[a] Comprará euros en el mercado al contado a partir de hoy, durante tres meses.
	\item[b] Comprará euros a futuros y los revenderá en el mercado al contado en tres meses (a partir de hoy).
	\item[c] Venderá euros en el mercado al contado (a partir de hoy) durante tres meses.
	\item[d] Venderá euros a futuros hoy y los comprará en el mercado al contado en tres meses, contados a partir de hoy.
\end{enumerate}

\textbf{35.} Indique qué sentencia es \textbf{FALSA} sobre el riesgo de cambio:

\begin{enumerate}
	\item[a] Si un banco español concede un préstamo en coronas a una empresa sueca, se beneficiará si la corona se deprecia contra el euro.
	\item[b] Si una empresa española vende ordenadores en España que previamente compra en Estados Unidos, le perjudicará que el dólar se aprecie contra el euro.
	\item[c] Si una compañía matriz española avala un préstamo en yenes a una empresa filial radicada en Japón, saldrá perjudicada si el euro se deprecia contra el yen.
	\item[d] Marque d) si no ha marcado ninguna de las anteriores. 
\end{enumerate}

\seccion{Test 2011}

\textbf{34.} Suponiendo dos países, País A y País B. El tipo de cambio spot es de 1 Moneda A/Moneda B, el tipo forward a un año es 2 Moneda A/Moneda B, el país A tiene un tipo de interés anual del 10\% y el país B tiene un tipo de interés anual del 5\%. ¿Qué beneficio neto obtendría al cabo de un año un arbitrajista que obtiene un préstamo de 100 Monedas A en el momento actual?
\begin{enumerate}
	\item[a] 200 Monedas A.
	\item[b] 210 Monedas A.
	\item[c] 100 Monedas A.
	\item[d] No existen posibilidades de arbitraje.
\end{enumerate}

\notas

Ver Wang y Sarno más allá de Gandolfo.

\textbf{2013}: \textbf{32.} B. \textbf{35.} A

\textbf{2011}: \textbf{34.} C

\bibliografia

Mirar en Palgrave:
\begin{itemize}
    \item foreing exchange market microstructure
    \item foreign exchange markets, history of
\end{itemize}

Aguilar García. \textit{Temario de CECO 2016}. Tema 14: Mercado de divisas: operaciones e instrumentos.

Dornbusch, R. (1986) \textit{Exchange Rate Economics: 1986)} NBER Working Paper Series -- En carpeta del tema

Gandolfo, G. \textit{International Finance and Open-Economy Macroeconomics}. Ch. 2, 3

James, J.; Warsh, I. W.; Sarno, L. \textit{Handbook of Exchange Rates} (2012) Ch. 1 Foreign Exchange Market Structure, Players and Evolution

MacDonald, R. \textit{Exchange Rate Economics. Theories and evidence.} (2007) -- En carpeta de economía internacional

Sarno, L. \textit{Taylor, M.} \textit{The economics of exchange rates} (2002) -- En carpeta de economía internacional

Shama, S. \textit{A Foreign Exchange Primer} (2008) -- En carpeta de economía internacional

Taylor, M. P. (1995) \textit{The Economics of Exchange Rates} Journal of Economic Literature Vol. XXXIII -- En carpeta del tema

Wang, P. \textit{The Economics of Foreign Exchange and Global Finance} (2005) 2nd Edition -- En carpeta de economía internacional

\end{document}
