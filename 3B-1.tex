\documentclass{nuevotema}

\tema{3B-1}
\titulo{La información financiera de las empresas: estados de situación y de circulación. Métodos de análisis económico y financiero de la empresa.}

\begin{document}

\ideaclave

Mencionar el nuevo EINF introducido en 2018 -- Estado de Información No Financiera. Ver \url{https://diarioresponsable.com/opinion/27134-informacion-no-financiera-las-memorias-de-sostenibilidad-ya-son-obligatorias}

Si existe una máxima que resume la utilidad de la información financiera, es aquella que reza: ``\textit{no se puede gestionar lo que no se conoce}''. Así, en el contexto de la gestión de una empresa lo que debe conocerse es precisamente la relación de transacciones, obligaciones y patrimonio a las que la actividad de las empresas dan lugar y se conoce como información financiera. Para que pueda utilizarse para el fin por el que se recoge, la información financiera requiere de una presentación organizada y sistemática que permita al usuario comparar a fin de invertir en activos de la empresa, valorar para comprar o vender o determinar las obligaciones tributarias que se desprenden de la actividad empresarial. Así, la información financiera es en definitiva un conjunto de datos organizado que se puede utilizar con fines diversos pero que se estructura de acuerdo con unas pautas que la hacen susceptible de ser analizada. La exposición tiene por \textbf{objeto} tratar de contestar a las preguntas que se desprenden lógicamente de la idea anterior: ¿en qué consiste la información? ¿para qué sirve? ¿cómo se organiza?, y ¿cómo se analiza? La \textbf{estructura} de la exposición comienza por  presentar el concepto y los principios básicos que ordenan y estructuran la información financiera, así como el contexto legal que regula su elaboración y compilación. Posteriormente se examinan las cuentas anuales del Plan General Contable de 2007 dada su preeminencia en la práctica empresarial española. Por último se plantean los fundamentos del análisis de los estados contables desde el punto de vista de la liquidez, la solvencia y el resultado.

En primer lugar es necesario esclarecer el \textbf{concepto} de \marcar{información financiera}. Entendemos que el término hace referencia a un conjunto de datos organizado de forma sistemática y estructurado temporalmente que describe el patrimonio de una empresa, los derechos y obligaciones que la afectan y los flujos financieros y monetarios que la afectan en un periodo determinado. Los \textbf{usuarios} de la información financiera son tanto internos --en tanto que actúen como gestores de la empresa- o externos --en tanto que no tengan un interés directo en la gestión de la empresa pero sí puedan verse potencialmente afectados por las decisiones tomadas en su seno como es el caso de clientes, proveedores o Administración Pública-. La información financiera debe, para cumplir proporcionar efectivamente un valor a sus usuarios, con una serie de \textbf{requisitos} generales entre los cuales es habitual mencionar la relevancia, la fiabilidad, la integridad, la comparabilidad y la claridad. Para cumplir con estos requisitos, la información financiera debe compilarse obligatoriamente de acuerdo con los \textbf{principios} de empresa en funcionamiento, devengo, uniformidad, prudencia, no compensación e importancia relativa. El método de partida doble, de acuerdo con el cual todo apunte contable debe reflejarse doblemente con igual cuantía pero signo opuesto da lugar a la identidad contable básica por la que la suma del pasivo y el patrimonio neto son iguales al activo. Los elementos de la información financiera son en último término cuantías monetarias a las que debe asignarse un valor y para ello existen diferentes \textbf{criterios de valoración} entre los que se cuentan el criterio de coste histórico, el valor razonable, el valor neto realizable, el valor actual, el valor en uso, el valor residual, etc... La valoración de cada elemento del pasivo admite uno o varios criterios de valoración especificado en el propio PGC. El \textbf{ciclo contable} puede dividirse en tres fases: \textit{i)} recogida de datos, \textit{ii)} elaboración, \textit{iii)} aprobación y registro de las cuentas. La legislación relevante a la compilación de la información financiera a nivel internacional tiene como elemento clave las Normas Internacionales de Información Financiera del \textit{International Accounting Standards Board}, creadas inicialmente con el objetivo de armonizar la información financiera en la Unión Europea y trasladadas al ordenamiento español por el Plan General de Contabilidad de 2007. En EEUU y otras jurisdicciones estos principios no han sido adoptados de forma general y rigen los llamados \textit{Generally Accepted Accounting Principles} americanos o los GAAP locales aplicables. En el ámbito nacional, la legislación relevante incluye como elemento central al PGC de 2007 pero también al Código de Comercio, la Ley de Auditoría de Cuentas, el PGC de Pequeñas y Medianas Empresas y el Real Decreto 602/2016 por el que se modifican los dos PGC.

El \marcar{PGC de 2007} dispone la obligación de compilar cinco documentos que en conjunto se denominadas \textbf{Cuentas Anuales}, y son las siguientes: el balance de situación, la cuenta de pérdidas y ganancias, el estado de cambios en el patrimonio neto, el estado de flujos de efectivo y la memoria. El \textbf{balance de situación} da cuenta del capital empleado y el capital utilizado y resume el estado global de la empresa respecto a su patrimonio y con ello, respecto a su solvencia y su liquidez. Se presenta habitualmente en dos columnas. En una de ellas se incluyen los elementos del activo ordenados de menor a mayor liquidez, y en otra los elementos del patrimonio neto y el pasivo por ese orden, es decir, de menor a mayor exigibilidad. La \textbf{cuenta de pérdidas y ganancias} presenta de forma desagregada en función de los destinatarios finales el valor generado por la empresa en el periodo en cuestión. Así, se estructura en cuatro epígrafes: el resultado de explotación, el resultado financiero, el resultado antes de impuestos, el impuesto sobre el beneficio y el resultado del ejercicio. El \textbf{estado de cambios en el patrimonio neto} resume las variaciones en el patrimonio neto y se divide en dos apartados, el \textbf{estado de ingresos y gastos reconocidos} y el \textbf{estado total de cambios en el patrimonio neto}. El \textbf{estado de flujos de efectivo} es un sumario de las variaciones en los activos monetarios a disposición de la empresa en función de las actividades que los han dado lugar. Se divide en cuatro epígrafes: los flujos de efectivo de las actividades de explotación, los flujos de efectivo de las actividades de inversión, los flujos de efectivo de las actividades de financiación y el efecto de las variaciones del tipo de cambio. Por último, la \textbf{memoria} amplía y complementa con informaciones adicionales el resto de las cuentas anuales e incluye información relevante a efectos fiscales y en relación a las normas aplicadas. Otros documentos no especificados en el PGC pero habitualmente aportados junto con las cuentas anuales son el Informe de Auditoría y el Informe de Gestión. Existen dos posibles \textbf{variaciones} respecto de las cuentas anuales estándar: las cuentas abreviadas y las cuentas del PGC de PYMES. Se caracterizan por dispensar de la obligación de presentar estado de flujos de efectivo y estado de cambios en el patrimonio neto, así como por requerir el cumplimiento de dos de tres requisitos durante dos ejercicios consecutivos y cumplidos a fecha de cierre de ejercicio. Los requisitos concretos son iguales para las PyMES y para la formulación del balance y la memoria abreviados, y son más laxos para la formulación de la cuenta de pérdidas y ganancias abreviada.

La información financiera tiene como objetivo último ser analizada para extraer conclusiones acerca de la capacidad de la empresa para hacer frente a sus compromisos inmediatos, su capacidad para hacer frente a sus compromisos en caso de liquidación, y su capacidad para generar valor. Esta actividad se denomina generalmente como \marcar{análisis de estados contables}. El primer aspecto de la contabilidad de una empresa es el \textbf{análisis de la liquidez}. Para ello, se parte de los dos conceptos centrales: el fondo de maniobra y las necesidades operativas de fondos. Las necesidades operativas de fondos son iguales a la suma de las existencias, los adelantos a clientes y la tesorería necesaria menos el pasivo espontáneo generado por los adelantos de proveedores. Este variable captura así el capital que necesita la empresa para llevar a cabo su ciclo operativo. Por otro lado, el fondo de maniobra es el activo corriente financiado con recursos permanentes. Es decir, con cargo al patrimonio neto y al pasivo no corriente. O de otro modo, el fondo de maniobra captura el activo corriente que no es necesario financiar recurriendo a pasivo de corto plazo. De esta forma, en la medida en que las necesidades operativas de fondos sean mayores al fondo de maniobra, la empresa tendrá una necesidad de financiación que deberá cubrir con deuda a corto plazo. En la medida en que esta necesidad de financiación aumente, la liquidez de la empresa se verá comprometida porque, si bien puede liquidar activos para hacer frente a deudas corrientes en la medida en que el fondo de maniobra sea positivo, necesitará una mayor cantidad de activo corriente para mantener el ciclo operativo en funcionamiento y por tanto el cumplimiento de obligaciones deudoras entrará en tensión con la capacidad de seguir llevando a cabo las actividades que le generan un flujo de caja. Más allá de esta relación entre fondo de maniobra y necesidades operativas de fondos son habituales una serie de ratios que tratan de capturar diferentes aspectos de la liquidez de la empresa tales como la rotación, el periodo de maduración, el \textit{acid test}, el ratio de tesorería o el coeficiente básico de financiación. 

La \textbf{solvencia} de la empresa puede caracterizarse de forma general y con la mayor simplicidad a partir del signo del patrimonio neto: si es negativo, la empresa no puede hacer frente a sus obligaciones aún liquidando todo el activo, si es positivo, puede en teoría hacerlo. Dado que el valor contable de los activos no es sino una aproximación de su valor en caso de ser vendidos en el mercado --cuando esto sea posible-, son necesarios otros indicadores para aproximar mejor la solvencia de la empresa tales como el coeficiente de endeudamiento, que relaciona pasivo en relación a patrimonio neto.

El \textbf{análisis del resultado} tiene por objetivo global cuantificar la capacidad de la empresa para generar valor a partir del capital de que dispone. Para ello es preciso valorar la rentabilidad pero también el riesgo o la volatilidad de la creación de valor. En cuanto a la rentabilidad, si bien la medida genérica más básica es simplemente el cociente entre la rentabilidad y la inversión, las medidas más relevantes en el análisis financiero son la \underline{rentabilidad económica} y la \underline{rentabilidad financiera}, que resultan respectivamente del cociente del beneficio operativo entre activo total y el cociente entre el beneficio neto y el patrimonio neto. El riesgo económico de las empresas se puede medir como la volatilidad del beneficio antes de impuestos e interés o como la volatilidad del EBITDA. El riesgo financiero puede cuantificarse como la volatilidad del beneficio neto. En algunas empresas que no generan beneficios pero respecto de las cuales es importante cuantificar su comportamiento de cara a la capacidad para generar flujos de caja pueden utilizarse otros indicadores particulares como rotación del activo, ventas por empleado, visitas web por periodo, etc... Las medidas de apalancamiento operativo están íntimamente relacionadas con el riesgo económico y miden la relación entre el aumento porcentual del beneficio operativo o el BAIT dada una variación porcentual de las ventas. El punto muerto mide la cantidad de unidades que la empresa debe vender para recuperar al menos los costes fijos incurridos. Además, y sin ánimo de exhaustividad dada la enorme cantidad de ratios e indicadores utilizados, son destacables algunos ratios como el beneficio neto por acción, el dividendo por acción, el flujo de caja por acción, el PER (\textit{price earnings ratio}) o el P/B (\textit{price to book value ratio}). Por la facilidad con la que se calcular a partir de datos directamente extraíbles de los estados financieros, su uso es muy habitual en el análisis financiero.

A lo largo de la exposición se han mostrado los conceptos básicos relativos a la información financiera, las cuentas anuales del PGC de 2007 y una serie de indicadores básicos al análisis financiero de uso habitual. Antes de \marcar{concluir}, es preciso señalar que la información financiera está sujeta a un grado de discrecionalidad cuyos resultados se ven modulados por los incentivos de los agentes que compilan y presentan la información financiera. Por ello, no basta sólo con el informe del auditor para garantizar que las cuentas anuales aportan efectivamente una visión realista de la situación de la empresa, sino que es además fundamental el examen crítico y desagregado de la información disponible por parte del analista financiero. A fin de cuentas, será su capacidad para extraer una opinión acertada lo que determinará el éxito de la labor de análisis y la utilidad de la información financiera.

%Para gestionar una empresa, es necesario primero saber qué gestionar. Es decir, disponer de información acerca del estado en que se encuentra una empresa. Tanto a nivel económico (qué se produce, qué beneficios y qué costes tiene el funcionamiento de la empresa), como a nivel financiero (de que forma, con quién y en qué cuantías tiene compromiso la empresa). La información que permite realizar esta gestión se recoge en los llamados estados de situación y circulación, que se diferencian por el carácter stock y flujo de sus variables. En la primera parte del tema, se exponen los principios y los conceptos básicos en relación a esta información financiera. En la segunda parte, se entra en el detalle de la estructura de ese conjunto de información denominado estados de situación y circulación, más concretamente aquellos que forman parte del Plan General de Cuentas de 2007. En la tercera parte del tema se plantean los métodos básicos de análisis de esa información financiera.

\seccion{Preguntas clave}
\begin{itemize}
    \item ¿Qué es la información financiera de las empresas?
    \item ¿Para qué sirve?
    \item ¿Quién la utiliza?
    \item ¿Cómo se estructura?
    \item ¿Cómo se analiza?
\end{itemize}


\esquemacorto

\begin{esquema}[enumerate]
	\1[] \marcar{Introducción} 2-2
		\2 Contextualización
			\3 Gestión requiere información
			\3 Información financiera
		\2 Objeto
			\3 Qué es la información financiera
			\3 Para qué sirve
			\3 Quién la utiliza
			\3 Cómo se estructura
			\3 Cómo se analiza
		\2 Estructura
			\3 Información financiera
			\3 Cuentas anuales del PGC 2007
			\3 Análisis de estados contables
	\1 \marcar{Información financiera} 5-7
		\2 Concepto
			\3 Conjunto de datos
			\3 Describen
		\2 Usuarios
			\3 Internos
			\3 Externos
		\2 Requisitos
			\3 Relevancia
			\3 Fiabilidad
			\3 Integridad
			\3 Comparabilidad
			\3 Claridad
		\2 Principios
			\3 Empresa en funcionamiento
			\3 Devengo
			\3 Uniformidad
			\3 Prudencia
			\3 No compensación
			\3 Importancia relativa
		\2 Partida doble
			\3 Idea clave
			\3 Formulación
			\3 Libro mayor
			\3 Libro diario
		\2 Criterios de valoración
			\3 Valor a coste histórico
			\3 Valor razonable
			\3 Valor neto realizable
			\3 Valor actual
			\3 Valor en uso
			\3 Costes de venta
			\3 Valor según coste amortizado
			\3 Costes de transacción de activos y pasivos financieros
			\3 Valor contable o valor en libros
			\3 Valor residual
		\2 Ciclo contable
			\3 Idea clave
			\3 Duración
			\3 Apertura de cuentas
			\3 Recogida de datos
			\3 Cierre de la contabilidad
			\3 Aprobación
			\3 Depósito de cuentas anuales
		\2 Legislación
			\3 Internacional
			\3 Nacional
	\1 \marcar{Cuentas anuales del PGC 2007} 10-17
		\2 Estructura del Plan General de Cuentas
			\3[I] Marco conceptual de la contabilidad
			\3[II] Normas de registro y valoración
			\3[III] Cuentas anuales
			\3[IV] Cuadro de cuentas
			\3[V] Definiciones y relaciones contables
		\2[I] Balance de situación
			\3 Idea clave
			\3 Patrimonio completo
			\3 Dos columnas
			\3[i] Activo no corriente
			\3[ii] Activo corriente
			\3[i'] Patrimonio neto
			\3[ii'] Pasivo no corriente
			\3[iii'] Pasivo corriente
		\2[II] Cuenta de Pérdidas y Ganancias (CPyG)
			\3 Idea clave
			\3 Operaciones continuadas
			\3 Operaciones descontinuadas
		\2[III] Estado de Cambios en el Patrimonio Neto (ECPN)
			\3 Idea clave
			\3[i] Estado de Ingresos y Gastos Reconocidos (EIGR)
			\3[ii] Estado total de Cambios en el Patrimonio Neto (ETCPN)
		\2[IV] Estado de Flujos de Efectivo (EFE)
			\3 Idea clave
			\3[i] Flujos de efectivo de las actividades de explotación
			\3[ii] Flujos de efectivo de las actividades de inversión
			\3[iii] Flujos de efectivo de las actividades de financiación
			\3[iv] Efecto de las variaciones del tipo de cambio
		\2[V] Memoria
			\3 Concepto
			\3 PYMES y abreviado
		\2 Otros
			\3 Informe de auditoría
			\3 Informe de gestión
		\2 Variantes
			\3 Cumplimiento de requisitos
			\3 Cuentas anuales NO necesarias
			\3 Modelo abreviado: Cuenta de PyG
			\3 Modelo abreviado: balance y memoria
			\3 Pymes
			\3 Normal
	\1 \marcar{Análisis de los estados contables} 11-28
		\2 Liquidez
			\3 Idea clave
			\3 Fondo de maniobra y Necesidades Operativas de Fondos
			\3 Coeficiente básico de financiación
			\3 Rotación
			\3 Periodo de maduración
			\3 Solvencia a corto plazo / Ratio corriente / Liquidez general
			\3 Acid test/quick ratio
			\3 Ratio de tesorería
		\2 Solvencia
			\3 Ratio de solvencia
			\3 Endeudamiento:
			\3 Debt-to-equity
			\3 Multiplicador del equity
			\3 Apalancamiento bancario
			\3 Capital regulatorio
			\3 Equilibrio financiero
		\2 Resultado
			\3 Rentabilidad
			\3 Márgenes
			\3 Análisis Du-Pont
			\3 Riesgo
			\3 Ventas y efectivo
			\3 Punto muerto
			\3 Apalancamiento operativo
			\3 Apalancamiento financiero
			\3 Ratios de mercado
	\1[] \marcar{Conclusión} 2-30
		\2 Recapitulación
			\3 Conceptos básicos de la información financiera
			\3 Cuentas anuales del PGC-2007
			\3 Indicadores básicos de análisis de estados
		\2 Idea final
			\3 Contabilidad es una versión posible
			\3 Necesario examen crítico de cuentas

\end{esquema}

\esquemalargo














\begin{esquemal}
	\1[] \marcar{Introducción} 2-2
		\2 Contextualización
			\3 Gestión requiere información
				\4 No se puede gestionar lo que no se conoce
				\4 Información financiera para describir realidad de empresa
			\3 Información financiera
				\4 Necesaria presentación organizada y sistemática
				\4 Instrumento de toma de decisiones
				\4 $\to$ Comparar para invertir
				\4 $\to$ Valorar para comprar/vender
				\4 $\to$ Determinar obligaciones  tributarias
				\4 Impacto legal, tributario, financiero...
		\2 Objeto
			\3 Qué es la información financiera
			\3 Para qué sirve
			\3 Quién la utiliza
			\3 Cómo se estructura
			\3 Cómo se analiza
		\2 Estructura
			\3 Información financiera
			\3 Cuentas anuales del PGC 2007
			\3 Análisis de estados contables
	\1 \marcar{Información financiera} 5-7
		\2 Concepto
			\3 Conjunto de datos
				\4 Organizado
				\4 Sistemático
				\4 Temporal
			\3 Describen
				\4 Patrimonio
				\4 Derechos y obligaciones
				\4 Flujos financieros
		\2 Usuarios
			\3 Internos
				\4 Directivos
				\4 Empleados
			\3 Externos
				\4 Inversores
				\4 Proveedores
				\4 Clientes
				\4 Administración pública
		\2 Requisitos
			\3 Relevancia
				\4 Conocimiento debe implicar alguna diferencia
			\3 Fiabilidad
				\4 Reflejar el estado real de las cuentas
				\4 Sin errores ni sesgos
			\3 Integridad
				\4 Datos completos sin omisiones relevantes
			\3 Comparabilidad
				\4 Debe permitir poner en relación con otras
			\3 Claridad
				\4 Debe permitir la formación de un juicio
		\2 Principios
			\3 Empresa en funcionamiento\footnote{Este principio puede no aplicarse en contextos de transmisión global de la empresa o liquidación, en cuyo caso se aplicarán los criterios de valoración que resulten adecuados a la situación.}
				\4 Se supone continuación
				\4 No se supone transmisión global
				\4 No se trata de determinar patrimonio neto
				\4[] Con el fin de transmisión o liquidación
			\3 Devengo
				\4 Hechos económicos registrados cuando ocurren
				\4[] Cuando se produce el cambio de titularidad económica
				\4 Independiente de cobros y pagos
				\4[] Independiente de movimiento de efectivo
			\3 Uniformidad
				\4 Criterios constantes para transacciones similares
				\4 Alteraciones criterios: en memoria
			\3 Prudencia
				\4 Mantener prudencia en estimaciones y valoraciones
				\4 Riesgos:
				\4[] Tener en cuanto se conozcan
				\4 Beneficios a contabilizar
				\4[] Sólo los obtenidos a fecha de cierre
				\4 Amortizaciones y depreciaciones
				\4[] Registrados aunque generen pérdidas
			\3 No compensación
				\4 Partidas de activo y pasivo no se compensan
				\4 Ingresos y gastos no se compensan
			\3 Importancia relativa
				\4 Admisible no aplicación de criterio contable
				\4[] Cuando importancia relativa sea pequeña
				\4 Aplicación eficiente de principios contables
		\2 Partida doble
			\3 Idea clave
				\4 Principio universal de contabilidad financiera
				\4 Toda anotación por duplicado
				\4 Origen finales de la edad media
				\4 Equilibrio constante entre activo y pasivo
				\4[] Todo derecho en favor de agente
				\4[] $\to$ Implica obligación en favor de otro
				\4[] $\then$ Con terceros
				\4[] $\then$ Consigo mismo residualmente
			\3 Formulación
				\4 Identidad contable básica: P + PN = A
				\4 Todo apunte, en debe y haber
				\4 Cargos y abonos
				\4[] Todo cargo tiene abono correspondiente
				\4[] $\to$ Y v.v.
				\4 Asiento contable
				\4[] Anotación que involucra al menos dos partidas
			\3 Libro mayor
				\4 Registro de todos los hechos contables
				\4 Permite análisis financiero instantáneo
				\4[] Obligaciones y derechos permanente registradas
				\4 Cuatro tipos de cuenta
				\4[] Activo
				\4[] $\to$ Primer apunte siempre en el debe
				\4[] $\to$ Deber mayor al haber
				\4[] $\to$ Saldos deudores
				\4[] Pasivo
				\4[] $\to$ Primer apunte siempre en el haber
				\4[] $\to$ Haber mayor al deber
				\4[] $\to$ Saldos acreedores
				\4[] Ingresos y Gasto
				\4[] $\to$ Ingresos al haber
				\4[] $\to$ Gastos a debe
				\4[] $\then$ Saldo acreedor si ganancias
				\4[] $\then$ Saldo deudor si pérdidas
			\3 Libro diario
				\4 Registro cronológico de asientos contables
				\4 Obligatorio llevar en empresas
		\2 Criterios de valoración\footnote{Ver \href{https://www.iberley.es/temas/normas-criterios-valoracion-59284}{Iberley sobre criterios de valoración en PGC}}
			\3 Valor a coste histórico
				\4 Coste de adquisición o producción
				\4 Importe efectivo pagado por el activo
				\4 Importe efectivo de materias primas
			\3 Valor razonable
				\4 Importe obtenible por activo/pasivo
				\4 Sin deducir costes de transacción
				\4 No es valor razonable si:
				\4[] Resultado de
				\4[] $\to$ Transacciones forzadas
				\4[] $\to$ Urgentes
				\4[] $\to$ Liquidación involuntaria
				\4 Valor fiable de mercado utilizado para calcular
			\3 Valor neto realizable
				\4 Valor obtenible por enajenación
				\4 Deduciendo costes estimados necesarios
			\3 Valor actual
				\4 Importe de flujos a recibir
				\4 Actualizados a descuento adecuado
			\3 Valor en uso
				\4 Suma de flujos capitalizados
				\4 Flujos a obtener derivados de uso del activo
				\4 Supuestos explícitos y razonables sobre flujos
			\3 Costes de venta
				\4 Costes atribuibles a venta de activo
				\4 No se incurrirían
				\4[] Si no se toma decisión de vender
				\4 No incluye gastos financieros ni impuestos
				\4 Sí gastos legales para transferir propiedad
			\3 Valor según coste amortizado
				\4 Valor inicial de un activo financiero
				\4[--] Menos reembolsos del principal producidos
				\4 Equivale a VActual de flujos pendientes
				\4[] Descontados a:
				\4[] $\to$ Tipo de interés efectivo
				\4[] $\to$ TIR de adquisición
			\3 Costes de transacción de activos y pasivos financieros
				\4 Costes atribuibles a diferentes transacciones
				\4[] Compra
				\4[] Emisión
				\4[] Enajenación
				\4[] Asunción de pasivo financiero
				\4[] $\to$ No existiría sin transacción
				\4 Incluidos
				\4[] Honorarios y comisiones pagadas a agentes
				\4[] Impuestos y derechos sobre transacción
				\4 Excluidos
				\4[] Primas o descuentos por transacción
				\4[] Gastos financieros
				\4[] Costes de mantenimiento
				\4[] Costes administrativos internos
			\3 Valor contable o valor en libros
				\4 Importe neto de un activo o pasivo
				\4[] Por el que activo/pasivo registrado en balance
				\4 Descontando:
				\4[] Amortizaciones acumuladas
				\4[] Correcciones valorativas
			\3 Valor residual
				\4 Importe obtenible por venta
				\4 Deducidos costes de la venta
				\4 Considerando vida útil
				\4[] Periodo durante el que se espera utilizar activo
				\4[] Si activo puede devolverse/revertirse
				\4[] $\to$ Periodo que resta de la concesión
				\4 Vida económica
				\4[] Periodo en el que el activo será utilizable
		\2 Ciclo contable
			\3 Idea clave
				\4 Secuencia cronológica de actuaciones
				\4 Determinar cambio en situación económica
			\3 Duración
				\4 Generalmente un año
				\4 Posible periodos más cortos
			\3 Apertura de cuentas
				\4 Registro de elementos en el balance
				\4[] Activo y pasivo
				\4 Cuentas de activo
				\4[] Se cargan
				\4 Cuentas de pasivo
				\4[] Se abonan
			\3 Recogida de datos
				\4 Iniciado tras apertura
				\4 Registro de transacciones económicas
				\4[] En libros contables obligatorios y auxiliares
				\4 Se extiende a lo largo de periodo
			\3 Cierre de la contabilidad
				\4 Saldo de todas la cuentas
				\4[] Cargo de cuentas acreedoras
				\4[] Abono de cuentas deudoras
				\4 Todas las cuentas tienen saldo nulo
			\3 Aprobación
				\4 Socios aprueban cuentas anuales
				\4 Acuerdan distribución del resultado
				\4 Sujeto a calendario legal
			\3 Depósito de cuentas anuales
				\4 Obligación de presentación en Registro Mercantil
		\2 Legislación
			\3 Internacional
				\4 NIIF/IFRS del IASB\footnote{NIIF/IFRS: Normas Internacionales de Información Financiera/\textit{International Financial Reporting Standards}. IASB: \textit{International Accounting Standards Board}. Hasta 2001 conocidas como NIC (Normas Internacionales de Contabilidad).}
				\4 Directiva 2013/34 de la Unión Europea (pymes)
				\4 Estados Unidos:
				\4[] \textit{Generally Accepted Accounting Principles} del FASB
			\3 Nacional
				\4 Código Civil
				\4 Código de Comercio de 2018
				\4 Plan General de Contabilidad de 2007
				\4 LSC -- Ley de Sociedades de Capital de 2010
				\4 LA -- Ley de Auditoría de Cuentas de 2015
				\4[] Modificado en 2007
				\4 Plan General de Contabilidad de Pequeñas y Medianas Empresas de 2007
				\4 RD 602/2016 de modificación de Planes Generales de Contabilidad
	\1 \marcar{Cuentas anuales del PGC 2007} 10-17
		\2 Estructura del Plan General de Cuentas
			\3[I] Marco conceptual de la contabilidad
				\4 Aplicación obligatoria
				\4 Conceptos y principios anteriores
			\3[II] Normas de registro y valoración
				\4 Aplicación obligatoria
				\4 Reglas de valoración de grupos de cuentas
			\3[III] Cuentas anuales
				\4 Normas de elaboración de las cuentas
				\4 Modelos normales
				\4 Modelos abreviados
			\3[IV] Cuadro de cuentas
			\3[V] Definiciones y relaciones contables
		\2[I] Balance de situación
			\3 Idea clave
				\4 Capital empleado y utilizado
				\4 Panorama general de la solvencia y la liquidez
			\3 Patrimonio completo
				\4 Bienes
				\4 Derechos
				\4 Obligaciones
			\3 Dos columnas
				\4 Activo
				\4[] Menor a mayor liquidez
				\4 Patrimonio neto y pasivo
				\4[] Menor a mayor exigibilidad
			\3[i] Activo no corriente
				\4 Inmovilizado intangible
				\4 Inmovilizado material
				\4 Inversiones inmobiliarias
				\4 Inversiones en empresas del grupo
				\4 Inversiones financieras a largo plazo
			\3[ii] Activo corriente
				\4 Existencias
				\4 Deudores comerciales y otras cuentas
				\4 Inversiones en empresas del grupo
				\4 inversiones financieras a corto plazo
				\4 Periodificaciones\footnote{Gastos e ingresos anticipados.}
				\4 Efectivo y otros líquidos equivalentes
			\3[i'] Patrimonio neto
				\4 Fondos propios\footnote{Incluye capital, prima de emisión, reservas, participaciones en patrimonio propias con saldo deudor y resultado de ejercicios anteriores.}
				\4 Ajustes por cambio de valor
				\4 Subvenciones, donaciones y legados
			\3[ii'] Pasivo no corriente
				\4 Provisiones a largo plazo
				\4 Deudas a largo plazo
				\4 Deudas con empresas del grupo
				\4 Pasivos por impuesto diferido
				\4 Periodificaciones a largo plazo
			\3[iii'] Pasivo corriente
				\4 Pasivos vinculados con activos no corrientes
				\4 Provisiones a corto plazo
				\4 Deudas a corto plazo
				\4 Acreedores comerciales y otras cuentas a pagar
				\4 Periodificaciones a corto plazo
		\2[II] Cuenta de Pérdidas y Ganancias (CPyG)
			\3 Idea clave
				\4 Aumento del patrimonio neto en el periodo
				\4[] Al margen de:
				\4[] $\to$ Las aportaciones de los socios
				\4[] $\to$ Otros cambios de volumen y valor
				\4[] $\then$ Resultado de los ingresos y los gastos
			\3 Operaciones continuadas
				\4[A1] Resultado de explotación
				\4[] Ingresos netos por ventas y otros
				\4[] (Gastos de personal y otros)
				\4[] (Amortizaciones y depreciaciones)
				\4[A2] Resultado financiero
				\4[] Ingresos financieros y dif. positivas de cambio
				\4[] (Costes financieros y dif. negativas de cambio)
				\4[A3] Resultado antes de impuestos (A1 + A2)
				\4[-A4] Impuesto sobre el beneficio
				\4[=] Resultado del ejercicio (A3 - A4)
			\3 Operaciones descontinuadas
		\2[III] Estado de Cambios en el Patrimonio Neto (ECPN)
			\3 Idea clave
				\4 Cambios anuales en el patrimonio neto
				\4 Debidos a:
				\4[] Resultados imputados a PN
				\4[] Cambio de criterios contables
				\4[] Subsanación de errores
				\4[] Ampliación del capital social
			\3[i] Estado de Ingresos y Gastos Reconocidos (EIGR)
				\4 Cambios derivados de CPyG
				\4 Ingresos y gastos imputables a PN
			\3[ii] Estado total de Cambios en el Patrimonio Neto (ETCPN)
				\4 Saldo total del EIGR
				\4 Variaciones PN por operaciones con socios y propietarios
				\4 Otras variaciones
				\4 Ajustes PN debidos a cambios en criterios contables
		\2[IV] Estado de Flujos de Efectivo (EFE)
			\3 Idea clave
				\4 Sumario del origen y uso de activos monetarios
				\4 Clasificación por actividades
				\4 Variación neta
				\4 Principio de caja, NO devengo
			\3[i] Flujos de efectivo de las actividades de explotación
				\4 Resultado antes de impuestos
				\4 Ajustes del resultado
				\4 Cambios en el capital corriente
				\4 Otros flujos de efectivo de las actividades de explotación
			\3[ii] Flujos de efectivo de las actividades de inversión
				\4 Pagos por inversiones
				\4 Cobros por desinversiones
			\3[iii] Flujos de efectivo de las actividades de financiación
				\4 Cobros y pagos por instrumentos de pasivo financiero
				\4 Pagos por remuneraciones y de otros instrumentos de patrimonio
			\3[iv] Efecto de las variaciones del tipo de cambio
		\2[V] Memoria
			\3 Concepto
				\4 Ampliación de información de balance y CPyG
				\4 Informaciones adicionales
				\4 Actividad, normas aplicadas, situación fiscal, otras informaciones...
			\3 PYMES y abreviado
				\4 Muy reducido
		\2 Otros
			\3 Informe de auditoría
				\4 Obligatorio para sociedades que incumplen límites PYMES
				\4 Favorable
				\4 Desfavorable
				\4 Denegada
			\3 Informe de gestión
				\4 Obligatorio modelo normal
				\4 No sujeto a principios y normas contables
				\4 Evolución de los negocios
				\4 Previsiones
				\4 Punto de vista de los directivos
		\2 Variantes
			\3 Cumplimiento de requisitos
				\4 Al menos dos de tres
				\4 Durante dos ejercicios consecutivos
				\4 Cumplidos a fecha de cierre de ejercicio
			\3 Cuentas anuales NO necesarias
				\4 EFE
				\4 ECPN
			\3 Modelo abreviado: Cuenta de PyG
				\4[i)] Activo < 11.400.000 €
				\4[ii)] Cifra de negocios < 22.800.000 €
				\4[iii)] Empleados $\leq$ 250 empleados
			\3 Modelo abreviado: balance y memoria
				\4[i)] Activo < 4 millones de €
				\4[ii)] Cifra de negocios < 8 millones de €
				\4[iii)] Empleados $\leq$ 50 empleados
			\3 Pymes
				\4[i)] Activo < 4 millones de €
				\4[ii)] Cifra de negocios < 8 millones de €
				\4[iii)] Empleados $\leq$ 50
			\3 Normal
				\4 Todas la que incumplan requisitos anteriores
				\4 Sociedades con monedas diferentes al €
				\4 Sociedades cotizadas
				\4 Otras
	\1 \marcar{Análisis de los estados contables} 11-28
		\2 Liquidez
			\3 Idea clave
				\4 Evaluar capacidad de:
				\4[] Satisfacer necesidades c/p
				\4[] Obtener recursos necesarios para operativa de empresa
			\3 Fondo de maniobra y Necesidades Operativas de Fondos\footnote{Leer 4.2 Vernimmen y págs. 533-536 de Stolowy et al.}
				\4 Fondo de Maniobra:
				\4[] $\to$ Activo corriente financiado con recursos permanentes\footnote{Es decir, con pasivo no corriente y patrimonio neto.}
				\4[] $\to$ $\text{PN}+\text{PNC}-\text{ANC}$
				\4[] $\to$ $AC$ - $PC$
				\4 Necesidades Operativas de Fondos:
				\4[] $\to$ Existencias + Deudores + Tesorería - Pasivo espontáneo\footnote{Adelantos de proveedores a la empresa.}
				\4 Si NOF > FM:
				\4[] $\to$ La empresa necesita financiación
				\4[] $\to$ Puede hacer frente a PC liquidando AC
				\4[] $\then$ Pero necesita capital para mantener ciclo operativo
			\3 Coeficiente básico de financiación
				\4 Valorar capacidad para mantener activo necesario
				\4[] ¿Pasivo exigible supera activo necesario?
				\4[] ¿Hay tensiones de liquidez que ponen en riesgo actividad?
				\4 $\dfrac{\text{Patrimonio neto}+\text{Pasivo no corriente}}{\text{Activo no corriente}+\text{Capital corriente necesario}}$
				\4 $\geq 1$: estabilidad financiera
				\4 $<1$: inestabilidad financiera
			\3 Rotación
				\4 Relación entre flujos de ciclo operativo y stocks
				\4 Rotación del activo:
				\4[] $\dfrac{\text{Ventas}}{\text{Activo}}$
				\4 Rotación de fabricacion, existencias...:
				\4[] Valores medios del stock
			\3 Periodo de maduración
				\4 Tiempo entre inversión y recuperación
				\4[] $\to$ Duración del ciclo operativo de la empresa
			\3 Solvencia a corto plazo / Ratio corriente / Liquidez general
				\4 $\dfrac{\text{Activo corriente}}{\text{Pasivo corriente}}$
			\3 Acid test/quick ratio\footnote{En algunas formulaciones se descuenta del numerador el ANCMV (Activo No Corriente Mantenido para la Venta), lo cual tiene sentido si el ANCMV forma parte del activo corriente. Entendemos por ANMCV activos financieros muy líquidos que la empresa puede convertir en efectivo muy rápidamente a bajo coste.}
				\4 $\frac{\text{Activo corriente}-\text{Existencias y anticipos a proveedores} }{\text{Pasivo corriente}}$
				\4 $\frac{Disponible + Realizable + ANMCV}{\text{Pasivo corriente}}$
			\3 Ratio de tesorería
				\4 Relación entre:
				\4[] Activos disponibles para pagar
				\4[] Obligaciones de corto plazo
				\4 $\frac{\text{Tesorería+ANMCV}}{\text{Pasivo corriente}}$
		\2 Solvencia
			\3 Ratio de solvencia
				\4 Ratio básico de medición capacidad de pago
				\4[] $\frac{\text{Activo}}{\text{Pasivo en sentido estricto}}$
			\3 Endeudamiento:
				\4 $\frac{\text{Pasivo}}{\text{Activo total}}$
			\3 Debt-to-equity
				\4 $\frac{\text{Deuda total}}{\text{Patrimonio neto}}$
			\3 Multiplicador del equity
				\4 $\frac{\text{Activo total}}{\text{Patrimonio Neto}}$
			\3 Apalancamiento bancario
				\4 $\frac{\text{Capital}}{Activo total}$
			\3 Capital regulatorio
				\4 $\frac{\text{Capital}}{\text{Activos ponderados por riesgo}}$
			\3 Equilibrio financiero
				\4 Fondo de maniobra: $\text{Activo corriente}-\text{Pasivo corriente}$
				\4 PN > 0 y FM > 0: equilibrio financiero
				\4 PN > 0 y FM < 0: suspensión de pagos
				\4 PN < 0: quiebra
		\2 Resultado
			\3 Rentabilidad
				\4 Genérica: $\frac{\text{Rentabilidad}}{\text{Inversión}}$
				\4 Económica: $\frac{\text{BAIT}}{\text{Activo total}} = M \cdot R$
				\4 Financiera\footnote{Siendo $A_f$ el apalancamiento financiero tal que $\frac{\text{BAI}\cdot \text{Activo}}{\text{BAIT}\cdot \text{PN}}$.} \footnote{$R_F = \frac{\text{BN}}{\text{PN}} = \frac{\text{BAI}-\text{GF}}{\text{PN}} = \frac{\text{BAI}}{\text{PN}} \cdot \frac{\text{A}}{\text{A}} - \frac{\text{GF}}{\text{PN}} = R_E \cdot \frac{\text{PN}+D}{\text{PN}} - \frac{GF}{PN}\cdot \frac{D}{D} = R_E + R_E \cdot \frac{D}{\text{PN}} - i\cdot \frac{D}{\text{PN}} = R_E + \left( R_E - i \right) \cdot \frac{D}{\text{PN}}$}: $R_F = \frac{\text{BN}}{\text{PN}} = \text{RE}\cdot A_F$ $= R_E + (R_E-i)\frac{D}{\text{PN}}$
				\4 ROCE\footnote{\textit{Return on Capital Employed} o retorno sobre el capital empleado}: $\frac{\text{BAIT}}{\text{PN}+\text{Pasivo de l/p}}$
			\3 Márgenes
				\4 Bruto
				\4[] $\frac{\text{Ingresos por ventas} - \text{Coste de ventas}}{\text{Ingresos de las ventas}}$
				\4 Ebitda
				\4[] $\frac{\text{EBITDA}}{\text{Ingresos por ventas}}$
				\4 Operativo
				\4[] $\frac{\text{BAIT}}{\text{Ingresos por ventas}}$
				\4 Neto
				\4[] $\frac{\text{BN}}{\text{Ingresos por ventas}}$
			\3 Análisis Du-Pont
				\4 Contexto
				\4[] RoE
				\4[] $\to$ Rentabilidad del equity
				\4[] \% Beneficio neto por valor del patrimonio neto
				\4 Objetivo
				\4[] Desagregar componentes de RoE
				\4[] $\to$ Margen neto
				\4[] $\to$ Rotación del activo
				\4[] $\to$ Apalancamiento financiero
				\4 Formulación
				\4[] $\text{RoE} = \frac{\text{BN}}{\text{PN}} = \frac{\text{BN}}{\text{PN}} \cdot \frac{\text{Ventas}}{\text{Ventas}} \cdot \frac{\text{AT}}{\text{AT}} = \text{Mar.neto} \cdot \text{Rotación} \cdot \frac{\text{AT}}{\text{PN}} $
				\4[] Margen neto: $\frac{\text{BN}}{\text{Ventas}}$
				\4[] Rotación: $\frac{\text{Ventas}}{\text{Activos}}$
				\4[] Multiplicador del patrimonio neto: $\frac{\text{AT}}{\text{PN}}$
			\3 Riesgo
				\4 Riesgo económico: $\text{Var}(\text{BAIT})$
				\4 Riesgo financiero: $\textbf{Var}( \text{BN})$
			\3 Ventas y efectivo
				\4 Crecimiento de las ventas
				\4 Gastos financieros por intereses entre ventas: $\frac{\text{GF}}{V}$
			\3 Punto muerto
				\4 Umbral de rentabilidad: $\dfrac{\text{CF}}{P - \text{CV}}$
			\3 Apalancamiento operativo
				\4 Elasticidad BAIT-Ventas: $\frac{\frac{ \text{BAIT}_t - \text{BAIT}_{t-1} }{ \text{BAIT}_{t-1}} }{\frac{ \text{Ventas}_t - \text{Ventas}_{t-1} }{ \text{Ventas}_{t-1} }}$
				\4 Sin costes fijos, AO=1
				\4 Con costes fijos, AO>1 si prod. > punto muerto.
			\3 Apalancamiento financiero
				\4 Medida de sensibilidad de:
				\4[] Beneficios netos
				\4[] $\to$ Respecto a beneficio operativo
				\4 $\frac{\Delta \% \text{Beneficio neto}}{\Delta \% \text{EBIT}}$
			\3 Ratios de mercado
				\4 Beneficio neto por acción
				\4 Dividendo por acción
				\4 Cash flow por acción: $\text{FCFE}/\text{Número de acciones}$
				\4 Payout ratio = $\frac{\text{Dividendo}}{\text{Beneficio neto}}$
				\4 P/E (PER) = $\frac{\text{Precio}}{\text{Beneficio neto por acción}}$
				\4 P/B = $\frac{\text{Precio}}{\text{Valor contable por acción}}$
	\1[] \marcar{Conclusión} 2-30
		\2 Recapitulación
			\3 Conceptos básicos de la información financiera
			\3 Cuentas anuales del PGC-2007
			\3 Indicadores básicos de análisis de estados
		\2 Idea final
			\3 Contabilidad es una versión posible
				\4 Criterios contables flexibles
				\4 Relativa discrecionalidad
				\4[] $\Rightarrow$ Incentivos a dar imagen positiva
			\3 Necesario examen crítico de cuentas
				\4 Papel del auditor
				\4 Tener en cuenta más información
				\4 Exigir desglose clientes, obligaciones...
				\4 Atender a flujos de caja
\end{esquemal}




























\conceptos

\concepto{EBITDA, beneficio operativo, beneficio antes de intereses e impuestos, beneficio neto, earnings y y otros términos de la cuenta de resultados}

En la habitual sopa de términos del campo del análisis financiero y contable, es habitual confundir términos. Más aún cuando entran en juego términos traducidos del inglés. Examinemos de cerca algunos de los términos más relevantes.

La diferencia entre \textit{earnings} y \textit{profits} es relativamente oscura y depende en gran medida del contexto, aunque puede señalarse con cierta confianza que \textit{earnings} hace referencia a diferencias generales entre ingresos y costes mientras que profits hace referencia a diferencias entre ingresos y costes pero con un matiz de totalidad. De esta forma \textit{earnings} es más habitual cuando se hace referencia a beneficios que no tienen en cuenta algunos gastos tales como gastos financieros o impuestos. 

EBITDA significa \textit{earnings before interest, tax, depreciation and appreciation} y se traduce en español como \textit{beneficio operativo} debido al hecho de que captura el beneficio derivado del ciclo operativo de la empresa, sin tener en cuenta el resultado del ciclo de inversión o el ciclo financiero de la empresa. El EBIT corresponde al \textit{earnings before interest and tax} y en español es habitual traducirlo como BAII o BAIIT, acrónimos de \textit{beneficio antes de intereses e impuestos}. El \textit{beneficio neto} es la traducción habitual de \textit{net income}, o beneficio una vez tenidos en cuenta los gastos financieros, los impuestos y las depreciaciones, amortizaciones y otros ajustes contables así como elementos extraordinarios o ingresos no recurrentes.

\concepto{Diferencia entre fondos propios y patrimonio neto}

El patrimonio neto engloba todo el capital no exigible, es decir, todo lo que no es pasivo. Dentro de esos fondos no exigibles a la empresa que constituyen el patrimonio neto, los fondos propios son aquellos aportados por los accionistas o generados directamente por la empresa. De esta forma, los fondos propios incluyen el capital, la prima de emisión, las reservas, las participaciones en patrimonio propio con saldo deudor y resultados de ejercicios anteriores. El patrimonio neto comprende, además de los fondos propios, capital disponible no aportado por los dueño tales como subvenciones, donaciones y legados, así como los ajustes derivados de cambios de valor.

\preguntas

\seccion{Test 2015}

\textbf{24.} Señale la respuesta correcta referida al ROE (\textit{return on equity}), ROA (\textit{return on assets}) y el margen y la rotación de los activos de una empresa a partir de los siguientes datos:

\begin{itemize}
	\item Activos totales: 600 M de €
	\item Recursos propios: 200 M de €
	\item Pasivo exigible: 400 M de €
	\item Ventas anuales: 3.000 M de €
	\item BAIT (Beneficio antes de Intereses e Impuestos): 1.200 M de €
	\item Tipo de interés anual ponderado de los pasivos exigibles: 15 \%
	\item Tipo impositivo: 0\%
\end{itemize}

\begin{enumerate}
	\item[a] El ROE es igual a 5,7; el ROA igual a 2 y el rendimiento económico de los activos proviene en mayor medida de su rotación que del margen.
	\item[b] El ROE es igual a 1,9; el ROA igual a 2 y el rendimiento económico de los activos proviene en mayor medida de su rotación que del margen.
	\item[c] El ROE es igual a 1,7; el ROA igual a 2 y el rendimiento económico de los activos proviene en mayor medida del margen que de su rotación.
	\item[d] El ROE es igual a 14,7: el ROA igual a 5 y el rendimiento económico de los activos proviene en mayor medida de su rotación que del margen.
\end{enumerate}

\seccion{Test 2011}

\textbf{23.} El fondo de maniobra es igual a:
\begin{enumerate}
	\item[a] La diferencia entre el activo corriente y el pasivo corriente.
	\item[b] La diferencia entre los pasivos a largo plazo y los activos a largo plazo.
	\item[c] Aquella parte de los activos corrientes que se financian con pasivos a largo plazo.
	\item[d] Todas las anteriores son correctas.
\end{enumerate}

\seccion{Test 2008}
\textbf{23.} La rentabilidad de los activos o rentabilidad económica:

\begin{enumerate}
	\item[a] Nos indica el volumen de ventas para el cual la empresa cubre todos los costes.
	\item[b] Nos indica la remuneración obtenida por los accionistas por su aportación de capital.
	\item[c] Se define como los beneficios después de intereses e impuestos sobre fondos propios.
	\item[d] Se puede descomponer en dos factores, el margen de beneficios antes de intereses e impuestos y la rotación del activo total.
\end{enumerate}


\seccion{Preguntas cante 13 de marzo 2017}
\begin{itemize}
    \item Me gustaría que hablase un poco del papel de los auditores. La nueva Ley de Auditoría distingue y separa las labores de consultoría y auditoría, ¿puede comentar esto? Las empresas tienden a compensar al auditor no elegido con contratos de consultoría. ¿Cuál es el papel de la auditoría interna en relación a la gobernanza interna de las empresas?
    \item Ha afirmado usted que las empresas no pueden aumentar la rentabilidad y reducir la volatilidad al mismo tiempo. ¿Por qué?
\end{itemize}

\notas

\textbf{2015}. \textbf{24}. A

\textbf{2011.} \textbf{23}. ANULADA. El concepto de pasivo a largo plazo puede dar lugar a ambigüedad y entenderse que incluye el patrimonio neto, de manera que la opción D fuese correcta. Por otro lado, si se entiende ``pasivo'' en sentido estricto, la opción A sería la correcta.

\textbf{2008.} \textbf{23}. D

Hay que reformar la parte de estructura del plan y la de variantes para reflejar el hecho de que los planes normal y abreviados forman parte del PGC 2007, mientras que el PGC de pymes es un plan separado que además, incluye una variatne propia, las microempresas (además de las pymes).

\bibliografia

Muñoz Jiménez, J. \textit{Contabilidad financiera} -- En carpeta Finanzas

Garrido Miralles, Íñiguez Sánchez - Análisis de estados contables (en Biblioteca)

Gurriarán, R. \textit{El Fondo de Maniobra y las Necesidades Operativas de Fondos} 

Pablo Fernández. \textit{Papers en SSRN}. \url{https://papers.ssrn.com/sol3/cf_dev/AbsByAuth.cfm?per_id=12696}

Fernández, P. \textit{Cash flow is cash and is a fact: net income is just an opinion} (2006) Working Paper IESE -- En carpeta del tema

Stolowy, H.; Lebas, M. J.; Yuan, D. \textit{Financial Accounting and Reporting. A Global Perspective} (2013) Fourth Edition -- En carpeta Finanzas

Vernimmen, P. et al. \textit{Corporate Finance. Theory and Practice} Fifth Edition (2017). Ch. 2, 3, 4, 5, 7, 9, 10, 11, 12, 13, 14

\end{document}
