\documentclass{nuevotema}

\tema{3B-39}
\titulo{La Política Agrícola de la Unión Europea. Problemas económicos y procesos de reforma. La política pesquera común.}

\begin{document}

\ideaclave

AÑADIR DATOS SOBRE SECTORES AGRÍCOLA Y PESQUERO

SECTOR EXTERIOR PESQUERO: \href{https://ec.europa.eu/fisheries/facts_figures_en?qt-facts_and_figures=5}{Trade and Fisheries statitics}

El sector primario tiene unas características peculiares en relación a otros sectores, tanto el sector agrícola como la pesca. 

El sector agrícola es muy vulnerable a factores exógenos por razones climatológicas y estacionales. Pero a la vez, su importancia estratégica es enorme, dado que permite asegurar las necesidades alimenticias de la población y disminuir los efectos psicológicos que tiene el aumento de los precios de los alimentos. Por su parte, el sector pesquero se caracteriza por la presencia de externalidades en la producción, dado su carácter de recurso natural con capacidad limitada para renovarse.

Es común a ambos sectores la fuerte dependencia económica que muestran determinados territorios respecto a estas actividades. Concretamente, grandes áreas del entorno rural dependen casi exclusivamente de la agricultura, mientras que existen zonas costeras cuya actividad económica se basa fundamentalmente en el sector pesquero y naval. 

Por estas características, la Unión Europea lleva a cabo desde los años 60 políticas específicas para estos dos sectores: la Política Agraria Común (PAC) y la Política Pesquera Común (PPC). Ambos instrumentos constituyen intervención pública en el mercado. El objetivo global de estas políticas es optimizar el funcionamiento de los recursos agrícolas y pesqueros. Los objetivos específicos son múltiples: aumentar la \textbf{eficiencia} de los mercados, asegurar la \textbf{sostenibilidad} de los recursos naturales y \clave{garantizar} el suministro de alimentos a precios razonables a la población europea. Además, el hecho de que el mercado único permita la libre circulación de productos agrícolas, obliga a que las instituciones comunitarias implementen mecanismos que permitan a los consumidores confiar en la calidad y seguridad de productos con origen en otros EEMM.

Para lograr estos objetivos, la PAC y la PPC se articulan basicamente en torno a dos tipos de intervención: intervención sobre precios y mercados, y reformas estructurales de los mercados para mejorar su eficiencia y sostenibilidad a largo plazo.

La intervención sobre precios y mercados en el sector agrícola se articula en torno a los llamados OCM (organización común del mercado). Hasta 2007, existían veintiún OCM específicas para diferentes mercados agrícolas. Sin embargo, con la entrada en vigor de la OCM, todas las OCM pasan a estar contenidas en un sólo OCM. 

La evolución histórica de la PAC y la PPC ha trascurrido paralela a la evolución de la UE en su conjunto y del sistema económico internacional. La política de intervención de la PAC se basaba inicialmente en un sistema de precios garantizados, que después pasaron a articularse en torno a un sistema de compensaciones y finalmente, a un sistema de ayudas directas. Actualmente, las medidas de intervención previstas en la OCM son medidas de "seguridad", en el sentido de que se prevé su aplicación sólo cuando sea estrictamente necesario y no de forma sistemática como en el pasado. En gran medida, las políticas comunes de pesca y agrícola han evolucionado como resultado de la necesidad de adaptarse a las sucesivas ampliaciones de la UE. 

En la actualidad y en el marco de las negociaciones de la OMC, la agricultura es uno de los temas más controvertidos. Los países en desarrollo presionan para la liberalización de los mercados agrícolas europeos. Asimismo, se han producido choques norte-norte al hilo de esta cuestión. A pesar de ello, el sector agrícola y pesquero continúa siendo uno de los sectores más restringidos al comercio internacional, y en gran medida la PAC y la PPC son resultado de ello.

En la actualidad, ambas políticas se encuentran en proceso de reforma dadas las críticas de las cuales han sido objeto, y de sus carencias evidentes. En cuanto a la pesca, el fracaso corresponde al intento por mantener los caladeros europeos en niveles sostenibles a largo plazo. En cuanto a la PAC, se critica el excesivo gasto, la regresividad de las ayudas directas --muy superiores a la ayuda prestada a economías en desarrollo- y los daños medioambientales inducidos.

\seccion{Preguntas clave}

\begin{itemize}
    \item ¿Qué son la PAC y la PPC?
    \item ¿Por qué llevan a cabo?
    \item ¿Qué actuaciones implican?
    \item ¿Cuáles son sus problemas?
\end{itemize}


\esquemacorto

\begin{esquema}[enumerate]
	\1[] \marcar{Introducción}
		\2 Contextualización
			\3 Unión Europea
			\3 Competencias de la UE
			\3 Sector agroalimentario de la UE
			\3 Comercio intra-UE en agroalimentario
			\3 Factores comunes de agricultura y pesca
			\3 Políticas nacionales pre-integración
		\2 Objeto
			\3 ¿Qué son la PAC y la PPC?
			\3 ¿Por qué se llevan a cabo?
			\3 ¿Qué actuaciones implican?
			\3 ¿Cuáles son sus problemas?
		\2 Estructura
			\3 PAC
			\3 PPC
	\1 \marcar{Política Agrícola Común}
		\2 Justificación
			\3 Seguridad del suministro
			\3 Rentas del sector agrícola
			\3 Eficiencia productiva
			\3 Estabilidad de los mercados
			\3 Seguridad alimentaria
			\3 Medio ambiente
		\2 Objetivos
			\3 Garantizar suministro y calidad
			\3 Mejorar nivel de vida agricultores
			\3 Proteger el medio ambiente
			\3 Precios asequibles y estables
		\2 Antecedentes
			\3 Idea clave
			\3 Pre-PAC
			\3 Creación de la PAC en 1962
			\3 Años 70: modernización y expansión
			\3 Años 80
			\3  Reforma McSharry de 1992
			\3 Agenda 2000
			\3 Reforma Fischler II de 2003
			\3 Chequeo de 2009
		\2 Marco jurídico
			\3 OMC
			\3 TFUE
			\3 Principios fundamentales de la PAC
			\3 Marco Financiero Plurianual
			\3 Reglamentos básicos
			\3 Legislación estatal
		\2 Marco financiero
			\3 Evolución del gasto
			\3 MFP 2014-2020
			\3 MFP 2021-2027
			\3 FEAGA -- EAGF
			\3 FEADER -- EAFRD
			\3 Restricciones presupuestarias
		\2 Actuaciones
			\3 \underline{Pilar I: ayudas directas y medidas de mercado}
			\3 {Organización Común de Mercado}
			\3 {Pagos directos}
			\3 \underline{Pilar II: desarrollo rural}
		\2 Valoración
			\3 Instrumento de integración
			\3 Contrafactual
			\3 Productividad
			\3 Rentas agrícolas
			\3 Estabilidad de mercados
			\3 Seguridad del suministro
			\3 Precios accesibles
		\2 Retos
			\3 Debate PAC post 2020
			\3 Ronda de Doha
			\3 Brexit
			\3 Donald Trump
			\3 Papel del Parlamento Europeo
			\3 MFP 2021-2027
			\3 Covid-19
	\1 \marcar{Política Pesquera Común}
		\2 Justificación
			\3 Recurso común
			\3 Internacional
			\3 Mercado de pescado
			\3 Sector pesquero
		\2 Objetivos
			\3 Estabilizar mercado
			\3 Mejorar estructura del sector
			\3 Conservar recursos
			\3 Negociar internacionalmente
		\2 Antecedentes
			\3 Tratado de Roma 1957
			\3 Años 70
			\3 PPC de 1983
			\3 Reforma de 1992
			\3 Reforma de 2002
			\3 Reforma 2013: actualidad
			\3 Brexit
		\2 Marco jurídico
			\3 Reglamento 1303/2013 -- Disposiciones comunes a FEIE
			\3 Reglamento 508/2014 - FEMP
			\3 Reglamento 1379/2013 - OCM Pesca
			\3 Reglamento 1380/2013 - PPC
			\3 Competencias
		\2 Marco financiero
			\3 FEMP -- Fondo Europeo Marítimo y de Pesca
			\3 FEIE -- Fondos Estructurales y de Inversión Europeos
		\2 Actuaciones
			\3 Agencia Europea de Control de la Pesca
			\3 Organización de mercado
			\3 Reformas estructurales
			\3 Conservación
			\3 Supervisión y control de la actividad pesquera
			\3 Relaciones extra-UE
			\3 Planes Operativos de España para FEMP
		\2 Valoración
			\3 Conservación
			\3 Exceso de capacidad
			\3 Compromiso científicos--pescadores
		\2 Retos
			\3 Deterioro caladeros
			\3 Exceso de capacidad
			\3 Regulación compleja
			\3 Conflicto intereses
			\3 Centralización vs descentralización
	\1[] \marcar{Conclusión}
		\2 Recapitulación
			\3 PAC
			\3 PPC
			\3 Por qué
			\3 Cómo
		\2 Idea final
			\3 Complejidad
			\3 Político
			\3 Evolución futura

\end{esquema}

\esquemalargo














\begin{esquemal}
	\1[] \marcar{Introducción}
		\2 Contextualización
			\3 Unión Europea
				\4 Institución supranacional ad-hoc
				\4[] Diferente de otras instituciones internacionales
				\4[] Medio camino entre:
				\4[] $\to$ Federación
				\4[] $\to$ Confederación
				\4[] $\to$ Alianza de estados-nación
				\4 Origen de la UE
				\4[] Tras dos guerras mundiales en tres décadas
				\4[] $\to$ Cientos de millones de muertos
				\4[] $\to$ Destrucción económica
				\4[] Marco de integración entre naciones y pueblos
				\4[] $\to$ Evitar nuevas guerras
				\4[] $\to$ Maximizar prosperidad económica
				\4[] $\to$ Frenar expansión soviética
				\4 Objetivos de la UE
				\4[] TUE -- Tratado de la Unión Europea
				\4[] $\to$ Primera versión: Maastricht 91 $\to$ 93
				\4[] $\to$ Última reforma: Lisboa 2007 $\to$ 2009
				\4[] Artículo 3
				\4[] $\to$ Promover la paz y el bienestar
				\4[] $\to$ Área de seguridad, paz y justicia s/ fronteras internas
				\4[] $\to$ Mercado interior
				\4[] $\to$ Crecimiento económico y estabilidad de precios
				\4[] $\to$ Economía social de mercado
				\4[] $\to$ Pleno empleo
				\4[] $\to$ Protección del medio ambiente
				\4[] $\to$ Diversidad cultural y lingüistica
				\4[] $\to$ Unión Económica y Monetaria con €
				\4[] $\to$ Promoción de valores europeos
				\4[$\to$] Objetivos de la UE
				\4[] Paz y bienestar a pueblos de Europa
			\3 Competencias de la UE
				\4 Tratado de la Unión Europea
				\4[] Atribución
				\4[] $\to$ Sólo las que estén atribuidas a la UE
				\4[] Subsidiariedad
				\4[] $\to$ Si no puede hacerse mejor por EEMM y regiones
				\4[] Proporcionalidad
				\4[] $\to$ Sólo en la medida de lo necesario para objetivos
				\4 Exclusivas
				\4[] i. Política comercial común
				\4[] ii. Política monetaria de la UEM
				\4[] iii. Unión Aduanera
				\4[] iv. Competencia para el mercado interior
				\4[] v. Conservación recursos biológicos en PPC
				\4 Compartidas
				\4[] i. Mercado interior
				\4[] ii. Política social
				\4[] iii. Cohesión económica, social y territorial
				\4[] iv. Agricultura y pesca \footnote{Salvo en lo relativo a la conservación de recursos biológicos marinos, que se trata de una competencia exclusiva de la UE}
				\4[] v. Medio ambiente
				\4[] vi. Protección del consumidor
				\4[] vii. Transporte
				\4[] viii. Redes Trans-Europeas
				\4[] ix. Energía
				\4[] x. Área de libertad, seguridad y justicia
				\4[] xi. Salud pública común en lo definido en TFUE
				\4 De apoyo
				\4[] Protección y mejora de la salud humana
				\4[] Industria
				\4[] Cultura
				\4[] Turismo
				\4[] Educación, formación profesional y juventud
				\4[] Protección civil
				\4[] Cooperación administrativa
				\4 Coordinación de políticas y competencias
				\4[] Política económica
				\4[] Políticas de empleo
				\4[] Política social
			\3 Sector agroalimentario de la UE
				\4 Poco peso relativo PIB
				\4 Poco peso relativo población
				\4 Potencia exportadora agrícola\footnote{Entre mayo de 2016 y abril de 2017, superávit exportador de 18.600 millones de euros.}
				\4 Importador productos pesqueros
			\3 Comercio intra-UE en agroalimentario
				\4 Totalmente liberalizado
				\4 Incluyendo agroalimentario
			\3 Factores comunes de agricultura y pesca
				\4 Ley de Engel
				\4[] Demanda de bienes alimentarios
				\4[] $\to$ Relativamente inelástica
				\4[] Crece menos que renta
				\4[] $\then$ Cada vez menos \% a alimentario
				\4 Oferta inelástica en corto plazo
				\4[] Shocks de oferta y demanda
				\4[] $\to$ Fuerte efecto sobre precio
				\4[] $\then$ Elevada volatilidad de precio
				\4 Efecto sobre medioambiente
			\3 Políticas nacionales pre-integración
				\4 Necesario sustituir tras
				\4[] Apertura intra-UE
				\4 Aparición de nuevos ganadores y perdedores
				\4 Atienden a otros objetivos
				\4[] Equilibrio interno
				\4[] Equilibrios de economía política interna
		\2 Objeto
			\3 ¿Qué son la PAC y la PPC?
			\3 ¿Por qué se llevan a cabo?
			\3 ¿Qué actuaciones implican?
			\3 ¿Cuáles son sus problemas?
		\2 Estructura
			\3 PAC
			\3 PPC
				\4 Justificación
				\4 Objetivos
				\4 Antecedentes
				\4 Marco jurídico
				\4 Marco financiero
				\4 Actuaciones
				\4 Valoración
				\4 Retos
	\1 \marcar{Política Agrícola Común}
		\2 Justificación
			\3 Seguridad del suministro\footnote{Actualmente, los mercados intracomunitarios agrícolas comunitarios están muy integrados, y un shock nacional puede fácilmente compensarse con importaciones de otros estados miembros. Se aduce que es necesaria un política de soporte de precios para mantener la seguridad alimentaria, cuando este tipo de políticas realmente mantienen la \textit{autosuficiencia} alimentaria, no la seguridad. La seguridad alimentaria depende en último término del ingreso y de la disponibilidad de alimentos, y precios altos pueden impedir a consumidores pobres la compra de alimentos aunque la producción sea autosuficiente.}
				\4 Estabilidad mundial + mercados perfectos
				\4[] $\then$ estabilidad regional
				\4 Problema: mercados imperfectos
				\4[] Almacenamiento insuficiente
				\4[] Costes de transacción y transporte
				\4[] Intervenciones pública ineficientes
				\4[] Restricciones financieras
				\4[] $\then$ No hay mercado mundial integrado
			\3 Rentas del sector agrícola
				\4 Determinadas regiones muy dependientes de agricultura
				\4[] Especialmente sensibles a shocks idiosincráticos
				\4 Garantizar nivel de vida equitativo
			\3 Eficiencia productiva
				\4 Uso de técnicas ineficientes
				\4 Falta de aprovechamiento de economías de escala
				\4[] Atomización de unidades productivas en algunos mercados
				\4 Restricciones de acceso al crédito
				\4 Baja eficiencia técnica relativa \footnote{El Agraa, pág. 310. Por ejemplo, debido a los jóvenes de zonas agrícolas que prefieren trabajar en otros sectores, aumentando la heterogeneidad del capital humano agrícola e impidiendo así que alcancen su potencial técnico.}
				\4 Dificultades acumulación de capital humano
				\4[] Potenciales agricultores jóvenes migran a ciudades
				\4[] $\to$ Capital humano no se acumula en agricultura
				\4[] $\then$ Baja productividad
			\3 Estabilidad de los mercados\footnote{La integración regional ha reducido la importancia de este motivo, así como las mejoras en el almacenamiento.}
				\4 Baja elasticidad-precio de demanda
				\4[] Ley de Engel
				\4[] $\to$ \% renta a alimentos crece muy poco
				\4 Nula elasticidad-precio de oferta en c/p
				\4[] $\then$ precios muy volátiles
			\3 Seguridad alimentaria\footnote{En inglés \textit{food safety}, frente al \textit{food security} al que se refiere el primer motivo.}
				\4 A priori, seguridad desconocida
				\4 Confianza: bien público a proveer
				\4 Necesario proveer a nivel UE
			\3 Medio ambiente
				\4 Agricultura impacta MA
				\4[] Emisiones gases de efecto invernadero
				\4[] Destrucción de hábitats
				\4[] Agricultura extensiva detrimento de intensiva
				\4 Consecuencias transnacionales
				\4[] Clima estable como bien público global
				\4[] Contaminación transfronteriza
				\4 Productores muy organizados a nivel nacional
				\4 Necesaria intervención europea\footnote{Como sucede en ocasiones en la UE, los gobiernos nacionales utilizan las políticas comunitarias para tratar de reducir el impacto político de medidas que perjudican a corto plazo a grupos muy organizados, tales como los agricultores.}
		\2 Objetivos
			\3 Garantizar suministro y calidad
				\4 Productos agrícolas
			\3 Mejorar nivel de vida agricultores
				\4 Estabilizar ingresos
				\4 Aumentar productividad
				\4 Mejorar relación productividad-ingresos
			\3 Proteger el medio ambiente
				\4 Reducción de emisiones
				\4 Reducción de contaminantes
				\4 Mitigar cambio climático
			\3 Precios asequibles y estables
				\4 Evitar fluctuaciones excesivas
				\4 Maximizar poder adquisitivo de consumidores
		\2 Antecedentes\footnote{Ver Zobbe (2001) y Garzon (2006).}
			\3 Idea clave
				\4 Elemento central de CEuropeas y UE
				\4[] Peso elevado en presupuesto
				\4[] Importante peso de agri. en economía política
				\4 Inmovilismo durante décadas
				\4[] Desde creación en 1962 hasta 1992
				\4[] $\to$ Reformas muy poco relevantes
				\4[] $\to$ Agotamiento del modelo
				\4 Contencioso internacional de UE vs resto del mundo
				\4[] Moneda de cambio en negociaciones comerciales
				\4[] Contexto internacional influye reformas internas UE
				\4[] Instrumento para aumentar poder de negociación UE
				\4[] $\to$ ``Atarse al mástil'' gracias a PAC
			\3 Pre-PAC
				\4 Proteccionismo a nivel nacional
				\4 Intereses agrícolas muy protegidos
				\4 EEMM originales importadores de alimentos
				\4[] Salvo Holanda
				\4[] $\to$ Escasez post-guerra mundial
				\4 Mención en Tratado de Roma del 57
				\4[] Sin especificar qué actuaciones
				\4[] Recoge objetivos
				\4[] Afirma necesidad de política agrícola en CEE
				\4 Schultz (1945)
				\4[] Análisis teórico muy influyente
				\4[] Introduce idea de inestabilidad en agricultura
				\4[] $\to$ Diferencia de elasticidades
				\4[] Políticas de estabilización de precios
			\3 Creación de la PAC en 1962
				\4 Tratado de Roma (1957) sobre CE
				\4[] Creación de UA en 12 años
				\4[] Política agrícola común se establecerá
				\4[] Tratado poco detallado en cuanto a PAC
				\4 Símbolo integración inicial
				\4 Impulsores principales:
				\4[] Sicco Mansholt, Hallstein
				\4 Intereses ofensivos en agricultura
				\4[] Francia
				\4[] $\to$ Esfuerzo modernizador de agricultura
				\4[] $\to$ Mecanización
				\4[] Holanda
				\4[] $\to$ Exportador agrícola neto
				\4[] Ambos países
				\4[] $\to$ Ampliar tamaño de mercado exportador
				\4 Intereses defensivos
				\4[] Alemania
				\4[] $\to$ Énfasis reconstrucción sector industrial
				\4[] $\to$ Agricultura poco competitiva
				\4[] $\to$ Fijación de precios agrícolas
				\4 Moneda de cambio negociación Mercado Único
				\4[] Alemania acepta PAC
				\4[] Holanda y Francia aceptan mercado único
				\4 Precios de intervención respecto alemanes
				\4[] Evitar pérdida competitividad agricultura ALE
				\4 Elevado arancel exterior común
				\4[] Mercado protegido frente a competencia exterior
				\4[] Francia expande su mercado potencial
				\4 Política de control de precios
				\4[] Evita incertidumbre a productores
				\4[] $\to$ Saben que hay suelo a precios
				\4[] Prevalece frente a subvenciones a productores
				\4[] $\to$ Precios más altos para consumidores
				\4[] $\to$ Requiere protección frente a importaciones
				\4[] $\to$ Menos impuestos distorsionantes
				\4[] $\to$ Crecimiento fuerte reduce impacto sobre renta
				\4 Principios esenciales de PAC se consolidan
				\4[] i. Unidad de mercado
				\4[] $\to$ Libre comercio bienes agrícolas dentro UE
				\4[] ii. Solidaridad financiera
				\4[] $\to$ Presupuesto comunitario financia intervenciones
				\4[] $\to$ EEMM contribuyen a presupuesto
				\4[] $\to$ Impacto presupuestario a cargo de UE
				\4[] iii. Preferencia comunitaria
				\4[] $\to$ Protección arancelaria elevada
				\4[] $\to$ Evitar competencia exterior
				\4 Problemas originales de la PAC
				\4[] Persisten durante décadas
				\4[] Estructura de negociación evita reformas
				\4[] $\to$ Consenso hasta reformas de años 80 QMV
				\4[] Agricultores no perciben exceso de oferta
				\4[] $\to$ UE siempre compra excesos
				\4[] Incentivos perversos en negociaciones entre EEMM
				\4[] $\to$ Producciones heterogéneas entre miembros
				\4[] $\to$ Cada EM trata de $\uparrow$ PIntervención de su producto
				\4[] $\then$ Acuerdos posibles solo aceptando precios elevados
				\4[] $\then$ Sesgo inflacionario de PAC
				\4 Incentivos a producción excesiva
				\4[] Única forma de aumentar beneficios
				\4[] $\to$ Vender producción al estado
				\4[] $\then$ Incentivos aumentar producción sin límite
				\4 Distorsión mercados internacionales
				\4[] Contexto pre-OMC
				\4[] $\to$ No existe acuerdo agrícola
				\4[] $\to$ Subsidios a exportación agrícola no prohibidos
				\4[] $\then$ Sólo necesario notificar a perjudicados
				\4[] $\then$ Cuotas de importación y exportación habituales
				\4[] Exceso de oferta doméstica
				\4[] UE trata de colocar producción en mercados int.
				\4[] $\to$ Subvención a exportaciones
				\4[] $\then$ Compensar diferencia PIntervención vs PMundial
				\4 Oposición francesa a QMV en PAC
				\4[] Veto unilateral se mantiene posible
				\4[] $\to$ Anquilosamiento del debate
				\4[] $\to$ Posiciones fijas
				\4[] $\to$ Falta de reformas
				\4 Plan Mansholt de 1968
				\4[] Reformar sistema de precios de intervención
				\4[] Primar reformas estructurales
				\4[] Reducir número de explotaciones
				\4[] Aumentar concentración
				\4[$\then$] Alemania se opone
				\4[] Reducido tamaño medio de explotaciones
				\4[] Afirma elevados costes socioecónomicos de transición
				\4[$\then$] Francia neutral
				\4[] Beneficiada de precios de intervención
				\4[] Estructura de explotaciones más concentrada que ALE
				\4[$\then$] Holanda apoya plan
				\4[$\then$] Aprobación de versión reducida en 1971
				\4[] Incentivos a abandono actividad agrícola
				\4[] Formación e información a agricultores
			\3 Años 70: modernización y expansión
				\4 Modernización de la agricultura europea
				\4[] Beneficios permiten reinversión
				\4[] Rentabilidad elevada atrae inversión
				\4[] Avances tecnológicos
				\4 Aumento producción
				\4[] Europa deja de ser importadora neta
				\4 Tendencia autosuficiencia
				\4[] $\to$ Ingresos de importación se reducen
				\4[] $\then$ Presión sobre presupuesto comunitario
			\3 Años 80
				\4 Contexto macroeconómico
				\4[] Recesión en muchos EEMM
				\4[] Inestabilidad monetaria
				\4[] $\to$ Volatilidad del TCN
				\4[] $\to$ Inflación
				\4 UE es uno de principales exportadores mundiales
				\4[] Pero mercados muy distorsionados
				\4[] Miedo a competencia exterior si apertura
				\4 Enormes excedentes
				\4[] Sobre inversión en 70s y 80s
				\4[] Instrumentos PAC utilizados al máximo
				\4[] $\to$ Precios de intervención
				\4[] $\to$ Subsidios a exportación
				\4 Subsidios a la exportación
				\4[] Protestas PEDs agrícolas
				\4 Enorme coste presupuestario de PAC
				\4[] Aumento desde 1973 a 1989
				\4[] $\to$ Se dobla el \% del PIB para PAC
				\4 Tensiones internas dentro de UE
				\4[] Reino Unido apenas recibe subsidios PAC
				\4[] Subsidios agrícolas pagados con presupuesto
				\4[] Aranceles agrícolas contribuyen a presupuesto
				\4[] $\then$ Transferencia de importadores netos a exportadores
				\4 Reclamaciones británicas: cheque británico (1984)
				\4[] Consejo de Fontainebleau de 1984
				\4[] Compensación al RU de 66\% de saldo neto negativo
				\4[] $\to$ En función de renta nacional
				\4[] $\to$ GER, NED, SWE y AUS sólo 25\% de lo que les correspondería
				\4 Imposición de cuotas en 1984
				\4[] Tratar de reducir excedentes de producción
				\4[] Asociadas a extensión agrícola utilizada
				\4[] Multas si producción supera cuota
				\4[] Especialmente controvertidas en producción de leche
				\4[] Mercado de cuotas
				\4[] $\to$ Son activo negociable
				\4[] $\then$ En vigor hasta 2015
				\4 Negociaciones de Ronda de Uruguay 1986
				\4[] Exportaciones agrícolas principal contencioso
				\4 Intentos fallidos de reforma
				\4[] Bajar precios de intervención
				\4[] $\to$ Necesarios pagos compensatorios
				\4[] $\then$ Para ser políticamente viable
				\4[] Introducción de cuotas
				\4[] $\to$ A producción
				\4[] $\to$ A inputs
				\4[] $\to$ A tierras roturadas
				\4[] $\to$ Fertilizantes
			\3  Reforma McSharry de 1992
				\4 Pocas reformas hasta entonces
				\4[] Apenas cambios superficiales
				\4[] Introducción de cuotas mayor cambio
				\4 Contexto
				\4[] Negociaciones Ronda de Urugay
				\4[] EEUU, AUS, JAP impulsores de liberalización
				\4[] Intereses agrícolas heterogéneos
				\4[] $\to$ UE, JAP fuertemente defensivos
				\4[] $\to$ EEUU, AUS, CAN, PEDs ofensivo
				\4 Cambios en opinión pública contra PAC
				\4[] Precios alimentación elevados
				\4[] Medio ambiente preocupa cada vez más
				\4[] Agotamiento de recursos naturales
				\4[] Incremento de regulación food safety
				\4[] Preocupación por desarrollo PEDs
				\4 Cambio de enfoque precios a ingresos
				\4[] Hasta ahora
				\4[] $\to$ Mantener precios en determinado nivel
				\4[] A partir de ahora
				\4[] $\to$ Garantizar ingresos agrícolas
				\4[] $\then$ Independientemente de mercado
				\4 Contexto de Comisión Delors\footnote{Se extiende del 85 al 95.}
				\4[] Impulso general a políticas europeas
				\4[] Preparación para Maastricht
				\4 Reducción significativa de precios de intervención
				\4[] Acercar precios UE a precios mundiales
				\4[] $\to$ Cereales
				\4[] $\to$ Vacuno
				\4[] $\to$ Leche y derivados
				\4 Introducción de pagos directos
				\4[] Compensar cambio de precios de intervención
				\4[] $\to$ Pagos directos para mitigar pérdida de ingresos
				\4 Reducir gasto en subsidios a exportación
				\4[] Cumplir con nuevos acuerdos OMC
				\4 Permitió concluir Ronda de Uruguay
				\4 Reformas estructurales: aumento de fondos
				\4[] Incentivos a jubilación anticipada
				\4[] Incentivos abandono tierras poco productivas
				\4[] Fondos estructurales
				\4[] $\to$ Incluyen fondos agrícolas estructurales
				\4[] Fondo reforma MAmbientales
				\4[] $\to$ Por primera vez también a nivel UE
			\3 Agenda 2000
				\4 Preparación para ampliación de 2003
				\4 Preparación para Ronda de Doha
				\4 Tres escenarios posibles
				\4[] i. Mantener soporte de precios y controlar vía cuotas
				\4[] $\then$ Mantenimiento de status quo
				\4[] ii. Abandonar todo soporte de precios e intervención
				\4[] $\then$ Abandonar todos los parámetros de PAC
				\4[] iii. Proceso gradual de reducción de soporte + pagos directos
				\4[] $\then$ Profundización de reforma McSharry
				\4[] $\to$ Se eligió la tercera opción
				\4 Reducciones de los límites de precios
				\4[] Limitar gasto agrícola
				\4 Aumento de compensaciones a productores
				\4[] Pero parciales respecto $\downarrow$ precios
				\4[] $\then$ Compensación no es completa
				\4 Configuración en dos pilares
				\4[] Pilar I:
				\4[] $\to$ Intervención precios
				\4[] $\to$ Pagos directos
				\4[] $\to$ Posible modular por criterios socioeconómicos
				\4[] $\to$ Posible ligar a MAmbiente
				\4[] Pilar II:
				\4[] $\to$ Pagos directos
				\4[] $\to$ Ligados a objetivos sociales y MA
			\3 Reforma Fischler II de 2003
				\4 Reforma parcial de Agenda 200
				\4 Desacoplamiento completo de pagos a agricultores
				\4[] Consolidación pagos directos
				\4[] $\to$ No ligados a producción
				\4 Aumento de condicionalidad
				\4[] Condicionados a otros objetivos sociales
				\4[] Esquema único de pagos
				\4 Nuevas áreas en Pilar II
				\4[] Calidad
				\4[] Sistemas de certificación
				\4[] Promoción de certificados de calidad
				\4[] Aumento de cofinanciación en MA
			\3 Chequeo de 2009
				\4[] Simplificaciones, correcciones
				\4 Mayor modulación de ayudas en función de tamaño
				\4 Mayor gasto en renovables, clima, gestión del agua
				\4 Mayores subvenciones a jóvenes
				\4 Extensión del desacoplamiento a más actividades
		\2 Marco jurídico
			\3 OMC
				\4 GATT-47: nada en la práctica
				\4 GATT-94: Acuerdo Agrícola\footnote{ Ver \url{https://www.wto.org/english/tratop_e/agric_e/negs_bkgrnd13_boxes_e.htm}}
				\4[] Objetivo: apertura mercados
				\4[] Caja ámbar:
				\4[] $\to$ Soporte distorsionante de producción y comercio
				\4[] $\to$ 5\% de soporte sobre producción permitido\footnote{Para desarrollados.}
				\4[] $\to$ Reducción obligada si supera
				\4[] Caja azul:
				\4[] $\to$ Soporte distorsionante no ligado a producción
				\4[] $\to$ Se exige a productores limitar producción
				\4[] $\to$ Sin límites de ayudas
				\4[] Caja verde:
				\4[] $\to$ Subsidios no distorsionantes
				\4[] $\to$ Soporte de precios no permitido
				\4[] Caja de desarrollo:
				\4[] $\to$ Flexibilidad adicional para PEDs
				\4[] Arancelización
				\4[] Reducción subsidios exportación
				\4[] Apertura mínima 5\%
			\3 TFUE
				\4 Artículo 3 - Competencias exclusivas
				\4[] Unión Aduanera
				\4[] Política comercial
				\4 Artículo 4 - Competencias compartidas
				\4[] Política Agraria Común
				\4 Artículos 38 a 44
				\4[] PAC propiamente
			\3 Principios fundamentales de la PAC
				\4[] \marcar{U}nidad de mercado
				\4[] \marcar{S}olidaridad financiera
				\4[] \marcar{P}referencia comunitaria
			\3 Marco Financiero Plurianual
				\4 2014-2020
				\4 Epígrafe 2:
				\4[] Crecimiento sostenible y recursos naturales
				\4[] $\sim 39\%$ de los créditos de compromiso
			\3 Reglamentos básicos
				\4 Reglamento 1303/2013 -- Disposiciones comunes a FEIE
				\4 Reglamento 1305/2013 -- Reglamento FEADER
				\4 Reglamento 1306/2013 -- Reglamento FEAGA
				\4 Reglamento 1307/2013 -- Pagos directos
				\4 Reglamento 1308/2013 -- OCM
				\4 Aprobados en 2013
				\4[] Previos a MFP
				\4[I] Desarrollo rural
				\4[II] Horizontal sobre pagos y controles
				\4[III] Pagos directos a agricultures
				\4[IV] Medidas de mercado
				\4 Reglamentos sobre sistemas de control de ayudas
			\3 Legislación estatal
				\4 Reales Decretos de 2014 sobre aplicación de PAC
				\4 Rd 1378/2018 sobre aplicación de la PAC
				\4 Real Decreto de 2019 de reforma de decretos
				\4 Otros reales decretos
		\2 Marco financiero\footnote{Ver \url{https://cohesiondata.ec.europa.eu/funds/eafrd} para visión general del gasto en FEIE.}
			\3 Evolución del gasto
				\4 66\% principios 80s
				\4 35\% último MFP
				\4 Disminución 2014 a 2020
				\4 Prevista disminución hasta $\sim 30\%$ en MFP 2021-2027
			\3 MFP 2014-2020
				\4 408.000 M €
				\4 \underline{Primer pilar}
				\4[] Medidas de mercado: 4,3\% total PAC
				\4[] Pagos directos: 71,3\% total PAC
				\4 \underline{Segundo pilar}
				\4[] Medidas de desarrollo rural: 24,4\%
			\3 MFP 2021-2027
				\4 Disminución absoluta y relativa
				\4[] Muy leve disminución absoluta
				\4[] Disminución relativa y en inflación
				\4 Paso a $\sim$ 30\%
				\4[] Especialmente en desarrollo rural
				\4[] Mantenimiento aproximado de pagos directos
			\3 FEAGA -- EAGF\footnote{Fondo Europeo Agrícola de Garantía / \textit{European Agricultural Guarantee Fund}.}
				\4 Dividido del FEOGA en 2005
				\4 Fondo Europeo Agrícola de Garantía
				\4 Tratamiento en Balanza de pagos
				\4[] Cuenta Corriente
				\4[] $\to$ Cuenta del Ingreso Primario
			\3 FEADER -- EAFRD\footnote{Fondo Europeo Agrícola de Desarrollo Rural / \textit{European Agricultural Fund for Rural Development}.}
				\4 Fondo Europeo Agrícola de Desarrollo Rural
				\4 Cuantía
				\4[] 100.000 M de €
				\4[] + $\sim 60.000$ M de EEMM por cofinanciación
				\4 Segundo Pilar
				\4 Tratamiento en Balanza de Pagos
				\4[] Cuenta de Capital
				\4[] $\to$ Transferencias de Capital
			\3 Restricciones presupuestarias
				\4 Ingresos presupuestarios
				\4 Ciclo económico
				\4 Influencia opinión pública
		\2 Actuaciones
			\3 \underline{Pilar I: ayudas directas y medidas de mercado}
			\3 {Organización Común de Mercado}\footnote{Ver \url{https://www.europarl.europa.eu/factsheets/en/sheet/108/first-pillar-of-the-cap-i-common-organisation-of-the-markets-cmo-in-agricultural}}
				\4 Reglamento 1308/2013 de la UE
				\4 Regulación global de mercados agrícolas europeos
				\4 Unificación en 2007 (antes 21 OCM)
				\4 Gran complejidad
				\4[] $\to$ 232 artículos en la OCM de la PAC
				\4[] $\to$ + reglas de actos delegados y de ejecución
				\4[] \underline{Vertiente externa}
				\4 Certificados de importación y exportación
				\4 Derechos de importación
				\4 Contingentes arancelarios
				\4 Regímenes especiales de importación
				\4[] Vigilancia, autorización, certificación
				\4 Importaciones especiales
				\4 Medidas de salvaguardia
				\4 Perfeccionamiento activo
				\4[] Importar para transformar en Europa
				\4[] $\to$ A arancel bajo o exento
				\4[] Para reexportar en el exterior
				\4[] $\to$ Mejorar protección efectiva
				\4 Restituciones por exportación
				\4[] Cantidades pagadas a empresas europeas
				\4[] $\to$ Por exportar prod. agrícolas o transformados
				\4[] Situar productos europeos a mismo nivel de precios
				\4[] $\to$ Evitar desaparezcan de mercados mundiales
				\4[] $\then$ Dado que precios en Europa son más altos
				\4 Perfeccionamiento pasivo
				\4[] Exportar bien para transformar en exterior
				\4[] $\to$ Reimportar con arancel más bajo
				\4[] \underline{Vertiente interna}
				\4 Medidas de intervención
				\4[] $\to$ $\sim 3.000$ M de € en últimos años
				\4[] $\to$ Tendencia a reducción progresiva
				\4[] $\then$ Compras a precios mínimos
				\4 Almacenamiento privado
				\4 Regímenes de ayudas a sectores
				\4 Autorizaciones en sectores concretos
				\4 Sectores individuales
				\4[] Sistemas de regulación de producción
				\4[] Inventarios
				\4[] Declaraciones obligatorias de producción
				\4 Organizaciones de productores
				\4[] Normas para sectores específicos
				\4[] Sistemas contractuales
				\4[] Sectores específicos
				\4 Fondo de reserva para emergencias
				\4[] $\to$ Financiado con reducciones anuales a pagos directos
				\4[] $\to$ Apoyo a sector en situaciones excepcionales
				\4[] \underline{Competencia}
				\4 Regímenes de ayudas nacionales
				\4 Pagos a mercados y segmentos especiales
				\4[] \underline{Medidas generales}
				\4 Medidas para evitar perturbaciones
				\4 Medidas frente a enfermedades animales
			\3 {Pagos directos}
				\4 Marco común europeo pero variaciones nacionales
				\4[] $\to$ Régimen de pago básico\footnote{Ver \url{https://ec.europa.eu/info/sites/info/files/food-farming-fisheries/key_policies/documents/basic-payment-scheme_en.pdf}}
				\4 Obligatorio ofrecer en todos los países:
				\4[] $\to$ Pago básico por hectárea
				\4[] $\to$ Pago por métodos sostenibles
				\4[] $\to$ Pago a agricultores jóvenes
				\4[] $\then$ ``Pagos obligatorios''
				\4 Vinculados a dimensiones segmentadas:
				\4[1] Pago por ha.
				\4[2] Protección del MA
				\4[3] Agricultores jóvenes
				\4[4] Refuerzo primeras ha.
				\4[5] Zonas con limitaciones naturales
				\4[6] Determinadas zonas o cultivos
				\4[7] Pequeños agricultores
			\3 \underline{Pilar II: desarrollo rural}\footnote{\url{https://ec.europa.eu/agriculture/rural-development-2014-2020_en}}
				\4 118 programas de desarrollo rural
				\4 Acuerdos de asociación nacional/National Frameworks
				\4[] Coordinar actuaciones del FEADER
				\4[] $\to$ En marco de FEIE
				\4[] Comisión debe aprobar
				\4[] $\to$ Antes de aprobar PDRural
				\4 PDR -- Programa de desarrollo rural nacional
				\4[] Planificación de actuaciones a nivel nacional
				\4 Programas de desarrollo regional
				\4[] Concretan actuaciones a nivel de región
				\4[] $\to$ Especialmente relevante para España
				\4 Estrategia Europa 2020
				\4 Cofinanciación variable
				\4 Programas EEMMs
				\4 Eligen entre menú de medidas
				\4 Transferencia de conocimientos
				\4 Sostenibilidad del desarrollo
				\4 Transición a renovables en medio rural
				\4 Reducción de emisiones y contaminación
		\2 Valoración
			\3 Instrumento de integración
				\4 Etapas iniciales de la UE
				\4 Experimento de integración
				\4 Moneda de cambio Francia-Alemania
			\3 Contrafactual
				\4 PAC menos proteccionista
				\4 Más flexible exteriormente
			\3 Productividad
				\4 $\uparrow$ explotaciones ineficientes
			\3 Rentas agrícolas
				\4 Dudoso efecto
				\4 A l/p, ingreso/trabajo = coste de oportunidad
				\4 Incentivos oferta laboral
				\4 Propietarios tierra muy beneficiados
			\3 Estabilidad de mercados
				\4 Precios muy estables
				\4 Ingreso inestable
				\4[] Efecto elasticidades bajas
			\3 Seguridad del suministro
				\4 Dudosa
				\4 Aumento de la producción
				\4[] Pero aumento también de insumos
				\4 Aumento de los precios
			\3 Precios accesibles
				\4 $\text{Precio} > \text{Precios mundiales}$
				\4 Coste pagos directos
				\4 Pérdidas irrecuperables eficiencia
		\2 Retos
			\3 Debate PAC post 2020
				\4 Actualmente en marcha
				\4 Introducción de gestión del riesgo
				\4 Reducción pagos directos frente a pilar II?
				\4 Unificación pilares?
				\4[] Pilar único apoyo agrorrural
				\4[] Cofinanciado
				\4 Coordinación políticas de cohesión
			\3 Ronda de Doha
			\3 Brexit
			\3 Donald Trump
			\3 Papel del Parlamento Europeo
			\3 MFP 2021-2027
			\3 Covid-19
	\1 \marcar{Política Pesquera Común}
		\2 Justificación
			\3 Recurso común
				\4 No excluible
				\4 Rival
				\4 Agotable
			\3 Internacional
				\4 Bancos pesqueros: múltiples EMs
			\3 Mercado de pescado
				\4 Volatilidad de precios
				\4 Incentivos a sobreexplotación
				\4 Sostenibilidad a largo plazo
			\3 Sector pesquero
				\4 Sobrecapacidad
				\4 Concentrado regionalmente
				\4 Pocas alternativas de empleo
		\2 Objetivos
			\3 Estabilizar mercado
				\4 Evitar volatilidad excesiva de precios
			\3 Mejorar estructura del sector
				\4 Reducir excesos de capacidad
				\4 Aumentar eficiencia media de productores
			\3 Conservar recursos
				\4 Evitar sobreexplotación de bancos de pescado
				\4 Acercar pesca a Optimum Sustainable Yield
				\4 Incentivos individuales a pescar más
				\4[] $\to$ Explotación superior a Maximum Sustainable Yield
			\3 Negociar internacionalmente
				\4 Proveer acceso a caladeros internacionales
				\4 Aumentar poder de negociación UE
		\2 Antecedentes
			\3 Tratado de Roma 1957
				\4 Pesca forma parte de agricultura
			\3 Años 70
				\4 Adopción de ZEE: 200 millas
				\4 Adhesión Reino Unido, Dinamarca, Irlanda
				\4 Competencia UE sobre recursos pesqueros
			\3 PPC de 1983
				\4 Adopción PPC propiamente
				\4 Principio de estabilidad relativa\footnote{La \textit{estabilidad relativa} hace referencia al mantenimiento de los derechos históricos de pesca de cada país a la hora de fijar las capturas máximas autorizadas, de tal manera que cada país mantiene la cuota que le correspondía históricamente.}
				\4 Medidas conservacionistas
				\4 Totales autorizados de capturas (TAC)
				\4 Cuotas
				\4 Adhesión España, Portugal
				\4 Retirada Groenlandia
				\4 Reunificación Alemania
			\3 Reforma de 1992
				\4 Desequilibrio flotas-recursos
				\4 Reducción flotas
				\4 Medidas estructurales desempleo
			\3 Reforma de 2002
				\4 Reformas anteriores: fracaso
				\4 Intento mejora sostenibilidad
				\4 Ayudas: sólo seguridad y trabajo a bordo
				\4 Agencia Europea Control de la Pesca (Vigo)
				\4[] Funcionando desde 2005
				\4 Consejos consultivos Regionales
			\3 Reforma 2013: actualidad
				\4 Problemas políticos asociados crecientes
				\4[] Percepción de excesiva centralización en Bruselas
				\4[] Descartes criticados por opinión pública
				\4[] $\to$ Objeto de críticas de euroescépticos
				\4 Libro Verde de 2009
				\4[] Comisión autocrítica sobre PPC
				\4[] $\to$ Sin objetivos claros de largo plazo
				\4[] $\to$ Corto plazo prevalece sobre largo plazo
				\4[] $\to$ Excesiva microgestión desde Bruselas
				\4[] $\to$ Exceso de capacidad sin atajar
				\4[] $\to$ Poca voluntad política real
				\4 Prohibición de los descartes
				\4 Consolidación del principio MSY
				\4[] Objetivo central de la gestión de recursos pesqueros
				\4 Ligera descentralización
				\4 Fracasa intento de implantar permisos transferibles
				\4[] No fue aceptado por mayoría en PE ni CdUE
				\4 Ver Lado (2013)
			\3 Brexit
				\4 Acuerdo transitorio para 2020
				\4[] Sin cambios respecto a situación anterior
				\4 Problemas en la negociación
				\4[] UK rechaza negociar acceso en acuerdo post-brexit
				\4[] Dudas sobre acceso a caladeros
		\2 Marco jurídico
			\3 Reglamento 1303/2013 -- Disposiciones comunes a FEIE
			\3 Reglamento 508/2014 - FEMP
			\3 Reglamento 1379/2013 - OCM Pesca
			\3 Reglamento 1380/2013 - PPC
			\3 Competencias
				\4[] \underline{Legislación}
				\4 Parlamento Europeo
				\4 Consejo de la UE
				\4[$\then$] Codecisión
				\4[] \underline{Acuerdos internacionales}
				\4 Parlamento: aprobación
				\4 Consejo: ratificación
		\2 Marco financiero
			\3 FEMP -- Fondo Europeo Marítimo y de Pesca
				\4 6400 M de € para MFP 2014-2020
			\3 FEIE -- Fondos Estructurales y de Inversión Europeos
				\4 FEMP es parte de FEIE
				\4 Otros fondos complementan acciones
		\2 Actuaciones
			\3 Agencia Europea de Control de la Pesca\footnote{Ver \href{https://europa.eu/european-union/about-eu/agencies/efca_es}{CE sobre Agencia Europea de Control de la Pesca.}}
				\4 Creada en 2005
				\4 Sede en Vigo
				\4 63 personas
				\4 Organización
				\4[] Consejo de Administración
				\4[] $\to$ 6 representantes de Comisión
				\4[] $\to$ 1 representante de cada EEMM
				\4[] Director ejecutivo
				\4[] $\to$ 5 años renovables
				\4[] $\to$ Elegido por Consejo de Administración
				\4[] Comité consultivo
				\4[] $\to$ Representantes de consejos consultivos
				\4[] $\to$ Asesoramiento a director ejecutivo
				\4 Apoyo a vigilancia costera
				\4 Planes de despliegue conjunto
				\4[] Vigilar poblaciones
				\4[] Velar por OSY y sostenibilidad
				\4 Igualdad de condiciones de industrias europeas
			\3 Organización de mercado
				\4 Retiradas compensadas del mercado\footnote{Los pescadores reciben una compensación por la retirada del producto del mercado general, que se dedica a la producción de piensos, obras de beneficiencia o a fines no alimentarios.}
				\4 Aplazamientos\footnote{Transformaciones en otras variedades de producto apto para consumo humano que permitan su venta futura.}
				\4 Retiradas y aplazamientos por org. de productores
				\4 Almacenamiento privado
				\4 Fomento organizaciones de productores\footnote{Aunque sólo existan cuatro que operen a nivel estatal.}
				\4 Estandarización de etiquetado: mercado único
				\4 Incentivos pesca sostenible
			\3 Reformas estructurales
				\4 Reconversión de las flotas
				\4 Jubilaciones
				\4 Fomento acuicultura
				\4 Mejoras seguridad, higiene, prácticas
				\4 Investigación
				\4 Inversión artes más selectivas
				\4 Infraestructura portuaria
			\3 Conservación
				\4 Limitación de las capturas
				\4 Limitación del esfuerzo pesquero\footnote{Por ejemplo, días de pesca autorizada.}
				\4 Medidas técnicas: especies prohibidas, tamaños...
				\4 Planes plurianuales para poblaciones amenazadas
				\4 Gestión de la flota
			\3 Supervisión y control de la actividad pesquera
				\4 Velar por cumplimiento legislación
				\4 Imposición de trazabilidad
				\4 Agencia Europea del Control de la Pesca
			\3 Relaciones extra-UE
				\4 ZEE: 35\% mares, 90\% recursos
				\4 21 acuerdos pesqueros
				\4 Acuerdos asociación
				\4[] Generalmente con países ACP
				\4[] Promover explotación de recursos
				\4[] Pago para administrar, controlar, investigar...
				\4[] Objetivo: promover pesca sostenible + acceso
			\3 Planes Operativos de España para FEMP
				\4 Co-financiación del 75\%
				\4[1] \marcar{C}aladeros
				\4[] Reforma de la flota
				\4[] Reconversión de embarcaciones
				\4[] Aumentar valor añadido de capturas
				\4[] Diversificación de caladeros
				\4[] Promoción del emprendimiento de jóvenes pescadores
				\4[] Uso de capturas indeseadas
				\4[2] \marcar{A}cuicultura
				\4[] Investigación de nuevos cultivos
				\4[] Promoción de principales especies cultivadas
				\4[] $\to$ Atún, lenguado, trucha, mejillón
				\4[] $\to$ Lubina, rodaballo, besugo
				\4[3] \marcar{P}olítica de Pesca Común
				\4[] Cumplimiento de política de pesca común
				\4[] Refuerzo de medidas de control e inspección
				\4[] Desarrollo de programas específicos
				\4[] Monitorización de Capturas Máximas Permitidas
				\4[4] \marcar{D}esarrollo local
				\4[] Diversificación del empleo en zonas pesqueras
				\4[] Inversión en actividades accesorias
				\4[] Mejora y capitalización de activos medioambientales
				\4[5]\marcar{M}arketing y procesamiento
				\4[] Industria de procesamiento es 2ª de Europa
				\4[] 15\% de producción y empleo del sector en UE
				\4[] Problemas habituales de industria española
				\4[] $\to$ Infracapitalización
				\4[] $\to$ Internacionalización
				\4[] $\to$ Bajo valor añadido
				\4[] Mejorar estrategias comerciales
				\4[] Valorización de descartes para humanos y no humanos
				\4[] Competitividad vía organizaciones de productores
				\4[6] \marcar{A}sistencia técnica del programa operativo
				\4[7] \marcar{P}olítica Marítima Integrada
				\4[] Investigación sobre ecosistemas marinos
				\4[] Exploración de plataforma continental
				\4[] Investigación oceanográfica
				\4 {Acuerdos de reciprocidad}
				\4[] Intercambio de cuotas de pesca
				\4{Acuerdos multilaterales}
				\4[] En el seno de OROP\footnote{Organizaciones Regionales de Ordenación Pesquera}
				\4[] Objetivo principal: frenar pesca ilegal
				\4[] {Convenios internacionales}
				\4[] Convenio Naciones Unidas Derecho del Mar
				\4[] Acuerdos de la FAO
		\2 Valoración
			\3 Conservación
				\4 Graves problemas de sobreexplotación
				\4 Necesarios mayores esfuerzos
			\3 Exceso de capacidad
				\4 Persiste problema
			\3 Compromiso científicos--pescadores
				\4 Punto medio insatisfactorio
		\2 Retos
			\3 Deterioro caladeros
			\3 Exceso de capacidad
				\4 Persiste a pesar PPC
			\3 Regulación compleja
				\4 Esfuerzo de simplificación
				\4 Persistente
				\4 Dificulta cumplimiento
			\3 Conflicto intereses
				\4 Entre países: cuotas acceso
				\4 Nacionales: presiones pescadores
			\3 Centralización vs descentralización
				\4 Intenso debate
				\4 Argumentos a favor y en contra
				\4 Descentralización: tendencia a incumplimiento
				\4 Centralización: problemas de adaptación políticas
	\1[] \marcar{Conclusión}
		\2 Recapitulación
			\3 PAC
			\3 PPC
			\3 Por qué
				\4 Externalidades
				\4 Sectores vulnerables
				\4 Seguridad y alimentaria y del suministro
			\3 Cómo
				\4 Actuaciones
				\4 Marco financiero
				\4 Marco jurídico
		\2 Idea final
			\3 Complejidad
				\4 Regulación extensa
				\4 Multitud de excepciones
				\4 Aspectos técnicos biológicos, medio ambiente
			\3 Político
				\4 Presiones grupos de interés
				\4 Path-dependency
			\3 Evolución futura
				\4 Hacer frente a complejidad y factores políticos
				\4 + Desarrollo tecnológico
				\4 + Evolución estructura económica
				\4 + Cambio climático
\end{esquemal}


























\conceptos

\concepto{Cajas verde, azul y ámbar}

En terminología de la OMC y más concretamente del Acuerdo Agrícola, se denominan cajas verde, ambar y roja a la suma de las ayudas permitidas, a reducir y prohibidas, respectivamente. Esta terminología se utiliza también para clasificar las ayudas nacionales a la agricultura en el contexto de la PAC.

En la \textbf{caja ámbar} se incluyen las medidas que se considera tienen un efecto distosionador sobre el comercio y la producción, lo cual incluye medidas de soporte de precios y medidas y subsidios relacionados con la producción, así como todas aquellas ayudas que no se incluyan en las cajas azul y verde. En la \textbf{caja azul} se incluyen aquellas medidas de apoyo a los productores locales que al mismo tiempo los obligan a limitar la producción con el objetivo de reducir las distorsiones. En la \textbf{caja verde} se incluyen otras ayudas desvinculadas con la producción sin límite de cuantía permitida, tales como ayudas al i+D, mejoras técnicas, inversión, protección del medio ambiente... Deben tener origen directo en los fondos públicos sin implicar precios más altos para los consumidores o soporte alguno de precios. Las ayudas \comillas{desacopladas} forman parte de esta caja verde, al no tener relación con los precios o la producción.

\concepto{Principio de estabilidad relativa}

En el marco de la Política Pesquera Común, el principio de estabilidad relativa establece que las cuotas de pesca asignadas a cada país permanecen constantes en el tiempo. 

\preguntas

\seccion{Test 2017}
\textbf{41.} En las reformas de la Política Agraria Común (PAC) del 2013.

\begin{itemize}
	\item[a] Los nuevos derechos de pago único se establecen en virtud de referencias regionales directamente relacionadas con el desempleo agrario.
	\item[b] Desaparecen en su totalidad las llamadas ayudas acopladas.
	\item[c] Se fortalecen y protegen los llamados derechos históricos de los agricultores.
	\item[d] Se introduce, por primera vez, el concepto de agricultor activo como criterio fundamental para la obtención del nuevo sistema de ayudas.
\end{itemize}


\seccion{Test 2016}
\textbf{46.} En relación con el Fondo Europeo Marítimo y de Pesca, cuál de las afirmaciones siguientes es verdadera:

\begin{enumerate}
    \item[a] Para el periodo 2014-2020, Francia es el país que recibirá más fondos del FEMP, seguido de España.
    \item[b] Entre sus objetivos primordiales se encuentran facilitar el acceso a la financiación y la diversificación de las economías de las comunidades pesqueras, pero no ayudar a los pescadores en la transición a la pesca sostenible.
    \item[c] El rendimiento máximo sostenible implica optimizar las capturas al máximo sin afectar a la productividad futura de las poblaciones. Dicha limitación debe aplicarse, a más tardar, en el año 2020.
    \item[d] La inversión en acuicultura supone alrededor del 40\% de la inversión total a realizar por el FEMP.
\end{enumerate}

\textbf{48.} ¿Qué es el régimen de pago básico en la PAC?
\begin{enumerate}
    \item[a] Pago efectuado en función de las superficies agrarias poseidas por el agricultor.
    \item[b] Pago mínimo por cabeza de ganado.
    \item[c] Ayuda mínima asociada a ciertos cultivos.
    \item[d] Ayuda básica a la exportación de productos agrícolas excedentarios.
\end{enumerate}

\seccion{Test 2014}
\textbf{42.} En la PAC se denomina desacoplamiento a:
\begin{itemize}
    \item[a] Desvinculación completa de los pagos respecto a la producción agrícola.
    \item[b] Una reducción de los precios de los precios de intervención.
    \item[c] Fijar un tope máximo al gasto de la PAC
    \item[d] Desvinculación de la PAC de algunas Organizaciones Comunes de Mercado
\end{itemize}

\seccion{Test 2013}
\textbf{42.} Para una Organización Común de Mercado, el precio mínimo garantizado es:
\begin{itemize}
    \item[a] Precio suelo
    \item[b] Precio techo
    \item[c] Precio de intervención
    \item[d] Precio umbral
\end{itemize}

\seccion{Test 2011}
\textbf{43.} La reforma de la Política Agraria Común (PAC) del año 2003 supuso:
\begin{itemize}
    \item[a] El llamado desacoplamiento que trata de desvincular las producciones agrarias de los programas de desarrollo rural.
    \item[b] El pago único que sustituye definitivamente a los programas de desarrollo rural.
    \item[c] La condicionalidad que implica que la cuantía del pago único/ayuda directa está condicionado/a al respeto del medio ambiente y del bienestar y sanidad animal.
    \item[d] El incremento considerable de los precios de los productos agrícolas.
\end{itemize}

\seccion{Test 2005}
\textbf{44.} En relación con la Política Agrícola de la Unión Europea, la reforma de 2003 supone:
\begin{itemize}
    \item El desmantelamiento de las Organizaciones Comunes de Mercado (OCM) así como de los instrumentos previos de fijación de precios por productos.
    \item[b] La introducción en la mayoría de las OCM de un nuevo sistema de pago único por explotación que disocia la recepción de ayudas de la producción, así como la reducción progresiva de las ayudas directas que aún se mantienen.
    \item[c] La reducción progresiva de las ayudas directas sólo para algunas OCM y manteniendo su vinculación con las decisiones de producción.
    \item[d] La introducción de un pago único por explotación para todas las OCM y eliminación del resto de las actuales ayudas directas.
\end{itemize}

\seccion{Test 2004}
\textbf{42.}
Cuando en la UE se habla de la \comillas{línea directriz agraria o agrícola} se está haciendo alusión:
\begin{enumerate}
    \item[a] A la evolución prevista para la parte más importante de los gastos de la Política Agrícola Común (PAC) contenidos en la rúbrica 1 del presupuesto de la UE.
    \item[b] A las modificaciones introducidas al inicio de cada campaña agrícola para adaptar desde el punto de vista reglamentario y presupupestario las normas báscias aplicables a cada Organización Común de Mercado.
    \item[c] A las directivas y reglamentos que desarrollan los cinco objetivos fundamentales de la PAC, recogidos en el artículo 33 del Tratado de la Unión (antiguo artículo 39 del Tratado CEE).
    \item[d] A las directrices u orientaciones básicas presentadas por la Comisión para su aprobación en el Consejo y en el Parlamento Europeo en materia de reforma de la PAC.
\end{enumerate}

\seccion{8 de marzo de 2017}
\begin{itemize}
    \item ¿Qué subvenciones agrarias se incluyen en la Caja Azul?
    \item ¿Por qué dice que la PAC genera un importante problema financiero?
    \item ¿Por qué se produjo un conflicto Norte-Norte (entre EEUU y UE) en el tema agrícola durante la Ronda de Uruguay?
    \item ¿Qué motivos políticos determinan una dotación tan importante para la PAC?
    \item ¿Por qué la UE es intervencionista en la agricultura y liberal en lo rural (e.g. Cabify, AirBnB)? ¿Cuál es el futuro de la PAC en 20 años?
    \item ¿Cuáles serían las consecuencias económicas de la no-PAC?
    \item ¿Qué porcentaje de los presupuestos de la UE van destinados a la PAC? ¿Y a cada pilar en concreto? 
    \item ¿Qué es el FEADER?
\end{itemize}

\seccion{22 de marzo de 2017}
\begin{itemize}
    \item ¿Es lógico que la PAC se lleve el 50\% del presupuesto?
    \item ¿Qué pasaría si la PAC desapareciese en España?
    \item ¿Qué es la Caja Azul?
    \item No ha mencionado el problema de los excedentes, ¿a qué se deben?
    \item ¿Ha oído hablar de la multifuncionalidad?
    \item ¿Qué porcentaje del presupuesto va a cada pilar?
    \item Ha dicho que se podría fomentar el turismo rural en áreas pesqueras. ¿Cómo?
    \item ¿El pago único depende sólo de las hectáreas? ¿Qué otros criterios se aplican?
\end{itemize}


\notas

\textbf{2017}. \textbf{41.} D

\textbf{2016}. \textbf{46.} C \textbf{48.} A

\textbf{2014}. \textbf{42.} A.

\textbf{2013}. \textbf{42.} C.

\textbf{2011}. \textbf{43.} C.

\textbf{2005}. \textbf{44.} B.

\textbf{2004}. \textbf{42.} A


\bibliografia

Mirar Palgrave:
\begin{itemize}
	\item European Union Common Agricultural POlicy (CAP)
	\item fisheries
\end{itemize}

El Agraa, A. \textit{The European Union. Economics and policies}


European Union. \textit{Agriculture and Rural Development}. \url{https://ec.europa.eu/agriculture/cap-overview_en}

Garzon, I. (2006) \textit{Reforming the Common Agricultural Policy. History of a Paradigm Change} Palgrave McMillan -- En carpeta del tema
Lado, E. P. (2016) \textit{The CFP Reform of 2013} The Common Fisheries Policy: The Quest for Sustainability. Ch. 16 -- En carpeta del tema.

Ministerio de Agricultura, Pesca y Alimentación. \textit{Estadísticas pesqueras 2008--2018} \url{http://www.mapama.gob.es/es/estadistica/temas/estadisticas-pesqueras/} -- En carpeta del tema desde noviembre 2017

Olsen y McCormick. \textit{The European Union. Politics and policies}

Parlamento Europeo, \textit{Fichas técnicas PAC}. \url{http://www.europarl.europa.eu/atyourservice/es/displayFtu.html?ftuId=FTU_5.2.10.html}

Parlamento Europeo. \textit{Fichas técnicas PPC}. \url{http://www.europarl.europa.eu/atyourservice/es/displayFtu.html?ftuId=FTU_5.3.1.html}

Schultz, T. W. (1945) \textit{Agriculture in an Unstable Economy} Committee for Economic Development Research Study -- En carpeta del tema

Zobbe, H. (2001) \textit{The Economic and Historical Foundation of the Common Agricultural Policy in Europe} Fourth European Historical Economics Society Conference -- En carpeta del tema



\end{document}
