\documentclass{nuevotema}

\tema{3A-11}
\titulo{Análisis teórico de la empresa en el marco de la teoría de la organización industrial.}

\begin{document}

\ideaclave


\seccion{Preguntas clave}

\begin{itemize}
	\item ¿Por qué existen las empresas?
	\item ¿Por qué la producción tiende a organizarse en empresas?
	\item ¿Qué teorías explican la existencia de empresas?
	\item ¿Qué factores limitan el tamaño de las empresas?
	\item ¿Qué es la teoría de la organización industrial?
	\item ¿Por qué los mercados adoptan determinadas estructuras?
	\item ¿Qué estructura de empresas deben adoptar los mercados?
	\item ¿Qué teorías analizan la estructura de las empresas de un mercado?
\end{itemize}

\esquemacorto

\begin{esquema}[enumerate]
	\1[] \marcar{Introducción}
		\2 Contextualización
			\3 Economía y microeconomía
			\3 Empresas
			\3 Teoría de la organización industrial
		\2 Objeto
			\3 ¿Por qué existen las empresas?
			\3 ¿Por qué la producción tiende a organizarse en empresas?
			\3 ¿Qué teorías explican la existencia de empresas?
			\3 ¿Qué factores limitan el tamaño de las empresas?
			\3 ¿Qué es la teoría de la organización industrial?
			\3 ¿Por qué los mercados adoptan determinadas estructuras?
			\3 ¿Qué estructura de empresas deben adoptar los mercados?
			\3 ¿Qué teorías analizan la estructura de las empresas de un mercado?
		\2 Estructura
			\3 Teoría de la empresa
			\3 Teoría de la organización industrial
	\1 \marcar{Teoría de la empresa}
		\2 Idea clave
			\3 Contexto
			\3 Objetivos
			\3 Resultados
		\2 Predecesores
			\3 Adam Smith
			\3 Karl Marx
			\3 J. S. Mill
			\3 John Bates Clark
			\3 Alfred Marshall
			\3 Frank Knight
			\3 Schumpeter
		\2 La existencia de la empresa
			\3 Idea clave
			\3 Enfoque neoclásico/tecnológico
			\3 Ronald Coase: costes de transacción
			\3 Alchian y Demsetz (1972)
			\3 Teoría de principal-agente
			\3 Williamson: costes de transacción
			\3 Enfoque de los derechos de propiedad
			\3 Maximización de las ventas -- Baumol
			\3 Maximización de la utilidad gerencial -- Williamson
			\3 Maximización del crecimiento -- Marris
			\3 Ineficiencia X de Leibenstein
			\3 Cyert y March (1963): teoría conductual
			\3 La empresa multinacional
	\1 \marcar{Teoría de la organización industrial}
		\2 Idea clave
			\3 Contexto
			\3 Objetivo
			\3 Resultados
		\2 Escuela austríaca
			\3 Idea clave
			\3 Eficiencia dinámica
			\3 Destrucción creativa
			\3 Monopolio
		\2 Escuela de Harvard -- Estructura-conducta-resultado
			\3 Idea clave
			\3 Formulación
			\3 Implicaciones
			\3 Valoración
		\2 Escuela austríaca
			\3 Neo-austriaca
			\3 Schumpeter
		\2 Escuela de Chicago
			\3 Idea clave
			\3 Formulación
			\3 Implicaciones
			\3 Valoración
		\2 Contestabilidad
			\3 Idea clave
			\3 Formulación
			\3 Implicaciones
			\3 Valoración
		\2 Teoría de juegos
			\3 Idea clave
			\3 Formulación
			\3 Implicaciones
			\3 Valoración
		\2 Nueva Organización Industrial Empírica
			\3 Idea clave
			\3 Formulación
			\3 Implicaciones
			\3 Valoración
		\2 Escuela de Toulouse
			\3 Idea clave
			\3 Formulación
			\3 Implicaciones
			\3 Valoración
	\1[] \marcar{Conclusión}
		\2 Recapitulación
			\3 Teoría de la empresa
			\3 Teoría de la organización industrial
		\2 Idea final
			\3 Análisis de la competencia
			\3 Regulación y privatización
			\3 Empresas públicas

\end{esquema}

\esquemalargo



\begin{esquemal}
	\1[] \marcar{Introducción}
		\2 Contextualización
			\3 Economía y microeconomía
				\4 Definición de Robbins
				\4[] Decisiones respecto a bienes escasos
				\4[] $\to$ Con usos alternativos
				\4[] $\to$ Para satisfacer necesidades humanas
				\4 Microeconomía
				\4[] Estudio de decisiones a nivel individual
				\4[] $\to$ Empresas
				\4[] $\to$ Consumidores
				\4[] $\to$ Gobiernos
			\3 Empresas
				\4 Papel en la sociedad
				\4[] Cita de Churchill
				\4[] $\to$ ``caballo que mueve la sociedad''
				\4[] Elemento fundamental de creación de VAñadido
				\4 Análisis microeconómico de la empresa
				\4[] Empresa como caja negra
				\4[] $\to$ Proceso maximizador del beneficio
				\4[] $\then$ No analiza cómo ni por qué
				\4[] Basada en concepción de Robbins de ciencia económica
				\4[] $\to$ Gestión de recursos escasos....
				\4 Existencia de empresas es hecho indiscutible
				\4[] Estatutos legales
				\4[] Relaciones jerárquicas y de propiedad
				\4[] Fenómeno tecnológico y organizativo
				\4[] Necesario explicar:
				\4[] $\to$ Por qué existen
				\4[] $\to$ En qué medida existen
				\4[] $\then$ Teoría de la firma
			\3 Teoría de la organización industrial
				\4 Construyendo sobre teoría de la firma
				\4 Empresas actúan en contexto inst. y económico
				\4[] Marco legal
				\4[] Barreras de entrada
				\4[] Demanda
				\4[] Información disponible
				\4 Contexto determina estructura de mercado
				\4[] Número de empresas
				\4[] Distribución del tamaño
				\4[] $\then$ Poder de mercado
				\4[] $\then$ Precios y costes
				\4[] $\then$ Producción
				\4[] $\then$ Beneficios empresariales
				\4[] $\then$ Bienestar social
				\4 Diferentes análisis de la estructura de un mercado
		\2 Objeto
			\3 ¿Por qué existen las empresas?
			\3 ¿Por qué la producción tiende a organizarse en empresas?
			\3 ¿Qué teorías explican la existencia de empresas?
			\3 ¿Qué factores limitan el tamaño de las empresas?
			\3 ¿Qué es la teoría de la organización industrial?
			\3 ¿Por qué los mercados adoptan determinadas estructuras?
			\3 ¿Qué estructura de empresas deben adoptar los mercados?
			\3 ¿Qué teorías analizan la estructura de las empresas de un mercado?
		\2 Estructura
			\3 Teoría de la empresa
			\3 Teoría de la organización industrial
	\1 \marcar{Teoría de la empresa}
		\2 Idea clave
			\3 Contexto
				\4 Empresas son actor elemental en economías avanzadas
				\4[] Mayor parte de creación de valor añadido
				\4 Conjuntos de:
				\4[] $\to$ Contratos
				\4[] $\to$ Propiedad de activos tangibles e intangibles
				\4[] $\to$ Relaciones laborales
				\4[] $\to$ Administración
				\4[] $\to$ Propietarios
				\4 Existencia generalizada
				\4[] No hay alternativa en economías avanzadas
				\4 Pero manifestaciones muy heterogéneas
				\4[] Integradas vertical y horizontalmente
				\4[] Dependientes de conjuntos de proveedores
				\4[] Múltiples grados de externalización
				\4[] Diferentes estructuras organizativas
			\3 Objetivos
				\4 Explicar existencia de empresas
				\4 Valorar beneficio social de existencia de empresas
				\4 Explicar límites al crecimiento de las empresas
			\3 Resultados
				\4 Múltiples teorías y enfoques
				\4 Énfasis sobre diferentes fenómenos
				\4[] Incertidumbre
				\4[] Costes de transacción
				\4[] Derechos de propiedad
				\4[] Problema de agencia
				\4[] ...
				\4 Interés renovado tras 90s
				\4[] Premio Nobel de 1991 a Coase
				\4 Relación con organización industrial
		\2 Predecesores
			\3 Adam Smith
				\4 Cree que empresario individual es preferible
				\4 Escéptico respecto a empresas
				\4[] Separación de gestión y propiedad
				\4[] $\to$ Gestión ineficiente
				\4 Contexto del s. XVIII
				\4[] Número reducido de ``compañías''
				\4[] Grandes distancias entre
				\4[] $\to$ propietarios
				\4[] $\to$ administradores
				\4[] Comunicaciones poco desarrolladas
				\4[] $\then$ Corrupción
				\4[] $\then$ Agendas propias de administradores locales
			\3 Karl Marx
				\4 Empresas son instrumento de alienación
				\4[] Alienación como pérdida de capacidad de trabajadores
				\4[] $\to$ Para saberse dueños de sus propios actos
				\4[] $\to$ Para tomar conciencia de mercancía producida
				\4 División del trabajo en seno de empresas
				\4[] Permite alienar trabajador del producto
				\4 Precede análisis principal--agente en seno de empresa
			\3 J. S. Mill
				\4 Énfasis en forma legal y gobernanza
				\4 Problema de agencia justifica empresa
				\4[] Propietarios deben sufrir riesgo empresarial
				\4[] $\to$ Para que tengan interés en buen funcionamiento
				\4 Empresas como instrumento para acumular capital
				\4[] Sociedades de responsabilidad limitada
				\4[] $\to$ Instrumento legal
				\4[] $\to$ Exigencia de transparencia sobre activos y deuda
				\4[] $\then$ Suficiente garantía para acreedores
				\4[] $\then$ Permite capitalistas provean capital
				\4 Crecimiento y progreso promueve aparición de empresas
				\4[] Aumento de la riqueza y el capital
				\4[] Mejora de la capacidad de gestión
				\4[] $\then$ Aparición de nuevas empresas
			\3 John Bates Clark
				\4 Empresa sirve para coordinar producción
				\4 Sin empresa, inputs desorganizados
				\4 Empresario remunerado por trabajo
				\4[] Papel de coordinador
			\3 Alfred Marshall
				\4 Empresas aparecen:
				\4[] Para realizar ecs. de escala
				\4[] Por la ambición competitiva de los managers
				\4 Análisis dinámico y narrativo de la empresa
				\4[] Empresas como árboles en el bosque
				\4[] $\to$ Nacen, crecen y mueren
				\4[] $\to$ En agregado, el bosque se mantiene estable
				\4 Análisis matemático de la empresa
				\4[] Inicia análisis formal
				\4[] $\to$ Maximización del beneficio
				\4[] $\to$ Equilibrio estático
				\4[] $\to$ Análisis de c/p, m/p, l/p variando input fijo
				\4 Empresario como factor de producción
				\4[] Recibe remuneración como otros ff.pp.
			\3 Frank Knight
				\4 Precedido por Von Thünen
				\4[] Beneficio empresarial es compensación
				\4[] $\to$ Por riesgo no asegurable
				\4 Análisis del riesgo e incertidumbre
				\4[] Contribución destacada del autor
				\4 Diferentes aversiones al riesgo
				\4[] Heterogeneidad entre individuos
				\4 Trabajadores más aversos que empresarios
				\4[] Empresa permite redistribuir riesgo
				\4[] $\to$ Trabajadores aseguran sus rentas
				\4[] $\to$ Empresarios asumen riesgo a cambio de bfcio.
			\3 Schumpeter
				\4 Empresa como motor de innovación
				\4 Enfoque tecnológico
				\4[] Toma como dado lo que es y lo que no es posible
				\4 Innovación es de hecho fuerza relevante
				\4[] Lo que es posible no está dado
		\2 La existencia de la empresa
			\3 Idea clave
				\4 Sistema de precios
				\4[] Instrumento óptimo de:
				\4[] $\to$ Transmisión de información
				\4[] $\to$ Intercambio de bienes y servicios
				\4 Sin embargo, aparición de empresas
				\4[] Organizaciones en las que internamente
				\4[] $\to$ Decisiones basadas en autoridad
				\4[] $\to$ Relaciones de l/p con contratos poco explícitos
				\4 Aparente contradicción
				\4[] Si mercado alcanza en general asignaciones óptimas
				\4[] $\to$ Empresas no deberían existir
				\4[] $\to$ Todas las decisiones en contexto de mercado
			\3 Enfoque neoclásico/tecnológico
				\4 Robbins
				\4[] Economía como ciencia de la decisión
				\4[] $\to$ Asignar fines a recursos escasos
				\4 Concepción neoclásica de la empresa
				\4[] Caja negra de transformación de inputs
				\4[] $\to$ Elegir combinaciones de inputs
				\4[] $\then$ Que maximizan beneficio
				\4[] Elegir plan de producción
				\4[] $\to$ Minimiza costes
				\4[] $\to$ Dado precio de output
				\4[] $\then$ Maximiza beneficio
				\4 Formulación
				\4[] $\underset{\vec{x} \in X}{\max} \quad \vec{p}\cdot \vec{x}$
				\4 Valoración
				\4[] Formulación muy general y tratable
				\4[] Muy útil para valorar respuesta a cambios exógenos
				\4[] No explica:
				\4[] $\to$ organización interna de la empresa
				\4[] $\to$ Existencia de la empresa sólo parcialmente
			\3 Ronald Coase: costes de transacción
				\4 Coase (1937)
				\4[] ``La naturaleza de la firma''
				\4 Empresas son alternativa a mecanismo de precios
				\4[] Característica definitoria
				\4[] $\to$ Sustitución de precios por autoridad
				\4 Plantea pregunta seminal
				\4[] Si sistema de precios permite coordinación
				\4[] $\to$ ¿Por qué es necesario organizar la producción?
				\4[] $\to$ ¿Por qué aparecen empresas basadas en autoridad?
				\4[] Ejemplo: distribución de espacio en un supermercado
				\4[] -- Distribución vía autoridad
				\4[] $\to$ Gerente decide donde poner qué
				\4[] -- Mercado
				\4[] $\to$ Gerente subasta espacios entre proveedores
				\4 Organización vía precios implica costes:
				\4[] Descubrir cuáles son los precios relevantes
				\4[] $\to$ Puede mitigarse
				\4[] $\to$ No puede eliminarse del todo
				\4[] Coste de contratación
				\4[] $\to$ Redactar y acordar contrato
				\4[] Contratos de largo plazo
				\4[] $\to$ Incertidumbre sobre condiciones futuras
				\4 Organización vía empresa permite
				\4[] Evitar descubrimiento constante de precios
				\4[] Sustituir contratos múltiples c/p por contrato l/p
				\4[] Simplificar definición de contratos internos a empresa
				\4[] $\to$ Sólo límites de lo que se espera de trabajadores
				\4[] $\to$ Detalles sin especificar
				\4 Trade-off del crecimiento de la empresa
				\4[] Producción dentro de la empresa
				\4[] $\to$ Aumenta costes burocráticos y de gestión
				\4[] $\to$ Reduce costes de transacción
				\4[] Aumento del tamaño de la empresa
				\4[] $\to$ Hasta CTransacción > CBurocráticos
				\4 ¿Por qué todo no se produce por una gran empresa?
				\4[] Rendimientos decrecientes de la función de gestión
				\4[] Coste de organizar dentro de la empresa
				\4[] $\to$ Aumenta con número de transacciones organizadas
			\3 Alchian y Demsetz (1972)
				\4 Crítica de Coase (1937)
				\4[] Trabajadores no tienen por qué cumplir órdenes
				\4[] $\to$ Más allá de lo que cumpliría un contratista externo\footnote{``Un trabajador no tiene obligación de cumplir órdenes más allá de la obligación que tiene un frutero de vender las frutas que le orden el cliente.''}
				\4[] Represalias disponibles para propietarios
				\4[] $\to$ No difieren mucho de disponibles para externos
				\4 Empresas son procesos de producción conjunta
				\4[] Cada miembro contribuye su parte
				\4[] Actuación conjunta vale más que por separado
				\4 Necesario supervisar contribución
				\4[] Para que se lleve a cabo correctamente
				\4[] $\to$ Necesarios incentivos para supervisar
				\4 Empresa existe porque coordina proceso productivo
				\4[] Es elemento central de conjunto de contratos
				\4[] $\to$ Coordina conjunto de inputs
				\4[] $\to$ Permite utilización eficiente
				\4[] $\then$ Empresa como ``cocinero'' que ejecuta receta
				\4[] $\then$ Inputs como ingredientes de la receta
				\4[$\then$] Empresario es input productivo
				\4[] Gestión y supervisión son input necesario
				\4[] Permite transformar conjunto de otros inputs
				\4[] $\to$ En output que otros agentes demandan
				\4[] Supervisa rendimiento de trabajadores
				\4 Incentivos necesarios para supervisión
				\4[] Coinciden con derechos de propietarios de empresas
				\4[] i) Derecho a beneficio residual
				\4[] ii) Observar comportamiento de los ff.pp.
				\4[] iii) Ser contrapartida común a contratos con ff.pp.
				\4[] iv) Alterar composición del ``equipo''
				\4[] v) Vender conjuntamente los derechos anteriores a terceros
				\4[$\then$] Empresa existe para coordinar inputs
				\4[$\then$] Empresa existe para incentivar supervisión
			\3 Teoría de principal-agente
				\4 Idea clave
				\4[] Managers como agentes
				\4[] $\to$ Conocen funcionamiento interno de la empresa
				\4[] $\to$ Toman decisiones
				\4[] $\then$ Objetivos pueden desviarse de los de accionistas
				\4[] Accionistas como principales
				\4[] $\to$ No conocen gestión diaria
				\4[] $\then$ Deben formular esquema de incentivos
				\4 Separación de gestión y propiedad
				\4[] Habitual en grandes empresas
				\4[] Necesario alinear objetivos de directivos y accionistas
				\4 Formulación basada en enfoque neoclásico
				\4[] $\to$ Empresa como conjunto de planes de producción
				\4 Teoría de la gestión de la empresa
				\4[] Teoría de principal y agente es punto de partida
			\3 Williamson: costes de transacción
				\4 Inicia análisis formal de costes de transacción
				\4[] Explicitación de causas
				\4 Empresa como estructura de gobernanza
				\4[] Que trata de minimizar costes de transacción
				\4 Dos causas de costes de transacción
				\4[] Racionalidad limitada
				\4[] $\to$ Agentes económicos satisfacen utilidad
				\4[] $\then$ No la maximizan
				\4[] Oportunismo
				\4[] $\to$ Agentes actúan de modo no cooperativo
				\4[$\then$] Costes de transacción para mitigar anteriores
				\4[] $\to$ Contratos explícitos
				\4[] $\to$ Descubrimiento de precios
				\4[] $\to$ Revelación de información
				\4 Tres factores favorecen actividad dentro de empresa
				\4[] i) Especificidad de los activos
				\4[] $\to$ Coste económico de cortar relación duradera
				\4[] $\to$ Más especificidad aumenta poder de negociación
				\4[] $\then$ Especificidad genera cuasirrentas
				\4[] $\then$ Empresa puede verse sometida a ``chantaje''
				\4[] ii) Incertidumbre
				\4[] $\to$ Aumenta incompletitud de contratos
				\4[] iii) Frecuencia elevada de la transacción
				\4[] $\to$ Multiplica problemas anteriores
				\4 Relación jerárquica para organizar producción
				\4[] Reduce fuertemente constes de transacción
				\4 Sin especificidad
				\4[] Posible contratar en el mercado
				\4[] $\to$ Ahorrar costes de supervisión
				\4 Sin incertidumbre
				\4[] Contratos casi completos son viables
				\4 Si carácter ocasional
				\4[] Costes fijos de gestión no compensan
				\4 Representación gráfica
				\4[] \grafica{williamsontresfactores}
			\3 Enfoque de los derechos de propiedad\footnote{Ver Ricketts (2002), págs. 118 y ss.}
				\4 Grossman, Hart, Holstrom y Moore
				\4 Empresa es un conjunto de activos
				\4[] Sujetos a propiedad común
				\4 Contratos incompletos son causa de empresa
				\4[] Propiedad de activos físicos
				\4[] $\to$ Derecho residual de control
				\4[] $\then$ Mayor flexibilidad de actuación
				\4 Posibilidad de negociación dada incompletitud
				\4[] Posterior a acuerdo contractual
				\4[] Introduce costes habituales
				\4 Derechos de propiedad sobre activos fijos no-humanos
				\4[] Permite mitigar costes
				\4[] $\to$ De negociación ex-post
				\4[] $\to$ De cautividad por especificidad
				\4 Sensibilidad de inversión a derechos de propiedad
				\4[] Determina quién posee activos fijos\footnote{\textit{``The greater the sensitivity of a contractor’s ex ante investments to control of a physical asset, the more powerful the case for the assignment of ownership to that contractor.''}.}
				\4[] Cuando inversiones por un agente dependen mucho
				\4[] $\to$ De la propiedad de un activo fijo
				\4[] $\then$ Activo fijo estará en manos de ese agente
				\4 Complementariedad de activos fijos (idea equivalente)
				\4[] Cuanto mayor sea la complementariedad
				\4[] $\to$ Más deseable que propiedad sea común
				\4 Límites de la firma
				\4[] Dentro de la firma
				\4[] $\to$ Cuando hay complementariedad
				\4[] $\to$ Cuando propiedad común reduce cautividad
				\4[] $\to$ Cuando hay incentivos a inversión complementaria
				\4[] Fuera de la firma
				\4[] $\to$ Cuando complementariedad apenas aporta nada
				\4 Diferencias con anteriores
				\4[] Coase y costes de transacción
				\4[] $\to$ Firma para reducir costes de transacción
				\4[] Williamson
				\4[] $\to$ Empresa como estructura de gobernanza
				\4[] $\to$ Permite superar incertidumbre de contratos incompletos
				\4[] $\to$ Permite reducir costes de transacción
				\4[] Derechos de propiedad
				\4[] $\to$ Empresa para optimizar inversión
				\4 Ejemplos:
				\4[] Chef de cocina y diseñador de interiores
				\4[] $\to$ ¿Quién debe poseer el local del restaurante?
				\4[] Inversión de diseñador de interiores
				\4[] $\to$ No requiere local del restaurante
				\4[] $\then$ Local no es complementario con KHumano de diseñador
				\4[] Inversión de chef de cocina
				\4[] $\to$ Requiere local de restaurante
				\4[] $\to$ Requiere inversión en cocina y similar
				\4[] $\then$ Local es complementario con KHumano de chef
				\4[] Si chef posee local:
				\4[] $\to$ Puede buscar otro diseñador fácilmente
				\4[] $\to$ Diseñador puede buscar otro cliente fácilmente
				\4[] $\then$ Realizará inversiones óptimas
				\4[] Si diseñador posee local:
				\4[] $\to$ Tendrá que contratar a chef
				\4[] $\to$ Chef muy específico a restaurante
				\4[] $\then$ Diseñador cautivo de KHumano de chef
				\4[] $\then$ No optimizará inversión en restaurante
				\4[] Derechos de propiedad que optimizan inversión
				\4[] $\to$ Chef propietario del local
				\4[] $\then$ Derechos de propiedad determinan qué es la empresa
			\3 Maximización de las ventas -- Baumol
				\4 Baumol (1962)
				\4 Directivos maximizan ventas
				\4[] Sujetos a restricción de beneficios
				\4 Tres razones para maximizar ventas
				\4[i] Stakeholders consideran buen indicador
				\4[] Accionistas, prestamistas, trabajadores
				\4[] $\to$ Toman ventas como indicador de salud de empresa
				\4[ii] Muchas variables está ligadas a evolución de ventas
				\4[] Sueldos, imagen, influencia, análisis estratégico
				\4[] $\to$ Ligados a ventas crecientes
				\4 En contexto de empresa monopolista
				\4[] Empresa no produce hasta que IMg = CMg
				\4[] $\to$ Produce hasta que IMg=0
			\3 Maximización de la utilidad gerencial -- Williamson
				\4 Los directivos maximizan su propia f. de u.
				\4 Elementos de decisión de los directivos
				\4[] Gastos en personal
				\4[] $\to$ Determinan poder personal
				\4[] Emolumentos de los gerentes
				\4[] $\to$ Determinan poder económico
				\4[] Inversión discrecional
				\4[] $\to$ Señalizan poder y prestigio
			\3 Maximización del crecimiento -- Marris
				\4 Evaluación de gestión individual de directivos
				\4[] Difícil de cuantificar
				\4[] Basada en indicadores imperfectos
				\4 Tamaño de departamentos como indicador
				\4[] Directivos incentivos a aumentar:
				\4[] $\to$ Tamaño de su departamento
				\4[] $\to$ Actividades que llevan a cabo
			\3 Ineficiencia X de Leibenstein
				\4 Empresas se desvían de comportamiento optimizador
				\4[] No siempre minimizan costes de producción
				\4 Inexistencia de mercados dentro de empresas
				\4[] Reduce presión competitiva
				\4 Individuos no maximizan beneficio
				\4[] Más bien, satisfacen preferencias
				\4[] $\to$ En la medida en que resulta rentable
				\4 Contratos son incompletos respecto a esfuerzo
				\4[] Empleados acomodan nivel de esfuerzo
				\4 Competencia imperfecta
				\4[] La presión competitiva es menor a CPerfecta
				\4 Minimización de coste como excepción
				\4[] Nunca es completa en la práctica
				\4[] Minimización como óptimo realmente inalcanzable
				\4[] $\to$ Necesario altísimo grado de presión externa
			\3 Cyert y March (1963): teoría conductual
				\4 Idea clave
				\4[] Decisiones en empresa
				\4[] $\to$ No son resultado de optimización racional
				\4[] $\then$ Reglas heurísticas son factor determinante
				\4 Empresa como conjunto de coaliciones
				\4[] Empresas pequeñas
				\4[] $\to$ operan bajo guía del empresario
				\4[] Empresas grandes
				\4[] $\to$ Coalición de intereses de diferentes grupos
				\4[] $\then$ Directivos
				\4[] $\then$ Accionistas
				\4[] $\then$ Trabajadores
				\4[] $\then$ Proveedores
				\4[] Necesario compatibilizar objetivos diversos
				\4[] $\to$ Producción
				\4[] $\to$ Inventarios
				\4[] $\to$ Cuota de mercado
				\4[] $\to$ Ventas
				\4[] $\to$ Beneficios
				\4 Comportamiento no optimizador
				\4[] Generalmente, sólo busca satisfacer preferencias
				\4[] Comportamiento optimizador implica costes
				\4[] $\to$ Que pueden no ser asumibles
				\4 Proceso de toma de decisiones
				\4[] Examen de etapas del proceso de decisión
				\4[] Primer nivel
				\4[] $\to$ Asignación de recursos financieros
				\4[] Segundo nivel
				\4[] $\to$ Combinación de factores de producción
				\4[] Recabar información implica costes
				\4[] $\to$ Distorsiones en proceso de toma de decisiones
				\4 Holgura organizativa
				\4[] Para:
				\4[] $\to$ Mantener a grupos de interés dentro de empresa
				\4[] $\to$ Amortiguar fluctuaciones
				\4[] $\then$ Necesario pagar más de lo estrict. necesario
			\3 La empresa multinacional
				\4 Evidencia empírica robusta
				\4[] Empresas exportadoras son más grandes
				\4 OLI -- Ownership, Location, Internationalization
				\4 Melitz (2003)
				\4[] Empresas heterogéneas
				\4[] Diferentes productividades
				\4[] $\to$ Diferentes costes marginales
				\4[] Producción para mercado nacional
				\4[] $\to$ Un coste fijo
				\4[] Producción para mercado internacional
				\4[] Coste fijo añadido
				\4[] Empresas capaces de superar costes fijos
				\4[] $\to$ Se mantienen en el mercado
				\4[] Apertura de comercio internacional
				\4[] $\to$ Aumenta competencia en mercado doméstico
				\4[] $\then$ Menores mark-ups
				\4[] $\then$ Menores beneficios operativos
				\4[] $\then$ Empresas menos productivas salen del mercado
				\4[] $\to$ Aparece posibilidad de exportar
				\4[] $\then$ Empresas más productivas rentabilizan exportación
				\4[] $\then$ Empresas más rentables tienden a crecer
				\4[$\then$] Aumento de tamaño y productividad con apertura comercial
				\4 Helpman, Melitz y Yeaple (2004)
				\4[] Crecimiento multinacional de empresas enfrenta trade-off
				\4[] $\to$ Reduce costes de transporte y aranceles
				\4[] $\then$ Más beneficio
				\4[] $\to$ Reduce posibilidad de realizar economías de escala
				\4[] $\then$ Aumenta beneficio
				\4[] Empresas más productivas capaces de replicar a menor coste
				\4[] $\to$ Pueden compensar coste de transporte con más plantas
	\1 \marcar{Teoría de la organización industrial}
		\2 Idea clave
			\3 Contexto
				\4 Empresas interaccionan con otros agentes económicos
				\4[] Consumidores
				\4[] Empresas
				\4[] Proveedores
				\4 Estructuras de mercado de competencia imperfecta
				\4[] Causa de:
				\4[] $\to$ Beneficio extraordinario
				\4[] $\to$ Incentivos a entrada en mercado
				\4[] $\to$ Equilibrios pareto subóptimos
			\3 Objetivo
				\4 Explicar y predecir interacción de empresas
				\4 Valorar optimalidad de estructuras de mercado
			\3 Resultados
				\4 Evolución de prescripciones normativas
				\4 Cambios en valoración de monopolio
				\4[] Diferentes prescripciones normativas
				\4 Reglas de intervención en mercados
				\4 Medidas cuantitativas de concentración
				\4 Metodología de análisis empírico
		\2 Escuela austríaca
			\3 Idea clave
			\3 Eficiencia dinámica
			\3 Destrucción creativa
			\3 Monopolio
		\2 Escuela de Harvard -- Estructura-conducta-resultado
			\3 Idea clave
				\4 Contexto
				\4[] Años 30 y 40
				\4[] Desarrollo de la econometría
				\4[] Oligopolios y monopolios generalizados
				\4[] Ley Sherman de 1890
				\4[] Principales impulsores
				\4[] $\to$ Clark Bates, Clark (1940), Mason (1949)
				\4[] $\to$ Bain en años 50
				\4[] Asociada a universidad de Harvard
				\4 Objetivo
				\4[] Explicar comportamiento de empresas
				\4[] Predecir evolución de mercados
				\4[] Fundamentar intervención pública pro-competitiva
				\4 Resultados
				\4[] Paradigma de pensamiento dominante
				\4[] $\to$ Hasta años 70
				\4[] Aún relevante en la actualidad
				\4[] Secuencia causal predominante
				\4[] i) Estructura del mercado
				\4[] $\to$ Número y distribución de tamaño de vendedores
				\4[] $\to$ Número y distribución de tamaño de compradores
				\4[] $\to$ Diferenciación del producto
				\4[] $\to$ Condiciones de entrada
				\4[] ii) Conducta de las empresas
				\4[] $\to$ Fijación de precios
				\4[] $\to$ Determinación de cantidades a producir
				\4[] iii) Resultados
				\4[] $\to$ Beneficios
				\4[] $\to$ Cantidades vendidas
				\4[] $\to$ Precios que prevalecen
				\4[] $\to$ Entrada y salida de empresas
			\3 Formulación
				\4 Estructura
				\4[] Competencia perfecta asume
				\4[] $\to$ Muchos vendedores y compradores
				\4[] $\to$ Producto homogéneo
				\4[] Realmente, múltiples desviaciones posibles
				\4[] $\to$ Definen estructura de un mercado
				\4[] -- Número y distribución de vendedores
				\4[] $\to$ En CPerfecta, precio iguala coste de oportunidad
				\4[] $\to$ En CImperfecta, posibles restricciones de oferta
				\4[] $\then$ Precio por encima de coste de oportunidad
				\4[] $\then$ Consumidores dispuestos a pagar COport. no pueden
				\4[] $\then$ Cuantas menos empresas, más cerca de monopolio
				\4[] $\then$ Sujeto a excepciones: competencia à la Bertrand
				\4[] -- Número y distribución de compradores
				\4[] $\to$ Efecto simétrico al de número y dist. de vendedores
				\4[] $\to$ Countervailing power\footnote{Fenómeno que hace referencia al aumento de la concentración en un lado del mercado como reacción a la concentración en el otro --y la subsecuente aparición de poder de mercado.}
				\4[] -- Diferenciación del producto
				\4[] $\to$ Productos más diferenciados son peores sustitutos
				\4[] $\then$ Cada empresa se acerca más a un monopolio
				\4[] $\then$ Acerca estructura a monopolio
				\4 Conducta
				\4[] Competencia perfecta
				\4[] $\to$ Vender/comprar a precio exógeno
				\4[] $\then$ Hasta optimizar preferencias/beneficios
				\4[] Competencia imperfecta
				\4[] $\to$ Múltiples conductas posibles
				\4[] $\to$ Amplio margen para determinar estructura
				\4[] -- Colusión
				\4[] $\to$ Coordinación de precios/cantidades
				\4[] $\then$ Elevar precios por encima de coste marginal
				\4[] $\then$ Obtener beneficios extraordinarios
				\4[] $\to$ Entrada de nuevas empresas dificulta colusión
				\4[] $\then$ Problemas de coordinación
				\4[] $\then$ Incentivos a romper acuerdo colusivo
				\4[] -- Conducta estratégica
				\4[] $\to$ Tratar de desalentar entrada de nuevas empresas
				\4[] $\then$ Manteniendo precios bajos
				\4[] $\then$ Incurriendo en costes hundidos
				\4[] $\to$ Integración horizontal y vertical
				\4[] -- Publicidad y actividades de I+D
				\4[] $\to$ Señalizar diferenciación de producto
				\4[] $\to$ Dificultar entrada a nuevos competidores
				\4[] $\then$ Evitando se conozca su producto
				\4 Resultado
				\4[] En competencia perfecta a l/p
				\4[] $\to$ Oferta iguala demanda a coste marginal
				\4[] $\to$ Empresas salen/entran hasta $\Pi = 0$
				\4[] $\then$ Producción a escala eficiente
				\4[] $\then$ Empresas ineficientes desaparecen
				\4[] $\then$ Eficientes entran y se mantienen
				\4[] En competencia imperfecta
				\4[] $\to$ Diferentes patrones de entrada/salida
				\4[] $\to$ Diferentes tasas de beneficio
				\4[] $\to$ Incentivos variables a progreso técnico
				\4[] -- Rentabilidad
				\4[] $\to$ Beneficio económico es posible y aparece
				\4[] $\to$ Generalmente, más concentración aumenta bfcio.
				\4[] -- Eficiencia
				\4[] $\to$ Mayores beneficios reducen eficiencia
				\4[] $\to$ Diferentes explicaciones
				\4[] $\then$ Ineficiencia X
				\4[] $\then$ Excedente del consumidor no extraído
				\4[] $\then$ Exceso de capacidad
				\4[] $\then$ ...
				\4[] $\then$ ``best of all monopoly profits is a quiet life''
				\4[] -- Progreso tecnológico
				\4[] $\to$ Sin beneficio potencial por entrar
				\4[] $\then$ No aparecen incentivos a entrada
				\4[] $\to$ Beneficios económicos en competencia imperfecta
				\4[] $\then$ Estimulan nuevas tecnologías y entrada
			\3 Implicaciones
				\4 Interacciones y causalidad
				\4[] Supuesto predominante sobre orden de causalidad
				\4[] Condiciones básicas $\then$ Estructura $\then$ Conducta $\then$ Resultado
				\4[] $\to$ Tendencia a considerar estructura como exógena
				\4 Análisis empírico
				\4[] Motivo principal de formulación del modelo
				\4[] $\to$ Marco de análisis de industrias reales
				\4[] Técnicas estadísticas varias
				\4[] $\to$ Estadísticas microeconómicas industriales
				\4[] $\to$ Medidas de concentración
				\4[] Uso incipiente decomputación
				\4 Delimitación de mercados
				\4[] Necesario definir claramente límites de mercados
				\4[] $\to$ ¿Metro y coche compiten en mismo mercado?
				\4[] $\to$ ¿Límites geográficos?
				\4 Medidas de concentración
				\4[] Caracterización cuantitativa de estructura
				\4[] $\to$ Cuántos competidores en oferta y demanda
				\4[] $\to$ Qué distribución de tamaño
				\4[] Ratio de concentración
				\4[] $\to$ Fracción de industria controlada por $n$ mayores
				\4[] $\then$ \fbox{$\text{CR}_x = \sum_{i=1}^X S_i$}
				\4[] Índice de Herfindahl-Hirschmann
				\4[] $\to$ Suma de cuadrados de cuota de mercado
				\4[] $\then$ \fbox{$\text{HH} = \sum_i (q_i)^2$}
				\4[] $\then$ Mayor índice cuanta mayor concentración
				\4[] $\then$ Comparación de concentraciones depende de industria
				\4[] Índice de Lerner
				\4[] $\to$ Margen sobre CMg como \% de precio
				\4 Política económica
				\4[] Regulación de conductas
				\4[] $\to$ Precios públicos y regulados
				\4[] Modulación de estructura
				\4[] $\to$ Prohibición de barreras anti-competitivas
				\4[] Incentivos a conductas
				\4[] $\to$ Impuestos y subsidios
				\4[] $\to$ Incentivos al empleo
				\4[] Políticas de defensa de la competencia
				\4[] $\to$ Abuso de posición dominante
				\4[] $\to$ Anti-cárteles
				\4[] $\to$ Competencia deseal...
				\4[] Políticas macroeconómicas
				\4[] $\to$ Estímulos de demanda
				\4[] $\to$ Fine-tuning
			\3 Valoración
				\4 Influencia persistente
				\4 Causalidad realmente mucho más compleja
				\4[] Resultados influyen:
				\4[] $\to$ Industria
				\4[] $\to$ Comportamiento
				\4[] Comportamiento influye
				\4[] $\to$ Estructura
				\4[] Tendencia a considerar también causalidad inversa
				\4 Énfasis sobre tamaño de la empresa
				\4[] Elemento central de valoración normativa
				\4[] $\to$ ``Sospecha'' persistente respecto empresas grandes
				\4 Críticas
				\4[] Ausencia de visión dinámica
				\4[] $\to$ Beneficios pueden ser estímulo a innovación
				\4[] Competencia se abre siempre camino
				\4[] $\to$ Stigler: competencia es ``mala hierba, no flor''
		\2 Escuela austríaca
			\3 Neo-austriaca
				\4 Rechazo a teoría de precios neoclásica
				\4 Economía es continuo proceso de descubrimiento
				\4 Mercados nunca están en equilibrio
				\4[] Estructura de empresas nunca en equilibrio
				\4 Sí existe orden
				\4[] Caracteriza evolución de la estructura industrial
				\4 Política económica
				\4[] Marco institucional debe ser objetivo del gobierno
				\4[] $\to$ Permitir innovación
				\4[] Eliminar monopolios no debe ser objetivo principal
				\4[] $\to$ En el largo plazo, monopolios desaparecen
			\3 Schumpeter
				\4 Innovación en el seno de la empresa
				\4[] Motor de cambio de estructura industrial
				\4 Destrucción creativa
				\4[] Proceso constante de transformación industrial
				\4[] Empresas viejas e ineficientes
				\4[] $\to$ Desaparecen
				\4[] Empresas que innovan y aumentan eficiencia
				\4[] $\to$ Reemplazan anteriores
				\4 Política económica
				\4[] Fomentar y permitir innovación es principal objetivo
				\4[] Énfasis en intervención es erróneo
				\4[] Innovación es causa de poder de mercado temporal
				\4[] $\to$ No debe tratar de frenarse
		\2 Escuela de Chicago
			\3 Idea clave
				\4 Contexto
				\4[] Competencia como fenómeno persistente
				\4[] $\to$ Stigler: ``competencia es mala hierba''
				\4[] Malos resultados empíricos de ECR en 70s
				\4[] Posner, Pork, Peltzman, Stigler...
				\4 Objetivo
				\4[] Reconsiderar competencia como fuerza determinante
				\4[] Aplicar teoría de precios a organización industrial
				\4[] Valorar propiedades de eficiencia
				\4[] $\to$ De resultados espontáneos de libre competencia
				\4 Resultado
				\4[] Análisis de causalidad no explicada en ECR
				\4[] Beneficio económico puede indicar buen comportamiento
				\4[] $\to$ Reducción de costes
				\4[] Concentración no es necesariamente indeseable
				\4[] $\to$ ECR no necesariamente lleva a óptimo
				\4[] Sin modelo explícito coherente
				\4[] Fuerte influencia sobre política econ. en 80s
			\3 Formulación
				\4 Competencia perfecta como punto de partida
				\4 Primitivas determinan resultado final
				\4[] Preferencias
				\4[] Tecnologías
				\4[] Dotaciones
				\4 Libertad de entrada es motor de dinámica
				\4[] Permite competencia
				\4 Diferencias de productividad entre empresas
				\4[] Inducen movilidad de factores productivos
				\4[] Tienden a ajustar mercado a óptimo paretiano
			\3 Implicaciones
				\4 Concentración no necesariamente indeseable
				\4[] Permite incentivar innovación
				\4 Política económica
				\4[] Intervención no siempre es deseable
				\4[] $\to$ Puede provocar distorsiones mayores
				\4[] $\to$ Fallos del sector público generalizados
				\4[] Mercados tienden a autorregularse
				\4[] Necesario énfasis en condiciones mínimas
				\4[] $\to$ Derechos de propiedad bien definidos
				\4[] $\to$ Seguridad jurídica
				\4[] $\to$ Certidumbre de políticas públicas
				\4 Subastas de monopolio
				\4[] Regulación ex-post puede sustituirse por subastas
				\4[] Monopolios subastados
				\4[] $\to$ Permiten extraer renta de monopolio
				\4[] Importante es mantener competencia en subasta
				\4[] $\to$ No necesariamente en mercado
			\3 Valoración
				\4 Influencia científica
				\4[] Predominante
				\4 Análisis matemático
				\4[] Predominante
				\4[] A nivel teórico y empírico
		\2 Contestabilidad
			\3 Idea clave
				\4 Contexto
				\4[] Baumol (1982)
				\4[] Baumol, Panzar y Willig (1982)
				\4[] Bailey y Baumol (1984)
				\4[] Regulación de monopolios naturales
				\4[] $\to$ Considerada casi axiomática
				\4[] Presión pro-desregulación
				\4[] $\to$ Friedman, Hayek
				\4[] $\to$ Escuela de Chicago en 70s
				\4 Objetivo
				\4[] Mostrar condiciones bajo las cuales
				\4[] $\to$ Monopolios naturales pueden alcanzar eq. óptimo
				\4[] $\to$ Ec. de escala no necesariamente anticompetitivas
				\4 Resultado
				\4[] Requisitos de contestabilidad
				\4[] No se prescribe libre mercado incondicionalmente
			\3 Formulación
				\4 Mercado perfectamente contestable
				\4[] No hay precio de equilibrio
				\4[] $\to$ Que permita entrante potencial rebajar precio
				\4[] $\then$ Y obtener un beneficio positivo
				\4 Requisitos
				\4[] Ausencia de barreras de entrada
				\4[] Sin costes hundidos
				\4[$\then$] No hay precios que induzcan beneficio persistente
				\4[] Siempre puede entrar un nuevo competidor y eliminarlos
			\3 Implicaciones
				\4 Mercados perfectamente contestables
				\4[] Benchmark de comparación
				\4[] $\to$ Para estructuras alejadas de competencia perfecta
				\4[] $\then$ Monopolios y oligopolios
				\4[] A diferencia de competencia perfecta
				\4[] $\to$ Sí pueden alcanzarse
				\4 Amenaza de hit-and-run induce $P=\text{CMe}$
				\4 Regulaciones y retrasos en entrada
				\4[] Permiten a incumbente mantener precios altos
			\3 Valoración
				\4 Fuerte influencia en teoría regulatoria
				\4 Competencia perfecta pierde peso como benchmark
				\4[] Carece de sentido en muchos mercados
				\4 Contestabilidad como objetivo más realista
		\2 Teoría de juegos
			\3 Idea clave
				\4 Desarrollo matemático a partir de 40s
				\4 Edgeworth, Von Neumann, Nash, Shapley
				\4 Análisis de interacción estratégica
				\4[] Decisión que considera reacción de terceros
				\4 Aplicación a organización industrial
				\4[] Empresas toman decisiones estratégicas
				\4[] $\to$ Considerando decisiones de otros
				\4[] $\to$ Postulando respuesta de otros agentes
			\3 Formulación
				\4 Descripción formal de juegos
				\4[] Aplicación a organización industrial
				\4[] $\to$ Empresas como jugadores
				\4[] $\to$ Estrategias posibles
				\4 Conceptos de solución
				\4[] Resultado que interacción de agentes determina
				\4[] $\to$ Estructura y resultado de mercados
			\3 Implicaciones
				\4 Reformulación de eqs. ya conocidos en estructuras habituales
				\4[] Cournot-Nash
				\4[] $\to$ Sin incentivos a cambiar cantidad
				\4[] Bertrand-Nash
				\4[] $\to$ Sin incentivos a cambiar precios
				\4[] ...
				\4 Análisis formal de estructuras de mercado
			\3 Valoración
				\4 Herramienta esencial de teoría de OI moderna
				\4 Caracterización de mercados concretos en términos matemáticos
				\4 Predicciones relativamente testables empíricamente
		\2 Nueva Organización Industrial Empírica
			\3 Idea clave
				\4 Métodos empíricos aplicados a ECR
				\4[] Generalmente, regresiones de:
				\4[] $\to$ Mark-up sobre coste marginal
				\4[] $\to$ Explicado por concentración de industria
				\4 Críticas a enfoque empírico de ECR
				\4[] Estimación de formas reducidas
				\4[] $\to$ No necesariamente muestran causalidad
				\4[] $\to$ Sensibles a cambios en primitivas
				\4[] Algunas firmas tienen ventajas de costes (Demsetz)
				\4[] $\to$ Aumentan beneficios y mark-up
				\4[] $\then$ Aunque industria sea competitiva
				\4 Nuevos desarrollos matemáticos
				\4[] Teoría de juegos
				\4[] Regresiones de panel
				\4 Autores
				\4[] Breshanan, Geroski, Lau, Porter...
				\4 Análisis económico de industrias idiosincrático
				\4[] Cada industria es diferente
				\4[] Detalles de cada industria son reelvantes
				\4[] Análisis de sección cruzada entre industrias
				\4[] $\to$ Omite información relevante
			\3 Formulación
				\4 Coste marginal no es observable
				\4[] Estimaciones de mark-up no son fiables
				\4 Industrias tienen características idiosincráticas
				\4[] Tecnológicas, informacionales, legales
				\4 Estimación de variación conjetural
				\4[] $\to$ Respuesta que las empresas estiman de otras
				\4 Subsumir mercado en modelo de tª de juegos
				\4[] Mercados aproximables generalmente a
				\4[] $\to$ Cournot
				\4[] $\to$ Stackelberg
				\4[] $\to$ ...
				\4 Estimación de comportamiento competitivo
				\4[] Comparación con valores de modelo teórico propuesto
			\3 Implicaciones
				\4 Regulación basada en costes marginales
				\4[] Fuertemente criticada
				\4 Énfasis en variables observables
				\4[] Precios y cantidades
				\4[] $\to$ Frente a estimaciones de costes
			\3 Valoración
				\4 Enfoque de análisis predominante en la actualidad
				\4[] Especialmente en contexto académico
		\2 Escuela de Toulouse
			\3 Idea clave
				\4 Jacques Laffont, Jean Tirole (Nobel 2014)
				\4 Énfasis en idiosincrasias de cada mercado
			\3 Formulación
				\4 Profundización en el análisis estratégico
				\4 Análisis de barreras de entrada estratégicas
				\4[] Incursión en costes hundidos
				\4[] $\to$ Reducción de costes marginales
				\4[] $\to$ Amenaza creíble de guerra de precios
				\4 Acuerdos de integración vertical
				\4[] Responden generalmente a motivos de eficiencia
				\4[] $\to$ A priori, no sancionables
				\4 Análisis de sustitutos y complementos estratégicos
				\4 Interacciones público privadas
				\4 Análisis de los efectos de la regulación
				\4[] Regulación sectorial de utilities y monop. naturales
				\4[] $\to$ Tasa de rendimiento
				\4[] $\to$ Condiciones de prestación de servicios
				\4[] Regulación de concentración
				\4[] $\to$ Integración vertical y horizontal
				\4[] $\then$ Efectos sobre eficiencia y poder de mercado
				\4[] Patentes y propiedad intelectual
				\4[] $\to$ Efectos de largo alcance en contexto dinámico
				\4 Problemas de los reguladores
				\4[] -- Selección adversa
				\4[] $\to$ Firmas reguladas tienen más y mejor información
				\4[] $\to$ Conocen tecnología y coste de inputs
				\4[] -- Riesgo moral
				\4[] $\to$ Regulados toman acciones ocultas
				\4[] Necesario considerar información en regulación
				\4 Análisis de plataformas
				\4[] Mercados con doble vertiente
				\4[] $\to$ Plataforma con compradores
				\4[] $\to$ Plataforma con vendedores
				\4[] $\then$ ¿Cómo intermediar óptimamente?
				\4[] $\then$ ¿Necesario regular?
			\3 Implicaciones
				\4 Evitar reglas rígidas anti-cartel
				\4[] Análisis específico a cada mercado
				\4[] Pragmatismo en regulación de sectores
				\4 Análisis teórico es punto de partida fundamental
			\3 Valoración
	\1[] \marcar{Conclusión}
		\2 Recapitulación
			\3 Teoría de la empresa
			\3 Teoría de la organización industrial
		\2 Idea final
			\3 Análisis de la competencia
			\3 Regulación y privatización
			\3 Empresas públicas
\end{esquemal}


































\graficas

\begin{axis}{4}{Representación gráfica de la teoría de Williamson sobre loslímites de la empresa. }{}{Costes}{williamsontresfactores}
	% extensión de ejes
	\draw[-] (4,0) -- (6,0);
	\node[below] at (6,0){Especificidad};
	
	% Coste dentro de la empresa
	\draw[-] (0,3.5) -- (5.5,0.5);
	\node[right] at (5.5,0.5){\tiny Empresa };
	
	% Coste en el mercado
	\draw[-] (0,0.5) -- (5.5,3.5);
	\node[right] at (5.5,3.5){\tiny Mercado};
	
	% Separación entre zonas óptimas
	\draw[dashed] (2.75,2) -- (2.75,0);
	% zona de contratación óptima en mercado
	\draw[-{Latex}] (2.5,-0.5) -- (0.5,-0.5);
	\node[below] at (1.5,-0.5){\tiny En mercado};
	% zona de integración óptima en la empresa
	\draw[-{Latex}] (3,-0.5) -- (5.5,-0.5);
	\node[below] at (4,-0.5){\tiny En empresa};
\end{axis}


\conceptos

\concepto{Indice de Hirschmann-Herfindahl}

\preguntas

\seccion{9 de marzo de 2017}

\begin{itemize}
	\item ¿Cual es realmente la distinción entre las teorías de Coase y la de los derechos de propiedad? 
	\item ¿Porque le han dado el premio Nobel a Hart? ¿Le suena el concepto de contratos incompletos?
	\item Defina riesgo moral y selección adversa
	\item ¿La saturación implica un aumento de la competencia siempre?
\end{itemize}

\seccion{Test 2016}
\textbf{5.} Si, de acuerdo con el modelo de Hart, todos los contratos que firman las empresas fueran completos

\begin{itemize}
	\item[a] Las empresas serían más rentables.
	\item[b] Aumentaría la dimensión de las empresas.
	\item[c] El derecho de quiebras sería innecesario.
	\item[d] No sería necesario que las empresas contaran con consejeros externos.
\end{itemize}

\seccion{Test 2013}
\textbf{12.} Señale la respuesta correcta en relación con el índice Hirschmann-Herfindahl (H):
\begin{itemize}
	\item[a] Solo tiene relación con el número de empresas que hay en un mercado.
	\item[b] Sirve para definir el mercado relevante.
	\item[c] Depende del número de empresas y del coeficiente de variación.
	\item[d] No sirve para definir el número equivalente de empresas de igual tamaño.
\end{itemize}


\notas

Leer \comillas{firm theory} del Palgrave extraído en la carpeta del tema.

\textbf{2016:} \textbf{5.} C

\textbf{2013:} \textbf{12.} C

\bibliografia

Mirar en Palgrave:
\begin{itemize}
	\item advertising
	\item anti-trust enforcement
	\item Averch-Johnson effect
	\item communications
	\item \textbf{competition}
	\item competition and selection
	\item \textbf{contracting in firms}
	\item corporations
	\item economic organization and transaction costs
	\item energy economics
	\item \textbf{entrepreneurship}
	\item \textbf{firm boundaries}
	\item \textbf{firm, theory of the}
	\item ideal output
	\item \textbf{industrial organization}
	\item industrial relations
	\item information sharing among firms
	\item international coordination of regulation
	\item labour-managed firms
	\item marginal and average cost pricing
	\item \textbf{market structure}
	\item monopoly
	\item natural monopoly
	\item oligopoly
	\item predatory pricing
	\item price discrimination (theory)
	\item profit and profit theory
	\item regulation
	\item regulation and deregulation
	\item trade and environmental regulations
\end{itemize}

Alchian, A. A.; Demsetz, H. (1972) \textit{Production, Information Costs, and Economic Organization} American Economic Review, Vol. 62. Issue 5. December 1972 -- En carpeta del tema.

Baumol, W. J. (1962) \textit{On the Theory of Expansion of the Firm} American Economic Review. Vol. 52, No. 5 -- En carpeta del tema

Bailey, E.; Baumol, W. J.(1984) \textit{Deregulation and the Theory of Contestable Markets} Yale Journal on Regulation. Vol 1. Issue 2, article 2 -- En carpeta del tema

Boudreaux, D. J.; Holcombe, R. G. (1989) \textit{The Coasian and Knightian Theories of the Firm} (1989) Managerial and Decision Economics -- En carpeta del tema

Coase, R. H. (1937) \textit{The Nature of the Firm} Economica. Vol 4. No. 16 -- En carpeta del tema

Coase, R.H (1960) \textit{The Problem of Social Cost} Journal of Law and Economics. Vol. III. October 1960 -- En carpeta del tema

Coase, R. H. (1991) \textit{The Sveriges Riksbank Prize in Economic Sciences in Memory of Alfred Nobel 1991} Lecture to the memory of Alfred Nobel -- En carpeta del tema.

Demsetz, H. (1983) \textit{The Structure of Ownership and the Theory of the Firm} Journal of Law and Economic. Vol. 26. No. 2. -- En carpeta del tema

Garrouste, P.; Saussier, S. (2004) \textit{Looking for a theory of the firm: future challenges} Journal of Economic Behavior \& Organization -- En carpeta del tema

Grossman, S. J.; Hart, O. D. (1986) \textit{The Costs and Benefits of Ownership: A Theory of Vertical and Lateral Integration}

\textbf{Hart, O. (1989) \textit{An Economist's Perspective on the Theory of the Firm} Columbia Law Review, Vol. 89, No. 7. -- En carpeta del tema}

Holmstrom, B.; Roberts, J. (1998) \textit{The Boundaries of the Firm Revisited} Journal of Economic Perspectives. Vol. 12. Number 4 -- En carpeta del tema

Holmstrom, B. R.; Tirole, J. (1989) \textit{The Theory of the Firm} Handbooks of Industrial Organization. Vol. 1

Economist, The. (2017) \textit{Coase's Theory of the firm} \url{https://www.economist.com/economics-brief/2017/07/27/coases-theory-of-the-firm} -- En carpeta del tema

Moss, S. (1984) \textit{The History of the Firm from Marshall to Robinson and Chamberlin: The Source of Positivism in Economics} Economica. Vol 51. No. 203 -- En carpeta del tema

Tirole, J. (1988) \textit{The Theory of Industrial Organization} MIT Press -- En carpeta Organización Industrial

Williamson, O. E. (1971) \textit{The Vertical Integration of Production: Market Failure Considerations} American Economic Review. Vol. 61. No. 2 -- En carpeta del tema

Williamson, O. E. (2002) \textit{The Theory of the Firm as Governance Structure: From Choice to Contract} Journal of Economic Perspectives. Vol. 16. Number 3. Summer 2002 -- En carpeta del tema

Williamson, O. E. (2009) \textit{Transaction Cost Economics: The Natural Progression}


\end{document}
