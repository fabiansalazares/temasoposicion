\documentclass{nuevotema}

\tema{3B-10}
\titulo{Teoría de la integración económica.}

\begin{document}

\ideaclave

\seccion{Preguntas clave}
\begin{itemize}
	\item ¿Qué es la integración económica?
	\item ¿Qué tipos de integración económica existen?
	\item ¿Cómo se modelizan sus efectos?
	\item ¿Cómo se valoran los efectos de la inmigración?
	\item ¿Por qué se lleva a cabo la integración económica?
\end{itemize}

\esquemacorto

\begin{esquema}[enumerate]
	\1[] \marcar{Introducción}
		\2 Contextualización
			\3 Concepto de integración económica
			\3 Procesos recientes de integración económica
			\3 Análisis de la integración económica
		\2 Objeto
			\3 ¿Qué es la integración económica?
			\3 ¿Qué formas de integración económica existen?
			\3 ¿Cómo se modelizan sus efectos?
			\3 ¿Por qué se lleva a cabo la integración económica?
		\2 Estructura
			\3 Formas de integración económica
			\3 Optimalidad de la integración
			\3 Geografía económica de la integración
			\3 Economía política de la integración
			\3 Otros efectos de la integración económica
			\3 Valoración empírica de la integración
	\1 \marcar{Formas de integración económica}
		\2 Idea clave
			\3 Número de integrantes
			\3 Grado de compromiso
			\3 Marco jurídico internacional
			\3 Tabla resumen
		\2 Tratado de preferencia comercial
			\3 Idea clave
			\3 Características básicas
		\2 Acuerdo de libre comercio
			\3 Idea clave
			\3 Características básicas
		\2 Unión aduanera
			\3 Idea clave
			\3 Características
		\2 Mercado común
			\3 Idea clave
			\3 Características
		\2 Mercado único
			\3 Idea clave
			\3 Características
		\2 Unión económica
			\3 Idea clave
			\3 Características
		\2 Unión económica y monetaria
			\3 Idea clave
			\3 Características
		\2 Unión política y económica
			\3 Idea clave
			\3 Características
	\1 \marcar{Optimalidad de la integración}
		\2 Antecedentes
			\3 Teoría clásica
			\3 Teoría neoclásica
			\3 Implicaciones
		\2 Desviación y creación de comercio: Viner (1950)
			\3 Idea clave
			\3 Formulación
			\3 Implicaciones
			\3 Valoración
		\2 Efectos sobre el consumo: Meade y Lipsey
			\3 Idea clave
			\3 Formulación
			\3 Implicaciones
		\2 Economías de escala y supresión de comercio: Corden (1976)
			\3 Idea clave
			\3 Formulación
			\3 Implicaciones
		\2 Efectos sobre la relación relativa de intercambio
			\3 Idea clave
			\3 Formulación
			\3 Implicaciones
		\2 Integración vía apertura unilateral vs UA
			\3 Idea clave
			\3 Cooper y Massel (1960): reducción unilateral
			\3 Johnson (1965): Preferencia por industria
			\3 Wonnacott y Wonnacott (1981): Aranceles en socios y no socios
			\3 Implicaciones
			\3 Valoración
		\2 Factores de deseabilidad de integración económica
			\3 Idea clave
			\3 Número de miembros
			\3 Diferencias entre costes de producción
			\3 Costes de transporte
			\3 Volumen de comercio previo entre miembros
			\3 Aranceles previos a integración entre miembros
			\3 Arancel externo común
			\3 Sustituibilidad de los bienes comerciados
			\3 Elasticidad de oferta y demanda
			\3 Número de bloques
	\1 \marcar{Geografía económica de la integración}
		\2 Krugman (1991): núcleo y periferia tras integración
			\3 Idea clave
			\3 Formulación
			\3 Implicaciones
			\3 Valoración
		\2 Dinámicas de especialización regional
			\3 Idea clave
			\3 Krugman (1993) y (2001)
			\3 Implicaciones
	\1 \marcar{Economía política de la integración}
		\2 Aspectos generales de la economía política de la integración
			\3 Idea clave
			\3 Stolper-Samuelson
			\3 Redistribución de beneficios del comercio
			\3 Modelo de factores específicos
			\3 Aversión a la pérdida
			\3 Aversión a incertidumbre
			\3 Aversión a desigualdad
			\3 Concentración de intereses
			\3 Implicaciones
			\3 Valoración
		\2 Trilema de Rodrik de la integración
			\3 Idea clave
			\3 Formulación
			\3 Implicaciones
			\3 Valoración
	\1 \marcar{Otros efectos de la integración económica}
		\2 Remuneración de ff.pp.
			\3 Idea clave
			\3 Implicaciones
		\2 Movilidad de factores
			\3 Idea clave
			\3 Movilidad de trabajo
			\3 Movilidad de capital
		\2 Heterogeneidad de empresas
			\3 Idea clave
			\3 Implicaciones
		\2 Crecimiento endógeno
			\3 Idea clave
			\3 Implicaciones
	\1 \marcar{Valoración empírica de los efectos de la integración}
		\2 Idea clave
			\3 Cuantificar efectos de la integración
			\3 Construcción de contrafactuales
			\3 Dificultades de la valoración empírica
		\2 Modelos de gravedad
			\3 Idea clave
			\3 Modelos simples
			\3 Modelos de resistencia multilateral
			\3 Ejemplos
		\2 Modelos de intensidad de comercio
			\3 Idea clave
			\3 Ejemplos (comparando con pre-integración)
		\2 Modelos CGE
			\3 Idea clave
			\3 Ejemplos de aplicación
	\1[] \marcar{Conclusión}
		\2 Recapitulación
			\3 Formas de integración económica
			\3 Optimalidad de la integración
			\3 Geografía económica de la integración
			\3 Economía política de la integración
			\3 Otros efectos de la integración económica
			\3 Valoración empírica de la integración
		\2 Idea final
			\3 Relaciones con otras áreas
			\3 Estado de la integración en la actualidad
			\3 Desintegración en contexto de crisis Covid
			\3 Instituciones en las que actúan policy-makiers

\end{esquema}

\esquemalargo



























\begin{esquemal}
	\1[] \marcar{Introducción}
		\2 Contextualización
			\3 Concepto de integración económica
				\4 Economías/países/jurisdicciones acuerdan
				\4[] Reducir barreras a movimiento:
				\4[] $\to$ Bienes y servicios
				\4[] $\to$ Trabajo y capital
				\4 Intensidad variable de la integración
				\4[] ¿Qué movimiento se permite?
				\4[] ¿Hasta qué punto se eliminan las barreras?
				\4[] ¿Qué barreras se eliminan?
			\3 Procesos recientes de integración económica
				\4 Constantes a lo largo de la historia
				\4[] Nuevos procesos de integración
				\4[] Fracaso de procesos
				\4[] $\to$ Desintegración y aparición de barreras
				\4 Imperios
				\4[] Desde el punto de vista económico
				\4[] $\to$ Ejemplos de integración económica
				\4[] Zollverein alemán
				\4 Entre-guerras
				\4[] Imperio británico
				\4[] Alemania nazi
				\4[] Neocolonialismo
				\4 Tras segunda guerra mundial
				\4[] Papel clave de GATT y OMC
				\4[] $\to$ Prohíbe tratamiento discriminatorio
				\4[] $\to$ Permite reducciones generalizadas
				\4[] Integración europea
				\4[] $\to$ CECA $\to$ CEE $\to$ UEM
				\4[] NAFTA
				\4[] MERCOSUR
				\4[] $\to$ BRA, ARG, PAR, URU
				\4[] ASEAN
				\4[] ...
			\3 Análisis de la integración económica
				\4 Sin corpus diferenciado
				\4 Herramientas de modelización habituales
				\4[] Teoría clásica del comercio internacional
				\4[] Crecimiento neoclásico y endógeno
				\4[] Modelos de economía política
				\4[] Modelos de localización
		\2 Objeto
			\3 ¿Qué es la integración económica?
			\3 ¿Qué formas de integración económica existen?
			\3 ¿Cómo se modelizan sus efectos?
			\3 ¿Por qué se lleva a cabo la integración económica?
		\2 Estructura
			\3 Formas de integración económica
			\3 Optimalidad de la integración
			\3 Geografía económica de la integración
			\3 Economía política de la integración
			\3 Otros efectos de la integración económica
			\3 Valoración empírica de la integración
	\1 \marcar{Formas de integración económica}
		\2 Idea clave
			\3 Número de integrantes
				\4 Afecta a los incentivos a unirse
			\3 Grado de compromiso
				\4 Determina profundidad de la integración
				\4 Mayor profundidad de integración implica:
				\4[] $\to$ Costes de negociación
				\4[] $\to$ Costes administrativos
				\4[] $\to$ Costes de heterogeneidad
				\4 También beneficios
				\4[] Examinado posteriormente
				\4[] Pero costes dificultan integración
				\4[] $\to$ Aunque en teoría fuese beneficiosa
			\3 Marco jurídico internacional
				\4 GATT impone no discriminación:
				\4[] Principio de nación más favorecida
				\4[] $\to$ Para todos, menor arancel para uno determinado
				\4[] Principio de trato nacional
				\4[] $\to$ Mismo trato a mercancías una vez han entrado
				\4 Excepciones a no discriminación
				\4[] Permiten integración económica
				\4[] Artículo XXIV de GATT: requisitos
				\4[] i) Eliminación substancial de barreras\footnote{Interpretado generalmente como aplicable a la práctica totalidad de los productos en cuestión.}
				\4[] ii) Barreras post-integración no superiores a previas
				\4[] iii) Si protección post-acuerdo perjudica a tercero
				\4[] $\to$ Necesario compensar a perjudicado
				\4[] Artículo V de GATS
				\4[] $\to$ Cobertura sustancial
				\4[] $\to$ Eliminación de toda discriminación
			\3 Tabla resumen
				\4[] \grafica{variantesdeintegracion}
		\2 Tratado de preferencia comercial\footnote{\textit{Preferential trade agreement}.}
			\3 Idea clave
				\4 Término con doble significado
				\4[] $\to$ Acuerdos de libre comercio y uniones aduaneras
				\4[] $\to$ Preferencias a países en desarrollo
				\4 Instrumento de ayuda a PEDs
				\4 Necesario reportar a WTO
			\3 Características básicas
				\4 Asumiendo segundo significado (PEDs)
				\4 Reducción parcial de aranceles
				\4[] ``Enabling clause'' de Ronda de Tokio: requisitos
				\4[] $\to$ Generalizado pero no ``substancial
				\4[] $\to$ No discriminatorio
				\4[] $\to$ No recíproco
		\2 Acuerdo de libre comercio\footnote{\textit{Free trade agreement}.}
			\3 Idea clave
				\4 Forman parte de concepto de RTA\footnote{``\textit{Regional trade agreement}''.}
				\4 Ejemplo: NAFTA
				\4 Eliminación de barreras aranc. y no arancelarias
				\4 Recíproco entre partes
				\4 Enorme proliferación en últimas décadas
				\4[] Más de 450 acuerdos notificados
				\4[] Solapamiento de acuerdos
				\4[] Aumento de complejidad jurídica
				\4[] Debilita sistema multilateral
				\4[] $\to$ ``\textit{spaghetti bowl}''
			\3 Características básicas
				\4 Eliminación de barreras entre partes
				\4[] $\to$ Aranceles
				\4[] $\to$ Posibles otras barreras
				\4 Basados en artículo XXIV de GATT y V de GATS
				\4[] $\to$ Excepción al principio de NMF
				\4[] $\to$ Cubrir >90\% del comercio entre partes
				\4[] $\to$ Fijar arancel promedio inferior a inicial\footnote{Algunos productos pueden tener arancel superior.}
				\4[] $\to$ Países perjudicados por UA pueden exigir compensaciones
				\4[] $\to$ Posibles acuerdos provisionales\footnote{Vetables por otros miembros de la OMC. Ningún acuerdo provisional se ha notificado desde 2015. La práctica habitual es notificar un acuerdo provisional como si fuera definitivo y aplicarle un plazo de caducidad, en el cual las partes negocian otro acuerdo verdaderamente definitivo.}
				\4 Mantenimiento de barreras con terceros
				\4[] Independientes de ALC
				\4 Requiere normas de origen complejas
				\4[] Para evitar que:
				\4[] $\to$ Importaciones entren por país con menor arancel
				\4[] $\then$ Administrativamente costoso
		\2 Unión aduanera\footnote{\textit{Customs union}.}
			\3 Idea clave
				\4 Eliminación de barreras arancelarias y no arancelarias
				\4 Arancel externo común
				\4 Ejemplos:
				\4[] UE + Turquía\footnote{Notar que es una unión aduanera parcial que afecta a los productos industriales en general pero no al acero ni a productos agrícolas.}
				\4[] CEE original de Tratado de Roma
				\4 Permite:
				\4[] $\to$ libre movimiento de bienes
				\4[] $\to$ Eliminar normas de origen entre miembros
				\4[] $\to$ Ahorro en costes administrativos y aduaneros
			\3 Características
				\4 Eliminación de aranceles entre partes
				\4 Introducción de arancel externo común
				\4[] En términos globales, no superior a previo
				\4[] Para algunos miembros
				\4[] $\to$ Puede ser superior a previo
				\4[] $\then$ Cálculo de impacto global de nuevo arancel
		\2 Mercado común
			\3 Idea clave
				\4 Ejemplo:
				\4[] Mercosur
				\4 Primera fase hacia mercado único
				\4 Liberalización de bienes
				\4[] + reducción barreras a servicios
				\4[] + reducción barreras a ff.pp.
			\3 Características
				\4 Eliminación de aranceles
				\4 Introducción de arancel externo común
				\4 Liberalización de prestación servicios
				\4[] Armonizar regulación
				\4[] Aceptar licencias de terceros países
				\4 Libre movimiento de ff.pp.
				\4[] Permitir migración y circulación
				\4[] Reducir barreras fiscales
		\2 Mercado único
			\3 Idea clave
				\4 Etapa final de mercado común
				\4 Objetivo de Acta Única Europea
				\4[] $\to$ ``mercado interior''
			\3 Características
				\4 Movimientos de ByS y ff.pp.
				\4[] $\to$ Circulan igual que dentro de un país
				\4 Profundización de mercado común
				\4[] No sólo libertad de movimiento de iure
				\4[] $\to$ También de facto
				\4[] $\then$ Armonización administrativa
				\4[] $\then$ Eliminación de otras barreras
		\2 Unión económica
			\3 Idea clave
				\4 Ejemplo: UE pre-euro, con SGP
				\4 Mercado único requiere mayor coordinación
				\4[] $\to$ Ciclos económicos asíncronos
				\4[] $\to$ Shocks asimétricos
			\3 Características
				\4 Eliminación de aranceles
				\4 Arancel externo común
				\4 Libre movimiento de ff.pp.
				\4 Política económica armonizada
				\4[] Armonización fiscal
				\4[] Coordinación fiscal
		\2 Unión económica y monetaria
			\3 Idea clave
				\4 Ejemplo: UEM de UE
				\4 Eliminar fluctuaciones de tipo de cambio
				\4[] Tipos de cambio fijados irreversiblemente
				\4 Autoridad monetaria única
			\3 Características
				\4 Eliminación de aranceles
				\4 Arancel externo común
				\4 Libre movimiento de ff.pp.
				\4 Política económica armonizada
				\4 Política monetaria común
		\2 Unión política y económica
			\3 Idea clave
				\4 Último estadio de integración
				\4 Soberanía compartida
			\3 Características
				\4 Libre circulación de ByS y ff.pp.
				\4 Política económica armonizada
				\4 Política monetaria común
				\4 Soberanía compartida con todos los miembros
	\1 \marcar{Optimalidad de la integración}
		\2 Antecedentes
			\3 Teoría clásica
				\4 Énfasis en tecnología
				\4 Ventaja absoluta
				\4[] Producir donde menores costes absolutos
				\4[] $\to$ Integración permite importar de donde menor coste
				\4[] $\then$ Integración aproxima a libre comercio
				\4[] $\then$ Integración deseable
				\4 Ventaja comparativa
				\4[] Producir donde ventaja comparativa
				\4[] $\to$ Integración permite importar de donde menor coste
				\4[] $\then$ Integración deseable
			\3 Teoría neoclásica
				\4 Generalización de clásica y H-O
				\4 Heckscher-Ohlin
				\4[] Énfasis en dotaciones
				\4 Integración aproxima a libre comercio
				\4[] Aumenta especialización dentro de bloque
				\4[] Induce producción donde menor coste
			\3 Implicaciones
				\4 Integración es en general deseable
				\4 Constituye un paso hacia libre comerci
				\4 Prevalente hasta años 50
		\2 Desviación y creación de comercio: Viner (1950)
			\3 Idea clave
				\4 Trabajo pionero en análisis de integración
				\4[] ¿Integración económica es siempre beneficiosa?
				\4 Fija temas principales del análisis
				\4[] $\to$ Cómo afecta integración a integrados
				\4[] $\to$ Cómo afecta integración a terceros
				\4 Costes de producción
				\4[] $\to$ Como criterio de eficiencia
			\3 Formulación
				\4 Tres países:
				\4[] $\to$ A, B y C
				\4 Unión aduanera:
				\4[] $\to$ A y B
				\4 \underline{Creación de comercio}
				\4 Antes de integración
				\4[] Costes de producción
				\4[] $\to$ A: 10
				\4[] $\to$ B: 11
				\4[] $\to$ C: 12
				\4[] Precios en B con arancel del 30\%:
				\4[] $\to$ Importación de A: 10 + 30\%=13
				\4[] $\to$ Importación de C: 12 + 30\%=15.6
				\4[] $\then$ $P_B < P_A, P_C$
				\4[] $\then$ País B no importa nada
				\4 Después de integración:
				\4[] Precios en B con arancel sólo a C
				\4[] $\to$ Importación de A: 10
				\4[] $\to$ Importación de B: 12+30\% = 15.6
				\4[] $\then$ $P_A < P_B < P_C$
				\4[] $\then$ País B importa de A
				\4[] $\then$ Nuevo flujo comercial
				\4 \underline{Desviación de comercio}
				\4 Antes de integración:
				\4[] Costes de producción:
				\4[] $\to$ A: 10
				\4[] $\to$ B: 12
				\4[] $\to$ C: 8
				\4[] Precios en B con arancel del 30\%:
				\4[] $\to$ Importación de A: 10 + 30\%=13
				\4[] $\to$ Importación de C: 8 + 30\%=10.4
				\4[] $\then$ $P_C < P_B < P_A$
				\4[] $\then$ País B importa de C
				\4 Después de integración:
				\4[] Precios en B con arancel sólo a C
				\4[] $\to$ Importación de A: 10
				\4[] $\to$ Importación de C: 8+30\%: 10.4
				\4[] $\then$ $P_A < P_B < P_C$
				\4[] $\then$ País B importa de A
				\4[] $\then$ Ha desaparecido un flujo comercial (C $\to$ B)
				\4[] $\then$ Se ha sustituido por productor menos eficiente A
			\3 Implicaciones
				\4 Creación de comercio
				\4[] Antes no se importaba de nadie
				\4[] Se pasa a importar de integrado
				\4[] Integrado es el más eficiente
				\4[] $\then$ Se reducen costes totales
				\4 Desviación de comercio
				\4[] Se deja de importar de no integrado
				\4[] Se importa de integrado
				\4[] No integrado es + eficiente que integrado
				\4[] $\then$ Integración destruye un vínculo comercial
				\4[] $\then$ Aumentan costes totales
				\4 Integración deseable
				\4[] Creación de comercio netamente superior a desviación
				\4 Integración puede no ser deseable
				\4[] Pueden aumentar costes totales
				\4[] $\then$ Empeora bienestar global
				\4 Optimalidad de integración depende del caso concreto
				\4[] No se pueden ordenar inequívocamente
				\4[] $\to$ Liberalización unilateral vs integración
			\3 Valoración
				\4 Racionaliza decisión de integración
				\4[] No siempre positiva
				\4[] $\to$ Posible preferir liberalización unilateral
				\4[] $\to$ Posible preferir status quo
				\4[] Perspectiva global
				\4 No tiene en cuenta:
				\4[] Efectos sobre el consumo de integración
				\4[] $\to$ ¿Los consumidores prefieren?
				\4[] Efectos sobre recaudación arancelaria
				\4[] Efectos de equilibrio general
				\4[] $\to$ Enfoque de eq. parcial
				\4[] Economías de escala
				\4 Problemas de contrastación
				\4[] Difícil estimación de costes
				\4[] $\to$ En distintos países
				\4[] $\to$ Antes y después de integración
		\2 Efectos sobre el consumo: Meade y Lipsey
			\3 Idea clave
				\4 Contexto
				\4[] Viner (1950) fuerte impacto
				\4[] $\to$ Atención a potencial impacto negativo de UAs
				\4[] $\to$ Énfasis en costes de producción
				\4[] Pero bienestar resulta de consumo
				\4[] $\to$ Viner análisis incompleto
				\4 Objetivo
				\4[] Considerar efectos de apertura sobre consumo
				\4[] Comparar no integración con UA y libre comercio
				\4 Resultados
				\4[] Libre comercio es first best
				\4[] $\to$ Óptimo de pareto mundial
				\4[] Pre-integración e integración son second-best
				\4[] $\to$ Dada imposibilidad de libre comercio
				\4[] $\to$ Ordenación pre-integración y UA depende de caso concreto
				\4[] Demanda de consumo es generalmente elástica
				\4[] $\to$ Integración reduce precios
				\4[] $\to$ Reducción de precios aumenta consumo
				\4[] $\then$ Apertura aumenta bienestar
				\4[] $\then$ Posible integración óptima aunque desviación
			\3 Formulación
				\4 Generalmente mostrado en:
				\4[] Eq. parcial
				\4[] Competencia perfecta
				\4[] Dos países A y B
				\4[] $\to$ Establecen unión aduanera
				\4[] Precios mundiales inalterados
				\4 Situación pre-integración
				\4[] País A:
				\4[] $\to$ Arancel más alto que B
				\4[] $\to$ Costes de prod. más altos que B
				\4[] $\to$ Demanda más alta que B
				\4[] $\to$ Producción alta pero insuficiente
				\4[] $\to$ Demanda baja pero más que producción
				\4[] $\then$ Importa
				\4[] País B:
				\4[] $\to$ Arancel más bajo que A
				\4[] $\to$ Costes de producción más bajos que B
				\4[] $\to$ Demanda más baja que B
				\4[] $\to$ No exporta ni importa con arancel previo
				\4 Situación post-integración
				\4[] Arancel externo común:
				\4[] $\to$ Media ponderada de A y B
				\4[] Cambio en A:
				\4[] $\to$ Aumenta demanda
				\4[] $\to$ Baja producción
				\4[] $\then$ Aumentan importaciones
				\4[] $\then$ Consumidores locales aumentan excedente
				\4[] $\then$ Empresas locales reducen excedente
				\4[] $\then$ Caen costes de producción
				\4[] $\then$ Desaparece ingreso arancelario
				\4[] Cambio en B:
				\4[] $\to$ Aumenta producción
				\4[] $\to$ Baja demanda
				\4[] $\then$ Aumentan exportaciones
				\4[] $\then$ Consumidores locales reducen excedente
				\4[] $\then$ Empresas locales aumentan excedente
				\4 \grafica{efectosconsumo}
			\3 Implicaciones
				\4 Integración reduce precios
				\4[] Aumento de demanda y consumo
				\4[] $\to$ Aumento de utilidad de consumidores
				\4 Integración puede ser deseable aún con desviación neta
				\4[] Efectos sobre aumento de consumo pueden compensar
				\4 Más margen para integración deseable
				\4[] Excedente del consumidor aumenta
				\4[] $\to$ Aumenta deseabilidad de integración
				\4 Integración como second-best
				\4[] No mejora libre comercio mundial
				\4[] Pero puede mejorar situación pre-integración
		\2 Economías de escala y supresión de comercio: Corden (1976)
			\3 Idea clave
				\4 Contexto
				\4[] Asumida competencia perfecta y R=E
				\4[] $\to$ Análisis de producción y precios anteriores
				\4[] Producción habitualmente muestra economía de escala
				\4[] $\to$ Reducción de coste medio con cantidad
				\4 Objetivos
				\4[] Caracterizar efectos de EEscala
				\4[] $\to$ sobre optimalidad IEconómica
				\4 Resultados
				\4[] Corden (1972)
				\4[] Economía de escala son factor adicional
				\4[] $\to$ Favorable a integración
				\4[] Reducción de costes medios
				\4[] $\to$ Resultado de integración
				\4[] Posible supresión de comercio
				\4[] $\to$ Integración reduce coste medio dentro de bloque
				\4[] $\to$ Reducción de coste medio suprime importaciones
				\4[] $\then$ Integración suprime comercio
				\4[] $\then$ Integración hace ineficiente comercio
			\3 Formulación
				\4 Pre-integración
				\4[] Bien producido en ambos potenciales integrados
				\4 Efecto convencional de creación de comercio
				\4[] Fin de arancel reduce coste de importación
				\4[] $\to$ Aumenta comercio entre miembros
				\4 Efecto de reducción de costes
				\4[] Ganancia adicional por concentración de producción
				\4[] Costes caen con mayor producción
				\4 Efecto de supresión de comercio
				\4[] País con menores costes medios pre-integración
				\4[] $\to$ Importaba de fuera de unión a pesar de CMe bajo
				\4[] Integración unión aduanera
				\4[] $\to$ Aumenta comercio entre nuevos socios
				\4[] $\to$ Reduce costes medios aún más
				\4[] $\then$ Sustitución de importaciones por intra-UA
				\4[] No es negativo
				\4[] $\to$ Reducción de costes en todo caso
			\3 Implicaciones
				\4 Supresión de comercio posible
				\4[] Eliminación de comercio con no-integrado
				\4[] Efecto positivo
				\4 Desviación de comercio sigue siendo posible
				\4[] Si no integrado sigue siendo productor más barato
				\4 Más probable integración favorable
				\4[] Efectos adicionales pro-integración
				\4[] $\to$ Reducción de costes de producción
				\4[] $\to$ Supresión de comercio con menos eficiente
				\4 Integración induce cambios tecnológicos
				\4[] Avanza otros programas de investigación
		\2 Efectos sobre la relación relativa de intercambio
			\3 Idea clave
				\4 Contexto
				\4[] Equilibrio general con países grandes
				\4[] $\to$ O con unión aduanera grande respecto al mundo
				\4[] Integración tiene efecto sobre precios mundiales
				\4[] $\to$ Reduce demanda hacia exterior de UA
				\4 Objetivos
				\4[] Caracterizar efecto de UA sobre RRI
				\4[] Valorar optimalidad
				\4 Resultados
				\4[] Mundell et al
				\4[] $\to$ UA pueden afectar RRI
				\4[] $\to$ Efectos sobre RRI no benefician igual a todos
				\4[] $\then$ Posibles conflictos entre miembros de UA
				\4[] Kemp y Van
				\4[] $\to$ Posible encontrar combinación de CETariff+redistribución
				\4[] $\to$ Redistribución en forma de sumas fijas
				\4[] $\then$ Hace óptima UA para todos sus miembros
				\4[] $\then$ UA mejora paretiana frente a no integración
			\3 Formulación
				\4 Desviación de comercio
				\4[] Reducción de importaciones de resto del mundo
				\4 Presión a la baja sobre precios de importaciones
				\4[] Reduce precio relativo a exportaciones
				\4 Efectos heterogéneos sobre miembros
				\4[] Algunos miembros podrían ser exportadores de bien abaratado
				\4[] $\to$ Pierden poder adquisitivo
				\4[] Otros miembros importadores de bien cuya demanda cae
				\4[] $\to$ Aumentan poder adquisitivo
				\4 Equilibrio parcial
				\4[] Caída de importaciones del RM
				\4[] Reduce precio mundial
				\4[] $\to$ Mejora RRI
				\4[] $\then$ Miembros de UA importan + por --
				\4 Equilibrio general
				\4[] Vanek (1965)
				\4[] Kemp y Wan (1976)
				\4[] Con múltiples bienes y países
				\4[] Existe combinación de:
				\4[] $\to$ Arancel externo común
				\4[] $\to$ Compensación a no miembros
				\4[] Que da lugar a:
				\4[] $\to$ Mantenimiento de comercio exterior
				\4[] $\to$ Igual o mayor bienestar en miembros
				\4[] $\then$ En teoría, posibles UAs que mejoren bienestar
			\3 Implicaciones
				\4 UA más deseable
				\4[] Asumiendo:
				\4[] $\to$ No hay represalias
				\4[] $\to$ Miembros ganadores compensan a perdedores\footnote{Al aumentar el arancel externo respecto al que algunos miembros impusieran antes de la integración, los consumidores de determinados países pueden verse perjudicados y exigir compensación a los ganadores dentro de la UA.}
				\4 Motivo de artículo XXIV de GATT
				\4[] Implícitamente, prevé efecto sobre RRI
				\4[] $\to$ Obliga a compensar a no-miembros
				\4 Aumenta poder de negociación de UA
				\4[] Respecto a países terceros
				\4[] UE es ejemplo claro
		\2 Integración vía apertura unilateral vs UA
			\3 Idea clave
				\4 Contexto
				\4[] Análisis de Viner, Meade, Lipsey
				\4[] $\to$ Integración reduce distorsión de arancel propio
				\4[] $\to$ Aumenta bienestar del consumidor nacional
				\4[] $\to$ Reduce costes de producción nacional
				\4[] $\then$ ¿Reducción compensa otros efectos?
				\4[] Simplemente con esos efectos
				\4[] $\to$ Cooper y Massel (1965)
				\4[] $\to$ Apertura unilat. $\Rightarrow$ creación máx. de comercio
				\4[] $\to$ Apertura unilat. evita desviación de comercio
				\4[] $\then$ Apertura unilateral debería ser óptima
				\4[] Asume supuesto implícito
				\4[] $\to$ Liberalizador unilat. exporta sin aranceles
				\4[] Realmente, exportaciones sufren aranceles:
				\4[] $\to$ Aranceles son herramienta de negociación
				\4[] $\to$ Cuanto mayor UA, otros más incentivados a gravar
				\4[] Otros efectos de integración económica
				\4[] $\to$ Posibles contrapartidas sobre exportaciones
				\4[] $\then$ Reducción a aranceles nacionales
				\4[] $\to$ Otros objetivos no económicos
				\4[] $\then$ Diversificación de exportadores
				\4[] $\then$ Relaciones diplomáticas, militares
				\4[] Posible comparar apertura unilat. vs integración
				\4 Objetivos
				\4[] Comparar apertura unilateral vs integración
				\4[] Identificar factores que hacen preferible integración
				\4 Resultado
				\4[] Debate de largo plazo
				\4[] Regionalismo vs multilateralismo
				\4[] Aparición de argumentos a favor de integración
				\4[] $\to$ Explican por qué no simple apertura unilateral
				\4[] $\to$ Explicar incentivos de integración vs apertura
			\3 Cooper y Massel (1960): reducción unilateral
				\4 Cooper y Massel (1975)
				\4 En contexto de Lipsey (1960)
				\4[] $\to$ Dos países
				\4[] $\to$ Consumo y producción
				\4 Reducción unilateral de país importador neto
				\4[] $\to$ Aumenta bienestar de consumidores
				\4[] $\to$ Mantiene cierto grado de recaudación por arancel
				\4 Integración en UA de país importador neto
				\4[] $\to$ Aumenta bienestar de consumidores
				\4[] $\to$ Pierde recaudación por arancel
				\4 Reducción unilateral siempre mejor que integración
				\4[] $\to$ Salvo que socio sea productor a menor coste mundial
				\4 Introduce cuestión
				\4[] $\to$ ¿Qué razones para integrarse en UA?
			\3 Johnson (1965): Preferencia por industria
				\4 Johnson (1965)
				\4 Algunos países tienen preferencia por industria nacional
				\4[] FBS no sólo basada en consumo individual
				\4[] $\to$ Consumo colectivo de bienes industriales relevante
				\4[] $\then$ Producción nacional de bienes es relevante
				\4 Consideraciones muy variadas
				\4[] Nacionalismo
				\4[] Seguridad nacional
				\4[] Economía política
				\4 Integración frente a apertura unilateral
				\4[] $\to$ Permite mantener industria nacional
			\3 Wonnacott y Wonnacott (1981): Aranceles en socios y no socios
				\4 Wonnacott y Wonnacott (1981)
				\4 Preferencia por industria nacional innecesaria
				\4[] $\then$ Justificable con incentivos puramente económicos
				\4 Unión aduanera permite reducir aranceles a exportaciones
				\4[] Mayores aranceles previos a integración
				\4[] $\then$ Más poder de negociación frente a potenciales socios
				\4[] $\then$ Permiten obtener más reducción para exportaciones
				\4 Cuanto mayor sea la UA resultante
				\4[] Más probable sufra aranceles exteriores
				\4[] Más efecto sobre RRI favorable a UA
				\4[] $\to$ Otros países tratan de obtener arancel óptimo
				\4[] $\then$ UA aumenta poder de negociación frente a terceros
			\3 Implicaciones
				\4 Existen incentivos puramente económicos a UA
				\4[] Capacidad para alterar RRI frente a no integrados
				\4[] Posibilidad de reducir aranceles sufridos por exportadores
				\4 Decisión entre UA y reducción unilateral
				\4[] Depende de aranceles que enfrentan exportadores
				\4[] $\to$ Si enfrentan, menos deseable
				\4[] Depende de tamaño de UA
				\4[] $\to$ Si grande, puede afectar RRI
				\4[] $\to$ Si grande, puede ganar poder de negociación
			\3 Valoración
				\4 Debate de largo plazo OMC vs acuerdos regionales
				\4 Acuerdos regionales aumentan importancia reciente
				\4 OMC estancada
				\4 Cabe preguntarse por bienestar resultante
		\2 Factores de deseabilidad de integración económica
			\3 Idea clave
				\4 Optimalidad de IEconómica
				\4[] No es unívoca
				\4 Depende de múltiples variables
				\4 Ceteris paribus
				\4[] Asumiendo resto de factores constantes
				\4[] $\to$ Valorar efecto de var. sobre optimalidad
			\3 Número de miembros
				\4 Cuanto más miembros integrados
				\4[] Más probable productor eficiente se integre
				\4[] $\to$ Menos probable desviación de comercio
				\4[] $\to$ Más beneficio para consumidores
				\4 Óptimo first-best
				\4[] $\to$ Todos los países sean miembros
				\4[] $\then$ Libre comercio mundial
				\4[$\then$] Más miembros hacen más deseable integración
			\3 Diferencias entre costes de producción
				\4 Mayores diferencias previas a integración
				\4[] Mayor efecto creación si integración
				\4 Más diferencia entre precios de integrados y RM
				\4[] Mayor efecto desviación de comercio
				\4[$\then$] Más diferencias iniciales, más deseable integración
				\4[$\then$] Más diferencias con RM eficiente, menos deseable
			\3 Costes de transporte
				\4 Cuanto menores los costes de transporte
				\4[] Más eficiente será la integración
				\4[$\then$] Menos CdT, más deseable
			\3 Volumen de comercio previo entre miembros
				\4 Mayor proporción de CI entre integrados pre-integración
				\4[] Menos creación con integración
				\4 Desviación no depende de comercio previo
				\4[$\then$] Cuanto mayor comercio previo, menos deseable
			\3 Aranceles previos a integración entre miembros
				\4 Aranceles muy elevados previos a integración
				\4[] Más creación de comercio con integración
				\4[] Más bienestar de consumidores
				\4[$\then$] Cuanto mayores los aranceles previos, más deseable
			\3 Arancel externo común
				\4 Menor arancel externo común
				\4[] Menor desviación de comercio
				\4 Artículo XXIV de GATT
				\4[] Area de integración no debe aumentar protección previa
				\4[] $\to$ Para no aumentar distorsión
				\4[] $\to$ Para no aumentar desviaciones de comercio
				\4[$\then$] Cuanto mayor el AEC, menos deseable la integración
			\3 Sustituibilidad de los bienes comerciados
				\4 Bienes comerciados entre integrados sustituibles
				\4[] Productores más eficientes sustituyen a menos eficientes
				\4 Bienes comerciados entre integrados complementarios
				\4[] Integrados no se sustituyen mutuamente
				\4[] Menos creación de comercio
				\4[$\then$] Cuanto más sustituibles, más deseable la integración
			\3 Elasticidad de oferta y demanda
				\4 Mayor elasticidad de demanda
				\4[] Aumenta demanda con bajada de precio por integración
				\4[] $\to$ Aumenta bienestar de consumidores
				\4[] $\then$ Mayor aumento cuanto más elasticidad demanda
				\4[] $\then$ Mayor bienestar cuanta más elasticidad
				\4 Mayor elasticidad de oferta
				\4[] Aumenta producción por aumento de demanda
				\4[] $\to$ Más elasticidad de oferta, menos aumento de coste
				\4[] $\then$ Menos aumento de costes con integración
				\4[$\then$] Cuanto más elásticas S y D, más deseable integración
			\3 Número de bloques
				\4 Mayor número de bloques regionales
				\4[] Mayor competencia entre bloques por exportación
				\4[] Menor poder de mercado de bloques individuales
				\4[] Menos efecto de RRI por imposición de arancel
				\4[] $\to$ Menos incentivo a imponer aranceles
				\4[] $\then$ Menos desviación de comercio
				\4 Muy pocos bloques y grandes
				\4[] Fuerte efecto individual sobre RRI
				\4[] $\to$ Incentivo a aumentar aranceles
				\4[] $\then$ Desviación de comercio
				\4[$\then$] Cuanto mayor el nº bloques, más deseable
	\1 \marcar{Geografía económica de la integración}
		\2 Krugman (1991): núcleo y periferia tras integración
			\3 Idea clave
				\4 Contexto
				\4[] Von Thünen
				\4[] $\to$ Análisis de localización óptima en ciudades
				\4[] $\to$ Pionero en economía espacial
				\4[] Marshall
				\4[] $\to$ Economías de escala tecnológicas con concentración
				\4[] i. Más facilidad para encontrar mano de obra
				\4[] ii. Spill-overs de información
				\4[] iii. Producción de inputs intermedios no comerciables
				\4[] $\then$ Difícil formalización
				\4[] $\then$ Análisis canónico hasta NEG
				\4[] Teoría clásica del CI
				\4[] $\to$ Economías son puntos sin dimensión espacial
				\4[] $\to$ Factores inmóviles entre países
				\4[] Hotelling
				\4[] $\to$ Análisis pionero
				\4[] $\to$ Formalización de decisión de localización empresas
				\4[] Salop (1979)
				\4[] $\to$ Entrada de empresas en contexto espacial
				\4[] Dixit y Stiglitz (1977)
				\4[] $\to$ Formalización de competencia monopolística
				\4[] $\to$ Análisis formal de equilibrio general
				\4[] $\then$ Agentes prefieren variedad
				\4[] $\then$ Incentivos a entrada de nuevas empresas/variedades
				\4[] Evidencia empírica
				\4[] $\to$ Aparición de núcleos y cinturones industriales
				\4[] $\to$ Concentración de población donde hay desarrollo industrial
				\4[] $\to$ Trabajo industrial móvil con patrones persistentes
				\4 Objetivos
				\4[] Explicar dinámicas de aglomeración
				\4[] $\to$ En contexto de desarrollo industrial
				\4[] $\to$ En contexto de bajada de precios de transporte
				\4[] Explicar patrón de comercio
				\4[] $\to$ En contexto de movilidad del trabajo
				\4[] $\to$ En contexto de costes de transporte y H-M-Effect
				\4 Resultados
				\4[] Dinámicas de aglomeración-dispersión
				\4[] $\to$ Dependen de parámetros clave
				\4[] Patrón de comercio endógeno
				\4[] $\to$ Depende de evolución de aglomeración-dispersión
				\4[] Integración comercial
				\4[] $\to$ Puede alterar estructura de población
				\4[] $\to$ Puede inducir aglomeración-dispersión
				\4[] Movimientos de trabajadores
				\4[] $\to$ Endógenos
				\4[] $\to$ Posible concentración geográfica
			\3 Formulación
				\4 Dos bienes consumidos
				\4[] Agrícola homogéneo $C_A$
				\4[] Manufacturado compuesto $C_M = \left( C_i^{\frac{\epsilon-1}{\epsilon}} \right)^{\frac{\epsilon}{\epsilon-1}}$
				\4 Dos factores de producción
				\4[] Campesinos inmóviles entre países
				\4[] $\to$ Distribuidos entre los dos países de manera exógena
				\4[] Obreros móviles entre países
				\4 Dos países/regiones A y B
				\4[] Campesinos repartidos equitativamente entre países
				\4[] Obreros con distribución inicial arbitraria
				\4[] $\to$ Sujeto a variación endógena
				\4 Demanda de bienes
				\4[] Obreros y campesinos iguales demandas
				\4[] $\to$ Se distribuye entre agrícola y manufacturero
				\4 Coste de transporte
				\4[] Agrícola sin coste de transporte
				\4[] Manufacturero con costes tipo iceberg
				\4[] $\to$ Para que llegue 1 hace falta enviar $\tau > 1$
				\4 Decisión de localización de obreros
				\4[] Donde haya mayor salario
				\4[] $\to$ Donde salario nominal compre más salario real
				\4[] $\then$ Donde haya más variedades más baratas
				\4[] Dos efectos contrapuestos afectan localización
				\4[] $\to$ Efecto competencia
				\4[] $\to$ Efecto demanda
				\4 Efecto demanda
				\4[] Localización cerca de la demanda
				\4[] $\to$ Permite superar costes de transporte
				\4[] Si coste fijo superior a costes de transporte
				\4[] $\to$ Preferible concentrar producción
				\4[] Cuanta mayor población obrera
				\4[] $\to$ Más se retroalimenta el efecto demanda
				\4[] Producción de variedades manufactureras concentradas
				\4[] $\to$ Aumenta salario real de obreros en aglomeración
				\4[] $\then$ Tendencia hacia concentración donde ya se produce manufact.
				\4 Efecto competencia
				\4[] Costes de transporte
				\4[] $\to$ Reducen competencia con variedades en otro país
				\4[] $\then$ Permiten aumentar precios
				\4[] Cuanta más población campesina sobre total
				\4[] $\to$ Mayor es la demanda que no se mueve
				\4[] $\then$ Más incentivos a localizarse donde no se producen variedades
				\4 Parámetros iniciales determinan resultado
				\4[] Preferencia por la variedad $\epsilon$
				\4[] $\to$ Menos posibilidad de sustituir variedades por baratas
				\4[] $\then$ Más cercanía a variedades baratas más importante
				\4[] $\then$ Aumenta importancia de tener más variedades
				\4[] Costes de transporte
				\4[] $\to$ Reduce competencia con variedades en otro país
				\4[] $\to$ Actúa a favor de la aglomeración
				\4[] Peso del sector manufacturero en población
				\4[] $\to$ Aumenta efecto de movimiento de L sobre demanda
				\4[] $\then$ Actúa a favor de la aglomeración
				\4 Dinámica del movimiento de obreros y comercio
				\4[] Asumiendo
				\4[] $\to$ Preferencia suficiente por la variedad
				\4[] $\to$ Suficiente peso del sector manufacturero
				\4[] Dispersión en equilibrio
				\4[] $\to$ Costes de transporte elevados +  pob. obrera reducida
				\4[] $\to$ Elevado efecto competencia
				\4[] $\to$ Poco efecto demanda
				\4[] $\then$ Tendencia a dispersión
				\4[] $\then$ \grafica{krugman91dispersion}
				\4[] Múltiples equilibrios
				\4[] $\to$ Costes de transporte intermedios + pob. obrera moderada
				\4[] $\to$ Si población dispersa, tendencia a dispersión
				\4[] $\to$ Si población inicialmente aglomerada, tendencia aglom.
				\4[] $\then$ Equilibrio depende de shock/condición inicial
				\4[] $\then$ Múltiples equilibrios dispersos y aglomerados
				\4[] $\then$ \grafica{krugman91multiplesequilibrios}
				\4[] Aglomeración en equilibrio
				\4[] $\to$ Costes de transporte reducidos + elevada pob. obrera
				\4[] $\to$ Poco efecto competencia
				\4[] $\to$ Efecto demanda elevado
				\4[] $\then$ Tendencia a aglomeración
				\4[] $\then$ \grafica{krugman91aglomeracion}
				\4[] Sin costes de transporte y con costes de congestión
				\4[] $\to$ Posible dispersión de nuevo
				\4[] $\to$ Sin home-market effect
				\4[] $\to$ Costoso concentrarse
				\4[] $\to$ Sin costes de exportar
				\4[] $\then$ Dispersión máxima
			\3 Implicaciones
				\4 Comercio internacional
				\4[] Especialización asimétrica posible
				\4[] $\to$ Núcleo especializado en industria
				\4[] $\to$ Núcleo y periferia producen agrícolas
				\4[] $\to$ Mayor demanda en núcleo induce importación agrícola
				\4 Movimiento de factores
				\4[] Permite explicar patrón migratorio en siglo XIX y XX
				\4[] Campo a la ciudad
				\4[] $\to$ Al reducirse CdTransporte
				\4[] $\to$ Al aumentar demanda de bienes industriales
				\4 Núcleo y periferia
				\4[] Núcleo
				\4[] $\to$ Concentración de obreros
				\4[] $\to$ Concentración de variedades industriales
				\4[] $\to$ Salarios elevados
				\4[] $\to$ Exportación de producto manufacturado
				\4[] $\to$ Importación de productos agrícolas
				\4[] Periferia
				\4[] $\to$ Sin obreros
				\4[] $\to$ Sin variedades industriales
				\4[] $\to$ Salarios reducidos
				\4[] $\to$ Importación de producto manufacturado
				\4[] $\to$ Exportación de productos agrícolas
				\4 Integración comercial induce aglomeración
				\4[] Posible aumento desigualdades regionales
				\4[] Posibles tensiones de economía política
			\3 Valoración
				\4 Premio Nobel a Krugman en 2008
				\4[] Culmina programa de comp. monop. y EEscala en CI
				\4 Abre programa de investigación
				\4[] Geografía económica basada en
				\4[] $\to$ Externalidades pecunarias
				\4 De manera paradójica, mundo se vuelve más clásico\footnote{Ver conclusión de Krugman (2008) Nobel Prize Lecture.}
				\4[] En últimas décadas
				\4[] Aumenta comercio basado en VComparativa
				\4[] $\to$ Cadenas de valor global
				\4[] $\to$ Especialización
				\4[] $\to$ IDE vertical frente a horizontal
		\2 Dinámicas de especialización regional
			\3 Idea clave
				\4 Nueva Economía Geográfica
				\4[] Krugman (1991), Fujita, Venables
				\4[] Explicar localización de actividad económica
				\4 Determinantes clave:
				\4[] Costes de transporte/Barreras al comercio
				\4[] $\to$ Impiden interacciones espaciales
				\4[] Beneficio de concentración
				\4[] $\to$ Rendimientos crecientes a escala
			\3 Krugman (1993) y (2001)
				\4 Integración y dinámicas de especialización
				\4 Contexto
				\4[] Similar a Rose (2000) y Glick y Rose (2001)
				\4[] Incertidumbre ante efectos de UEM
				\4[] Optimismo generalizado
				\4[] Poco análisis de especialización post-integración
				\4 Objetivo
				\4[] Valorar efectos de integración europea sobre:
				\4[] $\to$ Especialización regional
				\4[] $\to$ Correlación de shocks
				\4[] ...utilizando estados EEUU como referencia
				\4 Resultado
				\4[] UM aumenta:
				\4[] i. Comercio en general
				\4[] ii. Especialización regional
				\4[] Si i>ii y comercio intraindustrial aumenta mucho
				\4[] $\to$ Shocks más correlacionados
				\4[] Si especialización regional más importante
				\4[] $\to$ Menos correlación de shocks
				\4[] $\to$ UM elimina PM
				\4[] $\to$ Shocks regionales más idiosincráticos en UE que USA
				\4[] $\to$ Movimiento de ff.pp. menor en UE que USA
				\4[] $\to$ Transferencias fiscales menor en UE que USA
				\4[] $\then$ Política fiscal regional impotente tras integración
				\4[] $\then$ Integración puede complicar policy-making
				\4[] $\then$ Dinámicas inestables tras integración
				\4[] $\then$ Posible aumento desempleo en regiones afectadas
			\3 Implicaciones
				\4 Reducción de costes de transporte y barreras
				\4[] Característica fundamental de integración
				\4[] $\then$ Transformación de geografía económica
				\4 Concentración de la actividad
				\4[] Consecuencia de integración
				\4[] $\to$ Previsible
				\4[] $\to$ Constatada. P.ej: Europa
				\4[] Aplicable a múltiples niveles
				\4[] $\to$ Internacional
				\4[] $\to$ Integración entidades subnacionales
	\1 \marcar{Economía política de la integración}
		\2 Aspectos generales de la economía política de la integración\footnote{Con Baldwin (1989)Posible construir a partir de \textit{trade policy, political economy of} en Palgrave.}
			\3 Idea clave
				\4 Contexto
				\4[] Economía política
				\4[] $\to$ Análisis de efectos de política económica
				\4[] $\then$ Sobre intereses de diferentes grupos sociales
				\4[] $\then$ Como resultado de intereses de diferentes grupos
				\4[] Efectos de política comercial
				\4[] $\to$ Afectan distinto a diferentes sectores
				\4 Objetivo
				\4[] Caracterizar efectos sobre diferentes sectores
				\4[] Valorar diferentes modelos de economía política de la integración
				\4[] Entender impacto de estructura política sobre pol. arancelaria
				\4 Resultados
				\4[] Efectos de aranceles sobre diferentes grupos sociales
				\4[] $\to$ Beneficios y perjuicios
				\4[] $\to$ Diferentes grados de concentración
				\4[] $\to$ Diferente capacidad de respuesta
			\3 Stolper-Samuelson
				\4 En contexto Heckscher-Ohlin
				\4 Tras apertura comercial
				\4[] $\to$ Factor intensivo de sector de especialización
				\4[] $\then$ Aumenta pago al factor
				\4[] $\to$ Factor intensivo de sector que pierde producción
				\4[] $\then$ Coste de factores cae
				\4 Sector de factor intensivo en bien de especialización
				\4[] $\then$ Presión hacia reducción de aranceles
				\4 Sector de factor intensivo en bien que pierde producción
				\4[] $\then$ Presión hacia mantenimiento de aranceles
				\4 Países ricos
			\3 Redistribución de beneficios del comercio
				\4 Permite a perdedores aceptar reducción de aranceles
				\4[] Pero costes de redistribución
				\4[] $\to$ Negociación entre sectores
				\4[] $\to$ Votaciones
				\4[] $\to$ Adquisición de información
				\4[] $\then$ Posible no sea rentable redistribuir
			\3 Modelo de factores específicos
				\4 Dos factores de capital inmóviles
				\4 Desarme arancelario mutuo
				\4[] $\to$ Aumenta beneficios que pasan a exportar
				\4[] $\to$ Reduce beneficio en sectores que ahora importan
				\4[] $\then$ Flujo de trabajo de un sector a otro
				\4[] $\then$ Caída de PMgK en sector perjudicado
				\4 Diferentes intereses dentro de un mismo factor
				\4[] $\to$ Capital vs trabajo no siempre oposición homogénea
			\3 Aversión a la pérdida
				\4 Behavioral economics
				\4 Empíricamente, aversión a pérdida
				\4[] $\to$ Mayor que ganancia
				\4 Apertura arancelaria
				\4[] $\to$ Induce beneficio en un sector
				\4[] $\to$ Aumenta pérdidas en otro
				\4 Si aversión a pérdida mayor que ganancia por beneficio
				\4[] $\then$ Oposición más fuerte
			\3 Aversión a incertidumbre
				\4 Apertura aumenta incertidumbre
				\4[] $\to$ ¿Efectos de equilibrio general serán positivos?
			\3 Aversión a desigualdad
				\4 Apertura al comercio puede aumentar desigualdad
				\4[] $\to$ Sector de especialización más rico
				\4[] $\to$ Sector que reduce producción más pobre
				\4 Seres humanos muestran cierta aversión a la desigualdad
				\4[] $\to$ Factor de oposición a apertura
			\3 Concentración de intereses
				\4 Efectos de reducción arancelaria
				\4[] $\to$ Difusos sobre consumidores
				\4[] $\to$ Muy concentrados sobre industria desprotegida
				\4 Perjuicio concentrado
				\4[] $\to$ Facilita coordinación entre perjudicados
				\4[] $\then$ Facilita oposición política a apertura
			\3 Implicaciones
				\4 Oposición a apertura más fuerte que presión apertura
				\4 Instituciones multilaterales pueden catalizar
				\4[] Commitment liberalizador
				\4[] Aumenta poder de negociación de liberalizadores
				\4 Redistribución puede ser necesaria
				\4[] Mejora aceptación de apertura
				\4[] También es costosa
			\3 Valoración
				\4 Programa de investigación con muchas vertientes
				\4 Interacciones con sociología, ciencia política, demografía..
				\4 Ciencia económica no siempre ha examinado
				\4[] Supuestos demasiado fuertes
				\4[] $\to$ ¿Planificador social?
				\4[] $\to$ ¿Funciones de bienestar social?
				\4[] $\then$ ¿Realmente existen?
				\4[] $\then$ ¿Realmente consideradas en decisiones de PComercial?
		\2 Trilema de Rodrik de la integración
			\3 Idea clave
				\4 Contexto
				\4[] Fenómeno de últimas décadas
				\4[] Democracias representativas sufren:
				\4[] $\to$ Movimientos contrarios a integración
				\4[] $\to$ Dificultades para compensar perdedores del comercio
				\4[] Aparición de tensiones políticas
				\4 Objetivos
				\4[] Caracterizar consecuencias de integración económica
				\4[] $\to$ Sobre sistema político
				\4[] Valorar restricciones a integración
				\4[] $\to$ Dadas por sistema político
				\4 Resultados
				\4[] Rodrik (2000)\footnote{JEP Winter 2000.}
				\4[] Imposible integración económica con:
				\4[] $\to$ Democracia representativa
				\4[] $\to$ Soberanía nacional
				\4[] Necesario dejar una de las tres
			\3 Formulación
				\4 Restricción empírica postulada
				\4 Economías abiertas deben elegir 2 de 3:
				\4[I] Integración económica
				\4[II] Democracia
				\4[III] Soberanía nacional
				\4 Tres alternativas:
				\4[A] Camisa de fuerza de oro
				\4[] Integración económica+Estado nación soberano
				\4[] Sin transferencias fiscales entre estados
				\4[] Flujos de capital y comerciales libres
				\4[] Mercados internacionales limitan PEconómica nacional
				\4[] Sólo se proveen BPúblicos compatibles con MFinancieros
				\4[] Necesarias políticas autoritarias/represivas
				\4[] $\to$ Ante crisis de deuda/balanza de pagos
				\4[B] Federalismo supranacional
				\4[] Integración económica+democracia
				\4[] Apertura comercial y financiera plena
				\4[] Estados nación pierden soberanía
				\4[] $\to$ Entidad supranacional asume soberanía
				\4[] Transferencias fiscales entre estados
				\4[] $\to$ Posibles déficits exteriores y fiscales
				\4[] Democracia a nivel supranacional
				\4[] $\to$ Entidad supranacional se convierte en nación
				\4[C] Compromiso à la Bretton Woods
				\4[] Democracia+soberanía nacional
				\4[] Sin plena integración comercial+financiera
				\4[] Barreras a movimiento de capital generalizados
				\4[] Estados pueden evitar endeudamiento exterior
				\4[] Posible provisión democrática de bienes públicos
				\4[] $\to$ En la medida en que permita cap. productiva nacional
				\4[] $\to$ Como lo decidan votantes/responsable soberano
				\4[] Sin transmisión de soberanía a ent. supranacional
			\3 Implicaciones
				\4 Integración económica restringe sistema político
				\4 Sistema político restringe integración económica
				\4 Opciones intermedias pueden ser inestables
				\4[] Bretton Woods acaba cayendo
				\4[] $\to$ Movilidad de capitales creciente
				\4[] $\to$ Dilema de Triffin
			\3 Valoración
				\4 Fuerte impacto en debate público sobre integración
				\4 Debate sobre validez externa
				\4[] ¿Realmente hay menos democracia representativa?
				\4[] $\to$ ¿Es sólo percepción fruto de tensiones políticas?
	\1 \marcar{Otros efectos de la integración económica}
		\2 Remuneración de ff.pp.
			\3 Idea clave
				\4 Consecuencia de teorema de Stolper-Samuelson
				\4[] Aumento de precio de bien
				\4[] $\to$ Aumento de remuneración de f.p. intensivo
				\4 Barreras al comercio
				\4[] Benefician al factor escaso
				\4[] Penalizan factor abundante
			\3 Implicaciones
				\4 Integración genera ganadores y perdedores
				\4[] Dentro de área de integración
				\4[] $\to$ Economía política es relevante
				\4[] Factor escaso
				\4[] $\to$ Presión a no integración
				\4[] Factor abundante
				\4[] $\to$ Presión hacia integración
				\4 Presión de sectores exportadores
				\4[] Sector exportador competitivo de miembro
				\4[] $\to$ Presionará hacia integración
				\4[] $\to$ Ej. Alemania en UE
		\2 Movilidad de factores
			\3 Idea clave
				\4 Contexto
				\4[] Integración económica en primeras fases
				\4[] $\to$ Movimiento facilitado de mercancías
				\4[] Integración en fases avanzadas
				\4[] $\to$ Movimientos de factores de población también posibles
				\4 Objetivos
				\4[] Efectos de integración sobre migración
				\4[] Efectos de integración sobre movimiento de capital
				\4 Resultados
				\4[] Migración omnipresente y por olas
				\4[] $\to$ Relativamente menos crecimiento que CI y mov. de K
				\4[] Migrantes se autoseleccionan
				\4[] Modelos de migraciones
				\4[] $\to$ Borjas
				\4[] $\to$ Redes
				\4[] $\to$ Centro-periferia con economías de escala y CMonop
				\4[] $\to$ McDougall
				\4[] $\to$ ...
			\3 Movilidad de trabajo
				\4 Determinantes
				\4[] Borjas (1987): autoselección
				\4[] $\to$ Integración económica induce autoselección de migrantes
				\4[] $\then$ Correlación salario en origen y destino+varianza destino
				\4[] $\then$ Determina patrón migratorio tras integración
				\4[] $\then$ Posible fuga de capital humano
				\4[] Redes migratorias
				\4[] $\to$ Presencia de comunidad de mismo origen
				\4[] $\then$ Aumenta salario de migrantes de misma comunidad
				\4[] $\then$ Aumenta posibilidad de integración
				\4[] $\then$ Aumenta vínculos culturales
				\4 Efectos
				\4[] McDougall
				\4[] $\to$ Igualación de PMgL
				\4[] $\to$ Caída de PMgK en país de inmigración
				\4[] $\to$ Aumento de PMgK en país de emigración
				\4[] Aumento de población
				\4[] $\to$ Aumento de demanda
				\4[] $\then$ Aumento de home-market effect
				\4[] $\then$ Dinámicas de aglomeración
				\4[] Efectos sobre decisiones de población nativa
				\4[] $\to$ Desplazamiento a trabajos más cualificados
				\4[] $\to$ Aprovechamiento de complementariedades
				\4[] Efectos sobre estado de bienestar
				\4[] $\to$ Alesina et al: aumento diversidad cultural
				\4[] $\then$ Reduce disposición a estado de bienestar
				\4[] Brain drain y brain gain
				\4[] ...
			\3 Movilidad de capital
				\4 Determinantes
				\4[] Mayor productividad del capital
				\4[] Seguridad jurídica
				\4[] Crecimiento de la productividad
				\4[] Déficits de cuenta corriente
				\4[] Productividad por trabajor
				\4[] ...
				\4 Efectos
				\4[] Aumento de inversión
				\4[] Aumento de transferencia tecnológica
				\4[] Integración en CVGs
				\4[] Sudden-stops
				\4[] Burbujas especulativas
				\4[] IDE horizontal y vertical
		\2 Heterogeneidad de empresas
			\3 Idea clave
				\4 Melitz (2003)
				\4 Explicar un hecho empírico:
				\4[] Empresas exportadoras suelen ser:
				\4[] $\to$ Más grandes
				\4[] $\to$ Más eficientes
				\4[] Apertura al comercio induce:
				\4[] $\to$ Empresas más eficientes en media
				\4[] $\to$ Empresas más grandes en media
				\4 Factores que explican:
				\4[] Exportación implica costes fijos
				\4[] Empresas eficientes cubren costes fijos
				\4[] Empresas poco eficientes
				\4[] $\to$ No pueden exportar
				\4[] $\to$ Sufren competencia de extranjeras
			\3 Implicaciones
				\4 Integración afecta a estructura industrial
				\4 Integración puede aumentar eficiencia media
				\4 Cambios en estructura industrial
				\4[] Aumento de tamaño y eficiencia si
				\4[] $\to$ Aumenta acceso a mercado extranjero
				\4[] Pérdida de empresas, tamaño y eficiencia si:
				\4[] $\to$ Aumenta competencia de integrados
				\4[] $\to$ Reducción de acceso a terceros mercados
		\2 Crecimiento endógeno
			\3 Idea clave
				\4 Modelos estáticos tienen en cuenta efectos
				\4[] -- One-off
				\4[] -- de corto plazo
				\4 Estimaciones de ganancias de bienestar de c/p
				\4[] Ganancias muy reducidas
				\4[] Apenas algún punto \% de PIB
				\4 Estimaciones de ganancias de bienestar de l/p
				\4[] Mucho mayores
				\4[] Tienen en cuenta:
				\4[] $\to$ Cambios en tasa de crecimiento económico
				\4[] $\to$ Aumento de comercio intraindustrial
			\3 Implicaciones
				\4 Más allá de efectos creación y desviación
				\4 Integración provoca
				\4[i] Economías de escala y reducción de costes
				\4[] Aumentan con crecimiento
				\4[ii] Presión competitiva reduce ineficiencias
				\4[] Aumentan con número de empresas e integración
				\4[iii] Impulso a la inversión extranjera directa
				\4[] Si integración de mercados de capital
				\4[iv] Estímulos a innovación
				\4[] Si integración de trabajo y tecnologías
				\4[v] Aumento de acumulación de capital
				\4[] Aumento de producción aumenta ahorro
				\4[] $\to$ Aumento de capital
				\4[] Modelos de crecimiento endógeno
				\4[] $\to$ Crecimiento más allá de EEstacionario
	\1 \marcar{Valoración empírica de los efectos de la integración}
		\2 Idea clave
			\3 Cuantificar efectos de la integración
				\4[] Necesario cuantificar consecuencias
				\4[] $\to$ Justificar o no la integración
				\4 Valorar modelos teóricos de integración
				\4[] ¿Efectos predichos tienen lugar?
				\4[] ¿Cómo se comparan con no-integración?
			\3 Construcción de contrafactuales
				\4 Pilar central de valoración empírica
				\4[] A pesar de complejidad en macroeconomías
				\4 ¿Qué sucedería sin integración?
				\4[] $\to$ Modelizar evolución con integración
				\4[] $\then$ Para comparar escenarios y decidir
				\4[] $\then$ Para comparar con integración y valorar
			\3 Dificultades de la valoración empírica
				\4 Problemas de endogeneidad
				\4[] ¿Integración causa efectos?
				\4[] ¿Efectos causan integración?
				\4 No linealidad
				\4[] Especialmente a nivel macro
				\4 Procesos endógenos para contrafactual
				\4[] Muy dificilmente predecibles
		\2 Modelos de gravedad
			\3 Idea clave
				\4 Basados en pura estimación econométrica
				\4[] Estimación de CI (endógena)
				\4[] $\to$ A partir de vars. exógenas
				\4 Exógenas habituales
				\4[] A favor del comercio
				\4[] $\to$ Tamaño de la economía
				\4[] $\to$ Población
				\4[] $\to$ PIB per cápita
				\4[] Resistencia al comercio
				\4[] $\to$ Costes de transporte
				\4[] $\to$ Distancia físca
				\4[] $\to$ Similitud de lenguas
				\4[] $\to$ Instituciones similares
			\3 Modelos simples
				\4 Estimación de flujos bilaterales
				\4 Muy parecidos a ec. de gravedad de Newton
				\4[] $X_{ij} = \phi \frac{\text{PIB}_i^{\alpha_1} \text{PIB}_j^{\alpha_2}}{\text{m}_{ij}^{\alpha_3}}$
				\4[] $\then$ $\ln X_{ij} = \phi + \alpha_1 \ln \text{PIB}_i + \alpha_2 \ln \text{PIB}_j + \alpha_3 \ln \text{m} + \epsilon_{ij}$
				\4 Constante gravitacional
				\4[] Parámetro $\phi$ en anterior
				\4[] Representa barreras al comercio no explicadas
				\4[] $\to$ Lenguas, instituciones, fronteras
			\3 Modelos de resistencia multilateral
				\4 Anderson y Van Wincoop (2003) y otros
				\4 Reminiscente de equilibrio general
				\4 Estimar flujos bilaterales teniendo en cuenta:
				\4[] Precios y características esenciales
				\4[] $\to$ De terceros países
			\3 Ejemplos
				\4 Efecto de la distancia
				\4[] Mucho mayor en servicios que en mercancías
				\4 Efectos frontera
				\4[] Fronteras tienen efecto muy pronunciado
				\4[] Misma distancia, presencia/ausencia de frontera
				\4[] $\to$ Efecto negativo apreciable
		\2 Modelos de intensidad de comercio
			\3 Idea clave
				\4 Comparar CI entre miembros y RM en:
				\4[] $\to$ Pre-integración
				\4[] $\to$ Post-integración
				\4[] $\to$ Contrafactual sin integración
				\4 Calcular tasa de concentración comercial
				\4[] Cociente entre:
				\4[] $\to$ Exp. e Imp. entre integrados
				\4[] $\to$ Exp. e Imp. entre integrados y RM
			\3 Ejemplos (comparando con pre-integración)
				\4 UE
				\4[] Ligero aumento de concentración comercial
				\4 NAFTA
				\4[] Fuerte aumento de concentración
				\4 MERCOSUR
				\4[] muy fuerte aumento concentración
				\4 ASEAN
				\4[] Disminución de la concentración
		\2 Modelos CGE
			\3 Idea clave
				\4 Derivados de input-output
				\4[] Output resultado de:
				\4[] $\to$ Cantidades de input aplicado
				\4 Teniendo en cuenta efecto de $\Delta$ precios
				\4[] Elasticidades afectan a equilibrios
				\4 Variedades estáticas y dinámicas
				\4 Calibración
				\4[] 1. Asignar valores a parámetros
				\4[] $\to$ Mediante econometría o estimaciones
				\4[] 2. Replicar datos conocidos
			\3 Ejemplos de aplicación
				\4 NAFTA
				\4[] Predecían especial beneficio para México
				\4 TTP
				\4 Brexit
				\4[] Efectos negativos en mayoría de modelos
	\1[] \marcar{Conclusión}
		\2 Recapitulación
			\3 Formas de integración económica
			\3 Optimalidad de la integración
			\3 Geografía económica de la integración
			\3 Economía política de la integración
			\3 Otros efectos de la integración económica
			\3 Valoración empírica de la integración
		\2 Idea final
			\3 Relaciones con otras áreas
				\4 Movimientos de factores de producción
				\4 Economía política
				\4 Integración monetaria y financiera
				\4 Organización industrial
			\3 Estado de la integración en la actualidad
				\4 UE
				\4[] Proceso más importante de últimos siglos
				\4[] Evidencia indica que:
				\4[] $\to$ UEM ha aumentado comercio que ya era elevado
				\4[] $\to$ Efectos positivos sobre productividad
				\4[] $\to$ Convergencia este-oeste y sur-norte
				\4 NAFTA
				\4[] Ha aumentado convergencia:
				\4[] $\to$ entre México--USA y CAN
				\4[] $\to$ Entre estados mejicanos
				\4[] Mejora generalizada de productividad
				\4 MERCOSUR
				\4[] Empresas argentinas mejoraron productividad
				\4[] $\to$ Accedieron a mejor tecnología
			\3 Desintegración en contexto de crisis Covid
				\4 Ayudas estatales en áreas de integración
				\4[] Inevitables en sectores muy afectados
				\4[] No todas economías miembros pueden hacerlo
				\4[] Inducen tensiones entre miembros
				\4[] $\to$ Pueden presionar hacia desintegración
				\4 Consideraciones de seguridad nacional
				\4[] Aumentan peso en contexto de pánico
				\4[] Pueden inducir preferencia por industria nacional
			\3 Instituciones en las que actúan policy-makiers
				\4 Intereses nacionales
				\4[] Cuanto más concentrados
				\4[] $\to$ Más presión en contra/a favor
				\4 Productores vs consumidores
				\4[] Productores generalmente más concentrados
				\4[] $\to$ Defensa de intereses más efectiva
				\4[] Ejemplo: agricultores japoneses
				\4 Multilateralismo y regionalismo
				\4[] Forma de escapar a intereses nacionales
				\4[] $\to$ Contrarrestar con presión internacional
				\4[] $\to$ Aumentar ``commitment'' a apertura
				\4 Trade-off del policy-maker
				\4[] Decisiones deben contraponer:
				\4[] $\to$ Daño y presión de grupos perjudicados
				\4[] $\to$ Bienestar social e interés general disperso
				\4 Calidad institucional
				\4[] Condiciona poder de grupos de interés
				\4[] $\to$ Factor importante en voluntad de integración

\end{esquemal}



















































\graficas

\begin{tabla}{Características principales de cada grado de integración económica.}{variantesdeintegracion}
	\begin{tabular}{c | c | c | c | c | c | c | c | c  }
& PTA & ALC & UA & MC & MU & UE & UEM & UEP \\ \hline
Reducción de aranceles & \cellcolor{green!25}\cmark & \cellcolor{green!25}\cmark & \cellcolor{green!25}\cmark & \cellcolor{green!25}\cmark & \cellcolor{green!25}\cmark & \cellcolor{green!25}\cmark & \cellcolor{green!25}\cmark & \cellcolor{green!25}\cmark \\ \hline

Eliminación de aranceles & \cellcolor{red!25}\xmark & \cellcolor{green!25}\cmark & \cellcolor{green!25}\cmark & \cellcolor{green!25}\cmark & \cellcolor{green!25}\cmark & \cellcolor{green!25}\cmark & \cellcolor{green!25}\cmark & \cellcolor{green!25}\cmark \\ \hline 

Arancel externo común & \cellcolor{red!25}\xmark & \cellcolor{red!25}\xmark & \cellcolor{green!25}\cmark & \cellcolor{green!25}\cmark & \cellcolor{green!25}\cmark & \cellcolor{green!25}\cmark & \cellcolor{green!25}\cmark & \cellcolor{green!25}\cmark \\ \hline 

Movimiento de ff.pp. & \cellcolor{red!25}\xmark & \cellcolor{red!25}\xmark & \cellcolor{red!25}\xmark & \cellcolor{green!25}\cmark & \cellcolor{green!25}\cmark & \cellcolor{green!25}\cmark & \cellcolor{green!25}\cmark & \cellcolor{green!25}\cmark \\ \hline 

Cuatro libertades & \cellcolor{red!25}\xmark & \cellcolor{red!25}\xmark & \cellcolor{red!25}\xmark & \cellcolor{red!25}\xmark & \cellcolor{green!25}\cmark & \cellcolor{green!25}\cmark & \cellcolor{green!25}\cmark & \cellcolor{green!25}\cmark \\ \hline 

Armonización pol. fiscal & \cellcolor{red!25}\xmark & \cellcolor{red!25}\xmark & \cellcolor{red!25}\xmark & \cellcolor{red!25}\xmark & \cellcolor{red!25}\xmark & \cellcolor{green!25}\cmark & \cellcolor{green!25}\cmark & \cellcolor{green!25}\cmark \\ \hline

Política monetaria única & \cellcolor{red!25}\xmark & \cellcolor{red!25}\xmark &  \cellcolor{red!25}\xmark &  \cellcolor{red!25}\xmark &  \cellcolor{red!25}\xmark &  \cellcolor{red!25}\xmark & \cellcolor{green!25}\cmark & \cellcolor{green!25}\cmark \\ \hline 

Soberanía compartida & \cellcolor{red!25}\xmark & \cellcolor{red!25}\xmark & \cellcolor{red!25}\xmark & \cellcolor{red!25}\xmark & \cellcolor{red!25}\xmark & \cellcolor{red!25}\xmark & \cellcolor{red!25}\xmark & \cellcolor{green!25}\cmark\\ \hline

	\end{tabular}
\end{tabla}

\begin{axis}{4}{Efectos de una unión aduanera sobre el consumo y la producción en un contexto de equilibrio parcial sobre los países que se integran.}{Q}{P}{efectosconsumo}
	% Arancel externo común
	\draw[dashed] (0,2) -- (4,2);
	\draw[dashed] (6,2) -- (10,2);
	\node[left] at (0,2){\small CET};
	\node[left] at (6,2){\small CET};

	% Precio mundial
	\draw[dashed] (0,1) -- (4,1);
	\draw[dashed] (6,1) -- (10,1);
	\node[left] at (0,1){\small W};
	\node[left] at (6,1){\small W};
	
	% Ejes del país B
	\draw[-] (6,4) -- (6,0) -- (10,0);
	\node[below] at (10,0){Q};
	\node[left] at (6,4){P};

	%%%%%%%%% PAÍS A
	% Subtítulo del país
	\node[] at (2,-1){País A};
	
	% Arancel inicial
	\draw[dashed] (0,2.5) -- (4,2.5);
	%\draw[dashed] (6,2.5) -- (10,2.5);
	\node[left] at (0,2.5){$T_A$};
	
	% Demanda
	\draw[-] (1.5,4) -- (3.5,0.5);
	
	% Oferta
	\draw[-] (0,1.3) -- (3.5,3.5);
	
	% Equilibrio pre-integración
	\draw[dashed] (1.9,2.5) -- (1.9,0); % producción nacional
	\node[below] at (1.9,0){\small $Q_2$};
	\draw[dashed] (2.35,2.5) -- (2.35,0); % demanda nacional
	\node[below] at (2.31,0){\small $Q_3$};
	
	% Equilibrio post-integración
	\draw[dashed] (1.1,2) -- (1.1,0);
	\node[below] at (1.1,0){\small $Q_1$};
	\draw[dashed] (2.65,2) -- (2.65,0);
	\node[below] at (2.65,0){\small $Q_4$};
	
	% Aumentos del excedente
	\draw [white, fill=green, opacity=0.2] (1.9,2.5) -- (1.1,2) -- (1.9,2); % por reducción de costes nacionales
	\node[] at (1.7,2.2){\tiny A};
	
	\draw [white, fill=green, opacity=0.2] (2.35,2.5) -- (2.35,2) -- (2.65,2); % por aumento de excedentes nacionales
	\node[] at (2.42,2.2){\tiny B};

	% Reducción del excedente
	\draw [white, fill=red, opacity=0.2] (1.9,2) -- (1.9,1) -- (2.35,1) -- (2.35,2); % por aumento de excedentes nacionales
	\node[] at (2.07,1.5){\tiny C};
	
	%%%%%%%%% PAÍS B
	% Nombre del país B
	
	\node[] at (8,-1){País B};
	
	% Arancel inicial
	%\draw[dashed] (0,1.5) -- (4,1.5);
	\draw[dashed] (6,1.5) -- (10,1.5);
	\node[left] at (6,1.5){$T_B$};

	% Demanda
	\draw[-] (6,2.6) -- (10,0.2);
	
	% Oferta
	\draw[-] (6,0.65) -- (10,2.5);
	
	% Equilibrio pre-integración
	\draw[dashed] (7.84,1.5) -- (7.84,0); % producción y demanda nacional
	\node[below] at (7.84,0){\small $Q_2$};

	% Equilibrio post-integración
	\draw[dashed] (7,2) -- (7,0);
	\node[below] at (7,0){\small $Q_1$};
	\draw[dashed] (8.9,2) -- (8.9,0);
	\node[below] at (8.9,0){\small $Q_3$};
	
	% Aumento del excedente
	\draw [white, fill=green, opacity=0.2] (7,2) -- (7.84,1.5) -- (8.9,2);
	\node[] at (7.84,1.75){\tiny D};
\end{axis}

Las áreas verdes representan el aumento del excedente como resultado de la integración económica y el paso de aranceles $T_A$ y $T_B$ en los países A y B respectivamente, a un arancel externo común CET. Las áreas verdes representan el aumento del excedente de consumidores y productores. El área roja, la pérdida de excedente.

El área A corresponde a la reducción de los costes de producción de la industria de A, como consecuencia de la disminución de la producción que resulta de la aplicación de un arancel menor. El área B representa el aumento del excedente de los consumidores del país A como resultado del aumento de su demanda (por el menor precio del producto) y la reducción del precio desde $T_A$ hasta CET. El área D representa la diferencia entre el aumento del beneficio de los productores de B y la disminución del excedente de los consumidores de B, como resultado del aumento del precio en B desde $T_B$ hasta CET. El área roja C corresponde a la pérdida de bienestar que resulta de la disminución de la recaudación arancelaria en A tras integrar el mercado y pasar a importar producto no sujeto a arancel proveniente de B. Éste área roja corresponde a la diversión de comercio: una reducción de excedente que resulta de que deje de comerciarse con algún país como resultado de la integración. En la medida en que la suma de las áreas A, B y D sea más grande que el área C, el efecto creación de comercio predomina sobre el efecto desviación y la integración aparecerá como deseable.



\begin{axis}{4}{Modelo de Krugman (1991) del núcleo y la periferia. Diagrama de fase de la dinámica con costes de transporte elevados que resultan en un equilibrio con la población de obreros dispersada.}{}{$\dot{\lambda}$}{krugman91dispersion}
	% Extensión del eje de ordenadas hacia abajo y de abscisas hacia la derecha
	\draw[-] (0,0) -- (0,-4);
	\draw[-] (4,0) -- (8,0);
	\node[below] at (8.4,0){$\lambda$};	
	% Límite con lambda = 1
	\node[below] at (8,-0.2){$1$};
	\draw[-] (8,0.2) -- (8,-0.2);
	% Centro con lambda = 0.5
	\node[below] at (4,-0.3){0,5};	
	\draw[-] (4,0.2) -- (4,-0.2);

	% Dinámica de dot{lambda}
	\draw[-] (0,3) to [out=10, in=100](4,0) to [out=-80, in=200](8,-3);

	% Flechas de tendencia
	% Hacia la derecha desde origen
	\draw[-{Latex}] (0.5,0.5) -- (3.5,0.5);

	% Hacia la izquierda desde límite derecho
	\draw[-{Latex}] (7.5,-0.5) -- (4.5,-0.5);

\end{axis}

La variable $\lambda$ representa la concentración de la población de trabajadores obreros móviles en un uno de los países en cuestión.


\begin{axis}{4}{Modelo de Krugman (1991) del núcleo y la periferia. Diagrama de fase de la dinámica con costes de transporte intermedios que resultan en múltiples equilibrios posibles.}{}{$\dot{\lambda}$}{krugman91multiplesequilibrios}
	% Extensión del eje de ordenadas hacia abajo y de abscisas hacia la derecha
	\draw[-] (0,0) -- (0,-4);
	\draw[-] (4,0) -- (8,0);
	\node[right] at (8.4,0){$\lambda$};	
	% Límite con lambda = 1
	\node[below] at (8,-0.2){$1$};
	\draw[-] (8,0.2) -- (8,-0.2);
	% Centro con lambda = 0.5
	\node[below] at (4,-0.3){0,5};	
	\draw[-] (4,0.2) -- (4,-0.2);

	% Dinámica de dot{lambda}
	\draw[-] (0,-4) to [out=80,in=180](3,1) to [out=0, in=180](5,-1) to [out=0, in=260](8,4);

	% Flechas de tendencia
	% Hacia límite izquierdo, aglomeración
	\draw[-{Latex}] (1.4,0.5) -- (0.2,0.5);
	% Hacia centro desde izquierda, dispersión
	\draw[-{Latex}] (1.8,0.5) -- (3.6,0.5);
	% Hacia centro desde derecha, dispersión
	\draw[-{Latex}] (6.8,0.5) -- (4.4,0.5);
	% Hacia derecha, aglomeración	
	\draw[-{Latex}] (7.2,0.5) -- (8,0.5);

\end{axis}


\begin{axis}{4}{Modelo de Krugman (1991) del núcleo y la periferia. Diagrama de fase de la dinámica con costes de transporte reducidos pero presentes: el equilibrio tiende a la aglomeración.}{}{$\dot{\lambda}$}{krugman91aglomeracion}
	% Extensión del eje de ordenadas hacia abajo y de abscisas hacia la derecha
	\draw[-] (0,0) -- (0,-4);
	\draw[-] (4,0) -- (8,0);
	\node[right] at (8.4,0){$\lambda$};	
	% Límite con lambda = 1
	\node[below] at (8,-0.2){$1$};
	\draw[-] (8,0.2) -- (8,-0.2);
	% Centro con lambda = 0.5
	\node[below] at (4,-0.3){0,5};	
	\draw[-] (4,0.2) -- (4,-0.2);

	
	% Dinámica de dot{lambda}
	\draw[-] (0,-4) to [out=80, in=260](8,4) ;

	% Flechas de tendencia
	\draw[-{Latex}] (3.7,0.5) -- (0.3,0.5);
	\draw[-{Latex}] (4.3, -0.5) -- (7.7,-0.5);

\end{axis}


\conceptos

\concepto{Creación de comercio}

Cuando la creación de una unión aduanera induce la aparición de un flujo comercial que antes no existía, de tal manera que el bien en cuestión se producía localmente, tiene lugar el fenómeno de la \textit{creación de comercio}. El país donde se produce el bien y de donde se importa tras la puesta en funcionamiento de la unión aduanera no tiene por qué ser más eficiente que los productores del resto del mundo para que se produzca una ganancia neta de eficiencia debida a la reducción de costes de producción. Así, la creación de comercio es deseable aunque el nuevo país productor del bien y exportador dentro de la unión aduanera se encuentre efectivamente protegido de productores más eficientes del resto del mundo.

\concepto{Desviación de comercio}

La integración comercial provoca la desaparición de un intercambio comercial que antes sí tenía lugar, desplazándose la producción hacia una economía dentro de la unión aduanera que produce el bien en cuestión de manera más costosa que como se producía antes de implementar la unión aduanera.

\concepto{GATT-47 e integración económica: artículo V}

<<1. The provisions of this Agreement shall apply to the metropolitan
customs territories of the contracting parties and to any other customs
territories in respect of which this Agreement has been accepted under
Article XXVI or is being applied under Article XXXIII or pursuant to the
Protocol of Provisional Application. Each such customs territory shall,
exclusively for the purposes of the territorial application of this
Agreement, be treated as though it were a contracting party; Provided that
the provisions of this paragraph shall not be construed to create any rights
or obligations as between two or more customs territories in respect of
which this Agreement has been accepted under Article XXVI or is being
applied under Article XXXIII or pursuant to the Protocol of Provisional
Application by a single contracting party.

2. For the purposes of this Agreement a customs territory shall be
understood to mean any territory with respect to which separate tariffs or
other regulations of commerce are maintained for a substantial part of the
trade of such territory with other territories.

3. The provisions of this Agreement shall not be construed to
prevent:
(a) Advantages accorded by any contracting party to adjacent
countries in order to facilitate frontier traffic;
(b) Advantages accorded to the trade with the Free Territory of
Trieste by countries contiguous to that territory, provided that
such advantages are not in conflict with the Treaties of Peace
arising out of the Second World War.

4. The contracting parties recognize the desirability of increasing
freedom of trade by the development, through voluntary agreements, of
closer integration between the economies of the countries parties to such
agreements. They also recognize that the purpose of a customs union or of
a free-trade area should be to facilitate trade between the constituent
territories and not to raise barriers to the trade of other contracting parties
with such territories.

5. \textbf{Accordingly, the provisions of this Agreement shall not prevent,
as between the territories of contracting parties, the formation of a
customs union or of a free-trade area or the adoption of an interim
agreement necessary for the formation of a customs union or of a free-
trade area; Provided that:} (a) with respect to a \underline{customs union}, or an interim agreement leading to a formation of a customs union, the duties and other
regulations of commerce imposed at the institution of any such
union or interim agreement in respect of trade with contracting
parties not parties to such union or agreement shall not on the
whole be higher or more restrictive than the general incidence of
the duties and regulations of commerce applicable in the
constituent territories prior to the formation of such union or the
adoption of such interim agreement, as the case may be; (b) with respect to a \underline{free-trade area}, or an interim agreement leading
to the formation of a free-trade area, the duties and other
regulations of commerce maintained in each if the constituent
territories and applicable at the formation of such free-trade area
or the adoption of such interim agreement to the trade of
contracting parties not included in such area or not parties to
such agreement shall not be higher or more restrictive than the
corresponding duties and other regulations of commerce existing
in the same constituent territories prior to the formation of the
free-trade area, or interim agreement as the case may be; and
(c) any interim agreement referred to in sub-paragraphs (a) and (b)
shall include a plan and schedule for the formation of such a
customs union or of such a free-trade area within a reasonable
length of time.

6. If, in fulfilling the requirements of sub-paragraph 5 (a), a
contracting party proposes to increase any rate of duty inconsistently with
the provisions of Article II, the procedure set forth in Article XXVIII shall
apply. In providing for compensatory adjustment, due account shall be
taken of the compensation already afforded by the reduction brought
about in the corresponding duty of the other constituents of the union.

7. (a) Any contracting party deciding to enter into a customs
union or free-trade area, or an interim agreement leading to the formation
of such a union or area, shall promptly notify the CONTRACTING PARTIES
and shall make available to them such information regarding the
proposed union or area as will enable them to make such reports and
recommendations to contracting parties as they may deem appropriate.
(b) If, after having studied the plan and schedule included in an
interim agreement referred to in paragraph 5 in consultation with the
parties to that agreement and taking due account of the information made available in accordance with the provisions of sub-paragraph (a), the CONTRACTING PARTIES find that such agreement is not likely to result in the formation of a customs union or of a free-trade area within the period contemplated by the parties to the agreement or that such period is not a reasonable one, the CONTRACTING P ARTIES shall make recommendations to the parties to the agreement. The parties shall not maintain or put into force, as the case may be, such agreement if they are not prepared to modify it in
accordance with these recommendations. (c) Any substantial change in the plan or schedule referred to inparagraph 5 (c) shall be communicated to the CONTRACTING PARTIES, which may request the contracting parties concerned to consult with them if the change seems likely to jeopardize or delay unduly the formation of the customs union or of the free-trade area.

8.
For the purposes of this Agreement:
(a) A customs union shall be understood to mean the substitution of
a single customs territory for two or more customs territories, so
that (i) duties and other restrictive regulations of commerce (except,
where necessary, those permitted under Articles XI, XII, XIII,
XIV, XV and XX) are eliminated with respect to substantially
all the trade between the constituent territories of the union
or at least with respect to substantially all the trade in
products originating in such territories, and, (ii) subject to the provisions of paragraph 9, substantially the
same duties and other regulations of commerce are applied
by each of the members of the union to the trade of
territories not included in the union;
(b) A free-trade area shall be understood to mean a group of two or
more customs territories in which the duties and other restrictive
regulations of commerce (except, where necessary, those
permitted under Articles XI, XII, XIII, XIV, XV and XX) are
eliminated on substantially all the trade between the constituent
territories in products originating in such territories.

9. The preferences referred to in paragraph 2 of Article I shall not be
affected by the formation of a customs union or of a free-trade area but
may be eliminated or adjusted by means of negotiations with contracting
parties affected.* This procedure of negotiations with affected contracting
parties shall, in particular, apply to the elimination of preferences required
to conform with the provisions of paragraph 8 (a)(i) and paragraph 8 (b).

10. The CONTRACTING P ARTIES may by a two-thirds majority
approve proposals which do not fully comply with the requirements of
paragraphs 5 to 9 inclusive, provided that such proposals lead to the
formation of a customs union or a free-trade area in the sense of this
Article.

11. Taking into account the exceptional circumstances arising out of
the establishment of India and Pakistan as independent States and
recognizing the fact that they have long constituted an economic unit, the
contracting parties agree that the provisions of this Agreement shall not
prevent the two countries from entering into special arrangements with
respect to the trade between them, pending the establishment of their
mutual trade relations on a definitive basis.

12. Each contracting party shall take such reasonable measures as
may be available to it to ensure observance of the provisions of this
Agreement by the regional and local governments and authorities within
its territories.>>

\preguntas

\seccion{Test 2018}

\textbf{28.} Señale cuál será el efecto neto del establecimiento de una unión aduanera entre dos países.

\begin{itemize}
	\item[a] El efecto neto será siempre positivo, pues la eliminación de los aranceles existentes generará un efecto creación de comercio.
	\item[b] El efecto neto será siempre negativo, ya que desaparece el ingreso arancelario.
	\item[c] El efecto podrá ser positivo o negativo.
	\item[d] El efecto neto podrá ser positivo o nulo, pero nunca negativo.
\end{itemize}

\seccion{Test 2017}
\textbf{28.} La creación de comercio requiere que:

\begin{itemize}
	\item[a] El productor mundial más eficiente de un bien no sea miembro de la unión aduanera.
	\item[b] El productor mundial más eficiente del bien debe ser miembro de la unión aduanera.
	\item[c] La producción se desplace de un país no miembro a un miembro.
	\item[d] La producción del bien se desplace al país miembro más eficiente.
\end{itemize}

\seccion{Test 2015}
\textbf{30.} En un proceso de integración económica, un país A decide llevar a cabo una zona de libre comercio o unión aduanera con otro país B que es más eficiente en la producción de un determinado bien X pero menos que un país tercero C. Tras la integración, todas las importaciones del país A proceden del país socio B. En estas condiciones (señale la \underline{falsa}):

\begin{itemize}
	\item[a] Cuanto más elevadas sean la elasticidad de la demanda y la elasticidad de la oferta en un país que va a integrarse en una unión aduanera, mayor será la creación de comercio. 
	\item[b] Kemp y Wan mostraron cómo era posible elegir un arancel exterior común de tal forma que el resultado final fuese una mejora para los países integrados que no empeorase a los que no formaban parte del acuerdo.
	\item[c] Los efectos beneficiosos serán mayores cuanto menores hayan sido los aranceles previos entre los países que se unen ya que la creación de comercio será mayor.
	\item[d] Cuando los países que se unen tienen economías que rivalizan entre sí, la creación de comercio y el beneficio común será mayor.
\end{itemize}

\seccion{Test 2013}
\textbf{33.} En una unión aduanera la amplitud del efecto creación de comercio depende de:

\begin{itemize}
	\item[a] que se produzca el efecto desviación de comercio en el país que entra a formar parte de la Unión aduanera
	\item[b] que se produzca el efecto cambio en la dirección del comercio en el país que entra a formar parte en la unión aduanera.
	\item[c] la elasticidad precio de las curvas de oferta y demanda del bien de importación analizado en el país que entra a forma parte en la unión aduanera.
	\item[d] Ninguna de las anteriores.
\end{itemize}

\seccion{Test 2009}
\textbf{27.} Cuál de las siguientes afirmaciones respecto a las teorías de las uniones aduaneras es \textbf{CORRECTA}:

\begin{itemize}
	\item[a] La teoría tradicional de las uniones aduaneras de Viner, compara dos situaciones subóptimas por lo que, a priori, no es posible conocer si la creación de una unión aduanera es deseable o no.
	\item[b] Cooper y Massell concluyen que, bajo determinadas condiciones, la reducción unilateral y no discriminatoria de los aranceles puede ser preferible para un país a integrarse en una unión aduanera.
	\item[c] Wonnacott y Wonnacott defienden que integrarse en una unión aduanera puede ser preferible a la reducción unilateral y no discriminatoria de aranceles si permite que los países que pasan a formar parte de aquélla actúen como un país grande frente al resto del mundo y mejoran sus términos de intercambio y bienestar.
	\item[d] Todas las anteriores.
\end{itemize}

\seccion{Test 2004}
\textbf{26.} De acuerdo con el análisis tradicional de la integración comercial en una unión aduanera, en condiciones de competencia perfecta y equilibrio parcial, debido a J. Viner, lo deseable para el país que se integra sería que:

\begin{itemize}
	\item[a] Se produjera creación de comercio (esto es, la aparición de nuevas importaciones procedentes de los nuevos socios, que anteriormente no se llevaban a cabo), pero no se produjera desviación de comercio (esto es, la aparición de nuevas importaciones procedentes de los nuevos socios, que anteriormente se adquirían al resto del mundo).
	\item[b] Se produjera tanto creación de comercio (esto es, la aparición de nuevas importaciones procedentes de los nuevos socios, que anteriormente no se llevaban a cabo), como desviación de comercio (esto es, la aparición de nuevas importaciones procedentes de los nuevos socios, que anteriormente se adquirían al resto del mundo).
	\item[c] Se produjera creación de comercio (esto es, la aparición de nuevas importaciones procedentes de los nuevos socios, que anteriormente se adquirían al resto del mundo), pero no se produjera desviación de comercio (esto es, la aparición de nuevas importaciones procedentes de los nuevos socios, que anteriormente no se llevaban a cabo).
	\item[d] Se produjera tanto creación de comercio (esto es, la aparición de nuevas importaciones procedentes de los nuevos socios, que anteriormente se adquirían al resto del mundo), como desviación de comercio (esto es, la aparición de nuevas importaciones procedentes de los nuevos socios, que anteriormente no se llevaban a cabo).
\end{itemize}

\notas

\textbf{2018:} \textbf{28.} c

\textbf{2017:} \textbf{10.} D

\textbf{2015:} \textbf{30.} C

\textbf{2013:} \textbf{33.} C

\textbf{2009:} \textbf{27.} D

\textbf{2004:} \textbf{26.} A

\bibliografia

Mirar en Palgrave:
\begin{itemize}
	\item comparative advantage *
	\item customs unions *
	\item economic integration *
	\item effective protection
	\item Euro
	\item European Union (EU) trade policy *
	\item European Union Single Market: Design and Development *
	\item European Union Single Market: Economic Impact *
	\item genuine economic and monetary union *
	\item Heckscher-Ohlin trade theory
	\item international outsourcing
	\item Mercosur
	\item North-American Free Trade Agreement (NAFTA)
	\item optimal tariffs
	\item regional and preferential trade agreements *
	\item tariffs
	\item terms of trade
	\item The European Union's Common Agricultural Policy (CAP)
	\item theory of Economic Integration: a review *
	\item tradable and non-tradable commodities
	\item trade policy, political economy of *
	\item trade costs
\end{itemize}

Jovanović, M. N. \textit{The Economics of International Integration} (2006) Elgar Publishing -- En carpeta Economía Internacional

Krugman, P. R. (1991) \textit{Increasing Returns and Economic Geography} Journal of Political Economy, Vol. 99, No. 3 -- En carpeta del tema

Rodrik, D. (2000) \textit{How Far Will International Economoic Integration Go} Journal of Economic Perspectives. Winter 2000. -- En carpeta del tema

UNCTAD (2019) \textit{World Investment Report} -- En carpeta del tema

Viner, Jacob, (2014). \textit{The Customs Union Issue, Oxford University Press}, \url{https://EconPapers.repec.org/RePEc:oxp:obooks:9780199756124}

Wonnacott P.; Wonnacott, R. \textit{Is Unilateral Tariff Reduction Preferable to a Customs Union? The curious case of the missing foreing tariffs} (1981) American Economic Review 

\end{document}
