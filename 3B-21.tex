\documentclass{nuevotema}

\tema{3B-21}
\titulo{El Sistema Económico Internacional hasta la ruptura del sistema de Bretton-Woods. }

\begin{document}

\ideaclave

\seccion{Preguntas clave}
\begin{itemize}
	\item ¿Qué es el sistema económico internacional?
	\item ¿Cómo evolucionó hasta la ruptura de Bretton Woods?
	\item ¿Qué eventos principales tuvieron lugar?
	\item ¿En qué contexto se produjeron los grandes cambios?
	\item ¿Qué consecuencias se derivaron?
\end{itemize}

\esquemacorto

\begin{esquema}[enumerate]
	\1[] \marcar{Introducción}
		\2 Contexto
			\3 Sistema económico y monetario internacional
			\3 Mercados internacionales de capital
			\3 Tendencias de largo plazo hasta caída de Bretton Woods
		\2 Objeto
			\3 ¿Qué es el sistema económico internacional?
			\3 ¿En qué consistía el patrón oro?
			\3 ¿Cómo evolucionó hasta la ruptura de Bretton Woods?
			\3 ¿Qué eventos principales tuvieron lugar?
			\3 ¿En qué contexto se produjeron los grandes cambios?
			\3 ¿Qué consecuencias se derivaron?
		\2 Estructura
			\3 Antecedentes
			\3 El Patrón Oro
			\3 Entreguerras
			\3 Bretton Woods y posguerra
			\3 Caída de Bretton Woods
	\1 \marcar{Antecedentes}
		\2 Contexto
			\3 Económico
			\3 Político
			\3 Teórico
		\2 Eventos
			\3 Primeras formas de dinero
			\3 Dinero metálico
			\3 Predominio de la plata
			\3 Uso creciente del oro
			\3 Bimetalismo
			\3 Situación a principios del siglo XIX
			\3 Unión Monetaria Latina
			\3 Caída del bimetalismo
		\2 Consecuencias
			\3 Emergencia de sistema monetario basado en oro
			\3 Deflación en 1870s y 1880s
	\1 \marcar{El patrón oro}
		\2 Contexto
			\3 Periodo
			\3 Económico
			\3 Político
			\3 Teórico
		\2 Eventos
			\3 Funcionamiento del patrón oro
			\3 Reglas del juego
			\3 Credibilidad de los bancos centrales
			\3 Pánicos bancarios
			\3 Cooperación internacional
			\3 Inestabilidad en la periferia
			\3 Guerra Mundial y suspensión de convertibilidad
		\2 Consecuencias
			\3 Integración de la economía mundial
			\3 Liberalización y crecimiento
			\3 Fin de patrón oro
	\1 \marcar{Entreguerras}
		\2 Contexto
			\3 Económico
			\3 Político
			\3 Teórico
		\2 Eventos
			\3 Restablecimiento de la convertibilidad
			\3 Fuentes de inestabilidad
			\3 Gran Depresión
			\3 Abandonos de convertibilidad
		\2 Consecuencias
			\3 Inestabilidad cambiaria
			\3 Descoordinación de política monetaria
			\3 Cooperación a finales de 30
			\3 Aumento de proteccionismo
	\1 \marcar{Bretton Woods y posguerra}
		\2 Contexto
			\3 Económico
			\3 Político
			\3 Teórico
		\2 Eventos
			\3 Negociaciones de Bretton Woods
			\3 Creación de instituciones financieras multilaterales
			\3 Plan Marshall
			\3 Sistema de Bretton Woods
			\3 Devaluaciones iniciales en Europa
			\3 Escasez de dólares
			\3 Déficits crecientes en USA
			\3 Medidas para frenar flujos desestabilizantes
		\2 Consecuencias
			\3 Estabilidad cambiaria
			\3 Buen funcionamiento de sistema de pagos
			\3 Sin deflaciones
			\3 Restricciones de capital
			\3 Éxito relativo de FMI
			\3 Generación endógena de liquidez
	\1 \marcar{Caída de Bretton Woods}
		\2 Contexto
			\3 Económico
			\3 Político
			\3 Teórico
		\2 Eventos
			\3 Desequilibrios previos
			\3 Ventas masivas de dólares en 1971
			\3 Nixon Shock (1971)
			\3 Acuerdos de Smithsonian en Washington (1971)
			\3 Crisis del petróleo del 73
			\3 Abandono definitivo en Acuerdos de Jamaica (1976)
		\2 Consecuencias
			\3 Política monetaria como instrumento de PEconómica
			\3 Regímenes cambiarios polares
			\3 Volatilidad de tipos de cambio
			\3 Impulso a integración europea
	\1[] \marcar{Conclusión}
		\2 Recapitulación
			\3 Antecedentes
			\3 El patrón oro
			\3 Entreguerras
			\3 Bretton Woods y posguerra
			\3 Caída de Bretton Woods
		\2 Idea final
			\3 Influencia del pasado reciente
			\3 Política, economía y sistema monetario
			\3 Reminiscencias del pasado en la actualidad

\end{esquema}

\esquemalargo













\begin{esquemal}
	\1[] \marcar{Introducción}
		\2 Contexto
			\3 Sistema económico y monetario internacional
				\4 Sistema económico internacional
				\4[] Conjunto de:
				\4[] $\to$ Relaciones comerciales
				\4[] $\to$ Marco institucional y legal
				\4[] $\to$ Flujos financieros
				\4[] $\then$ Entre economías mundiales
				\4 Sistema monetario
				\4[] Componente elemental de sistema económico internacional
				\4[] Conjunto de instituciones y flujos financieros que permiten
				\4[] $\to$ Solucionar desequilibrios de balanza de pagos
				\4[] $\to$ Acceso a crédito
				\4[] $\to$ Canalizar pagos en divisas
				\4[] $\then$ Para aprovechar ganancias del comercio int.
				\4[] $\then$ Para suavizar patrón de consumo intertemporal
			\3 Mercados internacionales de capital
				\4 Pilar central de sistema monetario
				\4 Historia de sistema monetario y económico
				\4[] $\to$ Íntimamente ligado a hª mercados de capital
				\4 Diferentes regímenes de mercados de capital
				\4[] Ayudan a distinguir fases de evolución SEInternacional
				\4[] $\to$ Exposición dividida en fases
			\3 Tendencias de largo plazo hasta caída de Bretton Woods
				\4 Avances tecnológicos
				\4[] $\to$ Transporte: tren, barco, avión, combustión
				\4 Comunicaciones
				\4[] Telégrafo, teléfono
				\4[] $\to$ Salto cualitativo en trans. de información
				\4 Sindicatos
				\4[] Creciente tasa de sindicación
				\4[] Movimientos obreros en todo el mundo
				\4[] Cada vez mayor rigidez salarial
				\4 Protección social
				\4[] Incipiente desarrollo de sistemas en Occidente
				\4[] Papel creciente del estado en economía
				\4[] $\to$ Extremada en economías planificadas
				\4 Laissez faire vs planificación
				\4[] Tensión creciente
				\4[] Ambas se expanden en sus ámbitos de influencia
		\2 Objeto
			\3 ¿Qué es el sistema económico internacional?
			\3 ¿En qué consistía el patrón oro?
			\3 ¿Cómo evolucionó hasta la ruptura de Bretton Woods?
			\3 ¿Qué eventos principales tuvieron lugar?
			\3 ¿En qué contexto se produjeron los grandes cambios?
			\3 ¿Qué consecuencias se derivaron?
		\2 Estructura
			\3 Antecedentes
			\3 El Patrón Oro
			\3 Entreguerras
			\3 Bretton Woods y posguerra
			\3 Caída de Bretton Woods
	\1 \marcar{Antecedentes}
		\2 Contexto
			\3 Económico
				\4 Economía mundial fragmentada
				\4[] Grandes ``bloques regionales''
				\4[] Poco comercio entre ambos
				\4[] $\to$ Aunque existían algunas rutas relevantes
				\4[] $\to$ P. ej.: ruta de la seda en todas sus variantes
				\4 Avances en navegación
				\4[] Permiten comercio de larga distancia
				\4 Descubrimiento de América
				\4[] Comienzo de globalización
				\4[] Asia y América conectados con Europa
				\4 Revolución industrial
				\4[] Enorme aumento de la producción
				\4[] Comienza gran divergencia
				\4[] $\to$ Europa y satélites vs resto del mundo
				\4[] $\to$ China deja de ser principal economía mundial
				\4 Liberalización y comercio a partir de 1860s
				\4[] Bajadas generalizadas de aranceles
				\4[] Costes de transporte caen fuertemente
				\4[] Aumentan transacciones con exterior
				\4[] $\to$ Circulación de moneda extranjera
			\3 Político
				\4 Imperios como factores de estabilidad
				\4 Frecuentes guerras entre imperios/bloques
			\3 Teórico
				\4 Escolástica medieval
				\4 Escuela de Salamanca a comienzos del periodo moderno
				\4 Mercantilismo en s. XVI y XVII en toda Europa
				\4 Críticas al mercantilismo
				\4[] En s. XVII en Inglaterra: William Petty y otros
				\4[] Ens. XVIII en Francia: fisiócratas
				\4 Mecanismo flujo-especie
				\4[] David Hume
				\4[] Déficit CC resulta de desequilibrio metal precioso
				\4[] $\to$ Metal fluye hasta ajustar desequilibrios
				\4 Adam Smith
				\4[] Comienzo de economía como tal
				\4[] Germen de casi todos programas de investigación
				\4[] Enorme impacto sobre pensamiento económico en Europa
		\2 Eventos
			\3 Primeras formas de dinero
				\4 Conchas, piedras, marfil...
			\3 Dinero metálico
				\4 Metal precioso sirve como dinero
				\4[] Desde tiempo muy antiguo
				\4 Imperio romano, China...
			\3 Predominio de la plata
				\4 Plata es metal monetario dominante
				\4[] Cobre demasiado pesado
				\4[] Oro demasiado ligero
				\4[] $\to$ Algunas excepciones
				\4[] $\to$ Suecia: estándar de cobre
			\3 Uso creciente del oro
				\4 A lo largo de edad media, aumenta su uso
				\4 Italia en siglo XIII
				\4[] Oro para saldar grandes transacciones
				\4 Uso generalizado progresivo
				\4[] $\to$ Para saldar grandes cuentas
				\4[$\then$] Mezcla de metales de uso general
				\4[] Plata, cobre, oro
			\3 Bimetalismo\footnote{Ver \href{https://www.nber.org/papers/w20852.pdf}{Meissner (2015)}, Flandreau (1996) y \textit{bimetallism} en Palgrave}
				\4 Idea clave
				\4[] Sistema monetario basado en dos monedas
				\4[] $\to$ Generalmente, plata y oro
				\4[] Monedas de plata y oro son curso legal
				\4[] $\to$ Sirven para saldar deudas
				\4[] Cecas acuñan moneda de oro o plata libremente
				\4[] $\to$ Contra presentación de oro o plata
				\4[] $\to$ Relación de intercambio oro/plata fija
				\4[] Valor nominal igual a valor metálico
				\4[] $\to$ Para ambas monedas
				\4[] $\to$ Aunque puede existir moneda representativa
				\4 Ventajas del bimetalismo
				\4[] I. Medio de intercambio para amplio rango de transacciones
				\4[] II. Límites a manipulación de oferta monetaria
				\4[] $\to$ De forma similar a monometalismo
				\4[] $\to$ Límite superior a OMonetaria mayor
				\4[] III. Estabilización de shocks en países monometálicos
				\4[] $\to$ $\uparrow$ de ROP\footnote{Ratio Oro-Plata o unidades de plata por una de oro.} induce $\uparrow$ de oro desde bimetálico
				\4[] $\then$ Estabilización automática de precio de oro
				\4[] IV. Estabilización del tipo de cambio
				\4[] $\to$ Por mecanismo anterior de estabilización
				\4[] $\then$ País bimetálico también se beneficia
				\4 Inestabilidad
				\4[] Mecanismo de estabilización puede volverse inestable
				\4[] Déficits de CC sostenidos o divergencia excesiva ROP
				\4[] $\to$ Salida excesiva de metal infravalorado en ceca
				\4[] $\then$ Agotamiento de metal infravalorado y persiste divergencia
				\4[] $\then$ Monometalismo de metal sobrevalorado
			\3 Situación a principios del siglo XIX
				\4 Economías con oro
				\4[] Gran Bretaña, plenamente
				\4[] $\to$ Newton infravaloró plata en 1717
				\4[] $\then$ Plata se exportó para comercio internacional
				\4[] $\then$ Oro desplazó a plata para comercio nacional
				\4[] Monometalismo con oro desde 1821
				\4 Economías con plata
				\4[] Imperio Austro-Húngaro
				\4[] Escandinavia
				\4[] Rusia
				\4[] Lejano Oriente
				\4[] América
				\4 Países bimetálicos
				\4[] Francia
				\4[] $\to$ Principal economía bimetálica
				\4[] $\to$ Vínculo entre países con oro y plata
				\4[] $\then$ Papel estabilizador de ROP internacional
				\4[] Ratio Oro-Plata (ROP) en Francia
				\4[] $\to$ Aproximadamente, = precio que mercado internacional
				\4[] Holanda
				\4[] Bélgica
				\4[] Estados Unidos
				\4[] $\to$ Principal fuente de shock de oferta
				\4[] $\to$ Descubrimientos de oro (1848) y plata (1859)
			\3 Unión Monetaria Latina
				\4 Acuerdo de 1865 para estandarizar unidades
				\4[] Monedas de miembros son equivalentes
				\4[] Misma proporción de oro y plata
				\4[] Relación plata-oro 15,5:1
				\4[] $\then$ Circulación libre en toda la unión
				\4 Miembros
				\4[] Francia, Bélgica, Italia, Suiza
				\4[] $\to$ Posteriormente Grecia y otros
				\4 Problemas
				\4[] Países reducen cantidad de oro en monedas
				\4[] $\to$ Pero intercambiables a ROP fijo
				\4[] $\then$ Otros países se ven obligados a hacer lo mismo
				\4 Relativo éxito
				\4[] Monedas de la UML circulan en toda la Unión
				\4[] Fluctuaciones entre oro y plata
				\4[] Incentivos a freeriding y degradación de la moneda
				\4[] $\to$ Emitir monedas con menos metal de lo acordado
				\4[] $\then$ Financiación de déficits por vía monetaria
				\4 Especificaciones sobre:
				\4[] Oro y plata correspondiente a valor facial de monedas
				\4[] $\then$ ROP\footnote{Relación Oro-Plata}
				\4[$\then$] Causa de UML y desaparición
				\4[] Acuerdan UML para reducir problema de devaluaciones
				\4[] Devaluaciones unilaterales destruyen
				\4[] $\to$ Estados papales y Grecia especialmente
				\4[] Estallido de IGM
				\4[] $\to$ Fin de UML de facto
				\4[] 1927: fin de UML de iure
			\3 Caída del bimetalismo\footnote{Ver \href{https://www.nber.org/papers/w20852.pdf}{Meissner (2015).} }
				\4 Aumento de ROP en años 30\footnote{Es decir, la plata perdió valor frente al oro.}
				\4[] $\to$ Cae valor de la plata
				\4[] $\then$ Sobrevalorada en Francia
				\4[] $\then$ Entra plata en Francia
				\4[] $\then$ Sale oro de Francia
				\4 Descubrimiento de oro en California (1848)
				\4[] Caída de ROP
				\4[] $\to$ Aumenta valor de la plata
				\4[] Plata infravalorada en Francia
				\4[] $\then$ Fluye hacia Lejano Oriente
				\4 Descubrimiento de plata en Nevada (1859)
				\4[] Aumento de ROP
				\4[] Oro infravalorado en Francia
				\4[] $\to$ Influjo de plata hacia Francia
				\4[$\then$] Muy fuertes oscilaciones en circulación
				\4[$\then$] Descontento con sistema bimetálico
				\4 Mantenimiento de bimetalismo
				\4[] En la medida en que había países en plata u oro
				\4[] Efectos de red hacen posible bimetalismo
				\4[] $\to$ Beneficios por utilizar ambos
				\4[] $\then$ A pesar de fluctuaciones
				\4 Gran Bretaña crece más rápido que otros
				\4[] Incentivos a adoptar oro para comerciar con Gran Bretaña
				\4 Caída definitiva de bimetalismo
				\4[] Insatisfacción creciente con bimetalismo
				\4[] Alemania cambia a monometálico con oro en 1871
				\4[] $\then$ Reacción en cadena
				\4[] $\then$ Externalidades de red aumentan ventaja de oro
				\4[] Otros países cambian a oro
				\4[] $\to$ Dinamarca, Holanda, Suecia, Noruega, UML
				\4[] Francia prohíbe acuñar monedas de plata en 1873
				\4[] $\to$ Evitar que todo oro salga y entre plata
				\4[] $\then$ Evitar efectos negativos de Ley de Gresham
				\4[] EEUU restablece convertibilidad tras guerra civil
				\4[] $\to$ En 1879, ya efectivamente en oro
				\4[] A finales del XIX
				\4[] $\to$ EEUU ya en oro aunque cierta oposición interna\footnote{Como resultado de la deflación, el partido demócrata --representante en aquel tiempo de los intereses agrícolas- promueve la vuelta al bimetalismo y llega a instaurar la compra obligatoria de plata por parte de la reserva federal para aumentar la oferta monetaria, frenar la deflación y permitir una disminución del valor real de las deudas que estaba ahogando a los agricultores endeudados.}
				\4[] $\to$ Rusia y Japón en oro
				\4[] $\to$ India en oro
				\4[] $\to$ Hispanoamérica cambia a oro
				\4[] $\then$ Sólo España se mantiene en papel inconvertible
				\4[] $\then$ Sólo China y Centroamérica en plata
		\2 Consecuencias
			\3 Emergencia de sistema monetario basado en oro
				\4 Caída de bimetalismo es ascenso monometalismo
			\3 Deflación en 1870s y 1880s
				\4 Oferta de oro no sigue el ritmo de crecimiento
				\4 Deflación generalizada durante una década
				\4 Oro se mantiene a pesar de sesgo deflacionario
				\4[] Externalidades de red demasiado poderosas
				\4[] Un sólo país que acepte acuñar plata
				\4[] $\to$ Se arriesga a que todo el oro salga si infravalorado
				\4[] $\then$ Se expone a fluctuaciones del TC
				\4[] $\then$ Aumentan efectos desfavorables respecto a otros
	\1 \marcar{El patrón oro}\footnote{Ampliable con Bordo y Schwartz (1984).}
		\2 Contexto
			\3 Periodo
				\4 Década de 1870s hasta inicio de la IGM
				\4[] Suspensión de convertibilidad
			\3 Económico
				\4 Shocks de productividad positivos
				\4[] En muchos sectores industriales
				\4[] Industrialización acelerada
				\4 Deflación generalizada
				\4[] Output crece enormemente
				\4[] Oferta monetaria no sigue ritmo
				\4[] Plata ya no circula
				\4 Aumento del comercio internacional
				\4[] Liberalización generalizada
				\4[] Circula moneda extranjera
				\4 Países centrales del sistema económico mundial
				\4[] Gran Bretaña
				\4[] Francia
				\4[] Alemania
				\4[] Estados Unidos
				\4 Reservas internacionales
				\4[] Aparecen como instrumento habitual
				\4[] Pagan interés
				\4[] Países mantienen depósitos o deuda en Londres
				\4[] $\to$ Respaldan moneda nacional con divisas
				\4[] Otros mantienen saldos en París y Berlín
				\4 Circulación de papel moneda convertible en oro
				\4[] Convertible a tasa fija
				\4[] Ampliamente en toda Europa y América
				\4[] Instrumento habitual de transacción
				\4[] Salvo situaciones excepcionales, libre conversión
				\4[] $\then$ EEUU en Guerra Civil
				\4[] $\then$ España
			\3 Político
				\4 Presión pro-plata en Estados Unidos
				\4 Aparición progresiva de socialismo y sindicatos
				\4 Relativa estabilidad en Europa
			\3 Teórico
				\4 Marginalistas y neoclásicos
				\4 Desarrollos del mecanismo de flujo-especie
				\4[] Mill analiza papel de interés
				\4[] $\to$ Como herramienta para gestionar reservas
				\4 Desarrollo teoría de banca central
				\4[] Henry Thornton, debates currency vs banking school
		\2 Eventos
			\3 Funcionamiento del patrón oro
				\4 Varios tipos de patrón oro
				\4 Patrón oro puro
				\4[] Oro es único dinero
				\4 Patrón-oro mixto
				\4[] Oro circula junto a papel moneda
				\4 Patrón oro de lingotes/bullion gold standard
				\4[] Oro no circula en transacciones internas
				\4[] Compra de lingotes posible en banco central
				\4[] $\to$ Para cualquier agente con papel moneda
				\4 Patrón oro-cambio o patrón oro de intercambio
				\4[] Bancos centrales venden y compran divisa
				\4[] $\to$ De país que mantiene convertibilidad
				\4 Papel moneda precio fijo en relación a oro
				\4 Cada divisa, un precio en oro
				\4 Mecanismo flujo-especie explica
				\4[] Déficits de CC inducen salidas de oro
				\4[] $\to$ Divisa nacional convertida en oro
				\4[] $\to$ Oro convertido en divisa extranjera
				\4[] Salidas de oro reducen oferta monetaria
				\4[] Caídas de oferta monetaria reducen precios
				\4[] $\to$ Déficit de CC se reduce
				\4[] $\then$ CC se equilibra antes de agotar reservas
				\4 Realmente, flujo-especie requiere cambios
				\4[] $\to$ Política monetaria de bancos centrales
				\4[] $\to$ Flujos de capital
				\4 Política monetaria
				\4[] Bancos centrales ajustan oferta monetaria
				\4[] $\to$ Antes de que salida/entrada de oro ajuste
				\4[] Reducen/aumentan tipo de interés
				\4[] $\to$ Al que descuentan letras
				\4[] Operaciones de mercado abierto
				\4[] $\to$ Cuando Banco de Inglaterra pierde cuota de mercado
				\4[] $\then$ Su tipo de descuento es menos relevante
				\4[] $\to$ Drenaje de liquidez vendiendo deuda + repos
				\4[] Si hay salida de oro
				\4[] $\to$ Aumentan tipo al que descuentan letras
				\4[] $\then$ Menos descuento de letras
				\4[] $\then$ Menos oferta monetaria
				\4[] $\then$ Contracción de actividad y déficit CC
				\4[] $\then$ No es necesaria salida de oro
				\4 Flujos de capital
				\4[] Ante déficit en CC y aumento interés
				\4[] $\to$ Fuerte entrada de capital
				\4[] $\to$ Compra de divisa nacional a cambio de oro
				\4[] Agentes estiman banco central intervendrá
				\4[] $\to$ Y financian déficit CC para obtener interés
				\4[] $\then$ Actuación de BC aparece innecesaria
				\4[] Resulta de credibilidad muy alta de BC
				\4[] Reminiscencia de Krugman (1991)
				\4[] $\to$ Modelo de bandas de fluctuación
			\3 Reglas del juego
				\4 En periodo de entreguerras, Keynes describe patrón oro
				\4 Afirma componente fundamental es respeto a ``reglas del juego''
				\4 Economías comprometidas con sostenimiento de sistema
				\4 Toman medidas anteriores
				\4[] $\to$ Aumento de interés
				\4[] $\to$ Operaciones de mercado abierto
				\4 compromiso con ``reglas del juego''
				\4[] Aumenta credibilidad
				\4[] Mercados hacen trabajo de bancos centrales
				\4[] $\to$ Krugman (1991)
				\4 En realidad, cumplimiento relativo de reglas
				\4[] Países centrales esterilizan flujos de oro
				\4[] $\to$ Compran crédito doméstico cuando caen reservas
				\4[] $\then$ Mantienen OMonetaria cuando sale oro
				\4[] $\then$ Sin presión deflacionaria
				\4[] $\to$ Venden crédito doméstico cuando aumentan reservas
				\4[] $\then$ Mantienen OMonetaria cuando entra oro
				\4[] $\then$ Evitan presión inflacionaria
				\4[] Reaccionan a medidas de otros países centrales
				\4[] $\to$ Reduciendo efectividad de medida inicial
			\3 Credibilidad de los bancos centrales
				\4 BCentrales pueden permitirse actuar
				\4 Sindicatos aún poco desarrollados
				\4[] Salarios y precios flexibles
				\4[] $\then$ Evitan ajuste en cantidades
				\4 Modelos macro poco desarrollados
				\4[] Poca percepción de relación dinero-output
				\4[] $\then$ Menos presión contra contracción para ajuste CC
				\4 Países centrales del sistema
				\4[] Gran Bretaña, Alemania, Francia
				\4[] Incentivo muy fuerte a mantener paridad
				\4[] $\to$ Agentes entienden lo mantendrán a toda costa
			\3 Pánicos bancarios
				\4 Función de prestamista de último recurso sin desarrollar
				\4 Crisis bancarias relativamente frecuentes
				\4[] Numerosos episodios desde años 30
				\4[] En varios países del sistema
				\4[] $\to$ BCentrales no actúan como prestamistas de último recurso
				\4 Crisis de Baring en 1890
				\4[] Baring Brothers aseguran deuda argentina
				\4[] Argentina entra en default
				\4[] $\to$ Baring no tiene reservas suficientes
				\4[] $\then$ Sistema financiero británico a punto del colapso
				\4[] Depósitos en libras comienzan a liquidarse
				\4[] $\to$ Sale oro a pesar de subida de interés
				\4[] $\then$ Libra esterlina bajo presión devaluatoria
				\4[] Bancos centrales de Francia y Rusia prestan oro
				\4[] $\to$ Precedente clave
				\4 Suspensión de convertibilidad
				\4[] Se entiende como medida excepcional
				\4[] Sujeta a condiciones objetivas
				\4[] $\to$ No a discreción de autoridad monetaria/gobierno
				\4[] $\then$ Agentes no temen suspensión en general
			\3 Cooperación internacional
				\4 Necesaria
				\4[] Si un país sube tipos
				\4[] $\to$ Los demás subirán también
				\4[] $\then$ Subida inefectiva
				\4[] Si un país baja tipos
				\4[] $\to$ Los demás mantienen
				\4[] $\then$ País se queda sin reservas
				\4[] Necesario grado de cooperación
				\4[] $\to$ Apoyar economía que lo necesita
				\4[] Liderazgo del Banco de Inglaterra
				\4[] $\to$ A menudo, toma primer paso
				\4[] $\then$ Resto de bancos siguen
				\4 Habitual
				\4[] Entre países centrales
				\4[] $\to$ Para mantener sistema
				\4[] Especialmente Francia e Inglaterra
			\3 Inestabilidad en la periferia
				\4 Convertibilidad en periferia es menos importante
				\4[] Menos apoyo a bancos centrales periféricos
				\4 Movimientos populistas en Estados Unidos
				\4[] Exigen aumento crédito contra deflación
				\4[] Demócratas apoyan
				\4[] $\to$ Presión sobre dólar
				\4[] $\then$ Miedo a suspensión de convertibilidad
				\4[] Problema desaparece hacia 1900
				\4[] $\to$ Descubrimientos de oro a finales de 1890s
				\4[] $\to$ Reserva fraccionaria
				\4 Otros países periféricos suspenden convertibilidad
				\4[] De manera frecuente
				\4[] Permiten depreciación de moneda
				\4[] Argentina, Brasil, Chile, Italia, Portugal
			\3 Guerra Mundial y suspensión de convertibilidad
				\4 Precaria situación del Banco de Inglaterra
				\4[] Grandes reservas int. en libras esterlinas
				\4[] Reservas de oro insuficientes
				\4 Venta masiva de libras en 1913
				\4 Suspensión de facto de convertibilidad de libras
				\4[] Restricciones legales
				\4[] Presión del gobierno
				\4[] $\to$ Freno a conversiones de libras por oro
				\4[] $\then$ Aunque convertibilidad no se suspende de iure
		\2 Consecuencias
			\3 Integración de la economía mundial
				\4 Periodo de integración toca a su fin
			\3 Liberalización y crecimiento
				\4 Distorsión del comercio internacional
			\3 Fin de patrón oro
				\4 Suspensiones generalizadas de convertibilidad
				\4 Divergencias en niveles de precios
				\4 Papel moneda reemplaza circulación de oro
	\1 \marcar{Entreguerras}
		\2 Contexto
			\3 Económico
				\4 Fuerte inflación a lo largo de IIGM
				\4 Diferentes tasas en distintos países
				\4[] Precios se doblan en UK y USA
				\4[] Triplican en Francia
				\4[] Cuadruplican en Alemania
				\4 Desequilibrios externos se agrandan
				\4[] Superávit en USA
				\4[] Déficits en Europa
				\4 Transferencias de Alemania a vencedores
				\4 Mayores rigideces nominales
			\3 Político
				\4 Aumento poder sindicatos
				\4 Huelgas en territorios ocupados
				\4 Convulsión política general en post-guerra
				\4 Revolución en Rusia
				\4 Fascismo en Italia
			\3 Teórico
				\4 Tesoro británico y neoclásicos
				\4[] Economías tienden a equilibrio interno
				\4 Keynes
				\4[] Economías no tienden a equilibrio externo
		\2 Eventos
			\3 Restablecimiento de la convertibilidad
				\4 Estados Unidos lidera
				\4[] Levantamiento de embargo a exportaciones de oro (1919)
				\4[] Restablecimiento de convertibilidad en (1922)
				\4[] $\to$ A precio pre-guerra
				\4[] $\then$ Dólar es moneda de referencia
				\4 RU restablece convertibilidad en 1925
				\4[] Precio de pre-guerra
				\4[] Patrón oro de lingotes
				\4[] $\to$ Monedas de oro fuera de circulación
				\4[] $\to$ Compra libre de lingotes en Banco de Inglaterra
				\4[] Keynes critica restablecimiento
			\3 Fuentes de inestabilidad
				\4 Tipos de cambio inestables y escalonados
				\4[] No se fijan a la vez
				\4[] A medida que países restauran convertibilidad
				\4[] Incentivos a fijar TC estratégicamente
				\4[] $\then$ Primeros en fijar tienen moneda sobrevalorada
				\4[] $\then$ Déficits RU persistente
				\4[] $\then$ USA, Francia superávits persistentes
				\4 Reparaciones de guerra
				\4[] Alemania transfiere capital a aliados
				\4[] Reduce presión para ajustar BP en RU, FRA, USA
				\4[] Aparece hiperinflación en Alemania
				\4 Salarios y precios menos flexibles que pre-IGM
				\4[] Aumento del poder de sindicatos y movimientos obreros
				\4 Barreras al comercio acrecentadas
				\4[] Proteccionismo de la IGM se mantiene
				\4[] Ajuste exterior más difícil
				\4[] Déficits y superávits persistentes
				\4 Dilema de Triffin
				\4[] Países demandan activo de reserva
				\4[] $\to$ Para transacciones comerciales
				\4[] $\to$ Como activo seguro
				\4[] Si activo de reserva es divisa de país central
				\4[] $\to$ País central debe incurrir en déficits comerciales
				\4[] EEUU y RU como países centrales
				\4[] $\to$ Demanda internacional de dólares y libras
				\4[] $\to$ USA y RU incurren en déficits
				\4[] $\to$ Reservas de oro insuficientes
				\4[] $\then$ Presión sobre paridad
				\4 Tres centros financieros
				\4[] Pre-IGM: Londres es centro financiero
				\4[] Años 20: Nueva York, Londres y París
				\4[] Capitales fluyen entre los tres centros
				\4[] $\then$ Desestabilización de reservas
				\4 Desequilibrios en la distribución del oro
				\4[] Francia acumula mayoría de reservas de oro oficial
				\4[] Reservas de divisas en dólares y libras
				\4[] $\to$ Mucho más que francos franceses
				\4 Posición financiera precaria de RU
				\4[] Aún menor ratio oro-libras que pre-IGM
				\4 Equilibrio interno se convierte en objetivo
				\4[] Presión para mantener equilibrio interno
				\4[] Dificultades para primar equilibrio externo
				\4 Poca credibilidad de las autoridades
				\4[] Agentes no confían en mantenimiento de la convertibilidad
				\4 Reglas del juego violadas a menudo
				\4[] Países esterilizan salidas/entradas
				\4[] Impacto sobre credibilidad y ajuste
				\4 Sin liderazgo del Banco de Inglaterra
				\4[] Sin papel de líder en ningún área
				\4[] Fed utiliza pol. descuento para tres objetivos
				\4[] $\to$ Fortalecer la libra
				\4[] $\to$ Controlar especulación en Wall Street
				\4[] $\to$ Equilibrio interno
				\4[] $\then$ Fed tampoco ejerce de líder
			\3 Gran Depresión
				\4 Explosión de emisión de bonos y acciones en USA
				\4[] Enorme crecimiento desde años 20
				\4[] Americanos invierten en bonos y acciones extranjeras
				\4 Superávits comerciales USA
				\4[] Permiten financiar déficits en Europa y periferia
				\4 Subida de tipos de interés en USA en 1928
				\4[] Fed trata de :
				\4[] $\to$ Frenar boom de Wall Street
				\4[] $\to$ Reducir caída ratio de cobertura dólar-oro
				\4[] Capital deja de salir de USA
				\4 Problemas en periferia
				\4[] Sin entrada de capitales
				\4[] Aumento de tipos se transmite
				\4[] $\to$ Quiebras generalizadas en periferia
				\4 Deflación en periferia
				\4[] Mecanismo de flujo-especie comienza a operar
				\4[] Principios de 1929:
				\4[] $\to$ Se reduce déficit CC en periferia
				\4[] $\to$ Se mitigan problemas de financiación
				\4 Crack del 29
				\4[] Octubre negro
				\4[] Caídas bruscas y generalizadas de acciones
				\4[] $\then$ Quiebras bancarias
				\4[] $\then$ Despidos masivos
				\4[] $\then$ Contracción brusca de la demanda
				\4 Banking school
				\4[] Oferta monetaria es endógena a producción
				\4[] $\to$ No hay que intervenir
				\4[] Necesario dejar quebrar y liquidar
				\4[] $\to$ Despidos masivos
				\4[] $\to$ Profundiza caída de demanda
				\4[] $\then$ Contracción muy fuerte de oferta monetaria
				\4 Liquidacionismo
				\4[] von Hayek, Robbins, Schumpeter, Mellon
				\4[] Quiebras y liquidaciones son necesarias
				\4[] $\to$ Limpiar balanzas
				\4[] $\to$ Reasignar capital a inversiones productivas
				\4[] Presidente Hoover acepta al principio
				\4[] $\to$ Se arrepiente hacia 1932
				\4 Patrón oro
				\4[] Precios y TC fijos
				\4[] Deflación como única vía de ajuste
				\4[] $\to$ Aumento del valor real de deuda interna y externa
				\4[] $\then$ Quiebras de agricultores
				\4 Periferia
				\4[] Incipiente recuperación se frena
				\4[] Demanda de exportaciones muy débil
				\4[] $\to$ Depresión en EEUU, Europa
			\3 Abandonos de convertibilidad
				\4 Países periféricos primeros en suspender
				\4[] Brasil, Argentina, Canadá, Australia en 1929, 1930
				\4[] $\to$ Permiten flotación de sus monedas
				\4 Países centrales resisten
				\4[] Autoridades quieren inyectar crédito
				\4[] $\to$ Pero inconsistente con mantener patrón oro
				\4 Expectativas desestabilizantes
				\4[] Rumores de bajada de tipos de interés
				\4[] $\to$ Provocan salidas de capitales
				\4[] $\to$ Liquidación de balances en Londres y Nueva York
				\4 Austria y Alemania
				\4[] Quiebras bancarias en 1931
				\4[] $\to$ Suspensión de convertibilidad en verano
				\4 Reino Unido
				\4[] Venta masiva de libras tras suspensión alemana
				\4[] Desempleo masivo
				\4[] $\to$ Necesaria subida brutal de interés
				\4[] $\then$ Inaceptable para gobierno
				\4[] Suspensión de convertibilidad en septiembre 1931
				\4 Estados Unidos
				\4[] Dolar se aprecia fuertemente
				\4[] Demanda se desploma aun más
				\4[] Temor a suspensión de convertibilidad
				\4[] $\to$ Liquidaciones de dólares
				\4[] Glass-Steagall
				\4[] $\to$ Elimina restricción a aumento oferta monetaria
				\4[] Devaluación del dólar en 1933
				\4[] $\to$ Grupo de países fija paridad con dólar
				\4[] $\then$ FIL,CUB, Centroamérica, CAN, ARG
				\4 Países que mantienen patrón oro
				\4[] Depresión se agudiza fuertemente
				\4[] República Checa, Bélgica, Francia, Suiza...
				\4[] $\to$ Todos suspenden para 1936
		\2 Consecuencias
			\3 Inestabilidad cambiaria
				\4 Flotación de divisas generalizada
				\4 Volatilidad
			\3 Descoordinación de política monetaria
				\4 Decisiones unilaterales perjudican a vecinos
				\4 Expansión monetaria en USA
				\4[] Depreciación del dolar
				\4[] $\to$ Apreciación del franco
				\4[] $\then$ Shock negativo en Francia
				\4[] $\then$ Expansión conjunta habría evitado
			\3 Cooperación a finales de 30
				\4 Acuerdo Tripartito: FRA, UK, USA
				\4[] Francia limita devaluación del franco
				\4[] UK y USA aceptan no devaluar
			\3 Aumento de proteccionismo
				\4 Devaluaciones y expansiones monetarias inducen
				\4 Aumento de aranceles generalizado
	\1 \marcar{Bretton Woods y posguerra}
		\2 Contexto
			\3 Económico
				\4 Caída del output al terminar IIGM
				\4 Enormes costes de resasignación de factores
				\4[] Hombres empleados en ejército
				\4[] Industrias reconvertidas
				\4 Desequilibrios financieros
				\4[] Aumento fuerte de endeudamiento
				\4[] Impagos de países derrotados
				\4 Incertidumbre sobre sistema futuro
				\4[] ¿Convertibilidad?
				\4[] ¿Moneda de reserva?
			\3 Político
				\4 Conferencias de Yalta y Bretton Woods
				\4 Aumento de poder soviético tras guerra
				\4 Europa tutelada por Estados Unidos
				\4 Estados Unidos adopta papel hegemónico
			\3 Teórico
				\4 Nurkse para la Liga de Naciones
				\4[] Crítica a sistema cambiario entreguerras
				\4[] Crítica a libre flotación
				\4[] Propone evitar fijación unilateral de TCN
				\4[] $\to$ Tipos fijos ajustables con permiso inst. supranac.
				\4 Meade
				\4[] Defensa de tipos fijos ajustables
				\4 Friedman
				\4[] Emergente defensa de tipos flexibles
				\4[] Dinero es importante
				\4[] Apertura comercial y financiera
				\4[] Universidad de Chicago
				\4 Keynes
				\4[] Teoría General consolidada en política económica
				\4[] Síntesis neoclásica
		\2 Eventos
			\3 Negociaciones de Bretton Woods
				\4 Dos voces principales
				\4[] Representando a USA y UK
				\4[] $\to$ USA posición predominante
				\4[] $\to$ UK posición debilitada tras IIGM
				\4[] $\then$ Harry Dexter White
				\4[] $\then$ Keynes
				\4 Keynes prevé dos problemas futuros:
				\4[] $\to$ UK necesitará financiar BP
				\4[] $\to$ USA sufrirá otra depresión futura
				\4[] $\then$ Resto del mundo sufrirá déficits
				\4[] Tensión futura entre eq. interno y externo
				\4[] $\to$ Si países no pueden financiar deseq. BP
				\4[] Creación de moneda internacional
				\4[] $\to$ ``bancor''
				\4[] $\to$ Emitida por institución multilateral
				\4[] $\to$ Usada por gobiernos y BCentrales para saldar BP
				\4[] Paridades ajustables
				\4[] $\to$ Keynes enfatiza y White acepta
				\4 White
				\4[] Prevé un gran problema:
				\4[] $\to$ Existencia de oferta grande y elástica de moneda internacional
				\4[] $\then$ Resto del mundo capturará parte de output americano
				\4[] $\then$ Inaceptable tras IIGM
				\4[] Limitar oferta de crédito de FMI
				\4[] $\to$ Vía pool de divisas y oro
				\4[] $\then$ Cuantía mucho menor que propuesta de Keynes
				\4[] $\then$ White: $\$2$ bln, Keynes $\$23$ bln
				\4[] $\then$ Menos reservas, más necesidad de ajustes
			\3 Creación de instituciones financieras multilaterales
				\4 Fondo Monetario Internacional
				\4[] Financiar déficits de balanza de pagos
				\4[] $\to$ Evitar ajustes bruscos
				\4 Banco Mundial
				\4[] Canalizar capital a proyectos de reconstrucción
				\4[] Reducir tentación soviética
				\4 Cita de Keynes
				\4[] ``El Fondo es un banco y el Banco es un fondo''
				\4 International Trade Organization
				\4[] Evitar restricciones al comercio
				\4[] $\to$ Para estabilizar CC vía barreras comerciales
				\4[] Fracaso de la ITO
				\4[] $\to$ Aprobación del GATT
				\4[] $\then$ Rondas sucesivas de liberalización
			\3 Plan Marshall
				\4 Ayuda entre 1948 y 1951
				\4 Enormes volúmenes de transferencias a Europa
				\4 Eclipsa primeros años de FMI
			\3 Sistema de Bretton Woods
				\4 Tres elementos centrales
				\4[I] TCN fijos respecto a oro/dólar
				\4[] Todos los miembros del FMI deben fijar
				\4[] Dólar fijado a oro
				\4[] $\then$ TCNs fijados respecto dólar y oro
				\4[] Ajustables sólo si ``desequilibrio fundamental''
				\4[] $\to$ Sin definir explícitamente
				\4[] $\then$ Entendido como conflicto entre eqs. interno y externo
				\4[II] Convertibilidad de de la cuenta corriente
				\4[] Convertibilidad plena de monedas
				\4[] $\to$ Para transacciones de la cuenta corriente
				\4[] $\then$ Tan pronto como fuese posible
				\4[] Controles de capital son posibles
				\4[] $\to$ Pero no controles de cuenta corriente
				\4[] Toda divisa obtenida vía cuenta corriente
				\4[] $\then$ Debe ser convertible en moneda local
				\4[] No se alcanzó completamente hasta 1958
				\4[] Restricciones a convertibilidad de cuenta financiera
				\4[] $\to$ Plenamente aceptables
				\4[] $\to$ Fuertes restricciones iniciales
				\4[] $\to$ Progresiva relajación/evasión
				\4[III] Acceso a préstamos de FMI
				\4[] Evitar crisis de balanza de pagos
				\4[] $\to$ Sin divisas suficientes para financiar pagos
			\3 Devaluaciones iniciales en Europa
				\4 Finales de 40 y primeros 50
				\4 Europa tiene grandes déficits
				\4 Creación de Unión Europea de Pagos en 1950
				\4[] Marco de liberalización comercial
				\4[] Eliminación de restricciones de cuenta corriente
				\4[] Complementan FMI a nivel europeo
				\4[] $\then$ Éxito en estabilización
				\4 Problemas en Francia
				\4[] Crisis de Suez y Argelia
				\4[] $\to$ Déficit fiscal y público
				\4[] Programa de austeridad en 1957
				\4[] $\to$ Tras vuelta de De Gaulle
			\3 Escasez de dólares
				\4 Superávits de CC en USA
				\4 Hasta principios de años 60
				\4 Reservas de oro muy holgadas
				\4[] Sin expectativa alguna de devaluación
				\4[$\then$] Presión hacia revaluación del dólar
				\4 Dólares superan reservas en 1960
				\4[] Las reservas de oro americanas
				\4[] $\to$ No pueden comprar todos los dólares emitidos
			\3 Déficits crecientes en USA
				\4 Reservas de oro no cubren oferta monetaria en 1959
				\4 A partir de 1960
				\4 Déficits por cuenta corriente comienzan a aparecer
				\4 Salidas de capital cada vez más fácil
				\4[] Euromercados
				\4[] Sobrevalorar facturas de importación
				\4[] Empresas multinacionales
				\4 Guerra de Vietnam
				\4[] Expansión fiscal cada vez mayor
				\4 Victoria de Kennedy en 1960
				\4[] Precio del oro sube a $>\$40$ la onza en 1960
				\4[] Frente a precio oficial de $35$
			\3 Medidas para frenar flujos desestabilizantes
				\4 Derechos Especiales de Giro (1969)
				\4[] Creación y primera asignación en 1969
				\4[] $\to$ Primera enmienda al Acuerdo Constitutivo del FMI
				\4[] Suplementar oferta de dólares
				\4[] $\to$ Con activo de reserva alternativo
				\4[] Acuerdos de Jamaica de (1976)
				\4[] $\to$ Segunda enmienda a Acuerdo Constitutivo de FMI
				\4[] $\to$ Entrada en vigor en 1978
				\4 Interest Equalization Tax (1963)
				\4[] Impuesto a inversión en el exterior
				\4[] $\to$ Desincentivar compras de acciones y bonos extranjeros
				\4[] $\then$ Frenar salida de capitales\footnote{El déficit comercial americano implica una entrada neta de capital, esto es, un saldo negativo en la cuenta corriente. Por otro lado, la inversión directa y en cartera americana implica una salida neta de capitales, o un saldo positivo de estos componentes de la cuenta financiera (en la convención de signos del Sexto Manual). Para equilibrar la balanza de pagos, será necesario un saldo negativo en el resto de componentes de la cuenta financiera --incluida la variación de reservas- muy elevado, por lo que la sostenibilidad del sistema de fijación del tipo de cambio del dólar se verá aún más comprometido. La \textit{Interest Equalization Tax} tiene por objetivo reducir el saldo positivo de esos componentes de la cuenta financiera que dificultan el ajuste. }
				\4[] Relativamente efectivo
				\4[] $\to$ Capital sigue saliendo de EEUU
				\4[] Eliminada en 1974
				\4 Regulación Q\footnote{La regulación Q data de 1933, y ha sido enmendada en múltiples ocasiones hasta la actualidad. } en Estados Unidos
				\4[] Límite al interés que depósitos pueden pagar
				\4[] $\to$ Herramienta de represión financiera
				\4[] Incentiva creación de nuevos instrumentos:
				\4[] $\to$ Euromercados
				\4[] $\to$ Fondos del mercado monetario
				\4[] $\to$ Cuentas NOW\footnote{Negotiable Orders of Withdrawal.}
				\4[] Incentiva salida de capital de EEUU
				\4[] Derogación del límite al interés en 2010
				\4[] $\to$ Reforma Dodd-Frank
				\4 Prohibición de tenencia de oro en extranjero
				\4[] En Estados Unidos
				\4 Compras de oro obligatorias
				\4[] A precio fijo
				\4[] Penas de multa o cárcel por tenencia de oro
				\4 Restricciones a préstamos al extranjero
				\4[] En Estados Unidos
				\4[] Obligatorias a partir de 1968
				\4 Promoción de exportaciones americanas
				\4[] Créditos a la exportación via EXIM Bank
				\4[] Menos requisitos para obtener visado americano
				\4[] $\to$ Incrementar divisas vía turismo
				\4 Prohibición de interés de depósitos extranjeros
				\4[] En Alemania y Suiza
				\4 Acuerdos swap
				\4[] BCentrales acuerdan no vender divisas
				\4[] $\to$ Cuando agentes privados las presentan para liquidar
				\4 Gold Pool
				\4[] UK, SWI y CEE en 1961
				\4[] Aumento de incentivo a liquidar dólares por oro
				\4[] Acuerdo para evitar conversión dólares a oro
				\4[] Vender oro de sus propias reservas
				\4[] $\to$ Reducir presión sobre dólar
				\4[] Estados Unidos no corresponde
				\4[] $\to$ No se compromete a defender dólar
				\4[] Francia sale de gold pool en 1967
				\4[] $\to$ Otros países
		\2 Consecuencias
			\3 Estabilidad cambiaria
				\4 Devaluaciones espaciadas y poco frecuentes
				\4 Más frecuentes devaluaciones que revaluaciones
				\4[] $\then$ Progresiva revaluación del dólar
				\4 EEUU sin incentivos a devaluar
				\4[] Privilegio de ser moneda de reserva
			\3 Buen funcionamiento de sistema de pagos
				\4 Permite aumento de comercio
				\4[$\to$] Fuerte crecimiento en 50s y 60s
			\3 Sin deflaciones
				\4 Devaluaciones son alternativa a aumento de tipos
				\4 Políticas monetarias autónomas
			\3 Restricciones de capital
				\4 Limitan flujos de capital a lo largo del periodo
				\4 Aíslan de flujos de capital desestabilizantes
				\4 Relativo éxito
				\4 Visión actual distorsionada
				\4[] Actualidad, pocas restricciones al capital
				\4[] Difícil imponer restricciones efectivas
			\3 Éxito relativo de FMI
				\4 Supervisión desincentiva devaluaciones frecuentes
				\4 Apenas ejerce autoridad sobre devaluaciones
				\4[] $\to$ Cuando efectivamente se producen
			\3 Generación endógena de liquidez
				\4 BC y gobiernos complementan oro con divisas
				\4 Déficit de CC americano aumenta oferta de dólares
				\4[$\then$] Aumento de reservas y liquidez
				\4 Dilema de Triffin
				\4[] I. Sistema financiero global necesita activo de reserva
				\4[] $\to$ Demanda de dólares
				\4[] II. Aumento de la oferta de dólares
				\4[] $\to$ Dificulta mantenimiento de paridad oro/dólar
				\4[$\then$] Dilema
				\4[] Reducir déficit contrayendo actividad
				\4[] Mantener déficit dificultando sostenibilidad de sistema
				\4 Críticas francesas a privilegio exorbitante
				\4 Dólares superan reservas de oro en 1960
				\4[] Comienzan tensiones
				\4[] Francia comienza a valorar liquidación de dólares
				\4[] $\to$ A cambio de oro depositado en EEUU
	\1 \marcar{Caída de Bretton Woods}
		\2 Contexto
			\3 Económico
				\4 Presión creciente sobre dólar
				\4[] Salidas de capital constantes
				\4[] Medidas de control insuficientes
				\4[] Mercados de capital cada vez más porosos
				\4[] Reservas insuficientes
				\4[] Precio de dólar en mercados privados supera precio Fed
				\4[] $\to$ Francia amenaza con liquidar dólares
				\4 Miedo a inflación en Alemania
				\4[] Rechazo a compra masiva de dólares
				\4 Libra esterlina bajo presión
				\4[] Salidas de capital constantes
				\4 Inflación en EEUU
				\4[] No muy alta en términos absolutos
				\4[] Sí demasiado alta para mantener competitividad
				\4 Europa y Japón se acercan a convergencia
			\3 Político
				\4 Guerra de Vietnam
				\4[] Expansión fiscal en Estados Unidos
				\4 Sindicatos en Reino Unido
				\4[] Presión a subida de salarios
				\4 Guerra fría
				\4[] EEUU, Europa, Japón socios frente a URSS
				\4[] Incentiva cooperación a pesar de desequilibrios
				\4 Guerra del Yom Kippur
				\4[] Tensión en Oriente Próximo
				\4[] Estados Unidos apoya Israel
				\4[] $\to$ OPEC introduce embargo en 1973
			\3 Teórico
				\4 Agotamiento de Síntesis Neoclásica
				\4 Monetarismo
				\4[] Dinero es importante
				\4[] $\to$ Tiene fuertes efectos sobre ciclo
				\4 Robert Lucas
				\4[] Dinero afecta output
				\4[] Efectos se agotan si muy utilizados
		\2 Eventos
			\3 Desequilibrios previos
				\4 Déficits fiscales crecientes por Vietnam
				\4 Aumento de flujos de capital
				\4 Aumento del precio del oro en mercados privados
				\4 Inflación en Estados Unidos
			\3 Ventas masivas de dólares en 1971
				\4 Compras de marcos alemanes y florines holandeses
				\4[] Alemania y Holanda dejan apreciar sus monedas en mayo
				\4[] $\to$ Otros países europeos también
				\4 Ventas de dólares no se frenan
				\4 Suiza redime dólares por oro
				\4 Rumores de liquidación dólares por oro
				\4[] De Francia e Inglaterra
			\3 Nixon Shock (1971)
				\4 Fin de semana del 13 de agosto de 1971
				\4[] $\to$ Nixon toma decisión final
				\4 Cierre de ventanilla de oro
				\4[] Suspensión de convertibilidad a $\$35$ la onza
				\4 Impuesto del $10\%$ a las importaciones
				\4 Presión a otros países para que revalúen
				\4[] Evitar pérdida de prestigio
				\4[] $\to$ Asociada a devaluación
			\3 Acuerdos de Smithsonian en Washington (1971)
				\4 Diciembre del 71
				\4 Devaluación del dólar respecto al oro
				\4[] Pero sin reapertura de la ventanilla de oro
				\4[] Mejora ligera de competitividad americana
				\4[] $\to$ Aunque no compensa otras políticas
				\4[] $\to$ Política fiscal y monetaria expansiva pre-elecciones
				\4 Bandas de flotación del $\pm 2.25$ respecto al dólar
				\4[] $\to$ ``túnel''
				\4[] $\to$ Permite $4,5\%$ entre monedas\footnote{Si una moneda europea A se deprecia $2,25\%$ frente al dólar y otra moneda B se aprecia otro $2,25\%$, en la práctica hay un margen de flotación del $4.5\%$. Si la moneda A se aprecia un $4,5\%$ --pasando del límite inferior al superior-- y la moneda B se deprecia otro $4,5\%$ --pasando del límite superior al inferior--, tiene lugar  una variación del $9\%$ entre monedas aun manteniéndose dentro del sistema. Considerando este margen de fluctuación como excesivo, los países de la CEE establecieron la ``serpiente en el túnel''.}
				\4 Serpiente en el túnel
				\4[] Margen de flotación entre monedas europeas
				\4[] $\to$ Inferior al margen del túnel
				\4[] $\then$ Margen bilateral del $2,25\%$ respecto al dólar.
				\4 Libra sale del acuerdo en junio del 1972
				\4 Intento de negociación de un nuevo acuerdo
				\4[] $\to$ Fracaso
				\4 Japón sale del acuerdo
				\4[] $\to$ Libre flotación
				\4 Flotación conjunta de europeos en el 73
				\4[$\then$] Abandono de Acuerdo de Smithsonian en 1973
			\3 Crisis del petróleo del 73
				\4 Embargo de envíos de petróleo a EEUU
				\4 Aumento de inflación en Estados Unidos
				\4 Empeora presión devaluatoria del dólar
			\3 Abandono definitivo en Acuerdos de Jamaica (1976)
				\4 Segunda enmienda a artículos del FMI
				\4 Se permite todo régimen cambiario
				\4[] $\to$ Incluida libre flotación
				\4[] $\then$ Salvo fijación de TCN respecto oro
				\4[] Entrada en vigor en 1978
		\2 Consecuencias
			\3 Política monetaria como instrumento de PEconómica
				\4 PM ya no está supeditada a mantener TC
				\4 Gobiernos pueden utilizar libremente para eq. interno
				\4 Financiación monetaria de déficits impulsa inflación
			\3 Regímenes cambiarios polares
				\4 Fijos ajustables no parecen viables
				\4[] $\to$ Aumento de flujos de capital
				\4[] $\then$ Necesarios volúmenes enormes de TC
				\4 Economías grandes y relativamente aisladas
				\4[] $\to$ Libre flotación
				\4[] $\then$ Estados Unidos y Japón
				\4 Economías pequeñas y dependientes de comercio internacional
				\4[] $\to$ Fijación fuerte del TC
				\4[] Miembros de CEE
				\4[] $\to$ Mantener paridades muy importante dada PAC
				\4[] $\to$ Serpiente Europea
				\4[] $\then$ Sistema Monetario Europeo
				\4[] Pequeñas economías abiertas
				\4[] $\to$ Juntas de conversión
				\4[] $\to$ Regímenes fijos
			\3 Volatilidad de tipos de cambio
				\4 Fluctuaciones constantes
				\4 Alrededor del $3\%$ mensual
				\4 TCReal mucho más volátiles
				\4 Intervención pública habitual en mercados cambiarios
			\3 Impulso a integración europea
				\4 Políticas europeas ya implementadas
				\4[] Unión aduanera
				\4[] Política agrícola común
				\4 Estabilidad cambiaria necesaria
				\4[] Aprovechar ventajas de integración comercial
				\4[$\then$] Impulso a integración monetaria
				\4[] Informe Werner en 1970
				\4[] Serpiente en el túnel
				\4[] Sistema Monetario Europeo en 1979
	\1[] \marcar{Conclusión}
		\2 Recapitulación
			\3 Antecedentes
			\3 El patrón oro
			\3 Entreguerras
			\3 Bretton Woods y posguerra
			\3 Caída de Bretton Woods
		\2 Idea final
			\3 Influencia del pasado reciente
				\4 Diseño de políticas
				\4[] Influenciada por contexto contemporáneo
				\4 Valoración de decisiones pasadas
				\4[] Debe considerar contexto
				\4[] $\to$ Diferente información
				\4[] $\to$ Recuerdo reciente de crisis pasadas inmediatas
			\3 Política, economía y sistema monetario
				\4 Sistema monetario y económico no es exógeno
				\4[] Interacción compleja con política
				\4 Equilibrio interno como objetivo
				\4[] Aumenta importancia enormemente
				\4[] Difícil implementar determinados ajustes
				\4[] $\to$ Fuerte influencia sobre sistema monetario
			\3 Reminiscencias del pasado en la actualidad
				\4 Oro sigue siendo activo reserva relevante
				\4[] Aunque ningún país fije su TCN respecto oro
				\4 Dólar mantiene status de activo de reserva
				\4[] Aunque pérdida relativa respecto euro y yen
				\4 Instituciones de Bretton Woods
				\4[] FMI, GBM, GATT/OMC
				\4[] $\to$ Siguen existiendo y son relevantes
\end{esquemal}


























\preguntas

\seccion{Test 2015}
\textbf{34.} Señale la respuesta verdadera relativa al funcionamiento del patrón oro hasta 1914:

\begin{itemize}
	\item[a] La principal responsabilidad de un Banco Central era el mantenimiento de la estabilidad de precios.
	\item[b] En la Conferencia de Génova se decidió que los países más pequeños podían mantener como reservas las monedas de los países grandes, cuyas reservas internacionales consistirían únicamente en oro.
	\item[c] El mecanismo de ajuste precio-flujo de especie descrito por Ricardo establecía que las variaciones del tipo de cambio facilitarían el ajuste de los déficits/superávits de las balanzas comerciales.
	\item[d] El propio funcionamiento del sistema podía tener efectos negativos en términos de empleo para aquellos países que tenían que exportar oro. 
\end{itemize} 

\seccion{Test 2013}

\textbf{36.} La desmonetización del oro se produjo:
\begin{itemize}
	\item[a] En 1978, como consecuencia de la entrada en vigor de los Acuerdos de Jamaica, firmados en 1976.
	\item[b] En 1971, como consecuencia de los Acuerdos de Washington.
	\item[c] En 1968, como consecuencia de la Cumbre de Río de Janeiro.
	\item[d] Ninguna de las anteriores.
\end{itemize}

\textbf{37.} La crisis comercial que se produce durante la Crisis del '29 fue causada por:
\begin{itemize}
	\item[a] Existencia de proteccionismo.
	\item[b] Devaluaciones competitivas.
	\item[c] Descenso de la liquidez mundial.
	\item[d] Todas las anteriores.
\end{itemize}

\seccion{Test 2011}

\textbf{36.} Los Derechos Especiales de Giro:
\begin{itemize}
	\item[a] Se establecieron para adaptar las reservas de divisas a las necesidades mundiales a largo plazo.
	\item[b] Se crearon tras la Segunda Enmienda al Convenio Constitutivo del Fondo Monetario Internacional.
	\item[c] Actualmente su valoración se establece en función de una cesta de 8 monedas que corresponden a las de aquellos países con silla permanente en el Directorio Ejecutivo.
	\item[d] Puede utilizarse como medio de pago en todo tipo de transacciones económicas.
\end{itemize}

\seccion{Test 2009}

\textbf{36.} La principal novedad que significó el sistema monetario de Bretton Woods frente a los sistemas anteriores fue:
\begin{itemize}
	\item[a] El establecimiento, mediante un acuerdo internacional, de un nuevo patrón monetario que regulara las relaciones entre países.
	\item[b] La designación de tipos de cambio fijos como piedra básica del sistema.
	\item[c] Tipos de cambio absolutamente inalterables aunque un país presentase desequilibrios fundamentales de Balanza de Pagos.
	\item[d] La fijación por parte del Fondo Monetario Internacional de la paridad de las monedas participantes en el sistema frente al oro.
\end{itemize}

\seccion{Test 2006}

\textbf{37.} Señale la respuesta correcta:

De la denominada ``Conferencia Monetaria y Financiera de Naciones Unidas'' celebrada en Bretton-Woods en julio de 1944, se derivó la decisión de,
\begin{itemize}
	\item[a] Crear el grupo del Banco Mundial.
	\item[b] Establecer un sistema multilateral de pagos que evitase que los países incurriesen en déficit externos.
	\item[c] La creación de la Unión de Compensación Internacional. 
	\item[d] Crear el Fondo Monetario Internacional, el Banco Internacional de Reconstrucción y Desarrollo, y una organización internacional de comercio.
\end{itemize}

\notas

\textbf{2015:} \textbf{34.} D

\textbf{2013:} \textbf{36.} A \textbf{37.} D

\textbf{2011:} \textbf{36.} A

\textbf{2009:} \textbf{36.} A

\textbf{2006:} \textbf{37.} D

\bibliografia

Mirar en Palgrave:
\begin{itemize}
	\item Ancient Greece, the economy of
	\item Bank of England
	\item banking crises
	\item Baring crisis of 1890
	\item banking school, currency school, free banking school
	\item \textbf{bimetallism}
	\item \textbf{Bretton Woods system}
	\item bubbles in history
	\item bullionist controversies
	\item commodity money
	\item continuity in economic history
	\item economic history
	\item economic interpretation of history
	\item external debt
	\item foreign exchange, history of
	\item free banking
	\item free banking era
	\item \textbf{gold standard}
	\item \textbf{Great Depression}
	\item Great Depression (mechanisms)
	\item Great Depression, monetary and financial forces in
	\item Gresham's law
	\item history and comparative development
	\item history of economic thought
	\item international economics, history of
	\item \textbf{international finance}
	\item \textbf{international financial institutions}
	\item international indebtedness
	\item International Monetary Fund
	\item Keynes, John Maynard
	\item Kindleberger, Charles P.
	\item medieval guilds
	\item New Deal
	\item national debt
	\item public debt
	\item sovereign debt
	\item \textbf{silver standard}
	\item specie-flow mechanism
	\item thirld world debt
	\item \textbf{World Wars, economics of}
\end{itemize}

Aliber, R. Z.; Kindleberger, C. P. \textit{Manias, Panics, and Crashes} (2015) Palgrave MacMillan -- En carpeta Historia Económica

Bordo, M. D.; Schwartz, A. J. (1984) \textit{A Retrospective on the Classical Gold Standard, 1821--1931} University of Chicago Press -- En carpeta Historia Económica

Dellas, H.; Tavlas, G. S. \textit{Milton Friedman and the case for flexible exchange rates and monetary rules} (2017) Bank of Greece Working Paper -- En carpeta del tema

De Long, B. J.; Eichengreen, B. \textit{The Marshall Plan: History's Most Succesful Structural Adjustment Program} (1991) NBER Working Paper Series -- En carpeta del tema

Eichengreen, B. \textit{Globalizing Capital} (2008) Princeton University Press

Meissner, C. M. (2015) \textit{The Limits of Bimetallism} NBER Working Paper Series. WP 20852


\end{document}
