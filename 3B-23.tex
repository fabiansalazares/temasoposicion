\documentclass{nuevotema}

\tema{3B-23}
\titulo{El Fondo Monetario Internacional. Estructura y políticas. Implicaciones sobre las políticas de estabilización de los países en desarrollo.}

\begin{document}

\ideaclave

Hablar de ESR, GFS en supervisión multilateral e individualizada para países.

Ver BdE (2020) sobre ANÁLISIS EFECTIVIDAD PROGRAMAS DEL FMI EN LA ÚLTIMA DÉCADA.

PARA Propuestas de reforma
https://voxeu.org/article/imf-75-reforming-global-reserve-system

PARA Propuestas de reforma: cuotas en el FMI y reforma del sistema, bloqueo americano en 2019, NABs que expiran en 2020s  https://piie.com/system/files/documents/pb18-9.pdf 

La estabilidad del sistema financiero internacional es un factor determinante para el crecimiento de la producción mundial y el bienestar de la población. El elemento central de esa estabilidad es el acceso a la financiación del déficit exterior. Así, en la medida en que la globalización del comercio y los flujos de capital han inducido una interdependencia creciente entre economías, ha aumentado la necesidad de un mecanismo de suavización de las fluctuaciones de la oferta de financiación exterior. Tras el desastre de la Segunda Guerra Mundial, los principales actores de la economía mundial se propusieron implementar un sistema multilateral y basado en reglas que suavizase el impacto de las habituales crisis periódicas --que a menudo restringían de forma súbita la circulación del ahorro- y que contribuyese a reducir la posibilidad de nuevas guerras a escala mundial. Una de las piezas centrales de este sistema era la provisión de financiación exterior y el control del sistema internacional de cambios implantado en Bretton Woods. El organismo encargado de esta labor es el Fondo Monetario Internacional.

Examinemos en primer lugar los \marcar{objetivos, las actuaciones y la trayectoria histórica} del FMI. Los \textbf{objetivos} globales del Fondo Monetario se enumeran en el artículo primero de sus artículos de constitución, detallando y desarrollando el objetivo global de mantenimiento del sistema financiero internacional. Entre ellos, la promoción de la cooperación monetaria internacional y el comercio, la estabilidad de los tipos de cambio, el buen funcionamiento del sistema multilateral de pagos, el fomento de la confianza mutua de los agentes económicos que permita suavizar los ajustes de la balanza de pagos y la reducción de los desequilibrios en ésta última. Las \textbf{actuaciones} que el Fondo lleva a cabo para lograr estos objetivos se pueden resumir en aquellas que giran en torno a la supervisión de las políticas económicas y las recomendaciones relacionadas, el apoyo a economías con problemas mediante la asistencia financiera financiero y el diseño de programas de ajuste, el apoyo a países en desarrollo, y el mantenimiento del sistema multilateral de pagos así como su expansión y mejora. 

La \textbf{trayectoria histórica} del FMI ha trascurrido a lo largo de más de medio siglo, y es relevante a efectos de entender las idiosincrasias de su actuación presente, su organización y los efectos que el contexto económico han tenido en su configuración. En la etapa desde su creación hasta la caída del sistema de Bretton Woods, en los años 70, las actuaciones del FMI fueron encaminadas en su mayoría a garantizar el sistema de tipos de cambio fijo y permitir una liberalización ordenada de las cuentas corrientes. En las dos primeras décadas el Fondo consolidó una metodología de actuación que le ayudaría posteriormente a hacer frente a los retos de los años 80. En esta nueva década, el sistema de tipos flotantes se consolidó y redujo el papel del Fondo como supervisor del sistema de tipos de cambio. El nuevo sistema dio lugar a una serie de fenómenos que crearon nuevos desafíos para el Fondo, en especial la supervisión y la prevención de las crisis financieras de primera generación que afectaron especialmente a Latinoamérica. En los años 90, nuevas crisis financieras con diferentes características supusieron la creación de nuevos instrumentos de supervisión y la intervención como actor determinante en la crisis asiática y la crisis rusa de 1998. Los años 2000 se iniciaron con la crisis argentina de 2001 y terminaron con la más grave crisis financiera desde la Gran Depresión: la Gran Recesión de 2008. La intervención del Fondo fue determinante en la Crisis del Euro y la provisión de ayuda a Grecia, Irlanda y Portugal. En resumen, la trayectoria del FMI se ha visto modulada una constante presión por parte de la opinión pública mundial que ha obligado al Fondo a reformular sus políticas y llevar a cabo una labor de introspección que se refleja en la evolución de sus actuaciones, así como en su estructura organizativa y financiera.

La \marcar{estructura} \textbf{organizativa} del FMI se basa en 3 órganos principales: la \underline{Junta de Gobernadores}, el \underline{Directorio Ejecutivo} y la \underline{Gerencia}. El primero es el órgano supremo que refrenda formalmente las decisiones del Directorio Ejecutivo, encargado de la toma de decisiones del día a día y la aprobación de programas de ayuda en primer término. La composición de éste último órgano está determinada por el peso relativo de cada miembro en la organización, el cual se determina a su vez de acuerdo con una fórmula que tiene en cuenta el PIB, la apertura al comercio y el capital internacional, la volatilidad de los movimientos de capitales y los ingresos corrientes, y las reservas de divisas en posesión de cada economía. El cálculo de estos pesos relativos supone a menudo un elemento contencioso de las negociaciones de reforma pues constituyen al mismo tiempo el factor clave en la contribución que cada país realiza al fondo y el poder de decisión de cada miembro. El cálculo de la cuota ha sido objeto de sucesivas reformas y la última, que entró en vigor en 2016, supuso un relativo éxito para España al aumentar su peso en la contribución de fondos y situarla como la 5a economía europea en derechos de voto. La \underline{Gerencia} ejerce el liderazgo de la institución y la gestión administrativa y se estructura en torno a un Director Gerente --tradicionalmente europeo- y cuatro Subdirectores Gerentes originarios de los principales países accionistas y de países receptores de programas de ayuda.

La organización del Fondo a nivel \textbf{financiero} viene dada por la semejanza de su balance y el de una cooperativa de crédito. El activo se compone de las reservas de divisas en posesión del fondo, los derechos de cobro sobre programas de ayuda concedidos, reservas de oro y los inmuebles y otro patrimonio del fondo. El pasivo está conformado por las cuotas, que se asemejan al capital de un banco comercial, los préstamos recibidos por los estados miembros y una partida de reservas que equilibra el balance. El Fondo no tiene permitido asumir pérdidas, y aplica unos márgenes de intermediación sobre los fondos prestados. La unidad de cuenta del fondo es el Derecho Especial de Giro. Esta suerte de moneda es en realidad un derecho de adquisición potencial de divisas de estados miembros y define su valor de acuerdo al valor en dólares de una cantidad fija de divisas que se recalcula cada cinco años y que en la actualidad se compone de dólares estadounidenses, euros, yenes japoneses, libras esterlinas y desde 2016, de yuan renminbi chinos. Existe además el Fondo para la Reducción de la Pobreza y el Crecimiento en el que el FMI participa en colaboración con el Banco Mundial y que está separado del balance del Fondo.

Las \marcar{políticas implementadas por el FMI} giran alrededor de tres ejes principales: políticas de \textbf{asistencia financiera}, \textbf{políticas de supervisión} y \textbf{políticas de asistencia técnica}. Las primeras se pueden clasificar a su vez en función del destino de los fondos y los requisitos de obtención. La llamada \underline{asistencia estándar} se caracteriza por la libertad de acceso a unas cantidades determinadas en función de la cuota aportada y cuyos instrumentos principales son los llamados acuerdos \textit{Stand-By}, la asistencia del \textit{Extended Fund Facility} y el \textit{Instrumento de Financiación Rápida}. Todos ellos se ven modulados en caso de concurrir circunstancias especiales que aumentan el límite de fondos prestables y las condiciones de devolución. Las líneas de \underline{asistencia precautoria} son el resultado de un esfuerzo por reducir el estigma internacional de la asistencia del Fondo y permiten a economías sujetas a tensiones de la balanza de pagos contar con un colchón de liquidez disponible en caso de problemas pero que no necesariamente debe movilizarse. Entre ellos, el \textit{Flexible Credit Line} se caracteriza por la condicionalidad \textit{ex-ante} y el \textit{Precautory Liquidity Line} lo hace a su vez por la condicionalidad ex-post. En ambos casos, su utilización se limita hasta ahora a apenas un puñado de países. La \underline{asistencia para países pobres} o condicional dispone a su vez de varios instrumentos diferenciados en función de la cuantía y las razones de su posible utilización (necesidades duraderas estructurales o shocks inesperados de cuantía reducida) que permiten a economías con problemas acceder a crédito destinado a financiar su desarrollo, Las políticas de supervisión se enmarcan en las famosas \textit{consultas del artículo IV} y se suponen en general un esfuerzo de control periódico de las reformas y las políticas nacionales que puedan afectar a la estabilidad del sistema financiero, así como una evaluación de los regímenes de tipo de cambio y sus características factuales frente a lo que los países declaran, y una supervisión multilateral que trata de evaluar la situación de estabilidad global. La fijación de estándares es el último elemento de esta labor de \textbf{supervisión} y se concreta en la publicación de estándares internacionales de información financiera tales como el Manual de la Balanza de Pagos o la certificación de sistemas estadísticos nacionales. La \textbf{asistencia técnica}, por último, permite a los estados miembros hacer uso de la experiencia de \textit{policy making} acumulada del Fondo. Resulta especialmente relevante en el caso de economías en desarrollo sin cuerpos funcionariales con la suficiente formación o recursos para afrontar el diseño y la ejecución de políticas necesarias para permitir un desarrollo sostenido.

En el marco de estas tres referencias de actuación, el FMI ha conformado a lo largo de las décadas el llamado sistema o modelo de \marcar{programación financiera}. Se trata de la herramienta básica de diseño de intervenciones del Fondo. La razón de ser de la programación financiera es disponer de un marco de diseño de políticas de intervención con el que poder predecir la evolución de la economía, formular un diagnóstico y unos objetivos macroeconómicos, e implementar un plan de intervención que haga un uso coherente de los instrumentos disponibles para el \textit{policy maker}. La \textbf{formulación} de la programación financiera comienza con la \underline{construcción de cuatro proyecciones macroeconómicas} para los sectores real, monetario, fiscal y exterior así como con una visita de los técnicos del FMI de carácter similar a una \textit{due diligence} en el sector privado. En esta visita, los técnicos tratan de comprobar el estado efectivo de las cuentas públicas y las estadísticas nacionales: en definitiva, conocer el estado real de la economía. A continuación, y una vez presentado un \underline{diagnóstico} concretado en una serie de desequilibrios y realizado un análisis de sostenibilidad de la deuda, se formula una serie de \underline{objetivos de intervención} específicos a la economía examinada y se ligan a los \underline{instrumentos} de actuación que efectivamente están al alcance. El éxito o el fracaso de la aplicación del \textit{financial programming} tiene que ver con factores tales como el gradualismo del ajuste propuesto, el volumen de financiación prestado, la propiedad local o \textit{ownership} del programa de intervención, el contenido de la reforma estructural y la reestructuración de la deuda o ausencia de ella. Con el paso de las décadas, el modelo de programación financiera se ha convertido en un referente de la actuación del FMI en todos sus programas, y a pesar de las críticas, ha resistido bien la presión de los cambios en el sistema monetario y económico global. 

El impacto global de la existencia del FMI desde la Segunda Guerra Mundial hasta la actualidad arroja un balance claramente positivo. Ha realizado una enorme contribución a la estabilidad del sistema económico mundial, ha permitido a numerosas economías recuperar la senda del crecimiento y ha evitado ajustes bruscos de la balanza de pagos. Sin embargo, sus actuaciones más relevantes están unidas de forma inherente a momentos de crisis y por ello ha tendido también a ser utilizado como chivo expiatorio y ha cometido errores. La labor de la \textit{Independent Evaluation Office} es muy relevante en este sentido y consiste en evaluar la actuación del fondo y ejercer una muy necesaria autocrítica en una institución con tal poder y responsabilidad. La evolución futura del fondo estará sin duda ligada a los cambios en el contexto económico y político internacional. Las amenazas actuales al multilateralismo suponen un importante reto para el Fondo y su capacidad de adaptación a las nuevas realidades de la economía global.

\seccion{Preguntas clave}

\begin{itemize}
    \item ¿Qué es el FMI?
    \item ¿Para qué sirve?
    \item ¿Cómo ha evolucionado?
    \item ¿Qué políticas lleva a cabo?
    \item ¿Es positivo el balance de sus actuaciones?
    \item ¿Qué papel juega el FMI en relación con los países en desarrollo?
    \item ¿Qué perspectivas futuras tiene el FMI como institución?
\end{itemize}

\esquemacorto

\begin{esquema}[enumerate]
	\1[] \marcar{Introducción}
		\2 Contextualización
			\3 Sistema Monetario Internacional
			\3 Mecanismos globales de estabilización
			\3 Multilateralización de la asistencia
		\2 Objeto
			\3 Qué es y para qué sirve FMI
			\3 Cómo se estructura
			\3 Cómo actúa
			\3 Qué efectos tiene sobre PED
		\2 Estructura de la exposición
			\3 Funciones y antecedentes
			\3 Estructura del FMI
			\3 Políticas del FMI
			\3 Políticas de estabilización en países en desarrollo
	\1 \marcar{Funciones y antecedentes}
		\2 Funciones
			\3 Artículo I: objetivos
			\3 Supervisión e influencia
			\3 Soporte economías en problemas
			\3 Apoyo países en desarrollo
			\3 Mantenimiento sistema multilateral de pagos
		\2 Antecedentes
			\3 Creación hasta fin de Bretton Woods
			\3 Años 80
			\3 Años 90
			\3 Años 2000
			\3 Crisis financiera
	\1 \marcar{Estructura del FMI}
		\2 Organizativa
			\3 Junta de gobernadores
			\3 Comité Monetario y Financiero Internacional
			\3 Comité del Desarrollo
			\3 Directorio Ejecutivo
			\3 Gerencia del FMI
			\3 Personal del FMI
			\3 Toma de decisiones
		\2 Financiera
			\3 Balance del FMI: General Resources Account
			\3 Activo
			\3 Pasivo
			\3 Cuenta de Resultados
			\3 Unidad de cuenta
			\3 Fondo para la Reducción de la Pobreza y el Crecimiento
			\3 Reforma de cuotas y recursos de 2019
	\1 \marcar{Políticas del FMI}
		\2 Asistencia financiera
			\3 Liquidez de emergencia -- SLL
			\3 Asistencia estándar
			\3 Asistencia precautoria
			\3 Asistencia para países pobres: PRGT
		\2 Supervisión
			\3 Políticas económicas: artículo IV
			\3 Supervisión Sector Financiero
			\3 Supervisión de tipos de cambio
			\3 DSA para países con acceso al mercado
			\3 DSA para PEDs
			\3 EBA -- External Balance Assessment models
			\3 Supervisión multilateral
			\3 Fijación de estándares
		\2 Asistencia técnica
			\3 Diseño de políticas
			\3 Formación y capacitación de funcionarios
			\3 Programación financiera
		\2 Evaluación de la efectividad
			\3 Idea clave
			\3 Revisión de la condicionalidad y el diseño
			\3 Resultados
			\3 Problemas de evaluación
	\1 \marcar{Políticas de estabilización del FMI en países en desarrollo}
		\2 Idea clave
			\3 Contexto
			\3 Objetivos
			\3 Resultados
		\2 Sudden stops y crisis de balanza de pagos
			\3 Idea clave
			\3 Formulación
			\3 Implicaciones
			\3 Valoración
		\2 Controles de capital
			\3 Idea clave
			\3 Tipos de controles de capital
			\3 Evolución histórica
			\3 Actualidad
			\3 Implicaciones
		\2 Programación financiera
			\3 Idea clave
			\3 Formulación
			\3 Variables determinantes del éxito de la intervención
			\3 Autoevaluación de actuaciones
			\3 Valoración
	\1[] \marcar{Conclusión}
		\2 Recapitulación
			\3 Funciones y antecedentes
			\3 Estructura del FMI
			\3 Políticas del FMI
			\3 Políticas de estabilización en países en desarrollo
		\2 Idea final
			\3 Impacto desde Guerra Mundial
			\3 Constante evolución
			\3 Futuro del FMI

\end{esquema}

\esquemalargo













\begin{esquemal}
	\1[] \marcar{Introducción}
		\2 Contextualización
			\3 Sistema Monetario Internacional
				\4 Dinámico y complejo
				\4 Inestabilidad periódica
				\4 Historia: Bretton Woods, Nixon Shock, Años 90
			\3 Mecanismos globales de estabilización
				\4 FMI
				\4 Banco Mundial
				\4 WTO
			\3 Multilateralización de la asistencia
				\4 Mayor capacidad de estabilización frente a shocks
				\4 Centralización de conocimiento
				\4 Supervisión y acumulación de experiencia
		\2 Objeto
			\3 Qué es y para qué sirve FMI
			\3 Cómo se estructura
			\3 Cómo actúa
			\3 Qué efectos tiene sobre PED
		\2 Estructura de la exposición
			\3 Funciones y antecedentes
				\4 Funciones
				\4 Antecedentes
			\3 Estructura del FMI
				\4 Organizativa
				\4 Financiera
			\3 Políticas del FMI
				\4 Asistencia financiera
				\4 Supervisión
				\4 Asistencia técnica
				\4 Evaluación de la actividad
			\3 Políticas de estabilización en países en desarrollo
	\1 \marcar{Funciones y antecedentes}
		\2 Funciones
			\3 Artículo I: objetivos
				\4[i] Promover cooperación monetaria internacional
				\4[ii] Promover expansión del comercio internacional
				\4[] Crecimiento del comercio equilibrado
				\4[] Contribuir a aumentar:
				\4[] $\to$ Niveles de empleo
				\4[] $\to$ Renta
				\4[] $\to$ Desarrollo de recursos productivos
				\4[iii] Fomentar estabilidad cambiaria
				\4[] Evitar depreciaciones competitivas
				\4[iv] Establecer sistema multilateral de pagos
				\4[] Eliminar restricciones cambiarias de CC
				\4[] $\to$ Evitar distorsiones del CC
				\4[v] Permitir ajuste de BP sin medidas destructivas
				\4[] Suavizar ajustes de balanza de pagos
				\4[] $\to$ Proveyendo ayuda temporal
				\4[vi] Reducir duración y grado de desequilibrios en BP
			\3 Supervisión e influencia
				\4 Evaluación políticas macroeconómicas
				\4 Polo de conocimiento alto nivel macroeconomía
				\4 Diseminación información macro
				\4 Analizar economías (misiones artículo IV)
				\4[] Alertar sobre desequilibrios
				\4[] Monitorizar sistema financiero
			\3 Soporte economías en problemas
				\4 Apoyo financiero economías con desequilibrios
				\4 Diseño planes de ajuste
				\4 Transferencia conocimiento
				\4 'Sello' de confianza del FMI
				\4[] $\to$ genera confianza en los inversores
			\3 Apoyo países en desarrollo
				\4 Canalización recursos financieros
				\4 Préstamos concesionales
				\4 Conjuntamente con Banco Mundial
				\4 Proveer acceso mercados financieros
			\3 Mantenimiento sistema multilateral de pagos
				\4 Eliminación restricciones pagos
				\4 Estabilidad del sistema
				\4 Prevención devaluaciones competitivas
		\2 Antecedentes
			\3 Creación hasta fin de Bretton Woods
				\4 Mantenimiento sistema de tipos de cambio fijos
				\4[] Proveer divisas para equilibrar balanzas
				\4[] Organización similar a una cooperativa de crédito
				\4 Progresiva liberalización de cuentas corrientes
				\4 Consolidación metodología FMI: programación financiera
				\4 Relativo éxito primeras décadas
				\4 A partir de 60s, fuertes desequilibrios emergen
				\4[] Déficit cuenta corriente EEUU
				\4[] Incremento de flujos financieros
				\4 Crisis del Reino Unido 1976
				\4 Hasta aquí, mayoría intervenciones acuerdos stand-by
			\3 Años 80
				\4 Consolidación sistema tipos flotantes
				\4 Sin papel supervisor sistema de tipos fijos
				\4 Consolidación papel supervisor en crisis
				\4 Crisis de primera generación en Latinoamérica
				\4 Baker (sin reestructuración)
				\4 Brady (reducción de deuda a cambio de reformas)
			\3 Años 90
				\4 Crisis del Tequila (1994) México
				\4[] Paquete de ayuda de 18.000 M de \$
				\4 Crisis Asiática 1997-1998
				\4[] Experiencia gestión de crisis
				\4[] Enorme crecimiento cuantías programas de ayuda
				\4 Crisis Rusia 1998:
				\4[] Primer caso FMI admite insolvencia
				\4[] Acreedores incurren en pérdidas
				\4 Financial Sector Assessment Programs (1999-)
				\4[] Programa conjunto con Banco Mundial
				\4[] Evaluar vulnerabilidades sector financiero
				\4[] Proponer medidas prevención de crisis
			\3 Años 2000
				\4 Crisis Argentina 2001
				\4[] Presiones políticas en Junta de Gobernadores
				\4[] Para aprobar apoyo financiero Argentina
			\3 Crisis financiera
				\4 Criticado por no prever riesgos en Europa
				\4 Grecia
				\4[] Acuerdo Stand-by 2010
				\4[] Extended Fund Facility 2012
				\4 Influyente idea Unión Bancaria
				\4 Propone ruptura vínculo sistema bancario-deuda soberana
				\4 Extended Arrangement Irlanda (2010-2014)
				\4 Extended Arrangement Portugal (2011-2013)
				\4 Aumento de los recursos
				\4[] Vía New Arrangements to Borrow (500.000 M USD)
	\1 \marcar{Estructura del FMI}
		\2 Organizativa\footnote{\href{https://www.imf.org/en/About/Factsheets/Sheets/2016/07/27/15/24/How-the-IMF-Makes-Decisions}{IMF (2019): How the IMF Makes Decisions}}
			\3 Junta de gobernadores
				\4 \underline{Funciones}
				\4[] Máximo órgano de decisión
				\4[] Refrendo formal decisiones DEjecutivo
				\4[] Aprobar incrementos de la cuota
				\4[] Aprobar emisiones de SDR
				\4[] Admisión de nuevos miembros
				\4[] Expulsión de miembros
				\4[] Enmiendas a los artículos de asociación
				\4[] Elige miembros de Directorio Ejecutivo
				\4[] Mayoría necesaria para decisiones estratégicas:
				\4[] $\to$ 85\% votos
				\4[] Poder de veto EEUU
				\4 \underline{Composición}
				\4[] 189 países miembros\footnote{España entró en el 1958.}
				\4[] Dos gobernadores por país
				\4[] $\to$ Titular y alternativo
				\4[] $\then$ Ministro o gobernador banco central
			\3 Comité Monetario y Financiero Internacional
				\4 \underline{Funciones}
				\4[] Definición líneas estratégicas de trabajo
				\4[] Comunicados semestrales
				\4[] Diálogo entre ministros/gobernadores
				\4[] Pérdida de peso frente a G7, G20 (2009) y BIS
				\4[] Precedido por Comité Interino
				\4 \underline{Composición}
				\4[] Similar al directorio
				\4[] Ministros y gobernadores de bancos centrales
				\4[] Observadores de otras instituciones internacionales
				\4[] $\to$ Banco Mundial
				\4[] $\to$ Financial Stability Board
			\3 Comité del Desarrollo
				\4 \underline{Funciones}
				\4[] Asesorar Junta de Gobernadores
				\4[] Cuestiones relativas a PEDs
				\4 \underline{Composición}
				\4 25 miembros representan conjuntamente
				\4 Conjunto directorios ejecutivos de:
				\4[] $\to$ FMI
				\4[] $\to$ Banco Mundial
			\3 Directorio Ejecutivo
				\4 \underline{Funciones}
				\4[] Día a día de la institución
				\4[] Aprobación programas
				\4[] Mayoría simple/consenso
				\4[] $\to$ Método habitual de decisión
				\4[] Materias estratégicas: 85\%
				\4 \underline{Composición}
				\4[] 24 sillas
				\4[] 8 países silla propia
				\4[] USA, CHI, JAP, GER, FRA, UK, RUS\footnote{Conjunto con Siria.}, KSA
				\4[] 16 sillas restantes agrupan países
				\4[] España:
				\4[] $\to$ 1,92\% de los votos
				\4[] $\to$ Comparte silla con MEX, HispanoAm.
			\3 Gerencia del FMI
				\4 \underline{Funciones}
				\4[] Liderazgo de la institución
				\4 \underline{Composición}
				\4[] Director Gerente:
				\4[] $\to$ Europeo por costumbre
				\4[] $\to$ 5 años renovables
				\4[] $\to$ Nombrado por Directorio Ejecutivo
				\4[] $\to$ Generalmente por consenso
				\4[] Subdirectores gerentes (4):
				\4[] $\to$ accionistas principales
				\4[] $\to$ receptores de programas
				\4[] Debate reparto de cargos e influencia
				\4[] Por debajo: directores de departamento
			\3 Personal del FMI
				\4 Predominio univ. anglosajonas
				\4 Diversidad nacionalidades
			\3 Toma de decisiones
				\4 Principio de accionista
				\4 Voto: cuotas + votos básicos
				\4 \fbox{Cuota: $(0,5 \textrm{PIB}+0,3 A + 0,15 V + 0,05 R)^k$}
				\4[] PIB: $0,6 \textrm{PIB}_{\textrm{pm}} + 0,4 \textrm{PIB}_{\textrm{ppa}} $
				\4[] A: $\sum_{t=1}^5 \textrm{Ing. corr.}_t + \sum_{t=1}^5 \textrm{Pagos corr.}_t$
				\4[] V: $\sigma$ de media movil de 3 años ing. corr. y flujos de capital
				\4[] R: media de reservas de 12 meses\footnote{Incluido DEG, oro.}
				\4[] k: 0,95
				\4 Aumentos de las cuotas
				\4[] 2006
				\4[] Reforma 2010
				\4[] $\to$ Entrada en vigor en 2016
				\4[] $\then$ Tras cumplimiento de condiciones
				\4[] España: 5a economía europea
				\4[] Corrige infrarrepresentación
				\4[] 2019: posible revisión de cuotas
				\4 Facciones: G7, G20, G24, BRICS, UE...
		\2 Financiera
			\3 Balance del FMI: General Resources Account
				\4 FMI como cooperativa de Bancos Centrales
				\4 Reforma 2010: enorme aumento de cuotas
			\3 Activo\footnote{Ver Financial Statements, Statements of Financial Position at July 31 2019 -- En carpeta del tema}
				\4 \textbf{Total: 690.000 M USD}
				\4 Divisas
				\4 Derechos de cobro sobre programas
				\4 Oro
				\4 Inmuebles y patrimonio
			\3 Pasivo\footnote{Ver PIIE (2018)}
				\4 \textbf{Total: 690.000 M USD/512.000 M SDR\footnote{A 1.38 USD/SDR.}}
				\4 Cuotas: 690.000 M \$
				\4[] Último cálculo en 2016
				\4[] Prevista revisión para 2019
				\4[] $\to$ Pospuesta\footnote{\href{https://www.reuters.com/article/us-imf-quotas-japan/imf-to-postpone-planned-quota-increase-due-to-u-s-resistance-source-idUSKBN1WN1GC}{Reuters sobre propuesta de aumento de cuotas pospuesto.}}
				\4 Estrictamente fuera del balance
				\4[] NABs:\footnote{New Arrangements to Borrow. Expiran en 2019 pero pueden ampliarse un año más. Compromiso de 39 países para contribuir fondos adicionales en caso de necesidad. } 265.000 M \$
				\4[] $\to$ Aprobado doblar en 2021 los NAB
				\4[] $\to$ Periodo 2021-2025
				\4[] $\to$ Contexto de dificultad para renovar cuotas
				\4[] Préstamos bilaterales 450.000 M \$\footnote{Ver \url{https://www.imf.org/en/News/Articles/2019/11/05/pr19395-2016-bilateral-borrowing-agreements-about-us433bil-terms-extend-by-an-additional-yr-end2020}. Ampliación de un año hasta finales de 2020.}
				\4[] $\to$ Prevista expansión a 365.000 M de SDR desde 2021
				\4[] $\then$ Para periodo 2021-2025
				\4[] $\then$ Mantener capacidad cercana a 1 B de USD
				\4[$\then$] 1.000.000 M de USD disponibles para prestar
				\4[$\then$] 1.77 del PIB mundial
			\3 Cuenta de Resultados
				\4 Márgenes de intermediación
				\4 No puede asumir pérdidas
			\3 Unidad de cuenta
				\4 Derechos Especiales de Giro (DEGs)
				\4[] Derecho potencial de adquisición de divisas
				\4[] No son parte de cuotas
				\4[] No son parte de recursos para préstamo
				\4[] Asignados en función de cuotas
				\4[] Divisas: USD, EUR, RMB, JPY, GBP
				\4[] RMB añadido en 2016
				\4[] Creado en 1969
				\4 Valor del DEG:\footnote{\url{http://www.imf.org/external/np/fin/data/rms\_sdrv.aspx}.}
				\4[] DEG compra cantidad fija $\vec{x}$ de divisas
				\4[] ¿Cuántos USD para comprar $\vec{x}$ de divisas?
				\4[] $\to$ Cantidad necesaria = valor en USD de 1 DEG
				\4[] Cantidades fijas consultables
				\4 Emitidos\footnote{Ver \href{https://www.imf.org/external/np/fin/tad/extsdr1.aspx}{FMI: Asignación de DEGs}}
				\4[] Hasta 2020, ~204.000 M de DEGs
				\4[] $\then$ 280.000 M de USD
			\3 Fondo para la Reducción de la Pobreza y el Crecimiento
				\4 Separado del Balance General
				\4 Préstamos concesionales
				\4 Conjunto con Banco Mundial
				\4 Condcionalidad
				\4 Sujetos a PRSP
				\4[] Poverty Reduction Strategic Papers
				\4[] $\to$ Programas de reducción de pobreza asociados
			\3 Reforma de cuotas y recursos de 2019\footnote{Ver PIEE (2018).}
				\4 EEUU tiene actualmente poder de veto
				\4[] $16,5\%$ de votos en junta de gobernadores\footnote{Ver \url{https://www.imf.org/external/np/sec/memdir/members.aspx}.}
				\4[] $\to$ Necesario $>85\%$ para tomar decisiones
				\4 Se opone a reforma:
				\4[] $\to$ Aumento de las cuotas
				\4[] $\to$ Quién paga las cuotas
				\4 Potenciales características de la reforma
				\4[] Más recursos como \% mundial
				\4[] Más peso de PEDs
				\4 Prórrogas durante varios años
				\4[] $\to$ Cuotas sin reformar
				\4 NAB -- New Arrangements to Borrow
				\4[] Compromiso de préstamo a FMI
				\4[] $\to$ Si cuotas no son suficientes
				\4[] Principalmente países del G-7
				\4[] Activados sólo en emergencias
				\4[] Tienen caducidad
				\4[] $\to$ En 2020 y años siguientes
				\4[] $\then$ ¿Qué pasa después?
				\4 Posición americana
				\4[] Oposición a reducir peso en derechos de voto
				\4[] $\to$ Miedo a perder veto
				\4[] FMI ya tiene suficientes recursos
				\4[] No es necesario aumentar
				\4[] $\to$ Aunque caigan en términos relativos a PIB mundial
				\4[] $\to$ Aunque caigan en términos absolutos
				\4 Controversia cuotas europeas
				\4[] Algunos países demandan reducción de peso europeo
				\4[] Países europeos participan por separado
				\4[] $\to$ No como UE
				\4[] $\then$ Cuotas y votos básicos por separado
				\4[] En la práctica, coordinación informal de voto
				\4[] $\to$ Actúan como un sólo miembro
	\1 \marcar{Políticas del FMI}
		\2 Asistencia financiera\footnote{Ver \url{https://www.imf.org/en/About/Factsheets/IMF-Lending}.}
			\3 Liquidez de emergencia -- SLL\footnote{Ver \href{https://www.imf.org/en/News/Articles/2020/04/21/pr20180-imf-executive-board-covid-19-response-new-sll-enhance-adequacy-global-financial-safety-net}{FMI (2020)}}
				\4 Creada en abril de 2020
				\4 Préstamos revolving hasta 145\% de cuota
				\4 Tensiones transitorias de balanza de pagos
				\4[] Derivadas fundamalmente de cuenta financiera y reservas
				\4[] Resultado de volatilidad mercados de capital
				\4 Requisitos similares a FCL
				\4[] Buena trayectoria previa de implementación de políticas
				\4[] Fundamentales sólidos
				\4 Válido para 7 años -- hasta 2025
				\4[] Posible extensión posterior
				\4 Consecuencias previstas sobre balance FMI
				\4[] Países que reciban no participarán en NAB
				\4[] Recursos potencialmente bloqueados por periodo largo
				\4[] $\to$ Por carácter revolving
			\3 Asistencia estándar
				\4 Pago de interés de los DEG
				\4[] Más recargos según importe y vencimiento
				\4 \underline{Acuerdos Stand-By} -- SBA
				\4[] Programa estándar del FMI
				\4[] Plazo corto de repago
				\4[] Acceso normal: <145\% de la cuota
				\4[] Acceso excepcional: >145\% hasta 435\%
				\4[] Duración del programa de condicionalidad
				\4[] $\to$ Uno a dos años
				\4[] Plazo de pago
				\4[] $\to$ De 3 a 5 años
				\4[] Interés:
				\4[] $\to$ interés de SDR\footnote{Calculado a partir de tipos de interés de monedas de l} + 100pb + sobrecargos por exceso de cuota
				\4[] $\to$ creciente con tiempo
				\4[] $\to$ creciente con \% de cuota
				\4[] $\to$ pago mínimo por crédito comprometido
				\4 \underline{Extended Fund Facility} -- EFF\footnote{\url{https://www.imf.org/en/About/Factsheets/Sheets/2016/08/01/20/56/Extended-Fund-Facility}}
				\4[] Énfasis en condicionalidad estructural
				\4[] Países de bajo crecimiento
				\4[] Insuficiencias crónicas de la balanza de pagos
				\4[] Duración del programa de condicionalidad
				\4[] $\to$ Tres a cuatro años
				\4[] Periodo más largo de repago
				\4[] $\to$ De 4.5 a 10 años
				\4[] Mayor condicionalidad que Stand-By
				\4[] Más énfasis en reformas estructurales
				\4[] Mayor probabilidad de cumplimiento de objetivos
				\4 \underline{Instrumento de Financiación Rápida} -- RFI \footnote{Rapid Financing Instrument.}\footnote{\url{https://www.imf.org/en/About/Factsheets/Sheets/2016/08/02/19/55/Rapid-Financing-Instrument}}
				\4[] Acceso inmediato a fondos
				\4[] $\to$ Sin necesidad de programa/revisiones
				\4[] Máximo 50\% de cuota por año
				\4[] Aumentado hasta 60\% si desastre natural
				\4[] 100\% de la cuota en términos acumulados
				\4[] 133\% si derivado de un desastre natural grave
				\4[] Sin necesidad de programa o revisiones
				\4[] Pero colaboración requerida
				\4[] Devolución de 3 años y cuatro meses a cinco años
				\4 \underline{Programas con Marco de Acceso Excepcional}
				\4[] Permiten acceso sin límite a fondos
				\4[] Necesaria solvencia de la deuda
				\4[] Necesaria fuerte presión sobre balanza de de pagos
				\4[] Ajuste necesario es factible
				\4[] Más condiciones
				\4[] Cumplimiento de condicionalidad más pobre
				\4[] Menor probabilidad de alcanzar objetivos
			\3 Asistencia precautoria
				\4 \underline{Flexible Credit Line} -- FCL
				\4[] Condicionalidad ex-ante revisada bianualmente
				\4[] Países con fundamentos muy sólidos
				\4[] $\to$ Condicionalidad \textit{ex-ante}
				\4[] México, Colombia, Polonia
				\4[] Devolución de 3 años y tres meses a 5 años
				\4[] Sin límite de acceso
				\4[] Interés similar a otros programas de FMI
				\4 \underline{Precautory Liquidity Line} -- PLL\footnote{\url{https://www.imf.org/en/About/Factsheets/Sheets/2016/08/01/20/45/Precautionary-and-Liquidity-Line}}
				\4[] Países con fundamentos sólidos
				\4[] Condicionalidad ex-ante y ex-post
				\4[] Objetivo: reducir estigma de recurrir a FMI
				\4[] Duración del programa:
				\4[] $\to$ 6 meses, o de 1 a 2 años
				\4[] 125\% de cuota anual para programas de 6 meses
				\4[] $\to$ 250\% de acceso total acumulado en programas siguientes
				\4[] 250\% de cuota anual para programas mayores
				\4[] $\to$ 500\% total acumulado
				\4[] Interés similar  a otros programas
				\4[] $\to$ Más pago por ``compromiso'' si decide acceder
				\4[] Marruecos
			\3 Asistencia para países pobres: PRGT\footnote{Poverty Reduction and Growth Trust.}
				\4 Característica común:
				\4[] $\to$ Subsidio de tipos de interés
				\4[] $\to$ Diferentes programas de asistencia
				\4 \underline{Standby Credit Facility}
				\4[] Problemas normales de balanza de pagos
				\4 \underline{Extended Credit Facility}
				\4[] Problemas estructurales de largo alcance
				\4[] Plazo de repago largo (5.5 a 10 años)
				\4[] + carencia
				\4 \underline{Rapid Credit Facility}
				\4[] Importes modestos
				\4[] Shocks inesperados
		\2 Supervisión
			\3 Políticas económicas: artículo IV
				\4 \textit{``article IV consultations''}
				\4 Supervisión periódica de economías individuales
				\4 Anuales países relevantes
				\4 Bianuales países pequeños
				\4 Fuerte impacto público
				\4 Generalmente, conclusiones tenidas en cuenta
			\3 Supervisión Sector Financiero
				\4 Financial Sector Assessment Program (FSAPs)
				\4[] Fundamento de artículo IV
				\4[] Estabilidad del sistema financiero en conjunto
				\4[] $\to$ No de instituciones financieras concretas
				\4 Incluyen Financial System Stability Assessment (FSSAs)
				\4[] $\to$ \textit{stress tests}
				\4 Sistemas financieros importancia sistémica
				\4[] FSAPs cada 5 años
			\3 Supervisión de tipos de cambio
				\4 Recoger regímenes \textit{de jure} en informe anual
				\4 Recoger regímenes \textit{de facto}
				\4 Fijación libre de regimen de tipo salvo \textit{peg} con oro
				\4 Publicación base de datos COFER\footnote{Composition of Foreign Exchange Reserves}
				\4 Análisis desequilibrios externos
				\4[] Valoraciones Sector Exterior (External Balance Assessments)
				\4[] Informes sobre el Sector Exterior (External Sector Report)
			\3 DSA para países con acceso al mercado
				\4 Debt Sustainability Analysis
				\4 \url{https://www.imf.org/external/pubs/ft/dsa/mac.htm}
			\3 DSA para PEDs
				\4 \url{https://www.imf.org/external/pubs/ft/dsa/lic.aspx}
			\3 EBA -- External Balance Assessment models
				\4 Gap de cuenta corriente
				\4 Gap de TCER
			\3 Supervisión multilateral
				\4 Elaboración informes estabilidad global
				\4 World Economic Outlook
				\4 Global Financial Stability Report
				\4 Fiscal Monitor
				\4 ESR -- External Sector Reports
				\4 Gran impacto mediático
				\4 Tendencia creciente a analizar \textit{spillovers}
			\3 Fijación de estándares
				\4 Fijación estándares estadísticos
				\4 VI Manual de la Balanza de Pagos
				\4 Estadísticas Financieras Públicas
				\4 Estadísticas mercado de valores
				\4 Sello de calidad de estadísticas nacionales
				\4[] Diferentes niveles
				\4 Supervisión cumplimiento estándares
		\2 Asistencia técnica
			\3 Diseño de políticas
				\4 Generalmente resultado de supervisión
				\4 España 2012: recapitalización sector financiero
				\4 Italia 2012: reforma fiscal
				\4 Grecia, Portugal, Irlanda
			\3 Formación y capacitación de funcionarios
				\4 Especial utilidad en países en desarrollo
				\4 Gestión deuda pública
				\4 Suministro sistemas informáticos
				\4 Cursos de formación en diferentes ámbitos
			\3 Programación financiera
				\4 A continuación
		\2 Evaluación de la efectividad\footnote{Ver BdE (2020).}
			\3 Idea clave
				\4 FMI autoevalúa el resultado de sus actuaciones
				\4 Revisiones periódicas
				\4 Parte fundamental de institución
				\4[] ¿Qué se hizo bien?
				\4[] ¿Qué funcionó?
				\4[] ¿Qué metodología para valorar resultados?
			\3 Revisión de la condicionalidad y el diseño
				\4 Completada en 2019
				\4 Propuesta de marco sistemático de evaluación
				\4 Construcción de indicadores para revisar condicionalidad
			\3 Resultados
				\4 Reducción de desequilibrios internos y externos
				\4[] Evidencia favorable
				\4[] Tienden a reducirse tras implementar programas
				\4 Otros factores relevantes
				\4[] Crecimiento
				\4[] Distribución de la renta
				\4[] Acceso a mercados financieros
				\4[] Inflación
				\4[] Desigualdad
				\4[] $\to$ Evidencia poco concluyente
				\4 Fatiga de las reformas
				\4[] Fenómeno recurrente
				\4[] Cada vez más difícil seguir reformando
				\4[] $\to$ Incentivos a reformar se debilitan
				\4[] $\to$ Formación de coaliciones anti-reforma
			\3 Problemas de evaluación
				\4 Correlación o causalidad
				\4[] Resultados finales son causa de:
				\4[] $\to$ Programa aplicado?
				\4[] $\to$ Condiciones previas?
				\4 Programas más duros tienen peores resultados
				\4[] ¿Programas más duros son peores?
				\4[] ¿Peores perspectivas inducen programas más duros?
	\1 \marcar{Políticas de estabilización del FMI en países en desarrollo}
		\2 Idea clave
			\3 Contexto
				\4 Crisis exteriores
				\4[] Posibles y de hecho tienen lugar
				\4[] $\to$ En países con todo tipo de niveles de renta
				\4 Países en desarrollo
				\4[] Mayor susceptibilidad
				\4[] $\to$ Características estructurales de economía
				\4[] $\to$ Dependencia exterior
				\4[] $\to$ Poca diversificación de exportaciones
				\4[] $\to$ ...
				\4 Crisis de PEDs
				\4[] Problemas de financiación exterior
				\4[] Recesiones
				\4[] Inestabilidad política
				\4[] Impagos de deuda
				\4[] ...
				\4 Papel de FMI
				\4[] ``Banco es fondo y fondo es banco''
				\4[] Conjunto de políticas de asistencia anteriores
				\4[] $\to$ Aplicadas sobre PED en crisis determinado
			\3 Objetivos
				\4 Políticas de estabilización
				\4[] Restablecer equilibrio externa
				\4[] Retomar senda de endeudamiento sostenible
				\4[] Maximizar output
				\4 Servir como palanca de reforma
				\4[] Proveer incentivos políticos domésticos
				\4 Aprovechar conocimiento acumulado
				\4 Valorar efectividad de intervenciones
			\3 Resultados
				\4 Asistencia financiera en crisis de BP
				\4[] Generalizada
				\4[] Aún más importante actualmente con crisis de pandemia
		\2 Sudden stops y crisis de balanza de pagos
			\3 Idea clave
				\4 Contexto
				\4[] Dornbusch: not speed but the sudden stop that kills
				\4[] Ajustes bruscos son muy costosos
				\4[] $\to$ Recesión
				\4[] $\to$ Crisis financieras y bancarias
				\4 Objetivos
				\4[] Entender causas y consecuencias del ajuste brusco
				\4[] Evitar ajustes bruscos
				\4[] $\to$ Permitir suavización del ajuste
				\4[] $\to$ Bases para sostenibilidad de trayectoria
				\4 Resultados
				\4[] Liberalizaciones de CFinanciera deben evitar $\uparrow$ riesgos
				\4[]
			\3 Formulación
				\4 $S-I+\text{CK} = \underbrace{\text{NX}+\text{RP} + \text{RS}}_{\text{CC}} +\text{CK} = \text{VNA} - \text{VNP}$
				\4 Financiación de déficit en CC y CK
				\4[] Requiere entrada de capital
				\4 Sudden-stop y flow reversals
				\4[] Reducción brusca de aumento de pasivos netos
				\4[] $\to$ Extranjeros venden pasivos nacionales a residentes
				\4[] $\then$ Caída del precio de los pasivos nacionales
				\4[] $\then$ Aumento del coste de la financiación
				\4[] $\then$ Depreciación del tipo de cambio
				\4[] $\then$ Impide financiar déficit de CC
				\4 Factores de riesgo
				\4[] Libre movimiento de capital
				\4[] Préstamos de corto plazo
				\4[] Endeudamiento en moneda extranjera
				\4[] Pequeño sector exportador
				\4[] $\to$ Ante reversión, necesaria reasignación a sector exterior
				\4[] $\then$ Si pequeño, más reorganización necesaria
				\4[] Aumento de percepciones globales del riesgo
				\4[] $\to$ Medidas de volatilidad
				\4[] $\to$ Aumento de prima de riesgo exigida
				\4[] TCN Fijo + Libre movimiento de K
				\4[] $\to$ Vulnerabilidad clásica
				\4[] $\to$ Incentivo a ataques especulativos
				\4[] Stock de reservas pequeño
				\4[] $\to$ Países asiáticos aprenden lección tras crisis
			\3 Implicaciones
				\4 PEDs especialmente vulnerables
				\4[] Más factores de riesgo
				\4 Estructura del sector financiero doméstico es relevante
				\4 Necesarios mecanismos de emergencia
				\4[] Stand-by
				\4 Necesario reducir impacto del pánico
				\4[] Programas de asistencia precautoria
				\4[] $\to$ FCL, PLL
				\4[] $\then$ Buenos resultados
			\3 Valoración
				\4 Suceso recurrente
				\4[] Exceso de endeudamiento exterior
				\4[] Entradas de capital
				\4[] Acumulación de desequilibrios
				\4[] Solicitud de ayuda a FMI en contexto de sudden-stop
				\4[] Efectos de sudden-stop se hacen sentir a pesar de asistencia
				\4[] $\to$ FMI como chivo expiatorio por crisis
				\4 Papel del FMI es clave
				\4[] A pesar de errores históricos
				\4[] $\to$ Sesgo retrospectivo
				\4 Papel en crisis del covid
				\4[] Avalancha de solicitudes de asistencia de PEDs
				\4[] FCL: Perú, Chile, Colombia...
		\2 Controles de capital\footnote{Ver \href{https://www.imf.org/external/pubs/ft/op/op190/pdf/part1.pdf}{IMF (2000)} y \href{https://ecpr.eu/Filestore/PaperProposal/419f54b4-73ce-42cf-b3fe-4e3e6dd844d0.pdf}{Dierckx (2011)} y \href{https://www.imf.org/external/pubs/ft/op/op190/index.htm}{IMF (2000): Capital Controls: Country Experiences with Their Use and Liberalization}.}
			\3 Idea clave
				\4 Contexto
				\4[] Problemas causados por flujos de capital volátiles
				\4[] $\to$ Sudden stops
				\4[] $\to$ Aparición de burbujas especulativas
				\4[] $\to$ Desestabilización del tipo de cambio
				\4[] Sistema monetario internacional
				\4[] $\to$ Determina características de movimientos de capital
				\4[] $\to$ FMI es agente central
				\4[] Controles de capital para estabilizar economía
				\4[] $\to$ Debate de largo plazo
				\4[] $\to$ FMI participante en debate
				\4 Objetivos
				\4[] Evitar movimientos desestabilizantes de capital
				\4[] Evitar fluctuaciones excesivas del tipo de cambio
				\4[] Evitar sudden-stops y flow reverlsas
				\4 Resultados
				\4[] Evolución histórica de la postura del FMI
				\4[] Debate de largo plazo sobre controles de K
			\3 Tipos de controles de capital
				\4 Concepto
				\4[] Regulación de movimientos de capital
				\4[] Impuestos a movimientos internacionales de capital
				\4 Controles de salida
				\4[] Impuestos de salida
				\4[] Control de cambios
				\4 Controles de entrada
				\4[] Impuestos a entradas de capital
				\4[] Restricciones a propiedad de empresas domésticas
				\4[] Control de cambios
				\4 Control vía mercado
				\4[] Desincentivar flujos aunque sean posibles
				\4 Control directo
				\4[] Prohibir movimientos de capital por vía regulatoria
			\3 Evolución histórica
				\4 Antecedentes
				\4[] Movimientos de capital en patrón oro pre-guerra
				\4[] $\to$ Relativamente libres
				\4[] $\to$ Cuantías elevadas
				\4[] Entreguerras
				\4[] $\to$ Inestabilidad cambiaria
				\4[] $\to$ Limitaciones crecientes
				\4[] $\to$ Flujos desordenados
				\4 Marco de Bretton Woods
				\4[] $\to$ Controles de K generalizados
				\4[] $\to$ Comienzan a levantarse en 60s
				\4[] Caída de Bretton Woods
				\4[] Consenso post Bretton Woods
				\4[] $\to$ Movimientos de capital  pueden ser aceptables
				\4[] FMI acepta uso en determinados casos
				\4[] $\to$ Reducir vencimiento medio de financiación
				\4 Caída de Bretton Woods
				\4[] Debate sobre nuevo sistema monetario
				\4[] Estados Unidos en contra de controles
				\4 Institucionalización de liberalización en los 90
				\4[] Estados Unidos principal promotor
				\4[] Europa adopta política de liberalización de K
				\4[] $\to$ Tras Acta Única y Maastricht
				\4[] FMI asume postura oficialmente en segunda mitad de 90s
				\4[] Propuestas de institucionalizar en artículos
				\4[] $\to$ Pero crisis de los 90 frena
				\4 Crisis de los 90
				\4[] FMI receta en países asiáticos
				\4[] $\to$ Subidas de tipo de interés
				\4[] $\to$ Programas de asistencia
				\4[] $\to$ Planes de reforma vía programación financiera
				\4[] $\to$ Mantener flujos de capital
				\4[] Malasia en 1998
				\4[] $\to$ Entra en crisis cuando THA y COR recuperan
				\4[] $\to$ Impone controles de capital en 1998
				\4[] $\then$ Produce recuperación más rápida
				\4[] $\then$ Menor recesión
				\4[] $\then$ Entendido como fracaso de políticas de FMI
				\4[] $\then$ Empuja reflexión sobre controles de K
			\3 Actualidad
				\4 Controles de capital tienen ventajas e inconvenientes
				\4[] FMI admite
				\4[] $\to$ Implementados con éxito en algunas situaciones
				\4 Controles de K no sustituyen pol. macro. necesarias
				\4[] Sostenibilidad de deuda
				\4[] Sector exportador robusto
				\4[] Flexibilización de economías
				\4 Controles de K no solucionan problemas de l/p
				\4[] Acceso a financiación en crisis
				\4[] Acceso a capital para inversión
			\3 Implicaciones
				\4 Debate de largo plazo sobre flujos
				\4 Capacidad administrativa elevada necesaria
				\4[] Controles vía mercado más efectivos
				\4[] $\to$ Con capacidad administrativa elevada
				\4[] $\to$ Con sistema financiero desarrollado
				\4 Controles selectivos son poco efectivos
				\4[] Agentes encuentran rápidamente maneras de evitar
				\4[] Regulaciones
				\4 Fluctuación histórica pro y contra
				\4[] Nunca posturas absolutas en FMI
				\4[] $\to$ A pesar de simplificaciones historiográficas
				\4[] Atención a pros y contras fluctúa
		\2 Programación financiera
			\3 Idea clave
				\4 Herramienta básica FMI
				\4[] Polak, años 50
				\4[] Referente metodológico del Fondo
				\4 Sistematización de intervención
				\4[] Medir efectos de políticas
				\4[] Comparar
				\4[] Acumular experiencia
				\4 Construcción de escenarios
				\4[]  Escenario base sin intervención
				\4[]  Escenario con intervención
				\4 Asociación objetivos-instrumentos
				\4[]  Establecer efectos esperados
				\4[]  Proponer menú a autoridades locales
				\4[]  Seguimiento cumplimiento
			\3 Formulación
				\4 Construcción escenarios macro
				\4[]  Sector real
				\4[]  Sector fiscal
				\4[]  Sector Monetario
				\4[]  Sector exterior
				\4[]  Sector bancario
				\4[]  Visita del FMI
				\4 \textit{Due diligence}
				\4[] Verificar datos estadísticos
				\4[] Estado real de la economía
				\4 Diagnóstico
				\4[]  Constatación desequilibrios
				\4[]  Análisis sostenibilidad de la deuda
				\4 Objetivos macro típicos
				\4[]  Restaurar competitividad
				\4[]  Restaurar disciplina monetaria y fiscal
				\4[]  Reducir inflación
				\4[]  Restaurar crecimiento
				\4[]  Formulación objetivos cuantitativos
				\4[]  Reformas estructurales
				\4 Instrumentos de ajuste
				\4[] Ajuste de la demanda agregada
				\4[] Política fiscal contractiva
				\4[] $\to$ Reducir absorción interna
				\4[] $\to$ Mejorar sostenibilidad de deuda pública
				\4[] Política monetaria contractiva
				\4[] $\to$ Estimular entrada de flujos de capital
				\4[] Reformas estructurales
				\4[] Devaluación
				\4 Reestructuración de la deuda
				\4[] Pública
				\4[] Externa
				\4[] Países con acceso a mercados
				\4[] Países de renta baja
				\4 Análisis de sensibilidad
				\4[] Estimación efecto de shocks
				\4[] $\to$ Umbrales de desviación
				\4[] Estimación probabilidad de éxito
				\4[] $\to$ Valorar shocks que no deben producirse para tener éxito
			\3 Variables determinantes del éxito de la intervención
				\4 Gradualismo del ajuste
				\4[] Programas de shock más creíbles
				\4 Volumen de financiación
				\4[] Cantidades mayores disuaden especuladores
				\4 Propiedad local / ``ownership''
				\4[] Convicción interna de necesidad de ajuste\footnote{Un ejemplo paradigmático de un programa con un elevado \textit{ownership} es el Plan de Estabilización de 1959 que España presentó al FMI y que de hecho se considera en ocasiones como una iniciativa puramente interna.}
				\4[] Utilización del FMI como chivo expiatorio
				\4 Contenido de la reforma estructural
				\4 Reestructuración de la deuda
				\4[] Deseable para crecimiento
				\4[] Muy dificil de negociar
				\4[] Riesgo moral
				\4 Composición de los ajustes
			\3 Autoevaluación de actuaciones
				\4 IEO -- Independent Evaluation Office
				\4[] Evaluación independiente y objetiva
				\4[] $\to$ Políticas
				\4[] $\to$ Programas de actuación
				\4[] Dependiente del Directorio Ejecutivo
			\3 Valoración
				\4 Efectos habituales
				\4[] Ajustes de balanza de pagos
				\4[] Se habrían producido también sin FMI
				\4[] $\to$ FMI precisamente para reducir intensidad de ajuste
				\4 Éxitos
				\4[] A menudo, difícil de valorar
				\4[] $\to$ Países ya en situación complicada
				\4[] $\to$ FMI como chivo expiatorio
				\4 Fracasos
				\4[] Programas de ajuste en crisis asiática
				\4[] Énfasis en defender TCN fijo
				\4[] $\to$ Subiendo tipos de interés
				\4[] $\then$ Contracción fuerte de la economía
				\4[] $\then$ TCN aún más difícil de defender
	\1[] \marcar{Conclusión}
		\2 Recapitulación
			\3 Funciones y antecedentes
			\3 Estructura del FMI
				\4 Organizativa
				\4 Financiera
			\3 Políticas del FMI
				\4 Asistencia financiera
				\4 Supervisión
				\4 Asistencia técnica
				\4 Evaluación de la actividad
			\3 Políticas de estabilización en países en desarrollo
		\2 Idea final
			\3 Impacto desde Guerra Mundial
				\4 Enorme contribución
				\4 Estabilidad
				\4 Conocimiento funcionamiento macroeconomía
				\4 Apertura economías
				\4 Chivo expiatorio de crisis
			\3 Constante evolución
				\4 Herramientas auto-mejora
				\4[] Independent Evaluation Office
				\4 Sujeto a críticas muchos frentes
				\4 Evolución unida a shocks y transformaciones
				\4 Creciente colaboración con Bancos de Desarrollo
			\3 Futuro del FMI
				\4 Capacidad de adaptación
				\4 Amenazas al multilateralismo
				\4 Evolución de bloques económicos
\end{esquemal}






























\preguntas

\seccion{Test 2019}

\textbf{34.} Señale la respuesta correcta en relación al Fondo Monetario Internacional:

\begin{itemize}
	\item[a] En la práctica se financia mediante las cuotas o préstamos que le conceden los Bancos Centrales de los Países miembros así como a través de la emisión de bonos en los mercados.
	\item[b] El Directorio Ejecutivo está compuesto por 24 representantes de los 189 países miembros, de los cuales 8 tienen silla propia que no necesitan compartir con otros países.
	\item[c] A 1 de octubre de 2016, el valor del DEG se basa en una cesta de 4 monedas principales: el dólar de EEUU., el euro, el yen japonés y la libra esterlina.
	\item[d] Con la décimo quinta revisión de las cuotas del Fondo, en septiembre de 2019, España ha aumentado su porcentaje de votos hasta el $3,5\%$.
\end{itemize}

\seccion{Test 2018}

\textbf{35.} En relación a la ejecución de la función supervisora del Fondo Monetario Internacional, señale la afirmación \underline{\textbf{CORRECTA}}:

\begin{itemize}
	\item[a] A nivel global, el FMI publica informes sobre la evolución de los mercados financieros y la economía internacional como el World Economic Outlook (WEO), el Global Financial Stability Report (GFSR) y el Fiscal Monitor.
	\item[b] A nivel regional, el FMI informes sobre la evolución económica reciente y las perspectivas de distintas regiones como el Regional Economic Outlook Report.
	\item[c] A nivel nacional, mediante las consultas del Artículo IV, el FMI envía misiones a los distintos países y elabora informes sobre su política económica. 
	\item[d] Todas las opciones anteriores son correctas.
\end{itemize}

\seccion{Test 2016}
\textbf{43}. En relación a los préstamos del FMI, señale cuál de las siguientes afirmaciones es falsa:

\begin{enumerate}
	\item[a] La Línea de Crédito Flexible (LCF) incluye estrictos criterios de habilitación predefinidos que se conocen también como condicionalidad ex ante pero no se basa en la condicionalidad tradicional de los programas.
	\item[b] El Instrumento de Financiamiento Rápido se brinda en formas de compras directas sin necesidad de un programa propiamente dicho ni de exámenes y puede emplearse de manera repetida. En 2010-2013 se utilizó para Grecia, Irlanda, Portugal y Chipre.
	\item[c] Normalmente, el acceso a la Línea de Precaución y Liquidez en el marco de un acuerdo de seis meses no puede superar el 150\% de la cuota del país en el momento de la aprobación.
	\item[d] La LCF funciona como una línea de crédito renovable que puede usarse inicialmente por uno o dos años y que opera sin límite de acceso a los recursos del FMI.
\end{enumerate}

\seccion{Test 2015}
\textbf{35}. Señale la respuesta verdadera con respecto a los Derechos Especiales de Giro (DEG) del FMI:

\begin{enumerate}
	\item[a] Actualmente, 5 divisas forman parte de la cesta de monedas.
	\item[b] La inflación (medida por el deflactor del PIB) media de los últimos 5 años es uno de los dos criterios utilizados para valorar la inclusión de una divisa en la cesta de monedas.
	\item[c] Varios organismos interancionales utilizan el DEG como unidad de cuenta.
	\item[d] El renminbi chino es la divisa con menor peso en la cesta de monedas.
\end{enumerate}

\seccion{Test 2013}
\textbf{43}. En relación con la Línea de Crédito Flexible - Flexible Credit Line (FCL) del Fondo Monetario Internacional, seleccione la respuesta \textbf{FALSA}:

\begin{enumerate}
	\item[a] No tiene un límite prefijado de acceso en términos de porcentaje de cuota.
	\item[b] Tiene condicionalidad ex-post igual que los Acuerdos Stand-By tradicionales del Fondo.
	\item[c] Puede ser utilizada como una línea precautoria.
	\item[d] Está pensado para países con buenos fundamentos macroeconómicos y buen historial en materia de política económica.
\end{enumerate}


\seccion{Test 2009}
\textbf{37}. La cuota que pagan los países miembros al FMI desempeña una serie de funciones entre las que destaca:

\begin{enumerate}
	\item[a] Establece el volumen mínimo de recursos financieros que un país está obligado a suministrar al Fondo Monetario Internacional.
	\item[b] Determina el poder de voto de un país en las decisiones que se adoptan en el seno de la organización.
	\item[c] Establece el tipo de cambio de la moneda nacional en el sistema monetario internacional.
	\item[d] Determina la cantidad mínima de financiación que los países miembros pueden obtener del Fondo Monetario Internacional.
\end{enumerate}


\seccion{Test 2007}
\textbf{37}. ¿Cuál es el mayor problema a que se enfrenta el FMI en sus tareas de supervisión, prevención y resolución de crisis tras los episodios de crisis en los años 90?

\begin{enumerate}
	\item[a] El cuestionamiento de su status de acreedor preferente por los países de renta media y la banca privada tras la crisis asiática a finales de los 90.
	\item[b] Conseguir captar fondos suficientes mediante préstamos de sus miembros y emisiones para tener una incidencia real en episodios de crisis de tamaños antes desconocidos. 
	\item[c] Adaptar sus esquemas de supervisión a un contexto global en el que el tamaño de los mercados financieros internacionales y la dificultad para predecir sus movimientos se han multiplicado por los problemas de información asimétrica y la innovación financiera.
	\item[d] La necesidad de proceder a un incremento general de cuotas para actualizar la potencia de sus recursos al tamaño de los mercados financieros dado el alto porcentaje de sus recursos comprometidos.
\end{enumerate}

\textbf{38}. La corrección de los desequilibrios globales depende en parte de la revaluación del cambio de algunas divisas, sobre todo las asiáticas. Impedir las devaluaciones competitivas de los años 30 está entre los objetivos fundacionales del FMI. Señale cuál de las siguientes afirmaciones es \textbf{CORRECTA}:

\begin{enumerate}
	\item[a] Los estatutos del FMI prohíben los controles de capital y las intervenciones masivas, sistemáticas y unidireccionales en los mercados por parte de los bancos centrales.
	\item[b] Con el abandono del patrón cambios-oro a mediados de los setenta y la proliferación de políticas de flotación más o menos libres, este objetivo perdió sentido y la Cuarta Enmienda a los Estatutos del FMI de facto se suprimió.
	\item[c] La dificultad de determinar hasta que punto una moneda se aleja de su nivel de equilibrio, por la poca fiabilidad de los modelos disponibles, convierte la tarea de supervisión del FMI del cumplimiento de esta obligación fundamental de sus miembros en una misión complicada.
	\item[d] El FMI encuadra la política cambiaria de cada miembro dentro de una clasificación de esquemas estándar. Dentro de cada uno de estos regímenes cambiarios el FMI obliga a los Bancos Centrales a dejar apreciar su moneda en caso de que se produzca un desalineamiento fundamental respecto del tipo de equilibrio.
\end{enumerate}


\seccion{Test 2006}
\textbf{35}. En relación con los Derechos Especiales de Giro ¿cuál de las siguientes afirmaciones es correcta?

\begin{enumerate}
	\item[a] Los DEG fueron creados por el FMI en 1945 con objeto de servir como unidad de cuenta de las cuotas de los países miembros en la institución.
	\item[b] Los DEG se definen en términos de una cesta de monedas, que actualmente se compone del euro, yen japonés, libra esterlina, dólar USA y renminbi chino.
	\item[c] El tipo de interés que el FMI carga a sus prestatarios por los préstamos que realiza viene dado por el tipo de interés del DEG más un reducido margen, que le FMI utiliza para financiar sus gastos internos. 
	\item[d] Hasta la fecha se han producido dos asignaciones generales de DEGs a los países miembros. Aún no se ha producido ninguna asignación extraordinaria de DEGs.
\end{enumerate}

\seccion{Test 2005}
\textbf{36}. Indica cuál de las siguientes afirmaciones es \textbf{FALSA}:

\begin{enumerate}
	\item[a] En terminología del Fondo Monetario Internacional (FMI) se entiende por condicionalidad el conjunto de medidas de política económica que deben cumplir los países miembros para acceder a los recursos del FMI.
	\item[b] En el ámbito de la Asistencia Técnica, el FMI ha puesto en marcha el Financial Sector Assessment Program (FSAP) o Programa de Evaluación del Sector Financiero (PESF), para analizar fortalezas y deficiencias de los sectores financieros de los países miembros.
	\item[c] El Convenio Constitutivo y los Estatutos del FMI impieden a esta institución otorgar facilidades financieras concesionales, esto es a tipo de interés significativamente inferiores a los de mercado.
	\item[d] Los acuerdos de derechos de giro (stand-by arrangements) permiten a los países miembros disponer de un volumen de crédito para resolver un problema de balanza de pagos a corto plazo.
\end{enumerate}

\seccion{Test 2004}
\textbf{35}. La Línea de Crédito Contingente (LCC) existente en 1999-2003 en el Fondo Monetario Internacional ha consistido en:

\begin{enumerate}
	\item[a] Una línea especial de sus servicio de financiamiento compensatorio.
	\item[b] Un mecanismo para hacer frente al contagio de las crisis financieras.
	\item[c] Una línea extraordinaria de su servicio para el crecimiento y la lucha contra la pobreza.
	\item[d] Una extensión, para hacer frente a catástrofes naturales, de su servicio de acuerdos de giro (\textit{stand-by arrangements}).
\end{enumerate}

\notas

\textbf{2019}: \textbf{34}. B

\textbf{2018}: \textbf{35}. D

\textbf{2016}: \textbf{43}. C

\textbf{2015}: \textbf{35}. C

\textbf{2013}: \textbf{43}. B

\textbf{2009}: \textbf{37}. B

\textbf{2007}: \textbf{37}. C \textbf{38}. B

\textbf{2006}: \textbf{35}. D

\textbf{2005}: \textbf{36}. C

\textbf{2004}: \textbf{35}. B

\bibliografia

Mirar en Palgrave:
\begin{itemize}
    \item arrears
    \item banking crises
    \item Bretton Woods system
    \item collective action
    \item currency crises
    \item dollarization
    \item emerging markets
    \item financial structure and economic development
    \item fiscal and monetary policies in developing countries
    \item foreign exchange markets, history of
    \item golden rule
    \item international capital flows
    \item international financial institutions (IFIs)
    \item International Monetary Fund
    \item international monetary institutions
    \item international reserves
    \item macroeconomic forecasting
    \item monetary approach to the balance of payments
    \item moneylenders in developing countries
    \item sovereign debt
    \item third world debt
\end{itemize}

Banco de España (2020) \textit{La efectividad de los programas del FMI en la última década} Documentos ocasionales nº 2007 -- En carpeta del tema

Bruegel Institute. \textit{The IMF role in the Euro Area Crisis}.  \url{ http://bruegel.org/2016/08/the-imfs-role-in-the-euro-area-crisis-financial-sector-aspects/}

Cecchetti, S. Schoenholt, K. (2018) \textit{Sudden stops: A primer on balance-of-payments crises} Voxeu.org \href{https://voxeu.org/content/sudden-stops-primer-balance-payments-crises}{Enlace}

Eichengreen, B.; Gupta, P. (2016) \textit{Managing Sudden Stops} World Bank Group. Policy Research Working Paper -- En carpeta del tema

De Gregorio, J.; Eichengreen, B.; Ito, T.; Wyplosz, C. \textit{IMF Reform: The Unfinished Agenda} (2018) International Center for Monetary and Banking Studies -- En carpeta del tema

Dierckx, S. (2011) \textit{The IMF and Capital Controls: Towards Postneoliberalism} 

Economic Parliament. (2018) \textit{The ESM and the IMF: comparison of the main features} Economic Governance Support Unit of the European Parliament. In-Depth Analysis -- En carpeta del tema \url{https://www.europarl.europa.eu/RegData/etudes/IDAN/2017/614485/IPOL_IDA(2017)614485_EN.pdf}

Gandolfo, G. \textit{International Finance and Open-Economy Macroeconomics}. (2016)

Garrido, I.; Moreno, P.; Serra, X.; \textit{LA REFORMA DE LAS CUOTAS Y LA REPRESENTACIÓN EN EL FMI} Boletín Económico del Banco de España, abril de 2016 -- En carpeta del tema

Goldberg, P. K.; Knetter, M. M. \textit{Goods Prices and Exchange Rates: What Have We Learned?} (1997) Journal of Economic Literature

IMF. \textit{IMF Support for low income countries} \url{http://www.imf.org/en/About/Factsheets/IMF-Support-for-Low-Income-Countries}

IMF. \textit{A guide to commmittees, groups and clubs} (2018) -- \url{https://www.imf.org/en/About/Factsheets/A-Guide-to-Committees-Groups-and-Clubs}

IMF (2000) \textit{Capital Controls: Country Experiences with Their Use and Liberalization} IMF Occasional Paper 190 \href{https://www.imf.org/external/pubs/ft/op/op190/index.htm}{Disponible aquí} -- En carpeta del tema

Peterson Institute for International Economics. (2018) \textit{IMF Quota and Governance Reform Once Again} Edwin M. Truman. Policy Brief -- En carpeta del tema

Reinhart, C.; Calvo, G. (2000) \textit{When Capital Inflows Come to a Sudden Stop: Consequences and Policy Option} Reforming the International Monetary and Financial System: IMF -- En carpeta del tema 
\end{document}
