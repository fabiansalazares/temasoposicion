\documentclass{nuevotema}

\tema{4A-9}
\titulo{Análisis de los sectores de bienes intermedios: siderurgia, química y metales no férreos. La industria del automóvil.}

\begin{document}

\ideaclave

\input{/home/alibey/Oposicion/Resumenes_4o/Importancia_cuantitativa_Sectores.tex}

\seccion{Preguntas clave}

\begin{itemize}
	\item 
\end{itemize}

\esquemacorto

\begin{esquema}[enumerate]
	\1[] \marcar{Introducción}
		\2 Contextualización
			\3 Sectores de la economía española
			\3 Sector en cuestión
			\3 Ejemplos relevante
		\2 Objeto
			\3 Análisis estático
			\3 Análisis dinámico
			\3 Políticas públicas
		\2 Estructura
			\3 Análisis estático
			\3 Análisis dinámico
			\3 Política económica
	\1 \marcar{Bienes intermedios}
		\2 Idea clave
			\3 Concepto
			\3 Objetivos
			\3 Resultados
		\2 Siderurgia y ferroaleaciones
			\3 Análisis estático
			\3 Análisis dinámico
			\3 Políticas públicas
			\3 Estatuto de Consumidores Electrointensivos
			\3 Foro Mundial Sobre Exceso de Capacidad del Acero
		\2 Metales no férreos
			\3 Análisis estático
			\3 Aluminio
			\3 Zinc
			\3 Cobre
			\3 Estaño
			\3 Mercurio
			\3 Plomo
			\3 Wolframio
			\3 Tierras raras
			\3 Políticas públicas
		\2 Materiales de construcción
			\3 Delimitación
			\3 Empresas
			\3 Trabajo
			\3 Cerámico
			\3 Cemento
			\3 Vidrio
			\3 Hormigón y mortero preparado
			\3 Piedra natural
			\3 Implicaciones
	\1 \marcar{Industria química}
		\2 Análisis estático
			\3 Delimitación del sector
			\3 Importancia
			\3 Oferta
			\3 Demanda externa
			\3 Empresas
			\3 Química básica
			\3 Química intermedia
			\3 Especialidades químicas
		\2 Análisis dinámico
			\3 Evolución
			\3 Actualidad
			\3 Perspectivas
		\2 Políticas públicas
			\3 Acuerdo productos químicos en seno de WTO
			\3 Fuerte regulación en general
			\3 Tendencia a concentración geográfica
			\3 REACH
			\3 Políticas de I+D
			\3 Nanotecnología
			\3 Incorporación de DMarco de Residuos y IPPC
			\3 Ley de Residuos y Suelo contaminados
			\3 PEMAR -- Plan Estatal Marco de Gestión de Residuos 2016--2022
			\3 Programa Estatal de Prevención de Residuos 2014-2020
			\3 Estrategia Española de Economía Circular 2030
			\3 Impuesto sobre Plásticos de Un Solo Uso
	\1 \marcar{Industria del automóvil}
		\2 Análisis estático
			\3 Delimitación del sector
			\3 Importancia
			\3 Modelos teóricos relevantes
			\3 Oferta
			\3 Demanda interna
			\3 Sector exterior
		\2 Componentes de automóviles
			\3 Idea clave
			\3 Empresas
			\3 Empleo
			\3 Exportaciones
		\2 Distribución de vehículos: concesionarios
			\3 Idea clave
			\3 Evolución
			\3 Perspectivas
		\2 Análisis dinámico
			\3 Evolución
			\3 Actualidad
			\3 Elevada incertidumbre
			\3 Crisis Covid
			\3 Carsharing
			\3 Transporte colaborativo
			\3 Perspectivas
		\2 Políticas públicas
			\3 Justificación
			\3 Objetivos
			\3 Antecedentes
			\3 Marco jurídico
			\3 Marco financiero
			\3 Actuaciones
			\3 IPPC
			\3 Planes PIMA
			\3 Plan Renove
			\3 Plan Moves II: ayudas compra coches eléctricos
			\3 WLTP -- World Harmonized Light-duty Vehicle Test
			\3 Plataforma Tecnológica del Hidrógeno y Pilas de Combustible
			\3 Valoración
			\3 Retos
	\1[] \marcar{Conclusión}
		\2 Recapitulación
			\3 Bienes intermedios
			\3 Industria del automóvil
		\2 Idea final

\end{esquema}

\esquemalargo

\begin{esquemal}
	\1[] \marcar{Introducción}
		\2 Contextualización
			\3 Sectores de la economía española\footnote{Presentación Kingdom of Spain del Tesoro Público, diciembre de 2019 (fuente: INE).}
				\4 Porcentaje sobre VAB
				\4 Servicios: 74,7\%
				\4 Industria: 15,4\%
				\4 Construcción: 6,5\%
				\4 Sector primario: 3\%
			\3 Sector en cuestión
			\3 Ejemplos relevante
		\2 Objeto
			\3 Análisis estático
			\3 Análisis dinámico
			\3 Políticas públicas
		\2 Estructura
			\3 Análisis estático
			\3 Análisis dinámico
			\3 Política económica
	\1 \marcar{Bienes intermedios}
		\2 Idea clave
			\3 Concepto
				\4 Transformación de materias primas
				\4[$\to$] Bienes semielaborados
				\4 Diferencias con bienes de consumo
				\4[] Bienes intermedios se utilizan para producir
				\4[] $\to$ Otros bienes y servicios
				\4 Diferencias con bienes de equipo
				\4[] Bienes intermedios se agotan
				\4[] $\to$ En momento de incorporación a proceso productivo
			\3 Objetivos
				\4 Caracterizar principales sectores
				\4 Distinguir aspectos esenciales
				\4 Políticas públicas relativas a cada sector
			\3 Resultados
				\4 Enorme importancia sector industrial
				\4 Permite reducir costes sectores downstream
				\4 Eslabonamientos hacia abajo
		\2 Siderurgia y ferroaleaciones\footnote{Ver Sahuquillo y \href{https://www.mincotur.gob.es/Publicaciones/Publicacionesperiodicas/EconomiaIndustrial/RevistaEconomiaIndustrial/406/LARREA\%20Y\%20GARCIA.pdf}{Larrea Basterra, M.; García Berezo, A. La Siderurgia en España y su futuro.}}
			\3 Análisis estático
				\4 Concepto
				\4[] Transformación de metales férreos brutos
				\4[] $\to$ En manufacturas elaboradas
				\4 Delimitación del sector
				\4[] CNAE División 24: acero, hierro, ferroaleaciones
				\4[] CNAE Division 25: productos metálicos no maquinaria o equipo
				\4[] Fabricación de productos metálicos
				\4[] $\to$ Excepto maquinaria y equipo
				\4[] Otros sectores expuestos a metal
				\4[] $\to$ Fabricación productos electrónicos, ópticos...
				\4[] $\to$ Material y equipo electrónico
				\4[] $\to$ Maquinaria y equipo ncop.
				\4[] $\to$ Vehículos...
				\4[] $\to$ Otro material de transporte
				\4[] $\to$ Reparación de maquinaria y equipo
				\4 Subsectores
				\4[] Metalurgia básica
				\4[] $\to$ Obtención de hierros y aceros
				\4[] $\to$ Primeros laminados
				\4[] Transformación de metales
				\4[] $\to$ Laminación en frío
				\4[] $\to$ Forja y estampación
				\4[] $\to$ Trefilerías
				\4[] $\to$ Laminados de precisión
				\4 Importancia
				\4[] Casi 2\% de PIB CNAE 24 y 25
				\4[] $\to$ 19.000 M de €
				\4[] 4,6\% de PIB industrial
				\4[] 310.000 trabajadores directos
				\4[] $\to$ 60.000 en metalurgia, acero, hierro, ferroaleaciones
				\4[] $\to$ 250.000 en fabricación productos metálicos no equipo
				\4[] +20.000 en recogida de chatarra
				\4[] Productor nº 16 mundial en 2016
				\4[] $\to$ Por debajo de GER, FRA, ITA
				\4[] Más importancia que otros países UE
				\4[] Mucho más productivo que en media UE
				\4[] Efectos arrastre hacia delante y atrás
				\4[] $\to$ Automóvil
				\4[] $\to$ Electrodomésticos
				\4[] $\to$ Minería del hierro
				\4[] $\to$ Minería del carbón
				\4[] $\to$ Construcción
				\4[] $\to$ Industria naval
				\4[] $\to$ Industria de defensa
				\4[] $\to$ Aeronáutica
				\4[] $\to$ Equipos mecánicos y eléctricos
				\4 Ramas
				\4[] Siderurgia
				\4[] Ferroaleaciones
				\4 Tipos de productos
				\4[] Largos
				\4[] $\to$ Principal componente
				\4[] $\to$ Más de 50\% de total
				\4[] Planos
				\4[] Semis y transformados
				\4 Modelos teóricos relevantes
				\4[] Economías de escala
				\4[] Oligopolio
				\4[] Competencia monopolística
				\4[] Análisis de costes
				\4 Oferta
				\4[] Empresas de gran tamaño
				\4[] Cerca de 20 instalaciones productoras de acero
				\4[] 50 instalaciones de transformación y laminación
				\4[] Costes fijos muy elevados
				\4[] Enormes economías de escala
				\4[] Necesario producir en continuo
				\4[] Tendencia hacia concentración creciente
				\4[] Pequeño \% sobre producción mundial
				\4[] Concentración geográfica en País Vasco y Asturias
				\4[] $\to$ Corporación Siderúrgica Integral
				\4[] $\then$ Posteriores Aceralia, Arcelor, ArcelorMittal
				\4[] Sobrecapacidad
				\4[] $\to$ Tras desmantelamiento Sagunto en 80s
				\4[] Celsa en Cataluña
				\4[] $\to$ Productos largos
				\4[] $\to$ Primer grupo siderúrgico privado español
				\4 Competencia
				\4[] Elevada fragmentación a nivel mundial
				\4[] Costes de transporte no excesivos
				\4[] $\to$ Competencia fuerte a nivel mundial
				\4 Demanda interna
				\4[] Sobrecapacidad respecto demanda interna
				\4[] Sensible a ciclo
				\4 Demanda externa
				\4[] China principal productor mundial
				\4[] $\to$ Enorme sobrecapacidad
				\4[] India
				\4[] Este de Europa
				\4[] Alemania e Italia
				\4[] Sector relativamente fácil de proteger
				\4[] $\to$ Bienes pesados y fácilmente detectables
				\4[] Protección en bloque NAFTA tras USMCA
				\4[] $\to$ Aumentado con aranceles Trump
				\4[] Fuertes caídas recientes en destinos españoles
				\4[] $\to$ Portugal
				\4[] $\to$ Italia
				\4[] Saldos positivos en última década
				\4[] $\to$ Caída en 2018
				\4[] $\to$ Exceso de capacidad en 2019
				\4 Importaciones
				\4[] Francia, Portugal, Alemania, Italia, Turquía
				\4[] $\to$ Principales proveedores
				\4[] Emergentes asiáticos también relevantes
				\4[] Sobre todo en productos planos
			\3 Análisis dinámico
				\4 Evolución
				\4[] Larga tradición histórica
				\4[] $\to$ Yacimientos de carbón y de hierro
				\4[] $\to$ Industria de armamento
				\4[] Industrialización inicial
				\4[] $\to$ Comienza por sector siderúrgico
				\4[] $\then$ Ferrocarril motor de demanda inicial
				\4[] $\then$ País Vasco
				\4[] $\then$ Substrato armería Irún, Eibar...
				\4[] $\then$ Relaciones comerciales con Inglaterra
				\4[] $\to$ Relativamente débil respecto Europa del Norte
				\4[] Periodo autárquico
				\4[] $\to$ Énfasis en industria pesada
				\4[] $\to$ Creación de ENSIDESA\footnote{Empresa Nacional de Siderurgia S.A.} en 1957
				\4[] $\to$ Cuellos de botella: demanda < oferta
				\4[] Plan de Estabilización 1959
				\4[] $\to$ Liberalización importaciones siderúrgicas
				\4[] $\to$ No estrangular industrialización
				\4[] $\then$ Aumento de importaciones
				\4[] $\then$ Evidencia problemas de competitividad
				\4[] $\then$ Demanda elevada mantiene a flote y creciente
				\4[] $\then$ Reconversión se aplaza
				\4[] Crisis años 70
				\4[] $\to$ Inflación y crisis en Occidente
				\4[] $\to$ Caída exportaciones
				\4[] $\to$ Pérdida de competitividad
				\4[] $\to$ Aumento de costes energéticos
				\4[] $\to$ Aumento de desempleo
				\4[] Años 80
				\4[] $\to$ Comienzo reconversión
				\4[] $\to$ Disminución de capacidad instalada
				\4[] $\to$ Controversia Informe Kawasaki
				\4[] $\then$ Afirma necesidad gran planta en Sagunto
				\4[] $\then$ Presión sindical: Asturias y PV ganan partida
				\4[] $\then$ Acceso a fondos europeos para reconversión
				\4[] Años 90
				\4[] $\to$ Segunda reconversión industrial
				\4[] $\to$ Fusión ENSIDESA y Altos Hornos Vizcaya
				\4[] $\then$ Corporación Siderúrgica Integral
				\4[] $\then$ Aceralia en 1997
				\4[] $\then$ Arcelor en 2002
				\4[] $\then$ ArcelorMittal en 2006
				\4[] Crisis financiera
				\4[] $\to$ Caída inferior a competidores como Alemania
				\4[] $\to$ Sin embargo, apenas se recupera posteriormente
				\4[] $\then$ Pérdida de peso mundial
				\4 Actualidad
				\4[] Pérdida de peso reciente
				\4[] Importante peso en inversión anual
				\4[] $\to$ Maquinaria
				\4[] $\to$ Mejora de instalaciones
				\4[] $\to$ Mantener competitividad
				\4[] Elevados costes energéticos
				\4 Perspectivas
				\4[] Sector en fuerte crecimiento desde 2018
				\4[] $\to$ Países emergentes tiran de la demanda
				\4[] $\to$ Especialmente China, India
				\4[] Relativamente maduro en desarrollados
				\4[] Creciente competencia internacional
				\4[] $\to$ China acusada de dumping
				\4[] $\to$ Restricciones tras victoria Trump
				\4[] $\to$ Tecnología relativamente estable
				\4[] $\to$ Costes laborales a la baja
				\4[] $\then$ Sobrecapacidad mundial
				\4[] Protección medioambiental
				\4[] $\to$ Sujeta a ETS europeo
				\4[] $\to$ Industria fácil de controlar respecto emisiones
				\4[] Sustitutivos
				\4[] $\to$ Plásticos
				\4[] $\to$ Fibras de carbono
				\4[] $\to$ Aluminio
				\4[] $\to$ ...
				\4[] Exceso de capacidad internacional
				\4[] Débil demanda en destinos españoles
				\4[] $\to$ Portugal
				\4[] $\to$ Italia
				\4[] $\then$ Fuertes caídas de demanda
				\4 Amenazas
				\4[] Sobrecapacidad en China
				\4[] Sustitución por otras materiales
				\4 Fortalezas
				\4[] Cercanía con sectores downstream
				\4[] $\to$ Especialmente coches
				\4[] $\to$ Bienes de equipo País Vasco
				\4[] $\to$ Bienes de equipo en Europa
				\4[] Elevado nivel tecnológico
				\4[] $\to$ En relación a conjunto de economía
				\4[] $\to$ Relativamente bajo respecto otros industriales
			\3 Políticas públicas
				\4 Periodo autárquico
				\4 Planes de desarrollo
				\4 Reconversión industrial años 80
				\4 Regulación sobre ayudas públicas en la UE
				\4 ETS
				\4 Ajuste de carbono en fronteras UE
				\4[] No implementado
				\4[] Posible implementacón
				\4[] Extranjeros no sufren ETS
			\3 Estatuto de Consumidores Electrointensivos
				\4 Actualmente
				\4[] Precios eléctricos muy elevados
				\4[] Electricidad es input esencial de metalurgia
				\4 Aumentar certidumbre de precios eléctricos
				\4 Actividades coste electrico puede ser >40\%
				\4 En proyecto
			\3 Foro Mundial Sobre Exceso de Capacidad del Acero
				\4 Foro mundial con principales productores
				\4 Examinar y debatir cuestión de exceso de capacidad
				\4 Especialmente tras medidas comerciales Trump
				\4 China y Brasil abandonan foro
				\4[$\to$] Vaciado de contenido
		\2 Metales no férreos
			\3 Análisis estático
				\4 Delimitación del sector
				\4[] Extraído el mineral
				\4[] $\to$ Llevar a cabo primeras transformaciones
				\4[] CNAE Grupo 244: metales preciosos y no férreos
				\4[] $\to$ Metales preciosos
				\4[] $\to$ Metales no férreos
				\4[] $\to$ Producción
				\4[] $\to$ Fundición
				\4 Importancia
				\4[] Esencial para sector industrial
				\4[] $\to$ Bienes de equipo
				\4[] $\to$ Bienes de consumo
				\4[] $\to$ Automóvil
				\4[] Cualitativo
				\4[] $\to$ Factor de dependencia exterior
				\4[] $\to$ Consideraciones estratégicas y militares
				\4[] $\to$ Impacto medioambiental elevado
				\4[] $\to$ Emisiones no difusas
				\4 Modelos teóricos relevantes
				\4[] Costes fijos
				\4[] Monopolios naturales por inversiones elevadas
				\4 Oferta
				\4[] Trabajo
				\4[] Capital
				\4 Demanda interna
				\4 Demanda externa
			\3 Aluminio
				\4 Importancia
				\4[] Muy numerosas aplicaciones industriales
				\4[] $\to$ No se corroe
				\4[] $\to$ Buen conductor eléctrico y térmico
				\4[] $\to$ Muy ligero
				\4[] $\to$ Baja densidad
				\4[] Elevado consumo eléctrico en producción
				\4 Distribución geográfica
				\4[] Planta de Lérida
				\4[] $\to$ Explotación de bauxita
				\4[] Plantas de Lugo y Avilés
				\4[] $\to$ Fabricación de productos de aluminio
				\4[] Alibérico
				\4[] $\to$ Producción de envases de aluminio
				\4 Políticas públicas
				\4[] INI/SEPI
				\4[] $\to$ Propietaria de INESPAL
				\4[] $\to$ 9 plantas alumineras
				\4[] $\to$ Provisión de electricidad
				\4[] $\then$ Venta en 1998 a ALCOA
				\4[] $\then$ Fin de contratos preferentes electricidad
				\4[] Estatuto de Consumidor electrointensivo
				\4[] $\to$ Por desarrollar
			\3 Zinc
				\4 Importancia
				\4[] Fabricación de:
				\4[] $\to$ Latón
				\4[] Galvanización:
				\4[] $\to$ Evitar oxidación del hierro
				\4[] Envases
				\4[] Industria farmacéutica
				\4 Distribución geográfica
				\4[] Mina en Cantabria cerrada en 2003
				\4[] $\to$ Mayor explotación de Europa
				\4[] AZSA\footnote{Asturiana de Zinc.} en Asturias
				\4[] $\to$ Uno de los mayores productores europeos
				\4[] Explotaciones en Murcia y Jaén agotados
				\4[] Importancia cada vez menor
				\4[] $\to$ Parte de Glencore tras venta por Xstrata
				\4[] $\to$ Producción de lingotes de zinc
				\4[] $\to$ Otros subproductos
				\4[] $\to$ Récords de producción pre-Covid
			\3 Cobre
				\4 Importancia
				\4[] Excelente conductor eléctrico
				\4[] $\to$ Solo después de plata
				\4[] Relación coste-conductividad
				\4[] $\to$ Componente esencial electricidiad
				\4[] Eslabonamientos con todos sectores industriales
				\4[] $\to$ Electricidad
				\4[] $\to$ Otros usos
				\4 Distribución geográfica
				\4[] Minas históricas de Río Tinto
				\4[] Recuperación de chatarras
				\4[] $\to$ Distribuida respecto a áreas industriales
				\4[] KME multinacional italiana-alemana
				\4[] $\to$ Controla gran parte de Producción
				\4 Reciclaje ilimitado
				\4[] Reduce dependencia exterior
			\3 Estaño
				\4 Importancia
				\4[] Fabricación de bronce
				\4[] Aleación con cobre
				\4[] Importancia muy reducida
				\4[] Sustituido por aluminio
			\3 Mercurio
				\4 Aplicaciones muy diversas en industria
				\4 Antigua mina de Almadén, ciudad Real
				\4[] Clausurada en 2002
				\4 Alta toxicidad reduce uso y extracción
			\3 Plomo
				\4 Importancia
				\4[] Muy dúctil y maleable
				\4[] Pigmentos en pinturas
				\4[] Baterías
				\4[] Antiguamente, gasolina
				\4 Distribución geográfica
				\4[] Antiguamente, explotaciones muy fragmentadas
				\4[] $\to$ Productividad muy baja
				\4[] Explotaciones en Murcia y Jaén agotados
				\4[] Importancia cada vez menor
				\4 Medioambiente
				\4[] Tendencia
			\3 Wolframio
				\4 Punto de fusión muy elevado
				\4 Gran dureza
				\4 Carácter estratégico militar
			\3 Tierras raras
				\4 Conjunto de 15 elementos químicos
				\4 Relativamente escasos
				\4 Enorme importancia industrial
				\4[] Microchips
				\4[] Magnetismo
				\4 China, Australia, EEUU mayores productores
				\4 Potencial geológico de España
				\4 Yacimientos asociados a metales presentes en España
				\4 Medioambiente
				\4[] Elevado consumo de agua
				\4[] $\to$ Problema en zonas tensionadas
				\4[] Vertidos de materiales
			\3 Políticas públicas
				\4 Política industrial general
				\4 Regulación medioambiental
				\4[] Poco favorable a extracción de metales
				\4[] $\to$ Aumenta dependencia exterior
				\4 Consideración estratégica de tierras raras
		\2 Materiales de construcción
			\3 Delimitación
				\4 CNAE División 26 Otros productos minerales no metálicos
				\4 Importancia
				\4[] Input esencial en construcción
				\4[] Otros procesos industriales
				\4[] Intensivo en capital
				\4[] 0.5\% de VAB y empleo en 2015
				\4 Modelos teóricos relevantes
				\4[] Oligopólicos
				\4[] Elevados costes fijos
				\4[] Extracción de recursos medioambientales
				\4 Oferta
				\4[] Sobrecapacidad muy elevada
				\4 Demanda externa
				\4[] Superávit comercial estructural
				\4[] $\to$ Especialmente en baldosas y cerámicas
				\4 Amplia variedad dentro de subsectores
				\4 Varios subsectores principales
				\4[] Cerámico
				\4[] Cemento
				\4[] Hormigón
				\4[] Mortero
				\4[] Vidrio
				\4[] Metales
				\4[] ...
			\3 Empresas
				\4 Elevado número de empresas
				\4 Destrucción de empresas en crisis
				\4 Inflexión en 2014
				\4[] Vuelta a crecimiento
			\3 Trabajo
				\4 Cualificación relativamente baja
				\4 Paro por debajo de media nacional
				\4 100.000 empleados
			\3 Cerámico
				\4 Demanda mundial creciente
				\4 Azulejos, baldosas, ladrillos, tejas...
				\4 España muy competitiva
				\4[] 3er productor mundial
				\4 Comunidad Valenciana, Castellón
				\4[] Aglomeración de industrias relativas
				\4 Contenciosos comerciales con Marruecos
				\4[] Alegaciones de dumping
				\4 Principales exportadores mundiales
				\4[] China
				\4[] Italia
				\4[] España
				\4[] $\to$ Pamesa
				\4[] $\to$ STN Group
				\4 Principales consumidores
				\4[] China
				\4[] India
				\4[] Brasil
				\4[] Vietnam
				\4[] ...
				\4 Costes energéticos
				\4[] My relevantes
				\4[] Pérdida de competitividad España
				\4 Exportaciones
				\4[] Tercer mayor exportador tras China e Italia
				\4[] Menor valor añadido que Italia
				\4[] Sobre todo a Europa
				\4[] Crecimiento a Argelia
				\4 Importaciones
				\4[] Muy escasa importación
				\4[] Algunos productos muy alto valor añadido
				\4 Saldo
				\4[] Superavitario desde 2009
				\4[] 6.000 M de € superávit
			\3 Cemento
				\4 Material de construcción a partir de:
				\4[] Caliza
				\4[] Arcillas
				\4[] Transformación química mediante calor
				\4 Rama más importante de transformación
				\4[] En trasformación de minerales no metálicos
				\4 18 millones de toneladas en 2018
				\4 Plantas productivas repartidas por territorio
				\4[] 33 plantas productivas
				\4 Exceso de capacidad tras crisis
				\4[] Apenas utilización entre 50\% y 60\%
				\4 Unas 30 plantas nacionales relevantes
				\4 Muy intensivo en energía
				\4[] Sufre elevado coste energético en España
				\4 Muy afectado por crisis financiera
				\4[] Caída edificación, obra pública, vivienda
				\4 Demanda nacional
				\4[] Apenas 35\% de capacidad productiva
				\4[$\then$] Elevada infrautilización
				\4[$\then$] Concentración del negocio
				\4 Demanda por subsectores
				\4[] Vivienda y edificación no residencial
				\4[] $\to$ Superan ligeramente el 50\%
				\4[] Obra civil
				\4[] $\to$ Supera el 45\%
				\4 Energético intensivo
				\4 Elevadas emisiones de CO2
				\4 Fortísima caída tras crisis
				\4 Volumen apenas 1/4 de 2007
				\4 No se recupera desde 2013
				\4 \underline{Sector exterior}
				\4 Poco orientado al exterior
				\4[] Por relación peso--precio
				\4[] Superávit pero muy poca cuantía
				\4[] Francia, UK, Portugal principales destinos
				\4[] Turquía, Argelia competidores
				\4 Empresas mundiales
				\4[] Holcim-Lafargue
				\4[] Cemex
				\4[] CRH
				\4[] Heidelbergcement
				\4 Cemex
				\4[] Principal cementera en España
				\4[] Cierres en 2019
				\4[] Altos costes energéticos
				\4 Fuerte superávit exterior
				\4[] A pesar de CdTransporte elevados por kg
				\4[] Hacen poco rentable actividad exportadora
				\4[] $\to$ Pero España muy internacionalizada
				\4 Muy poca importación
				\4 Exceso de capacidad
				\4 Sujeto a derechos de emisión de CO2
				\4[] Emisiones fáciles de localizar
				\4[] Progresivo aumento de precios derechos
				\4 Demanda interna relativamente débil tras crisis
				\4 Fuerte interacción con sector de la construcción
			\3 Vidrio
				\4 Fusión de silicio a alta temperatura
				\4 Productos muy estandarizados
				\4 Márgenes muy reducidos
				\4 Beneficio por volumen
				\4 Elevada concentración del sector
				\4 Poco comercio internacional
				\4[] Costes de transporte elevados por kg
			\3 Hormigón y mortero preparado
				\4 Cemento+agua+áridos+otros
				\4 Subsectores
				\4[] Hormigón
				\4[] Mortero preparado
				\4[] Prefabricados de hormigón
				\4 Casi 2000 plantas productoras
				\4 Caída similar a cemento tras crisis
				\4 Recuperación muy lenta
				\4 Distribución geográfica
				\4[] Andalucía
				\4[] Castilla y León
				\4[] Galicia
				\4 Saldo exterior positivo
				\4[] Especialmente preparados de hormigón
			\3 Piedra natural
				\4 Granito, pizarra, mármol
				\4 Comercio internacional
				\4[] Sólo en segmento de calidad alta
				\4[] $\to$ Elevados CdTransporte
				\4[] $\to$ Muy elevado peso
			\3 Implicaciones
				\4 Tendencia alcista en 2018
				\4[] En casi todos los sectores
				\4 Crisis covid fuerte impacto global y España
	\1 \marcar{Industria química}
		\2 Análisis estático
			\3 Delimitación del sector
				\4[] Transformación de productos por reacciones químicas
				\4[] CNAE División 20: Industria química
				\4[] Principales Grupos
				\4[] $\to$ Fabricación productos químicos básicos
				\4[] $\to$ Refino de petróleo
				\4[] $\to$ Pesticidas y agroquímicos
				\4[] $\to$ Pinturas, barnices, revestimientos, tintas
				\4[] $\to$ Otros: perfumes, explosivos, colas, fibras artificiales...
			\3 Importancia\footnote{Ver \href{https://www.investinspain.org/invest/es/sectores/Industria-Quimica/Descripcion/index.html}{Ver Invest in Spain}.}
				\4[] 13\% de VAB industrial
				\4[] Casi 700.000 empleos inducidos totales
				\4[] Inputs esenciales en todo el sector industrial
				\4[] $\to$ Industria pesada
				\4[] $\to$ Bienes de consumo
				\4[] $\to$ Bienes de equipo
				\4[] Impacto medioambiental elevado
				\4[] Elevada capacidad de arrastre de capital humano
				\4[] Más del 50\% de producción dedicado a exportación
			\3 Oferta
				\4[] 9.000 VAB en 2018 (INE, Ramas)
				\4[] 1,1\% del empleo (2015, Sahuquillo)
				\4[] Empleo según INE CNAE:
				\4[] $\to$ Química+refino: 100.000 (CNAE 20)
				\4[] $\to$ Caucho y plásticos: 100.000 (CNAE 22)
				\4[] Elevado peso en inversión industrial I+D
				\4[] $\to$ Casi el 25\% de la inversión total en I+D+I
				\4[] Emplea +20\% de personal investigador en industria
				\4[] Capital extranjero muy relevante
				\4[] Concentración geográfica
				\4[] $\to$ Cataluña (Tarragona)
				\4[] $\to$ Madrid
				\4[] $\to$ Huelva
				\4[] $\to$ Algeciras
				\4[] $\to$ País Vasco
				\4[] $\to$ Zonas costeras
				\4[] Empleo
				\4[] $\to$ Casi 200.000 empleos directos
				\4[] $\to$ Muy baja temporalidad
				\4[] $\to$ Salario bastante por encima de la media
				\4[] Inversión
				\4[] $\to$ Elevado peso en inversión anual
				\4[] $\to$ Elevado peso inversión sobre \% VA
				\4 Demanda interna
			\3 Demanda externa
				\4 >50\% dedicado a exportación
				\4 Importante en exportaciones industriales
				\4[] $\to$ Tercer sector tras bienes de equipo y automóvil
				\4[] Refino de petróleo superavitario
				\4 Exportaciones
				\4[] $\to$ 27.000 M de € en 2019
				\4[] $\to$ Tercer mayor sector de exportaciones
				\4[] $\then$ Tras automóvil y ABT
				\4 Importaciones
				\4[] $\to$ 30.000 M de € en 2019
				\4[] $\to$ Tercer mayor sector de importaciones
				\4[] $\then$ Tras automóvil y maquinaria
				\4 Saldo
				\4[] $\to$ Ligero déficit persistente
				\4 EXPOQUIMIA en Barcelona
				\4[] $\to$ Feria importante del sector químico
				\4[] $\to$ Trianual
			\3 Empresas
				\4 FEIQUE
				\4[] Federación Empresarial de la Industria Química en España
				\4 Fertiberia
				\4 Solvay
				\4[] Cantabria
				\4[] Andalucía
				\4 Ercros (española)
				\4 Armando Álvarez (embalajes plásticos)
			\3 Química básica
				\4 Fertilizantes
				\4 Fertiberia
				\4 Demanda muy estacional
				\4 Fuerte interacción con sector agrícola
			\3 Química intermedia
				\4 Caucho
				\4[] Elevada dependencia de automoción
				\4 Plástico
				\4[] España entre 10 mayores productores
				\4[] Ligado a industria alimentaria
				\4[] Énfasis regulatorio
				\4[] $\to$ Reconversión en otros materiales
				\4[] $\to$ Alternativas plantean también problemas
				\4[] Sujeto de nuevos impuestos
				\4[] $\to$ Impuesto sobre Plásticos de un Sólo Uso (proyecto)
				\4[] $\to$ Recurso de financiación europeo
				\4 Pasta de papel
				\4[] Dos fases:
				\4[] 1. Fabricación de celulosa o pasta
				\4[] 2. Transformación en papel o cartón
				\4[] Producto fuertemente estandarizado
				\4[] $\to$ Competencia en precios
				\4[] Parte del sector en reciclado
				\4[] Muy contaminante
				\4[] Competencia con sector de reciclado
				\4[] $\to$ Aunque también demanda productos químicos
			\3 Especialidades químicas
				\4 Especialidades farmacéuticas (21\%)
				\4[] 21\% cifra de negocios
				\4 Materias primas, caucho, plástico
				\4[] 19\%
				\4 Química orgánica
				\4[] 15\%
				\4 Otros sectores
				\4[] Perfumería y cosmética
				\4[] Pinturas y tintas
				\4[] Materias primas farmacéuticas
				\4[] Detergentes
				\4[] Gases industriales
				\4[] Química inorgánica
				\4[] Fertilizantes
				\4[] Agroquímica
				\4[] Colorantes
				\4[] Fibras
				\4[] ....
		\2 Análisis dinámico
			\3 Evolución
				\4[] Planes de Desarrollo
				\4[] Crecimiento mayor que otros europeos
			\3 Actualidad
				\4[] Cluster químico de Tarragona
				\4[] $\to$ Mayor clúster químico Mediterráneo y Sur
				\4[] Andalucía y Huelva
				\4[] $\to$ Promocionados par anueva inversión
				\4[] Murcia y Valencia
				\4[] $\to$ Importantes focos a nivel mediterráneo
			\3 Perspectivas
		\2 Políticas públicas
			\3 Acuerdo productos químicos en seno de WTO
				\4 España es parte vía UE
				\4 Reducción general
				\4 Homogeneización de tipos
				\4[] $\to$ Reducir picos elevados
			\3 Fuerte regulación en general
				\4 Evitar impacto medioambiental
				\4 Evitar efectos salud humana
			\3 Tendencia a concentración geográfica
				\4 Mitigar impacto medioambiental
				\4 Rechazo de industrias en muchos lugares
				\4 Economías de escala elevadas
			\3 REACH
				\4 Entada en vigor en 2007
				\4[] Regulation on:
				\4[] $\to$ Registration
				\4[] $\to$ Evaluation
				\4[] $\to$ Authorisation
				\4[] $\to$ Restriction
				\4[] $\to$ Chemicals
				\4 Agencia Europea de Productos Químicos
				\4 Obligación de identificar y gestionar riesgos
				\4[] $\to$ Productos químicos vendidos y producidos
				\4 Sustancias químicas sin registro
				\4[] $\to$ No pueden importarse o producirse
				\4 Evaluación de dosieres
				\4[] $\to$ Por parte de la AEPQ
			\3 Políticas de I+D
				\4 Importante peso del sector privado
				\4 Poco peso relativo políticas públicas
			\3 Nanotecnología
				\4 Subvenciones empresas vía CDTI
				\4 CSIC
				\4 Otros organismos autonómicos
			\3 Incorporación de DMarco de Residuos y IPPC
				\4 CCAA
				\4[] Planes autonómicos prevención y vigilancia
				\4 EELL
				\4[] Recogida, transporte y tratamiento
				\4[] Potestad de vigilancia e inspección
				\4 Objetivo global
				\4[] Reducir residuos generados para 2020
				\4[] $\to$ Un 10\% respecto de 2010
				\4 Anteproyecto de nueva ley en desarrollo
				\4[] Impuesto sobre plásticos de un sólo uso
				\4[] $\to$ 0,45 € por kg
			\3 Ley de Residuos y Suelo contaminados
				\4 Ley 22/2011 de residuos y suelos contaminados
				\4[] Regular disposición de residuos
				\4[] Prevenir generación
				\4[] Fuera de ámbito de aplicación
				\4[] $\to$ Emisiones a la atmósfera
				\4[] $\to$ Residuos radioactivos
				\4[] $\to$ Residuos mineros
			\3 PEMAR -- Plan Estatal Marco de Gestión de Residuos 2016--2022
				\4 Instrumento para política española de residuos
				\4 Objetivos nacionales y autonómicos
				\4 CCAA deben cumplir objetivos mínimos
			\3 Programa Estatal de Prevención de Residuos 2014-2020
				\4 Desarrollo de política de prevención de residuos
				\4 Análisis de medidas de prevención
				\4 Líneas estratégicas
				\4[] Reducción de residuos generados
				\4[] Reducción de contenido de sustancias nocivas
				\4[] Reducción de impactos adversos
			\3 Estrategia Española de Economía Circular 2030
				\4 Basado en Pacto por la economía circular
				\4[] Conjunto de acciones para aumentar circularidad
				\4 Estrategia aún en desarrollo
				\4 Ministerios implicados
				\4[] Agricultura y Pesca
				\4[] Transición Ecológica
				\4[] Economía
				\4 Aumentar trazabilidad de residuos
				\4 Aprovechamiento materias primas en residuos
				\4 Esencial
			\3 Impuesto sobre Plásticos de Un Solo Uso
				\4 Impuesto indirecto especial
				\4 Fabricación, importación, adquisición UE
				\4[] Envases de plástico no reutilizables
				\4[] $\to$ Que se vendan en España
				\4 0.45 euros por kilogramo de envase
				\4 Tratar de reducir comercialización
				\4[] 70\% de aquí a 2030
				\4 Prohibir distribución gratuita
				\4 Obligatorio especificar precio en ticket de venta
	\1 \marcar{Industria del automóvil}\footnote{Ver \href{https://www.tendencias.kpmg.es/2019/02/claves-automocion-2019/}{KPMG 2020 sobre cambios esperables en sector del automóvil.}}
		\2 Análisis estático
			\3 Delimitación del sector
				\4 Concepto
				\4[] CNAE División 29 Fabricación vehículos a motor
				\4[] Apartados respectivos sobre:
				\4[] $\to$ Comercialización
				\4[] $\to$ Reparación
				\4[] $\to$ Repuestos
				\4[] $\to$ Alquiler
				\4 Subsectores
				\4[] Turismos
				\4[] $\to$ Suv Medio
				\4[] $\to$ Suv Pequeño
				\4[] $\to$ Utilitario
				\4[] $\to$ Compacto
				\4[] $\to$ Medio
				\4[] $\to$ Monovolumen peqeuño
				\4[] $\to$ Monovolumen grande
				\4[] Comerciales ligeros
				\4[] Camiones y transporte pesado
				\4 Diferenciación del producto
				\4[] Relativamente elevada
				\4 Características de la demanda
				\4[] Procíclica
				\4[] Demanda de demostración y emulación
				\4[] Sectores industriales arrastran
				\4[] $\to$ En segmento de vehículos industriales
				\4[] Medioambiente cada vez más relevante
				\4[] Mercado de segunda mano mucho peso
				\4 Fuentes estadísticas
				\4[] INE
				\4[] $\to$ Estadística de Fabricación de Vehículos
				\4[] $\to$ Estadística de Matriculación de Vehículos
				\4[] $\to$ Estadística del Parque Nacional de Vehículos
				\4[] $\to$ Transferencia de vehículos
			\3 Importancia
				\4 Cualitativa
				\4[] Arrastre en todo el sector industrial
				\4[] $\to$ Siderurgia
				\4[] $\to$ Productos metálicos
				\4[] $\to$ Software
				\4[] $\to$ Alta tecnología
				\4[] $\to$ Materiales
				\4[] $\to$ Caucho
				\4[] $\to$ Plásticos
				\4[] $\to$ ...
				\4 Cuantitativa
				\4[] VAB en fabricación:
				\4[] $\to$ 12.000 M de € en 2018
				\4[] Empleos en fabricación:
				\4[] $\to$ 160.000 empleos directos
				\4[] Total incluyendo:
				\4[] $\to$ Fabricación
				\4[] $\to$ Venta
				\4[] $\to$ Reparación
				\4[] $\then$ VAB: 30.000 M de €
				\4[] $\then$ Empleos: 460.000
				\4[] Tercer sector de exportación en 2019
				\4[] $\to$ Tras bienes de equipo y ABT
				\4[] $\to$ Tendencia a la baja
				\4[] $\to$ Esencial en exportación de mercancías
				\4[] $\then$ Fuertemente dependiente de exportaciones
			\3 Modelos teóricos relevantes
				\4 CVG
				\4 Ide Horizontal
				\4 Demanda de características
				\4 Hotelling localización
				\4 Oligopolios
				\4 Comercio interindustrial
				\4 Renta permanente
			\3 Oferta
				\4 Factores
				\4 Parque de vehículos
				\4[] 35 millones de vehículos
				\4[] $\to$ 25 millones son turismos
				\4[] $\to$ 5 millones son camiones y furgonets
				\4[] $\to$ 5 millones el resto
				\4 Vehículos producidos
				\4[] Más de 2.800.000 en 2019
				\4[] $\to$ Estable respecto a 2018
				\4 Producción por segmentos
				\4[] Turismos
				\4[] $\to$ Principal segmento
				\4[] $\to$ 2.200.000 vehículos
				\4[] $\to$ Sobre todo suv pequeños y utilitarios
				\4[] $\to$ Estable en últimos años
				\4[] Comerciales ligeros
				\4[] $\to$ 500.000 vehículos
				\4[] $\to$ Tendencia al alza
				\4[] Pesados
				\4[] $\to$ Resto
				\4[] $\to$ Tendencia a la baja en últimos años
				\4 Fabricantes a nivel mundial
				\4[] 91 millones de vehículos en 2019
				\4[] España 9 posición mundial
				\4[] Tras:
				\4[] 1. China
				\4[] 2. EEUU
				\4[] 3. Japón
				\4[] 4. Alemania
				\4[] 5. India
				\4[] 6. México
				\4[] 7. Corea del Sur
				\4[] 8. Brasil
				\4[] Superando a Francia
				\4[] Único junto con Brasil que crece en 2019
				\4 Distribución geográfica
				\4[] PSA
				\4[] $\to$ Vigo
				\4[] $\to$ Zaragoza
				\4[] $\to$ Madrid
				\4[] Ford
				\4[] $\to$ Valencia
				\4[] Mercedes-Benz
				\4[] $\to$ Cantabria
				\4[] $\to$ Vitoria
				\4[] Nissan
				\4[] $\to$ Ávila
				\4[] $\to$ Cantabria
				\4[] $\to$ Barcelona (proceso de cierre)
				\4[] Volkswagen
				\4[] $\to$ Navarra
				\4[] IVECO
				\4[] $\to$ Madrid
				\4[] $\to$ Valladolid
				\4[] Renault
				\4[] $\to$ Valladolid
				\4[] $\to$ Palencia
				\4[] $\to$ Sevilla
				\4[] Seat
				\4[] $\to$ Barcelona
			\3 Demanda interna
				\4 Matriculaciones
				\4[] Cayeron en 2019
				\4[] Canales de venta
				\4[] $\to$ Empresa
				\4[] $\to$ Alquilador
				\4[] $\to$ Particulares
				\4 Particulares
				\4[] Fuerte tendencia a la baja
				\4[] Incertidumbre sobre regulación futura
				\4[] Políticas distintas por territorios
				\4[] $\to$ Grandes ciudades
				\4[] $\to$ Áreas rurales
				\4[] $\to$ ...
				\4 Se rompe tendencia positiva desde 2012
				\4 Madrid comunidad que más matricula
				\4 Vehículos comerciales aumentan ventas
				\4 Industriales y autobuses estables
				\4 Empresas aumentan fuertemente en últimos años
				\4 Alquilador aumenta fuertemente desde últimos años
				\4 Particulares parece ser segmento maduro
				\4 Principales marcas
				\4[] Sobre todo, las que producen también en España
				\4[] $\to$ Más Toyota sin fábricas en España
				\4[] $\to$ Más Audi y BMW
			\3 Sector exterior
				\4 Exportaciones
				\4[] Pequeño aumento en unidades exportadas
				\4[] 80\% dedicado a exportación
				\4[] Mucho mayor en vehículos que en componentes
				\4[] $\to$ Induce saldo positivo en total vehículos+componentes
				\4[] Vehículos terminados
				\4[] $\to$ Cercanas a 34.000 M de €
				\4[] Componentes
				\4[] $\to$ Cercanas a 10.000 M de €
				\4 Destinos de exportación
				\4[] Principalmente dentro de UE
				\4[] $\to$ Grandes economías de la UE mayores receptores
				\4[] $\to$ Orden casi igual a tamaño relativo
				\4[] Fuera de la UE
				\4[] $\to$ Turquía
				\4[] $\to$ Suiza
				\4[] $\to$ México
				\4[] $\to$ EEUU
				\4 Importaciones
				\4[] Vehículos
				\4[] $\to$ Cercanas a 20.000 M de €
				\4[] Componentes
				\4[] $\to$ Cercanas a 20.000 M de €
				\4 Saldo neto en 2019
				\4[] Automóviles terminados
				\4[] $\to$ Desde casi equilibrio en 2005 a fuerte superávit
				\4[] $\then$ Hasta +13.700 M en vehículos terminados
				\4[] $\then$ 500.000 vehículos más producidos que comprados
				\4[] Componentes de vehículos
				\4[] $\to$ Fuertemente deficitario
				\4[] $\to$ Compensa en gran medida el saldo positivo en vehículos
				\4[] Total vehículos y componentes
				\4[] $\to$ Saldo positivo muy inferior
				\4[] $\then$ Cercano a +3.000 M de €
				\4 Competidores
				\4[] Sector con protección relativamente fuerte
				\4[] $\to$ A nivel de bloques comerciales
				\4[] Francia, Italia, Turquía competidores en UE
				\4[] Marruecos respecto a PEDs de Norte de África
				\4 Acuerdos y negociaciones comerciales
		\2 Componentes de automóviles
			\3 Idea clave
				\4 Relativamente elevada producción nacional
				\4 Fuertemente deficitario
				\4[] Compensa superávit en vehículos
				\4[] Mayor parte de VA en montaje
			\3 Empresas
				\4 Gestamp
				\4 Grupo Antolín
				\4 FagorEderlan
				\4 CIE Automotive
			\3 Empleo
				\4 Cercano a 230.000 personas
			\3 Exportaciones
				\4 Sobre todo a Europa
				\4 Creciente Marruecos como destino
		\2 Distribución de vehículos: concesionarios
			\3 Idea clave
				\4 Distribución selectiva
				\4[] Sólo algunos distribuidores autorizados
				\4 Venta a empresas y particulares
			\3 Evolución
				\4 Norma WLTP
				\4[] Entrada en vigor en septiembre 2018
				\4 Necesario dar salida a vehículos obsoletos
				\4[] Impulsa descuentos y ventas
				\4 Covid
				\4[] Cierre concesionarios y actividad
				\4[] Caída inicial
				\4[] Recuperación relativamente lenta
				\4[] $\to$ Incertidumbre
			\3 Perspectivas
				\4 Aumento de ventas a empresas
				\4[] Reducción de márgenes comerciales
				\4 Se preve desaparición de concesionarios
				\4 Más fácil contactar y comparar ofertas
				\4[] Internet y similares
				\4 Elevada incertidumbre
				\4[] Tecnologías a vender
		\2 Análisis dinámico
			\3 Evolución
				\4 Inicios de la industria
				\4 Primeras fábricas en España
				\4 Autarquía
				\4[] ENASA
				\4[] $\to$ Pegaso
				\4[] $\then$ IVECO
				\4 Planes de Desarrollo
				\4 Años 80
				\4 Años 90
				\4 Cadenas de valor global
				\4 Crisis financiera
			\3 Actualidad
				\4 Covid-19
				\4 Propulsión eléctrica
				\4 Baterías
			\3 Elevada incertidumbre
			\3 Crisis Covid
				\4 Fuerte reducción inicial de ventas
				\4 Paralización de producción
				\4[] Caída de demanda
				\4[] Disrupciones oferta relativamente leves
			\3 Carsharing
				\4 Alquiler de vehículos por periodos cortos
				\4 Internet para matching
				\4 Aumenta utilización del capital
				\4 Aumento demanda vehículos eléctricos
				\4 Áreas urbanas y elevada densidad
			\3 Transporte colaborativo
				\4 Alquiler de vehículos con conductor
				\4 Mayor utilización del capital
				\4 Posible caída demandad de vehículos
				\4 Inicialmente, transporte de vehículos
				\4 Extensión potencial a transporte de mercancías
			\3 Perspectivas
				\4 Propulsión eléctrica
				\4[] Sin emisiones
				\4[] Fuerte impulso políticas públicas
				\4[] Notable reducción de complejidad
				\4[] $\to$ Menos componentes de motor
				\4[] $\to$ Menores economías de escala
				\4[] $\to$ Reducir probable barreras de entrada
				\4 Baterías
				\4[] Tecnología avanzada
				\4[] Aumenta barreras de entrada
				\4[] $\to$ Compensa reducción de complejidad
				\4[] Elevado coste
				\4[] Metales pesados
				\4[] Tierras raras
				\4 Pila de combustible
				\4[] Hidrógeno
				\4[] Más autonomía
				\4[] Interacción con sector eléctrico
				\4[] $\to$ Producción vía electricidad
				\4 Economía colaborativa
				\4[] Blablacar
				\4[] Alquiler de muy corto plazo
				\4 Vehículo autónomo
				\4[] Tecnología de frontera
				\4[] Posible aumento de utilización del capital
		\2 Políticas públicas
			\3 Justificación
				\4 Peso en economía española
				\4 Importancia en exportaciones
				\4[] Mantener superávit exterior
				\4 Emisiones de efecto invernadero
				\4[] Transporte cercana 50\% emisiones
				\4[] Automóviles gran parte
				\4 Contaminación en áreas urbanas
				\4[] Coches grandes emisores
				\4 Renovación de parque de automóviles
				\4[]
				\4 Seguridad vial
				\4 Nuevos modelos de negocio digitales
				\4[] Ecosistemas relativos a venta de automóviles
				\4[] $\to$ Aplicaciones de navegación
				\4[] $\to$ Interconexión entre automóviles
				\4[] $\to$ 5G
				\4 Conducción autónoma
				\4 Nuevos modelos de negocio
				\4 Maduración de mercado europeo
				\4 Presión competitiva en Asia
			\3 Objetivos
				\4 Mantener ventaja comparativa en montaje
				\4 Incorporar nuevas tecnologías
				\4[] Conducción autónoma
				\4[]
				\4 Reducir impacto medioambiental
				\4 Transformación energética
			\3 Antecedentes
			\3 Marco jurídico
			\3 Marco financiero
			\3 Actuaciones
			\3 IPPC
			\3 Planes PIMA
				\4[] Sectores difusos
			\3 Plan Renove
				\4 Ayudas para compra de todo tipo automóviles
				\4 Todo tipo de automóviles
				\4[] $\to$ Combustión, eléctricos, híbridos
				\4 Sujeto a límites de emisiones
				\4 Necesario achatarrar coche > 10 años
			\3 Plan Moves II: ayudas compra coches eléctricos\footnote{Ver \href{https://cincodias.elpais.com/cincodias/2020/09/18/companias/1600426614_692764.html}{Cinco Días (2020): Algunas autonomías no convocan ayudas.}.}
				\4 Ayudas a compra de coches eléctricos
				\4 Ejecución limitada en algunas CCAA
			\3 WLTP -- World Harmonized Light-duty Vehicle Test
				\4 Estándar global sobre emisiones CO2
				\4 Vehículos ligeros
				\4 Vehículos deben homologarse
				\4[] Para poder valorar emisiones CO2
				\4 Entrada en vigor en 2018
				\4[] Impulsó venta de vehículos
				\4[] Necesario vender vehículos obsoletos
				\4[] $\to$ Aplicación de descuentos
			\3 Plataforma Tecnológica del Hidrógeno y Pilas de Combustible
				\4 Proyecto financiado por Ministerio de Ciencia
				\4 Agrupar investigación en hidrógeno
				\4 Especialmente importante para ind. automóvil España
				\4 Aumentar viabilidad comercial hidrógeno
			\3 Valoración
				\4 Transformaciones pueden afectar competitividad España
				\4[] Baterías
				\4[] Otras tecnologías de propulsión
				\4 Dependencia de componentes importados
				\4 Problemas de innovación en tecnologías frontera
				\4[] Baterías
				\4[] Conducción autónoma
			\3 Retos
				\4 Covid
				\4[] Incertidumbre
				\4[] Caída de renta disponible
				\4 Transición al vehículo eléctrico
				\4[] Necesario financiar tomas de electricidad
				\4[] Implementación de corredores eléctricos
				\4[] $\to$ Posible cargar vehículo
				\4[] $\then$ A lo largo de todo el trayecto
				\4[] Elevado coste de inversión
				\4[] $\to$ Zonas poco densas muy difíciles
				\4[] $\then$ Poca rentabilidad de inversión
				\4[] $\then$ Poca rentabilidad de fondos públicos
				\4 Regulación de conducción autónoma
				\4[] Aspecto muy difícil
				\4[] Implicaciones económicas profundas
				\4[] Sector inmobiliario
				\4[] $\to$ ¿Pagar por vivir cerca de trabajo rentable?
				\4 Aumentar tasa de utilización de stock de K
				\4[] Vía mecanismos de compartición
				\4[] $\to$
				\4[] $\then$ Nuevos modelos de negocio
				
	\1[] \marcar{Conclusión}
		\2 Recapitulación
			\3 Bienes intermedios
			\3 Industria del automóvil
		\2 Idea final
\end{esquemal}

\graficas

\conceptos

\preguntas

\notas

\bibliografia

Mirar en Palgrave:
\begin{itemize}
	\item 
\end{itemize}

KPMG Tendencias (2019) \textit{Principales tendencias según informe Global Automotive Executive Survey} \href{https://www.tendencias.kpmg.es/2019/02/claves-automocion-2019/}{Disponible aquí}

Larrea Basterra, M. Berezo García, A. (2017) \textit{La siderurgia en España y su futuro.} \href{https://www.mincotur.gob.es/Publicaciones/Publicacionesperiodicas/EconomiaIndustrial/RevistaEconomiaIndustrial/406/LARREA%20Y%20GARCIA.pdf}{Disponible aquí} -- En carpeta del tema

\end{document}
