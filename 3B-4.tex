\documentclass{nuevotema}

\tema{3B-4}
\titulo{Análisis del crecimiento de la empresa. Métodos de valoración de empresas. Especial referencia a los procesos de fusión, adquisición y alianzas estratégicas.}

\begin{document}

\ideaclave

REFORMAR LA SECCIÓN TEÓRICA PARA INCLUIR Lucas (1978) y Evans (1986)

Existen diferentes concepciones de la empresa, relacionadas a su vez con distintas justificaciones de su existencia. La teoría neoclásica entiende la empresa como un proceso de decisión cuyo objetivo es la obtención del máximo beneficio produciendo outputs a partir de una serie de inputs. Coase matizó esta definición, entendiendo la empresa como un conjunto de contratos que permite minimizar los costes de gestión y transacción. La empresa es también un agente social que permite satisfacer las demandas de la población de una serie de bienes y servicios. Desde el punto de vista financiero, la empresa es un capital invertido con el objetivo de crear valor para el accionista. Sea cual sea la definición de empresa que adoptemos, es un hecho que las empresas experimentan cambios constantes, que crecen, reducen sus dimensiones y desaparecen, se fusionan con otra empresas o las adquieren, se alían con otras empresas... Existen numerosos ejemplos recientes: la adquisición de Opel por el grupo PSA, la expansión internacional de las empresas españolas en las últimas décadas, el menor tamaño medio de las empresas españolas respecto de otros países europeos, los cambios en las mayores empresas del mundo por cotización en bolsa en la última década... Por ello, y tanto desde el punto de vista teórico como desde el punto de vista práctico del directivo de empresa que toma decisiones que repercuten en el tamaño de la empresa, es necesario valorar las empresas para obtener medidas cuantitativas de su actividad. El \textbf{objeto} de esta exposición es dar respuesta a una serie de preguntas tales como: ¿cómo crecen las empresas? ¿por qué lo hacen? ¿cómo se valoran las empresas? ¿en qué consiste una fusión, una adquisición o una alianza estratégica? La \textbf{estructura} de la exposición se divide en tres partes: en primer lugar, exponemos los rasgos característicos del crecimiento atendiendo a los hechos empíricos y a las principales explicaciones teóricas, posteriormente analizamos los procesos de crecimiento externo tales como las fusiones, las adquisiciones y las alianzas estratégicas; y realizamos por último un examen de los principales métodos de valoración.

El crecimiento de la empresa no es un concepto inequívoco de único significado. Si caracterizar una empresa requiere de numerosas dimensiones y variables, el \marcar{análisis del crecimiento} es posible también en relación a muchas variables. Así, la \textbf{definición de crecimiento} puede ser relativa al número de empleados, en volumen de ventas, en perímetro de mercado, en valor añadido, en tamaño del balance, en beneficios... Algunas de estas medidas de crecimiento requieren definir el perímetro de mercado, especialmente la cuota de mercado. En general, los estudios teóricos y empíricos del crecimiento se basan en un conjunto de las medidas anteriores, aunque dando menos relevancia al beneficio por tratarse de una variable con fuerte componente financiero exógeno a las empresas. 

El análisis de la \textbf{evidencia empírica} del crecimiento empresarial se inició en gran medida por Gibrat en los años 30. El autor formuló una ecuación que describe un proceso dinámico generador del tamaño de las empresas que aunque no establece relación causal alguna, aproxima con notable precisión algunos hechos estilizados y por ello, se ha consolidado como benchmark de comparación de otros modelos. En su versión débil, la Ley de Gibrat puede reducirse a afirmar que el crecimiento de las empresas no está relacionado con su tamaño. En su versión fuerte, caracteriza el tamaño de una empresa como el resultado de una secuencia de shocks aleatorios desde un tamaño inicial, relación a menudo expresada como que el logaritmo del tamaño es igual a la suma de shocks aleatorios indexados temporalmente. De este proceso estocástico emerge una distribución log-normal del tamaño de las firmas. En general, la Ley de Gibrat muestra un ajuste notablemente preciso de la distribución de firmas grandes. No así para empresas de menor tamaño, en las que aparece una correlación negativa entre tamaño y crecimiento. Así, cabe resumir algunos de los hechos estilizados con los que la Ley de Gibrat debe compararse. Respecto al tamaño, la distribución de empresas muestra efectivamente una distribución log-normal a nivel agregado. Sin embargo, observaciones de la distribución a nivel sectorial muestra perfiles multimodales (con múltiples picos), y sólo una convergencia temporal hacia distribuciones log-normales unimodales. La varianza del crecimiento muestra un cierto sesgo anticíclico en general, de tal manera que en momentos bajos del ciclo la distribución de crecimiento de las empresas reduce su varianza de forma compatible con un cierto ``rigor'' anticrecimiento en momentos de crisis. La curtosis de las distribuciones de crecimiento es por lo general procíclica, de manera que el grosor de la colas de las distribuciones aumenta en fases alcistas. La edad de la empresa muestra en general una correlación negativa con el crecimiento. El rendimiento financiero no muestra por lo general correlaciones consistentes. La productividad muestra una correlación ambigua que depende fuertemente de la variable considerada. El ciclo macroeconómico se muestra generalmente relacionado con las tasas de crecimiento.

En relación a estos hechos empíricos, existe en la literatura una serie de \textbf{modelos teóricos del crecimiento} que tratan de explicitar los mecanismos causales determinantes del crecimiento. El \underline{modelo neoclásico} más simple parte del supuesto de comportamiento racional para caracterizar el crecimiento como resultado de un problema de optimización de los beneficios empresariales. Así, según este modelo, una empresa aumentará o reducirá su tamaño en la medida en que ello resulte en mayores beneficios. \underline{Coase (1937)} desarrolló este resultado y lo enmarcó en el contexto de la reducción de costes. Así, una empresa que maximiza beneficios debe minimizar los costes de transacción y de gestión. En la medida en que resulte menos costoso llevar a cabo una actividad en el seno de la empresa porque los costes de transacción con terceros superan los costes de gestión, la empresa crecerá. En la medida en que gestionar un aumento del tamaño sea más costoso que externalizar la actividad y negociar con un tercero, la empresa no crecerá o decrecerá para desprenderse de esa actividad. La teoría de las fases de crecimiento planteada por Greiner (1972) caracteriza el crecimiento de la empresa en términos de los problemas que los directivos deben afrontar y las crisis que dan lugar a nuevos problemas. Así, en una primera fase el crecimiento de la empresa depende del liderazgo como problema básico a resolver. A continuación, y tras una crisis de liderazgo, la puesta en práctica  de procesos de dirección efectiva se convierte en el problema central, que a su vez desemboca en una crisis de autonomía respecto de agentes externos a la empresa. En la siguiente fase, los directivos deben afrontar el problema de la delegación efectiva y deben posteriormente hacer frente a una crisis de control de la actividad. A continuación aparece el problema de la coordinación entre diferentes agentes internos, lo que desemboca en crisis de excesivo peso de la burocracia interna. En la última fase, cuando la empresa alcanza la madurez, el reto principal consiste en implementar una colaboración eficiente entre las diferentes unidades de decisión de la empresa. Este tipo de modelos de crecimiento basados en fases no gozan en la actualidad de demasiado seguimiento pero dieron lugar a una extensa literatura en el pasado, que aún es relevante en la actualidad. La \underline{teoría del crecimiento de Penrose}, planteada por primera vez en el libro seminal \textit{The Growth of Firms in Theory and Practice} conceptualiza el crecimiento como el resultado de un proceso de aprendizaje por parte de los directivos. Cuando los directivos aprenden a gestionar la empresa y los diferentes problemas que plantean en cada fase, liberan recursos en términos de esfuerzo personal y tiempo destinados a la gestión operativa y los destinan a proyectos de crecimiento. \underline{Marris (1964)} y posteriormente otros autores como Baumol desarrollan un enfoque del crecimiento basado en las motivaciones de los directivos. Conectando con la teoría de la agencia que comenzaba a desarrollarse, estos modelos entienden que los managers tratan de maximizar su propia función de utilidad y ésta depende fundamentalmente del tamaño de la empresa y del rendimiento financiero. Mayor tamaño implica mayores prebendas y prestigio personal, mientras que mejor rendimiento financiero reduce la posibilidad de despido. A su vez, mayor tamaño se relaciona con menor rendimiento financiero provocado por inversiones poco rentables. Así, el tamaño óptimo pasa a ser aquel que da lugar al mínimo rendimiento satisfactorio para los accionistas.

A la vista de los diferentes modelos positivos del crecimiento, cabe adoptar un enfoque normativo y preguntarse: \textbf{¿por qué debe crecer una empresa?} El crecimiento de una empresa no es inevitable, y las empresas pueden preferir en determinados contextos no crecer. Además, la búsqueda de oportunidades de inversión es costosa para los directivos. Algunas de las \underline{ventajas} del crecimiento que la literatura señala como más relevantes son el alivio de tensiones internas relacionadas con las aspiraciones de empleados, la realización de economías de escala y alcance, la reducción de costes medios, la diversificación de riesgos que permite estabilizar flujos de caja, la imposición de barreras de entrada a competidores que permitan aumentar el poder de mercado, o el simple resultado de la utilización del crecimiento como métrica de rendimiento para los directivos. Entre las \underline{desventajas} del crecimiento se pueden citar la pérdida de control sobre las actividades de la empresa por parte de accionistas y alta dirección, la aparición de problemas de coordinación, el desconocimiento de nuevos mercados y por ello el aumento del riesgo, las barreras legales o simplemente, la preferencia de los managers por menores responsabilidades y riesgos en determinados contextos.

Las \textbf{estrategias de crecimiento} de las empresas son fundamentalmente tres: replicación, diversificación y crecimiento externo. La \underline{replicación} hace referencia al aumento de la escala de las operaciones, dentro de una misma actividad. La \underline{diversificación} hace referencia al aumento en la variedad de las operaciones. El \underline{crecimiento externo} es el objeto de la segunda parte de la exposición, y fundamentalmente consiste en el crecimiento basado en recursos productivos ajenos distintos de los que la empresa posee inicialmente.

Desde el punto de vista financiero, los \marcar{procesos de crecimiento externo} tienen por objetivo declarado --como toda decisión de los directivos-, aumentar el valor de la empresa. La puesta en común de recursos productivos se plantea en el contexto del crecimiento externo como una forma de realizar sinergias entre factores que no se estaba produciendo hasta el momento y que resultarían, de efectivamente realizarse, en creación de valor. De forma equivalente, se trata de poner en común recursos con la expectativa de que el todo valga más que la suma de las partes. La realización de sinergias tiene fundamento en dos fenómenos: el aumento de los ingresos y la reducción de los costes. El aumento de ingresos puede deberse a precios más altos debidos a mayor poder de mercado, a aumentos de la productividad, a las mejoras derivadas de compartir know-how o procesos tecnológicos que aumentan la calidad del producto... Las reducciones de costes a menudo se derivan de la realización de economías de escala, la eliminación de ineficiencias en la gestión, ahorros fiscales, independencia respecto a proveedores, mayor diversificación de riesgos... El \underline{diseño de la operación} es también un elemento clave del éxito. Én términos abstractos, un diseño acertado es aquel que permite crear el máximo valor. En la práctica, el diseño se concreta en dar respuesta a preguntas como: ¿cuánto obtiene cada parte? ¿cómo se financia la operación? ¿cuándo se lleva a cabo? ¿qué condiciones han de cumplirse tras la operación? ¿qué requisitos deben producirse para que la operación efectivamente tenga lugar? ¿cuándo se lleva a cabo?  Además, los procesos de crecimiento externos a menudo están motivados por razones de carácter industrial. Las operaciones de carácter vertical son aquellas en las que se integran dos empresas que producen en diferentes etapas de un mismo proceso productivo. En las operaciones horizontales, se trata de empresas en la misma etapa del mismo proceso productivo, y el objetivo suele ser aumentar el poder de mercado. Los conglomerados o holdings resultan de operaciones de integración en los que las empresas integradas realizan actividades diferentes. Los \underline{factores} que afectan al número de operaciones de crecimiento externo dependen de una serie de factores entre los cuales destacan factores macroeconómicos como la fase del ciclo o las dificultades para el crecimiento orgánico en economías avanzadas, legales como el entorno regulatorio y microeconómicos como las posibilidades de realización de economías de escala o alcance que aparecen en mercados emergentes.

Las \textbf{fusiones} son operaciones societarias en virtud de las cuales dos empresas pasan a constituir una sola y los accionistas de una y otra pasan a serlo de la empresa resultante de la fusión.  La relación de intercambio determina la proporción relativa de acciones de la nueva sociedad distribuidas a los accionistas de las sociedades fusionantes. Así, dadas dos empresas A y B, la relación de intercambio de A por B o  acciones de A a entregar por cada acción de B corresponde con el cociente entre el valor de B y el de A multiplicado por el cociente entre el número de acciones de A y B. Existen diferentes modalidades de fusión en función de la forma que tome la empresa resultante. Así, en ocasiones ambas empresas mantienen personalidad jurídica separada a pesar del intercambio de acciones, en otras una de las empresas se integra plenamente en otra y pasa a existir una sola personalidad jurídica, y en otras aún se crea una tercera empresa participada por los accionistas de una y otra que a su vez posee la totalidad del capital social de las dos empresas originales. Las fusiones se caracterizan por el hecho de que la realización de las sinergias es incierta a los participantes, y por ello ambas partes incurren en un riesgo mayor.

Las \textbf{adquisiciones} son procesos de crecimiento externo en los cuales una empresa adquiere la propiedad de otra a cambio de dinero. Así, los accionistas que venden su participación en la empresa adquirida obtienen de forma inmediata el valor de las sinergias potenciales que hayan sido capaces de negociar frente al adquiriente. Las adquisiciones pueden realizarse con el acuerdo de los gestores y de la totalidad de los accionistas del adquirido, pero también en su contra. Esto es especialmente posible en empresas cotizadas con muchos accionistas anónimos a los cuales ofrecer un precio de compra más alto que el del mercado. Existen por ello numerosas \underline{estrategias defensivas} que los directivos y/o grupos de accionistas pueden implementar para evitar la adquisición. Por ejemplo, mantenimiento de acciones con derechos especiales (en las legislaciones donde esta práctica se permita), obligación de aprobación previa a la entrada en accionariado, aumentos restringidos de capital, recompras de acciones, obligaciones de notificar a gestores participaciones que superen un umbral, contra-ofertas de adquisición en acuerdo con un tercero, compra de acciones de empresa que pretende adquirir por parte de empresa objetivo, emisión de warrants ``pildora venenosa'', acciones legales... Existen asimismo numerosas \underline{modalidades} de adquisición. Por ejemplo, adquisiciones financiadas mediante emisión de deuda, adquisiciones legales obligatorias una vez que la participación de un accionista supera determinado umbral, \textit{leveraged buy-outs} cuando la adquisición se financia a través de muy elevado apalancamiento para incentivar a los managers a generar caja en mercados maduros, management LBOs cuando son los propios directivos los que inician la operación...

Las \textbf{alianzas estratégicas} entre empresas son operaciones de puesta en común de recursos de distintos tipos (físicos, capital financiero, know-how, trabajadores...) sin transferencia de acciones y con un cuerpo normativo propio específico al caso. Existen tantas clases de alianzas estratégicas como ejemplos concretos, pero pueden clasificarse en grupos de sociedades en las que distintas empresas actúan coordinadamente bajo la supervisión de una misma dirección generalmente constituida por un grupo de accionistas, uniones temporales de empresa sin personalidad propia y horizonte limitado que acuerdan colaborar para realizar un objetivo concreto, franquicias, joint-ventures, sociedades de garantía recíproca...

Las \textbf{separaciones} de empresa son el proceso inverso de los procesos de crecimiento externo. Consisten en la separación de parte del activo para constituir una nueva sociedad, que puede estar participada por los mismos accionistas que la sociedad original o por otros diferentes si está unido a un proceso de adquisición. Este tipo de operaciones de separación tienen sentido cuando la empresa está excesivamente diversificada y no es atractiva para el inversor. También son habituales las separaciones en conglomerados ineficientes en los que las empresas con beneficios apoyan a empresas con pérdidas. 

La \marcar{estimación del valor de la empresa} es fundamental en todo proceso de crecimiento externo. Como hemos visto anteriormente, tanto en un proceso de fusión como en un proceso de adquisición es necesario estimar el valor de las empresas participantes para poder tomar una decisión informada acerca de lo apropiado o no de llevar a cabo la operación. La valoración de una empresa puede llevarse a cabo adoptando distintos enfoques. Las valoraciones desde un enfoque global tratan de estimar el valor del conjunto de los activos y pasivos y a partir de ahí hallar el valor del patrimonio neto. También es posible valorar por partes la empresa y después sumar sus valores. La elección de un enfoque u otro depende fundamentalmente de la existencia de sinergias entre las diferentes masas de activo y de la especificidad de los activos. Así, por ejemplo, una empresa dedicada al alquiler de aeronaves puede ser valorada por partes de manera precisa, mientras que una empresa de consultoría en la que el activo físico apenas es relevante y sí lo es las relaciones entre los trabajadores y la capacidad para actuar coordinadamente deberá ser valorada adoptando un enfoque de valoración global. 

La \textbf{valoración por descuento de flujos} es aplicable tanto al enfoque global como al enfoque por partes. Consiste en actualizar una serie de flujos de caja que se estima que la empresa podrá generar en el futuro y sumar esos flujos, obteniendo así una estimación directa del valor intrínseco de la empresa. Los flujos a descontar son los llamados \underline{flujos de caja libre} (FCF), que son aquellos disponibles para remunerar a los proveedores de capital. El FCF puede estimarse restando al EBITDA los impuestos a pagar, la inversión en activo fijo prevista (CAPEX) y el aumento de las necesidades operativas de fondos del periodo. Si el objetivo es hallar el valor del equity, habrá de descontarse a la rentabilidad exigida al equity. Si el objetivo es estimar el valor total de la empresa incluyendo deuda, habrá que descontar el FCF al WACC. Será necesario también postular el llamando horizonte de visibilidad, a partir del cual no es posible realizar estimaciones precisas de los flujos a obtener y pueden realizarse supuestos sobre los flujos que generará la empresa aplicando simplemente la fórmula de Gordon. La imputación de sinergias debe ser especialmente prudente. Cuando se trata de valorar participaciones minoritarias en el accionariado, puede ser preferible descontar los dividendos estimados y no los FCF. El valor de la deuda puede estimarse a partir de su valor de mercado o su valor en libros. Salvo cuantías muy reducidas, es preferible estimar el valor de mercado en base al precio que costaría reponer tal cantidad de deuda. 

La \underline{valoración por múltiplos} consiste en estimar el valor de una empresa aplicando a una variable conocida como el valor en libros o el beneficio neto, un múltiplo que corresponde a otras empresas similares. Existen múltiples posibilidades respecto del múltiplo que aplicar. El PER, P/E o price-earnings-ratio es el cociente el precio de la acción y el beneficio neto por acción. El ratio P/B es el cociente entre el precio del total del equity y el valor en libros. Otros ratios relacionan alguna medida del beneficio operativo con el valor de la emrpesa. Cuando se trata de empresas que no generan aún beneficios, es posible utilizar otro tipo de ratios para estimar la capacidad de beneficios futuros tales como la rotación de activos, las visitas a páginas webs, las toneladas de producto por unidad de tiempo, el volumen de ventas, el número de suscripciones a servicios....

Los métodos basados en el \textit{goodwill} son una serie de reglas arbitrarias aplicadas al valor en libros y al beneficio neto. Fruto de su carácter arbitrario, la variedad de reglas aplicables es inmensa. Su mayor ventaja es la inmediatez en la aplicación derivado de su carácter heurístico. Destacan especialmente tres reglas. El método clásico consiste en estimar el valor de una empresa como el valor en libros del activo más una cantidad resultante de aplicar un coeficiente al beneficio neto. El método indirecto o alemán es similar al clásico pero se capitaliza el beneficio neto de un periodo como si se tratase de una perpetuidad y se aplica una suma ponderada al 50\% con el activo contable. El método anglosajón estima el valor de la empresa como el valor en libros más el llamado ``superbeneficio'' estimado como la diferencia entre el beneficio obtenido por la empresa y el rendimiento obtenido si se hubiese invertido el activo a tasa $i$, capitalizado por el interés de la deuda multiplicado por un coeficiente.

Aunque los métodos basados en múltiplos son sin duda útiles por su inmediatez y simplicidad y de ello es prueba su amplia utilización en la práctica, es necesario entender las razones por las que los ratios se desvían de las medias, por qué los ratios toman determinados valores, así como prestar atención a elementos que puedan introducir sesgos en el valor de los ratios tales como primas de control o similares.

La valoración por opciones conceptualiza las diferentes masas patrimoniales de la empresa como opciones financieras y después aplica los métodos habituales a éstas. Así, la deuda puede entenderse como la venta de una acción put a los accionistas, ya que éstos últimos pueden desprenderse de sus obligaciones con los acreedores a cambio del valor que resulte de la liquidación de la empresa. El equity puede entenderse como la compra de una opción call a los deudores. La paridad put-call puede utilizarse para relacionar valores de una y otra. 

A lo largo de la exposición hemos examinado los hechos más relevantes del crecimiento de las empresas, las particularidades de los procesos de crecimiento externo tales como fusiones, adquisiciones y alianzas estratégicas y por último, los métodos de valoración de empresas. El crecimiento de las empresas es en realidad un mercado en sí mismo, de tal manera que directivos de empresa e inversores compran y venden oportunidades de crecimiento como si de activos tangibles se tratase. Además, es necesario considerar el crecimiento no sólo como un fenómeno relevante para gestores y accionistas, sino también para el conjunto de la economía. Un tamaño adecuado de las empresas de una economía permite mejorar la eficiencia, generar empleo y evitar cuellos de botella e ineficiencias derivadas tanto de empresas demasiado pequeñas como de conglomerados excesivamente grandes.


\seccion{Preguntas clave}

\begin{itemize}
    \item ¿Cómo crecen las empresas?
    \item ¿Por qué crecen las empresas?
    \item ¿Cómo pueden crecer las empresas?
    \item ¿Cómo se valora una empresa?
    \item ¿Por qué y como se adquiere una empresa?
    \item ¿Por qué y como se fusionan dos empresas?
    \item ¿Qué otras formas de crecimiento o decrecimiento externo existen?
\end{itemize}

\esquemacorto

\begin{esquema}[enumerate]
	\1[] \marcar{Introducción}
		\2 Contextualización
			\3 Concepto de empresa
			\3 Ejemplos recientes de crecimiento externo
			\3 Impacto crecimiento
		\2 Objeto
			\3 Cómo crecen las empresas
			\3 Por qué crecen las empresas
			\3 Cómo se valoran las empresas
			\3 Cómo se fusionan, adquieren y alían las empresas
		\2 Estructura
			\3 Análisis del crecimiento
			\3 Valoración
			\3 Fusiones y adquisiciones
	\1\marcar{Análisis del crecimiento de la empresa}
		\2 Concepto de crecimiento
			\3 Empleados
			\3 Ventas
			\3 Beneficio operativo
			\3 Cuota de mercado
			\3 Valor añadido creado
			\3 Activos o tamaño del balance
			\3 Patrimonio neto
		\2 Justificación del crecimiento
			\3 Idea clave
			\3 Ventajas de crecer
			\3 Desventajas de crecer
		\2 Modelos teóricos del crecimiento
			\3 Neoclásico
			\3 Modelo de crecimiento interno
			\3 Teoría de las fases de crecimiento: Greiner (1972)
			\3 Teoría del crecimiento de Penrose (1959)
			\3 Maximización utilidad del manager -- Marris (1964)
			\3 Crecimiento de la empresa internacional
		\2 Evidencia empírica
			\3 Distribución del tamaño de las empresas
			\3 Ley de Gibrat (1931)
			\3 Tasa de crecimiento de las empresas
		\2 Estrategias de crecimiento
			\3 Replicación
			\3 Diversificación
			\3 Crecimiento externo
	\1 \marcar{Métodos de valoración}
		\2 Idea clave
			\3 Enfoques de valoración global
			\3 Suma de las partes
			\3 Valoración fundamental
			\3 Valoración pragmática
		\2 Valoración fundamental por descuento de flujos
			\3 Idea clave
			\3 Descuento de flujos de caja libres (FCF)
			\3 Descuento de dividendos
			\3 Valor de la deuda
			\3 Análisis Du-Pont
		\2 Valoración por múltiplos
			\3 Idea clave
			\3 \underline{Menú de múltiplos}
			\3 PER
			\3 Forward PER
			\3 P/B o Precio-Valor-Contable
			\3 Ratios beneficio operativo / valor de empresa
			\3 Múltiplos basados en otras medidas
			\3 \underline{Métodos basados en el goodwill}
			\3 Método clásico
			\3 Método indirecto o alemán
			\3 Método directo o anglosajón
			\3 Valoración
		\2 Suma de las partes
			\3 Concepto
			\3 \underline{Valoración de activos}
			\3 Valor a coste histórico
			\3 Valor razonable
			\3 Valor neto realizable
			\3 Valor actual
			\3 Valor en uso
			\3 Costes de venta
			\3 Valor según coste amortizado
			\3 Costes de transacción de activos y pasivos financieros
			\3 Valor contable o valor en libros
			\3 Valor residual
			\3 Implicaciones
		\2 Valoración por opciones
			\3 Deudores
			\3 Accionistas
			\3 Paridad put call
			\3 Valoración
		\2 Comparación de métodos
			\3 Ventajas descuento de flujos
			\3 Desventajas descuento de flujos
			\3 Método de múltiplos
			\3 Ventajas suma de las partes
			\3 Desventajas
	\1 \marcar{Procesos de crecimiento externo}
		\2 Idea clave
			\3 Definición
			\3 Objetivo
			\3 Diseño de operación
			\3 Factores
		\2 Fusiones
			\3 Idea clave
			\3 Relación de intercambio
			\3 Modalidades
		\2 Adquisiciones
			\3 Idea clave
			\3 Estrategias defensivas
			\3 Modalidades
		\2 Alianzas estratégicas
			\3 Idea clave
			\3 Modalidades
		\2 Separaciones
			\3 Idea clave
			\3 Justificación
	\1[] \marcar{Conclusión}
		\2 Recapitulación
			\3 Análisis del crecimiento de las empresas
			\3 Procesos de crecimiento externo
			\3 Valoración de empresas
		\2 Idea final
			\3 Complejidad crecimiento
			\3 Desigual valoración del crecimiento
			\3 Consecuencias bienestar

\end{esquema}

\esquemalargo

\begin{esquemal}
	\1[] \marcar{Introducción}
		\2 Contextualización
			\3 Concepto de empresa
				\4 Teoría neoclásica
				\4 Teoría de contratos
				\4[] Coase
				\4 Agente social
				\4 Independientemente de definición
				\4[] Empresas sujetas a cambios de tamaño
			\3 Ejemplos recientes de crecimiento externo
				\4 Adquisición Opel por PSA
				\4 Posible fusión Deutschebank y Commerzbank
				\4[] Finalmente cancelada
				\4 Posible fusión Fiat-Chrysler
				\4 Expansión internacional empresas españolas
				\4[] Telefónica en América Latina y RU
				\4[] BBVA en Turquía
				\4[] Santander en América y UK
				\4 Menor tamaño medio de empresas españolas
				\4[] Efectos macroeconómicos
				\4 Cambios en 5 mayores empresas
				\4[] Hace una década: petroleras, Walmart
				\4[] Hoy: GAFAM
			\3 Impacto crecimiento
				\4 Bienestar
				\4[] Tienden a pagar mayores salarios
				\4 Productividad
				\4[] Aprovechamiento de economías de escala
				\4[] Más capacidad para atraer talento
				\4 Empleo
				\4 Riesgo
				\4[] Más tamaño $\to$ mejor absorción de shocks
		\2 Objeto
			\3 Cómo crecen las empresas
			\3 Por qué crecen las empresas
			\3 Cómo se valoran las empresas
			\3 Cómo se fusionan, adquieren y alían las empresas
		\2 Estructura
			\3 Análisis del crecimiento
			\3 Valoración
			\3 Fusiones y adquisiciones
	\1\marcar{Análisis del crecimiento de la empresa}
		\2 Concepto de crecimiento
			\3 Empleados
			\3 Ventas
			\3 Beneficio operativo
			\3 Cuota de mercado
				\4 Definir perímetro del mercado
			\3 Valor añadido creado
			\3 Activos o tamaño del balance
				\4 Valor en libros
				\4 Cotización en mercado organizado
				\4 Valor de reposición
				\4 Valor de liquidación
			\3 Patrimonio neto
		\2 Justificación del crecimiento
			\3 Idea clave
				\4 Crecimiento no es inevitable
				\4[] Firmas pueden preferir no crecer
				\4[] Crecimiento en varias dimensiones
				\4[] $\to$ No tienen porque crear valor para accionista
				\4 Crecer implica encontrar oportunidades
				\4[] $\to$ Búsqueda es costosa para managers
			\3 Ventajas de crecer
				\4 Economías de escala y alcance
				\4[] Reducción del coste medio
				\4 Diversificación de riesgos
				\4[] Permite estabilizar flujos de caja
				\4 Barreras de entrada
				\4[] Reducción de coste medio
				\4[] $\to$ Dificulta entrada de competidores
				\4 Crecimiento como medida de rdto. de gestores
				\4[] Medida simple de performance
				\4 Alivio tensiones internas
				\4[] Relacionadas con aspiraciones de empleados
			\3 Desventajas de crecer
				\4 Pérdida de control por accionistas
				\4 Problemas de coordinación
				\4 Desconocimiento nuevos mercados
				\4 Barreras legales
				\4 Aumento de costes administrativos
		\2 Modelos teóricos del crecimiento
			\3 Neoclásico
				\4 Supuesto de máxima racionalidad
				\4 Crecimiento:
				\4[] resultado de optimización de beneficios
				\4 Ejemplo:
				\4[] Crecimiento en producción y ff.pp empleados
				\4[] $\to$ Porque ingreso marginal supera coste marginal
				\4 Costes de transacción de Coase (1937)
				\4[] Máx. beneficios implica minimizar
				\4[] $\to$ Costes de transacción
				\4[] $\to$ Costes de autoridad/jerarquía/gestión
				\4[] Producir dentro de empresa
				\4[] $\to$ Reduce costes de transacción
				\4[] $\to$ Aumenta costes de gestión
				\4[] $\then$ Empresa crece si c. gestión < c. transacción
			\3 Modelo de crecimiento interno
				\4 Contexto
				\4[] Entender potencial de crecimiento
				\4[] $\to$ A partir de recursos generados internamente
				\4[] Variables contables, no financieras
				\4 Objetivo
				\4[] Caracterizar crecimiento de capital empleado
				\4[] Asumiendo constantes:
				\4[] $\to$ Estructura de K $\frac{D}{\text{PN}}$ de la empresa $\to$ constante
				\4[] $\to$ Ingresos/Capital $\to$ constante
				\4[] $\to$ $\frac{\text{EBITDA}}{\text{Capital total}}$/ROCE/$R_E$ $\to$ constante
				\4[] $\then$ Ingresos, prod., EBITDA crecen = que capital
				\4[] $\then$ Activos mantienen capacidad de generar riqueza
				\4 Resultados
				\4[] Dada estructura de capital constante
				\4[] $\to$ Equity crece tanto beneficio retenido
				\4[] $\to$ Deuda crece en proporción $\frac{D}{\text{PN}}$ a bfcio. retenido
				\4[] $\then$ Activo total crece a = \% que deuda y equity
				\4[] ¿De qué depende beneficio retenido?
				\4[] $\to$ ROCE/Rentabilidad económica
				\4[] $\to$ Coste de la deuda
				\4[] $\to$ Estructura del capital
				\4[] $\to$ Pay-out ratio
				\4[] $\then$ $\text{Crecimiento interno}$:
				\4[] \quad $R_F \cdot (1-\text{Payout })=\left( R_E + (R_E - i)\frac{D}{\text{PN}} \right) \cdot (1- \text{Payout})$
			\3 Teoría de las fases de crecimiento: Greiner (1972)
				\4 Enfoque de gestión
				\4[] $\to$ Problemas que afrontan los managers
				\4 Greiner (1972)
				\4[] Problemas que afrontan managers
				\4[] $\to$ Caracterizan fase del crecimiento
				\4[] $\to$ Relacionados con un tipo de crisis
				\4[] 1. Creatividad $\to$ liderazgo
				\4[] 2. Dirección $\to$ Autonomía
				\4[] 3. Delegación $\to$ Control
				\4[] 4. Coordinación $\to$ burocracia
				\4[] 5. Colaboración
			\3 Teoría del crecimiento de Penrose (1959)
				\4 Crecimiento resultado de aprendizaje
				\4[] Managers liberan recursos al aprender
				\4[] $\to$ Menos recursos destinados a gestión operativa
				\4[] $\to$ Más recursos disponibles para crecer
			\3 Maximización utilidad del manager -- Marris (1964)
				\4 Basada en teoría de la agencia
				\4[] F. de u. del manager = $f(\text{tamaño}, \text{rdto. financiero})$
				\4 Mayor rendimiento:
				\4[] Menor probabilidad despido
				\4 Mayor tamaño:
				\4[] Mayores prebendas
				\4[] $\to$ Más utilidad directa para manager
				\4[] Menor rendimiento financiero
				\4 Tamaño óptimo:
				\4[] $\to$ rendimiento mínimo satisfactorio
				\4 Menor crecimiento si propietarios controlan
				\4[] Más énfasis en rendimiento financiero
				\4[] Menos problema de agencia
				\4[] Evidencia empírica favorable
			\3 Crecimiento de la empresa internacional
				\4 Evidencia empírica robusta
				\4[] Empresas exportadoras son más grandes
				\4 Melitz (2003)
				\4[] Empresas heterogéneas
				\4[] Diferentes productividades
				\4[] $\to$ Diferentes costes marginales
				\4[] Producción para mercado nacional
				\4[] $\to$ Un coste fijo
				\4[] Producción para mercado internacional
				\4[] Coste fijo añadido
				\4[] Empresas capaces de superar costes fijos
				\4[] $\to$ Se mantienen en el mercado
				\4[] Apertura de comercio internacional
				\4[] $\to$ Aumenta competencia en mercado doméstico
				\4[] $\then$ Menores mark-ups
				\4[] $\then$ Menores beneficios operativos
				\4[] $\then$ Empresas menos productivas salen del mercado
				\4[] $\to$ Aparece posibilidad de exportar
				\4[] $\then$ Empresas más productivas rentabilizan exportación
				\4[] $\then$ Empresas más rentables tienden a crecer
				\4[$\then$] Aumento de tamaño y productividad con apertura comercial
				\4 Helpman, Melitz y Yeaple (2004)
				\4[] Crecimiento multinacional de empresas enfrenta trade-off
				\4[] $\to$ Reduce costes de transporte y aranceles
				\4[] $\then$ Más beneficio
				\4[] $\to$ Reduce posibilidad de realizar economías de escala
				\4[] $\then$ Aumenta beneficio
				\4[] Empresas más productivas capaces de replicar a menor coste
				\4[] $\to$ Pueden compensar coste de transporte con más plantas
		\2 Evidencia empírica
			\3 Distribución del tamaño de las empresas
				\4 Distribución log-normal a nivel agregado
				\4[] (Normal una vez aplicado un logaritmo)
				\4 Distribución multimodal a nivel sectorial
				\4[] Múltiples picos en la distribución
				\4 Convergencia temporal hacia log-normal
				\4 Diferencias tamaño medio por países
			\3 Ley de Gibrat (1931)
				\4 Proceso generador del tamaño
				\4[] $\to$ Sirve como benchmark de otros modelos
				\4[] $\to$ No analiza causas
				\4[] $\to$ Aproxima algunos resultados empíricos
				\4 Versión débil:
				\4[] $\to$ crecimiento sin correlación con tamaño
				\4 Versión fuerte:
				\4[] $\to$ crecimiento proceso estocástico
				\4[] $\log (x_t) \approx \log(x_0) + \sum_{s=1}^t \epsilon_s$
				\4[] $\epsilon$: var. aleatoria i.i.d, dist. normal
				\4[] $\to$ Emerge distribución log-normal de tamaños
				\4 Resultados empíricos:
				\4[] Se confirma para firmas grandes
				\4[] Rechazada firmas pequeñas
				\4[] $\to$ Correlación negativa crecimiento--tamaño
				\4[] $\then$ Para pequeñas, no es resultado de shocks aleatorios
			\3 Tasa de crecimiento de las empresas
				\4 Correlación temporal
				\4[] Resultados muy contradictorios
				\4[] Difícil decidir cuántos lags son relevantes
				\4 Ciclo económico
				\4[] Media procíclica
				\4[] $\to$ Más crecimiento en fase alcista
				\4[] Varianza procíclica
				\4[] $\to$ Crecimientos más dispersos en fase alcista
				\4[] $\to$ Existe ``rigor'' anticrecimiento en crisis
				\4[] $\to$ En ciclo adverso, todos los crecimientos bajos
				\4[] Kurtosis\footnote{Cómo de fat son las tails.} procíclica
				\4[] $\to$ En fase alcista, más valores extremos
				\4 Edad de la empresa
				\4[] Correlación negativa con crecimiento
				\4 Innovación
				\4[] Correlación difícil de probar con crecimiento
				\4 Rendimiento financiero
				\4[] Sin correlación robusta
				\4 Productividad
				\4[] Correlación ambigua
				\4[] Depende de parámetro analizado
				\4[] $\to$ ¿Productividad del trabajo?
				\4[] $\to$ ¿Productividad del capital?
		\2 Estrategias de crecimiento
			\3 Replicación
				\4 Aumentar escala de operaciones
			\3 Diversificación
				\4 Aumentar variedad de operaciones
			\3 Crecimiento externo
				\4 Puesta en común de medios de producción
				\4 Adquisiciones
				\4 Fusiones
				\4 Compras de activos
				\4 Alianzas
	\1 \marcar{Métodos de valoración}
		\2 Idea clave
			\3 Enfoques de valoración global
				\4 Directo
				\4[] Estimar valor del Patrimonio Neto
				\4 Indirecto
				\4[] Estimar valor del activo y restar pasivo
			\3 Suma de las partes
				\4 Combinación de métodos anteriores
				\4 Valorando componentes por separado
				\4[] y sumando
			\3 Valoración fundamental
				\4 Objetivo:
				\4[] determinar valor intrínseco
				\4 Métodos habituales de valoración de inversiones
				\4[] Descuento de dividendos
				\4[] Descuento de flujos de caja libres
			\3 Valoración pragmática
				\4 Por analogía con activos similares
		\2 Valoración fundamental por descuento de flujos
			\3 Idea clave
				\4 Actualizar flujos de caja
				\4 Sumar flujos de caja
				\4[$\Rightarrow$] Estimación directa valor intrínseco
			\3 Descuento de flujos de caja libres (FCF)
				\4 Flujos de caja disponibles para remunerar
				\4[] $\text{EBITDA} - \text{Impuestos} - \text{CAPEX} - \varDelta \text{NOF}$
				\4 Descuento a WACC si FCF antes de intereses:
				\4[] Valor empresa
				\4 Descuento a $k_e$:
				\4[] si FCF después de de intereses
				\4 Estimación horizonte de FCF (visibilidad)
				\4[] $\to$ ¿Hasta cuando podemos estimar flujos?
				\4 Crecimiento g constante: $V = \frac{\text{FCF normalizados}}{k-g}$
				\4 Prudencia si imputación sinergias
				\4[] Muy difícil conocer cuantías concretas
			\3 Descuento de dividendos
				\4 Valoración de participaciones minoritarias
				\4[] Sin control sobre cuantía dividendo
			\3 Valor de la deuda
				\4 De mercado
				\4 En libros
				\4 Preferible mercado
				\4[] Salvo deudas muy reducidas
				\4[$\Rightarrow$] ¿Cuánto cuesta ahora comprar esa deuda?
			\3 Análisis Du-Pont
				\4 Contexto
				\4[] RoE
				\4[] $\to$ Rentabilidad del equity
				\4[] \% Beneficio neto por valor del patrimonio neto
				\4 Objetivo
				\4[] Desagregar componentes de RoE
				\4[] $\to$ Margen neto
				\4[] $\to$ Rotación del activo
				\4[] $\to$ Apalancamiento financiero
				\4 Formulación
				\4[] $\text{RoE} = \frac{\text{BN}}{\text{PN}} = \frac{\text{BN}}{\text{PN}} \cdot \frac{\text{Ventas}}{\text{Ventas}} \cdot \frac{\text{AT}}{\text{AT}} = \text{Mar.neto} \cdot \text{Rotación} \cdot \frac{\text{AT}}{\text{PN}} $
				\4[] Margen neto: $\frac{\text{BN}}{\text{Ventas}}$
				\4[] Rotación: $\frac{\text{Ventas}}{\text{Activos}}$
				\4[] Multiplicador del patrimonio neto: $\frac{\text{AT}}{\text{PN}}$
		\2 Valoración por múltiplos
			\3 Idea clave
				\4 Relación de precio con variable
				\4 Comparación con similares
				\4 Generalmente, variable es generación retornos
				\4 Cálculo múltiplos empresas similares
				\4[] Necesaria prudencia si no cotizadas
				\4 Varios periodos
				\4 Estimación precio adecuado
				\4 Prudencia con medias/medianas
			\3 \underline{Menú de múltiplos}
				\4 Basados en:
				\4[] $\to$ Valor de empresa
				\4[] $\to$ Valor del patrimonio neto
			\3 PER\footnote{Price Earnings Ratio, generalmente referido como P/E.}:
				\4[] $\frac{\text{Precio acción}}{\text{Beneficio neto por acción}}$
				\4[] ¿Cuánto tardan beneficios netos en igualar precio?
				\4[] $\to$ Si se mantuviesen como en el pasado?
				\4[] Comparación con empresas similares
				\4[] Si P/E mayor que empresas similares
				\4[] $\to$ Precio de mercado considerado alto
				\4[] $\then$ Preferible vender
				\4[] Si P/E menor que empresas similares
				\4[] $\to$ Tarda relativamente poco en generar rendimiento
			\3 Forward PER
				\4[] Concepto similar a P/E
				\4[] Utilización de beneficio neto futuro estimado
				\4[] $\to$ ¿Cómo estimarlo?
			\3 P/B o Precio-Valor-Contable
				\4 $\frac{\text{Precio}}{\text{Valor en libros}}$
				\4 Valor en libros:
				\4[] $\to$ Valor contable de activos menos pasivos
				\4[] $\to$ Descontada amortización
				\4 Necesario considerar:
				\4[] $\to$ Activos valorados a coste de adquisición
				\4[] $\to$ Cambios en criterios contables
				\4[] $\to$ Diferentes normas contables
			\3 Ratios beneficio operativo / valor de empresa
			\3 Múltiplos basados en otras medidas
				\4 Útiles si empresa no genera aún beneficios
				\4[] $\to$ Rotación de activos
				\4[] $\to$ Visitas a página web
				\4[] $\to$ Toneladas producidas
				\4[] $\to$ Volumen de ventas
				\4[] $\to$ Suscripciones a servicios
				\4[] $\to$ ...
			\3 \underline{Métodos basados en el goodwill}
				\4 Enorme variedad
				\4[] Fruto de la arbitrariedad
				\4[] Utilizadas por su inmediated
				\4[] $\to$ Necesaria prudencia
				\4 Concepto de goodwill
				\4[] Valor de empresa
				\4[] $\to$ Por encima de activos por separado
				\4[] $\then$ Muy difícil valoración
				\4 Múltiplos basados en goodwill
				\4[] Reglas arbitrarias aplicadas sobre
				\4[] $\to$ Valor en libros
				\4[] $\to$ Beneficios neto
			\3 Método clásico
				\4[] Valor en libros + coeficiente aplicado a ben. neto
				\4[] $V = A + (n\cdot B)$
				\4[] $A$: activo contable
				\4[] $B$: beneficio neto
			\3 Método indirecto o alemán
				\4[] Similar a clásico
				\4[] Capitalizar BN como si perpetuidad
				\4[] Suma ponderada al 50\% con valor en libros
				\4[] $V = \frac{A}{2} + \frac{B/i}{2}$
			\3 Método directo o anglosajón
				\4[] Valor en libros + ``superbeneficio''
				\4[] Superbeneficio es diferencia entre:
				\4[] $\to$ Beneficio de la empresa
				\4[] $\to$ Rendimiento obtenido si se invertiese a tasa $i$
				\4[] \quad capitalizado a interés de deuda + coeficiente
				\4[] $V = A + \frac{B-iA}{t_m}$
				\4[] $t_m$: tipo de deuda multiplicado por coef.
			\3 Valoración
				\4[] Necesario entender:
				\4[] $\to$ ¿Por qué el ratio se desvía?
				\4[] $\to$ ¿Por qué el ratio es el que es?
				\4 Atención a primas de control\footnote{Si el precio se extrae de una compra de la totalidad de la empresa, que en determinados sectores y momentos puede ser habitual, es necesario tener en consideración que el comprar habrá pagado una prima por el control de la empresa. O equivalentemente, tener en cuenta que si se utilizan los precios de mercado los compradores no estarán generalmente abonando primas de control, especialmente en empresas cotizadas.}
		\2 Suma de las partes
			\3 Concepto
				\4 Valoración por separado de activos
				\4 Aplicación diferentes métodos según activo
				\4[$\Rightarrow$] Suma de diferentes valoraciones
			\3 \underline{Valoración de activos}
			\3 Valor a coste histórico
				\4 Coste de adquisición o producción
				\4 Importe efectivo pagado por el activo
				\4 Importe efectivo de materias primas
			\3 Valor razonable
				\4 Importe obtenible por activo/pasivo
				\4 Sin deducir costes de transacción
				\4 No es valor razonable si:
				\4[] Resultado de
				\4[] $\to$ Transacciones forzadas
				\4[] $\to$ Urgentes
				\4[] $\to$ Liquidación involuntaria
				\4 Valor fiable de mercado utilizado para calcular
			\3 Valor neto realizable
				\4 Valor obtenible por enajenación
				\4 Deduciendo costes estimados necesarios
			\3 Valor actual
				\4 Importe de flujos a recibir
				\4 Actualizados a descuento adecuado
			\3 Valor en uso
				\4 Suma de flujos capitalizados
				\4 Flujos a obtener derivados de uso del activo
				\4 Supuestos explícitos y razonables sobre flujos
			\3 Costes de venta
				\4 Costes atribuibles a venta de activo
				\4 No se incurrirían
				\4[] Si no se toma decisión de vender
				\4 No incluye gastos financieros ni impuestos
				\4 Sí gastos legales para transferir propiedad
			\3 Valor según coste amortizado
				\4 Valor inicial de un activo financiero
				\4[--] Menos reembolsos del principal producidos
				\4 Equivale a VActual de flujos pendientes
				\4[] Descontados a:
				\4[] $\to$ Tipo de interés efectivo
				\4[] $\to$ TIR de adquisición
			\3 Costes de transacción de activos y pasivos financieros
				\4 Costes atribuibles a diferentes transacciones
				\4[] Compra
				\4[] Emisión
				\4[] Enajenación
				\4[] Asunción de pasivo financiero
				\4[] $\to$ No existiría sin transacción
				\4 Incluidos
				\4[] Honorarios y comisiones pagadas a agentes
				\4[] Impuestos y derechos sobre transacción
				\4 Excluidos
				\4[] Primas o descuentos por transacción
				\4[] Gastos financieros
				\4[] Costes de mantenimiento
				\4[] Costes administrativos internos
			\3 Valor contable o valor en libros
				\4 Importe neto de un activo o pasivo
				\4[] Por el que activo/pasivo registrado en balance
				\4 Descontando:
				\4[] Amortizaciones acumuladas
				\4[] Correcciones valorativas
			\3 Valor residual
				\4 Importe obtenible por venta
				\4 Deducidos costes de la venta
				\4 Considerando vida útil
				\4[] Periodo durante el que se espera utilizar activo
				\4[] Si activo puede devolverse/revertirse
				\4[] $\to$ Periodo que resta de la concesión
				\4 Vida económica
				\4[] Periodo en el que el activo será utilizable
			\3 Implicaciones
				\4 Útil si activos poco específicos
				\4 Útil si goodwill poco relevante
				\4 Ejemplo: empresa de alquiler de maquinaria
				\4[] Activos no específicos:
				\4[] $\to$ Maquinaria
				\4[] Activos específicos
				\4[] $\to$ Cartera de clientes
				\4[] Valoración por partes
				\4[] $\to$ Maquinaria muy fácil valoración
				\4[] $\to$ Incorporable fácilmente a otra empresa
				\4 Ejemplo: empresa de consultoría
				\4[] Activos no específicos
				\4[] $\to$ Ordenadores, material de oficina
				\4[] Activos específicos
				\4[] $\to$ Coordinación entre empleados
				\4[] $\to$ Know-how
				\4[] $\to$ Cartera de clientes
				\4[] $\to$ ...
		\2 Valoración por opciones
			\3 Deudores
				\4 Venden put a accionistas
			\3 Accionistas
				\4 Compran call a deudores
			\3 Paridad put call
				\4[] $P + S_0 = C + X\cdot e^{-rT}$
			\3 Valoración
				\4 Aplicación de métodos de valoración de acciones
				\4 Apenas utilizado en la práctica
				\4 Literatura extensa sobre el problema
		\2 Comparación de métodos
			\3 Ventajas descuento de flujos
				\4 Cuantifica supuestos implícitos sobre futuro
			\3 Desventajas descuento de flujos
				\4 Dificultad estimar FC futuros
				\4 Dependencia valor terminal
				\4 Problemas de información
			\3 Método de múltiplos
				\4 Relativamente simple
				\4 Tiene en cuenta sentimiento del mercado\footnote{Si por ejemplo, el método de los múltiplos arroja una valoración superior al método de descuentos de flujos, los directivos deberían valorar una salida a bolsa, porque los inversores valoran más positivamente la capacidad de generación de flujos que los propios managers.}
			\3 Ventajas suma de las partes
				\4 Posible mayor precisión
				\4 Útil activos con poca especificidad y goodwill
				\4 Valoración empresas a desmembrar
			\3 Desventajas
				\4 Difícil valorar sinergias

	\1 \marcar{Procesos de crecimiento externo}
		\2 Idea clave
			\3 Definición
				\4 Puesta en común de factores productivos
				\4 Diferentes entidades jurídicas
			\3 Objetivo
				\4 Realización de sinergias
				\4 Todo vale más que suma de las partes
				\4[] $\Rightarrow$ Creación de valor
				\4 $V_A + V_B < V_{A+B}$
				\4 Motivos de realización de sinergias
				\4[] Aumento de ingresos
				\4[] $\to$ Mayor poder de mercado
				\4[] $\to$ Comparten know-how
				\4[] $\to$ Aumento productividad
				\4 Reducción de costes
				\4[] $\to$ Realización de economías de escala
				\4[] $\to$ Eliminación de ineficiencias en gestión
				\4[] $\to$ Razones fiscales
				\4[] $\to$ Mayor independencia de proveedores
				\4[] $\to$ Integración de activos específicos en empresa
				\4[] $\to$ Diversificación para reducir riesgos
			\3 Diseño de operación
				\4 Elemento determinante de creación de valor
				\4[] ¿Cuánto obtiene cada parte?
				\4[] ¿Cómo financiar la operación?
				\4[] ¿Cuándo se lleva a cabo la operación?
				\4[] ¿Qué condiciones tras operación?
				\4 Actividad
				\4[] Operaciones verticales
				\4[] $\to$ Diferentes etapas del proceso productivo
				\4[] Operaciones horizontales
				\4[] $\to$ Mismos procesos productivos
				\4[] Conglomerados
				\4[] $\to$ Actividades diferentes
			\3 Factores
				\4 Macroeconómicos
				\4[] Fase del ciclo
				\4[] Entorno regulatorio
				\4 Dificultades crecimiento orgánico en Europa
				\4 Microeconómicos
				\4[] Competencia:
				\4[] $\to$ necesaria masa crítica para competir
				\4[] Economías de escala y alcance
				\4[] Poder de mercado
				\4[] Objetivos personales de managers
		\2 Fusiones
			\3 Idea clave
				\4 Adquisición a cambio de acciones
				\4[] Dos sociedades forman una sola
				\4[] Accionistas de una y otra
				\4[] $\to$ Pasan a serlo de las dos
				\4 Necesario valorar ambas antes de fusión
			\3 Relación de intercambio
				\4 Para tener una acción de la nueva empresa...
				\4[] ...¿cuántas tengo que entregar de A y B?
				\4[] \fbox{$\text{R. Intercambio A por B} = \frac{V_B}{V_A} \cdot \frac{N_A}{N_B}$}
				\4[] Acciones de A entregadas por cada acción de B
				\4[] $V_B, V_A$: valor del patrimonio neto de B y A
			\3 Modalidades
				\4 Sociedades mantienen personalidad separada
				\4[] Una tercera sociedad posee las dos originales
				\4[] Accionistas de originales
				\4[] $\to$ Pasan a tener acciones de nueva empresa
				\4[] Nueva empresa
				\4[] $\to$ Detenta capital de empresas originales
				\4 Una sociedad se integra en otra
				\4[] Intercambio de acciones de empresa
				\4[] $\to$ Por acciones de empresa absorbente
				\4 Creación de nueva sociedad
				\4[] Intercambio de acciones de empresas originales
				\4[] $\to$ Por acciones de nueva empresa
				\4[] Desaparición de personalidad jurídica de empresas originales
		\2 Adquisiciones
			\3 Idea clave
				\4 Adquisición a cambio de dinero
				\4 Adquirido
				\4[] Sale del capital de empresa adquirida
				\4[] Recibe inmediatamente sinergias
				\4[] A diferencia de las fusiones
				\4[] $\to$ Partes arriesgan no realización de sinergias
				\4[] En adquisiciones
				\4[] $\to$ Adquirido recibe tantas como haya negociado
				\4 Adquiriente
				\4[] Único dueño del capital de ambas empresas
			\3 Estrategias defensivas
				\4 Una de las partes no desea participar
				\4[] Directivos
				\4[] Grupos de accionistas
				\4[] $\to$ Posible implementación de obstáculos
				\4 Acciones especiales con más votos
				\4[] Menos derechos de cobro de dividendos
				\4[] Más peso en junta de accionistas
				\4[] $\to$ Mantener control aunque adquieran resto de acciones
				\4 Entrada en accionariado sujeta a aprobación
				\4 Aumentos de capital restringidos
				\4[] Sólo socios pueden acceder
				\4[] $\to$ Aumenta control por accionistas originales
				\4[] $\then$ Dificulta toma de control
				\4 Recompras de acciones
				\4[] Compra de acciones a minoritarios
				\4[] $\to$ Aumento de control por originales
				\4 Obligación de notificar a gestores
				\4[] $\to$ Cuando un accionista supera un umbral
				\4 Contra-OPAs acordadas con un tercero
				\4[] Sólo permite evitar compra por cierto agente
				\4 Empresa objetivo compra acciones de adquiriente potencial
				\4 Emisión de warrants ``pildora venenosa''
				\4[] Adquiriente se ve obligado a negociar precio más alto
				\4 Acciones legales
			\3 Modalidades
				\4 Adquisición sin endeudamiento
				\4[] Directamente con fondos propios de adquiriente
				\4 LBO -- Leveraged Buy-Out
				\4[] Adquisición financiada por apalancamiento
				\4[] Management LBO
				\4[] $\to$ Los propios managers llevan a cabo LBO
				\4[] Equivale a aumento de apalancamiento total
				\4[] $\to$ Sustitución de equity por deuda
				\4[] $\to$ Con las dos empresas consideradas como un todo
				\4 Adquisiciones legales obligatorias
				\4[] Legislación obliga a OPAs
				\4[] $\to$ Toma de control de la sociedad
				\4[] $\to$ O exclusión de la cotización en bolsa
				\4[] $\to$ O reducción de capital
				\4[] Obligatorio presentar por el 100\%
		\2 Alianzas estratégicas
			\3 Idea clave
				\4 Puesta en común de recursos
				\4[] Físicos
				\4[] Capital financiero
				\4[] Know-how
				\4[] Trabajadores
				\4 Sin transferencia de acciones
				\4 Con cuerpo normativo propio
			\3 Modalidades
				\4 Personalidad jurídica
				\4[] Propia
				\4[] Sin personalidad propia
				\4[] $\to$ P.ej. agrupaciones de PYMES
				\4 Grupos de sociedades
				\4[] Sociedades separadas
				\4[] Con misma dirección
				\4[] Suelen estar unidas a accionariado común
				\4 Uniones temporales de empresa
				\4[] Sin personalidad propia
				\4[] Horizonte temporal limitado
				\4[] Realización de un objetivo concreto
				\4 Otras modalidades
				\4[] Franquicias
				\4[] Joint-ventures
				\4[] Sociedades de garantía recíproca
		\2 Separaciones
			\3 Idea clave
				\4 Separación masa de activos
				\4[] $\to$ Constitución de nueva sociedad
				\4 Problema similar a fusión y adquisición
				\4[] Pero a la inversa
				\4[] $\to$ ¿Crea valor separación?
				\4[] $\to$ $V_A + V_B > V_{A+B}$
			\3 Justificación
				\4 Excesiva diversificación
				\4[] Empresa no es atractiva para inversión
				\4 Conglomerados ineficientes
				\4[] Empresas con beneficios
				\4[] $\to$ Apoyan empresas ineficientes con pérdidas
				\4[] $\Rightarrow$ Producción subóptima
	\1[] \marcar{Conclusión}
		\2 Recapitulación
			\3 Análisis del crecimiento de las empresas
				\4 Cómo y por qué se produce
			\3 Procesos de crecimiento externo
				\4 Fusiones
				\4 Adquisiciones
				\4 Alianzas estratégicas
			\3 Valoración de empresas
				\4 Elemento fundamental crecimiento externo
		\2 Idea final
			\3 Complejidad crecimiento
				\4 Crecimiento externo: mercado en sí mismo
			\3 Desigual valoración del crecimiento
				\4 Empresas pequeñas: positivamente
				\4 Empresas grandes: negativamente
				\4 Evidencia empírica contradice\footnote{Las empresas grandes tienden a ser más productivas, generar más empleo y más estable. Además, algunos autores han señalado que la capacidad de las grandes firmas para diversificar sus mercados puede contribuir a su contestabilidad.}
			\3 Consecuencias bienestar
				\4 Crecimiento puede mejorar eficiencia
				\4 Genera empleo
				\4 Peligros crecimiento ineficiente
\end{esquemal}



























\conceptos

\concepto{Estrategias defensivas}

\concepto{Flujo de caja libre}

\concepto{Legislación europea sobre Ofertas Públicas de Adquisición} La popularidad creciente de ofertas públicas de adquisición internacionales dentro de la Unión Europea ha llevado a ésta a adoptar una directiva que armonice las legislaciones nacionales en la materia. La directiva señala una serie de principios básicos tales como igualdad de trato entre accionistas, obligación de dar a los accionistas tiempo suficiente para valorar la oferta, obligación de la compañía objetivo de actuar en beneficio de la compañía, prohibición de manipulación del precio de la acción, obligación de fundamentar suficientemente la financiación de la operación y respeto al funcionamiento normal de la empresa objetivo a pesar de la oferta.

Además de estos principios básicos, la directiva regula una serie de áreas tales como el principio de \textit{oferta obligatoria}, las \textit{estrategias defensivas}, las \textit{ofertas públicas de exclusión obligatorias} o la legislación sobre adquisiciones.

\concepto{Leveraged Buyout}

\concepto{Tamaño óptimo} ¿Existe un tamaño óptimo de la empresa? En el marco del modelo neoclásico, simplificado hasta el punto de que la única variable relevante es la producción, sí es posible que exista una cuantía que maximice los beneficios. En la medida en que la firma actúe racionalmente, producirá esa cantidad óptima y por tanto alcanzará el tamaño óptimo. Sin embargo, ni la producción es la única medida de \comillas{tamaño} de la empresa, ni la función que se maximiza en la práctica es la de beneficio. La función de utilidad a maximizar varía entre grupos, y por tanto el concepto de tamaño óptimo tiene poca utilidad fuera del marco puramente neoclásico. La variable a maximizar puede ser el tamaño del activo para los managers, el número de empleados para los sindicatos, la cifra de ventas para managers que busquen aumentar poder y prestigio, etc...

\concepto{Modelo de dividendos de Walter}: según este modelo, la política de dividendos de una empresa puede contribuir a la creación de valor. Si la rentabilidad de los proyectos de inversión en el seno de la empresa es superior a la rentabilidad exigida por el mercado a proyectos de similar riesgo, la política de dividendos óptima será aquella que los reduzca al máximo. Por el contrario, si los proyectos en el seno de la empresa ofrecen un retorno inferior al exigido por el mercado para proyectos de similar riesgo, el dividendo óptimo será el máximo posible. Así, los inversores podrán utilizar los fondos recibidos en proyectos que creen valor. Cuando las empresas aplican estas políticas de dividendos óptimas, su valor aumenta, dado que se crea efectivamente valor para los inversores.

\concepto{Métodos de valoración basados en el \textit{goodwill}}: esta familia de métodos de valoración consisten en la aplicación de una serie de reglas heurísticas al valor en libros o al valor en libros ajustado y al resultado del ejercicio. Así, la determinación del valor que arrojan estos métodos depende de la valoración \comillas{estática} de los activos de la empresa, y de una cuantificación del valor que la empresa generará en el futuro. Estos métodos de valoración adolecen de un grave problema: la arbitrariedad prácticamente absoluta a la hora de decidir qué fórmula aplicar a los datos con los que se cuenta. Sin embargo, han sido ampliamente utilizados en el pasado y aún lo son en el presente, por lo que resulta necesario conocerlos en líneas generales. En cualquier caso, y dada su arbitrariedad, es necesario tener en cuenta que son conceptualmente erróneos. Pablo Fernández afirma: <<[..] However, if he continues to read this section, he should not look for much \comillas{science} or much common sense at the methods that follow because they are very arbitrary.>>

Los ejemplos concretos más habituales de este clase de métodos son los siguientes:

\begin{description}
	\item[Clásico] Consiste simplemente en sumar al valor en libros (o a su valor ajustado) el beneficio neto multiplicado por un coeficiente determinado: $V = A + (n \cdot B)$ (siendo A el valor el libros y B el beneficio neto).
	
	\item[Indirecto o alemán] Similar al método clásico. Este método capitaliza el beneficio neto como si de una perpetuidad al tipo de interés de la deuda a largo plazo se tratase, y pondera los dos componentes del valor en igual medida: $V = \frac{A}{2} + \frac{B/i}{2}$.
	
	\item[Anglosajón o directo] La fórmula es la siguiente: $V = A + \frac{B- iA}{t_m}$. Se basa en sumar al valor en libros el \comillas{superbeneficio}, que no es sino la diferencia entre el beneficio de la empresa y lo que se hubiese obtenido de haber invertido el activo a interés $i$, y capitalizando a su vez esta diferencia al tipo de interés de la deuda multiplicado por un coeficiente entre 1,25 y 1,5 (lo que corresponde a $t_m$).
\end{description}

\concepto{Ofertas públicas de suscripción, venta y adquisición}

Las ofertas públicas son procesos públicos mediante los cuales se transmite la propiedad de un activo financiero. La información relativa a las ofertas públicas se recoge en los llamados folletos de emisión. Una vez puesta en marcha la oferta, los inversores expresan su interés. En la fase de finalización, se reparten los activos a los inversores que han aceptado las consecuencias, aplicando reglas de reparto en caso de que la demanda supere la oferta.
 
Las \textit{ofertas públicas de venta} consisten en el ofrecimiento al público de un activo financiero ya existente, tal como acciones o bonos de una empresa. Las \textit{ofertas públicas de suscripción} consisten en el ofrecimiento al público de un activo financiero que se emite expresamente en el momento de la oferta. Por ejemplo, una oferta pública de suscripción de acciones es una ampliación de capital abierta al público. 

\preguntas

\seccion{Test 2014}
\textbf{27.} El método conocido como Price Earning Ratio (PER) supone que:

\begin{enumerate}
	\item[a] El valor teórico de las acciones es superior al contable.
	\item[b] El valor de las acciones de una empresa viene reflejado por la capitalización que hace el mercado de los beneficios de la misma.
	\item[c] El valor de las acciones ordinarias está relacionado con los ingresos de ventas de las acciones.
	\item[d] El valor de las acciones de una empresa no está relacionado con la capitalización que hace el mercado de los beneficios de la misma.
\end{enumerate}

\seccion{Test 2013}
\textbf{27.} Una empresa que tiene un activo de 50 millones de euros ha obtenido el pasado año un beneficio de explotación de 5 millones de euros. Suponiendo un tipo de interés del mercado del 7\% anual y que la empresa ha repartido un 40\% del beneficio en forma de dividendo el valor de la empresa será, según el modelo de Gordon:
\begin{enumerate}
	\item[a] 10.000.000 euros.
	\item[b] 34.480.000 euros.
	\item[c] 200.000.000 euros.
	\item[d] Ninguno de los anteriores.
\end{enumerate}

\seccion{Test 2009}
\textbf{24.} Según la tesis del dividendo de Walter:

\begin{enumerate}
	\item[a] Es la capacidad de los activos de una empresa de generar renta y su clase de riesgo lo que determina el valor de sus acciones y no la política de reparto de dividendos.
	\item[b] Cuando el rendimiento de las inversiones en el seno de una empresa es mayor al rendimiento que ofrece el mercado, la empresa debe retener todos los beneficios y viceversa. Con ello consigue el máximo valor de mercado de sus acciones.
	\item[c] Si los dividendos están gravados por un impuesto sobre la renta, éstos deben acumularse en la empresa para evitar así el pago de este impuesto.
	\item[d] Ninguna de las anteriores.
\end{enumerate}

\seccion{Test 2007}
\textbf{24.} Una de las proposiciones originales de Modigliani-Miller (MM) es:

\begin{enumerate}
	\item[a] Supone que un incremento del endeudamiento no afecta al tipo de interés de la deuda de la empresa.
	\item[b] Supone que si la empresa no está endeudada la rentabilidad esperada de sus acciones es inferior a la rentabilidad esperada de sus activos.
	\item[c] Implica que el valor de la empresa dependerá únicamente de la rentabilidad de sus proyectos de inversión.
	\item[d] Establece que la rentabilidad esperada para los accionistas es independiente de las decisiones de financiación que tome la empresa.
\end{enumerate}

\seccion{Test 2005}
\textbf{25.} En lo que se refiere a métodos de valoración de acciones de empresas:

\begin{enumerate}
	\item[a] Los que se basan en el descuento de flujos son conceptualmente más correctos, aunque eliminan parcialmente el concepto de rentabilidad al apoyarse más en criterios de tesorería que de resultados.
	\item[b] El método indirecto, alemán o de los prácticos se basa en el concepto de fondo de comercio, al contrario que el método directo o anglosajón.
	\item[c] Los métodos de valoración basados en los beneficios o en los dividendos tienen la ventaja de que no son manipulables por decisiones discrecionales de la dirección de una empresa.
	\item[d] El PER (price earning ratio) es un método sencillo y ampliamente utilizado tanto para empresas que cotizan en bolsa como para las que no cotizan.
\end{enumerate}

\notas

\textbf{2014}: \textbf{27.} B

\textbf{2013}: \textbf{27.} ANULADA

\textbf{2009}: \textbf{24.} B

\textbf{2007}: \textbf{24.} C

\textbf{2005}: \textbf{25.} A


Ángel dice que cuando trabajaba en M\&A en Deloitte se utilizaba constantemente el método de valuación por \textit{goodwill}. Sin embargo, ni Damodaran ni Myers ni Vernimmen tratan el concepto de goodwill más allá de la diferencia entre el valor de la empresa adquirida y el precio pagado. Es posible que Ángel confunda el hecho de que para el comprador (y para el vendedor, en la medida en que debe intentar convencer al comprador de que el goodwill será grande) es vital valorar el goodwill porque captura las sinergias resultantes.

En el tema 36 de Vernimmen aparece el \comillas{internal growth model} que aparecía en los temas de CECO Viejo y de Juan. En el esquema se ha eliminado este modelo. El objetivo del modelo es mostrar la derivación del llamado \textbf{potencial de crecimiento}, una tasa de crecimiento de capital que mantiene constante el crecimiento de los fondos propios, de la deuda, el beneficio neto, el valor en libros por acción y el dividendo por acción.

\bibliografia

Para el primer apartado del tema (análisis del crecimiento de la empresa), se ha utilizado el survey de Alex Coad. Los temas de CECO y Miguel Tiana parecen también estar basados en este documento, parcialmente.

Vernimmen - caps. 31 (valuation techniques), 44 (taking control of a company), 45 (mergers and demergers). En general, Vernimmen para estos temas.

Mirar en Palgrave (Money and Finance):
\begin{itemize}
	\item firm level employment dynamics
	\item Gibrat's law
	\item growth and learning-by-doing
    \item leverage
    \item leveraged buy-out
    \item lognormal distribution
    \item Penrose, Edith Rose
    \item price earnings ratio
\end{itemize}

Monografías de Pablo Fernández en SSRN.

Evans, D. S. (1986) \textit{Tests of alternative theories of firm growth} Economic Research Reports. C. V. Starr Center for Applied Economics -- En carpeta del tema

Fernández, P. \textit{Valuing Companies by Cash Flow Discounting: 10 Methods and 9 Theories } \url{https://papers.ssrn.com/sol3/papers.cfm?abstract\_id=2684776}

Geroski, P. A. \textit{The Growth of Firms in theory and practice} (1999) -- En carpeta del tema

Lucas, R. E. Lucas (1978) \textit{The Bell Journal of Economics} The Bell Journal of Economics -- En carpeta del tema

Penrose, E. \textit{The Theory of the Growth of The Firm} (2009) 4th Edition -- En carpeta del tema


\end{document}
