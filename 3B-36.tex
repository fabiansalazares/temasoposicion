\documentclass{nuevotema}

\tema{3B-36}
\titulo{Procesos de integración no comunitarios. Organismos de cooperación internacional: la OCDE y otros organismos de cooperación.}

\begin{document}

\ideaclave

\seccion{Preguntas clave}

\begin{itemize}
	\item ¿En consiste la integración?
	\item ¿Por qué tienen lugar los procesos de integración?
	\item ¿Qué modalidades de integración?
	\item ¿A quién perjudica y a quién benefician los procesos de integración?
	\item ¿Qué procesos de integración existen actualmente al margen de la Unión Europea?
	\item ¿Qué son los organismos de cooperación internacional?
	\item ¿En qué consiste y para qué sirve la OCDE?
	\item ¿Qué otros organismos de cooperación existen?
\end{itemize}

\esquemacorto

\begin{esquema}[enumerate]
	\1[] \marcar{Introducción}
		\2 Contextualización
			\3 Concepto de integración económica
			\3 Procesos recientes de integración económica
			\3 Organismos de cooperación
		\2 Objeto
			\3 ¿Qué modalidades de integración existen?
			\3 ¿Qué ventajas tiene la integración frente al multilateralismo?
			\3 ¿A quién perjudican y a quién benefician?
			\3 A parte de la UE, ¿qué procesos de integración existen en la actualidad?
			\3 ¿Qué son los organismos de cooperación internacional?
			\3 ¿Para qué sirven los organismos de cooperación internacional?
			\3 ¿Cuáles son los principales organismos de cooperación internacional?
			\3 ¿Cómo se estructuran internamente?
		\2 Estructura
			\3 Formas de integración
			\3 Procesos de integración no comunitarios
			\3 Organismos de cooperación internacional
	\1 \marcar{Formas de integración}
		\2 Idea clave
			\3 Integración vs multilateralismo
			\3 Ventajas de la integración
			\3 Inconvenientes
			\3 Impacto de la integración
			\3 Integración profunda
		\2 Grados de de integración
			\3 Tratado de preferencia comercial
			\3 Acuerdo de libre comercio
			\3 Unión aduanera
			\3 Mercado común
			\3 Mercado único
			\3 Unión económica
			\3 Unión económica y monetaria
			\3 Unión política y económica
	\1 \marcar{Procesos de integración no comunitarios}
		\2 Acuerdos intercontinentales/mega-acuerdos
			\3 TTIP -- Transatlantic Trade and Investment Partnership
			\3 TPP -- Trans-Pacific Parnership
			\3 CPTPP/TPP11
			\3 RCEP
		\2 Europa
			\3 EFTA
			\3 EEE
			\3 CEFTA
			\3 Eurasian Economic Union
			\3 CISFTA
		\2 América
			\3 MERCOSUR
			\3 NAFTA y USMCA
			\3 USMCA
			\3 ALCA
			\3 Alianza del Pacífico
			\3 ALADI
			\3 Comunidad Andina de Naciones
			\3 Alternativa Bolivariana para América
			\3 UE-Centroamérica
			\3 MCCA
			\3 DR-CAFTA
		\2 Asia
			\3 CPTPP/TPP11
			\3 RCEP
			\3 ASEAN
			\3 APEC
			\3 SAFTA
			\3 BIMSTEC
			\3 PACER y PICTA
		\2 MENA
			\3 Idea clave
			\3 GCC -- Consejo de Cooperación del Golfo
			\3 Acuerdo de Agadir
			\3 GAFTA
			\3 Unión del Magreb Árabe
		\2 África
			\3 Idea clave
			\3 Unión Africana
			\3 Comunidad Ecónomica Africana
			\3 ECOWAS
			\3 UEMOA
			\3 UMOA
			\3 WAMZ
			\3 CEEAC
			\3 CEMAC
			\3 SADC
			\3 SACU
			\3 EAC -- East African Community
			\3 IGAD
			\3 COMESA
	\1 \marcar{Organismos de cooperación internacional}
		\2 OCDE
			\3 Función
			\3 Antecedentes
			\3 Organización
			\3 Actuaciones
			\3 Valoración
		\2 UNCTAD -- United Nations Conference on Trade and Development
			\3 Funciones
			\3 Organización
			\3 Actuaciones
			\3 Valoración
		\2 Otros organismos
			\3 Instituciones de Bretton Woods
			\3 G-20
			\3 G-8 y G-7
			\3 3G -- Global Governance Group
			\3 FSB -- Financial Stability Board
			\3 CIS
			\3 OPEP -- Organización de Países Exportadores de Petróleo
	\1[] \marcar{Conclusión}
		\2 Recapitulación
			\3 Formas de integración
			\3 Procesos de integración no comunitaria
			\3 Organismos de cooperación internacional
		\2 Idea final
			\3 Cadenas de valor internacionales
			\3 Acuerdos de nueva generación
			\3 Nuevo bilateralismo
			\3 Brexit
			\3 Papel de la UE

\end{esquema}

\esquemalargo













\begin{esquemal}
	\1[] \marcar{Introducción}
		\2 Contextualización
			\3 Concepto de integración económica
				\4 Economías/países/jurisdicciones acuerdan
				\4[] Reducir barreras a movimiento:
				\4[] $\to$ Bienes y servicios
				\4[] $\to$ Trabajo y capital
				\4 Intensidad variable de la integración
				\4[] ¿Qué movimiento se permite?
				\4[] ¿Hasta qué punto se eliminan las barreras?
				\4[] ¿Qué barreras se eliminan?
			\3 Procesos recientes de integración económica
				\4 Constantes a lo largo de la historia
				\4[] Nuevos procesos de integración
				\4[] Fracaso de procesos
				\4[] $\to$ Desintegración y aparición de barreras
				\4 Imperios
				\4[] Desde el punto de vista económico
				\4[] $\to$ Ejemplos de integración económica
				\4[] Zollverein alemán
				\4 Entre-guerras
				\4[] Imperio británico
				\4[] Alemania nazi
				\4[] Neocolonialismo
				\4 Tras segunda guerra mundial
				\4[] Papel clave de GATT y OMC
				\4[] $\to$ Prohíbe tratamiento discriminatorio
				\4[] $\to$ Permite reducciones generalizadas
				\4[] Integración europea
				\4[] $\to$ CECA $\to$ CEE $\to$ UEM
				\4[] NAFTA
				\4[] MERCOSUR
				\4[] $\to$ BRA, ARG, PAR, URU
				\4[] ASEAN
				\4[] ...
			\3 Organismos de cooperación
				\4 Catalizadores de más y mejor integración
				\4[] Foros de intercambio de ideas, problemas
				\4[] Marcos donde acordar reglas de integración
				\4 Aumento de organismos a partir de post-II GM
				\4[] En sistema previo, bilateralismo prevalece
				\4 Multilateralismo y plurilateralismo
				\4[] Eliminación de barreras entre grupos de países
				\4[] $\to$ Basándose en reglas definidas
				\4[] $\to$ Evitando acuerdos ad-hoc bilaterales
				\4[$\then$] Organismos de coop. internacional
				\4[] Instrumentos centrales de sist. económico mundial
				\4[] $\to$ Foros de negociación
				\4[] $\to$ Transferencia de información y coordinación
				\4[] $\to$ Solución de diferencias
				\4[] $\to$ Fomento de integración económica
				\4[] $\to$ Aprobación y reforma de reglas
		\2 Objeto
			\3 ¿Qué modalidades de integración existen?
			\3 ¿Qué ventajas tiene la integración frente al multilateralismo?
			\3 ¿A quién perjudican y a quién benefician?
			\3 A parte de la UE, ¿qué procesos de integración existen en la actualidad?
			\3 ¿Qué son los organismos de cooperación internacional?
			\3 ¿Para qué sirven los organismos de cooperación internacional?
			\3 ¿Cuáles son los principales organismos de cooperación internacional?
			\3 ¿Cómo se estructuran internamente?
		\2 Estructura
			\3 Formas de integración
			\3 Procesos de integración no comunitarios
			\3 Organismos de cooperación internacional
	\1 \marcar{Formas de integración}
		\2 Idea clave
			\3 Integración vs multilateralismo
				\4 Debate en 90s y 2000s
				\4[] Pre-GATT
				\4[] Acuerdos bilaterales
				\4[] $\to$ Obstáculo a liberalización
				\4[] ``Building blocks or stumbling blocks''
				\4[] $\to$ Baldwin (2007)
				\4 Actualmente
				\4[] Integración regional única forma de avanzar
				\4[] Ronda de Doha paralizada
				\4[] $\to$ Debate sobre ventajas de integración
				\4 Multilateralismo
				\4[] Liberalización no discriminatoria
				\4[] Cláusula NMF es elemento central
				\4[] $\to$ Reduce problemas de coordinación
				\4[] $\to$ Crea problema de free-riding
				\4[] Ventajas de cara a opinión pública
				\4[] $\to$ Negociación más transparente
				\4[] $\to$ Competidores menos identificables
				\4 Regionalismo
				\4[] Integración entre grupos de países
				\4[] ¿Por qué tiene lugar?
				\4[] Teoría del dominó
				\4[] $\to$ Integración crea ventajas para más integración
				\4[] Teorías de modelos de gravedad
				\4[] $\to$ Describir ventajas de integración
				\4[] Otras razones
				\4[] $\to$ Pragmática: paralización de WTO
				\4[] $\to$ Acuerdos más fáciles con menos miembros
				\4 Actualidad
				\4[] Más de 300 PTAs notificados en WTO
				\4[] Media de 13 acuerdos por miembro WTO
				\4[] PTA no regionales
				\4[] $\to$ Miembros de muy distintos países
				\4[] Mayoría de PTA son sur-sur
				\4[] Hasta 80s, mayoría de norte-sur
			\3 Ventajas de la integración
				\4 Reducción efectiva de aranceles
				\4 Creación de comercio entre miembros
				\4 Permite acuerdos en áreas más allá de WTO
				\4[] A menudo int. profunda no discriminatoria
			\3 Inconvenientes
				\4 Desviación de comercio
				\4 Pérdida de interés en multilateralismo
				\4 Aumento de poder de mercado de bloques
			\3 Impacto de la integración
				\4 Menor que indica número de PTAs
				\4[] $\uparrow$ de PTAs no ha implicado = $\uparrow$ comercio preferencial
				\4 Impacto de preferencias
				\4[] Muy pequeño
				\4[] Cuando países aplican arancel alto en NMF
				\4[] $\to$ También lo aplican en PTAs
				\4[] Liberalización vía PTAs
				\4[] $\to$ Poco impacto real sobre aranceles
				\4 Integración profunda
				\4[] Verdadero aspecto importante de integración
			\3 Integración profunda
				\4 Eliminar aranceles ya no sirve
				\4[] Necesaria integración en más áreas
				\4 Cadenas de valor
				\4[] Han sustituido a prod. en un país
				\4 Áreas  de integración profunda
				\4[] Propiedad intelectual
				\4[] Competición
				\4[] TBT
				\4[] SPS
				\4[] Contratación pública
				\4[] Servicios
				\4[] Estándares laborales
				\4[] Medioambiente
				\4[] Protección de inversiones
				\4 Aumenta dificultad de negociación
				\4[] Más margen para desacuerdo
				\4[] Más contrapartidas posibles
				\4[] PEDs no tienen experiencia alguna
				\4[] $\to$ P.ej: política de competencia
		\2 Grados de de integración
			\3 Tratado de preferencia comercial\footnote{\textit{Preferential trade agreement}.}
				\4 Término con doble significado
				\4[] $\to$ Acuerdos de libre comercio y uniones aduaneras
				\4[] $\to$ Preferencias a países en desarrollo
				\4 Instrumento de ayuda a PEDs
				\4 Necesario reportar a WTO
				\4 Evolución histórica
				\4[] UNCTAD presiona en 60s
				\4[] $\to$ Influencia en Ronda Kennedy
				\4[] En 1971, se aprueban waivers temporales a NMF
				\4[] ``Enabling clause'' de Ronda de Tokio (1979)
				\4[] $\to$ Permite excepción permanente NMF
				\4 Requisitos de preferencias a PEDS
				\4[] $\to$ Generalizado
				\4[] $\to$ No discriminatorio\footnote{Evitar que países desarrollados implementen preferencias para países ``amigos'' o aliados con los que mantienen algún tipo de relación de dependencia o de vínculo histórico, comercial o militar.}
				\4[] $\to$ No recíproco
			\3 Acuerdo de libre comercio\footnote{\textit{Free trade agreement}.}
				\4 Características básicas
				\4[] Eliminación de barreras entre partes
				\4[] $\to$ Aranceles
				\4[] $\to$ Posibles otras barreras
				\4 Artículo XXIV del GATT
				\4[] Fundamento jurídico de ALCs
				\4[] Denominados FTAs y RTAs
				\4 Artículo V del GATS
				\4[] Denominados ``Economic Integration Agreement''
				\4 Enorme proliferación en últimas décadas
				\4[] Más de 450 acuerdos notificados
				\4[] Solapamiento de acuerdos
				\4[] Aumento de complejidad jurídica
				\4[] Debilita sistema multilateral
				\4[] $\to$ ``\textit{spaghetti bowl}''
				\4 Mantenimiento de barreras con terceros
				\4[] Independientes de ALC
			\3 Unión aduanera\footnote{\textit{Customs union}.}
				\4[] Eliminación de aranceles entre partes
				\4[] Introducción de arancel externo común
				\4[] En términos globales, no superior a previo
				\4[] Para algunos miembros
				\4[] $\to$ Puede ser superior a previo
				\4[] $\then$ Cálculo de impacto global de nuevo arancel
			\3 Mercado común
				\4 Ejemplo:
				\4[] Mercosur
				\4 Primera fase hacia mercado único
				\4 Liberalización de bienes
				\4[] + reducción barreras a servicios
				\4[] + reducción barreras a ff.pp.
			\3 Mercado único
				\4 Etapa final de mercado común
				\4 Características
				\4[] Movimientos de ByS y ff.pp.
				\4[] $\to$ Circulan igual que dentro de un país
				\4 Profundización de mercado común
				\4[] No sólo libertad de movimiento de iure
				\4[] $\to$ También de facto
				\4[] $\then$ Armonización administrativa
				\4[] $\then$ Eliminación de otras barreras
			\3 Unión económica
				\4 Características
				\4[] Política económica armonizada
				\4[] Armonización fiscal
				\4[] Coordinación fiscal
				\4[] Coordinación de políticas monetarias
				\4[] $\to$ Ciclos económicos asíncronos
				\4[] $\to$ Shocks asimétricos
			\3 Unión económica y monetaria
				\4 Autoridad monetaria única
				\4 Tipo de cambio fijado irreversiblemente
				\4[] $\to$ Moneda única
				\4 Política monetaria común
			\3 Unión política y económica
				\4 Último estadio de integración
				\4 Características
				\4[] Soberanía compartida
				\4[] Libre circulación de ByS y ff.pp.
				\4[] Política económica armonizada
				\4[] Política monetaria común
				\4[] Soberanía compartida con todos los miembros
	\1 \marcar{Procesos de integración no comunitarios}
		\2 Acuerdos intercontinentales/mega-acuerdos
			\3 TTIP -- Transatlantic Trade and Investment Partnership
				\4 Suspendido tras Trump
			\3 TPP -- Trans-Pacific Parnership
				\4 Reducir dependencia de China en Pacífico
				\4 Aumentar influencia de EEUU
				\4 Firmado en 2016
				\4 Sin efecto por no ratificación
				\4[] Sólo Japón y NZ ratifican
				\4[] Trump 2016 declara voluntad de no ratificar
			\3 CPTPP/TPP11
				\4 11 miembros
				\4[] AUS, BRU, CAN, CHI, JAP
				\4[] MAL, MEX, NZ, PER, SIN
				\4[] VIE
				\4 Firmado en 2018
				\4 Efectivo desde 2018
				\4 Ratificados a enero de 2020
				\4[] 7 de 11
				\4[] Faltan:
				\4[] $\to$ Brunei
				\4[] $\to$ Chile
				\4[] $\to$ Malasia
				\4[] $\to$ Peru
				\4 Sucesor de TPP sin Estados Unidos
				\4[] Idéntico en 2/3 a TPP original
				\4[] $\to$ En momento de abandono de EEUU
				\4 Elimina intereses ofensivos de EEUU
				\4[] Eliminación de protección añadida a copyright
				\4[] Reducción posibilidad de empresas que demandan gobiernos
				\4 En general, más avanzado que RCEP
				\4 Miembros
				\4[] BRU, MAL, SIN, VIE, AUS, CAN, JAP,
				\4[] MEX, NZ, PER
				\4 Estándares mínimos de protección MAmbiente
				\4 Mecanismos de resolución de disputas
				\4 Obligación de informar sobre empresas públicas
				\4[] Controlar ayudas vía empresas públicas
				\4 Comisión del CPTPP
				\4[] Órgano de decisión
				\4[] Creado en 2018
				\4[] Dos encuentros en 2019
				\4 Posibles miembros futuros
				\4[] Estados Unidos
				\4[] Taiwan
				\4[] Reino Unido
				\4[] Colombia
				\4[] Indonesia
				\4[] Corea del Sur
				\4[] Tailandia
			\3 RCEP\footnote{\url{https://www.csis.org/analysis/last-rcep-deal}}
				\4 Propuesto inicialmente por China en 2011
				\4[] Más inclusivo que TPP en términos de miembros
				\4 Prevista firma en 2020
				\4 Potencial mayor área de comercio mundial
				\4 15 miembros tras retirada de India
				\4[] ASEAN (10)
				\4[] China
				\4[] Japón
				\4[] Corea del Sur
				\4[] Australia
				\4[] Nueva Zelanda
				\4 Más probable tras retirada de TPP
				\4[] Y aprobación posterior de CPTPP/TPP11
				\4 Interés de China
				\4[] Reducir influencia americana
				\4 Regulación de:
				\4[] Agricultura
				\4[] Comercio industrial
				\4[] Comercio electrónico
				\4[] Datos digitales
				\4[] Medioambiente
				\4[] Propiedad intelectual
				\4 Posible MSDisputas en inversiones
				\4 No prohíbe aranceles en contenidos digitales
				\4[] A diferencia de CPTPP y USMCA
				\4[] $\to$ Interés ofensivo habitual de EEUU
				\4 Reducción de aranceles
				\4 Acceso a mercado incrementado
				\4 Menor profundidad que CPTPP y USMCA
				\4 Reglas de origen comunes para todo el bloque
				\4[] Un sólo certificado de origen común
		\2 Europa
			\3 EFTA
				\4 European Free Trade Agreement
				\4 Creado en 1960
				\4[] Alternativa a CEE del Tratado de Roma de 1957
				\4[] Miembros originales:
				\4[] $\to$ AUS, DIN, NOR, POR, RU, SWE, SWI
				\4[] Posteriormente:
				\4[] $\to$ FIN, ISL, LIE
				\4 Miembros actuales
				\4[] ISL, LIE, NOR, SWI
				\4 Espacio Económico Europeo
				\4[] FTA entre EFTA (salvo Suiza) y UE
				\4 Actualidad
				\4[] Pérdida de importancia en favor de UE
				\4[] Países poco abiertos a aceptar nuevos miembros
				\4[] $\to$ RU se plantea ingreso tras Brexit
			\3 EEE
				\4 Espacio Económico Europeo
			\3 CEFTA
				\4 Central European Free Trade Agreement
				\4 Entrada en vigor en 1994
				\4[] Tras disolución de COMECON
				\4[] Inicialmente:
				\4[] $\to$ Grupo de Visegrado
				\4[] Acceden posteriormente:
				\4[] $\to$ SLV, ROM, BUL, CRO
				\4[] $\to$ Países de los Balcanes
				\4[] Salen del acuerdo los que entran en UE
				\4 Actualidad
				\4[] MAC, ALB, BYH, MOL, MON, SER, KOS
				\4[] Carácter de antesala pre-UE
				\4[] $\to$ Necesario acuerdo de asociación con UE
			\3 Eurasian Economic Union
				\4 Entrada en vigor en 2015
				\4 Miembros
				\4[] RUS, BEL, KAZ, KYR, ARM
				\4[] Todos miembros de la Com. Estados Independientes
				\4 Actualidad
				\4[] Mercado único
				\4[] Coordinación de:
				\4[] $\to$ Políticas macroeconómicas
				\4[] $\to$ Transporte
				\4[] $\to$ Competencia
				\4[] $\to$ Antitrust
				\4[] $\to$ Órgano de solución de disputas
				\4[] Planes de moneda única y mayor integración
				\4[] Diseño institucional similar a UE
				\4[] Vínculos con Asia
				\4[] $\to$ Especialmente Irán, China, Turquía
			\3 CISFTA
				\4 Community of Independent States Free Trade Agreement
				\4 Miembros
				\4[] Eurasian EU + MOL, UKR, UZB, TAJ
				\4 Actualidad
				\4[] EEU forma núcleo como mercado único
				\4[] FTA del CIS salvo Azerbaiyán
				\4[] Ucrania sin libre comercio con Rusia
		\2 América
			\3 MERCOSUR
				\4 Mercado Común del Sur
				\4 Miembros
				\4[] ARG, URU, PAR, BRA, VEN
				\4 Entrada en vigor en 1994
				\4[] Declaración de Ouro Preto
				\4 Actualidad
				\4[] Mercado común
				\4[] $\to$ Objetivo de medio plazo
				\4[] $\to$ Unión aduanera completada
				\4[] $\to$ Débil implementación de 4 libertades
				\4[] Mayor productor de alimentos del planeta
				\4[] Gigantescas reservas energéticas
				\4[] Venezuela suspendida del bloque
				\4[] Inestabilidad cambiaria
				\4[] $\to$ Devaluaciones de real y peso argentino
				\4[] $\then$ Búsqueda de soluciones bilaterales
				\4[] Bolivia en proceso de adhesión
				\4[] Negociando acuerdo con UE desde hace años
				\4[] $\to$ Avances muy lentos
				\4[] $\to$ Mercosur interés ofensivo en agricultura
				\4[] $\to$ UE exige protección MA y telecomunicaciones
				\4[] Acuerdo político en junio de 2019 con UE
				\4[] Sexto socio comercial de Unión Europea
				\4[] $\to$ Exp. de bienes de UE a Mercosur: $45.000$ M
				\4[] $\to$ Exp. de servicios de UE a Mercosur: $23.000$ M
				\4[] UE es mayor inversor extranjero en Mercosur
				\4[] $\to$ $>380.000$ M de €
			\3 NAFTA y USMCA
				\4 North American Free Trade Agreement

				\4 Entrada en vigor en 1994
				\4[] Origen en Canada-US FTA previo
				\4 Miembros
				\4[] US, CAN, MEX
				\4 Valoración
				\4[] NAFTA ha sido un éxito
				\4[] $\to$ Gran aumento del comercio entre firmantes
				\4[] $\to$ Muy beneficioso para todas las partes
				\4[] $\to$ Industrialización y más eficiencia en México
				\4[] Presidente Trump denuncia acuerdo en 2017
			\3 USMCA
				\4 US-Mexico-Canada Agreement
				\4 Antecedentes
				\4[] Negociaciones bilaterales
				\4[] Acuerdo en otoño de 2018
				\4 Renegociación de NAFTA aprobada en 2019
				\4 Reescritura de $\sim 60\%$ de NAFTA
				\4 Entrada en vigor 1 de julio de 2020
				\4[] Tras firma y ratificación de tres países
				\4 Protección de prop. intelectual
				\4 Facilitación de comercio en fronteras
				\4 Nuevas normas de origen de importaciones
				\4[] Especialmente relevantes en automóviles
				\4[] $\to$ $>70\%$ de acero debe ser USMCA para trato favorable
				\4 Agricultura
				\4[] Mejora acceso agrícola productos americanos en CAN
				\4 Copyright
				\4[] Aumento de protección copyright
				\4 Comercio de automóviles
				\4[] $\to$ Aumenta \% mínimo para exención arancelaria
				\4[] $\to$ Aumenta salario mínimo para exención
				\4[] $\then$ Incentivar producción local en América del Norte
				\4 Regulación laboral
				\4[] Mecanismo de resolución de disputas
				\4[] Especialmente relevante para México
				\4 Mecanismo de resolución de disputas
				\4[] Poco utilizado en NAFTA
				\4[] Fuertes debilidades
				\4[] MRDisputas de OMC prevalece
				\4[] Intento por mejorar y clarificar reglas
				\4 Farmacéuticos
				\4[] Reduce protección a industria
				\4[] Especialmente en términos de Prop. intelectual
				\4 Medio ambiente
				\4[] Gestión de residuos tratada explícitamente
				\4[] Sin referencias al cambio climático
				\4 Necesaria ratificación por 3 países
				\4 Entrada en vigor prevista en 2020
				\4 Revisión cada 6 años
				\4[] Derogación automática a los 16 años
				\4[] $\to$ Si no hay acuerdo de extensión
				\4[] Obligatorio notificar de negs. con terceros
			\3 ALCA
				\4 Área de Libre Comercio de las Américas
				\4 Firmado en 1994
				\4 Miembros
				\4[] Todos los países del continente americano
				\4[] Promovido por EEUU
				\4 Actualidad
				\4[] Paralizado a partir de 2005
				\4[] Fuerte oposición en algunos países
				\4[] $\to$ Sin acceso a exportaciones agrícolas a USA
				\4[] Proceso de reemplazo por tratados menos ambiciosos
			\3 Alianza del Pacífico
				\4 Miembros
				\4[] CHI, COL, MEX, PER
				\4[] ECU y otros en proces de adhesión
				\4 Actualidad
				\4[] Entrada en vigor en 2019
				\4[] Inicialmente, acuerdo de libre comercio
				\4[] Procesos previstos y en marcha:
				\4[] $\to$ Integración de Mercados de Valores nacionales
				\4[] $\to$ Libre circulación de personas
			\3 ALADI
				\4 Asociación Latinoamericana de Integración
				\4[] Sustituye ALALC
				\4[] $\to$ Asociación Latinoamericana de Libre Comercio
				\4[] Organismo internacional regional
				\4 Entrada en vigor en 1980
				\4 Miembros
				\4[] Países latinoamericanos
				\4[] $\to$ Salvo Centroamérica, Rep. Dominicana, GUY, SURINAM
				\4[] $\to$ México sí incluido
				\4 Actualidad
				\4[] Objetivos:
				\4[] $\to$ Creación de sistema preferencial para PMAs
				\4[] $\to$ Reducción de aranceles entre miembros
				\4[] $\to$ Marco de integración más profunda
			\3 Comunidad Andina de Naciones
				\4 Miembros
				\4[] COL, ECU, PER, BOL
				\4 Actualidad
				\4[] Unión Aduanera
				\4[] Fomento de Integración región andina
				\4[] $\to$ Proyecto de mercado común
				\4[] Estructura institucional desarrollada
				\4[] Avances escasos y paralización
				\4[] $\to$ Compite con MERCOSUR, Alianza del Pacífico
				\4[] $\to$ Abandono de Venezuela
				\4[] $\to$ Acuerdos bilaterales de PER y COL con UE
				\4[] $\then$ Proceso de desintegración
			\3 Alternativa Bolivariana para América
				\4 Alianza Bolivariana para los Pueblos de Nuestra América -- Tratado de Comercio de los Pueblos
				\4 Miembros
				\4[] VEN, BOL, CUB, NIC, Países isleños del Caribe
				\4 Actualidad
				\4[] Alternativa de VEN y CUB al ALCA
				\4[] Banco del ALBA
				\4[] Moneda virtual SUCRE
				\4[] Programas de cooperación
				\4[] Intercambios bilaterales de mercancías
				\4[] Intervencionismo estatal
				\4[] Inestabilidad actual
				\4[] $\to$ Crisis en Venezuela
				\4[] $\to$ Proceso de desintegración
			\3 UE-Centroamérica
				\4 Libre comercio
				\4[] Acceso a mercado europeo
				\4[] Eliminación de barreras NArancelarias
				\4 Promoción de integración centroamericana
				\4 Reducir deslocalización a Asia
				\4[] Especialmente, componentes electrónicos de Costa Rica
			\3 MCCA
				\4 Mercado Común Centroamericano
				\4 Entrada en vigor en 1960
				\4 Miembros
				\4[] GUA, SALVADOR, HON, NIC, CRC, PAN
				\4 Actualidad
				\4[] En gran medida sustituido por DR-CAFTA
			\3 DR-CAFTA
				\4 Dominican Republic--Central American FTA
				\4 Miembros
				\4[] USA, CRC, SALV, GUA, HON, NIC, RDOM
				\4 Actualidad
				\4[] Apertura generalizada ByS
				\4[] USA mantiene protección agrícola
		\2 Asia
			\3 CPTPP/TPP11
				\4 Comprehensive and Progressive TPP
				\4 11 miembros
				\4[] AUS, BRU, CAN, CHI, JAP
				\4[] MAL, MEX, NZ, PER, SIN
				\4[] VIE
				\4 Firmado en 2018
				\4 Efectivo desde 2018
				\4 Ratificados a enero de 2020
				\4[] 7 de 11
				\4[] Faltan:
				\4[] $\to$ Brunei
				\4[] $\to$ Chile
				\4[] $\to$ Malasia
				\4[] $\to$ Peru
				\4 Sucesor de TPP sin Estados Unidos
				\4[] Idéntico en 2/3 a TPP original
				\4[] $\to$ En momento de abandono de EEUU
				\4 Elimina intereses ofensivos de EEUU
				\4[] Eliminación de protección añadida a copyright
				\4[] Reducción posibilidad de empresas que demandan gobiernos
				\4 En general, más avanzado que RCEP
				\4 Miembros
				\4[] BRU, MAL, SIN, VIE, AUS, CAN, JAP,
				\4[] MEX, NZ, PER
				\4 Estándares mínimos de protección MAmbiente
				\4 Mecanismos de resolución de disputas
				\4 Obligación de informar sobre empresas públicas
				\4[] Controlar ayudas vía empresas públicas
				\4 Comisión del CPTPP
				\4[] Órgano de decisión
				\4[] Creado en 2018
				\4[] Dos encuentros en 2019
				\4 Posibles miembros futuros
				\4[] Estados Unidos
				\4[] Taiwan
				\4[] Reino Unido
				\4[] Colombia
				\4[] Indonesia
				\4[] Corea del Sur
				\4[] Tailandia
			\3 RCEP\footnote{\url{https://www.csis.org/analysis/last-rcep-deal}}
				\4 Propuesto inicialmente por China en 2011
				\4[] Más inclusivo que TPP en términos de miembros
				\4 Prevista firma en 2020
				\4 Potencial mayor área de comercio mundial
				\4 15 miembros tras retirada de India
				\4[] ASEAN (10)
				\4[] China
				\4[] Japón
				\4[] Corea del Sur
				\4[] Australia
				\4[] Nueva Zelanda
				\4 Más probable tras retirada de TPP
				\4[] Y aprobación posterior de CPTPP/TPP11
				\4 Interés de China
				\4[] Reducir influencia americana
				\4 Regulación de:
				\4[] Agricultura
				\4[] Comercio industrial
				\4[] Comercio electrónico
				\4[] Datos digitales
				\4[] Medioambiente
				\4[] Propiedad intelectual
				\4 Posible MSDisputas en inversiones
				\4 No prohíbe aranceles en contenidos digitales
				\4[] A diferencia de CPTPP y USMCA
				\4[] $\to$ Interés ofensivo habitual de EEUU
				\4 Reducción de aranceles
				\4 Acceso a mercado incrementado
				\4 Menor profundidad que CPTPP y USMCA
				\4 Reglas de origen comunes para todo el bloque
				\4[] Un sólo certificado de origen común
			\3 ASEAN
				\4 Association of Southeast Asian Nations
				\4 Entrada en vigor en 1967
				\4 Miembros (10)
				\4[] MYA, THA, LAO, VIE, CAM,
				\4[] MAL, INDO, PHI, BRU, SIN
				\4 Configuraciones adicionales
				\4[] ASEAN+3
				\4[] $\to$ Con Japón, China, India
				\4[] ASEAN+6
				\4[] $\to$ IND, AUS, NZ
				\4 Actualidad
				\4[] East Asia Summit
				\4[] $\to$ Foro con ASEAN+CHI,JAP, ROK, IND, AUS, NZ, US, RUS ...
				\4[] Asia-Europe Meeting con UE
				\4 Actualidad
				\4[] Relativamente poco comercio entre miembros
				\4[] Economías exportadoras a todo el mundo
				\4[] $\to$ Aumenta peso de comercio extra-ASEAN
				\4[] Objetivos de medio plazo
				\4[] $\to$ Mercado único
				\4[] $\to$ Cooperación en competencia
				\4[] $\to$ Propiedad intelectual
				\4[] $\to$ consumidores
				\4[] Grandes diferencias en nivel de desarrollo
			\3 APEC
				\4 Asia-Pacific Economic Cooperation
				\4 Miembros
				\4[] Iniciativa de Australia
				\4[] Países de la costa del Pacífico
				\4 Actualidad
				\4[] Facilitación de comercio entre miembros
				\4[] Tarjeta de negocios APEC
				\4[] $\to$ Viajar sin visado por motivo de negocios
				\4[] $\to$ Cooperación para reducir intensidad energética
			\3 SAFTA
				\4 South Asian Free Trade Agreement
				\4 Miembros
				\4[] BAN, BHU, INDIA, MALD, NEP, PAK, SRI
				\4 Actualidad
				\4[] Eliminación gradual de aranceles
				\4[] Ampliación de plazos
			\3 BIMSTEC
				\4 Cooperación en la Bahía de Bengala
				\4 Actualidad
				\4[] Proceso con muy pocos avances
				\4[] SAFTA y ASEAN más activos
				\4[] $\to$ Membresía solapada
			\3 PACER y PICTA
				\4 PICTA
				\4[] Islas del Pacífico
				\4[] Bienes y ampliacion a servicios
				\4 PACER
				\4[] PICTA + AUS y NZ
				\4[] Iniciativa australiana
				\4[] $\to$ Mantener influencia en región
				\4[] $\to$ Evitar fragmentación en Oceanía
		\2 MENA
			\3 Idea clave
				\4 Inestabilidad política
				\4 Asimetrías económicas
				\4[] Países exportadores de petróleo
				\4[] $\to$ Golfo, Libia, Argelia
				\4[] $\then$ Muy procíclicos
				\4[] $\then$ Esfuerzos por diversificar economía
				\4[] Países en desarrollo
				\4[] $\to$ Marruecos, Egipto, Túnez
				\4[] $\then$ Esfuerzos por industrializarse
				\4[] $\then$ Economías más diversificadas (especialmente MAR y TUN)
				\4 Unión Europea socio principal
				\4[] Salvo Irak
			\3 GCC -- Consejo de Cooperación del Golfo
				\4 Gulf Cooperation Council
				\4 Miembros
				\4[] BAH, KUW, OMA, QAT, KSA, UAE
				\4 Actualidad
				\4[] Unión aduanera
				\4[] $\to$ Completada en 2015
				\4[] Mercado común
				\4[] $\to$ En proceso de implementación
				\4[] $\to$ Movilidad de personas
				\4[] Proyecto de Unión Monetaria
				\4[] $\to$ Consejo Monetario del Golfo
				\4[] Proyectos de transporte de infraestructuras
				\4[] Paralización tras bloqueo a Qatar
			\3 Acuerdo de Agadir\footnote{http://www.agadiragreement.org/Pages/viewpage.aspx?pageID=243}
				\4 Miembros
				\4[] MOR, JOR, TUN, EGY
				\4 Entrada en vigor en 2019
				\4 Actualidad
				\4[] Profundizar GAFTA
				\4[] Preparar FTA Euro-Mediterranea
				\4[] $\to$ Compatible con EPAs con EU
			\3 GAFTA
				\4 Greater Arab Free Trade Area
				\4 Miembros
				\4[] Países de la Liga Árabe
				\4[] Salvo Somalia y Mauritania
				\4 Entrada en vigor en 2005
				\4[] Tras décadas de proyecto
				\4[] FTA de Liga Árabe
				\4 Actualidad
				\4[] Reducción progresiva de aranceles
				\4[] $\to$ Casi total en la actualidad
				\4[] Supervisión por Consejo de Seguridad de Liga Árabe
				\4[] Avances lentos en otras áreas
				\4[] Comercio interno relativamente pequeño
				\4[] $\to$ Respecto a UE y otras áreas
			\3 Unión del Magreb Árabe
				\4 Inactiva desde 2008
				\4 Proyecto de unión económica y política
				\4[] MAR, ALG, LIB, MAU, TUN
				\4 Parálisis actual
				\4[] Tensión Marruecos y Argelia
		\2 África
			\3 Idea clave
				\4 Muy poco comercio regional
				\4 Principales socios comerciales
				\4[] USA, EU, China
				\4[] $\to$ A pesar de distancia
				\4 Obstáculos a integración
				\4[] Inestabilidad política
				\4[] Carencias en infraestructuras
			\3 Unión Africana
				\4 Paraguas de procesos de integración
				\4 Reciente adhesión de Marruecos
				\4 Planes de largo plazo
				\4[] Avances muy lentos
				\4[] Reducir solapamiento de procesos
			\3 Comunidad Ecónomica Africana
				\4 Organización de la UA
				\4 Proyecto de largo plazo
				\4[] 1. Creación de bloques comerciales
				\4[] 2. FTAs y UAs consolidados en bloques
				\4[] 3. Unión Aduanera para todo el continente
				\4[] 4. Mercado común africano
				\4[] 5. Unión económica y monetaria
				\4[] $\to$ Sólo fase 1 completada
				\4[] $\then$ Casi imposible completar todas las fases
			\3 ECOWAS
				\4 Economic Community of Western African States
				\4 Miembros
				\4[] Países del Oeste de África
				\4[] $\to$ Mitad sur
				\4[] 15 estados
				\4[] Zona A
				\4[] $\to$ CABO VERDE, GAMBIA, GUINEA
				\4[] $\to$ GUINEA BISAU, LIBERIA, MALI,
				\4[] $\to$ SENEGAL, SIERRA LEONA
				\4[] Zona B
				\4[] $\to$ BENIN, BUR-FASO, GHANA
				\4[] $\to$ COSTA DE MARFIL, NIGERIA, NIGER
				\4[] $\to$ TOGO
				\4[] Marruecos solicita adhesión
				\4[] Mauritania estaba pero abandonó grupo
				\4 Actualidad
				\4[] Unión aduanera
				\4[] Coordinación y vigilancia macro
				\4[] Planes sectoriales plurianuales
				\4[] Comprende dos sub-bloques
				\4[] $\to$ UEMOA y UMOA
				\4[] $\to$ WAMZ -- West African Monetary Zone
			\3 UEMOA
				\4 Union Économique Monétaire Ouest-Africaine
				\4 Subgrupo de ECOWAS
				\4 Creada en 1994
				\4 Proyectos
				\4[] Proyecto de creación de mercado común
				\4[] Proyecto de armonización políticas sectoriales
				\4 Más integrada que otras zonas
				\4 No confundir con UMOA
			\3 UMOA
				\4 Unión Monétaire de l'Afrique de l'Ouest
				\4 Creada en años 60
				\4 Mismos miembros que UEMOA
				\4 Actividad exclusivamente monetaria
				\4 Franco CFA del oeste
				\4[] Banco central de África del Oeste
				\4[] Países:
				\4[] $\to$ Benín
				\4[] $\to$ Burkina Faso
				\4[] $\to$ Costa de Marfil
				\4[] $\to$ Guinea-Bisau
				\4[] $\to$ Mali
				\4[] $\to$ Níger
				\4[] $\to$ Senegal
				\4[] $\to$ Togo
				\4 Sustitución por ECO
				\4[] Desde julio de 2020
				\4[] Desaparece obligación de depositar 50\% reservas en BdF
				\4[] Francia y Tesoro Francés se retiran de gobernanza
			\3 WAMZ
				\4 Western African Monetary Zone
				\4 Proyecto de zona monetaria
				\4[] Países de ECOWAS que no tienen Franco CFA del oeste
				\4 Proyecto de moneda común ``eco''
			\3 CEEAC
				\4 Communauté Economique des États de l'Afrique Central
				\4 Creado en los 80
				\4 Paralizado durante décadas
				\4[] Conflictos, epidemias...
				\4 Reactivación reciente
				\4[] Impulso europeo
				\4[] Lograr más integración
			\3 CEMAC
				\4 Subgrupo de la CEEAC
				\4 Communauté Économique et Monétaire de l'Afrique Central
				\4[] Franco CFA de Centroáfrica
				\4[] Banco de los Estados de África Central
				\4 Miembros
				\4[] CAME, GAB, REP CENTRO, CONGO, GUINEA EQ., CHAD
				\4 Actualidad
				\4[] Libre circulación de personas
				\4[] $\to$ Parcialmente
				\4[] $\to$ Suspensiones frecuentes
			\3 SADC
				\4 South African Development Community
				\4 Miembros
				\4[] Todos desde RDCongo hasta Sudáfrica
				\4 Actualidad
				\4[] Planes de desarrollo
				\4[] Proyecto de moneda única
				\4[] Avances lentos
			\3 SACU
				\4 South African Customs Union
				\4 Creada en 1910
				\4 Miembros
				\4[] Namibia, Sudáfrica, Botswana, Lesotho, Swazilandia
				\4 Actualidad
				\4[] Unión aduanera más antigua en vigor
				\4[] Acuerdos con otros bloques
				\4[] $\to$ MERCOSUR, EFTA, USA
			\3 EAC -- East African Community
				\4 Creada en años 60
				\4[] Colapso en 70s
				\4[] Nuevo impulso en 2000s
				\4 Miembros
				\4[] BUR, KEN, RWA, SOUTH SUDAN,
				\4[] TAN, UGA
				\4 Actualidad
				\4[] Unión aduanera desde 2005
				\4[] Proyecto de mercado común
				\4[] Libre comercio con SADC y COMESA
				\4[] Planes de unión monetaria
				\4[] Negociación de Constitución regional
			\3 IGAD
				\4 Intergovernmental Authority on Development
				\4 Bloque comercial
				\4[] Países del cuerno de África
				\4[] Solapado con otros bloques
				\4 Énfasis en proyectos de desarrollo
				\4[] Evitar crisis alimentarias de los 90
			\3 COMESA
				\4 Common Market for Eastern and Southern Africa
				\4 Creada en 1993
				\4 Miembros
				\4[] 19 estados desde Libia y Egipto hasta Zimbabue
				\4 Actualidad
				\4[] Acuerdos con otros bloques comerciales
				\4[] $\to$ EAC
				\4[] $\to$ SADC
	\1 \marcar{Organismos de cooperación internacional}
		\2 OCDE
			\3 Función
				\4 Foro de países con ecs. de mercado
				\4 Coordinar políticas
				\4 Discutir buenas prácticas
				\4 Investigar y difundir información
				\4 Mejorar políticas públicas
			\3 Antecedentes
				\4 Fundada en 1961
				\4 Antecesor: OECE
				\4[] Organización Europea de Cooperación Económica (1948)
				\4[] $\to$ Plataforma distribución Plan Marshall
				\4[] $\to$ Normalización sistema de pagos
				\4[] $\to$ Creación Unión Europea de Pagos (1950-1958)
				\4[] $\to$ Lograr convertibilidad de moneadas
				\4[] $\to$ Acuerdo Monetario Europeo
				\4 Éxito de la OECE
				\4[] Interés de países fuera de Europa
				\4 Creación en 1961
				\4[] Miembros de OECE + USA, CAN
				\4[] Adhesión de Japón
				\4 Caída del telón de acero
				\4[] Ayuda a antiguas repúblicas comunistas
				\4[] $\to$ Transición a economía de mercado
			\3 Organización
				\4 36 miembros
				\4[] 22 EEMM de la UE
				\4[] $\to$ NO están: CRO, RUM, BUL, MAL, CYP
				\4[] UK
				\4[] SWI, NOR, TUR, ICE, ISR
				\4[] CAN, US, MEX, CHI
				\4[] JAP, KOR
				\4[] AUS, NZ
				\4 Procesos de adhesión
				\4[] CRC, COL
				\4[] RUS pospuesto en 2014
				\4 Colaboradores y observadores
				\4[] BRA, CHI, IND, INDO, ZAI
				\4 \underline{Órganos}
				\4 Consejo
				\4[] Órgano de gobierno
				\4[] Dirección estratégica
				\4[] Formado por:
				\4[] $\to$ Representantes permanentes
				\4[] $\to$ Secretario General
				\4[] Aprueba por consenso:
				\4[] $\to$ Plan bienal
				\4[] $\to$ Presupuesto
				\4[] $\to$ Decisiones legalmente vinculantes
				\4[] $\to$ Recomendaciones
				\4[] Reunión anual de ministros
				\4[] $\to$ Para presentar proyectos a Consejo
				\4 Comités y grupos de trabajo
				\4[] Especialistas de diferentes EM
				\4[] Más de 300 comités
				\4[] Proponer soluciones
				\4[] Asesorar estados miembros
				\4 Secretariado
				\4[] Apoyo a Comités y Consejo
				\4[] Organización en directorados
				\4 Presupuesto
				\4[] Cercano a 374 M de €
				\4[] Aportaciones nacionales según PIB
				\4[] EEUU mayor aportador
				\4[] España aporta 3\%
			\3 Actuaciones
				\4 Coordinación acuerdos doble tributación
				\4 Informes, estadísticas e investigación
				\4[] OCDE Economic Outlook
				\4[] $\to$ Dos veces al año
				\4[] $\to$ Previsiones para todos los países miembros
				\4[] OECD Factbook
				\4[] $\to$ Indicadores generales
				\4[] OECD Communications Outlook
				\4[] OECD Internet Economy
				\4 Vigilancia inter-pares
				\4[] Marco para comparar experiencias
				\4[] Identificar buenas prácticas
				\4[] Mejorar diseño de políticas
				\4 Actos legales
				\4[] Sin capacidad de sanción monetaria
				\4[] Decisiones
				\4[] $\to$ Vinculante salvo abstención
				\4[] $\to$ Efecto similar a tratados
				\4[] $\to$ Vía de aprobación de Códigos
				\4[] Recomendaciones
				\4[] $\to$ Sin carácter vinculante
				\4[] $\to$ Expresan voluntad política
				\4[] Acuerdos internacionales
				\4[] $\to$ Vinculantes
				\4[] Otros instrumentos no vinculantes
				\4 Consenso OCDE
				\4[] Normativa sobre créditos a la exportación
				\4[] Obligado cumplimiento en la UE
				\4[] Reconocido por la OMC
			\3 Valoración
				\4 Foro internacional más influyente
				\4 Reúne gran mayoría de comercio e inversión mundial
				\4 Incorporación de PEDs
				\4[] Esfuerzo por incorporar
				\4[] Imagen de ``club de países ricos''
				\4 Disparidad entre miembros
				\4[] Puede complicar toma de decisiones
				\4[] Afectar a relevancia de la organización
		\2 UNCTAD -- United Nations Conference on Trade and Development\footnote{Conferencia de las Naciones Unidas sobre Comercio y Desarrollo}
			\3 Funciones
				\4 Apoyar
			\3 Organización
				\4 195 miembros
				\4 Órgano subsidiario de la Asamblea General
				\4 Creada en 1964 por Asamblea General
				\4 Informa al Consejo Económico y Social
				\4 Órganos, miembros y presupuesto propios
				\4 Órganos
				\4[] Conferencia
				\4[] $\to$ Representantes de EEMM
				\4[] $\to$ 1 voto por EM
				\4[] $\to$ Cada 4 años
				\4[] $\to$ En 2020, conferencia en Barbados
				\4[] Junta de Comercio y Desarrollo
				\4[] $\to$ Supervisa UNCTAD entre conferencias
				\4[] $\to$ 165 miembros
				\4[] $\to$ Entre 1 y 3 veces al año
				\4[] Secretariado
				\4[] $\to$ Apoya órganos y reuniones
				\4[] $\to$ Presupuesto de $\sim 69$ millones USD anuales
				\4[] $\to$ Secretario General\footnote{Mukhisa Kituyi, de Kenia, desde 2013.}
				\4 Staff
				\4[] 470 funcionarios
			\3 Actuaciones
				\4 Promover colaboración con otros orgs. supranac
				\4[] Grupo interinstitucional de NU
				\4 Promover implementación de SPGs
				\4[] Primera contribución relevante:
				\4[] $\to$ SPGs de Ronda Kennedy
				\4 Elaborar informes
				\4[] Trade and Development Report
				\4[] World Investment Report\footnote{Interesante para tema 3B-33.}
				\4[] Trade and Environment Report\footnote{Interesante para tema 3B-32.}
				\4 Programas de cooperación y asistencia técnica
				\4[] Centrados en países en desarrollo
			\3 Valoración
				\4 Críticas
				\4[] Tendencia a favorecer:
				\4[] $\to$ Controles de precios
				\4[] $\to$ Formación de monopolios
				\4[] $\to$ ``Nuevo Orden Económico Internacional''\footnote{Conjunto de propuestas defendidas en los años 70 por algunos países en desarrollo en el seno de la UNCTAD que tenían por objetivo alterar a su favor sus relaciones relativas de intercambio, aumentar la ayuda al desarrollo y obtener acceso preferencial a los mercados de importación de países desarrollados. Se vinculaba con el movimiento de países no-alineados y en último término, trataba de transformar la arquitectura institucional de Bretton Woods.}
				\4[] Sesgo estructuralista inicial
				\4[] $\to$ Raúl Prebisch
				\4 Órgano de la Asamblea General
				\4[] Principal órgano
				\4[] Debatir
				\4[] $\to$ Oportunidades comerciales
				\4[] $\to$ Inversión
				\4[] $\to$ Desarrollo
				\4[] $\to$ Integración en economía mundial
				\4[] $\then$ Relativa a PEDs
				\4 Conferencias cada 4 años
		\2 Otros organismos
			\3 Instituciones de Bretton Woods
				\4 FMI
				\4 GATT/WTO
				\4 Grupo del BM
			\3 G-20
				\4 Función
				\4[] Coordinar respuestas a crisis
				\4[] Foro para dar voz a emergentes
				\4[] Coordinar política económica internacional
				\4 Antecedentes
				\4[] Creado en 1999
				\4[] Foro para dar voz a PEDs con G7
				\4[] Responder a crisis de deuda
				\4[] $\to$ Asia, Rusia, LTCM
				\4 Organización
				\4[] Sin staff o secretariado permanente
				\4[] Presidente rotatorio entre regiones
				\4[] Miembros
				\4[] $\to$ 19 países + UE
				\4[] $\to$ ARG, AUS, BRA, CAN, CHI
				\4[] $\to$ FRA, GER, IND, INDO, ITA
				\4[] $\to$ JAP, MEX, RUS, KSA, ZAI, ROK
				\4[] $\to$ TUR, UK, US
				\4[] $\to$ España invitado permanente
				\4[] $\to$ Dos invitados africanos
				\4[] $\to$ Presidente de ASEAN
				\4[] $\to$ País invitado por presidente rotatorio
				\4[] $\to$ Otras instituciones
				\4 Actuaciones
				\4[] Reuniones anuales
				\4[] $\to$ En país que ocupa presidencia
				\4[] $\to$ Gobernadores y ministros de finanzas
				\4[] $\to$ Banco Mundial y FMI
				\4[] 2019: Cumbre de Osaka
				\4[] 2020: Cumbre de Riad (Arabia Saudi)
			\3 G-8 y G-7
				\4 Función
				\4[] Foro de países desarrollados
				\4[] $\to$ Con mayor peso militar y económico
				\4 Organización
				\4[] GER, CAN, USA, FRA, ITA, JAP, RU
				\4[] RUS invitado hasta 2014 (G8)
				\4[] $\to$ Anexión de Crimea
			\3 3G -- Global Governance Group
				\4 Grupo informal de 30 países
				\4 Función
				\4[] Canalizar posturas en G20
				\4 Miembros
				\4[] Países no G20
				\4[] Incluyendo microestados y tercer mundo
			\3 FSB -- Financial Stability Board
				\4 Función
				\4[] Valorar vulnerabilidades del sistema financiero mundial
				\4[] Identificar y valorar medidas macroprudenciales
				\4[] Asesorar en mejores prácticas regulatorias
				\4[] Colaborar con IMF en ejercicios de alerta temprana
				\4[] Promover implementación de acuerdos
				\4 Antecedentes
				\4[] Creado en 2009
				\4[] $\to$ En marco del G20
				\4[] Consolidación como organización permanente en 2011
				\4[] Sustituye al FSF--Financial Stability Forum de 1999
				\4[] $\to$ Iniciativa del G7
				\4 Organización
				\4[] Sede en Basilea
				\4[] Miembros\footnote{https://www.fsb.org/about/fsb-members/}
				\4[] $\to$ 24 miembros\footnote{ARG, KOR, AUS, MEX, BRA, NED, CAN, RUS, KSA, RPC, SIN, ZAF, FRA, ESP, GER, SWI, HK, TUR, IND, UK, INDO, USA, ITA, JAP.} nacionales + UE
				\4[] $\to$ Envían gob. BC y ministro de finanzas
				\4[] Instituciones financieras internacionales
				\4[] $\to$ BIS
				\4[] $\to$ OCDE
				\4[] $\to$ IMF
				\4[] $\to$ GBM
				\4[] Instituciones de estandarización y supervisión
				\4[] $\to$ Comité de Basilea del BIS
				\4[] $\to$ Comité sobre el Sistema Financiero Global del BIS
				\4[] $\to$ Comité sobre Pagos e Infraestructura del BIS
				\4[] $\to$ IAIS\footnote{International Association of Insurance Supervisors.}
				\4[] $\to$ IASB\footnote{International Accounting Standards Board.}
				\4[] $\to$ IOSCO\footnote{International Organization of Securities Commissions.}
				\4[] Comité Plenario
				\4[] $\to$ Único órgano de toma de decisiones
				\4[] Comité Ejecutivo
				\4[] $\to$ Dirección operativa entre Comités Plenarios
				\4[] SCAV -- Comité de Valoración de Vulnerabilidades
				\4[] SRC -- Comité de Supervisión y Cooperación Regulatoria
				\4[] SCSI -- Comité de Implementación de Estándares
				\4 Actuaciones
				\4[] Supervisión de riesgos financieros globales
				\4[] Análisis de medidas macroprudenciales
				\4[] Elaboración de guías de supervisión y regulación
				\4[] Elaborar planes de contingencia
			\3 CIS
			\3 OPEP -- Organización de Países Exportadores de Petróleo
	\1[] \marcar{Conclusión}
		\2 Recapitulación
			\3 Formas de integración
			\3 Procesos de integración no comunitaria
			\3 Organismos de cooperación internacional
		\2 Idea final
			\3 Cadenas de valor internacionales
				\4 De exportación de mercancías a tareas
				\4[] Producción dividida en varias fases
				\4[] Países se especializan en fase
				\4 Cadenas de comercio muy complejas
				\4[] Inputs intermedios en muchos países
				\4[] $\to$ Importancia de integración consistente y profunda
			\3 Acuerdos de nueva generación
				\4 Más allá de barreras arancelarias
				\4 Aspectos incluidos
				\4[] Normativa técnica
				\4[] Tribunales para disputas de inversiones
				\4[] Mecanismos de solución de diferencias
				\4[] Normativa laboral y medioambiental
				\4[] ...
				\4 Reflejo de mayor complejidad del comercio
			\3 Nuevo bilateralismo
				\4 Administración Trump
				\4 Abandono de negociaciones multilaterales
				\4 Acuerdos bilaterales
				\4
			\3 Brexit
				\4 UK es una de las 6 mayores economías del mundo
				\4[] Pieza importante de sistema económico mundial
				\4[] Servicios financieros, alta tecnología, biotecnología...
				\4[] Participa en muchas cadenas de valor global
				\4 Salida de la UE
				\4[] Salida de unión aduanera
				\4[] $\to$ Necesitará renegociar acuerdos
			\3 Papel de la UE
\end{esquemal}

























\preguntas

\seccion{Test 2013}
\textbf{44.} Seleccionar cuál de las siguientes afirmaciones resulta \textbf{FALSA}:
\begin{itemize}
	\item[a] El G20 ha adquirido un papel relevante en la coordinación, a nivel global, de los principales ejes de política económica en respuesta a la crisis financiera y económica reciente, incluyendo políticas macroeconómicas, financieras y relativas a la arquitectura financiera internacional.
	\item[b] El Banco de Pagos Internacionales se ha reforzado, en el actual contexto, como institución que vela por evitar desequilibrios excesivos de balanza de pagos.
	\item[c] \textit{[sic]} El Consejo de Estabilidad Financiera (Financial Stability Board --FSB- en inglés-) coordina, a nivel internacional, el trabajo de las autoridades financieras nacionales y los organismos internacionales de estándares financieros y FSB desarrolla y promueve la implementación de buenas políticas en materia de supervisión y regulación financiera.
	\item[d] En el contexto de la reciente crisis, el G20 ha acordado un aumento de los recursos del Fondo Monetario Internacional para reforzar la capacidad de préstamo de dicho organismo.
\end{itemize}

\seccion{Test 2011}

\textbf{37.} La Organización para la Cooperación y el Desarrollo Económico (OCDE):
\begin{itemize}
	\item[a] Sustituyó a la Organización Europea para la Cooperación Económica.
	\item[b] Incluye únicamente países europeos.
	\item[c] Se estableció para repartir las ayudas del Plan Marshall.
	\item[d] Todas las anteriores respuestas son falsas.
\end{itemize}

\notas

\textbf{2013:} \textbf{44.} B

\textbf{2011:} \textbf{37.} A

\bibliografia

Mirar en Palgrave\footnote{Copiado de tema 3B-10.}:
\begin{itemize}
	\item comparative advantage *
	\item customs unions *
	\item economic integration *
	\item effective protection
	\item Euro
	\item European Union (EU) trade policy *
	\item European Union Single Market: Design and Development *
	\item European Union Single Market: Economic Impact *
	\item genuine economic and monetary union *
	\item Heckscher-Ohlin trade theory
	\item international outsourcing
	\item Mercosur
	\item North-American Free Trade Agreement (NAFTA)
	\item optimal tariffs
	\item regional and preferential trade agreements *
	\item tariffs
	\item terms of trade
	\item The European Union's Common Agricultural Policy (CAP)
	\item theory of Economic Integration: a review *
	\item tradable and non-tradable commodities
	\item trade policy, political economy of *
	\item trade costs
\end{itemize}


IMF. \textit{A guide to commmittees, groups and clubs} (2018) -- \url{https://www.imf.org/en/About/Factsheets/A-Guide-to-Committees-Groups-and-Clubs}

 https://voxeu.org/article/regional-trade-agreements-blessing-or-burden

\end{document}
