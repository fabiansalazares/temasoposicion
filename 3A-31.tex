\documentclass{nuevotema}

\tema{3A-31}
\titulo{Análisis macroeconómico del mercado de trabajo: teoría del desempleo de equilibrio; la NAIRU y la persistencia en el desempleo.}

\begin{document}

\ideaclave

Reformar con Gordon (2008) sobre historia de la Curva de Phillips

\seccion{Preguntas clave}
\begin{itemize}
	\item ¿Qué es el mercado de trabajo?
	\item ¿Cómo se analiza desde el punto de vista macroeconómico?
	\item ¿Qué modelos tratan de explicar su evolución?
	\item ¿Qué teorías tratan de explicar el desempleo?
	\item ¿Qué implicaciones de política económica se derivan?
	\item ¿Qué evidencia empírica existe al respecto?
	\item ¿Qué es la NAIRU?
	\item ¿Qué es la persistencia en el desempleo?
\end{itemize}

\esquemacorto

\begin{esquema}[enumerate]
	\1[] \marcar{Introducción}
		\2 Contextualización
			\3 Mercado de trabajo
			\3 Enfoques de estudio
			\3 Desempleo
		\2 Objeto
			\3 ¿Qué es el mercado de trabajo?
			\3 ¿Cómo se analiza desde el punto de vista macroeconómico?
			\3 ¿Qué modelos tratan de explicar su evolución?
			\3 ¿Qué teorías tratan de explicar el desempleo?
			\3 ¿Qué implicaciones de política económica se derivan?
			\3 ¿Qué evidencia empírica existe al respecto?
			\3 ¿Qué es la NAIRU?
			\3 ¿Qué es la persistencia en el desempleo?
		\2 Estructura
			\3 Hechos estilizados
			\3 Evolución del análisis macroeconómico del mercado de trabajo
			\3 Tasa natural de desempleo y NAIRU
	\1 \marcar{Hechos estilizados del mercado de trabajo}
		\2 Idea clave
			\3 Análisis agregado
			\3 Principales variables
			\3 Heterogeneidad
		\2 Evolución cíclica
			\3 Desempleo
			\3 Salarios
			\3 Actividad
			\3 Productividad
		\2 Hechos estilizados de largo plazo
			\3 Heterogeneidad geográfica del desempleo
			\3 Heterogeneidad demográfica del desempleo
			\3 Periodos de desempleo elevado muestran persistencia
			\3 Desempleo de largo plazo más elevado en Europa
			\3 Tendencia de l/p a aumento de participación
			\3 Aumento de la participación de mujeres
			\3 Ajuste más rápido en Japón y USA
	\1 \marcar{Desempleo e inflación}
		\2 Idea clave
			\3 Contexto
			\3 Objetivo
			\3 Resultado
		\2 Tipos de desempleo
			\3 Friccional
			\3 Estacional
			\3 Estructural
			\3 Cíclico
			\3 Involuntario
			\3 Subempleo
		\2 Tasa natural de paro
			\3 Idea clave
			\3 Formulación
			\3 Implicaciones
		\2 NAIRU
			\3 Idea clave
			\3 Formulación
			\3 Estimación empírica
			\3 Implicaciones
		\2 Histéresis
			\3 Idea clave
			\3 Evidencia empírica
			\3 Implicaciones
		\2 Persistencia
			\3 Idea clave
			\3 Formulación
			\3 Implicaciones
	\1 \marcar{Análisis teórico del mercado de trabajo}
		\2 Idea clave
			\3 Contexto
			\3 Objetivo
			\3 Resultados
		\2 Modelo clásico
			\3 Idea clave
			\3 Formulación
			\3 Implicaciones
			\3 Valoración
		\2 Keynes
			\3 Idea clave
			\3 Formulación
			\3 Implicaciones
			\3 Valoración
		\2 Síntesis neoclásica
			\3 Idea clave
			\3 Formulación
			\3 Implicaciones
			\3 Valoración
		\2 Curva de Phillips
			\3 Idea clave
			\3 Formulación
			\3 Implicaciones
			\3 Valoración
		\2 Monetarismo
			\3 Idea clave
			\3 Formulación
			\3 Implicaciones
			\3 Valoración
		\2 Neokeynesianos del desequilibrio
			\3 Idea clave
			\3 Leijonhufvud
			\3 Clower: hipótesis de la decisión dual
			\3 Barro y Grossman, Malinvaud
		\2 Nueva Macroeconomía Clásica
			\3 Idea clave
			\3 Formulación
			\3 Implicaciones
			\3 Valoración
		\2 RBC -- Real Business Cycle
			\3 Idea clave
			\3 Formulación
			\3 Implicaciones
			\3 Valoración
		\2 Nueva Economía Keynesiana--Primera generación
			\3 Idea clave
			\3 Contratos implícitos
			\3 Insiders y outsiders
			\3 Salarios de eficiencia
			\3 Modelo de negociación salarial
			\3 Fallos de coordinación
			\3 Implicaciones
		\2 Nueva Economía Keynesiana--Segunda generación
			\3 REFORMULAR CON GALI (2015) CH. 7  SOBRE DESEMPLEO
			\3 Idea clave
			\3 Formulación
			\3 Implicaciones
			\3 Extensiones
			\3 Valoración
		\2 Modelos de búsqueda - DMP
			\3 Idea clave
			\3 Formulación
			\3 Implicaciones
			\3 Valoración
	\1[] \marcar{Conclusión}
		\2 Recapitulación
			\3 Hechos estilizados en el mercado de trabajo
			\3 Empleo e inflación
			\3 Evolución de análisis teórico
		\2 Idea final
			\3 Problemas del mercado laboral europeo
			\3 Flexibilización de mercados laborales
			\3 Relación con otros conceptos

\end{esquema}

\esquemalargo












\begin{esquemal}
	\1[] \marcar{Introducción}
		\2 Contextualización
			\3 Mercado de trabajo
				\4 Especial importancia
				\4[] Trabajo remunerado es principal fuente de renta
				\4[] Condiciona actividad humana
				\4[] $\to$ Fracción importante del tiempo
			\3 Enfoques de estudio
				\4 Macroeconómico:
				\4[] Entender y predecir vars. agregadas
				\4[] $\to$ paro, ocupación, duración del paro...
				\4 Microeconómico:
				\4[] Entender y predecir decisión individual
				\4[] $\to$ ¿trabajar o no?
				\4[] $\to$ ¿cuánto tiempo dedicar al trabajo?
				\4[] $\to$ ¿qué salario exigir por el trabajo?
				\4[] $\to$ ¿cuánto trabajo aplicar al proceso productivo?
				\4[] $\to$ ¿qué relación entre trabajo y otros factores?
				\4[] $\to$ ¿qué salario ofrecer?
				\4[] $\to$ ¿cómo repartir trabajo entre miembros de la familia?
				\4[] $\to$ ¿cuánto tiempo dedicar a la búsqueda de empleo?
				\4[] $\to$ ¿cuánta educación obtener?
				\4 Relación entre ambos enfoques
				\4[] Análisis macroeconómico tiende a microfundamentar
				\4[] $\to$ Explicar vars. agregadas como res. decisiones micro
			\3 Desempleo
				\4 Principal variable de estudio
				\4 Importante coste social y económico
				\4[] Capacidad productiva sin utilizar
				\4[] Coste psicológico de desempleados
				\4[] Inestabilidad política
				\4 Importante esfuerzo prescriptivo de las teorías
				\4[] Tratar de reducir tasas de desempleo
		\2 Objeto
			\3 ¿Qué es el mercado de trabajo?
			\3 ¿Cómo se analiza desde el punto de vista macroeconómico?
			\3 ¿Qué modelos tratan de explicar su evolución?
			\3 ¿Qué teorías tratan de explicar el desempleo?
			\3 ¿Qué implicaciones de política económica se derivan?
			\3 ¿Qué evidencia empírica existe al respecto?
			\3 ¿Qué es la NAIRU?
			\3 ¿Qué es la persistencia en el desempleo?
		\2 Estructura
			\3 Hechos estilizados
			\3 Evolución del análisis macroeconómico del mercado de trabajo
			\3 Tasa natural de desempleo y NAIRU
	\1 \marcar{Hechos estilizados del mercado de trabajo}
		\2 Idea clave
			\3 Análisis agregado
				\4 Resumir miles de observaciones en var. agregada
				\4 Explicar fenómenos empíricos formulando modelos
			\3 Principales variables
				\4 Población en edad de trabajar
				\4[] Generalmente, 16-65 o 18-65 años
				\4 Población activa
				\4[] Dentro de pob. en edad de trabajar
				\4[] $\to$ Dispuestos a trabajar
				\4[] $\to$ Buscando empleo
				\4[] $\to$ Empleados actualmente
				\4 Tasa de actividad
				\4[] \% de población en edad de trabajar
				\4[] $\to$ Que está activa
				\4 Ocupados
				\4[] Subconjunto de población activa
				\4[] Personas que han trabajado en últimos días
				\4 Tasa de ocupación
				\4[] \% de población activa
				\4[] $\to$ Que trabaja
				\4 Desempleados
				\4[] Personas que  buscan empleo/quieren trabajar
				\4[] No encuentran trabajo acorde
				\4[] \% de población activa
				\4[] $\to$ Que no trabaja
			\3 Heterogeneidad
				\4 Vars. agregadas resumen conjuntos heterogéneos
				\4[] Pérdida de información
				\4 Conjuntos tienen características peculiares
				\4[] Que pueden ser muy diferentes a variable agregada
				\4[] Pueden dar lugar a conclusiones inexactas
		\2 Evolución cíclica
			\3 Desempleo
				\4 Claramente contracíclico
				\4 Aumento de parados componente fundamental
				\4 Aumento de población activa puede amortiguar
				\4[] ``Desanimados'' pueden volver a buscar
				\4[] También posible efecto contrario\footnote{Ejemplo: en algunas recesiones muy profundas, un miembro de la familia que previamente no buscaba trabajo, se incorpora a la población activa porque la principal fuente de ingresos de la familia ha desaparecido al haber sido despedido. Cuando la economía se recupera y uno de los miembros vuelve a estar empleado, el otro miembro deja de buscar trabajo.}
				\4 Ley de Okun
				\4[] Correlación empírica negativa
				\4[] \fbox{$\Delta y_t = \beta_0  + \beta_1 \Delta u_t + \epsilon_t$}
				\4[] $\to$ $\beta_1 < 0$ casi siempre
				\4[] $\then$ Más paro, menos output
				\4[] También en términos de output gap
				\4[] $\to$ En términos de output potencial y paro de equilibrio
				\4[] $\to$ Más output gap relacionado con menor desempleo
			\3 Salarios
				\4 Generalmente ligera prociclicidad
			\3 Actividad
				\4 Ambigua
				\4 Desanimados de largo plazo pueden salir
				\4[] En fases de contración
				\4[] $\to$ Puede ser contracíclica
				\4 Cuando miembro de familia que trabaja pierde empleo
				\4[] Otros miembros tratan de entrar también
				\4[] $\to$ Puede aumentar actividad en fase recesiva
			\3 Productividad
				\4 Hasta los 80s, prod. del trabajo procíclica
				\4 A partir de los 80s
				\4[] $\to$ Cambios en varias economías
				\4[] $\then$ Menor prociclicidad en EEUU
				\4[] $\then$ Productividad contracíclica en España
				\4[$\then$] Variable controvertida y poco robusta
		\2 Hechos estilizados de largo plazo
			\3 Heterogeneidad geográfica del desempleo
				\4 Japón y Estados Unidos bajo desempleo
				\4 Elevada heterogeneidad en Europa
				\4[] En general, desempleo más elevado
				\4 También fuerte heterogeneidad regional
			\3 Heterogeneidad demográfica del desempleo
				\4[] Entre grupos de edad, población, habilidades
				\4[] $\to$ Mayor desempleo entre jóvenes
				\4[] $\to$ Menor desempleo con más educación
			\3 Periodos de desempleo elevado muestran persistencia
				\4 Mercados laborales sufren fuerte efecto inicial
				\4[] Fuerte aumento del paro
				\4 Caída del paro muy lenta en recuperación
				\4[] Especialmente en sur de Europa
				\4[] Muy elevada persistencia
			\3 Desempleo de largo plazo más elevado en Europa
				\4 Mayor número de personas en paro > 1 año
			\3 Tendencia de l/p a aumento de participación
				\4 General en todos los países desarrollados
			\3 Aumento de la participación de mujeres
				\4 Tasas de participación crecen a largo plazo
			\3 Ajuste más rápido en Japón y USA
				\4 Tras shock que afecta al mercado de trabajo
				\4[] USA y Japón se reajustan más rápido
	\1 \marcar{Desempleo e inflación}
		\2 Idea clave
			\3 Contexto
				\4 Hilo conductor de modelos macro del mercado laboral
				\4[] Tensión entre uso de cap. productiva e inflación
				\4[] $\to$ Relación entre desempleo e inflación
				\4[] $\then$ Diferentes formas de explicar
				\4[] $\then$ Diferentes valoraciones sobre posible causalidad
				\4[] Tensión entre estabilidad y múltiples equilibrios
				\4[] $\to$ Diferentes formas de valorar estabilidad de economía
				\4[] $\then$ Modelos valoran diferente impacto de fluctuaciones
				\4 Múltiples equilibrios
				\4[] Políticas de demanda son efectivas sobre empleo
				\4 Equilibrio único y estable
				\4[] Políticas de oferta para reducir desempleo
			\3 Objetivo
				\4 Caracterizar relación entre desempleo en inflación
				\4 Explicitar supuestos de modelos
				\4 Explicar causas del desempleo de largo plazo
				\4 Valorar efectos de c/p sobre l/p en desempleo
			\3 Resultado
				\4 Varios conceptos para examinar mercado laboral
				\4 Tasa natural de desempleo
				\4[] Tasa de equilibrio tras ajuste de vars. nominales
				\4 NAIRU
				\4[] Concepto más general
				\4[] Desempleo que no acelera inflación
				\4 Histéresis
				\4[] Cambio permanente de NAIRU
				\4[] $\to$ Ante cambio puntual del paro
		\2 Tipos de desempleo
			\3 Friccional
				\4 Resultado de las fricciones entre oferta y demanda
				\4[] Procesos de contratación
				\4[] Despidos
				\4[] Personas que abandonan trabajo
				\4 Ejemplo:
				\4[] Empresas
				\4[] $\to$ no localizan desempleados inmediatamente
				\4[] Desempleados:
				\4[] $\to$ no toman primera vacante que aparece
				\4[] $\to$ No encuentran vacante deseada instantáneamente
				\4[] $\to$ Tarda en iniciar proceso de búsqueda
			\3 Estacional
				\4 Fruto de la variación del output en el año
				\4 Ejemplo:
				\4[] Empleo en turismo de playa
				\4[] $\to$ Aumenta en periodo estival
				\4[] $\then$ Desempleo aumenta en otoño e invierno
			\3 Estructural
				\4[] Dos concepciones
				\4 Desajuste estructural oferta y demanda
				\4[] Desajuste entre demanda de trabajo
				\4[] $\to$ Y características de la demanda
				\4[] Especialmente relevante
				\4[] $\to$ Habilidades de los trabajadores
				\4[] $\to$ Localización de los trabajadores
				\4[$\to$] Ejemplo de desempleo estructural
				\4[] Burbuja inmobiliaria forma personas en construcción
				\4[] Burbuja estalla y no hay demanda de construcción
				\4[] $\to$ Desempleados con habilidades en construcción
				\4[] $\then$ No encuentran trabajo
				\4[] $\then$ Desempleo aumenta
				\4[] Sector de energías renovables en auge
				\4[] $\to$ Demanda de ingenieros eléctricos
				\4[] $\to$ Imposible reasignar trabajadores en c/p
				\4[] $\then$ Hay vacantes que no se cubren
				\4 Desempleo compatible con inflación estable
				\4[] Utilizado en modelos matemáticos del MTrabajo
				\4[] Teorías tratan de explicar
				\4[] $\to$ Ver más abajo
			\3 Cíclico
				\4 Resultado de fluctuaciones cíclicas en output
				\4[] Se solapa con tipos anteriores
				\4 Diferentes teorías del ciclo económico
				\4 Origen en demanda
				\4[] Insuficiencias de demanda agregada
				\4[] $\to$ Mantienen trabajadores activos desocupados
				\4 Origen en oferta
				\4[] Shocks de productividad
				\4[] $\to$ Trabajadores sustituyen L intertemporalmente
			\3 Involuntario
				\4 Concepto fundamentalmente keynesiano
				\4 Desempleados dispuestos a trabajar
				\4[] A salario al que trabajan otros trabajadores
				\4[] $\to$ Pero no aceptan ser contratados
			\3 Subempleo
				\4 Factor trabajo utilizado por debajo de capacidad
				\4[] En términos de:
				\4[] $\to$ Horas trabajadas
				\4[] $\to$ Habilidades aplicadas
		\2 Tasa natural de paro
			\3 Idea clave
				\4 Wicksell
				\4[] Interés ``natural''
				\4[] $\to$ Equilibrio en mercado de capital
				\4[] $\then$ Sin variables nominales afectando
				\4 Friedman importa concepto de ``natural''
				\4 Tasa de paro natural
				\4[] Tasa de paro
				\4[] $\to$ Cuando inflación esperada y real coinciden
				\4[] $\to$ Resultado de sistema walrasiano de ecs. estructurales
				\4[] Con tasa de paro natural:
				\4[] $\to$ Output de largo plazo
				\4[] No es inmutable ni constante
				\4[] $\to$ Políticas de oferta pueden afectar
				\4[] $\to$ Políticas de demanda no alteran
			\3 Formulación
				\4[] \fbox{$\pi_t  = f(\bar{u} - u) + \pi^e_t$, $f(0)=0$}
				\4[] $\bar{u}$: tasa de paro natural
				\4[] $\to$ Inflación real y esperada coinciden
			\3 Implicaciones
				\4 Posibles desviaciones temporales
				\4[] Vía políticas de demanda
				\4 Connotación positiva
				\4[] ``Natural'' como ``libre de distorsiones''
				\4 No es posible explotar estructuralmente
				\4[] Tendencia a tasa natural o de equilibrio
		\2 NAIRU
			\3 Idea clave
				\4 Generalización del concepto de tasa natural
				\4 Tasa de paro que mantiene inflación estable
				\4[] En ausencia de shocks de oferta
				\4 No implica equilibrio único y estable
				\4[$\then$] Economía no tiene por qué tender
			\3 Formulación
				\4 Similar a tasa natural
			\3 Estimación empírica
				\4 Formulación genérica\footnote{Ver Gordon (1997).}
				\4[] Dornbusch (1975), otros autores en 80s
				\4[] Gordon (1997): modelo del ``triángulo''
				\4[] $\to$ Inercia -- Demanda -- Oferta
				\4[] $\pi_t = \alpha \pi_{t-1} + \beta D_t + \gamma S_t + v_t$
				\4[] $\to$ $\alpha$: inercia inflacionaria
				\4[] $\to$ $D_t$: exceso de demanda en mercado laboral
				\4[] $\to$ $S_t$: vector de shocks de oferta
				\4[] $\to$ $v_t$: vector de errores sin correlación serial
				\4 Exceso de demanda en mercado laboral
				\4[] Caracterizado como $D_t = \bar{U_t} - U_t^N$
				\4 Objetivo
				\4[] Separar:
				\4[] $\to$ Condiciones del mercado de trabajo
				\4[] $\to$ Cambios en las condiciones de oferta
				\4[] $\then$ Relación entre mercado laboral e inflación
				\4[] Estimar $\bar{U}_t$:
				\4[] $\to$ Conocida inflación
				\4[] $\to$ Conocido desempleo
				\4[] $\to$ Postulando shocks de oferta
				\4[] $\then$ Serie de $\bar{U}_t$
			\3 Implicaciones
				\4 Sin connotación positiva
				\4[] No implica desaparición de efecto de rigideces
				\4 Fluctuaciones posibles
				\4[] Diferentes teorías explicativas
				\4 Empleo no tiene por qué tender a NAIRU
				\4[] No es ``natural''
				\4[] $\to$ Sólo implica inflación estable
				\4 Enorme importancia en política macroeconómica
				\4[] Informar decisiones de política monetaria
				\4[] $\to$ Qué política monetaria llevar a cabo
				\4[] $\to$ Hasta qué punto se ha diseñado bien
				\4[] Valorar funcionamiento del mercado laboral
				\4[] $\to$ Para utilizar capacidad productiva
				\4[] Distinguir condiciones coyunturales de empleo
				\4[] $\to$ De problemas estructurales
		\2 Histéresis
			\3 Idea clave
				\4 Concepto original de la física
				\4[] Tras aplicar un estímulo a un sistema
				\4[] $\to$ No retorna a situación original
				\4[] $\to$ Aunque se deje de aplicar el estímulo
				\4 Aplicación al análisis macro del mercado laboral
				\4[] Blanchard y Summers (1986)
				\4[] Aumento de coyuntural del desempleo
				\4[] $\to$ Afecta a NAIRU
				\4[] $\then$ Economía no retorna a NAIRU original
				\4 Origen del concepto en economía
				\4[] Aumento del desempleo en Europa en 80s
				\4[] $\to$ No vuelve a niveles anteriores
				\4 Teorías de la histéresis
				\4[] Explicar por qué NAIRU cambia
				\4[] $\to$ Ante cambios en paro en un corto plazo
			\3 Evidencia empírica
				\4 Fuerte aumento del desempleo en 70s
				\4[] Especialmente en Europa
				\4[] Estados Unidos recupera tasas de paro anteriores
				\4[] $\to$ Persisten en Europa
				\4 Rigideces no pueden explicar persistencia
				\4[] Mercados se habrían ajustado en largo plazo
				\4 Sustitución no puede explicar
			\3 Implicaciones
				\4 Estabilidad del mercado laboral
				\4[] Histéresis implica mercado laboral no es estable
				\4[] $\to$ Pero tampoco divergente
				\4[] $\then$ Tampoco inestable
				\4 Historia es importante
				\4[] Evolución pasada de desempleo
				\4[] $\to$ Determina desempleo en el presente
		\2 Persistencia
			\3 Idea clave
				\4 Representación alternativa de cambios en NAIRU
				\4 Contraste entre ``histéresis pura'' y persistencia
			\3 Formulación
				\4 Histéresis no existe
				\4[] $\to$ Cambio en NAIRU no sucede realmente
				\4[] $\then$ Fruto de problemas de estimación
				\4 Desempleo es altamente persistente
				\4[] Variables se ajustan muy lentamente
				\4[] $\to$ Parece histéresis pero es ajuste hacia equilibrio
			\3 Implicaciones
				\4 Teorías de histéresis no explican paro
				\4 Mercado laboral tiende a equilibrio
	\1 \marcar{Análisis teórico del mercado de trabajo}
		\2 Idea clave
			\3 Contexto
				\4 Estadísticas sobre empleo
				\4[] Relativamente abundantes y buena calidad
				\4 Hechos estilizados robustos
			\3 Objetivo
				\4 Entender evolución de agregados sobre empleo
			\3 Resultados
				\4 Desempleo es constante histórica
				\4 Debate de largo recorrido
				\4[] ¿Ecs. alcanzan plena capacidad por sí solas?
				\4 Papel de las rigideces
				\4[] Varios tipos de rigideces en mercados de trabajo
				\4[] Variables no se ajustan hasta el equilibrio
				\4[] $\to$ ¿Por qué?
				\4[] $\to$ ¿Qué conclusiones en términos de pol. econ?
		\2 Modelo clásico
			\3 Idea clave
				\4 Contexto
				\4[] Modelo neoclásico aplicado al mercado de trabajo
				\4[] Mercados de bienes perfectamente competitivos
				\4[] Bien intercambiado (trabajo) homogéneo
				\4[] Ajuste de salario real para vaciar mercado
				\4[] $\to$ Variable de ajuste perfectamente flexible
				\4[] $\then$ Mercado en permanente equilibrio
				\4 Objetivo
				\4[] Representar equilibrio de mercado laboral
				\4[] $\to$ A nivel agregado
				\4[] Explicar desempleo
				\4[] $\to$ Como frenos institucionales a ajuste de SReal
				\4[] Caracterizar
				\4 Resultado
				\4[] Si el salario real se ajusta al eq. walrasiano
				\4[] $\to$ Todo el trabajo ofertado se vende
				\4[] $\to$ Se demanda todo el trabajo ofertado
				\4[] $\then$ No hay desempleo
				\4[] Salario real fuertemente procíclico
				\4[] $\to$ Si oferta de trabajo es inelástica
				\4[] Mercado de trabajo causa output
				\4[] $\to$ Equilibrio en mercado de trabajo determina L
				\4[] $\then$ Trabajo de equilibrio determina output
				\4[] $\to$ Exceso de oferta de trabajo no es persistente
				\4[] Políticas de demanda son inefectivas
			\3 Formulación
				\4 Mercado laboral con características walrasianas
				\4[] Salario nominal W como variable de equilibrio
				\4[] $\to$ Se ajusta hasta vaciar mercado
				\4[] $\then$ W que iguala oferta y demanda
				\4 Ley de Say
				\4[] Sin excesos de demanda agregados
				\4[] Sólo existen dos mercados
				\4[] $\to$ Bienes
				\4[] $\to$ Trabajo
				\4[] Exceso de demanda $\neq 0$ en un mercado
				\4[] $\to$ Exceso de oferta correlativo en otro
				\4 Mercado de bienes
				\4[] Perfectamente competitivo
				\4[] Empresas enfrentan demanda perfectamente elástica
				\4[] $\to$ Ingreso marginal es igual a precio
				\4 Mercado de trabajo
				\4[] Perfectamente competitivo
				\4[] Salario real se ajusta para equilibrar
				\4[$\then$] Oferta de bienes determinada por trabajo de equilibrio
				\4 Demanda de trabajo -- empresas
				\4[] Maximizan beneficio demandando L dados:
				\4[] $\to$ Precio del bien P
				\4[] $\to$ Función de producción $Y(L)$ con $Y'(L)>0$, $Y''(L) <0$
				\4[] $\to$ Salario nominal $W$
				\4[] Producen hasta igualar IMg con CMg
				\4[] $\to$ Demandan L para igualar IMg con CMg
				\4[] $\then$ $P = \frac{W}{Y'(L)}$
				\4[] $\then$ $Y'(L) = \frac{W}{P}$
				\4[] $\then$ Igualan productividad marginal a salario real
				\4 Oferta de trabajo -- trabajadores
				\4[] Maximizan utilidad ofertando L dados:
				\4[] $\to$ Precio del bien
				\4[] $\to$ Función de utilidad / preferencias
				\4[] $\to$ Salario
				\4[] $\then$ Igualan RMS ocio-consumo a salario real
				\4[] $\frac{u_l}{u_c} = \frac{W}{P}$
				\4[] Asumiendo agregación de oferta de trabajo posible
				\4 Oferta agregada de bienes resulta de eq. en trabajo
				\4[] Salario nominal perfectamente flexible
				\4[] $\to$ Se ajusta para mantener salario real en equilibrio
				\4[] $\to$ Dada cualquier variación de precios
				\4[] $\then$ Varía para inducir salario real de equilibrio
				\4[] $\then$ Trabajo determina output ofertado
				\4[] Tipo de interés se ajusta en fondos prestables
				\4[] $\to$ Equilibra oferta y demanda agregadas
				\4[] $\to$ No hay insuficiencias de inversión
				\4[] $\then$ Excesos de oferta agregada desaparecen
				\4[] Nivel de precios variable varía con oferta monetaria
				\4[] $\to$ Oferta exógena
				\4[] $\to$ Demanda depende de oferta monetaria
				\4[] $\then$ Dinero causa nivel de precios
				\4 Representación gráfica
				\4[] \grafica{modeloclasico}
				\4 Shock positivo de demanda sobre bien producido
				\4[] Aumento de precios
				\4[] $\to$ Salario real cae
				\4[] $\then$ Aumenta demanda de L
				\4[] $\then$ Cae oferta de L
				\4[] Aparece exceso de demanda de trabajo
				\4[] $\to$ Aumento del salario nominal
				\4[] $\then$ Aumento del salario real
				\4[] Aumento del salario real
				\4[] $\to$ Aumento de oferta de L
				\4[] $\to$ Caída de demanda de L
				\4[] $\then$ Desaparece exceso de demanda
				\4[] Vuelta a equilibrio inicial
				\4[] $\to$ Mismo trabajo inicial
				\4[] $\to$ Mismo output inicial
				\4[] $\to$ Precio y salario nominal más altos
				\4[] De forma inversa para shock de demanda negativo
				\4 Macroeconomía es estable hacia eficiencia
				\4[] En ausencia de distorsiones e intervenciones
				\4[] Bajo supuestos generales:
				\4[] Output tiende a equilibrio óptimo
				\4[] Oferta agregada es vertical
				\4[] \grafica{oadaclasico}
				\4[$\then$] Trabajo intercambiado causa demanda de output
			\3 Implicaciones
				\4 Curva de Phillips
				\4[] Vertical
				\4[] Sin relación entre variables nominales y empleo
				\4[] $\to$ Salarios nominales
				\4[] $\to$ Nivel de precios
				\4 Dicotomía clásica
				\4[] Variables nominales no tienen efecto sobre reales
				\4[] Aumentos de precio del bien afectan a salario nominal
				\4[] Salario real inalterado
				\4[] $\to$ Determinado por factores reales
				\4 Salario real > equilibrio causa desempleo
				\4[] Exceso de oferta en mercado laboral
				\4[] Sindicatos y salario mínimo
				\4[] $\to$ Mantienen salario real por encima de ajuste
				\4[] $\then$ Exceso de oferta de trabajo
				\4[] $\then$ Desempleo
				\4[] Solución a desempleo:
				\4[] $\to$ Flexibilizar mercado de trabajo
				\4[] $\to$ Reducir poder de mercado de sindicatos
				\4[] $\to$ Reducir salario real hasta equilibrio
				\4 Equilibrio sin desempleo si salario real flexible
				\4[] Si salario real efectivamente flexible
				\4[] $\then$ Todo el trabajo ofertado se demanda
				\4[] $\then$ Toda la demanda de trabajo se cubre
				\4[] Dado salario de equilibrio
				\4[] $\to$ Todos los que quieren trabajar pueden hacerlo
				\4[] $\then$ No hay desempleo
				\4 Ajuste de salario nominal elimina exceso de oferta
				\4[] Exceso de demanda
				\4[] $\to$ Salario nominal cae
				\4[] $\then$ Salario real cae
				\4[] $\to$ Cae oferta de trabajo
				\4[] $\to$ Aumenta demanda de trabajo
				\4[] $\then$ Exceso de oferta desaparece
				\4[] $\then$ Sin desempleo
				\4 Estímulos de dda. en bienes sin efecto en trabajo
				\4[] Empresas perfectamente competitivas en bienes
				\4[] $\to$ Toman precio como dado
				\4[] $\then$ Más demanda de bienes aumenta precios
				\4[] Efecto sobre demanda de trabajo
				\4[] $\to$ Cae salario real
				\4[] $\then$ Aumenta trabajo para igual salario
				\4[] $\then$ Curva demanda de trabajo hacia la derecha
				\4[] Efecto sobre oferta de trabajo
				\4[] $\to$ Cae salario real
				\4[] $\to$ Cae RMS ocio consumo $\frac{u_l}{u_c}$
				\4[] $\then$ Cae trabajo para igual salario
				\4[] $\then$ Curva oferta de trabajo hacia la izquierda
				\4[] Efecto sobre equilibrio
				\4[] $\to$ Precios más altos
				\4[] $\to$ Salarios nominales más altos
				\4[] Representación gráfica
				\4[] \grafica{clasicoestimulodemandabienes}
				\4[$\then$] Mismo trabajo intercambiado
				\4[$\then$] Mismo output producido
				\4[$\then$] Más output demandado pero mismo producido
				\4[$\then$] Tipo de interés real sube y reduce inversión
				\4[$\then$] Crowding-out de la inversión
				\4[$\then$] Sin efecto en output
				\4 Subsidio a empleo sí puede ser efectivo
				\4[] Trabajo menos costoso para empresas
				\4[] Dispuestas a demandar más L dado salario real
				\4[] $\to$ Dda. de trabajo se desplaza hacia derecha
				\4[] $\to$ Oferta de trabajo no se ve alterada
				\4[] $\then$ Más trabajo intercambiado
				\4[] $\then$ Más output
				\4[] Problema
				\4[] $\to$ ¿Cómo financiar subsidio?
				\4 Shocks de oferta afectan a salario real
				\4[] Curva de oferta se desplaza a la derecha
				\4[] Con oferta creciente
				\4[] $\to$ Aumenta salario real
				\4[] $\to$ Aumenta empleo
				\4[] $\then$ Aumenta output
				\4[] $\then$ Salario real procíclico
				\4[] $\then$ Empleo procíclico
				\4[] Con oferta decreciente
				\4[] $\to$ Aumenta salario real
				\4[] $\to$ Cae empleo
				\4[] $\then$ Efecto ambiguo sobre output
				\4[] $\then$ Prociclicidad ambigua
				\4[] Ampliación en modelo del salario real
			\3 Valoración
				\4 Referencia de otros modelos
				\4[] Objeto de crítica en modelos keynesianos
				\4[] Referencia básica en modelos de NMC y similares
				\4 No explica desempleo satisfactoriamente
				\4[] Todos los que quieren trabajar pueden
				\4[] Trabajadores no ofertan más L a salario de eq.
				\4 Salario real volátil
				\4[] Asumiendo oferta de trabajo poco elástica
				\4[] $\to$ A partir de estimaciones microeconométricas
				\4[] $\then$ Salario real fuertemente procíclico
				\4[] Incompatible con numerosa evidencia empírica
				\4[] $\to$ Encuentra que salario real levemente procíclico
				\4[] Solución habitual:
				\4[] $\to$ Calibración de modelos de corte neoclásico
				\4[] $\then$ Utilizar elasticidad alta que replica series
		\2 Keynes
			\3 Idea clave
				\4 Contexto
				\4[] Desempleo generalizado
				\4[] Trabajadores no encuentran trabajo a W de empleados
				\4[] Exceso de oferta a nivel agregado posible
				\4[] $\to$ Mercado de dinero puede mostrar ED $\neq 0$
				\4[] $\to$ Baja utilización del capital y trabajo
				\4[] Salario nominal rígido a la baja empíricamente
				\4[] Mercados de bienes no necesariamente competitivos
				\4[] $\to$ Posibles estructuras de competencia imperfecta
				\4[] $\then$ Competencia monopolística
				\4[] $\then$ Empresas deciden cuanto producir
				\4[] $\then$ Empresas enfrentan curvas decrecientes
				\4 Objetivos
				\4[] Explicar desempleo involuntario
				\4[] Considerar rigidez nominal a la baja de salarios
				\4[] Proponer soluciones para reducir desempleo
				\4 Resultado
				\4[] Políticas de demanda son efectivas
				\4[] Salario real no se ajusta para eliminar desempleo
				\4[] Salario real inicialmente contracíclico
				\4[] Salario real pro-cíclico si mark-up contracíclico
			\3 Formulación
				\4 Original de Keynes (1936)
				\4[] Explicación verbal
				\4[] Representación matemática posterior
				\4[] $\to$ Hicks, Hansen, Patinkin y otros
				\4 Ley de Say no se cumple
				\4[] Posible exceso de demanda agregado en bienes y trabajo
				\4[] Aparición de tercer mercado: dinero
				\4[] $\to$ Demanda de dinero por especulación
				\4[] $\then$ Dinero ya no es token para transacciones
				\4[] Posibles excesos de oferta en bienes y trabajo
				\4[] $\to$ Compensados con ED>0 en dinero
				\4 Mercado de bienes no es competitivo
				\4[] Empresas enfrentan demandas decrecientes
				\4[] $\to$ Para cada empresa una demanda decreciente
				\4[] $\then$ Empresas deciden cuanto producir
				\4[] $\then$ Empresas deciden cuanto trabajo contratar
				\4 Características no walrasianas de mercado de trabajo
				\4[] Existe rigidez nominal a la baja
				\4[] Trabajadores no aceptan bajadas de salario nominal
				\4[] $\to$ Negociado en cada submercado de trabajo
				\4[] $\to$ Aceptar bajada implica menor ingreso relativo
				\4[] SÍ aceptan bajadas de salario real vía precios
				\4[] $\to$ Afectan a todos los trabajadores
				\4[] $\to$ No hay pérdida relativa de poder adquisitivo
				\4[] Aún si fuese posible bajar salario nominal
				\4[] $\to$ Caería demanda agregada
				\4[] $\to$ Caerían precios
				\4[] $\then$ Aumentaría salario real
				\4[] $\then$ Aumentaría desempleo
				\4[] $\then$ Posible tendencia inestable
				\4 Salario real inicialmente más alto que equilibrio
				\4[] Exceso de oferta de trabajo
				\4 Rigidez nominal impide corrección de EOferta
				\4[] Salario nominal no puede bajar
				\4[] $\to$ Salario real no puede caer
				\4 Representación gráfica
				\4[] \grafica{modelokeynesiano}
				\4 Mercado de bienes puede no ser competitivo
				\4[] Empresas tienen poder de mercado en su variedad
				\4[] $\to$ Coste marginal + mark-up
				\4[] $\then$ Tienen capacidad para fijar el nivel de precios
				\4[] Eligen cuanto producir
				\4[] $\to$ Eligen cuánto trabajo demandar
				\4[] Pagan salario nominal más alto que necesario
				\4[] $\to$ Para contratar esa cantidad de trabajo
				\4[] $\to$ Trabajadores ofrecen más trabajo
				\4[] $\then$ Exceso de oferta de trabajo
			\3 Implicaciones
				\4 Curva de Phillips
				\4[] Horizontal/creciente en exceso de capacidad
				\4[] $\to$ Estímulo de demanda aumenta output
				\4[] $\then$ Independientemente de precios y salarios
				\4[] Salarios nominales pueden
				\4[] Creciente hasta verticalidad
				\4[] $\to$ Aunque no considera relevante en contexto años 30
				\4 Estímulo de demanda de bienes efectivo con desempleo
				\4[] En la medida en que logre aumentar los precios
				\4[] Efecto sobre demanda
				\4[] $\to$ Aumenta coste de producción
				\4[] $\to$ Para mantener mark-up aumenta precio
				\4[] $\then$ Cae salario real
				\4[] $\then$ Curva demanda hacia la derecha
				\4[] Efecto sobre oferta
				\4[] $\to$ Aumento de precios
				\4[] $\then$ Cae salario real
				\4[] $\then$ Curva oferta hacia la izquierda
				\4[] Salario nominal permanece constante
				\4[] $\to$ Menos salario real
				\4[] $\then$ Más trabajo demandado
				\4[] $\then$ Menos trabajo ofrecido
				\4[$\then$] Reducción del exceso de oferta
				\4[$\then$] Reducción del desempleo involuntario
				\4 Estímulo de demanda inefectivo sin desempleo
				\4[] Precios aumentan
				\4[] $\to$ Salario real cae
				\4[] Salario nominal no es rígido al alza
				\4[] $\to$ Aumenta salario nominal
				\4[] $\then$ Salario nominal vuelve a aumentar
				\4 Salario real contracíclico sin markup
				\4[] Aumento de demanda de trabajo y output
				\4[] $\to$ Implica salario real más bajo
				\4[] $\then$ Caída del desempleo
				\4[] $\then$ Contraciclicidad del salario real
				\4[] Contrario a evidencia empírica
				\4[] $\to$ Generalmente,
				\4 Salario real procíclico con mark-up contracíclico
				\4[] Keynes admite poco después de Keynes (1936)
				\4[] Salario real puede ser procíclico
				\4[] $\to$ Con mark-up variable
				\4[] Demanda óptima de trabajo
				\4[] $\underset{L}{\max} \quad \Pi = P(Y(L))Y(L) -WL$
				\4[] $\text{CPO}: \quad F'(L) \cdot \underbrace{\left( 1 - \frac{1}{\left| \epsilon_{Y-P} \right|}\right)}_{\frac{1}{\mu(L)}} = \frac{W}{P}$
				\4[] $\then$ $F'(L) = \mu(L) \cdot \frac{W}{P}$
				\4[] Si $\mu'(L) < 0$:
				\4[] $\to$ Posible $\uparrow \frac{W}{P}$ y $\downarrow F'(L)$
				\4[] $\then$ Posible salario real procíclico
				\4 Macroeconomía tiene varios equilibrios
				\4[] Oferta agregada es creciente
				\4[] $\to$ Hasta límite de capacidad productiva
				\4[] $\to$ Límite corresponde con eq. del mercado de trabajo
				\4[] \grafica{oadakeynes}
				\4[$\then$] Demanda de output causa trabajo intercambiado
			\3 Valoración
				\4 Enorme impacto en la práctica
				\4[] Estímulos de demanda son herramienta habitual
				\4[] Políticas de estabilización vía estímulo de dda.
				\4[] $\to$ Para tratar de aliviar desempleo
				\4 Evidencia empírica contraria
				\4[] Salarios reales no parecen contracíclicos
				\4[] Múltiple evidencia empírica contraria
				\4[] Posible compatibilizar añadiendo mark-up cíclico
				\4 Mark-up cíclico
				\4[] Keynes entiende modelo de Tª General como ilustrativo
				\4[] $\to$ Propone hacia 1939 otra interpretación
				\4[] Representable con mark-up anticlíco
				\4[] Empresas aplican un mark-up que depende de L
				\4[] $\to$ $\mu(L)$
				\4[] $\then$ $\mu'(L) > 0$: mark-up procíclico
				\4[] $\then$ $\mu'(L) < 0$: mark-up contracíclico
				\4[] Condición de óptimo de empresas cambia:
				\4[] $\to$ $\frac{W}{P} = F'(L)$ $\to$ $\frac{W}{P} = \frac{F'(L)}{\mu(L)}$
				\4[] Si $\mu(L)$ suficientemente contracíclico
				\4[] $\then$ Salario real puede ser procíclico
				\4[] Germen de modelos posteriores:
				\4[] $\to$ Competencia imperfecta
				\4[] $\to$ Mark-ups variables
				\4[] $\to$ Rigideces nominales en precios relevantes
		\2 Síntesis neoclásica
			\3 Idea clave
				\4 Contexto
				\4[] Fuerte impacto de Teoría General
				\4[] Políticas de demanda se popularizan
				\4[] Aumento del desempleo tras IIGM
				\4 Objetivo
				\4[] Establecer vínculo entre Keynes y neoclásico
				\4[] Entender relación entre paro y salarios
				\4 Resultados
				\4[] Marco de modelización general
				\4[] $\to$ Permite incorporar clásico y keynesiano
				\4[] Mercado de trabajo como pieza clave
			\3 Formulación
				\4 Bloque de oferta: mercado de trabajo/AS
				\4[] Oferta de trabajo
				\4[] $\to$ Maximización ocio-consumo + agregación
				\4[] Demanda de trabajo
				\4[] $\to$ Maximización del beneficio de empresas + agregación
				\4 Bloque de demanda: IS-LM/AD
				\4[] IS: $Y = C_0 + cY + I(r)$
				\4[] LM: $\frac{M}{P} = L(r,Y)$
				\4[] Trabajo: $\frac{F'(L)}{\mu(L)} = \frac{W}{P} = \frac{u_l}{u_c}$
				\4 Supuestos keynesianos
				\4[] Relevantes en el corto plazo
				\4[] Precios (P) rígidos
				\4[] Salario nominal rígido a la baja
				\4[] Demanda de bienes determina demanda de trabajo
				\4[] $\to$ Salario real no se ajusta a equilibrio
				\4[] $\then$ Salario real por encima de equilibrio
				\4[] Aumento de demanda y precios en bienes
				\4[] $\to$ Aumento de demanda de trabajo
				\4[] $\then$ Aumento de salarios a partir de cierto punto
				\4[] $\then$ Caída del desempleo
				\4[] $\then$ Relación decreciente salarios-desempleo
				\4[$\then$] Exceso de oferta en mercado laboral
				\4[$\then$] Exceso de capacidad
				\4[$\then$] Políticas de demanda sí son efectivas
				\4 Supuestos neoclásicos
				\4[] Relevantes en el largo plazo
				\4[] Precios y salarios flexibles
				\4[] Desempleo presiona salario nominal a la baja
				\4[] Aumentos de demanda presionan precios al alza
				\4[] $\to$ Aumento del empleo
				\4[] $\then$ Reducción del paro involuntario
				\4 Curva de Phillips
				\4[] Phillips (1956) + Lipsey (1960) + SySolow (1960)
				\4[] $\to$ Empíricamente, relación $\dot{w}-u$ decreciente
			\3 Implicaciones
				\4 Curva de Phillips
				\4[] Supuestos keynesianos
				\4[] $\to$ Decreciente en desempleo
				\4[] $\then$ Dicotomía clásica no se cumple
				\4[] Supuestos
				\4[] $\to$ Tiende a línea recta
				\4[] $\then$ Dicotomía clásica tiende a cumplirse en largo plazo
				\4 Modelo general con dos submodelos del mercado de trabajo
				\4 Representación gráfica
				\4[] Modelo con supuestos clásicos
				\4[] \grafica{islmadasclasico}
				\4[] Modelo con supuestos keynesianos
				\4[] \grafica{islmadaskeynesiano}
				\4 Desempleo involuntario posible en corto plazo
				\4[] Problemas de ajuste en mercado de trabajo
				\4 Sin desempleo involuntario en largo plazo
				\4[] Desempleo friccional y estructural posibles
			\3 Valoración
				\4 Marco básico de formulación de políticas
				\4 Sin fundamentar rigidez del mercado laboral
		\2 Curva de Phillips
			\3 Idea clave
				\4 Contexto
				\4[] Síntesis neoclásica
				\4[] $\to$ Salarios flexibles en el largo plazo
				\4[] Necesaria evolución compatible con modelo
				\4[] $\to$ Desempleo presiona salario nominal a la baja
				\4 Objetivo
				\4[] Probar empíricamente relación:
				\4[] $\to$ salario-desempleo
				\4[] $\to$ precios-output
				\4[] Incorporar soporte empírico a SNC
				\4[] $\to$ Mercado laboral funciona como previsto
				\4 Resultado
				\4[] Phillips (1957) en Econometrica
				\4[] Relación empírica entre positiva entre:
				\4[] $\to$ Inflación salarial
				\4[] $\to$ Reducción del desempleo
			\3 Formulación
				\4 Phillips (1958)\footnote{Ver ``Phillips curve'' en Palgrave.}
				\4[] \fbox{$\frac{\dot{W}}{W}= f(U_t)$, $f_U(U_t) < 0$}
				\4[] Relación estimada:
				\4[] $\to$ Inflación salarial nula con paro del $5,5\%$
				\4[] $\to$ Inflación salarial del $2,5\%$ con paro del $2\%$
				\4[] Variación del salario nominal
				\4[] $\to$ Función decreciente de nivel de paro
				\4[] (a igual tasa de variación del paro)
				\4[] Nivel de paro compatible con inflación media
				\4[] $\to$ Es positivo
				\4[] $\to$ Sin grandes cambios en periodo de observación
				\4[] Variación del salario nominal
				\4[] $\to$ Función creciente de velocidad de caída de paro
				\4[] (a igual tasa de paro)
				\4 Lipsey (1960)
				\4[] Análisis cuantitativo más completo
				\4[] $\to$ Regresiones econométricas
				\4[] Tasa de variación del salario
				\4[] $\to$ Variable a explicar
				\4[] Nivel de empleo
				\4[] $\to$ Explicativa
				\4[] Tasa de variación del desempleo
				\4[] $\to$ Explicativa
				\4[] Encuentra coeficientes negativos
				\4[] Racionaliza en términos de demand-pull
				\4[] $\to$ EDemanda en diferentes sectores
				\4[] $\then$ Empujar precios hacia arriba
				\4[] Relaciona con inflación general
				\4[] $\to$ Afirma relación similar inflación-output
				\4 Samuelson y Solow (1960)
				\4[] Hay una relación estructural explotable
				\4[] $\to$ Entre desempleo e inflación
				\4 Representación gráfica
				\4[] \grafica{curvadephillips}
			\3 Implicaciones
				\4 Evidencia favorable a modelo keynesiano
				\4[] Desempleo y caída de salarios
				\4[] $\to$ Relación consistente
				\4[] Compatible con:
				\4[] $\to$ Exceso de oferta de trabajo
				\4[] $\then$ Caída de salarios
				\4[] $\to$ Exceso de oferta de bienes
				\4[] $\then$ Caída de precios
				\4 Pieza que completa modelo neoclásico
				\4[] En el largo plazo, salarios/precios se ajustan
				\4[] Salario nominal cae si hay paro
				\4[] Aumento de precios reduce desempleo
				\4 Desempleo positivo con inflación constante
				\4[] $\exists$ paro con salarios creciendo a tasa constante
				\4[] $\to$ ¿Por qué?
				\4[] $\to$ ¿Qué fricciones permiten explicar?
				\4 Menú de política económica
				\4[] Samuelson y Solow (1960)
				\4[] Mercado laboral permite elegir posición en curva
				\4[] $\to$ Policy-makers deciden posición inflación-output
			\3 Valoración
				\4 Concepto central de mercado laboral
				\4 Objeto de críticas sucesivas de diferentes escuelas
				\4 Nuevos interrogantes
				\4[] Paro positivo con inflación estable
				\4[] $\to$ ¿Qué causa?
				\4[] ¿Es realmente explotable estructuralmente?
				\4[] $\to$ ¿Inflación de equilibrio no aumenta tras estímulo?
		\2 Monetarismo
			\3 Idea clave
				\4 Contexto
				\4[] Curva de Phillips considerada explotable
				\4[] $\to$ Policy-makers deciden combinación paro-inflación
				\4[] Análisis de Chicago y austriacos
				\4[] $\to$ No es posible explotar relación en l/p
				\4 Objetivo
				\4[] Considerar respuesta de trabajadores
				\4[] $\to$ A estímulos inflacionarios
				\4[] Explicar desempleo de equilibrio
				\4[] Valorar respuesta frente a salario real
				\4[] Considerar papel de expectativas
				\4 Resultados
				\4[] Relación no es explotable estructuralmente
				\4[] Trabajadores reconsideran salario real
			\3 Formulación
				\4 Oferta de trabajo
				\4[] Trabajadores deciden respecto a salario real
				\4[] $\to$ No nominal
				\4[] $\to$ No tienen ilusión monetaria
				\4[] Pero no conocen nivel de precios en presente
				\4[] $\to$ Deben plantear una estimación
				\4[] $\then$ Utilizan expectativas adaptativas
				\4[] A corto plazo, estímulo de demanda es efectivo
				\4[] $\to$ Aumenta precio por encima de estimación
				\4[] $\to$ Salario real efectivo por debajo de estimación
				\4[] $\then$ Empresas demandan más trabajo
				\4[] $\then$ Trabajadores aumentan oferta de trabajo
				\4[] $\then$ Aumento del output y el empleo
				\4 Curva de Phillips aumentada por las expectativas
				\4[] \fbox{$\frac{\dot{W}}{W} = f(\bar{U} - U_t) + \pi_t^e$}
				\4 Expectativas adaptativas
				\4[] Introducidas formalmente por Cagan (1956)
				\4[] $\to$ Estudio de hiperinflación
				\4[] Agentes estiman inflación futura a partir de:
				\4[] $\to$ Estimación para periodo pasado
				\4[] $\to$ Desviación entre realidad y estimación pasada
				\4[] \fbox{$E_{t}(P_{t+1}) = E_{t-1} (P_t) + \lambda \left( P_t - E_{t-1} (P_t) \right)$}
				\4 Tasa natural de paro
				\4[] Tasa de paro
				\4[] $\to$ Cuando inflación esperada y real coinciden
				\4[] $\to$ Resultado de sistema walrasiano de ecs. estructurales
				\4[] Con tasa de paro natural:
				\4[] $\to$ Output de largo plazo
				\4[] No es inmutable ni constante
				\4[] $\to$ Políticas de oferta pueden afectar
				\4[] $\to$ Políticas de demanda no alteran
				\4 Representación gráfica
				\4[] \grafica{curvadephillipsmonetarista}
			\3 Implicaciones
				\4 Expectativas son importantes
				\4[] Trabajadores reestiman inflación futura
				\4[] Política económica presente afecta a expectativas
				\4 Estímulos de demanda aumentan expectativa de inflación
				\4[] Para lograr mismo efecto
				\4[] $\to$ Cada vez es necesario mayor estímulo
				\4[] $\then$ Cada vez mayor inflación
				\4[] Dificil estimar efecto sobre inflación
				\4[] $\to$ Lags
				\4[] $\to$ Problemas de estimación
				\4[] $\then$ Mejor no utilizar
				\4[] $\then$ Preferible mantener inflación baja
				\4 Curva de Phillips no es explotable en l/p
				\4[] En el largo plazo, CPhillips vertical
				\4[] Aumento de expectativa de inflación
				\4[] $\to$ Reduce efectividad de mismo estímulo
			\3 Valoración
				\4 Inflación salarial en EEUU
				\4[] Aumenta a finales de 60s y 70s
				\4[] Compatible con desempleo elevado
				\4 Influencia general en PEconómica a partir de 70s
				\4[] Estímulos de demanda inefectivos
				\4[] Políticas de oferta preferibles
				\4[] $\to$ Reducir tasa natural de paro
		\2 Neokeynesianos del desequilibrio
			\3 Idea clave
				\4 Contexto
				\4[] Modelo keynesiano en términos verbales
				\4[] $\to$ Supuestos poco explícitos
				\4[] $\to$ características no walrasianas sin justificar
				\4[] Difusión de marco walrasiano en 50s
				\4 Objetivo
				\4[] Explicitar supuestos keynesianos
				\4[] Fundamentar problemas de ajuste del mercado laboral
				\4 Resultados
				\4[] Microfundamentación de Keynes
				\4[] Análisis microeconómico del desequilibrio
				\4[] Modelo keynesiano en ``lenguaje'' walrasiano
			\3 Leijonhufvud
				\4 Economía keynesiana y la economía de Keynes (1968)
				\4 Rechazo de IS-LM
				\4[] IS-LM malinterpreta mensaje central de Keynesd
				\4 Problemas de información y coordinación
				\4[] Entre trabajadores y empresas
				\4 Economía descentralizada
				\4[] $\to$ Problemas de información
				\4[] $\to$ Problemas de señalización
				\4[] $\then$ Coordinación subóptima entre agentes
				\4[] $\then$ Mercado de trabajo susceptible descoordinación
				\4 Contraste con Patinkin
				\4[] Ajuste lento de precios no basta para desempleo
				\4[] Keynes no es walrasiano
				\4[] $\to$ Keynes es énfasis sobre proceso de ajuste
				\4[] $\to$ No sobre existencia del equilibrio
			\3 Clower: hipótesis de la decisión dual
				\4 La contrarrevolución keynesiana: un examen teórico (1965)
				\4 Demanda nocional
				\4[] Demanda que tendría lugar
				\4[] $\to$ Si agentes pudiesen demandar cantidades óptimas
				\4 Demanda efectiva
				\4[] Demanda que tiene lugar
				\4[] $\to$ Dada renta tras intercambio efectivo
				\4[] $\then$ Que puede ser diferente al de óptimo
				\4 Ajuste de EDemanda en MLaboral
				\4[] Se realiza en relación a demanda efectiva
				\4[] $\to$ No a demanda nocional
				\4 Representación gráfica
				\4[] \grafica{coordinacionclower}
			\3 Barro y Grossman, Malinvaud
				\4 Caracterización de tipos de desempleo
				\4[] Como EDemanda y EOferta en lenguaje walrasiano
				\4 Paro clásico
				\4[] Mercado de bienes
				\4[] $\to$ Exceso de demanda
				\4[] Mercado de trabajo
				\4[] $\to$ Exceso de oferta
				\4[] Mercado de dinero
				\4[] $\to$ En equilibrio por definición
				\4 Paro keynesiano
				\4[] Mercado de bienes
				\4[] $\to$ Exceso de oferta
				\4[] Mercado de trabajo
				\4[] $\to$ Exceso de oferta
				\4[] Mercado de dinero
				\4[] $\to$ Exceso de demanda
				\4[] $\to$ DDinero por motivo intercambio+especulación
				\4 Inflación reprimida
				\4[] Mercado de bienes
				\4[] $\to$ Exceso de demanda
				\4[] Mercado trabajo
				\4[] $\to$ Exceso de demanda
				\4[] Mercado de dinero
				\4[] $\to$ Exceso de oferta
				\4[] $\then$ DDinero por motivo intercambio+especulación
				\4 Subempleo
				\4[] Mercado de bienes
				\4[] $\to$ Exceso de oferta
				\4[] Mercado de trabajo
				\4[] $\to$ Exceso de demanda
				\4 Representación gráfica
				\4[] \grafica{desequilibrios}
		\2 Nueva Macroeconomía Clásica
			\3 Idea clave
				\4 Contexto
				\4[] Estanflación en años 70
				\4[] $\to$ Paro e inflación al mismo tiempo
				\4[] $\to$ Ruptura de relación estructural
				\4[] Muth (1961)
				\4[] $\to$ Expectativas racionales
				\4[] Microfundamentación de la macroeconomía
				\4[] $\to$ Introducida en años 60
				\4[] $\to$ Marco walrasiano de modelización
				\4 Objetivo
				\4[] Incorporar HER
				\4[] $\to$ Empleo eficiente de la información
				\4[] Valorar efectos sobre empleo
				\4 Resultados
				\4[] Modelos incrementales del mercado de trabajo
				\4[] Lucas y Rapping (1969)
				\4[] $\to$ Primera microfundamentación
				\4[] $\to$ Primer análisis dinámico de mercado laboral macro
				\4[] Lucas (1972) y (1973)
				\4[] $\to$ CPhillips microfundamentada con inf. imperfecta
				\4[] $\to$ Relación no explotable por gobierno
				\4[] Kydland y Prescott (1977), Barro y Gordon (1983)
				\4[] $\to$ Modelización de incentivos de juego gobierno-trabajo
				\4[] $\then$ No explotabilidad como ENPS
				\4[$\then$] Curva de Phillips $\pi_t$--$u_t$ decreciente en c/p
				\4[$\then$] Curva de Phillips vertical en l/p
				\4[] Curva de Phillips vertical en l/p
				\4[] Estímulos de demanda inefectivos
				\4[] $\to$ Sólo en la medida que sorprendan a trabajadores
				\4[] Mercado de trabajo siempre en equilibrio
				\4[] Desempleo es óptimo de Pareto
				\4[] Poco énfasis en mercado laboral
			\3 Formulación
				\4 Demanda agregada
				\4[] Formulación muy simplificada
				\4[] $m_t - p_t = ky_t - k i_t$
				\4 Oferta de trabajo
				\4[] Trabajadores estiman salario real con HER mediante:
				\4[] $\to$ Salario nominal actual
				\4[] $\to$ Modelo subyacente del nivel de precios
				\4[] $\to$ Conocen incentivos de policy-makers
				\4[] $\then$ Aprovechan toda la información disponible
				\4[] $\then$ Sólo posibles errores con media cero
				\4[] $\then$ No cometen errores sistemáticos de estimación
				\4[] Ofertan trabajo considerando salario real
				\4[] $\to$ A partir de estimación de inflación
				\4[] Si inflación más alta que esperado
				\4[] $\to$ Salario real más bajo que esperado
				\4[] $\to$ Oferta de trabajo no se reduce
				\4[] $\to$ Demanda de trabajo aumenta
				\4[] $\then$ Estímulo es efectivo
				\4[] Si estímulos de demanda son habituales
				\4[] $\to$ Trabajadores asumen que inflación es volátil
				\4[] $\to$ $\uparrow$ de salario nominal implica $\uparrow$ P
				\4[] $\then$ Entienden aumento SNominal no es $\uparrow$ SReal
				\4[] $\then$ No ofertan más trabajo
				\4[] $\then$ No hay aumento del output
				\4 Curva de Phillips en términos de output natural
				\4[] \fbox{$\tilde{y}_t = y_t - y_t^n = \lambda \left( P_t - E(P_t) \right)$}
				\4 Representación gráfica
				\4[] \grafica{nmckypphillips}
			\3 Implicaciones
				\4 Curva de Phillips
				\4[] Relación creciente precios-output
				\4[] No explotable por gobierno
				\4[] Agentes estiman incentivos del gobierno
				\4[] $\to$ Predicen aumento de precios
				\4[] $\then$ $\uparrow$ precios no tiene efecto sobre mercado trabajo
				\4 Estímulos de demanda sistemáticos no afectan paro
				\4[] Sólo efectivos en plazo inmediato
				\4[] $\to$ Si ``sorprenden'' a agentes
				\4 Mercado de trabajo siempre en equilibrio
				\4[] Sin transiciones a desequilibrio
				\4[] Senda temporal de optimización
				\4 Desempleo involuntario no existe
				\4[] Fricciones y estructura son causas
				\4[] $\to$ Necesario mejorar matching
				\4[] $\to$ Adaptar oferta a demanda
			\3 Valoración
				\4 Marco de modelización predominante
				\4[] Microfundamentación+HER+agentes racionales
				\4[] Equilibrio general walrasiano
				\4 Políticas de demanda en desuso
				\4[] Énfasis en liberalización y flexibilización
				\4[] $\to$ Oferta es relevante aumentar renta
				\4 Base de modelo RBC
				\4[] Fluctuaciones del empleo
				\4[] $\to$ Resultado de shocks de productividad
		\2 RBC -- Real Business Cycle
			\3 Idea clave
				\4 Modelos de NMC
				\4[] $\to$ Macroeconomía es equilibrio general walrasiano
				\4[] Crítica de Lucas
				\4[] $\to$ Microfundamentación para tratar de evitar
				\4[] Modelo neoclásico de crecimiento
				\4[] $\to$ Referencia básica
				\4 Autores
				\4[] Kydland y Prescott (1982)
				\4[] Long y Plosser (1983)
				\4[] Otros nombres:
				\4[] $\to$  King, Rebelo, Benhabib, ...
				\4 Objetivos
				\4 Resultados
			\3 Formulación
				\4 Economía como equilibrio general walrasiano
				\4 Agente representativo maximiza utilidad
				\4[] $\to$ Ocio y consumo a lo largo de tiempo
				\4[] $\to$ Dado tiempo disponible
				\4[] $\to$ Dados impuestos de suma fija fijados por gobierno
				\4[] $\then$ Cuánto consumir en cada periodo
				\4[] $\then$ Cuánto trabajo en cada periodo
				\4[] $\then$ Cuánto invertir en capital en cada periodo
				\4[] $\then$ Cuánto invertir en bonos en cada periodo
				\4[] $\underset{C_t, N_t, B_{t+1}} \quad E_0 \sum_{t=0}^\infty \beta^t \left( u(C_t)+v(1-N_t) \right)$
				\4 Empresas maximizan beneficio
				\4[] $\underset{L_t, K_t}{\Pi_t} \quad A_t K_t^\alpha L_t^{1-\alpha} - W_t L_t - R_t K_t$
				\4[] $\then$ $W_t = (1-\alpha) A_t K_t^\alpha L_t^{-\alpha}$
				\4[] $\then$ $R_t = \alpha A_t K_t^{\alpha -1} L_t^{1-\alpha}$
				\4[] Dados:
				\4[] $\to$ Capital
				\4[] $\to$ Trabajo
				\4[] $\to$ Shocks tecnológicos
				\4 Gobierno
				\4[] $G_t - r_{t-1} D_{t-1} \leq T_t + D_{t+1} - D_t$
				\4[] $\then$ $G_t \leq T_t + D_t - (1+r_{t-1}) D_{t-1}$
				\4[] Determina gasto público sujeto a restricción
				\4[] Gasto público asumido no productivo
				\4[] $\to$ Sólo detrae de consumo disponible
				\4[] Financiado con:
				\4[] $\to$ Impuestos de suma fija
				\4[] $\to$ Emisión de deuda pública
				\4 Equilibrio de estado estacionario
				\4[] $C_t + I_t + G_t = Y_t$
				\4[] $I_t = K_{t+1} - (1-\delta) K_t$
				\4[] $Y_t = A_t K_t^\alpha L_t^{1-\alpha}$
				\4[] $A_t = A_{t-1}^\rho + \epsilon_t^A$
				\4[] $G_t= (1-\rho_G) \bar{G} + \rho_G \log G_{t-1} + \epsilon_t^G$
				\4 Equivalencia ricardiana
				\4[] Resultado general en modelos de RBC
				\4[] Irrelevante como se financie el gasto público
				\4[] Volumen de gasto público sí tiene efectos reales
			\3 Implicaciones
				\4 Shock productividad
				\4[] Transitorio:
				\4[] $\to$ $\left| \text{ES} + \text{ER}_i\right|$: elevado
				\4[] $\to$ $\text{ER}_D$: muy pequeño
				\4[] $\then$ Más probable que $\frac{d \, h}{d \, w} > 0$
				\4[] $\then$ Aumento fuerte de oferta de trabajo
				\4[] Permanente:
				\4[] $\to$ $\left| \text{ES} + \text{ER}_i\right|$: elevado
				\4[] $\to$ $\text{ER}_D$: elevado
				\4[] $\then$ ER directo compensa ES+ER indirecto
				\4[] $\then$ Menor aumento de oferta de trabajo
				\4 Shock gasto público
				\4[] Transitorio:
				\4[] $\to$ $\left| \text{ES} + \text{ER}_i\right|$: sin efecto
				\4[] $\to$ $\text{ER}_D$: muy pequeño
				\4[] $\then$ Más probable que $\frac{d \, h}{d \, w} > 0$
				\4[] $\then$ Aumento fuerte de oferta de trabajo
				\4[] Permanente:
				\4[] $\to$ $\left| \text{ES} + \text{ER}_i\right|$: sin efecto
				\4[] $\to$ $\text{ER}_D$: creciente cuanto más permanente
				\4[] $\then$ ER directo compensa ES+ER indirecto
				\4[] $\then$ Menor aumento de oferta de trabajo
				\4 Tabla resumen
				\4[] \grafica{resumenefectostrabajo}
				\4 Puzzle micro-macro de la ESI del trabajo
				\4[] Replicación de series reales implica
				\4[] $\to$ Muy elevada ESI del trabajo
				\4[] $\then$ Fluctuaciones del empleo muy grandes
				\4[] $\then$ Salarios reales solo levemente procíclicos
				\4[] En la práctica, a nivel micro
				\4[] $\to$ Muy baja elasticidad intertemporal del trabajo
				\4[] $\then$ Agentes no dispuestos a trabajar mucho más si $\uparrow w$
				\4[] $\then$ Sólo ligero aumento
				\4[] Múltiples intentos de fundamentar resultados
				\4[] $\to$ Margen extensivo e intensivo
			\3 Valoración
				\4 Cómo valorar capacidad de replicación
				\4[1] Calibración del modelo
				\4[] Elegir valores de parámetros en base a:
				\4[] $\to$ Teoría microecómica
				\4[] $\to$ Estimaciones microeconómicas
				\4[] $\to$ Teoría macro
				\4[] ...
				\4[2] Estimar estado estacionario
				\4[3] Introducir shocks
				\4[] Shocks de productividad: residuos de Solow
				\4[] Gasto público: ajustes estructurales
				\4[4] Comparar con series reales
				\4[] Los momentos de las series son similares?
				\4 Resultados habituales con modelos básicos de RBC
				\4[] Modelos más complejos mejoran resultados
				\4[] Introducen mayor complejidad y sup. ad-hoc
				\4 Buena replicación de:
				\4[] Primer momento de Y, C, I, N, w
				\4[] Volatilidades relativas del consumo e I
				\4[] $\to$ C mucho menos volátil que Y
				\4[] $\to$ I mucho más volátil que Y
				\4 Mala replicación
				\4[] Correlación entre trabajo y productividad
				\4[] RBC predice alta correlación
				\4[] $\uparrow$ $\frac{Y}{L}$ aumenta mucho horas trabajadas
				\4[] Pero en realidad, $W$ muy débilmente procíclico
				\4[] Reacción sobre todo en margen extensivo
				\4[] $\to$ Más que en horas de trabajo (intensivo)
				\4 Sectores múltiples y mercado de trabajo
				\4[] Efectos de transmisión de shocks entre sectores
				\4[] Estructura de mercados determina respuesta a shocks
				\4 Mercado de trabajo
				\4[] Rogerson (1984), Hansen (1985)
				\4[] Trabajo indivisible
				\4[] $\to$ Cambios en trabajo no son sólo cambios en horas
				\4[] $\then$ Sobre todo, cambios en número de empleados
				\4[] Incorporar respuesta de trabajo a shocks
				\4[] $\to$ Shocks implica variación más fuerte de trabajo
				\4[] $\Rightarrow$ Baja respuesta de horas trabajadas a shock
				\4[] $\Rightarrow$ Mejoran replicación de series reales
		\2 Nueva Economía Keynesiana--Primera generación
			\3 Idea clave
				\4 Contexto
				\4[] Keynesianismo pierde seguimiento
				\4[] Neokeynesianismo del desequilibrio
				\4[] $\to$ Introduce microfundamentación formal
				\4[] $\to$ Énfasis en desequilibrios persistentes
				\4[] NMC predominante en 70s y 80s
				\4[] $\to$ Consolidación de marco walrasiano en macro
				\4[] $\to$ Mercados en constante equilibrio
				\4[] $\to$ Desempleo no causado por rigideces
				\4[] $\to$ Negación de keynesianismo
				\4 Objetivos
				\4[] Microfundamentar rigideces nominales y reales
				\4[] $\to$ Que inducen resultados keynesianos
				\4[] $\then$ Exceso de capacidad persistente
				\4[] $\then$ Desempleo involuntario
				\4[] Explicar curva de Phillips empírica
				\4[] Explicar curva de Phillips de años 70
				\4[] $\to$ Inflación acelerada
				\4[] $\to$ Curva de Phillips incluso creciente
				\4[] Explicar desempleo de equilibrio
				\4[] Justificar políticas keynesianas
				\4 Resultados
				\4[] Inicialmente, modelos ad-hoc
				\4[] Equilibrio parcial
				\4[] Fundamentaciones de curva de Phillips
				\4[] $\to$ Sin modelo de equilibrio general
			\3 Contratos implícitos
				\4 Empresa:
				\4[] neutral al riesgo
				\4[] Incentivos a mantener relación contractual
				\4[] $\to$ Inversiones en capital humano
				\4[] $\to$ Costes fijos incurridos
				\4 Trabajadores
				\4[] Aversos al riesgo
				\4[] Quieren mantener flujo constante de consumo
				\4 Salario suma de dos componentes
				\4[] $\to$ Productividad marginal del trabajo
				\4[] $\to$ Prima o indemnización según estado de naturaleza
				\4[$\Rightarrow$] $\bar{w} = \text{PMgL} + \gamma$
				\4[] $\bar{w}$: fijo
				\4[] $\gamma$: positivo si $\bar{w} > \text{PMgL}$
				\4[$\then$] Aseguramiento frente a fluctuaciones
				\4[$\then$] Salario se mantiene rígido
				\4 Estado de la naturaleza adverso
				\4[] Empresa paga:
				\4[] $\to$ Productividad marginal del trabajo
				\4[] $\to$ Indemnización para cubrir diferencia
				\4[] Si coste de indemnizaciones elevado y despido bajo
				\4[] Empleado paga:
				\4[] $\to$ Prima de aseguramiento
				\4[] $\to$ Empresa despide empleados
				\4 Estado de la naturaleza favorable
				\4[] Empresa paga:
				\4[] $\to$ Productividad marginal del trabajo
				\4[] Empleado paga:
				\4[] $\to$ Prima de aseguramiento
				\4[] Salario real
				\4[] $\to$ Se mantiene constante
				\4 Ingresos reales constantes
				\4[]  Modelo muestra salario real rígido
				\4[]  Salario acíclico
				\4 Shocks de productividad positivos
				\4[] Aumenta productividad y output
				\4[] $\to$ Salario se mantiene constante
				\4[] $\to$ Aumenta trabajo
				\4[] $\Rightarrow$ Salario sin tendencia cíclica
				\4 Contradice hecho empírico
				\4[] Salario real levemente procíclico
				\4 Despidos concentrados
				\4 ¿Por qué empresa despide?
				\4[] Si productividad baja mucho
				\4[] $\to$ Carga financiera de salarios muy alta
				\4 Empresas transfieren carga despidiendo
				\4[] $\to$ Subsidios por desempleo toman relevo
				\4[] Si existen costes fijos de despido
				\4[] $\to$ Despidos concentrados
				\4 Despidos como prima de aseguramiento
				\4[] Empleados aceptan probabilidad de ser despedidos
				\4[] $\to$ Como pago por evitar reducción de $w$
				\4 Volatilidad del empleo aumentada
				\4 Factores que aumentan volatilidad
				\4[] Presencia de subsidio de desempleo
				\4[] $\to$ Empresas tienen incentivo a desviar hacia paro
				\4[] Empleados pueden exigir más seguridad salarial
				\4[] $\to$ Trabajadores se aseguran frente a desempleo
				\4[] $\to$ Aceptan salario menor a equilibrio
				\4[] $\to$ Empresas pagan salario > equilibrio con dda. baja
				\4[] $\then$ Salario real no se ajusta para vaciar mercado
			\3 Insiders y outsiders
				\4[] $\to$ Insiders negocian
				\4[] $\to$ Outsiders fuera del mercado de trabajo
				\4[] $\then$ Insiders maximizan su utilidad
				\4[] $\then$ Outsiders sufren desempleo involuntario
			\3 Salarios de eficiencia
				\4[] $\to$ W real de equilibrio costoso para empresas
				\4[] $\to$ Induce menor esfuerzo en trabajadores
				\4[] $\then$ Empresas mantienen salario real > equilibrio
				\4[] $\then$ Mercado laboral no
			\3 Modelo de negociación salarial
				\4 Layard y Nickell (1985), Carlin y Soskice (1990)
				\4[] $\to$ Trabajadores fuerzan mayor salario si menos paro
				\4[] $\then$ SRN -- Salario Real Negociado
				\4[] $\to$ Empresas tienen poder de mercado en bienes
				\4[] $\then$ SRP -- Salario Real Pagado
				\4[] $\to$ Suben precios si aumenta coste laboral
				\4[] $\to$ Si demandas salariales incompatibles con precio
				\4[] $\then$ Inflación se acelera
				\4[] $\then$ NAIRU es tasa de paro que compatibiliza
			\3 Fallos de coordinación
				\4 Idea clave
				\4[] Keynes:
				\4[] $\to$ Muy difícil para trabajo coordinar bajada de salarios
				\4[] $\then$ Más fácil aumentar oferta monetaria $\to$ Inflación
				\4[] Agentes no pueden coordinar decisiones
				\4[] $\to$ Equilibrios subóptimos son posibles
				\4[] $\then$ Desempleo como resultado
				\4 Diamond (1982), Robert (1987), Howitt
				\4[] Inspirado en Leijonhufvud, Clower, Patinkin..
				\4 Modelo de los cocos de Diamond
				\4[] Metáfora con cocos à la islas de Phelps
				\4[] Agentes viven en economía cerrada
				\4[] Pueden producir bienes recogiendo cocos de arboles
				\4[] Tabú impide consumir cocos que uno mismo recoge
				\4[] $\to$ Debe intercambiar con otro agente
				\4[] Para que un agente recoja cocos
				\4[] $\to$ Debe tener expectativa de que otros también
				\4[] $\then$ Debe creer que podrá intercambiarlos con otro
				\4[] Sin expectativa de intercambiar
				\4[] $\to$ Nadie tendrá incentivo a producir
				\4[] $\then$ Posibles múltiples equilibrios
				\4[] $\then$ Posible capacidad sin utilizar
				\4 Cooper y John (1988)
				\4[] Abandonan idea de rigideces nominales
				\4[] $\to$ Inicialmente entendido como alternativa a NEK 1aGEN
				\4[] $\then$ Posteriormente integrada con Ball y Romer (1991)
				\4[] Complementos estratégicos pueden determinar eq. macro
				\4[] Estrategia óptima de un agente
				\4[] $\to$ Depende positivamente de estrategias de otros
				\4[] $\then$ ``Si nadie produce/vende/baja precios, yo tampoco''
				\4[] Economías pueden quedarse atrapadas en desempleo
				\4[] $\to$ Aunque exista un equilibrio mejor
				\4 Ball y Romer (1991)
				\4[] Rigideces nominales son fallos de coordinación
				\4[] Incorpora instrumentos de modelización propios de NEK1G y 2G
				\4[] $\to$ Demanda à la Dixit-Stiglit
				\4[] $\to$ Equilibrio general
				\4[] Introduce agentes heterogéneos en NEK
				\4[] $\to$ Pionero en HANK
				\4[] Economía formada por dos empresas
				\4[] $\to$ Cada una provee inputs a la otra
				\4[] Shock de demanda negativo
				\4[] $\to$ Reducir precios es óptimo
				\4[] Ninguna quiere reducir unilateralmente primero
				\4[] $\to$ Beneficios caerían
				\4[] $\to$ Precios constantes aunque menos demanda
				\4[] $\then$ Ajuste en cantidades
				\4[] $\then$ Caída de empleo y output
				\4[] Si bajada coordinada de precios fuese posible
				\4[] $\to$ Saldos reales aumentan
				\4[] $\to$ Ambas empresas mejoran
				\4 Implicaciones
				\4[] Múltiples equilibrios posibles
				\4[] Desempleo persistente posible
				\4[] $\to$ Resulta de problema de coordinación
				\4[] Equilibrios múltiples subóptimos son posibles
				\4[] Intervención pública puede reducir desempleo
				\4[] $\to$ Servir como punto focal
				\4[] $\to$ Gasto público multiplicador de actividad
				\4 Valoración
				\4[] Dificil formular modelos tratables
				\4[] RBC aparece contemporáneamente
				\4[] Poca continuidad
				\4[] Enfoque de Lucas predominó
			\3 Implicaciones
				\4 NAIRU
				\4[] Non-Accelerating Inflation Rate of Unemployment
				\4[] Concepto más general que tasa natural de paro
				\4[] $\to$ No implica equilibrio
				\4[] $\then$ Generalización de tasa natural de paro
				\4[] $\then$ Admite diferentes explicaciones
		\2 Nueva Economía Keynesiana--Segunda generación
			\3 REFORMULAR CON GALI (2015) CH. 7  SOBRE DESEMPLEO
			\3 Idea clave
				\4 Contexto
				\4 Objetivos
				\4 Resultados
			\3 Formulación
				\4 Consumidores-trabajadores
				\4[] Maximización intertemporal
				\4[] Ocio-consumo
				\4[] Demanda de consumo agregado
				\4[] $\to$ Elasticidad de sustitución intertemporal constante
				\4[] Demanda de consumo de cada variedad
				\4[] $\to$ Función CES
				\4[] Demanda de ocio
				\4[] $\to$ Elasticidad de sustitución intertemporal constante
				\4 Empresas
				\4[] Competencia monopolística
				\4 Rigideces nominales
				\4[] Empresas no pueden cambiar precios a voluntad
				\4[] Diferentes submodelos de fijación de precios
				\4[] Más habitual: precios à Calvo
				\4[] $\to$ Sólo un \% puede cambiarlos en cada periodo
				\4 Ecuaciones de dinámica
				\4[DIS] IS dinámica
				\4[] \fbox{$\tilde{y}_t = \textrm{E}_t \left\lbrace \tilde{y}_{t+1} \right\rbrace - \frac{1}{\sigma} \left( \underbrace{i_t - \textrm{E}_t \left\lbrace \pi_{t+1} \right\rbrace}_{r_t} - r^n_t \right) $}
				\4[NKPC] Curva de Phillips Neo-Keynesiana básica
				\4[] \fbox{$\pi_t = \text{E}_t \left\lbrace \pi_{t+1} \right\rbrace + \textsc{k} \hat{y}_t + \textsc{k} (y $}
				\4[] \fbox{$\pi_t = \text{E}_t \left\lbrace \pi_{t+1} \right\rbrace + \textsc{k} (y_t - y^n_t + y^e_t - y^e_t ) $}
				\4[] \fbox{$\pi_t = \text{E}_t \left\lbrace \pi_{t+1} \right\rbrace + \textsc{k} (y_t - y^e_t) + \textsc{k}( y^e_t - y^n_t ) $}
				\4[] Donde:
				\4[] $\to$ $y_t - y^e_t$: desviación respecto a output eficiente
				\4[] $\to$ $y^e_t - y^n_t$: diferencia entre output eficiente y natural
				\4[] $\then$ Aparición de trade-off inflación-output
				\4[] $\then$ Acercarse más a $y_t$ eficiente requiere inflación
				\4[] $\then$ Necesario postular f. de pérdida inflación-output gap
				\4[TR] Regla de Taylor simple
				\4[] \fbox{$i_t = \rho + \phi_\pi \pi_t + \phi_y \tilde{y}_t + v_t $}
				\4[MP] Mercado de dinero
				\4[] \fbox{$m_t - p_t = y_t - \eta i_t$}
			\3 Implicaciones
				\4 Curva de Phillips
				\4[] Creciente inflación-output
				\4[] Decreciente salarios-empleo
				\4[] Forward-looking
				\4[] $\to$ Más inflación futura esperada
				\4[] $\then$ Más inflación presente
				\4 NAIRU
				\4[]
				\4 Histéresis
				\4[] Posible modelización con extensiones

			\3 Extensiones
				\4 Búsqueda
				\4 Coordinación
				\4 Política monetaria óptima
				\4 Agentes heterogéneos
			\3 Valoración
				\4 Base de modelo canónico macroeconomía actual
				\4 Enfoque de modelización dominante últimas 2 décadas
		\2 Modelos de búsqueda - DMP
			\3 Idea clave
				\4 Contexto
				\4[] Mercado de trabajo no es centralizado
				\4[] Encontrarse es costoso
				\4[] Ofertas de trabajo
				\4[] $\to$ Se reciben secuencialmente
				\4[] Tensión en el mercado de trabajo
				\4[] $\to$ Empíricamente, procíclica
				\4[] $\then$ Más vacantes por desempleado con $\uparrow$ Y
				\4[] Proceso de emparejamiento parados-vacantes
				\4[] $\to$ Puede alcanzar estado estacionario
				\4[] $\then$ Posible desempleo
				\4 Objetivos
				\4[] Caracterizar efectos de políticas de búsqueda
				\4[] $\to$ Sobre desempleo
				\4[] $\to$ Sobre salarios
				\4 Resultados
				\4[] Políticas que actúan sobre búsqueda de trabajo
				\4[] $\then$ Efectos sobre tasa de desempleo
			\3 Formulación
				\4[] Trabajadores
				\4[] $\to$ Aceptan oferta o esperan a siguiente
				\4[] $\to$ Ponderan trabajar vs seguir buscando
				\4[] Empresas
				\4[] $\to$ Mantienen vacante abierta o cierran
				\4[] $\to$ Ponderan producto del trabajo vs coste de búsqueda
				\4[] Instituciones del mercado de trabajo
				\4[] $\to$ Salario de reserva
				\4[] $\to$ Tasa de emparejamiento
				\4[] $\to$ Tasa de despido
				\4[] $\to$ Productividad del trabajo
				\4[] Equilibrio
				\4[] $\to$ Genera curva de Beveridge U--V
				\4[] $\to$ Genera desempleo con inflación estable
				\4[] $\to$ Genera desempleo óptimo
				\4[] Formulación con tres ecuaciones
				\4[] WS: Salarios negociados à la Nash
				\4[] \fbox{$w=(1-\beta)R + \beta(p+\frac{c}{q(\theta)})$}
				\4[] ZP: Costes y beneficio de mantener vacantes se igualan
				\4[] \fbox{$\frac{p-w}{r+\delta_m} = \frac{c}{q(\theta)}$}
				\4[] BV: Estado estacionario del empleo
				\4[] \fbox{$U^* = \frac{\delta_m}{\delta_m + f(\theta)}$}
			\3 Implicaciones
				\4 Shock de oferta aumenta $p$
				\4[] $\then$ Más salario de equilibrio
				\4[] $\then$ Más tensión en mercado laboral
				\4[] $\then$ Curva beveridge no se desplaza
				\4[] $\then$ Desplazamiento hacia NOeste a lo largo de BV
				\4 Aumento del salario de reserva
				\4[] $\then$ Más salario de equilibrio
				\4[] $\then$ Menos vacantes y menos tensión en mercado laboral
				\4[] $\then$ Desplazamiento hacia SEste a lo largo de BV
				\4 Aumento de la fiscalidad del trabajo
				\4[] $\to$ Rentas no salariales aumentan valor relativo
				\4[] $\to$ Aumenta poder de negociación de trabajadores
				\4[] $\then$ Más salario de equilibrio
				\4[] $\then$ Menos vacantes
				\4[] $\then$ Más desempleo
				\4 Aumento de la destrucción exógena del empleo
				\4[] $\to$ Cae beneficio por abrir vacante
				\4[] $\to$ Aumenta desempleo de equilibrio
				\4[] $\then$ Menor salario
				\4[] $\then$ Menor tensión en mercado de trabajo
				\4[] $\then$ Mayor desempleo
				\4 Políticas activas de empleo más efectivas
				\4[] $\to$ Caída de coste de mantener vacante
				\4[] $\to$ Entran más empresas con más vacantes
				\4[] $\then$ Aumento de la tensión en el mercado laboral
				\4[] $\then$ Caída del desempleo
				\4[] $\then$ Efecto ambiguo sobre el salario:
				\4[] \quad Mayor beneficio de empresas a repartir
				\4[] \quad Menor ahorro de empresas por cubrir reduce poder

			\3 Valoración
				\4[] Representar proceso de búsqueda y emparejamiento
				\4[] $\to$ Incorporable en modelos de equilibrio general
				\4[] Mercado de trabajo no es walrasiano
				\4[] $\to$ No se intercambia trabajo a un sólo precio
				\4[] $\to$ Emparejamiento tiene un coste informacional
				\4[] Función de emparejamiento
				\4[] $\to$ Empareja vacantes y desempleados
				\4[] $\then$ Dadas instituciones y tensión del mercado
				\4 Teorías de la histéresis
				\4[] Descapitalización de trabajadores
				\4[] $\to$ Cuando salen del mercado pierden habilidades
				\4[] $\to$ Mayor tiempo en paro, mayor pérdida
				\4[] $\then$ Aparición de desequilibrios estructurales
				\4[] Insiders y outsiders
				\4[] $\to$ Outsiders no participan en negociación
				\4[] $\to$ Cuando paro aumenta, salario real no cae
				\4[] $\then$ Desempleo se mantiene constante
				\4[] Variaciones de la productividad
				\4[] $\to$ Menos productividad+salario real rígido
				\4[] $\then$ Aumento del desempleo
				\4[] Centralización de la negociación colectiva
				\4[] $\to$ Hipótesis de Calmfors y Driffil
				\4[] $\then$ Relativamente robusta a evidencia empírica
				\4[] $\to$ Muy alta centralización
				\4[] $\then$ Agentes internalizan efecto macro
				\4[] $\then$ Ajuste más flexible de salario
				\4[] $\then$ Mejor comportamiento del empleo
				\4[] $\to$ Muy baja centralización
				\4[] $\then$ Negociación a escala de empresa
				\4[] $\then$ Trabajadores tienen menos poder
				\4[] $\then$ Ajuste más flexible de salarios
				\4[] $\then$ Mejor comportamiento del empleo
				\4[] $\then$ Menor desempleo de equilibrio
				\4[] $\to$ Centralización intermedia
				\4[] $\then$ Ninguno de los efectos anteriores
				\4[] $\then$ Resistencia a perder poder de compra relativo
				\4[] $\then$ Menos flexibilidad
				\4[] $\then$ Mayor desempleo de equilibrio
				\4[] Apertura de la economía
				\4[] $\to$ Más apertura impide ajuste en precios
				\4[] $\to$ Competencia con productos extranjeros
				\4[] $\then$ Puede afectar a NAIRU
				\4[] (VER BLANCHARD 2005 Y 2008)
				\4[] (VER 6. EVIDENCIA EMPÍRICA EN CECO NUEVO)
	\1[] \marcar{Conclusión}
		\2 Recapitulación
			\3 Hechos estilizados en el mercado de trabajo
			\3 Empleo e inflación
			\3 Evolución de análisis teórico
		\2 Idea final
			\3 Problemas del mercado laboral europeo
				\4 Tasas de paro relativamente más altas
				\4[] $\to$ Que Japón y EEUU
				\4 Problema arrastrado desde los 80
				\4 Especialmente grave en algunos países
				\4 Efectos desestabilizadores
				\4[] Políticos
				\4[] Efecto de crisis
			\3 Flexibilización de mercados laborales
				\4 Relativo consenso sobre flexibilización
				\4[]
			\3 Relación con otros conceptos
				\4 Teorías de negociación salarial
				\4 Modelos de búsqueda
				\4 Diseño de instituciones
				\4[] Seguridad social
				\4[] Políticas activas de empleo
				\4[] Fiscalidad
\end{esquemal}

\graficas

\begin{axis}{4}{Modelo clásico del mercado de trabajo: el salario nominal se ajusta para eliminar excesos de demanda.}{$L$}{$W/P$}{modeloclasico}
	% equilibrio
	\node[circle, fill=black, inner sep=0pt, minimum size=3pt] (a) at (1.79,2.27) {};
	\node[left] at (0,2.27){\tiny $(W/P)^*$};
	\draw[dashed] (0,2.27) -- (4,2.27);
	
	% DEMANDA
	\draw[-] (0.5,4) -- (3.5,0);
	\node[above] at (0.5,4){D};
	
	% exceso de demanda negativo
	\draw[-{Latex}] (1.24,3.01) -- (1.74,2.347);
	\draw[-{Latex}] (0.74,3.68) -- (1.24,3.01);
	
	% exceso de oferta negativo
	\draw[-{Latex}] (2.35,1.53) -- (1.85, 2.2);
	\draw[-{Latex}] (2.85,0.867) -- (2.35,1.53);
	
	% OFERTA
	\draw[-] (0,0.5) -- (3.5,4);
	\node[above] at (3.5,4){S};
	
	% exceso de demanda positivo
	%\draw[-{Latex}] (0,0.5) -- (.7,1.2);
	\draw[-{Latex}] (0.7,1.2) -- (1.2,1.7);
	\draw[-{Latex}] (1.2,1.7) -- (1.7,2.2);
	
	% exceso de demanda negativo
	\draw[-{Latex}] (2.85, 3.35)-- (2.35,2.85);
	\draw[-{Latex}] (2.35,2.85) -- (1.85,2.35);
	
	% precio alto
	\draw[dashed] (0,3.5) -- (4,3.5);
	\node[left] at (0,3.5){\tiny $(W/P)_1$};
	
	% precio bajo
	\draw[dashed] (0,1) -- (4,1);
	\node[left] at (0,1){\tiny $(W/P)_0$};
	
	% exceso de demanda positivo con precio bajo
	\draw[decoration={brace,mirror,raise=5pt},decorate]
	(0.45,1) -- node[below=9pt] {$\text{ED}>0$} (2.81,1);
	
	% exceso de demanda negativo con precio alto
	\draw[decoration={brace,raise=9pt},decorate]
	(0.97,3.27) -- node[above=12pt] {$\text{ED}<0$} (2.95,3.27);	
\end{axis}

\begin{axis}{4}{Representación del equilibrio entre oferta y demanda agregadas en el contexto de un modelo clásico del mercado de trabajo: los estímulos de demanda son inefectivos.}{Y}{P}{oadaclasico}
	% OA
	\draw[-] (2,0) -- (2,3.5);
	\node[above] at (2,3.5){\tiny OA};
	\node[below] at (2,0){\tiny $\bar{Y}$};
	
	% DA
	\draw[-] (0.5,3.5) -- (3.5,0);
	\node[above] at (0.5,3.5){\tiny DA};
	
	% Nivel de precios inicial
	\draw[dotted] (0,1.75) -- (2,1.75);
	\node[left] at (0,1.75){\tiny P};
	
	
	% Desplazamiento de demanda
	\draw[dashed] (1.5,3.5) -- (4.5,0);
	\node[above] at (1.5,3.5){\tiny DA'};
	
	% Nivel de precios tras estímulo
	\draw[dotted] (0,2.93) -- (2,2.93);
	\node[left] at (0,2.93){\tiny P'};
\end{axis}

\begin{axis}{4}{Modelo clásico del mercado de trabajo: ajuste del salario nominal ante un estímulo de demanda en el mercado de bienes.}{$L$}{$W$}{clasicoestimulodemandabienes}
	% Demanda inicial
	\draw[-] (0.5,3.5) -- (3.5,0.5);

	% Oferta inicial
	\draw[-] (0.5,0.5) -- (3.5,3.5);

	% Demanda tras estímulo
	\draw[dashed] (1.5,3.5) -- (4.5,0.5);

	% Oferta tras estímulo
	\draw[dashed] (0.5,1.5) -- (3.5,4.5);
	
	% Equilibrio inicial
	\draw[dashed] (0,2) -- (2,2) -- (2,0);
	\node[left] at (0,2){$W_0$};

	% Equilibrio final
	\draw[dashed] (0,3) -- (2,3) -- (2,2);
	\node[left] at (0,3){$W_1$};
	\node[below] at (2,0){$L^*$};
\end{axis}		

\newpage 

\begin{axis}{4}{Modelo keynesiano del mercado de trabajo: cuando el salario real es demasiado elevado, la rigidez nominal del salario impide su ajuste al equilibrio walrasiano.}{$L$}{$W/P$}{modelokeynes}
	% equilibrio
	\node[circle, fill=black, inner sep=0pt, minimum size=3pt] (a) at (1.79,2.27) {};
	\node[left] at (0,2.27){\tiny $(W/P)^*$};
	\draw[dashed] (0,2.27) -- (4,2.27);
	
	% DEMANDA
	\draw[-] (0.5,4) -- (3.5,0);
	\node[above] at (0.5,4){D};

	% OFERTA
	\draw[-] (0,0.5) -- (3.5,4);
	\node[above] at (3.5,4){S};

	% precio alto
	\draw[-] (0,3.5) -- (4,3.5);
	\node[left] at (0,3.5){\tiny $(W/P)_0$};
	
	% Trabajo demandado
	\draw[thick] (0.87,4) -- (0.87,0);
	\node[below] at (0.87,0){\tiny $L_D$};

	% Trabajo ofertado
	\draw[dotted] (3,3.5) -- (3,0);
	\node[below] at (3,0){\tiny $L_S$};
	
	% Trabajo de equilibrio walrasiano
	\draw[dotted] (1.8,2.2) -- (1.8,0);
	\node[below] at (1.8,0){\tiny $L^*$};
	
	% exceso de demanda negativo con precio alto
	\draw[decoration={brace,raise=9pt},decorate] (0.97,3.27) -- node[above=12pt] {\tiny $\text{ED}<0$} (2.95,3.27);	
	
	% desempleo involuntario
	\draw[decoration={brace,raise=13pt, mirror},decorate] (0.87,0) -- node[above=-25pt] {\tiny Desempleo involuntario} (1.8,0);

\end{axis}

\begin{dibujo}{4}{Representación gráfica del modelo neoclásico en el bloque de oferta (mercado de trabajo), el de demanda (IS-LM) y el equilibrio (AS-AD).}{x}{y}{islmadasclasico}
	%%%%%%%%%%%%%%%%%%% TRABAJO
	
	% Ejes
	\draw[-] (-8,4) -- (-8,0) -- (-4,0);
	\node[left] at (-8,4){W/P};
	\node[below] at (-4,0){L};
	
	
	% Demanda de trabajo
	\draw[-] (-7.5,3.5) -- (-4.5,0.5);
	\node[right] at (-4.5,0.5){$Y'(L) = \frac{W}{P}$};
	
	% Oferta de trabajo
	\draw[-] (-7.5,0.5) -- (-4.5,3.5);
	\node[right] at (-4.5,3.5){$\frac{u_l}{u_c} = \frac{W}{P}$};

	% Equilibrio
	\draw[dashed] (-8,2) -- (-6,2) -- (-6,0);
	\node[left] at (-8,2){$(W/P)^*$};
	\node[below] at (-6,0){$L^*$};
	
	%%%%%%%%%%%%%%%%%%% IS-LM
	
	% Ejes
	\draw[-] (-2,4) -- (-2,0) -- (2,0);
	\node[left] at (-2,4){r};
	\node[below] at (2,0){Y};
	
	% IS 
	\draw[-] (-1.5,3.5) -- (1.5,0.5);
	\node[right] at (1.5,0.5){IS};
	
	% LM
	\draw[-] (0,0) -- (0,4);
	\node[right] at (0,3.5){LM};
	\node[below] at  (0,0){$Y(L^*)$};
	
	% Equilibrio
	
	%%%%%%%%%%%%%%%%%%% AS-AD
	
	% Ejes
	\draw[-] (4,4) -- (4,0) -- (8,0);
	\node[left] at (4,4){P};
	\node[below] at (8,0){Y};
	
	% AD
	\draw[-] (4.5,3.5) -- (7.5,0.5);
	\node[right] at (7.5,0.5){AD};
	
	% AS
	\draw[-] (6,0) -- (6,4);
	\node[right] at (6,3.5){AS};
	\node[below] at  (6,0){$Y(L^*)$};
	
	% Equilibrio
	
\end{dibujo}

\begin{dibujo}{4}{Representación gráfica del modelo keynesiano en el bloque de oferta (mercado de trabajo), el de demanda (IS-LM) y el equilibrio (AS-AD).}{x}{y}{islmadaskeynesiano}
	%%%%%%%%%%%%%%%%%%% TRABAJO
	
	% Ejes
	\draw[-] (-8,4) -- (-8,0) -- (-4,0);
	\node[left] at (-8,4){W/P};
	\node[below] at (-4,0){L};
	
	
	% Demanda de trabajo
	\draw[-] (-7.5,3.5) -- (-4.5,0.5);
	\node[right] at (-4.5,0.5){$Y'(L) = \frac{W}{P}$};
	
	% Oferta de trabajo
	\draw[-] (-7.5,0.5) -- (-4.5,3.5);
	\node[right] at (-4.5,3.5){$\frac{u_l}{u_c} = \frac{W}{P}$};
	
	% Equilibrio
	\draw[dashed] (-8,3) -- (-5,3);
	\draw[-] (-7,3) -- (-7,0);
	\node[below] at (-7,0) {$L^*$};
	
	% Desempleo
	\draw[decoration={brace,raise=9pt},decorate]
	(-7,3.05) -- node[above=12pt] {$\text{ED}<0$} (-5,3.05);	
	
	%%%%%%%%%%%%%%%%%%% IS-LM
	
	% Ejes
	\draw[-] (-2,4) -- (-2,0) -- (2,0);
	\node[left] at (-2,4){r};
	\node[below] at (2,0){Y};
	
	% LM
	\draw[-] (-1.3, 0.5) -- (0,0.5) to [out=0, in=260](2,4);
	\node[right] at (2,4){LM};
	
	% IS
	\draw[-] (-1.8,4) -- (1,0.2);
	\node[right] at (1,0.2){IS};
	
	% Equilibrio
	
	%%%%%%%%%%%%%%%%%%% AS-AD
	
	% Ejes
	\draw[-] (4,4) -- (4,0) -- (8,0);
	\node[left] at (4,4){P};
	\node[below] at (8,0){Y};
	
	% AD
	\draw[-] (4.5,3.5) -- (6.5,0.5);
	\node[right] at (7.5,0.5){AD};
	
	% AS
	%\draw[-] (4.5,0.5) -- (7.5,3.5);
	\draw[-] (4.5,0.5) to [out=15,in=260](7,1.5) -- (7,4);
	\node[right] at (7,4){AS};
	
	% Equilibrio
	
\end{dibujo}


\begin{axis}{4}{Representación del equilibrio entre oferta y demanda agregadas en el contexto de un modelo keynesiano del mercado de trabajo: los estímulos de demanda son efectivos aumentando el output.}{Y}{P}{oadakeynes}
	% OA
	\draw[-] (0.5,0.5) to [out=0, in=270] (3,2) -- (3,4);
	\node[above] at (3,4){\tiny OA};
	
	% DA
	\draw[-] (0.5,3.5) -- (3.5,0);
	\node[above] at (0.5,3.5){\tiny DA};
	
	% Nivel de precios inicial
	\draw[dotted] (0,1.08) -- (2.6,1.08);
	\node[left] at (0,1.08){\tiny P};
	
	% output inicial
	\draw[dotted] (2.57,1.08) -- (2.57,0);
	\node[below] at (2.57,0){\tiny $\bar{Y}$};
	
	% Desplazamiento de demanda
	\draw[dashed] (1.5,3.5) -- (4.5,0);
	\node[above] at (1.5,3.5){\tiny DA'};
	
	% Nivel de precios tras estímulo
	\draw[dotted] (0,1.81) -- (2.98,1.81);
	\node[left] at (0,1.81){\tiny P'};
	
\end{axis}

\begin{axis}{4}{Ejemplo de curva de Phillips similar a la estimada por Phillips en 1958.}{}{$\dot{w}$}{curvadephillips}
	% extensión del eje de abscisas
	\node[below] at (6,0){$u$};
	\draw[-] (0,0) -- (-2,0);
	\draw[-] (0,0) -- (0,-2);
	\draw[-] (4,0) -- (6,0);
	
	% curva de phillips
	\draw[-] (1,4) to [out=280, in= 175](6,-1);
\end{axis}


\begin{axis}{4}{Curva de Phillips propuesta por Friedman que asume la hipótesis de expectativas adaptativas.}{$u$}{$\pi$}{curvadephillipsmonetarista}
	\draw[-] (0.1,3.5) to [out=275,in=175](4,-1);
	
	\draw[-] (0.7,4.2) to [out=275,in=175](4.7,-0.3);
	
	\draw[-] (1.4,4.9) to [out=275,in=175](5.4,0.4);
	
	% paro natural
	\draw[dashed] (1.69,0) -- (1.69,4);
	
	\node[below] at (1.69,0){$u^*$};
	
	% primer aumento del empleo
	\draw[dashed,-{Latex}] (0.64,1.42) -- (1.69,1.42);
	
	\draw[-{Latex}] (1.69,0) to [out=138, in=300](0.64,1.42);
	
	% segundo aumento del empleo
	\draw[dashed,-{Latex}] (0.81,3.47) -- (1.69,3.47);
	
	\draw[-{Latex}] (1.69,1.42) to [out=124, in=282](0.81,3.47);
	
	% paro menor que tasa natural con curva de Phillips inicial
	%\draw[dashed] (0.64,1.42) -- (0.64,0);
	
	% paro menor que tasa natural con curva de Phillips con curva de Phillips ajustándose a equilibrio
	%\draw[dashed] (0.81, 3.47) -- (0.81,0);
\end{axis}

\begin{axis}{4}{Desempleo involuntario como resultado de un problema de coordinación en Clower (1965) }{L}{c}{coordinacionclower}
	% Producción
	\draw[-] (0.7,1) to [out=60, in=185](4,2.8);
	\node[right] at (4,2.8){\tiny $Y$};
	
	% curva de indiferencia de equilibrio
	\draw[-] (0.99,2.17) to [out=10, in=250](3.19,3.29);
	
	% salario de equilibrio
	\draw[-] (0.7,1.73) -- (4,3.17);
	\node[right] at (4,3.17){\tiny $L^*$};
	
	% curva de indiferencia de desequilibrio
	\draw[-] (0.73,2.47) to [out=10, in=250](2.93,3.53);
	
	% salarios de desequilibrio
	\draw[-] (0.7, 1.22) -- (3.7,4.05);
	\node[right] at (3.7,4.05){\tiny $L' > L^*$};
	
	% demanda y oferta de trabajo de equilibrio walrasiano
	\draw[dashed] (2.45,2.5) -- (2.45,0);
	\node[below] at (2.40,0){\tiny $l^*$};
	
	% demanda de trabajo de desequilibrio
	\draw[dashed] (1.3,1.78) -- (1.3,0);
	\node[below] at (1.3,0){\tiny $l_\text{\tiny D}$};
	
	% oferta de trabajo de desequilibrio
	\draw[dashed] (2.6,3) -- (2.6,0);
	\node[below] at (2.7,0){\tiny $l_S$};
	
	% demanda y oferta de bienes de equilibrio walrasiano
	\draw[dashed] (2.45,2.5) -- (0,2.5);
	\node[left] at (0,2.5){\tiny $c^*$};
	
	% demanda de bienes de desequilibrio
	\draw[dashed] (2.6,3) -- (0,3);
	\node[left] at (0,3){\tiny $c_D$};
	
	% oferta de bienes de desequilibrio
	\draw[dashed] (1.3,1.78) -- (0,1.78);
	\node[left] at (0,1.78){\tiny $c_S$}; 
\end{axis}

La curva cóncava Y muestra la frontera de posibilidades de producción. Las dos curvas convexas paralelas son curvas de indiferencia de un consumidor representativo. La recta $L^*$ representa el salario de equilibrio walrasiano. La recta $L'$ cuya pendiente es mayor que $L^*$ representa un salario mayor al de equilibrio walrasiano que induce un exceso de oferta de trabajo ($l_S > l_D$) y un exceso de demanda de consumo. Definiendo la demanda nocional como aquella cantidad que un agente desearía consumir, y demanda efectiva como aquella cantidad a la que efectivamente accede, es posible argumentar que una situación de desequilibrio puede ser estable. En el marco walrasiano, los excesos de demanda que inducen variaciones de precios se derivan de diferencias entre demanda nocional y oferta. En el contexto del gráfico, un exceso de demanda derivado de la demanda nocional induciría un aumento del precio del consumo, reduciendo el salario real y por ello, la pendiente de la recta hasta alcanzar el equilibrio walrasiano. Sin embargo, si los excesos de demanda no fuese calculados a partir de la demanda nocional sino de la demanda efectiva, estos no serían tales y el sistema no tendería hacia el ajuste. La conclusión que Clower extrae es que cuando el ingreso aparece como una variable independiente en la función de exceso de demanda, la teoría de precios tradicional no permite derivar conclusiones sobre la estabilidad de una economía y su tendencia hacia el equilibrio.

\begin{tabla}{Posibles estados de desequilibrio recogidos por Malinvaud a partir del modelo de Barro-Grossman}{desequilibrios}
	\begin{tabular}{l l c c}
	& & \multicolumn{2}{c}{\textbf{Mercado de bienes}}\\ \cline{3-4}
	& & \textit{ES} & \textit{ED} \\ \hline
	\multirow{2}{*}{\textbf{Mercado de trabajo}} & \textit{ES} & Paro keynesiano & Paro clásico \\
	& \textit{ED} & Subempleo & Inflación reprimida \\ \hline
	\end{tabular}
\end{tabla}

\begin{axis}{4}{Curva de Phillips propuesta por Friedman que asume la hipótesis de expectativas adaptativas.}{$u$}{$\frac{\dot{w}}{w}$}{curvadephillipsmonetarista}
	\draw[-] (0.1,3.5) to [out=275,in=175](4,-1);
	
	\draw[-] (0.7,4.2) to [out=275,in=175](4.7,-0.3);
	
	\draw[-] (1.4,4.9) to [out=275,in=175](5.4,0.4);
	
	% paro natural
	\draw[dashed] (1.69,0) -- (1.69,4);
	
	\node[below] at (1.69,0){$u^*$};
	
	% primer aumento del empleo
	\draw[dashed,-{Latex}] (0.64,1.42) -- (1.69,1.42);
	
	\draw[-{Latex}] (1.69,0) to [out=138, in=300](0.64,1.42);
	
	% segundo aumento del empleo
	\draw[dashed,-{Latex}] (0.81,3.47) -- (1.69,3.47);
	
	\draw[-{Latex}] (1.69,1.42) to [out=124, in=282](0.81,3.47);
	
	% paro menor que tasa natural con curva de Phillips inicial
	%\draw[dashed] (0.64,1.42) -- (0.64,0);
	
	% paro menor que tasa natural con curva de Phillips con curva de Phillips ajustándose a equilibrio
	%\draw[dashed] (0.81, 3.47) -- (0.81,0);
\end{axis}

\begin{axis}{4}{Kydland y Prescott (1977): inconsistencia de la política monetaria óptima y sesgo inflacionario resultante.}{$u_t - u^*$}{$\pi_t$}{nmckypphillips}
	% Extensión de ejes
	\draw[-] (-3,0) -- (0,0); % abscisas
	\draw[-] (0,0) -- (0,-3); % ordenadas
	
	% Curvas de Phillips
	\draw[-] (-3,4) -- (3,-4);
	\draw[-] (-3,5.7) -- (3,-2.3);
	
	% Curvas de indiferencia de función de pérdida
	\draw[-] (-3,3) to [out=-20, in=90](0,0) to [out=270, in=20](-3,-3);
	\draw[-] (-3,3.53) to [out=-20, in=90](0.53,0) to [out=270, in=20](-3,-3.53);
	
	% Óptimo
	\node[circle,fill=black,inner sep=0pt,minimum size=4pt] (a) at (0,0) {};	
	\node[above] at (-0.45,0){O};
	
	% Equilibrio
	\node[circle,fill=black,inner sep=0pt,minimum size=4pt] (a) at (0,1.8) {};
	\node[right] at (0,1.8){B};
	
\end{axis}

\begin{tabla}{Comparación del efecto de diferentes shocks sobre el trabajo en un modelo RBC básico.}{comparacionrbctrabajo}
	\begin{tabular}{l | c | c}
					   & Transitorio & Permanente \\ \hline
Shock de productividad & Aumento fuerte: $|\text{ES}+\text{ED}_i| >> \text{ER}_d$ & Aumento moderado $|\text{ES}+\text{ED}_i| > \text{ER}_d$ \\ \hline
Shock de gasto público & Aumento mínimo $|\text{ES}+\text{ED}_i| = 0 < \text{ER}_d$ & Aumento moderado $|\text{ES}+\text{ED}_i| = 0 << \text{ER}_d$ \\ \hline
	\end{tabular}
\end{tabla}
\conceptos

\concepto{Tasa natural de paro, desempleo de equilibrio y NAIRU}

Los conceptos de tasa natural de paro y desempleo de equilibrio son sinónimos. El segundo surge como traducción del ``\textit{natural rate of unemployment}'' enunciado por Milton Friedman en su discurso de 1968. La tasa natural de paro es, de acuerdo con Friedman e inspirándose en el tipo de interés ``\textit{natural}'' de Wicksell, aquella tasa de paro que existiría cuando se cubriesen todos los excesos de demanda/oferta en el sistema walrasiano de ecuaciones que representaría la macroeconomía. 

La NAIRU (``non-accelerating inflation rate of unemployment'') guarda una relación cercana con el concepto de desempleo de equilibrio pero incorpora un matiz menos positivo y es más general. El concepto hace referencia simplemente a aquella tasa de paro compatible con una inflación constante, sin que esté implícita la desaparición de excesos de oferta y demanda que sí lo están en el concepto de tasa natural de paro y de forma general, en modelos de corte walrasiano en los que se tiende a un equilibrio estable que vacía el mercado.

\concepto{Rigidez nominal y real}

En general, el concepto de rigidez en los precios comprende aquellas situaciones en las que uno o varios precios no varían o varían menos de lo que sería necesario para mantener un determinado valor. Cuando los salarios reales no se ajustan de manera tendente a alcanzar un equilibrio dado, hablamos de rigidez real. Cuando es el salario nominal no varía a pesar de cambios en alguno de sus determinantes, nos encontramos ante rigidez nominal. Ambos fenómenos son compatibles y pueden coexistir. 

Para fundamentar la existencia de rigideces observadas, existe una muy amplia variedad de modelos. El modelo de Shapiro y Stiglitz sobre salarios de eficiencia en contexto de información imperfecta es un ejemplo de fundamentación de una rigidez real que induce desempleo al impedir el ajuste del salario real al equilibrio que vaciaría el mercado de trabajo. Los precios à la Calvo, Fischer o Taylor son ejemplos de supuestos ad-hoc que inducen rigideces nominales del nivel de precios. 


\concepto{Elasticidad de Frisch}

La elasticidad de Frisch mide la variación porcentual de la oferta de trabajo ante una variación porcentual del salario, manteniendo constante la utilidad marginal de la riqueza.

En otros términos, captura el cambio en la senda temporal de la oferta de trabajo dada una variación porcentual del salario en un periodo determinado, pero asumiendo que no hay efecto riqueza. Sin embargo, los cambios del salario en un periodo determinado tienen un efecto riqueza sobre el valor presente de la suma de los salarios. Así, la elasticidad de Frisch captura el \textit{efecto sustitución} del cambio en el salario y obvia \textit{efecto riqueza}.

\preguntas

\seccion{21 de marzo de 2017}

\begin{itemize}
	\item ¿Dónde sitúa las políticas de los sindicatos en relación a la maximización de su renta? ¿Tiene efectos sobre el desempleo?
	\item ¿Conoce países que apliquen la co-gestión sindicatos-empresarios para minimizar estos efectos?
	\item En el paro clásico, ¿en qué mercados se producen excesos de demanda y qué signo tienen éstos?
	\item ¿Qué diferencias y similitudes existen entre desempleo de equilibrio y NAIRU?
	\item ¿Qué teorías explican la histéresis?
\end{itemize}

\seccion{Test 2018}

\textbf{15.} Suponga una economía donde las expectativas de productividad son correctas y la tasa de inflación esperada para el presente año es la tasa de inflación efectiva del año anterior ($\pi^e_t = \pi_{t-1})$). Si se estima econométricamente una curva de Phillips con datos anuales, obteniéndose la expresión $\pi_t - \pi_{t-1} = 2\% - 0,25u_t$, (donde $\pi$ hace referencia a la tasa de inflación y $u$ a la tasa de paro) ello implica que:

\begin{itemize}
	\item[a] Si la tasa de paro efectiva fuera cero, la inflación caería un $0,25\%$ cada año.
	\item[b] Si la tasa de paro efectiva fuera el $10\%$, la tasa de inflación aumentaría un $2\%$ cada año.
	\item[c] La tasa natural de paro es el $8\%$.
	\item[d] La tasa de inflación esperada es siempre el $2\%$.
\end{itemize}

\textbf{16.} ¿Qué efectos causará en el mercado de trabajo un aumento del grado de competencia en el mercado de bienes?

\begin{itemize}
	\item[a] Un aumento del salario real y una reducción de la tasa natural de paro.
	\item[b] Una reducción del salario real y un aumento de la tasa natural de paro.
	\item[c] Una reducción de la tasa natural de paro sin afectar al salario real.
	\item[d] Un aumento de la tasa natural de paro sin afectar al salario real.
\end{itemize}

\seccion{Test 2016}

\textbf{17.} Si la Ley de Okun para un país viene definida por la siguiente ecuación:

\begin{equation*}
u_t - u_{t-1} = -0,4 (g_{YT} -3 \%)
\end{equation*}

en la que $u_t$ es la tasa de paro en el momento de $t$, $g_{yt}$ es la tasa de crecimiento del PIB en esa economía,

\begin{itemize}
	\item[a] Se puede reducir el desempleo si la producción crece al 3\%.
	\item[b] La tasa normal de crecimiento del PIB en esa economía será del 3\%.
	\item[c] Una causa de que el coeficiente 0,4 sea menor que 1 es el hecho de que las empresas ``atesoran empleo''. 
	\item[d] La b) y la c) son verdad.
\end{itemize}

\textbf{20.}

Indique cual de las siguientes afirmaciones es correcta:

\begin{itemize}
	\item[a] La tasa natural de desempleo surge de la suma de paro cíclico y friccional.
	\item[b] La curva de Phillips relaciona la tasa de desempleo en un año con el crecimiento económico en el mismo año.
	\item[c] La curva de la demanda agregada a largo plazo es vertical.
	\item[d] La curva de la oferta agregada a corto plazo puede tener pendiente positiva por la presencia de información imperfecta en el establecimiento de los precios.
\end{itemize}

\seccion{Test 2007}

\textbf{17.} Si, en el modelo de precios rígidos (pegajosos), los contratos de trabajo especifican que los salarios nominales están completamente indiciados por la inflación, la curva de oferta agregada a corto plazo:
\begin{itemize}
	\item[a] Tendrá pendiente positiva.
	\item[b] Será vertical.
	\item[c] Será horizontal.
	\item[d] Se desplazará hacia arriba.
\end{itemize}

\seccion{Test 2006}
\textbf{15.} En una economía cerrada con salario real rígido y precios del bien final flexibles, un aumento del salario real:
\begin{itemize}
	\item[a] Incrementa los precios y deja invariante el nivel de renta de equilibrio.
	\item[b] Disminuye el consumo y el empleo.
	\item[c] Disminuye el empleo y los precios de equilibrio.
	\item[d] No tiene efectos sobre el empleo.
\end{itemize}

\textbf{16.} Señale qué respuesta es verdadera ante una política monetaria expansiva en una economía cerrada con salario rígido y constante y con precios del bien final rígidos (suponga que este nivel de precios está situado por encima del nivel que se habría mantenido en equilibrio de no existir tal rigidez):
\begin{itemize}
	\item[a] Incrementa la producción de equilibrio, aumenta el empleo de equilibrio, si bien el paro de origen clásico permanece invariante.
	\item[b] No tiene efectos sobre la producción ni sobre el empleo, ya que las políticas de demanda no tienen efectos en este tipo de economía.
	\item[c] Aumenta el empleo, disminuye el paro total y el paro de origen keynesiano si bien, la producción no varía.
	\item[d] Aumenta la producción pero no tiene efectos sobre el empleo, por lo que el paro total no varía.
\end{itemize}

\seccion{Test 2005}
\textbf{16.} En una economía cerrada con desempleo, una rebaja en las cotizaciones de la Seguridad Social a cargo de la empresa provoca una caída en el tipo de interés, y un aumento en el consumo privado y en la inversión:

\begin{itemize}
	\item[a] Si el salario real es rígido y constante, y el precio del bien final es flexible.
	\item[b] Si el precio del bien final es rígido.
	\item[c] Si todos los precios y salarios son flexibles.
	\item[d] En ninguno de los casos anteriores.
\end{itemize}


\notas


Ver \url{https://voxeu.org/article/phillips-curve-dead-or-alive} sobre Curva de Phillips ``muerta'' en la actualidad

\textbf{2018}: \textbf{15.} C \textbf{16.} A

\textbf{2016:} \textbf{17.} D \textbf{20.} D

\textbf{2007:} \textbf{17.} B

\textbf{2006:} \textbf{15.} B \textbf{16.} A

\textbf{2005:} \textbf{16.} \textbf{A}

\bibliografia

Mirar en Palgrave:

\begin{itemize}
	\item cost-push inflation
	\item demand-pull inflation
	\item disguised unemployment
	\item employment, theories of
	\item full employment
	\item full employment budget surplus
	\item implicit contracts
	\item involuntary unemployment
	\item labour economics
	\item labour economics (new perspectives)
	\item labour market institutions
	\item labour markets
	\item labour supply
	\item labour supply of women
	\item labours share of income
	\item natural rate of unemployment
	\item output and employment
	\item Phillips curve
	\item Phillips curve (new perspectives)
	\item primary and secondary labour markets
	\item search models of unemployment
	\item segmented labour markets
	\item structural unemployment
	\item social networks in labour markets
	\item transitional labour markets: theoretical foundations and policy strategies
	\item underemployment equilibria
	\item unemployment
	\item unemployment and hours of work, cross country differences
	\item unemployment insurance
	\item unemployment measurement
\end{itemize}

Ball, L. Mankiw, G \textit{The NAIRU in theory and practice} (2002) Journal of Economic Perspectives

Blanchard, O. (2005) \textit{European Unemployment: the Evolution of Facts and Ideas} NBER Working Paper Series -- En carpeta del tema

Blanchard, O. \textit{A review of ``Unemployment. Macroeconomic Performance and the Labour Market'', by Richard Layard, Stephen Nickell, and Richard Jackman} (2006) -- En carpeta del tema

Blanchard, O. \textit{Should We Reject the Natural Rate Hypothesis?} (2018, Winter) Journal of Economic Perspectives -- En carpeta del tema

Cahuc, P.; Zylberberg, A. \textit{Labor Economics} (2004) Ch. 8 -- En carpeta Economía Laboral. También diapositivas de \url{www.labor-economics.org}

Chetty, R.; Guren, A.; Manoli, D.; Weber, A. \textit{Are Micro and Macro Labor Supply Elasticities Consistent? A Review of Evidence on the Intensive and Extensive Margins} (2011) -- En carpeta del tema

Fuhrer, J.; Kodrzycki, Y.; Little, J. S.; Olivei, G. \textit{Understanding Inflation and Implications for Monetary Policy: A Phillips Curve Retrospective} (2009) MIT Press -- En carpeta del tema

Gordon, R. J. (1997) \textit{The Time-Varying NAIRU and its implications for economic policy} Journal of Economic Perspectives -- En carpeta del tema

Gordon, R. J. (2008) \textit{The History of the Phillips Curve: Consensus and Bifurcation} Economica -- En carpeta del tema

Heijdra, B. J. \textit{Foundations of Modern Macroeconomics} (2017) 3rd ed. -- En carpeta Macro

Jimeno, J. F. \textit{El misterioso caso de la inflación desaparecida.} \url{http://nadaesgratis.es/juan-francisco-jimeno/el-misterioso-caso-de-la-inflacion-desaparecida}

Keane, M.; Rogerson, R. (2012) \textit{Micro and Macro Labor Supply Elasticities: A Reassessment of Conventional Wisdom} Journal of Economic Literature -- En carpeta del tema

Samuelson, P.; Solow, R. \textit{Analytical Aspects of Anti-Inflation Policy} (1960) The American Economic Review -- En carpeta del tema

Violante, G.; \textit{ Micro and macro labor supply elasticity} \url{http://www.econ.nyu.edu/user/violante/NYUTeaching/Macrotheory/Spring14/LectureNotes/lecture12_14.pdf} -- En carpeta del tema

Wells, P. (1990) \textit{Keynes's General Theory Critique of the Neoclassical Theories of Employment and Aggregate Demand} Faculty Working Paper No. 90-1680 -- En carpeta del tema

\end{document}
