\documentclass{nuevotema}

\tema{3A-17}
\titulo{La teoría de la competencia monopolística y la diferenciación de productos.}

\begin{document}

Lionel Robbins y Posteriormente Samuelson definieron la ciencia económica como el estudio de las decisiones de gestión de bienes escasos con usos alternativos para satisfacer necesidades humanas. En el contexto de esta decisión, la microeconomía es la rama de la ciencia económica que tiene por objetivo explicar y predecir las decisiones de agentes individuales tales como consumidores, empresas o gobiernos. Aunque la economía ha extendido el uso de sus herramientas a muy variados contextos, un aspecto central a explicar y predecir es la decisión en contexto de mercado. El tema que nos ocupa trata sobre un tipo de decisión muy concreta: la competencia entre empresas. Para organizar la modelización de la competencia en familias de modelos, la microeconomía ha planteado dos casos polares y dos casos intermedios. El monopolio es el caso polar que se caracteriza por la existencia de una sola empresa que enfrenta una demanda decreciente y toma una decisión sobre el precio o la producción sujeta a esa demanda y con el objetivo de maximizar sus beneficios. Dado que maximiza su beneficio y no el bienestar general, el resultado será generalmente subóptimo desde el punto de vista social. El modelo de competencia perfecta es el extremo opuesto: un número suficientemente elevado de empresas y consumidores ofrecen y demandan un producto homogéneo en un contexto de intercambio centralizado de tal manera que cada empresa enfrenta una demanda perfectamente elástica y sólo puede elegir cuanto produce: debe ofertar su producto al precio de equilibrio o no venderá nada. Uno de los resultados más importantes de toda la ciencia económica muestra como los equilibrios de competencia perfecta dados ciertos supuestos son óptimos de Pareto. ¿Cuáles son los casos polares intermedios? Por un lado, tenemos el oligopolio, cuya característica principal es la presencia de poder de mercado e interdependencia estratégica. Por otro, encontramos la competencia monopolística, cuyo rasgo característico es precisamente la ausencia de interdependencia estratégica aunque en presencia de poder de mercado. Tanto el oligopolio como la competencia monopolística permiten analizar el fenómeno más habitual al que se enfrentan las empresas: demandas decrecientes que se ven desplazadas por movimientos de otras empresas. Para que tenga lugar este fenómeno es necesario que aparezca el elemento común a todos los modelos presentados en esta exposición: la diferenciación de los productos analizados. Así, un mercado está compuesto en este marco de análisis por productos similares pero diferentes en alguna dimensión tal como su localización, su calidad u otro tipo de característica esencial. Los modelos de oligopolio con diferenciación de producto y competencia monopolística se diferencian así en las diferentes herramientas utilizadas para abordar la representación del problema. El \textbf{objeto} de la exposición consiste en dar respuesta a preguntas tales como: ¿qué es la diferenciación de producto? ¿qué es la competencia monopolística? ¿qué diferencia a estos modelos entre sí y de otros modelos microeconómicos? ¿qué resultados se derivan? ¿cómo deciden las empresas en este contexto de modelización? ¿qué aplicaciones tienen? La \textbf{estructura} de la exposición se divide en dos partes. En primer lugar, examinamos el oligopolio con diferenciación de producto. Posteriormente, presentamos los modelos de competencia monopolística más relevantes: el modelo de Chamberlin (1933) y el modelo de Dixit y Stiglitz (1977).

\ideaclave

\seccion{Preguntas clave}

\begin{itemize}
	\item ¿Qué es la competencia monopolística?
	\item ¿Qué caracteriza a esta familia de modelos?
	\item ¿En qué se diferencia de otros modelos de competencia?
	\item ¿Qué es la diferenciación de productos?
	\item ¿Las empresas prefieren diferenciarse o parecerse?
	\item ¿Qué factores determina su decisión?
	\item ¿Qué aplicaciones tienen los modelos presentados?
\end{itemize}

\esquemacorto

\begin{esquema}[enumerate]
	\1[] \marcar{Introducción}
		\2 Contextualización
			\3 Economía y microeconomía
			\3 Competencia entre empresas
			\3 Diferenciación y competencia monopolística
		\2 Objeto
			\3 ¿Qué es la diferenciación de producto?
			\3 ¿Qué es la competencia monopolística?
			\3 ¿Qué caracteriza a estos modelos?
			\3 ¿Cómo deciden las empresas en ese contexto?
			\3 ¿Qué resultados se derivan?
			\3 ¿Qué aplicaciones tienen estos modelos?
		\2 Estructura
			\3 Oligopolio con diferenciación de producto
			\3 Competencia monopolística
	\1 \marcar{Oligopolio con diferenciación de producto}
		\2 Idea clave
			\3 Contexto
			\3 Objetivo
			\3 Resultados
		\2 Bertrand con diferenciación
			\3 Idea clave
			\3 Formulación.
			\3 Implicaciones
			\3 Aplicaciones
		\2 Hotelling (1929)
			\3 Idea clave
			\3 Formulación
			\3 Implicaciones
			\3 Valoración
			\3 Aplicaciones
		\2 Salop (1972)
			\3 Idea clave
			\3 Formulación
			\3 Implicaciones
			\3 Valoración
		\2 Diferenciación vertical
			\3 Idea clave
			\3 Formulación
			\3 Implicaciones
			\3 Valoración
	\1 \marcar{Competencia monopolística}
		\2 Idea clave
			\3 Enfoque no direccional
			\3 Sistema de funciones de demanda
			\3 Evolución histórica
		\2 Modelo de Chamberlin (1933)
			\3 Idea clave
			\3 Formulación
			\3 Implicaciones
			\3 Valoración
		\2 Modelos de Dixit-Stiglitz
			\3 Idea clave
			\3 Formulación
			\3 Implicaciones
			\3 Aplicaciones
			\3 Valoración
	\1[] \marcar{Conclusión}
		\2 Recapitulación
			\3 Oligopolio con diferenciación de producto
			\3 Competencia monopolística
		\2 Idea final
			\3 Relevancia de la competencia monopolística
			\3 Análisis económico de la publicidad
			\3 Impacto general sobre ciencia económica

\end{esquema}

\esquemalargo

\begin{esquemal}
	\1[] \marcar{Introducción}
		\2 Contextualización
			\3 Economía y microeconomía
				\4 Definición de Robbins
				\4[] Decisiones respecto a bienes escasos
				\4[] $\to$ Con usos alternativos
				\4[] $\to$ Para satisfacer necesidades humanas
				\4 Microeconomía
				\4[] Estudio de decisiones a nivel individual
				\4[] $\to$ Empresas
				\4[] $\to$ Consumidores
				\4[] $\to$ Gobiernos
			\3 Competencia entre empresas
				\4 Central en comportamiento de las empresas
				\4 ¿Cómo compiten entre sí?
				\4[] $\to$ Cuánto produce cada empresa
				\4[] $\to$ A qué precio
				\4 Estructuras de competencia polares
				\4[] Monopolio
				\4[] $\to$ Una empresa que no compite contra nadie
				\4[] $\to$ Enfrenta demanda decreciente en precio
				\4[] $\to$ Decisiones no dependen de otras empresas
				\4[] $\to$ Empresa decide precio y cantidad
				\4[] Competencia perfecta
				\4[] $\to$ Empresa compite contra otras empresas
				\4[] $\to$ Cada empresa enfrenta demanda elástica
				\4[] $\to$ Empresa sólo decide producción
				\4 Estructuras intermedias
				\4[] Distinción atribuida a Chamberlin
				\4[] Oligopolio
				\4[] $\to$ Compiten con otras empresas
				\4[] $\to$ Decisión de una afecta a otras
				\4[] $\to$ Tienen en cuenta decisiones de otras
				\4[] $\to$ Poder de mercado con interdep. estratégica
				\4[] $\then$ Decisión de una empresa afecta decisión de otra
				\4[] Competencia monopolística
				\4[] $\to$ Compiten con otras empresas
				\4[] $\to$ Decisiones individuales son infinitesimales respecto mercado
				\4[] $\to$ Empresas no consideran respuesta de otras
				\4[] $\to$ Sin interdependencia estratégica
			\3 Diferenciación y competencia monopolística
				\4 Analizar hecho empírico habitual
				\4[] $\to$ Empresa enfrenta demanda decreciente
				\4[] $\to$ Decisión de empresa afecta a otras
				\4[] $\to$ Decisión de empresa depende de decisión de otras
				\4[] Productos similares deben tener alguna diferencia
				\4[] $\to$ Localización
				\4[] $\to$ Características esenciales
				\4[] $\to$ Calidad
				\4 Dos marcos de análisis con similar objetivos
				\4[] $\to$ Diferentes énfasis y herramientas
				\4[] Competencia monopolística
				\4[] $\to$ Todas empresas compiten contra todas
				\4[] $\to$ Diferenciación resumida en elast. de sust.
				\4[] $\to$ Sistema de funciones de demanda
				\4[] $\to$ Sin interdependencia estratégica
				\4[] Oligopolio con diferenciación de producto
				\4[] $\to$ Diferenciación definida explícitamente
				\4[] $\to$ Una o varias dimensiones
				\4[] $\to$ Con interdependencia estratégica
				\4[] $\to$ Cómo diferenciarse es factor relevante
		\2 Objeto
			\3 ¿Qué es la diferenciación de producto?
			\3 ¿Qué es la competencia monopolística?
			\3 ¿Qué caracteriza a estos modelos?
			\3 ¿Cómo deciden las empresas en ese contexto?
			\3 ¿Qué resultados se derivan?
			\3 ¿Qué aplicaciones tienen estos modelos?
		\2 Estructura
			\3 Oligopolio con diferenciación de producto
			\3 Competencia monopolística
	\1 \marcar{Oligopolio con diferenciación de producto}
		\2 Idea clave
			\3 Contexto
				\4 Oligopolio
				\4[] Número suficientemente reducido de empresas
				\4[] $\to$ Aparece interdependencia estratégica
				\4[] $\then$ Reacción de competidores es relevante
				\4 Cournot y Bertrand
				\4[] Diferentes empresas compitiendo
				\4[] Producto idéntico
				\4[] $\to$ Consumidores compran producto más barato
				\4 Evidencia empírica
				\4[] Consumidores no siempre compran el más barato
				\4[] $\to$ Aunque productos sean similares
				\4[] $\then$  Demandan diferentes productos a cada productor
				\4[] Productos con igual precio
				\4[] $\to$ Consumidores no siempre consumen el mismo
				\4[] $\then$ A veces todos prefieren el mismo
				\4[] $\then$ A veces diferentes agentes prefieren diferentes productos
				\4 Empresas deciden cuanto diferenciarse
				\4[] Diferentes decisiones según contexto
				\4 Diferenciación endógena tiene manifestaciones variables
				\4[] A veces, deciden máxima diferenciación
				\4[] Otras, mínima diferenciación
			\3 Objetivo
				\4 Explicar demanda de prod. homogéneo a distintos precios
				\4 Explicar distinta demanda entre productos similares a igual precio
				\4 Explicar entrada de empresas
				\4 Explicar decisión de diferenciación en contexto de inter. estratégica
			\3 Resultados
				\4 Diferenciación explícita
				\4[] Productos diferenciados explícitamente
				\4[] $\to$ Diferencias cuantificadas en una variable dada
				\4 Varias interpretaciones de diferenciación
				\4[] Mismo producto, diferente localización
				\4[] Diferente producto, diferentes preferencias
				\4 Aplicaciones en otros ámbitos
				\4[] Política: votante mediano
				\4 Tres formulaciones básicas:
				\4[(i)] Bertrand con productos diferenciados
				\4[] $\to$ Diferencias sin localizar explícitamente
				\4[] $\to$ Precio de competidores afecta demanda
				\4[] $\to$ Sustituibilidad imperfecta $\then$ dda. decreciente
				\4[] $\to$ Sin paradoja de Bertrand
				\4[(ii)] Hotelling (1929) y derivados
				\4[] $\to$ Ciudad lineal
				\4[] $\to$ ¿Dónde se localizan las empresas?
				\4[] $\to$ ¿A qué precio venden?
				\4[(iii)] Salop (1979)
				\4[] $\to$ Ciudad circular
				\4[] $\to$ ¿Cuántas empresas entran?
				\4 Posible explicar publicidad
				\4[] Diferenciación a ojos del consumidor
				\4 Posible explicar precio > coste marginal
				\4[] Precio es complemento estratégico
				\4[] Consumo de otras variedades induce menos utilidad
				\4[] $\to$ Posible subir precio de variedad propia
		\2 Bertrand con diferenciación
			\3 Idea clave
				\4 Contexto
				\4[] Oligopolio
				\4[] $\to$ Interdependencia estratégica
				\4[] Demandas diferenciadas para cada empresas
				\4[] $\to$ Dependen de precio que fija otra empresa
				\4[] Bienes parcialmente sustitutivo
				\4[] $\to$ Demanda de una empresa cae si aumenta dda. de otra
				\4 Objetivos
				\4[] Caracterizar precio de equilibrio
				\4[] Comparar resultado con Bertrand con producto homogéneo
				\4 Resultados
				\4[] Paradoja de Bertrand deja de cumplirse
				\4[] $\to$ Precio de ambas empresas mayor a coste marginal
				\4[] Precio de equilibrio aumenta
				\4 Diferencias de precios de diferentes bienes
				\4 $\to$ No implica captura total de la demanda
			\3 Formulación.\footnote{Realmente, la formulación es mucho más compleja. Pero para el cante puede servir. Una formulación correcta con demanda derivada de un problema de maximización de utilidad puede encontrarse en: \href{https://www.parisschoolofeconomics.eu/docs/caillaud-bernard/2016-io-2a-differentiation.pdf}{Caillaud -- PSE}}
				\4 Dos empresas 1 y 2
				\4 Equilibrio con empresas simétricas
				\4[] Asumiendo:
				\4[] $\to$ $\phi_{12}, \phi_{21} >0$
				\4[] $\to$ $\phi_{11}, \phi_{22} > \phi_{12}, \phi_{21}$
				\4[] $\then$ Bienes sustitutivos imperfectos
				\4[] $\then$ Precio propio afecta demanda más que ajeno
				\4[] Enfrentan demandas:
				\4[] $Q_1 = A -\phi_{11} P_1 + \phi_{12} P_2$
				\4[] $\to$ Demanda de 1 cae con precio de 1
				\4[] $\to$ Demanda de 1 aumenta con precio de 2
				\4[] $Q_2 = A -\phi_{22} P_2 + \phi_{21} P_1$
				\4 Problema de maximización de empresa 1
				\4[] $\underset{p_i}{\max} \quad p_1 \cdot (A- \phi_{11} p_1 + \phi_{12} p_2)- c \cdot (A - \phi_{11} p_1 + \phi_{12} p_2)$
				\4[] CPO: $\quad$ $A - 2 \phi_{11}p_1 + \phi_{12} p_2 + \phi_{11} c = 0$
				\4[] $\then$ \fbox{$p_1 = \frac{A+\phi_{12} p_2}{2 \phi_{11}} + \frac{c}{2}$}
				\4[] $\then$ \fbox{$\Pi = \left( \frac{A+\phi_{12} p_ 2}{2 \phi_{11}} \right) - \frac{c^2}{4}$}
				\4[] Precios > coste marginal
				\4[] Más precio cuanto mayor:
				\4[] $\to$ Coste marginal
				\4[] $\to$ Sensibilidad a precio de otra empresa
				\4[] $\to$ Precio de otra empresa
				\4[] Si son suficientemente pequeños:
				\4[] $\to$ Efecto de precio de 1 sobre demanda de 2
				\4[] $\to$ Efecto de precio de 2 sobre demanda de 1
				\4[] Bajada de precio no captura toda la demanda del otro
				\4[] $\to$ No compensa caída de beneficios
				\4[] $\then$ Ambos mantienen precios más elevados
				\4[] Funciones de reacción crecientes en precios
				\4 Representación gráfica
				\4[] \grafica{bertranddiferenciado}
			\3 Implicaciones
				\4 Bajada de precio no permite captura de mercado
				\4[] Solo captura parcial
				\4[] $\to$ Caída de ingresos puede hacer no rentable
				\4 Ruptura de paradoja de Bertrand
				\4[] $\to$ Precio no es igual a coste marginal
				\4 Posible obtención de beneficio económico
				\4 Más beneficios cuanta más diferenciación
				\4[] Cuanto menos afecte lo que pasa en el otro mercado
				\4[] Cuanto menos $\downarrow$ D por precio de otra empresa
			\3 Aplicaciones
				\4 Puede explicar publicidad
				\4[] Intento de aumentar diferenciación de producto
				\4[] $\to$ Menos competencia con otro mercado por misma demanda
				\4 Problemas habituales de definición de mercado
				\4[] Quienes son realmente sustitutos
				\4[] Quienes pueden entrar en mercado si precio sube demasiado
		\2 Hotelling (1929)
			\3 Idea clave
				\4 Contexto
				\4[] Diferentes precios para bien homogéneo
				\4[] $\to$ Fenómeno habitual
				\4[] Precios sobre coste en bienes homogéneos
				\4[] Tiendas que venden helados
				\4[] Localización concentrada para vender mismo bien
				\4[] Localización separada para vender mismo bien
				\4 Objetivos
				\4[] Caracterizar competencia en precios
				\4[] $\to$ Con diferenciación localizada dada
				\4[] Caracterizar decisión de diferenciación
				\4 Resultados
				\4[] Ruptura de paradoja de Bertrand
				\4[] $\to$ Con localizaciones exógenas
				\4[] $\to$ Con loc. endógenas y CdT cuadrático
				\4[] Localización más cerca de centro tiene dos efectos
				\4[] -- Efecto demanda:
				\4[] $\to$ Reducir CdT que enfrentan consumidores
				\4[] $\then$ Capturar parte de mercado
				\4[] $\then$ Acercarse al centro del espectro
				\4[] -- Efecto competencia
				\4[] $\to$ Mayor competencia en precios por menores CdT
				\4[] $\then$ Empresas deben fijar precios más bajos
				\4[] $\then$ Caída de ingresos
				\4[] Tipos de costes de transporte determinan localización
				\4[] $\to$ Si EDemanda > ECompetencia, sin diferenciación
				\4[] $\to$ Si ECompetencia < EDemanda, con diferenciación
				\4[] Optimalidad de EC con localización exógena
				\4[] $\to$ EGC es subóptimo
				\4[] $\to$ Óptimo es localización en $1/4$ y $3/4$
			\3 Formulación
				\4 Supuestos generales
				\4[] Consumidores distribuidos sobre línea $[0,1]$
				\4[] $\to$ En principio, localizados uniformemente
				\4[] Dos firmas localizadas en algún punto
				\4[] Costes de transporte crecientes en distancia
				\4[] $\to$ Cada consumidor paga
				\4[] $\to$ Lineales, cuadráticos...
				\4[] Cada consumidor demanda $\left\lbrace 0, 1 \right\rbrace$
				\4[] \grafica{hotelling}
				\4 Interpretación de la localización
				\4[] Como diferencias en preferencias de consumidores
				\4[] Como diferencias en características del producto
				\4 Acercarse a otra empresa tiene dos efectos opuestos
				\4[] Efecto demanda
				\4[] $\to$ Aumenta beneficios
				\4[] Efecto competencia
				\4[] $\to$ Reduce beneficios
				\4 Efecto demanda: aumenta beneficios
				\4[] Cuanto más cerca de la otra empresa
				\4[] $\to$ Más espacio en el que CdTransporte será mayor para otra empresa
				\4[] $\then$ Mayor demanda sobre la que empresa tiene ventaja
				\4 Efecto competencia: reduce beneficios
				\4[] Cuanto más cercanía entre empresas
				\4[] $\to$ Menores costes de transporte para consumidores entre ambas
				\4[] $\then$ Menos puede aumentarse el precio sin perder consumidores
				\4 Localización exógena
				\4[] Sólo decisión respecto a precio a fijar
				\4[] Demanda de $a$ depende de consumidor indiferente:
				\4[] $\to$ $\bar{u} - p_a - t \tilde {x}= \bar{u} - p_b - t (1 - \tilde{x})$
				\4[] Costes de transporte lineales
				\4[] $\to$ $t(\tilde{x}) = t \cdot x$
				\4[] $\then$ $\tilde{x} = x_a = \frac{1}{2} + \frac{p_B - p_A}{2t}$
				\4[] Costes de transporte cuadráticos
				\4[] $\to$ $t(\tilde{x}) = t \cdot x^2$
				\4[] $\then$ $\tilde{x} = x_a = \frac{1}{2} + \frac{p_B - p_A}{2t}$
				\4[] $\then$ Misma función de demanda que con lineales
				\4[] Problema de maximización del beneficio para empresa A:
				\4[] $\underset{p_A}{\max}\quad p_A \cdot x_a (p_a) - c \cdot x_a$
				\4[] $\to$ CPO: $p_A = \frac{c+p_B+t}{2}$
				\4[$\then$] \fbox{Si $p_A = p_B$: $p_A = c + t$}
				\4[$\then$] Precio más alto cuanto mayor coste de transporte
				\4[] Porque puede permitirse aumentar precio
				\4[] $\to$ A consumidores más cercanos a la empresa
				\4[] $\then$ No preferirán comprar a la otra empresa por CdTrans
				\4[] Representación gráfica
				\4[] \grafica{hotellinglocexogena}
				\4 Localización endógena
				\4[] Dos fases:
				\4[] 1. Decisión sobre localización
				\4[] 2. Decisión sobre precio
				\4[] Resolución de decisión óptima:
				\4[] $\to$ Inducción hacia atrás
				\4[] $\then$ Resolver precio óptimo de fase 2
				\4[] $\then$ Resolver localización dado precio óptimo
				\4[] Decisión sobre localización
				\4[] $\to$ Ponderar efecto demanda vs competencia
				\4[] Con costes cuadráticos $\then$ Máxima diferenciación
				\4[] $\to$ Moverse hacia centro $\uparrow$ \underline{mucho} competencia
				\4[] $\to$ Poder de mercado aumenta rápido hacia extremos
				\4[] $\to$ Más probable que predomine ECompetencia
				\4[] $\then$ Mucho que ganar moviéndose hacia extremos
				\4[] Costes lineales $\then$ Mínima diferenciación
				\4[] $\to$ Moverse hacia centro $\uparrow$ \underline{mucho} la competencia
				\4[] $\to$ Poder de mercado aumenta poco hacia extremos
				\4[] $\then$ Muy poco que ganar moviéndose hacia extremos
				\4[] $\then$ Efecto demanda prevalece
				\4 Localización endógena y precios exógenos
				\4[] Precios fijados exógenamente
				\4[] $\to$ P.ej: precios públicos
				\4[] No hay efecto competencia
				\4[] $\to$ No pueden bajar ni subir los precios
				\4[] Sólo efecto demanda
				\4[] $\to$ Tienden a moverse hacia el centro
				\4[] $\then$ Mínima diferenciación
				\4 Óptimo social con localización y precios endógenos
				\4[] Asumiendo mercado totalmente cubierto
				\4[] $\to$ Consumidor más alejado dispuesto a consumir\footnote{Es decir, en el equilibrio, el precio que fijan las empresas es tal que el consumidor más alejado de ambas obtiene utilidad no negativa por consumir, restando el precio que paga por el bien y el coste de transporte.}
				\4[] Objetivo de maximizador social
				\4[] $\to$ Minimizar costes de transporte
				\4[] $\then$ Equivalente a maximizar excedente social
				\4[] Localización óptima es $\frac{1}{4}$ y $\frac{3}{4}$
				\4[] $\then$ Minimiza cuadráticos y lineales
			\3 Implicaciones
				\4 Costes de transporte determinan poder de mercado
				\4[] $\to$ Interpretables como grado de diferenciación
				\4 Distribución de consumidores determina efecto demanda
				\4[] $\to$ Empresas buscan capturar máxima demanda
				\4[] $\to$ Interpretables como preferencias de consumidores
				\4 Diferenciación como objetivo
				\4[] Permite aumentar precios
				\4[] $\then$ Deseable
				\4[] $\to$ Trade-off con demanda cautiva
				\4 Contexto institucional determina equilibrio
				\4[] Posible modelizar máxima y mínima diferenciación
				\4[] Equilibrio no tiene porqué ser óptimo
				\4 Eficiencia
				\4[] Equilibrio no tiene por qué ser eficiente
				\4[] Óptimo de Pareto:
				\4[] $\to$ minimizar costes de transporte
				\4[] Con costes cuadráticos y lineales
				\4[] $\to$ Una en 1/4 y otra en 3/4
			\3 Valoración
				\4 Sin análisis de entrada
				\4[] Presencia de fronteras
				\4[] $\to$ Dificulta tratabilidad
				\4 No es posible valorar entrada
			\3 Aplicaciones
				\4 Modelo de Hotellings-Down
				\4 Programa de investigación de diferenciación localizada
				\4[] Salop (1979)
				\4 Modelos de economía espacial y geográfica
				\4[] NEG, Krugman, Fujita, Venables, etc...
		\2 Salop (1972)
			\3 Idea clave
				\4 Contexto
				\4[] Diferenciación crea oportunidades de entrada
				\4[] $\to$ Empresas venden por encima de coste marginal
				\4[] $\to$ Existe demanda que capturar con nuevas variedades
				\4[] Entrada no necesariamente socialmente-óptima
				\4[] $\to$ Puede ocurrir entrada excesiva
				\4 Objetivos
				\4[] Valorar entrada en contexto de diferenciación espacial
				\4[] Evitar efectos frontera de Hotelling
				\4[] Analizar optimalidad de entrada libre
				\4 Resultados
				\4[] Modelo de ciudad circular
				\4[] Salop (1979) formulación más conocida
				\4[] Submodelos de Hotelling encadenados en espacio circular
				\4[] Permite análisis de entrada óptima
			\3 Formulación
				\4 Supuestos generales
				\4[] Un círculo de tamaño 1
				\4[] Costes de transporte lineales
				\4[] Empresas con idénticas tecnologías de producción
				\4 Consumidor indiferente entre dos empresas
				\4[] $\bar{u} - p_i - t \tilde{x} = \bar{u} - p - t(\frac{1}{n} - \tilde{x})$
				\4[] $\then$ Dda. de $x_i$: \fbox{$x_i =2\tilde{x} = \frac{p - p_i}{t} + \frac{1}{n}$}
				\4 Maximización de los beneficios
				\4[] Dos etapas:
				\4[] 1. Entrar o no entrar
				\4[] 2. Precio a fijar
				\4[] Costes medios decrecientes
				\4[] $\to$ $C(q) = F + cq$
				\4[] Dos etapas
				\4[] 1. Firmas deciden entrar y dónde situarse
				\4[] 2. Fijación de precios
				\4[] Resolución
				\4[] $\to$ Inducción hacia atrás
				\4 Fijación de precios
				\4[] $\underset{p_i}{\max}\quad \Pi=\left(p_i -c \right) \cdot x_i - F = \left( p_i -c \right) \cdot \left( \frac{p - p_i}{t} + \frac{1}{n} \right) - F$
				\4[] CPO: \quad $\pdv{\Pi}{p_i} = \left(p_i -c \right) \cdot \frac{-1}{t} + \frac{p - p_i}{t} + \frac{1}{n} = 0$
				\4[] $\to$ Asumiendo simetría: $p = p_i$
				\4[] $\then$ $\frac{p_i -c }{t} = \frac{1}{n}$
				\4[] $\then$ \fbox{$p_i = c+\frac{t}{n}$}
				\4[] $\then$ \fbox{$\pi_i = (p_i - c) \frac{1}{n} - F = \frac{t}{n} \cdot \frac{1}{n} - F$}
				\4 Equilibrio de libre entrada
				\4[] Beneficios de empresas deben igualarse a 0
				\4[] $\pi_i =\left( c+  \frac{t}{n} - c \right) \cdot \frac{1}{n} - F = 0$
				\4[] $\to$ $\frac{t}{n^2} = F$
				\4[] $\then$ \fbox{$n^* = \sqrt{\frac{T}{F}}$}
				\4[] $\then$ Empresas entran hasta anular beneficios
			\3 Implicaciones
				\4 Precio por encima de coste marginal
				\4[] Resultado de diferenciación vía costes de transporte
				\4 Beneficio nulo
				\4[] Entran empresas hasta que:
				\4[] $\to$ Beneficio ecónomico iguala costes fijos
				\4 Entrada excesiva de empresas
				\4[] Costes totales
				\4[] $\to$ $C(n) = n\cdot \left( 2 \int_0^\frac{1}{2n} (tx) d \, x + F + \frac{c}{n} \right) = \frac{t}{4n} + nF + c$
				\4[] Minimización de costes totales en función de $n$:
				\4[] $\underset{n}{\min} \quad \frac{t}{4n} + nF + c$
				\4[] CPO: $\quad$ $n_S = \frac{1}{2} \sqrt{\frac{t}{F}}$
				\4[] $n_s = \frac{1}{2} \sqrt{\frac{t}{F}} > \sqrt{\frac{t}{F}} = n^*$
				\4[] $\then$ Factor clave es presencia de costes fijos
				\4[] Sin costes fijos:
				\4[] $\to$ Óptimo es infinitas empresas
				\4[] $\to$ Eq. competitivo y óptimo social son iguales
			\3 Valoración
				\4 Modelo canónico de diferenciación localizada
				\4 Buena caracterización de fenómeno de entrada
				\4 Resultados empíricos habitualmente confirman
		\2 Diferenciación vertical
			\3 Idea clave
				\4 Contexto
				\4[] Diferenciación horizontal
				\4[] $\to$ Si dos variedades se ofrecen a mismo precio
				\4[] $\then$  Habrá demanda positiva de ambas
				\4[] $\then$  Porque hay distintas preferencias
				\4[] $\then$ De gustibus...
				\4[] Diferenciación vertical
				\4[] Si dos variedades se ofrecen al mismo precio
				\4[] $\to$ Sólo habrá demanda de una
				\4 Objetivo
				\4[] Caracterizar relación costes y prod. de distinta calidad
				\4[] Caracterizar número de empresas en el mercado
				\4[] Predecir cuotas de mercado
				\4 Resultados
				\4[] Shaked y Sutton (1982)
				\4[] Variedades se distinguen por calidad
				\4[] Calidad y utilidad aportada
				\4[] $\to$ Relación monótona creciente
				\4[] Entrada depende de coste de mayor calidad
				\4[] $\to$ Si aumento de calidad implica poco $\uparrow$ coste
				\4[] $\then$ Consumidores prefieren variedades más caras
				\4[] Menor relación entre calidad y coste marginal
				\4[] $\to$ Menor número de variedades disponibles
			\3 Formulación
				\4 Consumidores
				\4[] Se distinguen por ingreso $t$
				\4[] Utilidad de consumidor $j$ depende de:
				\4[] $\to$ Variedad consumida $u_i$
				\4[] $\to$ Precio $p_i$
				\4[] $\to$ Ingreso $t$
				\4[] \fbox{$U_j=u_i \cdot (t_j - p_i)$}
				\4[] Ingreso sirve para justificar consumo de baja calidad
				\4 Variedades
				\4[] Se distinguen por utilidad que aportan
				\4[] $\to$ $u_n > u_{n-1} > u_{n-2} > ...$
				\4[] Precios crecientes con utilidad aportada
				\4[] $\to$ $p_n > p_{n-1} > p_{n-2} > ...$
				\4 Empresas
				\4[] Coste marginal creciente con calidad $c(u_i)$
				\4 Equilibrio
				\4[] Asumiendo precio iguala coste marginal
				\4[] Dependerá de coste de producir más calidad
				\4[] Si $c(q_i)$ muy poco creciente
				\4[] $\to$ Todos consumidores preferirán más calidad
				\4[] $\to$ Empresas alta calidad desplazan a baja calidad
				\4[] $\then$ Aparece límite a número de empresas que entran
				\4[] $\then$ Cuota de mercado no se fragmenta
				\4[] Si $c(q_i)$ suficientemente creciente
				\4[] $\to$ Fenómeno contrario
				\4[] $\to$ No todos demandan alta calidad porque precio es alto
				\4[] $\to$ Entra una empresa para cada calidad
				\4[] $\to$ Cuota de mercado muy fragmentada
			\3 Implicaciones
				\4 Posibles cuotas de mercado no fragmentadas
				\4[] Con preferencias homogéneas
				\4 Diferencias con contenido material
				\4[] Sin \textit{de gustibus non est disputandum}
				\4 Costes fijos pueden afectar resultado
				\4[] Interpretables como publicidad
				\4 Buena calidad barata reduce variedad
				\4[] Competencia en precios será más intensa
				\4[] Más calidad a menos precio
				\4[] $\to$ Empresas de baja calidad expulsadas de mercado
				\4[] $\to$ Empresas de + calidad: + costes pero + demanda
			\3 Valoración
				\4 Estudio de cuotas de mercado
				\4[] ¿Más cuota debida a mejor producto?
				\4 Comercio internacional
				\4[] Explicar comercio interindustrial
				\4 Crecimiento económico
				\4[] Modelos de crecimiento endógeno
				\4 Análisis de la publicidad
				\4[] Permite informar de diferente calidad
	\1 \marcar{Competencia monopolística}
		\2 Idea clave
			\3 Enfoque no direccional
				\4 Dimensión de diferenciación no es relevante
				\4[] Localización de diferencia no es relevante
				\4 Asimilable a existencia de muchas dimensiones
				\4 Todas variantes compiten contra todas
			\3 Sistema de funciones de demanda
				\4 Demanda que enfrenta cada productor
				\4 Sin explicitar aspecto diferenciador
				\4[] Sólo elasticidades de sustitución
				\4[] $\to$ Elemento determinante de equilibrio
			\3 Evolución histórica
				\4 Chamberlin (1933) expone de forma vaga
				\4 Spence (1972) y Dixit y Stiglit (1977)
				\4[] $\to$ Formalizan
				\4[] $\to$ Derivan resultados normativos explícitos
		\2 Modelo de Chamberlin (1933)
			\3 Idea clave
				\4 Chamberlin (1933)
				\4[] \textit{The Theory of Monopolistic Competition}
				\4 Inicia programa de investigación
				\4[] Teoría de la competencia monopolística
				\4 Caracteriza economía
				\4[] $\to$ Número elevado de empresas
				\4[] $\to$ Producen variedades diferenciadas
				\4[] $\to$ Miopes respecto a decisión de producción
				\4 Equilibrios subóptimos
				\4[] Excesiva competencia
				\4[] No aprovechan economías de escala
			\3 Formulación
				\4 Supuestos
				\4[] Muchas empresas idénticas
				\4[] Costes medios en forma de U
				\4 Corto plazo: sin libre entrada
				\4[] Número de empresas fijo
				\4[] Explicar a partir de gráfica
				\4[] \grafica{chamberlinsinlc}
				\4[] Empresa produce en punto 1
				\4[] Percibe curva de demanda individual $d$
				\4[] $\to$ Relativamente elástica
				\4[] $\then$ Cree posible $\uparrow$ beneficios aumentando cantidad
				\4[] Realmente enfrenta curva $D$ dda. agregada más inelástica
				\4[] $\to$ Curva $d$ se desplaza hacia abajo
				\4[] Situación se repite
				\4[] $\to$ Hasta precio igual a coste medio
				\4[] $\then$ Desaparece beneficio
				\4[] Número de empresas determina $Q^*$
				\4 Largo plazo: con libre entrada
				\4[] Empresas entran si beneficio positivo
				\4[] Explicar a partir de gráfica
				\4[] \grafica{chamberlinconlc}
				\4[] Empresa se sitúa inicialmente en \fbox{1}
				\4[] $\to$ Percibe demanda $d$
				\4[] $\to$ Beneficios positivos
				\4[] Empresas entran en mercado por beneficios positivos
				\4[] $\to$ Curva $D$ se desplaza a izquierda hasta $D'$
				\4[] $\to$ Se sitúan en \fbox{2}
				\4[] Empresas perciben demanda individual $d'$
				\4[] $\to$ Bajan precio para aprovechar dda. indiv. elástica
				\4[] En realidad, enfrentan demanda agregada
				\4[] $\to$ Precio menor que coste medio
				\4[] $\then$ Beneficio negativo
				\4[] Empresas salen del mercado hasta anular beneficio
				\4[] Demanda agregada se desplaza a derecha
				\4[] Nuevo equilibrio con beneficios nulos \fbox{3}
				\4[] $\to$ Para movimientos en dda. agregada
				\4[] $\to$ Para movimientos en dda. individual
			\3 Implicaciones
				\4 Economías de escala no se realizan
				\4[] Producción menor a escala eficiente
				\4[] Exceso de variedad en el mercado
				\4 Empresas no perciben poder de mercado correcto
				\4[] Demanda agregada e individual es distinta
			\3 Valoración
				\4 Modelo adolece de graves carencias
				\4[] No define claramente el mercado
				\4[] $\to$ ¿En qué grado un sustitutivo es o no relevante?
				\4[] Libre entrada y costes hundidos
				\4[] $\to$ Posible pero difícil
				\4[] $\to$ Supuesto definido difusamente
				\4 Bienestar de los consumidores
				\4[] No se explicitan preferencias
				\4[] No se puede valorar si prefieren más o menos variedad
				\4[] Bienestar no es valorable
				\4 Importancia del programa de investigación
				\4[] Precede a Dixit-Stiglitz
				\4[] Introduce idea de diferenciación entre sustitutivos
				\4[] $\to$ Sin caracterizar dimensión de diferenciación
				\4[] Introduce idea de término medio entre:
				\4[] $\to$ Monopolio
				\4[] $\to$ Competencia perfecta
				\4[] $\then$ Muy importante para organización industrial
		\2 Modelos de Dixit-Stiglitz
			\3 Idea clave
				\4 Contexto
				\4[] Modelos chamberlinianos
				\4[] $\to$ Equilibrio con mark-up
				\4[] $\to$ Entrada excesiva por producción < EEficiente
				\4[] Demanda de variedades por consumidores
				\4[] $\to$ Sin analizar explícitamente
				\4[] Optimalidad valorando bienestar
				\4[] $\to$ No tenida en cuenta
				\4[] Análisis formal en modelos chamberlinianos
				\4[] $\to$ Coherencia interna poco robusta
				\4[] $\to$ Supuestos poco explícitos
				\4 Objetivos
				\4[] ¿Entrada es excesiva si consideramos demanda?
				\4[] ¿Cuántas empresas entran teniendo en cuenta demanda?
				\4[] ¿Cuánto produce cada empresa en equilibrio?
				\4[] ¿Cómo afecta entrada a bienestar social?
				\4 Resultados
				\4[] Aplicación de funciones CES
				\4[] $\to$ Formulación se extiende a muchas otras áreas
				\4[] Entrada de empresas
				\4[] $\to$ Hasta anular beneficios
				\4[] $\to$ Más variedades cuanto mayor demanda total
				\4[] Equilibrio competitivo es óptimo de second-best
				\4[] $\to$ No es óptimo first-best porque precio > coste marginal
				\4[] $\to$ Es óptimo second-best porque entrada aumenta bienestar
			\3 Formulación\footnote{Basada en Caillaud (2016), diapositivas de PSE. Ver también paper original de Dixit y Stiglitz (1977) para implicaciones relativas a la optimalidad del equilibrio competitivo.}
				\4 Consumidores: etapas de decisión
				\4[] 1. Etapa: decisión entre homogéneo y compuesto
				\4[] $\to$ Bien homogéneo no diferenciado X
				\4[] $\to$ Bien compuesto por bienes diferenciados C
				\4[] $\then$ Dada función de producción CES de bien compuesto
				\4[] $\then$ Distribuyen renta entre ambos
				\4[] 2. Etapa: decisión entre bienes diferenciados
				\4[] $\to$ Cómo distribuir presupuesto de bien compuesto
				\4[] $\then$ Cuánto consumir de cada bien diferenciado
				\4[] $\then$ Maximizar cantidad de bien homogéneo consumido
				\4 Consumidores: primera etapa
				\4[] Maximizar utilidad distribuyendo renta W entre:
				\4[] $\to$ Bien homogéneo X a precio $P_X$
				\4[] $\to$ Bien compuesto C a precio $P_C$ (índice de precios)
				\4[] Problema de maximización
				\4[] $\underset{X, C}{\max} \quad U = X^{1-\gamma} C^\gamma$
				\4[] s.a: $\quad$ $P_x X + P_C C = W$
				\4[] $\to$ $L = X^{1-\gamma} C^\gamma - \lambda (P_x X + P_C C - W)$
				\4[] $\then$ CPO: $\quad \frac{P_C C}{P_X X} = \frac{\lambda}{1-\lambda}$
				\4[] $\then$ \quad $X = \frac{1-\gamma}{\gamma} \frac{P_C}{P_X}$
				\4[] $\then$ \fbox{$P_C C = W\cdot \gamma$}
				\4 Consumidores: segunda etapa
				\4[] Distribuir gasto $P_C C$ en bienes diferenciados $C(i)$
				\4[] $\to$ Max. cantidad de bien compuesto producido
				\4[] $\underset{c_i}{\max} \quad \left( \sum_{i=1}^n c_i^{\frac{\epsilon-1}{\epsilon}} \right)^{\frac{\epsilon}{\epsilon -1}}$
				\4[] s.a: $\quad$ $\sum_{i=1}^n P_i c_i \leq I \equiv W \cdot \gamma$
				\4[] $\then$ Demanda de $c_i$: \fbox{$c_i = \frac{I}{P_C} \cdot \left( \frac{P_C}{P_i} \right)^\epsilon$}
				\4[] $\then$ Elasticidad dda. \footnote{Demostración: $\epsilon_{i-p_i} = \frac{d C_i }{d \, P_i} \cdot \frac{P_i}{C_i} = \underbrace{-\epsilon \cdot \frac{I}{P} \cdot P^\epsilon \cdot P_i^{-\epsilon - 1} }_{\frac{d \, C_i}{d \, P_i}} \cdot \frac{P_i}{\underbrace{\frac{I}{P} \cdot P^\epsilon \cdot P_i^{-\epsilon}}_{C_i}} = - \epsilon $}. $i$ a $p_i$: \fbox{$\epsilon_{i-p_i} = - \epsilon$}
				\4[] $\then$ ÍPrecios de diferenciados: \fbox{$P_C = \left( \sum_i p_i^{1- \epsilon} \right)^{\frac{1}{1-\epsilon}}$}
				\4 Empresas de diferenciados: maximización de beneficios
				\4[] Maximizar beneficios dada demanda de bien diferenciado
				\4[] $\underset{q_i}{\max} \quad \Pi = p_i (q_i) q_i - c q_i - F$
				\4[] $\to$ CPO: \quad $\frac{p_i - c_i}{p_i} = \frac{1}{\left| \epsilon \right|}$
				\4[] $\then$ $p_i = \frac{\epsilon}{\epsilon-1} c$
				\4[] $\then$ $p_i \cdot \frac{\epsilon-1}{\epsilon} = c$
				\4 Equilibrio de libre entrada
				\4[] Entran empresas hasta que los beneficios se anulan:
				\4[] $\to$ $\Pi = p_i q_i - c q_i - F = 0$
				\4[] $\to$ $(p_i -c) q_i = F$
				\4[] $\then$ $(p_i - \frac{\epsilon - 1}{\epsilon} p_i) q_i = F$
				\4[] $\then$ $\underbrace{p_i q_i}_{\frac{I}{n \epsilon}} \cdot \frac{1}{\epsilon} = F$
				\4[] $\then$ $\frac{I}{n^* \epsilon}= F$ $\then$ \fbox{$n^* = \frac{I}{F \epsilon}$}
				\4[] $\to$ $ \left( \frac{\epsilon}{\epsilon-1}c_i - c_i \right) q_i= F$
				\4[] $\to$ $c q_i \left( \frac{1}{\epsilon -1 } \right) = F$

				\4[] $\then$ \fbox{$q_i = \frac{F}{c} \cdot \left( \epsilon -1 \right)$}
			\3 Implicaciones
				\4 Precio por encima de coste marginal
				\4[] $p_i = \frac{\epsilon}{\epsilon -1} c$
				\4[] $p_i \frac{\epsilon -1}{\epsilon} = c$
				\4[] $\then$ Igual a equilibrio competitivo de monopolio
				\4 Mark-up no depende de número de variedades
				\4[] Sólo de elasticidad de sustitución $\epsilon$
				\4 Producción de $c_i$ no depende de nº de variedades
				\4[] $q_i = \frac{F}{c} \cdot \left( \epsilon - 1 \right)$
				\4[] Cantidad producida no depende de número de variedades
				\4[] $\to$ Sí de elast. de sust., costes fijos y marginales
				\4 Número de variedades con libre entrada
				\4[] $n^* = \frac{I}{F\cdot \epsilon}$
				\4[] Depende positivamente de gasto total en diferenciados
				\4[] $\to$ Aumenta con tamaño del mercado
				\4[] Cae con coste fijo
				\4[] Cae con elasticidad de sustitutición
				\4[] $\to$ Menos posibilidad de aplicar mark-up a coste marginal
				\4[] $\then$ Menores beneficios
				\4 Eficiencia de primer orden
				\4[] ¿Es el eq. óptimo de Pareto...
				\4[] ...admitiendo beneficios negativos de empresas?
				\4[] $\to$ No, porque $p \neq \text{CMg}$
				\4 Eficiencia de segundo orden
				\4[] ¿Es eq. óptimo de Pareto...
				\4[] ...sujeto a que beneficios no sean negativos?
				\4[] $\to$ Dixit y Stiglitz muestran que sí
				\4 Exceso de capacidad
				\4[] Conclusión habitual de Chamberlin (1933)
				\4[] $\to$ No se agotan las economías de escala
				\4[] $\then$ CMonopolística produce exceso de capacidad
				\4[] Dixit-Stiglitz:
				\4[] $\to$ Escala eficiente no es necesaria
				\4[] $\to$ Porque consumidores prefieren variedad
				\4 Diferencias con oligopolio
				\4[] Con Cournot
				\4[] $\to$ Precio no converge a CMg con más empresas
				\4[] Con Bertrand
				\4[] $\to$ Precio no converge a CMg
			\3 Aplicaciones
				\4 Comercio intraindustrial
				\4 Heterogeneidad de empresas
				\4 Modelos macroeconómicos de NEK
				\4 ...
			\3 Valoración
				\4 Uno de los artículos más citados de la historia
				\4 Mejoras respecto a Chamberlin (1933)
				\4[] Chamberlin tiene graves carencias:
				\4[] $\to$ no explicita demanda
				\4[] $\to$ no formaliza conclusiones
				\4[] $\to$ no caracteriza eficiencia del equilibrio
				\4 Extensiones
				\4[] Dixit y Stiglitz (1977) considera varias
				\4[] Elasticidades variables y asimétricas
				\4[] $\to$ Entre variedades de producto
				\4[] $\to$ Diferentes conclusiones respecto eficiencia
				\4 Aplicaciones
				\4[] Macroeconomía: modelos DSGE de la NEK
				\4[] Comercio internacional: Nueva Teoría del CI
				\4[] Geografía económica: Krugman y Venables
	\1[] \marcar{Conclusión}
		\2 Recapitulación
			\3 Oligopolio con diferenciación de producto
			\3 Competencia monopolística
		\2 Idea final
			\3 Relevancia de la competencia monopolística
				\4 Enorme impacto en otras disciplinas
				\4 Modelos macro con sector monetario relevante
				\4[] Empresas tienen poder de mercado
				\4[] Rigideces nominales
				\4[] $\then$ Efectos reales de shocks nominales
				\4 Comercio intraindustrial
				\4 Crecimiento económico
				\4 ...
			\3 Análisis económico de la publicidad
				\4 Muy conectado con CMonopolística y DProducto
				\4 ¿Para que sirve la publicidad?
				\4[] $\to$ Transmitir información
				\4[] $\to$ Alterar preferencias de consumidores
				\4 Diferenciación de producto permite analizar
			\3 Impacto general sobre ciencia económica
				\4 Análisis más acertado de mercados reales
				\4 Ha permitido avance de muchas otras áreas
				\4 Similitud con propio Dixit-Stiglitz
				\4[] Modelización económica como bien compuesto
				\4[] Más variedad de modelos y mecanismos
				\4[] $\to$ Mejor comprensión de la realidad
\end{esquemal}



























\graficas

\begin{axis}{4}{Representación gráfica de las funciones de reacción de los precios en un contexto de Bertrand diferenciado con costes marginales fijos e iguales y con diferenciación.}{$P_1$}{$P_2$}{bertranddiferenciado}
	% Empresa 1
	\node[below] at (1,0){c};
	\draw[thick] (1,0) -- (3,4);
	
	% Empresa 2
	\node[left] at (0,1){c};
	\draw[thick, color=red] (0,1) -- (4,3);
	
	% Equilibrio
	\draw[dashed] (0,2) -- (2,2) -- (2,0);
	\node[below] at (2,0){$P_1^*$};
	
	\node[left] at (0,2){$P_2^*$};
	
\end{axis}

\begin{dibujo}{4}{Eje en el que se localizan los consumidores en un modelo de Hotelling}{x}{y}{hotelling}
	% Línea horizontal
	\draw[-] (-2,0) -- (2,0);
	% pequeñas líneas verticales al final de la línea
	\draw[-] (-2,-0.3) -- (-2,0.3); % izquierda
	\draw[-] (2,-0.3) -- (2,0.3);
	\node[below] at (-2,-0.5){0};
	\node[below] at (2,-0.5){1};

\end{dibujo}


\begin{dibujo}{4}{Modelo de Hotelling (1929): localizaciones exógenas y competencia en precios}{x}{y}{hotellinglocexogena}
	% Línea horizontal
	\draw[-] (-3,0) -- (3,0);
	
	% Pequeñas líneas verticales al final de la línea
	\draw[-] (-3,-0.15) -- (-3,0); % izquierda
	\draw[-] (3,-0.15) -- (3,0);
	\node[below] at (-3,-0.15){0};
	\node[below] at (3,-0.15){1};
	
	% Localización de empresas
	\node[above] at (-3,0.05){\small a};
	\node[above] at (3,0.05){\small b};
	
	% Precios de equilibrio
	\draw[-] (-3,1.2) -- (-3.2,1.2);
	\node[left] at (-3.2,1.2){$p^*$};
	\draw[-] (3,1.2) -- (3.2,1.2);
	\node[right] at (3.2,1.2){$p^*$};
	
	% Costes de cada localización de las dos empresas
	\draw[-] (-3,1.2) -- (1,4);
	\draw[-] (3,1.2) -- (-1,4);
	
	% Consumidor indiferente de equilibrio
	\draw[dotted] (0,3.2) -- (0,0);
	\node[below] at (0,0){$\tilde{x}$};
	
	% Costes tras bajada hipotética de precios
	\draw[dashed] (-3,0.6) -- (1,3.4);
	
	% Precio tras bajada hipotética
	\draw[-] (-3,0.6) -- (-3.2,0.6);
	\node[left] at (-3.2,0.6){$p^*-\alpha$}; % La bajada alpha es de -0.6
	
	
	% Consumidor indiferente tras bajada de precio de una empresa
	\draw[dotted] (0.44,3) -- (0.44,0);
	\node[below] at (0.44,-0.06){$x$};
	
	% Flechas indicando desplazamiento de consumidor indiferente
	\draw[-{Latex}] (0.05,1.5) -- (0.43,1.5);
	
\end{dibujo}

La relación entre rebajas $\alpha$ de precio y desplazamientos de consumidor indiferente $\Delta x$ crecen a medida que la función de costes de transporte reduce su convexidad. Así, si los costes de transporte crecen muy despacio, aumentos del precio tendrán un efecto muy fuerte sobre el consumidor indiferente, que se desplazará hacia la empresa que sube el precio. Cuando los costes de transporte crecen rápidamente con la distancia, se produce el efecto contrario: aumentos del precio tendrán consecuencias poco importantes sobre la localización del consumidor indiferente. Así, cuando los costes crecen fuertemente con la distancia, las empresas podrán subir los precios y perder menos clientes. Es decir, verán aumentado su poder de mercado. Por ello, cuando las empresas deciden localizarse de forma endógena y después competir en precios, la forma de la función de costes es determinante para las localizaciones de equilibrio. Cuando el coste de transporte crece muy poco con la distancia, la competencia en precios por un pedazo de mercado será muy fuerte, independientemente de su localización. En estos casos, el efecto demanda será más importante que el efecto de la competencia en precios, y las empresas preferirán situarse en el centro para tener un segmento de demanda cautiva lo más grande posible. Cuando los costes son cuadráticos, el coste de transporte crece mucho más rápido ante un aumento de la distancia, y las empresas preferirán situarse lo más lejos posible del centro para poder subir los precios. 

\begin{axis}{4}{Modelo de Chamberlin: equilibrio sin libre entrada de nuevas empresas}{Q}{P}{chamberlinsinlc}
	% Curva de costes medios
	\draw[-] (0.2,4) to [out=280, in=180](2.7,0.8);
	\draw[-] (2.7,0.8) to [out=0, in=265](4,4);
	
	% Curva de demanda agregada
	\draw[-] (1.19,4) -- (1.9,0.2);
	\node[above] at (1.19,4){\tiny $D$};
	
	% Curva de demanda individual inicial 
	\draw[-] (0.5,3.5) -- (4,1.5);
	\node[right] at (4,1.5){\tiny $d$};
	
	% Precio inicial
	\draw[dashed] (0,3) -- (1.38,3);
	\node[left] at (0,3){\tiny $P_0$};
	\node[circle,fill=black,inner sep=0pt,minimum size=3pt] (a) at (1.38,3) {};
	\node[right] at (1.38,3.07){\tiny $1$};
	
	% Curva de demanda individual real tras bajada de precio
	\draw[dashed] (0.5,2.6) -- (4,0.6);
	
	% Curva de demanda individual final tras desplazamiento
	\draw[dashed] (0.5,1.75) -- (3,0.32143);
	\node[right] at (3,0.32143){\tiny $d$};
	
	% Equilibrio final
	\node[circle,fill=black,inner sep=0pt,minimum size=3pt] (a) at (1.73,1.05) {};
	\node[right] at (1.73,1.14){\tiny $2$};
	\draw[dashed] (0,1.05) -- (1.73,1.05) -- (1.73,0);
	\node[left] at (0,1.05){\tiny $P^*$};
	\node[below] at (1.73,0){\tiny $Q^*$};
	
	% Escala eficiente
	\draw[dashed] (0,0.8) -- (2.7,0.8) -- (2.7,0);
	\node[below] at (2.7,0){\tiny $Q_\text{EE}$};
	\node[left] at (0,0.8){\tiny $P_\text{EE}$};
\end{axis}

\begin{axis}{4}{Modelo de Chamberlin: equilibrio con libre entrada y producción a la izquierda de escala eficiente}{}{P}{chamberlinconlc}
	% Extender eje de abscisas
	\draw[-] (4,0) -- (6,0);
	\node[below] at (6,0){Q};
	
	
	% Curva de costes medios
	\draw[-] (1.2,4) to [out=280, in=180](3.7,0.8);
	\draw[-] (3.7,0.8) to [out=0, in=265](5,4);
	
	% Escala eficiente
	\draw[dashed] (3.7,0.8) -- (3.7,0);
	\node[below] at (3.7,0){$q_\text{EE}$};
	
	% Curva de demanda agregada inicial
	\draw[-] (3.8,4) -- (4.9,0.2);
	\node[right] at (4.9,0.2){\tiny $D$};
	% equilibrio inicial
	\node[circle,fill=black,inner sep=0pt,minimum size=3pt] (a) at (4.08,3.03) {};
	\node[right] at (4.08,3.05){\tiny 1};
	
	% Curva de demanda agregada intermedia que induce beneficios negativos cuando empresas bajan precio
	\draw[dashed] (1.14,4) -- (2.24,0.2);
	\node[right] at (2.24,0.2){\tiny $D'$};
	% equilibrio inicial
	\node[circle,fill=black,inner sep=0pt,minimum size=3pt] (a) at (1.43,3) {};
	\node[right] at (1.43,3.05){\tiny 2};
	
	% Curva de demanda agregada final
	\draw[-] (1.8,4) -- (2.94,0.2);
	\node[right] at (2.94,0.2){\tiny $D''$};
	% equilibrio inicial
	\node[circle,fill=black,inner sep=0pt,minimum size=3pt] (a) at (2.7,1.07) {};
	\node[right] at (2.7,1.1){\tiny 3};

	% Curva de demanda individual inicial
	% curvas de demanda: y = a - 2/(3.5)x 
	\draw[-] (1.5,4.5) -- (5,2.5);
	\node[left] at (1.5,4.5){\tiny $d$};

	% Curva de demanda individual intermedia 
	% y = 3.817143 - (2/3.5)x
	\node[left] at (0.5, 3.53143){\tiny $d'$};
	\draw[dashed] (0.5, 3.53143) -- (3.5,1.81743);
	
	% Curva de demanda individual final
	\draw[-] (0.5,2.32) -- (3.5,0.61);
	\node[left] at (0.5,2.32){\tiny $d''$};

	% Precio inicial
%	\draw[dashed] (0,3) -- (1.38,3);
%	\node[left] at (0,3){\tiny $P_0$};
%	\node[circle,fill=black,inner sep=0pt,minimum size=3pt] (a) at (1.38,3) {};
%	\node[right] at (1.38,3.07){\tiny $1$};
%	
%	% Curva de demanda individual real tras bajada de precio
%	\draw[dashed] (0.5,2.6) -- (4,0.6);
%	
%	% Curva de demanda individual final tras desplazamiento
%	\draw[dashed] (0.5,1.75) -- (3,0.32143);
%	\node[right] at (3,0.32143){\tiny $d$};
%	
%	% Equilibrio final
%	\node[circle,fill=black,inner sep=0pt,minimum size=3pt] (a) at (1.73,1.05) {};
%	\node[right] at (1.73,1.14){\tiny $2$};
%	\draw[dashed] (0,1.05) -- (1.73,1.05) -- (1.73,0);
%	\node[left] at (0,1.05){\tiny $P^*$};
%	\node[below] at (1.73,0){\tiny $Q^*$};
%	
%	% Escala eficiente
%	\draw[dashed] (0,0.8) -- (2.7,0.8) -- (2.7,0);
%	\node[below] at (2.7,0){\tiny $Q_\text{EC}$};
%	\node[left] at (0,0.8){\tiny $P_\text{EC}$};
\end{axis}

\conceptos

\preguntas


\seccion{16 de marzo de 2017}
\begin{itemize}
    \item ¿Para qué sirven estos modelos de competencia monopolística en la realidad?
    
    \item Pero imagine que trabaja en competencia, ¿como sabemos que una posición es dominante? ¿El fabricante de diamantes De Beers tiene esa posición?
\end{itemize}

\seccion{Test 2018}

\textbf{10.} En el equilibrio a largo plazo en competencia monopolística:

\begin{itemize}
	\item[a] La empresa puede tener beneficios positivos.
	\item[b] El beneficio cero garantiza la eficiencia en la asignación de recursos.
	\item[c] Se produce mayor cantidad y se vende a un precio mayor que en competencia perfecta.
	\item[d] La empresa tiene exceso de capacidad.
\end{itemize}

\seccion{Test 2013}

\textbf{4.} En los modelos con diferenciación horizontal y localización: 

\begin{itemize}
	\item[a] Los consumidores tienen preferencias distintas.
	\item[b] Los bienes son homogéneos.
	\item[c] Los bienes tienen calidades distintas.
	\item[d] Los consumidores tienen rentas distintas.
\end{itemize}

\textbf{11.} Considere el modelo de ciudad lineal de Hotelling asentada en el intervalo cerrado [0,1] y donde hay dos establecimientos (A y B) localizados en los extremos A en 0 y B en 1. Los consumidores está distribuidos uniformemente y tienen un coste de desplazamiento para comprar o de transporte igual a 2 por unidad de distancia (X) recorrida al cuadrado.
Si el precio $P(A)=1$ y el precio de $P(B)=3$, ¿en qué tramo de la ciudad lineal se asentará el consumidor que es indiferente entre comprar en A o en B?

\begin{enumerate}
	\item[a] 1/3
	\item[b] 1/2
	\item[c] 2/3
	\item[d] 3/4
\end{enumerate}

\seccion{Test 2008}

\textbf{7.} En el contexto de un modelo de competencia monopolística, cuando las empresas funcionan con costes medios en forma de U a largo plazo:

\begin{itemize}
	\item[a] La elasticidad de la demanda a la que se enfrenta una empresa depende únicamente de la naturaleza del bien.
	\item[b] En el equilibrio cada empresa tiene un exceso de capacidad instalada, por lo cual produce con rendimientos a escala crecientes.
	\item[c] En el equilibrio las empresas producen el mínimo de los costes medios, esto es, con rendimientos a escala constantes.
	\item[d] En el equilibrio sólo se garantiza que la competencia entre las empresas sea máxima según el grado de diferenciación del mercado.
\end{itemize}

\notas

\textbf{2018:} \textbf{10.} D

\textbf{2013:} \textbf{4.} B

\textbf{2008:} \textbf{7.} B

\bibliografia

Mirar en Palgrave:

\begin{itemize}
	\item advertising
	\item Chamberling, Edward Hastings
	\item competition
	\item market structure
	\item monopolistic competition
	\item monopolistic competition and general equilibrium
	\item oligopoly
	\item product differentiation
	\item Robinson, Joan Violet
	\item spatial economics
\end{itemize}


Baldwin, Forslid, Martin, Ottaviano, Robert-Nicoud. \textit{Economic Geography and Public Policy} (2002) Appendix A: All you wanted to know about Dixit-Stiglitz but were afraid to ask -- En carpeta del tema

Caillaud, B. (2016) \textit{Product Differentiation} Paris School of Economics. Master APE. Class slides -- En carpeta del tema

Dingel, J. I. \textit{The basics of ``Dixit-Stiglitz lite''} (2009) -- En carpeta del tema

Dixit, A. K.; Stiglitz, J. E. \textit{Monopolistic Competition and Optimum Product Diversity} (1977) No. 3 (Jun. 1977) American Economic Review -- En carpeta del tema

Hotelling, H. \textit{Stability in competition} (1929) The Economic Journal -- En carpeta del tema

Machado, M. \textit{Industrial Organization - Master de Economía Industrial. Teaching Slides} -- \url{http://www.eco.uc3m.es/~mmachado/Teaching/OI-I-MEI/index.html}

Rotschild, R. \textit{The Theory of Monopolistic Competition:E.H. Chamberlin's Influence on Industrial Organisation Theory over Sixty Years} -- En carpeta del tema. PARA CHAMBERLIN.

Stiglitz, J. E. \textit{Monopolistic competition, the Dixit-Stiglitz model, and economic analysis} (2017) Research in Economics 71 -- En carpeta del tema

Sutton, J. \textit{Vertical Product Differentiation: Some Basic Themes} (1986) American Economic Review -- En carpeta del tema

\end{document}
