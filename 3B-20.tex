\documentclass{nuevotema}

\tema{3B-20}
\titulo{La coordinación internacional de las políticas macroeconómicas.}

\begin{document}

\ideaclave

Ver \href{https://www.piie.com/system/files/documents/wp16-9.pdf}{Blanchard (2016)} sobre Currency Wars, Coordination, Capital Controls

Los estados pueden tomar decisiones de política monetaria y fiscal de forma unilateral. Sin embargo, los efectos de estas decisiones afectan también a otras economías. Así, la política fiscal y monetaria se puede representar en un marco de teoría de juegos, dada esa presencia de interdependencia estratégica entre las acciones de unos y otros.

Sin embargo, las consecuencias de las políticas no son uniformes. Dependen del contexto, y de la reacción de los adversarios, así como de las percepción que los estados tengan en relación al modelo que mejor se ajuste a la realidad.

Este tema está íntimamente conectado con los dos siguientes temas: sistema económico internacional hasta bretton woods (21) y sistema económico internacional post bretton woods (22). En ambos temas, la coordinación de políticas es fundamental

El paper de Eichengreen explica muy bien los diferentes escenarios, desde una perspectiva moderna y con ejemplos históricos. El artículo de la Palgrave define de forma bastante clara por qué interesa la coordinación, y también por qué no. Entre los dos artículos es posible construir un tema que conteste a:

\begin{itemize}
    \item ¿Cuando deben coordinarse las políticas macroeconómicas?
    \item ¿Cómo pueden coordinarse las políticas macroeconómicas?
    \item ¿En qué contextos es deseable la cooperación macroeconómica?
    \item ¿De qué factores depende que la cooperación macroeconómica sea deseable?
    \item ¿En qué situaciones no es deseable la cooperación macroeconómica?
    \item ¿En qué momentos se han coordinado?

\end{itemize}

\esquemacorto

\begin{esquema}[enumerate]
	\1[] \marcar{Introducción}
		\2 Contextualización
			\3 Concepto de cooperación
			\3 Múltiples tipos de cooperación entre estados
			\3 Políticas macroeconómicas
			\3 Múltiples ejemplos a lo largo de la historia
			\3 Múltiples modelos de representación
		\2 Objeto
			\3 ¿Cuando deben coordinarse las políticas macroeconómicas?
			\3 ¿Cómo pueden coordinarse las políticas macroeconómicas?
			\3 ¿En qué contextos es deseable la cooperación macroeconómica?
			\3 ¿De qué factores depende que la cooperación macroeconómica sea deseable?
			\3 ¿En qué situaciones no es deseable la cooperación macroeconómica?
			\3 ¿Hasta qué punto debe integrarse la implementación de políticas macroeconómicas?
			\3 ¿En qué momentos se han coordinado?
		\2 Estructura
			\3 Análisis teórico
			\3 Evidencia empírica
	\1 \marcar{Análisis teórico}
		\2 Idea clave
			\3 Contexto
			\3 Objetivos
			\3 Resultado
		\2 Enfoques de modelización
			\3 Teoría de juegos
			\3 Mundell-Fleming
			\3 NOEM
		\2 Coordinación de política fiscal
			\3 Locomotora fiscal
			\3 Disciplina fiscal
			\3 Valoración
		\2 Coordinación de política monetaria
			\3 Devaluación competitiva
			\3 Apreciación competitiva
			\3 Valoración
		\2 Instrumentos de cooperación
			\3 Organismos de cooperación
			\3 Mecanismos informales
			\3 Reglas
		\2 Argumentos contra la cooperación
			\3 Idea clave
			\3 TCN flexible aísla de políticas externas
			\3 Devaluación competitiva provee liquidez
			\3 Integración reduce necesidad de cooperación
			\3 Pérdida de credibilidad de PM
			\3 Riesgo moral
			\3 Economía política nacional
			\3 Beneficios de segundo orden
	\1 \marcar{Evidencia empírica}
		\2 Patrón Oro
			\3 Contexto
			\3 Actuaciones
			\3 Implicaciones
		\2 Bretton Woods
			\3 Contexto
			\3 Actuaciones
			\3 Implicaciones
		\2 Volatilidad cambiaria en los 80
			\3 Contexto
			\3 Actuaciones
			\3 Implicaciones
		\2 Crisis financiera
			\3 Contexto
			\3 Actuaciones
			\3 Implicaciones
		\2 Pandemia de coronavirus
			\3 Contexto
			\3 Actuaciones
			\3 Implicaciones
		\2 Factores de coordinación
			\3 Aspectos técnicos frente a políticos
			\3 Existencia de marco institucional
			\3 Preservación de políticas ya existentes
			\3 Conflictos paralelos
			\3 Percepción sobre modelos compatible
	\1[] \marcar{Conclusión}
		\2 Recapitulación
			\3 Análisis teórico
			\3 Evidencia empírica
		\2 Idea final
			\3 Sentido de la coordinación macro internacional
			\3 Evolución de la cooperación macro
			\3 Situación económica actual

\end{esquema}

\esquemalargo












\begin{esquemal}
	\1[] \marcar{Introducción}
		\2 Contextualización
			\3 Concepto de cooperación
				\4 Toma de decisiones conjunta
				\4[] Entre dos unidades de decisión independientes
				\4[] $\to$ A partir de información compartida
				\4[] $\then$ Sobre efecto de decisiones
				\4[] $\then$ Sobre reparto de beneficios/costes
				\4 Objetivo
				\4[] Alcanzar un estado superior a decisión independiente
			\3 Múltiples tipos de cooperación entre estados
				\4 Política comercial
				\4 Regulación
				\4 Salud
				\4 Medio ambiente
				\4 Energía
				\4 Fiscalidad micro
				\4 Políticas macroeconómicas
			\3 Políticas macroeconómicas
				\4 Políticas fiscal y monetaria
				\4 Potenciales efectos generalizados
				\4 Potenciales spill-overs más allá de fronteras
				\4[] Por acción y omisión
			\3 Múltiples ejemplos a lo largo de la historia
				\4 Apoyo mutuo en patrón oro de pre-guerra
				\4 Bretton Woods
				\4 Crisis Financiera y estímulos G20
			\3 Múltiples modelos de representación
				\4 Herramientas de modelización de otros problemas económicos
				\4 Teoría de juegos
				\4 Mundell-Fleming
				\4 NOEM
		\2 Objeto
			\3 ¿Cuando deben coordinarse las políticas macroeconómicas?
			\3 ¿Cómo pueden coordinarse las políticas macroeconómicas?
			\3 ¿En qué contextos es deseable la cooperación macroeconómica?
			\3 ¿De qué factores depende que la cooperación macroeconómica sea deseable?
			\3 ¿En qué situaciones no es deseable la cooperación macroeconómica?
			\3 ¿Hasta qué punto debe integrarse la implementación de políticas macroeconómicas?
			\3 ¿En qué momentos se han coordinado?
		\2 Estructura
			\3 Análisis teórico
			\3 Evidencia empírica
	\1 \marcar{Análisis teórico}
		\2 Idea clave
			\3 Contexto
				\4 Economía internacional
				\4[]
				\4 Macroeconomía de economía abierta
				\4[] Entender y predecir efectos de políticas macro
				\4[] $\to$ Efectos más allá de fronteras nacionales
				\4[] $\to$
				\4 Interdependencia estratégica
				\4[] Concepto de análisis de teoría de juegos
				\4[] Toma de decisiones
				\4[] $\to$ Considerando efectos sobre decisión de otros
				\4 Macro policy-maker como agente decisor
				\4[] Políticas tienen efectos sobre:
				\4[] $\to$ Economía nacional
				\4[] $\to$ Otras economías
				\4[] $\to$ Segunda ronda
				\4 Información
				\4[] Permite:
				\4[] $\to$ Valorar decisión óptima
				\4[] $\to$ Negociar reparto de beneficios y costes
			\3 Objetivos
				\4 Valorar optimalidad de cooperación
				\4 Valorar optimalidad de no cooperación
			\3 Resultado
				\4 Modelos teóricos representan escenario concreto
				\4 Subsumir situación real en escenario concreto
				\4 Diferentes modelos identifican diferentes problemas
				\4[] Factores que favorecen cooperación
				\4[] Efectos de políticas macro sobre otros decisores
		\2 Enfoques de modelización
			\3 Teoría de juegos
				\4 Idea clave
				\4[] Representación formalizada de decisión estratégica
				\4[] Marco de análisis de interdependencia estratégica
				\4 Formulación
				\4[] Jugadores
				\4[] Estrategias
				\4[] Creencias
				\4[] Soluciones
				\4 Valoración
				\4[] Pieza central de análisis de cooperación
				\4[] Postulación explícita de efectos de políticas macro
				\4[] No siempre utilizado explícitamente
				\4[] $\to$ Muchas macroeconomías ``pequeñas''
				\4[] $\then$ Interdependencia estratégica no es relevante
			\3 Mundell-Fleming
				\4 Idea clave
				\4[] Contexto
				\4[] $\to$ IS-LM
				\4[] $\to$ Énfasis en demanda vs oferta
				\4[] Objetivo
				\4[] $\to$ Caracterizar efecto de PM y PF
				\4[] $\to$
				\4[] Resultados
				\4 Formulación
				\4[] En contexto de coordinación internacional
				\4[] $\to$ Dos países A y B
				\4[] $Y_A = C(Y) + I(r) + \underbrace{X(Y_B,S) - M(Y_A, S)}_{NX(Y_A,Y_B,S)}$
				\4[] (asumiendo $S$ es TCN directo)
				\4[] $\to$ $NX(Y,Y^*,S) = X(Y_B, S) - M(Y_A,S)$
				\4[] $\then$ $X$ aumentan con $\uparrow Y_B$ y $\uparrow S$
				\4[] $\then$ $M$ aumentan con $\uparrow Y_B$ y $\downarrow S$
				\4[] $\frac{M}{P} = L(Y,r)$
				\4[] $ \Delta R = NX (Y_A, Y_BS) - CF(r - r^*)$
				\4[] $\to$ Elasticidad CF a $r-r^*$ $\to$ $\infty$
				\4[] Asumiendo
				\4[] $\to$ Exceso de capacidad en A y B
				\4[] $\to$ No existe ni se espera inflación
				\4[] $\to$ Déficit y deuda pública sin efectos
				\4 Canal de transmisión: renta directo
				\4[] PM/PF expansiva en A aumentan renta en A
				\4[] $\to$ Aumento de demanda de importaciones en A
				\4[] $\then$ Aumento de exportaciones en B
				\4[] $\then$ Aumento de renta en B

				\4[] Tiene lugar con TCN fijo y flexible
				\4[] $\to$ Resulta de aumento de demanda agregada
				\4 Canal de transmisión: capital
				\4[] PM expansiva en A reduce interés en A
				\4[] $\to$ Capital sale de A hacia B
				\4[] $\then$ Exceso de demanda de divisa de B
				\4[] $\then$ Presión hacia apreciación de divisa de B
				\4[] PF contractiva en A aumenta interés en A
				\4[] $\to$ Capital sale de A hacia B
				\4[] $\then$ Exceso de
				\4[] Respuesta de país que sufre efecto
				\4[] $\to$ Depende de régimen cambiario comprometido
				\4[] País B fija TCN con A
				\4[] $\to$ PM expansiva en B para reducir entrada de K
				\4[] $\then$ TCN se mantiene
				\4 Canal de transmisión: relación real de intercambio
				\4[] Depreciación del TCN en A, precios constantes
				\4[] $\to$ Apreciación real de B
				\4[] $\then$ Aumento de exportaciones de A a B
				\4[] $\then$ Caída de exportaciones de B a A
			\3 NOEM
				\4 Idea clave
				\4[] Modelos DSGE
				\4[] $\to$ Microfundamentados
				\4[] $\to$ Equilibrio general
				\4[] $\to$ Estocasticidad
				\4[] $\to$ Análisis de bienestar posible
				\4[] Modelos neo-keynesianos/nueva síntesis neoclásica
				\4[] $\to$ Competencia imperfecta
				\4[] $\to$ Poder de mercado
				\4[] $\to$ Rigideces nominales y reales
				\4[] $\to$ Dinero no es neutral
				\4[] Aplicación a economía abierta
				\4[] $\to$ Modelización de regiones/economía mundial
				\4[] $\to$ Análisis de bienestar de políticas
				\4[] $\to$ Fácil extensión
				\4[] $\to$ Admite enorme variedad de supuestos
				\4[] Autores principales
				\4[] $\to$ Svensson y van Wijnbergen (1989)
				\4[] $\to$ Obstfeld y Rogoff (1995a)
				\4 Formulación
				\4[] Consumidores
				\4[] $\to$ Consumen bien compuesto de bienes diferenciados
				\4[] $\to$ Venden trabajo y capital a empresas
				\4[] $\to$ Posible demandan dinero
				\4[] $\to$ Invierten en activos financieros
				\4[] Empresas
				\4[] $\to$ Producen bien diferenciado
				\4[] $\to$ Enfrentan costes marginales crecientes
				\4[] $\to$ Sufren diferentes rigideces nominales sobre precios
				\4[] $\to$ Venden en mercado nacional y extranjero
				\4[] Gobierno
				\4[] $\to$ Determina oferta monetaria
				\4[] $\to$ Determina demanda autónoma de gasto público
				\4[] $\to$ Ofrecen activos financieros
				\4[] Equilibrio
				\4[] $\to$ Senda de optimización de consumo y trabajo
				\4 Canales de M-F
				\4[] Relevantes en modelos NOEM
				\4[] Modelizables con distintas intensidades
				\4 Canal de transmisión: rentas
				\4[] Rentas del capital
				\4[] $\to$ Posible incluir en modelo
				\4[] $\to$ PM expansiva aumenta valor activos extranjeros
				\4 Factores de modulación:  Tamaño relativo de países
				\4[] Mayor tamaño
				\4[] $\to$ Mayor efecto sobre terceros
				\4 Factores de modulación: divisa de facturación
				\4[] LCP
				\4[] $\to$ Local-currency pricing
				\4[] $\then$ Precios de exportación en divisa de importador
				\4[] $\then$ Demanda de importador insensible a TCN
				\4[] $\then$ Pass-through de importaciones relevante
				\4[] PCP
				\4[] $\to$ Producer-currency pricing
				\4[] $\then$ Precios de exportación en divisa de exportador
				\4[] $\then$ Demanda de
		\2 Coordinación de política fiscal
			\3 Locomotora fiscal
				\4 Contexto
				\4[] Shock de demanda afecta a varias economías
				\4[] Marco keynesiano
				\4[] $\to$ Políticas macro pueden estimular DA
				\4[] $\then$ Aumentar output y empleo
				\4[] Economías abiertas e interconectadas
				\4[] $\to$ Estímulo DA individual se transmite a otras
				\4 Formulación
				\4[] Conjunto de economías interconectadas
				\4[] Shock de demanda común a todas
				\4 PF individual sin coordinación
				\4[] Spill-over internacional de expansión fiscal
				\4[] $\to$ Parte del impulso se traslada a otras economías
				\4[] $Y = C_0 + c Y + I(r) +X (Y^*, S) - M_0 - mY$
				\4[] $\to$ $\frac{C_0 + I(r) + X_0 + X - M_0}{1-c+m}$
				\4[] $\then$ $m$
				\4[] Inflación doméstica + apreciación TCN
				\4[] $\then$ Apreciación del TCR
				\4[$\then$] Déficit de CC en país que estimula
				\4[$\then$] Superávit de CC en país que e
				\4[] Exportadores nacionales sufren
				\4[] Importadores extranjeros se benefician
				\4[] Incentivos en extranjeros a no expandir
				\4 PF coordinada
				\4[] Expansión fiscal de todas las economías
				\4[] Aumento de DA genera spill-overs mutuos
				\4[] TCR se mantienen estables
				\4[] Ningún país obtiene
				\4 Ejemplos
				\4[] Cumbre del G-20 en 2009
				\4[] $\to$ Expansión fiscal coordinada idea inicial
				\4[] $\then$ Crisis de deuda posterior en UE
				\4[] $\then$ Inicio de contracción en UE
				\4[] UE tras crisis financiera
				\4[] $\to$ Alemania podría expandir
				\4[] $\to$ Reducir presión para contraer en periféricos
				\4 Representación en marco de teoría de juegos
				\4[] Forma normal
				\4[] \grafica{locomotorafiscaljuegos}
				\4[] Funciones de reacción
				\4[] \grafica{locomotorafiscalreaccion}
			\3 Disciplina fiscal
				\4 Contexto
				\4[] Estados compiten por fondos en mercados financieros
				\4[] Mercados financieros integrados
				\4[] Déficits fiscales inducen externalidades negativas
				\4[] $\to$ Aumento general de tipos de interés
				\4[] $\to$ Tensiones en sistema financiero
				\4[] Riesgo moral
				\4[] $\to$ Si sólo un país imprudente fiscalmente
				\4[] $\then$ Amenaza de dejar caer más creíble
				\4[] $\to$ Si varios países expanden fiscalmente
				\4[] $\then$ Menos creíble amenaza de no inflactar/rescatar
				\4[] $\then$ Especialmente relevante en Unión Monetaria
				\4 Formulación
				\4[] Difícil representar en IS-LM
				\4[] Necesarios otros modelos
				\4[] $\to$ Sistema financiero
				\4[] $\to$ Incentivos a impago
				\4[] $\to$ Información asimétrica: riesgo moral
				\4 PF individual sin coordinación
				\4[] SPúblicos compiten por financiación
				\4[] $\to$ Incentivos individuales a endeudarse
				\4[] Aumenta demanda de capital en mercados financieros
				\4[] $\to$ Aumentan tipos de interés
				\4[] Más probable estados necesiten rescate/monetización
				\4[] $\to$ Más probable tenga lugar porque todos incumplen
				\4[] $\then$ Déficits excesivos
				\4[] $\then$ Más probable inflación y/o quiebras
				\4 PF coordinada
				\4[] Déficits públicos se mantienen bajos
				\4[] Tipos de interés moderados
				\4[] Menor crowding-out
				\4[] Amenaza creíble de no rescatar
				\4[] $\to$ Porque hay países que sí cumplen
				\4[] Disciplina en gasto público
				\4 Ejemplo
				\4[] UEM
				\4[] $\to$ Maastricth 1991
				\4[] $\to$ PEC
				\4[] $\to$ TCSG
				\4[] $\to$ Déficit excesivo como externalidad negativa
				\4[] $\then$ Temor a tener que transferir rentas
				\4 Representación en marco de teoría de juegos
				\4[] Forma normal
				\4[] \grafica{disciplinafiscaljuegos}
				\4[] Funciones de reacción
				\4[] \grafica{disciplinafiscalreaccion}
			\3 Valoración
				\4 Diferentes modelos del mundo
				\4[] Habitual policy-makers en diferentes economías
				\4[] $\to$ En una economía, perciben locomotora
				\4[] $\to$ En otra economía, perciben disciplina
				\4 Incentivos a postular diferentes modelos
				\4[] Países con mejor posición fiscal
				\4[] $\to$ Postulan juego de disciplina
				\4[] Países con posición fiscal débil
				\4[] $\to$ Prefieren postular disciplina
				\4[$\then$] Modelo de cooperación propuesto es endógeno
		\2 Coordinación de política monetaria
			\3 Devaluación competitiva
				\4 Contexto
				\4[] PM expansiva
				\4[] $\to$ Reducción de costes de financiación
				\4[] $\to$ Provisión de liquidez a menor coste
				\4[] $\to$ Expansión de balances
				\4[] Mercados financieros
				\4[] $\to$ Mercado de divisas
				\4[] $\to$ Mercados de activos financieros
				\4[] $\then$ Efectos de PM sobre mercado cambiario
				\4 Formulación
				\4[] Contexto M-F
				\4[] Dos países A y B
				\4[] A aumenta oferta monetaria
				\4[] --Efecto sobre cuenta financiera
				\4[] $\to$ Caída de interés en A respecto a B
				\4[] $\then$ Flujo de capital de A a B
				\4[] $\then$ $\uparrow$ demanda de divisa de B
				\4[] $\then$ $\downarrow$ oferta de divisa de A
				\4[] $\then$ Depreciación de moneda de A
				\4[] $\then$ Exportadores de A beneficiados
				\4[] $\then$ Exportadores de B perjudicados
				\4[] --Efecto sobre cuenta corriente
				\4[] $\to$ Estímulo de DA por canales transmisión PM
				\4[] $\then$ Caída de exportaciones netas
				\4[] $\then$ Más demanda de importación
				\4[] $\then$ Exportadores de B se benefician
				\4[] Si efecto cuenta financiera > cuenta corriente
				\4[] $\to$ Estímulo doble a DA
				\4[] $\then$ Por absorción y por exportaciones
				\4[] $\to$ Economías socias sufren PM
				\4[] $\to$ Economía que estimula se beneficia
				\4[] $\then$ Beggar-thy-neighbor
				\4[] $\then$ Empobrecimiento del vecino
				\4[] $\then$ Equilibrio no cooperativo empobrece
				\4 PM individual sin coordinación
				\4[] Si efecto cuenta financiera > cuenta corriente
				\4[] $\to$ Incentivos a PM expansiva unilateral
				\4[] Todos tienen incentivo a expandir
				\4[] $\to$ Expansión monetaria general
				\4[] $\then$ Posible inflación generalizada
				\4 PM coordinada
				\4[] Economías evitan expansión excesiva
				\4[] Interés relativamente bajo
				\4[] Expansión fiscal coordinada posible
				\4 Representación en marco de teoría de juegos
				\4[] Forma normal
				\4[] \grafica{devaluacioncompetitivajuegos}
				\4 Ejemplo
				\4[] Brasil en 2010:
				\4[] $\to$ Acusa a EEUU de iniciar guerra cambiaria
				\4[] EEUU responde:
				\4[] $\to$ Objetivo es estimular DA no $\downarrow$ TCN
				\4[] Cumbre del G-7 en 2013
				\4[] $\to$ Acuerdo
			\3 Apreciación competitiva
				\4 Contexto
				\4[] Inflación elevada
				\4[] Economías con dos sectores relativamente diferenciados
				\4[] $\to$ Bienes comerciables/exportador
				\4[] $\to$ Bienes no comerciables/doméstico
				\4[] Contracción monetaria
				\4[] $\to$ Reduce inflación
				\4[] $\to$ Aprecia moneda
				\4[] Economías abiertas
				\4[] $\to$ Apreciación tiene efectos fuertes
				\4[] $\then$ Reparte coste de contracción a sector bienes comerciables
				\4 Formulación
				\4[] Subida de tipo de interés
				\4[] $\to$ Atrae capital
				\4[] $\then$ Aumenta demanda de moneda local
				\4[] $\then$ Apreciación de moneda
				\4[] Apreciación de moneda
				\4[] $\to$ Reduce coste de importaciones
				\4[] $\to$ Perjudica exportadores nacionales
				\4[] $\then$ Reduce inflación nacional
				\4[] $\then$ Provoca inflación en socios
				\4 Equilibrio no cooperativo
				\4[] Economías suben tipos por separado
				\4[] $\to$ Contracción global excesiva
				\4[] $\to$ Inflación cae
				\4[] $\then$ Curva de Phillips $\pi$-u convexa
				\4[] $\then$ Enorme coste de inflación en términos de paro
				\4[] Déficits gemelos en país con moneda de reserva
				\4[] $\to$ Especialmente relevante para USA 80s
				\4[] $\to$ Déficit fiscal presiona al alza tipos de interés
				\4[] $\to$ Tipos de interés altos atraen capital
				\4[] $\to$ Entrada de capital aprecia TCN
				\4[] $\to$ TCN apreciado aumenta déficit por CC
				\4 Equilibrio cooperativo
				\4[] Coordinación de tipos de interés
				\4[] Intervención coordinada en mercados dedivisas
				\4[] Control de inflación sin coste excesivo de desempleo
				\4 Representación en marco de teoría de juegos
				\4[] Forma normal
				\4[] \grafica{apreciacioncompetitivajuegos}
				\4 Ejemplos
				\4[] Contracción de Volker en principios de 80s
				\4[] $\to$ Aumento de tipos de interés para $\downarrow$ inflación
				\4[] $\to$ Salida de capital de PEDs
				\4[] $\to$ Aumento coste de deuda
				\4[] $\then$ Crisis financiera en Latam
				\4[] Acuerdos Plaza de 1985
				\4[] $\to$ USA acepta devaluar
				\4[] $\to$ Otras divisas aceptan apreciar frente a dólar
				\4[] Taper tantrum de 2013
				\4[] $\to$ Fed propone retirada de QE
				\4[] $\then$ Fuerte aumento de tipos de interés
				\4[] $\then$ Salida de capital de emergentes
				\4[] $\then$ India se queja de política americana
				\4[] $\then$ Exige coordinación
			\3 Valoración
				\4 Problema similar al de política fiscal
				\4 Diferentes modelos del mundo
				\4[] Diferentes recetas de cooperación óptima
				\4[] $\to$ Determinada cooperación puede ser óptima o no
				\4[] $\then$ Depende de modelo subyacente en el que se crea
				\4 Percepciones del modelo son endógenas
				\4[] A menudo dependen de intereses
				\4 Factores de economía política
				\4[] Grupos de interés a nivel doméstico
				\4[] $\to$ Elemento clave de disposición a cooperar
		\2 Instrumentos de cooperación
			\3 Organismos de cooperación
				\4 Idea clave
				\4[] Organizaciones institucionalizadas
				\4[] Marco definido de:
				\4[] $\to$ Intercambio de información
				\4[] $\to$ Diseño de políticas
				\4[] $\to$ Mecanismos coercitivos
				\4[] No siempre constituidos formalmente
				\4 Ejemplos
				\4[] OECE
				\4[] $\to$ Tras GM
				\4[] $\to$ Europeos receptores del Plan Marshall
				\4[] $\to$ Coordinar apertura de CC y CF
				\4[] G-5 en años 80
				\4[] $\to$ EEUU, Japón, UK, Francia, Alemania
				\4[] $\to$ Clave en acuerdos del Hotel Plaza 85
			\3 Mecanismos informales
				\4 Idea clave
				\4[] Intercambio de información
				\4[] Diseño de políticas
				\4[] En contexto no institucional
				\4 Ejemplos
				\4[] Relaciones comerciales
				\4[] Relaciones personales entre policy-makers
				\4[] $\to$ Jackson Hole
				\4[] $\to$ WEF
				\4[] $\to$ Norman y Strong en pos-IGM\footnote{Montagu Norman gobernador del Banco de Inglaterra y Benjamin Strong, presidente de la Fed de Nueva York décadas de 1910 y principios de 1920. Una estrecha relación entre ambos permitió el intercambio de información y la cooperación a la hora de apreciar la libra hasta niveles considerados adecuados, entre otras intervenciones}
			\3 Reglas
				\4 Idea clave
				\4[] Acuerdos explícitos o implícitos
				\4[] Comportamiento obligado dadas circunstancias
				\4 Ejemplos
				\4[] ``Reglas del juego'' de Patrón Oro
				\4[] PEC y PDM en UE
				\4[]
		\2 Argumentos contra la cooperación
			\3 Idea clave
				\4 Hasta ahora
				\4[] Examinados contextos de cooperación óptima
				\4 No necesariamente
				\4[] Equilibrio no cooperativo puede ser preferible
				\4[] $\to$ Argumento similar a CPerfecta vs Monopolio
			\3 TCN flexible aísla de políticas externas
				\4
			\3 Devaluación competitiva provee liquidez
				\4 Contexto
				\4 Implicaciones
				\4 Valoración
			\3 Integración reduce necesidad de cooperación
				\4 Contexto
				\4 Implicaciones
				\4 Valoración
			\3 Pérdida de credibilidad de PM
				\4 Rogoff (1985)
			\3 Riesgo moral
				\4 Contexto
				\4 Implicaciones
				\4 Valoración
			\3 Economía política nacional
				\4 Contexto
				\4 Implicaciones
				\4 Valoración
			\3 Beneficios de segundo orden
				\4 Contexto
				\4 Implicaciones
				\4 Valoración
	\1 \marcar{Evidencia empírica}\footnote{Ver Eichengreen (2011)}
		\2 Patrón Oro
			\3 Contexto
			\3 Actuaciones
			\3 Implicaciones
		\2 Bretton Woods
			\3 Contexto
			\3 Actuaciones
			\3 Implicaciones
		\2 Volatilidad cambiaria en los 80
			\3 Contexto
			\3 Actuaciones
			\3 Implicaciones
		\2 Crisis financiera
			\3 Contexto
			\3 Actuaciones
			\3 Implicaciones
		\2 Pandemia de coronavirus
			\3 Contexto
			\3 Actuaciones
			\3 Implicaciones
		\2 Factores de coordinación\footnote{Págs. 1-2 Eichengreen (2011)}
			\3 Aspectos técnicos frente a políticos
			\3 Existencia de marco institucional
			\3 Preservación de políticas ya existentes
			\3 Conflictos paralelos
			\3 Percepción sobre modelos compatible
	\1[] \marcar{Conclusión}
		\2 Recapitulación
			\3 Análisis teórico
			\3 Evidencia empírica
		\2 Idea final
			\3 Sentido de la coordinación macro internacional
				\4 Participantes deben misma visión sobre:
				\4[] El problema que induce el equilibrio no cooperativo
				\4[] Naturaleza de los spillovers de sus acciones
				\4[] Reparto de los beneficios de la cooperación
			\3 Evolución de la cooperación macro
				\4 Fuertemente dependiente de otros factores
				\4 Relaciones personales
				\4[] Entre policy makers
				\4 Factores políticos y culturales
				\4 Path-dependency
				\4[$\then$] Más allá de exposición
			\3 Situación económica actual
				\4 Debate sobre optimalidad persiste
				\4 Enorme incertidumbre sobre efectos y evolución
				\4 Coordinación muy difícil
				\4[] Sin precedentes históricos de situación similar
				\4[] $\to$ Sólo aproximaciones parciales
				
\end{esquemal}























\graficas

\begin{tabla}{Representación en forma normal de una situación de locomotora fiscal: el equilibrio no competitivo implica contracción fiscal en ambos países.}{locomotorafiscaljuegos}
	\begin{tabular}{l || c | c}
		& \textbf{USA contrae} & \textbf{USA expande} \\ \hline \hline
		\textbf{UE contrae} & Recesión global & USA déficit CC, UE superávit y $\uparrow$ DA  \\ \hline
		\textbf{UE expande} & Europa incurre en déficit CC, USA se beneficia & Equilibrio cooperativo, expansión global \\ \hline
	\end{tabular}
\end{tabla}

\begin{axis}{4}{Locomotora fiscal: funciones de reacción de EEUU y UE y comparación de equilibrio cooperativo y no cooperativo.}{$\text{PF}_\text{UE}$}{$\text{PF}_\text{USA}$}{locomotorafiscalreaccion}
	%%%% UE

	% Función de reacción de UE
	\draw[-] (0,3) -- (1.5,0);
	\draw[-{Latex}] (0.5,2) -- (1,3);
	\node[above] at (1,3){$\text{PF}_\text{UE}$};

	% Curva de indiferencia de UE en eq. no cooperativo
	\draw[-] (0.3,1.4) to [out=300, in=180](1,1) to [out=0, in=240](1.7,1.4);

	% Curva de indiferencia de UE en eq. cooperativo
	\draw[-] (0.25,1.7) to [out=300, in=180](0.95,1.3) to [out=0, in=240](1.65,1.7);
	
	%%%% USA

	% Función de reacción de USA
	\draw[-] (0,1.5) -- (3,0);
	\draw[-{Latex}] (2,0.5) -- (3,1);
	\node[right] at (3,1){$\text{PF}_\text{USA}$};

	% Curva de indiferencia de USA en eq. no cooperativo
	\draw[-] (1.4,1.7) to [out=210, in=90](1,1) to [out=270, in=150](1.4,0.3);
	
	% Curva de indiferencia de USA en eq. cooperativo
	\draw[-] (1.65,1.6) to [out=210, in=90](1.25,0.9) to [out=270, in=150](1.65,0.3);
	
	% Equilibrio no cooperativo
	\node[circle, fill=black, inner sep=0pt, minimum size=3pt] (a) at (1,1) {};
	\node[left] at (0.98,0.9){\tiny E};
	
	
	% Equilibrio cooperativo
	\node[circle, fill=black, inner sep=0pt, minimum size=3pt] (a) at (1.5,1.5) {};
	\node[right] at (1.5,1.5){\tiny E*};
	
	% Bisectriz
	%\draw[dashed] (0,0) -- (3,3); 
	
\end{axis}

El punto E muestra el equilibrio no cooperativo en el que USA y UE tienen incentivos a no expandir para aprovechar los spill-overs de la expansión de la otra economía y obtener también un superávit por cuenta corriente. El punto $\text{E}^*$ muestra el equilibrio cooperativo como punto de tangencia entre las curvas de indiferencia de ambas economías. Ambas economías pueden alcanzar curvas de indiferencia superiores si coordinan la expansión y se comprometen a mantenerla a pesar de tener incentivos unilaterales a desviarse y a aprovechar la externalidad positiva de las otras economías.

\begin{tabla}{Representación en forma normal de una situación de disciplina fiscal: el equilibrio no competitivo implica déficit excesivo.}{disciplinafiscaljuegos}
	\begin{tabular}{l || c | c}
		& \textbf{Sur superávit público} & \textbf{Sur déficit} \\ \hline \hline
		\textbf{Norte superávit público} & Sin riesgo moral, sin crowding-out & Norte teme pagar rescate a sur  \\ \hline
		\textbf{Sur déficit público} & Sur teme pagar rescate a norte & Riesgo moral, todos incurren en déficits excesivos \\ \hline
	\end{tabular}
\end{tabla}

\begin{axis}{4}{Disciplina fiscal: funciones de reacción de EEUU y UE y comparación de equilibrio cooperativo y no cooperativo.}{$\text{PF}_\text{UE}$}{$\text{PF}_\text{USA}$}{disciplinafiscalreaccion}
	%%%% UE
	
	% Función de reacción de UE
	\draw[-] (0,3) -- (1.5,0);
	\draw[-{Latex}] (0.5,2) -- (1,3);
	\node[above] at (1,3){$\text{PF}_\text{UE}$};
	
	% Curva de indiferencia de UE en eq. no cooperativo

	\draw[-] (0.3,0.6) to [out=60, in=180](1,1) to [out=0, in=120](1.7,0.6);
	
	% Curva de indiferencia de UE en eq. cooperativo
	\draw[-] (0.37,0.3) to [out=60, in=180](1.07,0.7) to [out=0, in=120](1.77,0.3);
	
	%%%% USA
	
	% Función de reacción de USA
	\draw[-] (0,1.5) -- (3,0);
	\draw[-{Latex}] (2,0.5) -- (3,1);
	\node[right] at (3,1){$\text{PF}_\text{USA}$};
	
	% Curva de indiferencia de USA en eq. no cooperativo
	\draw[-] (0.6,1.7) to [out=-30, in=90](1,1) to [out=270, in=30](0.6,0.3);
	
	% Curva de indiferencia de USA en eq. cooperativo
	\draw[-] (0.25,1.75) to [out=-30, in=90](0.65,1.05) to [out=270, in=30](0.25,0.35);
	
	% Equilibrio no cooperativo
	\node[circle, fill=black, inner sep=0pt, minimum size=3pt] (a) at (1,1) {};
	\node[left] at (0.98,0.9){\tiny E};
	
	
	% Equilibrio cooperativo
	\node[circle, fill=black, inner sep=0pt, minimum size=3pt] (a) at (0.5,0.5) {};
	\node[right] at (0.5,0.5){\tiny E*};
	
	% Bisectriz
	\draw[dashed] (0,0) -- (3,3); 
	
\end{axis}

\begin{tabla}{Representación en forma normal de un contexto de devaluación competitiva: equilibrio no cooperativo implica inflación sin devaluación}{devaluacioncompetitivajuegos}
	\begin{tabular}{l || c | c}
		& \textbf{Fed no expande M} & \textbf{Fed expande M} \\ \hline \hline
		\textbf{BCE no expande M} & No hay depreciación competitiva & Euro se aprecia, exportadores europeos sufren  \\ \hline
		\textbf{BCE expande M} & Dolar se aprecia, exportadores americanos sufren & Todos expanden sin efecto \\ \hline
	\end{tabular}
\end{tabla}



\begin{tabla}{Representación en forma normal de un contexto de devaluación competitiva: equilibrio no cooperativo implica inflación sin devaluación}{apreciacioncompetitivajuegos}
	\begin{tabular}{l || c | c}
		& \textbf{Fed contrae M} & \textbf{Fed expande M} \\ \hline \hline
		\textbf{Alemania contrae M} & Todos contraen. Recesión global & Dolar se deprecia y aumenta inflación \\ \hline
		\textbf{Alemania expande M} & Dolar se aprecia, bajada de inflación & TCN estables, crecimiento, inflación  \\ \hline
	\end{tabular}
\end{tabla}


\preguntas

\seccion{Test 2015}

\textbf{33.} Sobre la coordinación internacional de las políticas monetarias, Rogoff (1985) considera que en un mundo de dos países con un sistema de tipos de cambio de flotación sucia (señale la verdadera):

\begin{itemize}
	\item[a] La coordinación de las políticas monetarias son siempre beneficiosas para ambos países.
	\item[b] La coordinación de las políticas monetarias puede ser contraproducente para el bienestar, ya que puede llevar a una mayor tasa de inflación.
	\item[c] La coordinación de las políticas monetarias no tiene ningún efecto, ni positivo ni negativo, ya que se consigue el mismo resultado de forma coordinada o actuando unilateralmente.
	\item[d] La coordinación de las políticas monetarias es efectiva siempre que exista un cierto grado de coordinación de las políticas fiscales.
\end{itemize}

\seccion{Test 2005}
\textbf{32.} Se facilita la coordinación de las políticas macroeconómicas y la obtención de resultados positivos para los países si:
\begin{itemize}
	\item[a] Los gobiernos actúan separadamente fijando sus objetivos, aunque éstos sean distintos entre los países cooperantes, siempre que éstos se encuentren protegidos por barreras arancelarias frente a terceros y entre ellos mismos.
	\item[b] Coinciden en la determinación de los instrumentos monetarios y fiscales a utilizar a corto plazo, aunque diverjan las reglas de actuación y no se establezcan canales de comunicación sobre propósitos y estrategias que quedan sujetas a la soberanía de las políticas macroeconómicas de cada país.
	\item[c] Los gobiernos se informan mutuamente acerca de sus intenciones políticas, los objetivos son parecidos aunque los instrumentos de política económica sean distintos y se eliminan actuaciones bilaterales agresivas como políticas de empobrecimiento al vecino o barreras arancelarias.
	\item[d] Cada uno fijo sus propios objetivos, confiando el ajuste y la coordinación a un sistema de cambios flexibles y a la inexistencia de barreras arancelarias. 
\end{itemize}

\seccion{9 de marzo de 2017}
\begin{itemize}
	\item ¿Hasta que punto hay que perseguir únicamente sólo los grandes déficits?
	\item ¿Habría que perseguir también los superávits? ¿Existen límites para superávits?
	\item ¿Cree usted que se está produciendo realmente armonización mundial en las distintas cuestiones de actualidad (corrupción, medioambiente, etc.)?
	\item ¿Cómo son las curvas de indiferencia en el modelo de Aldama (Hamada)? Concéntricas desde el punto de saturación.
	\item ¿Existe coordinación económica en el ámbito de la OCDE?
\end{itemize}

\seccion{13 de marzo de 2017}
\begin{itemize}
	\item Valore la coordinación de políticas macroeconómicas desde la crisis financiera.
	\item ¿Cuál es el papel del GATT y posteriormente de la OMC en la coordinación de políticas económicas a nivel internacional?
	\item A lo largo del temario se afirma que la competencia es óptima, que hay que competir. Pero sin embargo, del tema que usted ha cantado -y no sé si es culpa suya o del temario-, se desprende que la coordinación per se es deseable, en contraposición a la competencia. No termino de entender la deseabilidad de la coordinación. Por ejemplo, en España, los territorios compiten en relación al tipo del Impuesto sobre Sucesiones. ¿Es deseable coordinar políticas económicas? (Ésta, el profesor de universidad)
	\item ¿En qué situaciones se producen todos los elementos favorables para que tenga éxito la coordinación de políticas macroeconómicas?
\end{itemize}

\notas

\textbf{2015:} \textbf{33.} B

\textbf{2005:} \textbf{32.} C

Ver Rogoff (1985) sobre política monetaria y coordinación de políticas monetarias. Uno de sus resultados es que la coordinación puede reducir los incentivos a reducir inflación (a esto se refería el catedrático).

\bibliografia

Mirar en Palgrave:
\begin{itemize}
	\item control and coordination of economic activity
	\item coordination problems and communication
	\item economic integration
	\item international coordination in asylum provision
	\item international coordination of regulation
	\item \textbf{international policy coordination}
\end{itemize}


Begg, I.; Hodson, D.; Maher, I. (2003) \textit{Economic Policy Coordination in the European Union} National Institute Economic Review No. 183 -- En carpeta del tema


Dornbusch, R. \textit{Expectations and Exchange Rate Dynamics} (1976) Journal of Political Economy -- En carpeta del tema

Eichengreen, B. (2011) \textit{International Policy Coordination: The Long View} Ch. 2 Págs. 43-82 Globalization in the Age of Crisis: Multilateral Economic Cooperation in the Twenty-First Century -- En carpeta del tema

Frankel, J. F. (2015) \textit{The Plaza Accord, 30 years later} NBER Working Paper Series \href{https://www.nber.org/papers/w21813.pdf}{Disponible aquí} -- En carpeta del tema

Frankel, J. \textit{International Coordination} (2016) Faculty Research Working Paper Series - Harvard Kennedy School of Government -- En carpeta del tema


Hamada, K.  \textit{A Strategic Analysis of Monetary Interdependence}. (1976) Journal of Political Economy 

Kose, M. A.; Prasad, E.; Rogoff, K.; Wie, S-J. (2006) \textit{Financial Globalization: A Reappraisal} NBER Working Paper Series -- En carpeta del tema

McKibbin, W. \textit{The Economics of International Policy Coordination}. (1987) \url{https://ideas.repec.org/p/rba/rbardp/rdp8705.html} Por leer

Rey, H. (2018) \textit{Dilemma not trilemma: the global financial cycle and monetary policy independence} NBER Working Paper Series \href{https://www.nber.org/papers/w21162.pdf}{Disponible aquí} -- En carpeta del tema

Rogoff, K. \textit{Can international monetary policy cooperation be counterproductive?} (1992) Journal of International Economics -- En carpeta del tema \url{http://gen.lib.rus.ec/scimag/?q=Can+international+monetary+policy+cooperation+be+counterproductive%3F}
	
Taylor, M. P. (1995) \textit{The Economics of Exchange Rates} Journal of Economic Literature Vol. XXXIII -- En carpeta del tema


\end{document}
