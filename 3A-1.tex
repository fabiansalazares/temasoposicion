\documentclass{nuevotema}

\tema{3A-1}
\titulo{Objeto y método de la ciencia económica}

\begin{document}

\ideaclave

El origen de la palabra ``economía'' se atribuye al griego \textit{oikonomia}, concepto utilizado por Jenofonte en el 400 antes de Cristo en su obra Oikonomikos para hacer referencia a la buena gestión de la casa y la hacienda, por medio de una apropiada llevanza de los asuntos diarios y las cuentas del negocio. En la Edad Antigua y en la Edad Media, la economía no existe como tal, pero algunos de sus temas futuros comienzan a aparecer. La gestión de la política monetaria de los soberanos, la necesidad de aumentar el comercio entre regiones o las consideraciones morales de los contratos se tratan conjuntamente con problemas jurídicos, teológicos y morales. Es en el siglo XVIII cuando la ciencia económica comienza a separarse plenamente de otras disciplinas, y nace el término economía política que después pasaría a ser simplemente ``economía''. En la actualidad, la ciencia económica es una de las piedras angulares del estudio del comportamiento humano. Conceptos derivados de la economía son habituales en todas las actividades de gestión, y el debate económico es un elemento central de debate político y de la gestión pública. De este hecho surgen las dos preguntas básicas de la ciencia económica: ¿qué estudia la economía? ¿cómo lo hace? Contestar a estas dos preguntas básicas implica afrontar otras preguntas más específicas, que conforman el objeto de esta exposición. ¿Cuál es el objetivo de la economía? ¿qué fenómenos trata de entender y explicar? ¿cómo ha evolucionado el objeto de la economía? Por otro lado, en cuanto al cómo estudia la economía, la exposición trata de contestar a: ¿qué relación tiene la economía con el método científico en general? ¿qué rasgos caracterizan al método económico actual y pasado? ¿qué debates metodológicos han moldeado la ciencia económica? ¿qué innovaciones metodológicas se encuentran actualmente en la frontera de la investigación? ¿qué características debe tener un buen modelo económico? La exposición se divide en cuatro partes. En la primera, examinamos las diferentes formas de entender el objeto de la ciencia económica. A continuación examinamos la evolución del método científico de forma general, como generalización del método utilizado por la ciencia económica. Posteriormente tratamos los principales debates metodológicos que se han desarrollado en el seno de la ciencia económica. Por último, concluimos la exposición presentando algunas de los rasgos básicos que caracterizan lo apropiado de un modelo para explicar y predecir fenómenos económicos.

Desde su nacimiento como tal con la obra de Adam Smith, el \textbf{objeto de la ciencia económica} ha sido definido de diferentes formas. La primera aparición del término ``economía política'' se atribuye a Steuart en el siglo XVIII y apenas unas décadas antes de Adam Smith. Ambos autores conciben el objeto de la economía política como la provisión de máximo bienestar a la población y la generación de ingresos públicos para proveer servicios. Así, la economía política es una ciencia del \textit{debe ser} y no tanto del \textit{es}. En los economistas clásicos posteriores se produce una transformación hacia la segunda opción: Say y Senior, y también Ricardo y Malthus en menor medida entienden el objeto de la economía política como el descubrimiento de una serie de leyes que gobiernan la naturaleza de las relaciones de producción, distribución y consumo entre humanos. John Stuart Mill, el gran metodólogo del siglo XIX, matiza a ambos. Por un lado, afirma que entender el objeto de la economía política como el bueno gobierno de la nación supone confundir ciencia con arte. Por otro lado, rechaza que el objeto de la economía sea simplemente el descubrimiento de leyes relativas a la distribución, consumo y producción: las leyes que describen éstos fenómenos incluyen también relaciones subyacentes de carácter tecnológico. Así, Mill afirma que el objeto debe ser el descubrimiento de las leyes que rigen esos fenómenos, pero sólo en lo que resulta de la acción humana y separando aquellos elementos que son objeto de otras disciplinas de carácter natural o tecnológico. Alfred Marshall introduce el término ``economía'' para remplazar a la ``economía política'' previa. Marshall retorna a la concepción del objeto de la economía como un examen del debe ser: la economía es la ciencia de la vida diaria en lo que respecta a las acciones tomadas para alcanzar un nivel máximo de bienestar. En el siglo XX, Lionel Robbins primero y posteriormente Paul Samuelson, definen el objeto de la economía más habitual en la actualidad. Según estos autores, el objeto de la economía es el estudio del comportamiento humano cuando trata de gestionar una serie de recursos escasos para alcanzar la satisfacción de sus necesidades. Aunque esta definición del objeto de la economía ha recibido críticas, en la actualidad es la definición que se encuentra en la mayoría de libros de texto y que mejor parece adaptarse a los modernos modelos microeconómicos en los que se parte de un agente que trata de maximizar una función dadas una restricciones a su conjunto de decisión. Así, tenemos que el hilo conductor del debate acerca del objeto de la economía gira en torno al carácter positivo o normativo que ésta debe tomar. Si la economía es una ciencia positiva, debe tratar de reflejar la realidad de los fenómenos y dejar a un lado las consideraciones acerca de lo que \textit{debe ser}. Si la economía es una ciencia normativa, su objetivo es dilucidar qué acciones deben tomar los sujetos analizados tales como consumidores, empresas o gobiernos. En la práctica, la economía incorpora ambas facetas, aunque se aprecia en general una tendencia a rechazar, al menos nominalmente, la emisión de juicios de valor. Más allá de este debate, es interesante recordar la nota discordante que puso Jacob Viner afirmando que ``\textit{economía es lo que hacen los economistas}'', en una crítica velada a la delimitación de los contornos de la ciencia económica, y reflejo de que en la práctica, los economistas no prestan demasiada atención a lo que se supone debe ser el objeto de la economía y simplemente aplican sus métodos y su esfuerzo a las cuestiones que consideran merecen ser investigadas.

El \textbf{método científico} es un conjunto de técnicas utilizadas para investigar las causas de fenómenos observados, adquirir nuevos conocimientos e integrar el conocimiento obtenido con otras explicaciones de esos fenómenos. Las técnicas asociadas al método científico se aplican de forma iterativa con el fin de mejorar gradualmente la capacidad de explicación y predicción del conocimiento adquirido. Las técnicas incluyen la caracterización de los fenómenos observados y del objeto de la investigación, la deducción de hipótesis y predicciones a partir de principios básicos o explicaciones de otros fenómenos, y el contraste de ésas hipótesis y predicciones mediante la observación y contrastación con fenómenos reales. En general, se acepta que el método científico debe aportar conocimiento general más allá de un fenómeno concreto observado en una ocasión, ser neutral a juicios de valor y debe formular predicciones con suficiente contenido empírico como para poder ser contrastado. El método concreto para lograr conocimiento que cumpla estos requisitos ha sido objeto de debate y en la práctica es lo suficientemente diverso como para que existan diferentes formas de conceptualizar el método científico. 

En esa tarea de caracterización del método científico, es útil examinar la división entre deducción e inducción. Esta confrontación entre los dos conceptos toma también otras formas en todas las áreas de la investigación científica: análisis frente a síntesis, 
El \textit{método inductivo} consiste en extraer conclusiones generales a partir de una regularidad observada de forma repetida. Así, el método inductivo utiliza la verdad de las premisas como soporte para afirmar la verdad de la conclusión. Esta forma de razonar está limitada por el hecho de que observaciones pasadas de un fenómeno no suponen necesariamente que el fenómeno se repita en los mismos términos en el futuro. Así, nada permite afirmar con total certeza que un grupo de consumidores seguirá demandando un producto determinado a un precio dado, aunque haya un gran número de observaciones pasadas en las que así sucedió. El \textit{método deductivo} consiste en tomar una serie de axiomas o principios cuya verdad se considera indisputable o evidente y deducir de ellos una conclusión a través de reglas de derivación lógica. El método deductivo genera conclusiones verdaderos siempre que el proceso lógico de deducción se haya ejecutado correctamente. Sin embargo, las conclusiones derivadas del método deductivo no contienen necesariamente validez externa, en el sentido de que no tienen por qué corresponderse a priori con hechos empíricos.

A finales del siglo XIX y en la primera mitad del siglo XX comienza a conformarse el \textit{método hipotético-deductivo} con las contribuciones de Poincaré, los filósofos positivistas del Círculo de Viena. Hempel y Oppenheim formalizarán en 1965 una descripción de este método como representativo del método científico en general, y en la actualidad el método hipotético-deductivo es la representación habitual del método científico en la literatura de carácter didáctico o pedagógico. El método comienza por la observación de un fenómeno a explicar. Para explicarlo, se propone una hipótesis. De esta hipótesis se deducen una serie de resultados. Estos resultados hipotéticos se contrastan con hechos empíricos. En la medida en que estos hechos no contradigan la hipótesis, se realizan nuevos test para mejorar la confianza en la hipótesis. Si los hechos contradicen los resultados hipotetizados, la hipótesis se desecha. Estos resultados hipotéticos son predicciones acerca de fenómenos aún no observados, pero pueden también utilizarse como explicaciones de fenómenos ya observados, dando lugar a la llamada tesis de la simetría que relaciona la capacidad para explicar y predecir de un modelo científico. 

El \textit{falsacionismo} propuesto por Popper enlaza con la forma anterior de entender el método científico. Popper rechaza la posibilidad de verificar una hipótesis. Según el, una hipótesis sólo puede ser falsada, y el método científico se caracteriza por formular hipótesis que puedan ser falsadas. En la medida en que una teoría no excluya un estado de la naturaleza, no podrá ser falsada por medio de la realización de ese estado de la naturaleza y por ello no podrá ser falsada. El falsacionismo tendrá gran influencia sobre el método de la ciencia económica, con Milton Friedman y Mark Blaug como algunos de sus defensores principales. Sin embargo, la tesis de Duhem-Quine vendría a debilitar la tesis falsacionista: cuando contrastamos una hipótesis, en realidad estamos contrastando también otras hipótesis implícitas. 

Más allá de la caracterización de un método como científico, el estudio de la metodología consiste también en describir la dinámica del conocimiento científico en cuanto a los conjuntos de teorías en que tiende a formularse. El concepto de los \textit{paradigmas científicos} presentado por Kuhn (1962) trata de describir el proceso de avance del conocimiento científico. Un paradigma científico es una forma de entender el mundo y practicar la investigación científica que toma la forma de un conjunto de principios que dominan un área de investigación. La ciencia que se lleva a cabo dentro de un paradigma dado se denomina ciencia normal y es capaz de explicar un conjunto lo suficientemente amplio de fenómenos. Sin embargo, la existencia de fenómenos sin explicar abren la posibilidad de sustituir un paradigma predominante por otro. Cuando aparece una nueva forma de entender el mundo que es capaz de explicar algunos fenómenos previamente inexplicados y mantiene al mismo tiempo la mayor parte de la capacidad explicativa de paradigmas anteriores, se produce una revolución científica y un paradigma es sustituido por otro en su papel predominante. Sin embargo, en la medida en que sustituir un paradigma por otro es muy costoso, existe una fuerte resistencia a estos cambios a pesar de que las predicciones de las teorías de un paradigma puedan ser falsadas. Kuhn afirma que los paradigmas son inconmensurables en el sentido de que la diferencia entre unos y otros es muy dificil o imposible de caracterizar. Los paradigmas de Kuhn se convirtieron en una concepto central a la historiografía de la ciencia. Aunque fueron matizados posteriormente, los paradigmas constituyen hoy un instrumento básico a la hora de organizar las diferentes expresiones de conocimiento científico. 

En relación al falsacionismo de Popper y los paradigmas de Kuhn, Lakatos formula un nuevo modelo del progreso científico que trata de compatibilizar varios postulados. Primero, Lakatos afirma que los paradigmas científicos sí pueden ser descritos explícitamente, a diferencia de lo que afirmaba Kuhn. Segundo, el falsacionismo de Popper según el cual las teorías falsadas pueden y deben ser desechadas. Por último, la idea de que falsar una teoría no implica desechar todas las teorías relacionadas. Para compatibilizar estos puntos, Lakatos describe el concepto de programa de investigación como un conjunto de teorías agrupadas en torno a un núcleo rodeado por un cinturón protector. El núcleo está formado por teorías no falsables que no pueden ser abandonadas sin abandonar el programa en su conjunto. Por ello, generan un grado máximo de resistencia a ser desechadas. Por otra parte, el cinturón protector agrupa el conjunto de instrucciones, teorías y corolarios que se desprenden del núcleo. En palabras de Lakatos, el cinturón protector ejerce de escudo ante las anomalías no explicadas. Ante estas anomalías, las teorías del cinturón protector pueden ser sustituidas por otras capaces de explicar la anomalía pero que no contradicen las teorías del núcleo. El concepto de programa de investigación se ha convertido en un elemento habitual en la ciencia económica. Así, es habitual hablar del programa de investigación neoclásico, de Keynes o de Lucas entre otros. El análisis del método científico ha dado lugar también a enfoques críticos con las concepciones anteriores. Entre ellos resultado destacable el \textit{anarquismo metodológico de Feyerabend} que aboga por obviar las caracterizaciones rígidas del método científico y las critica por dogmáticas y contraproducentes. Feyerabend defiende el anarquismo metodológico como representación del eclecticismo que necesita la práctica científica para desarrollarse, así como lo dañino de la imposición de reglas estrictas. El enfoque sociológico del método científico examina el avance de la ciencia como resultado de la interacción de los diferentes grupos sociales relevantes en la investigación científica. El análisis de la retórica de la economía examina el éxito de los modelos económicos como resultado de su atractivo estético y su capacidad para persuadir al receptor de la idea de su valor científico. 

Aunque la discusión del método científico en general se ha reflejado en la práctica económica, los principales \textbf{debates metodológicos idiosincráticos a la ciencia económica} han moldeado la forma de investigar y explicar los fenómenos económicos de forma más directamente reconocible. Uno de los principales debates ha concernido el papel de las matemáticas que deben jugar a la hora de explicitar las diferentes teorías. ¿Son las matemáticas el lenguaje apropiado para caracterizar un modelo económico? ¿Qué áreas de las matemáticas han aportado herramientas a los economistas? Las respuestas a estas preguntas ha tendido a trascurrir paralela a la evolución de la propia ciencia matemática. Así, los economistas clásicos hicieron un uso parco de las matemáticas. Malthus fue uno de los pocos economistas clásicos en aplicar conceptos matemáticos, en concreto el concepto del óptimo. Ricardo no aplicó innovaciones matemáticas a explicación de fenómenos económicos, pero introdujo con enorme éxito la formulación de modelos abstractos que utilizan el método deductivo para derivar hipótesis a partir de un conjunto reducido de postulados. Fueron los autores marginalistas y los neoclásicos los que incorporaron plenamente el lenguaje matemático en la ciencia económica. El cálculo diferencial aparece como la principal herramienta de modelización y la formación matemática comienza a ser un hecho distintivo del economista neoclásico. Ejemplos de este hecho son Dupuit, Cournot, Jevons, Edgeworth, Walras o Marshall. La modelización del comportamiento de agentes individuales --que se denominaría posteriormente como microeconomía- comienza a expresarse predominantemente en términos matemáticos y más concretamente en términos relativos a la teoría de conjuntos que se había desarrollado al final del XIX y en las primeras décadas del siglo XX. A partir de los años 30 y en especial en el periodo de guerra y post-guerra, la economía comienza a incorporar el análisis estadístico y probabilístico que se generaliza en ciencias naturales y en la llamada investigación de operaciones que alcanza su apogeo con la guerra mundial y la posterior carrera armamentística. La ciencia económica adquiere en este periodo la denominación más general de ciencia de la decisión. En la actualidad, la ciencia económica hace uso de un amplio abanico de técnicas matemáticas relativamente avanzadas: análisis real, optimización dinámica, teoría de conjuntos, topología diferencial y métodos de inferencia estadística avanzados. Las matemáticas son consideradas como el lenguaje de la economía por los autores y escuelas más relevantes. A pesar de ello, existen escuelas al margen de la ortodoxia que critican el uso de las matemáticas para caracterizar fenómenos pretendidamente imposibles de representar con estas herramientas. Entre ellos destacan los llamados neo-austriacos así como los neomarxistas. Más allá de estas escuelas corrientes de carácter heterodoxo, existe cierta controversia por una supuesta excesiva matematización de la economía y la utilización de las matemáticas como instrumento retórico y de prestigio. 

El segundo gran debate metodológico, y uno de los más antiguos, concierne el uso de métodos deductivos frente a la inducción a partir de ejemplos empíricos y hechos históricos. El término alemán \textit{methodenstreit} se ha utilizado para denominar el duro debate que enfrentó a Carl Menger, padre de la escuela austriaca, y a Schmoller, referente del historicismo alemán. El primero defendía la utilización de las observaciones empíricas como método para ilustrar o validar modelos previamente formulados a partir de la deducción. El segundo criticaba todo modelo basado en deducción a partir de axiomas, y propugnaba la inferencia de leyes de la economía a partir del estudio de la historia. Aunque la postura de Menger se impuso, el \textit{methodenstreit} resultó en una mayor conciencia de la necesidad de contrastar empíricamente los modelos teóricos. En la actualidad, la econometría de series temporales y determinadas técnicas de análisis del impacto de políticas públicas como las regresiones de discontinuidad o los métodos de diferencias en diferencias (\textit{diff-in-diff}) son en cierto modo reminiscentes del énfasis historicista en el hecho empírico frente a la modelización desde la deducción. 

La controversia de los supuestos es un viejo debate metodológico que concierne la necesidad de formular modelos a partir de supuestos verdaderos. Aunque el mismo debate había enfrentado anteriormente a Senior, Mill y Cairnes, fue Friedman (1953) quien hizo renacer la controversia. Friedman afirmaba el error de juzgar una hipótesis en base a la realidad de todos los supuestos que requiere. Según Friedman, basta con que algunos supuestos sean ciertos si el modelo predice bien fenómenos aún no observados. Así, mientras que el supuesto de racionalidad perfecta no se verifica si examinamos el comportamiento de agentes reales, supone una aproximación razonable al comportamiento medio de agregados de agentes individuales y permite formular predicciones y explicaciones notablemente fieles a la realidad. Koopmans, Samuelson y otros autores interpretaron a Friedman como una afirmación de la irrelevancia absoluta de la verdad de los supuestos. En la actualidad, el debate sigue vivo en boca de algunas voces críticas con la economía, que la atacan en base a la utilización de supuestos poco realistas. A pesar de ello, la crítica no es especialmente relevante: los modelos económicos siguen basándose en supuestos poco realistas de la misma forma que la física asume que los cuerpos no tienen dimensión a la hora de caracterizar el efecto de la gravedad, o la geografía asume que el globo terráqueo es una esfera perfecta aunque no lo sea. La postura de Friedman ha prevalecido: los modelos son juzgados por su capacidad para predecir bien sin dar demasiada importancia a la realidad de la generalidad de los supuestos que subyacen a un modelo determinado.

El individualismo metodológico y su crítica es un debate recurrente en las ciencias sociales que ha afectado también a la economía, aunque de manera relativamente menor. El individualismo metodológico es una doctrina científica que afirma que la explicación a fenómenos sociales resulta en último término de fenómenos que acontecen a los individuos. El holismo metodológico y otras corrientes asociadas defienden la existencia de leyes a nivel macroscópico que se aplican al conjunto del sistema y que no son resultado de fenómenos de nivel individuales. Esta doctrina se asocia generalmente al pensamiento marxista y a la sociología. En economía, el individualismo metodológico es la doctrina dominante: el individuo en forma de empresa, consumidor, productor, etc... es el único actor que se considera relevante. 

El debate sobre la microfundamentación de los modelos macroeconómicos está relacionado con el individualismo metodológico, aunque se trata de conceptos diferentes. Mientras que el individualismo metodológico es un concepto de carácter teórico que se atribuye a los fenómenos económicos en general, la microfundamentación es un concepto de carácter aplicado relativo a modelos concretos. La microfundamentación consiste en la práctica de modelizar fenómenos agregados como resultado de la actuación de agentes microeconómicos que representan el comportamiento de una microeconomía como si de un agente individual se tratase, con los métodos de la microeconomía. Keynes fue pionero en formular modelos de la macroeconomía con una muy débil microfundamentación que representaban las fluctuaciones de nivel macro de manera directa y sin equiparar con procesos de decisión microeconómicos. Este enfoque metodológico se mostró muy efectivo a la hora de predecir el efecto de una gran variedad de políticas económicas. En los últimos 60 y sobre todo en la década de los 70, la metodología keynesiana fue objeto de una serie de críticas en relación a su capacidad para tener en cuenta la reacción de los agentes a cambios en políticas, a la formación de equilibrios y a la dificultad para valorar variaciones de bienestar dadas distintos estados agregados. Autores como Clower o Phelps en los 60, pero sobre todo Barro, Grossman, Lucas, Sargent y otros autores de los años 70 iniciaron un programa de investigación caracterizado por una microfundamentación explícita de la macroeconomía en todos sus aspectos. En la actualidad, modelos con débil microfundamentación como el IS-LM o el modelo de Mundell-Fleming siguen jugando un papel central en la docencia y en el policy-making por su capacidad para representar relaciones entre variables y efectos de política con métodos simples que no requieren conocimiento de técnicas matemáticas complejas o costosas. Sin embargo, en el contexto de investigación, los modelos DSGE caracterizados por su plena microfundamentación son hoy en día dominantes. 

La racionalidad de los agentes y el supuesto de maximización de las preferencias de los agentes es uno de los supuestos que más controversia generan en cuanto a la modelización microeconómica. ¿Son realmente los individuos de tomar las decisiones necesarias para obtener los grados más elevados de felicidad que pueden alcanzar, dado un contexto institucional y una dotación previa? ¿Conocen verdaderamente sus preferencias? ¿Disponen de la capacidad de proceso suficiente como para conocer esas acciones que realmente les llevan a la máxima felicidad? El agente maximizador que resulta de responder afirmativamente a estas preguntas se ha denominado ``homo economicus''. El debate metodológico concierne lo apropiado de formular modelos que asumen este ``homo economicus'' como decisor principal. Las críticas giran en torno a las dificultades que los seres humanos tienen para conocer sus propias preferencias, los costes que requiere la adquisición de información para la toma de decisiones y la falta de racionalidad que en la práctica muestran muchos seres humanos, en el sentido de que toman sistemáticamente decisiones que contradicen lo que aparentemente son sus preferencias. En la actualidad, los modelos principales siguen aplicando el supuesto de racionalidad perfecta y la hipótesis de expectativas racionales. Los avances en la caracterización de comportamiento no racionalidad han sido notables en los últimos años y se agrupan en torno al llamado \textit{behavioral economics}. Aunque su penetración se extiende a un número creciente de áreas, desde el marketing a las finanzas, la dificultad para formular un modelo general de las desviaciones respecto a la racionalidad, los problemas de tratabilidad y el carácter ad-hoc de muchos de sus preceptos constituyen obstáculos a su avance y eventual prevalencia respecto del supuesto de racionalidad perfecta.

La ciencia económica ha mostrado, en general, una notable capacidad para incorporar\textbf{innovaciones metodológicas}. En la frontera de investigación económica es habitual la utilización de métodos innovadores que en ocasiones se convierten en el origen de verdaderos programas de investigación. A menudo, la incorporación de nuevos métodos surge del contacto con otras disciplinas del conocimiento. La valoración de las innovaciones presentadas a continuación requiere aún y en general de varios años o incluso décadas. El enfoque de \textit{redes} es una de las más recientes innovaciones metodológicas. Su idea central consiste en la caracterización de la estructura de las relaciones entre distintos agentes como generadoras de procesos económicos diferenciados. Las aplicaciones son numerosas: desde formular la transmisión de una quiebra bancaria sobre todo el sistema en función de las interrelaciones entre entidades financieras, hasta representar el impacto de un shock de oferta en una industria determinada sobre otras industrias conectadas, pasando por el efecto de situar a alumnos brillantes en grupos con alumnos menos capaces en función de sus relaciones de amistad. La \textit{agent-based models} o modelización basada en agentes consiste en formular modelos en los que agentes heterogéneos interaccionan a partir de reglas de decisión simples que no están basadas en procesos de maximización. Este enfoque metodológico tiene origen en la modelización de procesos biológicos y físicos y por el momento ha logrado ser aplicado de forma exitosa en algunas áreas como la modelización de puntos focales o de Schelling, economía de la salud y gestión de recursos naturales. La \textit{teoría del caos} tiene su origen en la modelización meteorológica pero ha acabado transformándose en un área diferenciada de las matemáticas. La teoría del caos examina las fluctuaciones en sistemas dinámicos que no tienden a un estado estacionario ni tienen periodicidad determinada, aún manteniendo un carácter puramente determinista. La teoría del caos en economía apareció a finales de los 80 y principios de los 90 como una herramienta prometedora en la modelización del ciclo económico. Sin embargo, se trata de un programa relativamente estancado en la actualidad, por no haber mejorado la capacidad predictiva de modelos DSGE convencionales. Las \textit{redes neuronales y el deep learning} o aprendizaje profundo consisten en la construcción de modelos de reconocimiento de patrones en conjuntos de datos a partir de simulaciones de las redes de neuronas presentes en el cerebro animal. Este tipo de herramientas son capaces de ``aprender'' de sus errores reconociendo patrones en un proceso de feedback que resulta en una mejora constante de su capacidad predictiva. Sus aplicaciones abarcan un número enorme de disciplinas, pero en economía las aplicaciones principales se han centrado hasta ahora en la modelización de mercados financieros y de comportamiento de consumidores, en el análisis de series temporales y en la clasificación de diferentes tipos de empresas y consumidores. Además, comienza a plantearse la posibilidad de formular modelos en los que los agentes tomen decisiones basadas en el output de una red neuronal, con el objetivo de caracterizar la interacción entre racionalidad limitada y capacidad de aprendizaje. La \textit{economía experimental} recibió un reconocimiento expreso con la concesión del Premio Nobel de Economía a Vernon Smith en 2002 (junto con Kahneman). Desde los años 50, Joan Robinson y Paul Samuelson habían generalizado la percepción de que la economía era una ciencia social en la que los experimentos no eran posibles. Los trabajos de Kahneman y Tversky pero especialmente de Vernon Smith fueron esenciales para cambiar esta percepción. No sólo es posible formular experimentos en economía para caracterizar el impacto de cambios en los incentivos sobre la decisión de los agentes, sino que es posible utilizar estos experimentos para contrastar modelos teóricos y formular nuevas hipótesis que generalicen expresiones del comportamiento humano. La economía experimental se encuentra plenamente integrada hoy en el \textit{mainstream} metodológico, con numerosos programas experimentales que tratan de caracterizar el proceso de formación de equilibrios generales, la interacción estratégica entre agentes y la formación de precios en mercados simulados. 

Por último, es interesante examinar algunas de las \textbf{cualidades deseables de un modelo económico} dado. Las siete características presentadas a continuación muestran a menudo un cierto solapamiento y también \textit{trade-offs} entre ellas. En primer lugar, un modelo económico debe mostrar \textit{parsimonia}, en el sentido de que requiera el menor número de supuestos posibles para explicar o predecir un fenómeno determinado. Cualquier fenómeno puede ser explicado dado un número suficientemente elevado de supuestos, pero el hecho de requerir información adicional implica poder explicar un número cada vez menor de fenómenos. La \textit{tratabilidad} de un modelo caracteriza la posibilidad de extraer resultados por métodos analíticos sin incurrir en excesivos costes en términos de capacidad de proceso o conocimientos técnicos. La \textit{profundidad conceptual} como característica deseable refleja el hecho de que un modelo caracterice fenómenos no triviales con explicaciones que efectivamente sean capaces de señalar la causalidad entre un hecho y otro. Un modelo debe ser \textit{generalizable} para poder ser utilizado para explicar un número elevado de fenómenos similares en lo fundamental pero diferentes en lo superficial, con la introducción de supuestos adicionales de carácter secundario. \textit{falsabilidad} de un modelo se ha examinado anteriormente e implica la posibilidad de probar falso un modelo como resultado de la realización de un estado de la naturaleza que el modelo excluye. La \textit{consistencia empírica} refleja la concordancia entre las predicciones de un modelo y los hechos observados, e implica que el modelo no ha sido falsado. La \textit{precisión predictiva}, por último, refleja la capacidad de un modelo para formular predicciones cuantitativamente acertadas. Como requisito previo para ello, el modelo deber formular predicciones bien definidas. Se trata de una propiedad íntimamente relacionada con la consistencia empírica, pero diferente: es más fácil que un modelo sea consistente empíricamente si realiza predicciones muy generales y poco precisas.

Como conclusión, es importante tener en cuenta que la elección de un modelo económico frente a otro es a menudo una decisión con fuerte componente subjetivo y dependiente del contexto concreto. No existe en economía ni una caracterización unívoca del método ni un modelo general que represente todos los fenómenos relevantes. La elección de uno u otro está sujeta al fenómeno concreto a modelizar, así como a restricciones de tiempo de desarrollo y tratamiento analítico. Sin embargo, determinados rasgos del método económico tales como el comportamiento optimizador o el individualismo de los agentes se han utilizado en numerosas áreas de la ciencia social, en un proceso que ha generado críticas por la llamada tendencia al ``imperialismo económico''. La otra cara de la moneda es la capacidad de la economía para absorber conocimiento de otras disciplinas. La gran riqueza de modelos económicos existentes incluso para explicar un mismo fenómeno es en cierta medida resultado de la interacción progresiva con otras disciplinas del conocimiento: matemáticas, psicología, filosofía, ingeniería, física...

\seccion{Preguntas clave}
\begin{itemize}
	\item ¿Qué estudia la economía?
	\item ¿Cómo lo estudia?
	\item ¿Cómo ha evolucionado el método de la ciencia económica?
	\item ¿Qué debates metodológicos se plantean?
	\item ¿Qué innovaciones metodológicas se han producido en los últimos tiempos?
	\item ¿Qué cualidades deben tener los modelos económicos?
\end{itemize}

\esquemacorto

\begin{esquema}[enumerate]
	\1[] \marcar{Introducción}
		\2 Contextualización
			\3 Origen del término ``economía''
			\3 Edad antigua y edad media
			\3 Actualidad desde s. XVIII
		\2 Objeto
			\3 ¿Qué estudia la economía
			\3 ¿Cómo lo estudia?
		\2 Estructura
			\3 Objeto de la economía
			\3 El método científico
			\3 Debates metodológicos
			\3 Innovaciones metodológicas recientes
			\3 Cualidades deseables en un modelo económico
	\1 \marcar{Objeto de la economía}
		\2 Definiciones históricas de economía
			\3 Escolástica medieval
			\3 Steuart y Smith: buen gobierno
			\3 Say, Senior, Ricardo: descubrimiento de leyes naturales
			\3 John Stuart Mill: acción humana sobre producción y distribución
			\3 Alfred Marshall: gestión de asuntos corrientes humanos
			\3 Lionel Robbins y Paul Samuelson: gestión de escasez
			\3 Lipsey: economía no es sólo gestión de escasez
			\3 Economía como ciencia de la decisión óptima
			\3 Implicaciones
		\2 Debates sobre objeto de economía
			\3 Disciplina normativa o positiva
			\3 Imperialismo de la economía
			\3 Jacob Viner
	\1 \marcar{El método científico}
		\2 Idea clave
			\3 Conjunto de técnicas
			\3 Características del método científico
			\3 Diferentes concepciones del método científico
		\2 Método inductivo vs deductivo
			\3 Idea clave
			\3 Método inductivo
			\3 Método deductivo
		\2 Método hipotético-deductivo
			\3 Idea clave
			\3 Ejemplo de método H-D en economía
			\3 Tesis de la simetría
		\2 Falsacionismo
			\3 Idea clave
			\3 Influyente sobre
			\3 Ejemplo de teoría no falsable:
			\3 Tesis de Duhem-Quine
		\2 Paradigmas científicos de Kuhn
			\3 Idea clave
			\3 Normalidad y revolución
			\3 Valoración
		\2 Programas de investigación de Lakatos
			\3 Idea clave
			\3 Núcleo
			\3 Cinturón protector
			\3 Valoración
		\2 Otros enfoques
			\3 Anarquismo metodológico de Feyerabend
			\3 Enfoque sociológico
			\3 Enfoque retórico
	\1 \marcar{Debates metodológicos de la ciencia económica}
		\2 \textit{Methodenstreit}
			\3 Idea clave
			\3 Posturas enfrentadas
			\3 Valoración
		\2 Matemáticas
			\3 Idea clave
			\3 Posturas enfrentadas
			\3 Valoración
		\2 Controversia de los supuestos
			\3 Idea clave
			\3 Posturas enfrentadas
			\3 Valoración
		\2 Individualismo metodológico vs holismo
			\3 Idea clave
			\3 Posturas enfrentadas
			\3 Valoración
		\2 Microfundamentación
			\3 Idea clave
			\3 Posturas enfrentadas
			\3 Valoración
		\2 Homo economicus y racionalidad de agentes
			\3 Idea clave
			\3 Posturas enfrentadas
			\3 Valoración
		\2 Equilibrio general y parcial
			\3 Idea clave
			\3 Posturas enfrentadas
			\3 Valoración
	\1 \marcar{Innovaciones metodológicas recientes}
		\2 Idea clave
			\3 Frontera de la investigación
			\3 Valoración
		\2 Redes
			\3 Idea clave
			\3 Valoración
		\2 Modelos basados en agentes
			\3 Idea clave
			\3 Valoración
		\2 Caos
			\3 Idea clave
			\3 Valoración
		\2 Redes neuronales y deep learning
			\3 Idea clave
			\3 Valoración
		\2 Economía experimental
			\3 Idea clave
			\3 Valoración
	\1 \marcar{Cualidades deseables en un modelo económico}
		\2 Idea clave
			\3 Valoración de modelos
			\3 Cumplimiento de características
		\2 Profundidad conceptual
			\3 Idea clave
			\3 Justificación
		\2 Falsable
			\3 Idea clave
			\3 Justificación
		\2 Generalizable
			\3 Idea clave
			\3 Justificación
		\2 Tratable
			\3 Idea clave
			\3 Justificación
		\2 Parsimonia
			\3 Idea clave
			\3 Justificación
		\2 Consistencia empírica
			\3 Idea clave
			\3 Justificación
		\2 Precisión predictiva
			\3 Idea clave
			\3 Justificación
	\1[] \marcar{Conclusión}
		\2 Recapitulación
			\3 Objeto de la economía
			\3 Evolución del método
			\3 Debates metodológicos
			\3 Innovaciones metodológicas recientes
			\3 Cualidades deseables de los modelos económicos
		\2 Idea final
			\3 No hay un modelo único y general
			\3 Interdisciplinariedad

\end{esquema}

\esquemalargo

\begin{esquemal}
	\1[] \marcar{Introducción}
		\2 Contextualización
			\3 Origen del término ``economía''
				\4 Jenofonte (400 AC)
				\4[] En su obra Oikonomikos
				\4[] ``Oikonomía'' / $oiko\nu o \mu i \alpha$
				\4 Oikonomía como gestión de la casa y el negocio
				\4[] Buena llevanza de las cuentas domésticas
				\4[] Gestión apropiada de los recursos domésticos
			\3 Edad antigua y edad media
				\4 No existe economía como tal
				\4 Examen de moralidad de los intercambios
				\4 Economía y derecho se estudian a la vez
			\3 Actualidad desde s. XVIII
				\4 Posición central en la sociedad
				\4[] Economía central al debate político
				\4[] Conocimientos básicos necesarios en todas actividades
				\4[] Asociación entre ciencia económica y mercados
				\4[] Objeto de esta oposición ;)
				\4[$\Rightarrow$] Necesario entender qué y cómo estudia la economía
		\2 Objeto
			\3 ¿Qué estudia la economía
				\4 ¿Cuál es su objetivo?
				\4 ¿Qué fenómenos trata de entender y explicar?
				\4 ¿Cómo evolucionan respuestas a éstas preguntas?
			\3 ¿Cómo lo estudia?
				\4 ¿Qué relación con el método científico en general?
				\4 ¿Qué caracteriza al método económico en la actualidad y en el pasado?
				\4 ¿Qué debates metodológicos han moldeado la ciencia económica?
				\4 ¿Qué innovaciones metodológicas hay en la frontera de la investigación actual?
				\4 ¿Qué características debe tener un buen modelo económico?
		\2 Estructura
			\3 Objeto de la economía
			\3 El método científico
			\3 Debates metodológicos
			\3 Innovaciones metodológicas recientes
			\3 Cualidades deseables en un modelo económico
	\1 \marcar{Objeto de la economía}
		\2 Definiciones históricas de economía\footnote{Ver en \textit{Economics, definition of}.}
			\3 Escolástica medieval
				\4 Aspectos morales del intercambio
				\4 Transacciones consideradas acordes dcho. divino
			\3 Steuart y Smith: buen gobierno
				\4 Economía política es ciencia del buen gobierno
				\4[] Útil para gobernantes y estadistas
				\4[] Influencia del ``cameralismo'' alemán
				\4 Objetivos
				\4[] Proveer bienestar máximo a población
				\4[] Generar ingresos públicos para proveer servicios
			\3 Say, Senior, Ricardo: descubrimiento de leyes naturales
				\4 Economía política es descubrimiento de leyes
				\4 Leyes más o menos naturales que determinan:
				\4[] $\to$ Producción
				\4[] $\to$ Distribución
				\4[] $\to$ Consumo
			\3 John Stuart Mill: acción humana sobre producción y distribución
				\4 Objeto no es buen gobierno de una nación
				\4[] $\to$ Supone confundir ciencia con arte
				\4 Economía no es sólo descubrimiento de leyes
				\4[] Producción y distribución dependen de tecnología
				\4[] $\to$ Factor ajeno a economía
				\4[$\Rightarrow$] Economía es estudio de:
				\4 Leyes de producción y distribución de riqueza
				\4[] Dependen de:
				\4[] $\to$ Leyes físicas y conocimiento técnico
				\4[] $\to$ Instituciones, leyes, normas humanas
				\4[] Economía debe estudiar producción y distribución:
				\4[] $\to$ En lo que depende de acción humana
				\4[$\then$] Leyes psicológicas y morales
				\4[] $\to$ Que determinan producción
			\3 Alfred Marshall: gestión de asuntos corrientes humanos
				\4 Introduce ``economía'' por primera vez
				\4[] Reemplaza a ``economía política''
				\4 Economía es:
				\4[] Estudio del comportamiento humano diario
				\4[] destinado a alcanzar el máximo bienestar
				\4 Deseos humanos cambian con el tiempo
				\4[] Necesario comprender evolución de necesidades humanas
				\4[] Estudio de evolución forma parte de economía
			\3 Lionel Robbins y Paul Samuelson: gestión de escasez
				\4 Economía estudio de comportamiento humano
				\4[] Gestionando recursos escasos con usos alternativos
				\4[] Para satisfacer una serie de necesidades humanas
				\4[$\Rightarrow$] Economía es ciencia de la decisión
				\4 Definición generalizada actualmente
				\4[] Libros de texto, artículos, ...
			\3 Lipsey: economía no es sólo gestión de escasez
				\4 Economía no concierne sólo recursos escasos
				\4[] P. ej.: desempleo, producción dentro de FPP
				\4 Economía estudio de aprovechamiento óptimo
				\4[] De recursos que no son necesariamente escasos
				\4[] $\to$ Pero que pueden estar mal asignados
			\3 Economía como ciencia de la decisión óptima
				\4 Años 60 en adelante
				\4 Economía es conjunto de métodos
				\4[] Aplicados a amplio rango de comportamientos
				\4[] $\to$ No sólo relativos a producción y distribución
				\4[] $\then$ Al menos no en sentido amplio
				\4 Economía estudia decisión óptima
				\4[] Comportamiento tendente
				\4[] $\to$ A alcanzar satisfacción de objetivos
				\4[] $\then$ Ciencia de la racionalidad
				\4 Influencias clave
				\4[] Desarrollo de teoría de juegos
				\4[] Consolidación modelo microeconómico neoclásico
				\4 Expansión a otros dominios de ciencias sociales
				\4[] Gary Becker:
				\4[] $\to$ Capital humano
				\4[] $\to$ Familia
				\4[] Buchanan, Downs, Tullock...
				\4[] $\to$ Proceso político
			\3 Implicaciones
				\4 Def. Robbins-Samuelson sigue siendo más citada
				\4 Gestión de escasez en sentido muy amplio
				\4 Aplicación a variedad de problemas
				\4[] Cada vez más consolidada
		\2 Debates sobre objeto de economía
			\3 Disciplina normativa o positiva
				\4 Subyacente a todas las definiciones anteriores
				\4 Debate subyacente a la definición de economía
				\4[] Carácter normativo vs positivo
				\4 Normatividad
				\4[] La economía debe decir qué hacer
				\4 Positividad
				\4[] La economía consiste en describir realidad
				\4[] Qué hacer implica juicios de valor
				\4[] Ciencia debe mantenerse al margen
			\3 Imperialismo de la economía
				\4 Tendencia a aplicar métodos de economía
				\4[] En generalidad de problemas de carácter social
				\4[] $\to$ Política
				\4[] $\to$ Derecho
				\4[] $\to$ Sociología
				\4[] $\to$ Familia
				\4[] $\to$ Ciencia
				\4[] $\to$ Investigación
				\4[] $\to$ ...
			\3 Jacob Viner
				\4 ``Economía es lo que los economistas hacen''
				\4 Definiciones de economía tienden a ser ex-post
				\4[] Sirven para delimitar contornos de investigación ya hecha
				\4[] $\Rightarrow$ No es necesario definir economía
				\4[] $\Rightarrow$ En la práctica, economistas ignoran definiciones
	\1 \marcar{El método científico}
		\2 Idea clave
			\3 Conjunto de técnicas
				\4 Utilizadas para:
				\4[] Investigar causas de fenómenos
				\4[] Adquirir nuevos conocimientos
				\4[] Integrar conocimientos previos
				\4 Técnicas se aplican de forma iterativa
				\4[] Para mejorar predicción y explicación
				\4 Técnicas son generalmente:
				\4[] Caracterizar
				\4[] $\to$ observaciones
				\4[] $\to$ objeto de investigación
				\4[] Deducir predicciones a partir de principios
				\4[] Formular hipótesis dadas observaciones
				\4[] Contrastar hipótesis y predicciones
				\4[] $\to$ Utilizadas en grado variable
			\3 Características del método científico
				\4 General
				\4[] $\to$ Utilizable más allá de fenómeno concreto
				\4 Contrastable empíricamente
				\4[] $\to$ Proposiciones contrastables frente a realidad
				\4 Neutral
				\4[] $\to$ Sin juicios de valor
				\4[] $\to$ Especialmente delicado en ciencias sociales
			\3 Diferentes concepciones del método científico
				\4 Diferentes formas de alcanzar ese conocimiento
				\4 Evolución temporal
				\4[] Investigación sobre la investigación
				\4[] ¿De qué manera investigar?
		\2 Método inductivo vs deductivo
			\3 Idea clave
				\4 Dicotomía básica del conocimiento
				\4[] Paralelismos en otros ámbitos:
				\4[] $\to$ Síntesis y análisis
				\4[] $\to$ Confirmación y verificación
				\4 Dos maneras básicas de proceder
				\4[] De lo particular a lo general
				\4[] $\to$ Inducción
				\4[] De lo general a lo particular
				\4[] $\to$ Deducción
			\3 Método inductivo
				\4 Derivación de conclusiones generales
				\4[] A partir de hechos particulares
				\4 Conclusiones sujetas a crítica de Hume:
				\4[] No hay certeza de cumplimiento futuro
				\4 Ejemplo:
				\4[] Después de toda crisis anterior
				\4[] $\to$ Ha llegado una recuperación
				\4[] $\Rightarrow$ Todas las crisis terminan en recuperación
				\4[] No hay certeza de que vaya a seguir pasando
			\3 Método deductivo
				\4 Deducción lógica a partir de axiomas
				\4[1] Postular axiomas
				\4[2] Derivar resultados de axiomas
				\4[3] Extraer conclusiones
				\4[$\then$] De general a particular
		\2 Método hipotético-deductivo
			\3 Idea clave
				\4 Poincaré, círculo de Viena, Hempel y Oppenheim (1965)
				\4[1] Se observa un fenómeno que se pretende explicar
				\4[2] Se propone una hipótesis
				\4[3] Se deducen resultados de la hipótesis
				\4[4] Se contrastan resultados con experimentos empíricos
				\4[] $\to$ Si hechos no contradicen, nuevos tests
				\4[] $\to$ Si hechos contradicen, rechazo de la hipótesis
			\3 Ejemplo de método H-D en economía
				\4 Fenómeno
				\4[] Se observa correlación entre educación y enfermedad
				\4 Hipótesis
				\4[] Enfermedad causa malos resultados educativos
				\4 Deducción
				\4[] Si corregimos enfermedad, mejorarán resultados
				\4 Experimento
				\4[] Medicamentos antiparasitarios a algunos niños
				\4[] A otros niños, placebo
				\4 Resultado
				\4[] Niños con medicamento mejoran resultados
				\4[] $\to$ Hipótesis se considera probada
			\3 Tesis de la simetría
				\4 Explicación y predicción
				\4[] Surgen de misma estructura teórica
				\4 Si una teoría sirve para explicar
				\4[] $\to$ Debe servir para predecir y viceversa
		\2 Falsacionismo
			\3 Idea clave
				\4 Karl Popper principal impulsor
				\4 Complemento del método hipotético-deductivo
				\4 Teorías científicas deben poder ser falsadas
				\4[] Aunque no puedan ser verificadas
				\4[] Deben predecir algún estado de la naturaleza
				\4[] Cuya no-existencia puede verificarse
			\3 Influyente sobre
				\4 Milton Friedman
				\4 Blaug
				\4 En general, toda economía a partir de 1950s
			\3 Ejemplo de teoría no falsable:
				\4 Predicciones de Marx sobre evolución de capitalismo
				\4 Si crisis final llega $\to$ Confirmación
				\4[] Si no llega $\to$ Ya llegará
			\3 Tesis de Duhem-Quine
				\4 Contrastación de una hipótesis
				\4[] $\to$ Contrastación de varias hipótesis a la vez
				\4[] $\Rightarrow$ Necesaria precaución desechando teorías
				\4[] $\Rightarrow$ Pueden ser útiles aun aparentemente falsadas
		\2 Paradigmas científicos de Kuhn
			\3 Idea clave
				\4 Kuhn (1962)
				\4 ¿Cómo tiene lugar el avance científico?
				\4[] A través de nuevos paradigmas
				\4 Paradigma
				\4[] Forma de entender el mundo
				\4[] Forma de practicar la investigación científica
				\4[] Conjunto de principios
				\4[] $\to$ dominan un área de investigación
				\4 Caracterización de paradigmas
				\4[] no se pueden caracterizar
				\4[] la diferencia entre paradigmas es inconmensurable
			\3 Normalidad y revolución
				\4 Ciencia normal
				\4[] Ciencia llevada a cabo dentro del paradigma
				\4 Revolución científica
				\4[] Existe un hecho generalizado sin explicar
				\4[] $\to$ Aparece nuevo paradigma para explicar
				\4 Nuevo paradigma es capaz de:
				\4[] Explicar fenómeno sin explicar
				\4[] Mantener capacidad de explicación de paradigmas anteriores
				\4[$\Rightarrow$] Sustitución de un paradigma por otro
			\3 Valoración
				\4 Herramienta historiográfica
				\4[] Entender evolución del conocimiento científico
				\4[] Entender relación entre modelos
				\4[] Explicar permanencia de teorías a pesar de falsación
				\4 Fuerte influencia sobre toda historiografía
				\4[] Conocimiento científico
				\4[] $\to$ En términos de paradigmas
		\2 Programas de investigación de Lakatos
			\3 Idea clave
				\4 Lakatos
				\4 Tratar de compatibilizar:
				\4[] Los paradigmas sí pueden ser descritos
				\4[] Las teorías falsadas deben ser y son desechadas
				\4[] Falsación no implica:
				\4[] $\to$ desechar todas teorías relacionadas
				\4[] $\to$ Programas de investigación en vez de paradigmas
				\4 Programas de investigación
				\4[] Formadas por dos elementos
				\4[] $\to$ Núcleo
				\4[] $\to$ Cinturón protector
			\3 Núcleo
				\4 Conjunto de teorías no falsables
				\4 No pueden abandonarse sin abandonar programa
				\4 Resistencia máxima a desechar
			\3 Cinturón protector
				\4 Conjunto de instrucciones, corolarios, teorías
				\4[] Sí son falsables
				\4 Aplicaciones particulares de teorías del núcleo
				\4 Sirven de escudo ante anomalías
				\4 Aparición de anomalía
				\4 $\to$ cambio en círculo protector
			\3 Valoración
				\4 Muy habitual uso en economía
				\4 Teorías económicas caracterizadas como programas
				\4[] Programa neoclásico
				\4[] Programa de Keynes
				\4[] Programa de Lucas
				\4[] ...
		\2 Otros enfoques
			\3 Anarquismo metodológico de Feyerabend
				\4 Crítica del método científico moderno
				\4[] Método científico es dogmático y contraproducente
				\4 Necesario ``anarquismo metodológico''
				\4[] $\to$ Reglas científicas impide trabajo de científicos
				\4[] $\to$ No imponer reglas sobre científicos
			\3 Enfoque sociológico
				\4 Ciencia resulta de interacción entre grupos sociales
			\3 Enfoque retórico
				\4 Capacidad de convicción es elemento central
				\4 Formación de narrativas convincentes
				\4 Control de centros de producción científica
	\1 \marcar{Debates metodológicos de la ciencia económica}
		\2 \textit{Methodenstreit}
			\3 Idea clave
				\4 Cómo estudiar fenómenos económicos?
				\4[] ¿Deducción o inducción?
				\4 ¿Deducir leyes abstractas a partir de axiomas?
				\4 ¿Inducir leyes a partir de hechos históricos?
			\3 Posturas enfrentadas
				\4 Menger (1883)
				\4[] Escuela austriaca
				\4[] Deducción a partir de principios evidentes
				\4[] Formulación de leyes abstractas y generales
				\4[] Hechos empíricos útiles para:
				\4[] $\to$ Ilustrar teorías
				\4[] $\to$ Validar teorías
				\4 Schmoller (1883)
				\4[] Historicismo alemán
				\4[] Economía es sistema complejo que evoluciona
				\4[] Leyes a partir de axiomas no son útiles
				\4[] Ejemplos históricos son fuente de conocimiento
			\3 Valoración
				\4 En general, Menger triunfó
				\4[] Economía deduce leyes a partir de axiomas
				\4 Historicismo influyó en neoclasicismo
				\4[] Mayor énfasis sobre contraste empírico
				\4[] Teorías abstractas no son útiles por sí solas
		\2 Matemáticas
			\3 Idea clave
				\4 Matemáticas:
				\4[] $\to$ Lenguaje para formular modelos y conceptos
				\4 ¿Son las matemáticas el lenguaje apropiado?
				\4 ¿Son adecuadas para la economía?
				\4 Evolución del uso de las matemáticas
				\4[] Ligada a evolución de matemáticas
			\3 Posturas enfrentadas
				\4 Clásicos
				\4[] Apenas uso de matemáticas
				\4[] $\to$ Malthus principal exponente
				\4[] $\to$ Ricardo aplica método deductivo
				\4 Neoclásicos
				\4[] Inicio de matematización de la economía
				\4[] Cálculo diferencial es herramienta principal
				\4[] Cournot, Dupuit, Walras, Edgeworth, Marshall
				\4[] $\to$ Consolidación de microeconomía
				\4 Keynesianismo y posguerra
				\4[] Economía incorpora nuevos instrumentos matemáticos
				\4[] Econometría, estadística
				\4[] Economía e investigación de operaciones
				\4[] $\to$ Métodos analíticos para toma de decisiones
				\4 Actualidad
				\4[] Modelos avanzados requieren matemáticas complejas
				\4[] Amplia variedad de técnicas matemáticas:
				\4[] $\to$ Análisis real
				\4[] $\to$ Optimización dinámica
				\4[] $\to$ Teoría de conjuntos
				\4[] $\to$ Topología diferencial
				\4[] $\to$ Métodos estadísticos avanzados
			\3 Valoración
				\4 Matemáticas son el lenguaje de economía moderna
				\4 Críticas al uso de matemáticas son marginales
				\4[] Economía austriaca
				\4[] Economía marxista
				\4 Debate sobre excesiva matematización
				\4[] ¿Realmente son necesarias?
				\4[] ¿Matemáticas como barrera de entrada?
				\4[] ¿Matemáticas complejas aportan valor añadido?
		\2 Controversia de los supuestos
			\3 Idea clave
				\4 Asociada a Friedman (1953)
				\4[] Controversia similar entre J.S. Mill, Senior, Cairnes
				\4 La realidad de todos los supuestos no es importante
				\4[] No todos tienen por qué ser ciertos
				\4[] $\Rightarrow$ No hay que juzgar hipótesis por realidad de supuestos
				\4[] $\Rightarrow$ Basta con que algunos lo sean
				\4[] $\Rightarrow$ Relevante es bondad de predicción
				\4 Ejemplo:
				\4[] Competencia perfecta a menudo no es realista
				\4[] Supuestos sobre funciones dda. y oferta aproximan bien
				\4[] $\to$ Modelo predice suficientemente
			\3 Posturas enfrentadas
				\4 Koopmans, Samuelson y otros reinterpretan Friedman
				\4[] ``El realismo es totalmente irrelevante''
				\4[] ``Lo único importante es predicción''
			\3 Valoración
				\4 Debate actual
				\4 Frecuente afirmar
				\4[] ``La economía utiliza supuestos poco realistas''
				\4 En la práctica, crítica no es relevante
				\4[] Modelos siguen basados en supuestos falsos
				\4[] $\to$ De la misma forma que resto de ciencias
				\4 Friedman prevalece
				\4[] Modelos juzgados por capacidad de predicción
				\4[] Economistas siguen aplicando supuestos irrealistas
		\2 Individualismo metodológico vs holismo
			\3 Idea clave
				\4 Doctrina según la cual:
				\4[] La explicación a fenómenos sociales se basa:
				\4[] $\to$ En comportamiento individual de agentes
			\3 Posturas enfrentadas
				\4 Holismo metodológico
				\4[] Existen leyes macroscópicas sui generis
				\4[] $\to$ que se aplican al conjunto del sistema
				\4 Especialmente:
				\4[] Marxismo, sociología
			\3 Valoración
				\4 Ind. metodológico predomina en economía
				\4[] Todos fenómenos basados en individuos
				\4[] $\to$ Aunque emergencia no se niega del todo
				\4[] $\then$ Demasiada complejidad como para rechazar
				\4 Programa marxista es parte de heterodoxia
				\4[] Fuera del conjunto de modelos habituales
				\4[] Lugar minoritario en policy-making
		\2 Microfundamentación
			\3 Idea clave
				\4 Relacionado con individualismo metodológico
				\4[] Conceptos diferentes
				\4 Microfundamentación es un concepto aplicado
				\4[] Relativo a modelos concretos
				\4[] Definición:
				\4[] $\to$ Práctica de formular modelos agregados
				\4[] $\to$ A partir de conceptos microeconómicos
				\4[] $\to$ Macroeconomía
				\4[] Tratar de armonizar micro y macro
				\4[] $\to$ Mismas herramientas teóricas
				\4 Individualismo metodológico es concepto teórico
				\4[] Propiedad de fenómenos económicos en general
				\4[] $\to$ No de modelo concreto
				\4[] Implica posibilidad de microfundamentar
				\4[] Fenómeno de la emergencia
				\4[] $\to$ Frontera entre ind. metod. y microfundamentación
				\4[] $\to$ Aparición de fenómenos a nivel agregado
				\4[] $\to$ Por interacción entre agentes micro
			\3 Posturas enfrentadas
				\4 Equilibrio general walrasiano
				\4[] Arrow-Debreu (1954) demuestran existencia de equilibrio
				\4[] $\to$ Caracterizan equilibrio en términos micro
				\4[] $\to$ Agentes optimizadores deciden y alcanzan equilibrio
				\4[] Uso para crecimiento de largo plazo
				\4 Crítica de Lucas
				\4[] Modelos estructurales son relevantes
				\4[] Basados en:
				\4[] $\to$ Preferencias
				\4[] $\to$ Tecnologías
				\4[] $\to$ Dotaciones
				\4[] Efectos de políticas deben estimarse con modelos estructurales
				\4[] $\to$ Porque agentes reaccionan a cambios en régimen de políticas
				\4[] Reacciones estimadas bajo régimen diferente
				\4[] $\to$ Predicción errónea con nuevo régimen de políticas
				\4[] $\then$ Microfund. aporta coherencia a modelo
				\4 Economía keynesiana
				\4[] Relaciones ad-hoc entre variables agregadas
				\4[] Usado para c/p y estabilización de ciclo
			\3 Valoración
				\4 Triunfo Metodológico de la microfundamentación
				\4 Años 60
				\4[] Clower, Leijonhufvud, Phelps, Patinkin
				\4[] $\to$ Primera microfundamentación
				\4[] $\to$ Keynesianismo con conceptos walrasianos
				\4 Años 70
				\4[] Barro-Grossman, Drèze, Benassy, Lucas, NMC
				\4[] $\to$ Equilibrio general walrasiano
				\4[] $\to$ Explicar rigideces nominales en términos micro
				\4[] $\to$ Analizar consecuencias macro
				\4 Años 80
				\4[] Microfundamentar rigideces nominales
				\4[] En lenguaje puramente walrasiano
				\4[] Blanchard, Mankiw, Kiyotaki...
				\4 Actualidad
				\4[] Modelos no microfundamentados (p.ej. IS-LM)
				\4[] $\to$ Presentes en cursos básicos
				\4[] $\to$ Utilizados en policy-making
				\4[] Modelos microfundamentados
				\4[] $\to$ DSGE
				\4[] $\to$ Frontera de investigación
				\4[] $\to$ Modelos centrales de política económica
				\4[] $\to$ Mainstream programas de doctorado
		\2 Homo economicus y racionalidad de agentes
			\3 Idea clave
				\4 ¿Es útil suponer que agentes maximizan preferencias?
				\4[] ¿Realmente tratan de hacerlo?
				\4[] ¿Realmente pueden hacerlo?
				\4[] ¿Tiene sentido plantear otra hipótesis?
				\4[$\to$] Es útil representar el homo economicus?
			\3 Posturas enfrentadas
				\4 Agentes optimizadores de la utilidad
				\4[] $\to$ Conocen preferencias perfectamente
				\4[] $\to$ Conocen resultado de sus actos
				\4[] $\to$ Tienen capacidad de proceso necesaria
				\4[] Errores sistemáticos no tienen sentido en l/p
				\4[] $\to$ Es beneficioso corregirlos
				\4[] $\then$ Errores sistemáticos desaparecen
				\4[] $\then$ Tendencia a resultado compatible con racionalidad
				\4[] Errores de optimización son ruido blanco
				\4[] $\to$ No tienen estructura
				\4[] $\to$ En media, se compensan y desaparecen
				\4 Agentes no conocen verdaderas preferencias
				\4[] Marx: ``falsa conciencia''
				\4[] Conocer verdaderas preferencias es costoso
				\4[] Ilusión de la instrospección
				\4 Agentes no tienen información y no la adquieren
				\4[] Informarse es costoso
				\4[] No siempre beneficio supera a coste
				\4 Agentes no son racionales
				\4[] Agentes no actúan de acuerdo con sus fines
				\4[] Sesgos psicológicos, biológicos, sociales
			\3 Valoración
				\4 Homo economicus es supuesto habitual
				\4[] Resultados suficientemente precisos
				\4[] Alternativas no son fáciles de justificar
				\4 Behavioral economics
				\4[] Área cada vez más relevante
				\4[] Pero adolece de problemas
				\4[] $\to$ No es capaz de plantear modelo general
				\4[] $\to$ Supuestos ad-hoc dependientes de contexto
				\4[] $\to$ Problemas de replicabilidad
				\4[] $\to$ Menos tratable
				\4 Conjetura de Gary Lucas
				\4[] No es relevante que 90\% no puedan/intenten optimizar
				\4[] Relevante es 10\% que puede y tiene incentivos
				\4[] $\to$ Empujan agregado hacia optimización
		\2 Equilibrio general y parcial
			\3 Idea clave
				\4 Valorar todas interacciones o sólo algunas
				\4 Conjunto de todas las interacciones
			\3 Posturas enfrentadas
				\4 Análisis de equilibrio general
				\4 Análisis de equilibrio parcial
			\3 Valoración
				\4 Diferencia general y parcial
				\4[] Es gradiente
				\4 En la práctica fronteras a menudo difusas
				\4[] Eq. general aplica ceteris paribus implícitos
				\4[] Eq. parcial puede ser menos parsimonioso que gral.
	\1 \marcar{Innovaciones metodológicas recientes}
		\2 Idea clave
			\3 Frontera de la investigación
				\4 Métodos de aparición reciente
				\4[] Germen de nuevos programas de investigación
				\4[] Potenciales explicaciones de anomalías
				\4 Interdisciplinariedad
				\4[] A menudo surgen de relación con otras disciplinas
			\3 Valoración
				\4 Aún provisional
				\4 Necesarios años/décadas
		\2 Redes
			\3 Idea clave
				\4 Tomado de sociología
				\4[] Tablas I/O precursores en economía
				\4 Jackson, Bramoullé, Acemoglu, Economides
				\4 Estructura de relaciones entre agentes
				\4[] Potencial explicación de ciertos fenómenos
				\4[] $\Rightarrow$ Caracterizar estructura de relaciones
				\4 Herramientas matemáticas
				\4[] Teoría de grafos
				\4[] Econometría
				\4 Ejemplos
				\4[] Propagación de shocks industriales a través de sectores
				\4[] $\to$ Conexiones entre sectores son relevantes?
				\4[] Riesgo sistémico bancario
				\4[] $\to$ Relaciones entre bancos
				\4[] Crisis de deuda soberana
				\4[] $\to$ Efectos dominó según estructura de deuda
				\4[] Impacto desigual de políticas de empleo
				\4[] Impacto desigual políticas de educación
			\3 Valoración
				\4 Presencia aceptada en el mainstream
				\4 Aumento de publicaciones en última década
				\4 Utilización creciente en
				\4[] Macroeconomía
				\4[] Políticas públicas
		\2 Modelos basados en agentes
			\3 Idea clave
				\4 Tomado de biología y física
				\4 Elementos centrales
				\4[] Agentes heterogéneos
				\4[] Racionalidad limitada
				\4[] $\to$ Reglas heurísticas (rules of thumb)
				\4[] $\Rightarrow$ Emergencia de fenómenos a nivel agregado
				\4 Necesaria computación compleja
				\4[] Capacidad de proceso creciente necesaria
			\3 Valoración
				\4 Macroeconomía
				\4[] Por el momento, carácter experimental
				\4[] No logra mejorar capacidad predictiva
				\4 Microeconomía
				\4[] Aplicación exitosa en algunas áreas
				\4[] $\to$ Modelización de puntos focales/de Schelling
				\4[] $\to$ Economía de la salud, recursos naturales
		\2 Caos
			\3 Idea clave
				\4 Tomado de meteorología, matemática aplicada
				\4 Fluctuaciones macroeconómicas (ciclos)
				\4[] No tienden a estado estacionario
				\4[] No tienen periodicidad
				\4 Sistemas de ecuaciones diferenciales no lineales
				\4[] Pueden generar series similares
				\4[] Condiciones iniciales son muy importantes
			\3 Valoración
				\4 Programa prometedor a finales 80, principios 90
				\4 Se considera estancado hoy en día
				\4[] No es capaz de mejorar modelos DSGE
				\4 Programa ``durmiente''
		\2 Redes neuronales y deep learning
			\3 Idea clave
				\4 Reconocimiento de patrones en conjuntos de datos
				\4[] Por algoritmos que se asemejan a cerebro humano
				\4[] Permiten extrapolación a datos previamente desconocidos
				\4[] Algoritmos ``aprenden'' a reconocer a partir de:
				\4[] $\to$ Conjuntos de datos aportados
				\4[] $\to$ Errores cometidos en proceso de feedback
				\4 Potenciales aplicaciones en numerosas disciplinas
				\4 En economía:
				\4[] Mercados financieros
				\4[] Comportamiento de consumidores
				\4[] Series temporales
				\4[] Clasificación de empresas y consumidores
				\4[] Racionalidad limitada a partir de redes neuronales
				\4[] $\to$ Respuesta de agentes a partir de red neuronal
				\4[] $\to$ Propuesto por Sargent en 1986
			\3 Valoración
				\4 Aún en fase experimental
				\4[] En economía aún no hay aplicaciones claras
				\4 Finanzas y marketing campos más prometedores
		\2 Economía experimental
			\3 Idea clave
				\4 Samuelson y Robinson años 50:
				\4[] Economía no es ciencia experimental
				\4[] Experimientos no son posibles como biología, física
				\4 Smith, V. y Kahneman, D. Premios Nobel 2002
				\4[] Economía sí puede ser ciencia experimental
				\4[] Es posible diseñar experimentos en entornos controlados
				\4[1] Diseñar esquema de retribuciones y aplicar
				\4[] Exponer humanos a decisión respecto esquema retributivo
				\4[2] Contrastar hipótesis derivadas de modelos teóricos
				\4[] Con resultados del experimento
			\3 Valoración
				\4 Economía experimental ya dentro del mainstream
				\4 Aplicaciones habituales en microeconomía
				\4 Influencia en numerosas áreas
				\4[] Teoría de equilibrio general
				\4[] Teoría de precios
				\4[] Organización industrial
				\4 Críticas
				\4[] Comportamiento diferente si agentes conocen contexto
				\4[] Si saben que están siendo estudiados:
				\4[] $\to$ No se comportan como lo harían normalmente
				\4[] Difícil generalización
				\4[] Problemas de replicación de resultados
	\1 \marcar{Cualidades deseables en un modelo económico}
		\2 Idea clave
			\3 Valoración de modelos
				\4 Modelos alternativos para explicar mismo fenómeno
				\4 Necesario:
				\4[] Comparar
				\4[] Valorar alternativas
				\4 No siempre alternativas son excluyentes
				\4[] Posible plantear explicaciones alternativas
				\4[] A mismo fenómeno
			\3 Cumplimiento de características
				\4 A menudo solapamiento y trade-offs
				\4[] $\to$ Dificil cumplimiento de todas
		\2 Profundidad conceptual
			\3 Idea clave
				\4 Modelo debe explicitar predicción o explicación
				\4 Predicciones y explicaciones
				\4[] No deben ser triviales
				\4[$\then$] Deben aportar conocimiento no trivial
			\3 Justificación
				\4 Modelos deben servir para aumentar conocimiento
				\4 No basta describir, modelos deben:
				\4[] $\to$ Predecir
				\4[] $\to$ Explicar
		\2 Falsable
			\3 Idea clave
				\4 Modelo puede ser probado falso
				\4 Modelo explicita predicción
				\4[] Predicción excluye algún estado de la naturaleza
				\4 Exclusión de un estado de la naturaleza
				\4[] Implica posibilidad de probar falso
				\4[] Si resultado excluido tiene lugar
				\4[] $\to$ Modelo es falsado
			\3 Justificación
				\4 Predicciones deben poder ser contradichas
				\4 Modelo sin predicciones falsables
				\4[] Puede ser tautológico
				\4[] Puede ser imposible de valorar
				\4[] Puede ser irrelevante
		\2 Generalizable
			\3 Idea clave
				\4 Capacidad para explicar multitud de fenómenos
				\4[] Introduciendo distintos supuestos adicionales
				\4 Ejemplo:
				\4[] Modelo de oferta y demanda
				\4[] Introducción de supuestos empresas y consumidores
				\4[] $\to$ Se adapta a mercado concreto
			\3 Justificación
				\4 Formulación de nuevos modelos es costosa
				\4[] Un mismo modelo adaptable reduce costes
				\4 Generalizabilidad implica robustez principios básicos
				\4[] Elementos centrales del modelo son útiles
				\4[] explicando muchas situaciones
		\2 Tratable
			\3 Idea clave
				\4 Fácilmente analizable
				\4 Extracción de resultados y predicciones
				\4[] No requiere técnicas complejas
				\4[] No tiene un coste elevado
			\3 Justificación
				\4 Uso de técnicas complejas implica:
				\4[] Necesarios mayores conocimientos técnicos
				\4[] Más costoso formular predicciones
				\4 Mayores costes de uso
				\4[] Reducen utilidad del modelo
		\2 Parsimonia
			\3 Idea clave
				\4 Número reducido de supuestos
				\4[] Para generar resultados del modelo
			\3 Justificación
				\4 Cualquier fenómeno puede explicarse
				\4[] Aplicando un número arbitrario de supuestos
				\4 Supuestos adicionales restringen aplicabilidad
				\4[] Modelo puede explicar mejor un fenómeno
				\4[] $\to$ Pero explica cada vez menos fenómenos
		\2 Consistencia empírica
			\3 Idea clave
				\4 Las predicciones no son contrarias a hechos
				\4[] El modelo no ha sido falsado
				\4 Diferentes grados de consistencia
				\4[] Modelo puede hacer predicciones débiles (generales)
				\4[] $\to$ Fácil consistencia
				\4[] Modelo puede hacer predicciones fuertes (concretas)
				\4[] $\to$ Más difícil consistencia
			\3 Justificación
				\4 Si no es consistente
				\4[] Predice erróneamente
				\4[] $\then$ No puede utilizarse para predecir
				\4[] $\then$ Dudas sobre capacidad explicativa
		\2 Precisión predictiva
			\3 Idea clave
				\4 Predicciones cuantitativamente acertadas
				\4[] Predicción bien definida
				\4[] Error de predicción pequeño
				\4 Relación con consistencia empírica
				\4[] Consistencia empírica más fácil si:
				\4[] $\to$ Predicciones menos precisas
			\3 Justificación
				\4 Modelos se utilizan para
				\4[] Explicar fenómenos
				\4[] Predecir
				\4[] $\to$ toma de decisiones a partir de predicción
				\4 Toma de decisiones necesita
				\4[] $\to$ Predicciones precisas y acertadas
	\1[] \marcar{Conclusión}
		\2 Recapitulación
			\3 Objeto de la economía
			\3 Evolución del método
			\3 Debates metodológicos
			\3 Innovaciones metodológicas recientes
			\3 Cualidades deseables de los modelos económicos
		\2 Idea final
			\3 No hay un modelo único y general
				\4 No hay un modelo único y general
				\4[] Policy-makers deben encontrar modelo adecuado
				\4 Modelos utilizados deben adaptarse a:
				\4[] $\to$ Fenómeno a modelizar
				\4[] $\to$ Precisión y generalidad necesarias
				\4[] $\to$ Restricciones de tiempo de desarrollo y tratamiento
			\3 Interdisciplinariedad
				\4 Otra cara de la moneda del imperialismo económico
				\4[] Economía adapta métodos de otras disciplinas
				\4 Matemáticas
				\4[] Cálculo infinitesimal
				\4[] Teoría de conjuntos
				\4[] Estadística
				\4[] Métodos de optimización
				\4[] ...
				\4 Psicología
				\4[] Racionalidad limitada
				\4[] Framing
				\4[] ...
				\4 Filosofía
				\4[] Juicios de valor
				\4[] Epistemología
				\4 Ingeniería
				\4[] Economía como ciencia aplicada
				\4 Física
				\4[] Influencia newtoniana en clásicos
				\4[] Economía como descubrimiento de leyes naturales
\end{esquemal}































\conceptos

\concepto{Ley de Hume}

La Ley de Hume establece que una proposición normativa no puede derivarse exclusivamente de una proposición positiva.

\preguntas
\seccion{21 de marzo de 2017}
\begin{itemize}
    \item Defina "paradigma".
    \item ¿Cómo afectan las revoluciones científicas a la ciencia económica?
    \item Suponga que debe analizar el mercado de un bien desde el punto de vista de la economía normativa y de la economía positiva. ¿Cómo lo haría en cada caso?
    \item Comente la escuela austríaca.
    \item Ha dicho que el debate de los supuestos está superado. Paul Romer ha publicado en 2016 un artículo de gran impacto criticando precisamente el uso de supuestos poco acertados. ¿Qué puede decir al respecto?
    \item Comente el llamado imperialismo de la ciencia económica.
    \item ¿Cuál es la diferencia entre verificabilidad y falsabilidad?
    \item Comente las críticas al falsacionismo de Popper.
\end{itemize}

\seccion{28 de marzo de 2017}
\begin{itemize}
    \item No ha mencionado a Friedman en su exposición ni el papel del realismo de los supuestos, ¿podría decirme algo al respecto?
    
    \item Respecto a la crítica de Lucas, ¿podría explicarla? Usted ha dicho que es una crítica a la econometría, ¿a qué clase de modelos econométricos se refiere Lucas en su crítica?
    
    \item ¿Puede explicar la diferencia entre la economía normativa y la positiva y sus aplicaciones más importantes?
    
    \item Dentro de las corrientes del pensamiento económico, ¿podría decir cuál es la principal diferencia en el método de la escuela austriaca?
\end{itemize}

\notas

En la sección de idea clave, el resumen del método científico ocupa un espacio proporcionalmente muy superior al que hay que darle en la exposición. Pero lo he hecho así porque se trata de un apartado cuyos conceptos conozco peor que el resto, y puede ser útil leer un resumen más extenso a la hora de repasar.


\bibliografia

Mirar en Palgrave:
\begin{itemize}
	\item agent-based models
	\item analogy and metaphor
	\item assumptions controversy
	\item causality in economics and econometrics
	\item economic anthropology
	\item economic laws
	\item economic man
	\item economic sociology
	\item economics, definition of
	\item emergence
	\item endogeneity and exogeneity
	\item ethics and economics
	\item experimental econonomics
	\item explanation
	\item falsificationism
	\item history of economic thought
	\item identification
	\item individualism versus holism
	\item instrumentalism and operationalism
	\item law, economic analysis of
	\item macroeconomic forecasting
	\item mathematical economics
	\item mathematical methods in political economics
	\item mathematics and economics
	\item methodological individualism
	\item methodology of economics
	\item microfoundations
	\item models
	\item paradigms
	\item paradoxes and anomalies
	\item philosophy and economics
	\item pluralism in economics
	\item rhetoric of economics
	\item social interaction (theory)
	\item social networks, economic relevance of
	\item theory appraisal
\end{itemize}


Herbrich et al. \textit{Neural Networks in Economics: Background, Applications and New Development} (1999) -- En carpeta del tema

Blanchard, O. \textit{On the future of macroeconomic models} (2018) Oxford Review of Economic Policy -- En carpeta del tema. Número completo en: \url{https://academic.oup.com/oxrep/issue/34/1-2}

Blaug, M. \textit{Economic thought in retrospect.} Methodological postcript 

Blaug, M. \textit{The Methodology of economics} (1992)

De Vroey, M. \textit{A History of Macroeconomics: Keynes to Lucas and Beyond} (2016)

Gabaix, X.; Laibson, D. \textit{The Seven Properties of Good Models} (2008) The Foundations of Positive and Normative
Economics : A Handbook -- En carpeta del tema

Hausman, D. M. (1989) \textit{Economic Methodology in a Nutshell} Journal of Economic Perspectives. Spring 1989 -- En carpeta del tema

Hayek, F. (1989) \textit{The Pretence of Knowledge} American Economic Review. Nobel Lectures and 1989 Survey of Members. Originalmente pronunciado en 1974.

Samuels, W. J; Biddle, J. E.; Davis, J. B. \textit{A Companion to the History of Economic Thought} (2003)

Vines, D.; Wills, S. \textit{The rebuilding macroeconomic theory project: an analytical assessment} (2018) Oxford Review of Economic Policy -- En carpeta del tema. Número completo en: \url{https://academic.oup.com/oxrep/issue/34/1-2}



\end{document}
