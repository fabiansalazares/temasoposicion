\documentclass{nuevotema}

\tema{3A-9}
\titulo{Teoría de la producción.}

\begin{document}

\ideaclave

En su objetivo fundamental de comprender y predecir el funcionamiento de los mercados, la microeconomía examina el comportamiento de dos agentes fundamentales: consumidores y empresas. Aunque con numerosas e importantes matizaciones, la razón de ser básica de los consumidores consiste en maximizar su bienestar consumiendo una serie de bienes. Tales bienes no se encuentran generalmente a disposición de los consumidores de forma directa. Así, no son sino el resultado de un proceso de transformación de unos bienes en otros denominado producción. Los agentes encargados de llevarlo a cabo adoptan multitud de formas en la práctica y se mueven por diferentes motivaciones. Sin embargo, es habitual en el análisis macroeconómico de la producción englobar esos agentes bajo el concepto de ``empresa'', ``unidades productivas'' o ``productores''. 

Con el fin de llevar a cabo un análisis de los factores que determinan las decisiones de producción de las empresas y construir así un modelo de su comportamiento, la microeconomía realiza una serie de simplificaciones. La principal, representar el comportamiento como una elección entre diferentes combinaciones posibles de inputs y outputs . Es decir, entre combinaciones de bienes a utilizar para producir otras combinaciones de bienes. ¿Qué guía a las empresas a la hora de decidir una combinación u otra? El supuesto básico es que pretenden maximizar una función que representa a su vez el beneficio, o la diferencia entre ingresos por la venta de bienes producidos y costes derivados de la utilización de bienes. Partiendo de ese supuesto, es posible derivar toda una serie de resultados que describen variados aspectos del comportamiento de las empresas.

Así, tenemos que la modelización del comportamiento de las empresas habitual en la teoría microeconómica neoclásica se apoya en dos conceptos: la maximización de beneficios a través de la elección de un plan de producción determinado, y la caracterización de un conjunto de planes de producción entre los cuales la empresa pueda elegir. El presente tema se centra en examinar el segundo elemento. Es decir, la caracterización de las posibles decisiones de producción al alcance de las empresas. Si bien esta caracterización no permite por sí sola definir el comportamiento de la empresa, resulta absolutamente determinante y de ahí su enorme importancia para responder a la pregunta: ¿qué pueden hacer las empresas?

La exposición se divide en dos bloques. En el primero, se presenta la teoría microeconómica de la producción de forma general, sin atender a especificaciones concretas de conjuntos o funciones de producción. En el segundo, se introducen las funciones de producción más relevantes en la literatura, dando cuenta de su idea central, su formulación y sus aplicaciones.

La caracterización general de las teoría microeconómica de la producción comienza con las ideas clave. En primer lugar, la empresa se concibe como una ``caja negra'' que toma inputs y produce outputs, sin atender a cómo se realiza esa transformación, ni a qué otros procesos se llevan a cabo en el seno de la empresa. Se trata simplemente de caracterizar qué posibles combinaciones de inputs y outputs pueden elegirse por la empresa. La formulación de esta idea se lleva a cabo mediante la definición de conjuntos de vectores sobre el espacio $\mathbb{R}^{n+m}$ denominados planes de producción, siendo $n$ el número total de inputs y $m$ el de outputs tenidos en consideración. Si la idea de conjunto de producción recoge todas las combinaciones posibles de planes de producción, son necesarias y útiles también otras definiciones adicionales. Así, se define el conjunto de inputs necesarios como el conjunto que recoge todas las posibles combinaciones de inputs que permiten un nivel dado de output. A continuación se plantea la función de producción como una relación entre cantidades de input y máximas cantidades de output producibles. Así, la función de producción no es sino la frontera del conjunto de producción y captura todos aquellos planes de producción que utilizan con la mayor eficiencia posible los inputs disponibles.

Una vez presentados estos conceptos, se exponen una serie de propiedades que pueden cumplir los conjuntos de producción. Aunque no se pretende exponer de forma exhaustiva todas las propiedades que conjuntos de producción pueden posiblemente cumplir, las presentadas permiten caracterizar las tecnologías productivas más habituales en la modelización microeconómica. Una vez presentadas propiedades relativamente triviales como la propiedad del no-vacío, del conjunto cerrado, de la no producción gratuita, la inacción, libre disposición o irreversibilidad, se plantean las propiedades que verdaderamente determinan las funciones de producción más habituales. Estas propiedades se refieren al comportamiento del conjunto de producción ante aumentos en la escala de las operaciones (es decir, a aumentos proporcionales de todos los inputs y outputs) y caracterizan diferentes rasgos que los conjuntos de producción pueden mostrar, tales como rendimientos crecientes, decrecientes o constantes a escala. Asimismo, las propiedades de convexidad, aditividad y del cono convexo se solapan parcialmente con estas propiedades anteriores, y aportan información de extrema importancia a la hora de construir modelos matemáticamente tratables y económicamente significativos.

De las propiedades y las definiciones anteriores se desprenden una serie de implicaciones sobre las funciones de producción. De esta forma se examinan conceptos como los rendimientos a escala, la elasticidad de escala, la sustituibilidad de los factores fijadas las cantidades de output, la elasticidad de sustitución y el problema de agregación. El concepto de escala permite resumir en una sóla variable la dimensión de la producción de la empresa. Además, permite la interpretación de los planes de producción como la suma de varias empresas. El concepto de rendimientos a escala crecientes, constantes o decrecientes captura lo que sucede con la cantidad de output producido ante variaciones en la escala de la producción. La sustituibilidad no es sino el concepto que expresa en qué medida es posible sustituir un input por otro manteniendo fija la cantidad de output y del resto de inputs en consideración. La elasticidad de sustitución resume tal posibilidad de sustitución de inputs en una variable adimensional.

Los conceptos generales presentados hasta ahora no son útiles sino en la medida en que tengan aplicaciones concretas. Así, es importante ilustrar la teoría de la producción con ejemplos concretos de aplicación en la modelización teórica. El último bloque de la exposición se consagra a este objetivo, presentando las tres formas funcionales más relevantes en la literatura: la función CES o de elasticidad de sustitución constante, la función Cobb-Douglas y la función de Leontief. Las dos últimas funciones no son sino casos particulares de la primera función para valores extremos de un parámetro que captura la elasticidad de substitución de inputs. La función CES se presenta en su forma más habitual: la forma homotética. La función de Cobb-Douglas, quizás la forma funcional más común en toda la literatura micro y macroeconómica, se presenta en forma general y describiendo sus parámetros fundamentales. Aunque todas la funciones con forma Cobb-Douglas se caracterizan por mostrar eleasticidades de sustitución constantes e iguales a 1, varían en los rendimientos a escala que muestran y en la contribución a la producción total de los diferentes factores de producción. La función de producción de Leontieff se caracteriza por una elasticidad de sustitución infinitamente negativa. Así, en este tipo de funciones los inputs sólo contribuyen a aumentar la cantidad de output en la medida en que se añadan en una proporción determinada y constante. Las funciones de Leontief son la pieza fundamental de los modelos input-output, y permite modelar contextos en los que resulta razonable asumir una rigidez importante de los procesos productivos.

Por último, es necesario tener presente que la teoría de la producción no es sino un componente en la modelización del comportamiento de las empresas. Permite caracterizar las decisiones posibles, pero no señala nada en cuanto a cuál de esas decisiones de producción debe efectivamente tomar una empresa. En ocasiones, se diferencia entre eficiencia técnica y eficiencia económica para denominar a los dos componentes básicos de la modelización microecónomica del comportamiento empresarial. El presente tema caracteriza la eficiencia productiva, pero para completar la modelización del comportamiento empresarial es necesario introducir el concepto de función de beneficio. El beneficio no es sino la diferencia entre el ingreso y el coste, conceptos que implican necesariamente atender a los precios relativos de inputs y outputs en un contexto de intercambios voluntarios, lo cual implica a su vez la introducción de nuevos supuestos conductuales e institucionales que quedan fuera de la presente exposición.

\seccion{Preguntas clave}
\begin{itemize}
	\item ¿Qué pueden producir las empresas?
	\item ¿Cómo modelizar lo que pueden producir?
	\item ¿Qué funciones de producción son más habituales?
	\item ¿Cuáles son sus características principales?
	\item ¿Qué aplicaciones tiene la modelización de la producción?
\end{itemize}

\esquemacorto

\begin{esquema}[enumerate]
	\1[] \marcar{Introducción} 2'
		\2 Contextualización
			\3 Empresas
			\3 Microeconomía
			\3 Decisiones
		\2 Objeto
			\3 Qué pueden producir las empresas
			\3 Cómo modelizar las decisiones posibles
			\3 Qué modelos relevantes
		\2 Estructura
			\3 Teoría microeconómica de la producción
			\3 Progreso técnico de la función de producción
			\3 Principales funciones de producción
	\1 \marcar{Teoría microeconómica de la producción} 15'
		\2 Idea clave
			\3 Entender empresa como ``caja negra''
			\3 Caracterizar decisiones posibles
		\2 Formulación
			\3 Plan de producción
			\3 Conjunto de producción
			\3 {Función de transformación}
			\3 {Frontera de posibilidades de producción}
			\3 Conjunto de inputs necesarios
			\3 Función de producción
			\3 Propiedades de conjuntos de producción
			\3 Maximización de beneficios
			\3 Minimización de costes
		\2 Implicaciones sobre función de producción
			\3 Rendimientos a escala
			\3 Elasticidad de escala
			\3 Economías y deseconomías de escala
			\3 Sustituibilidad de factores
			\3 Elasticidad de sustitución
			\3 Agregación
		\2 Aplicaciones
			\3 Oferta
			\3 Equilibrio general
			\3 Organización industrial
			\3 Crecimiento económico
			\3 Comercio internacional
	\1 \marcar{Progreso técnico de la función de producción}
		\2 Idea clave
			\3 Contexto
			\3 Objetivos
			\3 Resultados
		\2 Corto plazo
			\3 Idea clave
			\3 Formulación
			\3 Implicaciones
		\2 Largo plazo
			\3 Idea clave
			\3 Formulación
			\3 Implicaciones
		\2 Muy largo plazo: progreso tecnológico
			\3 Idea clave
			\3 Progreso incorporado y no incorporado
			\3 Progreso técnico neutral
			\3 Progreso ahorrador de trabajo
			\3 Progreso ahorrador de capital
	\1 \marcar{Funciones de producción relevantes} 11'
		\2 CES (Elasticidad de sustitución constante)
			\3 Idea clave
			\3 Formulación
			\3 Aplicaciones
		\2 Cobb-Douglas
			\3 Idea clave
			\3 Formulación
			\3 Propiedades atractivas:
			\3 Aplicaciones
		\2 Leontief
			\3 Idea clave
			\3 Formulación
			\3 Aplicaciones
	\1[] \marcar{Conclusión}
		\2 Recapitulación
			\3 Teoría microeconómica de la producción
			\3 Progreso técnico de la función de producción
			\3 Funciones relevantes
		\2 Idea final
			\3 Costes
			\3 Estimación empírica
			\3 Producción en eq. general

\end{esquema}

\esquemalargo














\begin{esquemal}
	\1[] \marcar{Introducción} 2'
		\2 Contextualización
			\3 Empresas
				\4 Múltiples enfoques de análisis
				\4 Económico, jurídico, sociológico, financiero...
			\3 Microeconomía
				\4 Interacción de agentes individuales
				\4 Empresas: agente básico
				\4 Actividad fundamental:
				\4[] Transformación de bienes $\to$ Producción
				\4 Empresa como ``caja negra''
			\3 Decisiones
				\4 Producción
				\4[$\to$] Elección de cantidad de outputs
				\4[$\to$] Dada elección de inputs
				\4 Precio
				\4[$\to$] Al que ofrecer output
		\2 Objeto
			\3 Qué pueden producir las empresas
				\4[$\to$] Qué combinaciones de inputs y outputs son posibles
			\3 Cómo modelizar las decisiones posibles
				\4[$\to$] Qué propiedades tienen los conjuntos de prod.
			\3 Qué modelos relevantes
				\4[$\to$] Qué modelos se utilizan habitualmente
				\4[$\to$] Qué caracteriza a unos y a otros
		\2 Estructura
			\3 Teoría microeconómica de la producción
				\4 En qué consiste
				\4 Cómo se formula matemáticamente
				\4 Qué conclusiones aporta
				\4 Para qué sirve
			\3 Progreso técnico de la función de producción
			\3 Principales funciones de producción
				\4 Cuáles son
				\4 Por qué se caracterizan
				\4 Para qué sirven
	\1 \marcar{Teoría microeconómica de la producción} 15'
		\2 Idea clave
			\3 Entender empresa como ``caja negra''
				\4 Transforma en inputs en outputs
				\4 Sin examinar proceso de transformación
			\3 Caracterizar decisiones posibles
				\4 Decisiones = qué inputs y qué outputs
				\4 Mediante vectores
				\4 Objetivo:
				\4[$\to$] delimitar decisiones posibles
				\4 ¿Cómo?:
				\4[] $\to$ Caracterizando conjuntos de vectores factibles
		\2 Formulación
			\3 Plan de producción
				\4 Vector $(x_1, ..., x_n)$
				\4 Negativos: inputs
				\4 Positivos: outputs
				\4 Representación gráfica
				\4[] \grafica{planesdeproduccion}
			\3 Conjunto de producción
				\4 Conjunto planes de producción posibles
				\4 Represetación gráfica
				\4[] \grafica{conjuntodeproduccion}
				\4 Ejemplo: producción de $x_n$ a partir de inputs $\vec{x}_{-n}$
				\4[] $Y = \left\lbrace \left( -x_1, ..., -x_{n-1}, x_n \right): F(\vec{x}) \leq 0 \right\rbrace $
			\3 {Función de transformación}
				\4 $F(\vec{x}) \leq 0$
				\4 Define implicítamente:
				\4[] Conjunto de producción: $F(\vec{x}) \leq 0$
				\4[] FPP: $F(\vec{x}) = 0$
				\4 Relación Marginal de Transformación con plan $\bar{y}$:
				\4[] $\text{RMT}_{lk}(\bar{y}) = \frac{\partial F(\bar{y})/ \partial y_l}{\partial F(\bar{y}) / \partial y_k}$
			\3 {Frontera de posibilidades de producción}
				\4 Conjunto de outputs posibles
				\4[] Dada cantidad fija de inputs
				\4[] P.ej.: dados inputs fijos $\bar{k}, \bar{l}$, outputs $x$ e $y$
				\4[] FPP: combinaciones posibles de $x$ e $y$
				\4[] Representable mediante $y = f(x)$
			\3 Conjunto de inputs necesarios
				\4 ¿Qué vectores de inputs permiten producir $y$?
				\4[] $V(y) = \left\lbrace \left( x_1, ..., x_n \right): f(\vec{x}) \geq y \right\rbrace$
				\4 Representación gráfica
				\4[] \grafica{inputsnecesarios}
				\4[] $\to$ Isocuantas: fronteras de $V(y)$ al variar $y$
			\3 Función de producción
				\4 Output máximo dado un vector de inputs
				\4 Frontera del conjunto de producción
				\4[$\to$] Dados input $\vec{x}$, output $y$
				\4[] $y=f(\vec{x})$
				\4[] Definida implícitamente por $F(\vec{x}) = 0 = y - f(\vec{x})$
				\4 Representación gráfica
				\4[] \grafica{funciondeproduccion}
			\3 Propiedades de conjuntos de producción
				\4[(i)] \textit{No vacío}
				
				\4[(ii)] \textit{Cerrado}
				\4[] Y incluye su frontera
				
				\4[(iii)] \textit{Sin producción gratuita}
				\4[] $\nexists \vec{y} \in Y : y_i \geq 0, y_j \in \vec{y} > 0$ \, $\forall \, i $
				\4 Representación gráfica
				\4[] \grafica{producciongratuita}
				\4[(iv)] \textit{Posibilidad de inacción}
				\4[] Origen de coordenadas $\in Y$
				\4[] Precluye costes hundidos\footnote{En este contexto, costes hundidos podrían ser resultado de contratos irrevocables de compra de inputs.}
				\4[] \grafica{inaccion}
				
				\4[(v)] \textit{Libre disposición}
				\4[] Si $y \in Y$, $y' \leq y \then y' \in Y$
				\4[] Dados planes $y$, $y'$ que producen mismo output
				\4[] Si $y \leq y'$, $y' \in Y \then y \in Y$
				\4[] Posible producir mismo output con más input
				
				\4[] \grafica{libredisposicion}
				
				\4[(vi)] \textit{Irreversibilidad}
				\4[] Si $\vec{y} \neq 0 \in Y \then -\vec{y} \notin Y$
				\4[] No se puede producir input a partir de output
				\4[] \grafica{irreversibilidad}
				
				\4[(vii)] \textit{Rendimientos no crecientes}
				\4[] Para todo $\alpha \in [0,1]$:
				\4[] $y \in Y \Rightarrow \alpha y \in Y$
				\4[] \grafica{nocrecientes}
				
				\4[(viii)] \textit{Rendimientos no decrecientes}
				\4[] Para todo $\alpha \geq 1$
				\4[] $y \in Y \Rightarrow \alpha y \in Y$
				\4[] \grafica{nodecrecientes}
				
				\4[(ix)] \textit{Rendimientos constantes}
				\4[] Para todo $\alpha \geq 0$
				\4[] $y \in Y \then \alpha y \in Y$
				\4[] \grafica{constantes}
		
				\4[(x)] \textit{Aditividad o libre entrada}
				\4[] Para todos $y, y' \in Y$, $y+y' \in Y$
				\4[] Entrada de empresas siempre es posible
				\4[] Mismo output producible por varias empresas
				\4[] \grafica{libreentrada}
				
				\4[(xi)] \textit{Convexidad}
				\4[] $Y \text{es convexo} \iff \alpha y + (1-\alpha) y' \in Y $
				\4[] $\forall y, y' \in Y, \alpha \in [0,1]$
				\4[] Convexidad $\then$ rdtos. no crecientes
				\4[] Interpretable: extremos no son más productivos
				\4[] \grafica{convexidad}
				
				\4[xii] \textit{Cono convexo}
				\4[] Aditividad y convexidad $\Rightarrow$ cono convexo
				\4[] i.e: aditividad y convexidad
				\4[] $\iff$ $\forall y, y' \in Y, \alpha, \beta > 0: \alpha y + \beta y' \in Y$
				\4[] \grafica{conoconvexo}
				
				\4[xiii] \textit{Homoteticidad}
			\3 Maximización de beneficios
				\4[] $\underset{\vec{x} \in X}{\max} \quad \vec{p}\cdot \vec{x}$
				\4[] $\text{s.a.:} \quad F(\vec{x}) \leq 0$
				\4[] $\underset{y, \vec{x}}{\max} \quad p y - \vec{w} \cdot \vec{x}$
				\4[] $\text{s.a:} \quad y \leq f(\vec{x})$
				\4[] $\then$ $\text{CPO:} \quad $ \fbox{$\frac{\text{PMg}_i}{\text{PMg}_j} = \frac{w_i}{w_j}$} \quad $\forall \, i,j$
				\4 Implicaciones
				\4[] Demanda de $x_i$:
				\4[] $\to$ $x_i = x_i(\vec{p})$
				\4[] Oferta de $y$:
				\4[] $\to$ $y = y(\vec{p})$
				\4[] Beneficios:
				\4[] $\to$ $\pi(\vec{p})$
			\3 Minimización de costes
				\4[] $\underset{\vec{x}}{\min} \quad \vec{w} \cdot \vec{x}$
				\4[] $\text{s.a:} \quad f(\vec{x}) \geq y$
				\4 Implicaciones
				\4[] Dda. condicionada de $x_i$:
				\4[] $\to$ $x_i^c = x_i^c(\vec{p},y)$
				\4[] Función de costes:
				\4[] $\to$ $c(\vec{w}, y)$
		\2 Implicaciones sobre función de producción
			\3 Rendimientos a escala
				\4 Dados:
				\4[] Función $f(\vec{x})$
				\4[] Escalar $\lambda > 1$ aplicado a $\vec{x}$
				\4 Rendimientos crecientes a escala
				\4[] Si $f(\lambda \vec{x}) > \lambda f(\vec{x})$
				\4[] Si h.d.g. $> 1$
				\4 Rendimientos decrecientes a escala
				\4[] Si $f(\lambda \vec{x}) < \lambda f(\vec{x})$
				\4[] Si h.d.g. $< 1$
				\4 Rendimientos constantes a escala
				\4[] Si $f(\lambda \vec{x}) = \lambda f(\vec{x})$
				\4[] Si h.d.g. $= 1$
				\4 Historia del concepto de rdtos. decrecientes
				\4[] Ricardo ya plantea
				\4[] Calidad de la tierra como input fijo no aumentable
				\4[] $\to$ Aumento proporcional de K y L
				\4[] $\then$ Aumento menos que proporcional de output
				\4[] Factor de producción ``oculto'' no acumulable
				\4[] Dada cualquier función $f(\vec{x},m)$:
				\4[] $\to$ cóncava en $(\vec{x}, m)$
				\4[] $\to$ rendimientos constantes a escala en $(\vec{x},m)$
				\4[] Fijando el valor de $m$
				\4[] $\then$ Rendimientos decrecientes a escala
				\4[] Ejemplo práctico:
				\4[] $\to$ Capacidad organizativa como factor fijo
			\3 Elasticidad de escala
				\4 $\varDelta \%$ de output ante $\varDelta \%$ de inputs
				\4 \fbox{$\tau = \frac{df(t\vec{x})}{dt} \cdot \frac{t}{f(t \vec{x})} = \frac{\frac{df(tx)}{f(tx)}}{ \frac{dt}{t}} = \frac{d \, \ln f(t \vec{x})}{d \, \ln t}$}
			\3 Economías y deseconomías de escala
				\4 Relación entre costes y escala de producción.
				\4[] Economía de escala $\iff c(\lambda y) < \lambda c(y)$
				\4[] Deseconomía de escala $\iff c(\lambda y) > \lambda c(y)$
				\4 \underline{Economías de escala y rdtos. a escala}\footnote{(Bell, 1988).}
				\4[] $R \uparrow E$ $\then$ Economías de escala
				\4[] F. homotética $\then \left( R \uparrow E \iff \text{Economías de escala} \right) $ \footnote{Luego si no homotética: economía de escala $\nRightarrow$ Rdtos. crecientes a escala.}
				\4[] Demostración gráfica:
				\4[] \grafica{eenoimplicarce}
				\4 \underline{Deseconomías de escala y rdtos. a escala}
				\4[] Deseconomías de escala $\then R \downarrow E$
				\4[] F. no homotética: $ R \downarrow E \nRightarrow \text{Deseconomías de escala}$
			\3 Sustituibilidad de factores
				\4 ¿Cuánto pueden variar 2 factores si el resto son fijos?
				\4 RMT -- Relación Marginal de Transformación\footnote{Siguiendo MWG. En otras fuentes, RMT se refiere la relación de sustitución entre dos outputs, o pendiente de la FPP. V. pág. 129 de MWG.}
				\4[$\to$] Relación de intercambio entre 2 factores
				\4[] manteniendo fijo el resto.
				\4[] Diferencial total de $F(\vec{x})=0$
				\4[] $\frac{\partial F(\bar{x_1}) }{\partial x_1} dx_1 + ... + \frac{\partial F(\bar{x_n})}{\partial x_n} dx_n = 0$
				\4[] $\Rightarrow$ \fbox{$\left| \text{RMT}_{12} \right| \equiv  \frac{\partial F(\vec{x}) / \partial x_1}{\partial F(\vec{x})/ \partial x_2 } = - \frac{d \, x_2}{d \, x_1}$}
				\4[] Habitual entender RMT como:
				\4[] $\to$ R. intercambio entre outputs dado inputs fijos
				\4 RMST - Relación Marginal de Sustitución Técnica
				\4[] $\to$ Caso particular de RMT
				\4[] $\to$ RMT entre dos inputs dado output fijo
				\4[] \fbox{$ \text{RMST}_{lk} \equiv \left| - \frac{\partial f(l,k) / \partial l}{ \partial f(l,k) / \partial k } \right| = \frac{d \, k}{d \, l} $}
				\4[] Pdte. de frontera de $V(y)$
				\4[] \grafica{rmst}
			\3 Elasticidad de sustitución
				\4 Propiedad muy importante en estática comparativa
				\4[] $\to$ ¿Cómo responde demanda de inputs ante cambios en precios?
				\4 $\varDelta \%$ de la proporción entre dos inputs
				\4[] dado $\varDelta \%$ de la RMST o de precios relativos
				\4[] $\to$ Medida de la curvatura de las isocuantas
				\4[] $\then$ \fbox{$\sigma = \frac{d \ln (x_2/x_1)}{d \ln p_1 / p_2} = \frac{d \ln (x_2/x_1)}{d \ln \text{RMST}_{12}} > 0$}\footnote{ La condición de primer orden de un programa de optimización de los beneficios tal que: $\left\lbrace \max \quad  \vec{p} \vec{x} \quad s.a. \quad F(\vec{x}) = 0 \right\rbrace$
				corresponde a $\text{RMST}_{ji} \equiv \frac{d \, j}{d \, i} = \frac{\partial F / \partial y_i}{\partial F / \partial y_j} = \frac{p_i}{p_j}$ }
				\4 Propiedad muy importante en estática comparativa
				\4[] $\to$ ¿Cómo $\varDelta$ demanda de inputs ante $\varDelta$ en precios?
			\3 Agregación
				\4 Dadas $n$  empresas
				\4[] ¿Output total depende sólo de total de inputs?
				\4[] ¿Es necesario conocer uso individual de factores?
				\4[]$\to$ Depende de función de producción asumida
				\4 Generalmente, se asume agregabilidad
				\4[$\to$] Resultados razonables a nivel macro


		\2 Aplicaciones
			\3 Oferta
				\4 Agregada, a nivel macroeconómico
			\3 Equilibrio general
				\4 Interacción todos los agentes y todos mercados
			\3 Organización industrial
				\4 Poder de mercado
			\3 Crecimiento económico
				\4 Crecimiento exógeno
				\4 Crecimiento exógeno
			\3 Comercio internacional
				\4 Análisis de la ventaja comparativa
				\4 Análisis entrada de nuevos países
	\1 \marcar{Progreso técnico de la función de producción}
		\2 Idea clave
			\3 Contexto
				\4 Análisis anterior
				\4[] Conjuntos de inputs dados ``fijos''
				\4 Posibles cambios en:
				\4[] Inputs
				\4[] $\to$ Todos los inputs
				\4[] $\to$ Subconjuntos de inputs
				\4[] Conjuntos de producción
				\4[] $\to$ Funciones de producción
			\3 Objetivos
				\4 Caracterizar efectos de cambios en inputs
				\4[] Aplicando restricciones a inputs que pueden cambiar
				\4[] $\to$ Interpretar como cambios de corto plazo
				\4 Caracterizar formas de progreso técnico
				\4[] En términos de cambios en funciones de utilidad
			\3 Resultados
				\4 Diferentes representaciones para horizontes temporales
				\4 Imposición de restricciones a conjuntos de producción
				\4 Efectos sobre optimización de beneficios
				\4 Taxonomía de posibles cambios en f. de prod.
		\2 Corto plazo
			\3 Idea clave
				\4 No todos los inputs son fácilmente sustituibles
				\4 Algunos inputs tienen ofertas inelásticas
				\4 Cambios en planes de producción requieren tiempo
				\4[$\then$] Sólo algunas cantidades de inputs pueden variar
			\3 Formulación
				\4 Vector de inputs $\vec{x}$
				\4 Factores fijos y variables dentro de $\vec{x}$
				\4[] $\vec{x}_v$: posible variar cuantía
				\4[] $\vec{x}_f$: cantidades fijas de input
				\4 Algunos inputs no pueden aumentar/caer
				\4[] No tienen sustitutivos
				\4[] Requieren producción
			\3 Implicaciones
				\4 Ley de las proporciones variables
				\4[] Regularidad empírica postulada
				\4[] Inicialmente, PMg de factor variable positiva
				\4[] A partir de cierto punto
				\4[] $\to$ PMg de factor variable es negativa
				\4[] En términos de isocuantas
				\4[] $\to$ A partir de cierto punto, crecientes
				\4 Principio de LeChatelier Samuelson
				\4[] Elasticidad compensada de demanda de factores
				\4[] $\to$ Mayor en largo plazo que en corto
				\4[] Ejemplo
				\4[] $\to$ Caen salarios
				\4[] $\then$ En corto plazo, imposible sustituir K por L
				\4[] $\then$ L aumenta sólo por efecto demanda
				\4[] $\then$ En largo plazo, dda. de L aumenta por sust. con K
		\2 Largo plazo
			\3 Idea clave
				\4 Modelo general
				\4 Posible decidir la cantidad de todos los factores
				\4 Función de producción/conjunto de prod. inmutable
			\3 Formulación
				\4 Básicamente la de la primera parte del tema
			\3 Implicaciones
				\4 Menores o iguales costes que c/p
				\4[] $C_{LP}(y) \leq C_{CP}(y)$
				\4[] Porque es posible acceder a más conjuntos de prod
				\4[] $\to$ Como mínimo igual de eficiente
		\2 Muy largo plazo: progreso tecnológico\footnote{Ver Jones (1976).}
			\3 Idea clave
				\4 No sólo inputs pueden cambiar
				\4 F. de prod. también cambia con el tiempo
				\4 Posible representar cambio
				\4[] Taxonomía de efectos de cambios
				\4 Diferentes formas de cambio en f.de prods.
				\4[] $\to$ ¿Mismo efecto sobre prod. marginal?
				\4[] $\to$ ¿Efecto sobre inputs óptimos?
				\4[] $\to$ ¿Efecto sobre inputs ya aplicados
			\3 Progreso incorporado y no incorporado
				\4 En contexto dinámico
				\4 Progreso no incorporado
				\4[] Aumenta productividad de todo el capital
				\4[] $\to$ También el ya existente
				\4[] Interpretable como
				\4[] $\to$ Mejor utilización de ff.pp. ya existente
				\4 Progreso incorporado
				\4[] Aumenta productividad sólo de la nueva inversión
			\3 Progreso técnico neutral
				\4 Hicks
				\4[] A igual relación capital-trabajo
				\4[] $\to$ Proporción constante de PMg respectivas
				\4[] $\then$ Progreso no cambia PMg relativas
				\4[] Participación relativa en renta constante
				\4[] $\to$ $\frac{rk}{wL}$ constante
				\4 Harrod
				\4[] Progreso técnico que mantiene constante
				\4[] $\to$ La relación capital-output óptima
				\4[] $\to$ La tasa de beneficio/productividad de K
				\4[] Distribución de ingreso entre K y L
				\4[] $\to$ Se mantiene constante si $r=\text{PMgK}$
			\3 Progreso ahorrador de trabajo
				\4 Hicks
				\4[] Progreso aumenta prod. relativa de capital
				\4[] $\to$ Menos trabajo para misma producción
				\4[] $\then$ Ahorro de trabajo
				\4[] $\frac{F_K(t)}{F_L(t)} > \frac{F_K(0)}{F_L(0)}$
				\4[] Participación relativa en renta de K aumenta
				\4[] $\to$ $\frac{rK}{wL}$ aumenta con progreso
				\4 Harrod
				\4[] Progreso técnico que aumenta relación K-Y
			\3 Progreso ahorrador de capital
				\4 Hicks
				\4[] Progreso aumento prod. relativa del trabajo
				\4[] $\to$ Menos capital para misma producción
				\4[] $\then$ Ahorro de capital
				\4[] $\frac{F_K(t)}{F_L(t)} < \frac{F_K(0)}{F_L(0)}$
				\4[] Participación relativa en renta de L aumenta
				\4[] $\to$ $\frac{rK}{wL}$ cae con progreso
				\4 Harrod
				\4[] Progreso técnico que reduce proporción K-Y
	\1 \marcar{Funciones de producción relevantes} 11'
		\2 CES (Elasticidad de sustitución constante)
			\3 Idea clave
				\4 Elasticidad de sustitución es igual a parámetro
				\4[$\to$] $\varDelta \%$ de inputs es constante
			\3 Formulación
				\4 Forma general
				\4[] \fbox{$F(\vec{x}) = A \cdot \left( \sum_{i=1}^n x_i^\rho \right)^{1/\rho} = A \cdot \left( \sum_{i=1}^n x_i^{\frac{\epsilon-1}{\epsilon}} \right)^{\frac{\epsilon}{\epsilon-1}}$}
				\4 Forma con dos factores
				\4[] \fbox{$F(K,L) = A \left[ \delta K^{\rho} + (1-\delta) L^{\rho} \right]^{1/\rho}$}
				\4[] $A>0 , \quad 0 < \delta < 1 , \quad -1 < \rho \neq 0$
				\4 Elasticidad de sustitución
				\4[] $\epsilon = \frac{d \, \ln \left( x_i / x_j \right)}{d \, \ln \left( p_j/p_i\right)} =\frac{1}{1-\rho}$
				\4 Función con sustitutos perfectos
				\4[] $\rho = 1$ $\then$ $\epsilon \to \infty$
				\4[] $\then$ Elasticidad de sustitución es máxima
				\4 Función Cobb-Douglas
				\4[] $\rho = 0$ $\then$ $\epsilon \to 1$
				\4 Función de Leontieff
				\4[] $\rho \to -\infty$ $\then$ $\epsilon \to 0$
				\4[] $\then$ Elasticidad de sustitución es mínima
				\4 \textit{Transformaciones homotéticas}
				\4[] Transformaciones monótonas de la f. CES ordinaria
				\4[] $\to$ También son CES
				\4 \textit{Formas no homotéticas}
				\4 Existe una familia no homotética de funciones CES
			\3 Aplicaciones
				\4 Análisis empírico: función más habitual
				\4[$\to$] fácil parametrización $\then$ análisis econométrico
				\4 Competencia monopolística: Dixit-Stiglitz
				\4 Funciones de bienestar social: Samuelson
		\2 Cobb-Douglas
			\3 Idea clave
				\4 Contexto
				\4[] Propuesta originalmente por Wicksell
				\4[] Charles Cobb y Paul Douglas en 1928
				\4[] $\to$ ``A Theory of Production''
				\4[] $\then$ Muestran utilidad en análisis econométrico
				\4[] Observación empírica
				\4[] $\to$ Relación estable entre L y K
				\4[] Problema general
				\4[] $\to$ Cómo contribuyen diferentes factores a output
				\4 Objetivos
				\4[] Medir empíricamente contribución de L y K
				\4[] Estimación general de contribución de factores
				\4 Resultados
				\4[] Omnipresente literatura
				\4[] Muy tratable matemáticamente
				\4[] Función CES ordinaria con $\rho \to 0$
			\3 Formulación
				\4 Forma original
				\4[] $P = b L^k C^{1-k}$
				\4[] $\to$ $P$: producción
				\4[] $\to$ $b$: constante
				\4[] $\to$ $k$: relación entre inputs y output
				\4[] Forma logarítmica
				\4[] $\to$ $\ln P = \ln b + k \cdot \ln L + (1-k) \cdot \ln C$
				\4[] $\then$ Fácil estimación econométrica
				\4 \textit{General}
				\4[] \fbox{$F(\vec{x}) = A \prod_i x_i^{\alpha_i}$}
				\4[] $A$: parámetro de eficiencia
				\4[] $\alpha_i$: elasticidad de la producción con respecto a $x_i$
				\4[] $\sum_i$: elasticidad de escala de la producción y º de homogeneidad\footnote{La equivalencia entre la elasticidad de escala y el grado de homogeneidad se cumple también para funciones homogéneas no Cobb-Douglas.}
				\4 $\textit{Capital y trabajo} $
				\4[] $ F(K,L) = A K^\alpha L^{1-\alpha} $
				\4[] \grafica{cobbdouglas} % dibujar isocuantas, y sendas de expansión
			\3 Propiedades atractivas:
				\4[(i)] \underline{Elasticidad de sustitución constante}
				\4[] Igual a 1
				\4[(ii)] \underline{Elasticidad de escala}
				\4[] $\sum_{i=1} \alpha_i = \text{elasticidad de escala}$
				\4[] $\epsilon_i = \frac{d \, \ln f(\vec{x})}{d \, \ln x_i} = \frac{\text{PMg}_i}{\text{PMe}_i} = \alpha_i$
				\4[] Ej.: $f(k,l) = k^\alpha l^{1-\alpha} \Rightarrow \epsilon_k = \alpha, \epsilon_l = 1-\alpha$
				\4[(iii)] \textit{Pseudoconcavidad estricta}
				\4 $0 < \alpha_i < 1 \; \forall \, i \then$ Pseudoconcavidad estricta $\then$
				\4[] $\exists$ máx. único de f. de beneficio\footnote{Realmente, en una función Cobb-Douglas simple, si $\sum \alpha_i < 1$ la función será estrictamente cóncava. La pseudconcavidad hace referencia a la propiedad de comportarse en un intervalo como una función cóncava o cumplir algunas propiedades de éstas aunque no lo sea realmente. Respecto a la maximización de beneficios y el óptimo único, la utilización de pseudconcavidad es más general porque pueden existir variantes de Cobb-Douglas que presenten no-concavidades en algunos intervalos.}
				\4[(iv)] \textit{Grado de homogeneidad}
				\4[] $\sum_i \alpha_i =$ elasticidad de escala $=$ grado de homogen.
				\4[(v)] \textit{Agregación}\footnote{V. \textit{aggregation (production)} en Palgrave.}
				\4[] Si fciones. prod. son C-D + sendas de exp. paralelas:
				\4[] $\Rightarrow$ Agregables en f. prod. agregada también C-D.
				\4[(vi)] Neutralidad de Hicks y de Harrod
				\4[] Si la función tiene forma C-D, progreso técnico
				\4[] $\to$ Interpretable como neutral de Hicks
				\4[] $\to$ Interpretable como neutral de Harrod
				\4[] $F(K, L) = K^\alpha (A_t L)^{1-\alpha}= A_t^{1-\alpha} K^\alpha L^{1-\alpha} = \hat{A} K^\alpha L^{1-\alpha}$
				\4[(v)] Progreso Hicks-neutral implica Harrod-Neutral
				\4[(vi)] Cumplimiento de Condiciones de Inada implica C-B
				\4[] $\lim_{t \to \infty} \pdv{F}{x_i} = 0$
				\4[] $\lim_{t \to 0} \pdv{F}{x_i} \to \infty$
				\4[] $\then$ F es Cobb-Douglass
			\3 Aplicaciones
				\4 Omnipresentes
				\4 Macroeconomía
				\4 Crecimiento económico
				\4[] $\to$ Condiciones de Inada implican Cobb-Douglas\footnote{Más concretamente, el cumplimiento de las condiciones de Inada implica que la función de producción habrá de tener una elasticidad de sustitución que se aproxime asintóticamente a 1, lo cual implica forma Cobb-Douglas en el límite asintótico.}
				\4 Economía laboral
				\4 Muy habitual en economía aplicada.
		\2 Leontief
			\3 Idea clave
				\4 Caso particular de CES ordinaria: $\rho \to -\infty$
				\4 Los inputs se utilizan en proporciones fijas
				\4[$\to$] Cantidades que excedan de proporción fija
				\4[$\to$] No contribuyen a aumentar output.
			\3 Formulación
				\4 General
				\4[] $F(\vec{x}) = \min \left\lbrace \alpha_1 x_1, \ldots, \alpha_n x_n \right\rbrace $
				\4 Capital y trabajo
				\4[] $F(K,L) = \min \left\lbrace AK, BL \right\rbrace$
				\4[] \hyperref[fig:leontief]{Gráfica XVII}
			\3 Aplicaciones
				\4 Harrod-Domar
				\4 Análisis input-output
	\1[] \marcar{Conclusión}
		\2 Recapitulación
			\3 Teoría microeconómica de la producción
			\3 Progreso técnico de la función de producción
			\3 Funciones relevantes
		\2 Idea final
			\3 Costes
				\4 Costes y demanda determinan oferta de output
				\4[$\iff$] Decisión de producción es resultado de maximización de beneficios
				\4 Análisis anterior: posibilidades de producción
				\4 Función de producción
				\4[] Condiciona función de costes
				\4[] $\Rightarrow$ existencia y unicidad de vector que máx. beneficio
			\3 Estimación empírica
				\4 Análisis de dualidad simplifica estimación
				\4 Necesarios datos de calidad, en gran cuantía
				\4 Nuevas tecnologías
				\4[$\to$] Acceso a grandes cantidades de datos (big data)
			\3 Producción en eq. general
				\4 Determinante fundamental de bienestar\pagebreak
				\4 Interacción con muchos aspectos de una economía
				\4 Impuestos, consumo, poder de mercado, crecimiento...
\end{esquemal}



























\graficas

\begin{axis}{4}{Planes de producción}{$a_1$}{$a_2$}{planesdeproduccion}
	\draw[-] (-4,0) -- (0,0);
	\draw[-] (0,-4) -- (0,0);
	
	\node[circle,fill=black,inner sep=0pt,minimum size=4pt] (a) at (-3,2) {};
	\node[circle,fill=black,inner sep=0pt,minimum size=4pt] (a) at (-0.5,3) {};
	\node[circle,fill=black,inner sep=0pt,minimum size=4pt] (a) at (-1,1) {};
	\node[circle,fill=black,inner sep=0pt,minimum size=4pt] (a) at (-4,-2) {};
	\node[circle,fill=black,inner sep=0pt,minimum size=4pt] (a) at (3,2) {};
	\node[circle,fill=black,inner sep=0pt,minimum size=4pt] (a) at (3.2,-1.4) {};
\end{axis}

\begin{axis}{4}{Un conjunto de producción}{$x_1$}{$x_2$}{conjuntodeproduccion}
	\draw[-] (-4,0) -- (0,0);
	\draw[-] (0,-4) -- (0,0);
	
	\draw[-] (-4,2.2) to [out=345, in=100](0,0);
	\draw[-] (0,0) to [out=280, in=110](2,-4);
	
	\draw [blue, fill=yellow, opacity=0.2] (-4,2.2) to [out=345, in=100](0,0) to [out=280, in=110](2,-4) -- (-4,-4) -- (-4,2.2);
\end{axis}

\begin{axis}{4}{Conjunto de inputs necesarios para producir una cantidad $y$ de output}{$x_1$}{$x_2$}{inputsnecesarios}
	\draw[-] (.5,4) to [out=270,in=180](4,.5);
	\node[right] at (4,.5){$V(y)$};
	
	\draw[-] (1.5,4) to [out=270,in=180](4,1.5);
	\node[right] at (4,1.5){$V(2y)$};
	
	\draw[-] (2.5,4) to [out=270, in=180](4,2.5);
	\node[right] at (4,2.5){$V(3y)$};
\end{axis}

\begin{axis}{4}{Función de producción}{$x$}{$y$}{funciondeproduccion}
	\draw[-] (-4,0) -- (0,0);
	\draw[-] (0,-4) -- (0,0);

	\draw[line width=2pt] (-4,2.2) to [out=345, in=100](0,0);
	\draw[line width=2pt] (0,0) to [out=280, in=110](2,-4);
	
	\draw[-{Latex}] (-1,2.5) -- (-1.4,1.9);
	\node[right,above] at (-1,2.7){$y = f(x)$};
	
	\draw [blue, fill=yellow, opacity=0.2] (-4,2.2) to [out=345, in=100](0,0) to [out=280, in=110](2,-4) -- (-4,-4) -- (-4,2.2);
	
\end{axis}

\begin{axis}{4}{Conjunto que no cumple la propiedad de producción gratuita}{$x_1$}{$x_2$}{producciongratuita}
	\draw[-] (-4,0) -- (0,0);
	\draw[-] (0,-4) -- (0,0);
	
	\draw[-] (-4,3.4) to [out=345, in=110](0,1.5);
	
	\draw [blue, fill=yellow, opacity=0.2] (-4,3.4) to [out=345, in=110](0,1.5)  -- (0,-4) -- (-4,-4) --  (-4,3.4);
	
\end{axis}

\begin{axis}{4}{La propiedad de inacción en un conjunto de producción}{$x_1$}{$x_2$}{inaccion}
	\draw[-] (-4,0) -- (0,0);
	\draw[-] (0,-4) -- (0,0);
	
	\draw[-] (-4,0) -- (0,0);
	\draw[-] (0,-4) -- (0,0);
		
	\draw[-] (-4,2.2) to [out=345, in=100](0,0);
	\draw[-] (0,0) to [out=280, in=110](2,-4);
		
	\draw [blue, fill=yellow, opacity=0.2] (-4,2.2) to [out=345, in=100](0,0) to [out=280, in=110](2,-4) -- (-4,-4) -- (-4,2.2);
	
	\node[circle,fill=black,inner sep=0pt,minimum size=5pt] (a) at (0,0) {};
	
	\draw[-{Latex}] (0.3,0.3) -- (1.5,1.5);
	
	\node[right] at (1.5,1.5){$(0,0) \in Y$};
	
	\node[right] at (-2,-2){$Y$};
\end{axis}

\begin{axis}{4}{La propiedad de libre disposición en un conjunto de producción}{$a_1$}{$a_2$}{libredisposicion}
	\draw[-] (-4,0) -- (0,0);
	\draw[-] (0,-4) -- (0,0);
	
	\draw[-] (-4,0) -- (0,0);
	\draw[-] (0,-4) -- (0,0);
	
	\draw[-] (-4,2.2) to [out=345, in=100](0,0);
	\draw[-] (0,0) to [out=280, in=110](2,-4);
	
	\draw [blue, fill=yellow, opacity=0.2] (-4,2.2) to [out=345, in=100](0,0) to [out=280, in=110](2,-4) -- (-4,-4) -- (-4,2.2);
	
	\node[circle,fill=black,inner sep=0pt,minimum size=5pt] (a) at (-2,1.75) {};
	
	\node[above] at (-2,1.75){y};
	
	\node[below] at (-2,0.70){y'};
	
	\node[circle,fill=black,inner sep=0pt,minimum size=5pt] (a) at (-2,0.75) {};
\end{axis}

El conjunto cumple la propiedad de libre disposición porque para cualquier plan $y$ en la frontera del conjunto, se cumple que $y' \in Y$ si $y' \leq y$, 

\begin{axis}{4}{Un conjunto de producción que no cumple la propiedad de irreversibilidad.}{$x_1$}{$x_2$}{irreversibilidad}
	\draw[-] (-4,0) -- (0,0);
	\draw[-] (0,-4) -- (0,0);
	
	\draw[-] (-4,3) -- (0,0);
	\draw[-] (0,0) -- (2,-4);
	
	\node[circle,fill=black,inner sep=0pt,minimum size=5pt] (a) at (-2,1.5) {};
	
	\node[above] at (-1.9,1.6){$y \in Y$};
	
	\draw[dashed] (0,0) -- (4,-3);
	
	\node[circle, fill=black, inner sep=0pt, minimum size=5pt] (a) at (2,-1.5) {};
	
	\node[above] at (2.1,-1.4){$-y \notin Y$};
	
	\draw [blue, fill=yellow, opacity=0.2] (-4,3) -- (0,0) -- (2,-4) -- (-4,-4) -- (-4,3);
\end{axis}

\begin{axis}{4}{Un conjunto de producción que NO cumple la propiedad de rendimientos no crecientes a escala.}{$a_1$}{$a_2$}{nocrecientes}
	\draw[-] (-4,0) -- (0,0);
	\draw[-] (0,-4) -- (0,0);
	
	\draw[-] (0,0) to [out=160, in=280](-4,3);
	
	\node[circle, fill=black, inner sep=0pt, minimum size=5pt] (a) at (-3.63,2) {};
	\node[above] at (-3.53,2){$y$};
	
	\draw[dashed] (-3.63,2) -- (0,0);
	
	\node[circle, fill=black, inner sep=0pt, minimum size=5pt] (a) at (-2,1.102) {};
	\node[above] at (-1.8,1.102){$\alpha y' \not \in Y$};
	
	\draw [blue, fill=yellow, opacity=0.2] (0,0) to [out=160, in=280](-4,3) -- (-4,-4) -- (0,-4) -- (0,0);
\end{axis}

\begin{axis}{4}{Un conjunto de producción que NO cumple la propiedad de rendimientos a escala no decrecientes.}{$a_1$}{$a_2$}{nodecrecientes}
	\draw[-] (-4,0) -- (0,0);
	\draw[-] (0,-4) -- (0,0);
	
	\draw[-] (0,0) to [out=110, in=-20](-4,3);
	
	\draw [blue, fill=yellow, opacity=0.2] (0,0) to [out=110, in=-20](-4,3) -- (-4,-4) -- (0,-4) -- (0,0);
	
	\node[circle, fill=black, inner sep=0pt, minimum size=5pt] (a) at (-2,2.18) {};
	\node[above] at (-1.8,2.2){$y \in Y$};
	
	\node[circle, fill=black, inner sep=0pt, minimum size=5pt] (a) at (-3,3.24) {};
	\node[above] at (-2.6,3.24){$\alpha y' \notin Y$};
	
	\draw[dashed] (0,0) -- (-3.704,4);
\end{axis}

El conjunto de producción representado muestra rendimientos decrecientes a escala, porque un vector $y$ multiplicado por un escalar $\alpha > 1$ se encuentra fuera del conjunto en cuestión. Es decir, no es posible incrementar la escala de cualquier conjunto de producción en una proporción mayor a 1.

\begin{axis}{4}{La propiedad de rendimientos constantes a escala de un conjunto de producción}{$a_1$}{$a_2$}{constantes}
	\draw[-] (-4,0) -- (0,0);
	\draw[-] (0,-4) -- (0,0);
	
	\draw[-] (-4,3) -- (0,0);
	\draw [blue, fill=yellow, opacity=0.2] (-4,3) -- (0,0) -- (0,-4) -- (-4,-4) -- (-4,3);
\end{axis}

\begin{axis}{4}{Un conjunto de producción que cumple con la propiedad de libre entrada.}{$x_1$}{$x_2$}{libreentrada}
	\draw[-] (-4,0) -- (0,0);
	\draw[-] (0,-4) -- (0,0);
	
	\draw[-] (0,0) -- (-0.5,0) -- (-0.5,0.5) -- (-1,0.5) -- (-1,1) -- (-1.5,1) -- (-1.5,1.5) -- (-2,1.5) -- (-2,2) -- (-2.5,2) -- (-2.5,2.5) -- (-3,2.5) -- (-3,3) -- (-3.5,3) -- (-3.5,3.5) -- (-4,3.5) -- (-4,4);
	
	\draw [blue, fill=yellow, opacity=0.2] (0,0) -- (-0.5,0) -- (-0.5,0.5) -- (-1,0.5) -- (-1,1) -- (-1.5,1) -- (-1.5,1.5) -- (-2,1.5) -- (-2,2) -- (-2.5,2) -- (-2.5,2.5) -- (-3,2.5) -- (-3,3) -- (-3.5,3) -- (-3.5,3.5) -- (-4,3.5) -- (-4,4) -- (-4,0) -- (0,0);	
	
	\node[circle, fill=black, inner sep=0pt, minimum size=5pt] (a) at (-1,1) {}; % plan y
	\node[right] at (-1,1){$y$};
	
	\node[circle, fill=black, inner sep=0pt, minimum size=5pt] (a) at (-1.5,1) {}; % plan y'
	\node[left] at (-1.5,1){$y'$};
	
	\node[circle, fill=black, inner sep=0pt, minimum size=5pt] (a) at (-2.5,2) {}; % plan y' + y
	\node[below] at (-2.5,2){$y+y'$};
\end{axis}

Dado un conjunto de producción que cumple la propiedad de libre entrada aditividad, la suma de cualesquiera dos puntos pertenecientes al conjunto también está dentro del mismo.

\begin{axis}{4}{Un conjunto de producción que cumple la propiedad de convexidad}{$x_1$}{$x_2$}{convexidad}
	\draw[-] (-4,0) -- (0,0);
	\draw[-] (0,-4) -- (0,0);
	
	\draw[-] (0,0) to [out=100, in=-10](-4,3);
	\draw [blue, fill=yellow, opacity=0.2] (0,0) to [out=100, in=-10](-4,3) -- (-4,-4) -- (0,-4) -- (0,0);
	
	
	\node[circle, fill=black, inner sep=0pt, minimum size=5pt] (a) at (-2.5,1) {}; % plan y
	\node[circle, fill=black, inner sep=0pt, minimum size=5pt] (a) at (-1,.5) {}; % plan y'
	\draw[dashed] (-2.5,1) -- (-1,.5);
	
	\node[circle, fill=black, inner sep=0pt, minimum size=5pt] (a) at (-2.5,2.61) {}; % plan y
	\node[circle, fill=black, inner sep=0pt, minimum size=5pt] (a) at (-0.8,1.57) {}; % plan y'
	\draw[dashed] (-2.5,2.61) -- (-0.8,1.57);
\end{axis}

En un conjunto convexo, todas las combinaciones convexas de cualesquiera dos puntos dentro del conjunto, pertenecen también a dicho conjunto.

\begin{axis}{4}{Un conjunto de producción que es un cono convexo.}{$a_1$}{$a_2$}{conoconvexo}
	\draw[-] (-4,0) -- (0,0);
	\draw[-] (0,-4) -- (0,0);
	
	\draw[-] (-4,3) -- (0,0) -- (2,-4);
	\draw [blue, fill=yellow, opacity=0.2] (0,0) -- (-4,3) -- (-4,-4) -- (0,-4) -- (2,-4) -- (0,0);
	
	%\node[circle, fill=black, inner sep=0pt, minimum size=5pt] (a) at (-2.5,1) {};
	
\end{axis}

Dados $\alpha, \beta > 0$, un conjunto de producción es un cono convexo si para todos los pares de planes de producción $y, y' \in Y$ se cumple que $\alpha \cdot y + \beta \cdot y' \in Y$.

\begin{axis}{4}{Relación Marginal de Sustitución Técnica.}{$x_1$}{$x_2$}{rmst}
	\draw[-] (.5,4) to [out=270,in=180](4,.5);
	\node[right] at (4,.5){$V(y)$};
	
	\draw[-] (0.52,2.52) -- (2.52,0.52);
	\draw[dotted] (2.52,0.52) -- (0.52, 0.52);
	
	\draw[-] (1,0.52) to [out=90, in=210](1.52,1.52);
	
	\node[right] at (1.2, 1){\small $\alpha$};
\end{axis}

La RMST es igual a $\tan \alpha$.

\begin{axis}{4}{Isocuantas de función de producción Cobb-Douglas.}{$x_1$}{$x_2$}{cobbdouglas}
	\draw[-] (2.5,4) to [out=300, in=150](4,2.5);
	\draw[-] (1.9,4) to [out=292, in=158](4,1.9);
	\draw[-] (1.3,4) to [out=285, in=165](4,1.3);
	\draw[-] (.6,4) to [out=277, in=173](4,.6);
	\draw[-] (.1,4) to [out=272, in=178](4,.1);

	
	\draw[-] (0,0) -- (4,4);
\end{axis}

\begin{axis}{4}{Isocuantas de función de producción de Leontief $f(x) = \min \left\lbrace ax, by \right\rbrace$ .}{$x$}{$xy$}{leontief}
	\draw[dashed] (0,0) -- (4,4);
	
	\draw[-] (0.75,4) -- (0.75,0.75) -- (4,0.75);
	
	\draw[-] (1.5,4) -- (1.5,1.5) -- (4,1.5);
	
	\draw[-] (2.25, 4) -- (2.25,2.25) -- (4,2.25);
	
	\node[right] at (4,4){$y=\frac{a}{b}x$};
\end{axis}

\conceptos

\concepto{Elasticidad de escala}

Relación entre la variación porcentual de la escala de la producción, y la variación porcentual asociada de la producción. Así, dada una función de producción $f(\vec{x})$ a la que se aplica un factor de escala $t$ tal que $f(t\vec{x})$, la elasticidad de escala será:

\begin{equation}
\epsilon_{t} = \frac{df(t\vec{x})}{dt} \cdot \frac{t}{x} = \frac{\frac{df(tx)}{f(tx)}}{ \frac{dt}{t}}
\end{equation}

En funciones Cobb-Douglas, la elasticidad de escala es la suma de los exponentes de los inputs. Si la suma es menor que 1, la función tendrá rdtos. decrecientes a escala. Si igual a 1, rdtos. constantes a escala. Si mayor que 1, rdtos. crecientes a escala.

\textit{Buscar en internet forma general de funciones cuya elasticidad de escala es constante}

\concepto{Relación entre las funciones homotéticas y homogéneas}

Las funciones homotéticas son aquellas para las cuales se cumple que $f(\vec{x}) = f(\vec{y}) \iff f(\lambda \vec{x}) = f(\lambda \vec{y})$. Es decir, si dos vectores de inputs inducen la misma cantidad de output, esos mismos dos vectores multiplicados por el mismo escalar inducirán también la misma cantidad de output. 

Las funciones homogéneas de grado $k$ son aquellas para las cuales se cumple que $f(\lambda \vec{x}) = \lambda^k f(\vec{x})$ para todo $\lambda \in \mathbb{R}$. Esto es, la producción asociada a un vector de inputs dado se multiplica por un escalar elevado a $k$ cuando el vector de inputs es multiplicado por ese mismo escalar.

La propiedad de homogeneidad implica homoteticidad, pero no al revés. Tomemos dos vectores de inputs $\vec{x}$ y $\vec{y}$, y una función $f(\cdot)$ homogénea de grado $k$. Así, tenemos que:

\begin{align}
	f(\lambda \vec{x}) = \lambda^k f(\vec{x}) \\
	f(\lambda \vec{y}) = \lambda^k f(\vec{y})
\end{align}

Para que tal función fuese homotética, habría de cumplirse que:

\begin{align}
	f(\vec{x}) = f(\vec{y}) \Rightarrow f(\lambda \vec{x}) = f(\lambda \vec{y}) \quad \forall \, \lambda \in \mathbb{R}
\end{align}

Se cumple que:

\begin{align}
	f(\lambda \vec{x}) = f(\lambda \vec{y}) \Rightarrow \lambda^k f(\vec{x}) = \lambda^k f(\vec{y})
\end{align}

Luego:

\begin{align}
	f(\lambda \vec{x}) = f(\lambda \vec{y}) \Rightarrow f(\vec{x}) = f(\vec{y})
\end{align}


\concepto{Senda de expansión}

La senda de expansión es la línea que une los vectores óptimos de factores de inputs para un precio de factores dado cuando se aumenta la escala de producción. En la función Cobb-Douglas es una línea recta desde el origen.



\preguntas

\seccion{Test 2018}
\textbf{6.} Si la tecnología de una empresa se representa mediante la función de producción $Y=f(z_i) = \min{a_i, z_i}$, $(i=1, ..., n)$, donde $a\geq 0$ es una constante, la elasticidad a escala (E) será:

\begin{itemize}
	\item[a] $E > 1$
	\item[b] $E=1$
	\item[c] $E<1$
	\item[d] $E=\infty$
\end{itemize}

\seccion{Test 2015}
\textbf{7}. La función de producción $F(z_1, z_2) = z_1^{0,3} z_2^{0,2} + z_2^{0,5}$ presenta rendimientos a escala:

\begin{enumerate}
	\item[a] Constantes.
	\item[b] Crecientes.
	\item[c] Decrecientes.
	\item[d] Pueden ser constantes, crecientes o decrecientes, dependiendo del nivel de la producción.
\end{enumerate}

\seccion{Test 2013}

\textbf{3}. ¿Qué implicaciones tendría sobre la frontera de posibilidades de producción (FPP) liberar recursos destinados a defensa y destinarlos a educación?

\begin{enumerate}
	\item[a] Se desplazaría a la derecha.
	\item[b] Se desplazaría a la izquierda.
	\item[c] Pivotaría alrededor de un punto en un eje.
	\item[d] Habría un movimiento a lo largo de la FPP.
\end{enumerate}

\textbf{16}. La función de producción de una economía es de la forma:

\begin{equation}
y_s = k_s ^\alpha n_s^{1-\alpha} \\
\alpha \in (0,1) \\
s=t, t+1
\end{equation}

donde $y$, $k$ y $n$ representan, respectivamente, la producción, el stock de capital y el factor trabajo. Se sabe que la elasticidad de la producción respecto al capital es el doble de la correspondiente elasticidad respecto al trabajo. Se dispone también de la siguiente información:

$k_t=100$, $k_{t+1}=106$, $n_t = 90$, $n_{t+1}=92,7$

Obténgase, mediente una aproximación logarítmica, la tasa neta de crecimiento de la economía entre $t$ y $t+1$.

\begin{enumerate}
	\item[a] 4\%.
	\item[b] 1\%.
	\item[c] 5\%.
	\item[d] Ninguna de las respuestas anteriores. 
\end{enumerate}

\seccion{Test 2009}

\textbf{7}. Suponga que una empresa produce el bien $y$ de acuerdo con una tecnología de tipo Leontieff: $y = \min (\beta_1 x_1, \beta_2 x_2)$, siendo $x_1$, $x_2$ los factores de producción con precios $w_1$, $w_2$, respectivamente. ¿Qué propiedad característica tiene esta tecnología de producción?

\begin{enumerate}
	\item[a] Las demandas de los factores productivos $x_1$, $x_2$ son independientes de sus precios.
	\item[b] Las isocuantas asociadas son líneas rectas decrecientes en $R^2$.
	\item[c] La función de costes asociada es cuadrática en los precios de los factores, $C(w,y) = w_1^2w_2^2y$.
	\item[d] Ninguna de las anteriores.
\end{enumerate}

\seccion{Test 2008}

\textbf{3}. Una función de producción homogénea de grado $k$,

\begin{enumerate}
	\item[a] Presenta rendimientos decrecientes de escala si $k>1$.
	\item[b] Tiene primeras derivadas parciales homogéneas de grado $k$.
	\item[c] Da lugar a unos costes medios crecientes si $k<1$.
	\item[d] No permite calcular la función de costes de la empresa si $k=1$.
\end{enumerate}

\textbf{19}. Sea la función de producción:

\begin{equation}
	y=AN^{0,3}K^{0,7}
\end{equation}

Siendo A el progreso técnico, N el trabajo y K el capital. Calcule el residuo de Solow cuando ante un aumento del capital del 4\% y del trabajo del 3\%, la producción crece al 5\%.

\begin{enumerate}
	\item[a] 2,3 \%
	\item[b] 1,3 \%
	\item[c] 0,023
	\item[d] Ninguna de las anteriores.
\end{enumerate}

\seccion{Test 2007}

\textbf{8}. Si la función de producción $x=f(y_1, y_2) = y_1^{1/2} + y_2^{1/2}$, es \textbf{FALSO} que:

\begin{enumerate}
	\item[a] La elasticidad de escala es 1/2.
	\item[b] El grado de homogeneidad de la función es 1/2.
	\item[c] El grado de homogeneidad de la función es 1.
	\item[d] La tecnología presenta rendimientos decrecientes a escala.
\end{enumerate}

\seccion{Test 2005}

\textbf{7}. Suponga una empresa que produce el bien X atendiendo a la función de producción $x = f(y_1, y_2)=y_1^{1/2} + y_2^{1/2}$. En consecuencia, la elasticidad de escala sería:

\begin{enumerate}
	\item[a] Igual a 1.
	\item[b] Igual a 1/2.
	\item[c] Igual a 2.
	\item[d] Igual a 0.
\end{enumerate}

\notas

\textbf{2018}: \textbf{6}. ANULADA

\textbf{2015}: \textbf{7}. C

\textbf{2013}: \textbf{3}. D \textbf{16}. C

\textbf{2009}: \textbf{7}. A

\textbf{2008}: \textbf{3}. C \textbf{19}. B

\textbf{2007}: \textbf{8}. C

\textbf{2005}: \textbf{7}. B


\bibliografia


Mirar en Palgrave:

\begin{itemize}
	\item aggregation (production)
	\item CES production function
	\item Cobb-Douglas functions
	\item production functions
\end{itemize}

Chiang, A; Wainwright, K. \textit{Fundamental Methods of Mathematical Economics.} Ch. 12 Optimization with Equality Constraints.



Jones, H. G. (1976) \textit{An Introduction to Modern Theories of Economic Growth} McGrawHill -- Extracto en carpeta del tema

Kreps, D. \textit{A Course in Microeconomic Theory.} Chapter Seven. The neoclassical firm.

Levin, J.; Milgrom, P. \textit{Producer Theory}. \url{https://web.stanford.edu/~jdlevin/Econ%20202/Producer%20Theory.pdf}
	
MWG. \textit{Microeconomic Theory} Ch. 5 Production
	
Varian, H. \textit{Microeconomic Analysis}. Ch. 1 Technology

Mirar Ch. 5 de Gravelle

\end{document}
