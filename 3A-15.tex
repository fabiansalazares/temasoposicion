\documentclass{nuevotema}

\tema{3A-15}
\titulo{Teoría del monopolio. Regulación y control.}

\begin{document}

\ideaclave

El objetivo básico de la microeconomía es entender y predecir el comportamiento de los agentes económicos cuando interactúan en un contexto de mercado. Los posibles mercados que analizar son tan variados como lo es el tráfico económico, pero la microeconomía se centra en una serie de casos particulares que permiten ilustrar un resultado, una tendencia o un mecanismo concreto. Una de los aspectos principales a la hora de determinar el resultado de una interacción de mercado es la medida en que los agentes tienen poder para afectar el precio y la cantidad de equilibrio. Cuando los agentes se comportan como si no pudiesen afectar unilateralmente el precio de equilibrio, es habitual designar a los agentes como precio-aceptantes y al mercado como competitivo. El monopolio es el caso polarmente opuesto y su análisis teórico y práctico es el \textbf{objeto} de esta exposición. La característica básica de todo monopolio es la presencia de un sólo vendedor y un número elevado de compradores. Del hecho de que la empresa no tenga competidores inmediatos se deduce su capacidad para modular a voluntad el precio y la cantidad de unidades vendidas dentro de los límites impuestos por una función de demanda dada. La posición de monopolio tiene una serie de consecuencias sobre el bienestar de los agentes involucrados y plantea una serie de preguntas a las que trataremos de dar respuesta: ¿qué es un monopolio? ¿cómo se modeliza en teoría microeconómica? ¿es necesario permitir o regular los monopolios?, si es necesario regularlos, ¿cómo deben regularse? La \textbf{estructura} de la exposición parte de la presentación del modelo neoclásico estándar de monopolio. De éste se derivan una serie de extensiones a fin de representar un número mayor de situaciones que se presentan en la práctica y respecto de las cuales es preciso tomar decisiones de regulación. Por último, se analiza la necesidad de regular el monopolio, así como los medios utilizables para ello.

El \marcar{modelo estándar del monopolio}, habitualmente identificado con el paradigma neoclásico, se basa en los desarrollos de Cournot (1838) y Marshall. El supuesto central del modelo es la capacidad de monopolista para decidir el precio y la cantidad de equilibrio dentro de un menú de posibilidades determinado por la función de demanda agregada. El comportamiento maximizador del monopolista resultará en un precio y un nivel de producto intercambiado que efectivamente maximizará su nivel de beneficios pero que --siempre que la demanda no sea perfectamente elástica-, inducirá una cantidad comprada por los consumidores inferior a la que podrían consumir si se igualasen precio y coste marginal. En una \textbf{formulación} estándar, el monopolista maximiza el beneficio tomando como dada la función de demanda de los consumidores, de tal manera que la condición óptimo de primer orden requiere que el ingreso marginal del monopolista sea igual al coste marginal de producción. El ingreso marginal dependerá a su vez de la cantidad ofertada, a diferencia del mercado competitivo en el que el ingreso marginal es siempre igual al precio de equilibrio. La variable de optimización del problema del monopolista puede formularse indistintamente en términos del precio o la cantidad, si se utilizan respectivamente la función de demanda o la función de demanda inversa. Dado que ingreso marginal será siempre inferior al precio, y teniendo en cuenta que el excedente social se maximiza --con estos supuestos- igualando ingreso marginal y coste marginal, tenemos que la primera gran implicación del modelo es que el equilibrio no es un óptimo de Pareto. Del modelo se derivan una serie de implicaciones. El monopolista estará aplicando un \textit{mark-up} sobre el coste marginal cuya cuantía dependerá del valor absoluto de la inversa de la elasticidad de la demanda al precio. Esta relación porcentual entre mark-up aplicado y precio se denomina índice de Lerner y constituye también una medida del poder de mercado del monopolista. Se puede demostrar que el monopolista siempre decidirá producir una cantidad situada en el tramo elástico de la función de demanda. Si se situase en el tramo inelástico, siempre podría aumentar hasta cierto punto el precio y reducir la cantidad producida de tal manera que la reducción de costes superase a la caída de ingresos. En el tramo elástico, aumentar los precios reduce los ingresos totales pero reduce también los costes, por lo que la empresa tiene margen para reducir la cantidad hasta que ingreso y coste marginal se igualen. La cantidad producida no se situará necesariamente en la escala eficiente, de tal manera que los costes totales no tienen por qué minimizarse. Por último, el bienestar social no se maximiza: la maximización de excedente del productor sin tener en cuenta el bienestar social induce una pérdida irrecuperable de eficiencia conocida como \textit{triángulo de Harberger}.

Las aplicaciones del modelo neoclásico o estándar del monopolio han sido fundamentales para la regulación y la representación de innumerables mercados reales. El modelo ha fundamentado la necesidad de intervenir mercados monopolísticos a partir de la caracterización de los efectos negativos de su existencia, así como la eliminación de barreras legales a la entrada que han sido comunes en muchos mercados. En cualquier caso, es necesario tener presente que existen extensiones, variantes y alternativas a este modelo con implicaciones no necesariamente iguales en lo que respecta al equilibrio y a las políticas a adoptar para maximizar el bienestar.

El \textbf{análisis de la entrada} de nuevos competidores en un mercado inicialmente monopolístico es la primera \marcar{extensión} del modelo estándar que se examina. Si el modelo anterior postula la existencia de una sola empresa, cabe preguntarse: ¿qué sucede cuando es posible la entrada de otras empresas que compitan con el monopolista? ¿el resultado será eficiente? La idea de la \underline{contestabilidad} propuesta por Baumol, Panzar y Willig en 1982 plantea la posibilidad de que la sola amenaza de entrada induzca una mejora de bienestar. Ante la posibilidad de que entren competidores en el mercado, un monopolista con costes medios decrecientes produciría una cantidad tal que el precio de demanda y el coste medio se igualarían para reducir los beneficios hasta anularlos y eliminar todo incentivo a entrar en el mercado para posibles competidores. Aunque esta situación supondría una mejora respecto a la situación de monopolio, la condición de primer orden del problema de maximización del excedente social seguiría sin cumplirse al no igualarse precio y coste marginal.

El análisis del \underline{monopolio natural} es una situación relativamente habitual en la práctica. Una monopolio natural es aquel en el que la producción por un sólo productor de la cantidad que maximiza el beneficio del monopolista es menos costosa que si se llevase a cabo por dos productores. En términos matemáticos, los monopolios naturales se representan mediante funciones de coste subaditivas. En el caso de un sólo bien producido, una condición suficiente pero no necesaria es la presencia de economías de escala. En presencia de economías de escala, la posible entrada de competidores no es un factor relevante porque el monopolista podrá ``expulsar'' del mercado a sus competidores. Sin embargo, la subaditividad de la función de costes es también posible en presencia de costes medios crecientes. En estos casos, la presencia de un sólo monopolista no es sostenible y el mercado terminará por estabilizarse con varios productores. En esta situación, la ineficiencia de la presencia de varios productores puede inducir al gobierno a establecer barreras de entrada al mercado y a regular los precios para que reducir los efectos negativos sobre el bienestar del poder de mercado del monopolista.

Los \textbf{monopolios de bienes duraderos} son también un caso habitual en la teoría y en la práctica. Cuando un monopolista produce un bien que no se deteriora y que los consumidores pueden adquirir en varios momentos temporales, el monopolio compite contra las manifestaciones futuras de sí mismo. En estas situaciones, el monopolista deberá decidir si vende sólo hoy o también mañana, y si decide vender sólo hoy, cómo comprometerse a ello de tal manera que los consumidores consideren creíble su decisión y no decidan no comprar hoy para comprar mañana a un precio quizás inferior. En el análisis de este tipo de monopolios se ha aplicado el concepto de la inconsistencia dinámica para representar el problema del monopolista a la hora de hacer creíble su amenaza de no vender en el futuro. La \textit{conjetura de Coase} plantea la hipótesis de que en un contexto de horizonte infinito, la empresa no puede comprometerse a no vender en periodos futuros y por ello acaba vendiendo a coste marginal, en una suerte de competencia perfecta contra sí mismo. Algunos ejemplos de estrategias llevadas a cabo por los monopolistas en la práctica son la sustitución de la venta por el \textit{leasing} o la desincentivación de los mercados de segunda mano mediante la introducción de actualizaciones (por ejemplo, en el caso del mercado de los libros de texto).

El fenómeno de la \textbf{discriminación} llevado a cabo por un monopolista consiste en la venta del producto a diferentes precios en función del número de unidades vendidas a un consumidor dado o en función del consumidor de que se trate, de tal manera que el precio total deja de ser lineal en la cantidad vendida. La posibilidad de aplicar discriminación depende de una combinación de factores institucionales y tecnológicos que se resumen en la disposición del monopolio de la suficiente información para distinguir entre unidades vendidas y consumidores, y la capacidad para restringir la reventa. La literatura distingue de forma general tres grados de discriminación. En la \underline{discriminación de primer grado}, el monopolista es capaz de distinguir perfectamente a quién  vende y cuántas unidades le vende, de tal manera que es capaz de ofertar una cantidad única a cada consumidor a un precio individualizado. En este contexto, el problema de maximización del monopolista se convierte en un problema de maximización por cada consumidor, en el que las variables de decisión son la cantidad a vender y el precio total de esa cantidad, sujeto a la restricción de que el consumidor prefiera consumir a no consumir. La condición de primer orden es tal que la cantidad a vender es aquella que iguala utilidad marginal con coste. Este resultado es robusto a la agregación respecto a varios consumidores y permite deducir que se alcanza un óptimo de Pareto en el que el monopolista extrae todo el excedente social. La \underline{discriminación de segundo grado} se caracteriza por la posibilidad de fijar precios no lineales pero sin poder discriminar entre consumidores. Ejemplos de este tipo de discriminación son la implantación de tarifas no lineales como un plan de precios consistente en una cantidad fija a la que se añade una cantidad variable que depende del número de unidades producidas. Cuando la discriminación de segundo grado es posible, la decisión óptima del monopolista consiste en ofrecer un menú de precios que induzca a los consumidores a autoseleccionarse de tal manera que demanden las cantidades que maximicen el beneficio del monopolista. Si asumimos que existen consumidores con demanda alta y baja, y preferencias regulares, el resultado habitual desde el punto del excedente social es subóptimo. El menú de precios que maximice el beneficio del monopolista inducirá una cantidad consumida óptima por los consumidores con demanda alta, pero inferior a la óptima para los consumidores con demanda baja. La \underline{discriminación de tercer grado} consiste en la posibilidad del monopolista de diferenciar entre diferentes tipos de consumidores sin poder aplicar precios no lineales en función de la cantidad consumida. A priori, este tipo de situaciones pueden modelizarse como una simple maximización de varios problemas del monopolista. Sin embargo, la posibilidad de que la discriminación no sea perfecta, la presencia de costes marginales no constantes y la presencia de externalidades hacen necesarios refinamientos de los modelos. En general, un resultado habitual en este tipo de contextos es la fijación de un precio inferior en el mercado con demanda más elástica, de acuerdo con el modelo estándar de monopolio.

Los \textbf{monopolios de múltiples bienes} constituyen un caso especialmente relevante en la práctica. Se refieren a situaciones en las que una empresa monopolista vende varios productos en diferentes mercados. Cuando la demanda y la producción de los diferentes bienes son totalmente separables, la situación es idéntica a varios monopolios independientes. Sin embargo, lo habitual es que tanto demandas como producción se encuentren relacionadas. En este tipo de situaciones se pueden distinguir varios casos, en función de si es la demanda, los costes o ambos los factores separables. El caso más relevante en la práctica es aquel en el que las ofertas se pueden asumir independientes pero las demandas están interrelacionadas. En este tipo de situaciones, si los bienes son sustitutivos, el monopolista tenderá a fijar precios más elevados para ambos bienes que si sus demandas fuesen totalmente independientes. Para el caso de bienes con demandas complementarias, el monopolista tenderá a fijar precios inferiores a los que fijaría si las demandas fuesen independientes. 

El concepto de \marcar{regulación} del monopolio hace referencia a la intervención de los poderes públicos para establecer un marco de funcionamiento a un monopolio, generalmente a través de la imposición de un esquema de precios y una serie de restricciones a la entrada en el mercado. La \textbf{justificación de la regulación} de los monopolios parte del supuesto de que los monopolios naturales pueden potencialmente ser, en determinados contextos, más eficientes que los mercados en que participa más de una empresa. Sin embargo, varios factores alejan en la práctica a los monopolios naturales del óptimo tales como la aparición de ineficiencia, reducción de la calidad o menores cantidades producidas. La posibilidad de mejorar el monopolio natural sin regulación es un resultado teórico y empírico habitual. Sin embargo, también es necesario tener presente que existen críticas a la aplicación general de la regulación que señalan su utilización espuria tendente a favorecer los intereses de los propios regulados y perpetuar sus posición de predominio. La teoría de la regulación es un campo de enorme complejidad y extensión, pero los modelos teóricos pueden a grandes rasgos clasificarse en aquellos que asumen información completa por parte del regulador y ausencia de comportamiento estratégico, y aquellos que tienen en cuenta restricciones de información y presencia de comportamiento estratégico por parte de los regulados. En el primer tipo, cuando la \textbf{información del regulador es completa y no hay comportamiento estratégico}, el regulador conoce perfectamente las funciones de demanda y coste de los regulados, lo que impide que lleven a cabo comportamientos que distorsionen el efecto de las medidas regulatorias. En mercados contestables en los que el precio de venta iguala el coste medio de producción, un regulador puede tratar de aumentar la producción hasta que el precio igual al coste marginal. Sin embargo, en este contexto deberá tener en cuenta el coste de oportunidad de los fondos públicos utilizados para subvencionar al monopolista, por lo que existen pocas razones teóricas para regular este tipo de monopolios. En todo caso, los mercados perfectamente contestables son muy infrecuentes, y muchos monopolios naturales no son sostenibles. En estos contextos la imposición de barreras de entrada se hace necesaria para evitar la práctica del \textit{cream skimming} consistente en entrar en un mercado regulado para extraer el beneficio que estaba extrayendo el monopolista en exclusiva, sin atender a posibles obligaciones de servicio público impuestas por el regulador al monopolista tales como servir a ciertos mercados no rentables. Los \underline{precios de Ramsey-Boiteux} son un resultado importante en este marco de modelización con información perfecta. Cuando existe un sólo mercado, es posible alcanzar un óptimo fijando un precio igual al coste medio. Cuando el monopolista provee a varios mercados con diferentes funciones de demanda, la fijación del precio igual al coste medio es una situación que puede mejorarse modulando los precios en función de las elasticidades de cada demanda, respecto a la fijación de precios iguales a coste medio para todos los mercados. Dado que demandas más elásticas implican mayores pérdidas irrecuperables de eficiencia ante precios más elevados, es posible mejorar la situación de precios iguales a coste medio permitiendo al monopolista fijar precios más elevados en mercados con demandas inelásticas y reduciéndolo para demandas elásticas. Los precios de Ramsey-Boiteux relacionan elasticidad y mark-up aplicado en diferentes mercados y son el resultado de un programa de maximización del excedente social al que se aplica una condición de beneficio total mínimo, o viceversa. En este tipo de contextos de información perfecta también es aplicable la discriminación de segundo grado en forma de tarifas de dos partes adaptadas a la demanda.

En la práctica, es muy difícil conocer los costes de la empresa y la demanda de los consumidores. Cuando no se cumplen estos requisitos, la \textbf{información es limitada y hay comportamiento estratégico}, resulta necesario aplicar otro tipo de prácticas regulatorias. La regulación vía \underline{coste de servicio o tasa de retorno} fija un retorno máximo sobre el capital empleado en base a una valoración del capital y unos precios aplicables. Aunque este tipo de reglas pueden mejorar la regulación que se limite a fijar un precio, la imposición de restricciones distorsiona la decisión de inversión de la empresa, induciendo mayores intensidades de capital que las que corresponden a la minimización de costes. Las reglas \textit{price-cap} fijan una senda de precios a los que el monopolista debe ofrecer el servicio pero dejan a su libre disposición el resto de las decisiones. Este método dificulta la captura regulatoria por su sencillez, y fuerza mejoras de eficiencia periódicas a medida que el nivel general de precios aumenta más que el precio permitido a la empresa. Las reglas de tipo CPI-X imponen una senda de precios ligada al IPC reducida en un factor X dependiente de las mejoras de eficiencia que se producen en el sector en cuestión. 

Más allá de las herramientas anteriores, cabe destacar el papel que el \textbf{diseño de mecanismos} ha tenido en la regulación de monopolios. Asumiendo información asimétrica y comportamiento racional por el regulado, el diseño de mecanismos trata de diseñar estructuras de incentivos que resulten en equilibrios óptimos. El mecanismo de Loeb-Magat es un de los primeros ejemplos de aplicación de este tipo de técnicas y ha servido de punto de partida para multitud de modelos posteriores. En un contexto en que firma y regulador conocen la demanda pero sólo el regulado conoce los costes de producción, el regulador puede ofrecer un subsidio que decrece con el precio del servicio. El regulado tiene libertad para fijar el precio que desee, pero la subvención le incentivará a fijar un precio que tiende al precio marginal, bajo determinados supuestos generales. 

Por último, cabe resaltar el papel de las \textbf{instituciones regulatorias} de los monopolios. Los estados disponen de diferentes instrumentos y deben tratar de aplicarlas de manera óptima. En la práctica, se impone una combinación de diferentes herramientas. La regulación vía \underline{legislación} consiste en el establecimiento directo de una regulación determinada, pero se topa con el obstáculo de la falta de cualificación técnica de legisladores, la vulnerabilidad de éstos ante grupos de interés organizados y la excesiva influencia de cuestiones de corto plazo que alteran la voluntad política. Los \underline{contratos de franquicia} consisten en vender a las empresas el derecho a monopolizar de tal manera que los ingresos derivados de la venta compensen las rentas extraídas por los monopolistas. En la práctica, los contratos de franquicia con horizontes temporales cortos son preferibles porque mejoran la credibilidad de los contratos y los hacen menos vulnerables a cambios en las condiciones económicas. Las \underline{comisiones independientes de regulación} están formadas por un número reducido de miembros independientes que votan la aprobación de medidas regulatorias diseñadas con la ayuda de plantillas de técnicos. Mayores grados de independencia del poder ejecutivo y legislativo en este tipo de comisiones tienden a relacionarse con mejores resultados. Por último, los poderes públicos pueden optar por la \underline{provisión directa} de los servicios que en otro caso proveería el monopolista, por medio del control directo de los medios de producción. Aunque era habitual en la Europa previa a las privatizaciones de los 80 y 90, en la actualidad quedan pocos ejemplos. 

A lo largo de la exposición se han presentado el modelo tradicional del monopolio y las extensiones más relevantes, y se han examinado la regulación del monopolio como herramienta para mejorar el bienestar de los agentes. Como \marcar{conclusión}, cabe reflexionar sobre la relevancia del monopolio en la actualidad. El monopolio como estructura de mercado ha sufrido una transformación en el último siglo. Si en el pasado eran habituales las estructuras monopolísticas en ámbitos como el transporte aéreo, la producción de electricidad o el petróleo, entre otros mercados, en la actualidad los monopolios tienden a concentrarse en industrias aún en fase de crecimiento como el software o la publicidad en línea, y en algunos subsectores agrícolas y farmacéuticos. En la actualidad, diferentes tendencias de signo contrario se contraponen a la hora de generar monopolios. Por un lado, el mayor tamaño de los mercados aumenta las posibilidades de supervivencia de nuevos entrantes y empuja a las grandes empresas incumbentes hacia niveles de producción que superan la escala mínima eficiente. Por otro lado, la globalización de los mercados financieros y su mayor desarrollo facilita procesos de concentración industrial. Sea cual sea la tendencia predominante a nivel global, el reciente aumento de la alarma respecto del poder de mercado que acumulan las grandes empresas hace necesario continuar investigando y monitorizando los desarrollos de industrias tendentes a la concentración monopolística.


\seccion{Preguntas clave}
\begin{itemize}
	\item ¿Qué es el monopolio?
	\item ¿Cómo se modeliza?
	\item ¿Qué extensiones y variantes existen del modelo estándar?
	\item ¿Qué monopolios hay que regular?
	\item ¿Por qué hay que regular?
	\item ¿Quién lo regula?
	\item ¿Cómo regularlos?
\end{itemize}

\esquemacorto

\begin{esquema}[enumerate]
	\1[] \marcar{Introducción}
		\2 Contextualización
			\3 Economía y microeconomía
			\3 Demanda y oferta
			\3 Monopolio
			\3 Ciencia económica y monopolio
			\3 Estructura monopolística
		\2 Objeto
			\3 Qué es un monopolio?
			\3 Cómo se modeliza?
			\3 Hace falta regular?
			\3 Quién debe regular?
			\3 Cómo regular?
		\2 Estructura
			\3 Modelo estándar
			\3 Extensiones
			\3 Regulación y control
	\1 \marcar{Modelo estándar}
		\2 Idea clave
			\3 Contexto
			\3 Objetivos
			\3 Resultados
		\2 Formulación
			\3 Demanda
			\3 Maximización respecto a cantidad
			\3 Maximización respecto a precio
			\3 Representación gráfica
		\2 Implicaciones
			\3 Pendientes de ingreso y coste marginal
			\3 Elasticidad de la demanda en equilibrio
			\3 Poder de mercado
			\3 Beneficios en el corto plazo
			\3 Beneficios en el largo plazo
			\3 Nivel de producción ineficiente en el largo plazo
			\3 Bienestar
			\3 Pérdida de eficiencia
			\3 Impuestos
			\3 Aranceles y cuotas
		\2 Aplicaciones
			\3 Trusts y cárteles
			\3 Limitaciones legales a la entrada
			\3 Intervención estatal
	\1 \marcar{Extensiones}
		\2 Entrada
			\3 Idea clave
			\3 Monopolio natural
			\3 Contestabilidad
			\3 Guerra de desgaste
			\3 Problema de la delimitación
			\3 Monopolio como institución necesaria
		\2 Monopolios de bienes duraderos
			\3 Idea clave
			\3 Conjetura de Coase
			\3 Estrategias del monopolista
		\2 Discriminación
			\3 Idea clave
			\3 Primer grado
			\3 Segundo grado
			\3 Tercer grado
		\2 Múltiples bienes
			\3 Idea clave
			\3 Demandas y producción separables
			\3 Dda. y producción no separables
			\3 Monopolio dinámico
		\2 Escuela austríaca
			\3 Von Mises
			\3 Schumpeter
	\1 \marcar{Regulación}
		\2 Idea clave
			\3 Gestión de precios y costes
			\3 Imposición de barreras de entrada
		\2 Justificación de la regulación
			\3 Eficiencia
			\3 Posibilidad de mejorar resultado
			\3 Críticas a la regulación
		\2 Información completa, sin comportamiento estratégico
			\3 Idea clave
			\3 Mercados contestables
			\3 Multiproducto y precios lineales
			\3 Precios no lineales
		\2 Información limitada y comportamiento estratégico
			\3 Idea clave
			\3 Coste del servicio o tasa de retorno
			\3 Reglas Price Cap
			\3 Competencia referencial -- Yardstick competition
			\3 Diseño de mecanismos
		\2 Instituciones regulatorias
			\3 Idea clave
			\3 Legislación
			\3 Contratos de franquicia
			\3 Comisiones independientes de regulación
			\3 Provisión pública
	\1[] \marcar{Conclusión}
		\2 Recapitulación
			\3 Modelo estándar
			\3 Extensiones
			\3 Regulación
		\2 Idea final
			\3 Evolución del monopolio
			\3 Relevancia actual del análisis del monopolio

\end{esquema}

\esquemalargo














\begin{esquemal}
	\1[] \marcar{Introducción}
		\2 Contextualización
			\3 Economía y microeconomía
				\4 Definición de Robbins
				\4[] Decisiones respecto a bienes escasos
				\4[] $\to$ Con usos alternativos
				\4[] $\to$ Para satisfacer necesidades humanas
				\4 Microeconomía
				\4[] Estudio de decisiones a nivel individual
				\4[] $\to$ Empresas
				\4[] $\to$ Consumidores
				\4[] $\to$ Gobiernos
			\3 Demanda y oferta
			\3 Monopolio

			\3 Ciencia económica y monopolio
			\3 Estructura monopolística
				\4 Una sóla empresa
				\4[] Razones variadas para existencia de 1 sóla empresa
				\4 Consecuencias sobre eficiencia y bienestar
				\4[] Generalmente negativas
				\4 ¿Necesidad de intervención estatal?
		\2 Objeto
			\3 Qué es un monopolio?
			\3 Cómo se modeliza?
			\3 Hace falta regular?
			\3 Quién debe regular?
			\3 Cómo regular?
		\2 Estructura
			\3 Modelo estándar
			\3 Extensiones
			\3 Regulación y control
	\1 \marcar{Modelo estándar}
		\2 Idea clave
			\3 Contexto
				\4 Larga trayectoria
				\4 Diferentes formas de entender monopolio
				\4 Adam Smith, clásicos, neoclásicos, austriacos...
				\4 Irving Fisher:
				\4[] ``ausencia de competencia''
				\4 Cournot y Marshall definieron concepto moderno
			\3 Objetivos
				\4 Definir modelo formal de monopolio
				\4 Caracterizar propiedades positivas del equilibrio
				\4[] Cantidad producida y vendida
				\4[] Precio de venta
				\4 Caracterizar propiedades normativas del equilibrio
				\4[] Optimalidad en comparación con competencia perfecta
				\4[] Medida de la suboptimalidad
				\4 Caracterizar poder de mercado
				\4[] En qué consiste
				\4[] En qué medida se ejerce
			\3 Resultados
				\4 Modelo básico de monopolio
				\4[] Un producto homogéneo
				\4[] Múltiples consumidores
				\4[] Un sólo productor
				\4[] Información perfecta
				\4 Propiedades positivas
				\4[] Producción menor que CP
				\4[] Precio superior a CP
				\4[] Mark-up sobre precio marginal
				\4[] Producción en zona elástica de la demanda
				\4 Propiedades normativas
				\4[] Demanda a coste marginal insatisfecha
				\4[] Excedente social menor a CP
				\4[] Coste marginal de producción $\leq$ a beneficio social
		\2 Formulación
			\3 Demanda
				\4[(i)] Función de utilidad cuasilineal
				\4[] Supuesto no esencial
				\4[(ii)] Función de demanda decreciente
				\4[] $\underset{q}{\max} \quad \phi(q) + m$
				\4[] $\text{s.a:} \quad pq + m \leq w$
				\4[] $\text{CPO}: \quad p = \text{UMg}(q)$
			\3 Maximización respecto a cantidad
				\4[] $\underset{q}{\max} \quad \Pi(q) = p(q) \cdot q - c(q)$
				\4[] $\text{CPO}: \Pi'(q) =0 \then \quad \underbrace{p'(q) \cdot q + p(q)}_{\text{IMg}} - \underbrace{c'(q)}_{\text{CMg}} = 0$
				\4[] $\text{CSO}: \Pi''(q) = 0 \then \underbrace{p''(q) \cdot q + 2 p'}_{\text{IMg'}} \leq \underbrace{c''(q)}_{\text{CMg'}}$
				\4[] $\quad \quad \quad \Rightarrow p \left( \frac{1}{\epsilon_{q-p}} + 1 \right) = c'$
				\4[] $\to$ Monopolista iguala $\text{IMg}(q) = \text{CMg}(q)$
				\4[] $\to$ $\text{IMg}(q) < P$
			\3 Maximización respecto a precio
				\4[] $\underset{p}{\max} \quad p \cdot q(p) - c\left( q(p) \right)$
			\3 Representación gráfica
				\4[] \grafica{monopolioestandar}
		\2 Implicaciones
			\3 Pendientes de ingreso y coste marginal
				\4 Pendiente del ingreso marginal menor a la de CMg
				\4 Condición de segundo orden del óptimo
				\4[] $\Pi''=0$
				\4[] $\to$ $\frac{d \, \text{IMg}}{d \, q} - \frac{d \, \text{CMg}}{d \, q} < 0$
				\4[] $\then$ $\text{Img}'(q) < \text{CMg}'(q)$
			\3 Elasticidad de la demanda en equilibrio
				\4 Aumento de precios tiene tres efectos
				\4[] i. Aumenta ingreso por unidad vendida
				\4[] $\to$ Deseable aumentar precios
				\4[] ii. Cae cantidad de unidades vendidas
				\4[] $\to$ No deseable aumentar precios
				\4[] iii. Caen los costes por menor producción
				\4[] $\to$ Deseable aumentar precios
				\4 Monopolio agota zona inelástica de la demanda
				\4[] Si demanda inelástica a precio:
				\4[] $\to$ Unidades vendidas cae menos \% que precio
				\4[] $\then$ Efecto i > ii
				\4[] $\then$ Efecto iii no es relevante
				\4[] $\then$ Aumentar precios aumenta beneficio
				\4[] $\then$ Aumenta precios hasta que i > ii
				\4[] $\then$ Agota zona inelástica de la demanda
				\4 Posible también $\uparrow$ precios en zona elástica
				\4[] En zona elástica, Efecto i < ii
				\4[] $\to$ A priori deseable no aumentar precios
				\4[] Pero efecto iii puede hacer que $i + iii > ii$
				\4[] $\to$ Posible que aumentar precios sea deseable aún
				\4[] $\then$ Aumentar precios hasta que $i + iii = ii$
				\4[$\then$] Monopolio produce siempre en zona elástica de demanda
				\4[$\then$] $\frac{p-c}{p}$ siempre igual o menor a 1
				\4[] Luego necesariamente, $\frac{1}{\left| \epsilon_{y-p} \right|} \in \, \left[ 0, 1 \right]$
				\4[] $\then$ $\epsilon_{y-p} > 1$
			\3 Poder de mercado
				\4 Precio de venta: mark-up sobre coste marginal
				\4 Mark-up en función de elasticidad de la demanda
				\4[$\to$] \fbox{$\frac{p(q^*)-c}{p(q^*)} = \frac{1}{ \left| \epsilon_{q-p} \right|} = \text{Índice de Lerner}$}
				\4 Equilibrio siempre en tramo elástico de la demanda
				\4[] En todo el tramo, $\downarrow$ producción reduce costes
				\4[] En tramo inelástico, $\uparrow$ precio incrementa ingresos
				\4[] $\to$ Precio aumenta más que cae demanda
				\4[] En tramo elástico, aumentar precio reduce ingresos
				\4[] $\to$ Pero reducción de costes puede ser mayor
				\4[] $\Rightarrow$ Agota reducción de producción en inelástico
				\4[] $\Rightarrow$ reducir producción hasta IMg = CMg
				\4[] Analíticamente:
				\4[] i) CPO: $p(1+\frac{1}{\epsilon_{y-p}}) = c'$
				\4[] ii) Demanda decreciente en precio $\Rightarrow \epsilon_{y-p} < 0$
				\4[] iii) Precio y coste marginal son positivos
				\4[] $\to$ $1 > \frac{p(q^*)-c}{p(q^*)} = \frac{1}{\left| \epsilon_{y-p} \right| } > 0 \then |\epsilon| > 1$
				\4[] $\then$ Tramo elástico de la demanda
			\3 Beneficios en el corto plazo
				\4 Costes hundidos en el corto plazo
				\4[] Incurridos irrevocablemente
				\4[] Reducen beneficio
				\4[] Independientemente de producción
				\4[] $\to$ Aunque sea 0
				\4 Beneficios pueden ser negativos
				\4[] Costes hundidos pueden inducir negatividad
				\4[] Monopolista no puede salir del mercado
			\3 Beneficios en el largo plazo
				\4 Sin costes hundidos
				\4[] No hay costes incurridos irrevocablemente
				\4 Monopolista puede salir del mercado
				\4[] Si beneficios son negativos
				\4 Beneficios nunca son negativos
			\3 Nivel de producción ineficiente en el largo plazo
				\4 En competencia perfecta
				\4[] Empresas entran hasta eliminar beneficios económicos
				\4[] Producción en escala eficiente
				\4[] $\to$ Mínimo coste medio
				\4 En monopolio
				\4[] Empresas no entran
				\4[] $\to$ Por razones múltiples asumidas exógenas
				\4[] Producción óptima puede ser mayor o menor que EF
				\4[] $\to$ Condiciones $\text{IMg} = \text{CMg}$, $\text{IMg'} \leq \text{CMg'}$
				\4[$\to$] Producción $q^*$ tal que $\text{IMg}(q^*) = \text{CMg}(q^*)$
			\3 Bienestar
				\4 Beneficios del productor
				\4[] Superiores a los de competencia perfecta
				\4[] En competencia perfecta
				\4[] $\to$ Precio está fijado
				\4[] En monopolio
				\4[] $\to$ Monopolio elige precio dada recta
				\4[] $\then$ Más opciones
				\4[] $\then$ Necesariamente beneficio mayor o igual que CP
				\4 Excedente del consumidor
				\4[] Demanda insatisfecha a coste marginal
				\4[] Excedente de consumidores potencial
				\4[] $\to$ Que no llega a realizarse
				\4[] $\then$ Menos que en competencia perfecta
				\4[] $\then$ Menos
				\4 Maximiza excedente del productor
				\4[] $\to$ No bienestar social
				\4 Redistribución a favor del productor
				\4[] Consumidores reducen bienestar
				\4[] Productores aumentan
			\3 Pérdida de eficiencia
				\4 Coste marginal de producción
				\4[] Menor a precio de venta
				\4 Demanda no satisfecha
				\4[] Consumidores demandarían más unidades
				\4[] $\to$ Si precio bajase hasta CMg
				\4 Coste social de producción es coste marginal
				\4 Beneficio social mayor a coste marginal
				\4[] Aumentando nivel de producción:
				\4[] Consumidores podrían ganar más
				\4[] $\to$ Que productores podrían perder
				\4 Pérdida irrecuperable de eficiencia
				\4[] $\to$ Triángulo de Harberger
				\4[] \grafica{triangulodeharberger}
				\4[$\to$] Monopolio estándar no es sólo redistributivo
				\4[] también genera pérdidas de eficiencia
			\3 Impuestos\footnote{Ver tema 4B-9 sobre traslación e incidencia de los impuestos en competencia perfecta y mercados monopolistas, y tema 3B-7 sobre política comercial arancelaria y no arancelaria.}
				\4 Impuestos al producto en competencia perfecta
				\4[] Traslación hace irrelevante sujeto pasivo jurídico
				\4[] Ad-valorem y específico son equivalentes
				\4[] $\to$ Mismo efecto gravando uds. que precios
				\4[] Cuña fiscal igual a cuantía del impuesto
				\4 Impuestos al producto en monopolio
				\4[] Traslación hace irrelevante sujeto pasivo jurídico
				\4[] $\to$ Igual que en compentencia perfecta
				\4[] Ad-valorem y específico no son equivalentes
				\4[] Cuña fiscal puede ser mayor al impuesto
				\4[] $\to$ Con dda. isoelástica
				\4[] $\then$ Precio aumenta más que impuesto
				\4[] $\then$ Dda. muy inelástica, $\Delta P$ mucho más que $\Delta t$
				\4[] $\to$ Con dda. lineal
				\4[] $\then$ Precio aumenta la mitad que el impuesto
				\4 Impuestos al beneficio en competencia perfecta l/p
				\4[] Reduce número de empresas
				\4[] Aumenta precio de equilibrio
				\4[] Aumenta cantidad producida de equilibrio
				\4 Impuestos al beneficio en monopolio
				\4[] Redistribuye de empresa a estado
				\4[] Sin efecto sobre cantidad y precio producido
			\3 Aranceles y cuotas
				\4 Aranceles y cuotas en competencia perfecta
				\4[] Existe arancel equivalente a cuota
				\4[] $\to$ Misma cantidad producida
				\4[] $\to$ Misma cantidad importada
				\4[] $\to$ Mismo precio de venta
				\4 Cuotas y aranceles no son equivalentes en monopolio
				\4[] Aranceles
				\4[] $\to$ Monopolio nacional es precio aceptante
				\4[] $\to$ Coste de importaciones aumenta para todos importadores
				\4[] $\then$ Sin demanda residual para monopolio
				\4[] Cuotas
				\4[] $\to$ Demanda residual más alla de cuota de importación
				\4[] $\to$ Monopolio no es precio-aceptante en dda. residual
				\4[] $\then$ Comportamiento monopolístico en dda. residual
		\2 Aplicaciones
			\3 Trusts y cárteles
				\4 Evaluación de necesidad de intervención
				\4 Efectos negativos de su existencia
			\3 Limitaciones legales a la entrada
				\4 Justificación de ineficiencias
			\3 Intervención estatal
				\4 Generalmente, ha servido como justificación
				\4 Otros modelos de monop. extraen dif. conclusiones
	\1 \marcar{Extensiones}\footnote{Kreps + Varian (discriminación).}
		\2 Entrada
			\3 Idea clave
				\4 Contexto
				\4[] Modelo anterior supone implícitamente
				\4[] $\to$ Una sola empresa por asunción
				\4 Objetivo
				\4[] ¿Qué sucede cuando pueden entrar otras?
				\4[] $\to$ Qué equilibrio?
				\4[] $\to$ Es eficiente?
				\4 Resultados
				\4[] Entrada depende de posibilidad de b. positivos
				\4[] Eficiencia depende de subaditividad en costes
			\3 Monopolio natural
				\4 Definición:
				\4[] Un sólo productor es más eficiente que varios
				\4[] $\to$ Producción por monopolista es más eficiente
				\4[] Términos matemáticos:
				\4[] Función de costes es subaditiva
				\4[] $c(q_1 + q_2) < c(q_1) + c(q_2)$
				\4[] $\then$ Es posible que monopolio sea preferible
				\4[] $\then$ Necesarias mismas condiciones que contestabilidad
				\4 Monopolio natural con único producto
				\4[] Único producto
				\4[] $\to$ Cond. suficiente: EEscala/CMe decrecientes
				\4[] $\then$ No es condición necesaria
				\4[] \grafica{monopolionatural}
				\4 Monopolio natural con múltiples productos
				\4[] Multiples productos
				\4[] $\to$ Múltiples conjuntos de condiciones nec./suf.
				\4[] $\to$ Uno de ellos:
				\4[] Complementariedad de costes:
				\4[] $\uparrow$ de producción de un bien $\then$ $\downarrow$ CMg de resto
				\4 Sostenibilidad\footnote{Ver Conceptos.}
				\4[] Posibilidad de mon. natural no implica mon. natural
				\4[] Bajo determinados supuestos, puede ser inestable
				\4 Monopolio natural inestable:
				\4[] Demanda corta CMe en segmento creciente
				\4[] $\to$ Es decir, en zona con deseconomías de escala
				\4[] Entrante puede entrar en el mercado
				\4[] $\to$ Vendiendo a menor precio una cantidad menor
				\4[] $\then$ Capturando parte del mercado
				\4[] Reducción de precio y cantidad potencial
				\4[] $\to$ Arbitrariamente pequeña
				\4[] Sin incentivos a reducir hasta $P=\text{CMe}$
				\4[] $\to$ Existen beneficios económicos a capturar
				\4[] $\to$ Se mantienen incentivos a entrada
				\4[] $\then$ Monopolio natural no es estable
				\4 Monopolio natural estable
				\4[] Demanda corta CMe en segmento decreciente
				\4[] Entrante puede vender más a menor coste
				\4[] Dos supuestos alternativos sobre incumbente:
				\4[] i. Consumidores permiten a incumbente igualar oferta
				\4[] $\then$ Incumbente iguala para evitar entrada
				\4[] $\then$ Precio y cantidad se mantienen en monopolio
				\4[] ii. Consumidores no permiten igualar oferta
				\4[] $\to$ Incumbente y entrantes compiten sucesivamente
				\4[] $\then$ Equilibrio con $P=\text{CMe}$ y una empresa
				\4[] $\then$ Sin bfcio. potencial, ningún entrante potencial
				\4[] $\then$ Monopolio natural estable
				\4[] El incumbente puede aumentar producción ante entrada potencial
				\4[] $\to$ Si aumenta hasta CMe, no habrá incentivos a entrada
				\4[] $\then$ Entrante que baje precios pierde dinero
				\4 Necesidad de regulación
				\4[] Cuando producción superior a escala mínima eficiente
				\4[] Gobierno puede incentivar una sóla empresa
				\4[] $\to$ Para asegurar monopolio natural
				\4[] $\then$ Pero deberá regular precios
			\3 Contestabilidad
				\4 Idea clave
				\4[] Baumol, Panzar and Willig (1982)
				\4[] Caracterizar condiciones bajo las cuales
				\4[] $\to$ Monopolios producen a coste medio
				\4[] $\then$ Intervención pública innecesaria
				\4 Formulación
				\4[] Entrada y salida gratuita
				\4[] Sin costes hundidos
				\4[] $\to$ Coste fijo no es irrevocable e irrecuperable
				\4[] $\to$ Posible recuperar inversión en activo fijo
				\4[] $\then$ Posible estrategia hit-and-run
				\4[] Economías de escala
				\4[] $\to$ Coste medio decreciente
				\4[] Incumbente cree factible entrada de competidores
				\4[] $\to$ Toma medidas antes de que se produzca entrada
				\4[] $\then$ Entradas de competidores desincentivadas
				\4[] $\then$ Monopolio natural sostenible
				\4 Implicaciones
				\4[] Amenaza de entrada fuerza monopolista a:
				\4[] Producir cantidad $q_c$ tal que $\text{CMe}(q_c) = P(q_c)$
				\4[] $\then$ Beneficios iguales a cero
				\4[] $\then$ Eliminación de beneficio económico
				\4[] $\then$ Mejora eficiencia frente a monopolio sin entrada
				\4[] $\then$ Deseable que monopolios sean contestables
				\4[] $\then$ Si no hay CMe mínimo, $P=\text{CMe} > \text{CMg}$
				\4[] $\then$ Si hay CMe mínimo, $P=\text{CMe} < \text{CMg}$
				\4 Valoración
				\4[] Incumbentes tratan de incurrir en costes hundidos
				\4[] $\to$ Para ``atarse al mástil''
				\4[] $\to$ Amenaza creíble de mantener $P=CMe$
				\4[] $\then$ Señalizar disposición a eliminar beneficios
				\4[] $\then$ Otros competidores no dispuestos a entrar
				\4[] Realmente, muy dificil cumplimiento de supuestos
				\4[] ``Benchmark'' para mercados no competitivos
				\4[] $\to$ Más razonable que comparar con CPerfecta
			\3 Guerra de desgaste
				\4 Incumbente puede desincentivar entrada
				\4[] Tomando medidas para eliminar beneficio potencial de entrante
				\4 Ej.:
				\4[] Incumbente invierte en exceso de capacidad
				\4[] Convierte costes fijos en hundidos
				\4[] $\to$ Hace creíble amenaza de guerra de precios
				\4 Amenaza de guerra de precios
				\4[] Genera expectativa de pérdidas si entrada
				\4[] $\to$ Desincentiva potenciales entrantes
				\4[] $\then$ Puede fijar precio de monopolio
			\3 Problema de la delimitación
				\4 Antes de definir como monopolio natural
				\4[] Necesario delimitar el mercado
				\4 ¿Monopolio natural o competencia imperfecta?
				\4[] A menudo, cuestión subjetiva
				\4[] ¿Qué sustitutos se consideran?
				\4[] Ej.: Renfe monopolio en tren pero no en transporte
			\3 Monopolio como institución necesaria
				\4 Schumpeter, neo-austriacos
				\4 Modelo neoclásico
				\4[] Compara monopolio y comp. perfecta
				\4[] Producción de monop < prod. comp. perfecta
				\4 Objeto de comparación determina implicaciones
				\4[] Situación en la que inventor es monopolista
				\4[] Situación previa a invención: producción cero
				\4[] Situación post-invención: producción positiva
				\4[$\then$] Monopolio necesario para innovación
				\4[] Induce aparición de rentas a extraer
				\4[$\then$] Monopolios iniciales incentivan producción
				\4[] $\then$ Patentes
		\2 Monopolios de bienes duraderos\footnote{Palgrave, ``\textit{durable goods markets and aftermarkets}''.}
			\3 Idea clave
				\4 Bienes que pueden ser vendidos y usados en múltiples periodos
				\4 Comprador debe decidir
				\4[] Comprar ahora o comprar mañana
				\4 Monopolista debe decidir
				\4[] Vender sólo hoy o vender mañana también
				\4[] $\to$ Vender hoy reduce demanda mañana
				\4[] $\Rightarrow$ Monopolio es su propia competencia
				\4[] Si vende sólo hoy, puede fijar precio de monopolio
				\4[] ¿Puede comprometerse a no vender mañana?
				\4[] ¿Puede evitar reventa mañana?
				\4[] $\to$ Mañana tendrá incentivos a vender
				\4[] $\Rightarrow$ Inconsistencia dinámica
				\4[] $\to$ Compradores anticiparán
				\4[] $\Rightarrow$ Demandarán menos hoy $\Rightarrow$ $\downarrow$ precio
				\4 ¿Pueden comprometerse a no vender mañana?
				\4[] Depende del mercado
				\4[] Ej.: destrucción de modelos obras de arte
			\3 Conjetura de Coase
				\4 Horizonte infinito
				\4 Empresa no puede comprometerse a no vender en futuro
				\4[$\Rightarrow$] Precio será coste marginal
			\3 Estrategias del monopolista
				\4 Leasing en vez de venta
				\4[] Monopolista mantiene propiedad del bien
				\4[] $\to$ Incentivo a desviarse de óptimo desaparece
				\4 Eliminación de mercados de segunda mano
				\4[] Por ejemplo, libros de texto
				\4[] $\to$ Introducción de nueva edición anual
		\2 Discriminación
			\3 Idea clave
				\4 Monopolista puede vender a distintos precios
				\4[] $\to$ En función de la unidad
				\4[] $\to$ En función del consumidor
				\4[$\then$] Precios no lineales
			\3 Primer grado\footnote{Varian, pág. 243.}
				\4 Empresa puede distinguir
				\4[] Entre consumidores
				\4[] $\to$ Conoce su demanda individual
				\4[] Entre unidades vendidas
				\4[] $\to$ No hay reventa
				\4[$\then$] Puede cargar precios no lineales
				\4[$\then$] Puede diferenciar precios para cada consumidor
				\4 Empresa puede extraer todo el excedente
				\4[] Dos vías:
				\4[] $\to$ Take it or leave it
				\4[] $\to$ Precio por unidad + tarifa individual
				\4 Empresa ofrece un precio por cada única cantidad
				\4[] $\then$ oferta ``\textit{take it or leave it}''
				\4 Problema del monopolista
				\4[] $\underset{r,q}{\max} \quad \pi(r,q) = r - cq$
				\4[] $s.a: \quad u(q) \geq r$
				\4 Se cumple que $u(q) = r$
				\4[] $\then$ CPO: $u'(q) = c$
				\4 Eficiencia asignativa
				\4[] Coste marginal iguala utilidad marginal
				\4 Monopolista extrae todo el excedente social
				\4[] Beneficio + excedente del consumidor
			\3 Segundo grado\footnote{Varian, pág. 247. Ver también \href{http://www.eco.uc3m.es/~mmachado/Teaching/OI-I-MEI/slides/2.2.PriceDiscrimination_short.pdf}{Diapositivas de Matilde Machado (UC3M).}}
				\4 Empresa puede distinguir
				\4[] Entre unidades vendidas
				\4[] $\to$ No hay reventa
				\4[] $\then$ Pueden cargar precios no lineales
				\4 Empresa no puede distinguir
				\4[] Entre consumidores
				\4[] Aunque conoce distribución de características
				\4[$\then$] Puede cargar precios no lineales
				\4[$\then$] No puede diferenciar entre consumidores
				\4 Monopolista ofrece menú de precios
				\4[] Objetivo del menú:
				\4[] $\to$ Autoselección de consumidores
				\4[] $\to$ Extracción de máximo bienestar de cada consumidor
				\4[] $\then$ Aproximarse en lo posible a 1er grado
				\4[] $\then$ Maximizar beneficio
				\4[] Debe cumplir dos condiciones:
				\4[] $\to$ Todos los consumidores que deseen consuman
				\4[] $\to$ Consumidores se autoseleccionen
				\4 Precios no lineales sin discriminar consumidores
				\4[] $\to$ Tarificación no lineal
				\4[] $\to$ Posible discriminar unidades
				\4[] $\to$ Reventa asumida indisponible
				\4[] $\then$ Precio por unidad depende de cantidad
				\4 Ejemplo: consumidores dda. alta y baja
				\4[] Dos grupos de consumidores
				\4[] $\to$ Demanda muy alta
				\4[] $\to$ Demanda muy baja
				\4[] Imposible distinguirlos
				\4[] $\to$ Aunque se sabe que existen
				\4[] Opciones:
				\4[] i. Paquete $Q_0$ unidades a precio total $V_0$
				\4[] $\to$ Todos compran pack
				\4[] $\to$ Empresa extra igual excedente de ambos
				\4[] ii. Paquete $Q_1$ unidades a precio total $V_1$
				\4[] $\to$ Sólo demanda muy alta compra pack
				\4[] $\to$ Empresa no extrae excedente dda. baja
				\4[] iii. Ofrecer dos packs ($Q_B, V_B)$ y ($Q_A, V_A$)
				\4[] Decisión de demanda Alta
				\4[] $\to$ Dda. alta obtendría U positiva con A y B
				\4[] $\to$ Dda. alta obtendría mayor utilidad con A
				\4[] Decisión de demanda Baja
				\4[] $\then$ Prefiere pack B
				\4[] $\to$ Dda. baja. obtendría U nula consumiendo B
				\4[] $\to$ Dda. baja. obtendría U negativa con A
				\4[] $\then$ A demasiado cara para dda. baja
				\4[] $\then$ Prefiere pack B
				\4[] Con opción iii.
				\4[] $\to$ Empresa extrae máximo excedente de B
				\4[] $\to$ Empresa extrae parte del excedente de A
				\4[] $\then$ Maximización de excedente
				\4 Ejemplo: descuentos por volumen
				\4[] Tarifa en dos partes:
				\4[] $T = A + p q_i$
				\4[] $A \to$ Parte fija de la tarifa
				\4[] $p \to$ Parte variable de la tarifa
				\4[] $\then$ Precio/unidad cae a medida que $\uparrow$ cantidad
				\4[] $\then$ Más, consumo, menor precio
				\4 Eficiencia\footnote{Ver \href{http://faculty.econ.ucdavis.edu/faculty/bonanno/teaching/200C/2nd_degree.pdf}{Bonnano}}
				\4[] Respecto a ausencia de discriminación
				\4[] $\to$ Posible superioridad
				\4[] $\to$ Posible inferioridad
			\3 Tercer grado
				\4 Monopolistas pueden distinguir grupos de consumidores
				\4[] No pueden distinguir entre unidades vendidas
				\4[] $\to$ Pueden distinguir diferentes mercados
				\4[] $\to$ No pueden extraer ET de cada consumidor
				\4[] $\then$ Cargan precios lineales
				\4 Precios lineales pero discriminando consumidores
				\4 Si discriminación perfecta entre 2 consumidores:
				\4[] $\underset{q_1,q_2}{\max} \quad \pi(q_1,q_2) = p_1(q_1) \cdot q_1 + p_2(q_2) \cdot q_2 - c(q_1+q_2)$
				\4 Si costes marginales constantes
				\4[] $\to$ Equivale a dos problemas separados de máx.
				\4 Si costes marginales no constantes:
				\4[] CPO: $\text{IMg}_1 = \text{IMg}_2 = \text{CMg}$
				\4 Eficiencia
				\4[] El beneficio del monopolista aumenta/no decrece\footnote{O al menos, no decrece. Si el monopolista perdiese aplicando la discriminación, no la aplicaría y por ello, la posibilidad de discriminar mercados tiene un efecto no negativo.}
				\4[] El bienestar del consumidor cae/no aumenta\footnote{}
				\4[] $\to$ Beneficio $\uparrow$ más que $\downarrow$ bienestar de consumidor
				\4[] $\then$ Bienestar total puede aumentar respecto no discriminación
				\4[] $\then$ Bienestar mayor si mayor producción (condición necesaria)
				\4 Discriminación no perfecta
				\4[] $\to$ El precio para un consumidor afecta al otro
				\4 Distintos precios en cada mercado
				\4[] Mercado con demanda más elástica
				\4[] $\then$ Menor precio de monopolio
		\2 Múltiples bienes
			\3 Idea clave
				\4 Un monopolista vende en varios mercados
				\4[] ¿Demandas de distintos mercados están interrelacionadas?
				\4[] ¿Producción de un bien afecta a costes de otros bienes?
				\4[] $\to$ ¿Cómo optimiza el monopolista?
			\3 Demandas y producción separables
				\4 Precio de bienes no afectan demandas de otros bienes
				\4 Producción de un bien no afecta a costes de otros
				\4[$\Rightarrow$] Problemas de maximización separados
				\4[$\Rightarrow$] Varios monopolios de un producto
				\4[$\then$] Equivalente a discriminación de tercer grado
			\3 Dda. y producción no separables\footnote{Tirole, pág. 69. Sect. 1.1.2.}
				\4 Diferentes casos
				\4[] $\to$ Costes separados y demandas dependientes
				\4[] $\to$ Costes dependientes y demandas separadas
				\4 Costes separados y demandas dependientes
				\4[] Bienes sustitutivos:
				\4[] $\to$ Precios más altos que si ddas. independientes
				\4[] $\to$ Subida coordinada elimina sustitución
				\4[] Bienes complementarios
				\4[] $\to$ Precios más bajos que si ddas. independientes
				\4[] $\to$ Ddas. aumentan complementariamente
			\3 Monopolio dinámico\footnote{Tirole (1988), pág. 71.}
				\4 Distinto de bienes duraderos
				\4[] Cada periodo hay demanda
				\4[] $\to$ No es una demanda distribuida en distintos periodos
				\4 Bienes no duraderos pero vendidos en más de un periodo
				\4 ¿Ventas hoy afectan a demanda mañana?
				\4 Caso particular de monopolio multiproducto
		\2 Escuela austríaca
			\3 Von Mises
			\3 Schumpeter
	\1 \marcar{Regulación}\footnote{Handbook of Law and Economics, ch. 16}
		\2 Idea clave
			\3 Gestión de precios y costes
				\4 Fijación de precios
				\4 Subvenciones
				\4 Reglas de tarificación
				\4 ...
			\3 Imposición de barreras de entrada
				\4 Prohibición de entrada
				\4 Entrada regulada
				\4 Entrada sometida a subasta
				\4 ...
		\2 Justificación de la regulación
			\3 Eficiencia
				\4 Monopolios naturales se asumen más eficientes
				\4[] $\to$ Respecto a situación con más de una empresa
				\4 Varios problemas alejan de first-best:
				\4[] $\to$ Costes: ineficiencia X, subaditividad
				\4[] $\to$ Calidad
				\4[] $\to$ Cantidades
			\3 Posibilidad de mejorar resultado
				\4 En teoría y práctica
				\4 Herramientas regulatorias pueden $\uparrow$ beneficio social neto
				\4[] $\to$ Limitar entrada
				\4[] $\to$ Regular precios
				\4[] Teniendo en cuenta información disponible
			\3 Críticas a la regulación
				\4 Stigler y otros
				\4 Regulación es instrumento de monopolistas
				\4[] Demanda de regulación para perpetuar ventajas
		\2 Información completa, sin comportamiento estratégico
			\3 Idea clave
				\4 Regulador conoce perfectamente
				\4[] funciones de demanda
				\4[] funciones de coste
				\4 Regula:
				\4[] Precios
				\4[] Cantidades
				\4[$\to$] Para aumentar bienestar social
			\3 Mercados contestables
				\4 Precio igualará coste medio
				\4[] \grafica{precioigualacostemedio}
				\4 Posible reducir hasta coste marginal
				\4[] Subvencionando al monopolista
				\4[] $\to$ ¿Cuánto cuestan los fondos públicos?
				\4[] $\to$ ¿Qué distorsiones adicionales causarán?
				\4 En este contexto teórico, pocas razones para regular
				\4[] $\to$ Pero monopolios perfectamente contestables son raros
				\4[] $\to$ Muchos monop. naturales no son sostenibles
				\4[] $\Rightarrow$ Necesarias barreras de entrada
				\4[] Evitar el \textit{cream skimming} tras BdEntrada
				\4[] $\then$ Necesarias regulaciones de precios
			\3 Multiproducto y precios lineales
				\4 Ramsey (1927), Boiteux (1956)
				\4 Un sólo mercado
				\4[] Posible fijar P=CMe
				\4 Varios mercados
				\4[] Posible fijar P=CMe en todos los mercados
				\4 ¿Es posible mejorar?
				\4[] $\to$ Modular precios en función de elasticidades
				\4[] $\to$ Demandas + elásticas $\then$ + pérdida eficiencia
				\4[] $\then$ Discriminación de 3er grado puede aumentar eficiencia
				\4[] $\then$ Precio más alto donde menos afecta cantidad
				\4 Problema a resolver:
				\4[] (i) Dado un beneficio mínimo del monopolista
				\4[] ¿qué precios debe asignar para equilibrio Pareto-eficiente?
				\4[] ¿qué precios maximizan excedente social?
				\4[] (ii) Dado un excedente social mínimo
				\4[] ¿qué precios asignar para maximizar beneficio de monopolista?
				\4 Bienes con demanda inelástica
				\4[] $\to$ subir precios
				\4 Bienes con demanda elástica
				\4[] $\to$ Subidas menores
				\4 Formulación
				\4[] \fbox{$\frac{p_i - c_i}{p_i} = \frac{\lambda}{1+\lambda} \cdot \frac{1}{\left| \epsilon_i \right| - \sum_{k \neq i}^K \epsilon_{k-i}}$}
				\4[] \fbox{$\lambda = \left| \frac{\Delta \textrm{Beneficio social}}{\Delta \textrm{Beneficio del monopolista}} \right| $}: PSombra de FPúblicos
				\4[] $\epsilon_{ki} = \frac{d q_k}{q_k} \cdot \frac{p_i}{d p_i}$
				\4[] $\to$ Elasticidad cruzada de bien $k$ y precio de $i$
				\4[$\then$] Sustitutivos ($\epsilon_{ki} > 0$) aumentan precio de $i$.
				\4[$\then$] Complementarios ($\epsilon_{ki} <0$) bajan precio de $i$.
				\4[] Evitar el \textit{cream skimming}
				\4[] $\to$ Entrada de competidores en segmentos con beneficios
				\4[] $\then$ Necesarias regulaciones de entrada si precios
			\3 Precios no lineales
				\4 Discriminación de segundo grado
				\4[] $\to$ Tarifas de dos partes adaptadas a demanda
				\4 Necesario conocer muy bien la función de dda.
				\4 Dificil mantener a consumidores con demanda baja
				\4[] $\to$ Si parte fija demasiado elevada
		\2 Información limitada y comportamiento estratégico
			\3 Idea clave
				\4 Métodos anteriores suponen:
				\4[] -- Conocimiento de los costes de la empresa
				\4[] -- Conocimiento de la demanda
				\4[] -- Ausencia de comportamiento estratégico
				\4 En la práctica, no se cumplen todos
				\4[] Necesario considerar incentivos
			\3 Coste del servicio o tasa de retorno
				\4 Fijar un ingreso máximo
				\4[] Empresa puede extraer ese retorno como máximo
				\4 $R = E + sK$
				\4[] $R \to$ ingreso (depende de dda. y precios fijados)
				\4[] $E \to$ gastos operativos
				\4[] $s \to$ tasa de retorno del capital permitida
				\4[] $K \to$ capital utilizado
				\4 Maximización de beneficio sujeta a restricción
				\4[] $\underset{K,L}{\max} \quad \Pi = R(K,L) - wL - rK$
				\4[] $\text{s.a:} \quad \frac{R(K,L) - wL}{K} \leq s \then \Pi \leq (s-r) \cdot K$
				\4 Revisión periódica
				\4[] $\to$ Precios
				\4[] $\to$ Valoración del capital
				\4 Varios inconvenientes
				\4[] $\to$ ¿Cómo determinar $s$ óptimo?
				\4[] $\to$ Reduce incentivos para aumentar eficiencia
				\4[] $\to$ Requisitos informativos aumentan captura de regulador
				\4[] $\to$ ¿Cómo valorar el capital?\footnote{¿Precio marginal, histórico, reposición...?}
				\4[$\to$] Lags de revisión pueden ser perjudiciales
				\4[] Dañan a consumidores si costes bajan
				\4[] Dañan a empresa si costes suben
				\4[$\to$] Efecto de Averch-Johnson\footnote{Ver pág. 1298 de Joskow (Handbook of Law and Economics).}
				\4[] Regulador fija $s$
				\4[] $\to$ menor que $r_m$ en ausencia de regulación
				\4[] $\to$ mayor que coste de capital $r$
				\4[] Incentivos a invertir en exceso en K:
				\4[] $\frac{d L}{d K} = \left| \text{RMST}_{lk} \right| = \frac{ \partial f(k) / \partial k}{\partial f(l) / \partial l} = \frac{r}{w} + \mu(r-s)$, $r < s$
				\4[] $\then$ Empresa invierte en K más que en ausencia de regulación
				\4[] Representación gráfica
				\4[] \grafica{averchjohnson}
				\4[] (Puede estimular innovación por sust. de L y K)
			\3 Reglas Price Cap
				\4 Propuesto por primera vez
				\4[] Regular British Telecom (1983)
				\4 Funcionamiento
				\4[] Fijación de una senda de precios
				\4[] Empresa decide el resto
				\4 Ventajas
				\4[] $\to$ Fuerza mejoras de eficiencia periódicas
				\4[] $\to$ Sencillez de regla dificulta captura regulatoria
				\4 Lags en revisión de precios
				\4[] $\Rightarrow$ Incentivan mayor reducción de costes
				\4 CPI-X
				\4[] Senda de precios ligada a IPC menos tasa X
				\4[] X ligado a mejoras de productividad del sector
			\3 Competencia referencial -- Yardstick competition
				\4 Schleifer (1985)
				\4 Monopolios espaciales múltiples
				\4[] Cada uno en su ámbito de monopolización
				\4 Problema del regulador
				\4[] Fijar precio
				\4[] $\to$ Mantenga servicio
				\4[] $\to$ Minimice costes
				\4[] $\then$ Maximice excedente social
				\4 Coste de servicio
				\4[] Fijar precio igual a coste marginal
				\4[] Transferir cantidad para cubrir coste fijo
				\4 Crítica de coste de servicio
				\4[] Regulador fija precio
				\4[] $\to$ Para evitar que monopolio salga de mercado
				\4[] $\then$ Elimina incentivos reducir costes
				\4[] $\then$ Carece de información para conocer coste
				\4[] Regulador necesita:
				\4[] $\to$ Regla simple distinta a coste histórico
				\4 Formulación
				\4[] Coste marginal de proveer servicio:
				\4[] $c = c_0 - e$
				\4[] $\to$ $c$: coste final
				\4[] $\to$ $c_0$: coste inicial sin reducción
				\4[] $\to$ $e$: reducción vía esfuerzo
				\4[] Coste de esfuerzo de reducción de coste
				\4[] $C(e)$, $C'(e) > 0$, $C''(e) >0$
				\4[] Si regulador conoce forma $C(e$ y $e$:
				\4[] $\to$ Fija $e^*$ a ejercer tal que $C'(e) = 1$
				\4[] $\to$ Fija $P= c_0 - e*$
				\4[] $\to$ Transfiere $T=C(e)$
				\4[] $\then$ BMg iguala CMg de esfuerzo $1=C'(e)$
				\4[] $\then$ Óptimo social
				\4[] Si regulador sólo puede observar:
				\4[] $\to$ Coste total de esfuerzo realizado $C(e)$
				\4[] $\to$ Coste final $c$
				\4[] $\then$ Puede fijar $T=C(e)$ y $p$
				\4[] $\then$ No puede observar $C'(e)$ real
				\4[] $\then$ Óptimo de monop. no tiene por qué ser $C'(e) = 1$
				\4[] $\then$ Empresa sin incentivos a ejercer $e^*$
				\4 Comparar rentabilidad de industrias similares
				\4[] Opción de regulador
				\4[] Fija precio medio final $\bar{c} = \frac{1}{n-1} \sum_i c_j$
				\4[] $\to$ Media de costes finales de los demás
				\4[] Fija transferencia final $\bar{C}(e) = \frac{1}{n-1} \sum_j C_j(e)$
				\4[] Empresa sí tiene incentivos a reducir costes
				\4[] $\to$ Todo el coste que consiga reducir es beneficio
				\4[] $\to$ No tiene incentivos a declarar más coste marginal
				\4[] $\to$ No tiene incentivos a declarar más coste de esfuerzo
				\4[] $\then$ Porque beneficio iría a otros monopolistas
				\4 Vincular rentabilidad máxima permitida
				\4[] A rentabilidad de competidores comparables
				\4[] $\to$ Desincentivando declarar costes elevados
			\3 Diseño de mecanismos
				\4 Asumir información asimétrica
				\4[] Regulador tiene menos información que regulado
				\4 Tener en cuenta incentivos de regulado
				\4[] $\to$ Diseñar mecanismos que alineen objetivos
				\4 Mecanismo de Loeb-Magat
				\4[] Firma y regulador conocen demanda
				\4[] Sólo firma conoce costes
				\4[] Empresa anuncia precio
				\4[] Empresa recibe subsidio decreciente con precio
				\4[] $\then$ Firma tiende a fijar precio marginal
				\4 Campo muy fértil de investigación
				\4[] Regulación de monopolios sólo una aplicación
		\2 Instituciones regulatorias
			\3 Idea clave
				\4 Posible regular p y q con herramientas variadas
				\4 Estado debe elegir herramienta óptima
				\4 En la práctica, combinaciones de herramientas
			\3 Legislación
				\4 Establecimiento directo de regulaciones
				\4 Problemas:
				\4[] Falta de cualificación técnica de legisladores
				\4[] Falta de información actualizada
				\4[] Vulnerable a grupos de interés organizados
			\3 Contratos de franquicia
				\4 Competencia \textit{por} el mercado
				\4[] En vez de competencia \textit{dentro} el mercado
				\4 Empresas compiten por derecho a monopolizar
				\4[] $\to$ Estado trata de extraer máx. de renta a monop.
				\4 Requisitos
				\4[] Número suficiente de competidores
				\4[] Actúen de forma independiente
				\4[] Subasta diseñada adecuadamente
				\4 Diseño de mecanismos aplicado a contratos de franquicia
				\4 En la práctica:
				\4[] Contratos con horizontes temporales cortos
				\4[] Mejorar credibilidad de contratos
				\4[] $\to$ Si l/p, cambios econ. hacen insostenibles
			\3 Comisiones independientes de regulación
				\4 Formadas por número reducido de miembros
				\4[] Votan sí o no regulaciones impuestas
				\4[] Apoyadas por plantilla de técnicos
				\4 Estructura cuasi-judicial
				\4 Capacidad de aprobar regulación
				\4 Independencia deseable
			\3 Provisión pública
				\4 Estado controla medios de producción
				\4[] $\to$ Provee directamente el servicio o bien
				\4 Habitual en Europa pre-privatizaciones
				\4[] Telefonía, eléctricas, ferrocarril...
	\1[] \marcar{Conclusión}
		\2 Recapitulación
			\3 Modelo estándar
				\4 Microeconomía neoclásica
				\4 Derivada de Cournot y Marshall
			\3 Extensiones
				\4 Variaciones del modelo neoclásico
				\4[] Diferentes implicaciones
				\4 Entrada
				\4 Monopolio de bienes duraderos
				\4 Discriminación
				\4 Múltiples productos
				\4 Escuela austriaca
			\3 Regulación
				\4 Concepto
				\4 Justificación
				\4 Información completa
				\4 Información limitada
		\2 Idea final
			\3 Evolución del monopolio
				\4 Pasado:
				\4[] Líneas aéreas
				\4[] Utilities (eléctricas, agua, gas...)
				\4[] Petróleo (Standard Oil)
				\4[] ...
				\4 Actualidad
				\4[] Software
				\4[] Servicios
				\4[] Publicidad
				\4[] $\to$ Ej.: Google, Baidu
				\4[] Agricultura
				\4[] Preocupación creciente
				\4 Tendencias actuales contrapuestas
				\4[] Mayores mercados deberían dificultar monopolios tradicionales
				\4[] Pero globalización fomenta concentración
				\4[] Sistemas financieros más desarrollados facilitan entrada
			\3 Relevancia actual del análisis del monopolio
				\4 Componente fundamental economía neoclásica
				\4 Hecho empírico
				\4[] Aumento poder de mercado décadas recientes
				\4[] $\to$ En determinadas industrias
				\4[] $\to$ Nuevos monopolios naturales en plataformas
				\4[$\Rightarrow$] Análisis de monopolio es aun muy necesario
\end{esquemal}



























\graficas

\begin{axis}{4}{Maximización de beneficios de un monopolio en el modelo estándar.}{$q$}{$p$}{monopolioestandar}
	% demanda
	\draw[-] (0,3.5) -- (3.5,0);
	\node[right] at (2.7, 0.3){\footnotesize D};
	
	% coste marginal
	\draw[-] (0,0.75) -- (4,0.75);
	\node[right] at (3.6,0.55){\footnotesize CMg};
	
	% coste medio
	\draw[-] (0.25,4) to [out=272, in=180](4,0.9);
	\node[right] at (3.6,1.1){\footnotesize CMe};
	
	% ingreso marginal
	\draw[dashed] (0,3.5) -- (2,0);
	\node[right] at (.55,1.3){\footnotesize IMg};
	
	% cantidad óptima
	\draw[dotted] (1.58,0) -- (1.58,4);
	\node[below] at (1.58,0){$q_m$};
	
	% img(q*) = cmg(q*)

	\node[circle, fill=black, inner sep=0pt, minimum size=3pt] (a) at (1.58,0.75) {};
	
	% cme(q*)
	\node[circle, fill=black, inner sep=0pt, minimum size=3pt] (a) at (1.58,1.62) {};
	\draw[dotted] (1.58,1.62) -- (0,1.62);
	
	% p(q*)
	\node[circle, fill=black, inner sep=0pt, minimum size=3pt] (a) at (1.58,1.93) {};
	\draw[dotted] (0,1.93) -- (1.58,1.93);
	\node[left] at (0,1.93){$P_m$};
	
	
	% beneficio del monopolista
	\draw [white, fill=yellow, opacity=0.2] (1.58,1.62) -- (1.58,1.93) -- (0,1.93) -- (0,1.62) -- (1.58,1.62);	
	
	% excedente del consumidor
	\draw [white, fill=blue, opacity=0.2] (1.58,1.93) -- (0,1.93) -- (0,3.5);
	
\end{axis}

En azul, excedente del consumidor. En amarillo, beneficio del monopolio.

\begin{axis}{4}{Triángulo de Harberger de un monopolio estándar.}{$q$}{$p$}{triangulodeharberger}
	
	
	% demanda
	\draw[-] (0,3.5) -- (3.5,0);
	\node[right] at (2.7, 0.3){\footnotesize D};
	
	% coste marginal
	\draw[-] (0,0.75) -- (4,0.75);
	\node[right] at (3.6,0.55){\footnotesize CMg};
	
	% coste medio
	%\draw[-] (0.25,4) to [out=272, in=180](4,0.9);
	%\node[right] at (3.6,1.1){\footnotesize CMe};
	
	% ingreso marginal
	\draw[dashed] (0,3.5) -- (2,0);
	\node[right] at (1.8,0.3){\footnotesize IMg};
	
	% cantidad óptima
	\draw[dotted] (1.58,0) -- (1.58,4);
	\node[below] at (1.58,0){$q_m$};
	
	% img(q*) = cmg(q*)
	
	\node[circle, fill=black, inner sep=0pt, minimum size=3pt] (a) at (1.58,0.75) {};
	
	% cme(q*)
	%\node[circle, fill=black, inner sep=0pt, minimum size=3pt] (a) at (1.58,1.62) {};
	%\draw[dotted] (1.58,1.62) -- (0,1.62);
	
	% p(q*)
	\node[circle, fill=black, inner sep=0pt, minimum size=3pt] (a) at (1.58,1.93) {};
	\draw[dotted] (0,1.93) -- (1.58,1.93);
	\node[left] at (0,1.93){$P_m$};
	
	% beneficio del monopolista
	\draw [white, fill=yellow, opacity=0.2] (1.58,1.62) -- (1.58,1.93) -- (0,1.93) -- (0,0.75) -- (1.58,0.75);	
	
	% excedente del consumidor
	\draw [white, fill=blue, opacity=0.2] (1.58,1.93) -- (0,1.93) -- (0,3.5);
	
	% triángulo de Harberger
	\draw [white, fill=red, opacity=0.2] (1.58,0.75) -- (1.58,1.93) -- (2.74,0.75);
	\draw[-{Latex}] (3,2) -- (2,1.1);
	\node[right] at (3,2){\footnotesize Pérdida de eficiencia / Triángulo de Harberger};
\end{axis}

En azul, excedente del consumidor. En amarillo, beneficio del monopolio asumiendo que costes marginales fijos y que no hay costes fijos de tal manera que el coste medio es constante e igual al coste marginal. En rojo, pérdida de eficiencia o Triángulo de Harberger.


\begin{axis}{4}{Monopolio natural para una función de producción que muestra economías y deseconomías de escala en el intervalo relevante.}{}{P}{monopolionatural}
	\draw[-] (4,0) -- (10,0);
	\node[below] at (10,0){Q};
	
	% coste con una empresa
	\draw[-] (0.3,4) to [out=280, in=180](3.3,1);
	\draw[-] (3.3,1) to [out=0, in=260](6.3, 4);
	\node[above] at (6.1,4){$\text{CMe}_1$};
	
	% coste con dos empresas
	\draw[-] (4.3,4) to [out=280, in=180](7.3,1);
	\draw[-] (7.3,1) to [out=0, in=260](10.3, 4);
	\node[above] at (10.3,4){$\text{CMe}_2$};
	
	% límite del monopolio natural
	\draw[dashed] (5.30,1.8) -- (5.3,0);
	\node[below] at (5.3,0){$q_m$};
	
	% demanda que implica monopolio natural estable
	\draw[-] (0.8,4) -- (2.8,0);
	\node[above] at (0.8,4){D};
	
	% demanda que implica monopolio natural inestable
	\draw[-] (3.8,4) -- (5.8,0);
	\node[above] at (3.8,4){D};
	
	% límite de la estabilidad del monopolio natural
	\draw[dashed] (3.3,1) -- (3.3,0);
	\node[below] at (3.3,0){$q_e$};
\end{axis}

La cantidad $q_m$ señala la la producción a partir de la cual resulta más eficiente producir con dos empresas que con una. La figura muestra como los niveles de producción que superan $q_m$ implican unos costes medios superiores cuando una sóla empresa produce la cantidad señalada. El punto $q_e$ señala el límite del intervalo de producción para el que el monopolio natural es estable. A la derecha de $q_e$, el monopolista obtiene un beneficio positivo cuando iguala coste e ingreso marginales. Si la entrada en el mercado es libre, cualquier empresa podría producir una cantidad arbitrariamente inferior a $q_e$ y venderla a un precio inferior, obteniendo una fracción arbitraria del beneficio que el incumbente extraía inicialmente. El incumbente tendrá entonces incentivos a reducir de nuevo la cantidad producida, y el entrante potencial se verá a su vez incentivado a reducir su cantidad ulteriormente para extraer un beneficio inferior pero positivo. El proceso de reducción de la cantidad continuaría hasta que cada una de las empresas produjesen $q_e$ y se alcanzase un equilibrio subóptimo de dos empresas.

\begin{axis}{4}{Precio y cantidad producida por un monopolista en un mercado contestable}{Q}{P}{precioigualacostemedio}
	\draw[-] (0.2,3.5) -- (3,0);
	
	% Coste medio
	\draw[-] (0.2,3) to [out=270, in=177](4,1.1);
	\node[above] at (4,1.15){CMe};
	
	% Coste marginal
	\draw[-] (0,1) -- (4,1);
	\node[below] at (4,0.95){CMg};
	
	% precio de equilibrio de largo plazo
	\draw[dashed] (0,1.32) -- (1.93,1.32);
	\node[left] at (0,1.32){$P^*$};
	
	% cantidad de equilibrio de largo plazo
	\draw[dashed] (1.93,1.32) -- (1.93,0);
	\node[below] at (1.93,0){$Q^*$};
\end{axis}

\begin{axis}{4}{Representación gráfica del efecto Averch-Johnson: la restricción sobre la rentabilidad del capital induce una intensidad de capital superior a la que maximiza el beneficio.}{K}{$\pi$}{averchjohnson}
	% Beneficio en función de capital
	\draw[-] (0,0) to [out=70,in=180](2.2,3) to [out=0,in=100](3.75,0);
	\node[left] at (2,3.1){\tiny $\pi(K)$};
	
	% Restricción sobre la rentabilidad
	\draw[-] (0,0) -- (4,2.5);
	\node[right] at (4,2.5){\tiny $(s-r)K)$};
	
\end{axis}


\conceptos

\concepto{Coste de los fondos públicos}

Dado un gasto público exógeno y asumiendo un déficit en términos absolutos constante, la introducción de un subsidio o la reducción de un impuesto implicará necesariamente un aumento de los ingresos por otras vías. La distorsión producida por el impuesto adicional supondrá un coste que distorsionará las decisiones de los agentes y reducirá potencialmente el bienestar. El coste de los fondos públicos hace referencia a la disminución del bienestar derivada de esa distorsión fiscal adicional.

\concepto{Cream skimming}

De forma general, el \textit{cream skimming} hace referencia a la práctica de la provisión de servicios o bienes sólo a consumidores con alta disposición a consumir. De esta forma, el monopolista evita los costes asociados a mercados menos rentables.

En un contexto de subsidios cruzados en los que se permite a un monopolista cobrar un precio por encima del coste marginal en un mercado a cambio de proveer un servicio deficitario en otro, el concepto de \textit{cream skimming} hace referencia a la entrada de una tercera empresa en el mercado rentable. La entrada provocará una disminución de los beneficios de la empresa incumbente, posiblemente empujándola a pérdidas globales (teniendo en cuenta el mercado deficitario, que tiene obligación de abastecer). Un ejemplo paradigmático de cream skimming es la provisión de telefonía en Estados Unidos en los años 20. El mercado se encontraba dividido en dos segmentos: el segmento de larga distancia y el de llamadas locales. Para incentivar la provisión de un servicio universal, el gobierno obligaba a mantener bajo el precio de las llamadas locales a cambio de permitir la extracción de márgenes elevados en las llamadas de larga distancia. Aunque inicialmente los costes fijos impedían la entrada en el mercado de larga distancia, con el tiempo entrar en el mercado de larga distancia pasó a ser rentable. Esto creó oportunidades de entrada para potenciales competidores que se limitaban a proveer el servicio de larga distancia, extraían los elevados márgenes y afectaban a la sostenibilidad de la provisión del servicio de corta distancia.


\concepto{Economías de alcance}
Una función de producción muestra economías de alcance en la producción de múltiples outputs cuando es más barato producir varios outputs en una sóla empresa que producir el mismo vector de outputs en varias empresas por separado, con un output por empresa.

\concepto{Pérdida de eficiencia}

El concepto de pérdida de eficiencia hace referencia a la disminución cuantitativa en el bienestar que representa un equilibrio en relación a otro. En términos cualitativos, una pérdida de eficiencia describe a un equilibrio como socialmente subóptimo respecto a otro equilibrio factible dada una tecnología, unas preferencias y una dotación.

\concepto{Sostenibilidad de los monopolios naturales}

Ver página 88 de Sharkey (1982).

Un monopolio natural es sostenible cuando los entrantes potenciales no tienen incentivos para rebajar el precio y capturar toda o parte de la demanda. ¿Cuándo aparece esta ausencia de incentivos? Cuando reducciones del precio se sitúan necesariamente por debajo del coste medio para la cantidad producida, de tal manera que el beneficio sea negativo. Cuando la curva de demanda intersecciona la curva de coste medio a la izquierda de la escala eficiente y el monopolista fija un precio igual al coste medio en la intersección, cualquier disminución del precio aplicada por un potencial entrante resultará en una caída del precio por debajo del coste medio y por ello, en un beneficio negativo. Así, cuando la demanda intersecciona la curva de coste medio a la izquierda de la escala eficiente el monopolio natural es sostenible. ¿Qué sucede cuando la intersección se produce a la derecha de la escala eficiente? Supongamos que el monopolista incumbente ofrece su producto al precio que corresponde a la intersección entre la curva de demanda y la curva de coste medio. Un potencial entrante podría rebajar el precio una cantidad arbitrariamente pequeña de tal manera que los consumidores prefieresen su producto, y ofertar a ese precio una cantidad inferior a la que iguala el coste medio con el precio, obteniendo un beneficio positivo. Por ello, cuando aparecen deseconomías de escala o costes medios crecientes, un monopolio natural no es sostenible en el sentido de que potenciales competidores tienen incentivos a entrar en el mercado.


\concepto{Subaditividad y economías de escala} 

Los conceptos de subaditividad y economías de escala son dos posibles propiedades de las funciones de coste. La propiedad de economías de escala se caracteriza por una disminución del coste medio por unidad al aumentar la escala de producción. Formalmente, una función $c(x)$ experimenta economías de escala si y solo si $c(\lambda x) < \lambda c(x)$, $\forall \, \lambda \in \mathbb{R}$. Una función de costes es subaditiva si el coste de producir la suma de dos cantidades en el dominio de la función de costes es menor que la suma de los costes de producir por separado tales cantidades. Formalmente, una función $c(x): \mathbb{R} \to \mathbb{R}$ muestra subaditividad si $c(x_1 + x_2) < c(x_1) + c(x_2)$.

Todas las funciones de coste con economías de escala son subaditivas. Sin embargo, una función puede ser subaditiva sin mostrar necesariamente economías de escala. Es el caso, por ejemplo, de funciones de costes con la forma general $c(x) = f + x^2$. Esto es, funciones de coste con un elemento $f$ que captura el coste fijo de producir, y un elemento $x^2$ que captura el coste variable. En una función de este tipo, existen economías de escala\footnote{Resultado de minimizar la función de coste medio $\text{CMe}(x) = \frac{f}{x} + x$. La condición de primer orden es:
	
	\begin{align*}
		\text{CMe}'(x) = 1 - f \cdot x^{-2} = 0 \\
		\Rightarrow x^* = \sqrt{f}
	\end{align*}
	
Y dado que $\text{CMe}''(x) > 0 \, \forall x$, la CPO es suficiente además de necesaria. }en tanto que $x \in [0, \sqrt{f}]$. Sin embargo, a partir de tal valor $x=\sqrt{f}$, la función de costes sigue siendo subaditiva a pesar de comenzar a mostrar deseconomías de escala. En una función con esa forma general, esta subaditividad permanece\footnote{Demostracion:

Se trata de hallar los valores de $\bar{x}$ tal que $c(\sqrt{f} + \bar{x}) < c(\sqrt{f}) + (\bar{x})$. Sustituyendo por la función de costes, tenemos la inecuación:

\begin{align*}
	f + \left( \sqrt{f} \right)^2 + \bar{x}^2 + 2 \sqrt{f} \bar{x} < f + \left( \sqrt{f} \right)^2 + f + \bar{x}^2
\end{align*}

Reordenando y despejando tenemos que $\bar{x} < \frac{f}{2 \sqrt{f}}$

} hasta que $x=\sqrt{f} + \frac{f}{2 \sqrt{f}}$. De este modo, es posible que una función de costes sea subaditiva sin mostrar economías de escala en determinados intervalos y bajo determinados supuestos.

\concepto{Triángulo de Haberger v. pérdida de eficiencia}


\preguntas

\seccion{Test 2018}
\textbf{8.} Un monopolista se enfrenta a una función de demanda $p=120-3x$. Su función de costes es $\text{CT}=x^2 + 8x + 300$. Suponga que se produce una regulación consistente en imponerle un impuesto fijo de 100 unidades monetarias. El beneficio que obtendrá el monopolista tras la regulación será:

\begin{itemize}
	\item[a] $B=300$
	\item[b] $B=384$
	\item[c] $B=400$
	\item[d] $B=484$
\end{itemize}

\seccion{Test 2016}

\textbf{16.} Un monopolista, con una función de costes $c(x) = x$ se enfrenta a una función de demanda de un bien $x=10/p$
\begin{enumerate}
	\item[a] El beneficio máximo que puede obtener es 0.
	\item[b] El beneficio no depende de la cantidad producida.
	\item[c] La cantidad de producto con la que obtiene el máximo beneficio es 10.
	\item[d] Al monopolista le interesa producir la menor cantidad posible de producto.
\end{enumerate}

\seccion{Test 2015}
\textbf{9}. En un mercado la oferta está compuesta por 2 empresas idénticas cuya tecnología tiene asociada la siguiente función de costes: $C(q) = 40 + 10 q^2$, donde $q$ es el nivel de producción de cada empresa. Indique hasta qué nivel de producción puede considerarse que el mercado es un monopolio natural: 

\begin{enumerate}
	\item[a] $q<2$
	\item[b] $q<\sqrt{8}$
	\item[c] $q < 3$
	\item[d] Ninguno de los anteriores.
\end{enumerate}

\seccion{Test 2014}

\textbf{7.} El índice de Lerner de un monopolio que produce el bien a coste cero en un mercado en el que la demanda des $D(p) = \max \left\lbrace 1 - \frac{p}{4}, 0 \right\rbrace$ es:
\begin{enumerate}
	\item[a] $L = 0$
	\item[b] $L = 1/4$
	\item[c] $L = 1/2$
	\item[d] $L = 1$
\end{enumerate}

\textbf{9.} Si en un monopolio se elimina una legislación que prohíbe la discriminación de precios de tercer grado, entonces:
\begin{enumerate}
	\item[a] El beneficio del monopolio y el excedente total disminuyen.
	\item[b] El beneficio del monopolio aumenta y el excedente total disminuye.
	\item[c] El beneficio del monopolio y el excedente total aumentan.
	\item[d] El beneficio del monopolio disminuye y el excedente total aumenta.
\end{enumerate}

\seccion{Test 2013}

\textbf{6.} Respecto al equilibrio de monopolio sin discriminación de precios, la discriminación de precios de tercer grado genera:
\begin{enumerate}
	\item[a] Una reducción del nivel de producción.
	\item[b] Un aumento del excedente del productor.
	\item[c] Un aumento del precio en todos los mercados.
	\item[d] Una reducción del excedente de consumidor en todos los mercados.
\end{enumerate}

\textbf{9.} Hay un Monopolio Natural solo si:
\begin{enumerate}
	\item[a] Los costes fijos son altos.
	\item[b] El coste medio es decreciente.
	\item[c] Los costes son subaditivos.
	\item[d] Hay una única empresa produce más de un bien.
\end{enumerate}

\seccion{Test 2011}

\textbf{11.} En un monopolio regulado con un coste fijo y costes marginales constantes, un precio sombra de los fondos públicos estrictamente mayor que uno, y  distintos grupos de consumidores, los precios de Ramsey implican que los precios regulados óptimos son tales que:
\begin{enumerate}
	\item[a] Los costes fijos se cargan sobre todo a los consumidores con demandas más inelásticas.
	\item[b] Los costes fijos se cargan sobre todo a los consumidores con demandas más elásticas.
	\item[c] Los costes fijos se cubren exclusivamente con una subvención del estado.
	\item[d] Los costes fijos se cargan por igual a los distintos tipos de consumidores.
\end{enumerate}

\seccion{Test 2009}
\textbf{8.} Suponga que un monopolista vende un bien homogéneo en dos ciudades distintas, cobrando precios $p_1$ y $p_2$ respectivamente. Las ciudades están geográficamente separadas, de forma que no es posible para los consumidores realizar arbitraje entre los dos mercados. Si las curvas de demanda en ambas ciudades son iguales, entonces el monopolista:
\begin{enumerate}
	\item[a] Puede aumentar sus ingresos y beneficios cobrando un precio distinto en cada ciudad.
	\item[b] Puede aumentar sus ventas elevando el precio $p_1$ y mantiendo constante $p_2$.
	\item[c] Debe discriminar precios entre los dos mercados, cobrando un precio distinto en cada ciudad para maximizar su beneficio.
	\item[d] Debe cobrar el mismo precio en las dos ciudades para maximizar su beneficio.
\end{enumerate}

\seccion{Test 2008}

\textbf{6.} Si una empresa que es monopolista en un mercado cuya demanda, $P(x)$, tiene pendiente negativa, y produce con unos costes $C(x)$, es falso que:
\begin{enumerate}
	\item[a] Maximizará beneficios si produce siendo la elasticidad de la demanda $|E_x| \geq 1$.
	\item[b] Si maximiza beneficios el cociente del precio sobre el coste marginal será positivo.
	\item[c] Si maximiza los ingresos se sitúa en el tramo inelástico de la demanda.
	\item[d] Si maximiza los beneficios la pendiente, en valor absoluto, del ingreso marginal es menor que la pendiente, en valor absoluto, del coste marginal.
\end{enumerate}

\textbf{11.} Considere un monopolista que abastece un mercado y puede discriminar precios.
\begin{enumerate}
	\item[a] El precio será más alto en el mercado con demanda más elástica, si hace discriminación de tercer grado.
	\item[b] La cantidad ofrecida será menor si discrimina perfectamente en ambos mercados que si no lo hace.
	\item[c] El precio fijo al que vende cuando hace discriminación perfecta es menor que si hace discriminación de tercer grado.
	\item[d] El precio marginal al que vende cuando discrimina perfectamente es inferior al de no discriminación.
\end{enumerate}

\seccion{Test 2007}

\textbf{10.} Un monopolio maximizador de beneficios abastece un mercado con una demanda $P(x)$ segmentada en dos submercados, y produce con unos costes $C(x)$:
\begin{enumerate}
	\item[a] Sea cual sea el tipo de discriminación que haga, sólo venderá en el mercado en el que tenga más consumidores.
	\item[b] Si realiza discriminación de precios de tercer grado: $\text{IMg}_1 = \text{IMg}_2 = \text{CMg}$.
	\item[c] Si realiza discriminación perfecta, para todas las unidades vendidas $P(x) = \text{CMg}$.
	\item[d] Si establece tarifas individualizadas $P_1\left( 1-\frac{1}{|E_1|} \right) = P_2 \left( 1 - \frac{1}{|E_2|} \right) = \text{CMg}$.
\end{enumerate}

\seccion{Test 2006}

\textbf{11.} En el contexto de un monopolista que abastece dos demandas diferenciadas y que puede hacer discriminación de precios, señale la respuesta correcta:
\begin{enumerate}
	\item[a] Si hace discriminación de primer grado o perfecta, el bienestar social del equilibrio coincide con el de competencia perfecta.
	\item[b] Si establece una tarifa en dos partes con cuota común no excluyente, el precio de equilibrio coincide con el coste marginal.
	\item[c] Si hace discriminación de tercer grado, establece un precio menor en el mercado con menor elasticidad de la demanda.
	\item[d] Siempre lanza al mercado una cantidad inferior a la de competencia perfecta.
\end{enumerate}

\seccion{Test 2005}
\textbf{3.} Una empresa monopolista se enfrenta a una elasticidad--precio de la demanda igual a 5 (en valor absoluto). Si tiene unos costes medios y marginales constantes e iguales a $c$, el precio de equilibrio que maximiza sus beneficios es:
\begin{enumerate}
	\item[a] $P=\frac{5c}{4}$
	\item[b] $P=\frac{4c}{5}$
	\item[c] $P=c$
	\item[d] Ninguno, ya que para que pueda maximizar sus beneficios, la elasticidad de la demanda tiene que ser inferior a 1 (en valor absoluto).
\end{enumerate}

\textbf{8.} Suponga un monopolista maximizador de beneficios cuyos costes medios y marginales de producir el bien X son de 1 u.m. Esta empresa abastece dos mercados cuyas funciones de demanda son $x_1 = \frac{1}{p_1^2}$ y $X_2 = \frac{2}{p_2^3}$. En esta situación, si realiza discriminación de tercer grado, los precios a los que venderá el producto en cada mercado serán:

\begin{itemize}
	\item[a] $p_1 = p_2 = 1$
	\item[b] $p_1 = 2$ y $p_2=3$.
	\item[c] $p_1 = p_2 = 1,65$
	\item[d] $p_1 = 2$ y $p_2=1,5$.
\end{itemize}

\seccion{Test 2004}
\textbf{6.} Considere a un monopolista que se enfrenta a una curva de demanda $P(x) = 100 -x$, donde $x$ es la cantidad que produce. Si su función de costes es $C(x) = 50x + 3$, y elige el precio que maximiza sus beneficios, la pérdida irrecuperable de eficiencia provocada por el monopolio es igual a:
\begin{enumerate}
	\item[a] 100
	\item[b] 312,5
	\item[c] 625
	\item[d] 1250
\end{enumerate}

\textbf{12.} Suponga que hay una regulación de un monopolio natural consistente en fijar el precio igual al coste total medio. Esta regulación:
\begin{enumerate}
	\item[a] Implica que el monopolista tiene pérdidas, por lo que habrá que subvencionar la producción.
	\item[b] Provoca una pérdida irrecuperable de eficiencia.
	\item[c] Incentiva al monopolista a disminuir sus costes totales medios.
	\item[d] Consigue que se produzca la misma cantidad que se produciría en competencia perfecta.
\end{enumerate}

\notas

\textbf{2018}: \textbf{8.} B

\textbf{2016}: \textbf{16.} D

\textbf{2015} \textbf{9.} B

\textbf{2014} \textbf{7.} D \textbf{9.} C

\textbf{2013} \textbf{6.} B \textbf{9.} C

\textbf{2011} \textbf{11.} A

\textbf{2009} \textbf{8.} D. Se trata de averiguar la cantidad $q$ para la que se cumple $C(q) < 2\cdot C(\frac{q}{2})$.

\textbf{2008} \textbf{6.} C \textbf{11.} D

\textbf{2007} \textbf{10.} B

\textbf{2006} \textbf{11.} A

\textbf{2005} \textbf{3.} A \textbf{8.} D

\textbf{2004} \textbf{6.} B. Se trata de comparar el excedente total del equilibrio con precio de monopolista, con un hipotético equilibrio en el que la empresa vende a coste marginal a pesar de sufrir pérdidas por la presencia del coste fijo. \textbf{12.} B

\bibliografia

Mirar en Palgrave:
\begin{itemize}
	\item Averch-Johnson effect
	\item contestable market
	\item durable goods and aftermarkets
	\item marginal and average cost pricing
	\item mechanism design
	\item mechanism design (new developments)
	\item monopoly
	\item monopsony
	\item price discrimination
	\item price discrimination (empirical studies)
	\item price discrimination (theory)
	\item profit and profit theory
	\item Ramsey pricing *
	\item regulation
	\item rent seeking
\end{itemize}


Armstrong, M. \textit{Recent Developments in the Theory of Regulation}. (2005) \url{http://www.econ.ucl.ac.uk/downloads/armstrong/reg.pdf} 137 págs. Exhaustivo y muy técnico.

Brown, C. \textit{Sustainable and Unsustainable Natural Monopoly} \url{http://www.clt.astate.edu/crbrown/eleven2.htm} [Explicación sencilla de la sostenibilidad de los monopolios naturales]

Economist, the. \textit{Schumpeter Centenary}. Nov 19th 1983 (En carpeta del tema)

Holakouee, P. \textit{Ramsey-Boiteux Pricing} (2011) \url{http://www.edegan.com/wiki/images/4/4d/Theory_of_Incentives_in_Procurement_and_Regulation.pdf} -- En carpeta del tema

Jehle, G.; Reny, P. \textit{Advanced Microeconomic Theory}. 4.2 Imperfect Competition 170-174

Joskow, P. L; \textit{Ch. 16 Regulation of Natural Monopoly} Handbook of Law and Economics vol. 2 (2007) En carpeta del tema

King, S. P. \textit{Principles of price cap regulation} (1997) Infrastructure Regulation and Market Reform: Principles and Practice : Selected Papers Prepared for The Utility Regulation Training Program Held 

Kreps, D. \textit{A Course in Microeconomic Theory} Ch. 9 Monopoly

Kwoka, J. E. \textit{The Role of Competition in Natural Monopoly: Costs, Public Ownership, and Regulation} (2006) \url{http://ocean.sci-hub.bz/1bbae26630223295035e7de8215450b2/10.1007\%40s11151-006-9112-x.pdf}

Mas-Colell, A; Whinston, M.; Green, J. \textit{Microeconomic Theory} (1995) Ch. 12 Market Power

Stigler, G. J. \textit{The Theory of Economic Regulation} (1971) \url{http://www.rasmusen.org/zg604/readings/Stigler.1971.pdf} En carpeta del tema. Seminal de la teoría de la regulación y precursor del Public Choice.

Tirole. J. \textit{The Theory of Industrial Organization}. [Págs. 307-309 para monopolios naturales y sostenibilidad]

Posner, R. A. \textit{Natural monopoly and its regulation}. (1999) En carpeta del tema. 130 págs.

Sharkey, W. (1982) \textit{The Theory of Natural Monopoly} Cambridge University Press -- En carpeta del tema

Varian, H. \textit{Microeconomic Analysis} Ch. 14 Monopoly.

Wang, H.; Yang, B. Z. (2001) \textit{Fixed and Sunk Costs Revisited} The Journal of Economic Education, Vol. 32, No. 2 -- En carpeta del tema


\end{document}
