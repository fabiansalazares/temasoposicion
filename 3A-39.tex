\documentclass{nuevotema}

\tema{3A-39}
\titulo{La inflación: sus causas y sus efectos sobre la eficiencia económica y el bienestar.}

\begin{document}

\ideaclave

Reformar con Gordon (2008) sobre historia de la Curva de Phillips

\seccion{Preguntas clave}

\begin{itemize}
	\item ¿Qué es la inflación?
	\item ¿Qué factores determinan el aumento de los precios?
	\item ¿Qué modelos teóricos explican la inflación?
	\item ¿Qué evidencia empírica existe al respecto?
	\item ¿Cómo se mide la inflación?
	\item ¿Qué efectos tiene la inflación sobre la eficiencia?
	\item ¿Qué efectos tiene la inflación sobre el bienestar social?
	\item ¿Existe una tasa de inflación óptima?
	\item Si existe, ¿de qué depende?
\end{itemize}

\esquemacorto

\begin{esquema}[enumerate]
	\1[] \marcar{Introducción}
		\2 Contextualización
			\3 Concepto de inflación
			\3 Inflación a lo largo de la historia
			\3 Causas de la inflación
			\3 Efectos de la inflación
		\2 Objeto
			\3 ¿Qué es la inflación?
			\3 ¿Qué factores determinan el aumento de los precios?
			\3 ¿Qué modelos teóricos explican la inflación?
			\3 ¿Qué evidencia empírica existe al respecto?
			\3 ¿Cómo se mide la inflación?
			\3 ¿Qué efectos tiene la inflación sobre la eficiencia?
			\3 ¿Qué efectos tiene la inflación sobre el bienestar social?
			\3 ¿Existe una tasa de inflación óptima?
		\2 Estructura
			\3 Medición de la inflación
			\3 Análisis positivo de la inflación
			\3 Efectos de la inflación
	\1 \marcar{Medición de la inflación}
		\2 Idea clave
			\3 Contexto
			\3 Objetivo
			\3 Resultados
		\2 Índices de precios
			\3 Idea clave
			\3 Índice del coste de la vida
			\3 Laspeyres
			\3 Paasche
			\3 Índices encadenados
			\3 Índice ideal de Fisher
		\2 Índices de precios utilizados en la práctica
			\3 IPC -- CPI
			\3 Deflactor del PIB
			\3 Índices de precios de productor
		\2 Problemas empíricos
			\3 Elección de bienes
			\3 Cambios en características
			\3 Precios implícitos
			\3 Mercados volátiles
	\1 \marcar{Causas de la inflación}
		\2 Análisis tradicional
			\3 Teoría cuantitativa del dinero
			\3 Wicksell: proceso acumulativo
			\3 Keynes
			\3 Phillips (1958)
			\3 Inflación demand-pull o de demanda
			\3 Inflación cost-push o de costes
			\3 Lipsey (1960)
			\3 Síntesis neoclásica
			\3 Monetarismo
			\3 Modelo de Cagan (1956)
		\2 Nueva Macroeconomía Clásica
			\3 Idea clave
			\3 Curva de oferta de Lucas
			\3 Hiperinflación con HER: Sargent y Wallace (1973)
			\3 Sesgo inflacionario de política monetaria
			\3 Modelo del Ciclo Real
			\3 Aritmética monetarista desagradable
			\3 Teoría fiscal del nivel de precios
		\2 Nueva Economía Keynesiana
			\3 Idea clave
			\3 Precios pegajosos de Fischer
			\3 Precios escalonados de Taylor
			\3 Precios à la Calvo
			\3 Modelos de negociación salarial
			\3 Curva de Phillips neo-keynesiana
			\3 Implicaciones
		\2 Economía abierta
			\3 Idea clave
			\3 Paridad de poder adquisitivo y tipo de cambio
			\3 Pass-through
	\1 \marcar{Efectos de la inflación}
		\2 Idea clave
			\3 Contexto
			\3 Objetivos
			\3 Resultados
		\2 Inflación y tipos de interés
			\3 Idea clave
			\3 Fisher y monetarismo
			\3 Keynesianismo
			\3 Valoración
		\2 Valor real de la deuda
			\3 Idea clave
			\3 Formulación
			\3 Implicaciones
			\3 Valoración
		\2 Eficiencia económica
			\3 Idea clave
			\3 Eficiencia del sistema de precios
			\3 Economía política
			\3 Efectividad de la indexación
			\3 Precios públicos
		\2 Economía abierta
			\3 Efecto sobre competitividad
			\3 Efecto Balassa-Samuelson
		\2 Inflación óptima
			\3 Regla de Friedman
			\3 Regla de Phelps
			\3 Divina coincidencia
			\3 Regla de Taylor
			\3 Inflación óptima en la ZLB
		\2 Deflación
			\3 Idea clave
			\3 Efectos
			\3 Evidencia empírica
	\1[] \marcar{Conclusión}
		\2 Recapitulación
			\3 Causas de la inflación
			\3 Análisis positivo de la inflación
			\3 Efectos de la inflación
		\2 Idea final
			\3 Política monetaria en crisis de COVID-19
			\3 Inflación en la zona euro
			\3 Curva de Phillips en la actualidad
			\3 Criptomonedas e innovaciones financieras

\end{esquema}

\esquemalargo


















\begin{esquemal}
	\1[] \marcar{Introducción}
		\2 Contextualización
			\3 Concepto de inflación
				\4 Laidler y Parkin (1975)
				\4[] La inflación es:
				\4[] $\to$ Un proceso de aumento continuo de precios
				\4[] $\to$ O un proceso de caída continuada del valor del dinero
			\3 Inflación a lo largo de la historia
				\4 Omnipresente desde que existe el dinero
				\4 Schwartz (1973)
				\4[] Historia compacta de la inflación
				\4 Episodios de inflación documentados
				\4[] $\to$ Ya en la Antigua Grecia
				\4 Periodos alternos de inflación y deflación
				\4[] $\to$ A lo largo de Edad Media
				\4[] $\then$ Hasta Gran Depresión en años 30
				\4 Posteriormente
				\4[] $\to$ Tendencia general a aumentos ligeros y continuos
				\4[] $\then$ Salvo años 70 y determinadas regiones
			\3 Causas de la inflación
				\4 Inflación presente en contextos heterogéneos
				\4[] Causas de la inflación no siempre las mismas
				\4[] Difícil extraer relaciones inequívocas
				\4[] $\to$ Diferentes teorías
				\4[] $\to$ Evidencia empírica interpretable
				\4 Definición de Friedman (1970)\footnote{``\textit{Inflation is always and everywhere a monetary phenomenon, in the sense that it is and can be produced only by a more rapid increase in the quantity of money than in output.''}.}
				\4[] La inflación es siempre y en todo momento
				\4[] $\to$ Un fenómeno monetario
				\4[] En el sentido de que:
				\4[] $\to$ Es y sólo puede ocurrir cuando
				\4[] $\to$ ...la cantidad de dinero crece más rápido que el output
				\4[] Enorme influencia posterior
				\4[] $\to$ Aunque relativamente disputada
			\3 Efectos de la inflación
				\4 Casos extremos de inflación
				\4[] Destrucción del sistema de precios
				\4[] $\to$ Efectos negativos evidentes
				\4 Inflación moderada
				\4[] Debate sobre efectos
				\4[] $\to$ ¿Son siempre indeseables?
				\4[] $\to$ ¿Cierto grado de inflación es deseable?
		\2 Objeto
			\3 ¿Qué es la inflación?
			\3 ¿Qué factores determinan el aumento de los precios?
			\3 ¿Qué modelos teóricos explican la inflación?
			\3 ¿Qué evidencia empírica existe al respecto?
			\3 ¿Cómo se mide la inflación?
			\3 ¿Qué efectos tiene la inflación sobre la eficiencia?
			\3 ¿Qué efectos tiene la inflación sobre el bienestar social?
			\3 ¿Existe una tasa de inflación óptima?
				\4 Si existe, ¿de qué depende?
		\2 Estructura
			\3 Medición de la inflación
			\3 Análisis positivo de la inflación
			\3 Efectos de la inflación
	\1 \marcar{Medición de la inflación}
		\2 Idea clave
			\3 Contexto
				\4 Nivel de precios como valor del dinero
				\4[] Nivel de precios es cantidad de dinero
				\4[] $\to$ Necesaria para adquirir una cesta representativa
				\4 Inflación como cambio en el valor del dinero
				\4[] Cambio en el valor en un periodo determinado
				\4[] $\to$ Cambio en precio de cesta representativa
				\4[] $\then$ Aumento de precio de cesta representativa
				\4[] $\then$ Caída en valor del dinero
			\3 Objetivo
				\4 Cuantificar la variación de los precios
				\4 ¿Qué bienes deben ser tenidos en cuenta?
				\4 ¿En qué proporciones deben valorarse los $\Delta$ precios?
				\4 ¿Qué periodo es relevante?
				\4 ¿Es relevante la pérdida de bienestar de los agentes?
			\3 Resultados
				\4 Varias medidas de variación de precios
				\4 No todas las medidas son estimables en la práctica
				\4 Medidas divergentes
				\4 Métodos prácticas de aproximación
		\2 Índices de precios
			\3 Idea clave
				\4 Teoría de números índices
				\4[] Cómo agregar variaciones en series numéricas
				\4[] $\to$ En una sola serie que cuantifique variación
				\4[] $\then$ Aplicación a índices de de precios
				\4 Economías compuestas de muchos mercados
				\4[] Constantes cambios de precios en muchos de ellos
				\4[] $\to$ ¿Cómo medir variación general?
				\4[] $\then$ ¿Qué inflación ha tenido lugar?
			\3 Índice del coste de la vida
				\4 Cuánto ha variado coste de mantener utilidad constante
				\4[] Dadas:
				\4[] $\to$ Utilidad constante
				\4[] $\to$ Precios pasados y presentes
				\4[] $\then$ ¿Cuánto ha variado gasto necesaria?
				\4 Formulación
				\4[] \fbox{$I = \frac{e(p_1, u_0)}{e(p_0, u_0)}$}
				\4[] Donde:
				\4[] $\to$ $e(\vec{p}, u)$: función de gasto
				\4 Medida teórica de variación de precios
				\4[] Utilidad constante no es medible
				\4[] Si lo fuese,
				\4[] $\to$ ¿cómo agregar utilidades de muchos agentes?
				\4[] $\then$ ¿Qué medida de comparación entre utilidades?
			\3 Laspeyres
				\4 Cuánto aumentó el precio de cesta dada
				\4[] $\to$ Inflación como relación entre precios de cestas
				\4 Formulación
				\4[] \fbox{$I_L = \frac{p_t \cdot  q_0}{p_0 \cdot q_0}$}
				\4 Sobreestima inflación
				\4[] Cuando precio de un bien aumenta mucho
				\4[] $\to$ Agentes sustituyen consumo por otros similares
				\4[] $\then$ Cae peso de bien encarecido en cesta
				\4[] Numerador no tiene en cuenta cambio en peso
				\4[] $\to$ Si lo tuviese, variación de índice sería menor
			\3 Paasche
				\4 Cuánto hubiese costado comprar cesta actual
				\4[] Con precios del periodo anterior
				\4[] $\to$ Inflación como relación entre precios de cestas
				\4 Cesta de bienes consumida en periodo presente
				\4 Formulación
				\4[] \fbox{$I_P = \frac{p_t \cdot q_t }{p_0 \cdot q_t }$}
				\4 Subestima inflación
				\4[] Cesta actual es resultado de sustitución
				\4[] $\to$ De bienes encarecidos por otros más baratos
				\4[] $\then$ Bienes encarecidos infraponderados
				\4[] Si precio de un bien hubiese sido más barato
				\4[] $\to$ Habrían consumido más de ese bien
				\4[] $\then$ Habría aumentado su ponderación
				\4[] $\then$ Pero cesta presente no lo considera
				\4[] Variación de coste de cesta subestima encarecidos
				\4[] $\to$ Por infraponderación tras sustitución
			\3 Índices encadenados
				\4 Especialmente relevante en Laspeyres
				\4 Cesta de consumo varía en cada periodo
				\4[] Consumidores adaptan a cambios en precios relativos
				\4 Cesta del periodo inicial
				\4[] Óptima dados precios iniciales
				\4 Cuánto más nos alejamos del periodo inicial
				\4[] Más cambios acumulados en precios relativos
				\4[] $\to$ Más diferente es la cesta óptima
				\4[] $\then$ Menos representativo de cambio en valor es el $I_\text{Laspeyres}$
				\4 Encadenamiento como solución parcial
				\4[] En cada periodo, se recalcula cesta
				\4[] No soluciona subestimación totalmente
				\4[] $\to$ Se produce en tanto hay cambio en peso intraperiodo
				\4 Ejemplo:
				\4[] Variación entre periodo inicial y periodo 3
				\4[] Inflación con Laspeyres sin encadenar
				\4[] $\to$ $I_L = \frac{p_3 \cdot q_0}{p_0 \cdot q_0}$
				\4[] Inflación con Laspeyres encadenado
				\4[] $\to$ $I_L = \frac{p_3 \cdot q_2}{p_2 \cdot q_2} \cdot \frac{p_2 \cdot q_1}{p_1 \cdot q_1} \cdot \frac{p_1 \cdot q_0}{p_0 \cdot q_0}$
			\3 Índice ideal de Fisher
				\4 Media geométrica de Laspeyres y Paasche
				\4 Formulación
				\4[] \fbox{$I_F = \sqrt{I_L \cdot I_P}$}
		\2 Índices de precios utilizados en la práctica
			\3 IPC -- CPI
				\4 Índice de Laspeyres
				\4 Cesta de bienes más consumidos
				\4 Incluye bienes importados
				\4[] Sujeto a fluctuación del TC y aranceles
				\4 Habitualmente encadenado
			\3 Deflactor del PIB
				\4 Índice de Paasche
				\4 Relación entre:
				\4[] PIB nominal presente
				\4[] Precio de producción presente en periodo anterior
			\3 Índices de precios de productor
				\4 Similar a IPC pero sobre inputs industriales
		\2 Problemas empíricos
			\3 Elección de bienes
				\4 Cestas de bienes consumidos varían en el tiempo
				\4[] Ante cambios en precios
				\4[] $\to$ Consumidores/productores ajustan cestas
				\4[] $\then$ Ponderaciones ya no reflejan consumo/producción
			\3 Cambios en características
				\4 Bienes mejoran características esenciales
				\4[] Manteniendo precios
				\4 Difícil medir cambios en características
				\4[] Ejemplo: actualización de software
				\4[] $\to$ Mejora de características esenciales
				\4[] $\to$ Precio constante
				\4[] $\then$ ¿Cuánto ha cambiado el precio?
			\3 Precios implícitos
				\4 No todos los bienes tienen precio explícito
				\4[] Ejemplo: packs, bundles, descuentos
			\3 Mercados volátiles
				\4 Algunos precios fluctúan mucho en periodos cortos
				\4 Criterios arbitrarios para tomar precio relevante
				\4[] Media en periodo considerado?
				\4[] Media móvil ponderada?
				\4[] Último precio?
				\4[] Precio a mitad de periodo?
	\1 \marcar{Causas de la inflación}
		\2 Análisis tradicional
			\3 Teoría cuantitativa del dinero
				\4 Martín de Azpilcueta, Bodino, Newton, Hume
				\4 Inflación depende de:
				\4[] -- Oferta de metal precioso
				\4[] -- Volumen de transacciones
				\4 Inflación puede tener efectos reales en el muy c/p
				\4[] Pero ajuste de precios es rápido
				\4[] $\to$ E impacto sobre variables reales desaparece
				\4 Dicotomía clásica
				\4[] Variaciones en cantidad de dinero
				\4[] $\to$ Afectan a nivel de precios
				\4[] $\to$ No afectan a variables reales
				\4 Irving Fisher formaliza relación\footnote{Donde $M' V'$ es la cantidad de dinero en forma de depósitos y similares al dinero. Evidentemente, aparece en este punto el debate sobre el grado de \textit{moneyness} de las diferentes formas de dinero y cuasi-dinero.}:
				\4[] $M \cdot V + M' V'= P \cdot T$
				\4[] Tomando output como proxy de transacciones
				\4[] $\to$ $M \cdot V + M' V' = P \cdot Y$
				\4 Formulación alternativa de Cambridge
				\4[] Énfasis en demanda de dinero
				\4[] $M = k\cdot P \cdot T = \frac{1}{V} P \cdot T$
				\4[] $\to$ Velocidad constante no depende de interés
				\4[] $\to$ Dinero como fracción de volumen de transacciones
				\4 Inspiración de monetaristas posteriores
				\4[] Matizando velocidad constante
				\4 Inspiración de Keynes posterior
				\4[] Considerando efecto de interés sobre demanda/velocidad
			\3 Wicksell: proceso acumulativo\footnote{Ver \href{https://www.hetwebsite.net/het/essays/money/cumulative.htm}{HET Website}.}
				\4 Contribución más conocida de Wicksell
				\4 Compatibilizar dos puntos en conflicto:
				\4[] $\to$ Teoría cuantitativa del dinero
				\4[] $\to$ Ley de Say
				\4[] Fisher afirma TCD+Ley de Say
				\4[] $\to$ Pero Wicksell explica que por sí solos, incompatibles
				\4[] $\then$ Necesario mecanismo intermedio
				\4[] $\then$ Proceso acumulativo
				\4 Teoría cuantitativa del dinero
				\4[] Aumentos exógenos de la oferta monetaria
				\4[] $\to$ Resultan en aumentos proporcionales de nivel de precios
				\4 Ley de Say
				\4[] No existen excesos agregados de demanda de bienes
				\4[] $\to$ Oferta ``crea'' demanda en términos de Mill
				\4[] $\to$ Demanda racionada por oferta
				\4[] Independientemente de todo fenómeno monetario
				\4[$\then$] ¿Cómo tienen entonces lugar los $\uparrow$ P de la TCD?
				\4 Wicksell trata de explicar
				\4[] Como resultado de interés monetario menor que natural
				\4 Si $r>i$
				\4[] Resultado o origen de expansión monetaria
				\4[] Posible pedir prestado a tipo $i$
				\4[] Invertir y obtener $r$
				\4[] $\to$ Extraer beneficio de inversión
				\4[] $\then$ Demanda de dinero aumenta
				\4 Pero inversión debe ser igual a ahorro
				\4[] En contexto de pleno empleo
				\4 Sin embargo, cuando $r<i$, $I > S$
				\4[] Aumento incompleto de oferta porque pleno empleo
				\4[] ED de bienes de capital provoca
				\4[] $\to$ Aumento de producción de bienes de capital
				\4[] $\to$ Presión sobre precios de bienes de capital
				\4[] $\then$ Repercutidos a bienes de consumo
				\4[] $\then$ Aumento de nivel de precios
				\4[$\then$] Explicación de ocurrencia de TCD
				\4[$\then$] Dicotomía clásica no se cumple en corto plazo
				\4[$\then$] Dicotomía clásica se mantiene en largo plazo
				\4[$\then$] Inflación es fenómeno \textit{real}
				\4 Proceso acumulativo
				\4[] Si interés monetario por debajo de natural
				\4[] $\to$ Proceso inflacionario acumulativo
				\4 ¿Cuando acaba el proceso acumulativo?
				\4[] Bancos no tienen reservas para seguir prestando
				\4[] Dos opciones:
				\4[] i. Dejan de prestar
				\4[] $\to$ Demanda de crédito racionada
				\4[] $\to$ Cae demanda de inversión
				\4[] $\then$ Se reduce presión $I>S$
				\4[] ii. Aumenta el interés monetario
				\4[] $\to$ Dejan de prestar
				\4[] $\then$ Se reduce presión $I>S$
				\4[] $\to$ Piden prestados fondos en mercado monetario
				\4[] $\then$ Banco central no inyecto liquidez
				\4[] $\then$ Aumenta interés monetario
				\4 Origen de conceptos 'naturales' posteriores
				\4[] Tasa de desempleo natural
				\4[] Output tendencial
				\4[] Output gap
			\3 Keynes
				\4 Tratado sobre la Reforma Monetaria (1923)
				\4[] Principal análisis de la inflación
				\4[] Se adhiere a TCDinero
				\4[] $\to$ Crecimiento de oferta monetaria como causa
				\4[] Valora ventajas de inflación
				\4[] $\to$ Reducción de salario real
				\4[] $\to$ Aumento de beneficios empresariales
				\4[] Compara con inconvenientes
				\4[] $\to$ Distorsión del sistema de precios
				\4[] $\to$ Reducción de valor del capital
				\4[] $\to$ Acepta a Lenin: inflación destruye sistema
				\4 Introduce expectativas como causa de inflación
				\4[] Inflación esperada genera inflación
				\4 Acepta impuesto inflacionario
				\4[] Precede análisis monetarista posterior
				\4[] Inflación como ``método tributario''
				\4 Tratado sobre el Dinero (1930)
				\4[] Distinción entre dos tipos de inflación
				\4[] -- Inflación de beneficios
				\4[] $\to$ Aumento de beneficio por unidad de output
				\4[] $\then$ Precios crecen más rápido que costes
				\4[] $\then$ Favorece acumulación de capital
				\4[] $\then$ Estímulo a inflación de producción
				\4[] $\then$ Beneficiosa para crecimiento económico
				\4[] -- Inflación de rentas
				\4[] $\to$ Aumento de coste por unidad de output
				\4[] $\then$ Proceso de aumento de salarios
				\4[] $\then$ Aumento endógeno a demanda de trabajo
				\4[] $\then$ Distinta a inflación de costes/cost-push
				\4[] Inflación de beneficios como motor de progreso
				\4[] $\to$ Tarde o temprano, inflación pasa a ser de ingreso
				\4[] $\then$ Tiempo hasta transición determina crecimiento
				\4[] $\then$ Cuanto más tarde en ser de ingreso, mejor
				\4 Teoría General sobre el Empleo, el Interés y El Dinero (1936)
				\4[] Inflación puede servir para bajar salarios reales
				\4[] Pigou: deflación aumenta saldos reales
				\4[] $\to$ Aumento de saldos reales aumenta riqueza
				\4[] $\to$ Aumento de riqueza aumenta DA
				\4[] $\then$ Pero en la práctica, deprime actividad
				\4[] $\then$ Empresas ingresan menos
				\4 Keynesianismo
				\4[] Decantación de ideas keynesianas
				\4[] Especialmente en años 40, 50 y 60
				\4[] Inflación como herramienta para:
				\4[] $\to$ bajar salarios reales
				\4[] $\to$ Mantener bajo el coste de la deuda pública
				\4[] Pasiva ante otras herramientas más efectivas
				\4[] $\to$ Estímulos a demanda agregada
				\4[] $\to$ Política monetaria acomodaticia
			\3 Phillips (1958)
				\4 Idea clave
				\4[] Estimar empíricamente
				\4[] $\to$ Relación inflación salarial y empleo
				\4[] Diferentes fases históricas
				\4[] $\to$ Valorar robustez de relación
				\4[] Proponer explicación de relación
				\4[] $\to$ ¿Exceso de demanda?
				\4[] $\to$ ¿Presión de costes sobre salario?
				\4 Formulación
				\4[] Dos posibles explicaciones de $\Delta$ Salarios
				\4[] $\to$ Inflación de demanda: \textit{demand pull}
				\4[] $\then$ Competencia de empresas por factor trabajo
				\4[] $\then$ Relación positiva $\Delta$ salarios-trabajo
				\4[] $\to$ Inflación de costes: \textit{cost-push}
				\4[] $\then$ Shock externo aumenta coste de la vida
				\4[] $\then$ Trabajadores exigen mantener poder adquisitivo
				\4[] $\then$ Relación negativa salarios-trabajo
				\4[] Conjuntos de datos de dos fases:
				\4[] $\to$ 1861-1913
				\4[] $\to$ 1913-1948
				\4[] Estimación de curvas de aproximación
				\4[] Estimación de regresiones $\Delta$ salarios-paro
				\4 Implicaciones
				\4[] Relación decreciente persistente
				\4[] $\to$ Desempleo
				\4[] $\to$ Aumento de los salarios
				\4[] Relación asimétrica $\Delta$ salarios-paro
				\4[] $\to$ Menos paro aumenta más el $\Delta$ salarios
				\4[] $\to$ Más paro reduce más el $\Delta$ salarios
				\4[] Tasa de variación de salarios
				\4[] $\to$ Puede explicarse con desempleo
				\4[] $\then$ En mayoría de casos
				\4[] $\then$ Salvo que haya inflación de costes
				\4[] Inflación de demanda
				\4[] $\to$ Empresas compiten por factor trabajo
				\4[] Inflación de costes
				\4[] $\to$ Aumento de precios de bienes de importación
				\4[] $\then$ Aumento de salarios sin efecto sobre paro
				\4[] $\then$ Suceso relativamente raro
				\4 Valoración
				\4[] Artículo seminal
				\4[] Abre programa de investigación
				\4 Phillips (1958) encuentra relación empírica
				\4[] Diferentes periodos desde 1861 a 1957
				\4[] Regresión $\dot{w}$ contra $u_t$
				\4[] \fbox{$\ln (\delta w_t) = \beta_0 + \beta_1 \ln u_t$}
				\4[] $\to$ Coeficiente $\beta_1$ negativo
				\4[] $\to$ Inflación salarial > 0 para desempleo positivo
				\4[$\Rightarrow$] Postula rel. negativa entre $u$ y $\dot{w}$
				\4[$\Rightarrow$] Inflación salarial aún en presencia de desempleo
				\4[] \grafica{curvadephillips}
			\3 Inflación demand-pull o de demanda
				\4 Inflación causada por demanda elevada
				\4 Desplazamiento de DAgregada a la derecha
				\4[] Aumento de demanda agregada dados precios
				\4[] Generalmente, resultado de déficit público
				\4 Capacidad productiva utilizada al máximo
				\4[] Se demandan más output del que puede producirse
				\4 Ejemplos
				\4[] Estados Unidos en años 60
				\4[] $\to$ Fuerte aumento por Guerra de Vietnam
				\4[] $\to$ Bajadas de impuestos
				\4[] $\then$ Inflación
				\4 Críticas
				\4[] Realmente, son tipos de interés fijos los que causan
				\4[] Aumento del déficit debería aumentar interés
				\4[] $\to$ Crowd-out de inversión y gasto privado
				\4[] Si interés se mantiene constante por BCentral
				\4[] $\to$ Sin crowding out
				\4[] $\to$ Dinero crece más que output
				\4[] $\then$ Dinero es verdadera causa de inflación
			\3 Inflación cost-push o de costes
				\4 Inflación resulta de shock de oferta
				\4 Curva de oferta agregada se desplaza a izquierda
				\4[] Menor producción disponible a mismos precios
				\4 Shock se propaga a precios
				\4 Posible persistencia de la inflación
				\4[] Marco de negociación salarial es determinante
				\4[] Salarios indexados a inflación desatan espiral
				\4[] $\to$ Sindicatos fuerzan aumentos del salario
				\4[] $\to$ Empresas con poder de mercado suben precios
				\4[] $\then$ Carrera por no perder poder adquisitivo relativo
				\4[] $\then$ Espiral inflacionaria en movimiento
			\3 Lipsey (1960)
				\4 Idea clave
				\4[] Curva de Phillips a la derecha de ordenadas
				\4[] $\to$ Siempre hay paro
				\4[] $\to$ Imposible alcanzar paro 0 aun con inflación
				\4[] Incompatible con modelo walrasiano de mercado de trabajo
				\4 Paro friccional como explicación
				\4[] Desempleados no encuentran vacantes
				\4[] Vacantes no encuentran desempleados
				\4[] $\to$ Paro friccional
				\4 Pleno empleo
				\4[] No implica desempleo 0
				\4[] Entendido como ausencia de desempleo involuntario
				\4[] $\to$ Puede implicar paro positivo
				\4 Implicaciones
				\4[] Racionaliza relación observada por Phillips
				\4[] Exceso de demanda de trabajo empuja precios
				\4[] $\to$ Demand-pull
				\4[] Extrapola relación a inflación-desempleo
				\4[] $\to$ Inflación salarial como proxy de inflación
			\3 Síntesis neoclásica
				\4 Samuelson y Solow (1960)
				\4[] Proponen curva de Phillips como menú a c/p
				\4[] Gobiernos pueden elegir a c/p combinación $u$--$\pi$
				\4[] Decisión de gobiernos puede tener efectos a l/p
				\4[] $\to$ Adelanta idea de histéresis
				\4[] Debate CPhillips como herramienta para distinguir
				\4[] $\to$ Inflación demand-pull
				\4[] $\to$ Inflación cost-push
			\3 Monetarismo
				\4 Dinero es causa de inflación
				\4[] TCD es referente principal
				\4[] $\to$ Velocidad estable
				\4 Velocidad del dinero
				\4[] Depende de muchos factores
				\4[] Incluye:
				\4[] $\to$ Interés de corto plazo
				\4[] $\to$ Renta permanente
				\4[] $\to$ Interés de largo plazo
				\4[] $\to$ Interés del dinero
				\4[] $\to$ Expectativa de inflación...
				\4 Friedman y Schwartz
				\4[] Historia monetaria de los Estados Unidos
				\4[] ``Inflación es siempre y en todas partes...\footnote{Friedman (1970): <<\textit{Inflation is always and everywhere a monetary phenomenon in the sense that it is and can be produced only by a more rapid increase in the quantity of money than in output.}>> }
				\4[] ... un fenómeno monetario en el sentido de que...
				\4[] -..resulta de crecimiento de oferta de dinero
				\4[] $\to$ Más rápida que crecimiento de output''
				\4 Muy fuerte influencia en literatura posterior
				\4[] Dinero y política monetaria
				\4[] $\to$ Instrumento central de política económica
				\4 Política monetaria
				\4[] Causa inflación en último término
				\4[] Puede frenar espiral inflacionaria
				\4 Inflación > inflación esperada puede aumentar output
				\4[] Trabajadores estiman salarios reales más altos
				\4[] $\to$ Estimando inflación extrapolando pasado
				\4[] Realmente, precios más altos reducen SReal
				\4[] $\to$ Más oferta de trabajo
				\4[] $\to$ Más demanda y más output
				\4 Inflación depende de expectativas
				\4[] Trabajadores estiman $\pi$ dada $\pi$ pasada
				\4[] Posible estimular output vía inflación creciente
				\4[] $\to$ Trabajadores estimarán salario real más alto
				\4[] $\then$ Realmente, el salario será más bajo
				\4 Evolución de inflación es fenómeno dinámico
				\4[] Curva de Phillips no vertical en corto plazo
				\4[] En largo plazo, tiende a verticalidad
				\4[] Expectativas de inflación causan inflación presente
				\4[] Explotar deliberadamente relación precios-output
				\4[] $\to$ Para mantener output por encima de equilibrio
				\4[] $\then$ Requiere cada vez mayor inflación
				\4[$\then$] Paro por debajo de natural requiere inf. creciente
				\4[$\then$] Inflación tiene elevados costes de eficiencia
			\3 Modelo de Cagan (1956)\footnote{Extraído de Edmond, C.}
				\4 Idea clave
				\4[] Contexto
				\4[] $\to$ Análisis dinámico
				\4[] $\to$ Hiperinflaciones pre y post-II GM
				\4[] $\to$ Expectativas adaptativas
				\4[] $\to$ Monetarismo
				\4[] Objetivo
				\4[] $\to$ Explicar hiperinflaciones
				\4[] $\to$ Relacionar con aumento de oferta monetaria
				\4[] $\to$ Relacionar con impuesto inflacionario
				\4[] Resultados
				\4[] $\to$ Hiperinflaciones posibles sin gasto explosivo
				\4[] $\to$ A partir de cierto $\Delta M_t$, HEA causa inflación
				\4 Formulación
				\4[] Partiendo de $m_t - p_t = -\alpha i_t + y_t$
				\4[(1)] $m_t - p_t = -\alpha \pi^e_t$
				\4[(2)] $\pi_t^e = \lambda \pi_{t-1}^e + (1-\lambda) (p_t - p_{t-1})$
				\4[$\to$] $\alpha$: sensibilidad de la dda. de dinero al i.nominal
				\4[$\to$] $\lambda$: inercia de la estimación de la inflación
				\4 Solución: ecuación de precio en t a partir de:
				\4[(i)] Política monetaria del gobierno ($m_t$)
				\4[(ii)] Precios pasados
				\4[(iii)] Parámetros $\alpha$ y $\lambda$
				\4[] $\then$ $p_t = \beta_1 p_{t-1} + \beta_2 m_t - \beta_3 m_{t-1}$
				\4 Hiperinflación
				\4[] Determinados parámetros inducen senda divergente
				\4[] Crecimiento explosivo de $p_t$
				\4 Implicaciones
				\4[] Estimaciones de inflación basadas en inflación pasada
				\4[] $\to$ Pueden mantener dinámica inflacionaria en movimiento
				\4[] Gobierno
				\4[] $\to$ No es necesario crecimiento explosivo de la deuda
				\4[] $\then$ Para aumento explosivo de la inflación
				\4[] $\to$ Dada expectativa de inflación elevada
				\4[] $\then$ Necesitan más inflación para recaudar
				\4[] Con expectativas adaptativas sobre inflación
				\4[] $\to$ Expectativa de inflación basada en inflación pasada
				\4[] $\to$ Estimación miope de inflación
				\4[] $\then$ Mayor demanda de dinero
				\4[] $\then$ Gobierno puede superar ingreso con inflación creciente
				\4[] $\then$ Posible dinámica hiperinflacionaria
				\4[] Recomendaciones de política económica:
				\4[] $\to$ Evitar monetización
				\4[] $\to$ Anclar expectativas de inflación
				\4 Valoración
				\4[] Adaptación posterior a HER en Sargent y Wallace (1972)
				\4[] Mejora sustantiva de comprensión de hiperinflación
				\4[] $\to$ No sólo déficits+monetización son culpables
				\4[] $\to$ Dinámicas endógenas pueden provocar
		\2 Nueva Macroeconomía Clásica
			\3 Idea clave
				\4 Contexto
				\4[] Microfundamentación de fenómenos macroeconómicos
				\4[] Hipótesis de expectativas racionales
				\4[] Modelos dinámicos
				\4 Objetivo
				\4[] Explicar interacción entre inflación y output
				\4[] $\to$ Con agentes optimizadores que estiman con HER
				\4 Resultados
				\4[] Varias causas posibles de inflación
				\4[] Expectativas e incentivos tienen papel central
				\4[] Primeros modelos
				\4[] $\to$ Aumento de oferta monetaria
				\4[] Teorías fiscales de la inflación
				\4[] $\to$ En último término, política fiscal causa inflación
			\3 Curva de oferta de Lucas
				\4 Agentes conocen modelo subyacente de economía
				\4 No pueden distinguir entre:
				\4[] $\to$ Aumento del nivel general de precios
				\4[] $\to$ Aumento de demanda de trabajo que venden
				\4 Estiman
				\4[] Nivel de precios general a partir de:
				\4[] $\to$ Modelo completo de la economía
				\4[] $\to$ Volatilidad de shocks nominales y reales
				\4[] $\to$ Precios observados en su mercado de trabajo
				\4[] $\then$ $E(P_t | P^z_t, I_{t-1}) = \theta P^z_t + (1-\theta) E(P_t | I_{t-1})$
				\4[] \quad \quad $\theta=\frac{\sigma^2}{\sigma^2 - \tau^2}$
				\4[] \quad \quad $\sigma^2$: varianza de shocks nominales
				\4[] \quad \quad $\tau^2$: varianza de shocks reales
				\4 Comparan
				\4[] Con precios que reciben por su output
				\4 Ofrecen oferta de trabajo
				\4[] Dada relación precios observados y estimados
				\4 Curva de oferta de Lucas
				\4[] $y_t = y_t^n + \lambda (\pi_t - \pi_t^e )+\epsilon_t$
				\4 Inflación sí tiene efectos sobre output
				\4[] Sorpresas de inflación aumentan oferta de trabajo
				\4[] $\to$ Relación estadística output-precios
				\4[] $\then$ Curva de Phillips
				\4[] Pero relación no es explotable a voluntad
				\4 Relación inflación-output no es explotable
				\4[] Agentes conocen parámetros del modelo
				\4[] $\to$ Varianza de shocks nominales
				\4[] Intento de explotar consistentemente la relación
				\4[] $\to$ Agentes reconocen shocks como nominales
				\4[] $\then$ Toman observados como nominales
				\4[] $\then$ Efecto cada vez menor de inflación inesperada
			\3 Hiperinflación con HER: Sargent y Wallace (1973)
				\4 ``Hyperinflation and rational expectations''
				\4 Idea clave
				\4[] Influencia de Cagan (1956) sobre hiperinflación
				\4[] $\to$ Agentes esperan inflación dada inflación pasada
				\4[] $\to$ Gobierno necesita más M para financiar $g$
				\4[] $\then$ Hiperinflación posible sin déficit púb. desbocado
				\4[] $\then$ Por inercia de expectativas adaptativas
				\4[] Incipiente generalización de expectativas irracionales
				\4[] ¿Es posible hiperinflación con HER?
				\4 Formulación
				\4[] Gobierno financia déficit creando M
				\4[] Agentes demanda dinero en función de inflación
				\4[] $\to$ Asumiendo ec. de Fisher
				\4[] $\to$ Asumiendo dda. dinero en fción. de $y$ e $i$
				\4[] Estimación de inflación con HER
				\4[] $\to$ Teniendo en cuenta senda de déficit del gobierno
				\4[] $\to$ Teniendo en cuenta aumento de M e inflación pasadas
				\4[] Espiral hiperinflacionaria
				\4[] $\to$ Agentes estiman inflación
				\4[] $\to$ Gobiernos necesitan mas M para financiar dada $\pi$
				\4[] $\then$ Hiperinflación
				\4 Implicaciones
				\4[] Expectativas adaptativas pueden resultar de HER
				\4[] Hiperinflación posible sin supuesto ad-hoc
				\4[] $\to$ HEA resulta de HER
				\4[] Mejor estimación racional de aumento de M
				\4[] $\to$ Resulta de
				\4 Valoración
				\4[] Hiperinflaciones pueden ser racionales
				\4[] $\to$ No es necesario postular expectativas miopes
				\4[] Aplicación de HER más allá de oferta de trabajo
			\3 Sesgo inflacionario de política monetaria
				\4 Kydland y Prescott (1977)
				\4[] Primer análisis explícito de inconsistencia
				\4[] HER+función de pérdida inflación-output
				\4[] $\then$ Incentivos a explotar relación
				\4[] $\then$ Agentes conocen incentivos
				\4[] $\then$ ENEP no es óptimo de Pareto
				\4 Barro y Gordon (1983a) y (1983b)
				\4 Incentivos de bancos centrales causan inflación
				\4 Asumiendo curva de oferta de Lucas
				\4[] $y_t = y_t^n + \lambda (\pi_t - \pi_t^e )+\epsilon_t$
				\4[] $\to$ $y_t^n$: output sin ``sorpresa'' inflacionaria
				\4 Bancos centrales minimizan función de pérdida
				\4[] Más perdida con:
				\4[] $\to$ Más inflación
				\4[] $\to$ Más desviación respecto de output óptimo
				\4[] $\underset{\pi_t,y_t}{\min} \quad \sum_{t=0}^\infty \beta^t \left[ \pi_t^2 + \omega (y_t - y_t^*)^2 \right]$
				\4[] $\to$ $y_t^*$: output de óptimo de Pareto
				\4 Si output eficiente igual a output sin sorpresa
				\4[] Único objetivo de BCentral es minimizar inflación
				\4[] $\to$ Sin sesgo inflacionario
				\4 Si output eficiente es mayor a output sin sorpresa
				\4[] BCentral tiene dos objetivos incompatibles
				\4[] $\to$ Minimizar inflación
				\4[] $\to$ Aumentar output mediante sorpresa $\pi$
				\4 Agentes conocen función de pérdida de BCentral
				\4[] Esperan inflación no nula dados incentivos de BC
				\4[] $\to$ Necesaria mayor sorpresa inflacionaria para $\uparrow$ $y_t$
				\4[] $\then$ Sesgo inflacionario
				\4 Representación gráfica
				\4[] \grafica{kydlandprescott1977}
				\4[$\then$] Inflación causada por incentivos de BC
				\4[$\then$] Reputación del BCentral influye en inflación
				\4[] En contexto dinámico
				\4[] Folk-theorem aplicado al juego BCentral--agentes
				\4[] Estrategia disparador
				\4[] $\to$ Desviación de inflación induce represalias
				\4[$\then$] Reglas vinculantes pueden ser deseables
				\4[$\then$] Reputación como herramienta de compromiso
			\3 Modelo del Ciclo Real
				\4 Dicotomía clásica se cumple perfectamente
				\4[] Vars. reales y nominales independientes
				\4 Sin demanda de dinero ni precios rígidos
				\4[] Dinero es sólo un token
				\4[] $\to$ Ajuste inmediato de precios
				\4[$\then$] Inflación es fenómeno monetario
				\4[$\then$] Inflación no afecta a variables reales
			\3 Aritmética monetarista desagradable
				\4 Sargent y Wallace (1981)
				\4 Contexto de dominancia fiscal
				\4[] Gobiernos no pueden incurrir en default
				\4[] Pagan principales debidos en todo caso mediante
				\4[] $\to$ Impuestos fiscales
				\4[] $\to$ Impuestos inflacionarios
				\4[] En principio, financiación ortodoxa del déficit
				\4[] En último término, monetización de la deuda
				\4 Política fiscal causa inflación
				\4[] Asumiendo $r> g$ y déficit
				\4[] Stock de deuda crece en cada periodo
				\4[] Cuando se alcance límite de absorción de deuda
				\4[] $\to$ Gobierno deberá monetizar
				\4[] Agentes conocen restricción presupuestaria de gobierno
				\4[] $\to$ Saben de eventual monetización
				\4[] $\then$ Estiman inflación futura más elevada
				\4[] $\then$ Más monetización será necesaria
				\4 Aritmética desagradable
				\4[] Dada senda fiscal deficitaria exógena
				\4[] PM contractiva en presente
				\4[] $\to$ Agentes estiman inflación más alta en el futuro
				\4[] $\then$ Provoca inflación más elevada en presente
				\4[] Monetización presente y menos aumento de deuda
				\4[] $\to$ Reduce de hecho inflación
				\4[$\then$] Política fiscal puede causar inflación
				\4[$\then$] Financiación del déficit es causa de inflación
				\4[$\then$] Monetarización después causa inflación hoy
			\3 Teoría fiscal del nivel de precios
				\4 Inflación es fenómeno fiscal
				\4[] ¿Huevo o gallina?
				\4[] ¿Monetaria o fiscal?
				\4[] $\to$ TFP: PF $\to$ PM $\to$ Inflación
				\4 Dadas:
				\4[] Senda exógena de superávits primarios
				\4[] Stock de deuda inicial
				\4 Restricción presupuestaria
				\4[] $\frac{B_t}{P_t} = \text{Valor presente de superávits}$
				\4 Senda de precios se ajusta para cumplir RP
				\4[] Si superávits no son suficientes
				\4[] $\to$ Inflación para reducir valor real de deuda
				\4 Dinero sí influye en inflación
				\4[] PM expansiva aumenta interés nominal
				\4[] Interés nominal aumenta stock nominal de deuda
				\4[] $\to$ Precios deben aumentar
				\4[] $\then$ Inflación con origen fiscal
				\4 Críticas
				\4[] Evidencia empírica desfavorable
				\4[] $\to$ Déficit inesperados no parece provocar $\uparrow$ $\pi_t$
				\4[] Existen otras formas de determinar senda $M_t$ y $P_t$
				\4[] $\to$ P.ej.: regla de Taylor sobre interés
		\2 Nueva Economía Keynesiana
			\3 Idea clave
				\4 Contexto
				\4[] Modelos de equilibrio general walrasiano
				\4[] Microfundamentación+HER
				\4[] Fundamentación microeconómica de rigideces nominales
				\4[] $\to$ Causa y efecto de inflaciones
				\4[] Modelos de competencia monopolística
				\4[] $\to$ Dixit y Stiglitz (1977)
				\4 Objetivo
				\4[] Explicar efectos de inflación
				\4[] $\to$ Sobre variables reales
				\4[] $\to$ A menudo, causas de inflación postuladas
				\4[] Microfundamentar dinámicas inflacionarias
				\4[] $\to$ Persistencia
				\4[] $\to$ Correlación de inflación futura con PIB
				\4 Resultados
				\4[] Inflación como ajuste a mark-up de equilibrio
				\4[] Políticas de demanda y oferta pueden provocar
				\4[] Reconcilia marco de eq. general
				\4[] $\to$ Con modelo keynesiano
			\3 Precios pegajosos de Fischer
				\4 Ajuste de precios sólo posible cada cierto tiempo
				\4 Empresas fijan precios para varios periodos
				\4 Inflación es ajuste hacia precio deseado
				\4 Estímulo nominal tiene efectos reales
				\4[] Equilibrio similar a curva de oferta de Lucas
			\3 Precios escalonados de Taylor
				\4 Fracción de empresas cambia precios
				\4[] Diferentes fracciones cambian encada periodo
				\4 Dinámicas de inflación más complejas que Fischer
				\4[] Fracciones valoran cambios que harán las otras
				\4 Buen ajuste empírico a correlaciones PIB-inflación
				\4[] Inflación futura y PIB presente
				\4[] $\to$ Relación positiva
				\4[] Inflación pasada y PIB presente
				\4[] $\to$ Relación negativa
				\4 Persistencia en el nivel de precios
				\4[] Diferencia clave frente a precios pegajosos y Lucas
				\4[] Shocks se transmiten más allá de periodo
				\4[] $\to$ Cuanto menor sea fracción que cambia, mayor efecto
			\3 Precios à la Calvo
				\4 \% de empresas aleatorias pueden ajustar precios
				\4 Coste marginal asumido pro-cíclico
				\4 Inflación como ajuste hacia margen óptimo
				\4[] Aumento de demanda
				\4[] $\to$ Aumenta costes marginales
				\4[] $\to$ Sólo fracción de empresas puede ajustar precios
				\4[] $\then$ Compresión de márgenes iniciales
				\4[] $\then$ Aumento de precios progresivo para restablecer
			\3 Modelos de negociación salarial
				\4 Layard y Nickell (1985), (1986)
				\4[] Simplificado por Carlin y Soskice (1990)
				\4 Sindicatos tienen poder de mercado en trabajo
				\4 Empresas tienen poder de mercado en bienes
				\4 Negociación similar a monopolio bilateral
				\4[] Obvia posibles diferentes niveles de negociación
				\4 Inflación como dinámica de precios
				\4[] Que compatibiliza demandas de:
				\4[] $\to$ Sindicatos
				\4[] $\to$ Empresas
				\4 Shock de demanda que aumenta dda. de trabajo
				\4[] Mayor demanda de trabajo
				\4[] $\to$ Mayor demanda salarial de sindicatos
				\4[] $\then$ Desajuste salario real demandado y ofrecido
				\4 Espiral inflacionaria
				\4[] Ambas partes tratan de mantener sus rentas
				\4[] $\to$ Sindicatos aumentan salario
				\4[] $\to$ Empresas aumentan precios
				\4[] $\then$ Retroalimentación
				\4[] $\then$ Espiral de inflación
				\4[] $\then$ Resultados cualitativamente similares a monetarismo
				\4[$\then$] Inflación como resultado de demandas incompatibles
			\3 Curva de Phillips neo-keynesiana
				\4 Modelos de segunda generación de NEK
				\4 Agentes con HER
				\4 Consumidores demandan variedades de bien de consumo
				\4 Productores tienen poder de mercado en su variedad
				\4[] Fijan precio como mark-up sobre coste
				\4 Estiman senda futura de:
				\4[] $\to$ Shocks de oferta
				\4[] $\to$ Shocks de demanda
				\4[] $\to$ Costes
				\4[] $\to$ Precios
				\4 Equilibrio en estado estacionario
				\4[DIS] IS dinámica
				\4[] \fbox{$\tilde{y}_t = \textrm{E}_t \left\lbrace \tilde{y}_{t+1} \right\rbrace - \frac{1}{\sigma} \left( \underbrace{i_t - \textrm{E}_t \left\lbrace \pi_{t+1} \right\rbrace}_{r_t} - r^n_t \right) $}
				\4[NKPC] Curva de Phillips Neo-Keynesiana
				\4[] \fbox{$\pi_t = \text{E}_t \left\lbrace \pi_{t+1} \right\rbrace + \textsc{k} \tilde{y}_t $}
				\4[TR] Regla de Taylor simple
				\4[] \fbox{$i_t = \rho + \phi_\pi \pi_t + \phi_y \tilde{y}_t + v_t $}
				\4[MP] Mercado de dinero
				\4[] \fbox{$m_t - p_t = y_t - \eta i_t$}
			\3 Implicaciones
				\4 Inflación tiene efectos reales
				\4[] Aunque los agentes la esperen
				\4[] $\to$ Rigideces nominales
				\4 Desviación respecto a mark-up de equilibrio
				\4[] Genera movimiento en índices de precios
		\2 Economía abierta
			\3 Idea clave
				\4 Precio relativo de bienes internacionales
				\4[] Depende de:
				\4[] $\to$ Tipo de cambio nominal
				\4[] $\to$ Niveles de precios nominales relativos
				\4 Inflación altera niveles de precios relativos
				\4 TCN con dinámicas propias
				\4[] No siempre tendentes a ajustarse a PPA
			\3 Paridad de poder adquisitivo y tipo de cambio
				\4 Paridad de Poder Adquisitivo
				\4[] A medio camino entre:
				\4[] $\to$ Supuesto teórico
				\4[] $\to$ Regularidad empírica
				\4[] Cumplimiento aproximado
				\4[] $\to$ Aunque dependiente del contexto
				\4[] Tipo de Cambio Real constante
				\4[] $\to$ Mismo valor de bienes doméstico o extranjero
				\4[] $\then$ $S_t \cdot P^*_t = P_t$
				\4[] (donde $S_t$ es TCN directo)
				\4 Aumento de inflación en un país
				\4[] Necesita:
				\4[] $\to$ Depreciación de país con inflación
				\4[] $\to$ Apreciación de país sin inflación
				\4[] Pero mercado de divisas tiene otros determinantes
				\4[] $\to$ Cuenta financiera
				\4[] $\to$ Expectativas
				\4[] $\to$ Intervención de bancos centrales
				\4[] ...
				\4[] $\then$ Posibles desviaciones de PPA
				\4[] $\then$ Inflación aumenta precio relativo de doméstico
			\3 Pass-through
				\4 Pass-through de las importaciones
				\4[] Parámetro más relevante
				\4 Sensibilidad de precios en mercado nacional
				\4[] Ante variaciones en precios en extranjero
	\1 \marcar{Efectos de la inflación}
		\2 Idea clave
			\3 Contexto
				\4 Neutralidad del dinero
				\4[] Supuesto básico de modelos clásicos
				\4[] Nivel de precios no afecta vars. reales
				\4 Superneutralidad
				\4[] Supuesto más allá de neutralidad
				\4[] $\to$ Tasa de aumento de precios no afecta vars. reales
				\4 Relación entre dinero e inflación
				\4[] Validada empíricamente
				\4[] Términos concretos muy heterogéneos
				\4 Inflación esperada e inesperada
				\4[] Muy diferentes efectos
				\4[] Si agentes esperan inflación
				\4[] $\to$ Estiman variables reales con certeza
				\4[] $\then$ Realización de planes óptimos
				\4 Inflación altera muchas otras variables
				\4[] Precios relativos
				\4[] Tipos de interés
				\4[] Tipos de cambio
				\4[] ...
			\3 Objetivos
				\4 Ponderar efectos de crecimiento de precios
				\4[] Sobre diferentes variables
				\4[] $\to$ Especialmente, variables reales
				\4 Informar política económica
				\4[] En cuanto a decisiones que afectan inflación
			\3 Resultados
				\4 Anticipación de inflación es determinante
				\4 Variación de precios poco importante
				\4 Inflación sí tiene efectos reales
		\2 Inflación y tipos de interés
			\3 Idea clave
				\4 Generalmente, deuda e interés en términos nominales
				\4[$\to$] Inflación altera valor real de rentas
				\4 Fisher, Wicksell y monetaristas
			\3 Fisher y monetarismo
				\4 Identidad del interés nominal
				\4[] $r \equiv i - \pi$
				\4 Ecuación de equilibrio de los mercados financieros
				\4[] $i=r + \pi^e$
				\4 Estado estacionario
				\4[] $\pi = \pi^e$
				\4 Transiciones a equilibrio
				\4[] $\pi \neq \pi^e$
				\4[] $\to$ Efecto sobre tipos de interés real
				\4 Ajuste lento de interés nominal
				\4[] Implica efecto de inflación sobre vars. reales
				\4 Aumento de inflación con interés nominal rígido
				\4[] Reduce interés real esperado
				\4[] $\to$ Estímulo a expansión
				\4[] $\to$ Si dinero causa inflación
				\4[] $\then$ Dinero causa expansión vía efecto en interés
				\4 Deflación con interés nominal rígido
				\4[] Aumenta fuertemente interés real esperado
				\4[] $\to$ Estímulo a contracción y ahorro
				\4[] $\to$ Caída de demanda agregada
				\4[] $\then$ Deflación provoca contracción
				\4 Interés nominal elevado indica
				\4[] Inflación pasada
				\4[] Expansión del output pasada
			\3 Keynesianismo
				\4 Poca importancia a efectos de inflación sobre interés
				\4 Reducción de tipos implica expansión del output
			\3 Valoración
				\4 Sin discusión sobre identidad de Fisher
				\4 Controversia sobre ecuación
				\4[] Interés nominal es realmente rígido?
				\4[] $\to$ O se ajusta más despacio que inflación?
				\4[] Tipos bajos apuntan a
				\4[] $\to$ ¿expansión del output?
				\4[] $\to$ ¿contracción del output?
				\4[] $\then$ Debate muy actual en Europa
		\2 Valor real de la deuda
			\3 Idea clave
				\4 Contexto
				\4 Objetivos
				\4 Resultados
			\3 Formulación
				\4 $d_t = \frac{1+r}{1+g} b_{t-1} + \text{sp}_{t-1}$
				\4 $r = i - \pi$
				\4 Con $r>g$ senda de deuda divergente
				\4[] Tiende a infinito
				\4[] $\to$ Si no hay superávit primario
				\4[] $\then$ Si no se compensa senda de crecimiento
				\4 Inflación reduce tipo de interés real
				\4[] Hace más fácil $r < g$
				\4[] $\to$ Aumenta sostenibilidad
			\3 Implicaciones
				\4 Inflación reduce valor real de deuda
				\4 Inflación reduce superávits primarios necesarios
				\4 Inflación esperada debe aumentar tipo de interés
				\4[] Con HER, no debería tener efecto
				\4 Inflación como herramienta de financiación
				\4 Inflación puede tener origen fiscal
				\4[] Teoría fiscal del nivel de precios
				\4[] Asumiendo dominancia fiscal+HER
				\4[] $\to$ $\frac{D_t}{P_t} = \sum \text{superávits primarios}$
			\3 Valoración
				\4 Inflación es mecanismo habitual de impago real
				\4 Gobiernos utilizan para no pagar deuda doméstica
				\4[] Denominada en moneda local
				\4[] $\to$ Inflactar para reducir valor real
				\4 Expectativas reducen margen de actuación
				\4[] Aumento de tipos de interés nominales
		\2 Eficiencia económica
			\3 Idea clave\footnote{\href{https://www.nobelprize.org/uploads/2018/06/friedman-lecture-1.pdf}{Friedman (1976) Nobel Prize Lecture}}
				\4 Relación entre inflación y su volatilidad
				\4 Inflación más alta implica:
				\4[] Mayor margen para errar expectativas
				\4[] Mayores fluctuaciones de precios relativos
				\4 Si indexación + predicción perfecta posible
				\4[] $\pi$ no debería tener efecto sobre paro natural
				\4 Como no es posible
				\4[] Inflación tiene costes de eficiencia
			\3 Eficiencia del sistema de precios
				\4 Inflación es aumento de nivel de precios general
				\4 Precios relativos
				\4[] Unidades de un bien en términos de otro
				\4 Con inflación reducida
				\4[] Dos posibilidades
				\4[] $\to$ $\Delta$ en un sector se compensan con otro
				\4[] $\to$ $\Delta$ tienen escasa cuantía
				\4[] Compensación de hecho no se produce
				\4 Con inflación elevada
				\4[] Posibles aumentos fuertes en un sector
				\4[] $\to$ Sin compensar en otros
				\4[] $\then$ Distorsiones elevadas de precios relativos
				\4[] $\then$ Tensiones de economía política
				\4[] $\then$ Asignación ineficiente de recursos
			\3 Economía política
				\4 Ganadores y perdedores por $\Delta$ precios relativos
				\4 Comportamiento prudente se vuelve temerario
				\4[] Ahorrar se vuelve temerario
				\4[] Gastar rápidamente se vuelve prudente
				\4 Conflicto político aumenta
				\4[] Grupos de interés se organizan
				\4[] Capacidad del gobierno cae
				\4[] Aumenta presión para aumentar poderes del gobierno
			\3 Efectividad de la indexación
				\4 Inflación más volatil reduce efectividad
				\4 Si inflación elevada pero perfectamente constante
				\4[] Instituciones surgen para adaptarse
				\4[] $\to$ Indexación de contratos
				\4[] $\then$ Sin efectos reales sobre contratos
				\4 Si inflación es muy volátil
				\4[] Indexación debe reducir periodo de cálculo
				\4[] $\to$ Aumento de coste de renegociación
				\4[] $\to$ Rigidez nominal hasta nueva indexación
				\4[] $\to$ Aumento de stocks
				\4[] $\then$ Efectos reales de eficiencia
			\3 Precios públicos
				\4 Gobiernos proveen servicios de mercado
				\4[] Podrían ser provistos por sector privado
				\4[] Pero por razones varias, estado provee
				\4 Restricciones presupuestarias blandas
				\4[] Por principio de unidad de caja presupuestaria
				\4 Fijación de precios públicos
				\4[] Relación difusa con oferta y demanda
				\4 Inflación distorsiona precio relativo de precios públicos
				\4[] Necesario recalcular
				\4[] Decisión administrativa sujeta a costes de información
		\2 Economía abierta
			\3 Efecto sobre competitividad
				\4 Inflación doméstica mayor a extranjera
				\4[] Asumiendo TCN constante
				\4[] $\to$ Aprecia TCR
				\4 Índice de competitividad de las exportaciones
				\4[] \fbox{$\text{ITC}_t^a = \frac{\text{IPX}_t^a \cdot \text{IPR}_t^a}{100}$}
				\4 IPX -- Índice de Tipo de Cambio
				\4[] \fbox{$\text{IPX}_t^a = 100 \cdot \Pi_{i=1}^I \left( \frac{1}{\text{tc}_{it}} \right)^{n_i}$}
				\4[] $\to$ $n_i$: ponderación normalizada de cada moneda, $\sum_i n_i = 1$
				\4[] $\to$ $\text{tc}_{it}$: tipo de cambio directo
				\4[] $\then$ Media geométrica ponderada de TC bilaterales
				\4[] $\then$ Aumento (caída) de IPX indica apreciación (depreciación)
				\4 IPR -- Índice de Precios Relativos
				\4[] \fbox{$\text{IPR}_t^a = 100 \cdot \frac{\text{IPC}_\text{España,t}^a}{\Pi_{i=1}^I \left( \text{IPC}_{i,t}^a \right)^{n_i}}$}
				\4[] $\then$ Relación entre IPC de España...
				\4[] $\then$ ...y media geométrica ponderada de IPCs de referencia
				\4 Aumento de IPR
				\4[] Pérdida de competitividad nacional
				\4[] $\to$ Presión hacia déficit de cuenta corriente
			\3 Efecto Balassa-Samuelson
				\4 Inflación desestabiliza cumplimiento de PPA
				\4[] Países con mayor crecimiento de productividad
				\4[] $\to$ Mayor inflación en términos de IPC
				\4[] $\then$ Apreciación del TCR
				\4 IPC incluye comerciables y no comerciables
				\4[] Comerciables tienen precios iguales entre países
				\4[] $\to$ Aproximadamente
				\4[] $\to$ Resultado de competencia
				\4[] No comerciables pueden tener precios distintos
				\4[] $\to$ No hay competencia entre peluqueros indios y suecos
				\4 TCR no es constante usando IPC
				\4[] Aumenta con PMg en bienes comerciables
				\4[] $\to$ TCR más alto en países más productivos
				\4[] $\to$ Desarrollo aumenta TCR
				\4[] $\then$ Desviación sistemática de TCR
				\4[] $\then$ PPAAbsoluta no se cumple
				\4[] $\then$ PPARelativa a cumplir varía con desarrollo/productividad
				\4[] $\then$ PPA no es robusta a índice de precios utilizado
				\4 Formulación
				\4[] Países A y B
				\4[] Dos sectores
				\4[] $\to$ Bienes comerciables (T)
				\4[] $\to$ No comerciables (NT)
				\4[] Precios de bienes comerciables = en A y B
				\4[] $\to$ Por ley de único precio
				\4[] Productividad de no comerciables = en A y B
				\4[] $\to$ P.ej: peluquerías, taxis
				\4[] Productividad de comerciables distinta en A y B
				\4[] Salarios iguales en ambos sectores
				\4[] $\to$ Movilidad interna de L perfecta
				\4[] País A:
				\4[] $\to$ $w_T^A = P_T \cdot \text{PMg}_T^A = P_\text{NT}^A \cdot \text{PMg}_{NT} = w_\text{NT}^A$
				\4[] País B:
				\4[] $\to$ $w_T^B = P_T \cdot \text{PMg}_T^B = P_\text{NT}^B \cdot \text{PMg}_{NT} = w_\text{NT}^B$
				\4[] País A se desarrolla más que B
				\4[] $\to$ $\text{PMg}_A^T > \text{PMg}_B^T$ $\then$ $P_\text{NT}^A > P_\text{NT}^B$
				\4[] $\then$ IPC crece más en desarrollados (efecto Penn)
				\4[] Si TCN mantiene PPA para comerciables
				\4[] $\then$ TCN no mantiene PPA en no comerciables e IPC
				\4[] $\then$ Desviaciones permanentes de PPAAbsoluta
				\4[] $\then$ PPA comerciables compatible con no PPA general
				\4 Implicaciones
				\4[] Modelos de TCN basados en PPA
				\4[] $\to$ Implican TCR constante
				\4[] Efecto Balassa-Samuelson desestabiliza PPA
				\4[] $\to$ Modelos de TCN-PPA no robustos a índice de $\pi$
				\4[] En presencia de:
				\4[] $\to$ Bienes no comerciables
				\4[] $\to$ Divergencia en productividad marginal de L
				\4[] $\then$ PPA no es estable
				\4[] $\then$ Predicción sobre TCR basada en PPA y TCN no es estable

				\4 Contrastación empírica
				\4[] Muy difícil contrastación
				\4[] Evidencia débil a favor
				\4[] $\to$ Resultados poco robustos a medidas de productividad
				\4[] $\to$ PPA apenas se cumple entre comerciables
				\4[] $\to$ Resultados compatibles pero cuantitativamente pequeños
		\2 Inflación óptima
			\3 Regla de Friedman
				\4 Friedman (1969)
				\4[] Dinero es necesario como medio de transacción
				\4[] Coste marginal social de crear dinero es nulo
				\4[] Tenencia de dinero implica coste de oportunidad
				\4[] $\to$ Interés nominal perdido
				\4[] $\then$ Es costoso mantener dinero para transacciones
				\4[] Dado interés real exógeno
				\4[] $\to$ Inflación negativa igual a interés real
				\4[] $\then$ Interés nominal nulo
				\4[] $\then$ Mantener dinero no tiene coste nominal
				\4 Resultado relativamente robusto
				\4[] Principales modelos micro con dda. de dinero
				\4[] $\to$ Mantienen resultado de deflación óptima
				\4 Inflación óptima
				\4[] Deflación predecible
				\4[] $\to$ Igual a interés real
				\4[] Preferiblemente a través de regla conocida
				\4[] $\to$ De crecimiento de oferta monetaria
				\4 Regla de Friedman sin deflación
				\4[] Equiparar dinero con activos que generan interés
				\4[] $\to$ Dinero para transacción no implique coste
				\4[] Pagar interés por depósitos
				\4[] $\to$ Tasas equiparables a activos sustitutivos
				\4 Críticas
				\4[] Coste de mantener dinero es muy pequeño
				\4[] $\to$ No merece la pena deflación
				\4[] $\to$ (Al menos, con tipos de interés actuales)
				\4[] Salarios nominales rígidos a la baja
				\4[] $\to$ Nivel de precios difícilmente puede bajar
				\4[] Precios rígidos a la baja
				\4[] $\to$ Múltiples fundamentaciones
				\4[] Trampas de liquidez
				\4[] $\to$ Deflación es contractiva
				\4[] $\to$ Tipo nominal no puede bajar de cero
				\4[] $\to$ Estímulo vía tipos bajos imposible
				\4[] $\then$ Espiral deflacionaria
				\4[] Activos seguros indivisibles
				\4[] $\to$ Difícilmente pueden sustituir dinero
				\4[] $\then$ Tenencia de dinero tiene coste
			\3 Regla de Phelps
				\4 Impuestos fiscales introducen distorsiones
				\4[] Cuando no son de suma fija
				\4[] $\to$ Precios relativos
				\4[] $\to$ Efecto incentivo sobre K y L
				\4 Limitaciones del sistema tributario
				\4[] Existencia de bases no gravadas
				\4[] Evasión fiscal y ocultación
				\4 Dinero nacional usado por no residentes
				\4[] Dinero que circula fuera de fronteras nacionales
				\4[] $\to$ No contribuye a inflación
				\4[] $\to$ Puede generar señoreaje
				\4 Inflación es comparable a impuesto
				\4[] Gobierno extrae valor de saldos reales
				\4 Posible extraer ingresos fiscales vía señoreaje
				\4[] $\to$ Dirección contraria a coste de oportunidad
				\4[] $\then$ Inflación óptima no es la de Friedman
			\3 Divina coincidencia
				\4 Modelos DSGE de NEK
				\4[] Precios à la Calvo
				\4[] $\to$ Sólo $\%$ empresas pueden ajustar en cada periodo
				\4 Inflación es tendencia hacia mark-up óptimo
				\4[] Empresas que pueden suben precios
				\4[] $\to$ Para restablecer mark-up óptimo
				\4[] $\then$ Para cubrir output gap
				\4[] En la medida en que se debe restablecer mark-up óptimo
				\4[] $\to$ Mark-up no es óptimo
				\4[] $\then$ Empresas no maximizan beneficios
				\4[] $\then$ Consumidores no reciben beneficios óptimos
				\4 Ecuaciones de estado estacionario
				\4[DIS] IS dinámica
				\4[] \fbox{$\tilde{y}_t = \textrm{E}_t \left\lbrace \tilde{y}_{t+1} \right\rbrace - \frac{1}{\sigma} \left( \underbrace{i_t - \textrm{E}_t \left\lbrace \pi_{t+1} \right\rbrace}_{r_t} - r^n_t \right) $}
				\4[NKPC] Curva de Phillips Neo-Keynesiana
				\4[] \fbox{$\pi_t = \text{E}_t \left\lbrace \pi_{t+1} \right\rbrace + \textsc{k} \tilde{y}_t $}
				\4 Si $\tilde{y}_t = 0$ es óptimo
				\4[] Inflación óptima implica estabilidad de precios
				\4[] $\to$ Objetivos de inflación y output coinciden
				\4[] $\then$ Posible estabilizar inflación y output
				\4[] $\then$ No son necesarios más instrumentos
				\4[] $\then$ ``Divina coincidencia''
				\4[] $\then$ Mandato único equivale a mandato dual
				\4[] $\then$ No es necesario mandato estabilizador de paro\footnote{Aparece aquí la obligatoria comparación entre los respectivos mandatos de la Reserva Federal y el BCE. La Reserva Federal tiene el mandato de estabilizar la inflación y acercar la economía al pleno empleo, mientras que el BCE sólo tiene el mandato de mantener la inflación estable y cercana pero inferior a 2\% anual. Si se cumpliese la divina coincidencia en una economía dada, bastaría con mantener estable la inflación para estabilizar el output a su nivel óptimo. Si no se cumple, la estabilización de la inflación mantiene el output diferente de su nivel óptimo.}
				\4 Si output gap óptimo es positivo
				\4[] Necesaria inflación
				\4[] $\to$ Pero inflación tiene también costes
				\4[] $\then$ Estabilidad de precios no es óptima
				\4 Minimización de función de pérdida
				\4[] Cuando no se cumple la divina coincidencia
				\4[] Ponderar inflación vs desviación de output gap
				\4[] Ejemplo:
				\4[] $\underset{\pi, \tilde{y}}{\min} \quad \alpha \pi^2 + \beta(\tilde{y}_t - y^*_t)$
				\4[] $\text{s.a:} \quad \pi_t = \text{E}_t \left\lbrace \pi_{t+1} \right\rbrace + \textsc{k} \tilde{y}_t $
			\3 Regla de Taylor
				\4 Idea clave
				\4[] Contexto
				\4[] $\to$ Modelos NEK de segunda generación
				\4[] $\to$ Rigideces nominales y reales
				\4[] $\to$ HER
				\4[] Objetivo
				\4[] $\to$ Caracterizar PM que estabiliza inflación
				\4[] $\to$ Evitar análisis normativo de inflación óptima
				\4[] $\to$ Comparar con PM llevadas a cabo por bancos centrales
				\4 Formulación
				\4[] $i = r^* + \pi^* + \phi_t (\pi - \pi^*) + \phi_y (y_t - y_t^n)$
				\4[] Donde:
				\4[] $\to$ $i$: tipo de interés nominal de intervención
				\4[] $\to$ $r^*$: tipo de interés real de equilibrio l/p
				\4[] $\to$ $\pi$: inflación del periodo
				\4[] $\to$ $\pi^*$: inflación objetivo del banco central
				\4[] $\to$ $y$: output gap
				\4[] Regla original de Taylor (1993)
				\4[] $\to$ $i = r^* + \pi^* + 1.5(\pi - \pi^*t) + 0.5 y$
				\4 Implicaciones
				\4[] Estabilización implica sobrerreacción a inflación
				\4[] $\to$ Cuando no tiene lugar, inflación y volatilidad macro
				\4[] $\then$ Años 70
				\4[] $\to$ En gran moderación entre 80s y primeros 2000s
				\4[] $\then$ Se cumple aproximadamente
				\4[] $\then$ Baja volatilidad
				\4[] $\to$ Entre 2000s y GCF
				\4[] $\then$ no se cumple
				\4[] $\then$ Aumento de volatilidad
				\4[] $\then$ Conclusión discutida empíricamente
				\4[] Ecuación de Fisher
				\4[] $\to$ Cuando inflación no se desvía de objetivo
				\4[] $\to$ Cuando output no se desvía de natural
				\4[] $\then$ $i=r^* + \pi^*$
				\4[] $\then$ Interés nominal y real, inflación estacionarias
				\4[] Diferentes etapas de política monetaria
				\4[] $\to$ Años 70
				\4[] $\then$ Coeficiente por debajo de 1.5
				\4 Valoración
				\4[] Interés real natural inobservable
				\4[] $\to$ ¿Cómo estimar?
				\4[] $\then$ ¿Política acomodaticia o $r_n^t$ más bajo?
				\4[] Evidencia empírica mixta
				\4[] $\to$ Estabilidad con regla de Taylor
				\4[] $\to$ Estabilidad sin regla de Taylor
				\4[] $\to$ ...
			\3 Inflación óptima en la ZLB
				\4 Límite inferior a tipo de interés nominal
				\4[] $0\%$ o ligeramente inferior(ELB)
				\4[] Nadie querría invertir a tipo negativo
				\4[] $\to$ Si puede mantener dinero en metálico a $0\%$
				\4 Estímulo monetario limitado por ZLB
				\4[] Interés como instrumento de política monetaria
				\4[] $\to$ No puede reducir interés real más allá de límite
				\4 Expectativas de inflación como estímulo
				\4[] ``Forward-guidance''
				\4[] Solución teórica a límite de ZLB
				\4[] BCentral se compromete a mantener inflación alta
				\4[] $\to$ Más tiempo del consistente con regla de Taylor
				\4[] Inflación esperada aumenta para horizonte m/p
				\4[] $\to$ Tipos reales esperados bajos
				\4[] $\then$ Estímulo a demanda agregada y output
				\4 Sin forward-guidance, posible deflación
				\4[] Caída del output reduce inflación esperada
				\4[] $\to$ Caída de $\pi^e$ aumenta interés real
				\4[] $\then$ Profundización de caída del output
				\4[] $\then$ Posible espiral deflacionaria
		\2 Deflación
			\3 Idea clave
				\4 Contexto
				\4[] Contracción de demanda
				\4[] Shock de oferta keynesiano
				\4[] $\to$ Fuerte contracción de oferta inicial
				\4[] $\to$ Shock a expectativas y animal spirits
				\4[] $\then$ Caída de demanda aún mayor
				\4[] $\then$ Deflación
				\4 Objetivos
				\4[] Efectos de caída de nivel de precios
				\4 Resultados
				\4[] Transferencia de deudores a acreedores
				\4[] Reducción de los beneficios
				\4[] Aumento del desempleo con rigideces nominales en salarios
				\4[] $\to$ Empresas no pueden bajar sueldos
				\4[] $\to$ Precios de bienes y servicios se reducen
			\3 Efectos
				\4 Quiebras generalizadas
				\4[] Precios de bienes caen
				\4[] Costes fijos no caen de igual forma
				\4[] Reducción de márgenes y beneficio operativo
				\4[] Aumento del valor real de la deuda
				\4[] $\then$ Suspensión de pagos
				\4[] $\then$ Quiebras
				\4 Aumento de valor de saldos reales
				\4[] Puede reducir utilidad marginal de dinero
				\4[] $\to$ Aumentar utilidad marginal de consumo
				\4[] $\then$ Estimular consumo
				\4 Posponer consumo
				\4[] Depende de expectativas de agentes
				\4[] Con expectativas adaptativas o pasivas
				\4[] $\to$ Agentes esperan deflación se mantenga
				\4[] $\then$ No consumen ahora
				\4[] $\then$ Deflación persiste/aumenta
			\3 Evidencia empírica
	\1[] \marcar{Conclusión}
		\2 Recapitulación
			\3 Causas de la inflación
			\3 Análisis positivo de la inflación
			\3 Efectos de la inflación
		\2 Idea final
			\3 Política monetaria en crisis de COVID-19
			\3 Inflación en la zona euro
			\3 Curva de Phillips en la actualidad
			\3 Criptomonedas e innovaciones financieras
\end{esquemal}



































\graficas

\begin{axis}{4}{Ejemplo de curva de Phillips similar a la estimada por Phillips en 1958.}{}{$\dot{w}$}{curvadephillips}
	\node[below] at (6,0){$u$};
	\draw[-] (0,0) -- (-2,0);
	\draw[-] (0,0) -- (0,-2);
	\draw[-] (4,0) -- (6,0);
	
	\draw[-] (1,4) to [out=280, in= 175](6,-1);
\end{axis}

\begin{axis}{4}{Kydland y Prescott (1977): inconsistencia de la política monetaria óptima y sesgo inflacionario resultante.}{$u_t - u^*$}{$\pi_t$}{kydlandprescott1977}
	% Extensión de ejes
	\draw[-] (-3,0) -- (0,0); % abscisas
	\draw[-] (0,0) -- (0,-3); % ordenadas
	
	% Curvas de Phillips
	\draw[-] (-3,4) -- (3,-4);
	\draw[-] (-3,5.7) -- (3,-2.3);
	
	% Curvas de indiferencia de función de pérdida
	\draw[-] (-3,3) to [out=-20, in=90](0,0) to [out=270, in=20](-3,-3);
	\draw[-] (-3,3.53) to [out=-20, in=90](0.53,0) to [out=270, in=20](-3,-3.53);
	
	% Óptimo
	\node[circle,fill=black,inner sep=0pt,minimum size=4pt] (a) at (0,0) {};	
	\node[above] at (-0.45,0){O};
	
	% Equilibrio
	\node[circle,fill=black,inner sep=0pt,minimum size=4pt] (a) at (0,1.8) {};
	\node[right] at (0,1.8){B};
	
\end{axis}

El punto A muestra el óptimo alcanzable en presencia de commitment. En ausencia de commitment, el equilibrio es el punto B, en el que el desempleo es el mismo que en el óptimo pero la inflación es mayor. Una vez que se forman las expectativas, la autoridad monetaria tiene incentivos a tratar de situarse en una curva más a la derecha para reducir la función de pérdida aumentando la inflación. Sin embargo, los agentes estiman este comportamiento y la curva de Phillips se desplaza hacia arriba. En el equilibrio, el output es el natural pero la inflación es más alta que si hubiese existido la posibilidad de comprometerse a mantener la inflación baja.

\conceptos

\preguntas

\seccion{Test 2016}

\textbf{12.} La fórmula de actualización de un determinado tipo de rentas que perciben los consumidores que incluya entre sus componentes un índice de precios debería tener en cuenta:

\begin{itemize}
	\item[a] Que los índices de tipo Laspeyres tienden a subestimar el aumento del coste de vida, por lo que desde el punto de vista del bienestar de los rentistas convendría recurrir a un índice de tipo Paasche, si bien el deflactor del PIB no resulta adecuado ya que mide los cambios de precios en todos los aspectos de la economía, en oposición al IPC, que sólo analiza el gasto del consumidor.
	\item[b] Que los índices de tipo Paasche tienden a subestimar el aumento del coste de vida al suponer que la cesta del consumidor previa al aumento del nivel de precios generales puede alcanzarse con los precios finales.
	\item[c] Que los índices de tipo Laspeyres tienden a sobre-estimar el aumento del coste de vida ya que el consumidor puede alterar los productos de su cesta ante variaciones del precio.
	\item[d] Que en una economía sólo con bienes complementarios perfectos resultaría indiferente incluir en la fórmula el IPC o el deflactor del PIB desde el punto de vista del bienestar de los rentistas, dado que no existiría el efecto sustitución ante aumentos del nivel de precios.
\end{itemize}

\seccion{Test 2014}

\textbf{8.} Suponga que los precios en 2007 fueron $(p_x, p_y) = (2,3)$ y en 2008 $(p'_x, p'_y) (3,4)$. Si la cesta de bienes de un consumidor en 2007 fue $(2,2)$; entonces su IPC verdadero es:

\begin{itemize}
	\item[a] Menor del 40\%.
	\item[b] Exactamente el 40\%.
	\item[c] Mayor del 40\%.
	\item[d] Indeterminado.
\end{itemize}

\seccion{Test 2007}

\textbf{21.} La relación de Fisher puede describirse por la siguiente ecuación en la cual R es el tipo de interés nominal, $r$ es el tipo de interés real y $p$ es la tasa de inflación:

\begin{itemize}
	\item[a] $p=r+R$
	\item[b] $1+p = (1+r)/(1+R)$
	\item[c] $1+r = (1+p)/(1+R)$
	\item[d] $1+r = (1+R)/(1+p)$
\end{itemize}

\textbf{22.} Suponga que, como consecuencia de un aumento de las tensiones internacionales, los agentes de una economía esperan una subida de la tasa de inflación. En el contexto de un modelo de inflación basado en la curva de Phillips ampliada con expectativas, si el Banco Central desea mantener constante la actual tasa de inflación deberá:

\begin{itemize}
	\item[a] Reducir la tasa de crecimiento de la cantidad de dinero.
	\item[b] Aumentar la tasa de crecimiento de la cantidad de dinero.
	\item[c] Mantener constante la tasa de crecimiento de la cantidad de dinero.
	\item[d] No podrá evitar la subida de la tasa de inflación.
\end{itemize}

\seccion{Test 2005}
\textbf{18.} Señale la afirmación correcta en el marco de la teoría monetaria de la inflación

\begin{itemize}
	\item[a] Bajo expectativas racionales, los agentes cometen errores sistemáticos.
	\item[b] Bajo expectativas racionales, el futuro no juega ningún papel en la determinación del nivel de precios actual.
	\item[c] Cuando los agentes forman sus expectativas mediante un mecanismo adaptativo, aprenden de los errores de previsión cometidos en el pasado.
	\item[d] Cuando los agentes forman sus expectativas mediante un mecanismo adaptativo, las políticas monetarias futuras juegan un papel esencial en la determinación del nivel de precios actual.
\end{itemize}

\notas

Ver \url{https://voxeu.org/article/phillips-curve-dead-or-alive} sobre Curva de Phillips ``muerta'' en la actualidad

\textbf{2016:} \textbf{12.} C

\textbf{2014:} \textbf{8.} A

\textbf{2007:} \textbf{21.} D \textbf{22.} A

\textbf{2005:} \textbf{18.} C

Leer \comillas{fiscal theory of the price level} del Palgrave, extraído en la carpeta del tema.

\bibliografia

Mirar en Palgrave:
\begin{itemize}
	\item central bank independence
	\item cost-push inflation
	\item demand management
	\item demand-pull inflation
	\item expectations
	\item fiscal theory of the price level
	\item german hyperinflation
	\item hedonic prices
	\item hyperinflation
	\item index numbers
	\item inflation
	\item inflation accounting
	\item inflation and growth
	\item inflation dynamics
	\item inflation expectations
	\item inflation measurement
	\item inflation targeting
	\item inflationary gap
	\item monetary policy, history of
	\item neutrality of money
	\item phillips curve
	\item stagflation
	\item supply shocks in macroeconomics
\end{itemize}


* Andrade, P.; Galí, J.; Le Bihan, H.; Matheron, J. (2018) \textit{The Optimal Inflation Target and the Natural Rate of Interest} NBER Working Series -- En carpeta del tema

Banco de España. (2017) \textit{A Suite of Inflation Forecasting Models}.  \url{https://www.bde.es/f/webbde/SES/Secciones/Publicaciones/PublicacionesSeriadas/DocumentosOcasionales/17/Fich/do1703e.pdf}

Christiano, L. J., Fitzgerald, T. J. (2000) \textit{Understanding the Fiscal Theory of the Price Level} Economic Review of the Federal Reserve Bank of Cleveland  \url{http://faculty.econ.ucdavis.edu/faculty/kdsalyer/LECTURES/Ecn235a/Extra\_presentation\_papers/fiscal\_theory.pdf}

* Coibion, O.; Gorodnichenko, Y.; Wieland, J. (2012) \textit{The Optimal Inflation Rate in New Keynesian Models: Should Central Banks Raise Their Inflation Targets in Light of the Zero Lower Bound?} Review of Economic Studies -- En carpeta del tema

De Grauwe, P.; Polan, M. (2005) \textit{Is Inflation Always and Everywhere a Monetary Phenomenon?}  Scandinavian Journal of Economics -- En carpeta del tema

Edmond, C. (2007) \textit{Cagan's Model of Hyperinflation}  \url{http://pages.stern.nyu.edu/~cedmond/ge07pt/notes_cagan.pdf}

Gordon, R. J. (2008) \textit{The History of the Phillips Curve: Consensus and Bifurcation} Economica -- En carpeta del tema

Hall, R. H. et al (1982) \textit{Inflation: Causes and Effects} NBER Project Report. The University of Chicago Press -- En carpeta del tema

Humphrey, T. M. (1981) \textit{Keynes on Inflation} Federal Reserve Bank of Richmond Economic Review, January/February 1981 -- En carpeta del tema

* Nakamura, E.; Steinsson, J.; Sun, P.; Villar, D. (2018) \textit{The Elusive Costs of Inflation: Price Dispersion during the U.S. Great Inflation} Quarterly Journal of Economics -- En carpeta del tema

Phelps, E. S. (1973) \textit{Inflation in the Theory of Public Finance} The Swedish Journal of Economics - Vol. 75 March 1973 -- En carpeta del tema

Sanches, D. (2012) \textit{The Optimum Quantity of Money} Business Review of the Philadelphia Fed -- En carpeta del tema

Sargent, T. J.; Wallace, N. \textit{Rational expectations and the Dynamics of Hyperinflation} (1973) International Economic Review -- En carpeta del tema

\end{document}
