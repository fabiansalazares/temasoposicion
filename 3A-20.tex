\documentclass{nuevotema}

\tema{3A-20}
\titulo{Determinación de salarios: negociación, salarios de eficiencia y contratos implícitos}

\begin{document}

\ideaclave

El mercado de trabajo es, en toda economía, un sector de especial importancia. El trabajo remunerado no sólo constituye la principal fuente de renta de las familias, sino que condiciona en gran medida la vida de los seres humanos a través de su impacto en la gestión del tiempo y por ende, en el bienestar y la realización personal. La ciencia económica enfoca el análisis del mercado de trabajo desde diferentes perspectivas. Los análisis macroeconómicos del mercado de trabajo examinan los cambios en las variables agregadas para tratar de entender y predecir su evolución. Los modelos microeconómicos del mercado de trabajo examinan el proceso de decisión de agentes individuales, habitualmente representándolo como un problema de optimización matemática de una función objetivo como la utilidad o los beneficios. El modelo neoclásico ha sido el punto de partida del análisis, como en muchas otras áreas de la economía. Las conclusiones son las habituales: cuando el trabajo es un bien homogéneo que se intercambia a un único precio en todo el mercado, existe un equilibrio único y estable bajo condiciones relativamente generales y éste equilibrio se caracteriza por igualar ingreso y coste marginal de producción. Ningún agente tiene poder de mercado, de tal manera que oferentes enfrentan demandas completamente elásticas y demandantes enfrentan ofertas también completamente elásticas. Así, es habitual concluir que el modelo neoclásico predice ausencia total de desempleo, ya que cualquier trabajador que desee trabajar al salario de equilibrio puede hacerlo. Si aparece un exceso de oferta de trabajo, el salario de equilibrio disminuirá hasta eliminarlo. La realidad es notablemente distinta del panorama dibujado por el modelo neoclásico. El salario al que los trabajadores venden su trabajo a las empresas es el resultado de procesos de negociación de enorme variedad en los que toman parte trabajadores y empresarios, empresas y sindicatos, patronales y sindicatos... y en los que en ocasiones se negocian salarios de contratos individuales pero a menudo establecen acuerdos que establecen las condiciones de trabajo y el salario de un colectivo amplio de trabajadores. 

En estos procesos de negociación no existe un mercado agregado en el que el trabajo es un bien perfectamente homogéneo y sustituible, sino múltiples mercados de trabajo en los que las partes disponen de grados variables de poder de mercado y las partes no enfrentan ofertas y demandas perfectamente elásticas. Aparece así el fenómeno del desempleo, entendido como situación en la que se encuentra un trabajador que desea trabajar al salario que reciben trabajadores empleados, o a bajarlo, pero que sin embargo no encuentra demanda de trabajo con la que poder emplearse. La determinación del salario es uno de los pilares de los modelos que tratan de explicar el desempleo. Si los salarios no se ajustan de forma inmediata ante excesos de oferta de trabajo, aparecerán las diferentes formas de desempleo que es habitual encontrar en la literatura. Además, existen razones por las cuales tanto ofertantes como demandantes de trabajo pueden preferir de forma explícita no ajustar los salarios ante excesos de oferta. Los salarios de eficiencia y los contratos implícitos son las dos familias de modelos más importantes que han examinado este fenómeno.

La exposición tiene por \textbf{objeto} responder a las preguntas más relevantes respecto a la determinación de salarios, tales como: ¿de qué factores dependen los salarios en un sector determinado? ¿qué es la teoría de la negociación? ¿cómo se modeliza la negociación colectiva? ¿qué modelos de negociación son los más relevantes? ¿qué son los salarios de eficiencia? ¿qué son los contratos implícitos? ¿qué aplicación tienen al mercado laboral? Para responder a estas preguntas, la exposición se \textbf{estructura} en tres partes. En la primera, examinamos la negociación salarial, introduciendo en primer lugar los elementos básicos de la teoría de la negociación. A continuación, y de acuerdo con el título de la exposición, examinamos los modelos de salarios de eficiencia, que establecen una relación entre trabajo efectivo y salario. Por último, presentamos la teoría de contratos implícitos aplicada al mercado de trabajo.

\seccion{Preguntas clave}

\begin{itemize}
	\item ¿Cómo se modeliza la determinación del salario?
	\item ¿Cómo se negocian los salarios?
	\item ¿Qué es la teoría de la negociación (bargaining theory)?
	\item ¿Cuáles son los modelos de negocación colectiva más relevantes?
	\item ¿Qué son los salarios de eficiencia?
	\item ¿Qué son los contratos implícitos?
\end{itemize}

\esquemacorto

\begin{esquema}[enumerate]
	\1[] \marcar{Introducción}
		\2 Contextualización
			\3 Mercado de trabajo
			\3 Enfoques de análisis
			\3 Determinación del salario
		\2 Objeto
			\3 ¿De qué factores dependen los salarios?
			\3 ¿Cómo se modeliza la negociación colectiva?
			\3 ¿Qué es la teoría de la negociación?
			\3 ¿Qué modelos de negociación colectiva son los más importantes?
			\3 ¿Qué son los salarios de eficiencia?
			\3 ¿Qué son los contratos implícitos?
		\2 Estructura
			\3 Negociación salarial
			\3 Salarios de eficiencia
			\3 Contratos implícitos
	\1 \marcar{Negociación salarial}
		\2 Idea clave
			\3 Empresas, trabajadores y sindicatos
			\3 Proceso de negociación
			\3 Negociación aplicada al mercado de trabajo
			\3 Barreras conceptuales al análisis de neg. colectiva
		\2 Teoría de la negociación
			\3 Idea clave
			\3 Negociación axiomática
			\3 Negociación estratégica
		\2 Modelos de negociación empresa-sindicato
			\3 Idea clave
			\3 Dunlop (1944) -- Sindicato monopolístico
			\3 Right-to-manage -- Nickell y Andrews (1983)
			\3 Negociación débilmente eficiente
			\3 Negociación fuertemente eficiente
			\3 Insiders y outsiders
		\2 Estructura de la negociación colectiva
			\3 Idea clave
			\3 Formulación
			\3 Implicaciones
			\3 Valoración
		\2 Coordinación de la negociación colectiva
			\3 Idea clave
			\3 Formulación
			\3 Implicaciones
			\3 Valoración
		\2 Valoración
			\3 Sindicación en la práctica
			\3 Efectos de sindicación y negociación
			\3 Eficiencia de la negociación colectiva
			\3 Dispersión de salarios
			\3 Resultados globales
	\1 \marcar{Salarios de eficiencia}
		\2 Idea clave
			\3 Contexto
			\3 Objetivos
			\3 Resultados
		\2 Modelo básico de salarios de eficiencia
			\3 Idea clave
			\3 Formulación
			\3 Implicaciones
			\3 Valoración
		\2 Shapiro y Stiglitz (1984):información asimétrica
			\3 Idea clave
			\3 Formulación
			\3 Implicaciones
		\2 Extensiones
			\3 Bajada de salario si no ejercen esfuerzo
			\3 Otras fuentes de ineficiencia
		\2 Valoración
			\3 Implicaciones aproximadamente confirmadas
			\3 Dificil establecer causalidad
			\3 Persistencia del desempleo
	\1 \marcar{Contratos implícitos}
		\2 Idea clave
			\3 Contexto
			\3 Objetivo
			\3 Resultados
		\2 Formulación
			\3 Explicación basada en maximización de empresa
			\3 Explicación basada en salario compuesto
		\2 Implicaciones
			\3 Ingresos reales constantes
			\3 Salario acíclico
			\3 Despidos concentrados
			\3 Volatilidad del empleo aumentada
		\2 Valoración
			\3 Teoría de contratos aplicada a mercado de trabajo
			\3 Escasa incorporación en otros modelos
			\3 Programa de investigación relativamente abandonado
	\1[] \marcar{Conclusión}
		\2 Recapitulación
			\3 Negociación salarial
			\3 Salarios de eficiencia
			\3 Contratos implícitos
		\2 Idea final
			\3 Relación con otras disciplinas
			\3 Cambios recientes en desigualdad salarial

\end{esquema}

\esquemalargo














\begin{esquemal}
	\1[] \marcar{Introducción}
		\2 Contextualización
			\3 Mercado de trabajo
				\4 Especial importancia
				\4[] Trabajo remunerado es principal fuente de renta
				\4[] Condiciona actividad humana
				\4[] $\to$ Fracción importante del tiempo
			\3 Enfoques de análisis
				\4 Macroeconómico
				\4[] A partir de variables agregadas
				\4 Microeconómico
				\4[] Decisiones de agentes individuales
				\4 Modelo neoclásico del mercado de trabajo
				\4[] Análisis micro de oferta y demanda
				\4[] Conclusiones habituales de mod. neoclásicos
				\4[] $\to$ Ley de único precio en un mercado
				\4[] $\to$ Existencia de equilibrio único
				\4[] $\to$ Ingreso marginal iguala coste marginal
			\3 Determinación del salario
				\4 Modelo neoclásico:
				\4[] Precio es equilibrio entre oferta y demanda
				\4[] Mercado competitivo:
				\4[] $\to$ Ningún agente tiene poder de mercado
				\4[] $\then$ Sin poder sobre salario
				\4[] Desempleo no es relevante
				\4[] A precio de equilibrio
				\4[] $\to$ Cualquiera puede trabajar
				\4[] Si hay exceso de oferta de trabajo
				\4[] $\to$ Salario de equilibrio baja
				\4 En la práctica
				\4[] Determinación de salarios sujeta a negociación
				\4[] $\to$ Entre trabajadores y empresa
				\4[] $\to$ Entre empresa y sindicato
				\4[] $\to$ Entre patronal y sindicatos
				\4[] $\then$ Negociación colectiva
				\4[] $\then$ Partes tienen poder de mercado
				\4[] $\then$ Poder de negociación determina equilibrio
				\4 Fenómenos característicos del mercado de trabajo
				\4[] Existe desempleo
				\4[] $\to$ Salarios no eliminan excesos de oferta/demanda
				\4[] Existe negociación colectiva
				\4[] Existen sindicatos y patronales
				\4[] Empresas prefieren ajustar en cantidades
				\4[] $\to$ Antes que bajar salarios
				\4[] Sindicatos prefieren mantener salarios
				\4[] $\to$ Aunque arriesguen ser despedidos
				\4[] Contratos laborales son relaciones de l/p
				\4[] $\to$ Implican capital humano específico
				\4[] $\to$ No prevén todas las contingencias
				\4[] $\to$ Salarios no se ajustan dinámicamente
				\4[] $\then$ Salario se desvía de prod. marginal
		\2 Objeto
			\3 ¿De qué factores dependen los salarios?
			\3 ¿Cómo se modeliza la negociación colectiva?
			\3 ¿Qué es la teoría de la negociación?
			\3 ¿Qué modelos de negociación colectiva son los más importantes?
			\3 ¿Qué son los salarios de eficiencia?
			\3 ¿Qué son los contratos implícitos?
		\2 Estructura
			\3 Negociación salarial
			\3 Salarios de eficiencia
			\3 Contratos implícitos
	\1 \marcar{Negociación salarial}
		\2 Idea clave
			\3 Empresas, trabajadores y sindicatos
				\4 Modelo neoclásico estiliza mercado
				\4[] Trabajadores y empresas a título individual
				\4[] $\to$ Número suficientemente elevado
				\4[] $\to$ O supuestos behaviorales inducen CPerfecta
				\4[] $\then$ Salario de equilibrio competitivo
				\4[] $\then$ No hay poder de mercado
				\4 Sin embargo, hay poder de mercado
				\4[] Empresa negocia con trabajador
				\4[] Empresa negocia con trabajadores
				\4[] Empresas negocian con sindicatos
				\4[] ...
			\3 Proceso de negociación
				\4 Existe un excedente a repartir
				\4 ¿Cómo se repartirá?
				\4 ¿Qué factores determinan el reparto?
				\4 ¿Cómo representar proceso de negociación?
			\3 Negociación aplicada al mercado de trabajo
				\4 Trabajo genera un producto
				\4 Salario determina reparto del producto
				\4[] $\to$ ¿Cuánto para empresa?
				\4[] $\to$ ¿Cuánto para trabajadores?
				\4[] $\to$ ¿Entre cuántos trabajadores repartir?
				\4[] $\to$ ¿Se negocian también otras variables?
			\3 Barreras conceptuales al análisis de neg. colectiva
				\4 Partes compuestas de muchos agentes
				\4[] $\to$ ¿Cómo representar preferencias colectivas?
				\4[] Problemas habituales de agregación de prefs.
				\4[] Especialmente relevante con sindicatos
				\4[] Si todos trabajadores idénticos:
				\4[] $\to$ Maximizar salario esperado
				\4[] $\to$ Considerando demanda y total de trabajadores
				\4[] Si trabs. heterogéneos y decisión democrática
				\4[] $\to$ Teorema de Arrow
		\2 Teoría de la negociación
			\3 Idea clave
				\4 Modelos generales de negociación
				\4 Dos partes
				\4[] ¿qué utilidad obtiene cada uno?
				\4[] ¿cómo se reparten un beneficio?
				\4 Dos enfoques de modelización
				\4[] Enfoque axiomático
				\4[] $\to$ ¿qué propiedades tiene el reparto?
				\4[] Enfoque estratégico
				\4[] $\to$ ¿de qué decisiones depende el reparto?
			\3 Negociación axiomática
				\4 Nash (1950)
				\4 ¿Cómo debe ser reparto?
				\4[] $\to$ ¿Cómo debe ser $(u_1,u_2)$?
				\4[] Dada utilidad de reserva: $(d_1,d_2)$
				\4 Cuatro axiomas deseables:
				\4[] Pareto eficiencia
				\4[] Invariante a transformaciones positivas afines
				\4[] Independencia de las alternativas irrelevantes
				\4[] $\to$ Aumento de repartos alternativos no cambia óptimo
				\4[] Simetría
				\4[] $\to$ utilidad de ambos agentes es = de importante
				\4 Asignación que cumple con axiomas
				\4[] $(u_1, u_2) = \arg \underset{u \in G}{\max} \, (u_1-d_1)^\gamma(u_2 - d_2)^{1-\gamma}$
				\4[] $ \text{s.a.}: \, g(u_1,u_2) \leq 0$
				\4[] $\gamma$ interpretado como ``poder de negociación''
				\4[] $\to$ Cuantificado en parámetro $\gamma$
				\4[] Importancia relativa de cada agente:
				\4[] $\to$ Simetría puede relajarse
				\4[] $\to$ Simplemente, variando $\gamma$
			\3 Negociación estratégica
				\4 Stahl (1972), Rubinstein (1982)
				\4 Reparto es resultado de:
				\4[] Equilibrio de juego dinámico no cooperativo
				\4 Cada agente propone al otro un reparto
				\4[] Aceptar o rechazar $\to$ resultado del periodo
				\4[] Utilidad de reserva determina resultado
				\4[] $\to$ ¿Qué costes puedo infringir al otro?
				\4 Equivalencia con enfoque axiomático
				\4[] Bajo determinadas condiciones
				\4[] Solución caracterizable en términos de $\gamma$
		\2 Modelos de negociación empresa-sindicato
			\3 Idea clave
				\4 Teoría de la negociación anterior
				\4[] Aplicable en general
				\4[] Cualquier problema de reparto de un excedente
				\4 Mercado laboral
				\4[] Negociación sobre variables como:
				\4[] $\to$ Salario
				\4[] $\to$ Empleo
				\4[] $\to$ Otras dimensiones
				\4[] $\then$ Modelos específicos
			\3 Dunlop (1944) -- Sindicato monopolístico
				\4 Sindicato decide salario y empleo para:
				\4[] Maximizar utilidad de trab. representativo
				\4[] Empleo como trabajadores empleados
				\4[] $\to$ No horas de trabajo
				\4[] $\then$ $w$ y $L$ aportan utilidad positiva
				\4[] Sujeto restricción: demanda de empresa
				\4[] $\underset{w, L}{\max} \quad U(w,L)$
				\4[] $\text{s.a:} \quad L = L_D (w)$
				\4 Resultado no es Pareto-eficiente
				\4[] Podrían alcanzar mejor acuerdo
				\4[] Curvas isobeneficio-utilidad no son tangentes
				\4 Representación gráfica
				\4[] \grafica{dunlop}
			\3 Right-to-manage -- Nickell y Andrews (1983)
				\4 Generalización de Dunlop (1944)
				\4 Sindicato no tiene todo el poder
				\4 Negociación entre sindicato y empresa
				\4[] $\to$ Sobre el salario
				\4[] $\then$ Empresa decide empleo
				\4[] Negociación à la Nash
				\4[] Poder de negociación como parámetro
				\4[] $\to$ Determina equilibrio
				\4 Sólo salario de reserva es eficiente
				\4[] Curva isobeneficio-utilidad son tangentes
				\4[] Salario de reserva es mínimo
				\4[] Implica que empresa tiene todo poder de neg.
				\4 Representación gráfica
				\4[] \grafica{righttomanage}
			\3 Negociación débilmente eficiente
				\4 MacDonald y Solow (1981)
				\4 Empresas y sindicatos:
				\4[] negocian salario y empleo a la vez
				\4[$\then$] Empresa no decide empleo unilateralmente
				\4[$\then$] Posible alcanzar óptimos de Pareto
				\4 Óptimo concreto alcanzado
				\4[] Depende de poder de negociación de cada parte
				\4 Representación gráfica
				\4[] \grafica{negociaciondebil}
			\3 Negociación fuertemente eficiente
				\4 Empresa y sindicato negocian respecto a
				\4[] $\to$ Todas las variables relevantes
				\4 Por ejemplo, subsidio por desempleo
				\4 Posible mejorar débilmente eficiente
				\4 Ejemplo:
				\4[] Negociar mismo salario y subsidio ($b=w$)
				\4[] $\then$ Elimina riesgo para los trabajadores
				\4[] Empresa puede:
				\4[] $\to$ pagar menor salario
				\4[] $\to$ Contratar cantidad óptima
				\4[] $\then$ Óptimo de Pareto ``superior''
			\3 Insiders y outsiders
				\4 Modelos anteriores
				\4[] Trabajadores como grupo homogéneo
				\4[] Mismos objetivos de maximización
				\4[] Cambio de supuesto:
				\4[] $\to$ Empleados insiders y outsiders
				\4 Trabajadores desempleados
				\4[] Dispuestos a trabajar a mismo salario
				\4[] $\to$ O a rebajar arbitrariamente
				\4[] ¿Por qué empresas no contratan trabajadores...
				\4[] $\to$ ...dispuestos a trabajar por salario inferior?
				\4[] $\then$ ¿Por qué no reemplazar insiders por outsiders?
				\4 Costes de contratación y despido
				\4[] Ligados a:
				\4[] $\to$ Indemnizaciones por despido
				\4[] $\to$ Costes legales y administrativos
				\4[] $\to$ Capital humano específico acumulado
				\4[] $\then$ Impiden sustitución directa in. por out.
				\4 ¿Influye salario negociado en outsiders contratados?
				\4[] Pregunta clave del modelo insiders/outsiders
				\4[] Caracteriza salario negociado y empleo total
				\4[] $\to$ Frente a status-quo anterior
				\4 Insiders:
				\4[] buscan maximizar utilidad
				\4[] negocian salario con empresa que maximiza
				\4[] $\to$ Renta
				\4[] $\to$ Probabilidad de seguir trabajando
				\4[] $\to$ No tienen en cuenta a outsiders
				\4 Empresas:
				\4[] A partir de salario negociado con insiders
				\4[] Deciden cuanto trabajo contratar
				\4[] Si tenía demasiado trabajo:
				\4[] $\to$ Despide insiders
				\4[] $\to$ No contrata outsiders
				\4[] $\then$ Negocian salario más bajo que antes
				\4[] $\then$ Desempleo menos elevado que sin i/o
				\4[] Si necesita más trabajadores
				\4[] $\to$ Mantiene insiders
				\4[] $\to$ Contrata outsiders
				\4[] \quad Pero menos que si también negociasen $w$
				\4[] $\then$ Salario más alto que si todos negocian
				\4[] $\then$ Empresa contrata menos
				\4[] $\then$ Desempleo más elevado que con i/o
				\4 Discriminación salarial insiders--outsiders
				\4[] Insiders negocian también salario de outsiders
				\4[] Prefieren negociar salario outsiders bajo
				\4[] $\to$ Para que empresa aumente beneficios
				\4[] $\to$ Negociar salario más alto para ellos
				\4[] $\then$ Discriminación pero no desempleo
		\2 Estructura de la negociación colectiva\footnote{Ver Flanagan (1999).}
			\3 Idea clave
				\4 Contexto
				\4[] Nivel de negociación
				\4[] $\to$ Grado de centralización de la negociación
				\4[] Centralizada
				\4[] $\to$ Empresas vs sindicatos de todo un territorio
				\4[] Nivel intermedio
				\4[] $\to$ Empresas de un sector vs sindicatos de un sector
				\4[] Descentralizado
				\4[] $\to$ Trabajadores vs empresa o planta de producción
				\4[] Diferentes contextos institucionales determinados por
				\4[] $\to$ Legislación
				\4[] $\to$ Historia
				\4[] $\to$ Economía política
				\4 Objetivos
				\4[] Valorar efectos de nivel de negociación
				\4[] Considerar externalidades de las negociaciones
				\4 Resultados
				\4[] Predicciones sobre relación nivel de negociación-desempleo
				\4[] Dos familias de teorías
				\4[] $\to$ Relación decreciente centralización-desempleo
				\4[] $\to$ Relación en U invertida centralización desempleo
			\3 Formulación
				\4 Negociación centralizada
				\4[] Sindicatos federados en un sólo decisor
				\4[] Empresas asociadas en una patronal
				\4[] Negociación à la Nash
				\4[] $\to$ Sobre fracción de output para cada parte
				\4[] Imposible alterar precios relativos
				\4[] $\to$ Si $\uparrow$ salarial igual para todos sectores
				\4[] Resultado
				\4[] $\to$ Sin presión adicional sobre salarios
				\4[] $\to$ Sin presión a la baja sobre empleo
				\4[] $\to$ Modelos de negociación sección anterior
				\4 Negociación descentralizada con trabajo complementario
				\4[] Negociación por plantas o grupos de trabajadores
				\4[] Trabajo al que afecta acuerdo
				\4[] $\to$ Es complementario con otros sectores
				\4[] Ejemplo:
				\4[] $\to$ Ingenieros negocian con empresa
				\4[] $\to$ Operarios de planta negocian con empresa
				\4[] $\then$ Trabajo de ambos es complementario
				\4[] $\then$ Se afectan negativamente
				\4[] Aumento de salario negociado
				\4[] $\to$ Reduce demanda de otros trabajadores
				\4[] $\then$ Externalidad negativa
				\4[] Sin incentivos a internalizar externalidad
				\4[] $\to$ Exigen aumentos de salario
				\4[] Resultado
				\4[] $\to$ Presión salarial al alza
				\4[] $\to$ Caída del empleo
				\4 Negociación descentralizada con trabajo sustitutivo
				\4[] Negociación por plantas o empresas
				\4[] Trabajo al que afecta el acuerdo
				\4[] $\to$ Es sustitutivo con otros
				\4[] Ejemplo:
				\4[] $\to$ Operarios de máquinas en diferentes empresas
				\4[] Aumento de salario negociado
				\4[] $\to$ Aumenta demanda de otras empresas
				\4[] $\then$ Externalidad positiva
				\4[] Incentivos a no exigir demasiadas mejoras
				\4[] $\to$ Demanda puede irse a otras empresas
				\4[] Elasticidad de la demanda de trabajo
				\4[] $\to$ Aumenta fuertemente
				\4[] $\then$ Más poder de negociación
				\4[] Resultado
				\4[] $\to$ Menor presión inflacionaria
				\4[] $\to$ Más empleo
				\4[] Caso extremo
				\4[] $\to$ Negociación individual de cada trabajador
				\4[] $\then$ Demanda de trabajo perfectamente elástica
				\4[] $\then$ Trabajadores no tendrían poder
				\4 Negociación intermedia
				\4[] Negociación por sectores
				\4[] Trabajo al que afecta el acuerdo
				\4[] $\to$ Todo el que trabaja en un sector
				\4[] Empresas pueden repercutir aumentos a bienes
				\4[] $\to$ Aumenta precio relativo de bien del sector
				\4[] $\then$ Menos poder de negociación de empresas
				\4[] $\then$ Más incentivos a presionar por sindicatos
				\4[] Resultado
				\4[] $\to$ Más presión inflacionaria
				\4[] $\to$ Menos empleo
				\4 Economía abierta
				\4[] Salario real de consumo $\neq$ de producción
				\4[] Salario real de consumo
				\4[] $\to$ Medido en unidades de consumo
				\4[] Salario real de producción
				\4[] $\to$ Medido en cantidad de producto trabajado
				\4[] Nivel de precios depende también de importaciones
				\4[] $\to$ No es posible repercutir totalmente salarios a precios
				\4[] Incentivo a negociar aumento en todas situaciones
				\4[] $\to$ Aumenta salario real de consumo
				\4[] $\then$ Neg. centralizada sí puede aumentar salario real
				\4[] Descentralizada
				\4[] $\to$ Competencia con otras plantas nacionales y extranjeras
				\4[] $\then$ Menor poder de negociación de sindicatos
				\4[] Internalización de efectos sobre BPagos depende de:
				\4[] $\to$ Estructura institucional
				\4[] $\to$ Productividad
				\4[] Resultado:
				\4[] $\to$ Posible relación creciente centralización-desempleo
			\3 Implicaciones
				\4 Relación decreciente centralización-desempleo
				\4[] Si:
				\4[] $\to$ Centralización reduce inflación salarial
				\4[] $\to$ Descentralización con trabajo complementario
				\4 Relación en forma de U invertida: Calmfors y Driffil (1988)
				\4[] Si:
				\4[] $\to$ Centralización reduce inflación salarial
				\4[] $\to$ Descentralización con trabajo sustitutivo
				\4 Relación creciente inflación salarial-centralización
				\4[] Si:
				\4[] $\to$ Sindicatos no internalizan efectos BP
				\4[] $\to$ Descent. con trabajo sustit. con plantas extranjeroas
			\3 Valoración
				\4 Enfoques de estimación
				\4 Evidencia empírica
		\2 Coordinación de la negociación colectiva
			\3 Idea clave
				\4 Contexto
				\4[] Negociaciones son costosas
				\4[] Para reducir costes
				\4[] $\to$ Fijación de sendas temporales de salarios
				\4 Objetivos
				\4[] Valorar efectos de sincronización sobre empleo y salarios
				\4[] Relación entre sincronización y nivel de negociación
				\4 Resultados
				\4[] Menos sincronización aumenta salarios y desempleo
				\4[] Ajuste más lento
				\4[] Relación imperfecta nivel de neg.-sincronización
			\3 Formulación
				\4 Senda temporal del acuerdo
				\4[] Establece variación periódica del salario nominal
				\4 Determinación del salario real
				\4[] Depende de evolución del nivel de precios
				\4 Indexación de salarios a inflación
				\4[] Habitualmente, protección sólo parcial:
				\4[] $\to$ Revalorización parcialmente indexada
				\4[] $\to$ Suelos frente a deflación
				\4[] $\to$ Subidas de SNominal diferidas respecto a precios
				\4 Shocks en mercado de bienes
				\4[] Requieren ajuste de demanda de trabajo
				\4[] Con salarios nominalmente rígidos
				\4[] $\to$ Ajuste se produce vía precios
				\4 Escalonamiento de negociaciones
				\4[] Negociación depende de otros contratos en vigor
				\4[] Sindicatos rechazan bajadas o menores revalorizaciones
				\4[] $\to$ Pérdida relativa de poder adquisitivo
				\4[] En mejor de los casos
				\4[] $\to$ Aceptan iniciar senda de ajuste salarial
				\4 Proceso hasta alcanzar equilibrio
				\4[] Mayor salario real
				\4[] Mayor desempleo
				\4 Resultado de mayor solapamiento por asincronía
				\4[] Mayor presión salarial
				\4[] Mayor desempleo
			\3 Implicaciones
				\4 Sincronización reduce inflación y desempleo
				\4[] Evita rechazo a reducción relativa de salario
				\4 Correlación imperfecta sincronización y niveles
				\4[] Generalmente, negociación de nivel más centralizado
				\4[] $\to$ Mayor grado de sincronización
				\4[] Pero no siempre
				\4[] Centralización y poca sincronización:
				\4[] $\to$ Acuerdo de revalorización a nivel centralizado
				\4[] $\to$ Implementación lenta de acuerdo en niveles más bajos
			\3 Valoración
				\4 Relativamente menos atención que nivel de centralización
				\4 Modelos macro del ciclo NEK sí modelizan
				\4[] Staggered contracts de Taylor
		\2 Valoración
			\3 Sindicación en la práctica
				\4 Dos variables relevantes:
				\4[] $\to$ Generalmente relacionadas
				\4[] Tasa de sindicación
				\4[] Tasa de cobertura de acuerdo colectivo
				\4[] $\to$ TCobertura más relevante
				\4 Contextos legales muy heterogéneos
				\4[] Closed-shop
				\4[] $\to$ Afiliación obligatoria
				\4[] Convenios limitados a afiliados
				\4[] $\to$ No permitido en España
				\4[] $\to$ Habitual otros países europeos
				\4[] ...
			\3 Efectos de sindicación y negociación
				\4 Salarios
				\4[] Trabajos muestran diferencial entre:
				\4[] $\to$ Sindicados y no sindicados
				\4[] En países donde convenio sólo sindicados
				\4[] En economías con industrias con y sin convenio
				\4 Beneficios y productividad
				\4[] Beneficio empresarial
				\4[] $\to$ Efecto ambiguo
				\4[] $\to$ Habitual caída de precio de acción
				\4[] Productividad
				\4[] $\to$ Resultados ambiguos
				\4[] $\to$ En ocasiones, aumenta productividad
				\4[] $\to$ Internalización de objetivo de empresa por sindicatos es importante
			\3 Eficiencia de la negociación colectiva
				\4 Negociación fuertemente eficiente
				\4[] $\to$ Generalmente se rechaza
				\4 Right-to-manage
				\4[] ¿Empresas contratan sobre curva de demanda?
				\4[] Resultados contradictorios
			\3 Dispersión de salarios
				\4 Evidencia empírica relativamente robusta
				\4 Más centralización, menos dispersión
			\3 Resultados globales
				\4 Mucha ambigüedad y heterogeneidad
				\4 Difícil derivar conclusiones claras
				\4 Sólo dos indicios claros
				\4[] PMg de trabajo no iguala salario externo
				\4[] $\to$ Mercado de trabajo difícilmente competitivo
				\4[] Relación negativa entre L y salario negociado
				\4[] $\to$ Programa de investigación sobre desempleo
	\1 \marcar{Salarios de eficiencia}
		\2 Idea clave
			\3 Contexto
				\4 Competencia salarial en modelo neoclásico
				\4[] Si hay exceso de oferta de trabajo
				\4[] $\then$ Existe desempleo
				\4[] Desempleados ofrecerán trabajo más barato
				\4[] $\then$ Salario baja hasta eliminar exceso de dda.
				\4 Evidencia empírica generalizada sobre desempleo
				\4[] Existe desempleo
				\4[] $\to$ Trabajadores no pueden trabajar a salario de otros
				\4[] Salarios no bajan o bajan muy lentamente
				\4[] Trabajadores dispuestos a trabajar a salario de mercado
				\4[] $\to$ Pero no encuentran trabajo a ese salario
				\4[] Empresas podrían contratar más a menor salario
				\4[] $\to$ Pero no lo consiguen
				\4 Evidencia empírica generalizada sobre salario real
				\4[] Salario real es \underline{sólo} débilmente procíclico
				\4[] $\to$ Muy débil ajuste ante aumento de productividad
				\4[]  Muy pequeña variación ante $\Delta$ de la demanda
				\4[] $\to$ Ajustes sobre todo en cantidad
				\4[$\then$] Empleo y paro fluctúan más que salario
				\4[$\then$] Salario real aumenta ligeramente con ciclo
				\4 Trabajo puede medirse en unidades de eficiencia
				\4[] Cantidad de input efectivo en proceso de producción
				\4 Unidades de eficiencia dependen de:
				\4[] Horas de trabajo
				\4[] Otros factores
				\4[] $\to$ Esfuerzo aplicado
				\4[] $\to$ Salud de trabajadores
				\4[] $\to$ Calidad intrínseca de trabajadores atraídos
				\4[] ...
				\4 Propiedades intrínsecas de inputs:
				\4[] Generalmente independientes de su precio
				\4 Propiedades intrínsecas de trabajo
				\4[] Sí pueden depender de su precio
				\4 Ejemplos de efectos de más salario:
				\4[] Mejor alimentación y más salud
				\4[] $\to$ Más fuerza, más inteligencia
				\4[] $\then$ Mejor trabajo
				\4[] Trabajadores más productivos se presentan voluntarios
				\4[] $\to$ Más productividad media de candidatos
				\4[] Trabajadores incentivados a trabajar mejor
				\4[] $\to$ Porque pérdida de empleo más costosa
				\4[] Trabajadores menos propensos a cambiar de trabajo
				\4[] $\to$ Más probable que no encuentren trabajo mejor
				\4 Decisión de empresas sobre salario
				\4[] En contexto de interacción salario-eficiencia
				\4[] Bajada de salario
				\4[] $\to$ Reducción de costes
				\4[] Aumento de salario
				\4[] $\to$ Aumento de eficiencia
				\4[] Necesario ponderar decisiones
				\4[] No necesariamente óptimo bajar
				\4[] $\to$ Ante caída de demanda de output
			\3 Objetivos
				\4 Formular efectos de salario sobre eficiencia
				\4 Representar decisión óptima de empresas
				\4 Caracterizar efecto sobre desempleo y salario real
			\3 Resultados
				\4 Familia de modelos
				\4 Diferentes explicaciones de relación salarios-eficiencia
				\4[] Salop (1979)
				\4[] $\to$ Reducción de rotación con salarios más altos
				\4[] Calvo (1981)
				\4[] $\to$ Amenaza de despido en contexto estático
				\4[] Shapiro y Stiglitz (1984)
				\4[] $\to$ Desempleo de equilibrio
				\4[] $\to$ Salario superior a equilibrio
				\4[] $\then$ Evitar realizar menos esfuerzo
				\4[] $\then$ Salario de eficiencia
				\4 Explicación satisfactoria de rigideces reales
				\4[] Empresas no ajustar salario real a equilibrio competitivo
				\4 Explicación de salarios reales pro-cíclicos
				\4[] Caída del desempleo
				\4[] $\to$ Aumenta probabilidad de encontrar trabajo
				\4[] $\to$ Reduce penalización por despido
				\4[] $\then$ Necesario mayor salario para inducir esfuerzo
				\4 Explicación de desempleo sin rigideces nominales
				\4 Modelo de rigideces reales
				\4 Difícil estimación empírica
				\4[] ¿Cómo cuantificar medir esfuerzo?
				\4[] Problemas de identificación salario-esfuerzo
		\2 Modelo básico de salarios de eficiencia
			\3 Idea clave
				\4 Contexto
				\4[] Eficiencia de trabajo depende de precio
				\4[] Múltiples razones posibles
				\4[] Similar efecto en todas
				\4[] $\to$ Aumento de unidades de trabajo eficiente
				\4 Objetivos
				\4[] Caracterizar decisión salario-demanda de trabajo
				\4[] Valorar trade-off
				\4[] $\to$ Mayor eficiencia
				\4[] $\to$ Mayor coste
				\4 Resultados
				\4[] Aumento de salario pagado hasta
				\4[] $\to$ Elasticidad de eficiencia a salario igual a 1
				\4[] $\then$ Aumento de trabajo eficiente iguala coste
			\3 Formulación
				\4 Empresa maximiza beneficios
				\4[] Elige:
				\4[] $\to$ Salario
				\4[] $\to$ Trabajo a contratar
				\4[] Trabajo efectivo depende de:
				\4[] $\to$ Salario
				\4[] $\to$ Trabajo demandado
				\4 Problema de maximización
				\4[] $\underset{w,L}{\max} \quad \Pi = R(e(w) L ) - w L$
				\4[] $\text{CPO:} \quad \Pi_W = 0 \then R' \cdot e' (w) \cdot L - L = 0$
				\4 Óptimo:
				\4[] $\quad \quad \quad \quad \then$ (i) $R' = \frac{1}{e'(w)}$
				\4[] $\quad \quad \quad \, \, \, \, \Pi_L = 0 \then F' \cdot e(w) - w = 0$
				\4[] $\quad \quad \quad \quad \then$ (ii) $R' = \frac{w}{e}$
				\4[] $\then$ \fbox{ $\frac{d \, e}{d \, w} \cdot \frac{w}{e(w)} = 1$ }
				\4[] $\then$ $w$ depende exclusivamente de forma de $e(w)$
			\3 Implicaciones
				\4 Subida de salarios tiene dos efectos:
				\4[] Aumento de coste salarial
				\4[] $\to$ por $w\cdot L$
				\4[] Aumento de ingreso por más trabajo eficiente:
				\4[] $\to$ por $e(w) L$
				\4 Empresa sube salario hasta que:
				\4[] $\to$ Coste adicional iguala aumento de ingresos
				\4 Representación gráfica
				\4[] \grafica{eficienciabasico}
				\4[] Elasticidad de esfuerzo a salario es 1
				\4[] $\to$ Sin mejoras de esfuerzo que aprovechar
				\4[] $\to$ Sin reducción de costes que aprovechar
			\3 Valoración
				\4 Mecanismo básico de decisión
				\4 Salario nominal depende de forma de $e(w)$
				\4 Forma funcional de $e(w)$ determina salario
				\4[] Si elasticidad de eficiencia a salario > 1
				\4[] $\to$ Empresas prefieren pagar salario > equilibrio
		\2 Shapiro y Stiglitz (1984):información asimétrica
			\3 Idea clave
				\4 Contexto
				\4[] En contexto de modelo neoclásico
				\4[] $\to$ Esfuerzo de trabajadores es conocido
				\4[] Si esfuerzo desconocido y pleno empleo
				\4[] $\to$ Trabajador despedido encuentra trabajo inmediatamente
				\4[] $\then$ Trabajadores incentivados a no esforzarse
				\4[]  Shapiro y Stiglitz (1984)
				\4[] $\to$ Heredero de Salop (1979)
				\4[] Aplicación de modelización formal de riesgo moral
				\4 Objetivos
				\4[] Explicar desempleo como medida disciplinaria
				\4[] Caracterizar relación salarios-penalización por despido
				\4 Resultados
				\4[] Aplicación de programa de investigación de riesgo moral
				\4[] Modelo de rigideces reales del mercado laboral
				\4[] $\to$ Salario real no se ajusta para vaciar mercados
				\4[] Curva de Phillips salarios decrecientes en paro
				\4[] Salario real procíclico
				\4[] $\to$ Compatible con evidencia empírica
			\3 Formulación
				\4 Trabajadores
				\4[] Tres estados posibles
				\4[] $\to$ (E): trabajando y esforzándose
				\4[] $\to$ (S): trabajando y \textit{shirking}\footnote{Vagueando, escaqueándose.}
				\4[] $\to$ (U): desempleado
				\4 Utilidad de trabajadores
				\4[] $U = \int_0^\infty  u(t) e^{-\rho t} \, dt$ $\quad$ $\rho > 0$
				\4[] Si empleado: $u(t) = w(t) - e(t)$
				\4[] Si desempleado: $u(t) = 0$
				\4 Utilidad de estar trabajando depende de:
				\4[] $\to$ (+) salario
				\4[] $\to$ (-) esfuerzo realizado
				\4[] $\to$ (-) probabilidad de ser despedido por causas exógenas
				\4[] $\to$ (-) probabilidad de despido por vaguear
				\4[] $\to$ (+) utilidad de estar desempleado
				\4 Utilidad de estar desempleado depende de:
				\4[] $\to$ (+) Subsidio de desempleo
				\4[] $\to$ (+) Probabilidad de encontrar trabajo
				\4 Comparación (E) Esfuerzo vs (S) Shirking
				\4[] Si realizando esfuerzo:
				\4[] $\to$ Probabilidad de despido por shirking es 0
				\4[] $\then$ Menos ponderación a utilidad de desempleo
				\4[] Si shirking:
				\4[] $\to$ Probabilidad mayor de ser despedido
				\4[] $\then$ Mayor ponderación a utilidad de despido
				\4 Aumento de utilidad de desempleados
				\4[] Aumenta utilidad descontada de shirking
				\4[] $\then$ Necesario mayor salario para que $E \succ S$
				\4 Empresas
				\4[] Contratan trabajo
				\4[] Ingreso depende de trabajo efectivo
				\4[] $\to$ Trabajo efectivo crece con esfuerzo
				\4[] $\then$ Buscan $\uparrow$ esfuerzo hasta que $\uparrow$ ingreso = $\uparrow$ coste
				\4 ¿Qué salario ofrecer?
				\4[] Objetivo: valor descontado de E sea mayor que S
				\4[] $\to$ Ofrecen salario mínimo que lo induce
				\4[] $\to$ ``\textit{No-shirking condition}'' (NSC)
				\4[] $\then$ $w_\text{ns} = f(\underset{+}{L}, \underset{-}{q}, \underset{+}{\bar{e}})$
				\4[] Salario ofrecido depende de:
				\4[] $\to$ Trabajo que demandan (+)
				\4[] $\to$ $q$: Posibilidad de capturar (-)
				\4[] $\to$ $e$: esfuerzo exigido (+)
				\4 Equilibrio:
				\4[] Salario $w^*$ que cumple NSC
				\4[] Trabajo que maximiza beneficios dado $w^*$
				\4 Representación gráfica
				\4[] \grafica{shapirostiglitz}
			\3 Implicaciones
				\4 Desempleo necesario
				\4[] Sirve para disciplinar empleados
				\4[] En caso contrario
				\4[] $\to$ Encuentran empleo inmediatamente al despedir
				\4[] $\then$ Amenaza de despido no tiene efecto
				\4[] $\then$ Imposible inducir esfuerzo
				\4 Subsidios por desempleo
				\4[] $\to$ Necesario aumentar coste de desempleo vía $\uparrow$ tasa paro
				\4[] $\then$ Aumentan desempleo de equilibrio
				\4[] $\then$ Mecanismo diferente al de modelos de búsqueda
				\4 Salarios se ajustan lentamente ante shocks
				\4[] $\to$ Caída demanda aumenta paro
				\4[] $\to$ Aumento de paro permite bajar salarios
				\4[] $\to$ Proceso lento en el tiempo
				\4 Equilibrio es Pareto-subóptimo
				\4[] $\to$ Información asimétrica es costosa
		\2 Extensiones
			\3 Bajada de salario si no ejercen esfuerzo
				\4 En vez de despedir
				\4 NSC es una recta
				\4[] Variaciones de la demanda de trabajo
				\4[] $\to$ Recaen enteramente sobre empleo
				\4[] $\then$ Fluctuación más acorde con hechos estilizados
			\3 Otras fuentes de ineficiencia
				\4 Aumento de motivación, lealtad, confianza...
				\4 Modelizar respuesta psicológica
				\4 Otros modelos
		\2 Valoración
			\3 Implicaciones aproximadamente confirmadas
				\4 Firmas con más capital y tamaño
				\4[] Más vulnerables a esfuerzo bajo
				\4[] $\to$ Deberían pagar salarios más altos
				\4 Firmas con menor coste del capital
				\4[] Pueden permitirse más inversión
				\4[] En países en desarrollo pueden pagar más salario
				\4[] $\to$ Para aprovechar mejoras de eficiencia
			\3 Dificil establecer causalidad
				\4 ¿Cuál de las dos?:
				\4[] $\to$ Firmas más productivas pagan más salario
				\4[] $\to$ Salarios más productivos aumentan eficiencia
				\4 Pero salarios de eficiencia
				\4[] Buena explicación teórica de salarios más altos
			\3 Persistencia del desempleo
				\4 Salarios de eficiencia explicación relevante
				\4[] Junto con otros modelos
				\4 No predice valor concreto de desempleo
				\4[] Dificil cuantificación de resultados
	\1 \marcar{Contratos implícitos}\footnote{Ver Rosen (1985) en JEL.}
		\2 Idea clave
			\3 Contexto
				\4 Adam Smith
				\4[] Salarios dependen del riesgo de la remuneración
				\4[] $\to$ Más riesgo implica mayores salarios
				\4[] $\then$ Salarios se igualan corregidos por riesgo
				\4 Frank Knight
				\4[] Emprendedor como asegurador residual
				\4[] Empresario obtiene beneficio
				\4[] $\to$ Porque asume riesgos en múltiples contratos
				\4 Baily (1974), Gordon (1974), Azariadis (1975)
				\4 Laffont, Tirole y otros
				\4 Concepto de contratos implícitos
				\4[] Muchas otras aplicaciones
				\4[] Ejemplo clave:
				\4[] $\to$ Equity como compra de call
				\4[] $\to$ Deuda como venta de put
				\4 Contrato de trabajo no es contrato spot
				\4 Relación duradera
				\4[] Partes consideran suma de FC descontados
				\4 Flujos sujetos a fluctuaciones aleatorias
				\4[] $\to$ PMgL sujeta a fluctuaciones aleatorias
				\4 Énfasis en contrato en sí mismo
				\4[] Mercado de trabajo no es relación impersonal
				\4[] $\then$ Relaciones bilaterales de largo plazo
				\4 Patrón de despidos y contratación
				\4[] Difícil de explicar
				\4[] Empresas contratan y despiden de golpe
				\4[] En vez de ajustar horas de trabajo o salarios
			\3 Objetivo
				\4 Caracterizar efecto de contratos de larga duración
				\4[] $\to$ Sobre desempleo
				\4[] $\to$ Sobre salarios
				\4 Valorar efecto de subsidios de desempleo
				\4[] Sobre volatilidad del empleo
				\4[] Sobre salarios
			\3 Resultados
				\4 Familia de modelos muy amplia
				\4 Empleo como contrato duradero
				\4[] No es sólo un contrato spot
				\4 Contratos de trabajo incluyen contratos implícitos
				\4[] Aseguramiento de desempleados
				\4[] $\to$ Frente a fluctuaciones en productividad marginal
				\4 Contratos de trabajo aumentan volatilidad de empleo
				\4 Análisis de contratos especialmente relevante en L
				\4[] Relación de largo plazo
				\4[] Fuerte componente personal
				\4 Se desvía de visión keynesiana de desempleo involuntario
				\4[] No basado en explicaciones keynesianas tradicionales
				\4[] $\to$ Rigideces de precios
				\4[] $\to$ Ilusión monetaria
				\4[] $\to$ Mercados que no se vacían
				\4[] $\then$ Desempleo como fallo de mercado
				\4 Desempleo no es necesariamente fallo de mercado
				\4[] Puede resultar de preferencias de agentes
				\4[] $\to$ Reducir riesgo sobre salarios
				\4[] $\to$ Neutralidad al riesgo en empresas
				\4 Empleo más volátil cuanto más aseguramiento
				\4[] Empresas mayores incentivos a despedir
				\4 Trabajadores demandan seguridad con más subsidios desempleo
				\4[] Saben que empresas preferirán despedir
				\4[] $\to$ Si pueden desviar a subsidios de desempleo
				\4[] Exigirán mayor seguridad salarial
				\4[] $\to$ Saben podrán acceder a rentas por subsidios
		\2 Formulación
			\3 Explicación basada en maximización de empresa
				\4 Empresas resuelven:
				\4[] $\underset{L_i}{\max} \quad \sum_{i=1}^K p_i \left[ A_i F(L_i) - w L_i \right]$
				\4[] $\to$ Beneficio en cada estado $i$
				\4 Trabajadores obtienen:
				\4[] $\sum_{i=1}^K p_i U(C_i) - V(L_i) $
				\4[] Tienen utilidad de reserva $U_0$
				\4 Lagrangiano que maximizan empresas
				\4[] $\mathcal{L} = \sum_{i=1}^K p_i \left[ A_i F(L_i) - w L_i \right] +$
				\4[] $ + \lambda \left( \left[ \sum_{i=1}^K p_i ( U(C_i) - V(L_i) ) \right] - U_0 \right)$
				\4 Condiciones de primer orden:
				\4[] $U(C_i) = \frac{1}{\lambda}$ $\forall \, i$ $\then$ $U(C) = \frac{1}{\lambda}$
				\4[] $p_i A_i F'(L_i) = \lambda p_i V' (L_i)$
				\4[] \fbox{$A_i F'(L_i) = \frac{V'(L_i)}{U'(C)}$}
			\3 Explicación basada en salario compuesto
				\4 Empresa:
				\4[] neutral al riesgo
				\4[] Incentivos a mantener relación contractual
				\4[] $\to$ Inversiones en capital humano
				\4[] $\to$ Costes fijos incurridos
				\4 Trabajadores
				\4[] Aversos al riesgo
				\4[] Quieren mantener flujo constante de consumo
				\4 Salario suma de dos componentes
				\4[] $\to$ Productividad marginal del trabajo
				\4[] $\to$ Prima o indemnización según estado de naturaleza
				\4[$\Rightarrow$] $\bar{w} = \text{PMgL} + \gamma$
				\4[] $\bar{w}$: fijo
				\4[] $\gamma$: positivo si $\bar{w} > \text{PMgL}$
				\4[$\then$] Aseguramiento frente a fluctuaciones
				\4[$\then$] Salario se mantiene rígido
				\4 Estado de la naturaleza adverso
				\4[] Empresa paga:
				\4[] $\to$ Productividad marginal del trabajo
				\4[] $\to$ Indemnización para cubrir diferencia
				\4[] Si coste de indemnizaciones elevado y despido bajo
				\4[] Empleado paga:
				\4[] $\to$ Prima de aseguramiento
				\4[] $\to$ Empresa despide empleados
				\4 Estado de la naturaleza favorable
				\4[] Empresa paga:
				\4[] $\to$ Productividad marginal del trabajo
				\4[] Empleado paga:
				\4[] $\to$ Prima de aseguramiento
				\4[] Salario real
				\4[] $\to$ Se mantiene constante
		\2 Implicaciones
			\3 Ingresos reales constantes
				\4 Modelo muestra salario real rígido
			\3 Salario acíclico
				\4 Shocks de productividad positivos
				\4[] Aumenta productividad y output
				\4[] $\to$ Salario se mantiene constante
				\4[] $\to$ Aumenta trabajo
				\4[] $\Rightarrow$ Salario sin tendencia cíclica
				\4 Contradice hecho empírico
				\4[] Salario real levemente procíclico
			\3 Despidos concentrados
				\4 ¿Por qué empresa despide?
				\4[] Si productividad baja mucho
				\4[] $\to$ Carga financiera de salarios muy alta
				\4 Empresas transfieren carga despidiendo
				\4[] $\to$ Subsidios por desempleo toman relevo
				\4[] Si existen costes fijos de despido
				\4[] $\to$ Despidos concentrados
				\4 Despidos como prima de aseguramiento
				\4[] Empleados aceptan probabilidad de ser despedidos
				\4[] $\to$ Como pago por evitar reducción de $w$
			\3 Volatilidad del empleo aumentada
				\4 Factores que aumentan volatilidad
				\4[] Presencia de subsidio de desempleo
				\4[] $\to$ Empresas tienen incentivo a desviar hacia paro
				\4[] Empleados pueden exigir más seguridad salarial
		\2 Valoración
			\3 Teoría de contratos aplicada a mercado de trabajo
			\3 Escasa incorporación en otros modelos
			\3 Programa de investigación relativamente abandonado
	\1[] \marcar{Conclusión}
		\2 Recapitulación
			\3 Negociación salarial
			\3 Salarios de eficiencia
			\3 Contratos implícitos
		\2 Idea final
			\3 Relación con otras disciplinas
				\4 Muy relevantes para el análisis del m. de trabajo
				\4[] Derecho
				\4[] $\to$ Legislación a menudo compleja
				\4[] $\to$ Mercado en ocasiones muy regulado
				\4[] Psicología
				\4[] $\to$ Determinantes de motivación y esfuerzo
				\4[] Macroeconomía
				\4[] $\to$ Trabajo es elemento central
			\3 Cambios recientes en desigualdad salarial\footnote{Mirar en Palgrave: \textit{wage inequality, changes in}.}
				\4 Últimas décadas
				\4[] $\to$ Desde años 70
				\4[] $\to$ En general en economías desarrolladas
				\4[] $\to$ En economías muy distintas: USA vs Europa
				\4[] Mercados lab. han mostrado tendencias
				\4 Aumento de desigualdad salarial
				\4[] Brecha 90-10: aumento generalizado
				\4[] $\to$ Percentil 90 entre percentil 10
				\4 Modelos de determinación de salarios
				\4[] Jugado papel en explicación
				\4[] También otras teorías
				\4[] $\to$ Demanda relativa de trab. cualificado
				\4[] $\to$ Cambio tecnológico sesgado a high skills
				\4[] $\to$ Aumento del comercio internacional
				\4[] $\to$ Salario mínimo
				\4[] $\to$ Aumento poder de mercado y cuasirentas
				\4[] $\to$ ...
\end{esquemal}



























\graficas

\begin{axis}{4}{Modelo de Dunlop (1944): salario y empleo de equilibrio cuando sindicato optimiza utilidad de trabajador representativo respecto de salario y empleo sujeto a demanda de trabajo.}{$L$}{$w$}{dunlop}
	
	
	% demanda de trabajo
	\draw[-] (0,3.5) -- (3.5,0);
	\node[right] at (3,0.5){$L_D = L_D(L,w)$};
	
	% curva de utilidad de sindicato
	% equilibrio
	\draw[-] (0.5,3.51) to [out=275, in=170](2.4,1.9);
	% curva del sindicato inalcanzable dada función de demanda
	\draw[dashed] (1,4.01) to [out=275, in=170](2.9,2.4);
	
	% salario y trabajo de equilibrio
	\node[circle,fill=black,inner sep=0pt,minimum size=4pt] (a) at (0.98,2.52) {};
	
	% salario de equilibrio
	\draw[dashed] (0,2.52) -- (4,2.52);
	\node[left] at (0,2.52){\tiny $w^*$};
	
	% salario de reserva
	\draw[-] (0,1) -- (4,1);
	\node[left] at (0,1){$b$};
	
	% isobeneficio
	\draw[-] (0.02,2.23) to [out=30, in=150](1.97,2.23);
\end{axis}


\begin{axis}{4}{Modelo de negociación salarial ``\textit{right-to-manage}'': empresa y sindicato negocian salario y empresa decide nivel de empleo}{$L$}{$w$}{righttomanage}
	% demanda de trabajo
	\draw[-] (0,3.5) -- (3.5,0);
	\node[right] at (3,0.5){$L_D = L_D(L,w)$};
	
	% curva de utilidad de sindicato
	% equilibrio
	\draw[-] (0.5,3.51) to [out=275, in=170](2.4,1.9);
	
	% curva del sindicato inalcanzable dada función de demanda
	\draw[dashed] (1,4.01) to [out=275, in=170](2.9,2.4);
	
	% salario y trabajo de equilibrio
	\node[circle,fill=black,inner sep=0pt,minimum size=4pt] (a) at (0.98,2.52) {};
	
	% salario de equilibrio de monopolio
	\draw[dashed] (0,2.52) -- (4,2.52);
	\node[left] at (0,2.52){\tiny $w^*$};
	
	% salario de reserva
	\draw[-] (0,1) -- (4,1);
	\node[left] at (0,1){$b$};
	
	% isobeneficio de sindicato monopolista
	\draw[-] (0.02,2.23) to [out=30, in=150](1.97,2.23);
	
	\draw[-] (1.02,1.23) to [out=30, in=150](2.97,1.23);

	\draw[-] (1.57,0.71) to [out=30, in=150](3.49,0.71);
\end{axis}

\begin{axis}{4}{Modelo de negociación débilmente eficiente: empresa y sindicato negocian salario y trabajo empleado al mismo tiempo.}{$L$}{$w$}{negociaciondebil}		
	% isobeneficio de sindicato monopolista
	\draw[-] (0.02,2.23) to [out=30, in=150](1.97,2.23);
	
	\draw[-] (1.57,0.71) to [out=30, in=150](3.49,0.71);
	
	% isobeneficio de monopolista tangente a curva de utilidad de sindicato

	% demanda de trabajo
	\draw[-] (0,3.5) -- (3.5,0);
	\node[right] at (3,0.5){$L_D = L_D(L,w)$};
	
	% curva de utilidad de sindicato
	% equilibrio
	\draw[-] (0.5,3.51) to [out=275, in=170](2.4,1.9);
	
	% salario y trabajo de equilibrio
%	\node[circle,fill=black,inner sep=0pt,minimum size=4pt] (a) at (0.98,2.52) {};
	
	% salario de equilibrio de monopolio
	\draw[dashed] (0,2.52) -- (4,2.52);
	\node[left] at (0,2.52){\tiny $w^*$};
	
	% salario de reserva
	\draw[-] (0,1) -- (4,1);
	\node[left] at (0,1){$b$};
	
	% isobeneficio de empresa
	\draw[-] (0.02,2.23) to [out=30, in=150](1.97,2.23);
	
	\draw[-] (1.57,0.71) to [out=30, in=150](3.49,0.71);
	
	% isobeneficio de óptimo de Pareto
	\draw[-] (0.45,1.8) to [out=30, in=150](2.4,1.8);
	
	% equilibrio de óptimo de Pareto
	\node[circle,fill=black,inner sep=0pt,minimum size=4pt] (a) at (1.92,2.02) {};	
	
\end{axis}

\begin{axis}{4}{Representación gráfica de un modelo de salarios de eficiencia simple.}{$w$}{$e(w)$}{eficienciabasico}
	% Curva de eficiencia
	\draw[-] (0,0) to [out=5, in=200](3,2.5);
	\draw[-] (3,2.5) to [out=20, in=181](4,2.7);
	
	% Recta tangente a curva de eficiencia
	\draw[-] (0,0) -- (4,3.61);
	
	% eficiencia de equilibrio
	\draw[dotted] (0,2.1) -- (2.33,2.1) -- (2.33,0);
	\node[left] at (0,2.1){\tiny $e(w^*)$};
	\node[below] at (2.33,0){\tiny $w^*$};
\end{axis}

\begin{axis}{4}{Modelo de Shapiro y Stiglitz (1984): representación del equilibrio como intersección entre NSC y demanda de trabajo.}{}{$w$}{shapirostiglitz}
	% extender los ejes
	\draw[-] (4,0) -- (5,0);
	\node[below] at (5,0){NL};
	
	% Oferta de trabajo
	\draw[dotted] (4,0) -- (4,4);
	\node[below] at (4,0){\tiny $\bar{L}$};
	
	% Esfuerzo
	\draw[-] (0,1) to [out=0, in=270](4.65,4);
	\node[left] at (0,1){\tiny $\bar{e}$};
	
	% Demanda de trabajo
	\draw[-] (1.5,4) -- (4,1.5);
	\node[above] at (1.5,4){\tiny $L^D$};
	
	% Non-shirking condition
	\draw[dotted] (0,1.2) to [out=2, in=265](3.95,4);
	\node[above] at (4,4){\tiny NSC};
	
	% Equilibrio: Demanda de trabajo = NSC
	\node[circle,fill=black,inner sep=0pt,minimum size=4pt] (a) at (3.14,2.34) {};
	\node[above] at (3.14,2.37){\tiny E};
	
	% Salario de eficiencia
	\draw[dashed] (0,2.37) -- (4,2.37);
	\node[left] at (0,2.37){\tiny $w_\text{NSC}$};
	
	% Trabajo contratado
	\draw[dashed] (3.14,2.34) -- (3.14,0);
	\node[below] at (3.14,0){\tiny $L_\text{NSC}$};
\end{axis}


\preguntas

\seccion{Test 2011}
\textbf{16.} Indique cuál de las siguientes afirmaciones es la correcta:

\begin{itemize}
	\item[a] En los modelos insider-outsider, los sindicatos negocian salarios de forma que beneficien tanto a los ``insiders'' como a los ``outsiders''.
	\item[b] La tasa de paro natural representa al paro involuntario.
	\item[c] La NAIRU representa la tasa de paro de equilibrio en un mercado de trabajo en competencia perfecta.
	\item[d] Los modelos de salarios de eficiencia se caracterizan por establecer una relación positiva entre salarios y productividad.
\end{itemize}

\seccion{Test 2009}
\textbf{17.} Considere un mercado de trabajo en el que una empresa a corto plazo con tecnología $Y=AN^\alpha$, se encuentra con un sindicato cuyo número de afiliados es $\bar{L}=125$. Si $A=12$, $\alpha = 2/3$ y el salario de reserva es $R=2$, entonces el salario de equilibrio competitivo $\omega_\text{ec}$ y el empleo de equilibrio competitivo son respectivamente:

[ NOTA: $(125)^{1/3} = 5$ ]

\begin{itemize}
	\item[a] $\omega_\text{ec} = 2$ y $N_\text{ec} = 64$.
	\item[b] $\omega_\text{ec} = 1.6$ y $N_\text{ec} = 125$
	\item[c] $\omega_\text{ec} = 2$ y $N_\text{ec} = 125$.
	\item[d] Es necesario conocer la función objetivo del sindicato para determinar $\omega_\text{ec}$ y $N_\text{ec}$.
\end{itemize}

\seccion{Test 2008}
\textbf{22.} Para reducir el desempleo, el gobierno trata de estimular el empleo a través de ayudas fiscales a la contratación. Según el modelo de Solow de salarios de eficiencia, y suponiendo que estamos todavía muy lejos de una situación de pleno empleo:

\begin{itemize}
	\item[a] Los salarios reales disminuyen.
	\item[b] Los salarios reales aumentan.
	\item[c] No tiene ningún efecto sobre los salarios reales.
	\item[d] El desempleo aumenta
\end{itemize}

\notas

\textbf{2011}: \textbf{16.} D
\textbf{2009}: \textbf{17.} A
\textbf{2008}: \textbf{22.} C

\bibliografia

Mirar en Palgrave:
\begin{itemize}
	\item collective bargaining
	\item efficiency wages
	\item fixed factors
	\item implicit contracts
	\item labour economics
	\item labour economics (new perspectives)
	\item labour market institutions
	\item labour market search
	\item labour's share of income
	\item wage curve
	\item wage indexation
	\item wage inequality, changes in
\end{itemize}

Cahuc, P.; Zylberberg, A. \textit{Labor Economics} (2004) Ch. 6, Ch. 7

Flanagan, R. J. (1999) \textit{Macroeconomic Performance and Collective Bargaining: An International Perspective} Journal of Economic Literature. Vol. XXXVII. \href{https://pubs.aeaweb.org/doi/pdfplus/10.1257/jel.37.3.1150}{Disponible aquí} -- En carpeta del tema. 

Matsa, D. (2010) \textit{Capital Structure as a Strategic Variable: Evidence from Collective Bargaining} Journal of Finance. June 2010 -- En carpeta del tema

Romer, D. \textit{Advanced Macroeconomics} (2011) 4th edition. Ch 10 Unemployment -- En carpeta del tema

Rosen, S. (1985) \textit{Implicit Contracts: A Survey} Journal of Economic Literature, Vol. 23, No. 3 -- En carpeta del tema

\end{document}
