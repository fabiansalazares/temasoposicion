\documentclass{nuevotema}

\tema{3A-22}
\titulo{Economía del bienestar (I). Óptimo económico. Los criterios de compensación. La teoría del «second best».}

\begin{document}

\ideaclave

Lionel Robbins y posteriormente Samuelson definieron la ciencia económica como el estudio de las decisiones que tratan de gestionar una serie de recursos finitos con usos alternativos con el fin de satisfacer una serie de necesidades humanas. La economía del bienestar es un campo dentro de la ciencia económica que examina la medida y la maximización del bienestar social. El examen y la comparación de diferentes estados sociales que resultan de diferentes configuraciones institucionales y sociales son así los principales objetos de estudio de la economía del bienestar, pero también de la economía en general. Como afirmó Atkinson, el gran teórico de la desigualdad del siglo XX, la ciencia económica no existe sólo para describir el comportamiento humano y satisfacer la curiosidad y la vanidad de los economistas, sino para emitir recomendaciones y diseñar y valorar políticas que contribuyan a mejorar la vida de los ciudadanos. Por ello, la economía del bienestar forma parte del ``corazón'' de la ciencia económica y como tal, debe ocupar un lugar preeminente en la formación de un economista teórico y más aún en la de un policy-maker. Así, si la valoración de diferentes estados sociales de acuerdo a su deseabilidad social y la caracterización de los medios para lograrlos es el objetivo de la economía del bienestar, la primera línea de actuación concierne la caracterización del óptimo económico. Además de caracterizar y formalizar este concepto, es necesario examinar sus relación con las instituciones que definen un mercado y particularmente con los mercados competitivos, entender sus limitaciones y contemplar las posibles extensiones y soluciones básicas a estas limitaciones que son en definitiva los criterios de compensación. Así, el \textbf{objeto} de esta exposición consiste en responder a una serie de preguntas tales como: ¿qué es un óptimo económico? ¿para qué sirve? ¿qué limitaciones tiene? ¿qué son los criterios de compensación? ¿para qué sirven? ¿en qué consiste la teoría del second-best? La \textbf{estructura} de la exposición comienza por el examen del concepto de óptimo económico, pasa después a tratar los criterios de compensación y trata por último la teoría del second-best.

El \marcar{óptimo económico} es un concepto generalmente definido en un contexto de modelización microeconómica neoclásica de tal manera que depende de un conjunto de preferencias, una tecnología de producción y unas dotaciones. En este contexto, el óptimo económico puede definirse de forma genérica como la asignación de recursos que induce el máximo aprovechamiento de los recursos disponibles. De forma habitual, ``óptimo económico'' es utilizado como sinónimo de óptimo de Pareto.  El primer autor en formular el concepto hoy entendido como óptimo de Pareto fue Edgeworth en 1881, aunque fue Vilfredo Pareto el que estabilizó el concepto y lo introdujo en el incipiente programa neoclásico. Arrow y Hahn (1971) formalizaron el concepto en su representación habitual en la actualidad. Un óptimo de Pareto es una asignación respecto de la cual no existe otra más eficiente en el sentido de Pareto. Al concepto de óptimo económico subyace así la idea de eficiencia paretiana que presta su nombre al término. La eficiencia de Pareto es una caracterización, de entre otras posibles, del concepto de máximo aprovechamiento de unos recursos disponibles al que hacíamos referencia en un primero momento. Así, asignación de recursos es eficiente en el sentido de Pareto cuando no es posible mejorar la situación de uno de los agentes considerados sin empeorar al menos a otro agente. Así, una asignación A es Pareto-superior a otra B cuando el paso de B a A implica mejorar a al menos un agente sin empeorar a ningún otro. El concepto de eficiencia paretiana es una relación en todo caso ordinal, en el sentido de que permite ordenar diferentes asignaciones. Es decir, la eficiencia paretiana como criterio de ordenación de estados no permite inferir nada acerca de \textit{cuánto} más preferible es una asignación respecto de otras, sino simplemente que es preferible. Es posible matizar el concepto de eficiencia paretiana estableciendo dos subvariantes. La superioridad de Pareto en sentido débil de una asignación respecto de otra implica que mejora a todos los agentes considerados. La superioridad de Pareto en sentido fuerte es la que hemos utilizado previamente, de tal manera que una asignación es superior a otra cuando mejora a al menos uno de los agentes y deja indiferente al resto.

El óptimo de Pareto adopta \textbf{formulaciones} específicas en diferentes contextos de decisión. Son especialmente relevantes las caracterizaciones del óptimo económico en producción, consumo y el óptimo global. Cada uno de ellos consiste en una expresión de las condiciones de primer (y segundo orden) de un problema de maximización que representa el máximo producto, utilidad o ambos de un agente dada una restricción relacionada con una dotación inicial y el mantenimiento del bienestar o el nivel de producción de los demás agentes. Un \underline{óptimo en la producción} es una asignación de factores de producción a partir de una dotación inicial que inducen una producción de outputs tal que no es posible aumentar la producción de un output cualquiera sin reducir la de algún otro. Estas asignaciones se pueden caracterizar a partir de un programa de maximización de la producción de un bien dadas dos restricciones respecto a la suma de factores de producción y una en relación a una cantidad mínima de output de otro bien. La condición de primer orden es la igualdad de relaciones marginales de sustitución técnica de los factores de producción en la producción de ambos bienes. En una caja de Edgeworth, los óptimos de Pareto pueden representarse como los puntos de tangencia entre las curvas de indiferencia, si las soluciones son interiores. En caso de las soluciones de esquina, se representan como los puntos de intersección de las isocuantas en los vértices de la caja. Los óptimos en la producción inducen combinaciones de outputs que se pueden representar en el plano y se denominan fronteras de posibilidades de producción. Las FPP representan así las cantidades máximas producibles de un output, dadas cantidades fijas de los demás outputs. Aparece así el carácter fundamental todo óptimo Paretiano claramente señalado en esos puntos de la FPP: no es posible aumentar uno sin afectar a otro.

Un \underline{óptimo en el consumo} es una asignación de bienes que cumplen una restricción de consumo total tal que no es posible encontrar una reasignación que mejore la utilidad de uno de los agentes sin reducir la de otro. De forma similar al óptimo en producción, la caracterización del óptimo en consumo implica formular un problema de maximización cuya solución sean las asignaciones factibles que maximizan la utilidad de un agente manteniendo fijas las utilidades del resto de agentes. La condición de primer orden de este tipo de problemas es tal que las relaciones marginales de sustitución de los bienes para cada agente habrán de igualarse. En una caja de Edgeworth cuyos ejes representen las cantidades de bien de consumo, las asignaciones de óptimo de Pareto son aquellos puntos en los que las curvas de indiferencia son tangentes --si no hay soluciones de esquina-, y los vértices de la caja si las soluciones son de esquina. Las utilidades de óptimo de Pareto pueden representarse en la llamada Frontera de Posibilidades de Utilidad, que contiene las utilidades máximas que puede obtener un agente dado si se mantienen en cantidades fijas las de otro conjunto de agentes. 

Un \underline{óptimo global} es el término habitualmente utilizado para describir las asignaciones de bienes y factores de producción tales que no es posible encontrar reasignaciones que aumenten el bienestar de algún agente sin empeorar el de algún otro. El óptimo global se puede entender como el óptimo en la producción que induce además un óptimo en el consumo. Si en el óptimo en la producción individualmente considerando no teníamos en cuenta el bienestar de consumidor alguno, y en el óptimo en el consumo asumíamos como fijas las cantidades totales de bienes que se podían consumir, en el óptimo global los consumidores son tenidos en cuenta y además, las cantidades máximas de bienes son variables en la medida en que sean resultado de asignaciones óptimas en la producción --es decir, que se encuentren en la frontera de posibilidades de producción. Alternativamente, un óptimo global puede entenderse como la combinación entre un punto de la FPP y una asignación a los consumidores tal que no es posible mejorar el bienestar de uno sin empeorar el de otro. Todos los puntos sobre la FPP son óptimos en la producción, pero no todos inducen óptimos en el consumo: puede ocurrir que un punto de la FPP permita una asignación entre consumidores que mejore la asignación óptima dado otro punto de la FPP, y el primero será así Pareto-superior al segundo. La caracterización analítica de los óptimos globales resulta de un problema de maximización cuyo objetivo es hallar las cantidades de bienes que maximizan la utilidad de un agente manteniendo constante la utilidad de otro agente y manteniéndose en la frontera de posibilidades de producción. La condición de primer orden de estos problemas es tal que las relaciones marginales de sustitución deben igualarse con la relación marginal de transformación. Gráficamente, un óptimo global es así una asignación de bienes de consumo para las que las curvas de indiferencia de los consumidores son tangentes y su pendiente es igual a la de la frontera de posibilidades de producción. Desde otro punto de vista, es posible entender la caracterización de los óptimos globales como la búsqueda de los óptimos en el consumo para cada óptimo en la producción. Existen tres posibilidades respecto a la existencia de óptimos en el consumo, dado un óptimo en la producción. La primera, que exista sólo una asignación de bienes de consumo para la que se cumpla la condición de primer orden que iguale RMS de ambos agentes y RMT. En este caso, la frontera de posibilidades de utilidad resultante es un sólo punto en el plano de utilidades individuales. La segunda posibilidad, que existan varios puntos o que incluso exista un continuo de puntos, de tal manera que la frontera de posibilidades de utilidad sea una línea o un grupo de puntos. La tercera, que no exista ninguna asignación para la que se cumpla la condición de primer orden y que por ello, no haya combinación alguna que se pueda representar sobre el plano de utilidades individuales. Los óptimos globales se representan en la llamada \underline{Gran Frontera de Posibilidades de Utilidad}. Esta curva envuelve a las Fronteras de Posibilidades de Utilidad individuales de tal manera que muestra los óptimos de Pareto de entre los óptimos en el consumo obtenidos cuando se fijaba una cantidad fija de bienes de consumo. 

La optimalidad Paretiana no está ligada a priori a ninguna institución de mercado determinada. Sin embargo, en un resultado de enorme relevancia para la ciencia económica en general y la microeconomía en particular, el \underline{Primer Teorema Fundamental del Bienestar} muestra como todos los equilibrios competitivos son óptimos de Pareto cuando las preferencias no están saturadas y no hay externalidades. El \underline{Segundo Teorema Fundamental del Bienestar} muestra como es posible alcanzar cualquier óptimo de Pareto como un equilibrio competitivo tras una reasignación de las dotaciones cuando además de los supuestos necesarios para el cumplimiento del Primer Teorema, las preferencias y el conjunto de producción es convexo. La combinación de ambos teoremas permite separar la búsqueda de las asignaciones eficientes de las asignaciones socialmente deseables o equitativas. Este resultado permite justificar el desarrollo de la microeconomía sin atender a consideraciones de orden distributivo salvo que efectivamente este sea el objetivo explícito del investigador.

La \textbf{valoración} del concepto de óptimo económico parte necesariamente de su posición central en el análisis microeconómico. En la medida en que la microeconomía se utiliza con fines positivos en el sentido de limitarse a describir la naturaleza de los intercambios económicos, el óptimo económico permite representar las situaciones más allá de las cuáles es necesario utilizar juicios de valor para distinguir la deseabilidad de diferentes estados sociales. Esta ventaja a la hora de separar entre eficiencia asignativa y distributiva, que necesita de juicios de valor, está íntimamente ligada a las desventajas del óptimo de Pareto como criterio de ordenación de estados sociales. La primera y más obvia de ellas concierne el derecho de veto que el criterio otorga a todos los agentes respecto de cualquier cambio. A este derecho de veto se encuentra ligada la indeterminación que se produce cuando se trata de ordenar dos asignaciones que son Pareto-superiores a todas las demás. Así, el criterio de Pareto no es una ordenación completa y por ello el óptimo económico no es un óptimo absoluto, sino un punto más en una gama de posibles óptimos que no es posible comparar. Además, la maximización del bienestar social que caracterizan los óptimos globales y los óptimos en el consumo no son necesariamente deseables. Por ejemplo, una asignación tal que un agente recibe todo la renta y el resto nada en absoluto podría ser sin duda un óptimo económico pero difícilmente sería considerado deseable para casi cualquier sociedad imaginable. Las funciones de bienestar social son la etapa adicional lógica para suplir las indeterminaciones a las que da lugar el criterio de Pareto. 

Los \marcar{criterios de compensación} son reglas de ordenación de óptimos de Pareto que tratan de superar la indeterminación que de éste se deriva en la mayoría de situaciones. Los criterios de compensación permiten así comparar estados sociales sin explicitar juicios de valor, aplicando simplemente el llamado principio de compensación basado en la posibilidad de que unos agentes compensen a otros por un cambio entre estados sociales de tal manera que el cambio pueda interpretarse como una mejora global. Los criterios de compensación adolecen sin embargo de graves problemas de consistencia que en todo caso, no han impedido su utilización generalizada en determinados contextos de análisis de política económica. Existen varios criterios de compensación diferentes que aplicar y que no arrojan necesariamente los mismos resultados de ordenación de estados sociales. Es necesario tener en cuenta que los criterios de compensación no se limitan realmente a comparar asignaciones, sino asignaciones y las fronteras de posibilidades de utilidad a las cuales cada una pertenecen. 

El \textbf{criterio de Kaldor} (1939) ordena dos óptimos de Pareto dados en función de la posibilidad de que los ganadores tras un cambio de una asignación a otra puedan compensar a los perdedores de tal manera que el ganador mejore y el perdedor se mantenga indiferente entre el cambio. El \textbf{criterio de Hicks} (1940) señala que el paso de una asignación a otra es deseable si los perdedores no pueden compensar a los ganadores para que no se produzca el cambio efectivamente. El cumplimiento de uno u otro criterio no implica que otro también se cumpla. La \textbf{paradoja de Scitovsky}, que señaló el autor que le da nombre, consiste en mostrar como ambos criterios pueden ser inconsistentes consigo mismos, de tal manera que el paso de una asignación A a una B sea deseable, pero también lo sea el paso contrario, de B hacia A. El criterio de Scitovsky trata de solucionar esta paradoja, postulando que una asignación es preferida a otra cuando es preferida tanto en el sentido de Kaldor como el de Hicks. Sin embargo, \textbf{Samuelson} demostró que el criterio de Scitovsky daba lugar a intransitividades que sólo podían solucionarse volviendo al criterio de Pareto. \textbf{Little} criticó también los criterios de compensación de Kaldor, Hicks y Scitovsky por no tener en cuenta la distribución de utilidad a la hora de comparar asignaciones. Así, el criterio de Little añade la aversión a la desigualdad como elemento determinante de la comparación. Según este criterio una asignación será preferida a otra si es preferida aplicando los criterios de Kaldor y Hicks y además, se encuentra más cerca de la bisectriz que separa en dos partes iguales el primer cuadrante del eje de coordenadas. De esta forma, las asignaciones que inducen menor desigualdad son preferidas a las que no lo hagan. Este criterio choca sin embargo con la eficiencia Paretiana, porque puede señalar como preferida una asignación que reduce la desigualdad frente a otra en la que todos mejoran pero que la aumenta.

A la hora de \textbf{valorar} los criterios de compensación, es preciso tener presente que la compensación efectiva entre agentes no es necesaria, sino que basta como posibilidad hipotética. Por ello, que una asignación sea considerada superior a otra por un criterio de compensación no implica que efectivamente se vaya a producir una mejora, sino que teóricamente podría producirse. Esto es especialmente relevante para el caso del criterio de Hicks. Es notable que los criterios de compensación sigan siendo ampliamente utilizados a pesar de sus inconsistencias, intranstividades y resultados triviales. Sus aplicaciones más habituales han consistido en la valoración de políticas y ejecución de proyectos de obra pública y similares. 

La \marcar{teoría del second best} formulada por Lipsey y Lancaster en 1951 fue la consolidación formal de un concepto ya formulado anteriormente por otros autores tales como Pareto o Samuelson. Una vez caracterizado el óptimo de Pareto como el cumplimiento de una serie de condiciones de primer orden, parece a priori razonable postular que, si no es posible cumplir todas las condiciones de primer orden necesarias, sea deseable cumplir al menos el mayor número posible de éstas para acercarse lo más posible al óptimo Paretiano. La Teoría del Second Best desmiente este postulado como un resultado general. Para ello, se demuestra como la introducción de restricciones que impiden el cumplimiento de una o más de las condiciones de óptimo de Pareto, resultan en nuevas condiciones de óptimo que no consisten simplemente en cumplir el mayor número posible de condiciones originales. Expresado de forma sencilla, la teoría del second best muestra como la existencia de distorsiones puede hacer necesaria la introducción de restricciones adicionales para maximizar el bienestar, de modo que no basta con cumplir el mayor número de condiciones de óptimo de Pareto originales para maximizar la eficiencia asignativa. La Teoría del Second Best fundamenta numerosos ejemplos de situaciones en las que se introducen restricciones adicionales o se mantienen incumplimientos de condiciones de óptimo que a priori sería necesarias para alcanzar la máxima eficiencia. 

El monopolio contaminador es un ejemplo habitual. Una empresa produce en régimen de monopolio una cantidad inferior a la de competencia perfecta, y la vende a un precio superior al coste marginal. La producción del bien induce una externalidad negativa sobre otros agentes que no intervienen ni se benefician de la actividad del monopolista. En este contexto de monopolio, una de las condiciones de óptimo a priori sería la igualdad entre precio y coste marginal, de modo que debería producirse más cantidad de bien. Sin embargo, la presencia de la externalidad hace que lo óptimo sea producir por debajo de esa cantidad. Así, si con el fin de mejorar la eficiencia se impusiese una producción que igualase precio y coste marginal, se estaría de hecho reduciendo el bienestar. Es decir, para maximizar el bienestar social una vez introducida la distorsión no basta con acercarse lo más posible al cumplimiento de las condiciones de óptimo en ausencia de distorsión, sino introducir una distorsión adicional que aleje la producción de equilibrio de aquella que corresponde a la situación competitiva sin externalidad. 

Otro ejemplo habitual aparece en el campo de la tributación. Un impuesto óptimo es aquel que no distorsiona los precios relativos y no induce efectos sustitución que resultan en excesos de gravamen. Bajo supuestos generales, este impuesto óptimo se corresponde con un impuesto de suma fija cuya cantidad no depende de ninguna variable dependiente del comportamiento del agente. Sin embargo, en la práctica este tipo de impuestos son difícilmente aplicables por su carácter regresivo y el subsecuente rechazo político al que dan lugar. Cabe así preguntarse: si los impuestos de suma fija no son posibles, ¿es adecuado gravar al mismo nivel toda la renta o todo el consumo? Aunque un gravamen generalizado como éste acercaría el gravamen al correspondiente al de suma fija, de hecho se produciría una distorsión adicional porque el ocio no puede ser gravado y también supone consumo para los agentes. Así, tratar de acercarse al óptimo de suma fija por la vía de gravar toda la renta o todo el consumo implicaría la introducción de una distorsión adicional por el hecho de afectar el precio relativo del consumo. Por ello, para maximizar la eficiencia, sería necesario introducir una distorsión adicional como por ejemplo gravar en mayor medida los bienes de consumo complementario al ocio y gravar así de forma indirecta el ocio.

Además de numerosos ejemplos de aplicación, la teoría del second best también ha recibido críticas. El óptimo de second-best puede ser igual al de first-best en determinados contextos, de tal manera que no necesariamente existe una diferencia y la teoría del second-best puede no ser relevante en todos los casos. Se aduce también que cuando el sector que está sujeto a una distorsión está muy poco conectado con el resto de la economía, la aproximación al first-best puede ser óptima. Es necesario tener en cuenta que alcanzar el coste en términos informacionales de alcanzar el second-best debe ser tenido en cuenta también. A ésto último se añade el hecho de que los second-best se caracterizan de forma ad-hoc y no existen formulaciones generales que puedan guiar la toma de decisiones en casos concretos. Así, la teoría del second-best es más bien una apreciación en relación a la búsqueda de óptimos que una receta de política económica aplicable a una generalidad de casos.

A lo largo de la exposición se ha examinado el concepto del óptimo económico, los criterios de compensación como herramienta de comparación entre óptimos y la teoría del second best como apreciación respecto de situaciones en las que no es posible satisfacer todas las condiciones que caracterizan el óptimo económico. A lo largo de la exposición, el concepto de óptimo de Pareto está presente en todo momento. En la exposición hemos caracterizado el óptimo económico en contextos de competencia perfecta, pero la presencia de externalidades positivas y negativas en producción y consumo es En la moderna teoría económica, la eficiencia paretiana es efectivamente una herramienta fundamental para distinguir lo que es a priori deseable antes de aplicar juicios de valor sujetos inevitablemente a consideraciones subjetivas y se podría decir, arbitrarias. Sin embargo, es preciso tener en cuenta que el policy-maker no puede permitirse en la práctica el lujo de abstraerse de toda consideración distributiva, y que de hecho los juicios de valor son fundamentales para tomar una decisión. La teoría económica ha proporcionado herramientas para explicitar estos juicios de valor en forma de las funciones de bienestar social, y también ha planteado métodos de derivación de funciones de bienestar social a partir de conjuntos de preferencias individuales en la teoría de la elección social. 



\seccion{Preguntas clave}
\begin{itemize}
	\item ¿Qué es la economía del bienestar?
	\item ¿Qué es el óptimo económico?
	\item ¿Qué son los criterios de compensación?
	\item ¿Para qué sirven?
	\item ¿En qué consiste la teoría del second best?
\end{itemize}

\esquemacorto

\begin{esquema}[enumerate]
	\1[] \marcar{Introducción}
		\2 Contextualización
			\3 Economía
			\3 Economía del bienestar
			\3 Óptimo económico
		\2 Objeto
			\3 ¿Qué es un óptimo económico?
			\3 ¿Para qué sirve?
			\3 ¿Qué limitaciones tiene?
			\3 ¿Qué son los criterios de compensación?
			\3 ¿Para qué sirven?
			\3 ¿En qué consiste la teoría del second best?
		\2 Estructura
			\3 Óptimo económico
			\3 Criterios de compensación
			\3 Teoría del second-best
	\1 \marcar{Óptimo económico}
		\2 Idea clave
			\3 Contexto
			\3 Objetivos
			\3 Resultado
		\2 Formulación
			\3 Concepto
			\3 Eficiencia de Pareto
			\3 Óptimo en la producción
			\3 Óptimo en el consumo
			\3 Óptimo global
		\2 Implicaciones
			\3 Primer teorema del bienestar
			\3 Segundo teorema del bienestar
		\2 Valoración
			\3 Aplicabilidad
			\3 Veto
			\3 Indeterminación
			\3 Maximización del bienestar social
	\1 \marcar{Criterios de compensación}
		\2 Idea clave
			\3 Contexto
			\3 Objetivos
			\3 Resultados
		\2 Definiciones
			\3 Principio de compensación
			\3 Criterios de compensación
		\2 Criterio de Kaldor
			\3 Idea clave
			\3 Representación gráfica
		\2 Criterio de Hicks
			\3 Idea clave
			\3 Representación gráfica
		\2 Criterio de Scitovsky
			\3 Paradoja de Scitovsky
			\3 Solución
			\3 Representación gráfica
		\2 Criterio de Little
			\3 Idea clave
			\3 Formulación
			\3 Implicaciones
		\2 Criterio de Samuelson
			\3 Idea clave
			\3 Formulación
			\3 Implicaciones
		\2 Valoración de compensaciones
			\3 Idea clave
			\3 Medidas de compensación
		\2 Análisis coste-beneficio
			\3 Idea clave
			\3 Formulación
			\3 Implicaciones
			\3 Valoración
		\2 Valoración
			\3 Compensación efectiva no es necesaria
			\3 Ampliamente utilizados
	\1 \marcar{Teoría del second best}
		\2 Idea clave
			\3 Autores
			\3 Concepto
		\2 Ejemplos
			\3 Monopolio contaminador
			\3 Reducción selectiva de aranceles
			\3 Tributación óptima
		\2 Críticas
			\3 Óptimo de second-best puede ser igual a first-best
			\3 Sectores independientes
			\3 Falta de generalidad
	\1[] \marcar{Conclusión}
		\2 Recapitulación
			\3 Óptimo económico
			\3 Criterios de compensación
			\3 Teoría del second-best
		\2 Idea final
			\3 Criterio de Pareto como herramienta
			\3 Bienestar social

\end{esquema}

\esquemalargo

\begin{esquemal}
	\1[] \marcar{Introducción}
		\2 Contextualización
			\3 Economía
				\4 Definición de Robbins
				\4 Microeconomía
				\4[] Entender y predecir
				\4[] Comportamiento de agentes individuales
				\4[] Agrupados como
				\4[] $\to$ Consumidores
				\4[] $\to$ Empresas
			\3 Economía del bienestar
				\4 Corazón de la economía (Atkinson)
				\4[] Ciencia económica no existe sólo para
				\4[] $\to$ Describir comportamiento humano
				\4[] $\to$ Satisfacer curiosidad humana
				\4[] Existe sobre todo para
				\4[] $\to$ Emitir recomendaciones
				\4[] $\to$ Diseñar políticas
				\4[] $\Rightarrow$ Para mejorar sociedad
				\4 Valorar diferentes estados sociales
				\4[] $\to$ Definir herramientas de valoración
				\4[] $\to$ Explicitar supuestos de comparación
				\4[] $\to$ ¿Cuándo sociedad x es preferible a y?
			\3 Óptimo económico
				\4 Primera línea de la valoración de estados
				\4[] $\to$ ¿Cuáles aprovechan eficientemente recursos?
				\4[] $\to$ ¿Cuáles eliminan posibilidad de \textit{free lunch}?
				\4 ¿Qué limitaciones le afectan?
				\4[] Se trata de ordenar unos frente a otros
				\4[] Óptimo económico es primera aproximación
				\4[] $\to$ Todos los que aprovechan recursos disponibles
				\4[] $\to$ Dejando a un lado otras consideraciones
		\2 Objeto
			\3 ¿Qué es un óptimo económico?
			\3 ¿Para qué sirve?
			\3 ¿Qué limitaciones tiene?
			\3 ¿Qué son los criterios de compensación?
			\3 ¿Para qué sirven?
			\3 ¿En qué consiste la teoría del second best?
		\2 Estructura
			\3 Óptimo económico
			\3 Criterios de compensación
			\3 Teoría del second-best
	\1 \marcar{Óptimo económico}
		\2 Idea clave
			\3 Contexto
				\4 Edgeworth (1881) primero en formular
				\4 Pareto (1909)
				\4 Arrow y Hahn (1971)
				\4[] Formalización estándar
			\3 Objetivos
				\4 Caracterizar óptimo
				\4 Definir formalmente
				\4 Comparar y establecer superioridad
				\4 Analizar óptimo en diferentes casos
				\4 Caracterizar condiciones necesarias
				\4[] Para que equilibrios competitivos sean óptimos
				\4[] Para que óptimos puedan soportarse por eq. competitivos
			\3 Resultado
				\4 Óptimo de Pareto
				\4[] Asignación tal que no es posible:
				\4[] $\to$ Mejorar a algún agente
				\4[] $\to$ Sin empeorar a otro
		\2 Formulación
			\3 Concepto
				\4 Definido en un contexto neoclásico
				\4[] En base a:
				\4[] $\to$ Preferencias
				\4[] $\to$ Tecnología de producción
				\4[] $\to$ Dotaciones
				\4 Asignación de recursos
				\4[] Que induce máximo aprovechamiento
			\3 Eficiencia de Pareto
				\4 Caracterización formal
				\4[] De ``máximo aprovechamiento''
				\4[] Aceptada como criterio primario de comparación
				\4 Asignación es Pareto-eficiente si:
				\4[] No es posible mejorar a un agente
				\4[] Sin empeorar a otro
				\4 Pareto-superioridad
				\4[] Propiedad de asignación respecto a otra
				\4[] Posible mejorar a uno
				\4[] Sin empeorar a los demás
				\4 Relación de carácter ordinal
				\4[] No expresa cuánto es más eficiente una que otra
				\4[] $\to$ Sólo que una es tanto o más eficiente que otra
				\4 Óptimos económicos:
				\4[] Asignaciones para las cuales
				\4[] no existen otras que sean Pareto-superiores
				\4[] $\to$ Asignaciones Pareto-eficientes
				\4 Pareto-superioridad débil
				\4[] Es posible mejorar a todos
				\4 Pareto-superioridad fuerte
				\4[] Es posible mejorar a al menos un agente
				\4[] Manteniendo al resto indiferente
			\3 Óptimo en la producción
				\4 Máxima producción de un bien dados:
				\4[] $\to$ Restricción sobre ff.pp. m
				\4[] $\to$ Producción mínima de otro bien
				\4[$\Rightarrow$] Asignaciones factibles que aumentan la producción de X
				\4[] Sin reducir la de bien Y
				\4[] $\underset{L_x, K_x}{\max} \quad X = X(L_x, K_x) $
				\4[] $\text{s.a:} \quad L_x + L_y = \bar{L}$
				\4[] $\quad \quad \; \; K_x + K_y = \bar{K}$
				\4[] $\quad \quad \; \; Y(L_y, K_y) \geq \bar{Y}$
				\4[] $\mathcal{L} = X(L_x,K_x) - \mu \left( Y(\bar{L} - L_x, \bar{K} - K_y) - \bar{Y}  \right)$
				\4[] $\Rightarrow$ CPO: $\frac{\partial X / \partial L_x}{\partial X / \partial K_x} = \frac{\partial Y / \partial L_y}{\partial Y / \partial K_y}$
				\4[] \fbox{$\Rightarrow$  $\left| \text{RMST}^X_{LK} \right| = \left| \text{RMST}^Y_{LK} \right|$}
				\4 Caja de Edgeworth del óptimo en producción
				\4[] Representación gráfica de tangencias de isocuantas
				\4[] $\to$ Dadas cantidades fijas de factores
				\4[] \grafica{edgeworthproduccion}
				\4 Frontera de posibilidades de producción
				\4[] Combinaciones de X e Y que son óptimo de Pareto
				\4[] Niveles de producción de puntos en curva de contrato
				\4[] \grafica{fpp}
				\4 Pendiente de la FPP: RMT
				\4[] Relación entre disminución de Y y aumento de X\footnote{Teniendo en cuenta que $W/\text{PMgL}_X = \text{CMg}_X$.}
				\4[] \fbox{$ \left| \text{RMT}_{XY} \right| \equiv \frac{-d \, Y}{d \, X} =  \frac{\text{CMgX}}{\text{CMgY}} = \frac{\text{PMgL}_Y}{\text{PMgL}_X} = \frac{\text{PMgK}_Y}{\text{PMgK}_X} = \frac{P_X}{P_Y} $}
				\4[] $\frac{\text{CMgX}}{\text{CMgY}} = \frac{w/\text{PMgL}_x}{w/\text{PMgL}_y} = \frac{\text{PMgL}_Y}{\text{PMgL}_X} =  \frac{\text{PMgK}_Y}{\text{PMgK}_X} = \frac{r/\text{PMgK}_x}{r/\text{PMgK}_y}$
			\3 Óptimo en el consumo
				\4 Máxima utilidad de un consumidor dados:
				\4[] $\to$ Restricción sobre consumo total de bienes
				\4[] $\to$ Utilidad mínima de otro consumidor
				\4[$\Rightarrow$] Asignaciones factibles que mejoran agente A
				\4[] Sin perjudicar a agente B
				\4[] $\underset{X_A, Y_A}{\max} \quad U_A = U_A(X_A, Y_A)$
				\4[] $\text{s.a:} \quad U_B(X_B, Y_B) = \bar{U}$
				\4[] $\quad \quad \; \; X_A + X_B = \bar{X}$
				\4[] $\quad \quad \; \; Y_A + Y_B = \bar{Y}$
				\4[] $\mathcal{L} = U_A(X_A, Y_A) - \mu (U_B(X_B, Y_B) - \bar{U})$
				\4[] $\Rightarrow \text{CPO:} \quad \frac{\partial U_A / \partial X_A }{\partial U_A / \partial Y_A } = \frac{\partial U_B / \partial X_B }{\partial U_B / \partial Y_B } $
				\4[] $\Rightarrow$ \fbox{ $\text{CPO:} \quad \left| \text{RMS}^A_{XY} \right| = \left| \text{RMS}^B_{XY} \right|$}
				\4 Caja de Edgeworth del óptimo en consumo
				\4[] Representación gráfica de tangencias de CIndiferencia
				\4[] $\to$ Dadas cantidades fijas de bienes de consumo
				\4[] \grafica{edgeworthconsumo}
				\4 Frontera de posibilidades de utilidad
				\4[] Combinaciones de utilidad de A y B que son óptimo de Pareto
				\4[] Niveles de utilidad de puntos de curva de contrato
				\4[] \grafica{fpu}
			\3 Óptimo global
				\4[] $\underset{x_A,y_A,x_B,y_B}{\max} \quad U_A (X_A, Y_A)$
				\4[] $\text{s.a:} \quad U_B(X_B, Y_B) = \bar{U}$
				\4[] $\quad \quad \; \; X_A + X_B = X$
				\4[] $\quad \quad \; \; Y_A + Y_B = Y$
				\4[] $\quad \quad \; \; Y = f(X)$
				\4[] $\Rightarrow \text{CPO:} \quad \frac{-d \, Y}{d \, X} = \frac{\partial U_A / \partial X_A}{\partial U_A / \partial Y_A} = \frac{\partial U_B / \partial X_B}{\partial U_B / \partial Y_B} $
				\4[] $\Rightarrow$ \fbox{ $\text{CPO:} \quad \left| \text{RMT}_{XY} \right| = \left| \text{RMS}^A_{XY} \right| =  \left| \text{RMS}^B_{XY} \right| $}
				\4 FPP y caja de Edgeworth de consumo
				\4[] Caja de Edgeworth a partir de
				\4[] $\to$ Punto de FPP
				\4[] Pendiente de FPP en ese punto
				\4[] $\to$ Igual que curvas de indiferencia tangentes
				\4[] \grafica{optimoglobal}
				\4 ¿Cuántos puntos $(U_A, U_B)$ existen tal que..
				\4[] $\to$ $\text{RMT}_{XY} = \text{RMS}^A_{XY} =  \text{RMS}^B_{XY}$
				\4[] dada una combinación $(\bar{X},\bar{Y})$?
				\4[] Un sólo punto
				\4[] $\to$ FPU es un sólo punto
				\4[] Varios puntos
				\4[] $\to$ FPU es una línea o grupo de puntos
				\4[] Ningún punto
				\4[] $\to$ No hay FPU
				\4 Gran Frontera de Posibilidades de Utilidad
				\4[] De entre todas las FPU posibles
				\4[] $\to$ Puntos que son óptimo de Pareto
				\4[] $\Rightarrow$ Envolvente de curvas FPU
				\4[] $\Rightarrow$ Envolvente de puntos de óptimo
				\4[] \grafica{gfpu}
		\2 Implicaciones
			\3 Primer teorema del bienestar
				\4 Dados los supuestos:
				\4[] Ausencia de externalidades
				\4[] Preferencias no saturadas
				\4 Es posible demostrar que:
				\4[] Condiciones de óptimo de Pareto
				\4[] $\to$ Son iguales a condiciones de eq. competitivo
				\4[$\Rightarrow$] Eqs. competitivos son óptimos de Pareto
			\3 Segundo teorema del bienestar
				\4 Dadas:
				\4[] Preferencias convexas y no saturadas
				\4[] Conjunto de producción convexo
				\4 Es posible alcanzar cualquier Óptimo de Pareto
				\4[] $\to$ Como un equilibrio competitivo
				\4[] $\then$ Mediante un impuesto de suma fija a cada agente
				\4[$\Rightarrow$] Posible separar eficiencia de equidad
				\4[$\Rightarrow$] Cuestiones distributivas quedan al margen
		\2 Valoración
			\3 Aplicabilidad
				\4 Aplicable como principio rector
				\4 Caracterizar el problema
				\4 En la práctica
				\4[] Supuestos demasiado restrictivos
				\4[] Fallos de mercado son situación habitual
				\4[] No puede aplicarse tal cual
			\3 Veto
				\4 Inaplicable en términos de política económica
				\4[] Minorías pueden vetar cualquier cambio
			\3 Indeterminación
				\4 No es un orden completo de estados sociales
				\4 No ordena los óptimos
				\4[] $\to$ ¿Cuál es preferible?
			\3 Maximización del bienestar social
				\4 Óptimo económico no implica
				\4[] $\to$ deseabilidad social
				\4[] $\to$ Ordenación completa
				\4[] Todo para un agente y nada para el otro
				\4[] $\to$ Es óptimo de Pareto
				\4[] $\to$ Mucha gente lo consideraría indeseable
				\4[] $\Rightarrow$ Consideraciones distributivas son relevantes
				\4 Funciones de bienestar social
				\4[] Ordenar óptimos de Pareto
				\4[] $\to$ Explicitando criterios de ordenación
	\1 \marcar{Criterios de compensación}
		\2 Idea clave
			\3 Contexto
				\4 Utilitarismo de s.XIX
				\4[] Forma de consecuencialismo
				\4[] Sociedades valoradas sobre resultados de agentes
				\4[] $\to$ Utilidad alcanzada
				\4 Debate principios de s. XX
				\4[] Aceptado no es posible comparaciones CUtilidad
				\4[] Pero continúa debate sobre:
				\4[] $\to$ ¿Cómo comparar estados sociales?
				\4[] $\then$ ¿Es posible?
				\4 Escuela de Harvard
				\4[] Elección social
				\4[] No es posible encontrar juicios objetivos
				\4[] $\to$ Pero sí es posible con juicios de valor
				\4[] $\then$ Necesario explicitar juicios de valor
				\4[] $\then$ Relacionar preferencias indiv. con sociales
				\4 Escuela de Londres
				\4[] Sí es posible comparar sociedades
				\4[] $\to$ De manera objetiva
				\4[] $\to$ Sin comparaciones cardinales interpersonales
				\4[] $\then$ Criterios de compensación
			\3 Objetivos
				\4 Comparar entre diferentes estados sociales
				\4 Evitando juicios de valor
				\4[] Considerando sólo:
				\4[] $\to$ preferencias propias
				\4[] $\to$ Acuerdos entre agentes
			\3 Resultados
				\4 Criterios de compensación
				\4 Análisis de coste-beneficio
				\4[] Comparar costes vs beneficios
				\4[] $\to$ Decidir en función de resultado neto
		\2 Definiciones
			\3 Principio de compensación
				\4 Decisión entre estados puede basarse en:
				\4[] posibilidad de que unos compensen a otros
				\4[] $\to$ Y todos mejoren respecto a situación inicial
			\3 Criterios de compensación
				\4 Comparar estados sociales
				\4[] Sin explicitar juicios de valor
				\4[] $\to$ Simplemente aplicando principio de compensación
		\2 Criterio de Kaldor
			\3 Idea clave
				\4 $x$ es mejor que $y$ si:
				\4[] Los que ganan con $x$
				\4[] pueden compensar a los que pierden
				\4[] $\to$ Para que les compense el cambio
				\4 $x \succ y$ si:
				\4[] $\exists \, x' = \textrm{Reasignación de x} \, / x' \succ y$
			\3 Representación gráfica
				\4[] \grafica{kaldor}
		\2 Criterio de Hicks
			\3 Idea clave
				\4 $x$ es mejor que $y$ si:
				\4[] Los que pierden con $x$
				\4[] NO pueden compensar a los que ganan
				\4[] $\to$ Para que no les compense el cambio
			\3 Representación gráfica
				\4[] \grafica{hicks}
		\2 Criterio de Scitovsky
			\3 Paradoja de Scitovsky
				\4 Criterio de Kaldor puede ser inconsistente
				\4[] X sea preferido a Y
				\4[] $\to$ Ganador B puede compensar a perdedor A
				\4[] Y sea preferido a X
				\4[] $\to$ Ganador A puede compensar a perdedor B
				\4[] $\then$ Paso de X a Y y vv. no aumenta bienestar
			\3 Solución
				\4 Criterio de Scitovsky propone:
				\4[] Necesario cumplir Kaldor y Hicks
				\4[] Ganador pueda compensar a perdedor
				\4[] $\to$ Para llevar a cabo cambio
				\4[] Perdedor no pueda compensar a ganador
				\4[] $\to$ Para no llevar a cabo cambio
			\3 Representación gráfica
				\4 Mismo gráfico que para Hicks
		\2 Criterio de Little
			\3 Idea clave
				\4 Contexto
				\4[] Criterios anteriores no consideran distribución
				\4[] Posible desigualdad extrema
				\4[] $\to$ Siempre que sea posible compensar
				\4[] $\to$ Siempre que perdedor no pueda compensar
				\4 Objetivo
				\4[] Considerar igualdad en compensación
				\4 Resultados
				\4[] Introducción de medidas de igualdad
				\4[] Medida relativamente simple
				\4[] Relación con convexidad de preferencias
			\3 Formulación
				\4 Criterios de K-H:
				\4[] $\to$ ignoran consideraciones distributivas
				\4 Criterio de Little:
				\4[] Basado en Kaldor-Hicks-Scitovsky
				\4[] $\to$ Y añade aversión a la desigualdad
				\4 $x$ es mejor que $y$ si:
				\4[] i) $x$ es preferido a $y$ por Kaldor y Hicks
				\4[] ii) ganador tras cambio es el que peor estaba
				\4[] $\then$ Asignación final debe estar + cerca de bisectriz
			\3 Implicaciones
				\4 Simetría
				\4[] Todos los agentes tienen mismo peso con criterio de Little
				\4[] $\to$ Deseable a priori
				\4[] $\to$ Posibles excepciones con mayor complejidad
				\4 Puede ser contrario a Pareto
				\4[] Cambios a asignaciones donde todos mejoran
				\4[] $\to$ Pueden no ser preferidos porque aumentan desigualdad
		\2 Criterio de Samuelson
			\3 Idea clave
				\4 Basado en crítica de Gorman
				\4 Posible encontrar ciclo de compensaciones
			\3 Formulación
				\4 Criterio de Scitovsky genera intransitividades
				\4[] Posible que $x \succ y \succ z \succ x$
				\4 Solución:
				\4[] $x$ es mejor que $y$ si:
				\4[] se encuentra en una FPU totalmente exterior a la otra
				\4[] $\then$ Trivial porque equivale a criterio de Pareto
			\3 Implicaciones
				\4 Criterios de compensación no son realmente operativos
				\4 Criterio de Pareto es adecuado
				\4 Crítica a programa de investigación de compensación
		\2 Valoración de compensaciones
			\3 Idea clave
				\4 Daños y beneficios
				\4[] A menudo, muy difíciles de valorar
				\4 Necesarias estimaciones
				\4 Preguntar a implicados
				\4[] $\to$ ¿Cuánto acepta recibir a cambio de daño?
				\4[] $\to$ ¿Cuánto acepta pagar por recibir beneficio?
				\4[] $\to$ ¿Cuánto acepta pagar por reducir riesgo?
			\3 Medidas de compensación
				\4 WTA\footnote{Willingness to accept}--Disposición a aceptar
				\4[] Cuánto exige recibir para aceptar pérdida
				\4 WTP--Disposición a pagar
				\4[] Cuánto acepta pagar un agente por mejorar
				\4 VEV--Valor Estadístico de la Vida
				\4[] Vida humana no tiene valoración de mercado
				\4[] Pol. MAmbiental tiene efectos sobre riesgo de muerte
				\4[] Disposición a pagar/aceptar marginal
				\4[] $\to$ Por variación marginal de riesgo
				\4[] $\to$ $\text{VEV} = \frac{\text{WTP o WTA marginal}}{\text{Cambio marginal en riesgo}}$
		\2 Análisis coste-beneficio
			\3 Idea clave
				\4 Contexto
				\4[] Criterios de comparación como herramienta teórica
				\4[] Herramientas de valoración bienes no de mercado
				\4[] Necesario marco coherente
				\4[] $\to$ Valorar oportunidad de intervención
				\4[] $\then$ ¿Beneficios pueden compensar a costes?
				\4 Objetivos
				\4[] Operacionalizar criterios de compensación
				\4[] $\to$ ¿Cómo aplicar criterios a decisión real?
				\4[] $\to$ ¿Cómo valorar efectos sobre bienestar unos y otros?
				\4 Resultados
				\4[] Marco estándar de valoración de proyectos
				\4[] $\to$ Comparación de efectos de alternativas
				\4[] $\then$ ¿Compensan beneficios a costes?
				\4[] $\then$ ¿Cómo afectan a diferentes grupos de agentes?
			\3 Formulación
				\4[I] Definición del objetivo
				\4[II] Enumeración de alternativas
				\4[III] Cuantificación de beneficios y costes
				\4[IV] Elección de tasa de descuento
				\4[V] Consideración del riesgo
				\4[VI] Análisis de redistribución
				\4[VII] Elección de instrumento de compensación
				\4[VIII] Análisis de sensibilidad
				\4[IX] Comparación de resultados
			\3 Implicaciones
				\4 Necesaria cierta arbitrariedad
				\4[] Especialmente en valoración fuera de mercado
				\4 Compensación no tiene por qué producirse
				\4[] Aunque puede incluirse en proyecto
				\4[] $\to$ Y valorarse coste-beneficio de redistribución
			\3 Valoración
				\4 Ampliamente utilizado
				\4 Manifestación práctica de criterio de compensación
				\4 Enorme variedad de ejemplos
				\4[] Construcción de infrastructuras
				\4[] Líneas de transporte
				\4[] Provisión pública de bienes preferentes
				\4[] ...
		\2 Valoración
			\3 Compensación efectiva no es necesaria
				\4 Efectiva compensación
				\4[] No es importante
				\4[] $\to$ Sólo posibilidad de compensar
				\4 Considerar una asignación superior a otra
				\4[] $\to$ No implica que efectivamente todos mejoren
				\4[] $\then$ Margen para políticas que empeoren bienestar
			\3 Ampliamente utilizados
				\4 A pesar de:
				\4[] $\to$ inconsistencias
				\4[] $\to$ intransitividades
				\4[] $\to$ trivialidad
				\4[] $\Rightarrow$ Son ampliamente utilizados
				\4 Aplicaciones
				\4[] Valoración de proyectos
				\4[] Financiación de obra pública
				\4[] Apertura al comercio internacional
	\1 \marcar{Teoría del second best}\footnote{Mirar \textit{second best} en Palgrave.}
		\2 Idea clave
			\3 Autores
				\4 Lipsey y Lancaster (1951)
				\4[] Desarrollo formal
				\4[] Demuestran teorema
				\4 Otros autores anteriores ya plantearon
				\4[] Pareto
				\4[] Samuelson
			\3 Concepto
				\4 Marco de equilibrio general
				\4 Equilibrio óptimo
				\4[] Conjunto de condiciones de óptimo
				\4 Se introducen restricciones adicionales
				\4[] $\to$ Impiden cumplir todas las condiciones
				\4 Cumplimiento de condiciones originales de óptimo
				\4[] Tras introducir restricciones
				\4[] $\to$ Ya no tiene por qué ser deseable
				\4[] $\then$ Posible mejorar incumpliendo algunas
				\4[] $\then$ O introduciendo otras restricciones adicionales
				\4 Viaje a Hawaii
				\4[] Lo deseable es tener dinero para comprar el vuelo
				\4[] Si sólo tiene dinero para el 90\% del trayecto
				\4[] $\to$ Mejor que no compre billete para el 90\%
				\4[] $\then$ Le tirarán del avión en medio del océano
		\2 Ejemplos
			\3 Monopolio contaminador
				\4 Monopolio vende por encima de CMg
				\4[] $\to$ Menos unidades que corresponden a $P=\text{CMg}$
				\4 Condición de óptimo original es $P=\text{CMg}$
				\4 Introducción de exter. negativa de producción
				\4[] $\to$ ¿Cumplimiento de P=CMg acerca a óptimo?
				\4[] $\then$ No tiene por qué: $\uparrow$ producción $\to$ $\uparrow$ externalidad
				\4 Cantidad óptima ya no es la de $P=\text{CMg}$
				\4[$\then$] Aumento de la producción aumenta externalidad
				\4[$\then$] Cumplir óptimo original ya no es deseable
			\3 Reducción selectiva de aranceles
				\4[1.] A priori, aranceles elevados con todos los socios
				\4[2.] Reducción de aranceles a determinados países
				\4[3.] Creación de unión aduanera
				\4[$\then$] Puede aumentar ineficiencia
				\4[$\then$] Más comercio con países ineficientes
				\4[$\then$] Menos comercio con países más eficientes
				\4 Acercamiento parcial a condiciones de óptimo
				\4[] Eliminando aranceles a algunos
				\4[] $\to$ Puede ser peor que situación inicial
				\4 Meter modelo de desviación de comercio de Viner
			\3 Tributación óptima
				\4 Impuesto óptimo:
				\4[] No distorsiona precios relativos
				\4[] No distorsiona comportamiento
				\4 Impuesto de suma fija
				\4[] No depende de ninguna variable
				\4[] $\to$ Óptimo
				\4[] $\to$ Pero no posible por razones prácticas
				\4 Si óptimo no es posible:
				\4[] ¿Es adecuado gravar igual...
				\4[] ... toda la renta?
				\4[] ... o todo el consumo?
				\4[] No tiene por qué
				\4[] $\to$ No es posible gravar el ocio
				\4[] $\to$ Ocio es de hecho ``consumo'' y renta
				\4 Óptimo puede implicar distorsiones adicionales
				\4[] P.ej.: gravar más complementarios de ocio
				\4[] $\to$ Introduce restricción adicional
				\4[] $\to$ Restricción adicional acerca a óptimo
		\2 Críticas
			\3 Óptimo de second-best puede ser igual a first-best
				\4 No necesariamente debe haber una diferencia
				\4 La teoría puede no ser relevante
			\3 Sectores independientes
				\4 Sector sujeto a restricción adicional
				\4[] Muy poca relación con resto de la economía
				\4[$\then$] Aproximarse a first-best sigue siendo óptimo
				\4 En la práctica
				\4[] La mayoría de las intervenciones son pequeñas
				\4[] $\to$ Second-best sólo relevante en algunos casos
			\3 Falta de generalidad
				\4 Teoría de S-B es a menudo ad-hoc
				\4 No existe formulación general de S-B
				\4 No existen condiciones de óptimo S-B generales
	\1[] \marcar{Conclusión}
		\2 Recapitulación
			\3 Óptimo económico
			\3 Criterios de compensación
			\3 Teoría del second-best
		\2 Idea final
			\3 Criterio de Pareto como herramienta
				\4 Herramienta fundamental de economía moderna
				\4 Punto de partida de valoración de estados sociales
				\4[] Estados que aprovechan recursos eficientemente
				\4[] Descartar estados ineficientes
				\4 Especialización del trabajo:
				\4[] $\to$ Caracterizar estados óptimos
				\4[] $\to$ Elegir entre estados óptimos
			\3 Bienestar social
				\4 Necesario para para culminar proceso
				\4 De entre los óptimos:
				\4[] ¿cuál es preferible?
				\4[] $\to$ Necesario explicitar juicios de valor
\end{esquemal}

\graficas

\begin{dibujo}{4}{Representación gráfica de un óptimo en producción dadas unas dotaciones y un nivel mínimo de producción del bien Y}{x}{y}{edgeworthproduccion}
	% ejes que forman un cuadrado
	
	% eje al derecho, del agente A
	\draw[-{Latex}] (0,0) -- (0,4);
	\draw[-{Latex}] (0,0) -- (6,0);
	
	\node[below] at (6,-0.1){$K_x$};
	\node[left] at (0,3.9){$L_x$};
	
	\node[left] at (0,-0.3){$O_X$};
	
	% eje al revés, del agente B
	\draw[-{Latex}] (6,4) -- (0,4);
	\draw[-{Latex}] (6,4) -- (6,0);	
	
	\node[above] at (0.1,4.1){$K_y$};
	\node[right] at (6,0){$L_y$};
	
	\node[right] at (6,4.3){$O_Y$};
	
	% dotación inicial
	\node[circle, fill=black, inner sep=0pt, minimum size=3pt] (a) at (1.7,3.5) {}; 
	\node[left] at (1.63,3.53){ \tiny $\bar{e}$};
	
	% Curva de indiferencia de producto X que pasa por dotación inicial
	\draw[dashed] (1.7,3.5) to [out=280, in=170](4.6,1.2);
	
	% Curva de indiferencia de producto Y que pasa por dotación inicial
	\draw[-] (1.7,3.5) to [out=350, in=100](4.6,1.2);
	\node[right] at (4.3,2){\tiny $\bar{Y}$};
	
	% Curva de indiferencia de X que tangente con Y fija
	\draw[-] (2.6,4.4) to [out=280, in=170](5.4,2.1);
	\node[right] at (5.4,2.1){\tiny $X^*$};
	\node[circle, fill=black, inner sep=0pt, minimum size=3pt] (a) at (3.62,2.78) {};
	
	% Curvas de indiferencia de óptimo
	%
	% Curva de X
	\draw[-] (2,3.8) to [out=280, in=170](4.9,1.5);
	\node[right] at (4.9,1.5){\tiny $X'$};
	
	%
	% Curva de Y
	\draw[-] (1.12,2.92) to [out=350, in=100](4.02,0.62);
	\node[right] at (4.02,0.62){\tiny $Y'$};
	
	% Óptimo dentro de lente - punto de tangencia entre curvas de indiferencia 
%	\node[circle, fill=black, inner sep=0pt, minimum size=3pt] (a) at (3.02,2.19) {}; 
%	\node[above] at (3.03,2.21){\tiny $x^*$};
	
	% Óptimos de Pareto / núcleo
	\draw[thick] (2.76,1.85) to [out=50, in=210](3.6,2.75);
	
	\draw[-] (0,0) to [out=80,in=240](3.02,2.19) to [out=60,in=240](6,4);
\end{dibujo}

Partiendo de una dotación $\bar{e}$ de factores que induce una producción del bien Y tal que $\bar{Y}$, es posible reasignar los factores de tal manera que se mantenga el nivel de producción de Y pero aumente el nivel de producción de X hasta $X^*$. Así, el punto de tangencia entre las isocuantas de $X^*$ y de $\bar{Y}$ será Pareto-superior al punto $\bar{e}$ y además, un óptimo de Pareto porque no existe ninguna otra asignación que permita aumentar la producción de un bien sin reducir la de otro.

La línea gruesa muestra todos los óptimos que son Pareto-superiores a la dotación inicial $\bar{e}$. La línea continua de trazo normal que comprende a la línea gruesa muestra todos los óptimos de Pareto obtenibles a partir de todas las asignaciones iniciales. La frontera de posibilidades de producción se construye a partir de los niveles de producción asociados a los puntos de esta línea de trazo normal.

\begin{axis}{4}{Frontera cóncava de posibilidades de producción de los bienes X e Y.}{X}{Y}{fpp}
	% FPP
	\draw[-] (0,3.5) to [out=340, in=100](3.5,0);
	
	% Punto más alto en curva
	\node[circle, fill=black, inner sep=0pt, minimum size=5pt] (a) at (2,2.47) {};
	
	% punto más bajo en curva
	\node[circle, fill=black, inner sep=0pt, minimum size=5pt] (a) at (2.52,2){};
	
	% linea de cambio en Y y cambio en X
	\draw[-] (2,2.47) -- (2,2) -- (2.52,2);
	% cambio en Y
	\node[left] at (1.95,2.22){\tiny $-d \, Y$};
	% cambio en X
	\node[below] at (2.20,2){\tiny $d \, X$};
\end{axis}

\begin{dibujo}{4}{Representación gráfica de un óptimo en consumo dadas unas dotaciones y un nivel mínimo de utilidad del agente B}{x}{y}{edgeworthconsumo}
	% ejes que forman un cuadrado
	
	% eje al derecho, del agente A
	\draw[-{Latex}] (0,0) -- (0,4);
	\draw[-{Latex}] (0,0) -- (6,0);
	
	\node[below] at (6,-0.1){$X_A$};
	\node[left] at (0,3.9){$Y_A$};
	
	\node[left] at (0,-0.3){$O_A$};
	
	% eje al revés, del agente B
	\draw[-{Latex}] (6,4) -- (0,4);
	\draw[-{Latex}] (6,4) -- (6,0);	
	
	\node[above] at (0.1,4.1){$X_B$};
	\node[right] at (6,0){$Y_B$};
	
	\node[right] at (6,4.3){$O_B$};
	
	% dotación inicial
	\node[circle, fill=black, inner sep=0pt, minimum size=3pt] (a) at (1.7,3.5) {}; 
	\node[left] at (1.63,3.53){ \tiny $\bar{e}$};
	
	% Curva de indiferencia de producto X que pasa por dotación inicial
	\draw[dashed] (1.7,3.5) to [out=280, in=170](4.6,1.2);
	
	% Curva de indiferencia de producto Y que pasa por dotación inicial
	\draw[-] (1.7,3.5) to [out=350, in=100](4.6,1.2);
	\node[right] at (4.3,2){\tiny $\bar{U}_B$};
	
	% Curva de indiferencia de X que tangente con Y fija
	\draw[-] (2.6,4.4) to [out=280, in=170](5.4,2.1);
	\node[right] at (5.4,2.1){\tiny $U_A^*$};
	\node[circle, fill=black, inner sep=0pt, minimum size=3pt] (a) at (3.62,2.78) {};
	
	% Curvas de indiferencia de óptimo
	%
	% Curva de X
	\draw[-] (2,3.8) to [out=280, in=170](4.9,1.5);
	\node[right] at (4.9,1.5){\tiny $U_A'$};
	
	%
	% Curva de Y
	\draw[-] (1.12,2.92) to [out=350, in=100](4.02,0.62);
	\node[right] at (4.02,0.62){\tiny $U_B$};
	
	% Óptimo dentro de lente - punto de tangencia entre curvas de indiferencia 
	%	\node[circle, fill=black, inner sep=0pt, minimum size=3pt] (a) at (3.02,2.19) {}; 
	%	\node[above] at (3.03,2.21){\tiny $x^*$};
	
	% Óptimos de Pareto / núcleo
	\draw[thick] (2.76,1.85) to [out=50, in=210](3.6,2.75);
	
	\draw[-] (0,0) to [out=80,in=240](3.02,2.19) to [out=60,in=240](6,4);
\end{dibujo}

\begin{axis}{4}{Frontera de posibilidades de utilidad de los agentes A y B.}{$U_A$}{$U_B$}{fpu}
	% FPP
	\draw[-] (0,3.5) to [out=340, in=100](3.5,0);
	
	% Punto más alto en curva
	\node[circle, fill=black, inner sep=0pt, minimum size=5pt] (a) at (2,2.47) {};
	
	% punto más bajo en curva
	\node[circle, fill=black, inner sep=0pt, minimum size=5pt] (a) at (2.52,2){};
	
	% linea de cambio en Y y cambio en X
	\draw[-] (2,2.47) -- (2,2) -- (2.52,2);
	% cambio en Y
	\node[left] at (1.95,2.22){\tiny $-d \, U_B$};
	% cambio en X
	\node[below] at (2.20,2){\tiny $d \, U_A$};
\end{axis}

\begin{axis}{4}{Óptimo global de Pareto: producción y consumo.}{$X$}{$Y$}{optimoglobal}
	% FPP
	\draw[-] (0,3.5) to [out=340, in=100](3.5,0);
	
	% Recta tangente a FPP que muestra RMT
	\draw[-] (1.8,2.77) -- (3.2,1.334);
	
	% Punto de origen de la caja de Edgeworth sobre FPP
	\node[circle, fill=black, inner sep=0pt, minimum size=5pt] (a) at (2.45,2.1) {};
	
	% Caja de Edgeworth
	\draw[-] (0,2.1) -- (2.45,2.1) -- (2.45,0);
	\draw[-{Latex}] (0,0) -- (2.45,0);
	\draw[-{Latex}] (0,0) -- (0,2.1);
	\draw[-{Latex}] (2.45,2.1) -- (0,2.1);
	\draw[-{Latex}] (2.45,2.1) -- (2.45,0);
	
	\node[below] at (2.45,0){\tiny $X_A$};
	\node[left] at (0,2.1){\tiny $Y_A$};
	
	\node[above] at (0.2,2.1){\tiny $X_B$};
	\node[right] at (2.45,0.15){\tiny $Y_B$};
	
	
	% Recta dentro de caja de Edgeworth con misma pendiente que RMT
	\draw[-] (0.8,1.77) -- (2.2, 0.334);
	
	% Curva de indiferencia 
	\draw[-] (1.08,1.85) to [out=280, in=170](2.28,0.68);
	
	% Curva de indiferencia inversa
	\draw[-] (0.54,1.6) to [out=350, in=100](1.84,0.24);
	
	\node[circle, fill=black, inner sep=0pt, minimum size=5pt] (a) at (1.4,1.14) {};
\end{axis}

\begin{axis}{4}{Gran Frontera de Posibilidades de Utilidad como envolvente de los diferentes óptimos globales.}{$u_A$}{$u_B$}{gfpu}
	
	% FPUs de más alta y a la izquierda a más baja y a la derecha
	\draw[-] (0,3.3) to [out=-10, in=92](1.9,0);
	\draw[-] (0,3.1) to [out=-2, in=92](2.5,0);
	\draw[-] (0,2.69) to [out=-2, in=92](3.09,0);
	\draw[-] (0,2) to [out=-5, in=92](3.38,0);
	
	% GFPU como envolvente de las FPU
	\draw[thick] (0,3.5) to [out=-30,in=123](3.7,0);
	
\end{axis}

\begin{axis}{4}{El criterio de compensación de Kaldor.}{$u_A$}{$u_B$}{kaldor}
	% FPU a la que pertenece y
	\draw[-] (0,2.5) to [out=-3, in=92](3.5,0);
	
	% punto Y
	\node[circle, fill=black, inner sep=0pt, minimum size=5pt] (a) at (1.5,2.25) {};
	\node[below] at (1.5,2.15){Y};
	
	% área pareto superior a Y
	\draw[dashed] (1.5,3.35) -- (1.5,2.15) -- (2.7,2.15);
	
	% FPU a la que pertenece x
	\draw[-] (0,3.8) to [out=-4, in=92](2.5,0);
	
	% punto X
	\node[circle, fill=black, inner sep=0pt, minimum size=5pt] (a) at (2.35,1.2) {};
	\node[left] at (2.30,1.2){X};
	
	% Área Pareto-superior a X
	\draw[dashed] (2.35,2.4) -- (2.35,1.2) -- (3.5,1.2);
	
	% punto Z sobre FPU de X
	\node[circle, fill=black, inner sep=0pt, minimum size=5pt] (a) at (0.8,3.6) {};
	\node[above] at (0.8,3.6){Z};
	
	% área pareto superior a Z
	\draw[dashed] (0.8,4.8) -- (0.8,3.6) -- (2,3.6);
\end{axis}

El criterio de compensación de Kaldor afirmaría que Y es preferido a X. Tras pasar de X a Y, el agente A vería su utilidad disminuida y el agente B la vería aumentada. Sin embargo, la FPU correspondiente al punto Y permitiría que el agente B compensase al agente X de tal manera que ambos mejorasen con respecto al punto inicial X. Por ello, el criterio de compensación de Kaldor afirma que $Y \succ_K X$. 

De forma contraria, el criterio de Kaldor no indica que Z pueda ser preferido a Y. El paso de Z a Y implicaría un aumento del bienestar de A y una disminución del bienestar de B. A diferencia de la comparación anterior, en este caso el agente A no podría compensar al agente B de tal manera que ambos se encontrasen en una situación preferida a la de Z.

Se aprecia también en el gráfico como el criterio de Scitovsky señala la superioridad de Z respecto a Y. En un hipotético paso de Y a Z, el agente 2 (ganador) podría compensar al agente A (perdedor) para que su bienestar no empeorase con el cambio. Sin embargo, el agente A no podría compensar al agente B para que obtuviese el bienestar asociado a Z y aceptase evitar el cambio.

\begin{axis}{4}{Los criterios de compensación de Hicks y Scitovsky}{$u_A$}{$u_B$}{hicks}
	% FPU a la que pertenece y
	\draw[-] (0,2.5) to [out=-3, in=92](3.5,0);
	
	% punto Y
	\node[circle, fill=black, inner sep=0pt, minimum size=5pt] (a) at (1.5,2.25) {};
	\node[below] at (1.5,2.15){Y};
	
	% área pareto superior a Y
	\draw[dashed] (1.5,3.35) -- (1.5,2.25) -- (2.7,2.25);
	
	% FPU a la que pertenece x
	\draw[-] (0,3.8) to [out=-4, in=92](2.5,0);
	
	% punto X
	\node[circle, fill=black, inner sep=0pt, minimum size=5pt] (a) at (2.35,1.2) {};
	\node[left] at (2.30,1.2){X};
	
	% Área Pareto-superior a X
	\draw[dashed] (2.35,2.4) -- (2.35,1.2) -- (3.5,1.2);
	
	% punto Z sobre FPU de X
	\node[circle, fill=black, inner sep=0pt, minimum size=5pt] (a) at (0.8,3.6) {};
	\node[below] at (0.8,3.55){Z};
	
	% área pareto superior a Z
	\draw[dashed] (0.8,4.8) -- (0.8,3.6) -- (2,3.6);
	
\end{axis}

El criterio de compensación de Hicks muestra como Y no es superior a X. El agente A, que se ve perjudicado por el paso de X a Y, puede compensar a Y de tal manera que éste obtenga una utilidad al menos tan buena como la de Y, manteniéndose en la FPU correspondiente a X. Lo mismo sucede a la inversa, de tal manera que X no es superior a Y porque el agente que se ve perjudicado con el paso de Y a X puede compensar al otro para que el cambio de FPU no se produzca. 

El criterio de Hicks muestra que Z sí es superior a Y. El agente que se ve perjudicado con el paso de Y a Z no puede compensarle de ninguna forma sin cambiar de FPU para que el cambio no se produzca efectivamente.

La gráfica muestra también como el criterio de Scitovsky postula la superioridad de Z frente a Y. Ante un hipotético paso de Y a Z, el ganador con el cambio (B) puede compensar al perdedor (A) para que acepte el cambio y su utilidad se mantenga. Sin embargo, A no puede compensar a B para que no se produzca el cambio de Y a Z, porque no existen ningún punto en la FPU de Y que iguale el bienestar que B obtiene en Z.



\conceptos

\concepto{Curvas de contrato}

La curva de contrato es el espacio de puntos en una caja de Edgeworth que son óptimos de Pareto. Las soluciones interiores cumplen la propiedad de igualdad de relaciones marginales de sustitución. Las soluciones de esquina son aquellas para las que no se cumple la igualdad. La forma de las funciones de utilidad determina la forma de las curvas de contrato. Así, cuando las funciones de utilidad representan sustituibilidad perfecta entre bienes de tal manera que sus curvas de indiferencia son rectas perfectas, la igualdad de relaciones marginales de sustitución bien no es posible en ningún punto, bien aparece en todos los puntos, cuando el grado de sustituibilidad es idéntico. Cuando los grados de sustituibilidad no son idénticos, la curva de contrato ocupa dos lados de la caja (izquierda y arriba o abajo y derecha). En otras combinaciones de sustituibilidad y complementariedad pueden aparecer diferentes formas de la curva de contrato.

\concepto{Relación entre variación compensatoria, variación equivalente y excedente del consumidor en presencia de bien inferior.}

Supongamos un bien $x$, sobre cuyo precio se plantea la posibilidad de una bajada. Si la bajada se lleva efectivamente a cabo, ¿cómo habrá que compensar a un consumidor en términos de su renta para que la utilidad se mantenga al nivel inicial? La bajada del precio necesariamente lleva aparejada un aumento de la utilidad, ya sea porque el consumidor puede ahora consumir más unidades del bien en cuestión o porque puede consumir más unidades de otros bienes que le reportan utilidad, o por ambas razones. Así, si la bajada del precio tiene lugar, habrá que detraer renta del consumidor para inducirle su utilidad inicial. Sin embargo, las variaciones en la renta tienen efectos sobre la utilidad a través del consumo de bienes. En relación al consumo del bien $x$ cuyo precio se ha reducido, hay tres posibles efectos ante una detracción de su renta, que determinan la cuantía necesaria que hay que detraer (la \textit{variación compensatoria} o VC) para mantener la utilidad al nivel precio a la bajada de precios.

\begin{itemize}
	\item La detracción de renta no tiene efecto alguno sobre el consumo de $x$. Estamos ante un bien con demanda cuasilineal en otro bien, de tal manera que se demanda una cantidad fija de bien $x$ y todo el resto de la renta se destina a otro bien, que contribuye linealmente a aumentar la utilidad. Para reducir la utilidad al nivel inicial, bastará con detraer la cantidad que ha dejado de gastarse en bien $x$ tras la bajada de su precio y que había quedado libre para consumir en el otro bien. Esta cantidad a detraer no es sólo la variación compensatoria, si no también el llamado excedente del consumido (EC), o el área a la izquierda de la curva de demanda inversa marshalliana. Luego en términos formales tenemos que $\text{VC}=\text{EC}$.
	\item La detracción de renta tiene un efecto negativo sobre el consumo de $x$. Estamos ante un bien \textbf{normal}. En este caso, la utilidad no cae sólo por el menor consumo del otro bien (como en el caso anterior sin efecto renta), sino también por el menor consumo del bien $x$. Así, para reducir la utilidad al nivel inicial previo a la bajada del precio de $x$, será necesaria una menor reducción de renta que en el caso anterior. La variación compensatoria será así en este caso inferior al excedente del consumidor anterior. Luego en términos formales tenemos que $\text{VC} < \text{EC}$.
	\item La detracción de renta tiene un efecto positivo sobre el consumo de $x$. Estamos ante un bien \textbf{inferior}. En este caso, la bajada del precio provoca un aumento del consumo de $x$ menor que el que tendría lugar si no existiese este efecto positivo sobre el consumo de $x$. Ello resulta en un aumento de la utilidad inferior ante la caída de precio. Para reducir la utilidad al nivel inicial tras la caída en el precio, tendremos que detraer una cantidad de renta de igual forma que en el resto de los casos. Pero dado que los aumentos/detracciones de renta aumentan/reducen el el consumo de bien $x$ menos que en el primer caso sin efecto renta, será necesaria una mayor reducción de la renta. Luego en términos formales tenemos que $\text{VC} > \text{EC}$.
\end{itemize}

Si el objetivo es, mediante transferencias de renta, igualar la utilidad que se obtendría tras la potencial bajada de precio, pero sin que ésta llegue a producirse, estaremos tratando de caracterizar la variación equivalente o VE. Los resultados serán inversos a los obtenidos anteriormente para la variación compensatoria. Así, tendremos que $\text{VE} = \text{EC}$ si el bien no sufre efecto renta. Si el bien es normal y transferencias de renta aumentan el consumo del bien x, el gasto total en $x$ aumentará con la transferencia y estará disponible menos cantidad para el otro bien. Ello implica que serán necesarias transferencias de renta mayores para compensar esa desviación hacia $x$ y por tanto, $\text{VE} > \text{EC}$ con bienes normales. Si el bien es inferior, transferencias de renta reducirán el consumo de $x$, liberándose recursos para el consumo del otro bien adicionales a la propia transferencia de renta. Así, será necesario aumentar en menor cantidad la renta respecto al caso sin efecto renta en $x$ y por tanto, $\text{VE} < \text{EC}$ con bienes inferiores.

Poniendo en común ambos conjuntos de resultados, obtenemos la siguiente taxonomía:

\begin{itemize}
	\item Con bienes normales:
	\begin{itemize}
		\item Subidas de precio:
		\begin{itemize}
			\item $\text{VE} < \text{EC} < \text{VC}$
		\end{itemize}
		\item Bajadas de precio:
		\begin{itemize}
			\item $\text{VE} > \text{EC} > \text{VC}$
		\end{itemize}
	\end{itemize}
	\item Con bienes inferiores:
	\begin{itemize}
		\item Subidas de precio:
		\begin{itemize}
			\item $\text{VE} > \text{EC} > \text{VC}$
		\end{itemize}
		\item Bajadas de precio:
		\begin{itemize}
			\item  $\text{VE} < \text{EC} < \text{VC}$
		\end{itemize}
	\end{itemize}
	\item Con bienes sin efecto renta:
	\begin{itemize}
		\item $\text{VE} = \text{EC} = \text{VC}$
	\end{itemize}
\end{itemize}


\preguntas

\seccion{Test 2017}

\textbf{11.} Si una distribución de la renta entre 3 individuos es un óptimo de Pareto, entonces \underline{\textbf{no es posible}} encontrar una distribución alternativa en la que:

\begin{itemize}
	\item[a] Los 3 individuos estén mejor.
	\item[b] Al menos 2 individuos estén mejor y el otro permanezca igual.
	\item[c] Uno de los individuos esté mejor pero los otros 2 no estén peor.
	\item[d] Ninguna de las situaciones anteriores puede darse.
\end{itemize}

\seccion{Test 2016}

(Pregunta movida al tema 3A-39 sobre inflación)

\textbf{12.} La fórmula de actualización de un determinado tipo de rentas que perciben los consumidores que incluya entre sus componentes un índice de precios debería tener en cuenta:

\begin{itemize}
	\item[a] Que los índices de tipo Laspeyres tienden a subestimar el aumento del coste de vida, por lo que desde el punto de vista del bienestar de los rentistas convendría recurrir a un índice de tipo Paasche, si bien el deflactor del PIB no resulta adecuado ya que mide los cambios de precios en todos los aspectos de la economía, en oposición al IPC, que sólo analiza el gasto del consumidor.
	\item[b] Que los índices de tipo Paasche tienden a subestimar el aumento del coste de vida al suponer que la cesta del consumidor previa al aumento del nivel de precios generales puede alcanzarse con los precios finales.
	\item[c] Que los índices de tipo Laspeyres tienden a sobre-estimar el aumento del coste de vida ya que el consumidor puede alterar los productos de su cesta ante variaciones del precio.
	\item[d] Que en una economía sólo con bienes complementarios perfectos resultaría indiferente incluir en la fórmula el IPC o el deflactor del PIB desde el punto de vista del bienestar de los rentistas, dado que no existiría el efecto sustitución ante aumentos del nivel de precios.
\end{itemize}


\seccion{Test 2014}

(Pregunta movida al tema 3A-39 sobre inflación)

\textbf{8.} Suponga que los precios en 2007 fueron $(p_x, p_y) = (2,3)$ y en 2008 $(p'_x, p'_y) (3,4)$. Si la cesta de bienes de un consumidor en 2007 fue $(2,2)$; entonces su IPC verdadero es:

\begin{itemize}
	\item[a] Menor del 40\%.
	\item[b] Exactamente el 40\%.
	\item[c] Mayor del 40\%.
	\item[d] Indeterminado.
\end{itemize}

\seccion{Test 2011}

\textbf{9.} El segundo teorema del bienestar dice que bajo ciertas condiciones:

\begin{itemize}
	\item[a] Cualquier redistribución de las dotaciones iniciales, tiene asociado un equilibrio competitivo cuya asignación es eficiente en sentido de Pareto.
	\item[b] Cualquier asignación eficiente en sentido de Pareto puede alcanzarse como un equilibrio competitivo a través de una redistribución de las dotaciones iniciales.
	\item[c] Si una asignación es eficiente en sentido de Pareto y es el resultado de un equilibrio competitivo, se puede implementar para cualquier redistribución de las dotaciones iniciales.
	\item[d] La asignación asociada al equilibrio competitivo de cualquier redistribución de las dotaciones iniciales es eficiente en sentido de Pareto.
\end{itemize}

\seccion{Test 2009}

\textbf{12.} Considere un consumidor cuyas preferencias verifican todos los axiomas de comportamiento. En ese caso, si el bien $x_1$ \underline{es un bien inferior} la relación entre la variación compensada (VC), la variación equivalente (VE) y la variación del excedente del consumidor ($\Delta \text{EC}$), ante una subida del precio $p_1$ será la siguiente:

\begin{itemize}
	\item[a] $|\text{VC}| > |\Delta \text{EC} | > |\text{VE} |$
	\item[b] $| \text{VC}| = |\Delta \text{EC}| = |\text{VE}| $
	\item[c] $| \text{VC}| < |\Delta \text{EC}| < |\text{VE}| $
	\item[d] $|\text{VC}| > |\Delta \text{EC}| $ y $|\text{VE}| > |\Delta \text{EC}|$
\end{itemize}

\seccion{Test 2006}

\textbf{5.} Suponga un individuo que consume dos bienes, $x_1$ y $x_2$, a precios $p_1$ y $p_2$, con una renta M. Sea $\text{VE} = $ variación equivalente, $\text{VC} =$ variación compensatoria y $\text{VExc} =$ variación del excedente del consumidor. Si las preferencias del consumidor verifican todos los axiomas del comportamiento, entonces:

\begin{itemize}
	\item[a] Si para este consumidor el bien $x_1$ normal, ante un aumento en su precio, el bienestar aumenta y se verifica que: $| \text{VE} | < | \text{VExc} | < | \text{VC} |$. 
	\item[b] Si para este consumidor el bien $x_1$ independiente de la renta, ante un aumento en su precio, el bienestar aumenta y se verifica que: $|\text{VE}| < |\text{VExc}| < |\text{VC}|$.
	\item[c] Si para este consumidor el bien $x_1$ normal, ante una reducción en su precio, el bienestar aumenta y se verifica que: $|\text{VE}| < |\text{VExc}| < |\text{VC}|$.
	\item[d] Si para este consumidor el bien $x_1$ es independiente de la renta, ante una reducción en su precio, el bienestar del consumidor aumenta y se verifica que: $|\text{VE}=|\text{VExc} | = |\text{VC} |$. 
\end{itemize}

\notas

\textbf{2017:} \textbf{22.} D

\textbf{2016:} \textbf{12.} C

\textbf{2014:} \textbf{8.} A

\textbf{2011:} \textbf{9.} B

\textbf{2009:} \textbf{12.} C

\textbf{2006:} \textbf{5.} D


\bibliografia

Mirar en Palgrave:
\begin{itemize}
	\item compensation principle
	\item interpersonal utility comparisons
	\item optimality and efficiency
	\item Pareto efficiency *
	\item Pareto principle and competing principles *
	\item philosophy and economics
	\item second best
	\item welfare economics
\end{itemize}

\end{document}
