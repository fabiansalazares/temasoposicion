\documentclass{nuevotema}

\tema{3B-16}
\titulo{Análisis comparado de los distintos regímenes cambiarios. Intervención y regulación en los mercados de cambio.}

\begin{document}

\ideaclave

\seccion{Preguntas clave}

\begin{itemize}
	\item ¿Qué regímenes cambiarios existen?
	\item ¿Qué modelos teóricos permiten comparar regímenes cambiarios?
	\item ¿Qué ventajas e inconvenientes tienen?
	\item ¿En qué circunstancias es adecuado un régimen cambiario fijo o flexible?
	\item ¿Qué evidencia empírica existe al respecto?
	\item ¿Cómo intervienen las autoridades monetarias los mercados cambiarios?
	\item ¿Para qué sirven las intervenciones?
	\item ¿Cómo regulan las autoridades monetarias los mercados de cambio?
\end{itemize}

\esquemacorto

\begin{esquema}[enumerate]
	\1[] \marcar{Introducción}
		\2 Contextualización
			\3 Macroeconomía
			\3 Economías abiertas
			\3 Régimen cambiario
		\2 Objeto
			\3 ¿Qué regímenes cambiarios existen?
			\3 ¿Qué modelos teóricos comparan regímenes cambiarios?
			\3 ¿Qué ventajas e inconvenientes tiene cada régimen cambiario?
			\3 ¿Qué hace adecuado uno u otro régimen cambiario?
			\3 ¿Qué evidencia empírica existe al respecto?
			\3 ¿Cómo se interviene en el mercado cambiario?
			\3 ¿Para qué sirven las intervenciones?
			\3 ¿Cómo regulan las autoridades monetarias los mercados de cambio?
		\2 Estructura
			\3 Regímenes cambiarios
			\3 Análisis comparado de regímenes cambiarios
			\3 Intervención y regulación
	\1 \marcar{Regímenes cambiarios}
		\2 Idea clave
			\3 Contexto
			\3 Objetivos
			\3 Resultados
		\2 Regímenes de TC fijo ``duros''
			\3 Unión monetaria
			\3 Dolarización
			\3 Junta de conversión
		\2 Fijación suave del tipo de cambio nominal
			\3 Adjustable peg/fijo ajustable
			\3 Crawling peg
			\3 Bandas de flotación
		\2 Tipo de cambio flexible
			\3 Managed float o dirty float
			\3 Flotación libre
	\1 \marcar{Análisis comparado}
		\2 Idea clave
			\3 Contexto
			\3 Objetivo
			\3 Resultados
		\2 Análisis teórico
			\3 Clásicos
			\3 Bimetalismo vs oro vs plata
			\3 Nurkse
			\3 Meade
			\3 Friedman
			\3 Mundell y Fleming
			\3 Dornbusch: overshooting con tipos flexibles
			\3 Krugman (1991): bandas de flotación y eq. múltiples
			\3 Empleo friccional
			\3 Crisis cambiarias
			\3 Deuda externa
			\3 Economía política
			\3 Instituciones
		\2 Implicaciones
			\3 Factibilidad de regímenes de tipo de cambio fijo
			\3 Receta tradicional de amortiguación de shocks
			\3 Enfoque moderno
			\3 Controles de capital
		\2 Factores que determinan régimen óptimo
			\3 Idea clave
			\3[\textsc{i}] Tamaño y apertura
			\3[\textsc{ii}] Socio comercial
			\3[\textsc{iii}] Simetría de los shocks
			\3[\textsc{iv}] Movilidad del trabajo
			\3[\textsc{v}] Transferencias fiscales contracíclicas
			\3[\textsc{vi}] Transferencias corrientes contracíclicas
			\3[\textsc{vii}] Voluntad política
			\3[\textsc{viii}] Desarrollo financiero
			\3[\textsc{ix}] Origen de los shocks
		\2 Evidencia empírica
			\3 Patrón oro
			\3 Bretton Woods
			\3 Caída de Bretton Woods
			\3 Crisis cambiarias
			\3 Sudden stops y flow reversals
			\3 Hot money
			\3 Zona Euro
			\3 Miedo a flotar
	\1 \marcar{Intervención y regulación}
		\2 Intervención
			\3 Idea clave
			\3 \underline{Instrumentos}
			\3 Compraventa de divisas
			\3 Tipo de interés
			\3 Acumulación de reservas
			\3 Esterilización
			\3 Reputación del banco central
			\3 Coordinación de política monetaria
			\3 Expansión cuantitativa
			\3 Requisitos mínimos de divisas
			\3 \underline{Factores de éxito de la intervención}
			\3 Valoración
		\2 Regulación
			\3 Idea clave
			\3 Controles de capital
			\3 Control de cambio
			\3 Tipos de cambio múltiples
			\3 Valoración
	\1[] \marcar{Conclusión}
		\2 Recapitulación
			\3 Regímenes cambiarios
			\3 Análisis comparado de regímenes cambiarios
			\3 Intervención y regulación
		\2 Idea final
			\3 Debate histórico de política económica
			\3 Idiosincrasia de la economía
			\3 Supervisión macroprudencial
			\3 Crisis financieras
			\3 Euro como régimen de TC Fijo

\end{esquema}

\esquemalargo

\begin{esquemal}
	\1[] \marcar{Introducción}
		\2 Contextualización
			\3 Macroeconomía
				\4 Análisis de fenómenos económicos a gran escala
				\4 Énfasis sobre variables agregadas
			\3 Economías abiertas
				\4 Comercio internacional
				\4[] Intercambian ByS con otras economías
				\4[] $\to$ Precios relativos son importantes
				\4[] $\then$ Tipo de cambio es importante
				\4[] $\then$ DAgregada depende de exterior
				\4 Flujos financieros internacionales
				\4[] Intercambio de activos y pasivos
				\4[] Suavización intertemporal de rentas
				\4[] Dinámicas de deuda exterior
				\4 Tipo de cambio
				\4[] Precio más importante en una economía abierta
				\4[] $\to$ Relación entre bienes locales y extranjeros
				\4 Interacción de sector exterior y ec. doméstica
				\4[] Demanda exterior sobre demanda agregada y output
				\4[] Diferenciales de precios
				\4[] Condiciones de financiación
			\3 Régimen cambiario
				\4 Estados/aut. monetarias pueden afectar TCN
				\4[] Interviniendo el mercado de divisas
				\4[] Regulando intercambios de divisas
				\4 Concepto de régimen cambiario
				\4[] Conjunto de intervención+regulación
				\4[] Con el objetivo de:
				\4[] $\to$ Determinar TCN determinado
				\4[] $\to$ Alcanzar otros objetivos de PEconómica
				\4[] $\then$ Aparición de trade-offs
				\4[] $\then$ Necesario decidir qué régimen cambiario
				\4 Dimensión fundamental de un régimen cambiario
				\4[] TCN en nivel fijo o flexible
				\4[] $\to$ Qué TCN fijo
				\4[] $\to$ Qué grado de fluctuación
				\4[] $\to$ Qué actuaciones para lograrlo
		\2 Objeto
			\3 ¿Qué regímenes cambiarios existen?
			\3 ¿Qué modelos teóricos comparan regímenes cambiarios?
			\3 ¿Qué ventajas e inconvenientes tiene cada régimen cambiario?
			\3 ¿Qué hace adecuado uno u otro régimen cambiario?
			\3 ¿Qué evidencia empírica existe al respecto?
			\3 ¿Cómo se interviene en el mercado cambiario?
			\3 ¿Para qué sirven las intervenciones?
			\3 ¿Cómo regulan las autoridades monetarias los mercados de cambio?
		\2 Estructura
			\3 Regímenes cambiarios
			\3 Análisis comparado de regímenes cambiarios
			\3 Intervención y regulación
	\1 \marcar{Regímenes cambiarios}\footnote{\href{https://www.imf.org/en/Publications/Annual-Report-on-Exchange-Arrangements-and-Exchange-Restrictions/Issues/2019/04/24/Annual-Report-on-Exchange-Arrangements-and-Exchange-Restrictions-2018-46162}{Informe anual (2019) sobre regímenes cambiarios del FMI.}}
		\2 Idea clave
			\3 Contexto
				\4 Clasificación de regímenes cambiarios
				\4[] Más difusa cuanta más flexibilidad
				\4 Clasificación del FMI
				\4[] Mandato para reportar anualmente sobre regímenes cambiarios
				\4 De jure y de facto
				\4[] No siempre iguales
				\4[] Régimen de facto suele ser menos flexible que de iure
			\3 Objetivos
				\4 Identificar características definitorias
			\3 Resultados
				\4 Regímenes caracterizados por flexibilidad del TCN
				\4 Flexibilidad depende de:
				\4[] Margen de actuación de autoridades monetarias
				\4[] Compromiso de actuación para mantener un nivel de TCN
				\4 Regímenes polares
				\4[] Fijo y flexible son simplificaciones
				\4 Realidad de regímenes cambiarios
				\4[] Combinación de:
				\4[] $\to$ Instituciones legales y políticas
				\4[] $\to$ Compromiso de tipo de cambio
				\4[] $\then$ Múltiples variantes
				\4 Extremos de fijación/flexibilidad
				\4[] De más a menos fuerte compromiso de TC fijo
		\2 Regímenes de TC fijo ``duros''
			\3 Unión monetaria
				\4 Comparte con otros miembros de UM
				\4[] Política monetaria
				\4[] Autoridad monetaria
				\4[] Moneda física y unidad de cuenta
				\4[] Soberanía
				\4[] $\to$ Requiere cesiones a instituciones comunes
				\4[] $\to$ Generalmente también en otras áreas
				\4 Beneficios de señoreaje
				\4[] Repartidos entre miembros de UM
				\4 Régimen cambiario con terceros
				\4[] Puede ser cualquier otro
			\3 Dolarización
				\4 País renuncia a:
				\4[] Moneda física propia y unidad de cuenta
				\4[] $\to$ Adoptando país de tercero
				\4[] Política monetaria
				\4 Mantiene:
				\4[] Autoridad monetaria con funciones mínimas
				\4[] $\to$ Estadísticas
				\4[] $\to$ Cuentas del gobierno
				\4[] $\to$ Sistema de pagos nacionales
				\4[] $\to$ Gestión de reservas
				\4 Señoreaje
				\4[] País que emite moneda extrae todo el beneficio
			\3 Junta de conversión
				\4 País renuncia legal/constitucionalmente a:
				\4[] Política monetaria
				\4[] $\then$ Moneda nacional es token sobre divisa
				\4 Mantiene:
				\4[] Autoridad monetaria con funciones reducidas
				\4[] $\to$ Estatuto legal limitativo
				\4[] $\to$ Vender/comprar moneda nacional
				\4[] $\to$ Sin discrecionalidad alguna (en teoría)
				\4[] $\to$ Mantiene funciones propias de BC en dolarización
				\4 Señoreaje
				\4[] En teoría:
				\4[] $\to$ Sólo por ingresos de reservas de divisas\footnote{Las reservas que respaldan completamente la oferta monetaria de moneda local generan un interés que constituye un ingreso por señoreaje para el país en cuestión. Si los tipos de interés de las reservas son negativos, la autoridad monetaria puede incurrir en pérdidas y obtener de hecho un ``señoreaje'' negativo.}
				\4[] En la práctica:
				\4[] $\to$ AMonetaria puede ``hacer trampas''
		\2 Fijación suave del tipo de cambio nominal
			\3 Adjustable peg/fijo ajustable
				\4 País renuncia a:
				\4[] Política monetaria
				\4[] $\to$ Se compromete a defender tipo de cambio
				\4 Mantiene:
				\4[] Autoridad monetaria con PM
				\4[] $\to$ Sin estatuto legal limitativo de funciones
				\4[] $\to$ Posible cambio de pol. sin reforma const./legal
				\4[] $\to$ Posibles ajustes de tipo fijo cada cierto tiempo
				\4[] $\then$ Generalmente, fluctuación $\pm$ 1\%, más de 6 meses
				\4 Variantes
				\4[] Fijo en relación a punto determinado
				\4[] Bandas de fluctuación permitidas
				\4[] $\to$ Más ``fijo'' cuanto menores sean
				\4 Más habitual junto con managed float
			\3 Crawling peg
				\4 País se compromete a ajuste gradual
				\4[] Dado TCN objetivo declarado
				\4[] $\to$ Variaciones graduales hasta objetivo
				\4[] $\to$ Frecuentes en el tiempo
				\4 Política monetaria supeditada a régimen
				\4[] Pero ajustes frecuentes son aceptables
			\3 Bandas de flotación
				\4 País se compromete a mantener dentro de banda
				\4 Bandas amplias de fluctuación
				\4[] Sólo interviene si se acerca a bandas
				\4 Krugman (1991)
				\4[] Credibilidad de intervención es importante
				\4[] Si capacidad de BCentral es creíble
				\4[] $\to$ Mercados actúan solos para mantener en bandas
				\4[] $\to$ Intervención no es necesaria
				\4[] $\to$ TCN cae cuando se acerca a límite superior
				\4[] $\to$ TCN aumenta cuando se acerca a límite inferior
				\4[] Si capacidad de BCentral no es creíble
				\4[] $\then$ Situación similar a Krugman (1979)
				\4[$\then$] Expectativas juegan papel clave
		\2 Tipo de cambio flexible
			\3 Managed float o dirty float
				\4 TC fluctúa libremente pero BC interviene
				\4 Intervenciones relativamente frecuentes
				\4 Cierta opacidad respecto a criterios de intervención
				\4 Más habitual junto con fijo ajustable
				\4 Asociado al miedo a flotar
			\3 Flotación libre
				\4 Intervención excepcional
				\4 Sólo para evitar condiciones de extrema volatilidad
	\1 \marcar{Análisis comparado}
		\2 Idea clave
			\3 Contexto
				\4 Análisis comparado
				\4[] Extraer características que definen régimen cambiario
				\4[] Valorar efectos sobre macromagnitudes relevantes
				\4[] $\to$ Output
				\4[] $\to$ Empleo
				\4[] $\to$ Inflación
				\4[] $\to$ Saldo exterior
				\4[] $\to$ ...
				\4 Regímenes cambiarios
				\4[] Puede ser o no factibles
				\4[] Pueden tener diferentes efectos sobre
				\4[] $\to$ Niveles de variables stock
				\4[] $\to$ Varianza de variables flujo
				\4[] ...
				\4 Amplia variedad de evidencia empírica
				\4 Muchos modelos teóricos
				\4[] Enfatizan diferentes aspectos
				\4[] $\to$ Diferentes implicaciones sobre rég. óptimo
			\3 Objetivo
				\4 Caracterizar regímenes cambiarios
				\4 Describir teorías sobre efectos de shocks
				\4 Valorar optimalidad de regímenes en función de contexto
			\3 Resultados
				\4 Debate de l/p en torno a fijo vs flexible
				\4 Evidencia mixta
				\4[] Buenos y malos resultados con ambos regímenes
				\4 Acumulación de reservas muy frecuente
				\4[] A pesar de tipo de cambio nominalmente fijo
				\4 Acumulación de desequilibrios a pesar de flexibilidad
				\4 Habitual poca transparencia sobre régimen real
				\4 Régimen óptimo muy dependiente del contexto
				\4[] Instituciones
				\4[] Shock en cuestión
				\4[] Coyuntura internacional
				\4[] Sistema financiero
				\4[] ...
		\2 Análisis teórico
			\3 Clásicos
				\4 Debate sobre convertibilidad o no a oro
				\4 Currency school
				\4[] Evitar crecimiento excesivo de la oferta monetaria
				\4[] $\to$ Evitar inflación
				\4[] $\then$ Debe restablecerse convertibilidad libra oro
				\4[] Régimen cambiario óptimo
				\4[] $\to$ Libre convertibilidad con oro
				\4[] $\to$ Oro puede entrar y salir
				\4[] Tipo de cambio fijo vía paridad con oro
				\4 Banking school
				\4[] Oferta monetaria no es inflacionaria
				\4[] $\to$ Porque es endógena a actividad económica
				\4[] No es necesaria la convertibilidad con el oro
				\4[] $\to$ No se emitirán billetes en exceso
				\4[] Tipos de cambio pueden fluctuar
			\3 Bimetalismo vs oro vs plata
				\4 Países bimetálicos
				\4[] Plata y oro se venden a precio fijado en ceca
				\4[] $\to$ Plata y oro
				\4 Países en oro
				\4[] Gran Bretaña
				\4 Países en plata
				\4[] Hispanoamérica
				\4[] Austria-Hungría
				\4[] Rusia
				\4[] Escandinavia
				\4 Implicaciones
				\4[] Convertibilidad respecto a oro
				\4[] $\to$ Tipo de cambio fijo entre sí
				\4[] Convertibilidad respecto a plata
				\4[] $\to$ Tipo de cambio fijo entre sí
				\4[] Países bimetálicos (Francia)
				\4[] $\to$ Estabilizan relación oro-plata en mercado internacional
				\4[] $\then$ Permiten estabilidad cambiaria zona oro y plata
				\4 Abandono de bimetalismo
				\4[] Permite fluctuación zonas oro y plata
				\4[] $\to$ Más países cambian a oro
				\4[] $\then$ Emergencia de patrón oro
			\3 Nurkse\footnote{Ver Eichengreen, B. (2018) Ragnar Nurkse and the international  financial architecture.}
				\4 Defensa de TCN fijos a nivel de sistema internacional
				\4 International currency experience (1944)
				\4 Crítica general a fluctuaciones de TCN
				\4 Sistema monetario tras IGM
				\4[] Restauraciones de convertibilidad respecto a oro
				\4[] $\to$ Progresivamente
				\4[] $\to$ Países prefieren restaurar después
				\4[] $\then$ Tipos de cambio fijo tras restaurar
				\4[] $\then$ Paridades competitivamente fijadas
				\4[] $\then$ Inestabilidad del sistema
				\4 Tras caída de patrón oro
				\4[] Devaluaciones competitivas
				\4 Nurkse niega estabilidad de mecanismo de ajuste vía TCN
				\4[] Marshall-Lerner no siempre se cumple
				\4[] $\then$ Posible inestabilidad de devaluaciones
				\4 Devaluaciones competitivas perjudiciales
				\4[] Frecuentes en años 30
				\4[] Nurkse considera negativas
				\4 Régimen cambiario óptimo
				\4[] Tipos de cambio fijos
				\4[] Paridades fijadas cooperativamente
				\4[] Devaluaciones sólo posibles si circunst. excepcionales
			\3 Meade
				\4 Rigideces salariales y de precios
				\4[] $\to$ Dificultan ajuste ante shock real externo
				\4[] $\to$ Tipo de cambio fijo impide ajuste
				\4[] $\then$ Mejor tipos ``variables''
				\4 Defensa de régimen intermedio
				\4[] Tipos fijos pero ajustables a menudo
				\4[] Soft-peg
			\3 Friedman
				\4 Defensa de TCN flexibles a nivel de sistema internacional
				\4 Instrumentos de ajuste de la balanza de pagos
				\4[] i. Variaciones del tipo de cambio nominal
				\4[] $\to$ Vía fuerzas de mercado en TCN flexible
				\4[] $\to$ Vía regulación/intervención+mercado en TCN fijo
				\4[] ii. Variación de precios
				\4[] $\to$ Deflación si TCR demasiado apreciado
				\4[] $\to$ Aranceles
				\4[] iii. Controles directos a transacciones internacionales
				\4[] $\to$ Cuotas
				\4[] $\to$ Restricciones a importación
				\4[] iv. Reservas de divisas
				\4[] $\to$ Compra de divisas extranjeras para saldar superávit
				\4[] $\to$ Venta de divisas extranjeras para saldar déficit
				\4[] $\then$ Potencial efecto sobre oferta monetaria
				\4[] $\then$ Esterilización necesaria para mantener oferta monetaria
				\4 Argumenta por qué TCN flexible es el mejor instrumento
				\4[] Plantea problemas de los otros tres instrumentos
				\4[] Rebate críticas a TCN flexible como mecanismo óptimo
				\4 Devaluaciones
				\4[] Hasta que tiene lugar, aumentan desequilibrios
				\4[] Ajuste es más brusco y rápidoº
				\4[] $\to$ Mayores costes
				\4 Utilización de reservas
				\4[] Posible utilizar para pequeños ajustes transitorios
				\4[] $\to$ Aunque TCN flexible sería más rápido
				\4[] Para shock persistentes, inviable
				\4[] $\to$ Reservas son infinitas si déficit
				\4[] $\to$ Compra de divisas es inflacionaria
				\4[] $\then$ Necesario esterilizar
				\4[] $\then$ Esterilizar mantiene desequilibrio
				\4 Cambios en precios
				\4[] En teoría, es un mecanismo válido
				\4[] En la práctica, es inviable
				\4[] Los salarios son rígidos
				\4[] $\to$ Aún más al alza que a la baja
				\4[] $\to$ Ajustes de precios acaban siendo deflacionarios
				\4[] $\then$ Desempleo es objetivo esencial de política macro
				\4[] $\then$ Imposible soslayar
				\4[] $\then$ Acaba siendo impracticable
				\4 Controles directos sobre los intercambios
				\4[] Ingresos no pueden aumentarse administrativamente
				\4[] Medidas viables para aumentar exportación
				\4[] $\to$ Deben introducirse al margen de régimen cambiario
				\4[] Medidas que limiten importaciones
				\4[] $\to$ Extremadamente onerosas
				\4[] $\to$ Introducen distorsiones adicionales
				\4[] $\to$ Agentes encuentran formas de evitar controles
				\4[] $\then$ Acaban siendo coste adicional
				\4 Argumentos contra el TCNFlexible
				\4[] Libre flotación aumenta incertidumbre
				\4[] $\to$ TCNFlexible incierto se debe a condiciones subyacentes
				\4[] $\to$ Si economía sufre incertidumbre, TCNFlexible lo sufrirá
				\4[] $\to$ TCNFijo también sufriría incertidumbre si cond. subyacentes
				\4[] Variaciones de precios hacen innecesario TCNFlexible
				\4[] $\to$ Pero en l/p, precios dependen de factores monetarios
				\4[] $\to$ Espirales inflacionarias también posibles con TCNfijo
				\4[] Variaciones necesarias del TCN no se producen a tiempo
				\4[] $\to$ Los especuladores proveen capital necesario hasta ajuste
				\4[] $\to$ Perspectiva de ajuste a equilibrio incentiva especuladores
				\4 TCNFlexible protege frente a shocks dda. por factores nominales
				\4[] Si la demanda agregada en el extranjero cae
				\4[] $\to$ La demanda de exportaciones caerá
				\4[] Si un shock nominal afecta a la cuenta corriente
				\4[] $\to$ TCN demasiado alto/bajo
				\4[] $\to$ Nivel de precios demasiado alto/bajo
				\4[] $\then$ TCR alejado de nivel que equilibra CC
				\4[] $\then$ TCN flexible garantiza ajuste a equilibrio
				\4 Propuesta de Friedman
				\4[] Reglas de crecimiento de la oferta monetaria
				\4[] $\to$ Estabilizar inflación y output
				\4[] Tipos de cambio flexible a nivel mundial
				\4[] $\to$ Facilitar ajuste frente a shocks nominales externos
				\4[] Liberalización de flujos de capital
				\4[] $\to$ Mejorar asignación de capacidad productiva
				\4[] Liberalización de flujos comerciales
				\4[] $\to$ Maximizar bienestar de consumidores y output
				\4 Impacto
				\4[] Parcialmente tras caída de Bretton Woods
				\4[] $\to$ Sin régimen de TCN a partir de acuerdos Jamaica
			\3 Mundell y Fleming
				\4[IS] $Y = C(Y) + I(r) + NX(Y,E)$
				\4[] $Y = C_0 + c Y + I_0 + I(r) + X_0 + X(E) - (M_0 + m Y + M(E))$
				\4[] $\to$ $c,m < 0$, $I_r < 0$, $X_E > 0$, $M_E < 0$,
				\4[LM] $\frac{M}{P} = L(Y,r)$
				\4[] $\to$ $L_Y > 0$, $L_r > 0$
				\4[BP] $\Delta R = NX(Y,E) + CF(r, r^*)$
				\4[] $\Delta R = NX(Y,E) + K \cdot (r- r^*)$
				\4[] $\to$ $K \to \infty$: mov. perfecta de K
				\4[] $\to$ $K \to 0$: sin movilidad de K
				\4 Tres ecuaciones en tres incógnitas
				\4[] Asumiendo $\Delta R$ exógena = 0 en general
				\4[] $\to$ $Y$, $r$, $E$
				\4 Libre mov de K + TCN Flexible
				\4[] Shock negativo de demanda agregada\footnote{Atribuible a múltiples causas posibles, que en cualquier caso resultan en una reducción de la demanda agregada. Una posibilidad es que la absorción interna caiga como resultado de políticas fiscales contractivas o un shock a las expectativas que reduzca la inversión autónoma. Otra posibilidad es una reducción autónoma de la demanda de exportaciones netas, por un shock negativo a la demanda agregada en los socios comerciales. En ambos casos, la curva IS se desplaza hacia la izquierda, de tal manera que para un nivel dado de tipo de interés el output de equilibrio es menor.}
				\4[] $\to$ IS se desplaza a la izquierda
				\4[] $\to$ Caída de tipo de interés doméstico
				\4[] $\to$ Salida de capital
				\4[] $\to$ Depreciación del TCN
				\4[] $\to$ Depreciación del TCR
				\4[] $\to$ Aumento de exportaciones netas
				\4[] $\then$ Mitigación del impacto del shock sobre Y
				\4[] Shock negativo sobre oferta monetaria\footnote{Atribuible a múltiples causas, que en cualquier caso provocan un exceso de demanda de dinero. Una posibilidad es que la oferta monetaria se reduzca como resultado de una contracción del crédito bancario. Otra posibilidad es un aumento de la demanda de dinero independiente de la renta y del tipo de interés. En ambos casos, la curva LM  se desplaza hacia la izquierda, de tal manera que para un nivel dado de tipo de interés el output de equilibrio es menor.}
				\4[] $\to$ LM se desplaza a la izquierda
				\4[] $\to$ Aumento del tipo de interés doméstico
				\4[] $\to$ Entrada neta de capital
				\4[] $\to$ Apreciación del TCN
				\4[] $\to$ Apreciación del TCR
				\4[] $\to$ Caída de exportaciones netas
				\4[] $\then$ Amplificación del impacto del shock sobre Y
				\4[$\then$] TCNFlex mitiga shock DA, amplifica shock M
				\4 Libre mov de K + TCN Fijo
				\4[] Shock negativo de demanda agregada
				\4[] $\to$ IS a la izquierda
				\4[] $\to$ Caída del tipo de interés
				\4[] $\to$ Salida de capital
				\4[] $\to$ Necesario aumentar interés para frenar salida de K
				\4[] $\to$ Necesario reducir M para aumentar interés
				\4[] $\then$ Amplificación del impacto sobre Y
				\4[] Shock negativo sobre oferta monetaria
				\4[] $\to$ LM a la izquierda
				\4[] $\to$ Aumento del tipo de interés
				\4[] $\to$ Necesario reducir interés para frenar salida de K
				\4[] $\to$ Necesario aumentar M para reducir interés
				\4[] $\then$ Mitigación del impacto sobre Y
				\4[$\then$] TCNFijo amplifica shock DA, mitiga shock M
				\4 Si shocks de demanda agregada son más prevalentes
				\4[] Preferible tipo de cambio flexible
				\4 Si shock sobre oferta monetaria son más prevalentes
				\4[] Preferible tipo de cambio fijo
			\3 Dornbusch: overshooting con tipos flexibles
				\4 Contexto post-BW
				\4[] Tipos fijos respecto al dólar
				\4 Overshooting con TCN Flexible + rigideces de precios
				\4[] \grafica{dornbuschpm}
				\4 IS
				\4[] \fbox{$y = g + \delta(e + p^* - p) - \sigma i$}
				\4 LM
				\4[] \fbox{$m-p = \phi y - \lambda i$}
				\4 UIP -- Paridad descubierta de interés
				\4[] \fbox{$i = i^* + \dot{e}^e$}
				\4 Curva de Phillips
				\4[] \fbox{$\dot{p} = \pi(y-\bar{y})$}
				\4 HER sobre tipo de cambio nominal
				\4[] \fbox{$\dot{e}^e = \dot{e}$}
				\4 Diagrama de fase
				\4[] Espacio $p$--$e$
				\4[] $p$ en abscisas, $e$ en ordenadas
				\4 Shock de política monetaria
				\4[] Desplaza curva $\dot{e} = 0$ hacia la derecha
				\4[] $\to$ Salto a nueva senda de ajuste con precios rígidos
				\4[] $\then$ Depreciación instantánea más que futura
				\4[$\then$] Overshooting más allá de equilibrio de largo plazo
				\4 Implicaciones
				\4[] TCN flexible puede inducir mayor volatilidad
				\4[] $\to$ De TCN más allá de intrínseca a flotación
				\4[] $\to$ De TCR en la medida que precios sean rígidos
			\3 Krugman (1991): bandas de flotación y eq. múltiples
				\4 Krugman (1991)
				\4 Credibilidad de intervención es importante
				\4[] Si capacidad de BCentral es creíble
				\4[] $\to$ Mercados actúan solos para mantener en bandas
				\4[] $\to$ Intervención no es necesaria
				\4[] $\to$ TCN cae cuando se acerca a límite superior
				\4[] $\to$ TCN aumenta cuando se acerca a límite inferior
				\4 Si capacidad de BCentral no es creíble
				\4[] $\then$ Situación similar a Krugman (1979)
				\4[$\then$] Expectativas juegan papel clave
			\3 Empleo friccional
				\4 Régimen cambiario tiene efectos sobre empleo
				\4 Régimen de tipo de cambio fijo con principal socio
				\4 Fluctuaciones frecuentes
				\4[] Afecta recursos dedicados a sector exportador
				\4 Reasignaciones frecuentes
				\4[] Aumento del paro friccional
				\4 Régimen de tipo de cambio fijo
				\4[] Menor número de reasignaciones
				\4[] Reasignaciones más bruscas
				\4[] $\to$ Devaluaciones
				\4[] $\to$ Crisis cambiarias
				\4 Régimen de tipo de cambio flexible
				\4[] Reasignaciones más frecuentes
				\4[] Posiblemente más graduales
				\4[] Efectos de incertidumbre pueden detraer contratación
			\3 Crisis cambiarias
				\4 Concepto de crisis cambiaria
				\4[] Ajuste brusco de paridad
				\4[] Autoridades no son capaces de mantener
				\4 Régimen de tipo de cambio fijo
				\4[] Mucho más susceptible a crisis
				\4[] $\to$ Por ajuste brusco intrínseco a TC fijo
				\4 Diferentes modelos/tipos de crisis
				\4[] Crisis de primera generación
				\4[] $\to$ Financiación monetaria déficit fiscal
				\4[] $\to$ Aumento de crédito doméstico
				\4[] $\to$ Necesaria venta de reservas paralela si déficit CC
				\4[] $\then$ Agotamiento de reservas implica ruptura de paridad
				\4[] $\then$ Posible incompatibilidad
				\4[] $\then$ Incentivo a ataques especulativos
				\4[] Segunda generación
				\4[] $\to$ Minimización de función de pérdida inflación output
				\4[] $\to$ Régimen TCFijo requiere mantener dif. inflación
				\4[] $\to$ Equilibrios múltiples: puede ser o no sostenible
				\4[] $\to$ Expectativas de inflación elevada
				\4[] $\then$ Aumenta coste en output de mantener TCN fijo
				\4[] $\then$ Posible abandono de TCNFijo
				\4[] $\to$ Expectativas de inflación elevada
				\4[] $\then$ Reduce coste en output de mantener TCN fijo
				\4[] $\then$ Posible abandono de TCNFijo
				\4[] Tercera generación
				\4[] $\to$ Garantía implícita de gobierno sobre TCN
				\4[] $\to$ Endeudamiento de sector privado en divisa
				\4[] $\to$ Riesgo moral de bancos privados con dólares
				\4[] $\then$ Reversal flow o sudden stop de flujos
				\4[] $\then$ Enorme presión hacia depreciación
				\4[] $\then$ Aumento de tipos de interés desata recesión
				\4[] $\then$ Imposible refinanciar deuda
				\4[] $\then$ Caída del régimen de tipo fijo
			\3 Deuda externa
				\4 Régimen cambiario afecta deuda externa
				\4[] Especialmente países que no pueden endeudarse en moneda propia
				\4 Confianza en régimen cambario es esencial
				\4 Tipo de cambio fijo
				\4[] Sector privado t
			\3 Economía política\footnote{Ver Steinberg y Walter (2013) en carpeta del tema.}
				\4 Idea clave
				\4[] Régimen cambiario beneficia/perjudica grupos de interés
				\4[] $\to$ Régimen flexible o fijo
				\4[] $\to$ Nivel de TCN si fijo
				\4[] Asumiendo conocen preferencias y efectos de $\Delta$ TCN
				\4 Preferencias a favor de TCNFijo
				\4[] Agentes expuestos a mercado exterior
				\4[] $\to$ Reducen incertidumbre
				\4[] No todos quieren mismo nivel de TCN
				\4[] $\to$ Importadores quieren nivel apreciado
				\4[] $\to$ Exportadores quieren nivel depreciado
				\4[] Si paridad se desvía de preferencias
				\4[] $\to$ Pueden presionar hacia tipo flexible
				\4[] $\then$ Estimando que TCN tiende a equilibrio deseable
				\4 Preferencias a favor de TCNFlexible
				\4[] Agentes expuestos a paro
				\4[] $\to$ Prefieren equilibrio interno
				\4[] $\then$ Política monetaria independiente para equilibrio interno
				\4 Críticas
				\4[] Efecto de $\Delta$ TCN es difícil de conocer
				\4[] Grupos de interés no siempre pueden organizarse
				\4[] $\to$ ¿Es realmente relevante la economía política?
				\4[] $\to$ ¿Hasta qué punto puede conocerse su impacto?
			\3 Instituciones\footnote{Ver Steinberg y Walter (2013) en carpeta del tema.}
				\4 Idea clave
				\4[] Conjunto de normas y mecanismos legales
				\4[] $\to$ Transforman prefs. individuales en decisiones gobierno
				\4[] Instituciones democráticas/autoritarias
				\4[] $\to$ ¿Hacen óptimo uno u otro régimen cambiario?
				\4[] $\to$ ¿Tienden a llevar a cabo uno u otro régimen?
				\4 Autocracias
				\4[] Más habitual tipo de cambio fijo
				\4 Democracias
				\4[] Menos tendencia a tipo de cambio fijo
				\4[] Habitual intervención para estabilizar TCN
				\4[] Mucha heterogeneidad dentro de democracias
				\4 Independencia del banco central
				\4[] Evidencia mixta
				\4[] Casi cualquier tipo de conclusión extraíble
				\4[] $\to$ Poca robustez
		\2 Implicaciones
			\3 Factibilidad de regímenes de tipo de cambio fijo
				\4 TCN fijo no es necesariamente factible
				\4 Intentos múltiples por caracterizar factibilidad
				\4 Trinidad imposible de Mundell-Fleming
				\4[] Derivado del modelo de Mundell-Fleming
				\4[] Importante soporte empírico
				\4[] Sólo dos de tres características son posibles:
				\4[] \textsc{I} Tipo de cambio fijo
				\4[] \textsc{II} Libertad de movimiento de capital
				\4[] \textsc{III} Política monetaria independiente
				\4[] Supongamos I y II
				\4[] $\to$ Régimen de TC Fijo estable
				\4[] Añadimos III
				\4[] $\to$ PM expansiva utilizada para equilibrio interno
				\4[] Aparece divergencia de interés doméstico y mundial
				\4[] $\to$ Salida de capital y EDemanda de divisas
				\4[] $\then$ TCN Fijo imposible
				\4 Dilema, no trilema
				\4[] Propuesto por Rey (2015)
				\4[] Régimen cambiario es irrelevante
				\4[] $\to$ TCN no aísla de shock exteriores
				\4[] Autoridades monetarias deben elegir entre:
				\4[] \textsc{I} Movilidad de capital
				\4[] \textsc{II} Política monetaria independiente
				\4[] Existe un ciclo financiero global
				\4[] $\to$ Determinado por Fed y principales bancos centrales
				\4[] Con movilidad de capitales
				\4[] $\to$ Tipos de interés en USA, UE afectan resto del mundo
				\4[] $\to$ PM debe adaptarse a ciclo financiero global
				\4[] $\then$ Única solución es control de cuenta financiera
				\4 Trilema de Pisany-Ferry del sector financiero
				\4[] Planteado por Pisany-Ferry (2012)
				\4[] Asumiendo:
				\4[] $\to$ régimen de tipo fijo factible
				\4[] $\to$ Integración monetaria avanzada
				\4[] Sólo son posibles dos:
				\4[] \textsc{I} Interdependencia banca-sector público
				\4[] \textsc{II} Financiación monetaria del déficit prohibida
				\4[] \textsc{III} Sin corresponsabilidad respecto a deuda pública
				\4[] Con I y II:
				\4[] $\to$ Necesaria unión fiscal
				\4[] Con I y III:
				\4[] $\to$  Necesario prestamista de último recurso a nivel de AMonetaria
				\4[] Con II y III:
				\4[] $\to$  Necesaria unión financiera efectiva
			\3 Receta tradicional de amortiguación de shocks
				\4 Shock demanda agregada
				\4[] Resultado de:
				\4[] $\to$ Demanda de exportaciones netas
				\4[] $\to$ Absorción interna
				\4[] Régimen óptimo:
				\4[] $\to$ TCN flexible
				\4 Shock de oferta monetaria
				\4[] Resultado de:
				\4[] $\to$ Caída de oferta monetaria
				\4[] $\to$ Aumento de demanda de dinero
				\4[] Régimen óptimo
				\4[] $\to$ TCN fijo
				\4[] $\then$ Evitar amplificar shock negativo
			\3 Enfoque moderno
				\4 Shocks son mucho más complejos
				\4[] No son exógenos
				\4[] $\to$ En gran medida, endógenos a régimen cambiario
				\4[] $\to$ Endógenos a instituciones financieras y BC
				\4 Cambios de régimen por cada shock no son viables
				\4 No sólo se trata de reducir volatilidad de output
				\4[] También de inflación
				\4[] También de comercio
				\4[] También de reorganización de los ff.pp.
				\4 Trabajo muy difícilmente reasignable entre sectores
				\4[] Reduce deseabilidad de:
				\4[] $\to$ Tipo de cambio flexible
				\4[] $\to$ Tipo de cambio fijo suave con devaluaciones
				\4 Mecanismos de ajuste de TCN no siempre llevan a estabilidad
				\4[] Pass-through de importaciones es esencial
				\4[] En contexto de:
				\4[] $\to$ Déficit de CC
				\4[] $\to$ Depreciación de moneda local
				\4[] $\to$ Inicialmente, LCP\footnote{Los precios de las importaciones se fijan en la moneda del país importador.}
				\4[] $\to$ Muy poco pass-through de importaciones
				\4[] $\then$ Poca elasticidad de importaciones a TCN
				\4[] $\then$ Depreciación aumenta déficit
				\4[] $\then$ Posible espiral depreciación
				\4[] $\then$ Aumento inflación cuando aumente pass-through
			\3 Controles de capital
				\4 Flujos de capital hacen más difícil reg. flexible
				\4 Pueden ser útiles en TCN flexible y fijo
				\4 Permiten evitar sudden-stops y flow reversals
				\4 En TCN Fijo aumentan credibilidad de fijación
				\4 En TCN Flexible reducen volatilidad cambiaria
				\4[] especialmente cuando controles sobre flujos de c/p
		\2 Factores que determinan régimen óptimo\footnote{Basado en apartado 28.5, Handbook of Fixed Exchange Rates, capítulo por Frankel, J.}
			\3 Idea clave
				\4 No existe régimen perfecto en general
				\4 Contexto determina régimen óptimo
				\4[] Coyuntural
				\4[] Idiosincrático de la economía en cuestión
			\3[\textsc{i}] Tamaño y apertura
				\4 Países pequeños y abiertos
				\4[] Más beneficios del comercio
				\4[] $\to$ Mayores ventajas del TCFijo
				\4 Países grandes y cerrados
				\4[] Menos beneficios del comercio
				\4[] Más margen de actuación de PM
				\4[] $\to$ Mayores ventajas de TCFlexible
			\3[\textsc{ii}] Socio comercial
				\4 Existencia de un socio muy grande
				\4[] Con el que lleva a cabo elevado \%:
				\4[] $\to$ Comercio
				\4[] $\to$ Inversión
				\4[] $\to$ Perspectivas de relaciones futuras
				\4[] $\then$ TCFijo ofrece más ventajas
				\4 Fijación respecto a cesta
				\4[] Cuando hay varios socios dominantes
				\4 Países con economías muy diversificadas
				\4[] TCFlexible puede ser más estable que fijo
				\4[] $\to$ Sin inconvenientes de tipo fijo
			\3[\textsc{iii}] Simetría de los shocks
				\4 Correlación alta
				\4[] Shocks que afectan a país emisor de moneda ancla
				\4[] $\to$ Muy correlacionados con shocks domésticos
				\4[] $\to$ Movimiento de interés tenderá a ser similar
				\4[] $\then$ TCFijo es ventajoso
			\3[\textsc{iv}] Movilidad del trabajo
				\4 Mayor movilidad del trabajo
				\4[] Más flexibilidad ante shocks asimétricos
				\4[] $\to$ TCFijo menos vulnerable ante shocks
				\4[] $\then$ Menos razones para TCFlexible
			\3[\textsc{v}] Transferencias fiscales contracíclicas
				\4 Transferencias fiscales
				\4[] Especialmente relevante en UM
				\4[] También receptores de transferencias de K
				\4[] Mayores transferencias:
				\4[] $\to$ Más resistencia a shocks
				\4[] $\then$ TCFlexible menos ventajoso que sin trans.
			\3[\textsc{vi}] Transferencias corrientes contracíclicas
				\4 Especialmente, remesas
				\4 Países con poblaciones emigrantes en desarrollados
				\4[] Remesas contracíclicas a PEDs
				\4[] Responden a diferencial de posiciones cíclicas
				\4[] $\to$ Suavizan shocks domésticos en PEDs
				\4[] $\then$ TCFlexible menos ventajoso que sin trans.
			\3[\textsc{vii}] Voluntad política
				\4 Importancia de soberanía monetaria
				\4[] Sujeto de debate político
				\4[] Soberanía nacional asociada a moneda propia
				\4[] $\to$ Puede ser difícil abandonar moneda propia/TCFijo
			\3[\textsc{viii}] Desarrollo financiero
				\4 Países con sist. financieros subdesarrollados
				\4[] Menos beneficios de TCFlexible
				\4[] Mercados financieros poco profundos
				\4[] $\to$ Poco margen para acomodar shocks
				\4[] $\then$ Aumentan costes de shocks
				\4[] $\then$ Preferible TCFijo
			\3[\textsc{ix}] Origen de los shocks
				\4 Receta tradicional sigue siendo influyente
				\4 Grado de implementación variable
		\2 Evidencia empírica
			\3 Patrón oro
				\4 Idea clave
				\4 Implicaciones
				\4[] Confianza de agentes en mantenimiento TCN
				\4[] $\to$ Factor clave para estabilidad
				\4 Valoración
				\4[] TCN fijo muestra en general:
				\4[] $\to$ Aumento de intercambios comerciales
				\4[] $\to$ Aumento de flujos de capital
			\3 Bretton Woods
				\4 Idea clave
				\4[] Sistema monetario internacional de tipos fijos
				\4[] Reasignaciones de paridad posibles
				\4[] $\to$ Si ``desequilibrio fundamental''
				\4[] Unido a restricciones:
				\4[] $\to$ Inicialmente, sobre cuenta corriente
				\4[] $\to$ A lo largo de BW, sobre cuenta financiera
				\4 Implicaciones
				\4[] Flujos de capital crecientes desestabilizan TCFijos
				\4[] TCNFijo con oro+moneda reserva global
				\4[] $\to$ Aumento progresivo de demanda de liquidez
				\4[] $\then$ Déficits de país central
				\4[] $\to$ Aumento de desequilibrio reservas de oro-d´olar
				\4[] $\then$ Dilema de Triffin
				\4[] $\then$ Eventual ruptura de sistema
				\4 Valoración
				\4[] Fuerte crecimiento en desarrollados
				\4[] Gran aumento del comercio
				\4[] Aumento de los desequilibrios
				\4[] Flujos de capital aumentan progresivamente
			\3 Caída de Bretton Woods
				\4 Idea clave
				\4[] Desintegración de sistema TCFijo de iure con dólar-oro
				\4[] Nuevo ``mundo monetario''
				\4[] Nuevo equilibrio tarda en aparecer
				\4[] Europa busca nuevo acuerdo de TCN fijos
				\4 Implicaciones
				\4[] Cambios de régimen cambiario
				\4[] $\to$ Tardan en consolidarse
				\4[] TCN fijo se destruyen rápidamente
				\4[] $\to$ Pero transición a nuevo régimen no es inmediata
				\4 Valoración
				\4[] Enorme aumento de flujos comerciales y de capital
				\4[] Crisis cambiarias y monetarias relativamente frecuentes
				\4[] Consolidación de miedo a flotar
			\3 Crisis cambiarias
				\4 Idea clave
				\4[] Objetivos incompatibles
				\4[] $\to$ Mantenimiento de tipo fijo
				\4[] $\to$ Expansión fiscal financiado monetariamente
				\4[] Provoca crisis cambiarias
				\4[] $\to$ Ajuste brusco de tipo de cambio
				\4 Crisis cambiarias desde fin de Bretton Woods
				\4[] Relativamente frecuentes
				\4[] Movimientos de K aumentan
				\4[] $\to$ Bancos americanos a Latinoamérica
				\4[] $\to$ Crisis asiáticas
				\4[] $\to$ Brasil
				\4[] $\to$ Turquía
				\4[] $\to$ ...
				\4 Implicaciones
				\4[] TCNFijo requiere disciplina fiscal y monetaria
				\4[] Régimen cambiario con sistema financiero inestable
				\4[] $\to$ Incentivos a financiación en divisa
				\4[] $\then$ Aumento de desequilibrios
				\4[] TCNFijo puede aumentar desequilibrios
				\4[] $\to$ Y aumentar brusquedad del ajuste
				\4 Valoración
				\4[] Argumento habitual contra TCN Fijo
				\4[] Empuja hacia managed float/dirty float
				\4[] $\to$ Reducir probabilidad de crisis cambiaria
				\4[] $\to$ Evitar efecto negativo de fluctuación de TC
			\3 Sudden stops y flow reversals
				\4 Idea clave
				\4[] ``It's not the speed that kills, but the sudden stop''
				\4[] Frenos bruscos a entradas de capital
				\4[] $\to$ Presionan regímenes cambiarios
				\4[] Tipo de cambio fijo
				\4[] $\to$ Garantía implícita del gobierno
				\4[] $\then$ Se mantendrá fijo el TCN
				\4[] Fuerte interacción con regímenes de TCN fijo
				\4[] $\to$ Dificultan mantenimiento de TCN fijo
				\4 Implicaciones
				\4[] TCN fijo con movilidad de capital
				\4[] $\to$ Difícil de mantener
				\4[] $\then$ Si basado en entrada de capitales
				\4[] Controles de capital pueden reducir flujos inestables
				\4[] $\to$ Aumentar vencimiento medio
				\4[] $\to$ Reducir dependencia de deuda
				\4[] $\to$ Reducir inversión en no comerciables (inmobiliario)
				\4 Valoración
			\3 Hot money
				\4 Idea clave
				\4[] Movimientos de capital bruscos y grandes
				\4[] Expectativa de apreciación/devaluación
				\4[] Contexto de tipo fijo
				\4 Implicaciones
				\4[] Regímenes soft peg
				\4[] $\to$ Sujetos a flujos de capital
				\4[] $\then$ Expectativas de relevantes
				\4[] Efectos sobre economía real doméstica
				\4[] $\to$ Inestabilidad de precios
				\4[] $\to$ Cambios bruscos en tipos de interés
				\4[] $\to$ Inestabilidad financiera
				\4 Valoración
				\4[] Relativamente raros en patrón oro
				\4[] $\to$ Confianza en estabilidad de tipo de cambio
				\4[] Entreguerras más frecuentes
				\4[] $\to$ Enorme entrada de K en Francia 26-28
				\4[] $\then$ Expectativa de revaluación
				\4[] $\to$ Salidas de oro desde bloque oro en años 30
				\4[] $\then$ Expectativa de ruptura de paridad con oro
				\4[] Bretton-Woods, años finales
				\4[] $\to$ Aparecen de nuevo los movimientos hot money
				\4[] $\to$ Preceden devaluaciones/revaluaciones
				\4[] 1973, rumores fin de Smithsonian Agreement
				\4[] $\to$ Estados Unidos: control de rentas y tipos bajos
				\4[] $\to$ Bundesbank: contracción monetaria
				\4[] $\then$ Hot money flows salen de EEUU hacia Alemania
				\4[] $\then$ Expectativa de devaluación de dolar
			\3 Zona Euro
				\4 Idea clave
				\4[] Unión monetaria por excelencia actualmente
				\4[] Tipos fijos entre EEMM
				\4[] Aparecen:
				\4[] $\to$ Trilema de Pisany-Ferry
				\4[] $\to$ Crisis de deuda
				\4[] $\to$ Coste creciente de abandonar reg. tipo de cambio fijo
				\4 Implicaciones
				\4[] Aplicación pura de Trilema de Pisany-Ferry
				\4[] Presión creciente hacia
				\4[] $\to$ Integración fiscal
				\4[] $\to$ Desintegración
				\4 Valoración
				\4[] Régimen cambiario es éxito absoluto en plano técnico
				\4[] Debate de largo plazo sobre éxito/económico
				\4[] $\to$ En qué medida aumenta comercio entre EEMM?
				\4[] $\to$ En qué medida dificulta equilibrio interno?
				\4[] $\to$ En qué medida aumenta inestabilidad financiera?
			\3 Miedo a flotar
				\4 Idea clave
				\4[] Recomendación de TCN flexible con receta tradicional
				\4[] $\to$ Cierto consenso en numerosos casos
				\4[] $\then$ Hacer frente a shocks de demanda agregada
				\4[] $\then$ Acelerar ajuste exterior
				\4[] $\then$ Evitar acumulación de desequilibrios y ajuste brusco
				\4[] Sin embargo, autoridades monetarias:
				\4[] $\to$ Intervenciones frecuentes
				\4[] $\to$ Acumulan grandes reservas de divisas
				\4[] $\then$ Realmente muy pocos en libre flotación
				\4 Implicaciones
				\4[] Existen factores más allá de receta tradicional
				\4[] $\to$ Que los países tienen en cuenta
				\4[] Deuda externa
				\4[] $\to$ Aumenta su valor si depreciación
				\4[] Devaluaciones asociadas con recesiones en EM
				\4[] $\to$ Evitar salida de capitales
				\4[] Dolarización del sector privado
				\4[] $\to$ Realmente, CC apenas afecta mercado divisas
				\4[] Inestabilidad cambiaria afecta crédito exterior
				\4[] $\to$ Agentes prefieren reducir al máximo
				\4 Valoración
				\4[] Flotación libre pura es realmente construcción teórica
				\4[] $\to$ Apenas existen casos
	\1 \marcar{Intervención y regulación}\footnote{Sincronizar con tema 3B-14, parte Operaciones.}
		\2 Intervención
			\3 Idea clave
				\4 BCentrales son agentes en mercados financieros
				\4[] En mercados cambiarios
				\4[] $\to$ Compran y venden divisas y moneda local
				\4[] En mercado abierto
				\4[] $\to$ Compran y venden deuda nacional
				\4 Actuaciones en mercados financieros internacionales
				\4[] Afecta TCN que prevalece
				\4[] $\to$ Intervención afecta regímenes cambiarios
				\4[] $\then$ Herramienta para implementar régimen cambiario
				\4[] $\then$ Existen también otras razones para intervenir
			\3 \underline{Instrumentos}
			\3 Compraventa de divisas
				\4[] Aut. monetaria compra/vende directamente
				\4[] Ejemplo:
				\4[] $\to$ BoJ en septiembre de 2010
				\4[] $\to$ Compra 24 billones de dólares en un día
			\3 Tipo de interés
				\4[] Variar interés doméstico
				\4[] $\to$ Para atraer capital
				\4[] $\then$ Sostener TCN
				\4[] Ejemplo: Turquía en verano de 2018
			\3 Acumulación de reservas
				\4[] Compra de activos líquidos en divisas
				\4[] $\to$ A cambio de moneda nacional
				\4[] $\then$ Presión hacia depreciación moneda nacional
				\4[] Desde principios años 2000
				\4[] $\to$ Enorme aumento de reservas en emergentes
				\4[] ¿Por qué aumentar reservas?
				\4[] $\to$ Sostener TCN si presión a la baja
				\4[] $\to$ Mantener acceso a financiación de CC
				\4[] $\to$ Mantener gasto público en recesión
			\3 Esterilización
				\4[] Ante compraventa de divisas
				\4[] $\to$ ¿Qué sucede con balance de BCentral?
				\4[] $\to$ ¿Crece? ¿Decrece?
				\4[] Esterilización es mantener tamaño del balance
				\4[] $\to$ Aunque varíe la cantidad de reservas
				\4[] Cómo esterilizar
				\4[] $\to$ $\Delta$ de crédito doméstico igual a $\Delta$ de activos exteriores netos
				\4[] Debate sobre efectividad de interv. esterilizada
			\3 Reputación del banco central
				\4[] Percepción de agentes del mercado
				\4[] $\to$ Capacidad para intervenir con éxito
				\4[] Trayectoria de disciplina monetaria
				\4[] $\to$ Aumenta reputación
				\4[] $\then$ Aumenta poder para afectar expectativas
			\3 Coordinación de política monetaria
				\4[] Intervención coordinada con otros bancos centrales
			\3 Expansión cuantitativa
			\3 Requisitos mínimos de divisas
				\4 Herramienta también macroprudencial
				\4 Exigencia de \% mínimo en divisa
				\4[] Sobre activo total de banco
				\4[] Sobre depósitos totales
				\4 Evitar expansión excesiva de oferta monetaria
				\4 Tender a mantenimiento mínimo de reservas
			\3 \underline{Factores de éxito de la intervención}
				\4[i] Percepción fuerte o débil del mercado
				\4[] Mercados con opiniones ``fuertes'' sobre TCN de equilibrio
				\4[] $\to$ Muy difícil afectar vía intervención
				\4[] Mercados tienen opinión débil sobre evolución futura
				\4[] $\to$ Intervención sirve como ``nudge''
				\4[] $\then$ Permite fijar opinión de agentes
				\4[ii] Factor sorpresa de la intervención
				\4[] Intervención esperada puede haberse descontado ya
				\4[] Intervenciones inesperadas tienen más efecto
				\4[] $\to$ Incluso, overshooting
				\4[iii] Operaciones concertadas con otros bancos centrales
				\4[] Obviamente, BCNs interviniendo en distinta dirección
				\4[] $\to$ Tienen menor efecto
				\4[iv] Expectativas de los agentes respecto futuro
				\4[] Factor principal de éxito de intervención
				\4[] Agentes estiman intervención será exitosa
				\4[] $\to$ Ellos mismos actúan en la dirección de intervención
				\4[] $\then$ Intervención puede de hecho no ser necesaria
				\4[v] Cambios en fundamentales
				\4[] Intervención que no afecta fundamentales
				\4[] $\to$ Menor probabilidad de éxito a m/p
			\3 Valoración
				\4 ¿La intervención funciona?
				\4[] Debate académico y policy-making
				\4 Canales de actuación de la intervención
				\4[] Canal del balance
				\4[] $\to$ Variación en ofertas relativas
				\4[] $\to$ Ratio divisas/domésticos cambia aun con esterilización
				\4[] $\to$ No son sustitutivos perfectos
				\4[] $\then$ Agentes buscan equilibrar sus carteras
				\4[] $\then$ Efectos sobre TCN
				\4[] Canal de las expectativas
				\4[] $\to$ Qué TCN esperan los agentes dados anuncios
				\4[] $\to$ Señalización de intervención futura
				\4[] $\to$ Evitan actuar en contra de intervención
				\4[] $\to$ Pueden actuar a favor
				\4[] Canal de la coordinación
				\4[] $\to$ Intervención es señal para desencadenar flujos
				\4[] $\to$ Agentes privados y domésticos
				\4 Efectividad
				\4[] Generalmente, no esterilizada es efectiva
				\4[] Esterilizada puede ser efectiva también
				\4[] $\to$ Aunque resultados ambiguos
				\4[] $\to$ En el l/p, con algunos países no lo es
				\4 Efectividad de intervención esterilizada
				\4[] Debate de largo alcance
				\4[] Contrarios a efectividad
				\4[] $\to$ Mercados corrigen intervención rápidamente
				\4[] $\to$ Como mucho, efecto a corto plazo
				\4[] Favorables a efectividad
				\4[] $\to$ Mercados segmentados hacen efecto sea real
				\4[] $\to$ Efectos de c/p pueden inducir efectos l/p
		\2 Regulación
			\3 Idea clave
				\4 Mercado de divisas determina TCN
				\4 Gobierno puede ejercer coerción
				\4[] Imponer límites o prohibiciones:
				\4[] $\to$ Qué transacciones en divisas
				\4[] $\to$ A qué tasa pueden venderse divisas
				\4[] $\to$ Cuánto se puede comprar/vender
			\3 Controles de capital
				\4 Restricciones en cuenta financiera
				\4[] $\to$ Qué activos financieros pueden intercambiarse
				\4[] $\to$ Quién puede obligarse con no residentes
				\4[] $\to$ Quién puede comprar activos financieros en divisas
				\4[] $\to$ Cuánto capital puede circular
				\4[] $\to$ Impuestos a entradas de capital
				\4 Permitidos en BW y FMI
				\4 Años 80 y hasta crisis de 90s
				\4[] $\to$ Liberalización de mov. de K es necesaria
				\4 Actualidad
				\4[] $\to$ Algunas experiencias positivas con control de K
				\4[] $\to$ Aumentan costes de financiación de empresas
				\4[] $\to$ Reducen volatilidad de TC
				\4[] $\to$ Pueden reducir coste de ajuste en crisis monetarias
			\3 Control de cambio
				\4 Restricciones en mercado de divisas
				\4[] $\to$ Caso particular de control de K
				\4 Muy habituales en el pasado
				\4 Actualidad:
				\4[] $\to$ Persisten en formas muy débiles
				\4[] $\to$ Notificación a BC
				\4[] $\to$ Otros
			\3 Tipos de cambio múltiples
				\4 Distintos TC según uso
				\4[] $\to$ Importación
				\4[] $\to$ Turismo
				\4[] $\to$ ...
				\4 Objetivos habituales
				\4[] $\to$ Subvencionar sector exportador
				\4[] $\to$ Limitar salida de divisas
			\3 Valoración
				\4 Efectividad de la regulación
				\4[] Depende de:
				\4[] $\to$ Desarrollo de sistema financiero
				\4[] $\to$ Rigidez de regulación
				\4[] $\to$ Desajuste fundamental
				\4[] $\to$ Voluntad política y economía política
				\4[] Largo plazo
				\4[] $\to$ Aumenta dificultad para regular flujos
				\4[] $\to$ Aparición de nuevas formas de mov. K
				\4[] $\to$ Ejemplo: bitcoin en Venezuela
				\4 Aplicaciones efectivas
				\4[] China
				\4[] $\to$ Aislamiento de crisis asiática
				\4[] Malasia
				\4[] $\to$ Salida rápida tras crisis
				\4[] Chipre
				\4[] $\to$ Controles de corto plazo tras crisis bancaria
				\4 Aplicaciones fallidas
				\4[] Aumentan incentivos a no reformar
				\4[] Reducen disciplina fiscal
				\4[] Tentación de políticas monetarias expansivas
				\4[] $\to$ Que acaban disminuyendo efectividad de regulación
	\1[] \marcar{Conclusión}
		\2 Recapitulación
			\3 Regímenes cambiarios
			\3 Análisis comparado de regímenes cambiarios
			\3 Intervención y regulación
		\2 Idea final
			\3 Debate histórico de política económica
				\4 Continua en la actualidad
				\4 Decisión fundamental de cualquier gobierno
			\3 Idiosincrasia de la economía
				\4 Determina resultados de régimen cambiario
			\3 Supervisión macroprudencial
				\4 Cobra reciente importancia
				\4 Supervisión financiera con objetivos macro
				\4[] Evitar aparición de desequilibrios macro
				\4 Influencia de regímenes cambiarios
				\4[] Decisión potencialmente amplificadora
			\3 Crisis financieras
				\4 Tres generaciones de crisis
				\4 Cuarta generación
				\4[] Factores institucionales
				\4 Regimen cambiario es elemento central
				\4[] Especialmente, TCN Fijo
				\4[] $\to$ ¿Sigue siendo relevante?
			\3 Euro como régimen de TC Fijo
\end{esquemal}







































\begin{esquemal}
	\1[] \marcar{Introducción}
		\2 Contextualización
			\3 Macroeconomía
				\4 Análisis de fenómenos económicos a gran escala
				\4 Énfasis sobre variables agregadas
			\3 Economías abiertas
				\4 Comercio internacional
				\4[] Intercambian ByS con otras economías
				\4[] $\to$ Precios relativos son importantes
				\4[] $\then$ Tipo de cambio es importante
				\4[] $\then$ DAgregada depende de exterior
				\4 Flujos financieros internacionales
				\4[] Intercambio de activos y pasivos
				\4[] Suavización intertemporal de rentas
				\4[] Dinámicas de deuda exterior
				\4 Tipo de cambio
				\4[] Precio más importante en una economía abierta
				\4[] $\to$ Relación entre bienes locales y extranjeros
				\4 Interacción de sector exterior y ec. doméstica
				\4[] Demanda exterior sobre demanda agregada y output
				\4[] Diferenciales de precios
				\4[] Condiciones de financiación
			\3 Régimen cambiario
				\4 Estados/aut. monetarias pueden afectar TCN
				\4[] Interviniendo el mercado de divisas
				\4[] Regulando intercambios de divisas
				\4 Concepto de régimen cambiario
				\4[] Conjunto de intervención+regulación
				\4[] Con el objetivo de:
				\4[] $\to$ Determinar TCN determinado
				\4[] $\to$ Alcanzar otros objetivos de PEconómica
				\4[] $\then$ Aparición de trade-offs
				\4[] $\then$ Necesario decidir qué régimen cambiario
				\4 Dimensión fundamental de un régimen cambiario
				\4[] TCN en nivel fijo o flexible
				\4[] $\to$ Qué TCN fijo
				\4[] $\to$ Qué grado de fluctuación
				\4[] $\to$ Qué actuaciones para lograrlo
		\2 Objeto
			\3 ¿Qué regímenes cambiarios existen?
			\3 ¿Qué modelos teóricos permiten comparar regímenes cambiarios?
			\3 ¿Qué ventajas e inconvenientes tienen?
			\3 ¿En qué circunstancias es adecuado un régimen cambiario fijo o flexible?
			\3 ¿Qué evidencia empírica existe al respecto?
			\3 ¿Cómo intervienen las autoridades monetarias los mercados cambiarios?
			\3 ¿Para qué sirven las intervenciones?
			\3 ¿Cómo regulan las autoridades monetarias los mercados de cambio?
		\2 Estructura
			\3 Análisis comparado de regímenes cambiarios
			\3 Intervención y regulación
	\1 \marcar{Análisis comparado de regímenes cambiarios}
		\2 Idea clave
			\3 Concepto de análisis comparado
				\4 Características propias a regímenes cambiarios
				\4 Entender cómo evoluciona economía
				\4[] Dado un régimen cambiario
				\4[] Dada una coyuntura determinada
				\4 Comparar resultados
				\4[] Enfoque positivo
				\4[] Enfoque normativo
				\4 Contrafactuales
				\4[] Régimen cambiario es decisión compleja
				\4[] $\to$ Efectos sobre conjunto de la economía
				\4[] $\then$ Imposible experimentar
				\4[] Necesario construir contrafactuales
				\4[] $\to$ ¿Cómo se comportaría economía con otro régimen?
			\3 Historia del análisis
				\4 Nurkse (1944)
				\4[] TCN flexible no es deseable
				\4[] Inestabilidad de mercados financieros en años 20
				\4[] $\to$ Devaluaciones competitivas
				\4[] $\to$ Incertidumbre sobre comercio internacional
				\4[] $\to$ Tensiones políticas
				\4[] $\then$ Especulación es negativa
				\4[] $\then$ Mejor tipos fijos
				\4 Meade (1951)
				\4[] Rigideces salariales y de precios
				\4[] $\to$ Dificultan ajuste ante shock real externo
				\4[] $\to$ Tipo de cambio fijo impide ajuste
				\4[] $\then$ Mejor tipos ``variables''
				\4 Friedman (1953)
				\4[] Reglas monetarias+TCN flexible
				\4[] $\to$ Régimen cambiario deseable
				\4[] Contrario a PM discrecional+TCN fijo ajustable
				\4[] $\to$ Dificultan ajuste
				\4[] $\to$ Incentivos perversos de autoridad monetaria
				\4[] Hipótesis de Friedman
				\4[] $\to$ TCN Flex. inducen mov. especulativos 
				\4[] $\then$ Incentivos a ajustar paridad real
				\4[] $\then$ Especulación es negativa
				\4 Mundell (1961), (1963)
				\4[] Enfoque de Áreas Monetarias Óptimas
				\4[] Marco teórico para valorar políticas
				\4[] Régimen cambiario modula efectos de pol. econ.
				\4[] Movilidad del trabajo es factor determinante
				\4[] $\to$ Para que tipo fijo sea óptimo o no
				\4 Caída de Bretton Woods
				\4[] Divisas países desarrollados comienzan a fluctuar
				\4[] Flujos de capital crecen enormemente
				\4[] Volatilidad de TCN aumenta explosivamente
				\4[] $\to$ Aparición de mercados forward
				\4[] $\to$ EMS en CEE
				\4 Años 80
				\4[] Inestabilidad cambiaria
				\4[] Crisis de deuda en PEDs
				\4 Crisis de los 90
				\4[] Segunda y tercera generación de crisis
				\4[] Regímenes cambiarios son aspecto clave
				\4[] Emergentes y asiáticos acaparan reservas
				\4[] $\to$ A finales de los 90
				\4[] Recomendación bipolar
				\4[] $\to$ O totalmente fijo o totalmente flexible
				\4[] $\then$ Intermedios son inestables y fracasan
				\4 Actualidad
				\4[] Managed float y fijo ajustable son más comunes
				\4[] \$ y € son monedas de fijación más habituales
				\4[] Resultados y recomendaciones muy heterogéneos
				\4[] $\to$ Buenos y malos resultados con todos regímenes
				\4[] $\to$ Escuelas recomiendan uno y otro régimen
				\4[] Miedo a flotar
				\4[] $\to$ Especialmente relevante en emergentes
				\4[] $\to$ De iure, muchos regímenes flotantes
				\4[] $\to$ Realmente, fuerte intervención
				\4[] $\to$ Opacidad sobre régimen real
				\4[] $\to$ Cestas de fijación a veces ocultas
				\4[] $\to$ De iure y de facto diferentes
			\3 Enfoques de análisis
				\4 Factibilidad del régimen con TCN fijo
				\4 Ventajas e inconvenientes de TC fijo vs flexible
				\4 Protección frente a shocks
				\4 Análisis empírico
		\2 Regímenes cambiarios
			\3 Concepto
				\4 Regímenes polares
				\4[] Fijo y flexible son simplificaciones
				\4 Realidad de regímenes cambiarios
				\4[] Combinación de:
				\4[] $\to$ Instituciones legales y políticas
				\4[] $\to$ Compromiso de tipo de cambio
				\4[] $\then$ Múltiples variantes
				\4 Extremos de fijación/flexibilidad
				\4[] De más a menos fuerte compromiso de TC fijo
			\3 Unión monetaria
				\4 Comparte con otros miembros de UM
				\4[] Política monetaria 
				\4[] Autoridad monetaria
				\4[] Moneda física y unidad de cuenta
				\4[] Soberanía
				\4[] $\to$ Requiere cesiones a instituciones comunes
				\4[] $\to$ Generalmente también en otras áreas
				\4 Beneficios de señoreaje
				\4[] Repartidos entre miembros de UM 
				\4 Régimen cambiario con terceros
				\4[] Puede ser cualquier otro
			\3 Dolarización
				\4 País renuncia a:
				\4[] Moneda física propia y unidad de cuenta
				\4[] $\to$ Adoptando país de tercero
				\4[] Política monetaria
				\4 Mantiene:
				\4[] Autoridad monetaria con funciones mínimas
				\4[] $\to$ Estadísticas
				\4[] $\to$ Cuentas del gobierno 
				\4[] $\to$ Sistema de pagos nacionales
				\4[] $\to$ Gestión de reservas
				\4 Señoreaje
				\4[] País que emite moneda extrae todo el beneficio
			\3 Junta de conversión
				\4 País renuncia legal/constitucionalmente a:
				\4[] Política monetaria
				\4[] $\then$ Moneda nacional es token sobre divisa
				\4 Mantiene:
				\4[] Autoridad monetaria con funciones reducidas
				\4[] $\to$ Estatuto legal limitativo
				\4[] $\to$ Vender/comprar moneda nacional 
				\4[] $\to$ Sin discrecionalidad alguna (en teoría)
				\4[] $\to$ Mantiene funciones propias de BC en dolarización
				\4 Señoreaje
				\4[] En teoría:
				\4[] $\to$ Sólo por ingresos de reservas de divisas\footnote{Las reservas que respaldan completamente la oferta monetaria de moneda local generan un interés que constituye un ingreso por señoreaje para el país en cuestión. Si los tipos de interés de las reservas son negativos, la autoridad monetaria puede incurrir en pérdidas y obtener de hecho un ``señoreaje'' negativo.}
				\4[] En la práctica:
				\4[] $\to$ AMonetaria puede ``hacer trampas''
			\3 Adjustable peg/fijo ajustable
				\4 País renuncia a:
				\4[] Política monetaria
				\4[] $\to$ Se compromete a defender tipo de cambio
				\4 Mantiene:
				\4[] Autoridad monetaria con PM
				\4[] $\to$ Sin estatuto legal limitativo de funciones
				\4[] $\to$ Posible cambio de pol. sin reforma const./legal
				\4[] $\to$ Posibles ajustes de tipo fijo cada cierto tiempo
				\4[] $\then$ Generalmente, fluctuación $\pm$ 1\%, más de 6 meses
				\4 Variantes
				\4[] Fijo en relación a punto determinado
				\4[] Bandas de fluctuación permitidas
				\4[] $\to$ Más ``fijo'' cuanto menores sean
				\4 Más habitual junto con managed float
			\3 Crawling peg
				\4 País se compromete a ajuste gradual
				\4[] Dado TCN objetivo declarado
				\4[] $\to$ Variaciones graduales hasta objetivo
				\4[] $\to$ Frecuentes en el tiempo
				\4 Política monetaria supeditada a régimen
				\4[] Pero ajustes frecuentes son aceptables
			\3 Bandas de flotación
				\4 País se compromete a mantener dentro de banda
				\4 Bandas amplias de fluctuación
				\4[] Sólo interviene si se acerca a bandas
				\4 Krugman (1991)
				\4[] Credibilidad de intervención es importante
				\4[] Si capacidad de BCentral es creíble
				\4[] $\to$ Mercados actúan solos para mantener en bandas
				\4[] $\to$ Intervención no es necesaria
				\4[] $\to$ TCN cae cuando se acerca a límite superior
				\4[] $\to$ TCN aumenta cuando se acerca a límite inferior
				\4[] Si capacidad de BCentral no es creíble
				\4[] $\then$ Situación similar a Krugman (1979)
				\4[$\then$] Expectativas juegan papel clave
			\3 Managed float o dirty float
				\4 TC fluctúa libremente pero BC interviene
				\4 Intervenciones relativamente frecuentes
				\4 Cierta opacidad respecto a criterios de intervención
				\4 Más habitual junto con fijo ajustable
			\3 Flotación libre
				\4 Intervención excepcional
				\4 Sólo para evitar condiciones de extrema volatilidad
		\2 Factibilidad de regímenes con TCN fijo
			\3 Concepto
				\4 ¿Cuándo es posible adoptar TCN fijo?
				\4[] No siempre posible
				\4[] Régimen con TC fijo sujeto a restricciones
				\4[] Resultado de modelos teóricos+soporte empírico
			\3 Trinidad imposible
				\4 Resultado teórico
				\4[] Derivado de Mundell-Fleming
				\4[] Con importante soporte empírico
				\4 TC Fijo implica necesariamente una de las dos:
				\4[] $\to$ Controles de capital
				\4[] $\to$ PM monetaria supeditada a mantener TCN fijo
				\4[$\then$] Trinidad imposible/Trilema de política económica
				\4[] Sólo dos de tres características son posibles:
				\4[] \textsc{I} Tipo de cambio fijo
				\4[] \textsc{II} Libertad de movimiento de capital
				\4[] \textsc{III} Política monetaria independiente
				\4 Justificación teórica
				\4[] Supongamos I y II
				\4[] $\to$ Régimen de TC Fijo estable
				\4[] Añadimos III
				\4[] $\to$ PM expansiva utilizada para equilibrio interno
				\4[] Aparece divergencia de interés doméstico y mundial
				\4[] $\to$ Salida de capital y EDemanda de divisas
				\4[] $\then$ TCN Fijo imposible
			\3 Dilema, no trilema
				\4 Propuesto por Rey (2015)
				\4[] Régimen cambiario es irrelevante
				\4[] $\to$ TCN no aísla de shock exteriores
				\4 Autoridades monetarias deben elegir entre:
				\4[] \textsc{I} Movilidad de capital
				\4[] \textsc{II} Política monetaria independiente
				\4 Existe un ciclo financiero global
				\4[] $\to$ Determinado por Fed y principales bancos centrales
				\4 Con movilidad de capitales
				\4[] Tipos de interés en USA, UE afectan resto del mundo
				\4[] $\to$ PM debe adaptarse a ciclo financiero global
				\4[] $\then$ Única solución es control de cuenta financiera
			\3 Nuevo trilema del sector financiero
				\4 Planteado por Pisany-Ferry (2012)
				\4 Asumiendo:
				\4[] $\to$ régimen de tipo fijo factible
				\4[] $\to$ Integración monetaria avanzada
				\4[] Sólo son posibles dos:
				\4[] \textsc{I} Interdependencia banca-sector público
				\4[] \textsc{II} Financiación monetaria del déficit prohibida
				\4[] \textsc{III} Sin corresponsabilidad respecto a deuda pública
				\4 Con I y II
				\4[] Necesaria unión fiscal
				\4 Con I y III
				\4[] Necesario prestamista de último recurso a nivel de AMonetaria
				\4 Con II y III
				\4[] Necesaria unión financiera efectiva
		\2 Aislamiento frente a shocks exteriores\footnote{Ver Towbin y Weber (2011), \href{https://www.riksbank.se/globalassets/media/rapporter/pov/artiklar/engelska/2018/180326/20181-the-case-for-flexible-exchange-rates-after-the-great-recession.pdf}{Corsetti, Kuester y Müller (2018)} y Gandolfo (2015) págs. 172-174. }
			\3 Idea clave
				\4 Contexto
				\4[] Debate de largo plazo
				\4[] Demanda agregada sujeta a shocks
				\4[] $\to$ Demanda de exportaciones
				\4[] $\to$ Precios relativos frente a exterior
				\4[] Generalmente, preferible reducir varianza de output
				\4[] Régimen cambiario es factor en exportaciones netas
				\4[] $\to$ Determina relación real de intercambio
				\4[] $\to$ Determina saldo de cuenta corriente
				\4[] $\to$ Restringe rango de política monetaria viable
				\4[] Política monetaria
				\4[] $\to$ Instrumento para alcanzar equilibrio interno
				\4[] $\to$ Con TCFijo y mov. de K, se convierte en endógena
				\4 Objetivos
				\4[] Caracterizar efectos de reg. cambiarios sobre varianza de Y
				\4 Resultados
				\4[] En M-F básico, TCFlexible aísla de shocks exteriores
				\4[] $\to$ Mantiene precios relativos constantes
				\4[] $\to$ 
				\4[] Incluyendo:
				\4[] $\to$ Efectos balance
				\4[] $\to$ Ausencia de pass-through sobre precios de M
				\4[] $\then$ Posible que desaparezca el efecto aislante
			\3 Mundell-Fleming
				\4 Mundell-Fleming
				
				\4 NOEM
				\4[] Posible incluir otros supuestos relevantes
				\4[] 
			\3 Implicaciones
				\4 Shocks de demanda de bienes
				\4[] Tipo de cambio fijo
				\4[] $\to$ IS se desplaza la derecha
				\4[] $\to$ Tipo de interés por encima de mundial
				\4[] $\to$ Entrada de capital
				\4[] $\to$ Exceso de demanda de moneda local
				\4[] $\to$ Aumento de M para equilibrar interés con mundial
				\4[] $\then$ Nuevo equilibrio con $= r$, más $Y$
				\4[] $\then$ Amplificación del shock en términos de output
				\4[] Tipo de cambio flexible
				\4[] $\to$ IS se desplaza la derecha
				\4[] $\to$ Tipo de interés por encima de mundial
				\4[] $\to$ Entrada de capital
				\4[] $\to$ Exceso de demanda de moneda local
				\4[] $\to$ Apreciación de moneda local
				\4[] $\to$ Caída de demanda de exportaciones
				\4[] $\to$ Aumento de demanda de importaciones
				\4[] $\to$ IS vuelve a la izquierda hasta igualar $r$
				\4[] $\then$ Nuevo equilibrio con $=r$, $=Y$, TCN depreciado
				\4[] $\then$ Estabilización del shock en términos de output
				\4 Shocks de demanda de dinero
				\4[] Tipo de cambio fijo
				\4[] $\to$ LM se desplaza a la izquierda
				\4[] $\to$ Tipo de interés por encima de mundial
				\4[] $\to$ Entrada de capital
				\4[] $\to$ Exceso de demanda de moneda local
				\4[] $\to$ Aumento de M para equilibrar $r$ con mundial
				\4[] $\to$ LM vuelve hacia la derecha
				\4[] $\then$ Equilibrio con $=r$, $=Y$, más $M$
				\4[] $\then$ Estabilización del shock en términos de output
				\4[] Tipo de cambio flexible
				\4[] $\to$ LM se desplaza a la izquierda
				\4[] $\to$ Tipo de interés por encima de mundial
				\4[] $\to$ Entrada de capital
				\4[] $\to$ Exceso de demanda de moneda local
				\4[] $\to$ Apreciación de moneda local
				\4[] $\to$ Caen exportaciones netas
				\4[] $\then$ Equilibrio con $=$, menos $Y$, TCN apreciado
				\4[] $\then$ Amplificación del shock en términos de output
				\4 Resumen de efectos estabilizadores/amplificadores
				\4[] \grafica{estabilizacionamplificacion}
			\3 Valoración
		\2 Ventajas e inconvenientes
			\3 Tipo fijo: ventajas 
				\4 Amortiguar shocks nominales internos
				\4[] PM endógena a mantenimiento de TCN
				\4[] $\to$ Reduce efecto sobre 
				\4 Facilitar comercio e inversión
				\4[] Reduce incertidumbre sobre precio real
				\4[] Aumenta utilidad del dinero
				\4[] Especialmente relevante si TCN fijo
				\4[] $\to$ Con principal socio comercial
				\4 Evitar devaluaciones competitivas
				\4[] Devaluaciones competitivas como equilibrio de Nash subóptimo
				\4[] Sistema de tipos fijos
				\4[] $\to$ Herramienta de coordinación
				\4[] $\to$ Equilibrio second-best para evitar devaluación
				\4[] $\then$ Bretton Woods como respuesta a años 30
				\4 Evitar burbujas especulativas
				\4[] Burbujas especulativas entendidas como:
				\4[] $\to$ Desviación respecto de fundamentales
				\4[] $\to$ Expectativas auto-cumplidas
				\4[] $\then$ TCN fijo elimina eqs. no soportados por fundamentales
				\4 Referencia nominal para política monetaria
				\4[] Referencia clara y entendible
				\4[] Relativamente fácil de mantener
				\4[] Permite aprovechar prestigio de otro BCentral
				\4 Disciplina de políticas macroeconómicas
				\4[] Gobierno no puede estimular DA vía PM
				\4[] $\to$ Mayor presión para reformas por lado de oferta
				\4 Coordinación internacional de políticas
				\4[] Devaluación competitiva es eq. de Nash no coop.
				\4[] TCN fijo es señal para coordinación de políticas
			\3 Tipo fijo: inconvenientes
				\4 Mercados forward y opciones no se desarrollan
				\4[] Si agentes estiman TCFijo es viable
				\4[] $\to$ Mercados no se desarrollan
				\4[] Si agentes estiman TFijo no es viable
				\4[] $\to$ Coste prohibitivo de asegurar
				\4 Distorsiones de RRI persisten durante más tiempo
				\4[] Diferencial de inflación doméstico--extranjero
				\4[] Tipo de cambio se mantiene fijo
				\4[] $\then$ RRI empeora y TC no compensa
				\4[] Ejemplo: inflación nacional relativamente alta
				\4[] Tipo de cambio fijo
				\4[] $\to$ Todo diferencial de inflación se traslada a RRI
				\4 Depreciaciones bruscas del tipo de cambio 
				\4[] Cuando:
				\4[] $\to$ Autoridad monetaria no puede defender
				\4[] $\to$ Coste de defender es prohibitivo
				\4[] $\then$ Depreciación brusca
				\4 Necesario mantener reservas
				\4[] Costes de oportunidad y gestión
			\3 Tipo flexible: ventajas
				\4 Política monetaria independiente
				\4[] Posible utilizar para alcanzar equilibrio interno
				\4[] No supeditada a objetivo de equilibrio externo
				\4[] $\to$ Evita prociclicidad en crisis
				\4 Beneficios por señoreaje
				\4[] Moneda propia emitida por BC propio
				\4[] Ingresos de señoreaje a estado emisor
				\4[] Grados variables de pérdida de señoreaje
				\4[] $\to$ Parcial con compromiso de tipo fijo
				\4[] $\to$ Total con dolarización, junta de conversión, UMonetaria
				\4 Prestamista de último recurso
				\4[] Autoridad monetaria siempre puede prestar en moneda nacional
				\4[] Mantener estabilidad de sist. financiero en crisis
				\4 Evitar ataques especulativos
				\4[] Crisis monet. de 1a y 2a generación posibles con TFijo
				\4[] $\to$ En regímenes menos fijos que dolarización 
				\4[] $\to$ BCentral no dispuesto a pagar precio de defender moneda
				\4[] $\then$ Especulación puede provocar ajustes bruscos
				\4[] $\then$ Tipo flexible evita crisis de 1a y 2a gen.
				\4 Menores necesidades de reservas
				\4[] No son necesarias para cubrir EDemanda de divisas
				\4[] Se evita costes de:
				\4[] $\to$ Oportunidad
				\4[] $\to$ Gestión
			\3 Tipo flexible: inconvenientes
				\4 Dinámicas inestables de depreciación-inflación
				\4[] Pass-through no nulo induce
				\4[] $\to$ Inflación tras depreciación
				\4[] Si no se cumple M-L, depreciación induce:
				\4[] $\to$ Aumento de déficit comercial
				\4[] $\then$ Combinación de ambas pueden desestabilizar precios
				\4[] $\then$ Depreciación+inflación
				\4 Efectos adversos de burbujas especulativas
				\4[] Desvíen TC de fundamentales
				\4[] $\to$ Distorsión sobre comercio e inversión
				\4[] $\to$ Flujos de capital distorsionantes
				\4 Costes de reasignación entre sectores
				\4[] Variación de TC afecta competitividad relativa de sectores
				\4[] $\to$ Comerciable
				\4[] $\to$ No comerciable
				\4[] Variación de competitividad incentiva flujo de ff.pp. entre sectores
				\4[] $\then$ Reasignación de producción y costes 
				\4 Pérdida de disciplina de política monetaria
				\4[] Gobierno puede querer utilizar PM para afectar DA
				\4[] Menos presión para reformar
				\4[] Interacción con factores de economía política
				\4[] $\then$ Posible sesgo inflacionario

		\2 Evidencia empírica
			\3 Crecimiento
				\4 Aghion et al (2000)
				\4[] TCN fijo
				\4[] $\to$ Implica mayor volatilidad del TC real
				\4[] $\then$ Output más volátil que con TCFlexible
				\4[] $\to$ Efecto poco significativo sobre crecimiento
				\4[] $\to$ Más inversión pero menos comercio
				\4[] $\to$ Desarrollo financiero es factor muy relevante
				\4 Levy-Yeyati y Sturzenegger (2003)
				\4[] Países en desarrollo
				\4[] $\to$ TCFlexible impacto positivo sobre crecimiento
				\4[] $\to$ TCFijo impacto negativo
				\4[] Países desarrollados
				\4[] $\to$ Sin apenas efecto
			\3 Ajuste de balanza de pagos
				\4 Ghosh (2010): asimetrías en ajuste
				\4[] Con TCFlexible
				\4[] $\to$ Superávits se ajustan más rápido
				\4[] $\to$ Déficits se ajustan igual que TFijo
				\4 Chinn y Wei (2013): velocidad de ajuste
				\4[] 170 países
				\4[] 1971-2005
				\4[] Reversión a eq. de l/p
				\4[] $\to$ No es más rápido con tipo flexible
				\4[] $\to$ Sin efecto del régimen cambiario
				\4 Pancaro (2013): reversión de saldo de CC
				\4[] Sin efecto del régimen cambiario
				\4[] Output gap y déficits de CC
				\4[] $\to$ Sí afecta probabilidad de reversión
				\4 Lane y Milesi-Ferretti (2012); ajuste en Gran Recesión
				\4[] Ajuste exterior principalmente a través de:
				\4[] $\to$ Contracción de output
				\4[] $\to$ Contracción del gasto
				\4[] Tipo fijo
				\4[] $\to$ TCReal evoluciona de forma desestabilizante
				\4[] Tipo flexible
				\4[] $\to$ TCReal evoluciona paralelo a saldo de CC
			\3 Comercio
				\4 Rose (2000): volatilidad TC y CInternacional
				\4[] TCFijo es muy favorable para comercio
				\4[] $\to$ Pero hay una discontinuidad fuerte con UM
				\4[] Unión monetaria
				\4[] $\to$ Efecto significativo muy alto
				\4[] $\then$ ``Efecto Rose''
				\4[] Críticas basadas en varios argumentos:
				\4[] $\to$ Sección cruzada no es adecuada
				\4[] $\then$ Pero otros papers series temporales replican
				\4[] $\to$ Hay variables omitidas
				\4[] $\then$ Pero inclusión de otras mantiene generalmente
				\4[] $\to$ Endogeneidad del régimen cambiario
				\4[] $\then$ Pero cuasiexperimento de CFA desde Franco a EUR confirma\footnote{Tras la introducción del Euro, el franco CFA pasó a ser un régimen de tipo fijo en relación al euro y no al franco francés. Con el cambio, el comercio con todos los países de la zona euro aumentó fuertemente.}
				\4 Bacchetta y van Wincoop (2000) y otros
				\4[] Poco efecto c/p volatilidad de TC
				\4[] Posible causalidad inversa
				\4[] $\to$ Comercio afecta volatilidad de TC
			\3 Inflación
				\4 Ghosh et al (1997)
				\4 TCN Fijo
				\4[] Generalmente, menor inflación y menos volátil
				\4[] Menores crecimientos de M y velocidad del dinero
				\4[] $\to$ Efecto disciplina
				\4[] Excepciones:
				\4[] $\to$ Países que cambian paridad frecuentemente
				\4[] $\to$ Países con baja inflación menos afectados
			\3 Autonomía de la política monetaria
				\4 En la práctica, poca autonomía
				\4 Antes y después de BW
				\4[] PM en USA muy importante
				\4[] $\to$ Países siguen +/- la PM de Fed y BCE
			\3 Crisis financieras
				\4 TC Fijo en emergentes
				\4[] Mayor posibilidad de crisis
				\4[] Problema de clasificación
				\4[] $\to$ ¿Qué régimen está implementándose realmente?
			\3 Pass-through del tipo de cambio
				\4 Depende sobre todo de inflación
				\4[] Baja inflación en país importador
				\4[] $\to$ Poco pass-through de las importaciones
				\4 En menor medida, depende también de volatilidad del TC
				\4[] $\to$ Más volatilidad, más pass-through de las importaciones
				\4 Países OCDE
				\4[] Micro más importante que macro
				\4[] Régimen cambiario no es muy importante
			\3 Instituciones
				\4 Calvo y Mishkin (2004)
				\4[] Instituciones causan régimen cambiario
				\4[] Énfasis debe ponerse en contexto institucional
				\4[] $\to$ Régimen cambiario es factor de segundo orden
		\2 Factores que determinan régimen óptimo\footnote{Basado en apartado 28.5, Handbook of Fixed Exchange Rates, capítulo por Frankel, J.}
			\3 Idea clave
				\4 No existe régimen perfecto en general
				\4 Contexto determina régimen óptimo
				\4[] Coyuntural
				\4[] Idiosincrático de la economía en cuestión
			\3[\textsc{i}] Tamaño y apertura
				\4 Países pequeños y abiertos
				\4[] Más beneficios del comercio
				\4[] $\to$ Mayores ventajas del TCFijo
				\4 Países grandes y cerrados
				\4[] Menos beneficios del comercio
				\4[] Más margen de actuación de PM
				\4[] $\to$ Mayores ventajas de TCFlexible
			\3[\textsc{ii}] Socio comercial
				\4 Existencia de un socio muy grande
				\4[] Con el que lleva a cabo elevado \%:
				\4[] $\to$ Comercio
				\4[] $\to$ Inversión
				\4[] $\to$ Perspectivas de relaciones futuras
				\4[] $\then$ TCFijo ofrece más ventajas
				\4 Fijación respecto a cesta
				\4[] Cuando hay varios socios dominantes
				\4 Países con economías muy diversificadas
				\4[] TCFlexible puede ser más estable que fijo
				\4[] $\to$ Sin inconvenientes de tipo fijo
			\3[\textsc{iii}] Simetría de los shocks
				\4 Correlación alta
				\4[] Shocks que afectan a país emisor de moneda ancla
				\4[] $\to$ Muy correlacionados con shocks domésticos
				\4[] $\to$ Movimiento de interés tenderá a ser similar
				\4[] $\then$ TCFijo es ventajoso
			\3[\textsc{iv}] Movilidad del trabajo
				\4 Mayor movilidad del trabajo
				\4[] Más flexibilidad ante shocks asimétricos
				\4[] $\to$ TCFijo menos vulnerable ante shocks
				\4[] $\then$ Menos razones para TCFlexible
			\3[\textsc{v}] Transferencias fiscales contracíclicas
				\4 Transferencias fiscales
				\4[] Especialmente relevante en UM
				\4[] También receptores de transferencias de K
				\4[] Mayores transferencias:
				\4[] $\to$ Más resistencia a shocks
				\4[] $\then$ TCFlexible menos ventajoso que sin trans.
			\3[\textsc{vi}] Transferencias corrientes contracíclicas
				\4 Especialmente, remesas
				\4 Países con poblaciones emigrantes en desarrollados
				\4[] Remesas contracíclicas a PEDs
				\4[] Responden a diferencial de posiciones cíclicas
				\4[] $\to$ Suavizan shocks domésticos en PEDs
				\4[] $\then$ TCFlexible menos ventajoso que sin trans.
			\3[\textsc{vii}] Voluntad política
				\4 Importancia de soberanía monetaria
				\4[] Sujeto de debate político
				\4[] Soberanía nacional asociada a moneda propia 
				\4[] $\to$ Puede ser difícil abandonar moneda propia/TCFijo
			\3[\textsc{viii}] Desarrollo financiero
				\4 Países con sist. financieros subdesarrollados
				\4[] Menos beneficios de TCFlexible
				\4[] Mercados financieros poco profundos
				\4[] $\to$ Poco margen para acomodar shocks
				\4[] $\then$ Aumentan costes de shocks
				\4[] $\then$ Preferible TCFijo
			\3[\textsc{ix}] Origen de los shocks
				\4 Shocks de demanda interna
				\4[] $\then$ TCFijo preferible
				\4 Shocks de demanda de dinero
				\4[] TCFijo para amortiguar efecto de $\Delta$ de interés
				\4 Shocks de oferta o externos
				\4[] TCN puede ajustarse automáticamente
				\4[] $\to$ Reduce amplitud de variaciones de output
				\4[] $\then$ TCFlexible preferible
				\4[] $\then$ Incluidos shocks desastres naturales
	\1[] \marcar{Introducción}
		\2 Contextualización
			\3 Macroeconomía
				\4 Análisis de fenómenos económicos a gran escala
				\4 Énfasis sobre variables agregadas
			\3 Economías abiertas
				\4 Comercio internacional
				\4[] Intercambian ByS con otras economías
				\4[] $\to$ Precios relativos son importantes
				\4[] $\then$ Tipo de cambio es importante
				\4[] $\then$ DAgregada depende de exterior
				\4 Flujos financieros internacionales
				\4[] Intercambio de activos y pasivos
				\4[] Suavización intertemporal de rentas
				\4[] Dinámicas de deuda exterior
				\4 Tipo de cambio
				\4[] Precio más importante en una economía abierta
				\4[] $\to$ Relación entre bienes locales y extranjeros
				\4 Interacción de sector exterior y ec. doméstica
				\4[] Demanda exterior sobre demanda agregada y output
				\4[] Diferenciales de precios
				\4[] Condiciones de financiación
			\3 Régimen cambiario
				\4 Estados/aut. monetarias pueden afectar TCN
				\4[] Interviniendo el mercado de divisas
				\4[] Regulando intercambios de divisas
				\4 Concepto de régimen cambiario
				\4[] Conjunto de intervención+regulación
				\4[] Con el objetivo de:
				\4[] $\to$ Determinar TCN determinado
				\4[] $\to$ Alcanzar otros objetivos de PEconómica
				\4[] $\then$ Aparición de trade-offs
				\4[] $\then$ Necesario decidir qué régimen cambiario
				\4 Dimensión fundamental de un régimen cambiario
				\4[] TCN en nivel fijo o flexible
				\4[] $\to$ Qué TCN fijo
				\4[] $\to$ Qué grado de fluctuación
				\4[] $\to$ Qué actuaciones para lograrlo
		\2 Objeto
			\3 ¿Qué regímenes cambiarios existen?
			\3 ¿Qué modelos teóricos permiten comparar regímenes cambiarios?
			\3 ¿Qué ventajas e inconvenientes tienen?
			\3 ¿En qué circunstancias es adecuado un régimen cambiario fijo o flexible?
			\3 ¿Qué evidencia empírica existe al respecto?
			\3 ¿Cómo intervienen las autoridades monetarias los mercados cambiarios?
			\3 ¿Para qué sirven las intervenciones?
			\3 ¿Cómo regulan las autoridades monetarias los mercados de cambio?
		\2 Estructura
			\3 Análisis comparado de regímenes cambiarios
			\3 Intervención y regulación
\end{esquemal}

\graficas



\begin{axis}{4}{Modelo de Dornbusch (1976): efecto de un estímulo de política monetaria inesperado.}{p}{e}{dornbuschpm}
	% Línea de precio constante
	\draw[-] (0.5,0.5) -- (4,4);
	\node[above] at (4,4){\tiny $\dot{p}_0=0$};
	
	% Línea de tipo de cambio constante
	\draw[-] (0.5,4) -- (4,0.5);
	\node[above] at (0.5,4){\tiny $\dot{e}=0$};
	
	% Senda estable de punto de silla
	%\draw[-] (0.5,3) -- (4,1.5);
	% Hacia abajo y derecha
	\draw[-{Latex}] (0.5,3) -- (1,2.79);
	\draw[-{Latex}] (1,2.79) -- (1.5,2.57);
	\draw[-{Latex}] (1.5,2.57) -- (2,2.36);
	\draw[-{Latex}] (2,2.36)-- (2.25,2.25);
	% Hacia arriba e izquierda
	\draw[-{Latex}] (4,1.5) -- (3.5,1.71);
	\draw[-{Latex}] (3.5,1.71) -- (3,1.93);
	\draw[-{Latex}] (3,1.93) -- (2.5,2.143);
	\draw[-{Latex}] (2.5,2.143) -- (2.25,2.25);
	
	% NORTE: Precio creciente y tipo de cambio que se deprecia 
	\draw[-{Latex}] (2.4,3.5) -- (2.4,4);
	\draw[-{Latex}] (2.4,3.5) -- (2.9,3.5);
	
	% OESTE: Precio creciente y tipo de cambio que se aprecia
	\draw[-{Latex}] (0.5,2) -- (0.5,1.5);
	\draw[-{Latex}] (0.5,2) -- (1,2);
	
	% SUR: Precio decreciente y tipo de cambio que se aprecia
	\draw[-{Latex}] (2.5,1) -- (2,1);
	\draw[-{Latex}] (2.5,1) -- (2.5,0.5);
	
	% ESTE
	\draw[-{Latex}] (4.5,2.5) -- (4.5,3);
	\draw[-{Latex}] (4.5,2.5) -- (4,2.5);
	
	% Tipo de cambio de equilibrio
	\draw[dotted] (2.25,2.25) -- (0,2.25);
	\node[left] at (0,2.25){\tiny $\bar{e}_0$};
	
	% Precio de equilibrio
	\draw[dotted] (2.25,2.25) -- (2.25,0);
	\node[below] at (2.25,0){\tiny $\bar{p}_0$};
	
	% Línea de tipo constante tras estímulo de PM
	\draw[dashed] (1.5,4) -- (5,0.55);
	\node[above] at (1.5,4){\tiny $\dot{e}_1=0$};
	
	% Precio de equilibrio constante en el muy corto plazo tras estímulo
	\draw[dotted] (2.25,2.25) -- (2.25,3);
	
	% Tipo de cambio de overshooting
	\draw[dotted] (2.25,3) -- (0,3);
	\node[left] at (0.05,3.1){\tiny $e^*$};
	
	% Tipo de cambio de equilibrio tras estímulo
	\draw[dotted] (2.78,2.78) -- (0,2.78);
	\node[left] at (0,2.75){\tiny $\bar{e}_1$};
	
	% Precio de equilibrio tras estímulo
	\draw[dotted] (2.78,2.78) -- (2.78,0);
	\node[below] at (2.78,0){\tiny $\bar{p}_1$};
	
	% Nueva senda estable de punto de silla
	% y = (111/28) - (3/7)x
	%\draw[-] (0.5,3.75) -- (4,2.25); RESOLVER ECUACIÓN
	\draw[dashed,-{Latex}] (0.5,3.75) -- (1,3.54);
	\draw[dashed,-{Latex}] (1,3.54) -- (1.5,3.32);
	\draw[dashed,-{Latex}] (1.5,3.32) -- (2,3.11);
	\draw[dashed,-{Latex}] (2,3.11) -- (2.5,2.89);
	\draw[dashed,-{Latex}] (2.5,2.89) -- (2.78,2.78);
	
	\draw[dashed,-{Latex}] (4,2.25) -- (3.5,2.46);
	\draw[dashed,-{Latex}] (3.5,2.46) -- (3,2.68);
	\draw[dashed,-{Latex}] (3,2.68) -- (2.5,2.89);
	\draw[dashed,-{Latex}] (2.5,2.89) -- (2.78,2.78);
	
	%	\draw[dashed,-{Latex}] (4,2.25) -- (3.5,2.46);
	%	\draw[dashed,-{Latex}] (3.5,2.46) -- (3,2.68);
	%	\draw[dashed,-{Latex}] (3,2.68) -- (2.5,2.89);
	%	\draw[dashed,-{Latex}] (2.5,2.89) -- (2.25,3);
\end{axis}


\begin{tabla}{Efectos estabilizadores o amplificadores de shocks de demanda de bienes y dinero con regímenes cambiarios de tipo fijo o flexible.}{estabilizacionamplificacion}
	\begin{tabular}{l || c | c }
		& \textbf{TCN Fijo} & \textbf{TCN Flexible} \\ \hline \hline
		\textbf{Shocks de demanda de bienes} & Amplificación & Estabilización \\ \hline
		\textbf{Shocks de demanda de dinero} & Estabilización & Amplificación \\ \hline
	\end{tabular}
\end{tabla}

\preguntas

\seccion{Test 2017}

\textbf{32.} El ``terceto inconsistente'' o ``trinidad imposible'', se refiere a tres políticas que un país no puede adoptar simultáneamente. Éstas son:

\begin{itemize}
	\item[a] Libre movimiento de capitales, libre comercio y tipos de cambio fijos.
	\item[b] Libre movimiento de capitales, independencia de la política monetaria y tipos de cambio fijos.
	\item[c] Libre comercio, tipos de cambio flexibles y libre movimiento de capitales.
	\item[d] Independencia fiscal, libre movimiento de capitales y tipos de cambio flexibles.
\end{itemize}

\textbf{33.} En la Teoría de las Zonas Objetivo, ¿qué condición debe cumplirse para que ocurra el efecto ``luna de miel''?

\begin{itemize}
	\item[a] Las bandas deben ser lo suficientemente anchas para que no restrinjan el tipo de cambio de manera efectiva.
	\item[b] El compromiso de intervenir debe ser creíble. 
	\item[c] Las bandas deben ser lo suficientemente estrechas como para que el tipo de cambio se mueva muy poco.
	\item[d] El nivel de especulación en los mercados de tipo de cambio debe ser bajo.
\end{itemize}

\seccion{Test 2007}
\textbf{30.} Desde el punto de vista teórico, una ventaja de un sistema de tipo de cambio fijo frente a uno de tipo de cambio flexible sería que:

\begin{itemize}
	\item[a] Garantiza en todo momento la existencia de equilibrio en la balanza de pagos.
	\item[b] Tiende a aislar a la economía de los efectos de las perturbaciones exteriores.
	\item[c] Permite desarrollar una política monetaria autónoma a nivel nacional.
	\item[d] Facilita que la tasa de inflación sea menor, al llevar aparejada una mayor disciplina para las autoridades.
\end{itemize}

\seccion{Test 2006}
\textbf{29.} Considere los regímenes cambiarios conocidos como dolarización y currency board. Entre las ventajas relativas de cada uno de ellos pueden señalarse las siguientes:

\begin{itemize}
	\item[a] Una de las principales ventaja de la dolarización frente al currency board es que la dolarización gozará de mayor credibilidad, al ser un régimen más difícil de deshacer.
	\item[b] Una de las principales ventajas del currency board frente a la dolarización es que el currency board permite mantener el señoreaje mientras que la dolarización no lo permite.
	\item[c] La dolarización, por su parte, al eliminar completamente el riesgo de devaluación, en general permite disfrutar de menores tipos de interés para el endeudamiento exterior.
	\item[d] Todas las afirmaciones son correctas.
\end{itemize}

\seccion{Test 2004}

\textbf{37.} Ante las fuertes entradas de capital extranjero registradas en algunos países en desarrollo en la primera mitad de los años noventa, una de las respuestas de política económica fue la esterilización de tales entradas. Esa esterilización consiste en:
\begin{itemize}
	\item[a] Vender bonos públicos en la misma proporción que el aumento de las reservas en divisas.
	\item[b] Aumentar los activos externos del banco central, para mantener inalterado su balance.
	\item[c] Aumentar la inversión en el extranjero de los agentes nacionales.
	\item[d] Reducir el crédito interno en la misma proporción que el aumento de las reservas en divisas, para mantener inalterada la base monetaria.
\end{itemize}

\textbf{38.} Las juntas monetarias o cajas de conversión (\textit{currency boards}) de tipo ortodoxo son:
\begin{itemize}
	\item[a] Regímenes intermedios de tipo de cambio con un ajuste deslizante de la paridad.
	\item[b] Regímenes de tipo de cambio semi-fijo con respecto a una única moneda de referencia.
	\item[c] Regímenes de tipo de cambio fjo en los que la moneda nacional en circulación está respaldada plenamente por reservas en divisas.
	\item[d] Regímenes de tipo de cambio fijo con respecto a una única moneda de referencia, como por ejemplo, el de Argentina entre 1991 y 2002.
\end{itemize}


\notas

\textbf{2017:} \textbf{32.} B \textbf{33.} B

\textbf{2006:} \textbf{29.} D Ver pág. 738 de Appleyard y Field (2014): ``\textit{The seigniorage problem}''

\textbf{2007:} \textbf{30.} D

\textbf{2004:} \textbf{37.} D ¿Vender bonos públicos no es reducir el crédito interno? \textbf{38.} C 

Mirar documento en carpeta del tema sobre cálculo de currency baskets y pros y contras.

\bibliografia

Mirar en Palgrave:
\begin{itemize}
	\item \textbf{crawling peg}
	\item \textbf{currency boards}
	\item currency competition
	\item currency crises
	\item currency crises models
	\item exchange control
	\item \textbf{exchange rates}
	\item exchange rate dynamics
	\item exchange rate exposure
	\item \textbf{exchange rate target zones}
	\item exchange rate volatility
	\item financial liberalization
	\item \textbf{fixed exchange rates}
	\item \textbf{flexible exchange rates}
	\item foreign exchange markets, history of
	\item foreign exchange market microstructure
	\item foreign exchange reserve management
	\item foreign trade multiplier
	\item \textbf{gold standard}
	\item international capital flows
	\item international finance
	\item international indebtedness
	\item international liquidity
	\item International Monetary Fund
	\item international monetary institutions
	\item international monetary policy
	\item international real business cycles
	\item international reserves
	\item J-curve
	\item nominal exchange rates
	\item real exchange rates
\end{itemize}

Aghion, P.; Bacchetta, P.; Ranciere, R.; Rogoff, K. \textit{Exchange Rate Volatility and Productivity Growth: The Role of Financial Development} (2006) NBER Working Paper Series -- En carpeta del tema

Appleyard, D. R.; Field, A. \textit{International Economics} (2014) McGraw-Hill Irwin. 8th edition -- En carpeta economía internacional

Baxter, M.; Stockman, A. C. \textit{Business Cycles and the Exchange Rate System: Some International Evidence} (1989) NBER Working Paper Series -- En carpeta del tema

Calvo, G. A.; Mishkin, F. \textit{The Mirage of Exchange Rate Regimes for Emergin Market Countries} (2003) Journal of Economic Perspectives: Fall 2003 -- En carpeta del tema

Calvo, G. Reinhart, C. \textit{Fear of Floating} (2000) NBER Working Paper -- En carpeta del tema

Corsetti, G.; Kuester, K.; Müller, G. (2018) \textit{The case for flexible exchange rates after the Great Recession} Sveriges Riksbank Economic Review 2018:1 -- En carpeta del tema

Dellas, H.; Tavlas, G. S. \textit{Milton Friedman and the case for flexible exchange rates and monetary rules} (2017) Bank of Greece Working Paper -- En carpeta del tema

Dominguez, K. M.; Frankel, J. A. \textit{Does Foreign-Exchange Intervention Matter? The Portfolio Effect} (1993) American Economic Review -- En carpeta del tema

Dornbusch, R. \textit{Expectations and Exchange Rate Dynamics} (1976) Journal of Political Economy -- En carpeta del tema

Edwards, S.; Levy Yeyati, E. \textit{Flexible Exchange Rates as Shock Absorbers} (2003) NBER Working Paper Series -- En carpeta del tema

Edwards, S. \textit{Thirty Years of Current Account Imbalances, Current Account Reversals and Sudden Stops} (2004)

Eichengreen, B. (2018) \textit{Ragnar Nurkse and the international financial architecture} Baltic Journal of Economics. Vol. 18. No. 2 -- En carpeta del tema

Fischer, S. \textit{Exchange Rate Regeimes: Is the Bipolar View Correct?} (2001) Journal of Economic Perspectives -- En carpeta del tema

Gali, J. Perotti, R. \textit{Fiscal Policy and Monetary Integration in Europe} (2003) NBER Working Paper Series -- En carpeta del tema

Gandolfo, G. \textit{International Finance and Open-Economy Macroeconomics} (2016) Springer Verlag. Ch. 17 -- En carpeta Economía internacional

Ghosh, A.; Gulde, A. M.; Ostry, J.; Wolf, H. \textit{Does the Nominal Exchange Rate Regime Matter} (1997) NBER Working Paper Series -- En carpeta del tema

Ghosh, A.; Ostry, J.; Qureshi, M. \textit{Managing the exchange rate: It's not how much, but how} (2014) VOX CEPR Policy Portal -- \url{https://voxeu.org/article/managing-exchange-rate}

Husain, A.; Mody, A.; Rogoff, K. \textit{Exchange Rate Regimen Durability and Performance in Developing Versus Advanced Economies} (2004) NBER Working Paper Series -- En carpeta del tema

James, J.; Warsh, I. W.; Sarno, L. \textit{Handbook of Exchange Rates} (2012) Ch. 5,19, 26, 28. Wiley Publications -- En carpeta del tema y Economía Internacional (Libro completo)

Krugman, P. R. (1991) \textit{Exchange Rate Dynamics} Quarterly Journal of Economics. Vol. 106, No. 3 -- En carpeta del tema

Levy-Yeyati, E.; Sturzenegger, F. \textit{Classifying Exchange Rates Regimes: Deeds vs Words} (2005) European Economic Review -- En carpeta del tema

Levy-Yeyati, E.; Sturzenegger, F. \textit{To Float or to Fix: Evidence on the Impact of Exchange Rate Regimes on Growth} (2003) American Economic Review -- En carpeta del tema

Obstfeld, M.; Rogoff, K. \textit{The Mirage of Fixed Exchange Rates} (1995) Journal of Economic Perspectives -- En carpeta del tema

Obstfeld, M.; Shambaugh, J.; Taylor, A. M. \textit{The Trilemma in History: Tradeoffs among Exchange Rates, Monetary Policies, and Capital Mobility} (2004) NBER Working Paper Series -- En carpeta del tema

Obstfeld, M.; Shambaugh, J.; Taylor, A. \textit{Financial Stability, the Trilemma, and International Reserves} (2008) NBER Working Paper Series -- En carpeta del tema

Pilbeam, K. \textit{International Finance} (2006) 3rd Edition -- En carpeta Economía Internacional

Pisany-Ferry, J. \textit{The Euro crisis and the new impossible trinity} (2012) Bruegel Policy Contribution -- En carpeta del tema


Reinhart, C. M.; Rogoff, K. S. \textit{The Modern History of Exchange Rate Arrangements: A Reinterpretation} (2002) NBER Working Paper Series -- En carpeta del tema

Rey, H. \textit{Dilemma not Trilemma: The Global Financial Cycle and Monetary Policy Independence} (2015) NBER Working Papers. Revised February 2018 -- En carpeta del tema

Rogoff, K. S.; Husain, A. M.; Mody, A.; Brooks, R.; Oomes, N. \textit{Evolution and performance of Exchange Rate Regimes} (2003) IMF Working Paper

Rose, A. K. \textit{Exchange Rate Regimes in the Modern Era: Fixed, Floating and Flaky} (2011) Journal of Economic Literature -- En carpeta del tema

Rose, A. K. \textit{One money, one market: the effect of common currencies on trade} (2000) Economic Policy -- En carpeta del tema

Sarno, L.; Taylor, M. \textit{The economics of exchange rates} (2002) Cambridge University Press -- En carpeta Economía internacional

Sarno, L.; Taylor, M. \textit{Official Intervention in the Foreign Exchange Market: Is It Effective, and, if so, How Does it Work?} (2001) CEPR Discussion Paper -- En carpeta del tema

Steinberg, D., Walter, S. (2013) \textit{The Political Economy of Exchange-Rate Policy} Ch. 3. Handbook of Safeguarding Global Financial Stability -- En carpeta del tema

Taylor, M. P. (1995) \textit{The Economics of Exchange Rates} Journal of Economic Literature Vol. XXXIII -- En carpeta del tema

Towbin, P.; Weber, S. (2011) \textit{Limits of Floating Exchange RAtes: the Role of Foreign Currency Debt and Import Structure} IMF Working Paper WP/11/42 -- En carpeta del tema

\end{document}
