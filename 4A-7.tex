\documentclass{nuevotema}

\tema{4A-7}
\titulo{Análisis de los sectores de la industria agroalimentaria y de los bienes de consumo tradicionales.}

\begin{document}

\ideaclave

\input{/home/alibey/Oposicion/Resumenes_4o/Importancia_cuantitativa_Sectores.tex}

\seccion{Preguntas clave}

\begin{itemize}
	\item Industria agroalimentaria
\begin{itemize}
	\item ¿Qué es la industria agroalimentaria?
	\item ¿Cuál es su situación actual?
	\item ¿Cómo ha evolucionado?
	\item ¿Qué políticas se han llevado a cabo?
	\item ¿Qué políticas se llevan a cabo en la actualidad?
\end{itemize}

	\item Bienes de consumo tradicionales
\begin{itemize}
	\item ¿Qué es la industria de los bienes de consumo tradicionales?
	\item ¿Cuál es su situación actual?
	\item ¿Cómo ha evolucionado?
	\item ¿Qué políticas se han llevado a cabo?
	\item ¿Qué políticas se llevan a cabo en la actualidad?
\end{itemize}
\end{itemize}

\esquemacorto

\begin{esquema}[enumerate]
	\1[] \marcar{Introducción}
		\2 Contextualización
			\3 Sectores de la economía española
			\3 Sectores en cuestión: factores en común
			\3 Industria agroalimentaria
			\3 Bienes de consumo tradicionales
		\2 Objeto
			\3 Industria agroalimentaria
			\3 Bienes de consumo tradicionales
		\2 Estructura
			\3 Industria agroalimentaria
			\3 Bienes de consumo tradicionales
	\1 \marcar{Industria agroalimentaria}
		\2 Análisis estático
			\3 Delimitación del sector
			\3 Importancia
			\3 Modelos teóricos relevantes
			\3 Oferta
			\3 Demanda interna
			\3 Demanda externa
		\2 Análisis dinámico
			\3 Evolución
			\3 Actualidad
			\3 Circuitos cortos de distribución
			\3 Marcas de distribuidor vs marcas de fabricante
			\3 TIC en Agroalimentario
			\3 Medioambiente
			\3 Covid y demanda de productos agroalimentarios
			\3 Cambios en demanda
			\3 Distribución comercial
		\2 Políticas públicas
			\3 Empresas participadas por sector público
			\3 Centros de distribución MERCASA
			\3 DOP, IGP, ETG
			\3 Plan Integral de Política Industrial 2020
			\3 Internacionalización
			\3 Ley de la Cadena Alimentaria
			\3 Horarios de apertura
			\3 Licencias de grandes superficies
			\3 INIA
		\2 Alimentación
			\3 Cárnica
			\3 Aceites y aceitunas
			\3 Panadería y pastas
			\3 Bebidas
			\3 Conservas de pescado y mariscos
			\3 Conservas vegetales
			\3 Azúcar
			\3 Lácteos
			\3 Piensos
		\2 Bebidas
			\3 Vinos
			\3 Cerveza
			\3 Agua envasada
			\3 Espirituosas
			\3 Bebidas edulcoradas
		\2 Tabaco
			\3 Importancia
			\3 Demanda
			\3 Estructura industrial
		\2 Distribución alimentaria
			\3 Concepto
			\3 Importancia
			\3 Modalidades
			\3 Empresas
			\3 Trabajo
			\3 Canales
			\3 Marcas
			\3 Demanda
			\3 Internacional
			\3 Evolución
			\3 Perspectivas
			\3 Políticas públicas
	\1 \marcar{Bienes de consumo tradicionales}
		\2 Papel
			\3 Análisis estático
			\3 Análisis dinámico
			\3 Política económica
		\2 Artes gráficas y edición
			\3 Análisis estático
			\3 Análisis dinámico
			\3 Política económica
		\2 Productos farmacéuticos
			\3 Análisis estático
			\3 Análisis dinámico
			\3 Política económica
		\2 Industria textil y confección
			\3 Análisis estático
			\3 Análisis dinámico
			\3 Política económica
			\3 Industria textil
			\3 Algodón
			\3 Lana
			\3 Seda
			\3 Confección
		\2 Cuero y calzado
			\3 Análisis estático
			\3 Análisis dinámico
			\3 Política económica
		\2 Mueble y madera
			\3 Delimitación
			\3 Importancia
			\3 Oferta
			\3 Demanda interna
			\3 Sector exterior
		\2 Juguetes
			\3 Análisis estático
			\3 Análisis dinámico
			\3 Política económica
		\2 Otros sectores
			\3 Joyas y bisutería
			\3 Productos de limpieza
	\1[] \marcar{Conclusión}
		\2 Recapitulación
			\3 Industria agroalimentaria
			\3 Bienes de consumo tradicionales
		\2 Idea final

\end{esquema}

\esquemalargo

\begin{esquemal}
	\1[] \marcar{Introducción}
		\2 Contextualización
			\3 Sectores de la economía española\footnote{Presentación Kingdom of Spain del Tesoro Público, diciembre de 2019 (fuente: INE).}
				\4 Porcentaje sobre VAB
				\4 Servicios: 74,7\%
				\4 Industria: 15,4\%
				\4 Construcción: 6,5\%
				\4 Sector primario: 3\%
			\3 Sectores en cuestión: factores en común
				\4 Parte de industria manufacturera
				\4[] Conjuntamente, parte mayoritaria de ind. manuf. española
				\4 Contenido tecnológico relativamente bajo
				\4 Intensiva en trabajo relativamente poco cualificado
				\4 Elevado potencial exportador
			\3 Industria agroalimentaria
				\4 Transformación, transporte, empaquetado
				\4 Productos destinados a consumo alimentario
				\4 De consumo final e intermedio
				\4[] O ambos
			\3 Bienes de consumo tradicionales
				\4 Bienes destinados al consumo final
				\4 Relativamente bajo nivel tecnológico
				\4 Muy larga trayectoria histórica
				\4 Ejemplos relevantes
				\4[] Trabajos de la madera
				\4[] Textil
				\4[] Papel y artes gráficas
		\2 Objeto
			\3 Industria agroalimentaria
				\4 ¿Qué es la industria agroalimentaria?
				\4 ¿Cuál es su situación actual?
				\4 ¿Cómo ha evolucionado?
				\4 ¿Qué políticas se han llevado a cabo?
				\4 ¿Qué políticas se llevan a cabo en la actualidad?
			\3 Bienes de consumo tradicionales
				\4 ¿Qué es la industria de los bienes de consumo tradicionales?
				\4 ¿Cuál es su situación actual?
				\4 ¿Cómo ha evolucionado?
				\4 ¿Qué políticas se han llevado a cabo?
				\4 ¿Qué políticas se llevan a cabo en la actualidad?
		\2 Estructura
			\3 Industria agroalimentaria
			\3 Bienes de consumo tradicionales
	\1 \marcar{Industria agroalimentaria}\footnote{Ver \href{https://www.mapa.gob.es/es/alimentacion/temas/industria-agroalimentaria/_informeanualindustria2019-2020_tcm30-542507.pdf}{MAPAMA (2020) Informe Anual de la Industria Alimentaria Española Periodo 2019-2020}, \href{https://www.mapa.gob.es/images/es/20190807_informedeconsumo2018pdf_tcm30-512256.pdf}{Informe del Consumo Alimentario en España 2018} y \href{https://www.cajamar.es/es/pdf/observatorio-sector-agro.pdf}{Observatorio sobre el sector agroalimentario español en el contexto europeo. Informe 2018. Cajamar.}.}
		\2 Análisis estático
			\3 Delimitación del sector
				\4 Concepto
				\4[] Conjunto de actividades económicas
				\4[] Transformación industrial de alimentos
				\4[] $\to$ A partir de productos del sector primario
				\4[] Parte del sector industrial
				\4 Subsectores
				\4[] Alimentación: CNAE 10
				\4[] Bebidas: CNAE 11
				\4[] Tabaco: CNAE 12
				\4 Diferenciación del producto
				\4[] Muy elevada
				\4 Características de la demanda
				\4[] Muy estable
				\4[] Consumo agregado muy poco cíclico
				\4[] Desplazamiento de calidades según ciclo
				\4[] Preferencia por variedad
				\4[] $\to$ Creciente con renta
				\4 Ciclicidad
				\4[] Relativamente baja
				\4[] Bienes de consumo perecedero y primera necesidad
				\4[] $\to$ Poca elasticidad a renta
				\4[] Creció incluso durante crisis
				\4 Fuentes estadísticas
				\4[] INE
				\4[] $\to$ Consumo alimentario en hogares
				\4[] $\to$ Barómetro del clima de confianza del Sector Agroalimentario
				\4[] $\to$ Encuesta de Recogida de Leche y Productos Lácteos
				\4[] $\to$ Estadística de Industrias Lácteas
				\4[] $\to$ Previsiones Nacionales de Producciones Cárnicas
				\4[] $\to$ Precios coyunturales de productos ganaderos
			\3 Importancia
				\4 Cuantitativa
				\4[] 30.000 M de € de VAB en 2019
				\4[] $\to$ Industria alimentación, bebidas y tabaco\footnote{Ver INE, CNAE 10-12}
				\4[] España cuarto mayor exportador de UE
				\4[] 2,5\% del empleo total
				\4 Cualitativa
				\4[] Bienes de primera necesidad
				\4[] Efecto psicológico
				\4[] Eslabonamientos hacia atrás con agricultura
				\4[] $\to$ Absorbe 75\% producción agrícola
				\4[] Eslabonamientos hacia delante
				\4[] $\to$ Comercio y hostelería
				\4[] $\to$ Turismo
			\3 Modelos teóricos relevantes
				\4 Krugman sobre NEG
				\4 Dixit-Stiglitz
				\4 Ley de Engel
				\4 Estructuralismo
				\4 Oligopolio
				\4 Medio ambiente
			\3 Oferta
				\4 Trabajo
				\4[] Peso elevado de sector en relación a UE
				\4[] 430.000 empleados en 2018
				\4[] $\to$ Cárnica, panadería y bebidas principales sectores
				\4 Capital
				\4[] Relativamente baja capitalización
				\4[] Empresas de tamaño relativamente pequeño
				\4 Empresas
				\4[] 300.000 empresas
				\4[] Concentración relativamente elevada
				\4[] Estructura dual de empresas
				\4[] $\to$ Grandes empresas generan 70\% producción
				\4[] $\to$ Elevado número de PYMES
				\4 Distribución geográfica
			\3 Demanda interna
				\4 Factores culturales relevantes
				\4 Aumento de renta
				\4[] Aumenta demanda de productos diferenciados
				\4 Aumento de población extranjera
				\4[] Diversificación de importaciones
				\4[] Demanda nacional de nuevos productos
				\4[] $\to$ Posibles masas críticas
				\4[] $\to$ Aumenta CI con países de origen de inmigración
			\3 Demanda externa
				\4 Exportaciones
				\4[] Ind. agroalimentaria
				\4[] $\to$ Cuarto mayor exportador
				\4[] $\to$ Relativamente pequeño
				\4[] Destacan exportaciones de:
				\4[] $\to$ Carne de porcino y bovino
				\4[] $\to$ Aceite de oliva
				\4[] $\to$ Vino
				\4[] $\to$ Piensos
				\4 Importaciones
				\4[] Aumenta diversificación
				\4[] Productos de elevado valor añadido
				\4[] Competencia emergentes en África
				\4[] $\to$ Menores costes laborales
				\4[] $\to$ Creciente tecnificación
				\4[] $\to$ Capital humano en cumplimiento normas UE
				\4[] Destacan importaciones de:
				\4[] $\to$ Aceites no oliva
				\4[] $\to$ Quesos
				\4[] $\to$ Mariscos
				\4[] $\to$ Café y sucedáneos
				\4[] $\to$ Preparaciones de pescado
				\4[] $\to$ Azúcar, cerveza, whisky
				\4 Saldo neto
				\4[] Tasa de cobertura muy positiva
				\4[] Superávit agroalimentario de 15.000 M de €
				\4[] Transformados: 10.000 M de €
				\4[] No transformados: 5.000 M de €
				\4[] $\to$ Sector con mayor saldo positivo tras bienes de equipo
				\4[] $\then$ Incluye exportaciones agrícolas
				\4[] $\then$ ¿Hasta qué punto ``industria''
				\4[] Otras clasificaciones arrojan menores saldos positivos
				\4[] $\to$ Descontando agroalimentario no industrial
				\4[] Fuerte crecimiento en 2019 y últimos años
				\4[] Destacan saldos superavitarios
				\4[] $\to$ Frutas, hortalizas y legumbres
				\4[] $\to$ Cárnicos
				\4[] $\to$ Aceites y grasas
				\4[] $\to$ Bebidas
				\4[] Saldos negativos:
				\4[] $\to$ Pesqueros
				\4[] $\to$ Azucar, café, cacao
				\4[] $\to$ Lácteos y huevos
				\4[] $\to$ Tabaco
				\4 Competidores
				\4[] Italia
				\4[] $\to$ Aceites y grasas
				\4[] $\to$ Hortalizas
				\4[] Francia
				\4[] $\to$ Vinícola
				\4[] $\to$ Bebidas
				\4[] Marruecos
				\4[] $\to$ Frutas, verduras, hortalizas
				\4 Acuerdos y negociaciones comerciales
				\4[] Acuerdo agrícola
		\2 Análisis dinámico
			\3 Evolución
			\3 Actualidad
			\3 Circuitos cortos de distribución
			\3 Marcas de distribuidor vs marcas de fabricante
			\3 TIC en Agroalimentario
			\3 Medioambiente
			\3 Covid y demanda de productos agroalimentarios
			\3 Cambios en demanda
				\4 Aumento global demanda de variedad
				\4 Aumento demanda derivados de cereales
			\3 Distribución comercial
				\4 Elevada concentración aguas abajo
				\4 Poder de mercado hacia abajo
				\4 Tendencia a consolidación aguas arriba lenta
				\4[] Estructuras complejas de propiedad
				\4[] $\to$ Cooperativas
				\4[] $\to$ Sindicatos
				\4[] $\to$ Atomización
		\2 Políticas públicas
			\3 Empresas participadas por sector público
				\4 TRAGSA
				\4 Ebro Foods
				\4 Mercasa
				\4 CETARSA
			\3 Centros de distribución MERCASA
			\3 DOP, IGP, ETG
			\3 Plan Integral de Política Industrial 2020
				\4 Agroalimentario sector esencial
				\4 Subvenciones
			\3 Internacionalización
				\4 Instrumentos de promoción exterior
				\4 Énfasis en economías emergentes
				\4[] Demandas jóvenes y en crecimiento
			\3 Ley de la Cadena Alimentaria
				\4 Reforma en 2020
				\4[] Obligación de formalizar contratos > 2.500
				\4[] Restricciones a actividades promocionales
				\4[] Prohibición de vender bajo coste efectivo de producción
				\4[] Prohibición de modificaciones unilaterales
				\4[] Prohibición de de pagos adicionales
				\4[] Régimen sancionador
			\3 Horarios de apertura
			\3 Licencias de grandes superficies
			\3 INIA
				\4 Instituto Nacional de Investigación y Tecnología Agraria y Alimentaria
				\4 Creado en 1926
				\4 Sede en Madrid
				\4 Dependiente del Ministerio de Ciencia
		\2 Alimentación
			\3 Cárnica
				\4 Principal sector en empleados
				\4 Principal sector exportador
				\4 Especialmente cerdo
			\3 Aceites y aceitunas
				\4 Segundo sector exportador
			\3 Panadería y pastas
				\4 Segundo sector por empleados
				\4 Sin apenas exportaciones
			\3 Bebidas
				\4 Segundo sector por cifra de negocios
				\4 Segundo sector exportador
				\4[] Especialmente vino
			\3 Conservas de pescado y mariscos
				\4 Cercano a 4500 VAB
				\4 Elevado número de empresas
				\4[] Tendencia a mayor concentración
				\4 Concentración en Galicia
				\4 Tercer producto mundial conservas de atún
			\3 Conservas vegetales
				\4 España primer productor mundial
				\4[] Por competitividad frutas verduras
				\4 Concentración Navarra, Murcia, Rioja
			\3 Azúcar
				\4 Sistema previo de cuotas
				\4[] Cuota inferior a consumo nacional
				\4 Precios mínimos a remolacha hasta 2020
				\4[] Por acuerdo interprofesional
				\4 Fin de cuotas azucareras en 2017
			\3 Lácteos
				\4 Sector concentrado en norte de España
				\4 Adhesión a UE
				\4[] Cuota de producción asignada
				\4[] $\to$ Inferior a capacidad
				\4[] $\then$ Reconversión a cárnica
				\4 Fin de cuotas lácteas en 2015
				\4 Principales empresas
				\4[] Pascual
				\4[] Puleva
				\4[] Central Lechera Asturiana
			\3 Piensos
				\4 Transformación estructural desde 60s
				\4 Tercer mayor productor en UE
				\4 Especialización en algunos nichos
				\4[] Animales de compañía
				\4[] Ibéricos
		\2 Bebidas
			\3 Vinos
				\4 Segmentos
				\4[] Vino de mesa
				\4[] $\to$ Mayor cantidad de consumo
				\4[] VCPDR
				\4[] $\to$ Vino de Calidad Producido en una Región Determinada
				\4[] Granel
				\4[] $\to$ Menor valor por unidad de volumen
				\4[] Envasado
				\4[] $\to$ Precios mucho más elevados
				\4 Importancia
				\4[] Elevado potencial exportador
				\4[] España ventaja comparativa natural
				\4[] España 1º Europa superficie cultivada
				\4[] España 3º productor mundial
				\4[] España 3º en valor
				\4[] $\to$ Cultivos españoles relativamente poco productivos
				\4[] $\to$ Por detrás de Francia e Italia
				\4 Empresas
				\4[] Muy elevada atomización
				\4[] Comparativamente, España más atomizada
				\4[] $\to$ Especialmente respecto fuera UE
				\4 Tierra
				\4[] Caída en últimos años de superficie cultivada
				\4[] Relativamente estabilización reciente
				\4 Distribución
				\4[] Canal horeca casi la mitad
				\4[] $\to$ Especialmente en VCPDR
				\4[] Cambios tras crisis Covid
				\4[] $\to$ Aumento de e-commerce
				\4[] $\to$ Aumento de tiendas y supermercados
				\4 Competidores
				\4[] Francia e Italia
				\4[] $\to$ Superan en valor exportado
				\4[] Estados Unidos
				\4[] Argentina
				\4[] Sudáfrica
				\4[] Australia
				\4[] Chile
				\4[] China
				\4 Perspectivas
				\4[] Tendencia muy suave hacia reducción consumo
				\4[] $\to$ Aumento de cerveza
				\4[] Aumento de vinos alta calidad
				\4 Políticas públicas
				\4[] DOP/IGP
				\4[] $\to$ Especialmente importantes
				\4[] $\to$ Producciones menores en volumen
				\4[] $\to$ Mayores en valor
				\4[] Internacionalización
				\4[] $\to$ Medidas de promoción exterior (ICEX...)
			\3 Cerveza
				\4 Consumo creciente
				\4 Tendencia a marcas dominantes por regiones
				\4[] Históricamente
				\4 Aumento presencia otras marcas
				\4 Aumento marcas extranjeras
				\4 3er producto en UE
				\4 Relativamente alta penetración capital extranjero
				\4 Relativamente alto nivel tecnológico
				\4 Aparición cervezas artesanales
				\4[] Diferenciación
				\4[] Mercados locales
			\3 Agua envasada
			\3 Espirituosas
				\4 Elevado capital extranjero
				\4 Producción bajo licencia
				\4 Grandes grupos internacionales
				\4 Sujeto a problemas de falsificación
			\3 Bebidas edulcoradas
				\4 Tendencia decreciente
				\4 Algunos impuestos autonómicos
		\2 Tabaco
			\3 Importancia
			\3 Demanda
			\3 Estructura industrial
		\2 Distribución alimentaria
			\3 Concepto
				\4 Conjunto de actividades de servicios
				\4[] Poner en contacto oferta y demanda
				\4[] $\to$ Productos agroalimentarios
			\3 Importancia
				\4 Más de 90.000 M de € de facturación
				\4 Sector esencial por importancia de alimentación
			\3 Modalidades
				\4 Hipermercados
				\4[] Mayor variedad
				\4[] Descuentos
				\4[] Relativamente estable en crecimiento
				\4 Supermercados
				\4[] Tamaño menor a hipermercados
				\4[] Subclasificaciones
				\4[] $\to$ Grande
				\4[] $\to$ Mediano
				\4[] $\to$ Pequeño
				\4[] Super grande mayor crecimiento de todos
				\4 Hard discount
				\4[] Menor variedad de productos
				\4[] Descuentos muy fuertes
				\4 Conveniencia
				\4[] Proximidad
				\4[] Centro urbano
				\4[] Menores descuentos
				\4 Cash and carry
				\4[] Destinados a compra mayorista
				\4[] Pagar y traer uno mismo a su local
				\4[] $\to$ Alternativa a distribuidor tradicional que transportaba
			\3 Empresas
				\4 Por cuota de mercado entre nacionales
				\4[] Mercadona
				\4[] Carrefour
				\4[] Dia
				\4[] Eroski
				\4[] Lidl
				\4[] Auchan
				\4 Nacionales y regionales
				\4[] Importante segmento de regionales
				\4 Concentración relativamente baja
				\4[] En comparación a Europa
				\4[] $\to$ 50\%
			\3 Trabajo
				\4 Intensivo en trabajo
				\4 Trabajo temporal relativamente elevado
				\4 Asalarización elevada en grandes empresas
				\4 Autónomos en PYMES de alimentación
			\3 Canales
				\4 Online
				\4[] Relativamente menor a otros europeos
				\4[] $\to$ Oferta presencial de proximidad muy competitiva
				\4 Presencial
				\4 Horeca
			\3 Marcas
				\4 Marcas del distribuidor
				\4[] Aumento progresivo de cuota
				\4 Marcas exclusivas
				\4[] Muy reducido
				\4 Sin marcas
				\4[] Reducido
				\4 Resto de marcas
				\4[] Principal componente
			\3 Demanda
				\4 Cae ligeramente volumen en últimos años
				\4 Aumenta ligeramente en precios en los últimos años
				\4 Aumento de consumo de frescos
				\4 Aumento de consumo alimentos ecológicos
			\3 Internacional
				\4 Escasa presencia de empresas españolas en exterior
				\4 Elevada presencia capital extranjero en España
			\3 Evolución
			\3 Perspectivas
				\4 Franquicias frenan crecimiento
				\4 Crecimiento de canal online menor que otros
				\4[] Otros tipos de producto más susceptibles
				\4[] $\to$ No perecederos, fáciles de devolver
				\4 Supermercados regionales crecen
				\4 Ampliación a frescos en canal Online
				\4 Potencial segmento outlet en alimentación
				\4[] Emergente
				\4[] Sqrups primera iniciativa española
				\4 Canal horeca muy afectado por covid
			\3 Políticas públicas
				\4 Licencias de apertura grandes superficies
				\4 Horarios comerciales
				\4 Trazabilidad de cadena alimentaria
				\4 Reforma de la Ley de la Cadena Alimentaria
				\4[] Prohibición de venta a coste
				\4[] Obligación de pago electrónico > cantidad mínima
	\1 \marcar{Bienes de consumo tradicionales}
		\2 Papel
			\3 Análisis estático
				\4 Delimitación
				\4[] CNAE Divisiones 17: industria del papel
				\4[] CNAE Divisiones 18: artes gráficas y reproducción de soportes
				\4[] Muy variados productos
				\4[] $\to$ Celulosa
				\4[] $\to$ Pasta de papel
				\4[] $\to$ Tintas
				\4[] $\to$ Productos de impresión
				\4[] $\to$ Papeles adhesivos
				\4[] $\to$ Papeles especiales
				\4[] $\to$ Editoriales
				\4[] $\to$ Librerías
				\4[] $\to$ Embalajes
				\4[] $\to$ ...
				\4[] Diferentes tipos de fibra de celulosa
				\4[] $\to$ Fibra corta: papeles finos
				\4[] $\to$ Fibra larga: papeles resistentes y embalajes
				\4[] Subsegmentos en todos los productos
				\4[] $\to$ Embalajes planos
				\4[] $\to$ Embalajes out-of-home\footnote{Comida para llevar.}
				\4[] $\to$ Bolsas de papel
				\4 Importancia
				\4[] Cuantitativa
				\4[] $\to$ Cadena de papel: 4,5\% PIB en España\footnote{Ver Informe CESCE 2020, pág. 193}
				\4[] $\to$ Madera local >50\% de consumo
				\4[] Cualitativa
				\4[] $\to$ Relativamente menor en últimos años
				\4[] $\then$ Transición formatos digitales
				\4[] $\to$ Edición sigue siendo relevante por edición digital
				\4[] $\to$ Arrastre con sectores primario
				\4[] $\then$ Forestal
				\4[] $\then$ Plantaciones de eucalipto para fibra corta
				\4[] $\then$ Pino para fibra larga papeles resistentes
				\4 Modelos teóricos
				\4[] Cobweb
				\4[] $\to$ Mercados de materias primas con reparto
				\4[] Diferenciación de productos
				\4[] Modelos de difusión cultural y redes
				\4[] $\to$ Mercado de edición
				\4[] $\to$ Industria de artes gráficas
				\4 Oferta
				\4[] 10 plantas de celulosa
				\4[] 68 fábricas de papel
				\4[] Distribución geográfica fábricas papel
				\4[] $\to$ Principalmente Cataluña, Navarra, Zaragoza
				\4[] $\to$ Predomina noreste peninsular
				\4[] Elasticidad baja en corto plazo
				\4[] $\to$ Capacidad relativamente difícil de expandir
				\4[] $\to$ Madera relativamente inelástica
				\4[] Papeleras españolas relevantes a nivel nacional
				\4[] $\to$ ENCE Energía y Celulosa
				\4[] $\to$ Miquel y Costas
				\4[] Elevado uso de papel reciclado
				\4[] $\to$ Hasta 71\% de papel consumido es reciclado
				\4[] $\to$ Líder a nivel europeo con Alemania
				\4[] Intensiva en capital
				\4[] Relativamente intensiva en energía
				\4[] Cogeneración y biomasa
				\4[] $\to$ Mayor parte de energía utilizada en producción
				\4[] Reconversión de numerosas plantas
				\4[] $\to$ De papel de prensa
				\4[] $\to$ A envase y embalaje
				\4[] $\then$ Una sóla fábrica de papel de prensa actualmente
				\4 Demanda interna
				\4[] Implantación progresiva de bolsas de papel
				\4[] $\to$ Grandes supermercados
				\4[] Reducción progresiva papel de impresión
				\4[] Caídas anuales muy elevadas de papel de prensa
				\4[] Población acostumbrada a papel sanitario
				\4[] $\to$ Escaso margen de crecimiento
				\4[] Crecimiento elevado demanda de packaging
				\4[] No se han recuperado niveles pre-crisis de demanda
				\4[] Papel de prensa
				\4[] $\to$ Caídas anuales
				\4[] $\to$ Fabricado con papeles reciclados
				\4 Celulosa: Mercado exterior
				\4[] China
				\4[] $\to$ Fuerte aumento de demanda celulosa fibra corta
				\4[] $\to$ Papel para envases y embalajes
				\4[] $\to$ Papel tisú
				\4[] América del Sur
				\4[] $\to$ Mayor productor de celulosa fibra corta
				\4[] Madera para celulosa
				\4[] $\to$ Norteamérica, Brasil, Chile, Europa del Este
				\4[] Fuerte concentración internacional de productores
				\4[] $\to$ Especialmente en fibra corta
				\4[] España productor poco relevante a nivel internacional
				\4[] $\to$ Sí para consumo interno
				\4 Papel: mercado exterior
				\4[] España exporta >50\% de producción
				\4[] UE principal destino de exportaciones
				\4[] España cuarto productor europeo de cartón
				\4 Saldos comerciales
				\4[] Superávits y déficits en diferentes segmentos
				\4[] En términos globales, cifras poco relevantes
				\4 Edición y artes gráficas: mercado exterior
			\3 Análisis dinámico
				\4 Evolución
				\4[] Demanda de celulosa de fibra corta
				\4[] $\to$ Caídas notables en últimos años
				\4[] Demanda de tisú crece
				\4[] $\to$ Cambios en patrones de consumo
				\4[] $\to$ Aumento de la renta
				\4[] Demanda de papel de impresión
				\4[] $\to$ Alcanza máximo en primeros 2000
				\4[] $\to$ Caída posterior progresiva
				\4 Actualidad
				\4[] Recuperación pre-covid
				\4[] Fuerte caída post-covid
				\4[] $\to$ Shock inicial
				\4[] Recuperación progresiva de papel para embalajes
				\4[] $\to$ Considerado menos dañino ecológicamente
				\4[] $\to$ Sustituto de plásticos
				\4[] $\to$ Compras por internet
				\4[] Ligero exceso de oferta tras covid
				\4[] Progresiva recuperación de actividad
				\4 Perspectivas
				\4[] Buenas condiciones geográficas para madera
				\4[] Estrés hídrico
				\4[] Elevadas inversiones en descarbonización
				\4[] $\to$ Arrastre sector energético
				\4[] Sector maduro en impresión
				\4[] Sector en expansión en tisú y embalajes
				\4[] Mejoras de renta en países emergentes
				\4[] $\to$ Cambios hacia más consumo embalajes
				\4[] Covid impulsa transición a digital
				\4[] Fibras cortas ganan cuota de mercado
				\4[] Estándares medioambientales más estrictos
				\4[] Papel del idioma español
				\4[] $\to$ Centro internacional edición en castellano
			\3 Política económica
				\4 Justificación
				\4 Objetivos
				\4 Antecedentes
				\4 Marco jurídico
				\4 Marco financiero
				\4 Actuaciones
				\4[] UE:
				\4[] $\to$ Prohibición de bolsas de papel
				\4[] $\to$ Mercado de derechos de emisión
				\4[] España
				\4[] $\to$ Regulación medioambiental fábricas
				\4[] $\to$ Obligación de reciclar
				\4[] $\to$ Transposición directivas MA
				\4 Valoración
				\4 Retos
		\2 Artes gráficas y edición
			\3 Análisis estático
				\4 Delimitación
				\4[] Actividades:
				\4[] $\to$ Diseño gráfico
				\4[] $\to$ Preimpresión
				\4[] $\to$ Impresión
				\4[] $\to$ Encuadernación
				\4[] $\to$ Acabados
				\4[] $\to$ Edición
				\4[] $\to$ Prensa escrita
				\4[] Tipos de impresión
				\4[] $\to$ Offset para grandes tiradas
				\4[] $\to$ Impresión digital para papel
				\4[] Diferenciación
				\4[] $\to$ Relativamente reducida
				\4 Importancia
				\4[] Arrastre hacia delante
				\4[] Arrastre hacia atrás
				\4 Modelos teóricos
				\4 Oferta de artes gráficas
				\4[] Elevada atomización/baja concentración
				\4[] $\to$ Número muy elevado de empresas pequeñas
				\4[] $\to$ 5 mayores absorben sólo 10\% mercado
				\4[] Empresa familiar
				\4[] $\to$ Muy fuerte
				\4[] Propiedad nacional
				\4[] $\to$ Mayoritario
				\4[] Distribución geográfica artes gráficas
				\4[] $\to$ Madrid y Cataluña 20+\% cada una
				\4[] $\to$ Andalucía
				\4[] $\to$ Valencia
				\4 Oferta en editoriales
				\4[] Elevada concentración
				\4[] $\to$ 10 mayores empresas absorben 50\%
				\4[] Distribución geográfica editorial
				\4[] $\to$ Madrid y Cataluña 70\%+
				\4[] Principales grupos prensa escrita
				\4[] $\to$ Vocento
				\4[] $\to$ Unidad Editorial
				\4[] $\to$ Prisa
				\4[] $\to$ Planeta
				\4[] $\to$ Grupo Godó
				\4[] $\then$ 70\% ingresos totales
				\4[] $\to$ Grupo Prensa Ibérica (absorbe Zeta)
				\4[] Principales empresas editoras
				\4[] $\to$ Planeta
				\4[] $\to$ Anaya
				\4[] $\to$ Santillana
				\4[] $\to$ RB
				\4[] $\to$ Penguin Random House
				\4[] $\to$ Oxford University Press
				\4[] Importantes barreras de entrada en prensa escrita
				\4[] $\to$ Difusión inicial muy difícil de conseguir
				\4[] $\to$ Necesario mucho capital hasta ingresos publicitarios
				\4 Demanda interna
				\4[] Relativamente acíclico
				\4[] Sobre todo, demanda en castellano
				\4[] Pequeños mercados en lenguas regionales
				\4[] Importaciones
				\4[] $\to$ Reino Unido principal origen
				\4 Demanda externa
				\4[] España centro relevante de edición
				\4[] $\to$ Hacia Latinoamérica
				\4[] $\to$ Hacia Europa
				\4[] Ediciones de diarios españoles para Latinoamérica
				\4[] $\to$ Reducción progresiva tras fase de crecimiento
				\4[] $\to$ Desaparición edición El País
				\4[] Exportaciones relevantes a América
				\4[] $\to$ Argentina
				\4[] $\to$ Chile
				\4[] $\to$ Estados Unidos
			\3 Análisis dinámico
				\4 Evolución
				\4[] Entrada de algunos fondos de inversión últimos años
				\4[] Desaparición progresiva empresas pequeño tamaño
				\4[] $\to$ Se acelera en últimos años
				\4[] Reducción progresiva producción impresa
				\4[] Caída de inversión publicitaria
				\4[] Caída progresiva actividad editorial
				\4[] Aparición de nuevos canales distribución
				\4[] $\to$ Autoedición
				\4[] $\to$ Plataformas digitales
				\4 Actualidad
				\4[] Pequeño aumento en últimos periodos pre-crisis
				\4[] $\to$ Aumento demanda libros impresos
				\4[] $\to$ Autoedición relevante
				\4[] Publicidad digital compite con pub. en medios impresos
				\4[] $\to$ Caídas anuales
				\4 Perspectivas
				\4[] Sectores de demanda creciente
				\4[] $\to$ Envases impresos
				\4[] $\to$ Publicidad
				\4[] $\to$ Industria editorial
				\4[] Sectores en declive
				\4[] $\to$ Prensa escrita
				\4[] Libro electrónico continuará creciendo fuertemente
				\4[] $\to$ Aumento de cuota de mercado
				\4[] Transición de industria de papel
				\4[] $\to$ Progresivo menor peso de destino edición
			\3 Política económica
				\4 Justificación
				\4 Objetivos
				\4 Antecedentes
				\4 Marco jurídico
				\4 Marco financiero
				\4 Actuaciones
				\4 Valoración
				\4 Retos
		\2 Productos farmacéuticos
			\3 Análisis estático
				\4 Delimitación
				\4[] CNAE Divisiones 21: fabricación productos farmacéuticos
				\4[] Divisiones en:
				\4[] $\to$ Productos de base
				\4[] $\to$ Especialidades farmacéuticas
				\4 Importancia
				\4 Modelos teóricos
				\4 Oferta
				\4 Demanda interna
				\4 Demanda externa
			\3 Análisis dinámico
				\4 Evolución
				\4 Actualidad
				\4 Perspectivas
			\3 Política económica
				\4 Justificación
				\4 Objetivos
				\4 Antecedentes
				\4 Marco jurídico
				\4 Marco financiero
				\4 Actuaciones
				\4 Valoración
				\4 Retos
		\2 Industria textil y confección
			\3 Análisis estático
				\4 Delimitación
				\4[] Múltiples categorías en términos de CNAE
				\4[] CNAE División 13: industria textil
				\4[] CNAE División 14: confección de prendas de vestir
				\4[] Otras:
				\4[] $\to$ Fabricación de maquinaria textil
				\4[] $\to$ Intermediarios del comercio de textiles
				\4[] $\to$ Comercio al por mayor de textiles
				\4[] $\to$ Comercio al por menor
				\4 Importancia
				\4[] Cuantitativa\footnote{Ver Informe CESCE (2019) pág. 245.}
				\4[] $\to$ Industria textil es 0,7\% PIB total
				\4[] $\to$ Casi 3\% PIB incluyendo comercio
				\4[] $\then$ 13,2\% del comercio
				\4[] $\then$ 5,1\% de Industria
				\4[] $\to$ 4,3\% del empleo total
				\4[] $\then$ Casi 20\% empleo en comercio
				\4[] $\then$ 8\% en la industria
				\4[] Cualitativa
				\4[] $\to$ Arrastre sector primario
				\4[] $\then$ Cultivos textiles
				\4[] $\then$ Tejidos animales y pieles
				\4[] $\to$ Arrastre industria
				\4[] $\then$ Transformación de tejidos
				\4[] $\then$ Bienes de equipo
				\4[] $\then$ Refino y plásticos
				\4 Modelos teóricos
				\4[] Demanda de características
				\4[] Demanda de bienes diferenciados
				\4 Oferta
				\4[] Número de empresas creciente desde 2015
				\4[] Trabajo
				\4[] $\to$ 130.000 personas en CNAE 13-15
				\4[] $\to$ Aumento desde salida de crisis
				\4[] $\to$ Caída en 2018 y 2019
				\4[] $\to$ Ind. textil relativamente poco intensiva mano de obra
				\4[] $\to$ Confección más intensiva mano de obra
				\4[] Outlets
				\4[] Cadenas especializadas
				\4[] Principales empresas
				\4[] $\to$ Inditex
				\4[] $\to$ Tendam (antes Cortefiel)
				\4[] Zona este
				\4[] $\to$ Cataluña primera
				\4[] $\to$ Comunidad Valencia
				\4[] Plataformas logísticas
				\4[] $\to$ Grupo Inditex en Cataluña para algunas marcas
				\4[] Galicia
				\4[] $\to$ Más importante comunidad en trabajadores
				\4[] $\to$ Sede de grupo inditex
				\4[] $\to$ Plataformas logísticas de inditex
				\4[] Andalucía
				\4[] $\to$ Empresas de tamaño muy reducido
				\4[] $\to$ Mayor número de empresas
				\4[] $\to$ No tiene mayor número de empleados
				\4 Demanda interna
				\4[] Tendencia a atomización
				\4[] Demanda productos personalizados
				\4[] $\to$ Creciente frente a marcas lujo consolidadas
				\4[] $\to$ Especialmente alta gama
				\4[] Elasticidad renta extensiva reducida
				\4[] $\to$ En relación a gasto total
				\4[] Elasticidad-renta intensiva elevada
				\4[] $\to$ En relación a gasto y renta personal
				\4[] Apenas 4,4\% presupeusto familiar a ropa
				\4[] $\to$ Por encima de norte de Europa
				\4 Demanda externa
				\4 Saldo exterior
				\4[] Fuertemente deficitario
				\4[] Casi -6.000 M de € en 2019
				\4[] $\to$ Confección principal componente
				\4[] 81\% de tasa de cobertura
				\4[] Fuertemente Deficitario
				\4[] China, Sudeste Asiático, Marruecos, Turquía
				\4 Exportaciones totales
				\4[] 20.000 M de €
				\4 Importaciones totales
				\4[] 25.000 M de €
				\4 Saldo con UE
				\4[] Escasa cuantía
				\4[] Ligeramente superavitario
				\4 Apertura exterior creciente desde 2000
				\4 Destinos principales de exportación
				\4[] Francia
				\4[] Italia
				\4[] Portugal
				\4[] Alemania
				\4[] $\to$ Reino Unido
				\4 Muy difícil competencia exterior
				\4[] Al margen de grandes grupos empresariales
				\4[] $\to$ Inditex especialmente
				\4[] Competencia muy fuerte
				\4[] Muy difícil competencia en precios
				\4[] España relativamente competitiva en gama alta
				\4[] $\to$ Mayor diferenciación
				\4[] $\to$ Mayor importancia imagen de marca
			\3 Análisis dinámico
				\4 Actualidad
				\4[] Fuertes caídas en crisis Covid
				\4[] Confinamiento
				\4[] Reducción de demanda
				\4[] $\to$ Teletrabajo
				\4[] $\to$ Restricciones movilidad
				\4[] Fast fashion
				\4[] $\to$ Prendas baratas
				\4[] $\to$ Rotación rápida diseños
				\4[] $\then$ Gana peso en últimos años
				\4[] $\then$ Menor tiempo de uso de prednas
				\4[] $\then$ Mayor impacto medioambiental
				\4 Perspectivas
				\4[] Canal internet aumente peso
				\4[] Cadenas especializadas ganen cuota de mercado
				\4[] Recuperación progresiva tras Covid
				\4[] Intensificación de la competencia
				\4[] Hábitos cambiantes
				\4[] Reciclaje de tejidos
				\4[] Wearables
				\4[] $\to$ Fusión gadgets y prendas de vestir
			\3 Política económica
				\4 Justificación
				\4 Objetivos
				\4 Antecedentes
				\4 Marco jurídico
				\4 Marco financiero
				\4 Actuaciones
				\4 Valoración
				\4 Retos
			\3 Industria textil
				\4 Concepto
				\4[] Fabricación de fibras textiles
				\4[] $\to$ Preparación previa
				\4[] $\to$ Hilado
				\4[] No incluye confección
				\4 Segmentos
				\4[] Preparación e hilado
				\4[] Fabricación de tejidos textiles
				\4[] Acabado de textiles
				\4 Evolución
				\4[] Edad Media
				\4[] $\to$ Mesta
				\4[] $\to$ Comercio de lana
				\4[] Siglo XIX: desarrollo industria textil
				\4[] $\to$ Cataluña
				\4[] $\to$ Levante
				\4[] $\to$ Algunos puntos en Castilla
				\4[] Siglo XX
				\4[] $\to$ Elevada importancia en años 60-70
				\4[] $\then$ Confección fue hasta 6\% del VAB
				\4[] Pérdida de importancia progresiva
				\4[] $\to$ Casi 90\% de peso anterior en VAB
				\4 Textiles técnicos
				\4[] Ganan peso creciente
				\4[] $\to$ Frente a hogar y confección
				\4 Exportaciones
				\4[]
				\4 Importaciones
				\4[] Principales orígenes
				\4[] $\to$ China
				\4[] $\to$ Bangladesh
				\4[] $\to$ India
				\4[] $\to$ Filipinas
				\4[] $\to$ Turquía
				\4[] $\to$ Vietnam
				\4[] $\to$ Marruecos
				\4 Saldo
				\4[] Superavitario en industria textil
				\4[] $\to$ Déficit aparece posteriormente en confección
				\4[] Apenas 500 en total
			\3 Algodón\footnote{Sahuquillo}
				\4 Mayor importancia en España
				\4 Primera industria exportadora de origen vegetal
			\3 Lana
				\4 Primera industria exportadora de historia
				\4 Origen en Mesta y comercio con Flandes
				\4 Altos precios y costes
				\4 Sustitución por fibras sintéticas
				\4 Béjar
			\3 Seda
				\4 Muy reducido peso
				\4 Apenas utiliza seda natural
				\4 Principalmente, materias primas sustitutivas
				\4[] Rayón
				\4[] Nailón
				\4[] ...
				\4 Eslabonamientos con refinerías y petróleo
			\3 Confección
				\4 Empleo
				\4[] Destrucción progresiva de empleo desde años 90
				\4[] Recuperación post-cris
				\4[] Fuerte caída 2019
				\4 Tendencia decreciente confección esapñola
				\4 Saldo fuertemente deficitario con exterior
				\4[] Mayor parte de saldo total textil
		\2 Cuero y calzado
			\3 Análisis estático
				\4 Delimitación
				\4[] CNAE División 15: cuero y calzado
				\4[] Transformación:
				\4[] $\to$ de piel en cuero
				\4[] $\to$ de cuero en calzado y accesorios
				\4 Importancia
				\4[] Cualitativa
				\4[] $\to$ Arrastre con sector ganaderoº
				\4[] $\to$ Arrastre con bienes de equipo tec. media-baja
				\4[] $\to$ Arrastre con diseño y servicios marketing
				\4[] $\to$ Arrastre con distribución comercial
				\4[] Cuantitativa
				\4[] $\to$ España 2º exportador UE de curtidos tras Italia
				\4 Oferta
				\4[] Unos 60.000 empleados
				\4[] $\to$ Fuerte crecimiento desde 2013
				\4[] $\to$ Caída desde 2018
				\4[] Atomización de empresas
				\4[] $\to$ Elevada tasa de subcontratación de grandes empresas
				\4[] Relativamente elevada capitalización en cuero
				\4[] $\to$ En relación a competidores europeos
				\4[] Escasa capitalización del calzado
				\4[] Relativamente intensivo en trabajo
				\4[] Concentración en Comunidad Valencia
				\4[] $\to$ Especialmente Alicante
				\4[] $\then$ Elche, Elda
				\4[] Producción centrada en calzado no deportivo
				\4 Demanda interna
				\4 Demanda externa
				\4[] España exportadora calzado no deportivo
			\3 Análisis dinámico
				\4 Evolución
				\4[] Resistió a crisis relativamente bien
				\4[] Fuerte
				\4 Actualidad
				\4[] Calzado deportivo principal componente
				\4[] Muy bajo \% calzado deportivo
				\4 Perspectivas
				\4[] Aumento de diferenciación
				\4[] Diseño parte creciente de valor añadido
				\4[] Personalización tendencia creciente
			\3 Política económica
				\4 Justificación
				\4 Objetivos
				\4 Antecedentes
				\4 Marco jurídico
				\4 Marco financiero
				\4 Actuaciones
				\4 Valoración
				\4 Retos
		\2 Mueble y madera
			\3 Delimitación
				\4 Concepto
				\4[] CNAE División 16: industria de la madera y corcho
				\4[] CNAE Grupo 022: explotación de la madera
				\4[] CNAE División 31: fabricación de muebles
				\4 Subsectores
				\4[] Corcho
				\4[] Madera
				\4[] $\to$ Madera sin elaborar
				\4[] $\to$ Primera transformación: tablones, tablas..
				\4[] $\to$ Segunda transformación: muebles, ventanas...
				\4[] Muebles
				\4 Diferenciación
				\4[] Elevada
				\4 Ciclicidad
				\4[] Bastante elevada
			\3 Importancia
				\4 Cualitativa
				\4[] Relativamente bajo capital
				\4[] Impacto medioambiental
				\4 Cuantativa
				\4[] Cercana a 4.700 M de € en 2018
				\4[] Débil crecimiento
				\4[] Poca importancia en exportación
			\3 Oferta
				\4 Recursos forestales
				\4[] España cuarto puesto europeo
				\4[] $\to$ Tras Rusia,Suecia,Finlandia
				\4 Distribución geográfica
				\4[] Muebles
				\4[] $\to$ Cataluña y Valencia principales
				\4[] $\to$ Siguen resto, primera Madrid
			\3 Demanda interna
				\4 Fuertemente ligado a construcción
				\4 Muy fuerte aumento pre-burbuja
				\4 Máximo entre 2003 y 2008\footnote{Ver \href{https://www.ine.es/jaxiT3/Datos.htm?t=32449\#!tabs-tabla}{INE sobre industria del mueble}}
			\3 Sector exterior
				\4 España poco relevante
				\4[] Importador y exportador
				\4 Aumento reciente de exportaciones e interés
				\4[] Muebles de alta calidad
				\4 Imagen de marca relativamente débil
				\4 Competidores
				\4[] China
				\4[] Italia
				\4[] Polonia
				\4[] Vietnam
				\4 Saldo generalmente deficitario
				\4[] Aunque escasa cuantía
		\2 Juguetes
			\3 Análisis estático
				\4 Delimitación
				\4[] Grupo CNAE: 324 fabricación de juguetes
				\4[] Clase respectiva en comercio al por menor y mayor
				\4 Importancia
				\4 Modelos teóricos
				\4 Oferta
				\4 Demanda interna
				\4 Demanda externa
			\3 Análisis dinámico
				\4 Evolución
				\4 Actualidad
				\4 Perspectivas
			\3 Política económica
		\2 Otros sectores
			\3 Joyas y bisutería
				\4 Delimitación
				\4[] CNAE Grupo 321
				\4 Importancia
				\4 Modelos teóricos
				\4 Oferta
				\4[] Córdoba
				\4[] $\to$ Joyería isabelina
				\4[] Madrid, Cataluña, Valencia
				\4[] $\to$ Joyería gama media y alta
				\4[] Galicia
				\4[] $\to$ Piedras preciosas
				\4 Demanda interna
				\4 Sector exterior
				\4[] Sector ligeramente superavitario
				\4 Delimitación
				\4 Evolución
				\4[] Contracción tras crisis financiera
				\4[] Consolidación de empresas
				\4 Actualidad
				\4 Perspectivas
				\4[] España 5º sector europeo de lujo
				\4[] $\to$ Tras Francia, Italia, Alemania, Suiza
				\4[] Ligado a sector turístico
				\4[] $\to$ Incertidumbre
			\3 Productos de limpieza
				\4 Delimitación
				\4[] Muy elevado número de empresas
				\4[] Ligeramente procíclico
				\4[] Diferenciación relativamente baja
				\4[] $\to$ Competencia en precios
				\4[] Productos
				\4[] $\to$ Limpiahogares
				\4[] $\to$ Limpiadores de cocina
				\4[] $\to$ Limpieza de WC
				\4[] $\to$ Detergentes ropa
				\4 Importancia
				\4[] Cuantitativamente pequeña
				\4[] Cualitativamente importante
				\4[] $\to$ Sector esencial servicios, industria, hogar
				\4 Modelos teóricos
				\4 Oferta
				\4[] Principales empresas
				\4[] $\to$ Henkel
				\4[] $\to$ Procter y Gamble
				\4[] $\to$ Unilever
				\4[] $\to$ Persán
				\4[] $\to$ ....
				\4[] Capital español mayoritario
				\4[] Grandes empresas capital extranjero
				\4 Demanda interna
				\4[] Detergentes para ropa
				\4[] $\to$ Primer sector en ingresos
				\4[] Demanda de nuevos formatos
				\4[] $\to$ Especialmente limpieza vía máquinas
				\4[] $\then$ Unificar varios productos en uno
				\4[] Concienciación sostenibilidad
				\4 Demanda externa
				\4 Delimitación
				\4 Evolución
				\4[] Crisis financiera
				\4[] $\to$ Desaparición de empresas
				\4[] $\to$ Consolidación
				\4[] Tendencia creciente desde 2016
				\4 Actualidad
				\4 Perspectivas
				\4[]
	\1[] \marcar{Conclusión}
		\2 Recapitulación
			\3 Industria agroalimentaria
			\3 Bienes de consumo tradicionales
		\2 Idea final
			
\end{esquemal}


\graficas

\conceptos

\preguntas

\notas


\bibliografia

Mirar en Palgrave:
\begin{itemize}
	\item 
\end{itemize}


FIAB (2018) \textit{Informe Económico 2018} \href{http://fiab.es/es/archivos/documentos/FIAB_INFORME_ECONOMICO_2018.pdf}{Disponible aquí} -- En carpeta del tema


\end{document}
