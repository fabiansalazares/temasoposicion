\documentclass{nuevotema}

\tema{3B-46}
\titulo{El Sistema Europeo de Bancos Centrales y la Política Monetaria de la Eurozona: objetivos e instrumentos. La gobernanza económica del euro. La Unión Bancaria y otros desarrollos.}

\begin{document}

\ideaclave

IMPORTANTE Leer \href{https://www.bde.es/f/webbde/SES/Secciones/Publicaciones/PublicacionesSeriadas/DocumentosOcasionales/20/Fich/do2021.pdf}{Banco de España (2020): Endeudamiento y necesidades de financiación en la Unión Europea}.

Ver \href{https://www.consilium.europa.eu/en/press/press-releases/2020/04/09/report-on-the-comprehensive-economic-policy-response-to-the-covid-19-pandemic/}{Comunicado del Eurogrupo del 10 de abril de 2020} sobre medidas a adoptar al respecto de la crisis del COVID-19.

Ver \url{http://blognewdeal.com/andrea-lucai/una-funcion-de-estabilizacion-fiscal-para-la-ue/} sobre función de estabilización fiscal a nivel europeo. 

PARA Propuestas de reforma (zona euro)
https://voxeu.org/article/euro-area-architecture-what-reforms-are-still-needed-and-why

PARA Diseño de la política monetaria de la zona euro, modelo EAGLE -- Utilizado en ECB para estimar efecto de shocks en regiones, así como la interacción entre la zona euro y el resto del mundo. -- https://www.ecb.europa.eu/pub/pdf/scpwps/ecbwp1195.pdf

La Zona Euro y más generalmente la Unión Europea es el resultado de un proceso de integración económica y monetaria entre países con características relativamente diversas. Para lograr esta integración, la Unión Europea se ha dotado de una serie de instituciones y políticas. La coordinación de políticas monetarias o la emisión de moneda común es el primer paso del proceso y se basa en la integración de los bancos centrales nacionales en una estructura institucional. Sin embargo, la política monetaria no puede ejercerse de forma multipolar. Es decir, la competencia para determinar la orientación de la política monetaria debe tener una sola voz, por lo que se trata de una competencia exclusiva del Eurosistema. Esta competencia se ejerce a través de políticas convencionales y no convencionales, cuya línea divisoria a nivel temporal se encuentra fundamentalmente en la crisis financiera de 2008, debido a las disrupciones en la transmisión de la política monetaria que provocó.

La política económica sigue siendo, sin embargo, competencia de los estados miembros. Pero una unión monetaria no es viable si no existe un cierto grado de coordinación en las políticas económicas. Por ejemplo, no es posible mantener una monea única si alguno de los estados miembros se endeuda excesivamente y amenaza con abandonar la moneda única para poder pagar a sus acreedores en una nueva moneda. Tampoco es posible si un país incurre en un enorme déficit de la balanza de pagos y en un momento dado necesita devaluar su moneda para restaurar el equilibrio. En este punto entra en juego la gobernanza económica de la Unión Europea, que trata de evitar estos desequilibrios con potencial para provocar una ruptura y optimizar el desempeño de la economía europea.

La gobernanza económica se centra principalmente en la actuación del sector público, pero las turbulencias en los sectores financieros tienen efectos en y se retroalimentan con las turbulencias en el sector real, incluido el público. Tras la crisis financiera, se hizo evidente que la inestabilidad financiera en una economía podría tener efectos graves sobre el conjunto de la unión. Es decir, existía un vínculo entre riesgo soberano y riesgo bancario, que era necesario eliminar. Para ello, surge la Unión Bancaria, que a pesar de no ser una unión completa, sí ha logrado avances. Paralelamente, se hace patente la necesidad de avanzar en la reestructuración del mercado de capitales europeo, de tal manera que se reduzca la dependencia de la financiación bancaria y se mejore la eficiencia del mercado de capital. 

\seccion{Preguntas clave}

\begin{itemize}
	\item ¿Qué es el Sistema Europeo de Bancos Centrales?
	\item ¿Qué instituciones diseñan e implementan la política monetaria de la zona euro?
	\item ¿Qué objetivos tiene la política monetaria de la eurozona?
	\item ¿Qué instrumentos se utilizan?
	\item ¿Qué es la gobernanza económica del euro?
	\item ¿En qué consiste?
	\item ¿Qué es la Unión Bancaria?
	\item ¿Qué retos enfrenta la Unión Económica y Monetaria?
\end{itemize}

\esquemacorto

\begin{esquema}[enumerate]
	\1[] \marcar{Introducción}
		\2 Contextualización
			\3 Unión Europea
			\3 Competencias de la UE
			\3 Mercado interior e integración económica
			\3 Integración monetaria: política monetaria común
			\3 Instituciones
			\3 Marco de gobernanza
		\2 Objeto
			\3 ¿Qué es el SEBC?
			\3 ¿Qué instituciones diseñan e implementan la PM de la Zona Euro?
			\3 Qué objetivos tiene la PM de la eurozona?
			\3 ¿Qué instrumentos se utilizan?
			\3 ¿Cómo se organiza la gobernanza económica del euro?
			\3 ¿Qué es la Unión Bancaria?
			\3 ¿Qué retos enfrenta la UEM?
		\2 Estructura
			\3 Sistema Europeo de Bancos Centrales
			\3 Política Monetaria de la Eurozona
			\3 Gobernanza Económica del Euro
			\3 Unión Bancaria
	\1 \marcar{Sistema Europeo de Bancos Centrales}
		\2 Idea clave
			\3 Contexto
			\3 Objetivos
			\3 Resultados
		\2 Banco Central Europeo
			\3 Función
			\3 Antecedentes
			\3 Organización
			\3 Actuaciones
		\2 Eurosistema
			\3 Función
			\3 Antecedentes
			\3 Organización
			\3 Actuaciones
			\3 Balance del Eurosistema
		\2 SEBC
			\3 Función
			\3 Antecedentes
			\3 Organización
			\3 Actuaciones
	\1 \marcar{Política monetaria de la Eurozona}
		\2 Idea clave
			\3 Concepto de política monetaria
			\3 Particularidades área del Euro
			\3 Eurosistema
		\2 Objetivos
			\3 Estabilidad de precios
			\3 Pilar económico
			\3 Pilar monetario
		\2 Instrumentos
			\3 Facilidad permanente
			\3 Operaciones de mercado abierto
			\3 Reservas mínimas
			\3 Transición Eonia a €STR
			\3 Programas de compra de activos
			\3 Forward guidance
	\1 \marcar{Gobernanza Económica del euro}
		\2 Justificación
			\3 Interacción entre política monetaria y otras políticas
			\3 División de responsabilidades
			\3 Tensión entre soberanía nacional y Unión Europea
			\3 Efectividad de PM depende de PF previsible
			\3[$\then$] Necesario marco de gobernanza económica europeo
		\2 Objetivos
			\3 Reducir riesgo de desestabilización
			\3 Reducir incertidumbre
			\3 Garantizar sostenibilidad exterior y fiscal
			\3 Mejorar integración financiera
			\3 Gestionar crisis financieras
		\2 Antecedentes
			\3 Tratado de Maastricht -- 92 $\to$ 93
			\3 Pacto de Estabilidad y Crecimiento -- 1997
			\3 Brazo Preventivo
			\3 Brazo Correctivo
			\3 Reforma del PEC -- 2005
			\3 Crisis financiera 2007-2010
			\3 Informe de Larosière (2009)
		\2 Marco jurídico
			\3 Pacto de Estabilidad y Crecimiento
			\3 Orientaciones Integradas
			\3 PNR -- Programa Nacional de Reformas
			\3 Programa de Estabilidad
			\3 Six Pack (2011)
			\3 TCSG -- Tratado de Estabilidad, Coordinación y Gobernanza (2012) 
			\3 Two Pack (2013)
		\2 Marco financiero
			\3 EFSM -- MEEF (2010)
			\3 EFSF -- FEEF (2010)
			\3 ESM -- MEDE (2012)
			\3 ESRB -- Junta Europea de Riesgo Sistémico (2010)
			\3 ESFS -- Sistema Europeo de Supervisión Financiera (2011)
			\3 Fondo Europeo de Reconstrucción
		\2 Actuaciones
			\3 Brazo Preventivo del PEC
			\3 Brazo Correctivo del PEC
			\3 Fiscal Compact (2012)
			\3 Proc. de Desequilibrios Macroeconómicos -- PDM (2011)
			\3 Semestre Europeo
			\3 Autoridad Fiscal Europea y IFIndependientes
		\2 Valoración
			\3 Mejoras pre-crisis financiera
			\3 Flexibilidad vs cumplimiento estricto
			\3 Críticas al PEC
			\3 Economía política
			\3 Trilema de Rodrik
		\2 Retos
			\3 Eurobonos y SBBS
			\3 Instrumento sobre Convergencia y Competitividad
			\3 Impuesto sobre transacciones financieras
			\3 Presupuesto de la Zona Euro
	\1 \marcar{Unión Bancaria}
		\2 Justificación
			\3 Ruptura de vínculos entre EEMM
			\3 Competencia perversa entre EEMM
			\3 Círculo vicioso deuda bancaria--soberana
			\3 Economía política
		\2 Objetivos
			\3 Romper círculo vicioso público-privado
			\3 Integrar mercados de capital en UE
			\3 Reglas transparentes de ayuda financiera
			\3 Aumentar estabilidad del sistema financiero
			\3 Reducir impacto de crisis
			\3 Reducir discriminación entre nacional y europeo
		\2 Antecedentes
			\3 Antes de crisis
			\3 Informe de Larosière (2009)
			\3 Sistema Europeo de Supervisión Financiera -- ESFS
			\3 Crisis de deuda soberana de 2012
			\3 Puesta en marcha de Unión Bancaria
		\2 Marco financiero
			\3 Fondo Único de Resolución -- SRF
			\3 MEDE -- ESM
			\3 Loan Facility Agreements
		\2 Marco jurídico
			\3 \underline{Single Rulebook}
			\3 CRR I y CRD IV
			\3 CRR II/CRD V (2019)
			\3 BRRD II/SRMR II (2019)
			\3 DGSD
			\3 IFR/IFD (2019)
		\2 Actuaciones
			\3 Miembros
			\3 SSM -- Mecanismo Único de Supervisión
			\3 SRM -- Mecanismo Único de Resolución
			\3 EDIS -- Garantía de depósitos
		\2 Valoración
			\3 Avances respecto a pre-crisis
			\3 Unión Bancaria incompleta
			\3 Próxima crisis
		\2 Retos
			\3 Fondo Europeo de Garantía de Depósitos -- EDIS
			\3 Salvaguarda fiscal/backstop fiscal
			\3 Riesgo moral
			\3 Sovereign Bond-Backed Securities -- SBBS 
			\3 Externalidades más allá de Z€
	\1 \marcar{Otros desarrollos}
		\2 Unión del mercado de capitales
			\3 Justificación
			\3 Objetivos
			\3 Actuaciones
			\3 EMIR -- European Market Infrastructure Regulation
			\3 TARGET 2-Securities
			\3 Revisión de Plan de Acción en 2017
			\3 Valoración
			\3 Propuestas
		\2 Unión fiscal
			\3 Justificación
			\3 Requisitos
			\3 Propuesta de presupuesto zona euro
	\1[] \marcar{Conclusión}
		\2 Recapitulación
			\3 Sistema Europeo de Bancos Centrales
			\3 Política Monetaria Europea
			\3 Gobernanza de la Zona Euro
			\3 Unión Bancaria
		\2 Idea final
			\3 Instrumentos de PM en próxima crisis
			\3 Informe cinco presidentes

\end{esquema}

\esquemalargo

\begin{esquemal}
	\1[] \marcar{Introducción}
		\2 Contextualización
			\3 Unión Europea
				\4 Institución supranacional ad-hoc
				\4[] Diferente de otras instituciones internacionales
				\4[] Medio camino entre:
				\4[] $\to$ Federación
				\4[] $\to$ Confederación
				\4[] $\to$ Alianza de estados-nación
				\4 Origen de la UE
				\4[] Tras dos guerras mundiales en tres décadas
				\4[] $\to$ Cientos de millones de muertos
				\4[] $\to$ Destrucción económica
				\4[] Marco de integración entre naciones y pueblos
				\4[] $\to$ Evitar nuevas guerras
				\4[] $\to$ Maximizar prosperidad económica
				\4[] $\to$ Frenar expansión soviética
				\4 Objetivos de la UE
				\4[] TUE -- Tratado de la Unión Europea
				\4[] $\to$ Primera versión: Maastricht 91 $\to$ 93
				\4[] $\to$ Última reforma: Lisboa 2007 $\to$ 2009
				\4[] Artículo 3
				\4[] $\to$ Promover la paz y el bienestar
				\4[] $\to$ Área de seguridad, paz y justicia s/ fronteras internas
				\4[] $\to$ Mercado interior
				\4[] $\to$ Crecimiento económico y estabilidad de precios
				\4[] $\to$ Economía social de mercado
				\4[] $\to$ Pleno empleo
				\4[] $\to$ Protección del medio ambiente
				\4[] $\to$ Diversidad cultural y lingüistica
				\4[] $\to$ Unión Económica y Monetaria con €
				\4[] $\to$ Promoción de valores europeos
				\4[$\to$] Objetivos de la UE
				\4[] Paz y bienestar a pueblos de Europa
			\3 Competencias de la UE
				\4 Tratado de la Unión Europea
				\4[] Atribución
				\4[] $\to$ Sólo las que estén atribuidas a la UE
				\4[] Subsidiariedad
				\4[] $\to$ Si no puede hacerse mejor por EEMM y regiones
				\4[] Proporcionalidad
				\4[] $\to$ Sólo en la medida de lo necesario para objetivos
				\4 Exclusivas
				\4[] i. Política comercial común
				\4[] ii. Política monetaria de la UEM
				\4[] iii. Unión Aduanera
				\4[] iv. Competencia para el mercado interior
				\4[] v. Conservación recursos biológicos en PPC
				\4 Compartidas
				\4[] i. Mercado interior
				\4[] ii. Política social
				\4[] iii. Cohesión económica, social y territorial
				\4[] iv. Agricultura y pesca \footnote{Salvo en lo relativo a la conservación de recursos biológicos marinos, que se trata de una competencia exclusiva de la UE}
				\4[] v. Medio ambiente
				\4[] vi. Protección del consumidor
				\4[] vii. Transporte
				\4[] viii. Redes Trans-Europeas
				\4[] ix. Energía
				\4[] x. Área de libertad, seguridad y justicia
				\4[] xi. Salud pública común en lo definido en TFUE
				\4 De apoyo
				\4[] Protección y mejora de la salud humana
				\4[] Industria
				\4[] Cultura
				\4[] Turismo
				\4[] Educación, formación profesional y juventud
				\4[] Protección civil
				\4[] Cooperación administrativa
				\4 Coordinación de políticas y competencias
				\4[] Política económica
				\4[] Políticas de empleo
				\4[] Política social
			\3 Mercado interior e integración económica
				\4 Objetivo intermedio de la UE
				\4[] $\to$ Mercado interior
				\4[] $\to$ Estabilidad de precios
				\4[] $\to$ Economía social de mercado
				\4[] $\to$ Crecimiento económico sostenible
				\4[] $\to$ Establecimiento de UEM con moneda única
			\3 Integración monetaria: política monetaria común
				\4 Objetivo de UE artículo 3
				\4 Elemento básico de integración europea
				\4[] $\to$ Euro
				\4[] $\then$ Política monetaria única
				\4 Complementario a integración económica
				\4[$\then$] Necesita instituciones y marco de gobernanza
			\3 Instituciones
				\4 SEBC
				\4 Eurosistema
				\4 Banco Central Europeo
				\4 ESMA, EBA, EIOPA
				\4 ESRB
				\4 ESFS
				\4 EFSF, EFSM, ESM
			\3 Marco de gobernanza
				\4 Pacto de Estabilidad y Crecimiento
				\4 Semestre Europeo
				\4 Informe de Convergencia
				\4 Recomendaciones sobre el Empleo
				\4 Tratado de Convergencia, Estabilidad y Crecimiento
				\4 Unión Bancaria
				\4 Unión del Mercado de Capitales
				\4 ...
		\2 Objeto
			\3 ¿Qué es el SEBC?
			\3 ¿Qué instituciones diseñan e implementan la PM de la Zona Euro?
			\3 Qué objetivos tiene la PM de la eurozona?
			\3 ¿Qué instrumentos se utilizan?
			\3 ¿Cómo se organiza la gobernanza económica del euro?
			\3 ¿Qué es la Unión Bancaria?
			\3 ¿Qué retos enfrenta la UEM?
		\2 Estructura
			\3 Sistema Europeo de Bancos Centrales
			\3 Política Monetaria de la Eurozona
			\3 Gobernanza Económica del Euro
			\3 Unión Bancaria
	\1 \marcar{Sistema Europeo de Bancos Centrales}
		\2 Idea clave
			\3 Contexto
				\4 Finales de años 80, primeros 90
				\4 SME duro
				\4[] Franja de fluctuación reducida
				\4[] Movilidad de capitales tras Maastricht
				\4 Bancos centrales nacionales
				\4[] Diferentes grados de independencia
				\4[] Diferentes configuraciones institucionales
				\4 UEM
				\4[] Creación de unión monetaria plenamente integrada
				\4[] Política monetaria única
				\4[] Mantener grado de descentralización
			\3 Objetivos
				\4 Crear sistema institucional
				\4[] Coherente
				\4[] Descentralizado en mercado > 400 M de habitantes
				\4 Marco de gobernanza de PM
			\3 Resultados
				\4 Integración de instituciones nacionales
				\4[] Institución encargada de PM
				\4[] $\to$ Una en cada país
				\4[] Unión Económica y Monetaria
				\4[] $\to$ Integración monetaria es requisito
				\4[] $\to$ Necesaria coordinación de actuaciones
				\4 Marco institucional de política monetaria
				\4[] Conjunto de instituciones
				\4[] $\to$ Diseñan PM
				\4[] $\to$ Ejecutan PM
				\4[] $\then$ Alcanzar objetivos
				\4 Estabilidad de precios
				\4[] TFUE.127.1
				\4[] Objetivo principal de PM
				\4[] Sistema Europeo de Bancos Centrales
				\4[] $\to$ Encargado de garantizar
				\4[] Eurosistema
				\4[] $\to$ PM en la zona euro
				\4 Tres instituciones fundamentales
				\4[] Banco Central Europeo
				\4[] Eurosistema
				\4[] Sistema Europeo de Bancos Centrales
				\4[$\to$] Órganos de gobernanza solapados
				\4[$\to$] Instituciones comunes
		\2 Banco Central Europeo
			\3 Función
				\4 Garantizar funciones SEBC y Eurosistema
				\4[] Definir y ejecutar PM del Euro
				\4[] Realizar operaciones de divisas
				\4[] Gestionar reservas oficiales de divisas
				\4[] Buen funcionamiento sistemas de pago
				\4 Gestionar base monetaria
				\4 Coordinar información
				\4 Cooperación europea e internacional
				\4 Definir política cambiaria
				\4[] Compartida con ECOFIN
			\3 Antecedentes
				\4 Instituto Monetario Europeo
				\4 Creado en 1998
			\3 Organización
				\4 Comité ejecutivo
				\4[] Miembros:
				\4[] -- Presidente
				\4[] -- Vicepresidente
				\4[] -- Cuatro vocales
				\4[] $\to$ Mandatos no renovables 8 años
				\4[] $\to$ Dedicación exclusiva
				\4[] $\to$ Consejo Europeo nombra por QMV
				\4[] $\to$ A propuesta de Consejo de la UE
				\4 Consejo de Gobierno
				\4[] Miembros de Comité Ejecutivo de BCE
				\4[] Gobernadores de BCN de EEMM de Z€
				\4[] Presidido por Presidente de BCE
				\4[] Pueden asistir sin voto
				\4[] $\to$ Presidente del Consejo
				\4[] $\to$ Miembro de la Comisión Europea
				\4 Decisión
				\4[] Mayoría simple en general
				\4[] Miembros de Comité Ejecutivo
				\4[] $\to$ 1 voto/miembro, en todas las votaciones
				\4[] Gobernadores de BCNs
				\4[] $\to$ Sólo 15 en total votan, rotatorios\footnote{\url{https://www.ecb.europa.eu/ecb/orga/decisions/govc/html/votingrights.en.html}}
				\4[] $\to$ No representan a sus EEMM
				\4[] $\then$ Búsqueda de independencia
				\4[$\then$] 21 votos en un momento dado
				\4[] 15 de BCNs
				\4[] 6 de BCE
				\4 Capital
				\4[] Procede de BCNs
				\4[] $\to$ Todos los EEMM
				\4[] $\to$ Desembolsado: BCNs de Z€
				\4[] $\to$ Otros BCNs: parcialmente desembolsado\footnote{Como contribución a los costes operativos del BCE en relación a participación en SEBC. 3,75\% del total del capital suscrito.}
				\4[] Clave de capital
				\4[] $\to$ Población
				\4[] $\to$ PIB
				\4[] $\then$ Ajuste cada 5 años
				\4[] España: 8,3\%
			\3 Actuaciones
				\4 Principios rectores
				\4[] Independencia
				\4[] $\to$ Personal
				\4[] $\to$ Funcional
				\4[] Transparencia
				\4[] $\to$ Informes trimestrales de actividad
				\4[] $\to$ Estados financieros consolidados
				\4 Liderazgo de Eurosistema
				\4 Autorizar emisión de monedas y billetes
				\4 Intervención junto con BCNs
				\4 Supervisión entidades junto con supervisores nacionales
		\2 Eurosistema
			\3 Función\footnote{\url{https://www.bde.es/bde/en/secciones/eurosistema/inst/funciones/Las_funciones_d_f8d9baee75d0441.html}}
				\4 Definir e implementar la política monetaria
				\4 Coordinar operaciones en divisas
				\4 Mantener reservas oficiales de divisas
				\4 Mantener funcionamiento de sistema de pagos
				\4 Autorizar emisión de monedas y billetes
				\4 Supervisar instituciones de crédito
				\4 Recoger estadísticas necesarias para funciones
				\4 Contribuir a supervisión macroprudencial
			\3 Antecedentes
				\4 = que Banco Central Europeo
			\3 Organización
				\4 Miembros
				\4[] BCE
				\4[] Bancos centrales de EEMM con euro
				\4[] $\to$ Coexiste con SECB hasta adopción completa de €
				\4 Consejo de Gobierno
				\4[] Miembros de Comité Ejecutivo de BCE
				\4[] Gobernadores de BCN de EEMM de Z€
				\4[] Presidido por Presidente de BCE
				\4[] Pueden asistir sin voto
				\4[] $\to$ Presidente del Consejo
				\4[] $\to$ Miembro de la Comisión Europea
				\4 Decisión
				\4[] Mayoría simple en general
				\4[] Miembros de Comité Ejecutivo
				\4[] $\to$ 1 voto/miembro, en todas las votaciones
				\4[] Gobernadores de BCNs
				\4[] $\to$ Sólo 15 en total votan, rotatorios\footnote{\url{https://www.ecb.europa.eu/ecb/orga/decisions/govc/html/votingrights.en.html}}
				\4[] $\to$ No representan a sus EEMM
				\4[] $\then$ Búsqueda de independencia
			\3 Actuaciones
				\4 Definir y ejecutar PM de Z€
				\4 Realizar operaciones de divisas
				\4 Poseer y gestionar reservas de EEMM
				\4 Garantizar funcionamiento sistemas de pago
				\4 Otras
				\4[] Estabilidad financiera
				\4[] Supervisión prudencial
				\4[] Elaboración de estadísticas
				\4[] Monedas y billetes por delegación BCE
				\4 Reuniones
				\4[] 10 veces al año al menos
				\4[] Primer jueves de cada mes
			\3 Balance del Eurosistema\footnote{Ver \href{https://www.ecb.europa.eu/press/pr/wfs/2020/html/index.en.html}{Consolidated financial statements of the Eurosystem}}
				\4 (junio de 2020)
				\4 Tamaño del balance
				\4[] Cercano a 6,4 billones de € (agosto 2020)
				\4[] $\to$ Fuerte expansión en contexto de crisis covid
				\4[] $\to$ 5,2 billones de € en abril 2020
				\4 Activos
				\4[] 500.000 M € en oro
				\4[] 400.000 M € en reservas
				\4[] 1.590.000 M € en préstamos a IFs de Z€
				\4[] $\to$ Mayoría en LTRO, VLTRO, TLTROs...
				\4[] 3.450.000 M € en activos de residentes denominados en €
				\4[] $\to$ Programas de compras de activos
				\4[] $\to$ Por razones de política monetaria
				\4[] 300.000 M de € en otros activos
				\4 Pasivos
				\4[] 1.300.000 M de € en monedas y billetes
				\4[] 2.800.000 M de € pasivos frente a ECrédito
				\4[] $\to$ Requerimientos de reservas mínimas mayoría
				\4[] $\to$ Facilidad de depósito
				\4[] 830.000 M de € pasivos frente a residentes no ECrédito
				\4[] 280.000 M de € otros pasivos
				\4[] 500.000 M de € reserva de revalorización
				\4[] 100.000 M de € capital y reservas
		\2 SEBC
			\3 Función
				\4 Coordinar PM de EEMM
				\4[] Funciones subsidiarias a Eurosistema
				\4[] $\to$ Países que no han adoptado €
				\4[] $\then$ Heredero de Instituto Monetario Europeo
				\4 Elaboración de informe anual del BCE
				\4 Recopilación de información estadística
				\4 Funciones consultivas del BCE
				\4 No es autoridad monetaria de Euro
				\4[] Eurosistema sí
				\4[] PM de EEMM no-euro
				\4[] $\to$ Mantienen compromiso de estabilidad precios
				\4 Adoptar fijación de tipos de cambio
			\3 Antecedentes
				\4 Origen en Instituto Monetario Europeo
				\4 Carácter provisional
				\4[] Hasta que todos EEMM adopten €
				\4[] $\to$ Coexiste con SEBC
			\3 Organización
				\4 Miembros
				\4[] BCE
				\4[] Bancos centrales nacionales de todos EEMM
				\4 Consejo General
				\4 Miembros con voto:
				\4[] Presidente de BCE
				\4[] Vicepresidente de BCE
				\4[] Gobernadores 27 BCNs
				\4 Pueden acudir sin voto:
				\4[] Presidente de Consejo Europeo
				\4[] Vocales del Consejo Ejecutivo de BCE
				\4[] Un miembro de Comisión Europea
			\3 Actuaciones
				\4 Preparación entrada en Euro
				\4[] Fijación de tipos de cambio
				\4 Asesorar BCE
				\4 Estadísticas
				\4 Asuntos relativos a capital BCE
				\4 Estandarización contabilidad
				\4 Preparación informe anual BCE
				\4 Reuniones trimestrales
	\1 \marcar{Política monetaria de la Eurozona}
		\2 Idea clave
			\3 Concepto de política monetaria
				\4 Decisiones tomadas por BCentral o AMonetaria
				\4[] Respecto a:
				\4[] $\to$ Cantidad de dinero en circulación
				\4[] $\to$ Condiciones de financiación
				\4[] $\to$ Provisión de liquidez
				\4[] $\to$ Condiciones futuras
				\4 Para alcanzar determinados objetivos macro
				\4[] $\to$ Crecimiento sostenido de producto real
				\4[] $\to$ Tasa de paro
				\4[] $\to$ Estabilidad de precios
			\3 Particularidades área del Euro
				\4 Países soberanos comparten moneda
				\4[] Poder individual de decisión limitado
				\4[] BCE es independiente
				\4[] $\to$ PM para conjunto de zona euro
				\4[] $\to$ Influencia EEMM limitada
				\4 Conjunto de economías heterogéneas
				\4[] Insuficiente convergencia entre EEMM
				\4[] $\to$ Más convergencia nominal que real
				\4[] Diferente política fiscal
				\4[] Diferente estructura de mercados
				\4 Falta de integración entre EEMM
				\4[] Movimiento de L y K limitado
				\4[] Sistemas financieros relativamente estancos
			\3 Eurosistema
				\4 Encargada de diseñar e implementar
				\4 Implementación descentralizada
				\4[] BC nacionales grueso de operaciones
				\4[] BCE papel coordinador
		\2 Objetivos
			\3 Estabilidad de precios
				\4 Objetivo principal
				\4 Marco jurídico
				\4[] TFUE.127
				\4[] Decisión del Consejo de Gobierno BCE
				\4[] $\to$ Menor al 2\%
				\4[] $\to$ En términos del IAPC de Z€
				\4[] $\to$ Mantenida a medio plazo
				\4 Justificación
				\4[] Evitar efectos nocivos de inflación
				\4[] Mantener margen respecto a deflación
				\4[] Evitar uso de PM para estabilización c/p
				\4[] IAPC para PM consistente en Z€
				\4[] $\to$ Admite diferenciales de inflación
				\4[] $\to$ Cesta representativa a nivel europeo
				\4 Marco de toma de decisiones
				\4[] ¿Cómo afectará PM a inflación?
				\4[] Basado en dos pilares
				\4[] i. Análisis económico
				\4[] ii. Análisis monetario
			\3 Pilar económico
				\4 Perspectiva corto y medio plazo
				\4 Precios como interacción oferta--demanda
				\4 Indicadores utilizados
				\4[] Evolución del PIB
				\4[] Demanda agregada
				\4[] Componentes de la demanda
				\4[] Política fiscal
				\4[] Mercados de capital y trabajo
				\4[] Indicadores de precios y costes
				\4[] Tipo de cambio
				\4[] Economía mundial y BP
				\4[] Mercados financieros
				\4[] Situación patrimonial
			\3 Pilar monetario\footnote{Ver Issing (2005) .}
				\4 Perspectiva de medio a largo plazo
				\4 Justificación
				\4[] Relación empírica M y $\pi$
				\4 Relación crecimiento monetario e inflación
				\4 Ancla nominal firme para PM
				\4 Contrastar pilar económico
				\4 Crecimiento monetario de referencia
				\4[] 4,5\% del M3
				\4[] $\to$ No es regla de crecimiento
				\4[] $\then$ Sólo es valor de referencia
				\4 Agregados del Eurosistema\footnote{Ver pág. 164 de ECB (2019) Manual on MFI statistics (en carpeta del tema). }
				\4[] Diferentes delimitaciones de ``oferta monetaria''
				\4[M0] Billetes y monedas en circulación
				\4[M1] = M0 + Depósitos a la vista
				\4[M2] = M1 + Dep. hasta 2 años +
				\4[] + Dep. redimibles con máximo de 3 meses antelación
				\4[$\to$] Gran variedad de modalidades
				\4[M3] = M2 + activos comercializables de c/p de IFs +
				\4[] + Deuda de hasta 2 años + repos + money market
				\4[] $\to$ Activos muy líquidos
				\4[] $\to$ Activos menos sustituibles en conjunto
				\4[] $\to$ Relativamente más estable
				\4[] $\to$ Activos vencimiento mayor no incluidos\footnote{Aunque cuando están próximos al vencimiento, pueden ser tratados como sustitutos.}
		\2 Instrumentos
			\3 Facilidad permanente
				\4 Objetivo
				\4[] Ofrecer y absorber liquidez 1 día
				\4[] Gestión descentralizada
				\4[] $\to$ Miembros del Eurosistema
				\4[] Controlar EONIA/€STR
				\4[] $\to$ Tipos máximos y mínimos
				\4[] $\to$ Controlar volatilidad
				\4 Facilidad marginal de crédito
				\4[] Ofrece liquidez a un día
				\4[] $\to$ Con activos de garantía
				\4[] $\to$ Sin restricciones cuantitativas
				\4[] $\then$ Límite superior del pasillo interbancario
				\4 Facilidad marginal de depósito
				\4[] Acepta depósitos a un día
				\4[] $\to$ Generalmente sin restricciones
				\4[] $\to$ Posibles tipos negativos\footnote{Clasificada como medida no convencional.}
				\4[] $\then$ Límite inferior pasillo interbancario
				\4[] Actualmente $-0.5\%$
				\4[] Sistema por tramos desde octubre de 2019\footnote{Ver \url{https://www.ecb.europa.eu/press/pr/date/2019/html/ecb.pr190912_2~a0b47cd62a.en.html}.}
				\4[] $\to$ No aplicable a depósitos voluntarios
				\4[] $\to$ Aplicable a reservas mínimas exigibles
				\4[] $\then$ Múltiplo\footnote{Inicialmente fijado en $6$.} de RMExigibles: exento de $-0.5\%$
				\4[] $\then$ Parte exenta: remunerada a $0\%$
			\3 Operaciones de mercado abierto\footnote{Ver \href{https://www.ecb.europa.eu/mopo/implement/omo/html/index.en.html}{BCE sobre operaciones de mercado abierto, incluyendo programas de compras de activos}.}
				\4 VLTRO -- Very Long Term RO
				\4[] Programa RO obsoletos
				\4[] $\to$ Sólo en 2 ocasiones
				\4[] $\to$ Diciembre 2012 y febrero 2012
				\4[] $\to$ Vencimiento a 3 años
				\4[] $\to$ Entendido como ``no convencional''
				\4 MRO -- Main Refinancing Operations
				\4[] $\to$ Semanalmente
				\4[] $\to$ Vencimiento a una semana
				\4[] $\to$ Cubrir grueso de necesidades financieras
				\4 LTRO -- Long Term RO
				\4[] $\to$ Varios tipos de operaciones
				\4[] $\to$ Periodicidad regular
				\4[] $\to$ Calendario prefijado
				\4[] $\to$ Vencimiento hasta tres meses
				\4 TLTRO -- Targeted LTRO
				\4[] Vencimiento 1--4 años
				\4[] Proveer de liquidez adicional
				\4[] TLTRO I: junio de 2014
				\4[] TLTRO II: 2016, marzo
				\4[] TLTRO III: sep 2019 a marzo 2021 (2 años)
				\4[] Condicionadas: préstamo no-financieras y hog.
				\4[] Entendido como ``no convencional''
				\4 TLTRO III
				\4[] Creado en junio de 2019
				\4[] Modificado en marzo-abril 2020
				\4[] Interés bajado al -0.5\%
				\4[] Posible bajada al -1\%
				\4[] $\to$ Para entidades que cumplan umbral de préstamos
				\4 PELTRO -- Pandemic Emergency Longer-Term RO\footnote{Ver \href{https://www.ecb.europa.eu/press/pr/date/2020/html/ecb.pr200430_1~477f400e39.en.html}{ECB Press Release 30 04 2020}}
				\4[] 7 operaciones de financiación l/p
				\4[] Vencimiento entre 16 y 8 meses
				\4[] Provisión de liquidez de emergencia a sistema financiero
				\4[] Interés -25bp sobre tipo MRO (-0,25\% en abril 2020)
				\4[] Reglas de colateral a prestar reducidas
				\4 Fine Tuning/Operaciones de ajuste
				\4[] $\to$ Operaciones ad-hoc
				\4[] $\to$ Inyección y absorción de liquidez
				\4[] $\to$ Subastas estándar y operaciones bilaterales
				\4[] $\to$ Manejar picos y valles de liquidez
				\4[] $\to$ Estabilizar tipos de interés
				\4 Operaciones estructurales
				\4[] $\to$ Ad-hoc como fine-tuning
				\4[] $\to$ Ajustar posición de liquidez estructural
				\4[] $\to$ Sin periodicidad ni vencimiento fijo
				\4 Instrumentos financieros
				\4[] Reverse transactions
				\4[] $\to$ Repos
				\4[] $\to$ Préstamo colateralizado
				\4[] Outright transactions/Compras directas
				\4[] Emisión de deuda
				\4[] Swaps de tipo de cambio/FX swaps
				\4[] Toma de depósitos a plazo fijo
			\3 Reservas mínimas\footnote{\textit{Minimum Reserve Requirements}. Ver \href{https://www.bundesbank.de/en/tasks/monetary-policy/minimum-reserves/minimum-reserves-625912}{Bundesbank (2020) sobre ratio de reservas mínimas en Eurosistema.}}
				\4 Concepto
				\4[] Porcentaje mínimo aplicado a base de reservas
				\4[] $\to$ Mantener en cuentas corrientes con BCN de Z€
				\4[] $\to$ Remuneradas a tipo marginal MRO
				\4[] Base de reservas
				\4[] $\to$ Depósitos a la vista
				\4[] $\to$ Depósitos a plazo hasta 2 años
				\4[] $\to$ Papel comercial
				\4[] $\to$ Deuda hasta 2 años
				\4[] $\to$ Excluida deuda con otras ent. crédito en €Sistema
				\4[] $\to$ Excluida deuda con BCE y otros BCNs
				\4 Objetivo
				\4[] Estabilizar tipo de interés del mercado monetario
				\4[] Alterar condiciones estructurales de liquidez
				\4 Cumplimiento de requisito mínimo
				\4[] Permite acceso a:
				\4[] $\to$ Facilidades permanentes
				\4[] $\to$ Operaciones de mercado abierto
				\4 Ratio mínimo de reservas actualmente
				\4[] 1\% de base de reservas requerida
				\4[] $\to$ Desde 2012, anteriormente 2\%
			\3 Transición Eonia a €STR\footnote{Pronunciado ``ester''}
				\4 Ver \url{https://www.ecb.europa.eu/pub/economic-bulletin/focus/2019/html/ecb.ebbox201907_01~b4d59ec4ee.en.html}
				\4 Ver \url{https://blog.bankinter.com/economia/-/noticia/2019/8/8/que-estr-tipo-interes-corto-plazo-euro}
				\4 Ver \url{https://www.bde.es/f/webbde/INF/MenuHorizontal/Publicaciones/Boletines\%20y\%20revistas/InformedeEstabilidadFinanciera/ief_2019_1_Rec2_1.pdf}
			\3 Programas de compra de activos
				\4 Concepto
				\4[] Expansión del balance del BCE
				\4[] $\to$ Comprando activos de diferentes clases
				\4[] $\to$ Mercados secundarios de deuda pública
				\4[] Relacionado con estímulos post-crisis
				\4[] $\to$ Clasificado como no convencional
				\4[] $\to$ Más credit easing que quantitative easing
				\4 Objetivos
				\4[] Distinguibles dos fases
				\4[] 1. Pre-2014
				\4[] $\to$ Reparar mecanismos de transmisión PM
				\4[] $\to$ Ruptura interés oficial y privado
				\4[] 2. Post-junio de 2014
				\4[] $\to$ Evitar deflación
				\4[] $\to$ Reforzar posición acomodaticia
				\4 SMP -- Securities Market Programme
				\4[] 2010--Mayo
				\4[] Compra de deuda pública soberana de EEMM
				\4[] Principalmente ITA, ESP, POR, IRL, GRE
				\4[] Tensiones severas en mercados financieros
				\4[] Hasta 2012
				\4[] $\to$ OMT sustituye en finalidad
				\4[] Activos SMP mantenidos hasta vencimiento
				\4[] Liquidez inyectada se esterilizó
				\4[] $\to$ Esterilización suspendida en 2014\footnote{Ver página 21 de CECO Nuevo.}
				\4 CBPP1 -- Covered Bond Purchase Programme 1
				\4[] 2009--Julio hasta 2010--Junio
				\4[] 60.000 millones hasta vencimiento
				\4[] Compra de covered bonds
				\4[] Liquidez inyectada se esterilizó
				\4[] $\to$ Esterilización suspendida en 2014
				\4 CBPP2 -- Covered Bond Purchase Programme 2
				\4[] 2011--Noviembre hasta 2012--Octubre
				\4[] Liquidez inyectada se esterilizó
				\4[] $\to$ Esterilización suspendida en 2014
				\4 \underline{Asset Purchase Programme -- APP}
				\4[] Anunciado en 2014--septiembre
				\4[] Programas finalizados en diciembre 2018
				\4[] $\to$ Sin más compras netas
				\4[] Cuantías variables mensuales:
				\4[] $\to$ 60.000 M € entre marzo 2015-marzo 2016
				\4[] $\to$ 80.000 M € entre 2016 y 2017
				\4[] $\to$ Reducción progresiva hasta 15.000 M en últimos meses 2018
				\4[] Programas de compras vuelven desde 1 nov 2019
				\4[] $\to$ 20.000 M mensuales hasta que sea necesario
				\4[] Objetivos
				\4[] $\to$ Mejorar condiciones de financiación
				\4[] $\to$ Restablecer crédito al sector privado
				\4[] Devolución de principales se reinvierte
				\4[] Operativa
				\4[] $\to$ +90\% por BC Nacionales
				\4[] $\to$ Capital de BCN en BCE determina cuantía
				\4[] Incluye programas siguientes
				\4 CBPP3 -- Covered Bond Purchase Programme 3
				\4[] 2014--Octubre hasta 2018--Diciembre
				\4[] 260.000 millones
				\4[] Mercados primario y secundiario
				\4[] Activos susceptibles de ser comprados
				\4[] i. Cumplir criterios garantías de BCE
				\4[] ii. Calificación mínima BBB-
				\4[] iii. Tenencias <70\% de una emisión
				\4[] iv. Liquidados en Z€ y denominados en €
				\4[] Impacto
				\4[] $\to$ Depende de si emisión retenida o no\footnote{Es decir, dependiendo de si se compran a un inversor en el mercado secundario que había retenido los activos para su venta posterior, o directamente del emisor.}
				\4[] $\to$ Se buscan ciclos virtuosos de crédito
				\4[] $\to$ Valoración de impacto discutida
				\4 ABSPP -- Asset-Backed Securites Programme
				\4[] 2014--Noviembre hasta 2018--Diciembre
				\4[] Compra de activos Asset-Backed
				\4[] Mercados primario y secundario
				\4[] Bonos retenidos y no retenidos
				\4[] Límite cuota de emisión:
				\4[] $\to$ 70\% en general
				\4 \underline{Expanded APP -- EAPP}
				\4[] Comparado con QE de BoJ, BoE, Fed
				\4[] Enero de 2015
				\4[] Tenencias de deuda siempre inferiores a:
				\4[] $\to$ 30\% de stock de deuda nacional
				\4[] $\to$ 25\% de una emisión determinada
				\4[] Reparto entre BCNs en base a:
				\4[] $\to$ Cuota de capital en BCE
				\4[] $\then$ Banco de España: 8,34\% del capital\footnote{\url{https://www.ecb.europa.eu/ecb/orga/capital/html/index.en.html}.}
				\4 PSPP -- Public Sector Purchase Programme
				\4[] Algo más de 2 billones (españoles)
				\4[] 2015--Enero hasta 2018--Diciembre
				\4[] Deuda de:
				\4[] $\to$ Gobiernos centrales Z€
				\4[] $\to$ Agencias
				\4[] $\to$ Instituciones europeas
				\4[] $\to$ Gobiernos regionales y locales
				\4[] Límite cuota de emisión de 50\%
				\4[] Mercado secundario exclusivamente
				\4 CSPP -- Corporate Sector Purchase Programme
				\4[] 2016--Marzo hasta 2018--Diciembre
				\4[] 180.000 millones
				\4[] Mercados primario y secundario
				\4 OMT -- Outright Monetary Transactions
				\4[] Tras whatever it takes
				\4[] Octubre de 2012
				\4[] Disposición a comprar deuda soberana
				\4[] Bajo contexto del MEDE
				\4[] $\to$ Acuerdo intergubernamental
				\4[] $\to$ Estrictamente, no UE
				\4[] Sujeto a condicionalidad
				\4 PEPP -- Pandemic Emergency Pucharse Program\footnote{Ver \href{https://www.ecb.europa.eu/mopo/implement/pepp/html/index.en.html}{ECB sobre PEPP} y \href{Decisión del BCE de 24 de marzo de 2020}{https://www.ecb.europa.eu/ecb/legal/pdf/celex\_32020d0440\_en\_txt.pdf}}
				\4[] Garantizar transmisión de política monetaria
				\4[] Condiciones excepcionales
				\4[] 750.000 M de € en marzo de 2020
				\4[] +600.00 M de € en junio de 2020
				\4[] $\then$ 1.350.000 M de € en total
				\4[] Todo tipo de activos
				\4[] $\to$ Deuda pública
				\4[] $\to$ Bonos corporativos
				\4[] $\to$ Covered bonds
				\4[] $\to$ ABS
				\4[] Vencimiento entre dos meses y 30 años
				\4[] Permitida deuda pública griega
				\4[] $\to$ Restringido en otras operaciones
				\4[] Clave de capital es guía para stock total de activos
				\4[] $\to$ Pero compras realizadas de manera flexible
				\4[] Mínimo hasta junio de 2021
				\4[] $\to$ Extensible hasta fin de crisis Covid
			\3 Forward guidance
				\4 Comunicación de Bancos Centrales
				\4[] Desde 80s, habitual:
				\4[] $\to$ Condiciones económicas
				\4[] $\to$ Proyecciones utilizadas por BCentral
				\4[] $\to$ Análisis de políticas
				\4[] $\to$ Objetivos explícitos de tipos de interés
				\4 Objetivo
				\4[] Transmitir senda de interés c/p futura
				\4[] Comprometer PM futura de BCE
				\4[] Reducir riesgo de duración activos l/p
				\4 Diferentes referencias posibles
				\4[] $\to$ Calendario
				\4[] $\to$ Cualitativa
				\4[] $\to$ Cuantitativa
				\4[] $\to$ Resultados
				\4 Utilización en crisis financiera
				\4[] Fijar expectativas de agentes
				\4[] Tipos seguirán bajos más allá de recesión
				\4 Utilización por BC
				\4[] UE
				\4[] $\to$ Julio de 2012: ``whatever it takes''
				\4[] $\then$ Tanto estímulo como necesario
				\4[] $\to$ Noviembre 2013:
				\4[] $\then$ Compromiso reducción de tipos
				\4[] $\to$ Tipos seguirán bajos hasta $\uparrow$ inflación
				\4[] $\then$ 2012
				\4[] $\to$ Otros ejemplos tras crisis
				\4 Críticas
				\4[] Puede revelar pesimismo mayor del esperado
				\4[] Sensibilidad decreciente a shocks de información
	\1 \marcar{Gobernanza Económica del euro}
		\2 Justificación
			\3 Interacción entre política monetaria y otras políticas
				\4 PM es indivisible
				\4[] Igual para toda UEM
				\4[] $\to$ No sirve para shock idiosincrático en país dado
				\4 Efectos de PM dependen de otras políticas
				\4[] Tono de la política fiscal
				\4[] Estabilizadores automáticos
				\4[] Reformas estructurales
				\4[] $\then$ Modulan efectividad de PM
				\4 PM como herramienta de estabilización
				\4[] Con soberanía monetaria
				\4[] $\to$ Países disponen de herramienta
				\4[] Tras integración monetaria
				\4[] $\to$ PM no sirve para estabilizar shocks regionales
				\4 Potencial uso de PF para estabilizar
				\4[] Puede desestabilizar funcionamiento de AMonetaria
				\4[] $\to$ Miembros A€ endeudados en ``divisa''
				\4[] $\then$ Necesario evitar uso  desestabilizador
			\3 División de responsabilidades
				\4 Competencia sobre tributación
				\4[] En manos de EEMM
				\4[] $\to$ Apenas armonización tributaria
				\4 Presupuestos nacionales
				\4[] Cuantía mucho mayor que presupuesto europeo
				\4[] Parlamentos nacionales aprueban
				\4[] Gobiernos nacionales ejecutan
				\4[] UE no tiene control directo
				\4[] Mercados y gobiernos asumen bail-out implícito
				\4[] $\to$ Posible factor de destabilización
			\3 Tensión entre soberanía nacional y Unión Europea
				\4 Pertenencia a UEM restringe políticas nacionales
				\4[] Incentivo a utilizar UE como chivo expiatorio
				\4[] Coacción UE--EEMM es difusa
				\4[] $\to$ Tensiones políticas EEMM--UE
				\4[] $\then$ Necesario definir marco de PE nacional aceptable
				\4[] $\then$ PE nacional desvía objetivos
			\3 Efectividad de PM depende de PF previsible
				\4 Diseño de PM requiere información
				\4[] Estado del sistema financiero
				\4[] Evolución variables reales
				\4[] Lags de implementación y efectos
				\4[] Senda de PF que se implementará
				\4 Necesario marco de PE nacional
				\4[] Para anclar expectativas de BCE sobre déficit
				\4[] $\to$ Cumplir objetivos de política monetaria
			\3[$\then$] Necesario marco de gobernanza económica europeo
		\2 Objetivos
			\3 Reducir riesgo de desestabilización
				\4 Evitando EM se comporte irresponsablemente
			\3 Reducir incertidumbre
				\4 PF y reformas previsibles
				\4 Fijar proceso de toma de decisión
				\4 Sincronizar decisiones de política económica
			\3 Garantizar sostenibilidad exterior y fiscal
				\4 Reducir riesgo de impago deuda soberana
				\4 Evitar ajustes bruscos balanza de pagos, PF
				\4 Evitar acumulación excesiva de desequilibrios
			\3 Mejorar integración financiera
				\4 Integración financiera insuficiente
				\4[] Puede generar feedbacks desestabilizantes
				\4[] Ejemplo:
				\4[] $\to$ Integración aumenta inversión en cartera
				\4[] $\to$ Integración incompleta vincula riesgo bancario a soberano
				\4[] $\then$ Shock mercado inmobiliario precipita crisis deuda soberana
			\3 Gestionar crisis financieras
				\4 Contener efectos sistémicos de crisis nacionales
				\4 Aumentar transparencia de políticas ante crisis
		\2 Antecedentes
			\3 Tratado de Maastricht -- 92 $\to$ 93
				\4 Preparación para UEM
				\4 Introducción de límites finanzas públicas
				\4[] ``5 criterios de convergencia''
				\4[] i. 60\% de deuda
				\4[] ii. 3\% déficit público
				\4[] iii. $\pi <$ media de 3 países con menor $\pi$ + 1,5\%
				\4[] iv. Estabilidad cambiaria
				\4[] \quad Sin devaluar en dos años previos
				\4[] \quad Participación en ERM II durante dos años seguidos
				\4[] v. Interés de largo plazo (bonos 10 años)
				\4[] \quad $i$ > media de 3 miembros con menor $\pi$ + $2\%$
				\4[] $\then$ Permitir estabilidad de moneda única
				\4[] $\then$ Aún en funcionamiento para candidatos
			\3 Pacto de Estabilidad y Crecimiento -- 1997
				\4 Concretar límites déficit y deuda
				\4 Reforzar supervisión y cooperación fiscal
				\4 Engloba brazos preventivo y correctivo
			\3 Brazo Preventivo
				\4 Entra en vigor en 1998
			\3 Brazo Correctivo
				\4 Entra en vigor en 1999
			\3 Reforma del PEC -- 2005\footnote{Ver \url{https://ec.europa.eu/commission/presscorner/detail/en/IP_05_798}}
				\4 Presión de Alemania y Francia
				\4[] Habían tenido proc. déficit en 2003
				\4 Comisión Europea
				\4[] Flexibilizar reglas
				\4[] Hacer más fácil la ejecución
				\4 Cambios principales
				\4[] 3\% y 60\% se mantienen
				\4[] Déficit Excesivo depende de nuevos parámetros
				\4[] $\to$ Saldo estructural se valora
				\4[] $\to$ Nivel de deuda
				\4[] $\to$ Duración periodo de crecimiento bajo
				\4[] $\to$ Reformas para mejorar productividad
				\4[] $\to$ Condiciones ``excepcionales''
				\4 Brazo preventivo
				\4[] Introduce MTO calculado para cada país
			\3 Crisis financiera 2007-2010
				\4 Carencias gubernativas se evidencian en PEC
				\4[] i. Saldo estructural sobreestima EFiscal
				\4[] ii. Objetivo de deuda no se considera
				\4[] $\to$ Sólo déficit en supervisión
				\4[] iii. Falta de asesoramiento sobre cumplimiento PEC
				\4[] $\to$ EEMM elaboraban independientemente
				\4[] iv. Sin mecanismos que aseguren efectividad
				\4[] $\to$ Carácter discrecional de reglas
			\3 Informe de Larosière (2009)
				\4 2009
				\4 Recomendaciones en contexto de crisis financiera
				\4[] Regulación financiera
				\4[] Supervisión financiera
				\4[] Reducir fragmentación regulatoria
				\4[] Establecer agenda de reformas
				\4[] Implementar mecanismos de gestión de crisis
				\4 Propuestas
				\4[] Creación de EFSB
				\4[] $\to$ Organizar información sobre riesgos financieros
				\4[] Creación de sistema de supervisión
		\2 Marco jurídico
			\3 Pacto de Estabilidad y Crecimiento
				\4 Artículos 121-126 TFUE
				\4[] PEC en general
				\4 Artículo 136 TFUE
				\4[] Para zona euro
				\4 Múltiples reglamentos y directivas
				\4 Código de Conducta
				\4[] Acto atípico de la UE
				\4[] Define más claramente:
				\4[] $\to$ Alcance
				\4[] $\to$ Contenidos
				\4[] $\to$ Objetivos
				\4[] $\then$ Entendimiento común de los EEMM
			\3 Orientaciones Integradas
				\4 Artículos 121 y 148 de TFUE
				\4[] ``Integrated Guidelines''
				\4[] $\to$ A partir de 2005
				\4 Dos componentes
				\4[] $\to$ Orientaciones Generales de Política Económica\footnote{\textit{Broad Economic Policy Guidelines}.}
				\4[] $\to$ Orientaciones para las Políticas de Empleo
				\4 Recomendación del Consejo sobre OGPE
				\4[] Duración indefinida
				\4 Decisión del Consejo sobre Estrategia de Empleo
				\4[] Aprobación anual
				\4 Importancia
				\4[] Limitada en la actualidad
			\3 PNR -- Programa Nacional de Reformas
				\4 Basado en Integrated Guidelines
				\4 Anualmente
				\4 A presentar en abril junto con PEstabilidad
			\3 Programa de Estabilidad
				\4 Anualmente
				\4 Plan de presupuestario para 3 años
				\4 A presentar en abril junto con PNR
			\3 Six Pack (2011)
				\4 Entran en vigor en diciembre de 2011
				\4 6 medidas de reforma
				\4 Política fiscal
				\4[] 3 reglamentos y 1 directiva
				\4[] Pilar preventivo
				\4[] $\to$ Introduce regla de gasto
				\4[] $\to$ Introduce sanciones del pilar preventivo
				\4[] Pilar correctivo
				\4[] $\to$ Criterio de la deuda se operacionaliza
				\4[] $\to$ Incremento y graduación de sanciones
				\4[] $\to$ QMV inversa para aprobar sanciones\footnote{Necesaria QMV para rechazar imposición de sanciones, lo que aumenta el automatismo del procedimiento.}
				\4 Desequilibrios macroeconómicos
				\4[] 2 reglamentos
				\4[] Procedimiento de identificación y corrección
				\4[] Posibilidad de sancionar incumplimientos
			\3 TCSG -- Tratado de Estabilidad, Coordinación y Gobernanza (2012) \footnote{\textit{Treaty on Stability, Coordination and Governance.} En ocasiones también ``tratado de estabilidad fiscal''.} en la UEM
				\4 Firmado en 2012
				\4 Entrada en vigor en 2013
				\4 Incluye Fiscal Compact
				\4[] Obligación de implementar equilibrio fiscal
				\4[] $\to$ En legislación nacional
				\4[] No todos firmantes de TCG aceptan FCompact
			\3 Two Pack (2013)\footnote{Ver \href{https://ec.europa.eu/commission/presscorner/detail/en/MEMO_13_196}{EC (2013): presentación del Two Pack.}}
				\4 Entran en vigor en mayo de 2013
				\4 Dos reglamentos
				\4 Mejorar coordinación de proceso presupuestario
				\4[] Aumenta frecuencia de escrutinio europeo
				\4[] $\to$ Países con problemas o PDExcesivo
				\4[] $\then$ Informes trimestrales
				\4[] Calendario presupuestario común
				\4[] 1. Presentación borrador presupuestario
				\4[] $\to$ Antes del 15 de octubre
				\4[] $\to$ Obligación general
				\4[] 2. Planes presupuestarios a medio plazo
				\4[] $\to$ Antes del 30 de abril
				\4[] $\to$ Programas de estabilidad
				\4[] $\to$ Programas nacionales de reforma
				\4[] $\to$ Incentivar debate nacional sobre déficit
				\4 Incorpora a derecho UE partes de TCEG
				\4[] $\to$ Creación de entidades independientes
				\4[] $\to$ Monitorizar cumplimiento reglas numéricas
				\4[] $\to$ Deber de informar sobre emisión de deuda
		\2 Marco financiero
			\3 EFSM -- MEEF (2010)
				\4 European Financial Stabilisation Mechanism
				\4 Creado en mayo de 2010
				\4 Función
				\4[] Prestar ayuda miembros UE
				\4 Organización
				\4[] Bonos emitidos por CE
				\4[] Respaldo por presupuesto UE
				\4 Actuaciones
				\4[] Programas de asistencia
				\4[] IRL, POR, GRE (2015)
				\4[] ~48.000 millones de euros
				\4 Valoración
			\3 EFSF -- FEEF (2010)
				\4 European Financial Stability Facility
				\4 Creado en junio de 2010
				\4 Funciones
				\4[] Prestar ayuda miembros de área del Euro
				\4 Organización
				\4[] Vehículo especial
				\4[] Sociedad Anónima sede en Luxemburgo
				\4[] Muy elevada capacidad de ayuda
				\4[] $\to$ Hasta 750.000 millones
				\4[] Sin garantía presupuestaria
				\4 Actuaciones
				\4[] Ayuda GRE, IRL, POR
				\4 Valoración
			\3 ESM -- MEDE (2012)
				\4 European Stabilisation Mechanism
				\4 Creado en octubre de 2012
				\4 Función
				\4[] Sustituir EFSF y EFSM
				\4[] Proveer asistencia financiera
				\4[] Señalizar voluntad de estabilizar crisis
				\4 Organización
				\4[] Consejo de Gobernadores
				\4[] $\to$ Ministros de finanzas
				\4[] Decisión
				\4[] $\to$ Unanimidad
				\4[] $\then$ Ayuda financiera, decisiones trascendentes
				\4[] $\to$ QMV de 80\% por contribución a capital
				\4[] Acreedor preferente por detrás FMI
				\4[] Capital autorizado 700.000 millones
				\4[] Capital desembolsado: 80.000 millones
				\4[] Capacidad de préstamo: 500.000 millones
				\4 Actuaciones
				\4[] $\to$ Requiere beneficiario aprueba Fiscal Compact
				\4[] $\to$ Condiciones de mercado + margen
				\4[] $\to$ Condicionalidad estricta
				\4[] $\to$ CE+BCE+FMI supervisan cumplimento
				\4[] Préstamos a países
				\4[] $\to$ Países sin acceso efectivo a mercados
				\4[] $\to$ Acreedor preferente tras FMI
				\4[] Compras de bonos de EM
				\4[] $\to$ Outright Monetary Transactions
				\4[] $\to$ Mercados primarios y secundarios
				\4[] Líneas de crédito precautorias
				\4[] $\to$ Similares FCL y PCL
				\4[] Recapitalizaciones de instituciones financieras
				\4[] $\to$ Directas
				\4[] $\then$ No implica aumento nivel deuda de EM
				\4[] $\to$ A través de préstamos a gobiernos
				\4 Valoración
				\4[] Propuesta de reforma
				\4[] Conversión en Fondo Monetario Europeo
				\4[] $\to$ Introducir en acervo europeo
				\4[] $\to$ Decisiones por QMV
			\3 ESRB -- Junta Europea de Riesgo Sistémico (2010)
				\4 European Systemic Risk Board
				\4 Fundado en 2010
				\4[] Parte de ESFS
				\4[] Propuesto por de Larosière
				\4 Funciones
				\4[] Supervisión macroprudencial
				\4[] Prevención y mitigación riesgo sistémico
				\4[] Ámbito competencial
				\4[] $\to$ Crédito
				\4[] $\to$ Seguros
				\4[] $\to$ Shadow banking
				\4[] $\to$ Infraestructuras financieras
				\4[] $\to$ Otras instituciones
				\4 Organización
				\4[] Junta General
				\4[] $\to$ Presidente + VPresidente de ECB
				\4[] $\to$ Gobernadores de BCNs
				\4[] $\to$ 1 Miembro de la CE
				\4[] $\to$ Presidente de EBA
				\4[] $\to$ Presidente de EIOPA
				\4[] $\to$ Presidente de ESMA
				\4[] $\to$ Presidente+ 2 VPresidente de Comité Científico
				\4[] $\to$ Presidente de Comité Técnico Consultivo
				\4[] $\to$ Sin voto, representantes de supervisores nacionales
				\4[] $\to$ Sin voto, gobernadores países EEE
				\4[] Comité Director
				\4[] $\to$ Asistencia a Junta General
				\4[] Comité Técnico Consultivo
				\4[] $\to$ Asesoramiento y asistencia
				\4[] Comité Científico Consultivo
				\4[] $\to$ Investigación macroprudencial
				\4[] Secretaría
				\4[] $\to$ Gestión diaria y apoyo administrativo
			\3 ESFS -- Sistema Europeo de Supervisión Financiera (2011)
				\4 European System of Financial Supervision
				\4 Fundado en 2011
				\4 Miembros
				\4[] Autoridades de Supervisión Europea
				\4[] $\to$ EBA\footnote{\textit{European Banking Authority}.}
				\4[] $\to$ ESMA\footnote{\textit{European Securities and Markets Authority}.}
				\4[] $\to$ EIOPA\footnote{\textit{European Insurance and Occupational Pensions Authority}.}
				\4[] ESRB
				\4[] Supervisores Nacionales
			\3 Fondo Europeo de Reconstrucción
		\2 Actuaciones
			\3 Brazo Preventivo del PEC\footnote{Ver Assessment de 2018 de la Comisión Europea sobre España como ejemplo de análisis del Programa de Estabilidad y el grado de cumplimiento de la senda de ajuste hacia el MTO y otros temas relacionados.}
				\4 Asegurar política fiscal prudente
				\4[] Evitar entrar en ámbito de brazo correctivo
				\4 Valoración de cumplimiento de objetivos
				\4[] Una vez al año, para todos
				\4[] Dos veces al año, para zona €
				\4[] $\to$ Saldo estructural
				\4[] $\to$ Techo de gasto
				\4 Objetivos de Medio Plazo\footnote{\textit{Medium-Term Budgetary Objetives} (MTOs).}
				\4[] Específicos de cada país
				\4[] $\to$ Cada país elige sujeto a restricción
				\4[] $\to$ Trata de garantizar:
				\4[] \quad i) Margen de seguridad respecto a 3\%
				\4[] \quad ii) Sostenibilidad deuda pública
				\4[] \quad iii) Margen para estabilización cíclico
				\4[] Saldo estructural mínimo que sirve como MTO\footnote{Definido como el saldo cíclicamente ajustado menos medidas one-off, sobre \% del PIB.}
				\4[] $\to$ -1\% en general
				\4[] $\to$ -0.5 si firmante TCSG\footnote{Todos los miembros de la zona euro lo han firmado.}
				\4[] $\to$ Necesario estar por encima
				\4[] $\to$ Necesario ajustarse para alcanzar
				\4[] Senda de ajuste anual hacia MTO
				\4[] $\to$ Reducción del 0.5\% de saldo estructural
				\4[] Desviaciones permitidas de MTO o senda de ajuste
				\4[] i) Reformas estructurales profundas
				\4[] $\to$ Incluye algunas inversiones públicas
				\4[] $\to$ Especialmente, si crecimiento bajo potencial
				\4[] $\to$ Incluye inversiones europeas
				\4[] ii) Eventos inusuales fuerza mayor
				\4[] iii) Recesión severa en zona euro o UE
				\4 Programas de Estabilidad/Convergencia\footnote{``Programas de Convergencia para'' en el caso de EEMM que no son miembros del Área del Euro.}
				\4[] Fecha límite de presentación
				\4[] $\to$ 30 de abril
				\4[] Periodos de referencia
				\4[] $\to$ De $t-1$ hasta $t+3$
				\4[] Incluye:
				\4[] $\to$ MTO y senda de ajuste
				\4[] $\to$ Supuestos económicos subyacentes
				\4[] $\to$ Análisis de impacto de reformas
				\4[] $\to$ Justificación por no haber logrado objetivos
				\4[] $\then$ Evaluación cumplimiento ex-ante y ex-post
				\4[] Evaluación por Comisión y Consejo
				\4[] $\to$ ¿Se encuentra en MTO?
				\4[] $\to$ ¿Senda de ajuste hacia MTO?
				\4[] $\to$ ¿Cumple esfuerzo estructural?
				\4[] $\to$ ¿Cumple techo de gasto?
				\4 Techo de gasto\footnote{Introducido en six-pack.}
				\4[] Garantizar financiación de incremento de gasto
				\4[] $\to$ Aumentos del gasto sobre $\Delta$ \% crecimiento
				\4[] $\then$ Ligadas a aumento discrecional del ingreso
				\4[] Valoración del crecimiento neto del gasto
				\4[] $\to$ Como mínimo, anual
				\4[] $\to$ Zona Euro, dos veces al año\footnote{En primavera y otoño.}
				\4[] Límite al crecimiento neto de gasto anual
				\4[] Economías en MTO
				\4[] $\to$ $\Delta$ $\leq$ crecimiento potencial de m/p
				\4[] Economías no en MTO
				\4[] $\to$ $\Delta$ menor que crecimiento potencial de m/p
				\4 Desviaciones significativas
				\4[] Se abre Procedimiento Desviación Significativa:
				\4[] $\to$ DEEstructural > 0.5\% PIB en 1 año
				\4[] $\to$ DEEstructural > 0.25\% media de 2 años
				\4[] Prevé sanciones
				\4[] $\to$ Políticas y reputacionales
				\4[] $\to$ Financieras en último término\footnote{0.2\% del PIB como depósito sin interés, se convierte posteriormente en multa definitiva.}
			\3 Brazo Correctivo del PEC
				\4 Instrumento coercitivo
				\4[] Corregir déficits excesivos
				\4 Procedimiento de Déficit Excesivo
				\4[] Herramienta principal del brazo correctivo
				\4 Fases del PDExcesivo
				\4[] 1. Detección de incumplimiento
				\4[] $\to$ Déficit > 3\%
				\4[] $\to$ deuda >60\% que no disminuye\footnote{Si diferencia entre deuda pública y 60\% se reduce al 1/20 de media anual durante tres años.}
				\4[] 2. Remisión de dictamen a EMiembro
				\4[] 3. CdUE decide sobre existencia DExcesivo
				\4[] $\to$ A propuesta de la Comisión
				\4[] 4. CdUE adopta recomendación para EM
				\4[] $\to$ Exigiendo medidas en plazo 3/6 meses
				\4[] $\to$ Exigiendo plazo corrección del DEXcesivo
				\4[] $\to$ Exigiendo senda de corrección de déficit
				\4[] 5. Evaluación de CE y CdUE tras 6 meses
				\4[] 6a. Si se constata incumplimiento, ``step up''
				\4[] $\to$ Sanciones de hasta 0.2\% del PIB o +
				\4[] 6b. Si medidas efectivas, cierre PDE
			\3 Fiscal Compact (2012)
				\4 Marzo de 2012
				\4 Parte del TCSG
				\4 Todos salvo CZK y UK
				\4 Transposición a ordenamiento nacional
				\4[] Obligación de equilibrar presupuesto
				\4[] Medidas automáticas de correción
				\4[] Autoridad independiente de supervisión\footnote{AiREF en España.}
				\4[] MTO máximo del -0.5\%
				\4[] $\to$ ``\textit{Regla de oro}''
				\4[] $\to$ Salvo deuda <60\% PIB + bajo riesgo
				\4[] $\then$ Permite 1\% máximo en esos casos
			\3 Proc. de Desequilibrios Macroeconómicos -- PDM (2011)
				\4 \textit{Macroeconomic Imbalance Procedure}
				\4 Supervisión fiscal no es suficiente
				\4[] Necesario reducir otros desequilibrios
				\4[] $\to$ Acentúan efectos de crisis
				\4[] $\to$ Aumentan riesgo de nuevas crisis
				\4[] $\to$ Inducen divergencias de competitividad
				\4 Introducido en six-pack (2011)
				\4[] Dos reglamentos
				\4[] i. Identificación y corrección
				\4[] ii. Imposición de sanciones
				\4[] $\to$ Brazos preventivo y correctivo
				\4 Indicadores considerados
				\4[] 14 indicadores con umbral de cumplimiento
				\4[] $\to$ Indicadores principales
				\4[] 28 indicadores auxiliares
				\4[] $\to$ Complementar otros indicadores
				\4 Fases del PDM
				\4[] 1. Informe sobre el mecanismo de alerta
				\4[] $\to$ Paquete de otoño de la CE (noviembre)
				\4[] 2. Estado Miembro supera algún umbral
				\4[] $\to$ Un umbral por indicador
				\4[] $\to$ Problema específico si sobrepasado
				\4[] 3. Examen exhaustivo (in-depth review)
				\4[] \quad Dos posibilidades para Comisión
				\4[] \quad $\to$ No considerar problemática
				\4[] \quad $\to$ Formular recomendaciones preventivas
				\4[] \quad $\to$ Abrir Proc. Deseq. Excesivo (PDE)
				\4[] (si procede apertura PDE)
				\4[] 4. CE emite Recomendación medidas correctoras
				\4[] 5. País presenta plan de medidas correctoras
				\4[] 6. Evaluación medidas correctoras aplicadas
				\4[] \quad -- Cierre PDE
				\4[] \quad -- Emitir nueva recomendación (QMV Inversa)
				\4[] \quad -- Establecer sanciones
			\3 Semestre Europeo
				\4 Idea clave
				\4[] Marco temporal
				\4 Calendario
				\4[15 octubre] como máximo
				\4[] -- Planes presupuestarios nacionales
				\4[] -- Proyectos presupuestarios para año que viene
				\4[] -- Previsiones macroeconómicas independientes
				\4[Noviembre]
				\4[] -- Recepción borradores presupuestarios
				\4[] -- Fijar prioridades
				\4[] -- Orientación próximo año
				\4[] -- Paquete de Otoño
				\4[] $\to$ Estudio Prospectivo Anual sobre Crecimiento
				\4[] $\to$ Informe sobre Mecanismo de Alerta (para PDM)
				\4[] $\to$ Recomendación para el Área del Euro
				\4[] $\to$ Informe conjunto sobre Empleo
				\4[] Opinión sobre borradores presupuestarios
				\4[] $\then$ Sin veto, pero opinión muy importante
				\4[31 de diciembre]
				\4[] -- Adoptar presupuestos nacionales
				\4[Febrero]
				\4[] -- Informes sobre cada país
				\4[] -- Evaluación de política económica
				\4[Marzo-30 abril]
				\4[] -- Reuniones bilaterales Comisión--EEMM
				\4[] -- PNR Programas Nacionales de Reforma
				\4[] -- Programas de Estabilidad/Convergencia\footnote{Convergencia para países que no son miembros de área del euro. Se presentan cada año actualizaciones de los planes}
				\4[] $\to$ Plan presupuestario para tres años
				\4[Mayo-Julio]
				\4[] -- CSR -- Recomendaciones específicas para cada país
				\4[] -- Cierre de ciclo de Semestre Europeo
				\4[Agosto-octubre]
				\4[] -- Preparación para nuevo ciclo
				\4[] -- Incorporación recomendaciones a PGReformas
			\3 Autoridad Fiscal Europea y IFIndependientes\footnote{Instituciones Fiscales Independientes.}
				\4 IFIs creadas por Fiscal Compact
				\4 Autoridad Fiscal Europea
				\4 Justificación
				\4[] Presupuestos y sendas de ajuste
				\4[] $\to$ Condicionadas por político
				\4 Funciones
				\4[] Emitir opiniones técnicas e independientes
				\4[] $\to$ Presupuestos
				\4[] $\to$ Sostenibilidad medio plazo
				\4[] $\to$ Política fiscal y tributaria
		\2 Valoración
			\3 Mejoras pre-crisis financiera
			\3 Flexibilidad vs cumplimiento estricto
			\3 Críticas al PEC
			\3 Economía política
			\3 Trilema de Rodrik
		\2 Retos
			\3 Eurobonos y SBBS
			\3 Instrumento sobre Convergencia y Competitividad
			\3 Impuesto sobre transacciones financieras
			\3 Presupuesto de la Zona Euro
	\1 \marcar{Unión Bancaria}
		\2 Justificación
			\3 Ruptura de vínculos entre EEMM
				\4 Crisis induce desconfianza entre EEMM
				\4[] Opacidad verdadero estado IFinancieras
				\4[] Desconfianza datos económicos
				\4 Desconfianza rompe mercados
				\4[] Activos extranjeros no se venden
				\4[] Capital no fluye donde necesario
				\4[] Iliquidez se convierte en insolvencia
			\3 Competencia perversa entre EEMM
				\4 Atracción de capitales
				\4[] Incentivan relajación de supervisión/regulación
				\4[] Incentivan garantías públicas sistema depósitos
			\3 Círculo vicioso deuda bancaria--soberana
				\4 Bancos nacionales con problemas
				\4[] Aumentan riesgo para economía nacional
				\4[] $\to$ Gobiernos rescatan bancos nacionales
				\4 Rescates bancarios con dinero público
				\4[] Deterioran cuentas públicas
				\4[] Política monetaria ajena a gobiernos EEMM
				\4[] $\to$ No pueden financiarse
				\4[] $\then$ Aumenta riesgo de deuda soberana
				\4 Riesgo aumentado de deuda soberana
				\4[] Caen precios de títulos deuda pública
				\4[] Bancos nacionales tienen deuda nacional
				\4[] $\then$ Deterioro de balances de bancos nacionales
			\3 Economía política
				\4 Ganadores y perdedores con rescates bancarios
				\4[] A nivel nacional y europeo
				\4 Establecer reglas claras
				\4[] Prevenir:
				\4[] $\to$ Capturas de regulador
				\4[] $\to$ Presiones políticas
				\4[] $\to$ Special interests
		\2 Objetivos
			\3 Romper círculo vicioso público-privado
			\3 Integrar mercados de capital en UE
			\3 Reglas transparentes de ayuda financiera
			\3 Aumentar estabilidad del sistema financiero
			\3 Reducir impacto de crisis
			\3 Reducir discriminación entre nacional y europeo
		\2 Antecedentes
			\3 Antes de crisis
				\4 Principio de país de origen
				\4[] Supervisor nacional regula bancos nacionales
				\4[] Pasaporte europeo permite operar en toda UE
				\4 Coordinación entre reguladores
				\4[] Comités de reguladores
				\4[] $\to$ Bancarios
				\4[] $\to$ Seguros
				\4[] $\to$ Mercados de valores
				\4[] $\then$ Nacionales y europeos
				\4[] $\then$ EBA, ESMA, EIOPA
			\3 Informe de Larosière (2009)
				\4 Recomienda reformas generales
				\4 Creación de sistema de supervisión
				\4[] $\to$ Micro y macroprudencial
			\3 Sistema Europeo de Supervisión Financiera -- ESFS
				\4 En funcionamiento desde 2011
				\4 Integrar supervisión micro y macro
				\4 Supervisión micro
				\4[] EBA, ESMA, EIOPA
				\4[] Estándares de supervisión y regulación
				\4[] $\to$ Single Rulebook
				\4[] Supervisar aplicación de regulaciones
				\4 Supervisión macro
				\4[] ESRB -- European Systemic Risk Board
				\4[] Monitorizar riesgo sistémico
				\4[] Emitir alertas relevantes
				\4[] Monitorizar cumplimiento recomendaciones
				\4[] Misma sede que BCE
				\4[] Miembros:
				\4[] $\to$ BCE
				\4[] $\to$ ESMA, EBA, EIOPA
				\4[] $\to$ Bancos centrales nacionales
			\3 Crisis de deuda soberana de 2012
				\4 Respuesta relativamente contundente
				\4[] Extrema gravedad de la situación
				\4 MEDE + OMT
				\4[] Otoño de 2012
				\4[] Resultado de ``whatever it takes''
				\4 Informe de los cuatro presidentes
				\4[] Hoja de Ruta de Unión Bancaria
				\4[] $\to$ Asesoramiento del FMI
				\4[] $\to$ Acuerdo sobre regulación común sistema financiero
				\4[] $\to$ Recapitalizaciones directas por MEDE
				\4[] $\to$ Creación supervisor y mecanismo resolución únicos
			\3 Puesta en marcha de Unión Bancaria
				\4 En proceso
				\4 Impulso inicial 2013-2016
				\4 Relativo estancamiento actual
		\2 Marco financiero
			\3 Fondo Único de Resolución -- SRF
				\4 Contribuciones de instituciones
				\4[] De crédito y de inversión
				\4 Formación gradual
				\4[] Proceso 2016-2023
				\4[] Contribuciones lineales
				\4 Objetivo final
				\4[] 1\% de depósitos totales
				\4[] $\sim$ 25.000 millones actualmente
			\3 MEDE -- ESM
				\4 Rol potencial de garantía
				\4[] Backstop si FUR no es suficiente
				\4[] Aún no está bien definido
			\3 Loan Facility Agreements
				\4 Acordado en 2017
				\4 Líneas de crédito puente hasta 2024
				\4[] EEMM garantizan cuantía total SRF
				\4[] $\to$ hasta que no se cubra por bancos
		\2 Marco jurídico
			\3 \underline{Single Rulebook}
				\4 Marco institucional propio
				\4 Transferencia de competencias
				\4[] Necesarias para llevar a cabo
				\4 Aplicable a todos los EEMM
				\4[] Incluidos fuera de Z€
			\3 CRR I y CRD IV
				\4 Capital Requirements Regulation
				\4 2013
				\4 Transposición de Basilea III
				\4 Objetivo
				\4[] Prevención de crisis bancarias
				\4[] Aumentar capacidad de absorción de pérdidas
				\4 Disposiciones
				\4[] Incremento liquidez y capital
				\4[] Introduce ratio apalancamiento
			\3 CRR II/CRD V (2019)
				\4 Aprobadas en 2019
				\4 Entrada en vigor de mayoría de medidas en 2021
				\4 Implementar reformas de 2017 de Basilea III
			\3 BRRD II/SRMR II (2019)
				\4 Bank Recovery and Resolution Directive
				\4 Single Resolution Mechanism Regulation
				\4 BRRD (I) 2015
				\4 Regular resolución bancaria
				\4[] Marco de resolución ordenada
				\4[] Reducir coste para contribuyente
				\4[] Reducir incertidumbre sobre resolución
				\4[] $\then$ Reducir riesgo de contagio
				\4[] $\then$ Reducir probabilidad de pánicos bancarios
				\4 Bail-in
				\4[] Absorción de pérdidas por accionistas
				\4[] $\to$ Reducciones de capital
				\4[] $\to$ Quitas a acreedores
				\4[] $\then$ Minimizar inyecciones de fondos públicos
				\4 Artículo 55 sobre bail-in pasivos extra-EEA
				\4 Reforma de 2019 (BRRD II/SRMR II)\footnote{Ver \url{https://www.moodysanalytics.com/regulatory-news/jun-07-19-eu-publishes-brrd-ii-and-srmr-ii-in-the-official-journal}}
				\4[] Aplicables desde inicio 2021
				\4[] SRMR II:
				\4[] $\to$ Armonización de MREL a TLAC de FSB\footnote{TLAC: Total Loss-Absorbing Capacity. MREL: Minimum Requirements of Eligible Assets. El FSB estableció el concepto de TLAC en 2014. En la regulación europea a partir de 2014/2015, el TLAC se implementó como MREL, aunque no exactamente con el mismo tenor. El MREL exigido es diferente al ratio de capital mínimo en el marco de Basilea III y CRD II/CRD V. Ver \url{https://srb.europa.eu/sites/srbsite/files/list_of_public_qas_on_mrel_-_clean.pdf}}
			\3 DGSD
				\4 Deposit Guarantee Scheme Directive
				\4 2014
				\4 Eliminar equilibrios múltiples indeseables
				\4[] EEMM compiten por ofrecer más protección
				\4[] $\to$ Fugas de capital y $\uparrow$ de riesgo
			\3 IFR/IFD (2019)\footnote{\url{https://www.nortonrosefulbright.com/en/knowledge/publications/f6b2e0a7/the-new-prudential-regime-for-investment-firms}}
				\4 Entrada en vigor de mayoría de medidas en 2021
				\4 Regulación de empresas de inversión
				\4 Firmas de inversión implican menos riesgos...
				\4[] ...que instituciones de crédito reguladas
				\4[] $\to$ Menores requisitos prudenciales
		\2 Actuaciones
			\3 Miembros
				\4 Zona Euro
				\4 Voluntariamente, EEMM que no están en Euro
			\3 SSM -- Mecanismo Único de Supervisión
				\4 Single Supervisory Mechanism
				\4 Creado en 2013
				\4 Miembros
				\4[] $\to$ BCE
				\4[] $\to$ Autoridades nacionales de países participantes
				\4 Test de estrés
				\4[] Primeros en 2014
				\4 Entidades significativas
				\4[] 120 entidades
				\4[] $\to$ 80\% de activos bancarios
				\4[] Supervisadas directamente por BCE
				\4 Resto de entidades
				\4[] Delegadas en BCNacionales
				\4[] $\to$ Siguen manteniendo otras competencias\footnote{Blanqueo de capitales, protección al consumidor, sistemas de pago...}
			\3 SRM -- Mecanismo Único de Resolución
				\4 Single Resolution Mechanism
				\4 Dos componentes
				\4[] -- Junta Única de Resolución
				\4[] -- Fondo Único de Resolución
				\4 Junta Única de Resolución
				\4[] Single Resolution Board
				\4[] Operativa desde 2016
				\4[] Miembros:
				\4[] $\to$ Autoridades nacionales
				\4[] $\to$ Presidente
				\4[] $\to$ Vicepresidente
				\4[] $\to$ Cuatro miembros permanentes
				\4[] Sesión ejecutiva
				\4[] $\to$ Preparar decisiones
				\4[] $\to$ Decisiones operativas
				\4[] Sesión plenaria
				\4[] $\to$ Decidir resolución concreta
				\4 Fondo Único de Resolución
				\4[] Single Resolution Fund
				\4[] V. supra
			\3 EDIS -- Garantía de depósitos
				\4 Armonización de garantías
				\4 Garantía mínima
				\4[] 100.000 €
				\4 Tipos de depósitos
				\4[] Armonizados entre países
				\4 Contribución a garantías de depósitos
				\4[] Bancos contribuyen
				\4[] Contribuyentes quedan a priori al margen
				\4 Garantía común europea
				\4[] Proyecto
				\4[] No se ha implementado
				\4[] Oposición de algunos países
				\4[] $\then$ Reto para completar Unión Bancaria
		\2 Valoración
			\3 Avances respecto a pre-crisis
				\4 Supervisión mejorada
				\4 Coordinación basada en reglas
				\4[] En cierta medida superada
				\4[] $\to$ Transferencia de soberanía
			\3 Unión Bancaria incompleta
				\4 Argumentos a favor de mayor integración
				\4[] Necesario fiscal-backstop
				\4[] Necesario EDIS
				\4[] $\then$ Romper vínculo soberano
			\3 Próxima crisis
				\4 Test de situación actual
				\4[] $\to$ ¿Avances son suficientes?
				\4[] $\to$ ¿Estructura es resistente?
		\2 Retos
			\3 Fondo Europeo de Garantía de Depósitos -- EDIS
				\4 European Deposit Insurance
				\4[] Propuesto en 2015
				\4[] $\to$
				\4 Concepto
				\4[] Garantía de depósitos europea
				\4[] $\to$ No nacional
				\4[] $\then$ Fondo europeo garantice mínimo a depósitos
				\4 Justificación
				\4[] Tamaño
				\4[] $\to$ Seguro mejor cuanto más asegurados
				\4[] $\then$ Risk-spreading
				\4[] Consistencia
				\4[] $\to$ Supervisión europea necesita depósito europeo
				\4[] $\then$ Si no, contribuyentes nacionales pagan error europeo
				\4[] Desvincular riesgo bancario--soberano
				\4[] $\to$ Garantía nacional a cargo contribuyente nacional
				\4[] $\to$ Utilización de garantía aumenta coste
				\4[] $\to$ Aumenta presión sobre deuda soberana
				\4[] $\to$ Aumenta presión sobre bancos
				\4[] $\then$ Círculo vicioso
				\4[] Gestión de crisis
				\4 Propuesta de implementación
				\4[] 1. Reaseguro
				\4[] $\to$ Provisión de liquidez a fondos nacionales
				\4[] $\to$ Fondos devuelven liquidez
				\4[] 2. Aseguramiento completo
				\4[] $\to$ EDIS cubre pérdidas directamente
				\4[] $\then$ Sin vínculo localización del banco y país
			\3 Salvaguarda fiscal/backstop fiscal
				\4 ¿Qué pasa si FUR se queda sin dinero?
				\4 ¿Qué sucede si EDIS no tiene dinero?
				\4[$\then$] Necesaria salvaguarda fiscal
				\4[] Banco central puede prestar liquidez
				\4[] $\to$ No puede asumir + pérdidas que capital
				\4[] $\then$ Necesaria provisión pública de capital
				\4 Requisitos de capital bancario
				\4[] Tratar de hacer innecesario fiscal backstop
				\4[] $\to$ ¿Es suficiente?
				\4[] Múltiples voces:
				\4[] $\to$ Necesario fiscal backstop para $\uparrow$ confianza
				\4[] $\to$ USA, Japón, China sí pueden garantizar
				\4 ESM como herramienta
				\4[] Puede servir para recapitalizar
				\4[] $\to$ Último recurso
				\4[] $\to$ En la práctica, casi inutilizable
			\3 Riesgo moral
				\4 Intergubernamental en el tiempo
				\4[] Tendencia de gobiernos actuales
				\4[] $\to$ Dejar problema a gobiernos futuros
				\4 Intergubernamental entre EEMM
				\4[] Responsabilizan a UE de efectos
				\4[] $\to$ Derivados de su inacción
				\4 Bancos
				\4[] Posibilidad de ser rescatados
				\4[] $\to$ Reduce incentivos a reducir riesgo
				\4[] Relativa mitigación con BRRD y otros
			\3 Sovereign Bond-Backed Securities -- SBBS \footnote{Ver \url{https://ec.europa.eu/info/business-economy-euro/banking-and-finance/banking-union/sovereign-bond-backed-securities-sbbs_en}.}
				\4 Activo libre de riesgo europeo
				\4[] Herramienta para eliminar vínculo soberano
				\4[] Reducir concentración deuda soberana y bancos
				\4[] $\to$ ¿Cómo hacer que sea aceptable?
				\4 Eurobonos
				\4[] Bonos emitidos por Unión Europea
				\4[] $\to$ Muy difícil políticamente
				\4[] $\to$ Implica solidarizar deudas
				\4[] $\then$ Temor respecto a riesgo moral
				\4 SBBS: Alternativa políticamente aceptable
				\4[] CDOs de bonos soberanos europeos
				\4[] 1. Pool de deuda pública
				\4[] 2. Titulizar flujos de caja
				\4[] $\to$ Senior
				\4[] $\to$ Junior
				\4[] $\to$ Mezzanine
				\4 Obstáculos
				\4[] Impacto sobre bonos no incluidos
				\4[] $\to$ Reducción de demanda
				\4[] Colateralización
				\4[] $\to$ Oposición en países del norte
			\3 Externalidades más allá de Z€
				\4 Miembros de UE no euro
				\4 Países en desarrollo en órbita europea
				\4 Economía mundial
	\1 \marcar{Otros desarrollos}
		\2 Unión del mercado de capitales\footnote{Ver IMF (2009).}
			\3 Justificación
				\4 Eliminar obstáculos a flujos de capital
				\4[] $\to$ Integración mercados deuda y equity
				\4 Elevada bancarización y segmentación
				\4[] $\to$ Fuerte dependencia de bancos
				\4[] $\to$ Poca tenencia transfronteriza de activos financieros
				\4 Diversificar fuentes de financiación
				\4 Buffer frente a shocks sistémicos
			\3 Objetivos
				\4 i. Transparencia
				\4[] $\to$ Centralizar y estandarizar información sobre emisores
				\4[] $\to$ Reducir coste de información de pequeños emisores
				\4[] $\to$ Permitir no cotizadas acceder a mercado
				\4 ii. Regulación
				\4[] $\to$ Centralizar supervisión de intermed. sistétmicos
				\4[] $\to$ Aumentar convergencia supervisoria
				\4[] $\to$ Armonizar regulación de pensiones para portabilidad
				\4 iii. Insolvencia
				\4[] $\to$ Estándares mínimos de control
				\4[] $\to$ Sistematizar control de proceso regulatorio
			\3 Actuaciones
				\4 Reglamento sobre fondos de capital riesgo (2013)
				\4[] $\to$ Armoniza requisitos y condiciones
				\4[] $\to$ Mercado único de fondos
				\4 Reglamento de titulizaciones de 2017
				\4 Reglamento de folletos de 2017
				\4[] $\to$ Reduce carga administrativa
				\4[] $\to$ Facilitar acceso a empresas más pequeñas
				\4 Target 2-Securities
				\4[] $\to$ Sistema de intercambio de títulos-dinero
				\4[] $\to$ Intercambio simultáneo
				\4[] $\then$ Eliminar riesgo de contrapartida
				\4[] $\then$ Liquidación segura e inmediata de transacciones
			\3 EMIR -- European Market Infrastructure Regulation
				\4 Aprobada en 2012
				\4 Transparencia
				\4[] Obligación de proveer información a repositorios
				\4[] $\to$ Registros de derivados
				\4[] Publicación de posiciones agregadas de derivados
				\4[] $\to$ OTC
				\4[] $\to$ Cotizados en mercados oficiales
				\4 Reducción de riesgo de crédito
				\4[] Contratos OTC estandarizados
				\4[] $\to$ Deben compensarse y liquidarse en CCP
				\4[] Contratos fuera de CCP
				\4[] $\to$ Técnicas de mitigación del riesgo
				\4 Regulación de CCP
				\4[] Riesgo operativo
				\4[] Capital
				\4[] Transparencia
				\4 Acreditación de equivalencia
				\4[] Reconocimiento de CCP y repositorios fuera de UE
				\4[] Confirmar requisitos mínimos de fuera-UE
				\4[] Reconocimiento permite:
				\4[] $\to$ Utilización de CCP por residentes en UE
			\3 TARGET 2-Securities
				\4 Interconexión de depósitos centrales de valores
				\4 Liquidación de títulos valor
				\4[] Negociados en mercados oficiales
			\3 Revisión de Plan de Acción en 2017
				\4 Incrementar herramientas de la ESMA
				\4 Aumentar proporcionalidad de regulación de OPV
				\4 Revisar tratamiento prudencial de empresas de inversión
				\4 Mejorar regulación fin tech
				\4 Fortalecer crédito de bancos
				\4 Liderazgo en inversión sostenible
				\4 Comercialización transfronteriza de fondos
				\4 Desarrollo de ecosistemas de mercados de capital local
			\3 Valoración
				\4 Unión Bancaria sigue siendo objetivo principal
				\4 UMCapitales más compleja y de largo plazo
				\4 Doble objetivo de UMCapitales
				\4[] $\to$ Desintermediar sistema financiero
				\4[] $\to$ Desconcentrar financiación en mercados nacionales
			\3 Propuestas
				\4 Armonización fiscal mercados financieros
				\4 Simplificación de folletos
				\4 Mercado europeo de titulación
				\4 Armonizar:
				\4[] $\to$ prácticas contables y de auditoría
				\4[] $\to$ legislación societaria
				\4[] $\to$ derechos de propiedad
				\4 Plan de Acción de 2015
		\2 Unión fiscal
			\3 Justificación
				\4 Evitar política fiscal procíclica
				\4[] $\to$ Resultado de PEC
				\4 Superar limitaciones de presupuesto nacional
				\4[] $\to$ Estabilizadores automáticos insuficientes
				\4[] $\to$ Deuda demasiado elevada
				\4[] $\to$ Sin acceso a financiación
			\3 Requisitos
				\4 Convergencia económica real
				\4 Integración financiera
				\4 Cesión de soberanía presupuestaria
				\4[] $\then$ Reducir riesgo moral
			\3 Propuesta de presupuesto zona euro
				\4 Acuerdo franco-alemán noviembre 2018
				\4 Dentro de MFP 2021
				\4 Centrado en inversión, convergencia
				\4 Estabilización:
				\4[] $\to$ Marginal
				\4[] $\to$ Ligada a reformas estructurales
	\1[] \marcar{Conclusión}
		\2 Recapitulación
			\3 Sistema Europeo de Bancos Centrales
			\3 Política Monetaria Europea
			\3 Gobernanza de la Zona Euro
			\3 Unión Bancaria
		\2 Idea final
			\3 Instrumentos de PM en próxima crisis
				\4 Tipos de intervención en mínimos
				\4[] Paro relativamente alto
				\4[] Crecimiento positivo pero insuficiente
				\4 Retirada de programa de compras
				\4[] ¿Hay margen para implementar de nuevo?
				\4 Nuevos instrumentos de política monetaria
				\4[] Probablemente serán necesarios
			\3 Informe cinco presidentes
				\4 Cuatro frentes de actuación
				\4[] i. Unión económica genuina
				\4[] ii. Unión financiera
				\4[] iii. Unión fiscal
				\4[] iv. Unión política
				\4 Dependencia mutua entre frentes
				\4 Visión de largo plazo
				\4 Cesión de soberanía
				\4 Implementación gradual
				\4[] Fase I
				\4[] $\to$ Aumentar competitividad
				\4[] $\to$ Completar Unión Bancaria
				\4[] Fase II
				\4[] $\to$ Referencias vinculantes de convergencia
				\4[] $\to$ Preparar mecanismo de absorción de shocks
				\4 Obstáculos múltiples
				\4[] Economía política
				\4[] Diferencias culturales
				\4 Reminiscencia de Informe Werner
				\4[] $\to$ ¿Qué se ha conseguido a 2019?
\end{esquemal}




































\preguntas

\seccion{Test 2018}

\textbf{44.} La incorporación obligatoria de una regla de equilibrio presupuestario en las legislaciones nacionales de los estados miembros, dentro del marco de gobernanza actual de la eurozona, aparece prevista en:

\begin{itemize}
	\item[a] El Six-Pack.
	\item[b] El Mecanismo Europeo de Estabilidad (MEDE).
	\item[c] El Fiscal Compact.
	\item[d] El Two-Pack
\end{itemize}

\bigskip

\textbf{45.} En relación a los órganos rectores del Banco Central Europeo (BCE), señale la respuesta \textbf{\underline{CORRECTA}}:

\begin{itemize}
	\item[a] El Comité Ejecutivo está compuesto por el presidente del BCE, el vicepresidente del BCE y otros cuatro miembros.
	\item[b] El Consejo de Gobierno está formado por los miembros del Comité Ejecutivo y los gobernadores de los bancos centrales nacionales de los 28 Estados Miembros de la UE.
	\item[c] El Consejo General está formado por el presidente del BCE, el vicepresidente del BCE y los gobernadores de los bancos centrales nacionales de los 19 países de la zona euro.
	\item[d] Todas las opciones anteriores son correctas.
\end{itemize}

\seccion{Test 2017}
\textbf{44.} El Pacto de Estabilidad y Crecimiento (PEC):

\begin{itemize}
	\item[a] Tiene muchos aspectos relacionados con el control del déficit público, pero no dice nada sobre el control de la deuda pública.
	\item[b] Permite sancionar las tasas de inflación muy elevadas de los Estados, con la pérdida de préstamos del Banco Europeo de Inversiones (BEI).
	\item[c] Consta de un brazo preventivo, en el que el ajuste presupuestario anual exigido hacia el objetivo a medio plazo se puede modular de acuerdo a una matriz que tiene en cuenta el crecimiento real y potencial, el \textit{output gap} y el nivel de deuda pública.
	\item[d] Incluye el Tratado de Estabilidad Coordinación y Gobernanza (TECG), cuyos contenidos contemplan y destacan aspectos de carácter preventivo, si bien, en ninguno de sus apartados se tratan temas de carácter correctivo en torno a la disciplina fiscal/presupuestaria.
\end{itemize}

\textbf{45.} La política monetaria de la Eurozona:

\begin{itemize}
	\item[a] La formula y ejecuta el Consejo General, como órgano superior del Banco Central Europeo (BCE).
	\item[b] Ha tenido como objetivo prioritario mantener la estabilidad de los precios, una inflación inferior pero próxima al 2\%, si bien el Pacto de Estabilidad y Crecimiento (PEC), reforzado en el periodo 2011-2014, permite al Banco Central Europeo (BCE) la asunción de una inflación del 4\% antes de recurrir a políticas monetarias restrictivas.
	\item[c] Exige a las entidades de crédito el mantenimiento de unas reservas mínimas a fin de mantener una mejor estabilidad de tipos de interés en el mercado.
	\item[d] También tiene la función de autorizar la emisión de billetes y monedas y de costear los gastos de fabricación.
\end{itemize}

\textbf{46.} Señale la respuesta \underline{\textbf{FALSA}}:

\begin{itemize}
	\item[a] La Unión Bancaria nace a raíz del impacto de la crisis económica y de las distintas respuestas nacionales que amplifican la fragmentación de los mercados financieros en la zona euro.
	\item[b] En 2017, ha entrado en vigor el Fondo Europeo de Garantía de Depósitos.
	\item[c] El Mecanismo Único de Supervisión es el primer pilar de la Unión Bancaria y empezó a funcionar en noviembre de 2014.
	\item[d] El nuevo Código Normativo Único sobre requisitos de capital también incluye disposiciones de protección de los depositantes y de prevención y gestión de las quiebras bancarias.
\end{itemize}

\seccion{Test 2016}

\textbf{45.} El Mecanismo Europeo de Estabilidad (MEDE)
\begin{enumerate}
	\item[a] A diferencia de la FEEF, sus emisiones no están respaldadas por garantías concedidas por los estados miembros de la Zona Euro.
	\item[b] Está traspasando sus funciones a la Facilidad Europea de Estabilidad Financiera (FEEF), que terminará por sustituirlo.
	\item[c] Puede utilizarse como mecanismo para la concesión de ayuda financiera sin necesidad de supeditar dicha ayuda a concesiones estrictas.
	\item[d] Tiene como objetivo fundacional actuar como mecanismo temporal de estabilidad en la Zona Euro.
\end{enumerate}

\seccion{Test 2015}

\textbf{46.} Señale la respuesta verdadera relativa al Pacto de Estabilidad y Crecimiento (PEC):
\begin{enumerate}
	\item[a] Tiene dos componentes: la promoción de la estabilidad presupuestaria y el impulso de las reformas estructurales para promover el crecimiento en la UE.
	\item[b] Los objetivos presupuestarios a medio plazo del brazo preventivo se establecen en términos del saldo presupuestario nominal total.
	\item[c] Un nivel de deuda pública superior a 60\% del PIB no acarrea automáticamente la apertura de un procedimiento de déficit excesivo.
	\item[d] Las sanciones a las que se pueden enfrentar los países de la zona euro en caso de incumplimiento de la normativa del PEC pueden ser como máximo un 1\% de la renta disponible del país.
\end{enumerate}

\textbf{47.} Seleccione la respuesta correcta respecto a los pilares de la Unión Bancaria Europea:
\begin{enumerate}
	\item[a] El Mecanismo Único de Supervisión instaura un nuevo sistema de supervisión financiera formado por el Banco Central Europeo y las autoridades nacionales competentes de todos los países de la Unión Europea con el fin de velar por la seguridad y la solidez del sistema bancario europeo.
	\item[b] La Autoridad Bancaria Europea es una autoridad dependiente del Banco Central Europeo que trabaja para garantizar un nivel efectivo y coherente de regulación y supervisión prudencial bancaria.
	\item[c] El objetivo del Mecanismo Único de Resolución es garantizar la resolución ordenada de las entidades de crédito autorizadas en los Estados Miembros participantes en la Unión Bancaria, con un coste mínimo para los contribuyentes y la economía real.
	\item[d] El objetivo del futuro Fondo de Garantía de Depósitos Común es conseguir que la seguridad de los depósitos dependa del país en el que tiene su sede un banco y de la gestión y solidez de la entidad. 
\end{enumerate}

\seccion{Test 2014}

\textbf{38.} En la Unión Europea, en la actualidad, el fondo de rescate permanente para que un país recapitalice su sector financiero se denomina:
\begin{enumerate}
	\item[a] Facilidad Europea de Estabilidad Financiera (FEEF)
	\item[b] Mecanismo Europeo de Estabilidad (MEDE)
	\item[c] Fondo Europeo de Estabilidad Financiera (FEEF)
	\item[d] Ninguna de las anteriores.
\end{enumerate}

\textbf{39.} Los miembros de la Unión Europea que no firmaron el Tratado de Estabilidad, Coordinación y Gobernanza en la Unión Económica y Monetaria en marzo de 2012 son:
\begin{enumerate}
	\item[a] Grecia y Reino Unido.
	\item[b] Rep. Checa y Reino Unido.
	\item[c] Grecia y Rep. Checa.
	\item[d] Grecia, Rep. Checa y Reino Unido.
\end{enumerate}


\seccion{Test 2013} En el Banco Central Europeo, el capital desembolsado corresponde a:

\textbf{40.}
\begin{enumerate}
	\item[a] Todos los estados miembros de la Unión Europea
	\item[b] Los estados miembros de la Unión Monetaria Europea.
	\item[c] Los estados miembros de la Unión Monetaria Europea y del SME II.
	\item[d] También participan países no europeos.
\end{enumerate}

\textbf{41.} Los estados miembros de la Unión Europea que no pertecen a la eurozona:
\begin{enumerate}
	\item[a] Están presentes en el Comité Ejecutivo.
	\item[b] Están presentes en el Consejo General.
	\item[c] Están presentes en el Consejo de Gobierno.
	\item[d] No están presentes en ningún órgano del BCE.
\end{enumerate}

\seccion{Test 2011}

\textbf{45.} En el Consejo de Gobierno del Banco Central Europeo:
\begin{enumerate}
	\item[a] Sólo participa el Comité ejecutivo
	\item[b] Sólo participan los Gobernadores de los Bancos Centrales Nacionales de la UE
	\item[c] Participan el Comité Ejecutivo y los Gobernadores de los Bancos Centrales Nacionales de la zona euro.
	\item[d] Participan el Comité Ejecutivo y los Gobernadores de los Bancos Centrales Nacionales de la UE.
\end{enumerate}

\seccion{Test 2009}

\textbf{45.} En el Pacto de Estabilidad y Crecimiento de la Unión Europea, el Objetivo Presupuestario a Medio Plazo es:
\begin{enumerate}
	\item[a] Un objetivo presupuestario de medio plazo común para los países de la Unión Europea.
	\item[b] Un objetivo presupuestario de medio plazo diferenciado para cada país de la Unión Europea, con los objetivos de proporcionar un margen de seguridad respecto al límite de déficit del 3\% del PIB, avanzar hacia la sostenibilidad de las finanzas públicas y tener un margen de maniobra presupuestario para políticas fiscales activas.
	\item[c] Un objetivo presupuestario de medio plazo diferenciado para cada país de la zona euro y de aquellos cuya moneda está incluida en el Mecanismo de Tipos de Cambio II.
	\item[d] Un objetivo presupuestario de medio plazo común para los países de la zona euro y de aquellos cuya moneda está incluida en el Mecanismo de Tipos de Cambio-II, consisten en fijar un objetivo de equilibrio presupuestario a lo largo del ciclo económico.
\end{enumerate}


\seccion{Test 2007}

\textbf{43.} El Sistema Europeo de Bancos Centrales está formado por:
\begin{enumerate}
	\item[a] El Banco Central Europeo y los bancos centrales nacionales de los estados miembros de la Unión Europea que han adoptado el euro.
	\item[b] El Banco Central Europeo y los bancos centrales nacionales de los estados miembros de la UE.
	\item[c] Se creó con el Tratado.
	\item[d] El Banco Central Europeo y el Bundesbank alemán.
\end{enumerate}

\textbf{44.} ¿Cuál de los siguientes elementos caracterizan al Sistema Monetario Europeo II (SME-II)?

\begin{enumerate}
	\item[1] La adhesión al mismo es voluntaria.
	\item[2] El ancla del sistema es el euro.
	\item[3] Los miembros del mismo deben presentar informes de convergencia de manera regular.
	\item[4] Los márgenes de fluctuación son más, menos 15\%.
	\item[5] Se refuerza la coordinación económica y financiera entre los miembros.
	\item[6] El BCE defenderá de manera intramarginal las monedas del sistema atacadas de manera injustificada.
	\item[7] La gestión del sistem aes multilateral.
	\item[8] Las decisiones sobre realineamientos dependen en el último término de las autoridades nacionales.
\end{enumerate}

\begin{enumerate}
	\item[a] 1, 2, 3, 4, 5, 6, 7 y 8.
	\item[b] 2, 4, 5 y 7.
	\item[c] 2, 3, 4, 6 y 8.
	\item[d] 1, 2, 4, 5, 6.
\end{enumerate}

\textbf{45.} El informe One Market One Money (Informe Emerson) que justificó técnicamente dar el paso a la moneda única, giraba en torno a conceptos interrelacionados o cadenas de razonamientos como,
\begin{enumerate}
	\item[a] Eficiencia, estabilidad y apertura de mercados.
	\item[b] Reducción de precios, ganancias de competitividad, crecimiento del PIB.
	\item[c] Eficiencia micro, estabilidad macro, equidad entre países y regiones.
	\item[d] Supresión de barreras, reducciones de costes, ganancias de competitividad.
\end{enumerate}



\seccion{Test 2005}

\textbf{42.} El Pacto de Estabilidad y Crecimiento:
\begin{enumerate}
	\item[a] No incluye entre sus objetivos la estabilidad de precios en la UE.
	\item[b] Es un instrumento de disciplina fiscal de obligado cumplimiento que incluye mecanismos de flexibilidad en su aplicación.
	\item[c] Desarrolla unos criterios cuantitativos necesarios para asegurar una política monetaria adecuada.
	\item[d] Es un instrumento de disciplina fiscal de obligado cumplimiento, que incorpora un sistema de imposición automática de sanciones.
\end{enumerate}

\seccion{15 de marzo de 2017}
\begin{itemize}
	\item Respecto a la supervisión bancaria o European Systemic Risk Board, ¿opina que estas inciativas vacían de contenido a los Bancos Centrales de los distintos estados miembros?
	
	\item Respecto a la gobernanza económica, ¿existen límites a los superávit de cuenta corriente?
	
	\item En el año 2011, en plena crisis económica, el BCE decide aumentar los tipos de interés, ¿por qué?
\end{itemize}

\notas

\textbf{2018}: \textbf{44}. C \textbf{45}. A

\textbf{2017}: \textbf{44}. C \textbf{45}. C \textbf{46}. B

\textbf{2016}: \textbf{45}. A

\textbf{2015}: \textbf{46}. C \textbf{47}. C

\textbf{2014}: \textbf{38}. B \textbf{39}. B

\textbf{2013}: \textbf{40}. A \textbf{41}. B

\textbf{2011}: \textbf{45}. C

\textbf{2009}: \textbf{45}. B

\textbf{2007}: \textbf{43}. B \textbf{44}. A \textbf{45}. C

\textbf{2005}: \textbf{42}. B

\bibliografia

Valdez Molyneux

El Agraa

Mirar en Palgrave:
\begin{itemize}
    \item currency union
    \item Debt mutualisation in the ongoing Eurozone Crisis -- A tale of the 'North and the 'South
    \item euro
	\item European Banking Union
	\item European Central Bank
	\item European Central Bank and Monetary Policy in the Euro Area
	\item European Monetary Integration
	\item European Monetary Union
	\item European Union (EU) European Semester
	\item Euro Zone Crisis 2010
	\item fiscal federalism
	\item Greek crisis in perspective: origins, effects and ways-out
	\item Stability and Growth Pact
	\item Stability and Growth Pact of the European Union
\end{itemize}

Leer Valdez Molyneux ch.12 

Leer \comillas{Broken transmission} en The Economist.

\href{https://www.ecb.europa.eu/mopo/eaec/html/index.en.html}{Fuente de información general para todo el tema de política monetaria y banco central.}

\href{http://ec.europa.eu/eurostat/web/macroeconomic-imbalances-procedure/indicators/main-tables}{Indicadores del scoreboard}

Arestis, P.; Sawyer, M. (2013) \textit{Economic and Monetary Union Macroeconomic Policies. Current practices and alternatives}  -- En carpeta del tema

Banco de España. (2018) \textit{El Semestre Europeo 2018 y las recomendaciones específicas para España}  -- En carpeta del tema

Bruegel Institute. \textit{The IMF role in the Euro Area Crisis and financial sector aspects} \url{http://bruegel.org/2016/08/the-imfs-role-in-the-euro-area-crisis-financial-sector-aspects/}

Bénassy-Quéré, A.; Brunnermeir, M. K et al. \textit{How to reconcile risk sharing and market discipline in the euro area} VoxEU Article -- \url{https://voxeu.org/article/how-reconcile-risk-sharing-and-market-discipline-euro-area}

Comisión Europea. (2018) \textit{Assessment of the 2018 Stability Programme for Spain}  -- En carpeta del tema

Claeys, G.; Linta, T. (2019) \textit{The evolution of the ECB governing council's decision-making} Bruegel Blog Post -- \url{https://bruegel.org/2019/06/the-evolution-of-the-ecb-governing-councils-decision-making/}

Delgado-Téllez, M.; Kataryniuk, I.; López-Vicente, F.; Pérez, J. J. (2020) \textit{Endeudamiento supranacional y necesidades de financiación en la Unión Europea} Banco de España. Documentos ocasionales. Nº 2021 -- En carpeta del tema.

Delivorias, A. (2015) \textit{Monetary policy of the European Central Bank. Strategy, conducts and trends.}  European Parliamentary Research Service -- En carpeta del tema

European Commission. (2015) \textit{Completing Europe's Economic and Monetary Union}  ``The Five Presidents' Report'' -- En carpeta del tema

European Commission. (2014) \textit{The preventive arm of the Stability and Growth Pact in the reformed EU Governance.}  En carpeta del tema

European Central Bank (2019) \textit{Manual of MFI Balance Sheet Statistics}  -- En carpeta del tema

European Central Bank \textit{The Eurosystem's instruments.} \url{https://www.ecb.europa.eu/mopo/implement/html/index.en.html} -- En carpeta del tema

European Central Bank. (2014) \textit{The ECB's Forward Guidance} Monthly Bulletin, April 2014 -- En carpeta del tema

European Parliament. (2015) \textit{Monetary policy of the European Central Bank}  -- En carpeta del tema

European Parliament. (2015) \textit{Stability and Growth Pact -- An Overview of the Rules} Angerer Jost -- En carpeta del tema

Economic Parliament. (2018) \textit{The ESM and the IMF: comparison of the main features} Economic Governance Support Unit of the European Parliament. In-Depth Analysis -- En carpeta del tema \url{https://www.europarl.europa.eu/RegData/etudes/IDAN/2017/614485/IPOL_IDA(2017)614485_EN.pdf}

Fandl, M. (2018) \textit{Monetary and Financial Policy in the Euro Area}  Springer -- En carpeta macro

Gali, J. Perotti, R. (2003) \textit{Fiscal Policy and Monetary Integration in Europe}  NBER Working Paper Series -- En carpeta del tema

Gomes, S.; Jacquinot, P.; Pisani, M. (2010) \textit{The EAGLE. A model for policy analysis of macroeconomic interdependence in the Euro Area} ECB Working Paper-- En carpeta del tema

IMF (2019) \textit{A Capital Market Union for Europe} IMF Discussion Notes -- En carpeta del tema

Issing, O. (2005) \textit{The Monetary Pillar of the ECB}  ``The ECB and Its Watchers VII'' Conference 

Ministerio de Economía. Gobierno de España. \textit{Instrumentos Financieros en la UE} \url{http://www.mineco.gob.es/portal/site/mineco/menuitem.b6c80362d9873d0a91b0240e026041a0/?vgnextoid=e32f7cb59784c310VgnVCM1000001d04140aRCRD} (consultado el 4 de abril de 2019)

Sibert, A. (2010) \textit{The EFSM and the EFSF: Now and what follows}  European Parliament's Committee on Economic and Monetary Affairs

Unicredit (2017) \textit{The best way to track ECB rate-hike expectations} Rates Perspectives, No. 26 -- En carpeta del tema

Wolff, G. (2016) \textit{European Parliament Testimony on EDIS}  European Parliament Testimony -- Bruegel Institute -- En carpeta del tema


\end{document}
