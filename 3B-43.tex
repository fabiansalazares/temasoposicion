\documentclass{nuevotema}

\tema{3B-43}
\titulo{La política comercial de la Unión Europea}

\begin{document}

\ideaclave


Ver \href{https://ec.europa.eu/trade/policy/eu-position-in-world-trade/statistics/}{Eurostat Trade Statistics} para actualizar datos de socios comerciales.


Ver \url{https://commonslibrary.parliament.uk/brexit/no-deal-brexit-and-wto-article-24-explained/} sobre Brexit, artículo XXIV del GATT y potenciales acuerdos de libre comercio UE-Reino Unido

El debate librecambismo-proteccionismo ha tendido con el tiempo hacia un consenso de los economistas que señala la superioridad del primero frente al segundo. Así, entre los objetivos intermedios de la Unión Europea se encuentra el establecimiento de un mercado único. Este mercado único implica una unión aduanera y por tanto una política comercial común en relación al resto del mundo. Precisamente esta política comercial común llevada acabo por la UE es el objeto del tema. Para analizar las características de esta política comercial común hay que analizar las dos vertientes principales: la política comercial autónoma y la convencional. La autónoma es aquella que la Unión lleva a cabo de forma unilateral, mientras que la convencional es aquella que la Unión establece en base a acuerdos con terceras partes, incluyendo otros bloques comerciales y otros países. Además, la política comercial común debe justificarse en base a los beneficios del libre comercio. La atribución de competencias exclusivas debe también justificarse, ya que no es evidente la razón por la cual los estados miembros no llevan a cabo una política comercial externa independiente.

Se trata de un tema que complementa muy bien con los temas de la OMC y los bloques comerciales (33,34,35,36).

\begin{itemize}
	\item ¿Qué es la política comercial de la Unión Europea?
	\item ¿Por qué existe una política comercial europea?
	\item ¿Qué intereses comerciales trata de defender la Unión Europea?
	\item ¿En qué marco legal se lleva a cabo?
	\item ¿En qué consiste la política comercial autónoma de la UE?
	\item ¿En qué consiste la política comercial convencional de la UE?
\end{itemize}


\esquemacorto

\begin{esquema}[enumerate]
	\1[] \marcar{Introducción}
		\2 Contextualización
			\3 Unión Europea
			\3 Competencias de la UE
			\3 La UE en el comercio mundial
			\3 Política comercial común (PCC)
		\2 Objeto
			\3 ¿Por qué existe la política comercial europea?
			\3 ¿En qué consiste?
			\3 ¿Cuál es su marco jurídico?
			\3 ¿Qué intereses económicos trata de defender la UE?
			\3 ¿Qué actuaciones lleva a cabo la Unión Europea?
			\3 ¿Con qué objetivos?
		\2 Estructura
			\3 Aspectos generales de la política comercial
			\3 Política comercial autónoma
			\3 Política comercial convencional
	\1 \marcar{Aspectos generales de la política comercial europea}
		\2 Justificación
			\3 Mercado interior y UEM
			\3 Beneficios del CI en UE
			\3 Competitividad exterior
			\3 Promover desarrollo de PEDs
		\2 Objetivos
			\3 Preferencias de los EEMM
			\3 Intereses ofensivos
			\3 Intereses defensivos
		\2 Antecedentes
			\3 Tratado de la CECA de 1951
			\3 Tratado de Roma de 1957
			\3 Unión Aduanera
			\3 Años 70
			\3 Mercado interior
			\3 Ronda de Uruguay
			\3 Reformas PAC de 2000 y 2003
			\3 Ronda de Doha desde 2001 hasta 2015
		\2 Marco jurídico
			\3 Derecho originario
			\3 Derecho derivado
		\2 Actuaciones
			\3 Autónomas
			\3 Convencionales
			\3 Promoción exterior
	\1 \marcar{Política comercial autónoma}
		\2 Unión aduanera
			\3 Componentes
			\3 Evolución
			\3 Territorio
		\2 Política arancelaria
			\3 Arancel Aduanero Común
			\3 Reducciones arancelarias
			\3 Regímenes económicos aduaneros
		\2 Regímenes comerciales
			\3 Idea clave
			\3 Régimen de autorización
			\3 Régimen de vigilancia
			\3 Régimen de certificación
		\2 Defensa comercial
			\3 Idea clave
			\3 Antidumping
			\3 Antisubvención
			\3 Salvaguardias
			\3 Reglamento de obstáculos al comercio
			\3 Comité de Medidas de Defensa comercial
		\2 Sistema de preferencias generalizadas
			\3 Idea clave
			\3 SPG general
			\3 SPG+
			\3 EBA -- Everything But Arms
			\3 Valoración
	\1 \marcar{Política comercial convencional}
		\2 Negociación de acuerdos
			\3 Idea clave
			\3 Competencia
			\3 Procedimiento de negociación
		\2 Acuerdos multilaterales
			\3 GATT-47 y 94
			\3 GATS
			\3 TRIPS
			\3 Ronda de Doha
			\3 Conferencias ministeriales
		\2 Acuerdos plurilaterales
			\3 Idea clave
			\3 ITA -- Tecnologías de la Información
			\3 EGA -- Bienes medioambientales
			\3 TiSA -- Comercio de servicios
			\3 GPA -- Contratación pública
		\2 Acuerdos bilaterales
			\3 Idea clave
			\3 Europa
			\3 MENA
			\3 África
			\3 América
			\3 Asia
	\1[] \marcar{Conclusión}
		\2 Recapitulación
			\3 Aspectos generales de la política comercial
			\3 Política comercial convencional
			\3 Política comercial autónoma
		\2 Idea final
			\3 Valoración
			\3 Retos

\end{esquema}

\esquemalargo















\begin{esquemal}
	\1[] \marcar{Introducción}
		\2 Contextualización
			\3 Unión Europea
				\4 Institución supranacional ad-hoc
				\4[] Diferente de otras instituciones internacionales
				\4[] Medio camino entre:
				\4[] $\to$ Federación
				\4[] $\to$ Confederación
				\4[] $\to$ Alianza de estados-nación
				\4 Origen de la UE
				\4[] Tras dos guerras mundiales en tres décadas
				\4[] $\to$ Cientos de millones de muertos
				\4[] $\to$ Destrucción económica
				\4[] Marco de integración entre naciones y pueblos
				\4[] $\to$ Evitar nuevas guerras
				\4[] $\to$ Maximizar prosperidad económica
				\4[] $\to$ Frenar expansión soviética
				\4 Objetivos de la UE
				\4[] TUE -- Tratado de la Unión Europea
				\4[] $\to$ Primera versión: Maastricht 91 $\to$ 93
				\4[] $\to$ Última reforma: Lisboa 2007 $\to$ 2009
				\4[] Artículo 3
				\4[] $\to$ Promover la paz y el bienestar
				\4[] $\to$ Área de seguridad, paz y justicia s/ fronteras internas
				\4[] $\to$ Mercado interior
				\4[] $\to$ Crecimiento económico y estabilidad de precios
				\4[] $\to$ Economía social de mercado
				\4[] $\to$ Pleno empleo
				\4[] $\to$ Protección del medio ambiente
				\4[] $\to$ Diversidad cultural y lingüistica
				\4[] $\to$ Unión Económica y Monetaria con €
				\4[] $\to$ Promoción de valores europeos
			\3 Competencias de la UE
				\4 Tratado de la Unión Europea
				\4[] Atribución
				\4[] $\to$ Sólo las que estén atribuidas a la UE
				\4[] Subsidiariedad
				\4[] $\to$ Si no puede hacerse mejor por EEMM y regiones
				\4[] Proporcionalidad
				\4[] $\to$ Sólo en la medida de lo necesario para objetivos
				\4 Exclusivas
				\4[] i. Política comercial común
				\4[] ii. Política monetaria de la UEM
				\4[] iii. Unión Aduanera
				\4[] iv. Competencia para el mercado interior
				\4[] v. Conservación recursos biológicos en PPC
				\4 Compartidas
				\4[] i. Mercado interior
				\4[] ii. Política social
				\4[] iii. Cohesión económica, social y territorial
				\4[] iv. Agricultura y pesca \footnote{Salvo en lo relativo a la conservación de recursos biológicos marinos, que se trata de una competencia exclusiva de la UE}
				\4[] v. Medio ambiente
				\4[] vi. Protección del consumidor
				\4[] vii. Transporte
				\4[] viii. Redes Trans-Europeas
				\4[] ix. Energía
				\4[] x. Área de libertad, seguridad y justicia
				\4[] xi. Salud pública común en lo definido en TFUE
				\4 De apoyo
				\4[] Protección y mejora de la salud humana
				\4[] Industria
				\4[] Cultura
				\4[] Turismo
				\4[] Educación, formación profesional y juventud
				\4[] Protección civil
				\4[] Cooperación administrativa
				\4 Coordinación de políticas y competencias
				\4[] Política económica
				\4[] Políticas de empleo
				\4[] Política social
			\3 La UE en el comercio mundial
				\4 $\sim 14\%$ del CI mundial de bienes
				\4 Segundo mayor importador del mundo
				\4[] $\to$ Tras EEUU
				\4 Segundo mayor exportador del mundo
				\4[] Tras China
				\4 Mayor receptor mundial de IDE
				\4[] Por delante de EEUU
				\4[$\then$] PCComún enorme importancia a nivel mundial
				\4 Principales socios comerciales\footnote{Ver El Agraa 384 y ss.}
				\4[] 1. Comercio entre EEMM
				\4[] 2. EEUU y China
				\4[] 4. Reino Unido
				\4[] 5. EFTA
				\4[] 6. Rusia
				\4[] 7. Japón
				\4[] 8. Emergentes
				\4[] 9. Otros: $\sim$ 40\% de X y M
			\3 Política comercial común (PCC)
				\4 Conjunto de principios y actuaciones
				\4[] A nivel de instituciones europeas
				\4[] $\to$ Articulan relaciones comerciales de UE
				\4 Papel de UE predomina frente a EEMM
				\4[] $\to$ Iniciativa de negociación
				\4[] $\to$ Negociación
				\4[] $\to$ Recursos propios de presupuesto UE
		\2 Objeto
			\3 ¿Por qué existe la política comercial europea?
			\3 ¿En qué consiste?
			\3 ¿Cuál es su marco jurídico?
			\3 ¿Qué intereses económicos trata de defender la UE?
			\3 ¿Qué actuaciones lleva a cabo la Unión Europea?
			\3 ¿Con qué objetivos?
		\2 Estructura
			\3 Aspectos generales de la política comercial
			\3 Política comercial autónoma
			\3 Política comercial convencional
	\1 \marcar{Aspectos generales de la política comercial europea}
		\2 Justificación
			\3 Mercado interior y UEM
				\4 TUE.3.3 y 3.4
				\4 Libertad de circulación de ByS
				\4[] Implican eliminación de arancel interno
				\4 Unión aduanera
				\4[] Implica arancel externo común
				\4[] $\to$ Evitar problemas de origen
				\4[] $\to$ Política comercial externa coherente
				\4[] $\then$ Reducir distorsiones con PC común
			\3 Beneficios del CI en UE
				\4 Aumento de variedades
				\4 Aumento de competencia
				\4 Economías de escala
				\4[$\to$] UE es economía muy abierta
				\4[] Entre EEMM y respecto a terceros
				\4[] $\then$ Margen para obtener beneficios de CI
				\4 Especialización
				\4 Transferencia tecnológica
				\4 Home-market effects
			\3 Competitividad exterior
				\4 Maximizar demanda de export. nacionales
				\4 Mejorar términos de comercio de UE
				\4[$\then$] Mejorar competitividad sector exterior
			\3 Promover desarrollo de PEDs
				\4 CI es herramienta indispensable
				\4 Facilitar integración en mercados mundiales
				\4 Mejorar acceso europeo a inputs de PEDs
		\2 Objetivos
			\3 Preferencias de los EEMM
				\4 Múltiples y distintos intereses
				\4 Necesario agregar a nivel UE
				\4[] $\to$ Balance refleja prefs. de muchos sectores
			\3 Intereses ofensivos
				\4 NAMA -- Non Agricultural Market Access
				\4[] Ventaja comparativa sectores alto VA
				\4[] UE ya ha liberalizado acceso a su mercado
				\4 Contratación pública
				\4[] Muy amplia liberalización interna
				\4[] VC en muchos ámbitos
				\4[] $\to$ Infraestructuras de transporte
				\4[] $\to$ Infraestructura energética
				\4[] $\to$ Obras públicas
				\4[] ...
				\4 Competencia
				\4[] Marco de competencia bien definido en UE
				\4[] Interés en regulación similar extra-UE
				\4[] Evitar barreras privadas sustituyan públicas
				\4[] Poco apoyo a Comisión
				\4[] $\to$ Grupos defensa consumidor apoyan
				\4[] $\to$ USA rechaza de plano
				\4[] $\to$ PEDs argumentan falta de capacidad
				\4 Servicios
				\4[] Ventaja comparativa en ámbitos clave
				\4[] $\to$ Servicios financieros
				\4[] $\to$ Servicios a empresas
				\4[] $\to$ Turismo
				\4 Propiedad intelectual
				\4[] Sectores intensivos en PI crean VA en UE
				\4[] Interés en aumentar protección
				\4 SPS y TBT
				\4[] Mejorar reconocimiento mutuo
				\4[] Implementar normativa común
			\3 Intereses defensivos
				\4 Agricultura
				\4[] Gran interés defensivo de UE
				\4[] Arancel relativamente elevado
				\4[] $\to$ Comparado con muchos emergentes
				\4[] Grandes ayudas a sector agrícola en PAC
				\4[] $\to$ 39\% del MFP 2014-2020
				\4[] Controversias sobre ayudas PAC
				\4[] $\to$ ¿A qué caja del Acuerdo Agrícola pertenecen?
				\4[] $\then$ Actualmente, a caja verde
				\4[] $\then$ Críticas de PEDs y otros a PAC $\in$ caja verde
				\4 Denominaciones de origen
				\4[] Protección de variedades nacionales
				\4 Propiedad intelectual
				\4[] UE relativamente competitiva en prop. intelectual
				\4[] Protección deficiente en otros países
				\4[] $\to$ Daña ventaja competitiva de UE
		\2 Antecedentes
			\3 Tratado de la CECA de 1951
				\4 Libre comercio en carbón y acero
				\4[] Inputs esenciales para guerras
				\4 Evitar incentivos a buscar espacios vitales
				\4 Reducir tensiones militares
				\4 Integración frente a amenaza soviética
				\4 6 EEMM fundadores
				\4[] FRA, GER, ITA, NED, BEL, LUX
			\3 Tratado de Roma de 1957
				\4 Elimina aranceles entre miembros
				\4 Establece plan para UA en 12 años
				\4 Propone Política Agrícola Común
				\4[] Creada en 1962
			\3 Unión Aduanera
				\4 CEE presente en Ronda Kennedy 1964
				\4[] Territorio aduanero
				\4 Unión completada en 1968
				\4 Arancel Aduanero Común en vigor
			\3 Años 70
				\4 Papel destacado en Ronda Tokio
				\4 Reducción aranceles vía fórmula suiza\footnote{Aunque la fórmula fue propuesta originalmente por la delegación suiza en la Ronda de Doha, se utilizó ya en la Ronda de Tokio.}
				\4 Cassis de Dijon 1979\footnote{\href{https://circulodeempresarios.org/2013/01/29/el-cassis-de-dijon-y-la-unidad-del-mercado-nacional/}{Martínez Lage (2013).}}
				\4[] Sentencia de TJCE
				\4[] Producto legalmente comercializado en otro EM
				\4[] $\to$ Debe poder venderse legalmente en resto de Unión
				\4[] Prohibidas medidas equivalentes a cuotas
				\4 Dassonville 1974\footnote{Ver \href{https://en.wikipedia.org/wiki/Procureur\_du\_Roi\_v\_Beno\%C3\%AEt_and_Gustave_Dassonville}{Wikipedia: Dassonvile}.}
			\3 Mercado interior
				\4 Acta Única de 1987
				\4 Completa mercado interior
				\4[$\to$] Aumenta poder negociador de CEE
				\4[$\to$] Impulso liberalizador basado en reglas
			\3 Ronda de Uruguay
				\4 Tensiones previas a negociación
				\4[] PAC fuertemente criticada
				\4[] $\to$ UE subvenciona exportaciones
				\4[] $\to$ Exceso de producción europea
				\4[] $\to$ Elevada protección a importaciones
				\4 Muy importante influencia
				\4[] Como parte del ``quad''
				\4[] $\to$ USA, UE, JAP, CAN
				\4 Reducción unilateral de arancel comprometido
				\4[] Media del 4\%
				\4 Acción defensiva en agricultura
				\4[] Acepta reglas multilaterales
				\4[] $\to$ Acuerdo agrícola cajas roja azul y verde
				\4[] Muy poca liberalización efectiva
				\4 Catalizador de Reforma McSharry de 1992 de PAC
				\4[] Reducción de precios de intervención
				\4[] Ayudas públicas ligadas a restricción de producción
				\4[] $\to$ Caja azul
			\3 Reformas PAC de 2000 y 2003
				\4 1999-2000
				\4[] Continúa camino de McSharry
				\4[] Reducir precios de intervención
				\4[] Compensar agricultores parcialmente
				\4[] Introducir límites a producción
				\4 Reforma de 2003
				\4[] Desacoplamiento real
				\4[] Ayudas pasan a caja verde
				\4[] $\to$ Sin vinculación con producción
				\4[] $\to$ Pilar de desarrollo rural
			\3 Ronda de Doha desde 2001 hasta 2015
				\4 4 temas de Singapur
				\4[] Promovidos por UE
				\4[] $\to$ Inversiones
				\4[] $\to$ Competencia
				\4[] $\to$ Contratación pública
				\4[] $\to$ Facilitación de comercio
				\4 Giro progresivo de UE hacia cambio de enfoque
				\4[] De tratar de concluir Ronda de Doha
				\4[] $\to$ A concluir acuerdos plurilaterales
				\4[] $\to$ A integración regional
		\2 Marco jurídico
			\3 Derecho originario
				\4 TUE.3.3
				\4[] La Unión establecerá un mercado interior
				\4 TFUE.3 sobre competencias exclusivas
				\4[] Política comercial común
				\4[] Unión aduanera
				\4 TFUE.206 y ss.
				\4[] Política comercial común
				\4[] Establecimiento de UAduanera
				\4 Acuerdos comerciales
				\4[] Tratados internacionales UE--terceras partes
			\3 Derecho derivado
				\4 Reglamento sobre IDE
				\4 Reglamento del código aduanero de la Unión de 2013
				\4[] Modificado en 2019
				\4 Reglamento de SPG de 2012
				\4[] SPG general
				\4[] SPG+
				\4[] EBA
		\2 Actuaciones
			\3 Autónomas
				\4 Decisiones unilaterales de la UE
				\4[] $\to$ Arancel aplicado
				\4[] $\to$ Regímenes especiales
				\4[] $\to$ Preferencias generalizadas
				\4[] $\to$ Medidas de defensa comercial
				\4[] $\to$ ...
			\3 Convencionales
				\4 Acuerdos con terceras partes
				\4[] $\to$ Países
				\4[] $\to$ Bloques comerciales
			\3 Promoción exterior
				\4 No es competencia de la UE
				\4 Tendencia hacia armonización progresiva
				\4[] Evitar falsear competencia entre UE
				\4[] $\to$ En terceros mercados
	\1 \marcar{Política comercial autónoma}
		\2 Unión aduanera
			\3 Componentes
				\4 Arancel Aduanero Común
				\4[] $\to$ Perfil arancelario único
				\4[] $\to$ Código Aduanero Comunitario Modernizado
				\4 Mercado interior
				\4[] $\to$ Eliminación aranceles exp./imp.
				\4[] $\to$ Eliminación de ajustes fiscales en frontera
				\4[] $\to$ Armonización de estándares técnicos
				\4[] $\to$ Libertad de mov. de L y K
			\3 Evolución
				\4 Tratado de Roma de 1957
				\4[] Elimina aranceles entre miembros
				\4[] Plan de 12 años para establecimiento UE
				\4 Implementación completa en 1968
				\4[] Completa proceso desde T. Roma 1957
				\4 Ingresos arancelarios: recurso propio de la UE
				\4[] Aprox. 12\% en MFP actual
				\4 Acta Única Europea de 1986
				\4[] Entra en vigor en 1987
				\4[] Creación del mercado interior
			\3 Territorio
				\4 UE salvo excepciones:
				\4[] Ceuta
				\4[] Melilla
				\4[] Gibraltar
				\4 Otros países no-UE
				\4[] Turquía
				\4[] Andorra
				\4 Control de aduanas:
				\4[] EEMM mantienen cuerpos respectivos
				\4[] Aplican sistema uniforme de reglas
		\2 Política arancelaria
			\3 Arancel Aduanero Común
				\4 TARIC -- Tarif Integré de la Communauté
				\4[] Nomenclatura arancelaria
				\4[] $\to$ Designación de mercancías según producto
				\4[] Tipos arancelarios aplicables
				\4[] $\to$ Arancel Aduanero Común en sí
				\4[] $\to$ Media de tipos vigentes de aranceles
			\3 Reducciones arancelarias
				\4 Concepto
				\4[] Disminuir/eximir derecho normal aplicable
				\4 Objetivo
				\4[] Corregir desequilibrios coyunturales
				\4[] Favorecer determinadas importaciones por motivos
				\4[] $\to$ Económicos
				\4[] $\to$ Culturales
				\4[] $\to$ Científicos
				\4 Instrumentos
				\4[] Contingentes arancelarios
				\4[] $\to$ Imp. cantidad máxima con arancel inferior o nulo
				\4[] $\to$ Prohibido para productos terminados
				\4[] $\to$ Condiciones tasadas de utilización
				\4[] $\to$ Materias primas o productos semielaborados
				\4[] $\to$ Bienes escasos en la UE
				\4[] Suspensiones arancelarias
				\4[] $\to$ Dejar de aplicar o reducir temporalmente
				\4[] $\to$ Sin límite de cantidad de mercancía
				\4[] Franquicia arancelaria
				\4[] $\to$ Exenciones totales a los aranceles
			\3 Regímenes económicos aduaneros
				\4 Concepto
				\4[] Flexibilidad para determinadas situaciones
				\4[] Varios regímenes según circunstancia
				\4[i] Régimen de perfeccionamiento activo
				\4[] Importación de input sin aranceles
				\4[] $\to$ Para fabricación de productos de exportación
				\4[ii] Régimen de perfeccionamiento pasivo
				\4[] Exp. y reimportación con exención total/parcial
				\4[] $\to$ Para transformación fuera de la UE
				\4[iii] Régimen de depósito aduanero
				\4[] Depósito de mercancías extra-UE
				\4[] $\to$ Sin aplicación de medidas de pol. comercial
				\4[] $\to$ Sin derechos de importación
				\4[] Tiempo ilimitado
				\4[] $\to$ Hasta importación UE o exp. fuera UE
				\4[] Mayor flexibilidad en pago de aranceles
				\4[] $\to$ Pay-as-you-go para importador europeo
				\4[iv] Importación temporal
				\4[] Mercancías reexportadas sin transformar
				\4[v] Régimen de transformación bajo control aduanero
				\4[] Introducción de mercancía originalmente con arancel alto
				\4[] $\to$ En zona especial, entra sin arancel
				\4[] $\to$ Para transformación en otra mercancía
				\4[] $\to$ Mercancía transformada paga menor arancel
				\4[] $\then$ Transformación de productos dentro de UE
		\2 Regímenes comerciales
			\3 Idea clave
				\4 En general, régimen libre de M y X
				\4 Sin restricciones cuantitativas a miembros OMC
				\4[] $\to$ Exigido por normativa de OMC
				\4 Tres situaciones excepcionales
				\4[] $\to$ No implican restricción cuantitativa
				\4 Aplicación según país de origen
				\4[] Miembros de la OMC
				\4[] $\to$ Libre importación y exportación
				\4[] $\to$ Salvo salvaguardias
				\4[] No miembros OMC
				\4[] $\to$ Posibles restricciones cuantitativas
				\4[] Situación de embargo
				\4[] $\to$ Modificación de régimen comercial de M y X
			\3 Régimen de autorización
				\4 Necesaria licencia de importación
				\4 EEMM conceden autorización
			\3 Régimen de vigilancia
				\4 Necesario documento de vigilancia
				\4 Objetivo
				\4[] Seguimiento de operaciones de comercio
				\4[] Posible implementación medidas de salvaguardia
				\4[] $\to$ Recabar datos para valorar situación
			\3 Régimen de certificación
				\4 Necesario certificado de M o X
				\4 Verificar cumplimiento requisitos previos
				\4 Especialmente en productos agrícolas
				\4 Embargos
				\4[] Certificación previa a exportación
				\4[] P.ej.: exportación a Rusia de prod. industriales
		\2 Defensa comercial
			\3 Idea clave
				\4 Prácticas desleales en CI
				\4[] Causan perjuicio a los que sí cumplen
				\4[] $\to$ Necesarios instrumentos de defensa
				\4 OMC
				\4[] Regulación de instrumentos de defensa
				\4[] $\to$ Varios acuerdos
				\4 Solución de Diferencias en OMC
				\4[] Recursos sobre medidas aplicadas
				\4[] $\to$ Parte afectada en primer lugar
				\4[] $\to$ Parte afectada por medidas defensivas
			\3 Antidumping
				\4 Concepto de dumping
				\4[] Exportación a precio inferior a ``normal''
				\4[] Varias definiciones de ``normal''
				\4[] $\to$ Precio en mercado nacional
				\4[] $\to$ Coste de producción
				\4 Medidas antidumping
				\4[] Aranceles equivalentes a margen de dumping
				\4 Requisitos fundamentales
				\4[] i. Existencia probada de margen dumping
				\4[] ii. Perjuicio grave para industria europea
				\4[] iii. Medidas deben ser de interés comunitario
				\4 Procedimiento de investigación
				\4[] Llevado a cabo por Comisión
			\3 Antisubvención
				\4 Acuerdo Antisubvención de OMC
				\4[] Prohibir subvenciones discriminatorias
				\4[] Obligación de notificar subvenciones
				\4[] Similar antidumping
				\4[] Países y no empresas infractoras en este caso.
				\4 Dos vías de respuesta:
				\4[] -- Denuncia ante OSD
				\4[] -- Imposición de medidas compensatorias
				\4 Tres tipos de subvenciones
				\4[] Subvenciones prohibidas
				\4[] $\to$ Vinculadas a objetivos concretos de exportación
				\4[] $\to$  Vinculadas a uso de productos nacionales
				\4[] $\to$ Invocables frente a OSD
				\4[] Subvenciones recurribles
				\4[] $\to$ No están prohibidos
				\4[] $\then$ Pero pueden ser recurridos
				\4[] $\then$ Pueden imponerse medidas compensatorias
				\4[] $\then$ Deben causar efectos adversos a país que toma medidas
				\4[] $\then$ i) Hay daño a industria doméstica
				\4[] $\then$ ii) Hay perjuicio grave a exportaciones
				\4[] $\then$ iii) Perjudica acceso al mercado
				\4[] No recurribles
				\4[] $\to$ Ligadas a I+D, regiones desfavorecidas
				\4[] $\to$ Medioambiente
				\4[] $\to$ No invocables ante OSD
				\4[] $\to$ Posible recomendar abandono
				\4 Concepto de subvención
				\4[] Transferencia de gobierno extranjero a exportador
				\4[] Diferentes formas:
				\4[] $\to$ Donaciones
				\4[] $\to$ Préstamos
				\4[] $\to$ Incentivos fiscales
				\4[] $\to$ Bienes o servicios públicos
				\4[] $\then$ En condiciones más favorables que mercado
				\4 Medidas antisubvención
				\4[] Arancel correspondiente a diferencia
				\4[] $\to$ Precio de exportación subv. y no subv.
				\4 Requisitos fundamentales
				\4[] i. Subvenciones recurribles
				\4[] $\to$ i.e. concedidas a empresa o sector
				\4[] ii. Causar perjuicio actual o potencial a industria UE
				\4[] iii. Aplicación de medidas de interés comunitario
				\4 Procedimiento de investigación
				\4[] Llevado a cabo por Comisión Europea
			\3 Salvaguardias
				\4 Acuerdo de salvaguardias de WTO
				\4[] Permite imposición de protección unilateral
				\4[] $\to$ Establecer procedimientos mínimos
				\4[] Contexto de grave perjuicio a industria nacional
				\4[] Reducir uso de medidas de zona gris:
				\4[] $\to$ Restricciones voluntarias a la exportación
				\4[] $\then$ Prohibidas desde GATT-94
				\4[] Procedimiento de determinación de perjuicio grave
				\4[] $\to$ Actuaciones deben ser públicas
				\4[] $\to$ Procedimiento debe estar definido
				\4 Concepto de salvaguardia
				\4[] Medidas adoptadas ante:
				\4[] $\to$ Aumento repentino, importante, imprevisible
				\4[] $\to$ Importaciones de un bien extra-UE
				\4[] $\then$ Que ponen en peligro industria comunitaria
				\4[] Objetivos:
				\4[] $\to$ Permitir respiro transitorio
				\4[] $\to$ Margen de adaptación a industria nacional
				\4[] $\to$ Tiempo para materializar inversiones
				\4[] Sin carácter selectivo
				\4[] $\to$ No son contra país determinado
				\4[] $\to$ Para todas las importaciones
				\4 Procedimiento
				\4[] Industria afectada denuncia ante CEuropea
				\4[] CEuropea investiga daño causado
				\4[] $\to$ Decide si imponer medidas de salvaguardia
				\4 Medidas de salvaguardia
				\4[] Aranceles adicionales
				\4[] Cuotas
				\4[] Contingentes arancelarios
				\4[] Posibles medidas provisionales
			\3 Reglamento de obstáculos al comercio
				\4 Reglamento de la UE 3286/94
				\4 Objetivo
				\4[] Denunciar ante instituciones UE
				\4[] Obstáculos comerciales en 3os países
				\4 Procedimiento
				\4[] 1. Empresas/sector/EMiembro denuncia ante CE
				\4[] 2. CE inicia consultas bilaterales
				\4[] $\to$ Revisión de medidas proteccionistas
				\4[] 3. Aprobación de medidas de retorsión
				\4[] $\to$ Posible denuncia ante OSD
			\3 Comité de Medidas de Defensa comercial
				\4 Asiste CE en implementación de medidas de defensa comercial
				\4 Presidido por representante de Comisión
				\4 Representantes de todos los EEMM
				\4 Medidas sobre las que opina:
				\4[] Imponer o medidas definitivas o provisionales
				\4[] Iniciar revisiones de caducidad
				\4[] Modificación de medidas existentes
		\2 Sistema de preferencias generalizadas
			\3 Idea clave
				\4 Concepto
				\4[] Trato comercial favorable no recíproco
				\4[] Diferenciado a determinados productos
				\4[] $\to$ Procedentes de PEDs
				\4[] Aprobación mediante waiver
				\4[] $\to$ Aprobar marco sin extender a terceros
				\4[] $\then$ Exención a principio NMF
				\4 Evolución
				\4[] Implementación en UE en 1971
				\4[] Concepto de waiver en GATT en 1979
				\4 Actualidad
				\4[] Concentración en países más débiles
				\4[] Mayor graduación\footnote{El concepto de graduación hace referencia al porcentaje máximo de importaciones que el país que recibe trato favorable puede exportar a la UE. Cuando exporta más de esta cantidad (17.5\% en la actualidad), el país se gradúa.}
				\4[] Última reforma: 2016--2025
			\3 SPG general
				\4 Exención arancelaria
				\4[] Productos no sensibles
				\4[] $\to$ 66\% de todas las líneas de productos
				\4 Productos sensibles
				\4[] $-3.5\%$ respecto a tipo NMF
				\4[] $-20\%$ para textil y confección
				\4 Sensibilidad de los productos
				\4[] Existencia de producto similar en UE
				\4[] Incidencia de importación en UE
				\4[] $\then$ Sobre todo, productos industriales
				\4 Retirada de preferencias\footnote{\href{https://trade.ec.europa.eu/tradehelp/standard-gsp}{EC: GSP. Graduation.}}
				\4[] Cuando importaciones del país beneficiario
				\4[] $\to$ Sean $>17.5\%$ del total de M en agroalimentario y mineral
				\4[] Cuando valor de importaciones de beneficiario
				\4[] $\to$ Sea superior a 57\% del total de SPG de sección
				\4[] $\to$ Para textiles, del 47,2\%
				\4 Elegibilidad
				\4[] Por debajo de ingreso-medio alto
				\4[] $\to$ Según Banco Mundial
				\4[] No resulta aplicable otro esquema de SPG
				\4[] Respeto a 15 convenciones DDHH y derecho del trabajo
			\3 SPG+
				\4 Exención arancelaria
				\4[] Lista de productos cubiertos
				\4 Cumplimiento de dos requisitos
				\4[] i. Vulnerabilidad
				\4[] ii. Compromiso con convenciones internacionales
				\4[] $\to$ 27 convenciones internacionales
				\4[] $\then$ DDHH, laborales, medioambiente, gobernanza
				\4 Sin cancelar por volumen de exportación
			\3 EBA -- Everything But Arms
				\4 Exención total a todos los productos de PMA
				\4[] Según criterios de ONU
				\4 Mecanismo de retirada
				\4[] Si incumplen obligaciones
				\4[] Si perjuicio a productores comunitarios
			\3 Valoración
				\4 Críticas
				\4[] No contribuye realmente a desarrollo de PMAs
				\4[] Complejidad técnica dificulta utilización
				\4[] Tratamiento no suficientemente favorable
				\4[] Productos ``sensibles'' son los más competitivos
				\4[] Posibilidad de salvaguardias
				\4[] Aplicación a países que no lo necesitan
				\4 UE como importador de PMAs
				\4[] Mayor importador del mundo
				\4[] 76.000 millones de €
				\4 Última reforma
				\4[] Trata de corregir críticas
				\4[] Menor número de países
				\4[] Más productos no sensibles
				\4[] Más países beneficiarios de SPG+
				\4[] $\to$ Más supervisión de cumplimiento convenciones
	\1 \marcar{Política comercial convencional}
		\2 Negociación de acuerdos
			\3 Idea clave
				\4 Iniciativa en manos de UE
				\4 CdUE y PE participan en proceso
				\4[] Aprueban en última instancia
				\4[] Deben ser informados
				\4 Cambios tras Tratado de Lisboa
			\3 Competencia
				\4 Exclusiva de la UE
				\4 Comisión Europea
				\4[] Proponer inicio de negociaciones
				\4[] Llevar a cabo negociación
				\4 Consejo de la UE
				\4[] Autorizar apertura de negociación
				\4[] Aprobar acuerdo final
				\4 Parlamento Europeo
				\4[] Aprobar acuerdo final
			\3 Procedimiento de negociación
				\4 Artículos 207 y 218 del TFUE
				\4[1] CE presenta recomendaciones a CdUE
				\4[2] CdUE autoriza inicio negociación
				\4[] Otorga mandato negociador vinculante
				\4[] Define directrices de negociación
				\4[3] CE negocia con terceros
				\4[] Informando a:
				\4[] $\to$ Comité de Política Comercial\footnote{Comité especial designado por el Consejo de la Unión Europea para asistir a la Comisión. (Art. 206.3 del TFUE)}
				\4[] $\to$ Parlamento Europeo
				\4[4] CE presenta texto del acuerdo a CdUE
				\4[] CdUE aprueba por QMV
				\4[5] PE aprueba texto del acuerdo
				\4[6] Comisión firma en nombre de la UE
		\2 Acuerdos multilaterales
			\3 GATT-47 y 94
				\4 Acuerdo multilateral de OMC
				\4[] $\to$ UE es parte del acuerdo
			\3 GATS
				\4 Acuerdo multilateral de OMC
				\4[] $\to$ UE es parte del acuerdo
				\4 Importante impulsor del GATS
				\4 Interés ofensivo en servicios
				\4[] Salvo modo 4 (desplazamiento de trab.)
			\3 TRIPS
				\4 Acuerdo Multilateral de OMC
				\4[] $\to$ UE es parte del acuerdo
				\4 Interés ofensivo en propiedad intelectual
			\3 Ronda de Doha
				\4 Inicio en 2001
				\4 Rige single undertaking
				\4 Temas principales
				\4[] Agricultura
				\4[] NAMA
				\4[] Servicios
				\4[] Propiedad intelectual
				\4[] Facilitación del comercio
				\4[] Normas
				\4[] Solución de diferencias
				\4 Paralización actual
				\4[] Numerosos fracasos anteriores
				\4[] Agricultura es principal bloqueo
				\4[] $\to$ PAC es muy importante en UE
				\4[] $\to$ Intereses agrícolas muy organizados
				\4[] $\to$ Exportadores agrícolas exigen apertura
				\4[] $\then$ Muy difícil acuerdo
				\4[] UE ha tratado de mantener ronda viva
				\4[] Actualmente, negociación ``durmiente''
				\4[] $\to$ Esfuerzos trasladados a plurilaterales
				\4[] $\to$ Trump/USA presionan hacia bilaterales
			\3 Conferencias ministeriales
				\4 CM Ginebra 2011
				\4[] Adhesión de Rusia
				\4[] Aprobado waiver de servicios para PMAs
				\4 CM Bali 2013
				\4[] Paquete de Bali
				\4[] $\to$ Acuerdo de Facilitación de Comercio\footnote{Que ya ha entrado en vigor tras su firma por 2/3 de los miembros de la OMC.}
				\4[] $\to$ Seguridad alimentaria
				\4[] $\to$ Programa para futuras negociaciones
				\4[] Relativo éxito de UE
				\4 CM Nairobi 2015
				\4[] Paquete de Nairobi
				\4[] $\to$ Salvaguardias seguridad alimentaria
				\4[] $\to$ Restricciones a subvención de ex. agrícolas
				\4[] $\to$ Medidas de apoyo a PMAs
				\4[] Notificación de listas del waiver de servicios
				\4[] Sin acuerdo para continuar Ronda de Doha
				\4 CM Buenos Aires 2017
				\4[] Sin nuevos acuerdos
				\4[] Constata estancamiento
				\4 CM Astaná 2020
				\4[] Previsible choque de posturas
				\4[] $\to$ Sobre Órgano de Apelación
				\4[] $\to$ Reformas propuestas
		\2 Acuerdos plurilaterales
			\3 Idea clave
				\4 Vía de escape ante bloqueo Doha
				\4 Incluyen sólo a parte de OMC
				\4 Incentivos a free-riding
				\4[] Por cláusula NMF
			\3 ITA -- Tecnologías de la Información
				\4 Firmado en 1996
				\4 Eliminación recíproca de aranceles
				\4[] Bienes relacionados con TI
				\4 Éxito del acuerdo
				\4[] Más de 50 países han firmado
				\4[] Más del 97\% del comercio mundial
				\4 Ampliado en Nairobi 2015
			\3 EGA -- Bienes medioambientales
				\4 Negociación comienza en 2014
				\4[] Sin avances desde 2016
				\4[] Principales economías exp./imp.
				\4[] $\to$ 17 miembros OMC + UE
				\4 Liberalización comercio bienes MAmb.
				\4[] Ejemplos:
				\4[] $\to$ Turbinas solares
				\4[] $\to$ Filtros de agua
				\4[] $\to$ Equipos de medición
			\3 TiSA -- Comercio de servicios
				\4 Negociación comienza en 2013
				\4[] Sin avances desde final 2016
				\4 Ampliar GATS
				\4[] Más áreas de liberalización
				\4[] Más países
				\4 23 miembros negociando
				\4 Objetivo l/p es multilateralización
			\3 GPA -- Contratación pública
				\4 Entrada en vigor en 1996
				\4[] Ampliado y revisado en 2014
				\4 Liberalizar licitaciones públicas
				\4[] Igualdad de condiciones entre partes
				\4[] Licitaciones que superen un mínimo
				\4 Especialmente importante para la UE
				\4[] VC en provisión a Admón. Pública
		\2 Acuerdos bilaterales
			\3 Idea clave
				\4 Contexto
				\4[] Paralización Doha
				\4[] Crecimiento de demanda extra-UE en próximos años
				\4 Objetivos
				\4[] Aprovechar aumento de dda. mundial
				\4[] Tomar posiciones en PEDs
				\4 Resultados
				\4[] Mayor red de acuerdos bilaterales
				\4[] Acuerdos muy profundos de nueva generación
				\4[] DCFTA -- Deep and Comprehensive Free Trade Agreement
				\4[] $\to$ Acceso parcial al mercado único
				\4[] $\to$ Armonización de regulación
				\4[] $\to$ Compromisos sobre estado de derecho, etc...
			\3 Europa
				\4 EEE -- Espacio Económico Europeo
				\4[] Firmado en 1994
				\4[] UE+EFTA -- Suiza
				\4[] $\to$ UE + ISL + NOR + LIE
				\4[] Cuatro libertades del mercado único
				\4[] Acuerdos sectoriales
				\4[] Cooperación en otras áreas
				\4[] Áreas excluidas:
				\4[] $\to$ Unión aduanera y política comercial
				\4[] $\to$ PAC
				\4[] $\then$ Mayor mercado interior del mundo
				\4 Suiza
				\4[] Rechazo por referéndum al EEE
				\4[] Equiparar con EEE
				\4[] 7 acuerdos sectoriales
				\4[] UE es principal socio comercial de Suiza
				\4 Ucrania
				\4[] Acuerdo de Asociación general en 2014
				\4[] $\to$ Contiene proyecto de DCFTA
				\4[] DCFTA entra en vigor en 2017
				\4[] $\to$ Liberalización de comercio
				\4[] $\to$ Armonización de regulación
				\4[] $\to$ Facilitación de comercio
				\4[] $\to$ Contratación pública
				\4[] $\to$ Seguridad alimentaria
				\4[] $\to$ Competencia
				\4[] $\to$ ...
				\4[] $\to$ Protección de propiedad intelectuales
				\4[] $\to$ Prohibidas importaciones de Crimea y Sebastopol
				\4[] $\then$ Acceso selectivo al mercado interior
				\4[] Préstamos a Ucrania
				\4[] Socio estratégico en el este
				\4[] Fuerte crecimiento de exportaciones a Ucrania
				\4[] UE es mayor socio comercial de Ucrania
				\4[] Ucrania exporta materias primas y maquinaria
				\4[] UE exporta manufacturas, maquinaria, químicos
				\4[] Elevado stock de IDE de Europa hacia Ucrania
				\4 Moldavia
				\4[] DCFTA con UE desde 2016
				\4 Georgia
				\4[] DCFTA con UE desde 2016
				\4 Turquía
				\4[] Unión Aduanera desde 1995
				\4[] Productores industriales con excepciones
				\4[] $\to$ Siderurgia no incluida
				\4[] $\to$ Agricultura sólo ciertas concesiones recíprocas
				\4[] En revisión
				\4 Balcanes
				\4[] Acuerdos con ALB, SER, BOS, MON, MAC
				\4 Política Europea de Vecindad
				\4[] Marco de acuerdos de asociación:
				\4[] $\to$ UCR, MOL, GEO
			\3 MENA
				\4 Política Europea de Vecindad
				\4[] Marco de acuerdos de asociación:
				\4[] $\to$ MAR, TUN, ARG, EGI, ISR, JOR, LIB, PAL
				\4[] ARM, GEO, AZE
				\4 Marruecos
				\4[] Acuerdo de asociación de 2000
				\4[] Mecanismo de resolución de disputas
				\4[] Acuerdo de liberalización agrícola
				\4[] Explorando DCFTA\footnote{Deep and Comprehensive Free-Trade Agreement.}
				\4 Túnez
				\4[] Acuerdo de asociación en 1998
				\4[] Negociando DCFTA
			\3 África
				\4 Sudáfrica
				\4[] Acuerdo de libre comercio
				\4[] $\to$ 90\% comercio
				\4 SADC -- Comunidad Sudafricana para el Desarrollo
				\4[] Acuerdo comercial
				\4 ECOWAS
				\4[] Economic Community of Western African States
				\4[] Libre entrada para importaciones de ECOWAS
				\4[] Liberalización progresiva a entrada de productos UE
			\3 América
				\4 Estados Unidos
				\4[] Negociación de TTIP
				\4[] $\to$ 2013--2016
				\4[] Paralizado tras elección de Trump
				\4[] Muy ambicioso pero difícil de negociar
				\4 Canadá
				\4[] CETA en 2016
				\4[] Aplicado provisionalmente en algunas áreas
				\4[] $\to$ Pendiente de ratificaciones nacionales
				\4[] $\to$ Pendiente de fallo de ECJ sobre resolución de disputas
				\4 México
				\4[] En vigor TLC desde 2000
				\4[] $\to$ Gran éxito
				\4[] $\to$ Enorme aumento de flujos comerciales
				\4[] Renegociado y modernizado recientemente
				\4[] $\to$ Última ronda de negociación concluida
				\4[] $\to$ Acuerdo político concluido y firmado
				\4[] $\to$ Actualmente, revisión legal de acuerdos
				\4 Centroamérica
				\4[] Costa Rica, El Salvador, Guatemala,
				\4[] Honduras, Nicaragua y Panamá
				\4[] Pequeña importancia para UE
				\4[] Enorme importancia de UE para Centroamérica
				\4[] Acuerdo de Asociación
				\4[] $\to$ En vigor desde 2013
				\4[] $\to$ Barreras técnicas, servicios, comercial
				\4[] $\to$ Favorecer integración entre ellas
				\4[] $\to$ Cooperación al desarrollo
				\4[] $\to$ Fomentar paz seguridad, inst. democráticas
				\4 Comunidad Andina
				\4[] Colombia, Ecuador, Perú, Bolivia
				\4[] UE es socio muy importante
				\4[] $\to$ Segundo socio comercial
				\4[] $\to$ Uno de principales inversores
				\4[] $\to$ Mercado de exportación muy grande
				\4[] Notable potencial de crecimiento
				\4[] $\to$ Economías en plena expansión
				\4[] $\to$ Estabilización tras décadas de crisis
				\4[] $\to$ Sobretodo Colombia, Ecuador, Perú
				\4[] $\to$ Bolivia exportador de materias primas clave
				\4[] Acuerdo Multipartes
				\4[] $\to$ Con COL, ECU, PER
				\4[] $\to$ Liberalización comercial y barreras técnicas
				\4[] $\to$ Mejorar acceso a inversiones
				\4[] $\to$ Fomentar integración entre economías
				\4 Mercosur
				\4[] Negociación muy larga
				\4[] $\to$ Desde 1999
				\4[] Paralizada varios años
				\4[] Reanudada en 2016
				\4[] Acuerdo político en junio de 2019
				\4[] Sexto socio comercial de Unión Europea
				\4[] $\to$ Exp. de bienes de UE a Mercosur: $45.000$ M
				\4[] $\to$ Exp. de servicios de UE a Mercosur: $23.000$ M
				\4[] UE es mayor inversor extranjero en Mercosur
				\4[] $\to$ $>380.000$ M de €
				\4 Chile
				\4[] Acuerdo de asociación desde 2002
				\4[] Negociando nuevo acuerdo actualmente
				\4[] $\to$ Desde 2017
				\4[] UE es socio importante para Chile
				\4[] $\to$ Segunda fuente de importaciones
				\4[] $\to$ Tercer mercado de exportación
				\4[] Chile exportador de mercancías clave
				\4[] $\to$ Cobre y otros metales
			\3 Asia
				\4 Japón
				\4[] EU-Japan Economic Partnership Agreement
				\4[] Entrada en vigor en 2019
				\4[] Muy ambicioso:
				\4[] $\to$ Agricultura
				\4[] $\to$ Estándares laborales y medio ambiente
				\4[] $\to$ Protección al consumidor
				\4[] $\to$ Protección de datos
				\4[] $\to$ Salvaguardias sobre servicios públicos
				\4[] $\to$ Desarrollo sostenible
				\4[] $\to$ Cumplimiento de Acuerdo de París
				\4 China
				\4[] Inversiones
				\4[] $\to$ En negociación en febrero 2019
				\4[] $\to$ Eliminar barreras a la inversión
				\4[] $\to$ Aumentar protección a inversores
				\4[] $\to$ Reemplazar acuerdos bilaterales EEMM--China
				\4[] Servicios
				\4[] $\to$ Marco del TiSA
				\4[] $\to$ Sin avances desde 2016
				\4[] Bienes medioambientales
				\4[] $\to$ Marco del EGA
				\4[] $\to$ Sin avances desde 2016
				\4 India
				\4[] Inicio de negociaciones en 2007
				\4[] UE más ambiciosa que India
				\4[] $\to$ País tradicionalmente defensivo
				\4[] Negociaciones paralizadas desde 2013
				\4[] $\to$ Posible reanudación
				\4[] $\to$ Sin fecha de reanudación
				\4 Singapur
				\4[] Dos acuerdos firmados:
				\4[] $\to$ Acuerdo de Libre Comercio
				\4[] $\to$ Acuerdo de Protección de Inversiones
				\4[] Acuerdo comercial
				\4[] $\to$ Entrada en vigor en noviembre de 2019
				\4[] Acuerdo de protección de inversiones
				\4[] $\to$ Aprobado por PE
				\4[] $\to$ Pendiente de ratificación por EEMM
				\4 Vietnam
				\4[] Dos acuerdos firmados no ratificados:
				\4[] $\to$ Acuerdo de Libre Comercio
				\4[] $\to$ Acuerdo de Protección de Inversiones
				\4[] Pendiente de aprobación por PE
				\4 Indonesia
				\4[] En negociación para FTA
				\4 Filipinas
				\4[] En negociación desde 2015
				\4[] Sin acuerdos hasta la fecha
				\4 Tailandia
				\4[] Negociaciones en 2013 para FTA
				\4[] Paralizadas desde 2014
				\4 Myanmar/Birmania
				\4[] Conversaciones técnicas
	\1[] \marcar{Conclusión}
		\2 Recapitulación
			\3 Aspectos generales de la política comercial
			\3 Política comercial convencional
			\3 Política comercial autónoma
		\2 Idea final
			\3 Valoración
				\4 Poder de negociación
				\4[] Ha sufrido altibajos
				\4[] 1. Aumentó tras completar UA
				\4[] $\to$ En 1968
				\4[] 2. Presión ofensiva de USA en 70s
				\4[] $\to$ Frente a UE y Japón
				\4[] $\to$ USA abandona pol. industrial
				\4[] 3. Mucho poder en años 90s
				\4[] $\to$ Papel determinante en Uruguay
				\4[] 4. Crecimiento en emergentes
				\4[] $\to$ Muy fuerte a partir de 2000s
				\4[] $\to$ Se integran en comercio mundial
				\4[] $\to$ China entra en OMC
				\4[] $\then$ UE pierde peso relativo
				\4 Papel de UE en sistema multilateral
				\4[] Lidera impulso liberalizador
				\4[] Ha tratado de mantener Doha viva
				\4[] Ciertos debates internos
				\4[] $\to$ ¿Comercio sostenible debe negociarse?
				\4[] $\to$ ¿Reciprocidad o lib. unilateral?
				\4[] $\to$ ¿Protección agrícola debe mantenerse
			\3 Retos
				\4 Asegurar sistema multilateral
				\4[] Problemas en OSD
				\4[] EEUU no apoya bilateralismo
				\4 Reducción de barreras no arancelarias
				\4[] Aumento de las divergencias regulatorias
				\4[] Pueden obstaculizar más aún que aranceles
				\4 Fragmentación de cadenas de valor
				\4[] Sistema más sensible a barreras
				\4[] Mayor complejidad de protección
				\4 Heterogeneidad de EEMM
				\4[] Tensiones internas ante negs. comerciales
				\4[] $\to$ CETA
				\4[] $\to$ Acuerdo con Ucrania y Holanda
				\4[] $\to$ Oposición TTIP
				\4[] $\to$ Aumento voces proteccionistas
				\4 Poco que ofrecer en negociaciones
				\4[] UE más intereses ofensivos que defensivos
				\4[] Arancel medio bajo y pocos picos
				\4[] Agricultura es principal baza
				\4[] $\to$ Presión interna para mantener PAC
				\4[] $\to$ Interés ofensivo de PEDs
				\4[] $\then$ Potencial moneda de cambio
				\4 Papel del Parlamento Europeo
				\4[] Tratado de Lisboa aumentó competencia PE
				\4[] Pre-Lisboa:
				\4[] $\to$ CE y FAC del CdUE aprueban
				\4[] Post-Lisboa
				\4[] $\to$ PE debe aprobar también AComerciales
				\4 Gestionar Brexit
				\4[] UK fuerte impulso liberalizador
				\4[] $\to$ También reciprocidad
				\4[] $\to$ No partidario de negociar ``comercio sostenible''
				\4[] $\then$ ¿Nuevo equilibrio interno sin UK?
\end{esquemal}





























\conceptos

\concepto{Comité de Política Comercial}

Corresponde con el ``comité especial'' al que hace referencia el artículo 207.3 del TFUE. Asesora y asiste a la Comisión en tres aspectos:

\begin{itemize}
	\item Asuntos relacionados con la OMC.
	\item Negociaciones comerciales bilaterales.
	\item Nueva legislación europea en el ámbito de la política comercial.
\end{itemize}

El comité está presidido por el presidente rotatorio del Consejo de la Unión, actualmente Rumanía (febrero de 2019).

El comité adopta también otras configuraciones para materias específicas tales como servicios, inversiones, sectores del acero, textil y otros sectores industriales, y para acuerdos de reconocimiento mutuo.

\preguntas

\seccion{Test 2019}

\textbf{44.} Señale la afirmación correcta respecto de la Política Comercial Común de la Unión Europea (UE): 

\begin{itemize}
	\item[a] El Parlamento Europeo tiene la potestad para introducir modificaciones en los procesos de negociación y celebración de acuerdos de libre comercio entre la UE y países terceros.
	\item[b] En la OMC son miembros todos los países de la UE, pero no la UE en su conjunto.
	\item[c] La Comisión Europea habrá de ser autorizada por el Consejo de la UE y por el Parlamento Europeo para poder iniciar las negociaciones tendentes a la celebración de acuerdos de libre comercio entre la UE y países terceros.
	\item[d] La UE goza de competencia exclusiva en el ámbito de la Política Comercial Común, pero los acuerdos alcanzados pueden precisar la ratificación de los Estados miembros. 
\end{itemize}

\seccion{Test 2018}

\textbf{43.} ¿Cuál de los cuatro países que forman la European Free Trade Association (EFTA) no forma parte del Espacio Económico Europeo (EEE)?

\begin{itemize}
	\item[a] Islandia
	\item[b] Liechtenstein
	\item[c] Noruega
	\item[d] Suiza
\end{itemize}

\seccion{Test 2017}
\textbf{43.} En la negociación y finalmente conclusión de acuerdos comerciales entre la Unión Europea y países terceros:

\begin{itemize}
	\item[a] La Comisión debe informar al Parlamento Europeo a fin de que este pueda modificar los contenidos comerciales negociados si no cumplen con sus expectativas.
	\item[b] El Consejo debe pronunciarse por mayoría cualificada si se negocian/celebran acuerdos de comercio de servicios y de la propiedad intelectual e industrial.
	\item[c] El Consejo debe pronunciarse por unanimidad, especialmente si se negocia/celebran temas relacionados con servicios culturales y audiovisuales que puedan tener efectos negativos sobre la diversidad cultural y lingüística de la Unión Europea. 
	\item[d] La Comisión tiene plenas competencias para iniciar las negociaciones de cualquier acuerdo comercial, entre la Unión Europea y países terceros, sin la autorización del Consejo.
\end{itemize}

\seccion{Test 2015}
\textbf{45.} En el marco de la política comercial europea, señale la respuesta correcta:
\begin{itemize}
	\item[a] La Unión Europea acaba de firmar con EEUU un acuerdo de libre comercio, conocido por sus siglas en inglés TTIP (\textit{The Transatlantic Trade and Investment Partnership}).
	\item[b] Acaba de entrar en vigor el Acuerdo Multilateral de Libre Comercio Transpacífico, conocido por sus siglas en inglés TPP (\textit{Trans-Pacific Partnership}), que busca fomentar el comercio particularmente entre países a ambos lados del Pacífico. La UE es una de las partes que ha firmado el acuerdo.
	\item[c] Han entrado en vigor tres acuerdos comerciales entre la Unión Europea y Colombia, Perú y Centroamérica, respectivamente.
	\item[d] Ante los escasos avances, la Unión Europea ha decidido abandonar definitivamente las negociaciones para un acuerdo de libre comercio con Mercosur.
\end{itemize}

\seccion{Test 2009}
\textbf{44.} Respecto a la Política Comercial de la Unión Europea elija la respuesta \textbf{FALSA}
\begin{itemize}
	\item[a] La Política Comercial Autónoma está formada por las medidas adoptadas por la Unión Europea con carácter unilateral y discrecional respetando las obligaciones derivadas de los acuerdos multilaterales de la OMC.
	\item[b] En la Política Comercial común no se engloban acuerdos comerciales específicos con sus principales socios (Estados Unidos y Japón) y el comercio con ellos se canaliza a través de los mecanismos de la OMC y de acuerdos puntuales en sectores muy concretos.
	\item[c] El tráfico de perfeccionamiento pasivo es un tipo de régimen suspensorio que se aplica a las mercancías importadas de terceros países no comunitarios que se transforman en la UE y se reexportan fuera de ella.
	\item[d] La principal reforma que supone el nuevo Sistema de Preferencias Generalizadas (SPG) para el periodo 2009-2011 es la simplificación de los distintos sistemas en tres (SPG estándar, SPG + y SPG EBA) y el cambio en el mecanismo de graduación.
\end{itemize}

\seccion{Test 2005}
\textbf{40.} El Sistema de Preferencias Generalizadas de la UE,
\begin{itemize}
	\item[a] Permite una exención parcial arancelaria para aquellas mercancías que van a ser transformadas en terceros países no comunitarios.
	\item[b] Otorga unilateralmente determinados beneficios arancelarios a productos originarios procedentes de países en desarrollo beneficiarios.
	\item[c] Otorga rebajas arancelarias en el marco de acuerdos preferenciales de la UE.
	\item[d] Otorga unilateralmente determinados beneficios arancelarios a productos procedentes de países en desarrollo beneficiarios.
\end{itemize}

\notas

\textbf{2019:} \textbf{44.} D

\textbf{2018:} \textbf{43.} D

\textbf{2017:} \textbf{43.} C. Esta respuesta oficial es incorrecta. Ver artículo 207.4 y 218.8 del TFUE. El Consejo sólo debe pronunciarse por unanimidad para las materias especiales referidas en la respuesta, pero en general basta la mayoría cualificada.

\textbf{2015:} \textbf{45.} C

\textbf{2009:} \textbf{44.} C

\textbf{2005:} \textbf{40.} B

\bibliografia

Mirar en Palgrave:
\begin{itemize}
	\item European Union (EU) trade policy
\end{itemize}

Ver Boletín ICE sobre Política Comercial de la UE -- En carpeta del tema

Banco de España (2020) \textit{El tratado de libre comercio entre la UE y el MERCOSUR: principales elementos e impacto económico} Boletín Económico 1/2020. Artículos Analíticos -- En carpeta del tema

Bilal, S.; Hoekman, B. (2019) \textit{Perspectives on the Soft Power of EU Trade Policy} CEPR Press -- VoxEU.org Books -- En carpeta del tema

Comisión Europea. \textit{Acuerdos comerciales de la Unión Europea} \url{http://ec.europa.eu/trade/policy/countries-and-regions/negotiations-and-agreements/#\_being-negotiated}

Comisión Europea. \textit{The European Union Trade Policy 2016} (2016) -- En carpeta del tema


\end{document}
