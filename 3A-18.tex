\documentclass{nuevotema}

\tema{3A-18}
\titulo{Teoría neoclásica de la oferta y la demanda de trabajo. Análisis intertemporal de la oferta de trabajo. Teoría del capital humano.}

\begin{document}

\ideaclave

\seccion{Preguntas clave}
\begin{itemize}
	\item ¿En qué consisten los modelos neoclásicos del mercado de trabajo?
	\item ¿Cómo se modeliza la demanda?
	\item ¿Cómo se modeliza la oferta?
	\item ¿Cómo se representa la decisión de oferta de trabajo a lo largo del tiempo?
	\item ¿Qué es la teoría del capital humano?
\end{itemize}

\esquemacorto

\begin{esquema}[enumerate]
	\1[] \marcar{Introducción}
		\2 Contextualización
			\3 Mercado de trabajo
			\3 Enfoques de estudio
			\3 Modelos micro aplicadas al mercado laboral
		\2 Objeto
			\3 ¿En qué consiste el modelo neoclásico del mercado de trabajo?
			\3 ¿Cómo representa la oferta y la demanda?
			\3 ¿Qué implicaciones se derivan?
			\3 ¿Qué determina la oferta de trabajo en un contexto intertemporal?
			\3 ¿Qué es la teoría del capital humano?
			\3 ¿Qué implicaciones se derivan de ella?
		\2 Estructura
			\3 Teoría neoclásica del mercado de trabajo
			\3 Análisis intertemporal de la oferta de trabajo
			\3 Teoría del capital humano
	\1 \marcar{Teoría neoclásica del mercado de trabajo}
		\2 Idea clave
			\3 Contexto
			\3 Objetivo
			\3 Resultados
		\2 Oferta
			\3 Idea clave
			\3 Formulación
			\3 Implicaciones
			\3 Valoración
		\2 Demanda
			\3 Idea clave
			\3 Corto plazo
			\3 Largo plazo
			\3 Múltiples inputs
			\3 Implicaciones
			\3 Valoración
		\2 Equilibrio
			\3 Existencia, unicidad, estabilidad
			\3 Pleno empleo
		\2 Valoración
			\3 Inexistencia de pleno empleo
			\3 Poder de mercado en mercado de trabajo
			\3 Trabajo no es homogéneo
	\1 \marcar{Análisis intertemporal de la oferta de trabajo}
		\2 Idea clave
			\3 Modelo estático
			\3 Dimensión temporal
			\3 Oferta de trabajo y sustitución intertemporal
		\2 Formulación
			\3 Problema de maximización
			\3 Condiciones de primer orden
		\2 Implicaciones
			\3 Efectos de variación del salario
			\3 Elasticidad de Frisch
		\2 Valoración
			\3 Modelo del ciclo real
			\3 Elasticidades micro y macro de la oferta de trabajo
			\3 Problemas empíricos
	\1 \marcar{Teoría del capital humano}
		\2 Idea clave
			\3 Trabajo no es homogéneo
			\3 Relación consistente entre educación y salario
			\3 Diferentes teorías de la demanda de educación
			\3 Teoría del capital humano
		\2 Formulación
			\3 Supuestos generales
			\3 Decisión sobre años de educación
			\3 Decisión sobre pago de educación
			\3 Emigración y capital humano
		\2 Implicaciones
			\3 Educación como inversión
			\3 Optimalidad de la educación
			\3 Salarios de trabajo no cualificado
			\3 Habilidades
			\3 Demografía
			\3 Ecuación de Mincer
		\2 Valoración
			\3 Modelos alternativos
			\3 Política económica
	\1 \marcar{Conclusión}
		\2 Recapitulación
			\3 Teoría neoclásica del mercado de trabajo
			\3 Análisis intertemporal de la oferta de trabajo
			\3 Teoría del capital humano
		\2 Idea final
			\3 Aspectos sin analizar
			\3 Relevancia de los modelos anteriores

\end{esquema}

\esquemalargo

\begin{esquemal}
	\1[] \marcar{Introducción}
		\2 Contextualización
			\3 Mercado de trabajo
				\4 Especial importancia
				\4[] Trabajo remunerado es principal fuente de renta
				\4[] Condiciona actividad humana
				\4[] $\to$ Fracción importante del tiempo
			\3 Enfoques de estudio
				\4 Macroeconómico:
				\4[] Entender y predecir vars. agregadas
				\4[] $\to$ paro, ocupación, duración del paro...
				\4 Microeconómico:
				\4[] Entender y predecir decisión individual
				\4[] $\to$ ¿trabajar o no?
				\4[] $\to$ ¿cuánto tiempo dedicar al trabajo?
				\4[] $\to$ ¿qué salario exigir por el trabajo?
				\4[] $\to$ ¿cuánto trabajo aplicar al proceso productivo?
				\4[] $\to$ ¿qué relación entre trabajo y otros factores?
				\4[] $\to$ ¿qué salario ofrecer?
				\4[] $\to$ ¿cómo repartir trabajo entre miembros de la familia?
				\4[] $\to$ ¿cuánto tiempo dedicar a la búsqueda de empleo?
				\4[] $\to$ ¿cuánta educación obtener?
			\3 Modelos micro aplicadas al mercado laboral
				\4 Adaptados a diferentes fenómenos
				\4 Teoría neoclásica
				\4[] $\to$ Inf. perfecta sobre precios y características
				\4[] $\to$ Homogeneidad del bien en cada mercado
				\4[] $\to$ Ausencia de externalidades
				\4[] $\to$ Racionalidad perfecta
				\4[] $\to$ Enfoque de equilibrio
				\4 Modelos intertemporales
				\4[] Dimensión intertemporal de la decisión
				\4[] $\to$ Costes de ajuste
				\4[] $\to$ Descuento subjetivo
				\4[] $\to$ Transferencia intertemporal de renta
				\4 Capital humano
				\4 Modelos de búsqueda aplicada al mercado laboral
				\4 Modelos de negociación empresas-trabajadores
		\2 Objeto
			\3 ¿En qué consiste el modelo neoclásico del mercado de trabajo?
			\3 ¿Cómo representa la oferta y la demanda?
			\3 ¿Qué implicaciones se derivan?
			\3 ¿Qué determina la oferta de trabajo en un contexto intertemporal?
			\3 ¿Qué es la teoría del capital humano?
			\3 ¿Qué implicaciones se derivan de ella?
		\2 Estructura
			\3 Teoría neoclásica del mercado de trabajo
			\3 Análisis intertemporal de la oferta de trabajo
			\3 Teoría del capital humano
	\1 \marcar{Teoría neoclásica del mercado de trabajo}
		\2 Idea clave
			\3 Contexto
				\4 Neoclásico
				\4 Supuestos habituales de modelos estándar
				\4 Bien intercambiado: horas de trabajo
			\3 Objetivo
				\4 ¿De qué depende la oferta de trabajo?
				\4[] Decidir entre consumo vs ocio/trabajo
				\4[] Más consumo $\Rightarrow$ Menos ocio/Más trabajo
				\4 ¿De qué depende la demanda de trabajo?
				\4[] Beneficio marginal positivo del trabajo
				\4[] Más trabajo:
				\4[] $\to$ Más producción
				\4[] $\to$ Mayor coste salarial
			\3 Resultados
				\4 Bajo supuestos generales
				\4 Equilibrio único
				\4[] Demanda de trabajo iguala oferta
				\4[] Empresas maximizan beneficio $\to$ 0
				\4[] Consumidores maximizan utilidad
		\2 Oferta\footnote{Ch. 1 de Cahuc y Zylbergerg.}
			\3 Idea clave
				\4 Contexto
				\4[] Tiempo limitado a distribuir entre:
				\4[] $\to$ Ocio
				\4[] $\to$ Trabajo $\to$ Consumo
				\4[] $\then$ Decide entre ocio y consumo
				\4[] Precio del ocio
				\4[] $\to$ Salario $w$
				\4[] Unidad adicional de ocio:
				\4[] $\then$ $w$ unidades menos de consumo
				\4[] Ocio
				\4[] $\to$ Bien normal:
				\4[] $\to$ Más renta $\then$ más demanda de ocio
				\4[] $\to$ Más renta $\then$ Menos oferta de trabajo
				\4[] Consumo
				\4[] $\to$ Bien normal
				\4[] $\then$ Más consumo con más renta
				\4 Resultados
				\4[] Oferta de trabajo
				\4[] $\to$ Resultado de efectos contrapuestos
				\4[] Efecto sustitución:
				\4[] $\to$ $\uparrow$ salario encarece ocio
				\4[] $\then$ Aumenta oferta de trabajo
				\4[] Efecto renta:
				\4[] $\to$ $\uparrow$ salario aumenta renta
				\4[] $\then$ Disminuye demanda de trabajo
				\4[] ¿Qué efecto prevalece?
				\4[] $\then$ Determina oferta de trabajo
			\3 Formulación
				\4 Maximización de la utilidad
				\4[] $\underset{c,l}{\max} \quad u(c,l)$
				\4[] $\text{s.a:} \quad \quad c + w \cdot l \leq w \cdot l_0 + R_0 \equiv R(w)$
				\4[] $\text{CPO:} \quad$ \fbox{$w=\frac{\partial u / \partial l}{\partial u / \partial c} = \left| \text{RMS}_{l,c}\right|$}
				\4[] Donde:
				\4[] $c$ $\to$ consumo
				\4[] $l$ $\to$ ocio
				\4[] $l_0$ $\to$ tiempo total disponible
				\4[] $R$ $\to$ renta no salarial
				\4 Representación gráfica
				\4[] Tangencia entre CI y RP en espacio $l$--$c$
				\4[] \grafica{ofertadetrabajo}
			\3 Implicaciones
				\4 Demandas
				\4[] Ocio: $l^* = l(w, wl_0+R_0) = l(w, R(w))$
				\4[] $\Rightarrow$ Trabajo: $h^* = l_0 - l(w, wl_0 + R_0) = h(w,R(w))$
				\4[] Consumo: $c^* = c(w, l_0, wl_0 + R_0) = l(w,R(w))$
				\4 ER y ES sobre demanda de ocio
				\4[] Descomponiendo\footnote{Ver ``\textit{efecto renta directo e indirecto}'' en conceptos.} $\dfrac{d \, l}{d \, w}$ en $\text{ES}+\text{ER}_\text{indirecto}$ y $\text{ER}_\text{directo}$
				\4[] (i) $\dfrac{d \, l}{d \, w} = \underbrace{\pdv{l}{w}}_{\text{ES}+\text{ER}_i} + \underbrace{\pdv{l}{R} \cdot l_0}_{\text{ER}_\text{d}}$
				\4[] Descomponiendo $\pdv{l}{w}$ en ES y $\text{ER}_\text{indirecto}$ (Slutsky)
				\4[] (ii) $\pdv{l}{w} = \underbrace{\pdv{l^h}{w}}_{\text{ES}<0} - \underbrace{l \cdot \pdv{l}{R}}_{\text{ER}_i>0}$
				\4[] Sustituyendo (ii) en (i)
				\4[] \fbox{$\dfrac{d \, l}{d \, w} = \underbrace{\pdv{l^h}{w}}_\text{ES} + \underbrace{\pdv{l}{R}(l_0 - l)}_{\text{ER}_d - \text{ER}_i}$}
				\4 Efectos de salario sobre oferta de trabajo
				\4[] $\text{ES}<0$ siempre
				\4[] $\to$ Porque más salario hace más caro ocio
				\4[] $\text{ER} = \text{ER}_d - \text{ER}_i > 0$ siempre
				\4[] $\to$ Porque ocio es bien normal y $l_0 \geq l$
				\4[] Efecto global de salario sobre oferta de trabajo
				\4[] $\to$ Depende de si $\text{ES}$ mayor o menor que $\text{ER}$
				\4[] Si $\text{ES} > \text{ER}$
				\4[] $\to$ Más salario reduce demanda de ocio
				\4[] $\then$ Aumenta oferta de trabajo con salario
				\4[] Si $\text{ES} < \text{ER}$
				\4[] $\to$ Más salario aumenta demanda de ocio
				\4[] $\then$ Cae la oferta de trabajo con salario
				\4 Elasticidad-salario de la oferta de trabajo
				\4[] En términos de elasticidades y trabajo:
				\4[] (iii) \fbox{$\epsilon_{h-w} = \underbrace{\epsilon_{h-w}^h}_{\text{ES}} + \underbrace{s_{(wh)/R} \cdot \epsilon_{h-R}}_{\text{ER}}$}
				\4[] $\to$ $\epsilon_{h-w}$: elasticidad-salario del trabajo
				\4[] $\to$ $\epsilon_{h^h-w}$: elasticidad-salario compensada
				\4[] $\to$ $s_{((wh)/R)}$: proporción de renta debida a salario
				\4[] $\to$ $\epsilon_{h-R}$: elasticidad-renta del trabajo (<0)
%				\4[] Efectos contrapuestos
%				\4[] $\to$ $\text{ER}_d > 0$ (siempre\footnote{Porque ante aumentos de la renta monetaria, siempre querrá consumir más ocio.})
%				\4[] $\to$ $\text{ER}_i > 0$
%				\4[] $\to$ $\text{ER}_d > \text{ER}_i$
%				\4[] $\then$ $\text{ER}_d - \text{ER}_i > 0$
%				\4[] $\then$ Cae oferta de trabajo
%				\4[] Si $\text{ES}-\text{ER}_i$ < 0 (siempre\footnote{Porque ante caídas de la renta real disponible para ocio, siempre querrá consumir menos. Y además, el efecto sustitución es evidentemente negativo cuando aumenta el precio del ocio.}) 
%				\4[] $\then$ Aumenta oferta de trabajo
%				\4[] Si $\text{ES} - \text{ER}_i > \text{ER}_d$ $\then$ $\dfrac{d \, l}{d \, w} < 0 \then \dfrac{d \, h}{d \, w} > 0$
%				\4[] $\to$ Predomina efecto sustitución
%				\4[] $\to$ Más salario aumenta oferta de trabajo
%				\4[] Si $\text{ES} - \text{ER}_i < \text{ER}_d$ $\Rightarrow$ $\dfrac{d \, l}{d \, w} > 0 \then \dfrac{d \, h}{d \, w} < 0$
%				\4[] $\to$ Predomina efecto renta
%				\4[] $\to$ Más salario reduce oferta de trabajo
				\4[] Representación gráfica en ocio--consumo
				\4[] \grafica{eserconsumoocio}
				\4[] Representación gráfica salario--trabajo
				\4[] \grafica{esertrabajosalario}
				\4 Salario de reserva
				\4[] Mínimo salario $\ubar{w}$ para trabajar horas positivas
				\4[] Máximo salario $\ubar{w}$ para demandar máximo de ocio
				\4[]
				$\left.\frac{u_l}{u_c}\right|_{\tiny l=l_0} > w$ $\Rightarrow$ $h=0, l=l_0, w < \ubar{w}$
				\4[] \grafica{salariodereserva}
				\4 Oferta agregada
				\4[] Oferta individual:
				\4[] $\to$ $\text{ER}_d$ acaba predominando
				\4[] $\then$ Oferta se reduce a partir de cierto $w$
				\4[] Oferta agregada:
				\4[] $\to$ ER reduce trabajo pero aumenta participación
				\4[] $\then$ OAgregada generalmente creciente en salario
			\3 Valoración
				\4 Enorme literatura empírica
				\4[] ¿El modelo neoclásico explica oferta?
				\4[] ¿Qué parámetros caracterizan oferta?
				\4[] ¿Cómo estimar oferta?
				\4 Oferta individual respecto a salario
				\4[] Efectivamente, ES > ER Primero
				\4[] ER predominante después
				\4 Margen intensivo y extensivo
				\4[] Extensivo: trabajar o no
				\4[] Intensivo: trabajar más o menos horas
				\4[] Elasticidad de extensivo
				\4[] $\to$ Mayor que elasticidad de intensivo
				\4[] $\then$ Compatible con indivisibilidad de oferta
				\4 Hombres y mujeres
				\4[] Elasticidades de oferta de hombres
				\4[] $\to$ Muy reducidas
				\4[] $\to$ Menores que para mujeres
				\4[] $\then$ Fiscalidad afecta participación de mujeres
				\4[] Explicación:
				\4[] $\to$ Trabajo mujer más sustituible que hombre
		\2 Demanda\footnote{Ch. 4 de Cahuc y Zylberberg.}
			\3 Idea clave
				\4 Contexto
				\4[] Trabajo es input esencial de procesos productivos
				\4[] Todas las empresas demandan trabajo en mayor/menor medida
				\4[] Empresas en contexto neoclásico
				\4[] $\to$ ``Cajas negras'' convierten inputs en outputs
				\4[] $\then$ Maximizar beneficios dado precios de inputs y outputs
				\4[] Diferentes elasticidades de demanda en horizontes temporales
				\4[] $\to$ A corto plazo, oferta de K menos elástica
				\4[] $\then$ Difícil sustitución trabajo por capital
				\4[] Competencia perfecta en el mercado de trabajo
				\4[] $\to$ Empresas precio aceptantes respecto a salario
				\4[] $\to$ Trabajadores precio aceptantes respecto salario
				\4 Objetivos
				\4[] Efectos de $\Delta$ en salario sobre dda. de trabajo
				\4[] Efecto de $\Delta$ en coste de K sobre dda. de trabajo
				\4[] Efecto de $\Delta$ en precio de bienes sobre dda. de trabajo
				\4 Resultados
				\4[] Análisis de corto y largo plazo
				\4[] Corto plazo
				\4[] $\to$ Trabajo no es sustituible por capital
				\4[] $\to$ Demanda de trabajo no depende coste de K
				\4[] $\then$ Menor elasticidad de la demanda de trabajo
				\4[] Análisis de largo plazo
				\4[] $\to$ Trabajo y capital sustitutivos
				\4[] $\to$ Efecto producción
				\4[] $\to$ Efecto escala
				\4[] $\then$ ¿Cambio en producción cambia relación K-L?
				\4[] $\to$ Efecto sustitución
				\4[] $\then$ ¿Cambio en precio de input cambia rel. K-L?
			\3 Corto plazo
				\4 Trabajo es el único factor sobre el que decidir
				\4 Caracterizar efecto escala
				\4[] $\to$ ¿Cuánto producir?
				\4[] $\to$ ¿Cuánto trabajo emplear?
				\4[] $\to$ ¿Cómo afecta la competencia en el mercado de bienes?
				\4 Problema de maximización de la empresa
				\4[] Asumiendo $Y''(L) < 0$ $\to$ F. de prod. cóncava\footnote{$v$ es medida inversa de poder de mercado. Valores de $v$ próximos a 1 indican ausencia de poder de mercado/competencia perfecta.}
				\4[] $\underset{L}{\max} \quad \pi(L) = P(Y(L)) \cdot Y(L) - W L$
				\4[] $\text{CPO:} \quad \pi'(L) = P' \cdot Y' \cdot Y + P \cdot Y'-W = 0$
				\4[] $\then$ $P\cdot Y'\cdot \left( 1+\frac{1}{\epsilon_{Y-P}} \right) = W$
				\4[] $\then$ $ Y'(L) = \frac{W}{P} \cdot \left( \frac{1}{1+\frac{1}{\epsilon_{Y-P}} } \right) = \frac{W}{P} \cdot \left( \frac{1}{1-|\epsilon_{P-Y}|} \right) $
				\4[] $\Rightarrow$ Producción que iguala $\text{PMg}$ con salario real
				\4[] $\then$ $P = \frac{W}{Y'(L)} \cdot \left( \frac{1}{1-|\epsilon_{P-Y}|} \right) = \text{CMg}(L) \cdot \left( \frac{1}{1-|\epsilon_{P-Y}|} \right) $
				\4[] $\Rightarrow$ Empresa fija precio añadiendo mark-up a coste marginal
				\4[] $|\epsilon_{P-Y}| = 0$: demanda perfectamente elástica
				\4[] $|\epsilon_{P-Y}| > 0$: poder de mercado
				\4[] Dado que $F'(L) > 0, F''(L) < 0$
				\4[] $\then$ \fbox{$L^D = L(\overset{-}{W},\overset{+}{P}, \overset{-}{|\epsilon_{P-Y}|})$}
				\4 Competencia perfecta en el mercado de bienes:
				\4[] $\to$ Más producción
				\4[] $\then$ Más demanda de trabajo
				\4 Poder de mercado en el mercado de bienes:
				\4[] $\to$ Menos producción
				\4[] $\then$ Menos demanda de trabajo
			\3 Largo plazo
				\4 Empresa decide cuánto utilizar de cada factor
				\4 Caracterizar efectos sustitución y escala
				\4[] Efecto sustitución (ES):
				\4[] $\to$ Dado cambio en coste de factores
				\4[] $\to$ Dada cantidad a producir fija
				\4[] $\then$ cambio en $L/K$
				\4[] Efecto producción (EP):
				\4[] $\to$ Dado cambio en precio de bien producido
				\4[] $\to$ Dado cambio en coste de factores
				\4[] $\then$ Cambio en demanda de trabajo
				\4 Asumiendo:
				\4[] i) Función de producción es homogénea de grado $k$
				\4[] $\then$ Es homotética
				\4[] $\then$ Escala no altera proporciones de factores
				\4[] ii) $Y_L, Y_K >0, Y_{LL}, Y_{KK} < 0$
				\4[] iii) $Y_{LK} = Y_{KL} > 0$
				\4[] $\to$ $\theta^k Y(L,K) = Y(\theta L, \theta K)$
				\4 Problema de minimización de costes:
				\4[] Caracterizar efecto sustitución
				\4[] $\underset{L,K}{\min} \quad WL+RK$
				\4[] $\text{s.a:} \quad \quad Y(L,K) \geq \bar{Y}$
				\4[] $\mathcal{L} = WL+RK - \lambda (Y(L,K) - \bar{Y})$
				\4[] $\text{CPO:} \quad \quad \mathcal{L}_L = W - \lambda Y_L = 0$
				\4[] $\text{CPO:} \quad \quad \mathcal{L}_K = R - \lambda Y_K = 0$
				\4[] $\then$ \fbox{$\frac{Y_L}{Y_K} = \frac{W}{R}$}
				\4 Función de costes $C(\bar{Y},W,R)$
				\4[] cumple propiedades habituales
				\4[] i) Homogénea de grado 1 en W y R
				\4[] ii) Cóncava en R y W
				\4[] iii) Cumple lema de Shephard
				\4[] iv) Homogénea de grado $1/k$ en $\bar{Y}$:
				\4[] Donde $k$ es grado de homogeneidad de $Y(L, K)$
				\4[] Aplicando Lema de Shephard:
				\4[] $\to$ Posible derivar demandas condicionadas
				\4[] $\then$ \fbox{ $\bar{L}(W,R,\bar{Y}) = C_W(W,R, \bar{Y}) $}
				\4[] $\then$ \fbox{ $\bar{K}(W,R,\bar{Y}) = C_R(W,R, \bar{Y}) $}
				\4[] Elasticidad de sustitución entre factores
				\4[] $\to$ $\Delta$ \% de $L/K$ ante $\Delta$ \% de precio relativo
				\4[] $\to$ Caracteriza efecto sustitución
				\4[] \fbox{$ \sigma = \frac{\text{d} \, (K/L) / (K/L) }{\text{d} \, (W/R) / (W/R) } = \frac{d \ln (K/L)}{d \ln (W/R)}= \frac{C C_{WR}}{C_W C_R}$}
				\4 Elast-salario y elast-renta de la dda. condicional
				\4[] \fbox{$\epsilon_{\bar{L}-W} = -(1-s)\sigma = -\epsilon_{\bar{L}-R}$}
				\4[] $\to$ Donde $s=\frac{W\bar{L}}{C}$ partipación del trabajo en coste
				\4[] Especialmente importante a nivel empírico
				\4[] $\to$ ¿Cuánto cae la demanda de trabajo si aumenta el salario?
				\4[] $\to$ ¿Cuánto aumenta la demanda de trabajo si aumenta la rent. del K?
				\4 Problema de maximización de beneficios
				\4[] Caracterizar efecto Producción
				\4[] $\underset{L,K}{\max} \quad \Pi =P \cdot Y(K,L) - WL - RK$
				\4[] $\text{CPO:} \quad \quad P \cdot Y_L - W = 0$
				\4[] $\text{CPO:} \quad \quad P \cdot Y_K - R = 0$
				\4[] $\then$ \fbox{$\frac{Y_L}{Y_K} = \frac{W}{R}$}
				\4[] $\then$ $Y^*=Y(P, W, R)$
				\4 Función de beneficios $\Pi(P,W,R)$
				\4[] Derivable a partir de CPO
				\4[] Aplicando Lema de Hotelling:
				\4[] $\to$ Posible derivar demandas incondicionales
				\4[] $\then$ \fbox{$L^*(W,R) = -\Pi_W = C_W(W,R,Y^*(P,W,R))$}
				\4[] $\then$ \fbox{$K^*(W,R) = -\Pi_R = C_R(W,R,Y^*(P,W,R))$}
				\4 Elasticidades de la dda. incondicionada de trabajo:
				\4[] Diferenciando $L^* = C_W$ respecto de W:
				\4[] $\pdv{L^*}{W} = C_{WW} + C_{WY}\cdot Y^*_W$
				\4[] Multiplicando por $\frac{W}{L^*}$ y operando:
				\4[] \underline{Elasticidad de demanda de trabajo respecto salario}
				\4[] \fbox{$\epsilon_{L^*-W} = \underbrace{\epsilon_{\bar{L}-W}}_{\text{ES}} + \underbrace{\epsilon_{\bar{L}-Y} \epsilon_{Y-W}}_{EE}$}
				\4[] ES, EE < 0
				\4[] Si función de producción es homogénea:
				\4[] $\epsilon_{L-W} = - (1-s)\sigma - \frac{v}{v-\theta}s $
				\4[] $\epsilon_{L-R} = (1-s)\left( \sigma - \frac{v}{v-\theta} \right)$
				\4[] $\to$ Donde $v = \frac{1}{1+\epsilon_{P-Y}}$: markup bruto
				\4[] $\then$ Más poder de mercado $\to$ Menos efecto escala

				\4[] \underline{Elasticidad de demanda de trabajo respecto interés}
				\4[] \fbox{$\epsilon_{L^*-R} = \underbrace{\epsilon_{\bar{L}-R}}_{\text{ES}} + \underbrace{\epsilon_{\bar{L}-R} \epsilon_{Y-W}}_{EE}$}
				\4[] ES > 0
				\4[] ER < 0 si homogénea
				\4[] $\then$ ES+ER es ambigua si f. homogénea
				\4[] Si $\epsilon_{L^*-R} = ES+EE > 0$ $\then$ L y K sust. brutos
				\4[] Si $\epsilon_{L^*-R} = ES+EE < 0$ $\then$ L y K comp. brutos
			\3 Múltiples inputs
				\4 Generalizable a número arbitrario de inputs
				\4 Gran diferencia:
				\4[] Efecto sustitución ya no tiene por qué ser negativo
				\4[] Con 2 inputs, si $\uparrow$ precio de un factor
				\4[] $\to$ Será sustituido por otro
				\4[] $\then$ Necesariamente aumentará consumo de otro
				\4[] Con >2 inputs, si $\uparrow$ precio de factor $i$
				\4[] $\to$ No necesariamente sustituido por $j$
				\4[] $\to$ Puede ser sustituido por $k$
				\4[] $\to$ $k$ puede ser complementario de $i$
				\4[] $\then$ Consumo de $j$ no aumenta necesariamente
				\4[] $\then$ Consumo de $i$ no decrece necesariamente
			\3 Implicaciones
				\4 Aproximada concordancia con Leyes H-M
				\4[] Leyes de la demanda derivada de Hicks-Marshall
				\4[] La elast. de demanda de trabajo\footnote{O de cualquier otro factor de producción.} $\uparrow$ si...
				\4[1] ....$\uparrow$ la elasticidad de demanda del output
				\4[2] ...$\uparrow$ la elasticidad de sustitución del factor
				\4[3] ...$\uparrow$ la elasticidad de oferta de otros factores
				\4[4] ...$\uparrow$ el peso del factor en el coste total
				\4[] Resultados de modelo más o menos compatibles
				\4[] $\to$ Salvo peso del factor en coste total
				%\4[1] ...más elástica sea la dda. del output
				%\4[2] ...más alta sea la elast. de sustitución del factor
				%\4[3] ...más elástica sea la elast. de oferta de otros factores
				%\4[4] ...mayor sea el peso del factor en el coste total\footnote{Cuando la elasticidad del output sea mayor que la de la demanda de factor.}
			\3 Valoración\footnote{Ver sección 2.2 ``Main Results'' de Ch. 4 Labor demand de Cahuc and Zylberberg sobre estimaciones de elasticidades..}
				\4 Elasticidad de demanda es parámetro clave
				\4[] $\to$ Efecto del salario mínimo sobre empleo
				\4[] $\to$ Efecto de inmigración sobre salario
				\4[] $\to$ Efectos de negociación colectiva sobre empleo
				\4 Elasticidad de demanda compensada
				\4[] Ante $\Delta$ w >O, ¿cuánto trabajo sustituido
				\4[] ...asumiendo nivel de producción constante?
				\4[] $\to$ A nivel agregado
				\4[] estimaciones menores a 1 (VAbsoluto)
				\4[] $\to$ Media aproximada: 0.3
				\4[] Fuertemente dependiente del sector
				\4[] $\to$ Poco cualificado más sustituible por capital
				\4 Horas y trabajadores
				\4[] En la práctica, no son sustitutos
				\4[] $\to$ Horas (H)/trabajador (N) afecta productividad
				\4[] Posible variar modelo:
				\4[] $\to$ H y N en vez de L
				\4 Funciones CES
				\4[] Más habituales para modelizar f. de prod.
		\2 Equilibrio
			\3 Existencia, unicidad, estabilidad
				\4 Formulaciones simples lo asumen
				\4 Intersección de oferta y demanda
				\4[] Bajo supuestos generales
				\4[] $\to$ Un sólo equilibrio estable
			\3 Pleno empleo
				\4 Todos los que quieren trabajar
				\4[] $\to$ Pueden vender su trabajo
				\4 Toda la demanda a un precio determinado
				\4[] $\to$ Es satisfecha por oferta
		\2 Valoración
			\3 Inexistencia de pleno empleo
				\4 En la práctica, no se alcanza equilibrio
				\4[] $\to$ Agentes ofertan trabajo y no venden
				\4[] $\to$ Vacantes no se cubren
				\4[] $\to$ Oferta y demanda no se encuentran
			\3 Poder de mercado en mercado de trabajo
				\4 Negociación para determinar salario
				\4 Presencia de sindicatos, patronales
			\3 Trabajo no es homogéneo
				\4 Muchos mercados de trabajo
				\4 Calidad de trabajo ofrecido
				\4[] Es factor exógeno
	\1 \marcar{Análisis intertemporal de la oferta de trabajo}
		\2 Idea clave
			\3 Modelo estático
				\4 Decidir entre ocio y consumo
				\4[] Un periodo, una sola decisión
				\4 Sustituir ocio y consumo
				\4[] $\to$ Hasta que aporten mismo bienestar
			\3 Dimensión temporal
				\4 Decidir entre ocio y consumo
				\4[] A lo largo de diferentes periodos
				\4 Posible transferir riqueza entre periodos
				\4[] A tipo de interés determinado
				\4 Posible sustituir consumo entre periodos
				\4 Posible sustituir ocio entre periodos
				\4[$\then$] Cuánto trabajar hoy respecto a mañana?
				\4[$\then$] Cuánto consumir hoy respecto a mañana?
			\3 Oferta de trabajo y sustitución intertemporal
				\4 Objetivo es caracterizar:
				\4[] Cuánto consumo y cuánto ocio hoy
				\4[] Cuánto consumo hoy y cuanto mañana
				\4[] Cuánto trabajo hoy y cuánto mañana
				\4[] $\to$ Ante shocks salariales
				\4 Shocks salariales
				\4[] Se caracterizan por extensión temporal
				\4[] $\to$ Permanentes
				\4[] $\to$ Transitorios
				\4[] ¿Cómo afectan a oferta de trabajo hoy y mañana?
		\2 Formulación
			\3 Problema de maximización
				\4[] $\underset{\left\lbrace C_t \right\rbrace, \left\lbrace L_t \right\rbrace}{\max} \quad \mathcal{U} = \sum_{t=0}^{\infty} \beta^t U(C_t, L_t)$
				%\4[] $\text{s.a:} \quad \quad C_t = (1+r_t) A_{t-1} - A_t + R_t + w_t (1-L_t) \quad \forall t$
				%\4[] (Restricción intertemporal:
				\4[] $\text{s.a:} \quad \quad \sum_{t=0}^\infty C_t \cdot \frac{1}{(1+r)^t}= \sum_{t=0}^\infty (1-L_t)w_t \cdot \frac{1}{(1+r)^t}$
				%\4[] Donde:
				%\4[] $A_t$: activos en fecha $t$
				%\4[] $R_t$: renta exógena en $t$
%				\4[] $\mathcal{L} = \sum_{t=0}^{t=T} U(C_t, L_t) - \sum_{t=0}^T \lambda_t \left( C_t - (1+r_t) A_{t-1} + A_t + - B_t - w_t (1-L_t) \right)$ 
			\3 Condiciones de primer orden
				\4 CPOs:
				\4[] $U_{C,t} = \lambda_t$
				\4[] $U_{L,t} = \lambda_t \cdot w_t$
				\4[] $\lambda_t = (1+r_t) \lambda_{t+1}$
				\4[] $\to$ $\lambda_t$: utilidad marginal de riqueza en $t$
				\4[] $\to$ $\lambda_t = \lambda_t(\lambda_0(w_t), r_t)$
				\4 Demandas de Frisch de consumo y oferta de trabajo
				\4[] Manteniendo constante la utilidad marginal de la renta
				\4[] $\to$ Parámetro $\lambda_t$
				\4[] $C_t = C(w_t,\lambda_t(w_t),t)$
				\4[] $L_t = L(w_t,\lambda_t(w_t),t)$
				\4[] $\then$ Oferta de trabajo depende de:
				\4[] -- $w_t$: salario en el periodo t
				\4[] -- $\lambda_t (w_t)$: Utilidad marginal de la riqueza
		\2 Implicaciones
			\3 Efectos de variación del salario
				\4 $\Delta$ del salario tiene dos efectos
				\4[] $\to$ Sustitución
				\4[] $\to$ Riqueza
				\4 Efecto riqueza:
				\4[] Variación de la riqueza total
				\4[] Puede definirse en relación a $\lambda$
				\4[] Consumo y ocio bienes normales
				\4[] $\to$ Aumentan si aumenta riqueza
				\4 Efecto sustitución intratemporal
				\4[] Entre ocio y consumo
				\4[] Más salario
				\4[] $\to$ Más caro es el tiempo de ocio
				\4[] $\then$ Disminuye demanda de ocio
				\4[] $\then$ Aumenta oferta de trabajo
				\4 Efecto sustitución intertemporal
				\4[] Depende de ESI
				\4[] ¿Puedo trabajar más hoy y menos mañana?
				\4[] ¿Disposición a concentrar trabajo en un periodo?
			\3 Elasticidad de Frisch\footnote{Ver Reichling y Whalen (2012).}
				\4 Elasticidad-salario del trabajo
				\4[] Manteniendo constante $\lambda$
				\4[] $\to$ Con utilidad marginal de la renta constante
				\4 $\Delta$ de oferta de trabajo ante $\Delta w$
				\4[] $\to$ Suponiendo riqueza total futura constante
				\4[$\then$] Efecto sustitución sobre la oferta de trabajo
				\4[$\then$] $\Delta$ de oferta de trabajo neta de efecto renta
				\4 Aplicación
				\4[] Valorar efectos de shocks temporales de salario
				\4[] $\to$ Sobre oferta de trabajo
				\4[] $\then$ Efecto de variaciones de impuestos
				\4[] Diferentes estimaciones en margen extensivo/intensivo
				\4[] $\to$ Intensivo: cuántas horas trabajar
				\4[] $\to$ Extensivo: trabajar o no trabajar
				\4 Empiricamente (margen intensivo)
				\4[] Hombres menos elast. de Frisch que mujeres
				\4[] $\to$ Menos dispuestos a $\uparrow \downarrow$ trabajo ante $\Delta$ renta
				\4[] Hombres jóvenes menos e. Frisch que hombres cercanos a jubilación
				\4[] Mujeres sin niños menos e. Frisch que con niños
				\4 Empíricamente (margen extensivo)
				\4[] Años hasta jubilación son factor importante
				\4[] Cuantos menos años para jubilación
				\4[] $\to$ Más elasticidad de Frisch
				\4[] $\then$ Más gente deja de trabajar con $\downarrow$ renta
		\2 Valoración
			\3 Modelo del ciclo real
				\4 Modelo neoclásico intertemporal de oferta laboral
				\4[] $\to$ Componente básico del RBC
				\4 Utilizado para explicar
				\4[] Variaciones del empleo en recesión/crecimiento
				\4 Sustitución intertemporal del empleo
				\4[] $\to$ Principal fuente de $\Delta$ de empleo
				\4[] $\to$ Agentes muy dispuestos a concentrar trabajo
				\4[] $\then$ Explica fluctuaciones del empleo
				\4[] $\then$ Explica recesiones
			\3 Elasticidades micro y macro de la oferta de trabajo
				\4 Estimaciones microeconómicas:
				\4[] Elasticidades relativamente bajas
				\4[] $\Delta$ de salario $\then$ $\Delta$ pequeña de trabajo
				\4 Estimaciones macroeconómicas:
				\4[] Elasticidades relativamente altas
				\4 Diferencia macro-micro
				\4[] Uno de los grandes puzzles de macroeconomía
				\4 Explicaciones propuestas al puzle
				\4[] Elast. micro y macro miden cosas distintas
			\3 Problemas empíricos
				\4 Variaciones del salario real son pequeñas
				\4[] Necesaria elasticidad intertemporal muy elevada
				\4[] $\to$ Para explicar $\Delta$ observada del empleo
				\4[] $\to$ ¿Disposición a sust. intertemp. es tan alta?
				\4 Estudios empíricos desmienten
				\4[] EIS es muy baja, especialmente hombres
				\4[] Mujeres, un poco más alta pero < 1
	\1 \marcar{Teoría del capital humano}
		\2 Idea clave
			\3 Trabajo no es homogéneo
				\4 Modelo neoclásico:
				\4[] Trabajo como bien homogéneo
				\4 Realmente:
				\4[] Diferentes niveles de productividad
				\4[] Diferentes salarios
				\4[] Diferentes características de los trabajadores
				\4[] $\to$ Educación
				\4[] $\to$ Habilidades
				\4[] $\to$ Elasticidades
				\4[] Empleadores a menudo no pueden verificar
			\3 Relación consistente entre educación y salario
				\4 Correlación positiva en muchos mercados
				\4[] $\uparrow$ Educación $\longleftrightarrow$ $\uparrow$ Salario
				\4 ¿Por qué?
				\4[] ¿Más educación causa más salario?
				\4[] ¿Más salario causa más educación?
				\4[] ¿Más educación aumenta productividad que aumenta salario?
				\4[] ¿Más productividad aumenta salario y educación?
				\4[] $\then$ Diferentes teorías para explicar
			\3 Diferentes teorías de la demanda de educación
				\4 Teoría de la señalización
				\4[] Spence (1973) y otros
				\4[] Educarse para señalizar habilidad innata
				\4[] Sólo los que tienen habilidad innata
				\4[] $\then$ Pueden educarse a bajo coste
				\4 Teoría contable de la educación
				\4[] Educación es simplemente bien de consumo
				\4 Teoría del capital humano
				\4[] Educación permite aumentar productividad
				\4[] Productividad se remunera en mercado
				\4[] Educación es resultado de optimización
				\4[] $\to$ Coste de oportunidad de educación
				\4[] $\to$ Mayores ingresos futuros por educación
				\4[] $\then$ Educación de óptimo
			\3 Teoría del capital humano
				\4 Supuesto central
				\4[] Educación aumenta productividad
				\4[] $\to$ Productividad aumenta salario
				\4[] $\then$ Educación como inversión para $\uparrow$ salario
				\4 Smith, Knight, Schultz, Denison
				\4[] $\to$ Introducen: educación como inversión
				\4 Gary Becker (1964)
				\4[] Concepto de ``capital humano''
				\4[] Marco de análisis general
				\4[] Educación como decisión de agente racional
				\4[] $\to$ Qué nivel de educación alcanzar
				\4[] $\to$ Qué comparación realizar con otra inversiones
				\4[] $\to$ Quién paga coste de educación
		\2 Formulación
			\3 Supuestos generales
				\4 Productividad depende de educación
				\4[] Positivamente
				\4[] En diferente grado según individuo
				\4[] $\to$ Diferente retorno a la educación
				\4 Mercados competitivos de trabajo
				\4[] Salario es igual a productividad
				\4[] Trabajadores siempre venden trabajo
				\4[] Empresas no obtienen beneficios
				\4 Ingreso total depende de:
				\4[] Años trabajados $\longleftrightarrow$ Tiempo dedicado a estudio
				\4[] Coste de educación
				\4[] Ingreso por trabajo
				\4[] Remuneración del trabajo en relación a educación
				\4 Costes de educación
				\4[] Educación tiene dos costes
				\4[] $\to$ De oportunidad por no trabajar
				\4[] $\to$ De educación en sí mismo
			\3 Decisión sobre años de educación
				\4 Trabajador maximiza ingresos a lo largo de su vida
				\4 Trade-off entre:
				\4[] $\to$ Trabajar y ganar dinero hoy
				\4[] $\to$ No trabajar hoy, educarse y ganar más mañana
				\4 Elección óptima
				\4[] Educarse hasta que ingreso marginal de educación
				\4[] $\to$ Iguala coste de oportunidad por no trabajar
			\3 Decisión sobre pago de educación
				\4 Educación tiene un coste explícito
				\4 ¿Quién paga el coste?
				\4[] ¿Empresa o trabajador?
				\4[] $\to$ Depende del tipo de formación
				\4 Formación general
				\4[] Aumenta la productividad para todo trabajo
				\4[] $\to$ Aumento de sueldo en todo trabajo
				\4[] $\then$ Trabajadores financian educación
				\4[] ¿Cuánta educación obtener?
				\4[] $\to$ Iguale ingreso descontado y CMg
				\4 Formación específica
				\4[] Aumenta la productividad en trabajo concreto
				\4[] $\to$ Empresas no compiten por mayor productividad
				\4[] $\then$ Empresas financian educación
				\4[] Durante periodo de formación
				\4[] $\to$ Empresas pagan más que productividad
				\4[] $\then$ Empresa incurre en coste de oportunidad
				\4[] En el periodo tras formación
				\4[] $\to$ Productividad mayor que salario
				\4[] $\then$ Empresa recupera inversión en capital humano
			\3 Emigración y capital humano
				\4 Brain drain
				\4[] Emigración de trabajadores de elevada cualificación
				\4[] $\to$ Que reduce stock de capital humano nacional
				\4[] ¿Tiene efectos negativos sobre mercado de origen?
				\4[] $\to$ No, si mercado laboral perfec. competitivo
				\4[] $\then$ Trabajadores remunerados a producto marginal
				\4[] $\then$ Sin externalidades de cualificación sobre otros
				\4[] $\then$ Sin complementariedades de trabajo cualificado
				\4[] $\to$ No, si transferencias fiscales origen/destino
				\4[] $\then$ Origen se beneficia de complemetariedades en destino
				\4[] $\to$ Sí, si extern+complemen. no compensadas
				\4[] $\then$ Sin transferencias fiscales a nivel internacional
				\4[] Hay realmente brain drain?
				\4[] $\to$ Emigración cualificada aumenta en términos absolutos
				\4[] $\to$ Pero cualificación en mercados de origen aumenta también
				\4[] $\then$ Efecto escaso sobre cualificación media
				\4[] $\then$ Brain drain puede incluso caído en últimos años
				\4 Brain gain
				\4[] Fenómeno postulado
				\4[] Aumento de emigración de cualificados
				\4[] $\to$ Acaba aumentando cualificación en mercado de origen
				\4[] ¿Por qué?
				\4[] $\to$ Posibilidad de emigrar incentiva inversión en KHumano
				\4[] $\then$ Finalmente, no todos emigran
				\4[] $\to$ Envío de remesas a país de origen
				\4[] $\to$ Inversión a país de origen
				\4 Brain waste
				\4[] Profesionales altamente cualificados que emigran
				\4[] $\to$ Trabajan en destino en profesiones menos cualificadas
				\4 Complementariedades de trabajo cualificado y no cualificado
				\4[] En la práctica, es difícil discriminar inmigración
				\4[] $\to$ Cualificada atrae no cualificada
		\2 Implicaciones
			\3 Educación como inversión
				\4 Sacrificio presente para beneficio futuro
				\4 Aplicables criterios de inversión habituales
				\4 Educación como activo financiero
				\4[] Poco líquido
				\4[] No colateralizable
			\3 Optimalidad de la educación
				\4 Maximización de ingreso respecto a educación
				\4[] Educación aumenta productividad
				\4[] $\to$ Hasta que CMg iguala IMg individual
				\4 Asumiendo ausencia de externalidades
				\4[] $\to$ Decisión induce óptimo social
				\4[$\then$] T. Capital Humano
				\4[] Educación deseable en sí misma
				\4[] Contrasta con Spence (1973)
				\4[] $\to$ Educación como mal menor para señalizar
			\3 Salarios de trabajo no cualificado
				\4 Modelo explica abandono escolar en booms
				\4[] Trabajadores con pocas habilidades
				\4[] $\to$ Obtienen poco retorno por educación
				\4[] $\to$ Pueden obtener salarios elevados sin formarse
				\4[] $\then$ Aumenta coste de oportunidad de formarse
				\4 Fases de crecimiento elevado
				\4[] (Especialmente España)
				\4[] $\to$ $\uparrow$ Salario trabajo no cualificado
				\4[] $\to$ $\uparrow$ Coste de oportunidad de formarse
				\4[] $\then$ $\uparrow$ Abandono escolar de bajas habilidades
			\3 Habilidades
				\4 Retorno a la educación puede ser heterogéneo
				\4[] Diferentes individuos diferentes retornos
				\4[] $\to$ Diferentes ``habilidades''
				\4 Decisión óptima:
				\4[] Educarse hasta que beneficio marginal se anula
				\4[] Más habilidades
				\4[] $\to$ Más retorno a educación
				\4[] $\to$ Más ingreso por educarse
				\4[] $\Rightarrow$ Más educación
				\4[$\Rightarrow$] Habilidad determina educación
				\4[] Habilidades caracterizan heterogeneidad
				\4 Punto de conexión con teoría de la señalización
			\3 Demografía
				\4 Educación aumenta rentabilidad
				\4[] $\to$ Aumentan años de educación
				\4 Más años de educación reducen ventana fértil
				\4[] $\then$ Reducción de la natalidad
			\3 Ecuación de Mincer\footnote{Ver Patrinos (2016) para buena explicación general.}
				\4 Modelo econométrico basado en TCHumano
				\4[] Más habitual en trabajos empíricos
				\4[] ``Mincer equation''
				\4[] Caracterizar impacto sobre ingreso por $\Delta$ de:
				\4[] $\to$ Educación
				\4[] $\to$ Experiencia
				\4[] $\ln w = \ln w_0 + \rho s + \beta_1 x + \beta_2 x^2$
				\4[] $\to$ $s$: años de educación
				\4[] $\to$ $x$: años de experiencia
				\4[] $\to$ $\rho$: retorno de la educación
				\4[] $\to$ $\beta_1, \beta_2$: retorno de la experiencia
		\2 Valoración
			\3 Modelos alternativos
				\4 Señalización
				\4[] Spence (1973) y otros
				\4[] Educación señaliza habilidades innatas
				\4[] $\to$ No aumenta productividad en sí misma
				\4[] $\then$ Más educación agregada no aumenta bienestar
				\4 Educación como bien de consumo
				\4[] La educación es bien normal
				\4[] Mayor renta a lo largo de vida
				\4[] $\to$ Mayor gasto en educación
				\4[] $\then$ Educación es irrelevante como bien de capital
			\3 Política económica
				\4 Diseño de programas de becas y préstamos
				\4[] Teniendo en cuenta retorno de educación
				\4[] $\to$ Posible desagregar por tipo de educación
				\4[] $\then$ Valorar aumento de años de educación
				\4 Programas de desarrollo
				\4[] ¿Las becas sirven para reducir pobreza?
				\4[] ¿Realmente implica aumento de la TFP/Trabajo?
	\1 \marcar{Conclusión}
		\2 Recapitulación
			\3 Teoría neoclásica del mercado de trabajo
			\3 Análisis intertemporal de la oferta de trabajo
			\3 Teoría del capital humano
		\2 Idea final
			\3 Aspectos sin analizar
				\4 Mercado de trabajo es muy diverso y complejo
				\4 Fenómenos que modelos anteriores no explican:
				\4[] Comportamiento no walrasiano
				\4[] Asimetrías de información
				\4[] Determinación de salario
				\4[] $\to$ Negociación
				\4[] $\to$ Comportamiento estratégico de empresas
				\4[] $\to$ Salarios de eficiencia
				\4[] $\to$ Contratos implícitos
			\3 Relevancia de los modelos anteriores
				\4 A pesar de todo, muy relevantes
				\4 Base de modelos RBC y NEK
				\4 Modelos muy parsimoniosos
				\4 Explican gran cantidad de fenómenos
				\4 Caracterizan parámetros a estimar
\end{esquemal}



























\graficas

\begin{axis}{4}{Demanda de ocio y consumo como punto de tangencia entre recta presupuestaria y curvas de indiferencia}{$l$}{$c$}{ofertadetrabajo}
	% recta presupuestaria
	\draw[-] (0,2.5) -- (3,1) -- (3,0);
	
	% consumo máximo potencial
	\node[left] at (0,2.5){$wl_0 + R_0$};
	
	% consumo mínimo de renta autónoma
	\draw[dashed] (3,1) -- (0,1);
	\node[left] at (0,1){$R_0$};
	
	% tiempo máximo disponible
	\node[below] at (3,0){$l_0$};
	
	% curva de indiferencia
	\draw[-] (0.37,3) to [out=280, in=170](2.57,1.5);
	
	% equilibrio
	\node[circle, fill=black, inner sep=0pt, minimum size=3pt] (a) at (1.26,1.88) {};
	
\end{axis}

\begin{axis}{4}{Descomposición del efecto sustitución y el efecto renta sobre las demandas de ocio y trabajo ante un aumento del salario.}{$l$}{}{eserconsumoocio}
	% recta presupuestaria
	\draw[-] (0,2.5) -- (3,1) -- (3,0);
	
	% consumo máximo potencial
	\node[left] at (0,2.5){\tiny $wl_0 + R_0$};
	
	% consumo mínimo de renta autónoma
	\draw[dashed] (3,1) -- (0,1);
	\node[left] at (0,1){\tiny $R_0$};
	
	% tiempo máximo disponible
	\node[below] at (3,0){\tiny $l_0$};
	
	% curva de indiferencia
	\draw[-] (0.37,3) to [out=280, in=170](2.57,1.5);

	% equilibrio inicial
	\node[circle, fill=black, inner sep=0pt, minimum size=3pt] (a) at (1.26,1.88) {};
	
	% RP tras aumento de salario
	\draw[-] (0,3.8) -- (3,1);
	
	% consumo máximo potencial tras aumento de salario
	\node[left] at (0,3.8){\tiny $w'l_0 + R_0$};
	
	% CI de equilibrio con ES y ER global
	\draw[-] (0.82,3.45) to [out=280, in=170](3.02,1.95);
	
	% RP hipotética de ES
	\draw[-, color=gray] (0,2.93) -- (3,0.13);
	
	% equilibrio si sólo ES
	\node[circle, fill=gray, inner sep=0pt, minimum size=3pt] (a) at (0.8,2.2) {};

	% equilibrio tras aumento de salario (ES+ER)
	\node[circle, fill=black, inner sep=0pt, minimum size=3pt] (a) at (1.25,2.63) {};
\end{axis}

\begin{axis}{4}{Representación gráfica del efecto global de variaciones en el salario sobre la oferta de trabajo.}{$w$}{$h$}{esertrabajosalario}
	% oferta de trabajo dado salario
	\draw[-] (0,0) to [out=60, in=180](2.5,2) to [out=0, in=180](4,0.2);
	
	% Intervalo con ER < ES
	\draw[decorate,decoration={brace, mirror,amplitude=3pt},xshift=0pt,yshift=-0.3cm] (0,0) -- (2.5,0) node[black,midway,xshift=2pt, yshift=-0.33cm] {\footnotesize $\text{ES}>\text{ER}$};
	
	% Línea que separa intervalos
	\draw[dashed] (2.5,2) -- (2.5,0);
	
	% Intervalo con ER > ES
	\draw[decorate,decoration={brace, mirror,amplitude=3pt},xshift=0pt,yshift=-0.3cm] (2.5,0) -- (4,0) node[black,midway,xshift=2pt, yshift=-0.33cm] {\footnotesize $\text{ES}<\text{ER}$};
\end{axis}

El gráfico muestra como el efecto sustitución predomina hasta cierto punto a partir del cual el efecto renta es más fuerte. En el primer intervalo, la oferta de trabajo crece con el salario. Posteriormente, aumentos del salario reducen la oferta de trabajo. Se asume que la demanda de ocio es normal.

\begin{axis}{4}{Salario de reserva como salario por debajo del cual la oferta de trabajo de equilibrio es una solución de esquina.}{$l$}{$c$}{salariodereserva}
	% recta presupuestaria
	\draw[-] (0,2.5) -- (3,1) -- (3,0);
	
	% consumo máximo potencial
	\node[left] at (0,2.5){\tiny $\bar{w}l_0 + R_0$};
	
	% consumo mínimo de renta autónoma
	\draw[dashed] (3,1) -- (0,1);
	\node[left] at (0,1){\tiny $R_0$};
	
	% tiempo máximo disponible
	\node[below] at (3,0){\tiny $l_0$};
	
	% curva de indiferencia
	\draw[-] (2.33,2) to [out=280, in=170](3.83,0.7);
	
	% equilibrio
	\node[circle, fill=black, inner sep=0pt, minimum size=3pt] (a) at (3,1) {};
\end{axis}

El gráfico caracteriza el salario de reserva $\bar{w}$ como aquel para el que la demanda de ocio y consumo es tal que el agente no oferta trabajo. Para cualquier salario superior a $\bar{w}$, el punto de tangencia con la curva de indiferencia se situaría a la izquierda del equilibrio dibujado, de tal manera que la oferta de trabajo sería positiva. Para cualquier salario inferior, el equilibrio sería el mismo pero se trataría de una solución de esquina en la que la el valor absoluto de la pendiente de la curva de indiferencia no sería igual al salario. Así, cuando el salario fuese inferior a $\bar{w}$, la remuneración al trabajo y consiguiente aumento del consumo y la utilidad no sería suficiente para compensar la pérdida de utilidad que implica disminuir el tiempo de ocio.

\conceptos 

\concepto{Efecto renta directo e indirecto}

El análisis del efecto de variaciones en el precio del ocio --el salario- tiene algunas particularidades respecto al análisis de demanda de bienes de consumo en los que la renta no depende directamente del precio del bien. Así, en el análisis ocio-consumo, variaciones del salario/precio del ocio inducen dos efectos independientes: uno directo y otro indirecto. Para caracterizar los motivos de cada efecto y su valor, definamos formalmente la demanda de ocio en el óptimo:

\begin{equation*}
	l^* = l(w, w l_0 + R_0) = l(w, R(w))
\end{equation*}

El efecto de una variación del salario $d w$ corresponde a:

\begin{equation*}
	\frac{dl}{dw} = \underbrace{\pdv{l}{w}}_{\text{ES}+\text{ER}_i} + \underbrace{\pdv{l}{R(w)} \cdot \frac{d R}{w}}_{\text{ER}_d}
\end{equation*}

Comparemos esta situación con un análisis de demanda de bien convencional en el que la demanda de óptimo corresponde a:

\begin{equation*}
	x^* = x(p, M)
\end{equation*}

En esta situación en la que la renta $M$ no depende del precio $p$, tenemos que una variación del precio tiene un efecto sobre la demanda de $x$ de:

\begin{equation*}
	\frac{d x}{d p} = \pdv{x}{p}
\end{equation*}

Como vemos, la variación de la demanda de bien carece del segundo componente que sí aparece en el lado derecho de $\frac{dl}{dw}$. Es precisamente ese componente, $\pdv{l}{R(w)} \cdot \frac{d R}{w}$, el que representa el efecto directo de la variación del salario. El efecto indirecto puede caracterizarse a partir de la ecuación de Slutsky aplicada a $\pdv{l}{w}$:

\begin{equation*}
	\pdv{l}{w} = \underbrace{\pdv{l^h}{w}}_{\text{ES}} - l \cdot \underbrace{\pdv{l}{R}}_{\text{ER}_i}
\end{equation*}

El efecto renta indirecto es el efecto renta habitual del análisis de demanda: dada una renta monetaria, un aumento del precio reduce la cantidad de bien que se puede adquirir. El efecto renta directo es sin embargo un concepto que sólo aparece en el análisis de la demanda de ocio: la renta monetaria varía con el salario de forma directa, sin que efecto sobre la demanda de ocio tenga lugar indirectamente a través de la cantidad de ocio que se puede comprar dada una renta monetaria fija. Así, el efecto renta directo es resultado de variaciones \textit{en la renta monetaria}, y es por ello que se denomina efecto renta directo. 

\preguntas




\seccion{Test 2017}
\textbf{10.} ¿Cuál de las siguientes afirmaciones es correcta?

\begin{itemize}
	\item[a] Cuando aumenta la productividad del trabajo, siempre disminuye la demanda de trabajo porque no hace falta contratar a tantos trabajadores para producir la misma cantidad de producto.
	\item[b] Cuando aumenta la productividad del capital, el empresario siempre demanda menos trabajo porque le sale más rentable utilizar más máquinas y menos trabajo.
	\item[c] El efecto de cambios en la productividad de trabajo sobre la demanda de trabajo dependerá de la elasticidad-precio de la demanda del producto de la empresa.
	\item[d] La demanda de trabajo depende positivamente de la productividad del trabajo si el efecto renta es mayor que el efecto sustitución.
\end{itemize}

\seccion{Test 2015}

\textbf{3.} En el modelo de elección consumo-ocio, considere un único consumidor, cuyas preferencias son monótonas y estrictamente convexas, que no dispone de rentas no salariales. Tanto el bien consumido como el ocio son bienes normales. Se sabe que cuando el Estado le impone un impuesto proporcional sobre la renta, el consumidor decide trabajar un número de horas positivo. El Estado está pensando en la posibilidad de sustituir el impuesto sobre la renta por uno de cuantía fija que recaude lo mismo. Con el nuevo impuesto:

\begin{enumerate}
	\item[a] El consumidor empeorará.
	\item[b] El ocio aumentará.
	\item[c] La oferta de trabajo será mayor.
	\item[d] No se sabe si aumentará o disminuirá el trabajo pues depende del efecto renta y el efecto sustitución.
\end{enumerate}

\seccion{Test 2014}

\textbf{12.} Suponga que el mercado de trabajo es perfectamente competitivo pero no lo es el mercado de producto. Cuando el mercado de trabajo está en equilibrio el salario será:
\begin{enumerate}
	\item[a] Menor que el producto del precio por la productividad marginal del trabajo.
	\item[b] Igual al producto del precio por la productividad marginal del trabajo.
	\item[c] Mayor que el producto del precio por la productividad marginal del traabajo.
	\item[d] Ninguna de las respuestas es correctas.
\end{enumerate}

\seccion{Test 2013}

\textbf{13.} Considérese una economía de Robinson Crusoe en su versión estática. Las preferencias y tecnología del agente vienen dadas por las funciones $u(c,n)$ y $f(n)$, ambas de buen comportamiento, siendo $c$ el consumo y $n$, las horas dedicadas al trabajo. Se sabe que tanto el bien de consumo como el ocio son bienes normales. Supóngase que se produce un shock tecnológico en virtud del cual la nueva función de producción pasa a ser $f(n) + A$, donde $A > 0$. ¿Qué efectos cabe esperar sobre el consumo y las horas trabajadas?
\begin{enumerate}
	\item[a] Se elevará tanto $c$ como $n$.
	\item[b] Se elevará $c$, pero se reducirá $n$.
	\item[c] Se reducirá tanto $c$ como $n$.
	\item[d] Se elevará $c$, pero permanecerá inalterado $n$.
\end{enumerate}

\seccion{Test 2007}


\textbf{11.} Una empresa produce con la función $x=\ln (L)$, siendo $w(L)$ la oferta del factor trabajo, y $P(x)$ la demanda del mercado del bien, es \textbf{FALSO} que:
\begin{enumerate}
	\item[a] Si la empresa es precio aceptante en el mercado en el que vende el producto $x$ y al comprar el factor, demandará factor de forma que $\frac{P(x)}{L} = W(L)$.
	\item[b] Si la empresa es precio aceptante en el mercado en el que vende el producto $x$ y en el mercado del factor, demandará éste de forma que $P(x) L = W(L)$. 
	\item[c] Si la empresa es precio aceptante en el mercado en el que vende el producto $x$ y ejerce poder de monopsonio en el mercado del factor, demandará éste de forma que $P(x) \text{PMg}_L > W(L)$.
	\item[d] Si la empresa es precio aceptante en el mercado en el que vende el producto y ejerce poder de monopsonio en el mercado del factor, demandará éste de forma que $P(x) \text{PMG}_L = W(L) + L \frac{dW(L)}{dL}$.
\end{enumerate}

\seccion{Test 2006}

\textbf{20.} Suponga una empresa sometida a una restricción en el mercado de bienes cuyo objetivo de maximización de beneficio está descrito a continuación:

\begin{align*}
	\underset{\left\lbrace N \right\rbrace }{\max}\quad &p N^\alpha \bar{K}^{1-\alpha} - w N \\
	\text{Sujeto a:}\quad &N^\alpha \bar{K}^{1-\alpha} \leq Y_M 
\end{align*}

Sea $Y_M = 9$, $\bar{K} = 9$, $\alpha = 1/2$. Señale la respuesta \textbf{verdadera}:
\begin{enumerate}
	\item[a] Para cualquier salario real $w/p< \frac{1}{2}$, la demanda de empleo por parte de la empresa será constante e igual a 5.
	\item[b] Para cualquier salario real $w/p> \frac{3}{4}$, la demanda de empleo por parte de la empresa será constante e igual a 4.
	\item[c] Para cualquier salario real $w/p< \frac{1}{2}$, la demanda de empleo por parte de la empresa será constante e igual a 9.
	\item[d] Para cualquier salario real $w/p> \frac{3}{4}$, la demanda de empleo por parte de la empresa será constante e igual a 9.
\end{enumerate}

\seccion{Test 2004}

\textbf{17.} En general, un impuesto sobre el consumo se considera menos distorsionante que un impuesto sobre la renta del trabajo porque, además del consumo, el impuesto sobre la renta afecta adversamente:
\begin{enumerate}
	\item[a] La oferta de trabajo.
	\item[b] La demanda de trabajo.
	\item[c] La distribución de la renta.
	\item[d] Los ingresos impositivos.
\end{enumerate}

\seccion{Preguntas cante: 8 de marzo de 2017}

\begin{itemize}
    \item La literatura utiliza a menudo C y L como argumentos (en vez del ocio). ¿Podría pintar las curvas de indiferencia en ese caso?
    \item ¿Qué relación tiene el modelo intertemporal de Lucas con la elasticidad de sustitución de L (no de C)?
    \item ¿Qué efectos tiene en W* y en L* la existencia de un monopolio en el mercado de bienes?
    \item En el marco del tema, ¿por qué los inmigrantes mexicanos legales votaron a Trump?
\end{itemize}


\notas



\textbf{2017}: \textbf{18.} C

\textbf{2015}: \textbf{3.} C

\textbf{2014}: \textbf{12.} A

\textbf{2013}: \textbf{13.} B

\textbf{2007}: \textbf{11.} B

\textbf{2006}: \textbf{20.} C 

\textbf{2004}: \textbf{17.} A

\bibliografia

Mirar en Palgrave:
\begin{itemize}
	\item derived demand
	\item labour supply
\end{itemize}

Cahuc, P.; Zylberberg, A. \textit{Labor Economics} (2004) Ch. 1, 2, 4 -- En carpeta Economía Laboral. También diapositivas de \url{www.labor-economics.org}

Chetty, R.; Guren, A.; Manoli, D.; Weber, A. \textit{Are Micro and Macro Labor Supply Elasticities Consistent? A Review of Evidence on the Intensive and Extensive Margins} (2011) -- En carpeta del tema

Heijdra, B. J. \textit{Foundations of Modern Macroeconomics} (2017) 3rd ed. -- En carpeta Macro

Keane, M. P. (2011) \textit{Labour Supply and Taxes: A Survey} Journal of Economic Literature -- En carpeta del tema

Patrinos, H. A. \textit{Estimating the return to schooling using the Mincer Equation} (2016) Iza World of Labor -- En carpeta del tema

Reichling, F.; Whalen, C. (2012) \textit{Review of Estimates of the Frisch Elasticity of Labor Supply} Congressional Budget Office Working Paper Series -- En carpeta del tema

Smith, S. \textit{Labour Economics, 2nd Edition} (2003) 2nd Edition -- En carpeta del tema

\end{document}
