\documentclass{nuevotema}

\tema{3A-24}
\titulo{Economía del bienestar (III). Las funciones de bienestar social. Teoría de la elección colectiva. El teorema de la imposibilidad de Arrow y desarrollos posteriores.}

\begin{document}

\ideaclave

Añadir contraste Welfarismo -- No-Welfarismo en forma de la FBS.

Lionel Robbins y posteriormente Samuelson definieron la ciencia económica como el estudio de las decisiones que tratan de gestionar una serie de recursos finitos con usos alternativos con el fin de satisfacer una serie de necesidades humanas. La economía del bienestar es un campo dentro de la ciencia económica que examina la medida y la maximización del bienestar social. El examen y la comparación de diferentes estados sociales que resultan de diferentes configuraciones institucionales y sociales son así los principales objetos de estudio de la economía del bienestar, pero también de la economía en general. Como afirmó Atkinson, el gran teórico de la desigualdad del siglo XX, la ciencia económica no existe sólo para describir el comportamiento humano y satisfacer la curiosidad y la vanidad de los economistas, sino para emitir recomendaciones y diseñar y valorar políticas que contribuyan a mejorar la vida de los ciudadanos. Por ello, la economía del bienestar forma parte del ``corazón'' de la ciencia económica y como tal, debe ocupar un lugar preeminente en la formación de un economista y más aún de un policy-maker. Así, si la valoración de diferentes estados sociales de acuerdo a su deseabilidad social y la caracterización de los medios para lograrlos es el objetivo de la economía del bienestar, la primera línea de actuación concierne la caracterización del óptimo económico. Los óptimos económicos o de Pareto son aquellas asignaciones de recursos que no pueden ser modificadas para mejorar la situación de al menos un agente sin perjudicar a al menos un otro agente. La importancia del concepto de óptimo de Pareto reside en la posibilidad de separar la calificación de estados sociales en la búsqueda de eficiencia asignativa y distributiva. Así, el óptimo paretiano indica la eficiencia asignativa máxima que alcanzan una serie de estados pero no ordena éstos en relación a su deseabilidad. Para ordenar estados sociales que son óptimos de Pareto en relación a su deseabilidad, es inevitable postular algún criterio que pondere la importancia relativa del bienestar de diferentes individuos. Las funciones de bienestar social son una herramienta básica en economía del bienestar para explicitar los juicios de valor subyacentes a la ponderación de las utilidades de distintos agentes. La teoría de la elección colectiva, por otra parte, trata el problema de extraer criterios de ordenación entre óptimos a partir de las preferencias individuales de conjuntos de agentes. El teorema de la imposibilidad de Arrow es uno de sus principales resultados y ha dado lugar a una amplia literatura que ha contribuido a clarificar el problema de la valoración de estados sociales. Así, el \textbf{objeto} de la exposición consiste en dar respuesta a una serie de preguntas como: ¿qué es una función de bienestar social? ¿para qué sirve? ¿qué es la teoría de la elección colectiva? ¿qué papel juega dentro de la economía del bienestar? ¿en qué consiste el teorema de Arrow y qué implicaciones se derivan? ¿a qué desarrollos posteriores ha dado lugar? La exposición se \textbf{estructura} en dos partes. En la primera examinamos la función de bienestar social en sus aspectos generales y en cuanto a las características particulares de las formas más importantes. En segundo lugar, analizamos la teoría de la elección social planteando sus objetivos, el teorema de la imposibilidad de Arrow y sus críticas y desarrollos posteriores relacionados.

Como hemos introducido anteriormente, el criterio de Pareto permite ordenar estados sociales en función de la posibilidad de mejorar a algún agente sin perjudicar al resto. Aunque esto permite distinguir entre aquellas asignaciones que aprovechan los recursos disponibles y aquellas que no lo hacen, no aporta ningún criterio de decisión entre esos óptimos en los que se aprovechan los recursos. La teoría económica puede derivar muchas y muy relevantes proposiciones positivas absteniéndose de ordenar los diferentes óptimos de Pareto, pero en la práctica del policy-making y en último término, de la organización de una sociedad, es necesario decidir qué óptimos son deseables y se deben tratar de alcanzar. Las \marcar{funciones de bienestar social} tienen por objeto abordar este problema de forma que los juicios de valor --cuya determinación es externa a la ciencia económica- subyacentes a una ordenación sean explicitados formalmente. Fueron propuestas en la primera mitad del siglo XX por Bergson (1938) y Samuelson (1947). Las comparaciones interpersonales de utilidad son inevitables en esta labor, y se derivan de las preguntas que es necesario contestar para ordenar óptimos: ¿cuánto mejor está un agente que pasa de un óptimo dado a otro? ¿importa más la variación del bienestar que sufre un agente dado en comparación a la que afecta a otro? Un objetivo secundario de las funciones de bienestar social es la cuantificación del bienestar que inducen los diferentes óptimos. Si en el objetivo de ordenar diferentes estados sociales se trata de dar respuesta a ¿qué estado es mejor?, en el objetivo de cuantificar el bienestar social se trata de contestar a ¿cuánto mejor es un estado social que otro?

La \textbf{formulación} de las FBS de Bergson-Samuelson son funciones entre un conjunto X de estados sociales y la recta real. Un valor mayor en la recta real implica preferencia de un estado sobre otro, de forma similar a como las funciones de utilidad ordenan vectores de consumo en relación a unas preferencias ordinales subyacentes. En este caso, el dominio de las funciones es el conjunto de estados sociales posibles. Un estado social es un vector de utilidades individuales asociadas al estado social dado. Dado un conjunto de estados sociales, las FBS deben poder ordenar cualesquiera subconjuntos pero su razón de ser principal es ordenar aquellos estados caracterizados por la frontera de posibilidades de utilidad. De esta manera, la aplicación de una FBS a la elección de un óptimo social de entre los estados sociales que inducen óptimos de Pareto puede representarse gráficamente como el punto de tangencia entre la frontera de posibilidades de utilidad y la curva de indiferencia de la FBS más alejada del origen. 

La \textbf{forma de una FBS} depende de los juicios de valor respecto del bienestar de los agentes en cuestión. Para caracterizar estos juicios de valor que subyacen a las FBS, existe una gama de propiedades que pueden cumplir. La propiedad de \underline{individualismo o no paternalismo} caracteriza a las funciones habitualmente denominadas individualistas o welfaristas. En algunas ocasiones, el término FBS de Bergson-Samuelson hace referencia a FBS que cumplen esta propiedad. En FBS con esta propiedad, el bienestar depende exclusivamente de las preferencias individuales de los agentes considerados. Así, si la FBS toma como argumentos las funciones de utilidad individuales de cada agente que representan preferencias ordinales, y todos los agentes tienen funciones de utilidad tales que son indiferentes entre dos estados $x$ e $y$, la FBS deberá también ordenar a los dos estados sociales como igualmente deseables. De forma contraria, una FBS es paternalista cuando otros factores aparte de las funciones individuales entran en juego. Estos factores pueden estar sujetos a percepciones subjetivas de otros agentes tales como el grado de libertad, principios ideológicos, tradiciones religiosas... La propiedad de \underline{monotonía o Pareto-eficiencia} tiene a su vez dos versiones. En su versión débil, el cumplimiento de esta propiedad implica que cuando todos los agentes mejoren su utilidad, la FBS deberá reflejar una preferencia por ese cambio. En su versión fuerte, basta con que un agente mejore su situación y que el resto permanezcan igual para que la FBS refleje una preferencia por el cambio. La propiedad de \underline{simetría} hace referencia a la indiferencia de la FBS entre dos vectores de utilidades idénticos pero en orden inverso. En términos geométricos, se puede caracterizar como la simetría de las curvas de indiferencia de la FBS respecto de la bisectriz del primer cuadrante. La propiedad de \underline{anonimato} generaliza la propiedad de simetría. Una FBS con esta propiedad es indiferente respecto de la identidad de los agentes a la hora de valorar estados sociales, de tal manera que si dos estados sociales inducen mismos niveles de utilidad pero cambiando los agentes cuya utilidad cambia, habrán necesariamente de valorarse de igual forma. Aunque existen numerosas otras propiedades que una FBS puede cumplir, la última y más relevante es la que concierne a las \underline{comparaciones interpersonales de utilidad}. En el contexto de la teoría de la demanda neoclásica, no es necesario que las funciones de utilidad incorporen información cardinal para poder derivar funciones de demanda, por lo que es habitual caracterizar las preferencias de los agentes en términos puramente ordinales de modo tal que las funciones de utilidad están definidas para todas las transformaciones estrictamente crecientes. Sin embargo, a la hora de comparar estados sociales en base a las utilidades de los agentes, la cardinalidad sí es relevante. Por ejemplo, a la hora de comparar el bienestar social inducido por dos asignaciones iguales en cantidad total pero distintas en su distribución entre individuos, es relevante la respuesta a las preguntas ¿cuánto mejor está un agente que otro dada una misma variación de su consumo? ¿cuánto mejora a un agente una asignación respecto de otra? Cuando la FBS implica realizar comparaciones interpersonales de utilidad, no todas las transformaciones de las funciones de utilidad son posibles. Así, solo aquellas que mantengan cierta proporción constante entre diferencias de utilidad son válidas, y sólo aquellas que afecten a todas las funciones de utilidad por igual. En caso contrario, estaríamos alterando la ponderación de unos agentes respecto a otros en la función de bienestar social, afectando a la ordenación de estados sociales resultante y en definitiva cambiando la FBS en cuestión. La propiedad de comparaciones interpersonales de utilidad es en la práctica consustancial a la valoración de estados sociales por medio de las FBS. Sin embargo, hay que tener en cuenta que la FBS en sí misma no tiene necesariamente por qué acarrear información cardinal, de tal manera que puede simplemente limitarse a ordenar estados sociales sin afirmar nada respecto a cuánto es mejor uno que otro. En este caso, la FBS estará definida para cualquier transformación monótona.

Algunos ejemplos de FBS son especialmente habituales en la economía del bienestar. Las \textbf{FBS utilitaristas} toman forma de una suma lineal de las utilidades individuales, aplicando algún tipo de ponderación a cada una de las utilidades. Este tipo de FBS no muestran aversión alguna por la desigualdad, lo cual se manifiesta en términos matemáticos en el hecho de no ser estrictamente cuasicóncava o no ser estrictamente cóncava si la FBS incorpora información cardinal sobre el bienestar social. El nombre de este tipo de funciones trae causa en la doctrina filosófica utilitarista de autores como Beccaria, Hume, Bentham... Aunque puede parecer razonable a priori por su sencillez y compatibilidad con el criterio de Pareto, varias paradojas han sido descritas. La más importante, denominada del ``resultado repugnante'' hace referencia al hecho de que el simple aumento de los individuos de una sociedad puede hacerla preferible a otra con menos agentes, aunque en la primera se encuentren todos en la miseria. Una solución a esta paradoja implica la transformación de la FBS utiltarista simple en una media de las utilidades en vez de una simple suma. El teorema de Harsanyi mostró que dados algunos supuestos relativamente comunes sobre las preferencias de la sociedad, la FBS es necesariamente utilitarista. Así, si la sociedad maximiza el bienestar social esperado, los agentes maximizan la utilidad esperada y la sociedad es indiferente entre dos distribuciones de probabilidad sobre estados sociales cuando todos los agentes lo son, la FBS será necesariamente utilitarista. La \textbf{FBS de Rawls o maximin} tiene su origen en la doctrina del filósofo americano John Rawls. La idea rawlsiana de justicia social se basa en el concepto del velo de la ignorancia. A la hora de decidir qué principios deben caracterizar la valoración de estados sociales, Rawls argumenta que la decisión debe tomarse detrás de un ``velo de ignorancia'' respecto a la posición que cada agente ocupará cuando nazca. En este contexto, el autor afirma que una sociedad es tan buena como bienestar tenga el miembro que viva en peores condiciones. En este contexto, la FBS toma la forma de una función de mínimo sobre el vector de utilidades individuales. Esta forma funcional implica una aversión máxima al riesgo, aunque es de hecho indiferente a la desigualdad entre agentes. Así, dos sociedades pueden ser igualmente valoradas por una FBS de esta forma aunque bajo medidas generales de la desigualdad una de ellas sea mucho más desigual que la otra. Además, esta función es contraria al criterio de la optimalidad fuerte de Pareto dado que es indiferente entre dos sociedades en las que cualquier agente que no sea el peor aumente su utilidad. Las \underline{FBS flexibles o intermedias} son aquellas que toman formas intermedias entre los casos anteriores, de tal manera que son compatibles con el criterio de Pareto pero muestran un grado de aversión a la desigualdad por medio de la cuasiconcavidad estricta y también una tendencia a valorar más la mejora del agente en peor posición. Una función de tipo Cobb-Douglas es un ejemplo de este tipo de funciones flexibles o intermedias. 

La \marcar{teoría de la elección social} concierne la decisión entre diferentes ordenaciones de estados sociales a partir de conjuntos de preferencias individuales denominados \textit{perfiles de preferencias}. Si en la sección anterior examinábamos el problema de valorar diferentes estados sociales en función de la utilidad que obtiene un conjunto de agentes, en esta segunda sección de la exposición examinamos el problema de relacionar perfiles de preferencias individuales con ordenaciones de los estados sociales posibles. Esto es, examinamos el problema de la agregación de preferencias individuales para obtener preferencias sociales que cumplan determinados requisitos. Interpretando el problema en clave de organización de una sociedad, la teoría de la elección social trata de diseñar un mecanismo para encontrar ``constituciones'' a partir de unas preferencias. O de otra forma, como relacionar conjuntos de preferencias individuales con funciones de bienestar social que puedan utilizarse para ordenar estados sociales. La teoría de la elección social aparece en los años 40 y 50. En cierto modo trata de generalizar el concepto de función de bienestar social y es indudable que es este concepto el fundamento de la teoría de la elección social. Arrow (1950) y (1951) inician el camino de este programa de investigación. Posteriormente, otros autores como Sen, Black, Tullock o Buchanan lo continúan.

El \textbf{teorema de la imposibilidad de Arrow} presentado en Arrow (1951) --posteriormente republicado en Arrow (1963)- es el punto de partida de la teoría de elección social. De forma un tanto paradójica, el artículo tiene por objetivo demostrar la incompatibilidad entre cuatro propiedades que puede tener un funcional de decisión social. Un funcional de decisión social es una relación entre el conjunto de perfiles de preferencias individuales y el conjunto de posibles preferencias sociales de carácter racional. Es decir, aquellas ordenaciones de estados sociales que cumplan con los axiomas de completitud y transitividad. Así, dado un vector de preferencias individuales o perfil de preferencias, el funcional debe asignar una ordenación. Arrow demuestra que es imposible que el funcional asigne preferencias que cumplan con las siguientes {cuatro propiedades}. La primera, denominada de dominio completo (\fbox{U}, de \underline{unrestricted domain}) hace referencia, como su propio nombre indica, a la necesidad de que el dominio del funcional sea el conjunto completo de perfiles de preferencia individuales. la segunda propiedad, de criterio débil de Pareto (\fbox{WP}, \underline{weak pareto criterion}) implica que si todos los agentes individuales prefieren $x$ a $y$, la ordenación de estados sociales también deberá ordenar a $x$ como preferido a $y$. La propiedad de independencia de las alternativas irrelevantes (\fbox{IIA} - \underline{independence of irrelevant alternatives}) implica que a la hora de ordenar dos estados $x$ e $y$ sólo deberán tenerse en cuenta las preferencias individuales respecto a estos dos estados. Así, si de un perfil dado se extrae una ordenación social dada respecto de dos estados, un cambio en las ordenaciones individuales respecto de otros dos estados sociales no deberá alterar el orden señalado por las preferencias sociales. Por último, la propiedad de ausencia de dictador (\fbox{D}, \underline{dictatorship}) implica la inexistencia de un agente cuyas preferencias coincidan perfectamente con las de la ordenación agregada. De forma equivalente, existe un dictador si para cualquier $x$ e $y$, el dictador prefiere $x$ a $y$ si y solo si la ordenación social también lo hace. Habiendo definido estas cuatro propiedades, Arrow demuestra que es imposible encontrar relaciones de preferencia (ordenaciones racionales) que cumplan con esos requisitos cuando el conjunto de alternativas tiene tres o más elementos. Esto es, existen ordenaciones de estados sociales que cumplen las cuatro propiedades pero que dan lugar a intransitividades, y relaciones de preferencia que thenn al menos una de las propiedades. Tras la demostración pionera de Arrow (1951) aparecieron numerosas pruebas que utilizaban otras técnicas. En general, la estrategia utilizada por estos métodos de prueba consiste en mostrar como el cumplimiento de U, WP e IIA por una relación de preferencia implica la existencia de un dictador. El teorema de la imposibilidad ha sido interpretado como una crítica al sistema democrático que requiere de la agregación de preferencias individuales, como una crítica al populismo y una defensa de la democracia representativa, o incluso como una defensa de sistemas dictatoriales. Más allá de interpretaciones subjetivas, el enorme impacto del teorema de la imposibilidad en la ciencia económica resulta de haber introducido de pleno la modelización axiomática en economía del bienestar: primero, se plantean axiomas considerados razonables y deseables; segundo, se caracteriza la posibilidad o imposibilidad de respetar axiomas; tercero, se extraen conclusiones y proponen alternativas.

El teorema de la imposibilidad renovó el interés por el análisis matemático de las \underline{reglas de votación}. Las reglas de votación no son sino los funcionales de bienestar social que examina Arrow: relaciones entre el conjunto de perfiles de preferencias individuales y el conjunto de preferencias sociales. El teorema de la imposibilidad generalizó algunas paradojas que ya habían sido presentadas por autores como Condorcet o Borda. La votación mayoritaria por parejas entre tres o más alternativas induce ordenaciones que cumplen las cuatro propiedades U, D, WP e IIA pero que sin embargo no son racionales. La ausencia de racionalidad se concreta en que no siempre existe un ganador de Condorcet, es decir, una alternativa que gane a todas las demás cuando se enfrentan por parejas entre sí.

El \textbf{teorema de Gibbard-Satterthwaite} es un resultado de gran importancia en el ámbito del diseño de mecanismos que fue derivado como resultado de la aparición del teorema de Arrow y su formulación axiomática de la elección social. Uno de los supuestos implícitos al teorema de Arrow es que se conocen perfectamente las preferencias de los agentes, o que pueden conocerse con certeza a la hora de introducir el perfil de preferencias en el funcional. Sin embargo, en la práctica difícilmente esto sucede y los agentes no tienen porqué revelar sus verdaderas preferencias. El teorema de Gibbard-Satterthwaite demuestra que dados los supuestos del teorema de Arrow, no existe ninguna agregación de las preferencias individuales tal que los agentes tengan incentivos a revelar sus verdaderas preferencias al tiempo que no existe un dictador. Esto es, salvo que la relación de preferencia social coincida perfectamente con las preferencias de un agente, los agentes tienen incentivos a revelar preferencias de forma estratégica, que no tiene por qué corresponderse con la realidad. En el ámbito del diseño de mecanismos este es un resultado esencial porque pone de manifiesto una serie de condiciones a priori deseables bajo las cuales los agentes no revelan sus verdaderas preferencias, y la necesidad de abordar el problema de la compatibilidad de los incentivos.

Las \textbf{críticas} al teorema de la imposibilidad han sido numerosas, algunas por parte de autores de gran relevancia en el campo de la economía del bienestar y la elección social. \underline{Little} incidió en la distinción entre funciones de bienestar social y procesos de elección social, siendo éstos últimos el verdadero objeto de Arrow (1951). A partir de esta apreciación --que parece trivial en la actualidad pero no lo era tanto en el momento en que el autor la formuló-, es necesario plantearse que si el objetivo es ordenar estados sociales y el perfil de preferencias viene dado o no es relevante, el teorema de la imposibilidad no es relevante tampoco porque lo necesario es encontrar una función de bienestar social que maximice el bienestar y no una forma de relacionar cualesquiera perfiles de preferencias individuales con preferencias sociales. La crítica de \underline{Tullock} plantea que en la práctica, el dominio de los funcionales está fuertemente restringido y lo está de tal manera que la relación de preferencia social no then las cuatro propiedades, lo que hace irrelevante el teorema de la imposibilidad. \underline{Buchanan} (1954) afirmó que Arrow exige a ordenaciones sociales el cumplimiento de propiedades que no son exigibles a ordenaciones individuales, lo que lo hace irrelevante. Arrow (1963) contestó a esta crítica afirmando que sí son relevantes y puso el ejemplo de la práctica del ``\textit{agenda-setting}'' para modificar el resultado de votaciones por parejas y la importancia de encontrar mecanismos de decisión que sean ``path-independent'' o neutrales al calendario de votación. \underline{Hylland} criticó la relevancia del teorema de Arrow por utilizar capciosamente el término ``dictador'' para describir una propiedad que no tiene por qué ser indeseable bajo cualquier circunstancia. El término ``dictador'' tienen una connotación negativa que no se corresponde con la definición de la propiedad de inexistencia de dictador. Si el ``dictador'' se denomina ``conformista'' y se interpreta como un agente cuyas preferencias se limitan a aceptar la relación de preferencia social, la connotación negativa desaparece y se reduce la necesidad de implementar esta propiedad. Además, aún manteniendo la denominación de ``dictador'', Hylland señala que la existencia de un agente cuyas preferencias coinciden plenamente con las preferencias sociales no tiene por qué ser necesariamente peor que una preferencia social para la que no exista tal dictador pero que sin embargo sea contraria a las preferencias de muchos otros agentes. 

Si el \textbf{teorema de la imposibilidad} introdujo la axiomatización de la elección social, también supuso la semilla de una fértil literatura cuyo objetivo es encontrar ``escapes'' a la imposibilidad. Es decir, relajaciones de los requisitos exigidos en el teorema de la imposibilidad que resulten aceptables. La aparición de intransitividades en la relación de preferencia social cuando se cumplen los cuatro requisitos U, D, IIA y WP puede solventarse \underline{restringiendo el dominio del funcional de bienestar social}, es decir, restringiendo las preferencias individuales admitidas. La restricción necesaria para que siempre aparezca un ganador de Condorcet consiste en requerir que las preferencias puedan ordenarse en un eje y que sean unimodales. Así, las alternativas deben poder ordenarse de mayor a menor mediante algún criterio objetivo no dependiente del sujeto, y las preferencias individuales no pueden dar lugar a más de un máximo local si cada preferencia individual se representa como una función real sobre el eje de estados sociales. Esta restricción es en todo caso una violación de la propiedad D. En este contexto de preferencias ordenables en un eje y preferencias unimodales, se cumple además que el ganador de Condorcet es la opción preferida del agente cuya opción preferida se sitúa en la mediana de la distribución de ``picos'' de las distribuciones unimodales. Este resultado ha tenido reflejo en teorías sobre la localización de partidos en el espectro ideológico que predicen que los partidos tratarán de aparecer en el centro ideológico para atraer al votante mediano. La sustitución de la \underline{transitividad por la aciclidad} de la relación de preferencia social implica que se admitan preferencias con ciclos entre grupos de opciones, pero que no exista ningún ciclo global de manera que exista siempre una opción preferida entre todas las demás. Cuando las preferencias son acíclicas pero intransitivas, es posible encontrar ordenaciones de estados sociales que tengan una opción preferida a toda las demás y que cumpla las cuatro propiedades U, WP, IIA y D. La \underline{regla de Borda} es otra posible alternativa a la aparición de intransitividades que sin embargo implica abandonar la propiedad IIA. Este método de voto consiste en que cada agente asigne números a cada opción según su preferencia individual. A continuación se suman los números asignados y la relación de preferencia social se corresponde con el orden al que da lugar esa suma de votos individuales. Es posible demostrar, sin embargo, que la preferencia social entre dos alternativas varía cuando cambian las preferencias individuales respecto a otras alternativas que deberían ser irrelevantes si se cumpliese la propiedad IIA. La \underline{votación por unanimidad} es otra posible alternativa que viola también la propiedad U de dominio no restringido. Buchanan y Tullock (1962) examinaron el problema de determinar la mayoría mínima necesaria para tomar una decisión como suavización de la regla de unanimidad. Propusieron fijar la mayoría necesaria en una votación como aquel porcentaje que minimiza la suma del coste de no decidir y del coste de tomar una decisión no unánime, que aumenta y decrece respectivamente con el aumento de la mayoría necesaria para tomar una decisión. Así, cuanto más alto sea el porcentaje requerido para adoptar una decisión entre dos alternativas mayor será el coste en términos de no tomar decisiones por la dificultad de alcanzar porcentajes muy altos. Por otro lado, cuanto menor sea, mayor será el coste de tomar una decisión contraria a los deseos de parte de los agentes. Existe además el problema de que no adoptar una decisión es una decisión en sí misma y debe ser tenido en cuenta como tal. Las \underline{comparaciones interpersonales de utilidad} han sido el escape a la imposibilidad que más literatura ha generado y que se presenta como una solución más aceptable, aunque ha dado lugar también a numerosas confusiones y interpretaciones erróneas. Admitir comparaciones interpersonales de utilidad implica abandonar la no comparabilidad ordinal de las preferencias. Es decir, que se admitan comparaciones del tipo: ``\textit{en un estado social x, el agente A está mejor que el B}''. Sin embargo, no implica abandonar necesariamente el supuesto de IIA. La solución propuesta por Sen (1970) incide en esta vía de escape y más que basarse en la relajación de una propiedad, consiste en cambiar el marco en el que se plantea el teorema de Arrow por uno diferente. Otra opción más radical consiste en admitir información cardinal en las preferencias de los agentes, aun sin admitir comparabilidad. Es decir, que se tenga en cuenta la intensidad de las preferencias individuales pero no sea posible decir que un agente está mejor que otro porque tenga una ``valoración'' o una ``utilidad'' más alta asociada a un estado. Esta vía de escape resulta en una violación de la propiedad de IIA análogo al de la regla de Borda. Por último, si las preferencias individuales contienen información cardinal y son comparables entre individuos el teorema de imposibililidad no se cumple y es efectivamente posible ``escapar'' a la imposibilidad aunque esto implica una salida absoluta del marco de modelización de Arrow. Este escape es la base de las funciones de bienestar social examinadas en la primera parte de la exposición.

A lo largo de la exposición hemos examinado las funciones de bienestar social como herramienta para ordenar diferentes estados sociales en relación a unas preferencias individuales sociales y otros factores exógenos dados, y la teoría de la elección social como análisis del problema de relacionar perfiles de preferencias individuales y ordenaciones agregadas de estados sociales. Las explicaciones han transcurrido en un plano fundamentalmente abstracto y formal. La influencia de este tipo de desarrollos en el diseño de políticas públicas ha sido a menudo indirecto por su difícil aplicación teórica, en especial en relación al análisis de democracias representativas. En escalas más pequeñas, en las que deciden unas decenas o unos cientos de personas sobre un número claramente definido y reducido de alternativas, el teorema de la imposibilidad de Arrow sí ha sido un elemento configurador del debate y el diseño de mecanismos de voto. En lo que respecta a la valoración de estados sociales tomando como dadas un conjunto de utilidades postulado, la tensión entre welfarismo y no welfarismo sigue existiendo: ¿son importantes los valores sociales? ¿existen voluntades colectivas diferentes de la suma de las voluntades individuales? ¿existen, o son sólo realmente un espejismo resultado de la dificultad para agregar las preferencias de los agentes? ¿otro tipo de factores a priori diferentes a las preferencias son relevantes? ¿la utilidad de terceros agentes puede ser relevante para la utilidad de un agente dado? El análisis de la distribución de la renta a través de herramientas matemáticas como la dominancia estocástica, así como la construcción de indicadores sintéticos que tratan de resumir el bienestar de una población han sido el resultado práctico de las funciones de bienestar social como concepto abstracto. El gran éxito de la economía del bienestar en lo que respecta a los conceptos tratados en esta exposición no ha sido tanto la implementación directa de los conceptos presentados, sino el hecho de haber contribuido a explicitar los problemas de la agregación y los juicios de valor, aumentando en definitiva la transparencia y la legitimidad de las decisiones de carácter social. 

\seccion{Preguntas clave}
\begin{itemize}
	\item ¿Qué es una función de bienestar social?
	\item ¿Para qué sirve?
	\item ¿Qué es la teoría de la elección colectiva?
	\item ¿Qué papel juega dentro de la economía del bienestar?
	\item ¿En qué consiste el teorema de la imposibilidad de Arrow?
	\item ¿Qué implicaciones se derivan?
\end{itemize}

\esquemacorto

\begin{esquema}[enumerate]
	\1[] \marcar{Introducción}
		\2 Contextualización
			\3 Economía
			\3 Economía del bienestar
			\3 Criterios de decisión
		\2 Objeto
			\3 ¿Qué es una función de bienestar social?
			\3 ¿Cómo se puede decidir entre estados sociales?
			\3 ¿Qué limitaciones lógicas existen?
			\3 ¿Qué implica el teorema de la imposibilidad de Arrow?
			\3 ¿Qué caracteriza a los distintos criterios?
			\3 ¿Qué es un función de bienestar social?
		\2 Estructura
			\3 Funciones de bienestar social
			\3 Teoría de la elección social
	\1 \marcar{Funciones de bienestar social}
		\2 Idea clave
			\3 Contexto
			\3 Objetivos
			\3 Resultados
		\2 Formulación
			\3 Estados sociales
			\3 FBS de Bergson-Samuelson
			\3 Óptimo de Pareto
			\3 Frontera de posibilidades de utilidad
			\3 Óptimo social
		\2 Juicios de valor
			\3 Idea clave
			\3 Racionalidad de la preferencia social
			\3 Paternalismo vs libertarianismo
			\3 Welfarismo vs no-welfarismo
			\3 Monotonía/Pareto-eficiencia
			\3 Simetría
			\3 Cuasiconcavidad estricta de la FBS
			\3 Comparaciones interpersonales de utilidad
		\2 Formas funcionales básicas
			\3 FBS utilitarista
			\3 FBS rawlsiana o maximin
			\3 FBS flexible o intermedia
	\1 \marcar{Teoría de la elección social}
		\2 Idea clave
			\3 Contexto
			\3 Objetivos
			\3 Resultados
		\2 Teorema de la imposibilidad de Arrow (1951, 1963)
			\3 Idea clave
			\3 Funcional de decisión social
			\3 Axiomas requeridos
			\3 Teorema de la imposibilidad
			\3 Implicaciones de la imposibilidad
		\2 Reglas de votación y teorema de la imposibilidad
			\3 Votación por mayoría entre parejas
			\3 Regla de Borda
			\3 Reglas de unanimidad y mayoría óptima
		\2 Teorema de Gibbard-Satterthwaite
			\3 Idea clave
			\3 Implicaciones
		\2 Críticas
			\3 Little
			\3 Tullock
			\3 Buchanan
			\3 Hylland
		\2 Escapes al problema de la imposibilidad
			\3 Restringir dominio: preferencias unimodales
			\3 Aciclicidad en vez de transitividad
			\3 Comparaciones interpersonales de utilidad cardinal
			\3 Información no welfarista
	\1 \marcar{Public choice}
		\2 Idea clave
			\3 Contexto
			\3 Objetivos
			\3 Resultados
		\2 Formulación
			\3 Votantes
			\3 Políticos
			\3 Burocracia
			\3 Sistemas de voto
		\2 Implicaciones
			\3 Modelo de Hotelling-Downs
			\3 Heterogeneidad de las conclusiones normativas
			\3 Proceso de elección colectiva en democracias
			\3 Mayorías necesarias para aprobar legislación
			\3 Log-rolling y elección colectiva
			\3 Mecanismos de control de burocracia
		\2 Valoración
			\3 Influencia creciente desde 80s
			\3 Control del poder púbico
			\3 Política presupuestaria y fiscal
			\3 Implicaciones ideológicas
	\1[] \marcar{Conclusión}
		\2 Recapitulación
			\3 Funciones de bienestar social
			\3 Teoría de la elección social
			\3 Public choice
		\2 Idea final
			\3 Cita de Knight (1921)
			\3 Influencia en el debate político
			\3 Welfarismo frente a no-welfarismo
			\3 Distribuciones de renta y otras variables
			\3 Igualdad de resultados y oportunidades
			\3 Gran éxito de la teoría de la elección social

\end{esquema}

\esquemalargo


















\begin{esquemal}
	\1[] \marcar{Introducción}
		\2 Contextualización
			\3 Economía
				\4 Definición de Robbins
				\4 Microeconomía
				\4[] Entender y predecir
				\4[] Comportamiento de agentes individuales
				\4[] Agrupados como
				\4[] $\to$ Consumidores
				\4[] $\to$ Empresas
			\3 Economía del bienestar
				\4 Rama de la ciencia económica
				\4[] Explorar implicaciones de diferentes criterios éticos
				\4 Valorar diferentes estados sociales
				\4[] $\to$ Definir herramientas de valoración
				\4[] $\to$ Explicitar supuestos de comparación
				\4[] $\to$ ¿Cuándo sociedad x es preferible a y?
				\4 Corazón de la economía (Atkinson)
				\4[] Ciencia económica no existe sólo para
				\4[] $\to$ Describir comportamiento humano
				\4[] $\to$ Satisfacer curiosidad humana
				\4[] Existe sobre todo para
				\4[] $\to$ Emitir recomendaciones
				\4[] $\to$ Diseñar políticas
				\4[] $\then$ Para mejorar sociedad
			\3 Criterios de decisión
				\4 Cómo decidir entre sociedades?
				\4[] Criterio de Pareto es primera aproximación
				\4[] $\to$ Caracterizar conjunto de decisión
				\4[] $\then$ No permite decidir entre óptimos
				\4 Decisión entre sociedades
				\4[] Implica trade-offs
				\4[] Implica juicios de valor
				\4[] $\to$ Necesario explicitar y entender
				\4[] $\to$ Qué hace a una sociedad mejor o peor
		\2 Objeto
			\3 ¿Qué es una función de bienestar social?
			\3 ¿Cómo se puede decidir entre estados sociales?
			\3 ¿Qué limitaciones lógicas existen?
			\3 ¿Qué implica el teorema de la imposibilidad de Arrow?
			\3 ¿Qué caracteriza a los distintos criterios?
			\3 ¿Qué es un función de bienestar social?
		\2 Estructura
			\3 Funciones de bienestar social
			\3 Teoría de la elección social
	\1 \marcar{Funciones de bienestar social}
		\2 Idea clave
			\3 Contexto\footnote{Ver \href{https://www.hetwebsite.net/het/essays/paretian/paretosocial.htm}{HET website sobre comparaciones paretianas y economías del bienestar.}}
				\4 Bentham y utilitaristas
				\4[] Interpretación dominante en siglos XVIII y XIX
				\4[] $\to$ Mayor felicidad para mayor número de personas
				\4 Stuart Mill
				\4[] Ingreso puede redistribuirse sin sacrificar eficiencia
				\4 Economía del bienestar
				\4[] Explicitar efecto de distintos criterios éticos
				\4[] $\to$ Sobre decisión entre diferentes estados sociales
				\4 Teoría de la elección
				\4[] Neoclásicos, Pareto, Edgeworth....
				\4[] Caracterizar efectos de axiomas sobre decisión
				\4[] Entender y predecir decisión
				\4 Contexto social
				\4[] Múltiples agentes con diferentes:
				\4[] $\to$ Preferencias
				\4[] $\to$ Dotaciones
				\4[] $\to$ Bienestar
				\4 Debate Lionel Robbins -  Roy Harrod
				\4[] Lionel Robbins
				\4[] $\to$ Utilidad interpersonal no es comparable
				\4[] $\to$ No tiene sentido que economistas hagan juicios de valor
				\4[] $\to$ Elección entre óptimos es juicio normativo
				\4[] $\then$ Normatividad no concierne a los economistas
				\4[] $\then$ Políticas sólo llevar hacia óptimo de Pareto
				\4[] Roy Harrod
				\4[] $\to$ Imposible sólo acercarse a óptimo de Pareto
				\4[] $\to$ Todas las políticas causan algún perjuicio
				\4[] $\then$ Necesarios juicios de valor
				\4 Bergson (1938)
				\4[] Autor:
				\4[] $\to$ Pionero de análisis de economía soviética
				\4[] $\to$ Sistema de precios difícil caracterización en URSS
				\4[] $\then$ ¿Cómo valorar diferentes estados sociales?
				\4[] Marco matemático de decisión entre estados sociales
				\4[] Primera manifestación
				\4 Nueva Economía del Bienestar
				\4[] Primera mitad del siglo XX
				\4[] Comparaciones interpersonales de utilidad rechazadas
				\4[] $\to$ En términos empíricos
				\4[] Corriente de Harvard
				\4[] $\to$ Bergson, Samuelson
				\4[] $\to$ Utilidad individual incomparable
				\4[] $\to$ Elección de óptimo social es debate normativo
				\4[] $\then$ Pero economía también en ámbito normativo
				\4[] $\then$ Funciones de bienestar social
				\4[] $\then$ Teoría de la elección social
				\4[] Corriente de LSE
				\4[] $\to$ Kaldor, Hicks, Scitovsky
				\4[] $\to$ Utilidad individual incomparable
				\4[] $\to$ Elección social óptima puede positivizarse
				\4[] $\then$ Posible objetivizar óptimo social
				\4[] $\then$ Criterios de compensación, etc...
			\3 Objetivos
				\4 Explicitar juicios de valor entre estados sociales
				\4 Representar efectos de juicios de valor sobre decisión
				\4 Caracterizar diferencias de bienestar entre estados
			\3 Resultados
				\4 Estados sociales
				\4[] Descripción de todos los aspectos relevantes
				\4[] $\to$ Para caracterizar bienestar en una sociedad
				\4[] Representable como vector
				\4[] $\to$ Cada variable representa dimensión relevante
				\4 Espacio de estados sociales
				\4[] Conjunto de todos los posibles estados
				\4 Función de Bergson-Samuelson
				\4[] Aplicación desde el conjunto de estados sociales
				\4[] $\to$ A la recta real
				\4[] $\then$ Representar preferencia social sobre estados
				\4 Juicios de valor
				\4[] Determinan qué estado preferido a cual otro
				\4[] $\then$ Caracterizan forma de función de Bergson-Samuelson
				\4[] $\then$ Explicitan criterios de decisión
				\4 Óptimo de Pareto
				\4[] Permite caracterizar restringir estados a elegir
				\4[] $\to$ Tomando utilidad como criterio de decisión
				\4[] $\then$ Caracterizar frontera de posibilidades de utilidad
				\4[] Combinaciones Pareto-subóptimos no tienen sentido
				\4[] $\then$ Pero no permite decidir entre óptimos
				\4 Frontera de posibilidades de Utilidad
				\4[] Conjunto de óptimos de Pareto
				\4[] Dada una FPU
				\4[] $\to$ ¿Qué punto es el más deseable?
				\4 Comparaciones interpersonales de utilidad
				\4[] Inevitables en esta tarea
				\4[] $\to$ ¿Cuánto mejor está un agente en óptimo A y B?
				\4[] $\to$ ¿Importa más el $\Delta$ de bienestar de B o A?
				\4[] $\then$ FBS personifican juicios de valor que dan respuesta
		\2 Formulación
			\3 Estados sociales
				\4 Estados $x$ en conjunto $X$
				\4 Cada estado $x$
				\4[] Vector de asignaciones de recursos
				\4[] $\to$ U. de agente $i$ depende de $x$
			\3 FBS de Bergson-Samuelson
				\4 Resume deseabilidad social de diferentes estados
				\4 Relación W entre X y $\mathbb{R}$
				\4[] $W: \, \, X \to \mathbb{R}$
				\4 Mayor valor implica preferencia
				\4[] De forma similar a f. de utilidad
				\4[] $W(x) \geq W(y) \iff$ x igual o preferido a y
				\4 FBS individualista
				\4[] Considerada por Samuelson
				\4[] $W(X)$ depende de utilidades individuales
				\4[$\then$] $W = F(U_1, ..., U_n)$
				\4[] No necesariamente requiere utilidad cardinal\footnote{Existen demostraciones de este hecho.}
				\4[] $\to$ De misma forma que teoría demanda no requiere
				\4[] $\then$ Elección entre combinaciones de utilidad
				\4[] $\then$ Sin necesariamente comparar
			\3 Óptimo de Pareto
				\4 Contexto de bienestar individualista
				\4 Estado social en el que
				\4[] Ningún agente puede mejorar
				\4[] $\to$ Sin que empeore otro
				\4 Distinto de óptimo social
				\4[] En contexto individualista
				\4[] $\to$ Óptimo social debe ser óptimo de Pareto
				\4[] Pero no todos los óptimos de Pareto
				\4[] $\to$ Son óptimos sociales
			\3 Frontera de posibilidades de utilidad\footnote{Realmente, en un contexto en el que en la economía hay intercambio de dotaciones y producción de bienes, habría que utilizar la Gran Frontera de Posibilidades de Producción, que comprende todos los óptimos de Pareto de entre todos los óptimos de Pareto que resultan de un conjunto dado de dotaciones.}
				\4 Representación habitual de conjunto de decisión
				\4[] Conjunto de óptimos de Pareto
				\4 Implica supuestos fuertes
				\4[] Utilidad interpersonal comparable
				\4 Vectores de u. individuales asociadas a asignación $\vec{x}$
				\4[] $\left( u_1(\vec{x}), ..., u_I(\vec{x}) \right)$
				\4 Vectores de utilidad tal que
				\4[] No existen vectores que mejoren a todos
				\4[] $\to$ Sin perjudicar a ninguno
				\4[] $\then$ Óptimos de Pareto
				\4[] \grafica{fpu}
			\3 Óptimo social
				\4 Punto de FPU
				\4[] $\to$ Que induce máximo en FBS
				\4[] $\then$ Tangente FPU y curva de indiferencia de FBS
		\2 Juicios de valor
			\3 Idea clave
				\4 Contexto
				\4[] Programa de investigación de criterios de compensación
				\4[] $\to$ Trata de evitar juicios de valor
				\4[] $\to$ Criterios de comparabilidad sin juicio moral
				\4[] FBSocial de Bergson-Samuelson
				\4[] $\to$ Inevitables los juicios de valor
				\4[] $\then$ Explicitemoslos
				\4 Objetivos
				\4[] Caracterizar valoraciones de estados sociales
				\4[] Identificar juicios de valor postulados
				\4 Resultados
				\4[] Características de curvas de indiferencia de FBS
				\4[] $\to$ Interpretables como juicios de valor
				\4[] Determinantes de preferencia de estados sociales
				\4[] $\to$ ¿Preferencias son racionales?
				\4[] $\to$ ¿Utilidad y/o otros criterios?
				\4[] $\to$ ¿Desigualdad?
				\4[] $\to$ ¿Todos los agentes por igual?
				\4 ¿Qué determina preferencia?
				\4[] Necesarios juicios de valor
				\4[] $\to$ Juicio sobre qué es mejor que qué
				\4[] $\then$ ¿Importa sólo utilidad de agentes?
				\4[] $\then$ ¿Importa desigualdad entre agentes?
				\4 Forma general es sólo marco de formulación
				\4 FBS pueden cumplir algunas propiedades
			\3 Racionalidad de la preferencia social
				\4 Definido sobre curvas de indiferencia
				\4 Completitud
				\4[] Sobre cualquier $x \in X$
				\4[] $\to$ Pasa al menos una curva de indiferencia
				\4 Continuidad
				\4[] Curvas de indiferencia no se cortan
				\4 Transitividad
				\4[] Dos curvas de indiferencia no se cruzan
				\4 No saturación de la FBS
				\4[] Curvas de indiferencia son infinitesimales
				\4 Curvas de indiferencia decrecientes
				\4[] Decrecientes
				\4[] $\to$ Si mejora los dos, estado es preferible
				\4[] $\then$ Pareto-optimalidad débil
				\4[] Estrictamente decrecientes
				\4[] $\to$ Si mejoran al menos uno, es preferible
				\4[] $\then$ Pareto-optimalidad fuerte
			\3 Paternalismo vs libertarianismo
				\4 Preferencias individuales no siempre conocidas
				\4 No siempre conocibles
				\4 Paternalismo
				\4[] Estado o autoridad
				\4[] $\to$ Puede conocer mejor las preferencias de agentes
				\4[] $\to$ Que los propios agentes
				\4[] $\then$ Puede aplicar políticas para maximizar
				\4[] $\then$ Coactividad, limitación de libertad justificada
				\4[] $\then$ Intervención en economía justificada
				\4 Libertarianismo
				\4[] Individuos conocen mejor que nadie sus prefs.
				\4[] $\to$ No implica que las conozcan perfectamente
				\4[] $\then$ Estado no conoce prefs. mejor que individuos
				\4[] $\then$ Coacción e intervención debe evitarse
				\4[] Funciones de bienestar social
				\4[] $\to$ Útiles como construcción teórica
				\4[] $\to$ Sin aplicación real
				\4[] $\then$ Intentos de aplicar pueden causar daño
				\4 Paternalismo libertario
				\4[] Evitar coacción
				\4[] Formular preferencias de estados sociales
				\4[] Tratar de alcanzar mediante nudges
				\4[] $\to$ Plantear elecciones a agentes
				\4[] $\to$ Utilizar reglas heurísticas
				\4[] $\then$ ``Empujar'' a decisión considerada deseable
			\3 Welfarismo vs no-welfarismo
				\4 Welfarismo
				\4[] Toda la información relevante
				\4[] $\to$ Para juzgar estados sociales
				\4[] $\then$ Contenida en funciones de utilidad individual
				\4 No welfarismo
				\4[] No toda la información está contenida en utilidad indiv.
				\4[] Ejemplos:
				\4[] $\to$ Cumplimiento de ley sagrada
				\4[] $\to$ Deberes morales
				\4[] $\to$ Justicia
				\4[] $\to$ ``Interés general''
				\4[] $\to$ Oportunidades de elección
			\3 Monotonía/Pareto-eficiencia
				\4 Débil
				\4[] Si todos los agentes mejoran utilidad
				\4[] $\to$ FBS aumenta
				\4 Fuerte
				\4[] Si al menos un agente mejora utilidad
				\4[] Resto permanecen igual
				\4[] $\to$ FBS aumenta
			\3 Simetría
				\4 FBS es indiferente entre
				\4[] Vectores de utilidad idénticos pero distinto orden
				\4 Geométricamente
				\4[] C. de indiferencia simétricas respecto a bisectriz
			\3 Cuasiconcavidad estricta de la FBS
				\4 Implica aversión a la desigualdad
				\4 Combinaciones convexas de vectores de utilidad
				\4[] $\then$ Preferidas a vectores por separado\footnote{Asumiendo que la FBS se limita a ordenar estados y no valora cardinalmente la preferencia por uno u otro.}
				\4 Aversión a la desigualdad no implica cuasiconcavidad/convexidad estrictas
				\4[] Ojo con esto
				\4[] \grafica{adesigualdadsinconvexidad}
			\3 Comparaciones interpersonales de utilidad
				\4 Contexto de teoría de la demanda
				\4[] Interpretación cardinal no es relevante ni necesaria
				\4[] Puede derivarse función de demanda
				\4[] $\to$ Exclusivamente a partir de información ordinal
				\4[] $\then$ F. de u. definidas $\forall$ transformación monótona
				\4 Comparar estados sociales en base a utilidades
				\4[] Puede que información cardinal sí sea necesaria
				\4[] $\then$ ¿Cuánto mejor está A que B?
				\4[] $\then$ ¿Cuánto mejora un agente entre $x_1$ y $x_2$?
				\4 Si hay comparación de estados sociales
				\4[] No toda transformación de f. de u. es aceptable
				\4[] $\to$ Sólo aquellas que mantengan diferencias de utilidad
				\4[] $\to$ Sólo aquellas que afecten a todas las f. de u. igual
				\4[] $\then$ Si no, estaremos cambiando ponderación en FBS
				\4 FBS puede ser ordinal o cardinal
				\4[] Puede limitarse a:
				\4[] $\to$ Decidir qué estado es mejor que otros
				\4[] Puede también señalar
				\4[] $\to$ Cuánto es mejor un estado que otro
		\2 Formas funcionales básicas
			\3 FBS utilitarista
				\4 Idea clave
				\4[] Inspirado en utilitarismo de Bentham, Hume, Beccaria...
				\4[] Bienestar es suma de utilidades
				\4[] Diferentes ponderaciones posibles
				\4 Formulación
				\4[] Bienestar social es suma de utilidades
				\4[] $W(u_1(\vec{x}), ..., u_n(\vec{x})) =  \sum_{i=1}^n u_i(\vec{x})$
				\4[] \grafica{fbsutilitarista}
				\4[] Sin aversión a desigualdad
				\4[] $\then$ No es estrictamente cóncava
				\4 Implicaciones
				\4 Da lugar a paradojas
				\4[] P.ej.:
				\4[] ``resultado repugnante''
				\4[] $\to$ Más individuos aumentan utilidad
				\4[] $\then$ Aunque estén todos en la miseria
				\4[] $\then$ Solucionado valorando utilidad media
				\4[] Teorema de Harsanyi
				\4[] $\to$ Tres supuestos sobre preferencias de agentes
				\4[] $\then$ FBS utilitarista
				\4[] i.) No conocen asignación que tendrán, sólo distribución
				\4[] ii. Tienen funciones de utilidad VNM
				\4[] iii. Utilidades individuales separables de otros agentes
				\4[] $\then$ Agentes querrán FBS utilitarista


			\3 FBS rawlsiana o maximin
				\4 Idea clave
				\4[] Contexto
				\4[] Atribuida a John Rawls
				\4[] Objetivos
				\4[] $\to$ Bienestar de miembro en peor situación
				\4[] Resultados
				\4[] $\to$ Curvas de indiferencia cuadradas
				\4[] $\to$ Optimalidad de Pareto en sentido débil
				\4 Formulación
				\4[] Velo de la ignorancia
				\4[] Si agentes no saben que estatus tendrán
				\4[] $\to$ Intentarán evitar estados desfavorables
				\4[] $\then$ Maximizan el mínimo
				\4[] Bienestar social es utilidad mínima de agentes
				\4[] $W(u_1, ..., u_n) = \min \left\lbrace u_1, ..., u_n \right\rbrace$
				\4[] \grafica{fbsrawlsiana}
				\4 Implicaciones
				\4[] Pareto-optimalidad
				\4[] $\to$ Se eligen Pareto-débil superiores
				\4[] $\then$ Necesario que todos estén mejor
				\4[] $\to$ No se eligen Pareto-fuerte superiores
				\4[] $\then$ Si al menos uno está mejor
				\4 Aversión máxima al riesgo/desigualdad
				\4[]  $\to$ Utilidad de menos favorecido constante
				\4[] $\to$ Sociedad indiferente entre $+/-$ utilidad para un agente
				\4[] $\then$ Contrario a optimalidad fuerte de Pareto
				\4 Valoración
			\3 FBS flexible o intermedia
				\4 Idea clave
				\4[] Muestra aversión a la desigualdad
				\4[] $\then$ Cuasiconcavidad estricta
				\4[] Pero cumpliendo criterio de Pareto fuerte
				\4[] $\then$ Mejora de un agente aumenta bienestar social
				\4[] \grafica{fbsflexible}
	\1 \marcar{Teoría de la elección social}
		\2 Idea clave\footnote{Igersheim (2017).}
			\3 Contexto
				\4 Contexto histórico
				\4[] Introducción de funciones de bienestar social
				\4[] $\to$ Bergson (1938)
				\4[] $\to$ Samuelson (1947)
				\4[] Rechazo general de comparaciones interpers. de u.
				\4 Funciones de bienestar social
				\4[] $\to$ Ordenan estados sociales
			\3 Objetivos
				\4 Relacionar preferencias individuales con colectivas
				\4 Formular relación colectiva-individuales
				\4 Caracterizar límites de reglas de elección
				\4 ¿Cómo agregar preferencias?
				\4 ¿Cómo relacionar perfiles de preferencias
				\4[] $\to$ Con funciones que ordenen estados sociales?
				\4[] $\to$ Con funciones que midan bienestar social
				\4[] $\then$ ¿Cómo relacionar preferencias indiv. y sociales?
				\4[] $\then$ ¿Cómo encontrar funciones de bienestar social?
			\3 Resultados
				\4 Teoría de la elección social
				\4 Teoría de la elección social
				\4[] Encontrar ``constituciones''
				\4[] $\to$ Dadas preferencias
				\4[] $\then$ Relacionar conjuntos de preferencias con FBS
				\4[] $\then$ Hallar \textbf{funcionales de bienestar social}\footnote{Un funcional se diferencia de una función en que asigna funciones a elementos del dominio.}
		\2 Teorema de la imposibilidad de Arrow (1951, 1963)
			\3 Idea clave
				\4 Contexto
				\4[] Escuela de Harvard
				\4[] Funciones de bienestar de Bergson-Samuelson
				\4 Objetivos
				\4[]  Mostrar limitaciones
				\4[] $\to$ de agregación de preferencias ordinales
				\4[] Caracterizar conjuntos imposibles de características
				\4 Resultados
				\4[] ``Tercer Teorema Fundamental del Bienestar''
				\4[] Imposible cumplir ciertos requisitos deseables
				\4[] No existe funcional que relacione:
				\4[] $\to$ Conjunto de perfiles de preferencias individuales
				\4[] $\to$ Conjunto de preferencias sociales
				\4[] $\then$ Que cumpla 4 requisitos
			\3 Funcional de decisión social
				\4 Relaciones de preferencia
				\4[] Ordenaciones completas y transitivas de estados
				\4 Espacio de preferencias
				\4[] Conjunto $X$ de todas las relaciones de preferencia
				\4 Perfil de preferencias
				\4[] Vector de preferencias individuales
				\4[] $\left( \succeq_1, ..., \succeq_I \right)$
				\4[] (``preferencias'' luego racionales)
				\4 Espacio de perfiles de preferencias
				\4[] Conjunto $\mathcal{B}$ de todos los vectores de preferencias
				\4 Funcional de decisión social
				\4[] Relación $f$ entre conjuntos $\mathcal{B}$ y X
				\4[] Asignar a perfiles de preferencias $(x_1, ..., x_I)$
				\4[] $\to$ Una relación de preferencia social R
				\4[] $\to$ Expresa ordenación social de preferencias
				\4[] $\then$ $R = f \left( (\succeq_1, ..., \succeq_I) \right)$
			\3 Axiomas requeridos
				\4 Las $R$ asignadas deben cumplir
				\4[U] -- Dominio completo (Unrestricted domain)
				\4[] El dominio de $f$ incluye todos los perfiles posibles
				\4[] $\to$ Dominio de $f$ es $\mathcal{B}$
				\4[WP] -- Criterio débil de Pareto
				\4[] $\forall x,y \in X: x \succeq_i y \, \forall i \then x R y$
				\4[] Si todos los agentes prefieren $x$ a $y$
				\4[] $\then$ La relación de preferencia social también lo hará
				\4[IIA] -- Independencia de las alternativas irrelevantes
				\4[] La relación de preferencia social R entre $x$ e $y$
				\4[] $\to$ Depende sólo de prefs. indiv. entre $x$ e $y$
				\4[D] -- No dictatorial
				\4[] No existe un agente $i$ para el que se cumple:
				\4[] $\forall x, y \in X, x \succ_i y \iff x R y$
				\4[] No hay ningún agente cuyas preferencias
				\4[] $\to$ Siempre coincidan con la RP social
				\4[RP] La relación binaria asignada debe ser un orden de preferencias.
			\3 Teorema de la imposibilidad
				\4 Si el conjunto X tiene tres o más alternativas
				\4[] Se puede demostrar que no existe $f(\succeq_1, ..., \succ_I)$
				\4[] $\to$ Que asigne RP y cumpla U,WP,IIA,D
				\4 Diferentes pruebas
				\4[] Ej.:
				\4[] $\to$ Mostrar que U, WP e IIA implican $\exists$ dictador
				\4[] $\to$ Prueba diagramática
				\4[] $\to$ ...
			\3 Implicaciones de la imposibilidad
				\4 Críticas a la democracia
				\4[] Preferencias de ciudadanos no se pueden agregar
				\4 Críticas al populismo
				\4[] Necesaria democracia representativa
				\4 Teoría económica
				\4[] Introduce método axiomático
				\4[] $\to$ en economía del bienestar
				\4[] 1. Plantear axiomas considerados razonables
				\4[] 2. Caracterizar posibilidad de respetar axiomas
				\4[] 3. Extraer conclusiones
		\2 Reglas de votación y teorema de la imposibilidad
			\3 Votación por mayoría entre parejas
				\4 Votación por parejas
				\4[] Ganador de Condorcet
				\4[] $\to$ Opción que gana bilateralmente a las demás
				\4[] Cumple U, WP, IIA, D
				\4[] $\to$ Pero implica intransitividad
				\4[] $\to$ Implica posibilidad de agenda-setting
				\4[] $\to$ No hay ganador de Condorcet
				\4[] $\then$ No es relación de preferencia
				\4 Ejemplo:
				\4[] Tres opciones, $x, y, z$
				\4[] Tres agentes A, B, C con preferencias:
				\4[] $\to$ A: $x \succ_A y \succ_A z$
				\4[] $\to$ B: $y \succ_B z \succ_B x$
				\4[] $\to$ C: $z \succ_C x \succ_C y$
				\4[] Votaciones:
				\4[] $\to$ $x$ vs $y$: $x \succ y$
				\4[] $\to$ $x$ vs $z$: $z \succ x$
				\4[] $\to$ $y$ vs $z$: $y \succ z$
				\4[] $\then$ $x \succ y \succ z \succ x$
				\4[] $\then$ Intransitividad
				\4[] $\then$ Orden de votación importa: agenda-setting
				\4[] $\then$ No existe ganador de Condorcet
				\4[] $\then$ \fbox{Paradoja de Condorcet}
			\3 Regla de Borda
				\4 Asignar números según preferencia
				\4[] Sumando números para cada opción
				\4[] $\to$ Extraer orden de preferencia
				\4[] $\then$ then IIA
				\4 Ejemplo:
				\4[] Preferencias 1:
				\4[] $\to$ A: $x \succ_a z \succ_a y$
				\4[] $\to$ B: $y \succ_b x \succ_b z$
				\4[] Asignando 1 al más preferido y 3 al menos:
				\4[] $x=3, y=4, z=5$
				\4[] $\then$ Prefs. sociales: $x \succ_1 y \succ_1 z$
				\4[] Preferencias 2:
				\4[] $\to$ A: $x \succ_a y \succ_a z$
				\4[] $\to$ B: $y \succ_b z \succ_b x$
				\4[] Asignando 1 al más preferido y 3 al menos:
				\4[] $x=4, y=3, z=5$
				\4[] $\then$ Prefs. sociales: $y \succ_2 x \succ_2 z$
				\4[] Preferencias sociales han cambiado
				\4[] $\to$ Pero prefs. entre $x$ e $y$ NO han cambiado
				\4[] $\then$ Violación de IIA
			\3 Reglas de unanimidad y mayoría óptima
				\4 Unanimidad viola U
				\4[] Dominio restringido a acuerdo unánime
				\4 Buchanan y Tullock (1962)
				\4[] Mayoría necesaria debe igualar:
				\4[] $\to$ Coste de tomar decisión no unánime
				\4[] $\to$ Coste de no tomar decisión
		\2 Teorema de Gibbard-Satterthwaite
			\3 Idea clave
				\4 Teorema de Arrow asume preferencias conocidas
				\4[] O que pueden conocerse con certeza
				\4[] $\to$ ¿Es realmente posible?
				\4 Teorema muestra que:
				\4[] No existe agregación de prefs. individuales tal que:
				\4[] $\to$ Agentes incentivos a revelar verdaderas prefs.
				\4[] $\to$ No sea dictatorial
				\4 No existe ningún método de:
				\4[] Elección de resultado a partir de perfil de prefs.\footnote{Esto es una diferencia respecto de Arrow, en el que de un perfil de preferencias se obtiene un orden de preferencia y se toma como objetivo caracterizar qué propiedades pueden cumplir esos órdenes de preferencia obtenidos. En el marco de G-S, el teorema se formula en términos de relacionar perfiles de preferencias y directamente, una opción preferida dentro del conjunto de resultados.}
				\4[] En el que agentes tengan incentivos a
				\4[] $\to$ Revelar su verdadero perfil de preferencias
				\4[] Salvo que la elección sea dictatorial
				\4[] $\to$ El resultado lo elija siempre un agente
			\3 Implicaciones
				\4 Cuando hay tres o más opciones
				\4[] Salvo regla dictatorial
				\4[] $\to$ Siempre existen incentivos a voto estratégico
				\4 Resultado clave para diseño de mecanismos
				\4[] Revelación de preferencias es obstáculo a superar
				\4[] Necesario tener en cuenta incentivos para revelar
		\2 Críticas
			\3 Little
				\4 Es necesario diferenciar entre:
				\4[] $\to$ Funciones de bienestar social
				\4[] $\to$ Procesos de elección social
				\4 Objeto de Arrow son procesos de elección
				\4[] Cómo relacionar perfiles de preferencia
				\4[] $\to$ Con órdenes de preferencia
				\4 Si el objetivo es ordenar estados sociales
				\4[] Y perfil de preferencias dado o no relevante
				\4[] $\then$ El teorema de la imposibilidad es irrelevante
			\3 Tullock
				\4 En la práctica, dominio está restringido
				\4[] Bajo supuestos débiles y condiciones generales
				\4[] $\to$ Mayoría de perfiles $\then$ U, WP, IIA, D racional
				\4[$\then$] Teorema de Imposibilidad es irrelevante
			\3 Buchanan
				\4 Buchanan (1954)
				\4[] Arrow aplica exigencias a ordenaciones sociales
				\4[] Propiedades exigibles a ordenaciones individuales
				\4 Réplica: Arrow (1963)
				\4[] Sí son relevantes
				\4[] P. ej.:
				\4[] $\to$ agenda-setting $\to$ deseable ``path-independence''
				\4[] $\to$ Camino para llegar a ordenación social no debe importar
				\4 Contrarréplica de Buchanan
				\4[] Por supuesto, agenda-setting es problema teórico
				\4[] En la práctica de democracias
				\4[] $\to$ Muy difícil beneficiarse de agenda en sentido de Arrow
				\4[] Teorema de imposibilidad es poco relevante
				\4[] $\to$ Porque muy difícil manipular agenda
			\3 Hylland
				\4 El término ``dictatorial'' introduce un sesgo
				\4[] $\to$ Dictatorial = negativo
				\4 Pero puede denominarse ``conformista''
				\4[] Agente que siempre está de acuerdo con regla
				\4[] $\then$ D ya no parece tan necesaria
				\4 Existencia de ``dictador''
				\4[] No tiene por qué ser negativo
				\4[] Pueden existir reglas no dictatoriales
				\4[] $\to$ Que sea contraria a prefs. de muchos agentes
				\4[] $\then$ Preferible que haya un ``conformista''
		\2 Escapes al problema de la imposibilidad
			\3 Restringir dominio: preferencias unimodales
				\4 Restringiendo dominio
				\4[] $\to$ Incumpliendo D
				\4[] $\to$ Puede superarse imposibilidad
				\4 Alternativas pueden representarse en un eje
				\4[] P.ej: más izquierda, más derecha
				\4 Perfiles de preferencias restringidos:
				\4[] Preferencias multimodales no permitidas
				\4[] $\to$ Varios máximos locales
				\4[] \grafica{preferenciasmultimodales}
				\4[] Preferencias unimodales son las únicas permitidas
				\4[] $\to$ Un sólo máximo local
				\4[] \grafica{preferenciasunimodales}
				\4[] $\then$ Incumple U
				\4[] $\then$ Cumple WP, IIA, D, RP
				\4 Teorema del votante mediano
				\4[] Asumiendo:
				\4[] $\to$ Opciones ordenables en un eje
				\4[] $\to$ preferencias unimodales
				\4[] Opción preferida mediana
				\4[] $\to$ Es ganador de Condorcet
				\4[] Localización de partidos
				\4[] $\to$ Tratan de atraer a votante mediano
				\4[] $\then$ Tenderán al centro
				\4[] $\then$ Modelos de Hotelling-Down de ubicación de partidos
			\3 Aciclicidad en vez de transitividad
				\4 Incumplir RP
				\4[] Porque no orden de prefs. social es intransitivo
				\4 Sustituir por aciclicidad
				\4[] Ordenación social puede incluir ciclos
				\4[] $\to$ Entre grupos de opciones
				\4[] Pero no ciclos globales
				\4[] $\then$ Hay una opción preferida sobre todas las demás
				\4[$\then$] Cumplimiento de U, WP, IIA, D
			\3 Comparaciones interpersonales de utilidad cardinal
				\4 Sen (1970) y otros
				\4[] Abandona no comparabilidad ordinal
				\4[] Mantiene IIA
				\4 Teorema de Arrow no considera info. cardinal
				\4[] Sólo relevante información ordinal
				\4[] T. Arrow formulable en términos de funciones de u.
				\4[] $\to$ Mismo resultado de imposibilidad
				\4[] $\then$ Si no hay comp. interpers. cardinales de u.
				\4[] $\then$ Cardinalidad no comparable implica imposibilidad
				\4 Utilidad cardinalmente comparable
				\4[] $\to$ Teorema de imposibilidad no se cumple
				\4[] $\then$ Escape a impos. si es posible comparar utilidades
				\4[] $\then$ Neocardinalismo
			\3 Información no welfarista
				\4 Incorporar criterios más allá de resultados indiv.
				\4 Ejemplos:
				\4[] ``interés general''
				\4[] Principios religiosos
				\4[] Destino manifiesto
				\4[] ...
				\4 Posibilidad abre debate sobre welfarismo
				\4[] Hasta ahora, adoptado criterio welfaristas/utilitaristas
				\4[] $\to$ Formas de consecuencialismo\footnote{Las políticas o las decisiones debe valorarse en función de sus consecuencias. En un contexto welfarista, las políticas o los estados sociales se valoran en función de sus efectos sobre el bienestar de la población. El utilitarismo es una forma de welfarismo --o bajo ciertos criterios, equivalente- en la que el criterio para medir el bienestar de la sociedad es considerar la suma de las \textit{utilidades} de los agentes.}
				\4[] ¿Bienestar de población debe ser criterio central?
	\1 \marcar{Public choice}
		\2 Idea clave
			\3 Contexto
				\4 Economía política tradicional
				\4[] Valorar políticas en términos de efectos económicos
				\4[] $\to$ ¿Cuáles producen mejores efectos?
				\4[] $\to$ ¿Cómo conocer qué efectos son los mejores?
				\4[] $\then$ Escuela de Londres: podemos objetivar decisión
				\4[] $\then$ Escuela de Harvard: necesarios juicios de valor
				\4 Literatura ciencia política
				\4[] Considerada heterodoxa en economía
				\4[] $\to$ Pero muy relevante en la práctica
				\4[] Arthasastra (India)
				\4[] Poliano (Grecia/Imperio romano)
				\4[] Maquiavelo (Florencia s. XVI)
				\4[] $\to$ Maximizar rendimiento indiv. de pol. públicas
				\4 Herramientas de ciencia económica
				\4[] Consolidadas en siglo XX
				\4[] Lenguaje formal de modelización
				\4[] $\to$ Caracterizar incentivos individuales
				\4[] $\then$ También en contexto de diseño de políticas
				\4 Elección social en la práctica
				\4[] Agentes no conocen sus propias preferencias
				\4[] Agentes no conocen consecuencias de voto
				\4[] Agentes utilizan proceso político
				\4[] $\to$ Para maximizar su propia utilidad
				\4[] $\to$ No para maximizar bienestar social
				\4[] Desinformación habitual
				\4[] Implementación de decisión colectiva
				\4[] $\to$ Sujeta a problemas de información asimétrica
			\3 Objetivos
				\4 Entender decisiones políticas
				\4[] Aplicando herramientas de economía
				\4[] $\to$ Maximización de preferencias individuales
				\4 Analizar problemas de implementación de FBS-BS y Arrow
				\4[] $\to$ En contexto de toma de decisiones económicas
			\3 Resultados
				\4 Black (1956)
				\4[] Inicio del programa de investigación
				\4[] Matemáticas del proceso de elección colectiva
				\4[] $\to$ Redescubrimiento de Condorcet y similares
				\4 Downs (1957)
				\4[] ``An Economic Theory of Democracy''
				\4 Programa de investigación del public choice
				\4[] Buchanan, Tullock...
				\4[] $\to$ Incentivos de votantes
				\4[] $\to$ Incentivos de empleados del gobierno
				\4[] $\then$ Resultados de incentivos de grupos de interés
				\4[] $\then$ Crítica a utilidad de teoría de elección social
				\4 Premio Nobel (1986): Buchanan
				\4[] Teoría de la elección colectiva
				\4[] $\to$ Fundamentada a partir de contratos económicos
				\4 Agentes heterogéneos en proceso elección colectiva
				\4[] Diferentes incentivos
				\4[] Diferentes restricciones
		\2 Formulación\footnote{Ver Grandjean \textit{Gordon Tullock on Majority Voting: the Making of a Conviction} y artículo en Palgrave.}
			\3 Votantes
				\4 Maximizan utilidad personal
				\4[] Renta
				\4[] Libertad
				\4[] Consumo
				\4[] Variedad de bienes
				\4[] Bienestar de prójimo
				\4[] ...
				\4 Deciden sobre:
				\4[] $\to$ Qué votar
				\4[] $\to$ Cuánto tiempo dedicar obtener información
				\4 Dadas:
				\4[] Preferencias personales
				\4[] Contexto institucional
				\4[] Restricciones cognitivas
				\4 Desatención racional
				\4[] Informarse es costoso
				\4[] Generalmente, sólo informados de algún aspecto concreto
				\4[] $\to$ Interés particularmente relevante para ellos
				\4 Modelo de Downs-Hotelling
				\4 Teorema de Black sobre el votante mediano
			\3 Políticos
				\4 Electos por votantes
				\4[] Para implementar determinadas políticas propuestas
				\4[] Para implementar políticas ante imprevistos
				\4 Maximizan utilidad personal
				\4[] Ingresos
				\4[] Estatus social
				\4[] Poder
				\4[] Interés general
				\4[] ...
				\4 Teoría de elección social
				\4[] Deciden de acuerdo con agregación preferencias
				\4 Public choice
				\4[] Parcialmente
				\4[] $\to$ De acuerdo con agregación de preferencias
				\4[] Parcialmente
				\4[] $\to$ Lo que creen que les permite mantener poder
				\4 Mantener poder vs utilizar poder para óptimo social
				\4[] Lo que es necesario para mantener el poder
				\4[] $\to$ Lo que los electores creen que les conviene
				\4[] $\then$ No necesariamente lo que les conviene
				\4[] $\then$ No necesariamente igual a elección social
				\4[] $\then$ Posible desviación de óptimo social
			\3 Burocracia
				\4 Visión tradicional
				\4[] Parte de maquinaria gubernamental
				\4[] $\to$ Para ejecutar elección social óptima
				\4[] $\then$ Hacen lo que les dicen los superiores
				\4[] $\then$ Hacen lo correcto
				\4 Maximización de utilidad de empleados del estado
				\4[] Conservar puesto de trabajo
				\4[] Maximizar remuneración
				\4 Atenuación de control de gobierno--funcionarios
				\4[] Común a todas las burocracias
				\4[] Menor control cuanto más niveles de administración
				\4[$\then$] Desviaciones de órdenes de superiores
				\4[$\then$] Objetivos pueden ser o no congruentes con elección social
				\4 Funcionarios como agentes optimizadores
				\4[] Tratan de alcanzar su visión de interés público
				\4[] Trata de maximizar utilidad como cualquier otro agente
				\4[] $\to$ Ocio--consumo
				\4[] $\to$ Renta permanente
				\4[] ...
				\4 Tamaño de la burocracia y proceso político
				\4[] Burócratas también son votantes
				\4[] Con burocracias muy grandes
				\4[] $\to$ Importante peso en voto total
				\4[] $\then$ Empleados se convierten en empleadores
			\3 Sistemas de voto
				\4 Sujetos a problemas de sección anterior
				\4[] $\to$ Paradoja de Condorcet
				\4[] $\to$ Problemas de regla de borda
				\4[] $\to$ Agenda-setting
				\4[] ...
				\4 Enorme variedad
		\2 Implicaciones
			\3 Modelo de Hotelling-Downs
				\4 Idea clave
				\4[] Hotelling (1929)
				\4[] $\to$ Localización de empresas en espacio lineal
				\4[] $\to$ Preferencias/localización de consumidores+CdTransporte
				\4[] $\then$ Localización de empresas
				\4[] Elección colectiva vía democracia representativa
				\4[] $\to$ Partidos políticos como empresas
				\4[] $\to$ Localización de programas para atraer votantes
				\4[] $\then$ Localización de votantes determinan posición partidos
				\4[] Black (1948)
				\4[] Análisis formal del voto bipartidista
				\4[] $\then$ Teorema del votante mediano
				\4[] Downs (1957)
				\4[] $\to$ Aplicación de Hotelling (1929) para demostrar TVMediano
				\4 Formulación
				\4[] Recta de localización de preferencias políticas
				\4[] Votantes no distribuidos homogéneamente entre $[0,1]$
				\4[] $\to$ A diferencia de Hotelling (1929) simple
				\4[] Votantes con preferencias single peaked
				\4[] $\to$ Máximo global entendido como localización en $[0,1]$
				\4 Implicaciones
				\4[] Teorema del votante mediano en sistemas bipartidistas
				\4[] $\to$ Ya formulado por Black (1948)
				\4[] $\to$ Votante mediano en relación a pico de preferencias
				\4[] $\then$ Partidos políticos buscan cubrir a votante mediano
				\4[] $\then$ Localización tiende al centro del espectro
				\4[] Implícitamente
				\4[] $\to$ Preferencias exógenas
				\4[] $\to$ Partidos políticos endógenos
				\4[] $\then$ ¿Realmente es así?
				\4[] $\then$ ¿Partidos políticos no alteran opinión pública?
				\4[] $\then$ ¿Picos no cambian con discursos políticos?
				\4 Valoración
				\4[] Supuestos muy restrictivos
				\4[] $\then$ Preferencias ordenables en espacio unidimensional?
				\4[] $\then$ ¿Preferencias son realmente single-peaked?
				\4[] $\then$ Sistemas a menudo no son bipartidistas
				\4[] Buena aproximación con decisión sobre aspectos concretos
				\4[] Difícil aplicación general a proceso político
			\3 Heterogeneidad de las conclusiones normativas
				\4 Diferentes corrientes
				\4 No intervencionismo
				\4[] $\to$ No se conoce suficiente sobre elección colectiva
				\4[] $\then$ Probable causar más daño que beneficio
				\4 Más voto
				\4[] Asumiendo votantes más informados
				\4[] $\to$ En aspectos que les conciernen directamente
				\4[] Aumentar frecuencia y granularidad de voto
				\4[] $\to$ En teoría, mejores decisiones
				\4 Elitismo
				\4[] Dentro de public choice propiamente
				\4[] $\to$ Muy poco relevante
				\4[] Votantes no están informados ni deciden bien
				\4[] Estado sí puede hacerlo mejor
				\4[] $\to$ Decide en mayor número de aspectos
				\4[] $\then$ Difícil contrastar con análisis de opt. indiv.
				\4[] $\then$ Implica burócratas y políticos
			\3 Proceso de elección colectiva en democracias\footnote{La obra seminal es Buchanan y Tullock (1962) \textit{The Calculus of Consent}}
				\4 Legislaturas bicamerales
				\4[] Una de ellas representativa de voto proporcional
				\4 Aceptada generalmente en mayoría de democracias
				\4 Análisis centrado en Estados Unidos
			\3 Mayorías necesarias para aprobar legislación
				\4 Downs (1957)
				\4[] Mayoría simple (50\%+1) es eficiente
				\4 Tullock y Buchanan (1962)
				\4[] Mayoría simple tiene un coste de eficiencia
				\4[] $\to$ Los que votan en contra pierden
				\4[] $\to$ Cuanto más ajustada la mayoría, menos eficiencia
				\4[] $\then$ Unanimidad es óptimo
				\4 Trade-off eficiencia vs toma de decisiones
				\4[] Si sólo se aprueban leyes por unanimidad
				\4[] $\to$ Mucho más difícil reaccionar y decidir nada
				\4[] Si se aprueban leyes por mayoría simple
				\4[] $\to$ Pérdida de eficiencia para los que votan no
				\4[] $\then$ Mayor cuanto más ajustada la mayoría
			\3 Log-rolling y elección colectiva
				\4 Paquete de leyes+bien público+impuestos como un todo
				\4[] Presentado al público
				\4[] $\to$ Posible rechazo del aparato legislativo en su conjunto
				\4[] $\to$ Difícil rechazar leyes por separado
				\4 Log-rolling
				\4[] Diferentes sectores en cámaras legislativas
				\4[] $\to$ Con diferentes intereses
				\4[] $\to$ Requieren financiación específica para sus leyes
				\4[] Acuerdos entre diferentes partes
				\4[] $\to$ Votarse a favor mutuamente
				\4[] $\then$ Ambos proyectos se llevan a cabo
				\4[] $\then$ Necesario financiar ambos
				\4[] $\then$ Inflación legislativa
				\4[] $\then$ Tendencia de largo plazo a aumento del gasto
				\4 Aumento de mayorías necesarias para aprobar leyes
				\4[] Reduce coste en términos de eficiencia
				\4[] $\to$ Menor \% minoría oprimida por la mayoría
				\4 Crítica al aumento de \% necesario
				\4[] No decidir nada también es decidir
				\4[] $\uparrow$ de mayoría necesaria
				\4[] $\to$ Aumento de no-decisiones
				\4[] $\then$ Sesgo hacia cierta idea de sociedad óptima
				\4[] $\then$ ¿Es realmente lo que desean los ciudadanos?
			\3 Mecanismos de control de burocracia
				\4 Individualización de responsabilidades
				\4 Independencia de poder político
				\4[] Crea a su vez posible problema de legitimidad
				\4 Salarios de eficiencia
				\4[] Salarios como forma de alterar comportamiento de empleados
				\4[] Mayores salarios pueden incentivar a mejor comportamiento
				\4[] $\to$ Menos optimización de agendas propias
				\4[] $\to$ Más lealtad a superiores
				\4[] Difícil compaginar con:
				\4[] $\to$ Independencia
				\4[] $\to$ Imposibilidad de despido
				\4 ...
		\2 Valoración
			\3 Influencia creciente desde 80s
				\4 Caída de Unión Soviética
				\4 Privatizaciones
				\4 Reformas democráticas en antiguas dictaduras
			\3 Control del poder púbico
				\4 Esfuerzos por aumentar transparencia
				\4[] Facilitar control ciudadano de decisiones del estado
				\4[$\then$] Posible utilización perversa
				\4[] Utilización de transparencia selectiva
				\4[] $\to$ Opinión pública contra competidores
			\3 Política presupuestaria y fiscal
				\4 Introducción de reformas presupuestarias
				\4 Presupuesto de base cero
				\4 Presupuestos de ejecución
				\4 Presupuestos por programas
				\4 Presupuestos participativos
				\4 ...
			\3 Implicaciones ideológicas
				\4 Public choice implica crítica fundamental a Estado
				\4[] Pone de manifiesto
				\4[] $\to$ Problemas de implementación de elección colectiva
				\4 Concepción del estado como poder naturalmente benigno
				\4[] Fuertemente comprometida por public choice
	\1[] \marcar{Conclusión}
		\2 Recapitulación
			\3 Funciones de bienestar social
			\3 Teoría de la elección social
			\3 Public choice
		\2 Idea final
			\3 Cita de Knight (1921)\footnote{Knight, F. H. (1921) \textit{Review of Cassel} Quarterly Journal of Economics.}
				\4 El objetivo último de la teoría económica:
				\4[] Crítica en términos éticos y humanos
				\4[] $\to$ Del funcionamiento del sistema económico
				\4[] $\then$ Teoría del valor y precio es esencial
			\3 Influencia en el debate político
				\4 Dificultades de aplicación
				\4 Teoría enmarca análisis téorico y académico
				\4[] Pero en mayoría de los casos
				\4[] $\to$ Difícil aplicación teórica
			\3 Welfarismo frente a no-welfarismo
				\4 Debate sigue vivo
				\4 Valores son importantes?
				\4 Utilidades de terceros agentes
				\4[] $\to$ Relevante para utilidad de un agente dado?
			\3 Distribuciones de renta y otras variables
				\4 Frecuente objeto de análisis práctico
				\4[] Comparar momentos de distribuciones
				\4[] $\to$ Medias
				\4[] $\to$ Varianzas
				\4[] $\to$ Asimetría
				\4[] $\to$ Kurtosis
			\3 Igualdad de resultados y oportunidades
				\4 Es relevante resultado final
				\4[] ¿O posibilidad de alcanzar cualquier resultado?
			\3 Gran éxito de la teoría de la elección social
				\4 No tanto por informar debate político
				\4 Sino por aumentar claridad del debate
				\4[] ¿qué problemas tiene agregación social?
				\4[] ¿qué juicios de valor se aplican?
				\4[] $\then$ Aumentar transparencia de decisión social
\end{esquemal}



































\graficas

\begin{axis}{4}{Frontera de posibilidades de utilidad con dos agentes}{$u_1$}{$u_2$}{fpu}
	% GFPU	
	\draw[thick] (0,3) to [out=-10, in=100](3,0);
	
	% punto no óptimo
	\node[circle,fill=black,inner sep=0pt,minimum size=5pt] (a) at (1,1) {};
	\node[left] at (1,0.85){x};
	
	% vectores de utilidad preferibles a x
	\draw[dashed] (1,2.6) -- (1,1) -- (2.6,1);
	
\end{axis}

\begin{axis}{4}{Preferencias que implican una fuerte aversión a la desigualdad pero que no son convexas.}{$u_A$}{$u_B$}{adesigualdadsinconvexidad}
	% Bisectriz
	\draw[dashed] (0,0) -- (4,4);
	
	% Preferencias
	\draw[-] (0.5,4) to [out=270, in=150](1.75,2.25) -- (1,1) -- (2.25,1.75) to [out=330,in=180] (4,1.25);
\end{axis}

\begin{axis}{4}{Función de bienestar social utilitarista}{$u_1$}{$u_2$}{fbsutilitarista}
	% GFPU	
	\draw[-] (0,3) to [out=-10, in=100](3,0);
	
	% Curvas de indiferencia
	\draw[dashed] (1,4) -- (4,1);
	\draw[dashed] (2,4) -- (4,2);
	\draw[dashed] (-0.8,4) -- (4,-0.8);	
	
	% Curva de óptimo
	\draw[-] (0.03,4) -- (4,0.03);
	
	% Estado social óptimo
	\node[circle,fill=black,inner sep=0pt,minimum size=5pt] (a) at (2,2) {};
\end{axis}

\begin{axis}{4}{Función de bienestar social rawlsiana}{$u_1$}{$u_2$}{fbsrawlsiana}
	% GFPU	
	\draw[-] (0,3) to [out=-10, in=100](3,0);
	
	% curvas de indiferencia
	\draw[dashed] (1,4) -- (1,1) -- (4,1);
	\draw[dashed] (3,4) -- (3,3) -- (4,3);
	
	% Curva de óptimo
	\draw[-] (2,4) -- (2,2) -- (4,2);
	
	% Estado social óptimo
	\node[circle,fill=black,inner sep=0pt,minimum size=5pt] (a) at (2,2) {};
	
\end{axis}

\begin{axis}{4}{Tres agentes con diferentes preferencias unimodales}{Opción}{}{preferenciasunimodales}
	% puntos en el eje de abscisas
	\node[below] at (1,-0.3){1};
	\draw[-] (1,0.3) -- (1,-0.3);
	\node[below] at (2,-0.3){2};
	\draw[-] (2,0.3) -- (2,-0.3);
	\node[below] at (3,-0.3){3};
	\draw[-] (3,0.3) -- (3,-0.3); 
%	\node[below] at (4,-0.3){4};
%	\draw[-] (4,0.3) -- (4,-0.3);
	
	% puntos en el eje de ordenadas
	\node[left] at (-0.3,1){1};
	\draw[-] (-0.3,1) -- (0.3,1);
	\node[left] at (-0.3,2){2};
	\draw[-] (-0.3,2) -- (0.3,2);
	\node[left] at (-0.3,3){3};
	\draw[-] (-0.3,3) -- (0.3,3);
%	\node[left] at (-0.3,4){4};
%	\draw[-] (-0.3,4) -- (0.3,4);
	
	% primer agente: prefiere opción 1
	\draw[-] (0,3) -- (3,0);
	\node[circle,fill=black,inner sep=0pt,minimum size=5pt] (a) at (0,3) {};
	\node[circle,fill=black,inner sep=0pt,minimum size=5pt] (a) at (1,2) {};
	\node[circle,fill=black,inner sep=0pt,minimum size=5pt] (a) at (2,1) {};
	\node[circle,fill=black,inner sep=0pt,minimum size=5pt] (a) at (3,0) {};
	
	% segundo agente: prefiere opción 2
	\draw[-, color=red] (0,1) -- (2,3) -- (3,2);
	\node[circle,fill=red,inner sep=0pt,minimum size=5pt] (a) at (0,1) {};
	\node[circle,fill=red,inner sep=0pt,minimum size=5pt] (a) at (1,2) {};
	\node[circle,fill=red,inner sep=0pt,minimum size=5pt] (a) at (2,3) {};
	\node[circle,fill=red,inner sep=0pt,minimum size=5pt] (a) at (3,2) {};
	
	% tercer agente: prefiere opción 3
	\draw[-, color=blue] (0,0) -- (3,3);
	\node[circle,fill=blue,inner sep=0pt,minimum size=5pt] (a) at (0,0) {};
	\node[circle,fill=blue,inner sep=0pt,minimum size=5pt] (a) at (1,1) {};
	\node[circle,fill=blue,inner sep=0pt,minimum size=5pt] (a) at (2,2) {};
	\node[circle,fill=blue,inner sep=0pt,minimum size=5pt] (a) at (3,3) {};
\end{axis}

\begin{axis}{4}{Un agente con preferencias multimodales: la preferencia por opciones ordenadas respecto a un eje tiene más de un máximo local.}{Opción}{}{preferenciasmultimodales}
	% puntos en el eje de abscisas
	\node[below] at (1,-0.3){1};
	\draw[-] (1,0.3) -- (1,-0.3);
	\node[below] at (2,-0.3){2};
	\draw[-] (2,0.3) -- (2,-0.3);
	\node[below] at (3,-0.3){3};
	\draw[-] (3,0.3) -- (3,-0.3); 
	%	\node[below] at (4,-0.3){4};
	%	\draw[-] (4,0.3) -- (4,-0.3);
	
	% puntos en el eje de ordenadas
	\node[left] at (-0.3,1){1};
	\draw[-] (-0.3,1) -- (0.3,1);
	\node[left] at (-0.3,2){2};
	\draw[-] (-0.3,2) -- (0.3,2);
	\node[left] at (-0.3,3){3};
	\draw[-] (-0.3,3) -- (0.3,3);
	%	\node[left] at (-0.3,4){4};
	%	\draw[-] (-0.3,4) -- (0.3,4);
	
	% primer agente: prefiere opción 1
	\draw[-] (0,3) -- (2,1) -- (3,2);
	\node[circle,fill=black,inner sep=0pt,minimum size=5pt] (a) at (0,3) {};
	\node[circle,fill=black,inner sep=0pt,minimum size=5pt] (a) at (1,2) {};
	\node[circle,fill=black,inner sep=0pt,minimum size=5pt] (a) at (2,1) {};
	\node[circle,fill=black,inner sep=0pt,minimum size=5pt] (a) at (3,2) {};
	
\end{axis}

\conceptos

\concepto{Funciones y funcionales de bienestar social}

En Bergson (1937) se introduce el concepto de ``\textit{social welfare function}'' (SWF) para caracterizar la ordenación de estados sociales que resulta de unas preferencias dadas. Arrow (1951) utiliza el mismo término pero con un sentido diferente: un SWF es una regla de transformación entre ordenaciones de preferencias individuales y ordenación de preferencias sociales. El SWF de Arrow no es la ordenación final, sino el funcional (porque es una correspondencia de conjuntos de funciones a conjuntos de funciones) que relaciona perfiles de ordenaciones con una ordenación. Little sugirió a Arrow establecer una distinción entre ambos, y Arrow (1963) introduce el término ``\textit{social decision process}'' (SDP) para hacer referencia a las reglas de decisión que habían sido el objeto de su artículo. Así, las SWF pueden asimilarse a la función de utilidad que tendría un individuo dado (ya sea un rey, un ente imaginario que personifica la voluntad de la nación, un líder político, la ``gente'') respecto a lo deseable de las múltiples sociedades que pueden existir. Un SDP o regla de decisión social es sin embargo un mecanismo que transforma cualquier perfil de ordenaciones en una ordenación determinada. El objeto del estudio es claramente distinto, pero ha dado lugar a importantes confusiones y acaba siendo el mayor problema a la hora de entender el objeto de este tema.

\concepto{Funcionales de bienestar social, ordinalidad y cardinalidad}

``\textit{Arrow's social welfare function is a special case of a
	social welfare functional with the invariance requirement
	corresponding to ordinal non-comparability (i.e., if
	one n-tuple of utility functions is replaced by another
	obtained from the first by taking positive, monotonic
	transformations of each utility function - not necessarily
	the same for all - then the social ordering R determined
	by the first n-tuple will also be yielded by the second). It
	is obvious that Arrow's 'impossibility theorem' can be
	translated in the format of social welfare functionals
	with ordinal non-comparability. More interestingly, this
	result can be generalized to the case of cardinal non-
	comparability also. When individual utilities can be
	cardinally measured but not in any way interpersonally
	compared, the same impossibility result continues to
	hold (see Sen, 1970)}'' Ver \textit{social choice} en Palgrave, apartado 6

\preguntas
\seccion{8 de marzo de 2017}
\begin{itemize}
    \item ¿Cómo casan el teorema de la imposibilidad de Arrow con la existencia de óptimos únicos en otras ramas de la Economía?
    \item ¿Qué aplicación práctica tiene el tema? En concreto, en la Constitución Española.
\end{itemize}

\seccion{Test 2019}

\textbf{16.} En el contexto del Teorema de Imposibilidad de Arrow, considere una sociedad en que los votantes tienen preferencias estrictas sobre las alternativas y considere las siguientes funciones de bienestar social:

\begin{itemize}
	\item[i] Unanimidad: la alternativa ``x'' es socialmente preferida a la alternativa ``y'' si todos los votantes prefieren la alternativa ``x'' a la alternativa ``y''. Si al menos un votante prefiere ``x'' a ``y'' y otro votante ``y'' a ``x'', ``x'' e ``y'' son consideradas socialmente indiferentes.
	\item[ii] Mayoría: la alternativa ``x'' es preferida socialmente a la alternativa ``y'' si (y sólo si) el número de votantes que prefieren la alternativa ``x'' a la alternativa ``y'' supera al número de votantes que prefieren la alternativa ``y'' a la alternativa ``x''
\end{itemize}

Señale la afirmación \textbf{\underline{incorrecta}}:

\begin{itemize}
	\item[a] De acuerdo con la regla de unanimidad todos los votantes son dictadores.
	\item[b] La regla de unanimidad satisface el axioma de Independencia de alternativas irrelevantes.
	\item[c] Si todos los votantes tienen las mismas preferencias sobre alternativas, la regla de mayoría generaría unas preferencias sociales sin ciclos.
	\item[d] Las reglas de la mayoría y unanimidad coinciden cuando sólo hay dos votantes y dos alternativas.
\end{itemize}

\seccion{Test 2016}

\textbf{13.} En la teoría de la elección colectiva:

\begin{itemize}
	\item[a] La paradoja del voto de Condorcet no se produce si existen individuos con preferencias multimodales.
	\item[b] La regla de votación por puntos (regla de Borda) no cumple la propiedad de independencia de alternativas irrelevantes.
	\item[c] Una vía de construir la Función de Bienestar Social, cumpliría con las propiedades establecidas por Arrow para que las reglas de elección sean racionales, es a través de la existencia de tecnócratas tomando decisiones que actúen como dictadores benevolentes.
	\item[d] Bajo el enfoque contractualista de Rawls, las curvas de indiferencia social suponen neutralidad al riesgo.
\end{itemize}

\seccion{Test 2015}

\textbf{12.} Señale la respuesta correcta sobre los criterios de bienestar social:

\begin{itemize}
	\item[a] Según el criterio utilitarista, el bienestar social se calcula como la suma del bienestar de los individuos que componen la sociedad. Este criterio emplea funciones de utilidad individuales cardinales y evita las comparaciones interpersonales de utilidad. 
	\item[b] Una de las ventajas del criterio de Pareto es que su uso no implica realizar ningún juicio de valor.
	\item[c] Según el criterio de Kaldor, una asignación dada por el vector $x_1$ es socialmente preferida a otra $x_2$ si partiendo de $x_1$ se puede alcanzar mediante redistribución una tercera asignación $x_3$ que sea Pareto superior a $x_2$.
	\item[d] Según el criterio de Hicks, una asignación dada por el vector $x_1$ es socialmente preferida a otra $x_2$ si los individuos que perderían bienestar con el cambio de asignación pueden compensar a los individuos que mejorarían con el cambio para que este no se produzca.
\end{itemize}

\seccion{Test 2011}

\textbf{10.} Suponga tres individuos con órdenes de preferencias abc, bca y cab sobre tres opciones a, b, c. Para tomar una decisión votan en dos fases: en la primera, dejan una opción fuera y votan entre las otras dos por mayoría, en la segunda fase, votan por mayoría entre la opción ganadora en la primera ronda y la opción que quedó fuera de la votación de la primera ronda. El resultado de la segunda ronda determina la opción finalmente elegida. Si en cada ronda los individuos votan por su opción más preferida:

\begin{itemize}
	\item[a] La opción elegida es la a. 
	\item[b] La opción elegida es la b.
	\item[c] La opción elegida es la c.
	\item[d] La opción elegida puede ser a, b ó c.
\end{itemize}

\notas

Leer Nobel Prize Lecture de Angus Deaton para mejorar este tema.

\textbf{2019.} \textbf{16:} A

\textbf{2016.} \textbf{13:} B

\textbf{2015.} \textbf{12:} C

\textbf{2011.} \textbf{10:} D. Por la paradoja de Condorcet. La agregación de preferencias transitivas da lugar a intransitividades. Si primero se vota $a$ vs $b$, se elige $a$. Después, en $a$ vs $c$, elegimos $c$. Si primero se vota $a$ vs $c$, tenemos un resultado de $c$. A continuación, si se vota $c$ vs $b$, tenemos un resultado de $b$. Por último, si en la primera ronda se vota $b$ vs $c$ tenemos un resultado de $b$. En la segunda ronda tenemos $b$ vs $a$, y el resultado es $a$. De tal manera que las preferencias agregadas son intransitivas tal que $a \succ b \succ c \succ a$.

\bibliografia

Mirar en Palgrave:
\begin{itemize}
	\item Arrow's theorem * 
	\item constitutions, economic approach to
	\item interpersonal utility comparisons
	\item interpersonal utility comparisons (new developments)
	\item public choice
	\item social choice * 
	\item social choice (new developments) *
	\item social welfare function *
	\item strategic voting *
	\item voting paradoxes *
\end{itemize}

Arrow, K. J. \textit{Social choice and individual values} (1951), (1963)

Blankart, C. B.; Koester, G. B. (2005) \textit{Political Economics versus Public Choice} KYKLOS, International Review for Social Sciences -- En carpeta del tema

Buchanan, J. M.; Tullock, G. (1962) \textit{The Calculus of Consent} 

Grandjean, J. \textit{Gordon Tullock on Majority Voting: the Making of a Conviction} En carpeta del tema

Igersheim, H. \textit{The death of welfare economics: History of a controversy} (2017) CHOPE Working Paper -- En carpeta del tema. Interesante para comprender diferencias entre funciones de Bergson-Samuelson y las funciones de bienestar social a las que hacía referencia Arrow, así como la controversia entre el welfare economics y la teoría de la elección social.

Jehle y Rehny. Ch. 6 Social choice and welfare

Kreps. Ch. 5

MWG. Ch 21, 22

Stanford Encyclopedia of Philosophy. \textit{Arrow's Theorem} -- \url{https://plato.stanford.edu/entries/arrows-theorem}

Stanford Encyclopedia of Philosophy. \textit{Social Choice Theory} -- \url{https://plato.stanford.edu/entries/social-choice/}

Tullock, G. \textit{The General Irrelevance of the General Impossibility Theorem} (1967) Quarterly Journal of Economics -- En carpeta del tema



\end{document}
