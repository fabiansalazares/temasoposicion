\documentclass{nuevotema}

\tema{3A-27}
\titulo{Análisis de las tablas <<input-output>>}

\begin{document}

\ideaclave

\seccion{Preguntas clave}

\begin{itemize}
	\item
\end{itemize}

\esquemacorto

\begin{esquema}[enumerate]
	\1[] \marcar{Introducción}
		\2 Contextualización
		\2 Objeto
		\2 Estructura
	\1 \marcar{Análisis input-output}
		\2 Idea clave
	\1 \marcar{Extensiones}
	\1 \marcar{Aplicaciones del análisis input/output}
		\2 Macroeconomía
		\2 Comercio internacional
		\2 Finanzas públicas
		\2 Medio ambiente
		\2 Desarrollo económico
			\3 Ver ``linkages''
	\1[] \marcar{Conclusión}
		\2 Recapitulación
		\2 Idea final

\end{esquema}

\esquemalargo












\begin{esquemal}
	\1[] \marcar{Introducción}
		\2 Contextualización
		\2 Objeto
		\2 Estructura
	\1 \marcar{Análisis input-output}
		\2 Idea clave
		
	\1 \marcar{Extensiones}
	\1 \marcar{Aplicaciones del análisis input/output}
		\2 Macroeconomía
		\2 Comercio internacional
		\2 Finanzas públicas
		\2 Medio ambiente
		\2 Desarrollo económico
			\3 Ver ``linkages''
	\1[] \marcar{Conclusión}
		\2 Recapitulación
		\2 Idea final
\end{esquemal}























\preguntas

\notas

\bibliografia

Mirar en Palgrave:
\begin{itemize}
	\item Hawkins-Simon conditions
	\item input-output analysis
	\item Leontief Paradox
	\item linear models
	\item linkages
\end{itemize}

\end{document}