\documentclass{nuevotema}

\tema{3A-38}
\titulo{La financiación del déficit público. Sostenibilidad del déficit público. Aspectos monetarios de su financiación.}

\begin{document}

\ideaclave

Ver Mauro y Zhou (2020) sobre evidencia empírica respecto al diferencial entre $r$ y $g$, y las variaciones previas a defaults. En carpeta del tema y \href{https://www.imf.org/en/Publications/WP/Issues/2020/03/13/r-minus-g-negative-Can-We-Sleep-More-Soundly-49068}{aquí}.

\seccion{Preguntas clave}

\begin{itemize}
	\item ¿Qué es el déficit público?
	\item ¿Cómo puede financiarse el déficit público?
	\item ¿Qué fuentes de financiación son preferibles en cada contexto?
	\item ¿Qué efectos tienen las diferentes fuentes de financiación?
	\item ¿Qué efecto tienen los déficits públicos sobre la acumulación de capital?
	\item ¿Cuando un déficit público es sostenible?
	\item ¿En qué consiste la financiación monetaria del déficit?
	\item ¿Qué efectos tiene la financiación monetaria sobre otras variables?
\end{itemize}

\esquemacorto

\begin{esquema}[enumerate]
	\1[] \marcar{Introducción}
		\2 Contextualización
			\3 Macroeconomía
			\3 Importancia del sector público
			\3 Justificación de la intervención pública
			\3 Instrumentos de actuación
			\3 Déficit público
		\2 Objeto
			\3 ¿Qué es el déficit público?
			\3 ¿Cómo puede financiarse el déficit público?
			\3 ¿Qué fuentes de financiación son preferibles en cada contexto?
			\3 ¿Qué efectos tiene el déficit sobre la acumulación de capital?
			\3 ¿Qué efectos tiene el déficit público sobre la renta futura?
			\3 ¿Qué efectos tienen las diferentes fuentes de financiación?
			\3 ¿Cuando un déficit público es sostenible?
			\3 ¿En qué consiste la financiación monetaria del déficit?
			\3 ¿Qué efectos tiene la financiación monetaria sobre otras variables?
		\2 Estructura
			\3 Financiación del déficit público
			\3 Sostenibilidad del déficit y la deuda
			\3 Aspectos monetarios de la financiación
	\1 \marcar{Financiación del déficit público}
		\2 Idea clave
			\3 Contexto
			\3 Objetivo
			\3 Resultados
		\2 Instrumentos de financiación
			\3 Impuestos
			\3 Deuda
			\3 Financiación monetaria
			\3 Otras fuentes de financiación
		\2 Análisis teórico
			\3 Ricardo
			\3 Keynesianismo
			\3 IS-LM
			\3 Monetarismo
			\3 Blinder y Solow (1973)
			\3 Equivalencia ricardiana
			\3 Modelos neoclásicos modernos
		\2 Evidencia empírica
			\3 Consumo y ahorro
			\3 Interés
			\3 Cuenta corriente
			\3 Renta
	\1 \marcar{Sostenibilidad del déficit y la deuda}
		\2 Idea clave
			\3 Contexto
			\3 Objetivos
			\3 Resultados
		\2 Dinámica de la deuda
			\3 Idea clave
			\3 Formulación
			\3 Implicaciones
			\3 Análisis de sensibilidad
		\2 Políticas de sostenibilidad de la deuda
			\3 Consolidación fiscal
			\3 Default, quita o reestructuración
			\3 Reformas estructurales
			\3 Enajenación del patrimonio del estado
			\3 Represión financiera
			\3 Monetización del déficit
	\1 \marcar{Aspectos monetarios de la financiación}
		\2 Idea clave
			\3 Contexto
			\3 Objetivo
			\3 Resultados
		\2 Señoreaje e impuesto inflacionario
			\3 Idea clave
			\3 Formulación
			\3 Implicaciones
			\3 Valoración
		\2 Hiperinflación
			\3 Idea clave
			\3 Formulación
			\3 Implicaciones
			\3 Valoración
		\2 Financiación monetaria en economía abierta
			\3 Idea clave
			\3 Formulación
			\3 Implicaciones
			\3 Valoración
		\2 Aritmética monetarista desagradable
			\3 Idea clave
			\3 Formulación
			\3 Implicaciones
		\2 Teoría fiscal del nivel de precios
			\3 Idea clave
			\3 Formulación
			\3 Implicaciones
			\3 Valoración
	\1[] \marcar{Conclusión}
		\2 Recapitulación
			\3 Financiación del déficit público
			\3 Sostenibilidad del déficit y la deuda
			\3 Aspectos monetarios de la financiación
		\2 Idea final
			\3 Evolución histórica de los déficits
			\3 Unión Europea
			\3 España
			\3 Países en desarrollo

\end{esquema}

\esquemalargo


\begin{esquemal}
	\1[] \marcar{Introducción}
		\2 Contextualización
			\3 Macroeconomía
				\4 Análisis de fenómenos económicos a gran escala
				\4 Énfasis sobre variables agregadas
			\3 Importancia del sector público
				\4 Cualitativa
				\4[] Condiciona fuertemente las decisiones privadas
				\4[] $\to$ Poder coactivo
				\4[] $\to$ Superioridad de medios en países desarrollados
				\4 Cuantitativa
				\4[] Gasto público es 40\% de PIB en OCDE
				\4[] $\sim 5000000$ de gasto total en España
			\3 Justificación de la intervención pública
				\4 Marco básico de funcionamiento
				\4[] Marco legal de actuación
				\4[] Reducir incertidumbre de agentes económicos
				\4[] Garantizar derechos de propiedad
				\4 Eficiencia
				\4[] Presencia de fallos de mercado
				\4[] $\to$ Asignaciones ineficientes en sentido de Pareto
				\4 Equidad
				\4[] Sociedad realiza juicios de valor
				\4[] sobre deseable de asignaciones
				\4[] $\to$ Actúa para cambiarlas
				\4 Estabilización
				\4[] Suavizar fluctuaciones cíclicas
				\4[] Reducir impacto de shocks sobre bienestar
			\3 Instrumentos de actuación
				\4 Regulación
				\4[] Disposiciones legales y reglamentarias
				\4[] Cumplimiento mediante poder coactivo
				\4 Empresas públicas
				\4[] Ordenación de factores productivos
				\4[] directamente por el Estado
				\4[] $\to$ Proveer bienes y servicios
				\4 Presupuesto público
				\4[] Recaudar fondos mediante ingresos públicos
				\4[] Distribuirlos mediante gasto público
			\3 Déficit público
				\4 Gasto público no financiado por vía fiscal
				\4[] $\to$ SPúblico tiene necesidad de financiación
				\4 Necesario recurrir a fuentes alternativas
				\4[] $\to$ Pedir prestado
				\4[] $\to$ Expansión de base monetaria
				\4[] $\to$ Enajenar patrimonio público
				\4 Efectos sobre macroeconomía
				\4[] Corto plazo
				\4[] Largo plazo
				\4 Sostenibilidad
				\4[] Déficit es flujo de un stock (deuda)
				\4[] $\to$ Proceso dinámico
				\4[] $\to$ Necesario valorar convergencia hacia estabilidad
				\4 Financiación monetaria
				\4[] Medida disponible con:
				\4[] $\to$ Dinero fiduciario
				\4[] $\to$ Autonomía monetaria
				\4[] Ventajas e inconvenientes
		\2 Objeto
			\3 ¿Qué es el déficit público?
			\3 ¿Cómo puede financiarse el déficit público?
			\3 ¿Qué fuentes de financiación son preferibles en cada contexto?
			\3 ¿Qué efectos tiene el déficit sobre la acumulación de capital?
			\3 ¿Qué efectos tiene el déficit público sobre la renta futura?
			\3 ¿Qué efectos tienen las diferentes fuentes de financiación?
			\3 ¿Cuando un déficit público es sostenible?
			\3 ¿En qué consiste la financiación monetaria del déficit?
			\3 ¿Qué efectos tiene la financiación monetaria sobre otras variables?
		\2 Estructura
			\3 Financiación del déficit público
			\3 Sostenibilidad del déficit y la deuda
			\3 Aspectos monetarios de la financiación
	\1 \marcar{Financiación del déficit público}
		\2 Idea clave
			\3 Contexto
				\4 Ingresos fiscales a menudo insuficientes
				\4 Adm. fiscal es costosa
				\4[] Elevados costes iniciales
				\4[] Costes variables ligados a recaudación
				\4 Expansión del sector público
				\4[] Tendencia de largo plazo creciente
			\3 Objetivo
				\4 Caracterizar formas de financiación
				\4 Valorar alternativas
				\4 Caracterizar efectos
			\3 Resultados
				\4 Debate a lo largo de historia
				\4 Instituciones y desarrollo condicionan
				\4[] Instrumentos de financiación disponibles
				\4[] Reacción de otros agentes
		\2 Instrumentos de financiación
			\3 Impuestos
				\4 Ingresos obtenidos por vía coactiva
				\4 Sin contraprestación definida
				\4 Principal fuente de ingresos en mayoría de países
			\3 Deuda
				\4 Contracción de obligaciones con terceros
				\4 Ingresos presentes a cambio de pagos futuros
				\4 No implica coactividad
			\3 Financiación monetaria
				\4 En contexto de monopolio de emisión de moneda
				\4[] Sólo estado puede emitir moneda de curso legal
				\4 Emisión de monedas y billetes
				\4[] Para cubrir nec. de financiación
			\3 Otras fuentes de financiación
				\4 Venta de patrimonio público
				\4 Tasas
				\4[] Cubrir coste público de servicio prestado
				\4[] $\to$ Sin alternativas privadas
				\4[] No implica coactividad
				\4 Contribuciones especiales
				\4[] Pago coactivo derivado del aumento de valor de bienes privados
				\4[] Resultado de prestación de servicio público
				\4 Multas y sanciones
		\2 Análisis teórico
			\3 Ricardo
				\4 Impuestos
				\4[] Mal en cualquiera de sus formas
				\4[] Lamentablemente, necesarios en algunas ocasiones
				\4[] Sin teoría explícita de tributación óptima
				\4 Deuda
				\4[] Plantea posibilidad de equivalencia ricardiana
				\4[] Aunque rechaza considerar seriamente
				\4 Minimizar gasto público
				\4[] Sea cual sea su forma de financiación
				\4[] Perjudica a la acumulación de capital
				\4[] Perjudica al trabajo
			\3 Keynesianismo
				\4 Déficit públicos aumentan demanda agregada
				\4 Estímulo a demanda agregada
				\4[] En contexto de exceso de capacidad
				\4[] $\to$ Aumenta output y empleo
				\4[] $\then$ Multiplicador positivo
				\4 Modelo keynesiano simple
				\4[] $Y = C_0 + c Y + G - T$
				\4[] $\to$ $Y = \frac{C_0 + G - T}{1-c}$
				\4[] $\then$ $\frac{d \, Y}{d \, (G - T)} = \frac{1}{1-c}$
				\4 Expansión de output tras aumento del déficit
				\4[] Aumenta consumo y ahorro
				\4[] Aumenta retorno a la inversión
				\4[] $\to$ Aumenta también inversión
				\4 Déficit implica emisión de deuda
				\4 Demanda de dinero sensible a interés
				\4[] Aumento del déficit aumenta interés
				\4[] Aumento del interés reduce inversión
				\4[] $\to$ Parte del efecto expansivo se ve compensado
				\4 Financiación monetaria del déficit
				\4[] Banco central puede comprar deuda del gobierno
				\4[] Tipo de interés se mantiene bajo
				\4[] $\to$ Efecto más expansivo
				\4 Implicaciones
				\4[] Gobiernos pueden usar déficit para aumentar
				\4[] $\to$ Renta
				\4[] $\to$ Empleo
				\4[] $\to$ En la medida en que haya capacidad sin utilizar
				\4[] Política monetaria pasiva subordinada a fiscal
				\4[] $\to$ Para mantener tipo de interés bajo
			\3 IS-LM
				\4 Formalización del modelo keynesiano
				\4[] IS: $Y = C_0 + c(1-t)(Y+\text{TR}) + I(r) + G_0$
				\4[] LM: $\frac{M}{P} = L(Y,r)$
				\4 Multiplicadores
				\4[] Multiplicador del gasto público
				\4[] $\frac{d Y}{d G} = \frac{1}{1-c(1-t) + I_i \frac{L_Y}{L_i}}$
				\4[] Multiplicador de las transferencias
				\4[] $\frac{d Y}{d \text{TR}} = \frac{c}{1-c(1-t) + I_i \frac{L_Y}{L_i}}$
				\4[] Multiplicador de los impuestos
				\4[] Tomando $T=t\cdot y$
				\4[] $\frac{d Y}{d T} = \frac{-c}{1-c(1-t) + I_i \frac{L_Y}{L_i}}$
				\4 Implicaciones
				\4[] Aumento del gasto con déficit
				\4[] $\to$ Expansivo sobre output
				\4[] $\to$ Aumento del interés reduce multiplicador
				\4[] Aumento de oferta de bonos
				\4[] $\to$ Necesario mayor interés para equilibrar mercado
				\4[] $\to$ Reducción de inversión vía aumento del interés
				\4[] $\then$ Crowding-out financiero
				\4[] Aumento del gasto sin aumento del déficit
				\4[] $\to$ $d G = d T$
				\4[] $\then$ $\frac{d \, Y}{d \, G} d G + \frac{d \, Y}{d T} \cdot dT = \frac{1-c}{1-c(1-t) + I_i \frac{L_Y}{L_i}} \cdot d G$
				\4[] $\then$ Multiplicador menor que con aumento del déficit\footnote{Es decir, con $d T$}
				\4[] Sensibilidad de inversión y dda. dinero a interés
				\4[] $\to$ Determina grado de crowding-out de inversión
				\4[] $\then$ Inversión muy sensible, crowding-out elevado
				\4[] $\then$ Dda. dinero poco sensible, fuerte aumento de interés
				\4[] Efecto del déficit sobre interés
				\4[] $\to$ Determina efecto total sobre demanda agregada
			\3 Monetarismo
				\4 Demanda de dinero muy poco sensible a interés
				\4[] Para equilibrar mercado monetario
				\4[] $\to$ Aumentos de output...
				\4[] $\then$ ...requieren fuertes aumentos del interés
				\4 Demanda de inversión
				\4[] Muy sensible al tipo de interés
				\4[] Déficits públicos financiados con deuda
				\4[] $\to$ Aumento del interés de la deuda pública
				\4[] $\to$ Demanda de dinero poco sensible a interés
				\4[] $\then$ Necesario fuerte $\uparrow r$ de interés para eq. mercado monetario
				\4[] $\then$ Crowding-out de la inversión privada
				\4 Efecto riqueza
				\4[] Déficit aumenta DPública en manos de SPrivado
				\4[] Deuda pública es outside-money para SPrivado
				\4[] $\to$ Aumento de riqueza de sector privado
				\4[] Aumento de riqueza aumenta demanda de dinero
				\4[] $\to$ Desplaza curva LM hacia dentro
				\4[] $\then$ Efecto contractivo del aumento de riqueza
				\4[] $\then$ Reduce multiplicador del gasto vía déficit
				\4[] Aumento de riqueza aumenta demanda de consumo
				\4[] $\to$ Desplazamiento ulterior de IS hacia afuera
				\4[] $\then$ Efecto expansivo del aumento de riqueza
				\4[] $\then$ Cuestión empírica si es menor o mayor que el otro
				\4[] Monetaristas afirman tiene efecto total contractivo
				\4 Implicaciones
				\4[] Déficit público financiado con deuda
				\4[] $\to$ Aumenta interés y crowding-out
				\4[] $\to$ Induce efecto riqueza
				\4[] $\then$ Muy poco efecto sobre output
				\4[] Déficit público financiado con emisión de dinero
				\4[] $\to$ Mantiene interés bajo
				\4[] $\to$ Aumenta oferta monetaria
				\4[] $\then$ Efecto sobre output porque aumenta dinero
			\3 Blinder y Solow (1973)
				\4 Análisis dinámico del déficit financiado con deuda
				\4[] Déficit y deuda resultan de proceso dinámico
				\4[] $\to$ Deuda afecta a déficit y viceversa
				\4[] $\to$ Diferentes canales de actuación
				\4 Estabilidad de economía
				\4[] Economía en equilibrio sólo si no hay déficit
				\4[] Déficit puede desencadenar proceso estable o inestable
				\4 Supuestos monetaristas
				\4[] Efecto riqueza neto es contractivo
				\4[] Aumento del déficit induce contracción
				\4[] $\to$ Cae recaudación impositiva
				\4[] $\then$ Aumento ulterior del déficit
				\4[] $\then$ Economía tiende a inestabilidad
				\4 Implicaciones
				\4[] Déficit público impide economía en equilibrio
				\4[] $\to$ Dinámica puede ser estable o inestable
				\4[] Supuestos monetaristas implican inestabilidad
				\4[] $\to$ Pero en la práctica, déficits no desestabilizan
				\4[] Supuestos monetaristas implican:
				\4[] $\to$ Financiación monetaria mejor que déficit
				\4[] Pero dados supuestos y proceso dinámico
				\4[] $\to$ Financiación vía deuda es mejor que monetaria
				\4[] $\then$ Efecto expansivo más duradero
				\4[] $\then$ Pagos de interés y mayor riqueza
			\3 Equivalencia ricardiana
				\4 Teorema de la separación de Fisher
				\4[] Dadas:
				\4[] $\to$ Una renta inicial
				\4[] $\to$ Una tecnología de inversión
				\4[] $\to$ Un mercado financiero
				\4[] $\to$ Un agente consumidor
				\4[] La decisión de inversión óptima
				\4[] $\to$ No depende de las preferencias del consumidor
				\4[] $\to$ Es tal que PMg de inversión = interés
				\4[] $\then$ Da igual preferencias de gasto ahora o después
				\4 Contexto keynesiano previo
				\4[] Deuda de SPúblico frente a SPrivado es riqueza neta
				\4[] Déficit es estímulo a demanda agregada
				\4[] $\to$ Capacidad productiva sin utilizar
				\4[] $\then$ Déficits crean valor
				\4 Contexto teórico en años 70
				\4[] Hipótesis del ciclo vital/renta permanente
				\4[] $\to$ Modigliani y Friedman
				\4[] $\to$ Agentes consideran toda su renta
				\4[] Agentes racionales optimizadores
				\4[] Expectativas racionales
				\4[] Mercados financieros eficientes
				\4[] Modigliani y Miller (1958)
				\4[] $\to$ Valor de empresa independiente de financiación
				\4[] Generaciones solapadas
				\4[] $\to$ Diamond (1964)
				\4 David Ricardo (1820)
				\4[] Ensayo sobre la financiación
				\4[] Postula que impuestos y deuda son equivalentes
				\4[] $\to$ Financiar gasto con impuestos hoy
				\4[] $\to$ Financiar gasto con deuda hoy e impuestos mañana
				\4[] $\then$ SPrivado ahorrará hoy para pagar impuestos mañana
				\4[] Afirma no tiene relevancia práctica
				\4[] $\to$ Agentes privados planifican de forma miope
				\4 Barro (1974)
				\4[] Recupera intuición de Ricardo
				\4[] Formula resultado de equivalencia déficit-impuestos
				\4[] $\to$ En contexto de generaciones solapadas
				\4[] $\to$ En marco Diamond (1964)
				\4[] Supuestos:
				\4[] i) Vínculo altruista intergeneracional
				\4[] ii) Mercados financieros perfectos\footnote{O imperfectos de manera muy concreta que no restrinja la senda de ahorro/endeudamiento deseada por los agentes.}
				\4[] iii) Consumidores racionales con HER
				\4[] iv) Solidaridad intertemporal entre familias
				\4[] $\to$ No entre familias con diferentes PMg a consumir
				\4[] v) Impuestos no distorsionadores
				\4[] vi) Déficits públicos no crean valor
				\4[] $\to$ No tienen efectos ``keynesianos''
				\4[] $\to$ Economía en pleno empleo
				\4[] vii) Posibilidad de déficit no altera proceso político
				\4 Formulación de Barro
				\4[] Agentes con horizontes vitales finitos
				\4[] Supuestos anteriores
				\4[] Utilidad de generación $t+1$
				\4[] $\to$ Incorporada en utilidad de generación $t$
				\4[] $\then$ Resultado de equivalencia
				\4[] Agente representativo de horizonte infinito
				\4[] Gobierno que financia gasto exógeno
				\4[] $\to$ Con déficit--deuda e impuestos
				\4[] $\underset{c_t}{\max} \quad U = \sum_{t=0}^{\infty} \beta^t u(c_t)$
				\4[] $\text{s.a}: \, \quad \sum_{t=0}^\infty c_t \cdot \frac{1}{(1+r)^t} = \sum_{t=0}^\infty (y_t - \tau_t) \cdot \frac{1}{(1+r)^t} $
				\4[] $\quad \quad \quad \sum_{t=0}^\infty \tau_t \cdot \frac{1}{(1+r)^t} = \sum_{t=0}^\infty g_t \cdot \frac{1}{(1+r)^t}$
				\4[] Condición de primer orden:
				\4[] $\to$ $u'(c_t) = (1+r) \cdot \beta \cdot u'(c_{t+1})$
				\4[] Implicaciones
				\4[] $\to$ C = a RPermanente total $\sum_{t=0}^\infty (y_t - \tau_t) \cdot \frac{1}{(1+r)^t} $
				\4[] $\to$ Aumento de VPresente de impuestos tiene ERenta negativo
				\4[] $\to$ Distribución temporal de déficit--impuestos no afecta CPO
				\4[] $\then$ Senda déficit--impuestos no afecta consumo
				\4[] $\then$ Senda déficit--impuestos sí afecta ahorro
				\4[] $\then$ Deuda hoy aumenta ahorro hoy
				\4[] $\then$ Agentes ahorran para pagar impuestos mañana
				\4[] $\then$ Mismos resultados con oferta de trabajo
				\4 Implicaciones
				\4[] Aumentos del déficit
				\4[] $\to$ inducen $\uparrow$ del ahorro privado
				\4[] $\to$ Tipos de interés no se ven afectados
				\4[] Bajadas temporales de impuestos
				\4[] $\to$ Apenas afectan consumo
				\4[] $\uparrow$ del déficit sin cambio en VPresente de gasto
				\4[] $\to$ Sin efecto sobre demanda agregada y output
				\4[] DPública en manos privadas no es riqueza neta
				\4[] $\to$ Representa un activo hoy
				\4[] $\to$ Implica un pasivo mañana
				\4[] $\then$ En neto, activo hoy y deuda mañana se anulan
				\4[] $\then$ DPública en manos privadas tiene valor nulo
			\3 Modelos neoclásicos modernos
				\4 Modelos DSGE que incorporan gobierno
				\4[] Cumplen equivalencia ricardiana dados supuestos
				\4[] Representan también efectos no ricardianos de déficit
				\4[] $\to$ Relajando restricciones
				\4 Horizontes finitos sin altruismo intergeneracional
				\4[] Agentes no tienen en cuenta generaciones futuras
				\4[] Ejemplo: gobierno aumenta gasto y déficit hoy
				\4[] $\to$ Viejos de hoy no sufrirán $\uparrow$ impuestos
				\4[] $\then$ Aumentan consumo y estímulo a demanda agregada
				\4[] $\then$ Aumenta demanda de trabajo
				\4[] $\to$ Jóvenes de hoy tendrán que pagar deuda mañana
				\4[] $\to$ Jóvenes ofertan más trabajo hoy para ahorrar
				\4[] $\then$ Aumenta output
				\4[] $\then$ Aumenta oferta de trabajo
				\4[] $\to$ Resultado no ricardiano
				\4[] $\then$ Déficit más expansivo que impuestos
				\4 Restricciones de liquidez
				\4[] Agentes no pueden ahorrar o tomar prestado
				\4[] $\to$ Tanto como quisieran
				\4[] $\to$ A interés de mercado
				\4[] Cuando restricciones vinculan
				\4[] $\to$ Senda de ahorro-consumo no es óptima
				\4[] Déficit del gobierno puede hacer vinculantes restricciones
				\4[] $\to$ Agentes se financian más caro
				\4[] $\then$ Déficit vía deuda puede crear valor
				\4[] Ejemplo: Gobierno se financia más barato
				\4[] $\to$ Sector privado se endeuda a $\tilde{r}$
				\4[] $\to$ Gobierno se endeuda a a $r < \tilde{r}$
				\4[] $\then$ Agentes privados difieren pago vía gobierno
				\4[] $\then$ Agentes privados pagan menos interés
				\4[] $\then$ Gasto financiado con deuda a $r$ crea valor
				\4 Impuestos distorsionadores
				\4[] Impuestos de suma fija no afectan precios relativos
				\4[] $\to$ No alteran decisiones de consumo-ocio
				\4[] $\to$ No alteran dda. consumo intertemporal
				\4[] Impuestos sobre trabajo alteran precio del ocio
				\4[] $\to$ Aumentan demanda de ocio
				\4[] $\to$ Reducen demanda de consumo
				\4[] $\then$ Reducen inversión
				\4[] $\then$ Reducen ahorro
				\4[] $\then$ Afectan a output
				\4[] Impuestos pueden tener incidencia variable
				\4[] $\to$ Entre grupos de población
				\4[] $\to$ Entre generaciones
				\4[] Incertidumbre sobre impuestos distorsiona decisión
				\4[] $\to$ Preferible suavización impositiva
				\4[] $\then$ Deuda como elemento compensador
				\4[$\then$] Posible encontrar senda óptima de impuestos-deuda
				\4[] $\to$ Financiación del déficit sí importa
		\2 Evidencia empírica
			\3 Consumo y ahorro
				\4 Mayoría de estudios muestran relación positiva
				\4[] Más déficit implica más consumo
				\4[] $\to$ Contrario a evidencia ricardiana
				\4 Barro apunta dos experimentos naturales
				\4[] Carroll y Summers (1987): Canada y Estados Unidos
				\4[] $\to$ Similares tasas de ahorro hasta 70s
				\4[] $\to$ Aumento del déficit público en Canadá en 80s
				\4[] $\then$ Aumento del ahorro inmediatamente
				\4[] $\then$ Ahorro nacional (pub + priv) estable
				\4[] Israel en años 80
				\4[] $\to$ Fuerte aumento del déficit e inflación
				\4[] $\to$ Paralelo aumento inmediato del ahorro privado
				\4[] $\then$ Favorable a equivalencia
			\3 Interés
				\4 Evidencia mixta
				\4 Estudios favorables a una y otra postura
				\4[] --Aumento del déficit público aumenta interés
				\4[] --Sin relación entre déficit público e interés
				\4 Muy difícil contrastación
				\4[] Déficit endógeno a interés
				\4[] Tipos de interés diversos para deuda pública
			\3 Cuenta corriente
				\4 Déficit gemelos
				\4[] Déficit por cuenta corriente + público
				\4[] $\to$ Ocurrencia rel. habitual en ec. abierta
			\3 Renta
				\4 Keynesianos estiman formas reducidas
				\4[] Resultados favorables a efectos expansivos
				\4[] $\to$ Déficits aumentan output
				\4 Crítica a formas reducidas
				\4[] Mezclan efectos de muchas variables
				\4[] Puede existir endogeneidad o causalidad inversa
				\4[] $\to$ Ejemplo: crecimiento aumenta déficit
				\4 Déficits más expansivos en recesión
				\4[] Mayoría de estudios apuntan
				\4[] $\to$ Salvo problemas de sostenibilidad
	\1 \marcar{Sostenibilidad del déficit y la deuda}
		\2 Idea clave
			\3 Contexto
				\4 Modelos anteriores asumen deuda
				\4 Deuda pública es stock afectado por flujos
				\4[] Déficit público
				\4[] Intereses del stock de deuda
				\4[] Inflación
				\4[] Crecimiento del output
				\4 Relaciones complejas entre flujos
				\4[] Unos y otros factores tienen efectos recíprocos
				\4 Sostenibilidad
				\4[] Definición del FMI
				\4[] Deuda pública es sostenible si:
				\4[] $\to$ Dado el coste de la financiación que enfrenta
				\4[] $\to$ valor actual de superávits primarios $\geq$ deuda
				\4[] $\then$ Sin implementar una consolidación irrealista
				\4[] $\then$ Sin inflactar deuda
				\4 Solvencia
				\4[] Valor presente de obligaciones del gobierno
				\4[] $\to$  Inferior a valor presente de ingresos
				\4[] Ignora problemas de liquidez puntuales
				\4[] Sostenibilidad sí considera
				\4[] $\to$ Junto con disposición a pagar
				\4[] $\then$ Sostenibilidad es más estricta que solvencia
				\4 Iliquidez
				\4[] Carencia de activos líquidos
				\4[] Para hacer frente a compromisos en el corto plazo
			\3 Objetivos
				\4 Caracterizar evolución de deuda dados factores
				\4 Valorar posibilidad de financiación vía deuda
			\3 Resultados
				\4 Análisis de dinámica de la deuda
				\4[] Cuantificación de solvencia
				\4 Caracterización de alternativas de política económica
				\4[] Qué políticas pueden implementarse para sostenibilidad
				\4 Análisis de sensibilidad
				\4[] Qué impacto tienen diferentes factores sobre sostenibilidad
		\2 Dinámica de la deuda
			\3 Idea clave
				\4 Dados:
				\4[] Crecimiento del PIB
				\4[] Interés nominal
				\4[] Inflación
				\4[] Stock inicial de deuda
				\4[] Déficit
				\4[$\to$] ¿Cómo será la deuda en el futuro?
				\4[$\to$] ¿Podrá pagarse?
			\3 Formulación
				\4 Restricción presupuestaria del gobierno
				\4[]  $\underbrace{P_t G_t + i_t D_{t-1}}_{\text{Gastos totales}}= \underbrace{P_t T_t + (M_t - M_{t-1}) + D_t - D_{t-1}}_{\text{Ingresos totales}}$
				\4 Déficit primario
				\4[] $P_t G_t - P_t T_t = (M_t - M_{t-1}) + D_t - (1+i) D_{t-1}$
				\4 Asumimos financiación monetaria no disponible
				\4[] $M_t - M_{t-1} = 0$
				\4[] Aspectos monetarios examinados posteriormente
				\4 Dividiendo por renta nominal $P_t Y_t$
				\4[] Asumiendo que recaudación crece linealmente con economía
				\4[] $-\text{sp}_t \equiv g_t - t_t = d_t - (1+i_t) \frac{D_{t-1}}{P_t Y_t}$
				\4 Multiplicando y dividendo por $P_{t-1} Y_{t-1}$
				\4[] Denominando $g$ a $\frac{Y_t - Y_{t-1}}{Y_{t-1}}$, $\pi$ a $\frac{Y_t - Y_{t-1}}{Y_{t-1}}$
				\4[] $-\text{sp}_t =  d_t -  d_{t-1} \frac{(1+i_t)}{(1+g)(1+\pi)}$
				\4[] Interés real $(1+r) = \frac{1+i}{1+\pi}$
				\4[] $\then$ $-\text{sp}_t = d_t - \frac{1+r_t}{1+g_t} d_{t-1}$
				\4 Dinámica del endeudamiento
				\4[] \fbox{$d_t = -\text{sp}_t + \frac{1+r}{1+g} d_{t-1}$}
				\4[] Partiendo de $b_0$ e iterando hacia delante
				\4[] \fbox{$d_0 = \left( \frac{1+r}{1+g} \right)^{-T} d_T  - \sum_{t=1}^T \left( \frac{1+r}{1+g} \right)^{-t} \text{sp}_t$}
				\4 Condición de no-Juego de Ponzi
				\4[] Imposible acumular deuda indefinidamente
				\4[] $\to$ Valor presente de deuda debe tender a anularse
				\4[] $\then$ $\lim_{T \to \infty} \left( \frac{1+r}{1+g} \right)^{-T} d_T = 0$
				\4[] $\then$ Si hay deuda, serán necesarios súp. primarios
				\4 Restricción presupuestaria intertemporal
				\4[] Valor presente neto de saldos primarios
				\4[] $\to$ Deben igualar nivel inicial de deuda
				\4[] \fbox{$d_0 = \sum_{t=1}^T \left[ \left( \frac{1+r}{1+g} \right)^{-t} \text{sp}_t \right]$}
				\4 Convergencia de la deuda
				\4[] Deuda ya existente puede refinanciarse
				\4[] $\to$ Si condiciones de sostenibilidad constantes
				\4[] Deuda debe mantenerse constante como \% de PIB
				\4[] $\to$ Convergencia hacia valor estable
				\4[] $d_t=d_{t-1}$ $\then$ \fbox{$\frac{r-g}{1+g} d = \bar{\text{sp}}$}
				\4[] $\then$ $\bar{\text{sp}}$ es superávit que mantiene $b_t = b_{t-1}$
				\4[] $\then$ Con $r>g$, necesario superávit primario $\bar{\text{sp}} > 0$
				\4[] $\then$ Con $g>r$, posible déficit primario $\bar{\text{sp}} < 0$
			\3 Implicaciones
				\4 Convergencia y divergencia de la deuda
				\4[] Interés real mayor que crecimiento $r>g$
				\4[] $\to$ Deuda sobre \% aumenta por intereses
				\4[] $\to$ Sin déficit primario, cada vez mayor deuda
				\4[] $\then$ Necesario súp. primario para estabilizar/reducir
				\4[] Representación gráfica
				\4[] \grafica{rgtg}
				\4[] Interés real menor que crecimiento $r<g$
				\4[] $\to$ Convergencia a nivel estable
				\4[] $\to$ Nivel puede ser mayor o menor que deuda actual
				\4[] $\to$ Deuda constante $b_t=b_{t-1}$ con $d_t = \bar{d}$
				\4[] Representación gráfica
				\4[] \grafica{rltg}
				\4[$\then$] Necesarios superávits primarios si $r > g$
				\4[] Cuantía debe ser suficiente
				\4[] $\to$ Para evitar crecimiento explosivo de la deuda
				\4[$\then$] Posibles déficits primarios si $r<g$
				\4[] Si no superan $\bar{d}$
				\4[] $\to$ Deuda converge a nivel estable inferior a actual
				\4[] Si superan $\bar{d}$
				\4[] $\to$ Deuda converge a nivel estable superior a actual
			\3 Análisis de sensibilidad
				\4 Variables sujetas a incertidumbre y endogeneidad
				\4 Habitual estimación de escenarios
				\4[] Pesimista, conservador, favorable...
				\4 Postular:
				\4[] -- Relaciones entre variables
				\4[] -- Distribuciones de probabilidad sobre variables
				\4 Caracterizar escenarios
				\4[] Dinámicas de deuda en diferentes contextos
		\2 Políticas de sostenibilidad de la deuda
			\3 Consolidación fiscal
				\4 Reducción del déficit aumentando cap. de financiación
				\4[] Reducción de gasto público
				\4[] Aumento de ingresos
				\4 Efecto negativo sobre la demanda agregada
				\4[] Puede reducir output y dificultar consolidación
				\4[] $\to$ Posible empeoramiento de la dinámica
				\4 Credibilidad de la política fiscal
				\4[] ``Efectos no keynesianos de la consolidación''
				\4[] Consolidación fuerte señaliza compromiso con deuda
				\4[] $\to$ Reduce incertidumbre de flujos futuros
				\4[] $\to$ Menor temor a impago
				\4[] $\then$ Aumento de precio de activos de deuda
				\4[] $\then$ Menor coste de financiación
				\4 Efecto sobre el crecimiento
				\4[] Consolidaciones fiscales que reducen inversión
				\4[] $\to$ Pueden afectar a stock de capital futuro
				\4[] $\then$ Efectos negativos sobre crecimiento a l/p
				\4 Reglas de consolidación fiscal
				\4[] Normas legales que obligan a ajuste fiscal
				\4[] $\to$ Basadas en variables objetivamente medibles
				\4[] Tratan de eliminar sesgo deficitario
				\4[] $\to$ Forzar sostenibilidad automática de deuda pública
				\4[] $\then$ Reducir incertidumbre para acreedores
			\3 Default, quita o reestructuración
				\4 Reducción de stock de deuda
				\4 País puede ejercer facultades soberanas
				\4[] Negarse a pagar deuda
				\4 Negociación con acreedores
				\4[] Refinanciar deuda/aplazamiento de vencimiento
				\4[] $\to$ No depende sólo del gobierno en cuestión
				\4[] $\then$ Acreedores deben aceptar
				\4 Restricciones de acceso al mercado
				\4[] Países soberanos pueden impagar deuda
				\4[] Mercados reaccionan cerrando acceso a financiación
				\4[] Habitualmente, periodo posterior sin acceso a financiación
				\4 Condicionalidad
				\4[] Acreedores pueden imponer condiciones para reestructurar
				\4[] Ejemplo: programas de BM+FMI
			\3 Reformas estructurales
				\4 Aumentar crecimiento del output
				\4 Mejorar aprovechamiento de capacidad productiva
				\4 Aumentar crecimiento de la productividad de los factores
				\4 Problemas
				\4[] $\to$ Múltiples diagnósticos sobre cómo lograr
				\4[] $\to$ Grupos sociales se ven afectados y se oponen
				\4[] $\to$ Necesario repartir costes del progreso
				\4 En la práctica, es la opción más importante
				\4[] Relativamente fácil de implementar políticamente
				\4[] Es objetivo general al margen de dinámica de la deuda
			\3 Enajenación del patrimonio del estado
				\4 Privatizaciones de empresas públicas
				\4[] En teoría, su valor refleja flujos de caja futuros
				\4[] $\to$ Debería tener efecto neutro sobre déficit
				\4[] Compradores pueden estimar posibilidad de crear valor
				\4[] $\to$ Ofrecen precio superior a flujos descontados
				\4[] $\then$ Gobierno extrae parte de creación de valor
				\4 Venta de patrimonio público
				\4[] Reservas de oro
				\4[] Patrimonio artístico o inmobiliario
				\4[] ...
				\4 Utilización limitada
				\4[] Patrimonio público no es ilimitado ni renovable
				\4[] Privatizaciones se consideran ingreso ``one-off''
				\4[] $\to$ Ej. No computan para PDE  de Comisión Europea
			\3 Represión financiera
				\4 Canalizar ahorro privado hacia SPúblico
				\4[] Aplicando capacidad de coacción del estado
				\4 Restricción de intereses ofrecidos por bancos
				\4[] Reduce atractivo y aumenta demanda de deuda pública
				\4 Nacionalización de fondos de pensiones privados
				\4[] Para invertirlos en deuda pública
				\4 Obligación de invertir en deuda pública
				\4[] Para fondos de pensiones, bancos, etc...
				\4 Limitación a la salida de inversión exterior
				\4[] Ahorro cautivo deberá invertirse en ahorro nacional
			\3 Monetización del déficit
				\4 Expansión del balance del Banco Central
				\4[] Aumento de base monetaria
				\4[] $\to$ Para comprar deuda pública
				\4 Analizada a continuación
%			\2 Economía política
%				\3 Idea clave
%				\3 Disposición a pagar/willingness-to-pay
%				\3 Ciclo político
%				\3 Asimetrías de información
%			\2 Crisis de deuda
	\1 \marcar{Aspectos monetarios de la financiación}
		\2 Idea clave
			\3 Contexto
				\4 Economías modernas basadas en dinero fiduciario
				\4 Estado tiene monopolio de creación de base monetaria
				\4 Posible financiar déficit creando base monetaria
				\4 Crecimiento de la base monetaria afecta precios
			\3 Objetivo
				\4 Representar interrelación entre oferta monetaria y déficit
				\4 Cuantificar relación déficit--inflación
				\4 Valorar oportunidad de financiación monetaria
			\3 Resultados
				\4 Límites y efectos de financiación monetaria
				\4 Inflación como impuesto sobre dinero
				\4 Modelos de inflación basados en déficit
		\2 Señoreaje e impuesto inflacionario
			\3 Idea clave
				\4 Contexto
				\4[] Definición original de señoreaje:
				\4[] $\to$ Valor facial de moneda -- coste de acuñación
				\4[] Ejemplo: particular acude a ceca con 100g de oro
				\4[] $\to$ Recibe monedas con valor facial 95g de oro
				\4[] $\then$ Señoreaje: 5g de oro para Estado
				\4[] Actualidad
				\4[] $\to$ Coste marginal casi nulo de $\uparrow$ base monetaria
				\4[] $\to$ Valor facial a voluntad del BCentral emisor
				\4[] $\then$ Diferencia para el estado
				\4[] Valor real de señoreaje
				\4[] $\to$ Depende de nivel de precios
				\4[] Evolución de nivel de precios
				\4[] $\to$ Depende de base monetaria
				\4 Objetivo
				\4[] Representar efecto de fin. monetaria sobre inflación
				\4[] Diferenciar regímenes de tipo de cambio
				\4[] Cuantificar límites de financiación monetaria
				\4[] Relacionar déficits e hiperinflaciones
				\4 Resultados
				\4[] Existen límites a financiación monetaria
				\4[] Financiación monetaria puede inducir hiperinflación
			\3 Formulación
				\4 Asimilable a economía cerrada
				\4[] Reservas de divisas irrelevantes o inexistentes
				\4[] Sin pérdida de generalidad
				\4 Mercado de dinero
				\4[] $M_t = P_t L(i,Y) = P_t \cdot e^{-b \cdot i} \cdot Y_t$
				\4 Ecuación de Fisher
				\4[] $i_t = r_t + \pi^e$
				\4 Inflación en estado estacionario
				\4[] Dado equilibrio en mercado de dinero
				\4[] $\to$ Crecimiento de oferta de dinero iguala inflación
				\4[] $\then$ $\theta = \frac{M_t - M_{t-1}}{M_{t-1}} = \frac{P_t - P_{t-1}}{P_{t-1}} = \pi$
				\4 Restricción presupuestaria del gobierno
				\4[] $P_t \text{DEF}_t = D_t - (1+i_t)D_{t-1} + M_t - M_{t-1}$
				\4[] (incluyendo financiación vía deuda)
				\4 Financiación puramente monetaria del déficit
				\4[] $D_t - D_{t-1} = 0$, $i_t D_{t-1}=0$ s.p.g.
				\4[] $\text{DEF}_t = M_t - M_{t-1}$
				\4[] $\then$ \fbox{$ \frac{\text{DEF}_t}{P_t} = \frac{M_t - M_{t-1}}{P_t} = \frac{M_t - M_{t-1}}{M_{t-1}} \cdot \frac{M_{t-1}}{M_t} \cdot \frac{M_t}{P_t} = \frac{\theta}{1+\theta}\cdot \frac{M_t}{P_t}$}
				\4[] $\then$ Más $DEF_t$ dado $M_t$ y $P_t$ $\then$ Más $\theta$ $\then$ Más $\pi$
			\3 Implicaciones
				\4 Inflación como impuesto
				\4[] Demanda de dinero es base imponible
				\4[] Inflación reduce valor de saldos reales
				\4[] $\to$ Asimilable a impuesto sobre tenencia de saldos
				\4 Curva de Laffer de la financiación monetaria
				\4[] Relación entre inflación e ingresos por señoreaje
				\4[] $\to$ Crece inicialmente
				\4[] $\to$ Decrece a partir de máximo
				\4[] $\then$ Posible inflación alta y baja para déficit dado
				\4[] Representación gráfica
				\4[] \grafica{lafferinflacion}
				\4 Límite a financiación monetaria del déficit
				\4[] Si agentes estiman correctamente la inflación
				\4[] $\to$ Existe ingreso máximo de estado estacionario
				\4 Hiperinflación posible sin déficit explosivo
				\4[] Cagan (1956)
				\4[] No es necesario crecimiento explosivo de la deuda
				\4[] $\to$ Para aumento explosivo de la inflación
				\4[] Con expectativas adaptativas sobre inflación
				\4[] $\to$ Expectativa de inflación basada en inflación pasada
				\4[] $\to$ Estimación miope de inflación
				\4[] $\then$ Mayor demanda de dinero por baja inflación pasada
				\4[] $\then$ Gobierno puede superar ingreso con inflación creciente
				\4[] Gobierno tiene incentivos a aprovechar esa dinámica
				\4[] $\to$ Expectativas se van ajustando
				\4[] $\then$ Cada vez mayor coste de financiación monetaria
				\4[] $\then$ Posible dinámica hiperinflacionaria
				\4[] $\then$ Aunque déficit real no aumente explosivamente
			\3 Valoración
				\4 Caracterización simple de problemas de fin. monetaria
				\4[] Posible sesgo inflacionario
				\4[] Posibles dinámicas hiperinflacionarias
				\4[] Expectativas complejas de modelizar
				\4 Soporte a independencia de bancos centrales
				\4[] Evitar recurso a financiación monetaria
				\4[] $\to$ Reducir inflación y efectos asociados
				\4 Financiación ortodoxa
				\4[] Limitada a impuestos y deuda
				\4[] Tendencia creciente a evitar fin. monetaria
				\4[] $\to$ Especialmente en desarrollados
				\4 Uso habitual en PEDs
				\4[] Estructuras administrativas insuficientes
				\4[] $\to$ Insuficiente recaudación de impuestos
				\4[] Mercados de capital poco desarrollados
				\4[] $\to$ Difícil financiación vía deuda
		\2 Hiperinflación
			\3 Idea clave
				\4[] Modelo de Cagan (1956)\footnote{Extraído de Edmond, C.}
				\4 Contexto
				\4[] Análisis dinámico
				\4[] Hiperinflaciones pre y post-II GM
				\4[] Expectativas adaptativas
				\4[] Monetarismo
				\4 Objetivo
				\4[] Explicar hiperinflaciones
				\4[] Relacionar con aumento de oferta monetaria
				\4[] Relacionar con impuesto inflacionario
				\4 Resultados
				\4[] Hiperinflaciones posibles sin gasto explosivo
				\4[] A partir de cierto $\Delta M_t$, HEA causa inflación
			\3 Formulación
				\4[] Partiendo de $m_t - p_t = -\alpha i_t + y_t$
				\4[(1)] $m_t - p_t = -\alpha \pi^e_t$
				\4[(2)] $\pi_t^e = \lambda \pi_{t-1}^e + (1-\lambda) (p_t - p_{t-1})$
				\4[$\to$] $\alpha$: sensibilidad de la dda. de dinero al i.nominal
				\4[$\to$] $\lambda$: inercia de la estimación de la inflación
				\4 Solución: ecuación de precio en t a partir de:
				\4[(i)] Política monetaria del gobierno ($m_t$)
				\4[(ii)] Precios pasados
				\4[(iii)] Parámetros $\alpha$ y $\lambda$
				\4[] $\then$ $p_t = \beta_1 p_{t-1} + \beta_2 m_t - \beta_3 m_{t-1}$
				\4 Hiperinflación
				\4[] Determinados parámetros inducen senda divergente
				\4[] Crecimiento explosivo de $p_t$
			\3 Implicaciones
				\4 Estimaciones de inflación basadas en inflación pasada
				\4[] $\to$ Pueden mantener dinámica inflacionaria en movimiento
				\4 Gobierno
				\4[] $\to$ No es necesario crecimiento explosivo de la deuda
				\4[] $\then$ Para aumento explosivo de la inflación
				\4[] $\to$ Dada expectativa de inflación elevada
				\4[] $\then$ Necesitan más inflación para recaudar
				\4 Con expectativas adaptativas sobre inflación
				\4[] $\to$ Expectativa de inflación basada en inflación pasada
				\4[] $\to$ Estimación miope de inflación
				\4[] $\then$ Mayor demanda de dinero
				\4[] $\then$ Gobierno puede superar ingreso con inflación creciente
				\4[] $\then$ Posible dinámica hiperinflacionaria
				\4 Recomendaciones de política económica:
				\4[] $\to$ Evitar monetización
				\4[] $\to$ Anclar expectativas de inflación
			\3 Valoración
				\4 Adaptación posterior a HER en Sargent y Wallace (1972)
				\4 Mejora sustantiva de comprensión de hiperinflación
				\4[] $\to$ No sólo déficits+monetización son culpables
				\4[] $\to$ Dinámicas endógenas pueden provocar
		\2 Financiación monetaria en economía abierta
			\3 Idea clave
				\4 Contexto
				\4[] Tipo de cambio fijo
				\4[] Economía abierta al comercio
				\4 Objetivo
				\4[] Caracterizar efectos de financiación monetaria
				\4[] Valorar sostenibilidad de financiación monetaria
				\4 Resultados
				\4[] Financiación monetaria reduce reservas
				\4[] Tipo de cambio fijo insostenible con fin. monetaria
			\3 Formulación
				\4 Paridad de poder adquisitivo
				\4[] Asumimos PPA absoluta s.p.g.
				\4[] $s_t P^*_t = P_t$
				\4 Mercado de dinero
				\4[] $M_t = P_t e^{-bi} Y_t = s_t P^*_t e^{-bi} Y_t$
				\4[] $\to$ $\Delta M_t$ $\then$ $\Delta s_t$
				\4 Tipo de cambio fijo
				\4[] Gobierno comprometido a mantener $s_t$ constante
				\4[] Exceso de oferta de dinero
				\4[] $\to$ Presión sobre tipo de cambio
				\4[] $\then$ Obliga a vender reservas en mercado cambiario
				\4[] $\then$ Límite a venta de reservas
				\4[] $\then$ EOferta en FX aumenta si agentes esperan devaluación
				\4 Restricción presupuestaria del gobierno
				\4[] $P_t \text{DEF}_t  - s_t \cdot (i_t^* \cdot R_t) =  M_t - M_{t-1} - s_t \cdot (R_t - R_{t-1}) $
				\4[] $\to$ $R_t$: reservas de divisas
				\4[] $\to$ $s_t$: tipo de cambio directo
			\3 Implicaciones
				\4 Financiación monetaria hace insostenible TCN fijo
				\4[] Si gobierno monetiza deuda
				\4[] $\to$ Nivel de precios subirá ceteris paribus
				\4[] $\then$ Presión hacia depreciación por PPA
				\4[] Gobierno debe vender divisas para mantener TCFijo
				\4[] Reservas de divisas son finitas
				\4[] $\to$ Devaluación tarde o temprano
				\4[] $\then$ Ataques especulativos
				\4 Bienes no comerciables
				\4[] Nivel de precios nacional media ponderada
				\4[] $\to$ Entre precio de comerciables y no comerciables
				\4[] $\then$ $P_t = (s_t P^*_t)^\beta \cdot (P_t^N)^{1-\beta}$
				\4[] PPA se cumple para comerciables
				\4[] Eq. en mercado de dinero nacional
				\4[] $\to$ Depende de nivel de precios nacional
				\4[] $M_t = (s_t P^*_t)^\beta \cdot (P_t^N)^{1-\beta} e^{-bi}\cdot Y_t$
				\4[] Con muy poco peso de comerciables
				\4[] $\to$ Monetización apenas afecta $s_t$
				\4[] $\then$ Posible utiliza reservas durante más tiempo
				\4[] Gasto público sesgado hacia bienes no comerciables
				\4[] $\to$ Más gasto aumenta peso de no comerciables
			\3 Valoración
		\2 Aritmética monetarista desagradable
			\3 Idea clave
				\4 Contexto
				\4[] Kydland y Prescot (1977)
				\4[] $\to$ Incosistencia temporal de políticas económicas
				\4[] $\then$ Sin commitment, inflación aumenta
				\4[] Dominancia fiscal
				\4[] $\to$ Habitual si BCentral no es independiente
				\4[] $\to$ Situación habitual en 70s y 80s
				\4[] Sargent y Wallace (1981)
				\4[] ``Some unpleasant monetarist arithmetic''
				\4 Objetivos
				\4[] ¿Cómo interaccionan política fiscal y monetaria?
				\4[] ¿Bajo qué supuestos PF determina inflación?
				\4[] ¿Cuando PM puede controlar la inflación?
				\4[] $\then$ Dadas ciertas condiciones, PM depende de PF
				\4[] $\then$ Inflación depende de PF, no de PM
				\4[] Valorar efecto de PF sobre inflación
				\4[] $\to$ Dadas sendas de financiación alternativas
				\4[] $\then$ i) Deuda ahora y financiación monetaria después
				\4[] $\then$ ii) Financiación monetaria ahora y después
				\4 Resultados
				\4[] Financiación monetaria después
				\4[] $\to$ Puede inducir mayor inflación total
				\4[] Agentes estiman inflación posterior
				\4[] $\to$ Reaccionan ahora
				\4[] $\then$ Mayor inflación futura necesaria
				\4[] PM contractiva en presente y dominancia fiscal
				\4[] $\to$ Puede provocar inflación más alta mañana
			\3 Formulación
				\4 Cuatro fuentes de financiación del déficit\footnote{Primario y general.}
				\4[] $\to$ Más impuestos ($\uparrow T$)
				\4[] $\to$ Menor gasto ($\downarrow G$)
				\4[] $\to$ Emisión de deuda ($\uparrow B$)
				\4[] $\to$ Señoreaje ($\uparrow M$)
				\4 Senda de déficit fiscal exógena y arbitraria
				\4[] Autoridad fiscal determina exógenamente
				\4[] $\to$ Impuestos
				\4[] $\to$ Gasto
				\4[] $\then$ Déficit exógeno
				\4 Supuesto de no-default
				\4[] Cumplimiento de RPG del gobierno
				\4 Deuda crece más que economía
				\4[] $r > g$
				\4 Ecuación cuantitativa del dinero
				\4[] Se cumple en horizonte temporal relevante
				\4 Emisión de deuda tiene límites
				\4[] $\to$ Deuda se vende a menor precio o no se vende
				\4[] Si el gasto aumenta:
				\4[] $\to$ Señoreaje debe cubrir lo que $\uparrow T$, $\downarrow G$, $\uparrow B$ no cubre
				\4 Senda de precios de equilibrio no es única
				\4[] Pueden existir múltiples sendas de precios de equilibrio
				\4[] $\to$ Más inflación al principio, menos después
				\4[] $\to$ PM restrictiva al principio $\to$ expansiva después
				\4 Expectativas racionales
				\4[] Agentes conocen incentivos del gobierno y finanzas públicas
			\3 Implicaciones
				\4[] Germen de ``teoría fiscal del nivel de precios''
				\4[] Si política fiscal domina
				\4[] $\to$ PM debe adaptarse tarde o temprano
				\4[$\then$] Si B. Central no es independiente
				\4[]$\to$ inflación antes o después
		\2 Teoría fiscal del nivel de precios
			\3 Idea clave
				\4 Contexto
				\4 Objetivo
				\4 Resultados
				\4[] Inflación es fenómeno fiscal
			\3 Formulación
				\4 Dadas:
				\4[] Senda exógena de superávits primarios
				\4[] Stock de deuda inicial
				\4 Restricción presupuestaria
				\4[] $\frac{B_t}{P_t} = \text{Valor presente de superávits}$
				\4 Senda de precios se ajusta para cumplir RP
				\4[] Si superávits no son suficientes
				\4[] $\to$ Inflación para reducir valor real de deuda
			\3 Implicaciones
				\4 Dinero sí influye en inflación
				\4[] PM expansiva aumenta interés nominal
				\4[] Interés nominal aumenta stock nominal de deuda
				\4[] $\to$ Efecto sobre precios
			\3 Valoración
				\4 Críticas
				\4[] Evidencia empírica desfavorable
				\4[] $\to$ Déficit inesperados no parece provocar $\uparrow$ $\pi_t$
				\4[] Existen otras formas de determinar senda $M_t$ y $P_t$
				\4[] $\to$ P.ej.: regla de Taylor sobre interés
	\1[] \marcar{Conclusión}
		\2 Recapitulación
			\3 Financiación del déficit público
			\3 Sostenibilidad del déficit y la deuda
			\3 Aspectos monetarios de la financiación
		\2 Idea final
			\3 Evolución histórica de los déficits
			\3 Unión Europea
			\3 España
			\3 Países en desarrollo
\end{esquemal}



























\graficas

\begin{axis}{4}{Representación gráfica de una dinámica inestable de deuda en la que el tipo de interés real es mayor a la tasa de crecimiento nominal de la economía.}{$b_{t-1}$}{$b_t$}{rgtg}
	% Extensión de ejes
	\draw[-] (0,0) -- (-2,0); % abscisas
	\draw[-] (0,0) -- (0,-2); % ordenadas
	
	% Bisectriz
	\draw[dotted] (0,0) -- (4,4);
	
	% Dinámica divergente
	\draw[-] (0,0) -- (2,4);
	\node[above] at (2,4){\tiny $r > g$};
	
	% Evolución desde stock de deuda inicial no nulo
	\node[below] at (1.5,0){\tiny $b_0$};
	\draw[dashed] (1.5,0) -- (1.5,3);
	\draw[dashed,-{Latex}] (1.5,3) -- (3,3) -- (3,4.5);
	
	% Senda con superávit para restablecer equilibrio
	\draw[-] (0,-1.5) -- (2.75,4);
	
	% Equilibrio con superávit
	\node[circle,fill=black,inner sep=0pt,minimum size=4pt] (a) at (1.5,1.5) {};
	
	% Superávit mínimo
	\draw[decorate,decoration={brace, mirror,amplitude=3pt},xshift=-5pt,yshift=0cm] (0,0) -- (0,-1.5) node[black,midway,xshift=-25pt, yshift=0cm] {\tiny Superávit minímo};
	
\end{axis}

El gráfico muestra una dinámica de deuda resultado de un tipo de interés real superior a la tasa de crecimiento nominal de la economía. Cuando la economía tiene un stock de deuda inicial de $b_0$, la dinámica de la deuda es inestable y tiende a crecer explosivamente como muestra la línea discontinua. Para evitar este crecimiento explosivo sería necesario incurrir en el superávit primario mínimo representado en el eje de ordenadas.


\begin{axis}{4}{Representación gráfica de una dinámica estable de deuda en la que el tipo de interés real es menor a la tasa de crecimiento nominal de la economía.}{$b_{t-1}$}{$b_t$}{rltg}
	% Extensión de ejes
	\draw[-] (0,0) -- (-2,0); % abscisas
	\draw[-] (0,0) -- (0,-2); % ordenadas
	
	% Bisectriz
	\draw[dotted] (0,0) -- (4,4);
	
	% Dinámica convergente
	\draw[-] (0,1.5) -- (4,3);
	\node[right] at (4,3){\tiny $r<g$};
	
	% Evolución desde stock de deuda inicial no nulo
	\node[below] at (1,0){\tiny $b_0$};
	\draw[dashed] (1,0) -- (1,1.89);
	\draw[dashed,-{Latex}] (1,1.89) -- (1.89,1.89) -- (1.89,2.21) -- (2.21,2.21);
	
	% Déficit máximo
	\draw[decorate,decoration={brace, mirror,amplitude=3pt},xshift=-5pt,yshift=0cm] (0,1.5) -- (0,0) node[black,midway,xshift=-25pt, yshift=0cm] {\tiny Déficit primario};
\end{axis}

\begin{axis}{4}{Representación gráfica de la curva de Laffer de la financiación monetaria del déficit.}{}{}{lafferinflacion}
	% Extensión de abscisas
	\draw[-] (4,0) -- (6,0);
	\node[below] at (6,0){$\theta=\pi$};
	
	% Extensión de ordenadas
	\draw[-] (0,4) --(0,4.5);
	\node[left] at (0,4.5){Ingreso por señoreaje};
	
	% Ingresos por señoreaje
	\draw[-] (0,0) to [out=60, in=180](2.5,4) to [out=0, in=120](5.5,0);
	
	% Inflación que maximiza ingresos
	\draw[dotted] (0,4) -- (2.5,4) -- (2.5,0);
	\node[left] at (0,4){\tiny Ingreso máximo};
	\node[below] at (2.5,0){\tiny $\theta^*$};
	
	% Déficit a financiar
	\draw[-] (0,2.5) -- (6,2.5);
	\node[left] at (0,2.5){\tiny Déficit};
	
	% Inflación baja
	\draw[dashed] (0.88,0) -- (0.88,2.5);
	\node[below] at (0.88,0){\tiny $\pi_\text{baja}$};
	
	\draw[dashed] (4.43,0) -- (4.43,2.5);
	\node[below] at (4.43,0){\tiny $\pi_\text{alta}$};
\end{axis}

\conceptos

\concepto{Convergencia y divergencia de la riqueza neta con saldos primarios fijos}

El análisis de convergencia de series de riqueza neta con saldos primarios fijos puede aplicarse a la sostenibilidad de la posición neta de inversión internacional, a la sostenibilidad de la deuda externa o a la sostenibilidad de la deuda pública o privada. El objetivo general es caracterizar la evolución de la riqueza neta en un horizonte temporal infinito, cuando la economía crece a una tasa fija y exógena $g$, la riqueza neta crece a una tasa fija y exógena $r$ y el saldo primario es fijo y está exógenamente determinado. 

Denotamos la riqueza neta en $t$ como $B$ y el saldo primario en $t$ como $S_t$. La ecuación de la dinámica de la riqueza neta en términos nominales será:

\begin{equation*}
	B_t = \text{S}_t + (1+i)B_t
\end{equation*}

La riqueza neta se representa habitualmente en términos relativos al producto nominal. Para ello, dividimos y multiplicamos por $Y_t$, que denota el producto nominal en $t$. Asumiendo una inflación de $\pi$ y una tasa de crecimiento del producto real de $g$, la relación entre el producto en $t$ y en $t+1$ será tal que $Y_t = Y_{t+1} \cdot (1+\pi) (1+g)$. Tenemos así que:

\begin{equation*}
	\frac{B_t}{Y_t} =  \frac{S_t}{Y_t} + \frac{(1+i)}{(1+\pi) (1+g)} \cdot \frac{B_{t-1}}{Y_{t-1}} 
\end{equation*}

Representando en minúsculas las variables en relación al producto nominal, tenemos:

\begin{equation*}
	b_t = s_t + \frac{1+r}{1+g} b_{t-1}
\end{equation*}

Resulta útil caracterizar el saldo primario para el que la riqueza neta se mantendrá constante. Para ello, igualamos la riqueza neta presente y pasada tal que $b_t = b_{t-1}$ y despejamos el saldo primario que induce tal condición:

\begin{equation*}
	b = s + \frac{1+r}{1+g} b 
\end{equation*}

\begin{equation*}
	s = \frac{g-r}{1+g} b
\end{equation*}

De las anteriores ecuaciones pueden derivarse dos resultados generales en relación a la convergencia y la divergencia de la riqueza neta:

\begin{itemize}
	\item Si $g > r$, la riqueza neta puede converger a un valor finito o se mantiene estable:
	\begin{itemize}
		\item Si $b_{t-1} > 0$:
		\begin{itemize}
			\item Si $s>0$:
			\begin{itemize}
				\item Si $s > \frac{g-r}{1+g}b_{t-1}$, la riqueza neta converge a valor positivo mayor a $b_{t-1}$.
				\item Si $s = \frac{g-r}{1+g}b_{t-1}$, la riqueza neta se mantiene estable.
				\item Si $s < \frac{g-r}{1+g}b_{t-1}$, la riqueza neta converge a un valor positivo menor a $b_{t-1}$.
			\end{itemize}
			\item Si $s<0$:
			\begin{itemize}
				\item Si $s < \frac{g-r}{1+g}b_{t-1}$, la riqueza neta converge a valor negativo menor a $-b_{t-1}$.
				\item Si $s = - \frac{g-r}{1+g}b_{t-1}$, la riqueza neta converge a $-b_{t-1}$
				\item Si $s < \frac{g-r}{1+g}b_{t-1}$, la riqueza neta converge a un valor negativo mayor a $b_{t-1}$.
			\end{itemize}
		\end{itemize}
		
		\item Si $b_{t-1} < 0$:
		\begin{itemize}
			\item Si $s > 0$:
			\begin{itemize}
				\item Si 
				\item
				\item
			\end{itemize}
			\item Si $s < 0$:
			\begin{itemize}
				\item 
				\item
				\item
			\end{itemize}
		\end{itemize}
	
	\end{itemize}
	
	\item Si $g<r$, la riqueza neta puede divergir o mantenerse estable:
	\begin{itemize}
	\item Si $s = \frac{g-r}{1+g} b$, la riqueza neta se mantiene estable.
	\item Si $s < \frac{g-r}{1+g} b$, la riqueza neta diverge a $-\infty$.
	\item Si $s > \frac{g-r}{1+g} b$, la riqueza neta diverge a $+\infty$. 
	\end{itemize}
\end{itemize}

\preguntas

\seccion{Test 2018}

\textbf{19.} ¿Cuál de los siguientes \textbf{\underline{NO}} es uno de los supuestos de la proposición de Equivalencia Ricardiana, o proposición de equivalencia Barro--Ricardo?

\begin{itemize}
	\item[a] Existe altruismo intergeneracional, de modo que los agentes se preocupan por no dejar una elevada carga fiscal a sus herederos.
	\item[b] Los mercados de capitales son perfectos.
	\item[c] Existe perfecta previsión sobre la evolución futura de los impuestos.
	\item[d] El tipo de interés al que puede endeudarse el sector público es inferior al tipo de interés al que puede endeudarse el sector privado.
\end{itemize}

\bigskip
\textbf{20.} Suponga una economía con superávit primario, un tipo de interés nominal del 5\%, una inflación del 3\% y una tasa de crecimiento real del $2,5\%$ en cada periodo. 

\begin{itemize}
	\item[a] La deuda nunca se amortizará.
	\item[b] La deuda se amortizará después de un número finito de periodos.
	\item[c] La deuda seguirá una senda explosiva.
	\item[d] No es posible saber cómo evolucionará la deuda.
\end{itemize}

\seccion{Test 2017}

\textbf{21.} ¿Cuál de las siguientes afirmaciones es correcta?

\begin{itemize}
	\item[a] Si el déficit primario es constante, la ratio deuda pública-PIB converge a un valor constante cuando la tasa de crecimiento económico real es mayor que el tipo de interés real.
	\item[b] Si el saldo primario es nulo, la ratio deuda pública-PIB converge siempre a un valor constante.
	\item[c] La ratio deuda pública-PIB nunca converge a un valor constante.
	\item[d] Los gobiernos siempre pueden financiar sus gastos tanto con impuestos como con déficits, puesto que la deuda pública siempre es sostenible.
\end{itemize}

\bigskip

\textbf{22.} Para que se cumpla la \textit{equivalencia ricardiana}, es necesario que:

\begin{itemize}
	\item[a] Los agentes económicos tomen sus decisiones en función de sus expectativas futuras.
	\item[b] Los agentes económicos planifiquen sus decisiones de consumo e inversión a lo largo de un horizonte infinito.
	\item[c] No existan restricciones de liquidez y todos los agentes y el gobierno puedan prestar y endeudarse al mismo tipo de interés.
	\item[d] Todas las anteriores.
\end{itemize}

\seccion{Test 2014}

\textbf{24.} Señale la afirmación correcta:
\begin{itemize}
	\item[a] Para disminuir la relación de deuda pública sobre el PIB es necesario que la economía registre superávit público primario.
	\item[b] La financiación del déficit público mediante emisión de deuda pública es siempre menos inflacionista que la financiación mediante emisión de dinero.
	\item[c] La relación de deuda pública sobre el PIB será más sostenible cuanto menores sean los tipos de interés reales y los déficit públicos primarios y mayor el crecimiento económico.
	\item[d] La acuñacion de moneda reporta un mayor señoriaje que la impresión de papel moneda (billetes) por el mayor valor del material de fabricación de las monedas.
\end{itemize}

\seccion{Test 2013}
\textbf{20.} El problema de la inconsistencia intertemporal de las políticas económicas tiene su explicación en:

\begin{itemize}
	\item[a] Los cambios en las preferencias de las autoridades económicas.
	\item[b] Los cambios en las preferencias del público.
	\item[c] La reacción del público ante el anuncio mismo de la política futura.
	\item[d] Cualquiera de las respuestas anteriores es correcta.
\end{itemize}

\seccion{Test 2009}
\textbf{21.} Considere la ecuación para la dinámica de deuda de un gobierno que financia déficits primarios emitiendo bonos nominales a un período, y expresa dicha ecuación en ratios sobre PIB. Si el tipo de interés nominal es el 2.5\%, la inflación es el -1.5\% y el crecimiento real es el -2.5\%, y el ratio deuda sobre PIB actual es finito y positivo, si no cambiara nada. 

\begin{itemize}
	\item[a] El nivel de endeudamiento se amortizará con el paso del tiempo.
	\item[b] El nivel de endeudamiento convergerá a su ratio deuda-PIB de estado estacionario.
	\item[c] En esta situación no hay forma en que podamos corregir el crecimiento de la deuda.
	\item[d] Una política fiscal suficientemente contractiva podría permitirnos mantener los niveles actuales de endeudamiento.
\end{itemize}

\seccion{Test 2006}
\textbf{19.} Suponga que la evolución del stock de deuda en términos del PIB de una economía dada sigue la siguiente ecuación dinámica:

\begin{equation*}
b_t = \bar{d} + \frac{1+R}{1+n} b_{t-1}
\end{equation*}

donde $\bar{d}$ es el déficit primario en términos del PIB, $R$ es el tipo de interés real, y $n$ es el crecimiento real del PIB. Suponga una economía con un déficit primario positivo ($\bar{d} > 0$), un tipo de interés nominal del 5\%, una inflación del 3\% y un crecimiento de la economía del 2.5\%. La autoridad fiscal no puede apelar al Banco Centra para financiar su déficit. Suponga además que su stock de deuda actual es positivo y menor que la deuda de estado estacionario. En esta situación, si no cambiara nada:

\begin{itemize}
	\item[a] El nivel de endeudamiento del siguiente periodo será mayor.
	\item[b] El nivel de endeudamiento permanecerá invariable.
	\item[c] El nivel de endeudamiento del siguiente periodo será menor.
	\item[d] El nivel de endeudamiento se amortizará con el paso del tiempo.
\end{itemize}

\seccion{Test 2005}

\textbf{20.} Suponga que la evolución del stock de deuda en términos de PIB de una economía dada sigue la siguiente ecuación dinámica:

\begin{equation*}
b_t = \bar{d} + \frac{1+R}{1+n} b_{t-1}
\end{equation*}

donde $\bar{d}$ es el déficit primario en términos de PIB, $R$ es el tipo de interés real y $n$ es el crecimiento real del PIB. Suponga un déficit primario igual al 0,2\% del PIB ($\bar{d}=0.002$), un tipo de interés nominal del 4\%, una inflación del 2\% y un crecimiento real de la economía del 2,4\%. Suponga además que su stock de deuda actual es del 30\% sobre el PIB. En esta situación diga qué respuesta es \textbf{FALSA}, dada la evolución de la deuda por la ecuación dinámica descrita:

\begin{itemize}
	\item[a] Podría amortizarse la deuda si se pusiera en marcha una política fiscal contractiva qu eliminara el déficit.
	\item[b] Si no cambia nada, se amortizará la deuda después de un número finito de periodos.
	\item[c] Si no cambia nada, nunca se amortizará la deuda. 
	\item[d] Si no cambia nada, el nivel de endeudamiento de largo plazo será del 46,6\% del PIB.
\end{itemize}

\seccion{15 de marzo de 2017}

\begin{itemize}
    \item ¿Podría ampliar su explicación del concepto de señoreaje?
    
    \item Ha desperdiciado el tiempo haciendo un recorrido histórico demasiado extenso. Además, no ha escrito la restricción presupuestaria del gobierno que resulta verdaderamente útil para cualquier análisis: la restricción presupuestaria intertemporal del gobierno. ¿Podría escribirla? (el catedrático)
\end{itemize}

\notas

\textbf{2018:} \textbf{19.} D \textbf{20.} B

\textbf{2017:} \textbf{21.} A \textbf{22.} D

\textbf{2014:} \textbf{24.} C

\textbf{2013:} \textbf{20.} C

\textbf{2009:} \textbf{21.} D

\textbf{2006:} \textbf{19.} A

\textbf{2005:} \textbf{20.} B

\bibliografia

Mirar en Palgrave:

\begin{itemize}
	\item budgetary policy
	\item built-in stabilizaers
	\item burden of the debt
	\item fine tuning
	\item fiscal and monetary policies in development countries
	\item fiscal multipliers
	\item fiscal stance
	\item fiscal theory of the price level
	\item forced saving
	\item full employment budget surplus
	\item functional finance
	\item government budget constraint
	\item government budget restraint
	\item growth and cycles
	\item monetary and fiscal policy overview
	\item neo-ricardian economics
	\item new keynesian macroeconomics
	\item optimal fiscal and monetary policy (with commitment)
	\item optimal fiscal and monetary policy (without commitment)
	\item optimal savings
	\item over-saving
	\item public debt
	\item public works
	\item Ricardian equivalence theorem
	\item saving equals investment
	\item stabilization policy
	\item targets and instruments
	\item time consistency of monetary and fiscal policy
	\item vector autoregressions
\end{itemize}



Barro, R. (1974) \textit{Are government bonds net wealth?} Journal of Political Economy 
-- En carpeta del tema

Barro, R. (1989) \textit{The Ricardian Approach to Budget Deficits} Journal of Economic Perspectives. Spring 1989 -- En carpeta del tema

Bernheim, B. (1989) \textit{A Neoclassical Perspective on Budget Deficits} Journal of Political Economy. Spring 1989 -- En carpeta del tema

Blinder, A. S.; Solow, R. M. (1973) \textit{Does Fiscal Policy Matter} Journal of Public Economics -- En carpeta del tema

Christiano, L. J., Fitzgerald, T. J. \textit{Understanding the Fiscal Theory of the Price Level} Economic Review of the Federal Reserve Bank of Cleveland (2000) \url{http://faculty.econ.ucdavis.edu/faculty/kdsalyer/LECTURES/Ecn235a/Extra\_presentation\_papers/fiscal\_theory.pdf}

Cuerpo, C.; Ramos, J. M. (2014) \textit{Spanish Public Debt Sustainability Analysis} Hacienda Pública Española -- En carpeta del tema

Eisner, R. (1989) \textit{Budget Deficits: Rhetoric and Reality} Journal of Economic Perspectives. Spring 1989 -- En carpeta del tema

Gootzeit, M. J. (1987) \textit{Adam Smith of Balanced Budget Government Spending} History of Economics Society Bulletin -- En carpeta del tema

Gramlich, E. M. (1989) \textit{Budget Deficits and National Saving: Are Politicians Exogenous?} Journal of Economic Perspectives. Spring 1989 -- En carpeta del tema

Heijdra, B. J. \textit{Foundations of Modern Macroeconomics} (2017) 3rd ed. -- En carpeta Macro

Mauro, P.; Zhou, J. (2020) \textit{r minus g negative: Can We Sleep More Soundly?} IMF Working Papers -- En carpeta del tema

Seater, J. J. (1993) \textit{Ricardian Equivalence} Journal of Economic Literature. March 1993 -- En carpeta del tema

Sims, E. (2015) \textit{Graduate Macro Theory II: Fiscal Policy in the RBC Model} \url{https://www3.nd.edu/~esims1/fiscal_policy_sp2015.pdf} -- En carpeta del tema

Yellen, J. L. (1989) \textit{Symposium on the Budget Deficit} Journal of Economic Perspectives. Spring 1989 -- En carpeta del tema


\end{document}
