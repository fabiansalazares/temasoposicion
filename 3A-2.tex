\documentclass{nuevotema}

\tema{3A-2}
\titulo{Los economistas clásicos y Marx}

\begin{document}

\ideaclave

La evolución del pensamiento económico es un proceso complejo que resulta de interacción de varios factores. Entre ellos, destacan el contexto económico del momento, el pensamiento económico precedente y los avances en otras disciplinas como la filosofía, las matemáticas, la física o la biología. El conocimiento de la historia del pensamiento económico permite entender las raíces intelectuales del pensamiento actual, aproximarse al análisis de los fenómenos económicos en el pasado y valorar la importancia de los diferentes programas de investigación. Además, muchos de los conceptos económicos de la actualidad no son sino versiones reformuladas y simplificadas de teorías descritas por primera vez hace siglos. 

Los economía clásica engloba la obra del primer grupo de economistas que recibieron tal nombre de forma explícita. El concepto fue acuñado por Marx para designar a un grupo de economistas herederos intelectuales de Smith, Ricardo y Malthus a los cuales les atribuía la creación de un sistema teórico cuyo objetivo es perpetuar la posición social de la burguesía. Con el tiempo, el término ``clásico'' se ha transformado y ha dado lugar a diferentes significados. Los seguidores de Adam Smith, los economistas que creen en la tendencia del ajuste natural y automático hacia el equilibrio, los seguidores de Smith y Ricardo, los defensores del laissez faire... En esta exposición adoptamos el denominador común de esas definiciones y entendemos como economistas clásicos los principales autores que aceptan la libre circulación de bienes y la libertad de empresa e intercambio como principios generales de actuación, desde Adam Smith a finales del XVIII hasta John Stuart Mill a mediados del XIX, e incluyendo a Marx como añadido de adscripción controvertida a la economía clásica. Así, el objeto de la exposición consiste en dar respuesta a preguntas tales cómo: ¿qué aportaron los economistas clásicos? ¿qué trayectoria intelectual siguieron? ¿quiénes fueron sus influencias? ¿a quiénes influenciaron? ¿qué ideas y programas de investigación articularon su obra? La estructura de la exposición sigue aproximadamente el orden cronológico de aparición de los principales autores: Smith, Malthus, Ricardo, Mill y Marx, con una breve enumeración de otros economistas clásicos relevantes entre estos dos últimos.

\marcar{Adam Smith} es frecuentemente considerado como el padre de la ciencia económica. Nacido en Escocia, su vida trascurrió entre Oxford, Glasgow y Francia, donde acompañó como tutor al hijo de un gran aristócrata inglés. Sus tres grandes influencias fueron Hutcheson, David Hume y los fisiócratas franceses. El primero fue su profesor universitario y determinó las líneas generales de su programa de investigación. Con Hume llevó a cabo un intenso intercambio de ideas por vía epistolar y tomó contacto con los fisiócratas en su viaje a Francia. Las tres obras fundamentales de Adam Smith fueron la \textit{Teoría de los Sentimientos Morales}, \textit{Investigación sobre la naturaleza y las causas de la Riqueza de las Naciones} y \textit{Lecciones sobre Jurisprudencia}. El contexto histórico influyó de forma importante en su obra, aunque con un cierto retraso: la revolución industrial aún incipiente y las primeras fábricas que comenzaban a aparecer por toda Gran Bretaña no lograron desviar su atención de la agricultura como base de la economía y la manufactura tradicional como objetos centrales de investigación. La Teoría de los Sentimientos Morales sienta las bases filosóficas de su pensamiento posterior y trata de examinar los determinantes del comportamiento humano. Aunque no se trata de una obra con contenido económico directo, introduce la semilla de muchos programas de investigación posteriores tales como el análisis de los determinantes de la utilidad, las externalidades positivas del comportamiento humano o la estabilidad de la sociedad como sistema dinámico de funcionamiento complejo. 

\textit{La Riqueza de las Naciones} (1776) es la gran obra de Adam Smith con contenido pura y explícitamente económico. Además de presentar determinados conceptos con envidiable claridad y notable acierto, su inconmensurable aportación a la economía es realmente su capacidad para recoger la práctica totalidad de los programas de investigación de los dos siglos posteriores. Se divide en cinco libros, que a grandes rasgos definen los límites del pensamiento de Smith. 

En el \underline{primero}, el tema central es la \underline{teoría del valor y la distribución de la renta}. Introduce la idea de la división del trabajo como motor del crecimiento y lo fundamenta en la mayor productividad que se deriva de la especialización. Examina también la cuestión de la teoría del valor y distingue entre valor de uso y valor de intercambio, así como la formación de precios. Rechaza la teoría del valor-trabajo de forma general adscribiéndola a sociedades primitivas. Entiende la competencia como un proceso dinámico que lleva los precios hacia sus precios naturales, que define como la suma de los precios de los factores. Recoge algunas teorías respecto a los salarios, y establece los determinantes de los salarios y deriva de ellos el hecho de que los salarios no tienden a igualarse en términos monetarios sino en términos del desagrado que produce el trabajo por unidad de remuneración --el famoso \textit{toil and trouble}. Realiza también un análisis de la distribución de la renta que sentaría las bases del programa ricardiano posterior. Además, introduce la idea del ingreso nacional medido en relación al trabajo o el \textit{toil and trouble} que ha requerido. Esta presentación destruye definitivamente el mercantilismo que ya venía siendo atacado desde mediados del siglo XVII: es el trabajo y no la cantidad de metal precioso lo que hace ricas a las naciones. 

El \underline{segundo libro} se dedica básicamente al problema de la \underline{acumulación del capital y su relación con el producto y la renta}. Analiza el capital y lo entienden como el conjunto de bienes intermedios necesarios para la producción final. Distingue entre trabajo productivo --aquel que añade valor y permite aumentar la producción- y trabajo improductivo --aquel que se consume con su uso-. Preconiza también la Ley de Say afirmando que el ahorro no destruye poder de compra, y que el propósito de toda producción es el consumo. 

El \underline{tercer libro} es un \underline{análisis comparativo de la evolución de diferentes economías}: Roma, Europa e Inglaterra, aplicando conceptos desarrollados en otras partes del libro e introduciendo el análisis de la historia a partir de conceptos económicos.

El \underline{cuarto libro} se dedica a cuestiones variadas de \underline{teoría económica y a su aplicación práctica como política económica}. Aquí se encuentra una de sus más conocidas aportaciones: la idea de la mano invisible como reconciliación afortunada entre eficiencia económica e interés privado. Señala también las limitaciones de la mano invisible como mecanismo de aproximación a la eficiencia. Dadas estas limitaciones, Smith establece que la provisión de la defensa, la justicia y las infraestructuras son deberes básicos del estado. Además, el poder público debe proveer educación y debe intervenir para favorecer la aparición de industrias nacientes, justificando el uso de restricciones al libre comercio en determinadas situaciones excepcionales.

En el \underline{quinto libro}, Smith propone una primitiva \underline{teoría del estado} y como parte central de ella, analiza los rasgos característicos de las \underline{políticas fiscales óptimas}. Señala los cuatro principios básicos de la tributación: carga en función de la renta, simplicidad y fácil comprensión, pago lo más cerca posible en el tiempo del momento en que se produce el hecho imponible, y costes de administración reducidos tanto como sea posible. Se muestra a favor de impuestos sobre la tierra pero no sobre el producto de la tierra. También rechaza impuestos sobre el capital y los salarios porque entienden que no sólo son injustos y desincentivarían la producción, sino que acabarían siendo repercutidos sobre sujetos distintos del sujeto tributario directo, en un primitivo análisis de incidencia.

\marcar{Robert Malthus }(1766--1834) es, junto con David Ricardo, el otro gran economista clásico que contribuye a estabilizar la ciencia económica de la primera etapa clásica. Sus influencias son Smith y el propio Ricardo, a través de un intenso intercambio epistolar a lo largo de sus vidas. Además, el efecto de su formación matemática se deja sentir en el conjunto de su pensamiento. Malthus influyó multitud de conceptos posteriores, entre los cuales destacan el concepto de demanda efectiva y la búsqueda de valores óptimos. Marx se sirvió de Malthus para su crítica a la economía clásica en tanto que le acusaba de defensor de intereses de rentistas y terratenientes. La influencia sobre Ricardo fue muy fuerte aunque de manera indirecta: sus frecuentes intercambios sirvieron para clarificar y enfrentar sus respectivas ideas. Las dos \underline{principales obras} son el \textit{Ensayo sobre el Principio de la Población} (1798) y los \textit{Principios de Economía Política} (1820). Malthus desarrolla su obra en un contexto de tensiones económicas por las guerras napoleónicas, las secuelas de la independencia de EEUU y los problemas sociales que comienzan a hacerse evidentes como resultado del desarrollo industrial en Gran Bretaña. 

En el plano \textbf{metodológico}, la obra de Malthus es significativa por sus diferencias con la metodología de la mayoría de los clásicos. En primer lugar, Malthus muestra una tendencia a entender la economía como la búsqueda de la proporción adecuada. Esta idea de la proporción correcta adelanta el concepto posterior del óptimo que desarrollaría la economía neoclásica. Por otra parte, el autor rechaza frontalmente la idea de las \textit{leyes} de la economía porque entiende que los principios generales de la economía política están sujetos a un gran número de limitaciones y excepciones. 

La \textbf{teoría de la población} es uno de los elementos centrales del pensamiento malthusiano y la primera teoría de la población como tal. De forma sucinta, el modelo de la población de Malthus postula que si la población crece más rápidamente que progresa la tecnología, la población está condenada a la subsistencia en el largo plazo. El modelo parte de dos postulados: la población experimenta crecimiento geométrico, mientras que los recursos crecen de forma aritmética. En este contexto, deben existir una serie de frenos al crecimiento de la población. El concepto de frenos positivos son resultado de la insuficiencia de los recursos disponibles para proveer un medio de subsistencia a toda la población, y se manifiestan en forma de guerras, plagas, enfermedades, pobreza extrema, contaminación y en último término, inanición. Los frenos preventivos son aquellos que los seres humanos pueden establecer de forma consciente para evitar los frenos positivos, tales como el recurso a la prostitución, los métodos anticonceptivos o la abstinencia (la que el propio Malthus recomienda, como antiguo párroco). Con estos elementos, Malthus formula una dinámica de la población. Suponiendo una tasa de nacimientos constante, una tasa de fallecimientos que decrece con el aumento de los salarios, y unos salarios que decrecen con la población, es posible caracterizar la dinámica de la población hacia un estado estacionario. Cuando la población es demasiado grande, los fallecimientos aumentarán por los bajos salarios y se mantendrán por encima de la tasa de nacimientos, conduciendo a la economía hacía un equilibrio con menos población. De forma contraria, cuando la población es baja, la tasa de nacimientos estará por encima de los fallecimientos como resultado de salarios relativamente altos y la población tenderá a crecer. A pesar de la percepción habitual en la literatura no especializada de Malthus como un pesimista contrario al aumento de la población, en numerosas ocasiones el autor explicita que el aumento de la población es condición necesaria (aunque no suficiente) para el progreso social. El objetivo de su modelo es más bien caracterizar los límites al crecimiento ordenado de la población.

Respecto al \textbf{libre comercio}, la postura de Malthus es más tibia que la de otros economistas clásicos. Aunque acepta la idea del laissez faire como principio general, admite las excepciones de carácter pragmático de Smith. Además, y de forma especialmente notable, explicita que el laissez faire no siempre conduce a situaciones óptimas y puede llevar a situaciones de menor bienestar para el conjunto de la población, lo que puede llevar a justificar intervenciones más amplias de la autoridad pública. En un primer momento y en esa línea, rechazó derogar las \textit{corn laws} que regulaban la importación y los aranceles al grano en las primeras décadas del XIX. Posteriormente matizó su postura: se lamentaba de tener que apoyar las \textit{corn laws} a pesar de estar en desacuerdo con ellas desde el punto de vista de los principios. 

La teoría de Malthus sobre el \textbf{crecimiento económico} se encuentra dispersa en varias obras y no tiene el carácter sistemático que sí está presente en otros autores. Sin embargo, adelanta algunos conceptos que serían centrales en el siglo XX y especialmente en la economía keynesiana. Malthus define la demanda efectiva como poder junto con deseo de comprar. La demanda efectiva determina la producción de una economía por el lado de la demanda. Por ello, la distribución de la propiedad es importante para Malthus: para aprovechar todo el potencial de los medios de producción disponibles, es necesario una distribución de la renta que induzca una demanda efectiva óptima. Uno de los elementos de esta distribución óptima es la existencia de una clase de consumidores improductivos tales como sirvientes, militares, cómicos, rentistas, clérigos... en definitiva todo un conjunto de profesiones que no producen bienes materiales y que Malthus considera como ``improductivos''. Marx interpretará esto como una defensa del status quo y de los intereses de los rentistas. La postura de Malthus respecto de la Ley de Say dio lugar a grandes controversias. Ricardo y Marx entendían que Malthus afirmaba que el ahorro no es necesariamente igual a la inversión pero no porque los agentes económicos deseen acaparar producción, sino porque en determinados contextos no son capaces de satisfacer su demanda de inversión ante la ausencia de oportunidades adecuadas.

\marcar{David Ricardo} (1772-1823) fue el complemento de Malthus en muchas de sus aportaciones, así como el gran y primer exponente de la modelización económica a través de sistemas abstractos de axiomas a partir de los cuales se deducen conclusiones que sirven de explicación y predicción de fenómenos reales. En su relativamente corta vida, David Ricardo fue un muy exitoso inversor en bolsa y miembro del parlamento británico. Sus mayores influencias fueron Smith y James Mill, padre de J.S. Mill. Le influyeron también profundamente los largos intercambios epistolares con Malthus, su contrapunto teórico en numerosos debates. La obra de Ricardo irradió todo el pensamiento clásico posterior: James Mill, John Stuart Mill y Marx en el siglo XIX, Piero Sraffa y otros autores en el siglo XX. Su obra principal fueron los \textit{Principios de Economía Política y Tributación} (1817). También son relevantes sus dos panfletos sobre el precio del oro y la depreciación de la moneda, y el ensayo sobre el precio del grano y su impacto sobre los pagos a los factores de producción.

En lo \textbf{metodológico}, Ricardo supone un punto de inflexión respecto al pensamiento económico anterior y una de sus mayores aportaciones a la ciencia económica. Aunque los fisiócratas ya habían formulado modelos teóricos de la macroeconomía, Ricardo alcanza cotas nunca antes vistas en su capacidad para presentar modelos de elevada complejidad que derivan leyes generales a partir de principios fundamentales o leyes más básicas. Estas leyes derivadas sirven para caracterizar tendencias de largo plazo a las cuales la economía tiende aproximarse. Ricardo prefiere formular sistemas coherentes con el mayor grado de generalidad posible, un contrapunto a la tendencia de Malthus hacia tener en cuenta un número elevado de excepciones.

La \textbf{teoría de la renta diferencial} aparece en la segunda década del siglo de la mano de Ricardo, West, Torrens y Malthus. El concepto de rendimientos decrecientes sirve para construir la idea de renta diferencial. La generalidad de los economistas clásicos entienden que la agricultura tiende a producir rendimientos medios cada vez más bajos a medida que se incorporan dosis adicionales de mezclas homogéneas de capital y trabajo. Aunque los economistas clásicos no son capaces de formular una prueba lógica satisfactoria de este fenómeno, la regularidad empírica lo confirma. El precio de los bienes agrícolas depende del coste de producción, que es más elevado a medida que aumenta la producción por los rendimientos marginales decrecientes. La renta diferencial es para los clásicos el pago que recibe el poseedor de un bien cuya oferta es inelástica. Así, supongamos una porción de tierra que permite generar un beneficio de 120 € y otra tierra de peor calidad un beneficio de 100 €. Si la tierra de peor calidad es gratis, cabe preguntarse qué precio estarán dispuestos a pagar los agricultores por utilizar la tierra de calidad superior. Dado que pueden extraer un beneficio de 100 € sin pagar por la tierra, estarán dispuestos a pagar, como máximo, un precio de 20 € por utilizar la mejor tierra. Esos 20 € son la renta diferencial que extrae el propietario y constituyen un pago por el mero de hecho poseer un bien escaso. En términos literales de Ricardo, se trata de un pago por el uso de \textit{los poderes indestructibles y originales de la tierra}. En términos matemáticos, la renta puede definirse como la diferencia entre el producto medio multiplicado por la cantidad de factor productivo homogéneo y el producto marginal multiplicado por la cantidad de factor productivo homogéneo. Asumiendo que el producto marginal decrece a mayor velocidad que el producto medio por la ley de los rendimientos marginales decrecientes, la renta será cada vez mayor a medida que se agotan las tierras más productivas y las tierras del margen extensivo son cada vez peores. Asumiendo que el salario es una cantidad exógena determinada por factores culturales o biológicos, el remanente de restar al producto total el pago de rentas y salarios será cada vez menor. Este remanente representa la retribución a los capitalistas. A partir de estos supuestos y conclusiones, Ricardo deriva una teoría del crecimiento económico. Dada la tendencia a la reducción del retorno al capital hasta cero, los incentivos a la acumulación del capital se eliminarán y será necesario el progreso técnico o la apertura al comercio internacional para que aparezcan nuevas oportunidades de obtención de beneficios y la economía siga creciendo. En la medida que esto no se produzca, Ricardo concluye que la economía se estancará en un estado estacionario sin crecimiento del producto por habitante. En cierto modo, el modelo de Solow y Swann del siglo siguiente retomará el modelo Ricardiano de crecimiento en lo que respecta a caracterizar el crecimiento del producto en el estado estacionario como resultado de un proceso tecnológico exógeno. Es destacable por último que aunque estos autores entienden el concepto de manera similar, no así sus implicaciones de política económica. Si en Malthus las rentas son positivas porque sirven para evitar insuficiencias de demanda, Ricardo las considera negativas porque reducen el incentivo a acumular capital.

La \textbf{Teoría del Valor-Trabajo} de David Ricardo tendrá una importante influencia sobre todo el clasicismo posterior. Ricardo toma la teoría pura del valor que Adam Smith asociaba a economías primitivas y trata de resolver algunas de las complicaciones que aparecen en economías modernas. Aparece así la Teoría del Valor-Trabajo. El primer obstáculo a la determinación de los precios relativos en función de la cantidad relativa de trabajo incorporado surge de la \underline{existencia de la renta}. La renta es un coste que afecta al precio de los bienes y es diferente para cada bien e incluso dentro de un mismo bien, distorsionando el hipotético mecanismo de determinación de precios como resultado de las cantidades relativas de trabajo aplicado. Como solución, Ricardo propone tener en cuenta sólo el coste de las unidades producidas en el margen y no sujetas a renta. Así, en ese contexto la renta no influirá en el coste y no distorsionará los precios relativos. El segundo problema que se plantea es la \underline{heterogeneidad del trabajo}. Ricardo prefiere solucionar este problema simplemente obviando las diferencias entre diferentes calidades de trabajo, en cierto modo influenciado por la creencia propia de la Ilustración de considerar que todo el trabajo tiene el mismo potencial de mejora. El problema de la existencia de \underline{diferentes cantidades de capital en diferentes productos} es el mayor obstáculo teórico al que se enfrenta Ricardo. Si el capital consiste en el adelanto de los salarios a los trabajadores, el interés no es sino el pago que los capitalistas reciben por adelantar el salario un periodo. En la práctica, diferentes bienes requieren diferentes cantidades de capital y periodos de producción de diferentes duraciones, lo que implica costes del capital diferentes. En estos casos, el cociente de los costes de producción no resulta en un cociente exclusivamente entre cantidades de trabajo y por tanto, los precios relativos no dependen en exclusiva del trabajo.

Aunque se trata de soluciones poco satisfactorias incluso para el propio autor, es necesario tener en cuenta que Ricardo nunca concibió la Teoría del Valor-Trabajo sino como una mera aproximación a los precios relativos de los bienes. Además, explicitó excepciones a esta teoría tales como situaciones de monopolio o oferta inelástica.  

Ricardo examinó múltiples cuestiones además de la renta y la teoría del valor. El concepto de \textbf{ventaja comparativa }aparece en su obra aunque su autoría original es discutida actualmente y se atribuye alternativamente a Robert Torrens y/o a James Mill. La ventaja comparativa como explicación de los beneficios del comercio internacional y el patrón de comercio consiste en deducir que distintas economías pueden mejorar su bienestar si se especializan en los bienes con menor coste de oportunidad en términos del resto de bienes y después intercambian su producción con otras economías de acuerdo con sus preferencias. Ricardo expone este resultado a partir del famoso ejemplo de Inglaterra, Portugal, el vino y la tela. La \textbf{política monetaria} es uno de los temas a los que Ricardo dedica una atención especial en el contexto de la controversia de las primeras décadas del siglo XIX sobre la necesidad de restablecer la convertibilidad del papel moneda y el metal precioso. Ricardo apoya las tesis de la \textit{currency school} frente a la \textit{banking school}, argumentando que la posibilidad de emitir papel moneda no convertible resultará en un proceso de emisión excesiva e inflación que desestabilizará la economía. En general, Ricardo asume la teoría cuantitativa del dinero que Hume y otros habían ya formulado en la etapa pre-clásica y propugna la convertibilidad para evitar un crecimiento excesivo de la oferta monetaria.

Es frecuente considerar a \marcar{John Stuart Mill} (1806-1873) como el último de los economistas clásicos. En lo puramente económico, su obra matiza, extiende y mejora los temas principales del pensamiento clásico aunque no introduce ningún nuevo gran tema. Las influencias principales de Mill son Smith, su propio padre, Ricardo y el utilitarismo de influencia de Jeremy Bentham. Su obra tuvo un fuerte impacto sobre todo el utilitarismo posterior, sobre Alfred Marshall y en general, sobre todos los economistas en el periodo entre entre los inicios de la segunda mitad y la consolidación del neoclasicismo. Sus obras principales son los Principios de Economía Política (1848) en lo económico, y Sobre la Libertad (1859) y El Utilitarismo (1877) en los que se centra en la ética y la filosofía política. Los Principios de Economía Política se convirtieron en el principal libro de texto de economía hasta la aparición de Principios de Economía de Marshall varias décadas después. El libro refina los modelos clásicos de Ricardo, Smith, Malthus y James Mill, poniendo especial énfasis en los mecanismos que inducen transiciones entre unos equilibrios y otros. Además, introduce algunas contribuciones propias como la importancia de las instituciones como determinantes del progreso, un examen de la importancia de la educación más complejo que el de Smith y un análisis de la libertad del individuo como factor determinante del bienestar económico. En este contexto, Mill se muestra partidario del \textit{laissez faire} como regla general sujeta a posibles excepciones similares a las postuladas por Smith. Examina también el incipiente socialismo de mediados de siglo. Aunque Mill no es un socialista y critica los postulados de esta corriente, considera que debe ser tenida en cuenta como una opción factible.

Uno de los temas que Mill examina con especial detalle es la \textbf{distribución de la renta}. Afirma que la distribución de la renta no sigue leyes inexorables, sino que puede ser modificada por la decisión de individuos y gobiernos. Existe, según Mill, una relación compleja entre distribución y producción que debe ser considerada por la autoridad pública. A priori, la técnica determina la producción, pero la propia distribución del producto afecta a la técnica utilizada y por ello, indirectamente, al producto. Así, examina los diferentes modos de propiedad, enumerando las ventajas e inconvenientes de la propiedad común frente a la privada y concluyendo que la propiedad común está sujeta a problemas de free-riding, y la propiedad privada a problemas de agencia. Aunque es favorable a la propiedad privada, se muestra contrario a la herencia.

En lo que respecta al \textbf{crecimiento}, Mill pone el foco sobre la importancia de la educación. Una mejor educación de los ciudadanos redundará en un menor descuento del futuro y ello en más ahorro. Además, mayores grados de civilización resultan en menores riesgos y por ello una tendencia hacia tasas de beneficio e interés decrecientes hasta alcanzar un estado estacionario. Mill predice un futuro con tasas de crecimiento nulas y el fin de la lucha por alcanzar mayor riqueza relativa, ya que nadie deseará ser más rico una vez alcanzado un grado suficiente de bienestar y habiéndose eliminado la pobreza. En cuanto a la Ley de Say, Mill rechaza las críticas a este concepto aunque admite posibles excesos temporales de oferta.

En cuanto a la \textbf{política monetaria}, Mill refina los modelos anteriores. Así, actualiza el mecanismo flujo-especie de Hume para adaptarlo a la realidad del sistema financiero de su tiempo. Describe como influjos de metal precioso reducen el tipo de interés y mayores tasas de interés inducen influjos de capital. Así, la autoridad monetaria puede aumentar los tipos de interés para proteger las reservas de metal precioso.  En el debate del bullionismo entre banking school, free banking school y currency school, Mill adopta una posición intermedia. En tiempos normales, acepta las posturas del banking school. En situaciones de boom crediticio, prefiere la recomendación central de la currency school: es necesario limitar la emisión de papel moneda para poner límites a la expansión de la oferta monetaria y controlar la inflación.

La \textbf{teoría del valor} de Mill no es especialmente novedosa pero adelanta algunos de los temas del neoclasicismo, valorando el papel de la demanda en la determinación del precio de los bienes. Aunque afirma que el valor de uso solo determina el valor de intercambio de forma excepcional. Examina la determinación del precio como origen en demanda y oferta en función de la elasticidad de la oferta. Cuando la oferta es totalmente inelástica, es la demanda la que determina el precio. La oferta es el factor determinante cuando es perfectamente elástica. Cuando es relativamente elástica, el coste de producción es el determinante de largo plazo del precio relativo, pero está sujeto a fluctuaciones en el corto plazo.

En el ámbito del \textbf{comercio internacional}, Mill aporta algunos conceptos novedosos. Introduce la ecuación de demanda internacional para caracterizar el precio relativo de exportaciones e importaciones en un contexto de balanzas comerciales en equilibrio. Examina también las ganancias relativas que extraen unas economías y otras en función de la elasticidad de la demanda de importación. En la medida en que un país tenga una demanda de importaciones muy elástica, obtendrá más ganancias del comercio. En general apoya el libre comercio internacional, salvo para fomentar el desarrollo de industrias nacientes.

Por último, en cuanto a la \textit{tributación}, Mill rechaza los impuestos progresivos sobre la renta por entender que desincentivan el esfuerzo y sólo acepta la progresividad derivada de mínimos exentos. El impuesto de sucesiones es la herramienta de redistribución fundamental para Mill, de forma paralela a su rechazo de la herencia.

Antes de pasar a examinar a Karl Marx, es preciso nombrar a algunos \marcar{otros economistas clásicos} menores que sin embargo, aportaron algunos conceptos determinantes en la evolución de la ciencia económica posterior. \textbf{Jean-Baptiste Say} introdujo el debate sobre la igualdad entre el ahorro y la inversión. La ley que fue nombrada en su honor admite varias interpretaciones al respecto: ahorro e inversión son iguales como identidad; la igualdad entre ahorro en inversión es una condición de equilibrio; ahorro e inversión tienden a ser iguales en el largo plazo aunque sean posibles distorsiones temporales. La Ley de Say en su formulación más habitual se corresponde con esta última versión. \textbf{Nassau William Senior} desarrolló la teoría de la población de Malthus además de participar en numerosas controversias metodológicas de la época. \textbf{Henry Thornton} contribuyó al desarrollo de la teoría monetaria y la descripción del sistema bancario, así como a la teoría de la banca central. \textbf{Torrens} postuló el principio de la ventaja comparativa de forma paralela a Ricardo y a James Mill. Otros nombres relevantes son éste último autor, \textit{Sismondi, Saint-Simon o Piero Sraffa}, ya en el siglo XX pero continuador de la tradición clásica. 

\marcar{Karl Marx} (1818-1883) fue el autor que acuñó el término ``economía clásica''. Su propia adscripción a este grupo de economistas es controvertida. Por un lado, aplica y desarrolla temas clásicos como la teoría del valor-trabajo. Por otra parte, critica duramente el sistema conceptual de los economistas clásicos y su objetivo último es explicitar un nuevo sistema de pensamiento que describa la historia como resultado de la interacción de grupos sociales. Su trayectoria vital fue muy convulsa y trascurrió entre varios países europeos. Así como su obra, que se extiende más allá de la economía y entra de pleno en la sociología y la filosofía política. Las principales influencias de Marx son Hegel, Smith, Ricardo, la banking school y los socialistas utópicos de la primera mitad del XIX. La influencia posterior de Marx fue enorme en la sociología y la historia, pero no así en la economía, que tendió a considerar su obra como un programa de investigación con graves incoherencias. Sin embargo, algunos de sus temas sí tuvieron un cierto impacto sobre determinadas escuelas, como los economistas austriacos y su énfasis sobre el ajuste dinámico. Además, la posición central del marxismo en la organización política y económica de los países comunistas confieren al autor una posición destacada en la historia del pensamiento económico y en la historia económica. Las principales obras de Marx en lo que respecta al pensamiento económico son el \textit{Manifiesto Comunista} (1848) publicado junto con Engels, y sobre todo, su obra de referencia \textit{El Capital}, cuyo primer volumen aparece en (1867) y a la cual le siguieron dos volúmenes posteriores publicados a título póstumo a partir de notas recopiladas por Engels.

El \textbf{contexto filosófico y económico} en que Marx vive son determinantes de su pensamiento económico. En el plano filosófico, Marx bebe del materialismo hegeliano según el cuál son las condiciones materiales los principales determinantes de la realidad social. Al mismo tiempo, postula la dialéctica entre tesis y antítesis como mecanismo generador de síntesis que representan la evolución y transformación constante a la que está sujeta la realidad. Para Marx es fundamental la contextualización de un sistema económico como conjunto de relaciones fruto del momento histórico y las condiciones sociales concretas. Así, Marx enfatiza la influencia de lo social sobre lo individual y se aleja del individualismo metodológico que ya comenzaba a hacerse evidente en los clásicos. Marx critica la economía clásica por considerarla una herramienta de la burguesía dominante para mantener su posición de preeminencia. Según Marx, los clásicos no son definen la naturaleza de capital y beneficio, no reconocen el carácter histórico del sistema capitalista ni tienen en cuenta la explotación de los trabajadores. Así, entiende que los clásicos se centran en las relaciones de intercambio pero dejan de lado las condiciones en las que se lleva acabo la producción. Por esto, la consideración de Marx como economista clásico es ciertamente controvertida. 

Uno de los ejes del pensamiento de Marx es su \textbf{teoría del valor}. Ésta se basa en una crítica de la teoría del valor-trabajo postulada por Smith y Ricardo. En primer lugar, Marx distingue entre valor y precio. El primero es el valor de un bien en el sentido de cantidad de trabajo ` ``socialmente necesario'' para producir un bien determinado. Por otro lado, el precio de un bien es simplemente la relación de intercambio entre mercancías. En base a esta distinción entre precio y valor, Marx expone el concepto de \underline{explotación} central que será central en su pensamiento. La explotación es el hecho de que el precio del trabajo sea menor que su valor de uso. Así, los capitalistas producen bienes que venden a un precio superior al precio que pagan por el trabajo. El precio del trabajo está determinado por factores culturales y por el poder de mercado de los capitalistas. Constituye, en sus palabras, el ``valor necesario para reproducir la mano de obra''. La \underline{tasa de explotación} cuantifica esa relación entre precio de la producción y salario y se obtiene a partir del cociente entre plusvalía y precio del trabajo. La concepción del capital es otra materia a la que Marx dedica gran atención. Define el capital constante como el trabajo acumulado en forma de herramientas, máquinas, capital circulante, etc... y el capital variable como el trabajo humano directamente aplicado al proceso productivo. A partir de estos dos conceptos define la \textit{composición orgánica del capital} como la relación entre el capital constante y el capital variable utilizados en una industria determinada. Composiciones orgánicas del capital altas indican proporciones elevadas de capital constante, y bajas para proporciones reducidas de capital constante. El \underline{problema de la transformación} es el talón de Aquiles de su teoría, por el gran número de críticas y acusaciones de inconsistencia a las que dio lugar. Grosso modo, el problema consiste en reconciliar tres proposiciones. La \textit{primera} afirma que el valor relativo de los bienes es función creciente de la cantidad de trabajo incorporado, y dado que los precios relativos tienden a aproximarse al valor relativo, bienes con más trabajo incorporado se venderán a precios relativamente mayores. La \textit{segunda}, que los capitalistas derivan la plusvalía del capital variable. A más trabajo variable, más plusvalía y mayor tasa de beneficio. Por ello, la composición orgánica será clave a la hora de determinar la tasa de beneficio: composiciones orgánicas más altas deberán resultar en tasas de beneficio inferiores. Por último, la \textit{tercera}, recoge una regularidad empírica que indica que las tasas de beneficio tienden a igualarse entre industrias, de tal manera que industrias intensivas en capital no obtienen necesariamente menos beneficio. Esta observación empírica no es compatible con la teoría de la plusvalía de Marx. Así, debe existir algún obstáculo a la transformación del valor en precio y de ahí el nombre de ``problema de la transformación''. La solución que Marx propuso en el último volumen del capital consiste en postular que los sectores intensivos en capital venden por encima de su valor. En términos agregados, las diferencias tienden a compensarse y ello hace la teoría del valor-trabajo correcta en términos generales. Las fuertes críticas que esta solución suscitó fueron uno de los principales motivos de la pérdida de crédito del marxismo en el pensamiento económico posterior.

La \textbf{teoría del crecimiento de Marx} es también una teoría de la historia, con una énfasis particular en la interacción entre clases sociales y el papel de las crisis. De entrada, Marx acepta la Ley de Say como una condición de equilibrio pero entiende que apenas es posible que se cumpla. Según Marx, las crisis comienzan cuando los salarios suben como resultado del crecimiento económico. La subida de salarios reduce la inversión, lo que reduce la demanda agregada y obliga a las empresas a bajar precios y con ello, la tasa de beneficio, que a su vez provoca bajadas ulteriores de la inversión. La crisis se agudiza, aumenta el desempleo y el llamado ``ejército industrial de reserva''. Las empresas ineficientes desaparecen y caen los salarios, lo que aumenta la productividad y la tasa de beneficio, volviéndose a iniciar el ciclo del crecimiento económico. Eventualmente, los salarios volverán a subir y el proceso se desencadenará de nuevo. Las \underline{leyes del movimiento} de Marx son cuatro corolarios básicos que subyacen a la concepción de Marx del crecimiento. La \textit{primera}, la \textit{ley de la miseria creciente del proletariado}, afirma que los trabajadores cada vez son más pobres que los capitalistas en términos relativos, aunque su nivel de vida en términos absolutos mejore. La \textit{segunda} ley postula que la \textit{tasa de beneficio de las empresas tiende a decrecer con el tiempo}. En el corto plazo, los capitalistas tratan de aumentar la composición orgánica del capital para aumentar el poder de mercado y poder aumentar la plusvalía y la tasa de explotación para una misma cantidad de trabajo, pero en el largo plazo esto reduce la tasa de beneficio. La \textit{tercera} ley concierne la \textit{gravedad creciente de las crisis}. El ratio entre producción y capital es cada vez más pequeño, por lo que los beneficios son cada vez más sensibles al aumento de salarios y las fluctuaciones cíclicas son cada vez mayores. Por último, la \textit{cuarta} ley postula la \textit{concentración industrial creciente} como resultado de la absorción de empresas ineficientes en las crisis. Estas cuatro leyes describen de forma aproximada las llamadas ``contradicciones'' internas del capitalismo y conducirán al fin de éste y el inicio del socialismo. Los factores monetarios tienen en el modelo de Marx un papel secundario. Las crisis provocan excesos de demanda de dinero que a su vez reducen la liquidez disponible y dan lugar a suspensiones de pagos y quiebras que amplifican las crisis. La acumulación de saldos monetarios, por otra parte, tiene un efecto estimulador al inicio del nuevo ciclo.

A lo largo de la exposición hemos presentado la obra de los economistas clásicos más relevantes y de Karl Marx. En la historia del pensamiento económico, son numerosos los programas de investigación que a partir de cierto punto no tuvieron continuidad en sus propios términos. Hemos presentado en la exposición varios de estos programas: la teoría del valor-trabajo, el marxismo o la teoría de la renta diferencial. Sin embargo, estos temas fueron el germen de otros programas posteriores que contribuyeron enormemente al desarrollo de la ciencia económica, tales como la ley de Walras como refinamiento de la Ley de Say, la búsqueda de una teoría del valor más coherente y general, la teoría del crecimiento económico, el estudio de los ciclos económicos, la teoría monetaria o las ventajas del comercio internacional. 

\seccion{Preguntas clave}
\begin{itemize}
	\item ¿Quiénes fueron los economistas clásicos?
	\item ¿Cuáles fueron sus aportaciones?
	\item ¿Qué trayectoria intelectual siguieron?
	\item ¿Quiénes los influenciaron? ¿A quiénes influenciaron?
\end{itemize}

\esquemacorto

\begin{esquema}[enumerate]
	\1[] \marcar{Introducción}
		\2 Contextualización
			\3 Evolución de la ciencia económica
			\3 Historia del pensamiento económico
			\3 Economía clásica
		\2 Objeto
			\3 ¿Quiénes fueron los economistas clásicos?
			\3 ¿Qué aportaron?
			\3 ¿Qué trayectoria siguieron?
			\3 ¿Quiénes los influenciaron y a quienes influenciaron?
		\2 Estructura
			\3 Adam Smith
			\3 Robert Malthus
			\3 David Ricardo
			\3 Stuart Mill
			\3 Otros economistas clásicos
			\3 Karl Marx
	\1 \marcar{Adam Smith}
		\2 Trayectoria
			\3 Vida
			\3 Influenciado por:
			\3 Obras
			\3 Contexto histórico
		\2 Metodología
			\3 Narrativo
			\3 Método inductivo predomina
		\2 Teoría de los sentimientos morales
			\3 Idea clave
			\3 Temas
			\3 Dicotomía observador vs actor
		\2 La Riqueza de las Naciones
			\3 Libros
		\2 Teoría del valor
			\3 Medida del valor
			\3 Precios
			\3 Paradoja del valor
			\3 División del trabajo
		\2 Teoría de la distribución
			\3 Salarios
			\3 Beneficios
			\3 Rentas
			\3 Bienestar e ingreso total
		\2 Capital, ahorro e inversión
			\3 Concepto de capital
			\3 Capital corriente
			\3 Capital fijo
			\3 Inversión
			\3 Ahorro
		\2 Teoría monetaria
			\3 Mecanismo flujo-especie
			\3 Sistema bancario
		\2 Laissez-faire
			\3 Comercio internacional
			\3 Mano invisible
		\2 Teoría del estado
			\3 Intervención pública
			\3 Financiación del gasto
		\2 Influencia en España
			\3 Contexto
			\3 Seguidores de Adam Smith
			\3 Traducción
			\3 Oposición a Smith
	\1 \marcar{Robert Malthus}
		\2 Trayectoria
			\3 Vida
			\3 Influenciado por
			\3 Influenció a
			\3 Obras
			\3 Contexto histórico
		\2 Metodología
			\3 Doctrina de las proporciones
			\3 Rechazo a la generalización en forma de leyes
		\2 Teoría de la población
			\3 Idea clave
			\3 Ratios de crecimiento
			\3 Frenos positivos
			\3 Frenos preventivos
			\3 Equilibrio de economía Malthusiana
			\3 Argumentos a favor del crecimiento
			\3 Poor Laws
		\2 Libre comercio
			\3 Laissez faire
			\3 Corn Laws
		\2 Ahorro e inversión
			\3 ¿El ahorro es siempre inversión?
			\3 Excesos agregados de demanda
		\2 Crecimiento económico
			\3 Causas del progreso
			\3 Demanda efectiva
			\3 Distribución de la propiedad
	\1 \marcar{David Ricardo}
		\2 Trayectoria
			\3 Vida
			\3 Influenciado por
			\3 Influenció a
			\3 Obras
		\2 Metodología
			\3 Método deductivo
			\3 Formulación de sistemas
		\2 Teoría de la renta diferencial
			\3 1815: Panfletos de West, Torrens, Malthus y Ricardo
			\3 Rendimientos decrecientes
			\3 Renta diferencial
			\3 Distribución de la renta
			\3 Crecimiento de estado estacionario
		\2 Teoría del valor
			\3 Idea clave
			\3 Teoría pura del valor
			\3[1] Renta es un coste
			\3[2] Trabajo no homogéneo
			\3[3] Diferentes cantidades de capital
			\3 Oferta inelástica y monopolio
		\2 Comercio internacional
			\3 Ventaja comparativa
			\3 Patrón de especialización
			\3 Beneficios del comercio
			\3 Escape a estado estacionario
		\2 Política monetaria
			\3 Contexto histórico
			\3 Controversia bullionista
			\3 Critica a banking school
			\3 Defensor de la Currency School
		\2 Política fiscal y tributación
			\3 Impuestos
			\3 Deuda
			\3 Minimizar gasto público
	\1 \marcar{John Stuart Mill}
		\2 Trayectoria
			\3 Vida
			\3 Influenciado por
			\3 Influenció a
			\3 Obras
		\2 Principios de Economía Política (1848)
			\3 Idea clave
			\3 Contribuciones propias:
			\3 Laissez-faire
		\2 Teoría del valor
			\3 Acepta TVT de Ricardo
			\3 Valor es concepto relativo, no absoluto
			\3 Demanda y oferta
		\2 Distribución de la renta
			\3 Depende de instituciones, no de leyes
			\3 Propiedad privada frente a propiedad común
			\3 Rechaza derecho a la herencia
			\3 Beneficio empresarial
		\2 Inversión, ahorro y crecimiento
			\3 Tendencia decreciente de la tasa de beneficio
			\3 Futuro con crecimiento cero
			\3 Ley de Say
		\2 Política monetaria
			\3 Tipo de interés y tipo de cambio
			\3 Bullionismo, banking school y currency school
		\2 Comercio internacional
			\3 Ecuación de Demanda Internacional
			\3 Ganancias del comercio
		\2 Tributación
			\3 Impuesto sobre la renta
			\3 Impuesto de sucesiones
		\2 Teoría de la empresa
			\3 Problema de agencia
			\3 Instrumento de acumulación de capital
			\3 Papel en crecimiento
	\1 \marcar{Karl Marx}
		\2 Trayectoria
			\3 Vida
			\3 Influenciado por
			\3 Influenció a
			\3 Obras
		\2 Contexto económico y filosófico
			\3 Materialismo
			\3 Dialéctica e historia
			\3 Burguesía y ciencia económica
		\2 Teoría del valor
			\3 Idea clave
			\3 Valor y precio
			\3 Explotación
			\3 Capital
			\3 Problema de la transformación
		\2 Crecimiento económico y ciclos
			\3 Ley de Say
			\3 Crisis
			\3 Leyes del movimiento
			\3 Política monetaria
		\2 Críticas al modelo de Marx
			\3 Falsabilidad
			\3 Problema de la transformación
			\3 Remuneración a PMg frente a salarios de subsistencia
			\3 Gestión de empresa como factor de producción
			\3 Tiempo como factor de producción
	\1 \marcar{Otros economistas clásicos}
		\2 Jean Baptiste Say
			\3 Trayectoria
			\3 Ley de Say
			\3 Teoría del valor
		\2 Senior
			\3 Teoría de la población
			\3 Metodología
		\2 Henry Thornton
			\3 Trayectoria
			\3 Política monetaria
		\2 Robert Torrens
			\3 Comercio internacional
			\3 Arancel óptimo
		\2 Otros nombres
			\3 Fréderic Bastiat
			\3 Simonde de Sismondi
			\3 James Mill
			\3 Henri de Saint-Simon
			\3 Piero Sraffa
	\1[] \marcar{Conclusión}
		\2 Recapitulación
			\3 Adam Smith
			\3 Thomas Malthus
			\3 David Ricardo
			\3 John Stuart Mill
			\3 Karl Marx
			\3 Otros economistas clásicos
		\2 Idea final
			\3 Programas de investigación sin continuidad
			\3 Contribuciones duraderas y germen de programas

\end{esquema}

\esquemalargo










\begin{esquemal}
	\1[] \marcar{Introducción}
		\2 Contextualización
			\3 Evolución de la ciencia económica
				\4 Conjunción de múltiples factores
				\4 Contexto económico
				\4 Avances en otras discplinas
				\4[] Filosofía
				\4[] Matemáticas
				\4[] Biología
			\3 Historia del pensamiento económico
				\4 Permite entender origen de pensamiento actual
				\4 Permite entender problemas históricos
				\4 Permite valorar programas de investigación
			\3 Economía clásica
				\4 Primer grupo de economistas con tal nombre
				\4 Concepto acuñado por Marx
				\4 Diferentes definiciones:
				\4[] Seguidores de Adam Smith
				\4[] Economistas que creen en ajuste hacia equilibrio
				\4[] Economistas que aceptan Ley de Say
				\4[] Seguidores de Smith y Ricardo
				\4[] Defensores del laissez-faire
				\4[] Desde Smith hasta 1830
				\4 Generalmente:
				\4[] Economistas desde Smith hasta Mill
		\2 Objeto
			\3 ¿Quiénes fueron los economistas clásicos?
			\3 ¿Qué aportaron?
			\3 ¿Qué trayectoria siguieron?
			\3 ¿Quiénes los influenciaron y a quienes influenciaron?
		\2 Estructura
			\3 Adam Smith
			\3 Robert Malthus
			\3 David Ricardo
			\3 Stuart Mill
			\3 Otros economistas clásicos
			\3 Karl Marx
	\1 \marcar{Adam Smith}
		\2 Trayectoria
			\3 Vida
				\4 1723-1790
				\4 Universidad de Glasgow $\to$ alumno de Hutcheson
				\4 Oxford $\to$ como alumno y profesor
				\4 Periodo en Francia $\to$ fisiócratas
			\3 Influenciado por:
				\4 Fisiócratas
				\4[] Quesnay $\to$ influencia principal
				\4[] Turgot
				\4 Hutcheson
				\4[] Profesor de Smith en Glasgow
				\4[] Influencia programa de investigación
				\4 Hume
				\4[] Intercambio de ideas
				\4[] Relación epistolar de muchos años
			\3 Obras
				\4 Teoría de los Sentimientos Morales
				\4 Lecciones sobre Jurisprudencia
				\4 Investigación sobre la naturaleza y las Causas de la Riqueza de las Naciones
			\3 Contexto histórico
				\4 Revolución Industrial incipiente
				\4 Primeras fábricas
				\4 Agricultura es aún base de la economía
				\4 Ilustración escocesa
		\2 Metodología
			\3 Narrativo
				\4 Carente de formalización
				\4 A pesar de influencia fisiócrata
			\3 Método inductivo predomina
				\4 Influencia de Hume
				\4 Ejemplos históricos
				\4 Deducción relativamente menor
		\2 Teoría de los sentimientos morales
			\3 Idea clave
				\4 Fundamentos éticos y filosóficos de comportamiento
				\4 Influencia en ideas posteriores
				\4 ¿Qué hace felices a los seres humanos?
				\4 ¿Qué les mueve a actuar?
			\3 Temas
				\4 Empatía por el prójimo
				\4 Humanos tratan de emular riqueza y fortuna ajena
				\4 Externalidades positivas de la búsqueda de riqueza
				\4 Conservadurismo y reformas
				\4[] Las instituciones tienen ventajas no evidentes
				\4 Sociedad como sistema estable
				\4[] $\to$ Reformas pueden desestabilizar
			\3 Dicotomía observador vs actor
				\4 En fuero interno de seres humanos
				\4[] Hay dos agentes más o menos separados
				\4 Observador externo
				\4[] Puede valorar adecuado de actos
				\4[] Plantea decisiones ``racionales''
				\4 Actor
				\4[] Ejecuta decisiones
				\4[] No siempre escucha al observador
				\4[] Sujeto a influjos ``irracionales''
				\4[] $\to$ No actúa para alcanzar sus objetivos
				\4 Precede behavioral economics
				\4[] Reglas heurísticas para decidir
				\4[] $\to$ Kahneman y Tversky (1974)
		\2 La Riqueza de las Naciones
			\3 Libros
				\4[I] -- Teoría del valor y la distribución
				\4[II] -- Acumulación del capital y rentas
				\4[III] -- Historia económica
				\4[] Análisis comparativo del progreso
				\4[] Roma, Europa, Inglaterra
				\4[] Relaciones campo-ciudad
				\4[IV] -- Teoría económica y política económica
				\4[V] -- El estado y la política fiscal
		\2 Teoría del valor
			\3 Medida del valor
				\4 Valor de uso
				\4[] Beneficio que genera consumo
				\4 Valor de intercambio
				\4[] Cantidad de otros bienes que se puede obtener
				\4[] No siempre igual a precio
				\4[] Sí tiende a igualar con precio
			\3 Precios
				\4 En el corto plazo, \textit{precio de mercado}\footnote{Resultado de la acción de las fuerzas de oferta y demanda.}
				\4[] En el largo plazo, \textit{precio natural}
				\4 Determinantes del precio natural
				\4[] $\to$ Coste de los factores utilizados
				\4[] Sociedades con un solo factor (trabajo)\footnote{En términos de Smith, estas sociedades se encuentran en un ``\textit{early and rude state of society}''. Para explicar estas situaciones utilizó el ejemplo del castor y el ciervo (\textit{beavers and deers}).}.
				\4[] $\to$ Precio determinado por trabajo utilizado
				\4[] Sociedades avanzadas
				\4[] $\to$ Suma de costes monetarios de trabajo, tierra y capital
				\4[] Rechazo de la teoría del valor-trabajo\footnote{Que habían sugerido algunos precedesores sin llegar a desarrollar la idea completamente como haría posteriormente Ricardo.}
				\4[] Sin teoría de explícita del precio de los factores
				\4 ¿Cómo se alcanza el precio natural en el mercado?
				\4[] Ajuste constante de oferta y demanda
				\4[] Tendiente al precio natural
				\4[] Relativa falta de examen de la demanda
				\4 Competencia
				\4[] Sentido diferente al moderno
				\4[] Concepción dinámica
				\4[] Proceso que lleva hacia precio natural
			\3 Paradoja del valor
				\4 Agua
				\4[] Enorme utilidad
				\4[] $\to$ Fundamental para la vida
				\4[] Muy poco precio
				\4 Diamantes
				\4[] Muy poca utilidad en vida diaria
				\4[] $\to$ No pueden comerse ni fabricarse nada con ellos
				\4[] Bienes poseen:
				\4[] Valor de uso
				\4[] Valor de intercambio
				\4[] No necesariamente ocurren a la vez
				\4 ¿Por qué diamantes se venden más caros que el agua?
				\4[] Por el ``\textit{toil and trouble}'' de obtenerlos
				\4[] $\to$ Mucho mayor para los diamantes
				\4[$\then$] Teoría del valor trabajo al menos en intercambio
				\4[$\then$] Ignora papel de la demanda en determinación
				\4 Respuesta insatisfactoria
				\4[] Un diamante perfecto encontrado en la naturaleza
				\4[] $\to$ No vale menos que un diamante pulido con mucho esfuerzo
				\4[] $\then$ No se trata de toil and trouble
				\4 Solución posterior a paradoja
				\4[] Jevons, Menger, Walras
				\4[] Valor de las cosas depende de uso marginal
				\4[] Uso marginal de diamante:
				\4[] $\to$ Decoración
				\4[] $\to$ Cortar otros materiales
				\4[] Uso marginal de agua
				\4[] $\to$ Limpiar cosas de poco valor, jugar, etc...
				\4[] Como uso marginal de agua es menos importante
				\4[] $\to$ Tiene menos valor
				\4[] Si agua escasea
				\4[] $\to$ Uso marginal es mantenerse vivo
				\4[] $\then$ Valor relativo aumenta enormemente
			\3 División del trabajo
				\4 División del trabajo
				\4[] Ejemplo de las agujas
				\4[] Razones por las que aumenta producción
				\4[] Economía es una red de interrelaciones
				\4 Posibilidades de especialización
				\4[] Dependen del tamaño del mercado
				\4[] Adelanta economías de escala externas
				\4[] Crecimiento endógeno
				\4[] $\Rightarrow$ Rendimientos crecientes a escala
		\2 Teoría de la distribución
			\3 Salarios
				\4 Recapitula diferentes teorías de salarios
				\4[] $\to$ Fondo de salarios
				\4[] $\to$ Salarios de subsistencia
				\4[] $\to$ Negociación
				\4[] $\to$ Productividad
				\4 Aplica mecanismo de ajuste similar al de bienes
				\4[] Ajuste de salarios l/p a salario natural
				\4[] $\to$ Similar a la teoría de los salarios de subsistencia
				\4[] Empresarios utilizan fondo de salarios
				\4[] Para pagar salarios hasta que venden producto
				\4 Salarios relativos dependen de
				\4[(i)] Desagradable del trabajo
				\4[] $\to$ Precede análisis ocio-consumo
				\4[(ii)] Coste de obtención de habilidades
				\4[] $\to$ Precede capital humano
				\4[(iii)] Regularidad del empleo
				\4[] $\to$ Precede contratos implícitos
				\4[(iv)] Confianza y responsabilidad
				\4[] $\to$ Precede salarios de eficiencia
				\4[(v)] Probabilidad de cobrar el salario efectivamente
				\4[] $\to$ Precede análisis de riesgo y contratos implícitos
				\4[] $\Rightarrow$ Competencia no iguala salarios monetarios
				\4[] $\Rightarrow$ Sí iguala (des)ventajas de las ocupaciones
			\3 Beneficios
				\4 Diferente a salarios de directivos
				\4[] Beneficio =  interés + prima de riesgo
				\4[] Competencia reduce interés $\Rightarrow$ $\downarrow$ beneficio
				\4[] Pero crecimiento económico puede aumentar ambos\footnote{Como sucede en las colonias.}
			\3 Rentas
				\4 Excedente diferencial
				\4 Determinado por el precio, salario, beneficio
				\4 Salarios y beneficios causan precios
				\4 Rentas son lo que resta
			\3 Bienestar e ingreso total
				\4 El valor real de un bien es su coste psicológico total
				\4 Sociedad mejor si menor coste para mismo valor\footnote{Adam Smith denominó ``toil and trouble'' a ese coste.}
				\4[] Precursor de PIB como medida de bienestar
				\4[] El trabajo\footnote{En sentido de ``\textit{toil and trouble}''.} es la base de la riqueza
				\4[] $\to$ No la cantidad de metal precioso
				\4[] $\to$ El trabajo hace rico a las naciones
				\4[] $\to$ Crítica al mercantilismo
		\2 Capital, ahorro e inversión
			\3 Concepto de capital
				\4 Bienes intermedios entre
				\4[] $\to$ uso de inputs
				\4[] $\to$ producción final
				\4 Dos tipos de capital
				\4[] Fijo
				\4[] Corriente
				\4[] $\to$ Diferentes proporciones según industrias
			\3 Capital corriente
				\4 Reporta beneficio cuando se vende
				\4 Creado y vendido en ciclo de producción
				\4 Cuantificado en términos reales
			\3 Capital fijo
				\4 Máquinas, edificios, capital humano
				\4[] $\to$ Medios de producción producidos
			\3 Inversión
				\4 Deseabilidad de inversión
				\4[] Agri. > manufact. > com. interior > com. intncl. > transporte
				\4 Ingreso y valor añadido
				\4[] Define ingreso bruto y neto
				\4[] Ingreso bruto $\to$ similar a PIB
				\4[] Ingreso neto $\to$ similar a PIN\footnote{Es decir, PIB menos depreciación del capital.}
				\4 Dos definiciones de trabajo productivo
				\4[] $\to$ aquel que añada valor
				\4[] $\to$ aquel que perdura y que permite producir más
				\4[] $\then$ Trabajo improductivo el que se consumo con el uso\footnote{``\textit{[...] perish in the very instant of their performance.}''.}
			\3 Ahorro
				\4 Ahorro es gasto
				\4[] El ahorro no destruye poder de compra
				\4 Ahorro depende de instituciones y costumbres
				\4[] $\to$ No dice que dependa de interés o renta
				\4 Aproximadamente acepta Ley de Say
		\2 Teoría monetaria
			\3 Mecanismo flujo-especie
				\4 Acepta sólo parcialmente
				\4[] $\to$ sin mencionar explícitamente
				\4 Aprecia diferentes niveles de precios en Europa
			\3 Sistema bancario
				\4 Papel moneda sirve para no utilizar oro
				\4 Ley del reflujo debería:
				\4[] $\to$ Igualar valor de reservas y papel moneda
				\4[] $\then$ Banking school
				\4 Pero se ha emitido demasiado
				\4[] $\to$ Currency school
		\2 Laissez-faire
			\3 Comercio internacional
				\4 Argumentos en favor del libre comercio
				\4[] Hay que comprar donde se produce a menor coste\footnote{``\textit{[...] what is prudence in the conduct of every private family can scarce be folly in that of a great Kingdom''}.}
				\4[] Si todos lo hacen, bienestar general aumentará
				\4[] $\to$ Falacia de composición
				\4 Impacto de las recomendaciones de Smith
				\4[] Escaso
				\4[] Mercantilismo ya había quedado atrás
			\3 Mano invisible
				\4 La mano invisible
				\4[] Armonía entre:
				\4[] $\to$ Eficiencia económica
				\4[] $\to$ Intereses privados
				\4[] En términos modernos, mano invisible es competencia
				\4 Limitaciones de la mano invisible como equilibrador
				\4[] Industrias nacientes
				\4[] Medidas de retorsión
				\4[] Economías de escala externas
				\4[] Defensa
		\2 Teoría del estado
			\3 Intervención pública
				\4 Tres deberes del estado
				\4[] Defensa
				\4[] Justicia
				\4[] Infraestructuras y obras públicas
				\4 Gasto público
				\4[] Construir infraestructuras
				\4[] Mantener el comercio
				\4[] Sistema educativo
				\4[] Financiado por usuarios directos si posible
				\4 Justicia
				\4[] Sirve para proteger la propiedad
				\4[] Mantiene incentivos de acumulación de capital
			\3 Financiación del gasto
				\4 Principios de la tributación
				\4[(i)] Carga en función de la renta
				\4[(ii)] Impuestos sencillos, claros, sin incertidumbre
				\4[(iii)] Pagados al mismo tiempo que sucede el hecho imponible
				\4[(iv)] Minimizando costes de administración
				\4 Compañías y personas jurídicas
				\4[] Opuesto a los privilegios que reciben
				\4[] Consejo de administración reacciona demasiado lento
				\4[] $\to$ no es práctico en determinados sectores
				\4 Impuestos sobre la tierra
				\4[] Justos si gravan la tierra
				\4[] Ineficientes si gravan el producto de la tierra
				\4 Impuestos sobre capital y beneficios
				\4[] Beneficios son pago por riesgo y gestión
				\4[] $\to$ Impuesto aumentaría precios de bienes
				\4[] $\to$ Reduciría pagos de interés
				\4[] $\Rightarrow$ Primitivo análisis de incidencia
				\4[] $\Rightarrow$ No debe gravarse
				\4[] Capital difícil de gravar
				\4[] Desincentiva acumulación
				\4 Impuesto sobre salarios, personas y bienes
				\4[] Sobre salarios, serían repercutidos sobre empleador
				\4[] Empleador repercutiría sobre bienes
				\4 Impuesto sobre riqueza
				\4[] Injusto porque es volátil
				\4 Impuestos de capitación
				\4[] Injusto porque pobres pagan =
				\4 Deuda o impuestos
				\4[] En general, mejor financiarse con impuestos
		\2 Influencia en España\footnote{Ver \href{https://www.elmundo.com/portal/opinion/columnistas/la_primera_traduccion_de_la_riqueza__de_las_naciones_al_castellano.php}{Maya, G. M. (2016) La primera traducción de La Riqueza de las Naciones al Castellano} y \href{http://www.cepc.gob.es/Controls/Mav/getData.ashx?MAVqs=~aWQ9MzE0OTYmaWRlPTEwMzcmdXJsPTExJm5hbWU9UkVDUF8wMTlfMjIxLnBkZiZmaWxlPVJFQ1BfMDE5XzIyMS5wZGYmdGFibGE9QXJ0aWN1bG8mY29udGVudD1hcHBsaWNhdGlvbi9wZGY=}{Smith, R. S. (1957) La ``Riqueza de las Naciones'' en España e Hispanoamérica, 1780-1830}.}
			\3 Contexto
				\4 Mercantilismo alcanza cumbre primer cuarto sXVIII
				\4[] Uztáriz: \textit{``Theórica de Comercio y Marina''}
				\4[] $\to$ El propio Adam Smith conoce la obra
				\4[] $\to$ Sirve como referencia para criticar mercantilismo
				\4 Influencia creciente de fisiócratas e ilustrados
				\4[] Quesnay, Turgot, Condorcet
				\4[] Disponibles en castellano
			\3 Seguidores de Adam Smith
				\4 Influencia en Sociedades Económicas de Amigos del País
				\4 Gaspar Melchor de Jovellanos
				\4 Vicente Alcalá Galiano
				\4 Valentín de Foronda
			\3 Traducción
				\4 Primera traducción en castellano en 1794
				\4[] José Alonso Ortiz
				\4 Incluye pequeños cambios
				\4[] Defensa de Imperio Español
				\4[] Múltiples notas aclaratorias
			\3 Oposición a Smith
				\4 Principalmente en Cataluña
				\4 Ramón Lázaro de Dou y de Bassols
				\4 Defiende aranceles
				\4[] Herramienta para industrializar
				\4[] Defiende obra de Uztáriz
	\1 \marcar{Robert Malthus}
		\2 Trayectoria
			\3 Vida
				\4 1766-1834
				\4 Hijo de un rentista
				\4 Formación eclesiástica
			\3 Influenciado por
				\4 Matemáticas
				\4 Smith
				\4 Ricardo
			\3 Influenció a
				\4 Keynes
				\4[] Demanda efectiva
				\4[] Metodología: modelos no son leyes
				\4 Marx
				\4[] Criticó a Malthus
				\4[] Malinterpretó ideas sobre redistribución
				\4[] Le acusaba de pro-rentistas y terratenientes
				\4 David Ricardo
				\4[] Intercambio epistolar y amistad
				\4[] Desacuerdo frecuente
			\3 Obras
				\4 Ensayo sobre el Principio de la Población (1798)
				\4[] Numerosas ediciones posteriores con cambios
				\4 Principios de Economía Política (1820)
			\3 Contexto histórico
				\4 Revolución industrial
				\4 Coetáneo de David Ricardo
				\4 Post independencia de EEUU
		\2 Metodología
			\3 Doctrina de las proporciones
				\4 Influencia aristotélica
				\4 Búsqueda de proporciones idóneas en economía
				\4[] $\Rightarrow$ Adelanta el concepto de óptimo\footnote{Aunque sin utilizar este término.}
				\4 Reconoce dificultad de encontrar proporciones
				\4[] $\to$ La economía política no tiene aún recursos
			\3 Rechazo a la generalización en forma de leyes
				\4 Principios generales de ec. política
				\4[] Sujetos a limitaciones y excepciones
				\4[] $\Rightarrow$ Crítica a la idea de \textit{ley} de ec. política
		\2 Teoría de la población
			\3 Idea clave
				\4 Primera teoría de la población
				\4 Si la población crece más rápido que la tecnología
				\4[] $\Rightarrow$ población condenada a subsistencia
			\3 Ratios de crecimiento
				\4 Población: crecimiento geométrico
				\4 Recursos: crecimiento aritmético
			\3 Frenos positivos
				\4 Incremento de la tasa de defunción
				\4[] Guerras, plagas, enfermedades...
				\4[] Pobreza extrema, contaminación, inanición
				\4 Reducción de la esperanza de vida
			\3 Frenos preventivos
				\4 Reducción de la tasa de nacimientos
				\4[] Prostitución, anticonceptivos
				\4[] Abstinencia preventiva
				\4[] $\to$ Hasta asegurarse medios de subsistencia
			\3 Equilibrio de economía Malthusiana
				\4 Supuestos:
				\4[(i)] Tasa de nacimientos constantes
				\4[(ii)] Tasa de defunción decrece con salarios
				\4[(iii)] Ingreso pc. baja con población
				\4[$\to$] Ingreso pc y población de equilibrio: \grafica{salariomalthus}
				\4[$\Rightarrow$] Aumento de la mortalidad puede $\uparrow$ nivel de vida
				\4[$\Rightarrow$] Menos nacimientos puede $\uparrow$ nivel de vida
			\3 Argumentos a favor del crecimiento
				\4 A pesar de idea popular\footnote{Aunque es cierto que sus manifestaciones más rotundas se encontraban en apéndices al \textit{Ensayo} que fueron eliminados en ediciones posteriores.}
				\4[] $\uparrow$ Población es necesario para crecimiento
				\4[] Aunque no suficiente
			\3 Poor Laws
				\4 Efecto contraproducente
				\4 Aumentan pobreza porque aumentan población
				\4 Evolución de postura:
				\4[] Primero, abolir
				\4[] Segundo, abolición gradual
				\4[] Tercero, reforma administrativa
				\4[] $\to$ Enfoque pragmático de la política económica
		\2 Libre comercio
			\3 Laissez faire
				\4 Acepta como principio general
				\4 Señala excepciones y reservas
				\4[$\to$] Provisión de bienes públicos
				\4[$\to$] Tributación
				\4[$\Rightarrow$] Apoyo pragmático y condicional à la Smith
				\4 Resultado del laissez faire no siempre positivo
				\4[] Diverge en este punto de Smith
				\4[] Posibilidad de crecimiento económico perjudicial
				\4[] $\to$ Conflicto entre crecimiento y felicidad
				\4[] $\Rightarrow$ Justificación del intervencionismo
			\3 Corn Laws
				\4 Inicialmente, se opuso a abolición
				\4[$\to$] Excepción al laissez faire general que defendía
				\4[$\to$] Ejemplo de rechazo a ``leyes'' de política económica
				\4 Posteriormente moderó postura
				\4[$\to$] Se lamentaba de tener que apoyar Corn Laws
		\2 Ahorro e inversión
			\3 ¿El ahorro es siempre inversión?
				\4 Desacuerdo en literatura
				\4 Keynes, Ricardo entendían que decía:
				\4[] $\to$ Ahorro no es siempre igual a inversión
				\4[] No porque agentes prefieran acaparar
				\4[] $\to$ sino porque no encuentra inversión satisfactoria
				\4[] Múltiples ejemplos
			\3 Excesos agregados de demanda
				\4 Afirma evidencia empírica favorable
				\4 Keynes entiende como rechazo de Ley de Say
		\2 Crecimiento económico
			\3 Causas del progreso
				\4 Ideas dispersas en varias obras
				\4 Dos causas inmediatas
				\4[] Medios de producción
				\4[] Medios de distribución
				\4 Ideas estándar sobre la producción
				\4 Idea controvertida sobre distribución
				\4[] $\to$ Papel clave de la demanda efectiva
			\3 Demanda efectiva
				\4 Poder y deseo de comprar $\Rightarrow$ demanda efectiva
				\4 Determina producción por lado de la demanda
			\3 Distribución de la propiedad
				\4 Para que se aproveche potencial de medios
				\4[] $\to$ necesaria distribución óptima de ingreso
				\4[] $\to$ necesarios consumidores ``improductivos''\footnote{Tales como sirvientes, militares, actores, rentistas, clérigos... que no producen bienes materiales.}
				\4[$\Rightarrow$] Distribución como factor de producción
				\4[$\Rightarrow$] Defensa de distribución de riqueza
				\4 Distribución moderada
				\4[] $\to$ Defensa de la propiedad privada
				\4[] $\to$ Redistribución extrema es perjudicial
				\4 Malinterpretado por Marx
	\1 \marcar{David Ricardo}
		\2 Trayectoria
			\3 Vida
				\4 1772-1823
				\4 Inversor en bolsa y político
			\3 Influenciado por
				\4 Smith
				\4 Robert Malthus $\to$ relación epistolar
				\4 James Mill $\to$ discípulo y mentor al tiempo
				\4 Jean-Baptiste Say
			\3 Influenció a
				\4 James Mill
				\4 J. S. Mill
				\4 Malthus
				\4 Marx
				\4 Piero Sraffa
			\3 Obras
				\4 El Alto Precio del Lingote, una Prueba de la Depreciación del Papel Moneda (1809)
				\4 Ensayo sobre la Influencia del Bajo Precio del Grano en los Beneficios del Capital (1815)
				\4 Principios de Economía Política y Tributación (1817)
		\2 Metodología
			\3 Método deductivo
				\4 Capacidad deductiva admirada por contemporáneos
				\4 Derivación de leyes basadas en principios o leyes básicas
				\4 Tendencias a largo plazo derivadas de leyes
			\3 Formulación de sistemas
				\4 Tendencia a construir sistemas coherentes
				\4 Menos tendencia a aceptar excepciones
				\4[] A diferencia de Malthus
				\4[] $\to$ entendidas como ajuste a equilibrio de l/p
		\2 Teoría de la renta diferencial
			\3 1815: Panfletos de West, Torrens, Malthus y Ricardo
				\4 Formulación independiente de la misma teoría
				\4 Diferentes implicaciones de política económica
				\4 Malthus:
				\4[] Renta positivas para evitar insuficiencia de demanda
				\4 Ricardo:
				\4[] Renta negativa porque reduce incentivo a acumular capital
			\3 Rendimientos decrecientes
				\4 Concepto central de la economía clásica
				\4 Aplicable a la agricultura
				\4 Asumiendo dosis homogéneas de trabajo y capital:
				\4[] $\uparrow$ trabajo-capital $\to$ $\downarrow$ output medio
				\4 Múltiples intentos de prueba
				\4[] En general, fallidos
				\4[] Pero regularidad empírica lo confirma
				\4 Precio del bien agrícola
				\4[] Determinado por coste marginal de producción
				\4[] $\Rightarrow$ Mayor coste $\to$ mayor precio del bien
			\3 Renta diferencial
				\4 Renta es pago bien inelástico que se posee
				\4[] Tierra A permite beneficio 120 €
				\4[] Tierra B permite beneficio 100 €
				\4[] Tierra B es gratis
				\4[] ¿Qué renta deberá pagarse por usar A?
				\4[] $\Rightarrow$ $\textrm{Precio}_B = 120 - 100 = 20 \textrm{€}$
				\4 Formulación
				\4[] $\textrm{Renta} = f(N) - N\cdot f'(N) = \textrm{PMe}(N)\cdot N - N \cdot f'(N)$
				\4[] Donde:
				\4[] $f(N)$ $\to$ producto total
				\4[] $N$ $\to$ trabajo total
				\4[] $f'(N)$ $\to$ producto marginal del trabajo
				\4[] \grafica{modelodericardo}
				\4[$\then$] Renta es pago por escasez
				\4[] Existe porque hay ff.pp. con oferta fija
			\3 Distribución de la renta
				\4 Salarios
				\4[] En c/p: tamaño del fondo de salarios
				\4[] En l/p: teoría de salarios de subsistencia\footnote{Complementado en cierta medida con alusiones a los factores psicológicos que restringen el crecimiento de la población cuando ésta se acostumbra a unos niveles de vida más elevados. Ver Hicks y Hollander (1977).}
				\4 Renta
				\4[] Mejoras tecnológicas en agricultura reducen renta
				\4[] Aumento de la demanda aumenta renta
				\4 Beneficios
				\4[] Residuo después de pagar salarios y renta
			\3 Crecimiento de estado estacionario
				\4 Economía alcanza estado estacionario futuro
				\4[] Tasas de variación dejan de depender del tiempo
				\4[] $\to$ Sólo de proceso exógeno tecnológico
				\4 En estado estacionario:
				\4[] Renta captura todo el beneficio
				\4 Beneficios nulos
				\4[] $\to$ Sin incentivos a acumulación de capital
				\4[$\Rightarrow$] Necesario progreso técnico
				\4[$\Rightarrow$] Necesario permitir comercio internacional
				\4 Representación gráfica de estado estacionario
				\4[] \grafica{beneficiomodelodericardo}
		\2 Teoría del valor
			\3 Idea clave
				\4 Modelo de los precios relativos
				\4[] $\to$ ¿Por qué 1 de x se intercambia por 2 de y?
				\4[$\Rightarrow$] El trabajo utilizado determina precios relativos
			\3 Teoría pura del valor
				\4 Misma de Adam Smith para sociedades primitivas
				\4 Trabajo es perfectamente homogéneo
				\4 Ratio de uds. de trabajo $\to$ precio relativo
				\4[] Ciervo: 1 ud. de trabajo, castor: 2 ud. de trabajo
				\4[] $\Rightarrow$ 2 ciervos por cada castor
				\4 En sociedades modernas, mayor complejidad
				\4[] $\to$ Aparecen tres problemas
				\4[1] La renta es un coste y es diferente para cada bien
				\4[2] El trabajo no es homogéneo
				\4[3] Los bienes utilizan diferentes cantidades de capital
				\4[] $\to$ Teoría del Valor-Trabajo trata de resolver
			\3[1] Renta es un coste
				\4 Margen extensivo:
				\4[] Ultima unidad de tierra utilizada
				\4[] No se paga renta por utilizarla
				\4 Margen intensivo
				\4[] Última unidad de trabajo utilizada
				\4[] No se paga renta por utilizarlo
				\4 Solución al problema de la renta
				\4[] $\to$ Asumir que sólo es relevante el margen
				\4[] $\Rightarrow$ Renta no influye en coste $\Rightarrow$ precio relativo
			\3[2] Trabajo no homogéneo
				\4 En general, se ignoran diferencias
				\4 Influencia siglo XVIII:
				\4[] Todo el trabajo tiene el mismo potencial
				\4[] Es la educación (=capital) la que modifica trabajo
			\3[3] Diferentes cantidades de capital
				\4 Gran obstáculo a la Teoría del Valor-Trabajo
				\4[] Ricardo intentó sin éxito resolver
				\4[] Sólo soluciones parciales
				\4 Formulación del problema:
				\4[] El capital se utiliza para adelantar salarios un periodo
				\4[] Precio del bien 1: $p_1 = w\cdot a_1 \cdot (1+r_1)$
				\4[] Precio del bien 2: $p_2 = w\cdot a_2 \cdot (1+r_1)$
				\4[] $\Rightarrow$ Precio relativo: $\frac{p_1}{p_2} = \frac{a_1}{a_2}$
				\4[] $\Rightarrow$ Trabajo relativo explica precios relativos
				\4[] Bien 3: intensivo en capital, necesario adelantar 2 periodos
				\4[] Precio del bien 3: $p_3 = w \cdot a_3 \cdot (1+r_1)(1+r_2)$
				\4[] $\Rightarrow$ Precio relativo: $\frac{p_1}{p_3} = \frac{a_1}{a_3\cdot (1+r_2)}$
				\4[] $\Rightarrow$ Trabajo relativo no basta para explicar precios
				\4 Múltiples intentos fallidos de solucionar este problema
			\3 Oferta inelástica y monopolio
				\4 Teoría del valor-trabajo no es aplicable
		\2 Comercio internacional
			\3 Ventaja comparativa
				\4 Costes absolutos inferiores en todos bienes
				\4[] $\to$ Comercio int. sigue siendo beneficioso
				\4 Portugal
				\4[] Vino: 1 unidad de trabajo
				\4[] Tela: 1 unidad de trabajo
				\4 Inglaterra
				\4[] Vino: 4 unidades de trabajo
				\4[] Tela: 2 unidades de trabajo
				\4 Precios
				\4[] Portugal: 1 vino/tela
				\4[] Inglaterra: 0.5 vino/tela
				\4[] $\Rightarrow$ tela más barata en Inglaterra
				\4[] Inglaterra tiene ventaja comparativa en tela
				\4[] $\Rightarrow$ vino más barato en Portugal
				\4[] Portugal tiene ventaja comparativa en vino
			\3 Patrón de especialización
				\4 Inglaterra producirá tela y venderá a Portugal
				\4 Portugal producirá vino y lo venderá a Inglaterra
			\3 Beneficios del comercio
				\4 Reduce rentas de propietarios
				\4 Aumenta ingreso nacional
			\3 Escape a estado estacionario
				\4 Tierra se dedican a fines más productivos
				\4 Cambio en uso de la tierra
				\4[] Tierras poco productivas en uso agrícola
				\4[] $\to$ Otros fines más productivos
				\4[] $\then$ Reducción de rentas
				\4 Aumentan beneficios
				\4[] Aumentan incentivos a inversión
		\2 Política monetaria
			\3 Contexto histórico
				\4 Convertibilidad suspendida 1797-1821
				\4[] Tras pánico bancario por miedo a invasión napoleónica
				\4 Guerras napoleónicas
				\4 Inflación relativamente elevada
			\3 Controversia bullionista
				\4 Panfleto de 1809
				\4 Emisión de papel moneda sin convertibilidad
				\4[] $\then$ Emisión excesiva
				\4[] $\then$ Aumento de la inflación
				\4 Ricardo apoya teoría cuantitiva del dinero
				\4[] Frente a Banking School
				\4 Defiende convertibilidad de billetes en oro
				\4[] Para evitar emisión excesiva de notas de banco
			\3 Critica a banking school
				\4 Doctrina de las \textit{real bills}
				\4[]  Adam Smith y otros
				\4[]  Bancos deben descontar sólo letras ``reales''\footnote{Es decir, letras con suficiente garantía de repago y generalmente de corto plazo. En términos más sencillos, los bancos deben prestar sólo a deudores solventes y a corto plazo.}
				\4 Law of Reflux / Ley del Reflujo
				\4[] No es posible emitir más dinero del demandado
				\4[] $\to$ Exceso de emisión fluye de vuelta al banco emisor
				\4 Diferentes canales de reflujo
				\4[] Conversión en metal precioso
				\4[] Depósito en banco
				\4[] Depósito en otro banco y compensación en cámara
			\3 Defensor de la Currency School
				\4 ``Un Ensayo sobre el Alto Precio del Lingote''
				\4 Inflación en Inglaterra
				\4[] Resultado de elevada emisión de billetes de banco
				\4 Teoría cuantitativa del dinero
				\4[] Defiende implícitamente
				\4[] Causalidad de M a P e Y
		\2 Política fiscal y tributación
			\3 Impuestos
				\4 Mal en cualquiera de sus formas
				\4 Lamentablemente, necesarios en algunas ocasiones
				\4 Sin teoría explícita de tributación óptima
			\3 Deuda
				\4 Plantea posibilidad de equivalencia ricardiana
				\4[] Aunque rechaza considerar seriamente
			\3 Minimizar gasto público
				\4 Sea cual sea su forma de financiación
				\4[] Perjudica a la acumulación de capital
				\4[] Perjudica al trabajo
	\1 \marcar{John Stuart Mill}
		\2 Trayectoria
			\3 Vida
				\4 1806-1873
			\3 Influenciado por
				\4 Adam Smith
				\4 James Mill
				\4 Jeremy Bentham
				\4[] Utilitarista radical en sus inicios
				\4[] Utilitarismo: eje de su teoría
				\4 Ricardo
			\3 Influenció a
				\4 Utilitarismo posterior
				\4 Alfred Marshall
			\3 Obras
				\4[] Ensayos sobre algunas cuestiones no resueltas en economía política (1844)
				\4 Principios de Economía Política (1848)
				\4[] Sobre La Libertad (1859)
				\4[] Utilitarianismo (1863)
		\2 Principios de Economía Política (1848)
			\3 Idea clave
				\4 Libro de texto principal hasta Marshall
				\4 Recapitulación y extensión de Smith, Ricardo, Mill
				\4 Énfasis sobre los mecanismos
				\4[] $\to$ Explicar transiciones entre equilibrios
				\4 Refinamiento de modelos clásicos
				\4 Análisis de casi todos los temas del clasicismo
			\3 Contribuciones propias:
				\4[] $\to$ Filosofía política
				\4[] $\to$ Instituciones como determinante de progreso
				\4[] $\to$ Libertad del individuo
				\4[] $\to$ Educación como motor de progreso
			\3 Laissez-faire
				\4 En general, prefiere el laissez-faire
				\4 Excepciones
				\4[$\to$] Similares a las de Smith
				\4[$\to$] Énfasis en educación
				\4 Examen del socialismo como posibilidad
				\4[] No es un socialista
				\4[] Pero lo considera idea a tener en cuenta
		\2 Teoría del valor
			\3 Acepta TVT de Ricardo
				\4 Acepta como generalmente concluyente
				\4 Jevons criticará en 1871 influencia de J.S. Mill
				\4[] $\to$ ``perniciosa influencia de la autoridad''
			\3 Valor es concepto relativo, no absoluto
				\4 Valor es capacidad de compra
				\4[] sobre resto de bienes
				\4[] con precios relativos constantes
				\4 Abandona búsqueda de valor absoluto
				\4 Valor de uso determina valor de intercambio
				\4[] $\to$ Sólo excepcionalmente
			\3 Demanda y oferta
				\4 Precio de equilibrio
				\4[] $\to$ Igualdad de demanda y oferta
				\4 Tres tipos de oferta
				\4[i] Perfectamente inelástica
				\4[] $\to$ Demanda determina precio
				\4[ii] Perfectamente elástica
				\4[] $\to$ Coste de producción determina precio
				\4[iii] Relativamente elástica
				\4[] Coste de producción con matices
				\4 Distingue entre corto y largo plazo
		\2 Distribución de la renta
			\3 Depende de instituciones, no de leyes
				\4 Distribución de la renta no sigue reglas inexorables
				\4[$\to$] Individuos y gobierno pueden alterar
				\4 Relación compleja entre distribución y producción
				\4[] A priori, técnica independiente de economía
				\4[] Posteriormente analiza interrelaciones
				\4[] Distribución de renta afecta al producto
			\3 Propiedad privada frente a propiedad común
				\4 Ventajas e inconvenientes de propiedad común
				\4 Problemas de incentivos en ambos casos
				\4[] $\to$ Free-riding en propiedad común
				\4[] $\to$ Problema de agencia en propiedad privada
			\3 Rechaza derecho a la herencia
				\4 Derecho a la propiedad
				\4[] $\to$ no implica derecho a la herencia
				\4 En general contrario a herencia
			\3 Beneficio empresarial
				\4 Remuneración por:
				\4[] Interés / abstinencia
				\4[] Riesgo
				\4[] Salario del gestor
				\4 Controlando por los tres factores
				\4[] $\to$ igual beneficio en todas las industrias
		\2 Inversión, ahorro y crecimiento
			\3 Tendencia decreciente de la tasa de beneficio
				\4 Mayor educación implica mayor racionalidad
				\4 Mayor racionalidad implica menor descuento del futuro
				\4 Mayor racionalidad implica más ahorro
				\4 Más civilización implica menores riesgos
				\4[] $\Rightarrow$ Tasa de interés tiende a decrecer
				\4 Menor tasa de beneficio
				\4[] Reduce tasa de inversión
				\4[] $\then$ Menos acumulación de capital
				\4[] $\then$ Menos crecimiento
			\3 Futuro con crecimiento cero
				\4 Fin de la lucha por mayor riqueza relativa
				\4 Estado en el que nadie es pobre
				\4[] $\to$ Nadie desea mayor riqueza
			\3 Ley de Say
				\4 Admite incumplimientos temporales
				\4[] $\to$ Entiende Ley de Say como ecuación de Say
				\4[] $\to$ Posible exceso de oferta temporal
				\4 Rechaza críticas de Ley de Say como Malthus
				\4[] $\to$ Las entiende como obstáculo a progreso
		\2 Política monetaria
			\3 Tipo de interés y tipo de cambio
				\4 Extiende mecanismo flujo-especie de Hume
				\4 Aumento del tipo de interés
				\4[] Vía:
				\4[] $\to$ Venta de letras
				\4[] $\to$ Aumento del tipo de descuento del Banco de Inglaterra
				\4[] $\then$ Capital entra el país
				\4[] $\then$ Aumenta stock de oro/oferta monetaria
				\4 Reducción de tipo de interés
				\4[] Vía descuento de letras por Banco Central
				\4[] $\to$ capital de corto plazo sale del país
				\4[] $\Rightarrow$ ajuste del tipo de cambio
				\4 Implicación de política monetaria
				\4[] $\to$ Posible aumentar tipos de interés para proteger reservas
			\3 Bullionismo, banking school y currency school
				\4 En tiempos normales
				\4[] $\to$ Banking School
				\4 En situaciones de booms de crédito
				\4[] $\to$ Necesario limitar emisión de papel moneda
				\4[] $\to$ Currency school en estos casos
		\2 Comercio internacional
			\3 Ecuación de Demanda Internacional
				\4 Asumiendo balanzas comerciales en equilibrio
				\4 Valor de exportaciones e importaciones
				\4[] $\to$ Debe igualarse
				\4[$\Rightarrow$] RRI tal que cuenta comercial en equilibrio
				\4 Precursor de curvas de oferta recíproca de Marshall
			\3 Ganancias del comercio
				\4 País con demanda de importación más elástica
				\4[] $\to$ Obtiene más beneficio de comercio
				\4 Rechaza Leyes del Grano (Corn Laws)
				\4 Apoyo general al libre comercio
				\4 Salvo industrias nacientes si:
				\4[] $\to$ Presencia de economías de aprendizaje
				\4[] $\to$ Protección temporal
				\4[] $\to$ Industria debe poder madurar por sí sola
		\2 Tributación
			\3 Impuesto sobre la renta
				\4 Progresividad desincentiva esfuerzo
				\4[] $\to$ Se opone a progresividad
				\4[] $\to$ Sólo acepta mínimo exento
			\3 Impuesto de sucesiones
				\4 Favorable a gravar progresivamente
				\4 Herramienta básica de redistribución
		\2 Teoría de la empresa
			\3 Problema de agencia
				\4 Problema de agencia justifica empresa
				\4[] Propietarios deben sufrir riesgo empresarial
				\4[] $\to$ Para que tengan interés en buen funcionamiento
			\3 Instrumento de acumulación de capital
				\4 Énfasis en forma legal y gobernanza
				\4 Empresas como instrumento para acumular capital
				\4[] Sociedades de responsabilidad limitada
				\4[] $\to$ Instrumento legal
				\4[] $\to$ Exigencia de transparencia sobre activos y deuda
				\4[] $\then$ Suficiente garantía para acreedores
				\4[] $\then$ Permite capitalistas provean capital
			\3 Papel en crecimiento
				\4 Crecimiento y progreso promueve aparición de empresas
				\4[] Aumento de la riqueza y el capital
				\4[] Mejora de la capacidad de gestión
				\4[] $\then$ Aparición de nuevas empresas
	\1 \marcar{Karl Marx}
		\2 Trayectoria
			\3 Vida
				\4 1818--1883
				\4 Crece en Alemania
				\4 Exiliado a Francia
				\4 Expulsado de Francia $\to$ Londres
			\3 Influenciado por
				\4 Aristóteles
				\4 Hegel
				\4 Ricardo
				\4[] Teoría del valor trabajo
				\4 Smith
				\4[] Distinción entre valor de uso e intercambio
				\4 Banking School
				\4 Socialistas utópicos
				\4 Simonde de Sismondi
				\4 Saint-Simon
			\3 Influenció a
				\4 Marxismo posterior
				\4 Socialismo marxista
				\4 Comunismo
				\4 Post-keynesianos
				\4[] $\to$ Desequilibrio
				\4 Austriacos
				\4[] $\to$ Enfoque dinámico
				\4 Keynes
				\4[] Salarios no se ajustan a oferta y demanda
				\4 Incentivos a refutar con nuevas ideas
			\3 Obras
				\4 Manifiesto Comunista\footnote{Con Engels.} (1848)
				\4 El Capital -- Volumen I (1867)
				\4 El Capital -- Volúmenes II y III (póstumos)
				\4 La ideología alemana
		\2 Contexto económico y filosófico
			\3 Materialismo
				\4 Aristóteles
				\4[] Cuando dos cosas se intercambian
				\4[] $\to$ Es porque tienen igual valor
				\4 Hegel
				\4[] Condiciones materiales determinan realidad
			\3 Dialéctica e historia
				\4 Ser humano como ser social
				\4[] Relaciones sociales de producción definen sociedad
				\4[] Contexto social malea individuo
				\4[] $\to$ Modula percepción de los intereses en juego
				\4[] $\to$ No es un átomo que decide de forma autónoma
				\4[$\then$] Necesario contextualizar sistema económico
				\4[] $\to$ Relaciones económicas no son atemporales
				\4[] $\to$ Dependen de momento histórico y condiciones sociales
				\4 Cambio en las relaciones sociales
				\4[] Proceso dinámico
				\4[] Tesis + antítesis $\to$ síntesis
			\3 Burguesía y ciencia económica
				\4 Crea término 'economía clásica'
				\4 Economía clásica
				\4[] $\to$ Justificación del sistema por la burguesía
				\4[] $\to$ Herramienta para mantener sistema
				\4 Critica clásicos por:
				\4[] i) No definir naturaleza de capital y beneficio
				\4[] ii) No reconocer carácter histórico del capitalismo
				\4[] iii) No reconocer explotación de trabajadores
				\4[] $\to$ se centran en intercambio y no en producción
				\4[$\Rightarrow$] Encuadre como ``clásico'' discutido
		\2 Teoría del valor
			\3 Idea clave
				\4 Parte de Smith y sobre todo Ricardo
				\4[] $\to$ Teoría del Valor-Trabajo
				\4 Critica y extiende
			\3 Valor y precio
				\4 Valor de uso de un bien
				\4[] Capacidad para satisfacer necesidades humanas
				\4 Valor de intercambio
				\4[] Aparece cuando hay comercio
				\4[] Aristóteles:
				\4[] $\to$ Dos cosas se intercambian cuando tienen igual valor
				\4[] Propiedades físicas no pueden determinar igual valor
				\4[] $\to$ Porque son muy diferentes entre mercancías
				\4[] Debe existir algo en común a toda mercancía
				\4[] $\to$ Trabajo
				\4[] Medida del valor de intercambio
				\4[] $\to$ Trabajo 'socialmente necesario' para producir
				\4[] Trabajo en condiciones estándar y habilidad medias
				\4[] $\to$ valor absoluto (para Ricardo relativo)
				\4[] $\to$ compra-venta de bienes no crea valor
				\4[] $\then$ Trabajo es única fuente de valor
				\4 Valor de intercambio de trabajo en sí mismo
				\4[] Valor de medios de subsistencia que reproducen trabajo
				\4 Valor de uso del trabajo en sí mismo
				\4[] Capacidad para fabricar otras mercancías
				\4 Precio
				\4[] Relación de intercambio entre mercancías
			\3 Explotación
				\4 Precio del trabajo es menor que valor de uso
				\4[] Capitalistas pagan precio del trabajo
				\4[] Trabajo se aplica a fabricación de mercancías
				\4[] $\to$ Fabricación es valor de uso
				\4[] $\to$ Mercancías tienen mayor precio en mercado
				\4[] $\then$ Extraen la diferencia como plusvalía
				\4 Precio del trabajo o valor social del trabajo
				\4[] Determinado por:
				\4[] $\to$ factores culturales
				\4[] $\to$ poder de los capitalistas
				\4[] Valor necesario para reproducir mano de obra
				\4 Plusvalía
				\4[] Diferencia entre:
				\4[] $\to$ Medida monetaria por la que se intercambia el producto
				\4[] $\to$ Precio del trabajo/capital variable
				\4[] $\to$ Precio del capital fijo
				\4[] $\then$ $\text{Valor de mercancía} = c_v + c_f + \text{plusvalía}$
				\4[] $\then$ $\text{plusvalía} = \text{Valor de mercancía} - c_v - c_f$
				\4 Tasa de explotación
				\4[] $\to$ Relación entre plusvalía y precio de trabajo
				\4[] $T_\text{explotación} = \frac{\text{Plusvalía}}{\text{Precio del trabajo}}$
			\3 Capital
				\4 Capital constante
				\4[] Trabajo acumulado en forma de herramientas, máquinas..
				\4 Capital variable
				\4[] Trabajo 'directo' aplicado al proceso productivo
				\4 Composición orgánica del capital
				\4[] Relación entre capital constante y variable
				\4[] Si composición orgánica alta
				\4[] $\to$ Proporción elevada de capital constante
				\4[] Si composición orgánica baja
				\4[] $\to$ Proporción elevada de capital variable
			\3 Problema de la transformación\footnote{Ver \href{https://core.ac.uk/reader/190866304}{Bieri (2007)}.}
				\4 Reconciliar cuatro proposiciones
				\4[i] Precios relativos dependen de valor relativo
				\4[] Los bienes se venden por su valor
				\4[] Valor depende de cantidad de trabajo incorporado
				\4[ii] Capitalistas derivan plusvalía de capital variable
				\4[] Más trabajo variable $\to$ más plusvalía
				\4[] $\text{Tasa de beneficio} = \frac{s}{c_v+c_f}$
				\4[] $c_f, c_v$: capital fijo y capital variable
				\4[] $\then$ Tasa de beneficio depende de comp. orgánica
				\4[] Si comp. orgánica elevada:
				\4[] $\to$ Tasa de beneficio baja
				\4[] Si comp. orgánica baja:
				\4[] $\to$ Tasa de beneficio alta
				\4[] $\then$ Más trabajo directo debería implicar más beneficio
				\4[iii] Diferentes industrias, difer. comp. orgánica de K
				\4[iv] Tasas de beneficio se igualan entre industrias
				\4[] Observación empírica
				\4[] Industrias intensivas en K
				\4[] $\to$ No siempre tienen menos beneficio
				\4[$\to$] ¿Por qué no se observa ``+ CFijo $\then$ -- beneficio''?
				\4[$\then$] Trabajo relativo no se transforma en precios relativos
				\4[$\then$] Problema de la ``transformación''
				\4 Solución de Marx
				\4[] Los precios no reflejan valor en bienes concretos
				\4[] Sectores intensivos en K venden por encima de valor
				\4[] En términos agregados, diferencias se compensan
				\4[] Teoría del valor-trabajo correcta en términos agregados
				\4 Fuertes críticas posteriores
				\4[] Dudas sobre validez interna del modelo
		\2 Crecimiento económico y ciclos\footnote{Ver \href{https://www.socialist.net/marx-s-capital-chapters-23-25-accumulation.htm}{Socialist.net sobre acumulación de capital en Marx}.}
			\3 Ley de Say
				\4 Acepta como condición de equilibrio
				\4 Critica por dificultad de alcanzar equilibrio
				\4 Equilibrio de oferta y demanda agregada
				\4[] Es inestable
				\4 Desequilibrio no genera caos
				\4[] $\to$ Leyes del movimiento aportan regularidad
			\3 Crisis
				\4[1] Suben salarios, lo que reduce beneficio
				\4[2] Se reduce inversión
				\4[3] Cae demanda agregada
				\4[4] Bajan precios
				\4[5] Baja tasa de beneficio
				\4[6] Bajada ulterior de inversión
				\4[7] Crisis se agudiza
				\4[8] Aumenta desempleo
				\4[9] Aumenta ejercito industrial de reserva
				\4[10] Empresas ineficientes desaparecen
				\4[11] Aumenta productividad en términos agregados
				\4[12] Aumenta tasa de beneficio
				\4[13] Aumenta inversión
				\4[1] Suben salarios
				\4[] ...
			\3 Leyes del movimiento
				\4 Hipótesis básicas
				\4[] Inversión aumenta con beneficio
				\4[] Beneficio depende de salarios
				\4[i] Ley de la miseria creciente del proletariado
				\4[] Nivel de vida de trabajadores aumenta
				\4[] Pero posición relativa con capitalistas empeora
				\4[ii] Tasa de beneficio decreciente
				\4[] Tasa de beneficio depende de
				\4[] $\to$ Positivamente de tasa de explotación
				\4[] $\to$ Negativamente de composición orgánica del capital
				\4[] En corto plazo, acumulación de capital
				\4[] $\to$ Para reducir efectos de lucha de clases
				\4[] $\to$ Aumentar poder de mercado
				\4[] En el largo plazo, mecanización creciente
				\4[] $\Rightarrow$ bajada de beneficios
				\4[iii] Concentración industrial creciente
				\4[] Empresas eficientes absorben ineficientes
				\4[$\Rightarrow$] Crisis final del capitalismo
				\4[iv] Crisis cada vez más graves
				\4[] Cada vez menor ratio producción-capital
				\4[] Más sensibilidad a aumento de salarios
				\4[] $\to$ Oscilaciones cada vez mayores
			\3 Política monetaria
				\4 Factores monetarios amplifican ciclos
				\4[] Crisis provocan exceso de demanda de dinero
				\4[] Liquidez desaparece
				\4[] Quiebras, bancarrota
				\4[] Acumulación de saldos monetarios
				\4[] $\to$ Estimulan nuevo ciclo
				\4 Apoya la banking school
				\4[] $\to$ M y V se ajustan a necesidades del comercio
				\4 Teoría monetaria dinámica
		\2 Críticas al modelo de Marx
			\3 Falsabilidad
				\4 Especialmente respecto modelo de crecimiento
				\4 No excluye un estado de la naturaleza futuro
				\4[] Luego no puede demostrarse falsa
				\4 Crisis final del capitalismo no ha llegado aún
				\4[] Entonces, ya llegará
			\3 Problema de la transformación
				\4 Resolución muy poco satisfactoria
				\4 Existen industrias con composiciones órg. muy distintas
				\4 Empresarios de hecho sustituyen trabajo por K
				\4[] $\to$ ¿Por qué lo harían si plusvalía depende de trabajo?
			\3 Remuneración a PMg frente a salarios de subsistencia
				\4 Teoría de explotación asume salarios de subsistencia
				\4[] Remuneración para reproducir trabajo
				\4[] Influencia ricardiana de la remuneración a factores
				\4[] $\to$ Lo necesario para ``dejarlos igual''/reproducir
			\3 Gestión de empresa como factor de producción
				\4 Gestión es factor de producción
				\4 Es necesario remunerar
			\3 Tiempo como factor de producción
				\4 Necesario remunerar abstinencia temporal
				\4 Pago al capital no es plusvalía por explotación
	\1 \marcar{Otros economistas clásicos}
		\2 Jean Baptiste Say
			\3 Trayectoria
				\4 Vida
				\4[] 1767-1832
				\4 Influencias
				\4[] Smith, Mill, J.S. Mill
				\4 Obras
				\4[] Tratado de Economía Política (1803)
			\3 Ley de Say
				\4 Contexto de enorme aumento de la producción
				\4[] $\to$ ¿La demanda será suficiente?
				\4[] $\to$ ¿Se producirán bienes que nadie demandará?
				\4[] $\to$ ¿Todo lo que no se consume es inversión?
				\4[] $\to$ ¿La demanda limitará el crecimiento futuro
				\4 Ley de Say
				\4[] En equilibrio, no habrá excesos de demanda agregada
				\4[] Producción $\to$ Pago a factores $\to$ demanda
				\4[] Interpretación de Keynes:
				\4[] $\to$ Economía tiende al pleno empleo
				\4[] Interpretación habitual:
				\4[] ``la oferta crea su propia demanda''
				\4 Diferentes versiones de la Ley de Say
				\4[] Valor agregado oferta = valor agregado demanda
				\4[] $\to$ Ley de Walras
				\4[] Mercado de dinero siempre está en equilibrio
				\4[] $\to$ No puede haber excesos de demanda de dinero
				\4[] $\to$ No pueden existir excesos DA por definición
				\4[] $\then$ Identidad de Say
				\4[] DAgregada = OAgregada como condición de equilibrio
				\4[] $\to$ Ecuación de Say
				\4[] $\to$ Admite existencia de desequilibrios
				\4[] $\to$ Idea de ajuste hacia equilibrio
				\4[] $\to$ Idea de insuficiencia de demanda
			\3 Teoría del valor
				\4 Pionero en teoría de la utilidad
				\4 Utilidad como fundamento último del valor
				\4 Producción es creación de utilidad
		\2 Senior
			\3 Teoría de la población
				\4 Desarrolla a Malthus
				\4 Límites al crecimiento por previsión de escasez
				\4[$\Rightarrow$] No necesario que escasez tenga lugar
			\3 Metodología
				\4 Economía política es descubrimiento de leyes
				\4 Cambio frente a concepción de Smith
				\4[] Economía política es buen gobierno
		\2 Henry Thornton
			\3 Trayectoria
				\4 1760-1815
				\4 Actividad política y sistema bancario
			\3 Política monetaria
				\4 Padre de la banca central
				\4 Influencia en Keynes, Hayek, Wicksell
				\4 Posición intermedia entre currency y banking-school
				\4 Crítica a currency school
				\4[] No cree que se hayan emitido notas en cantidad excesiva
				\4[] $\to$ Emisión de billetes de banco no aumentó inflación
				\4 Crítica a banking school
				\4[] Niega totalmente real bills doctrine
				\4[] Aunque las letras sean ``reales''
				\4[] $\to$ Si\footnote{Donde $r$ es la productividad del capital y $i$ es el tipo de interés nominal de las letras.} $r>i$, demanda infinita de descuento
				\4[] $\then$ Expansión de la oferta monetaria
				\4[] $\then$ No hay manera de controlar inflación
				\4[] $\then$ Precede Wicksell
				\4[] $\then$ Precede austriacos
				\4 Defensa del bullionismo frente a letras reales
				\4[] Billetes deben ser convertibles en oro
				\4[] Fuerte debate tras suspensión de convertibilidad 1797
				\4[] $\to$ Pánico por rumor tropas francesas en Inglaterra
				\4[] $\then$ Parlamento decreta suspensión de convertibilidad
		\2 Robert Torrens
			\3 Comercio internacional
				\4 Ventaja comparativa
				\4 Libre comercio bajo reciprocidad
				\4[] Rechaza apertura incondicional
			\3 Arancel óptimo
				\4 Primer análisis
				\4 Se adelanta a Mill
		\2 Otros nombres
			\3 Fréderic Bastiat
				\4 Liberalismo en Francia
				\4 Predecesor liberales s.XIX y XX
			\3 Simonde de Sismondi
				\4 Crítica de laissez-faire
				\4 Introduce plusvalía antes que Marx
				\4 Defiende seguro de desempleo
			\3 James Mill
				\4 Influencia sobre Ricardo, Mill
				\4 Anti-bullionismo
				\4[] Contrario a Ricardo
				\4[] Defiende convertibilidad no se restablezca
				\4[] Considera ley del reflujo evita inflación
				\4[] $\then$ Doctrina de las letras reales/real bills
			\3 Henri de Saint-Simon
				\4 Análisis de clases
				\4 Defiende gobierno ``tecnocrático''
				\4[] Centrado en industriales y científicos
				\4 Influyente en socialismo utópico
				\4[] Pero defiende capitalismo y propiedad privada
			\3 Piero Sraffa
				\4 Economía clásica en el siglo XX
				\4 Reexamina y extiende a Ricardo
	\1[] \marcar{Conclusión}
		\2 Recapitulación
			\3 Adam Smith
			\3 Thomas Malthus
			\3 David Ricardo
			\3 John Stuart Mill
			\3 Karl Marx
			\3 Otros economistas clásicos
		\2 Idea final
			\3 Programas de investigación sin continuidad
				\4 Teoría del valor-trabajo
				\4[] Salvo Sraffa
				\4 Marxismo
				\4 Teoría de la renta ricardiana
			\3 Contribuciones duraderas y germen de programas
				\4 Ley de Say
				\4[] Como tendencia al equilibrio
				\4 Búsqueda de teoría del valor
				\4 Crecimiento económico de largo plazo
				\4 Capital humano
				\4 Ciclos
				\4 Ventajas del comercio internacional
\end{esquemal}











































\graficas

\begin{axis}{4}{Dinámica del ingreso per cápita y las tasas de nacimientos y defunción en una economía malthusiana.}{Ingreso per cápita}{TN \\ TD }{salariomalthus}
	% Tasa de Nacimientos
	\draw[-] (0,1.5) -- (3,1.5);
	\draw[-{Latex}] (4,1.5) -- (3,1.5);
	\draw[-{Latex}] (0,1.5) -- (2,1.5);
	\node[right] at (4,1.5){TN};
	
	% Tasa de Defunciones
	\draw[-] (2,1.75) -- (3,1.125);
	\draw[-{Latex}] (4,0.5) -- (3,1.125);
	\draw[-{Latex}] (0,3) -- (2,1.75);
	\node[right] at (4,0.5){TD};
	
	% Equilibrio
	\draw[dashed] (2.4,1.5) -- (2.4,0);
	\node[below] at (2.4,0){$y^*$};
\end{axis}

\begin{axis}{4}{Modelo ricardiano de la renta diferencial y la distribución del ingreso.}{Capital-y-trabajo}{Grano}{modelodericardo}
	
	% Producto medio
	
	\draw[thick] (0,3.9) -- (3.9,1.4);	
	\node[right] at (3.9,1.4){\small Producto medio};	
	\node[left] at (0,2.55){\small PMe};
	
	% Producto marginal
	
	\draw[thick] (0,3.9) -- (3.3,0.4);	
	\node[right] at (3.3,0.4){\small Producto marginal};	
	\node[left] at (0,1.67){\small PMg};
	
	% Salario
	
	\draw[thick] (0,1) -- (4,1);	
	\node[right] at (4,1){\small Salario};
	
	% Distribución del ingreso
	
	% Línea de trabajo utilizado
	
	\draw[-] (2.1,0) -- (2.1,2.55);	
	\node[below] at (2.1,0){\small $N$};
	
	% Trabajo de estado estacionario
	
	\draw[dashed] (2.73,0) -- (2.73,1);	
	\node[below] at (2.73,0){\small $N^*$};
	
	% Separación entre beneficio y renta
	% -> Es la línea horizontal entre el eje de ordenadas y la intersección entre producto marginal y la línea de trabajo utilizado
	
	\draw[-] (0,1.67) -- (2.1,1.67);
	
	% RENTA
	
	\draw[-] (0,2.55) -- (2.1,2.55);	
	\draw[blue, fill=red, opacity=0.2] (0,3.9) -- (0,1.67) -- (2.1,1.67);	
	\draw[pattern=north east lines, pattern color=blue, opacity=0.2] (0,1.67) -- (2.1,1.67) -- (2.1,2.55) -- (0,2.55);
	\node[] at (1,2.1){\small Renta};
	\node[above] at (2.1,2.58){\small R};
	\node[right] at (2.1,1.73){A};
		
	% BENEFICIOS
	
	\draw[blue, fill=green, opacity=0.2] (2.1,1) -- (2.1,1.67) -- (0,1.67) -- (0,1);	
	\node[] at (1,1.3){\small Beneficios};	
	\node[] at (2.25,0.82){K};
	
	% SALARIOS
	
	\draw[blue, fill=yellow, opacity=0.2] (0,0) -- (2.1,0) -- (2.1,1) -- (0,1);	
	\node[] at (1,0.5){\small Salarios};	
\end{axis}

El gráfico muestra la distribución del producto en función del factor variable utilizado. La renta es una variable diferencial que resulta de restar al producto total el pago de los salarios y el beneficio que obtiene el capitalista. De acuerdo con la teoría de los salarios de subsistencia del modelo clásico, la oferta de trabajo es infinitamente elástica al salario de subsistencia por lo que el salario es una cuantía fija en términos de grano, el único bien que consumen los trabajadores. Por ello, la cantidad destinada al pago de salarios depende exclusivamente de la cantidad de trabajo contratado.

El beneficio de los capitalistas es igual al producto de la unidad de capital-y-trabajo que se aplica a la \textit{peor} porción de tierra, multiplicado por la cantidad de capital-y-trabajo utilizado. Es decir, a la productividad marginal multiplicada por la cantidad total de capital-y-trabajo aplicado. Asumiendo que el capital utilizado es únicamente el salario ``adelantado'' a los trabajadores, la cantidad de capital será constante y proporcional al trabajo y por tanto tendremos dosis homogéneas de capital-y-trabajo.

La renta es el residuo que resulta de restar al producto total el pago de salarios y el beneficio de los capitalistas. En tanto que el salario sea inferior al producto marginal, los empresarios obtendrán un beneficio positivo. Sin embargo, este beneficio induce inversión y demanda de trabajo, lo que aumenta la producción pero disminuye la productividad marginal, hasta alcanzarse el punto $N^*$ de estado estacionario. El siguiente gráfico muestra la evolución de la tasa de beneficio.

\begin{axis}{4}{Evolución del beneficio en relación a la cantidad de factor variable utilizado en el modelo distribución de Ricardo.}{Población}{Producto total menos renta}{beneficiomodeloricardo}

	% Salarios totales
	\draw[-] (0,0) -- (4,4);
	
	% Producto total
	\draw[-] (0,0) to [out=75, in=185](4,3.6);
	
	
	% tasa de beneficio
	\node[circle, fill=black, inner sep=0pt, minimum size=5pt] (a) at (2,2) {};
	
	\node[above] at (2,2.94){A};
	
	\node[right] at (2,1.96){W};
	
	\node[below] at (2,0){N};
	
	% trayectoria
	\draw[-] (2,0) -- (2,2);
	\draw[-{Latex}] (2,2) -- (2,2.94);
	\draw[-{Latex}] (2,2.94) -- (2.94,2.94);
	\draw[-{Latex}] (2.94,2.94) -- (2.94,3.4);
	\draw[-{Latex}] (2.94,3.4) -- (3.4,3.4);
	\draw[-{Latex}] (3.4,3.4) -- (3.4,3.53);
	\draw[-{Latex}] (3.4,3.53) -- (3.53,3.53);
	
	% estado estacionario
	
	\draw[dashed] (3.53,3.53) -- (3.53,0);
\end{axis}

\conceptos

\concepto{Ley de Hierro de los Salarios} 

También conocida como Teoría de los Salarios de Subsistencia o Ley de Bronce de los Salarios, afirma que los salarios tienden a un mínimo de subsistencia natural determinado por el grado mínimo y necesario de satisfacción de las necesidades humanas. Así, si los salarios son más altos que este salario mínimo, los trabajadores tendrán más hijos y aumentará la población. Este aumento de la población reducirá los salarios de nuevo hasta ese nivel de subsistencia

\concepto{Ley del Reflujo} 

ver Sproul, 2010.

\concepto{Margen intensivo y extensivo}

De forma general, los conceptos de márgenes intensivo y extensivo hacen referencia al grado máximo de utilización de un factor de producción que implica beneficios no negativos, cuando éste puede ser utilizado en dos dimensiones diferenciadas. En un contexto de mercado de trabajo, el margen extensivo hace referencia al número de trabajadores involucrados en el proceso productivo, mientras que el margen intensivo hace referencia al número de horas trabajadas por trabajador. En un contexto de análisis clásico de las rentas, el margen extensivo hace referencia a la extensión de tierra utilizada, mientras que el margen intensivo hace referencia a la máxima cantidad de trabajo-capital que resulta rentable aplicar a una porción de tierra dada.

\concepto{Problema de la transformación de Marx}

(Ver \url{https://isj.org.uk/marxs-transformation-made-easy/})

<< \textit{On the basis of the analysis of value in the first volume of Capital each industry would have a different “rate of profit”, because each uses a different ratio of living and dead labour (different “organic compositions of capital”, as Marx put it). But in capitalism as it actually exists, profit rates tend to equalise across economies. Marx argues that capital flows between sectors, from those with low rates of profit to those with high ones, as capitalists seek to maximise their profitability. Those goods in sectors with a high rate of profit will be produced in greater quantities, lowering their prices, while those in sectors with a low rate of profit will become scarce and their prices will rise. Over time these changes in price will tend to equalise profit rates between sectors. This process is the “transformation” of values into what Marx called “prices of production”.} >> 

\preguntas

\seccion{Test 2017}
\textbf{2.} Indique cuál de las siguientes afirmaciones es \underline{\textbf{INCORRECTA}} en relación con la obra magna de J. S. Mill (1806-73), de \textit{Principios de Economía Política}:

\begin{itemize}
	\item[a] Subraya la importancia de la elasticidad de la demanda en la teoría del comercio internacional y formula la teoría de la demanda recíproca.
	\item[b] Defiende que la economía política debe orientarse más a establecer un conjunto de reglas normativas que a profundizar en el conocimiento de leyes o verdades positivas.
	\item[c] Analiza los efectos positivos que para un país tiene un la introducción de un arancel.
	\item[d] Discute la Ley de los Mercados de Say aduciendo que la existencia del dinero puede resultar en una situación de exceso general de oferta.
\end{itemize}

\seccion{Test 2016}
\textbf{3.} En la teoría de Marx los valores de las mercancías no coinciden con sus precios por:
\begin{itemize}
	\item[a] diferencias en la composición orgánica del capital en los diversos sectores de la economía.
	\item[b] diferencias en el nivel de competencia en los diversos sectores de la economía.
	\item[c] diferencias en el nivel de sindicalización de los trabajadores en los diversos sectores de la economía.
	\item[d] diferencias en el nivel de regulación en los diversos sectores de la economía.
\end{itemize}

\seccion{Test 2015}
\textbf{1.} Señale la respuesta correcta respecto a los economistas clásicos:
\begin{itemize}
	\item[a] Una idea importante del pensamiento de Adam Smith es que en un contexto competitivo, las decisiones individuales de las personas, guiadas únicamente por sus propios intereses, dan lugar a resultados que maximizan el bienestar de la sociedad. Esta noción ha sido posteriormente confirmada por la Teoría de Juegos, que demuestra que las decisiones individuales y descentralizadas de las personas generan siempre resultados eficientes en el sentido de Pareto.
	\item[b] La evolución demográfica y de la producción agraria de Inglaterra durante el siglo XIX confirmó empíricamente la teoría demográfica de Malthus, que se basaba en la idea de que la población crece más deprisa que la producción de alimentos.
	\item[c] Ricardo sostuvo que el crecimiento a largo plazo de la economía en el estado estacionario era relativamente alto y se basaba en las tasas de inversión relativamente altas de los capitalistas.
	\item[d] Una de las contribuciones de John Stuart Mill a la economía se produjo en la teoría del comercio internacional. En concreto, defendió que el reparto de las ganancias del comercio entre los países dependía de la intensidad relativa de sus respectivas demandas de importación.
\end{itemize}

\seccion{Test 2014}
\textbf{1.} La teoría del valor para Adam Smith se basa en:
\begin{itemize}
	\item[a] El valor de uso de los bienes.
	\item[b] El valor de cambio de los bienes.
	\item[c] El trabajo que incorporan los bienes.
	\item[d] La abundancia o escasez de los bienes.
\end{itemize}

\seccion{Test 2013}
\textbf{1.} La teoría del comercio internacional de J. S. Mill
\begin{itemize}
	\item[a] No hace ninguna aportación significativa al modelo de Ricardo.
	\item[b] Critica las ideas que sirvieron de base a los argumentos proteccionistas de List.
	\item[c] Fue utilizada para justificar la política de preferencias imperiales.
	\item[d] Aporta algunas ideas originales que serían desarrolladas en su día por Marshall.
\end{itemize}

\seccion{Test 2009}
\textbf{2.} La Riqueza de las Naciones de Adam Smith es CORRECTA:
\begin{itemize}
	\item[a] La clave del bienestar social está en el crecimiento económico, que se potencia a través de la división del trabajo.
	\item[b] Los límites de la división del trabajo vienen determinados por el tamaño del mercado y del ``stock de capital''. 
	\item[c] Es necesario suprimir todas aquellas disposiciones normativas que entorpecen la actividad económica.
	\item[d] Todas las anteriores.
\end{itemize}

\seccion{Test 2006}
\textbf{1.} Indique cuál de las siguientes proposiciones acerca de la \textit{Teoría del Valor Trabajo} es CORRECTA:

\begin{itemize}
	\item[a] Se formuló por vez primera en el libro de Carlos Marx ``El Capital''.
	\item[b] Fue utilizada por los economistas clásicos para intentar hallar una medida invariante del valor de las cosas.
	\item[c] Fue ignorada por David Ricardo, que construyó sus proposiciones teóricas exclusivamente a partir de la retribución del capital y de la renta de la tierra.
	\item[d] Ninguna de las anteriores.
\end{itemize}

\notas

\textbf{2017:} \textbf{2.} B

\textbf{2016:} \textbf{3.} A

\textbf{2015:} \textbf{1.} D

\textbf{2014:} \textbf{1.} C

\textbf{2013:} \textbf{1.} D

\textbf{2009:} \textbf{2.} D

\textbf{2006:} \textbf{1.} B

\bibliografia

Mirar en Palgrave\footnote{*: extraídos en carpeta del tema.}:
\begin{itemize}
	\item absolute and exchangeable value
	\item Banking School, Currency School, Free Banking School
	\item British classical economics *
	\item bullionist controversies
	\item bullionist controversies (empirical evidence) *
	\item circular flow *
	\item circulating capital *
	\item classical distribution theories *
	\item classical economics and economic growth *
	\item classical growth model *
	\item classical production theories *
	\item commodity fetishism
	\item commodity money
	\item Corn Laws, free trade and protectionism *
	\item dialectial reasoning
	\item exploitation
	\item labor theory of value
	\item laissez-faire, economists and
	\item land tax
	\item Malthus, Thomas Robert *
	\item malthusian economy
	\item Mandeville, Bernard
	\item Marx, Karl Heinrich
	\item Marxian transformation problem
	\item Marxian value analysis
	\item Marx's analysis of capitalist production
	\item Mill, James
	\item Mill, John Stuart
	\item money, classical theory of *
	\item physiocracy
	\item profit and profit theory
	\item Quesnay, François
	\item real bills doctrine *
	\item real bills doctrine versus the quantity theory *
	\item Ricardo, David *
	\item Say, Jean-Baptiste
	\item Say's law *
	\item Smith, Adam *
	\item Turgot, Anne Robert Jacques, Baron de l'Aulne
	\item utilitarianism and economic theory
	\item wages fund
\end{itemize}



Blaug, M. \textit{Economic Theory in Retrospect} (1997) 5th edition - En carpeta \textit{Historia del Pensamiento Económico}

Butler, E. \textit{The Condensed Wealth of Nations and the Incredibly Condensed Theory of Moral Sentiments} (2011) \url{https://static1.squarespace.com/static/56eddde762cd9413e151ac92/t/56fbaba840261dc6fac3ceb6/1459334065124/Condensed_Wealth_of_Nations_ASI.pdf} -- En carpeta del tema

Cremaschi, S.; Dascal, M. \textit{Malthus and Ricardo on Economic Methodology} (1996) History of Political Economy

Hicks, J.; Hollander, S. \textit{Mr. Ricardo and the Classics} (1977) Quarterly Journal of Economics

Glasner, D. \textit{Classical Monetary Theory and the Quantity Theory} (2000) History of Political Economy -- En carpeta del tema

Glasner, D. \textit{The Real-Bills Doctrine in the Light of the Law of Reflux} (1992) History of Political Economy -- En carpeta del tema

Goodhard, C.; Jensen, M. (2015) \textit{Currency school versus Banking School: an ongoing confrontation} Economic Thought, 4 (2) pp. 20-31 \href{http://eprints.lse.ac.uk/64068/1/Currency\%20School\%20versus\%20Banking\%20School.pdf}{Enlace} -- En carpeta del tema

Le Maux, L. \textit{The Banking School and the Law of Reflux in General} (2012) History of Political Economy -- En carpeta del tema

Robbins, L. \textit{A History of Economic Thought. The LSE Lectures} (1998) -- En carpeta \textit{Historia del Pensamiento Económico}

Rothbard, M. U. \textit{Francis Hutcheson: teacher of Adam Smith} \url{https://mises.org/library/francis-hutcheson-teacher-adam-smith}

Samuels, W. J; Biddle, J. E.; Davis, J. B. \textit{A Companion to the History of Economic Thought} (2003) Ch. 8 Classical Economics -- En carpeta \textit{Historia del Pensamiento Económico}

Screpanti, E; Zamagni, S. \textit{An Outline of the History of Economic Thought} (2005) -- En carpeta \textit{Historia del Pensamiento Económico}

Sproul, M. \textit{The Law of Reflux} (2010) \url{https://mpra.ub.uni-muenchen.de/24813/1/MPRA_paper_24813.pdf} -- En carpeta del tema

\end{document}
\begin{esquema}[enumerate]

\end{esquema}

