\documentclass{nuevotema}

\tema{3B-41}
\titulo{La Unión Europea y la cohesión económica y social: política regional y reforma de los fondos estructurales. Políticas social y de empleo. Implicaciones sobre el proceso de convergencia real en la Unión Europea.}

\begin{document}

\ideaclave

La economía tiene una dimensión espacial que los modelos tradicionales tienden a dejar de lado. Sin embargo, los efectos desiguales del crecimiento económico en unas regiones y otras son un efecto a tener en cuenta por el policy maker. A este efecto surge la política regional, que tiene por objetivo básico lograr o acelerar la convergencia real entre diferentes territorios de una economía. La llamada política de cohesión de la Unión Europea tiene este objetivo. Se articula a través de una serie de fondos tales como el Fondo Europeo de Desarrollo Regional, el Fondo Social Europeo (los llamados Fondos de Cohesión) y el Fondo Europeo Agrario de Desarrollo Rural.

Por otro lado, las condiciones laborales y los llamados derechos sociales son competencia de los EEMM, con el potencial para generar divergencias entre EEMM e incluso competitividad a la baja. Dado que el bienestar de los ciudadanos es parte del objetivo primario de la UE (TUE.3.1) y de los objetivos intermedios, la Unión Europea ha dispuesto en esta materia. Aunque de forma relativamente débil en relación a otros aspectos, se ha logrado establecer una serie de principios mínimos en relación a los derechos sociales de los ciudadanos y residentes de la Unión Europea.

\seccion{Preguntas clave}

\begin{itemize}
	\item ¿Qué son las políticas europeas de cohesión?
	\item ¿En qué consiste la política regional?
	\item ¿Para qué se utilizan los fondos estructurales?
	\item ¿Qué políticas sociales y de empleo lleva a cabo la UE?
	\item ¿Qué implicaciones tienen sobre la convergencia real en la UE?
\end{itemize}

\esquemacorto

\begin{esquema}[enumerate]
	\1[] \marcar{Introducción}
		\2 Contextualización
			\3 Unión Europea
			\3 Competencias de la UE
			\3 Divergencias regionales
			\3 Mercado de trabajo europeo
			\3 Políticas de cohesión económica y social
		\2 Objeto
			\3 ¿Qué son las políticas europeas de cohesión?
			\3 ¿En qué consiste la política regional?
			\3 ¿Para qué se utilizan los fondos estructurales?
			\3 ¿Qué políticas sociales y de empleo lleva a cabo la UE?
			\3 ¿Qué implicaciones tienen sobre la convergencia real en la UE?
		\2 Estructura
			\3 Política regional
			\3 Política social y de empleo
			\3 Implicaciones sobre la convergencia real
	\1 \marcar{Política regional}
		\2 Justificación
			\3 Desigualdad regional
			\3 Dimensión espacial del crecimiento
			\3 Dificultades para atraer inversión
			\3 Economías de escala financieras
			\3 Regulación europea sobre ayudas nacionales
			\3 Problemas relacionados con migración
		\2 Objetivos
			\3 Aprovechamiento de recursos ociosos
			\3 Fomentar desarrollo y competitividad
			\3 Reducir diferencias regionales
			\3 Interacción con políticas sectoriales
		\2 Antecedentes
			\3 Hasta años 70
			\3 Ampliación de 1973
			\3 Creacion del FEDER (1975)
			\3 Finales de los 80
			\3 Maastricht 91-93
			\3 Paquete Delors II (93-99)
			\3 Agenda de Lisboa (2000)
			\3 MFP 2007-2013
			\3 Tratado de Lisboa (2007)
		\2 Marco jurídico
			\3 TUE
			\3 TFUE
			\3 Principios de actuación (GGCCCA)
			\3 Categorías de región
			\3 Objetivos temáticos de los FEIE
			\3 Europa 2020
			\3 MEC 14-20 -- Marco Estratégico Común 2014-2020
			\3 Acuerdos de Asociación
			\3 Programas Operativos
			\3 Reglamentos
		\2 Marco financiero
			\3 MFP 2014-2020
			\3 Fondos EIE -- Estructurales y de Inversión Europeos
			\3 Otros
		\2 Actuaciones
			\3 Actuaciones temáticas comunes
			\3 Reparto de fondos
			\3 Semestre Europeo
			\3 Coacción europea
		\2 Valoración
			\3 Políticas de oferta vs demanda
			\3 Desigualdades regionales
			\3 Consistencia con otras políticas europeas
			\3 Inversión de capital
		\2 Retos
			\3 Muy pequeño tamaño relativo
			\3 Política regional en contexto de crisis
			\3 Acceso a fondos por regiones ricas
			\3 Dinámicas de aglomeración
			\3 MFP 2021-2027
	\1 \marcar{Política social y de empleo}
		\2 Justificación
			\3 Dinámicas de aglomeración
			\3 Perdedores de desarrollo e integración
			\3 Necesario garantizar niveles mínimos
			\3 Desempleo elevado
		\2 Objetivos
			\3 Compensar a perdedores de integración
			\3 Garantizar niveles mínimos
			\3 Mejorar coordinación de políticas nacionales
			\3 Mejorar funcionamiento de mercados laborales
			\3 Estrategia 2020
		\2 Antecedentes
			\3 Tratado de Roma (1957)
			\3 Carta Social Europea de 1961
			\3 PAS -- Programa de Acción Social de 1974
			\3 Estancamiento en los 80
			\3 Acuerdo de Schengen de 1985
			\3 Acta Única de 1987
			\3 Carta Comunitaria de Derechos Sociales fundamentales de 1989
			\3 Convención de Schengen de 1990
			\3 Tratado de Maastricht de 1992-3
			\3 Tratado de Amsterdam de 1997
			\3 Agenda 2000
			\3 Tratado de Lisboa
		\2 Marco jurídico
			\3 Principios de actuación
			\3 Carta Social Europea de 1961 y revisada en 1998
			\3 TUE
			\3 TFUE
			\3 Estrategia Europa 2020
			\3 Employment Guidelines/Orientaciones generales sobre el empleo
			\3 JER -- Informe Conjunto sobre el Empleo
			\3 NRPs -- Programas Nacionales de Reforma
			\3 CSR -- Recomendaciones Específicas por País
			\3 Reglamentos de FSE y Fondos EIE
		\2 Marco financiero
			\3 Fondo Social Europeo
			\3 EaSI
			\3 FEAG
			\3 FEAD
			\3 IEJ/YEI
		\2 Actuaciones
			\3 JEL -- Orientaciones generales para el empleo
			\3 Fondo Social Europeo
			\3 Youth Employment Initiative
			\3 Skills Agenda de 2016 -- Comunicación de la Comisión
			\3 EGF -- European Globalisation Adjustment Fund
			\3 EaSI -- Employment and Social Innovation
			\3 SURE -- Support to mitigate Unemployment Risks in an Emergency
			\3 Intercambio de buenas prácticas
			\3 Fomento de la movilidad
			\3 Propuestas para 2021-2027
		\2 Valoración
			\3 Pre-crisis y post-crisis
			\3 Dificultades de valoración
			\3 Comparación con resto del mundo
		\2 Retos
			\3 Desempleo juvenil
			\3 Resultados desiguales
			\3 Flexibilidad del mercado laboral
	\1 \marcar{Implicaciones sobre la convergencia real}
		\2 Idea clave
			\3 Concepto de convergencia
			\3 Aspectos teóricos
		\2 Evidencia empírica sobre convergencia real
			\3 Hasta 90s
			\3 Post-90s
			\3 Recientemente
			\3 Convergencia norte-sur
			\3 Convergencia oeste-este
			\3 Convergencia centro-periferia
			\3 Efectos de crisis sobre convergencia
			\3 Sincronización de ciclo económico
		\2 Impacto de las políticas de cohesión
			\3 Dificultades de medición
	\1[] \marcar{Conclusión}
		\2 Recapitulación
			\3 Política regional
			\3 Política social y de empleo
			\3 Implicaciones sobre la convergencia real
		\2 Idea final
			\3 Séptimo Informe sobre Cohesión
			\3 Futuro de UE

\end{esquema}

\esquemalargo
















\begin{esquemal}
	\1[] \marcar{Introducción}
		\2 Contextualización
			\3 Unión Europea
				\4 Institución supranacional ad-hoc
				\4[] Diferente de otras instituciones internacionales
				\4[] Medio camino entre:
				\4[] $\to$ Federación
				\4[] $\to$ Confederación
				\4[] $\to$ Alianza de estados-nación
				\4 Origen de la UE
				\4[] Tras dos guerras mundiales en tres décadas
				\4[] $\to$ Cientos de millones de muertos
				\4[] $\to$ Destrucción económica
				\4[] Marco de integración entre naciones y pueblos
				\4[] $\to$ Evitar nuevas guerras
				\4[] $\to$ Maximizar prosperidad económica
				\4[] $\to$ Frenar expansión soviética
				\4 Objetivos de la UE
				\4[] TUE -- Tratado de la Unión Europea
				\4[] $\to$ Primera versión: Maastricht 91 $\to$ 93
				\4[] $\to$ Última reforma: Lisboa 2007 $\to$ 2009
				\4[] Artículo 3
				\4[] $\to$ Promover la paz y el bienestar
				\4[] $\to$ Área de seguridad, paz y justicia s/ fronteras internas
				\4[] $\to$ Mercado interior
				\4[] $\to$ Crecimiento económico y estabilidad de precios
				\4[] $\to$ Economía social de mercado
				\4[] $\to$ Pleno empleo
				\4[] $\to$ Protección del medio ambiente
				\4[] $\to$ Diversidad cultural y lingüistica
				\4[] $\to$ Unión Económica y Monetaria con €
				\4[] $\to$ Promoción de valores europeos
			\3 Competencias de la UE
				\4 Tratado de la Unión Europea
				\4[] Atribución
				\4[] $\to$ Sólo las que estén atribuidas a la UE
				\4[] Subsidiariedad
				\4[] $\to$ Si no puede hacerse mejor por EEMM y regiones
				\4[] Proporcionalidad
				\4[] $\to$ Sólo en la medida de lo necesario para objetivos
				\4 Exclusivas
				\4[] i. Política comercial común
				\4[] ii. Política monetaria de la UEM
				\4[] iii. Unión Aduanera
				\4[] iv. Competencia para el mercado interior
				\4[] v. Conservación recursos biológicos en PPC
				\4 Compartidas
				\4[] i. Mercado interior
				\4[] ii. Política social
				\4[] iii. Cohesión económica, social y territorial
				\4[] iv. Agricultura y pesca \footnote{Salvo en lo relativo a la conservación de recursos biológicos marinos, que se trata de una competencia exclusiva de la UE}
				\4[] v. Medio ambiente
				\4[] vi. Protección del consumidor
				\4[] vii. Transporte
				\4[] viii. Redes Trans-Europeas
				\4[] ix. Energía
				\4[] x. Área de libertad, seguridad y justicia
				\4[] xi. Salud pública común en lo definido en TFUE
				\4 De apoyo
				\4[] Protección y mejora de la salud humana
				\4[] Industria
				\4[] Cultura
				\4[] Turismo
				\4[] Educación, formación profesional y juventud
				\4[] Protección civil
				\4[] Cooperación administrativa
				\4 Coordinación de políticas y competencias
				\4[] Política económica
				\4[] Políticas de empleo
				\4[] Política social
			\3 Divergencias regionales
				\4 Enormes diferencias entre regiones
				\4[] Algunas regiones >200\% renta media EU
				\4[] Otras regiones: <75\% renta media
				\4 Patrón de desigualdad entre países
				\4[] Regiones con rentas bajas
				\4[] $\to$ Predominan en algunos países
				\4[] Regiones con rentas altas
				\4[] $\to$ Especialmente concentradas en torno a capitales nacionales
				\4[] $\to$ Oeste y norte más desarrollado que sur y este
				\4[$\then$] Diferencias de renta generan problemas
				\4[] Desplazamientos de población
				\4[] Despoblación de regiones desfavorecidas
				\4[] Congestión en regiones ricas
				\4[] Inestabilidad política
				\4[] Problemas de envejecimiento y despoblación
				\4[] Crisis más largas y pronunciadas
				\4[] Recursos ociosos
			\3 Mercado de trabajo europeo
				\4 Resultados muy divergentes
				\4[] Elevadas tasas de paro en mediterráneo
				\4[] Cercano a pleno empleo en resto
				\4 Muy segmentado geográficamente
				\4[] Poca movilidad
				\4 Regulación heterogénea
				\4[] A nivel nacional
				\4[] Elevada descoordinación
			\3 Políticas de cohesión económica y social
				\4 Mitigar problemas derivados de:
				\4[] $\to$ Desigualdad interregional
				\4[] $\to$ Desigualdad intraregional
				\4 Evitar situaciones de desprotección social
				\4 Rúbrica 1b del MFP 2014-2020
				\4[] Cohesión para el crecimiento y el empleo
				\4[] $\to$ $\sim$ 34\% del presupuesto
				\4[] $\to$ $\sim$ 350.000 M de €
				\4[] $\then$ Elemento central de política económica europea
		\2 Objeto
			\3 ¿Qué son las políticas europeas de cohesión?
			\3 ¿En qué consiste la política regional?
			\3 ¿Para qué se utilizan los fondos estructurales?
			\3 ¿Qué políticas sociales y de empleo lleva a cabo la UE?
			\3 ¿Qué implicaciones tienen sobre la convergencia real en la UE?
		\2 Estructura
			\3 Política regional
			\3 Política social y de empleo
			\3 Implicaciones sobre la convergencia real
	\1 \marcar{Política regional}
		\2 Justificación
			\3 Desigualdad regional
				\4 Algunas regiones > 200\% PIB medio
				\4 Otras, < 50\% PIB medio
				\4[$\then$] Potenciales problemas
				\4[] Potencial de crecimiento desaprovechado
				\4[] Desafección proyecto europeo
				\4[] Tensiones políticas, comerciales, militares
				\4[] Migraciones
			\3 Dimensión espacial del crecimiento
				\4 Localización del crecimiento es relevante
				\4 Dinámica espacial del crecimiento
				\4[] Tendencias consistentes de aglomeración
				\4[] Modelos teóricos de economía espacial
			\3 Dificultades para atraer inversión
				\4 Carencias en infraestructuras
				\4 Escasa conectividad
			\3 Economías de escala financieras
				\4 UE puede movilizar más fondos
				\4 Canalizar a regiones que necesiten más
				\4[] Aunque EEMM ricos intentan capturar también
			\3 Regulación europea sobre ayudas nacionales
				\4 Ayudas nacionales prohibidas
				\4[] Salvaguardar mercado interior
				\4[] Evitar distorsión de competencia
				\4 Política regional para suplir
				\4[] Decidir a nivel UE donde es adecuado ayudar
			\3 Problemas relacionados con migración
				\4 Presión sobre salarios
				\4 Descapitalización de regiones emisoras
				\4 Saturación de servicios públicos
				\4 Presión sobre precios inmobiliarios
		\2 Objetivos
			\3 Aprovechamiento de recursos ociosos
				\4 Humanos
				\4 Naturales
				\4 Financieros
			\3 Fomentar desarrollo y competitividad
				\4 Especialización de regiones
				\4[] Realizar ventajas competitivas
				\4 Proveer bienes públicos a nivel UE
			\3 Reducir diferencias regionales
				\4 Apoyar especialmente a regiones determinadas
			\3 Interacción con políticas sectoriales
				\4 Transportes
				\4 Investigación y desarrollo
				\4 Medio ambiente
		\2 Antecedentes
			\3 Hasta años 70
				\4 Actuaciones dispersas
				\4[] Fondo Social Europeo
				\4[] Fondos de PAC
				\4[] Banco Europeo de Inversiones
			\3 Ampliación de 1973
				\4 RU, IRL, DIN
				\4 Declive industrial en RU, IRL
				\4 Política regional como contrapeso a PAC
			\3 Creacion del FEDER (1975)
				\4 Abordar desequilibrios regionales
				\4 Apoyo a reconversión industrial
				\4 Política regional aún inexistente
				\4[] FEDER asignado por cuotas
				\4[] EEMM eligen proyectos
				\4 Progresiva creación de política regional
				\4[] Proyectos fuera de cuota nacional
				\4[] Financiación según grado de consecución de objetivos
			\3 Finales de los 80
				\4 Ampliación a GRE, POR, ESP
				\4 Acta Única
				\4 Paquete Delors I (88-92)
				\4[] Política regional comienza propiamente
				\4[] Perspectiva plurianual
				\4[] Incremento de dotaciones de recursos
				\4[] Reglamento común a FEDER, FSE, FEOGA\footnote{FEOGA no existe en la actualidad y ha sido sustituido por el FEAGA -- Fondo Europeo Agrícola de Garantía.}
				\4 Principio de adicionalidad
			\3 Maastricht 91-93
				\4 Introduce Fondos de Cohesión
				\4 Fondo de Cohesión creado oficialmente en 1994
			\3 Paquete Delors II (93-99)
				\4 Aumento de dotación presupuestaria
				\4 Implementa Fondos de Cohesión
				\4 Reglamento de fondos estructurales
				\4[] Fondo de Cohesión
				\4[] IFOP -- Instrumento Financiero de Orientación de la Pesca
			\3 Agenda de Lisboa (2000)
				\4 Explicitó varios objetivos de cohesión
			\3 MFP 2007-2013
				\4 Adaptación a nuevos 10 miembros
				\4 Política regional reformada
				\4[] Limitación de contribuciones netas
				\4[] Asegurar fondos para UE-15
				\4 Aumento de fondos para objetivos de Agenda de Lisboa
			\3 Tratado de Lisboa (2007)
				\4 Explicitó aspecto territorial de pol. regional
				\4[] Implícito hasta ahora
				\4[] Desarrollo espacialmente equilibrado
		\2 Marco jurídico
			\3 TUE
				\4 Art. 3.3 sobre objetivos de la UE
				\4[] Promover cohesión territorial
			\3 TFUE
				\4 Título XVIII sobre cohesión
				\4 Creación de Fondos de Cohesión y FEDER
				\4 Competencia compartida UE y EEMM
			\3 Principios de actuación (GGCCCA)
				\4 \marcar{G}enerales de UE
				\4[] Atribución
				\4[] Subsidiariedad
				\4[] Proporcionalidad
				\4 \marcar{G}estión compartida
				\4[] EEMM diseñan y ejecutan actuaciones
				\4[] Comisión comprueba y audita
				\4 \marcar{C}omplementariedad
				\4[] Complementan intervención de los EEMM
				\4[] No debe contarse con ellos como único instrumento
				\4 \marcar{C}oherencia
				\4[] Con el resto de políticas de la UE
				\4 \marcar{C}ofinanciación
				\4[] Fondos Estructurales y de Inversión Europeos no financian completamente
				\4[] No pueden superar máximo entre 50\% y 85\%
				\4 \marcar{A}dicionalidad
				\4[] Fondos estructurales no deben reemplazar gasto nacional
				\4[] $\to$ No pueden implicar reducción de gasto nacional
			\3 Categorías de región
				\4 Menos desarrolladas
				\4[] PIB < 75\% de media UE
				\4 En transición
				\4[] Entre 75\% y 90\% de media UE
				\4 Más desarrolladas
				\4[] > 90\% de media UE
			\3 Objetivos temáticos de los FEIE
				\4[1] I+D y desarrollo tecnológico
				\4[2] Transformación digital
				\4[3] Competitividad de PYMES
				\4[4] Reducción de emisiones
				\4[5] Cambio climático: prevención y adaptación
				\4[6] Protección del MA y eficiencia
				\4[7] Transporte
				\4[8] Empleo de calidad y movilidad del trabajo
				\4[9] Inclusión social
				\4[10] Educación
				\4[11] Calidad institucional, eficiencia en AAPP
			\3 Europa 2020
				\4 Objetivos globales para la década
			\3 MEC 14-20 -- Marco Estratégico Común 2014-2020
				\4 Marco común para uso de los 5 fondos FEIE
				\4 Objetivos de Europa 2020
			\3 Acuerdos de Asociación
				\4 Acuerdos entre CE y EEMM bilateralmente
				\4 Planificación de uso y distribución de fondos
				\4 Incluye también a partes interesadas
			\3 Programas Operativos
				\4 Diseñados por estados miembros
				\4 Cómo se gastará dinero de FEIE
				\4[] Cuál de los 11 objetivos temáticos
				\4 Regionales
				\4[] Planificación de actuaciones por regiones
				\4 Sectoriales
				\4 Interregionales
				\4[] Cooperación territorial entre EEMM
				\4 En España
				\4[] Uno por comunidad autónoma
				\4[] Uno multirregional
				\4[] Uno para PYMES en marco FEDER
			\3 Reglamentos
				\4 Reglamento 2013/1300: Fondo de Cohesión
				\4 Reglamento 2013/1301: FEDER
				\4 Reglamento 2013/1303: Común a los FEIE
				\4 Reglamento 2013/1304: FSE
				\4 Reglamento 2013/1305: FEADER
				\4 Reglamento 2014/508: FEMP
		\2 Marco financiero
			\3 MFP 2014-2020
			\3 Fondos EIE -- Estructurales y de Inversión Europeos\footnote{Cuantías extraídas de \url{https://cohesiondata.ec.europa.eu/funds}.}
				\4 460.000 M de € en total en presupuesto UE\footnote{Ver \href{https://cohesiondata.ec.europa.eu/overview}{Comisión Europea: Regional funds available budget.}.}
				\4[] 640.000 M de € en total incluyendo financiación
				\4 FEDER -- Fondo Europeo de Desarrollo Regional
				\4[] $\to$ 200.000 M de € en MFP 14-20
				\4[] $\to$ 276.000 M de € con cofinanciación
				\4 Fondo de Cohesión
				\4[] 62.000 M de € en MFP 14-20
				\4[] $\to$ 10.000 M de € de los 74.000 corresponden a CEF\footnote{Connecting Europe Facility}
				\4[] 73.000 M de €
				\4 FSE -- Fondo Social Europeo
				\4[] 83.000 M de € en MFP 14-20 incluyendo cofinanciación
				\4[] 120.000 M de € con cofinanciación
				\4 FEADER -- Fondo Europeo Agrícola de Desarrollo Rural
				\4[] 100.000 M de € en MFP 14-20
				\4[] $\to$ $\sim$ $9\%$
				\4[] 150.000 M de € con cofinanciación
				\4 FEMP -- Fondo Europeo Marítimo y de Pesca
				\4[] 6.000 M de € en UE 14-20
				\4[] 8.000 M de € en total con cofinanciación
				\4 YEI -- Youth Employement Initiative
				\4[] 9.000 M de € en MFP 14-20
				\4[] 10.000 M de € con cofinanciación
			\3 Otros
				\4 EFSI -- European Fund for Strategic Investment
				\4[] Garantías del BEI y CE, FIE
				\4[] Objetivo: >300.000 M € en inversiones
				\4 FIE -- Fondo de Inversión Europeo
				\4[] Participación BEI + privado
				\4[] Financiación de capital riesgo a PYMES
				\4[] No presta directamente, vía bancos
				\4 Fondo de Solidaridad Europeo
				\4[] Ayuda para catástrofes graves
				\4[] 500 millones desde 2000
				\4 Préstamos y garantías del BEI
		\2 Actuaciones
			\3 Actuaciones temáticas comunes
				\4 Competitividad de PYMES
				\4 Transporte
				\4 Protección de medio ambiente
				\4 Empleo e inclusión social
				\4 Instituciones y admón. pública
			\3 Reparto de fondos
				\4[] Todas las categorías de región pueden recibir
				\4[] $\to$ Modulando cofinanciación
				\4 FEDER --  Fondo Europeo de Desarrollo Regional
				\4[] Creado en 1973-1975 por adhesión RU e IRL
				\4[] Todas las regiones pueden recibir
				\4[] Áreas de actuación por importancia:
				\4[1] I+D y desarrollo tecnológico
				\4[2] Transformación digital
				\4[3] Competitividad de PYMES
				\4[4] Reducción de emisiones
				\4[5] Cambio climático: prevención y adaptación
				\4[6] Protección del MA y eficiencia
				\4[7] Transporte
				\4[] Prioridades 1-4 especial relevancia
				\4[] Obligatorio dedicar \% a prioridades 1-4
				\4[] $\to$ Regiones más desarrolladas: 80\%
				\4[] $\to$ Regiones de transición: 60\%
				\4[] $\to$ Menos desarrolladas: 50\%
				\4[] Obligatorio dedicar \% mínimo a reducir emisiones
				\4[] $\to$ Regiones más desarrolladas: 20\%
				\4[] $\to$ Regiones de transición: 15\%
				\4[] $\to$ Regiones menos desarrolladas: 12\%
				\4[] Reconversión industrial
				\4[] Zonas rurales
				\4[] Desventajas demográficas o naturales graves
				\4[] Cooperación territorial
				\4[] Desarrollo urbano sostenible
				\4[] Investigación y desarrollo
				\4[] Nuevas tecnologías medioambientales
				\4[] 50\% dedicado a zonas urbanas en MFP 14-20
				\4 Fondo de Cohesión
				\4[] Objetivos temáticos de actuación
				\4[4] Reducción de emisiones
				\4[5] Cambio climático: prevención y adaptación
				\4[6] Protección del MA y eficiencia
				\4[7] Transporte
				\4[11] Calidad institucional, eficiencia en AAPP
				\4[] Creado en 1994
				\4[] Países con RNBpc < 90\%
				\4[] Reforzar cohesión económica
				\4[] Inversión en medio ambiente
				\4[] RTE-T
				\4[] Conectar Europa
				\4[] Asistencia técnica
				\4[] 10.000 M de € aportados a Connecting Europe Facility
				\4[] Polonia es mayor receptor
				\4[] $\to$ 23.000 M de € en 2014-2020
				\4 FEMP
				\4[] España es principal beneficiario
				\4[] Objetivos temáticos comunes por importancia
				\4[1] I+D y desarrollo tecnológico
				\4[3] Competitividad de PYMES
				\4[4] Reducción de emisiones
				\4[6] Protección del MA y eficiencia
				\4[8] Empleo de calidad y movilidad del trabajo
				\4[9] Inclusión social
				\4 FEADER
				\4[] \href{https://eur-lex.europa.eu/legal-content/EN/TXT/HTML/?uri=CELEX:32013R1305&from=en}{Reglamento 2013/1305}
				\4[] Áreas de actuación por importancia
				\4[1] I+D y desarrollo tecnológico
				\4[3] Competitividad de PYMES
				\4[4] Reducción de emisiones
				\4[5] Cambio climático: prevención y adaptación
				\4[6] Protección del MA y eficiencia
				\4[9] Inclusión social
				\4[] RDPs--Programas de Desarrollo Rural
				\4[] $\to$ Diseñados nacional o regionalmente
				\4[] $\to$ Al menos 4 de las 6 prioridades del FEADER
				\4[] Comisión Europea:
				\4[] $\to$ Supervisa y aprueba RDP
				\4[] Autoridades nacionales o regionales
				\4[] $\to$ Seleccionan proyectos y aprueban pagos
				\4[] 30\% de fondos en RDP a Protección MA y Cambio Climático
				\4[] 5\% de fondos para LEADER\footnote{Método de desarrollo local en el que las comunidades locales deciden el alcance y el contenido de los proyectos. }
			\3 Semestre Europeo
				\4 Calendario de coordinación de política regional
				\4 Orientación de reformas económicas y sociales
				\4 Seguimiento de reformas
				\4[$\then$] Recomendaciones Específicas por País
				\4[] CSR -- Country Specific Recommendations
				\4[$\then$] Programas Nacionales de Reformas
				\4[] NRP -- National Reform Programmes
			\3 Coacción europea
				\4 Incumplimiento de obligaciones con UE
				\4[] Suspender pagos a EEMM
				\4[] Casos de incumplimiento de obligaciones y calendario
		\2 Valoración
			\3 Políticas de oferta vs demanda
				\4 Objetivo declarado centrado en oferta
				\4 En ocasiones
				\4[] $\to$ Se convierten en políticas de demanda
			\3 Desigualdades regionales
				\4 Persistentes en algunos EEMM y regiones
			\3 Consistencia con otras políticas europeas
				\4 Elevada complejidad de sistema
				\4 Solapamientos e ineficiencia en ocasiones
				\4 Elevado número de regiones y destinatarios
				\4[] Aumenta cargas administrativas de supervisión
				\4[] Reduce eficiencia de uso de fondos
			\3 Inversión de capital
				\4 Crisis redujo fuertemente inversiones pública
				\4 Fondos EIE sostuvieron niveles de inversión
				\4[] Hasta 10\% de inversión de K en 2015-2017 en ESP
				\4[] Casi 60\% en POR
		\2 Retos
			\3 Muy pequeño tamaño relativo
				\4 Apenas 0,3\% de PIB total europeo
				\4 Sin impacto en términos de estabilización
				\4[] Aunque fondos a menudo se desvían para ello
				\4[] $\to$ Problema clave de política de cohesión
			\3 Política regional en contexto de crisis
				\4 Menor disposición a contribuir
				\4 Descontento opinión pública
				\4[] Recortes nacionales unidos a contribuciones netas
			\3 Acceso a fondos por regiones ricas
				\4 Vía para compensar aportaciones netas
				\4 Economía política
			\3 Dinámicas de aglomeración
				\4 ¿Aglomeración es escenario deseable?
				\4 Ciudades de tamaño medio
				\4[] ¿Deben recibir apoyo?
			\3 MFP 2021-2027\footnote{Ver \href{https://www.europarl.europa.eu/RegData/etudes/BRIE/2018/625141/EPRS_BRI(2018)625141_EN.pdf}{EPRS (2020)} sobre FEDER y FCohesión en 2021-2027 (en carpeta del tema). }
				\4 Reducción de fondos totales
				\4 Reducción de objetivos de 11 a 5
				\4[1] Innovación y transformación industrial
				\4[2] Transformación ecológica
				\4[3] Transporte y conectividad digital
				\4[4] Política social
				\4[5] Desarrollo urbano y rural
				\4 FEDER
				\4[] Gasto en los 5 objetivos
				\4[] Énfasis en 1 y 2
				\4 Regulación común para FEDER y FCohesión
				\4 Centrarse en prioridades centrales
				\4[] Tecnologías digitales
				\4[] Modernización industrial
				\4[] Descarbonización
				\4[] Cambio climático
				\4 Mantener fondos para todas las regiones
				\4 Mejorar definición de objetivos
				\4 Reducir carga administrativa
				\4 Simplificar trámites
				\4 Más coordinación vía Semestre Europeo
	\1 \marcar{Política social y de empleo}
		\2 Justificación
			\3 Dinámicas de aglomeración
				\4 Proceso dinámico que se retroalimenta
				\4 Más población concentrada
				\4[] Mayores salarios en áreas centrales
				\4[] $\to$ Menos salarios en zonas periféricas
				\4[] $\then$ Migración hacia centro
			\3 Perdedores de desarrollo e integración
				\4 Integración económica europea
				\4[] Deslocalizaciones de empresas
				\4[] Mayor competencia
				\4[] $\to$ Asignación más eficiente de recursos
				\4[] $\then$ Pero también pérdidas muy concentradas
			\3 Necesario garantizar niveles mínimos
				\4 Evitar conflictos sociales
				\4 Evitar desaprovechar potencial de crecimiento
			\3 Desempleo elevado
				\4 Especialmente en algunos EEMM
				\4 Funcionamiento deficiente de mercado laboral
				\4 Desempleo causa problemas adicionales
				\4[] Deterioro de capital humano
				\4[] Presión sobre cuentas públicas
				\4[] Conflictividad social
		\2 Objetivos
			\3 Compensar a perdedores de integración
				\4 Evitar desafección proyecto europeo
				\4 Reducir tensiones políticas y económicas
			\3 Garantizar niveles mínimos
				\4 Renta
				\4 Igualdad de oportunidades e inclusión social
				\4 Condiciones laborales
				\4 Educación
			\3 Mejorar coordinación de políticas nacionales
				\4 Intercambio de conocimiento sobre diseño de políticas
				\4 Trasladar políticas exitosas a otros EEMM
				\4 Evitar competencia a la baja en políticas sociales
			\3 Mejorar funcionamiento de mercados laborales
				\4 Fomentar flexibilización
				\4 Proveer incentivos compatibles con reducción desempleo
			\3 Estrategia 2020
				\4 Explicita objetivos
		\2 Antecedentes
			\3 Tratado de Roma (1957)
				\4 Recogía objetivos fundamentales
				\4[] Mejora condiciones de vida y trabajo
				\4[] Igualdad de remuneración
				\4 Creación de Fondo Social Europeo
				\4 Base legal para avances posteriores
			\3 Carta Social Europea de 1961
			\3 PAS -- Programa de Acción Social de 1974
				\4 Iniciativas contra pobreza
				\4 Apoyo a trabajadores desplazado
				\4 Clima de descontento tras Crisis del Petróleo
				\4 Creación de centro de formación profesional europeo
			\3 Estancamiento en los 80
				\4 Políticas nacionales muy diversas
				\4 Gobierno Thatcher contrario a acción europea sobre social y empleo
			\3 Acuerdo de Schengen de 1985
				\4 Acuerdo político
				\4 Eliminar barreras movimiento de trabajo
			\3 Acta Única de 1987
				\4 Sin avances relevantes
				\4 Competencia de armonización sobre estándares mínimos laborales
			\3 Carta Comunitaria de Derechos Sociales fundamentales de 1989
				\4 Opuesto por RU
				\4 No vinculante
				\4 Marco para desarrollos posteriores de Comisión
				\4 Principios laborales básicos
				\4[] Seguridad y salud
				\4[] Discriminación por raza o sexo
			\3 Convención de Schengen de 1990
				\4 Implementación del acuerdo
				\4 Entrada en vigor en 1995
			\3 Tratado de Maastricht de 1992-3
				\4 Protocolo social
			\3 Tratado de Amsterdam de 1997
				\4 Cambio de gobierno en RU
				\4[] Acepta Carta Comunitaria de Derechos Sociales
				\4 Integración de la carta en Derecho UE
				\4 Objetivo explícito de empleo y protección social
			\3 Agenda 2000
				\4 Explicita objetivos sociales para próxima década
			\3 Tratado de Lisboa
				\4 Reorganiza disposiciones anteriores
				\4 Retoma Carta Comunitaria
				\4[] TFUE.151
		\2 Marco jurídico
			\3 Principios de actuación
				\4 \marcar{R}espeto prácticas nacionales
				\4 \marcar{C}ompetitividad
				\4 \marcar{D}ialogo social
			\3 Carta Social Europea de 1961 y revisada en 1998
				\4 En vigor en España desde 1980
				\4 Revisada en proceso de ratificación (2019)
				\4[] $\to$ Se necesitaban cambios en legislación nacional
				\4 Condiciones mínimas de trabajo
				\4 Protección de niños y jóvenes
				\4 Formación y orientación profesional
				\4 Maternidad, familia, etc...
			\3 TUE
				\4 Objetivos de la Unión de 3.3
				\4[] Inclusión y justicia social
				\4[] No discriminación
				\4[] Pleno empleo de calidad
			\3 TFUE
				\4 Competencia peculiar
				\4[] Compartida\footnote{Artículo 4 TFUE.} en las áreas definidas en tratado
				\4 Coordinación de políticas nacionales
				\4 Libre circulación de trabajadores
				\4 Títulos IX y X
				\4[] Empleo y política social
				\4 Incorpora Carta DDSS fundamentales de 1989
			\3 Estrategia Europa 2020
				\4 Tasa de empleo: 75\% de la población entre 20 y 64
				\4 Pobreza o exclusión social: 20 millones menos
				\4 Abandono escolar: menos del 10\%
				\4 Estudios superiores: >40\% de personas entre 30-34
			\3 Employment Guidelines/Orientaciones generales sobre el empleo
				\4 Aprobadas inicialmente en 2010
				\4[] Estrategia Europa 2020
				\4 Reformadas en 2018
				\4 Objetivo para 2020
				\4 Comunes para toda UE
				\4 Prioridades y objetivos generales sobre empleo
				\4 Objetivos a los que EEMM deben tender
			\3 JER -- Informe Conjunto sobre el Empleo
				\4 Paquete de otoño: propuesta de JER
				\4 Joint Employment Report
				\4 Indicadores de seguimiento anuales
			\3 NRPs -- Programas Nacionales de Reforma
			\3 CSR -- Recomendaciones Específicas por País
				\4 Valorando políticas implementadas por cada país
				\4 Informes específicos para cada EEMM
			\3 Reglamentos de FSE y Fondos EIE
		\2 Marco financiero
			\3 Fondo Social Europeo
			\3 EaSI
				\4 European Programme for Employment and Social Innovation
				\4 Programa Europeo de Empleo e Innovación Social
				\4[] 1000 millones de € a precios 2013
			\3 FEAG
				\4 European Globalisation Adjustment Fund
				\4 Hasta 150 M de € anuales
				\4 Ayuda a recolocación en despidos > 500 personas
				\4 Ayuda a NiNis
				\4 No puede utilizarse para financiar inversiones
			\3 FEAD
				\4 Fondo de Ayuda Europea para los más Necesitados
				\4 Fund for European Aid to the Most Deprived
				\4 3800 millones para MFP 2014-2020
				\4 Generalmente, aportaciones a ONGs
				\4 Aprobación de programas de países miembros
			\3 IEJ/YEI
				\4 Iniciativa para el Empleo Juvenil
				\4 9.000 M de € para 2014-2020
		\2 Actuaciones
			\3 JEL -- Orientaciones generales para el empleo
				\4 4 directivas generales
				\4[v] Impulsar demanda de trabajo
				\4[] Reducir coste del empleo
				\4[] Eliminar trabas burocráticas al empleo
				\4[] Aumentar eficiencia de administración
				\4[vi] Mejora de la oferta de trabajo
				\4[] Inversión en capital humano
				\4[] $\to$ Habilidades
				\4[] $\to$ Competencias
				\4[] Políticas activas de empleo
				\4[] Educación debe orientarse al empleo
				\4[] Productividad objetivo central
				\4[vii] Eficiencia del mercado de trabajo
				\4[] Emparejamiento debe ser facilitado
				\4[] Dialogo social
				\4[] Intercambio de buenas prácticas
				\4[viii] Igualdad de oportunidades
				\4[] Modernización de sistemas de protección social
				\4[] Incentivos al trabajo
				\4[] Protección de rentas ante adversidad
				\4[] Énfasis en grupos de población desfavorecidos
			\3 Fondo Social Europeo\footnote{Ver \href{https://ec.europa.eu/esf/BlobServlet?docId=16259&langId=en}{Comisión Europea sobre Fondo Social Europeo}}
				\4 Reglamento 2013/1304
				\4[] Objetivos temáticos de actuación
				\4[8] Empleo de calidad y movilidad del trabajo
				\4[9] Inclusión social
				\4[10] Educación
				\4[11] Calidad institucional, eficiencia en AAPP
				\4 Actuaciones generales en la línea de JEL
				\4 Programas de inserción a parados de largo plazo
				\4 Programas de eliminación burocracia
				\4 Programas de inserción social
				\4[] Personas cercanas a pobreza
				\4 Objetivo general
				\4[] Mejorar eficiencia del mercado de trabajo
				\4[] Evitar histéresis
			\3 Youth Employment Initiative
				\4 Periodo 2014-2020
				\4 9.000 M de €
				\4 Complementado con fondos nacionales
				\4 Implementar programas de:
				\4[] Formación profesional
				\4[] Empleo subvencionado
				\4[] Búsqueda de empleo
				\4[] Continuación de estudios
			\3 Skills Agenda de 2016 -- Comunicación de la Comisión
				\4 Actuaciones en 10 sectores
				\4[] Aumentar capital humano
				\4[] Mejorar matching empresas--trabajadores
				\4 Formación de adultos
				\4[] Coordinado con EEMM
				\4 Marco de cualificaciones europeo
				\4 Formación digital
				\4 Sectores deficitarios en personal
				\4 Identificación de habilidades en inmigrantes
				\4 Formación profesional
				\4 Competencias clave en sistema educativo
				\4 Optimizar flujos de capital humano
				\4 Optimizar formación de posgrado
				\4[] Analizar rendimiento a nivel de grado
				\4[] Optimizar decisiones respecto posgrado
				\4[] $\to$ Educación
				\4[] $\to$ Experiencia
			\3 EGF -- European Globalisation Adjustment Fund
				\4 150 millones anuales
				\4 Para:
				\4[] despidos de >500 por 1 compañía
				\4[] Número elevado en sector/región localizados
				\4 Proyectos gestionados por EEMM o regiones
				\4 Cofinanciados
				\4 2 años
				\4 No financia protección social
				\4 Sólo políticas activas de empleo
				\4[] Búsqueda de empleo
				\4[] Asesoramiento
				\4[] Formación
				\4[] Mentoring y coaching
				\4[] Emprendimiento
				\4 Trabajadores son beneficiarios
				\4 Empresas no pueden beneficiarse
			\3 EaSI -- Employment and Social Innovation
				\4 Modernización de prácticas de empleo
				\4 Eures
				\4[] Cooperación servicios de empleo
				\4 Eje PROGRESS
				\4[] Investigación en políticas de empleo
				\4 Eje EURES
				\4[] Mejorar matching laboral intraeuropeo
				\4[] Movilidad laboral
				\4[] Mejorar servicios de empleo
			\3 SURE -- Support to mitigate Unemployment Risks in an Emergency
				\4 Marco de MEDE
				\4[] $\to$ Extrapresupuestario
				\4 Propuesto en 2020
				\4[] Crisis de Covid-19
				\4 Objetivo
				\4[] Evitar despidos que produzcan histéresis
				\4 Préstamos concesionales
				\4[] 100.000 M de € máximo
				\4 Permitir aumento de gasto público excepcional
				\4 Evitar presión sobre coste de financiación
				\4 Evitar despidos en contexto de crisis
				\4 Favorecer ERTEs y similares
				\4[] Parte de sueldo sin trabajar
				\4[] Administración se hace cargo parcialmente
				\4[] Reducir destrucción de empleo
				\4 Aprobado en mayo 2020
			\3 Intercambio de buenas prácticas
				\4 Comité de Protección Social
				\4 Comité de Empleo
			\3 Fomento de la movilidad
				\4 Interprofesional
				\4 Internacional
				\4 Estudiantes
				\4 Marco Europeo de Cualificaciones
				\4 Garantías juveniles
			\3 Propuestas para 2021-2027
				\4 Integración de fondos en ESF+
				\4[] Fondo Social Europeo
				\4[] Fondo Ayuda Europea a Desfavorecidos
				\4[] EaSI
				\4[] Programa Europeo de Salud
				\4 Reducción de cuantía
				\4[] De 100.000 a 120.000
		\2 Valoración
			\3 Pre-crisis y post-crisis
				\4 Niveles de empleo pre-crisis se alcanzan ahora
				\4 Histéresis en países del sur
				\4[] Especialmente GRE, ESP
				\4 Política social europea ayudó a reformar
				\4[] Especialmente en fase de crisis 2012-2014
			\3 Dificultades de valoración
				\4 Causalidad difícil de establecer
				\4[] ¿Política social resulta de crecimiento económico?
				\4[] ¿Al revés?
			\3 Comparación con resto del mundo
				\4 Niveles muy altos de protección social
				\4 Desigualdad comparativamente menor
				\4 Desempleo más elevado que otros países avanzados
		\2 Retos
			\3 Desempleo juvenil
				\4 Muy elevado en algunos EEMM
				\4 Efectos de l/p sobre población activa
			\3 Resultados desiguales
			\3 Flexibilidad del mercado laboral
				\4 Política social contribuye a más rigidez
				\4 ``Flexiseguridad''
				\4[] Combinar flexibilidad para empleadores con certidumbre para trabajadores
	\1 \marcar{Implicaciones sobre la convergencia real}
		\2 Idea clave
			\3 Concepto de convergencia
				\4 Tendencia hacia la reducción de diferencias
				\4[] Entre unidades económicas
				\4[] $\to$ Países
				\4[] $\to$ Regiones
				\4[] $\to$ Estados
				\4 Connotación años 50 y 60
				\4[] Tendencia hacia igualdad entre
				\4[] $\to$ Occidente capitalista y países comunistas
				\4 Sentido moderno
				\4[] Persistencia/desaparición de diferencias en PIBpc
				\4[] Ocasionalmente, también otras variables
				\4[] $\to$ Desempleo
				\4[] $\to$ Paro
				\4[] $\to$ Productividad por ocupado
				\4 Informe de Convergencia
				\4[] Responsabilidad del BCE
				\4[] Relativo a países candidatos a ingresar en Z€
				\4[] Principalmente variables nominales
				\4 Informe sobre Cohesión
				\4[] Comisión Europea
				\4[] Último publicado en 2017\footnote{Ver \url{https://ec.europa.eu/regional_policy/en/information/publications/reports/2017/7th-report-on-economic-social-and-territorial-cohesion}}
			\3 Aspectos teóricos
				\4 $\beta$-convergencia
				\4[] Convergencia es relación negativa entre:
				\4[] $\to$  Renta inicial
				\4[] $\to$ Tasa de crecimiento
				\4[$\then$] $\uparrow$ Renta, $\downarrow$ Crecimiento
				\4[$\then$] Países más pobres crecen más que ricos
				\4[] $\beta$-convergencia condicional
				\4[] $\to$ $\beta$-conv. entre países similares
				\4[] $\then$ Necesario controlar por otras características
				\4[] Problema:
				\4[] $\to$ ¿Qué factores son relevantes?
				\4[] $\to$ ¿Qué criterio para elegir factores?
				\4 $\sigma$-convergencia
				\4[] Convergencia es tendencia a reducción de
				\4[] $\to$ Varianza de una sección cruzada de países
				\4 Relación entre $\beta$ y $\sigma$-convergencia
				\4[] $\beta$-convergencia es condición necesaria
				\4[] $\to$ NO es condición suficiente
				\4[] Es decir:
				\4[] $\sigma$-convergencia $\then$ $\beta$-convergencia
				\4[] $\beta$-convergencia $\nRightarrow$ $\sigma$-convergencia
				\4[] Falacia de Galton (FALSO):
				\4[] $\to$ Reversión a la media en series temporales
				\4[] $\then$ Reducción de la varianza
				\4[] \grafica{betasigmaconvergencia}
				\4 $\sigma$-convergencia condicional
				\4[] Similar a $\beta$-convergencia condicional
				\4[] Reducción de $\sigma$ entre países similares
				\4 Convergencia de series temporales
				\4[] Convergencia es tendencia a reducción de
				\4[] diferencias en PIBpc en el infinito
				\4[] $\to$ Dado historial de crecimiento pasado
				\4[] En términos formales
				\4[] $\lim_{T\to \infty} E\left( \ln y_{i, t+T} - \ln y_{j, t+T} \, | \, F_t \right) = 0$
		\2 Evidencia empírica sobre convergencia real
			\3 Hasta 90s
				\4 $\beta$-convergencia entre EEMM
				\4[] Convergencia a tasas elevadas
				\4[] Convergencia aún mayor si condicional
				\4 $\sigma$-convergencia
				\4[] También tiene lugar
			\3 Post-90s
				\4 Velocidad de $\beta$-convergencia cae significativamente
				\4[] Especialmente, entre UE-15
				\4 Nuevos EEMM
				\4[] Algunos convergen muy rápido
				\4[] $\to$ CZK, SLK, SLV,
				\4[] Otros, convergencia positiva pero lenta
				\4 $\sigma$-convergencia
				\4[] Estancada entre regiones UE-15
				\4[] Sigue produciéndose a nivel UE-27
				\4[] Nuevos EEMM convergen con EEMM pre-ampliación
			\3 Recientemente\footnote{Ver resumen de VII informe de Cohesión (2017) en carpeta del tema. }
				\4 Estancamiento o incluso divergencia
				\4[] Crisis económica tiene impactos desiguales
				\4[] Especialmente acusado en desempleo
				\4 Caen niveles de inversión
				\4 Aumentan niveles de paro
				\4 Regiones con PIBpc muy superior a media
				\4[] Han crecido más rápido que menos desarrolladas
				\4[] Economías de aglomeración acentuadas
				\4 Regiones con PIBpc cercano a media
				\4[] Muchas regiones atrapadas en renta media
				\4[] Industrias manufactureras más pequeñas y débiles
				\4[] $\to$ Que las de ciudades grandes y pequeñas
				\4[] Sistemas de innovación no suficientemente grandes
				\4[] Afectadas por globalización
				\4 Dinámicas de aglomeración acentuadas
			\3 Convergencia norte-sur
			\3 Convergencia oeste-este
				\4 Muy rápida en últimos años
				\4 Especialmente:
				\4[] POL, HUN, SLV, SLK, CZK, Bálticos
				\4 Mucho más lenta en Rumania y Bulgaria
				\4 Algunos estudios atribuyen a fondos estructurales
				\4[] Polonia mayor receptor
				\4[] Otros países Visegrado importantes receptores
			\3 Convergencia centro-periferia
				\4 Muy débil en últimos años
				\4 Agravamiento tras crisis
				\4 Tendencia a aglomeración en centro
				\4[] Capital humano
				\4[] Trabajo cualificado
				\4[] Industrias de alta tecnología
				\4 Políticas estructurales no parecen tener gran efecto
				\4 Políticas de transporte incluso efecto contrario
				\4[] Efecto competencia en favor de efecto demanda
				\4[] $\to$ Preferible aglomerarse y exportar a periferia
			\3 Efectos de crisis sobre convergencia
				\4 Aumenta divergencias
				\4[] Especialmente regionales
				\4 Grandes urbes tienden a recuperar convergencia
				\4 Regiones desfavorecidas pierden peso
			\3 Sincronización de ciclo económico
				\4 Krugman (1993) y (2003)
		\2 Impacto de las políticas de cohesión
			\3 Dificultades de medición
				\4 Problemas habituales de identificación
				\4[] ¿Cómo establecer causalidad?
				\4[] ¿Qué supuestos para formular contrafactual?
				\4[] $\to$ Inversiones se habrían llevado a cabo igualmente
				\4[] $\to$ Menor coste de fondos puede mejorar convergencia
				\4[] $\to$ Crowding-out del gasto público
				\4[] $\to$ Crowding-in
				\4 Evidencia empírica
				\4 Positiva
				\4[] Tiende a ser dominante
				\4[] Ligero efecto positivo sobre convergencia
				\4[] Contribuyentes netos también beneficiados
				\4[] $\to$ Principales proveedores de bienes de K
				\4[] $\to$ Aumento de exportaciones
				\4[] $\to$ Aumento de productividad tras división del trabajo
				\4 Crítica
				\4[] Son otras políticas las que contribuyeron a convergencia
				\4[] $\to$ Mercado único
				\4[] $\to$ Convergencia nominal para entrada en euro
				\4[] $\to$ Globalización
				\4[] Evidencian costes excesivos de administración
				\4[] Impactos cercanos a 0 sobre crecimiento
				\4[] $\to$ A pesar de avalancha de fondos tras adhesión
				\4[] Velocidades de convergencia más elevadas
				\4[] $\to$ Cuando no existían políticas de cohesión
	\1[] \marcar{Conclusión}
		\2 Recapitulación
			\3 Política regional
			\3 Política social y de empleo
			\3 Implicaciones sobre la convergencia real
		\2 Idea final
			\3 Séptimo Informe sobre Cohesión
				\4 Generalmente optimista
				\4 Énfasis en desempleo
				\4[] Ha tardado mucho en recuperar niveles pre-crisis
				\4[] Resultados muy heterogéneos
				\4 Recomienda:
				\4[] Aumentar cofinanciación
				\4[] $\to$ Mejorar sentimiento de implicación
				\4[] Dejar parte sin asignar
				\4[] $\to$ Aumentar flexibilidad de políticas
				\4[] Normas de liberalización más estrictas
				\4[] Simplificaciones mucho más radicales
				\4[] $\to$ Para compensar complejidad creciente
				\4[] $\to$ Cierre de programas más rápido
				\4[] Complementariedad con instrumentos financieros
				\4[] $\to$ Especialmente con EFSI
				\4[] $\to$ Fondos de K riesgo europeos
			\3 Futuro de UE
				\4 En la actualidad en entredicho
				\4 Riesgos principales
				\4[] Inestabilidad económica y financiera
				\4[] Reformas lentas e insuficientes frente a próxima crisis
				\4 Política de cohesión seguirá siendo relevante
				\4[] Germen de posible presupuesto de estabilización
\end{esquemal}

\preguntas

\seccion{Test 2019}

\textbf{43.} Señale la afirmación correcta en relación al crecimiento económico y a la convergencia/divergencia entre los países y entre las regiones de la Unión Europea (UE) según el Séptimo Informe de Cohesión, publicado por la Comisión Europea en 2017:

\begin{itemize}
	\item[a] A lo largo del período 2001-2016, el PIB per cápita creció con menor rapidez en términos reales en los Estados miembros menos desarrollados, y está previsto que esta tendencia se mantenga en 2017 y 2018.
	\item[b] En la UE las disparidades regionales se acrecentaron considerablemente, en términos del PIB per cápita, entre el año 2000 y el año 2008.
	\item[c] Durante el periodo 2001-2015, las regiones de renta muy alta, con un PIB per cápita equivalente al 150\% o más de la renta media de la UE, han crecido más rápido que las regiones de renta alta, con un PIB per cápita entre el 120\% y el 149\% de la media de la UE.
	\item[d] Todas las afirmaciones son correctas.
\end{itemize}

\seccion{Test 2016}

\textbf{47.} El Fondo de Cohesión en la Unión Europea:
\begin{itemize}
	\item[a] Tiene como objetivo fortalecer la cohesión socioeconómica dentro de la Unión Europea corrigiendo los desequilibrios entre sus regiones.
	\item[b] Es el principal instrumento para apoyar la creación de empleo, ayudar a las personas a conseguir mejores puestos de trabajo y garantizar oportunidades laborales más justas para todos los ciudadanos de la UE.
	\item[c] Actualmente está destinado a los Estados Miembros cuya Renta Nacional Bruta per cápita es inferior al 90\% de la renta media de la Unión Europea.
	\item[d] Nunca financia proyectos de infraestructuras que estén incluidos en el marco del Instrumento de Interconexión para Europa.
\end{itemize}

\seccion{Test 2015}
\textbf{44.} Señale la respuesta \textbf{\underline{falsa}} con respecto a la Estrategia 2020 de la UE:
\begin{itemize}
	\item[a] El crecimiento inclusivo es una de las 3 prioridades que se refuerzan mutuamente de esta estrategia.
	\item[b] Uno de sus 5 objetivos está relacionado con la tasa de abandono escolar.
	\item[c] El desempleo femenino no forma parte de los 5 objetivos principales.
	\item[d] Se denomina semestre europeo según el cual los estados miembros de la UE coordinan sus políticas económicas y trabajan en la aplicación de la Estrategia 2020.
\end{itemize}

\seccion{Test 2011}
\textbf{40.} El Fondo de Cohesión en la Unión Europea:
\begin{itemize}
	\item[a] Fue creado con el objetivo de facilitar la cohesión económica y social entre las regiones de la UE.
	\item[b] Ha contribuido de forma decidida a la cohesión económica y social a partir de su creación en 1970.
	\item[c] Fue creado para ayudar a determinados países a cumplir los criterios de convergencia nominal del proceso de integración monetaria.
	\item[d] Nunca financia proyectos relacionados con infraestructuras.
\end{itemize}

\seccion{Test 2009}
\textbf{43.} Respecto de la Política Regional Comunitaria (PRC)
\begin{itemize}
	\item[a] Estuvo presente desde el inicio de la Comunidad Económica Europea ya que en el Tratado de Roma se crearon los instrumentos necesarios para su puesta en marcha y su desarrollo.
	\item[b] Fue, desde el inicio, una política no subsidiaria de las políticas de los estados miembros ya que financió medidas de apoyo regional decididas desde la Comunidad.
	\item[c] Se reconfiguró de forma sustancial en el Tratado de Amsterdam al recoger la necesidad de alcanzar la cohesión económica y social en el seno de la Comunidad Europea.
	\item[d] Recibió su mayor impulso en el año 1988 al incrementarse notablemente los recursos financieros destinados a fines específicamente regionales.
\end{itemize}

\seccion{Test 2005}
\textbf{45.} En el marco de la política regional de la Unión Europea:

\begin{itemize}
	\item[a] Los fondos estructurales suponen más del 90\% de las acciones estructurales frente al menos del 10\% restante de los Fondos de Cohesión.
	\item[b] Los Fondos estructurales se reparten el presupuesto en acciones estructurales al 50\% con los Fondos de Cohesión.
	\item[c] Las perspectivas financieras actuales (2000-2006) redujeron el número de iniciativas comunitarias a las 4 actuales: INTERREG III, LEADER+, EQUAL y URBAN II y mantuvieron el mismo número de objetivos prioritarios.
	\item[d] Las perspectivas financieras actuales (2000-2006) redujeron el número de objetivos prioritarios de 7 a 3 pero sin embargo mantuvieron el mismo número de iniciativas comunitarias.
\end{itemize}


\seccion{Test 2004}
\textbf{41.} ¿Desempeña el Comité de las Regiones una misión fundamental en el sistema institucional de la Unión Europea (UE)?
\begin{itemize}
	\item[a] Sí. su tarea esencial consiste en la gestión, bajo la supervisión de la Comisión Europea, de la política regional comunitaria.
	\item[b] Sí. Sin contar con su aprobación expresa no resulta posible aplicar los programas de gasto del FEDER.
	\item[c] No. No participa de forma activa en el sistema institucional de la UE. Fue creado por el Tratado de Niza como un órgano auxiliar destinado a facilitar la aplicación a las políticas comunitarias de los principios de subsidiariedadd y de cooperación reforzada establecidos en el Tratado de Maastricht.
	\item[d] No. El Comité de las Regiones fue creado a principios de los años 90. Nació con grandes ambiciones pero su papel en la UE es poco relevante. Su desarrollo institucional ha sido similar al del Comité Económico y Social desde su creación.
\end{itemize}

\textbf{43.} El Fondo de Cohesión de la UE es:
\begin{itemize}
	\item[a] Un fondo estructural establecido en 1986 y destinado a los Estados Miembros cuya renta per cápita es inferior al 90\% de la media comunitaria.
	\item[b] Un fondo creado para completar las acciones estructurales y de cohesión de la UE y presta ayuda financiera a cualquier región europea que presente problemas importantes de atraso estructural.
	\item[c] Un fondo orientado esencialmente al desarrollo de las infraestructuras y redes de transportes, comunicaciones y transmisión de energía en los Estados Miembros de menores niveles de renta per cápita. 
	\item[d] Un fondo especial que absorbe alrededor de un tercio de los gastos estructurales de la UE y que atiende en particular las necesidades en materia de cohesión de las regiones que no son Objetivo 1 de los fondos estructurales comunitarios.
\end{itemize}

\notas

\textbf{2019:} \textbf{43.} C

\textbf{2016:} \textbf{47.} C

\textbf{2015:} \textbf{44.} ANULADA

\textbf{2011:} \textbf{40.} C pero tiene que ser un error, mirar 177 TFUE.

\textbf{2009:} \textbf{43.} D

\textbf{2005:} \textbf{45.} A

\textbf{2004:} \textbf{41.} B \textbf{43.} C

Para el tema de la PAC, Victor propuso un esquema estándar para explicaciones de políticas, cuando llega el momento de explicar la situación actual:
\begin{enumerate}
    \item Justificación
    \item Objetivos
    \item Marco de funcionamiento
    \item Valoración y retos
\end{enumerate}
En la medida de lo posible, estandarizar este esquema, y aplicarlo a todos los temas incluido éste.

Hay que mejorar la parte de valoración y resultados de políticas regionales y sociales

\bibliografia

Mirar en Palgrave:
\begin{itemize}
	\item European Cohesion Policy
	\item European Employment Policy
	\item European Labour Markets
	\item European Unemployment Insurance
	\item globalization and the welfare state
	\item regional development, geography of
	\item regional distribution of economic activity
\end{itemize}


Comisión Europea. \textit{Séptimo informe sobre la cohesión económica, social y territorial} (2017) \url{https://ec.europa.eu/regional_policy/sources/docoffic/official/reports/cohesion7/7cr_es.pdf} -- En carpeta del tema

Eurostat. \textit{GDP at regional level} (2018) \url{https://ec.europa.eu/eurostat/statistics-explained/index.php/GDP_at_regional_level#Regional_gross_domestic_product_.28GDP.29_per_inhabitant}

Goecke, H., Hüther, M. \textit{Regional Convergence in Europe} (2016) \url{https://archive.intereconomics.eu/year/2016/3/regional-convergence-in-europe/} -- En carpeta del tema


\end{document}
