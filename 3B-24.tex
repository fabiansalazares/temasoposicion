\documentclass{nuevotema}

\tema{3B-24}
\titulo{Mercados financieros internacionales: emisores, instrumentos y mercados de renta fija y variable}

\begin{document}

\ideaclave

Ver \href{https://www.imf.org/external/pubs/ft/bop/2018/pdf/Clarification0518.pdf}{FMI (2018)} sobre funcionamiento de los swaps entre bancos centrales.

Los contratos financieros son uno de los principales engranajes de la economía mundial. Un contrato financiero genérico es un instrumento jurídico mediante el cual una parte se obliga a efectuar una corriente de pagos en el futuro, posiblemente sujetos a una serie de condiciones, a cambio de otra corriente de pagos de distinta cuantía y/o con diferente perfil temporal. Los contratos financieros permiten a los agentes económicos suavizar intertemporalmente su perfil de consumo, trasladando rentas del presenta al futuro y viceversa, así como asegurarse frente a riesgos o asumir riesgos a cambio de un retorno. Un mercado financiero es un concepto abstracto que hace referencia a un conjunto de agentes económicos que intercambian la titularidad de valores financieros que representan contratos, ya sea en el contexto de un entorno físico o como un intercambio virtual a través de medios electrónicos. Así, los mercados financieros son el entorno en el que las partes de un contrato financiero se encuentran y se obligan mutuamente. Este encuentro o ``matching'' puede tener lugar de manera centralizada, cuando lo hacen a través de un intermediario que se interpone entre las partes para hacer el proceso de obligación más fluido y eliminar riesgos, o de manera centralizada, cuando las partes se encuentran y se obligan de manera autónoma y directa. Los mercados financieros internacionales son una clase particular de mercado financiero especialmente relevante por sus potenciales aportaciones pero también por sus posibles consecuencias negativas. La apertura de un mercado financiero al exterior, más allá de unas fronteras nacionales y un conjunto determinado de agentes ligados a una economía, amplía las posibilidades de matching y reduce los costes ligados al contrato financiero. Así, se añaden a las posibles contrapartes nacionales un número mucho más elevado de contrapartes, aumentan las modalidades de contratación hasta donde la imaginación, las necesidades y la regulación lo permitan, a menudo se reducen los costes de intermediación por el hecho de que aumenta la competencia, y se produce también un efecto potencialmente positivo sobre los costes de búsqueda de información al aumentar la información disponible a los inversores como resultado de la mayor competencia entre emisores. 

La aparición de mercados financieros globalizados está fuertemente conectada con los avances en materia de comunicaciones y la reducción de los costes de transporte de mercancías a nivel mundial que han impulsado la integración y la interconexión de la economía mundial. Aunque este proceso ha tenido lugar a lo largo de los últimos siglos, y en la etapa final del siglo XIX y hasta la Primera Guerra Mundial el tráfico financiero internacional conoció un auge significativo, ha sido a partir de los años 80 del siglo XX cuando la liberalización de las cuentas financieras de la mayoría de los países desarrollados, así como la consolidación de las tecnologías de la información han generado una verdadera explosión de los volúmenes de transacción de productos financieros a escala global. Los mercados financieros internacionales tienen además el potencial de afectar a la economía real. Aunque permiten suavizar los shocks temporales que afectan a toda las economías, la mayor interconexión de las economías financieras permite también la transmisión de shocks positivos y negativos a nivel mundial que potencialmente pueden desembocar en situaciones de crisis o inestabilidad. Por todo ello, el análisis y comprensión global del funcionamiento de los mercados financieros internacionales es fundamental para comprender el funcionamiento de una economía moderna y proponer soluciones a los problemas que les afecta. Así, el \textbf{objeto} de la exposición consiste en responder a preguntas tales como: ¿cómo funcionan los mercados financieros internacionales? ¿qué agentes participan en ellos? ¿qué activos financieros se negocian? ¿cómo han evolucionado? ¿qué cambios han sufrido tras la Gran Crisis? La \textbf{estructura} de la exposición se divide en cuatro partes. En la primera, examinamos los agentes participantes. A continuación examinamos el mercado y los instrumentos negociados en el mercado de renta fija. Posterioremente, tratamos el mercado y los instrumentos de renta variable del mercado internacional. Por último, analizamos los cambios que se han producido en los mercados internacionales en el periodo post-crisis.

En los mercados financieros participan una gran variedad de \textbf{agentes}, aunque es posible englobarlos en diferentes grupos a partir de algunas de sus características. Las llamadas instituciones financieras son el principal tipo de agente que participa en los mercados financieros. Las IFs se caracterizan por ser personas jurídicas cuya principal actividad consiste en la negociación de activos financieros, ya sea deuda, acciones, instrumentos del mercado monetario o varios de éstos a la vez. La importancia de las IFs es enorme en los mercados financieros internacionales y por consiguiente, para cualquier economía desarrollada. Por un lado, las IFs son los principales emisores de pasivos, con alrededor de un 80\% del volumen total. Por otra parte, son los principales tenedores de activos financieros. Las IFs pueden tener un carácter colectivo, en el sentido de que múltiples agentes tienen derechos de propiedad sobre la institución, o tener un carácter individual en el sentido de ser propiedad de un sólo agente como un gobierno o un banco. Las instituciones financieras del mercado monetario son aquellas IFs que negocian pasivos de corto plazo de equivalencia cercana al dinero por su elevada liquidez e interés cercano a cero. Este tipo de instituciones compran y venden deuda con plazo inferior a un año. Los \textit{money market funds} son un tipo de institución financiera que emite pasivos e invierte sus ingresos en este tipo de activos de muy corto plazo, ofreciendo una rentabilidad a los compradores de sus pasivos muy reducida pero generalmente superior a los depósitos bancarios. A nivel internacional, la caída del sistema de Bretton Woods generó una demanda de liquidez en divisas que aumentó enormemente la importancia sistémica de estas instituciones. 

\textit{Las agencias financieras} son instituciones financieras emisoras de pasivos financieros asociadas a los gobiernos y que cuentan, por ello, con respaldo gubernamental más o menos explícito. Las agencias financieras conceden créditos o avales a sectores específicos de la economía y captan fondos emitiendo pasivos en mercados nacionales e internacionales. Ejemplos de estas instituciones son Fannie Mae o Freddie Mac en Estados Unidos o el ICO en España. Los fondos de titulización son instituciones financieras privadas que compran deuda no negociable directamente tal como préstamos particulares, crédito comercial, créditos al consumo o préstamos hipotecarios y emiten bonos cuyos pagos están ligados a la satisfacción de los flujos ligados a la deuda no negociable adquirida. Este tipo de agentes fue especialmente relevante en la explosión crediticia previa a la Crisis Financiera de 2008. 

Los \textit{fondos de inversión} son instituciones de inversión colectiva que compran toda clase de activos en mercados nacionales e internacionales con el fin de ofrecer un retorno a los inversores participantes del fondo. Así, estos fondos dividen su propiedad en diferentes participaciones que otorgan derechos sobre los rendimientos. Este tipo de instituciones están generalmente formadas por dos componentes: una sociedad gerente que se encarga de decidir el objeto de las inversiones, y una sociedad depositaria cuya finalidad es custodiar los valores que representan las inversiones realizadas. Los fondos de inversión pueden gestionarse de maneras muy diversas, pero en general se dividen en fondos de gestión activa y de gestión pasiva. Los fondos de gestión pasiva simplemente replican el rendimiento de un índice o una cartera de valores determinada previamente. Su presencia es cada vez más relevante en las carteras de inversión nacionales e internacionales. Los fondos de gestión activa toman decisiones de inversiones tratando de obtener los rendimientos relativos más elevados posibles, con estrategias de inversión a discreción del fondo. 

Los \textit{hedge funds} son instituciones de inversión privadas de participación restringida cuyo objetivo consiste en generar retornos absolutos para los propietarios aplicando estrategias de inversión activa y generalmente, tasas de apalancamiento muy elevadas. El papel de los fondos de inversión en varias crisis internacionales ha sido destacado: el ataque de Soros a la libra esterlina en los 90, la crisis asiática, el fondo LTCM... La asunción de riesgos, la concentración de las inversiones y el papel que juegan en el gobierno corporativo los llamados \textit{private equity funds} (hedge funds que tratan de obtener retornos a partir de inversiones de largo plazo y que se involucran en la gestión de las empresas en las que invierten a través de compras de paquetes accionariales) han generado críticas a la actuación de los hedge funds y llamadas a su regulación. En el contexto de la Unión Europea, los hedge funds, los private equity funds y los fondos de inversión inmobiliaria están regulados por la \textit{Alternative Investment Funds Manager Directive} de 2011, que incorpora entre otras medidas restricciones al perfil inversor de los hedge funds y a la estructura de remuneración de los gestores para evitar tomas de riesgos excesivas. 

Las \textit{sociedades no financieras} son también agentes habituales en los mercados financieros internacionales, tanto por la emisión de activos de renta fija y renta variable para financiar sus ciclos operativo y de inversión, tanto como inversores para colocar picos de tesorería y similares.

Los \textit{gobiernos} participan habitualmente en los mercados internacionales para financiarse mediante la emisión de bonos y la captación de deuda bancaria internacional. Algunos \textit{fondos soberanos} de titularidad estatal han alcanzado tamaños significativos, en especial aquellos ligados a países con cuentas corrientes fuertemente superavitarias tales como China, Noruega, Emiratos Árabes, Kuwait, Arabia Saudí o Qatar. Estos fondos soberanos adquieren una fuerte dimensión política y estratégica cuando actúan en los mercados internacionales.

Las \textit{organizaciones internacionales} participan en los mercados financieros internacionales principalmente a través de los bancos de desarrollo. Existe un mercado secundario de deuda de bancos internacionales relativamente activo, en el cual instituciones como el Banco Mundial, el Banco Europeo de Inversiones, el Banco Asiático de Inversiones o el Banco Asiático de Infraestructuras y Desarrollo son actos de especial importancia. El \textit{Fondo Monetario Internacional} también es un importante actor en el mercado financiero internacional a través de la implementación de programas de ayuda cuyo objetivo es recuperar la estabilidad de una economía y su capacidad para financiarse de forma autónoma. 

Por último, es relevante el papel de las \textit{entidades de estandarización y supervisión}. La provisión de estándares de contabilidad y auditoría que permitan comparar los estados contables de empresas a nivel internacional es esencial para reducir los costes de información y la transparencia de las emisiones de activos. La \textit{International Accounting Standards Board} elabora las Normas Internacionales de Información Financiera. En el ámbito de la auditoría, son relevantes la \textit{International Federation of Accountants} y la \textit{Public Interest Oversight Board}. En el ámbito de las finanzas islámicas son relevantes la \textit{Islamic Financial Services Board} y la \textit{Accounting and Auditing Organization for Islamic Financial Institutions}.

Una vez analizados los agentes que tienen un papel significativo en los mercados internacionales, podemos centrarnos en segmentos concretos, empezando por el \textbf{mercado y los instrumentos de renta fija}. Los activos financieros de renta fija son aquellos que otorgan al poseedor el derecho a percibir un flujo de rentas futuras determinado a priori o determinable a partir de una regla conocida. Los activos de renta fija no confieren a los poseedores derecho alguno sobre la gestión del emisor, salvo en caso de incumplimiento de sus obligaciones. Los activos de renta fija son el instrumento más habitual de financiación y representan los volúmenes más elevados en los mercados internacionales. 

Examinemos primero los diferentes segmentos de los mercados financieros internacionales en los que se intercambian títulos de corto plazo. Es habitual denominar este segmento como \textit{mercado monetario} por ser éste el segmento en el que se intercambian títulos de deuda con vencimientos muy cortos, prácticamente sustitutivos del dinero en condiciones normales por sus características de liquidez y vencimiento. Los \textit{depósitos y los certificados de depósito} son deuda bancaria de corto plazo, con vencimientos generalmente entre uno y seis meses, aunque también existen depósitos a la vista. Los agentes que participan en este segmento son casi siempre bancos que quieren cubrir una necesidad de liquidez o que quieren colocar un pico de liquidez en un momento determinado. Mientras que los depósitos son deuda bancaria no negociable, los certificados de depósito toman la forma de títulos valor y son negociables. Los llamados ``\textit{euromercados}'' son mercados en los que se negocian depósitos en moneda diferente a la moneda local. Así, el mercado de depósitos eurodólar hace referencia al intercambio de depósitos denominados en dólares en una jurisdicción en la que la moneda local es otra, como por ejemplo el Reino Unido o Europa. En los euromercados los depósitos a la vista no son habituales. Los \textit{repos o cesiones temporales de activos} son una modalidad de deuda a corto plazo muy habitual en los mercados internacionales y en transacciones con los bancos centrales en los que el prestatario vende un activo al prestamista y se compromete a recomprarlo en una fecha futura. El \textit{papel comercial} especialmente habitual en Estados Unidos y consiste en la emisión de obligaciones de corto plazo por parte de empresas no financieras. Este tipo de instrumentos se emite generalmente al descuento por empresas de reconocida solvencia, dada la ausencia de garantías por parte del emisor. Existe también un euromercado de notas en el que la moneda de emisión es diferente a la del país en el que se emite el papel. La colocación de este tipo de títulos se lleva a cabo generalmente por medio de subastas, sindicatos bancarios o directamente por medio de los departamentos de tesorería. Es también habitual el término de \textit{notas y euronotas} para hacer referencia a emisiones de deuda en las cuales la colocación de la emisión está garantizada por el intermedia contratado por el emisor, que se compromete a suscribir la emisión en lo que la demanda del mercado no sea capaz de cubrir en un primer momento. En lo que respecta a \textit{deuda gubernamental de corto plazo}, existen mercados muy líquidos y profundos de este tipo de instrumentos. La denominación de estos instrumentos varía según el país: letras del tesoro en España, treasury bills en Reino Unido y Estados Unidos, bons du trésor en Francia o schatzwechsel en Alemania, entre otros. Aunque este tipo de instrumentos tiene en principio una orientación al mercado doméstico, en Europa se ha producido una clara tendencia a la internacionalización. Este tipo de instrumentos tienen un vencimiento de hasta 1 año, y se subastan generalmente de acuerdo a un calendario previo, en episodios periódicos. El sistema de venta es casi siempre al descuento, de tal manera que los compradores pagan un precio ligeramente por debajo del valor nominal que recibirán en el vencimiento.

El \textit{mercado de notas y bonos} hace referencia al mercado de instrumentos con vencimientos superiores a un año o dieciocho meses que generalmente, pagan un cupón periódico a su tenedor. Algunos de estos instrumentos conllevan la posibilidad de convertirse en acciones del emisor. Esta conversión puede estar sujeta a un evento determinado, o quedar a disposición del comprador con la emisión paralela de un \textit{warrant} que confiere el derecho a comprar acciones a un precio y que son negociables por separado. En ocasiones, el pago de estos títulos de deuda está garantizado por un conjunto de activos, y se denominan ''debentures``. La colocación de estos títulos se lleva a cabo generalmente por sindicación bancaria, aunque para emisores especialmente solventes o de reconocido prestigio son posibles las colocaciones directas. La dimensión internacional de este segmento de mercado es especialmente relevante. Los \textit{eurobonos} son títulos de deuda denominados en una moneda diferente a la moneda local del país de emisión. Así, un ejemplo de un eurobono puede ser una empresa japonesa que vende bonos denominados en dólares a inversores europeos. Este tipo de emisiones deben explicitar la jurisdicción relevante a la emisión y la regulación a la que están sujetos. Generalmente, se listan en Londres o Luxemburgo. Este tipo de títulos suelen llevar aparejado tipos de interés más elevados, por el riesgo de tipo de cambio y de liquidez que pueden llevar aparejados por el hecho de que la moneda de denominación sea distinta de la moneda local. Los \textit{bonos extranjeros} son bonos emitidos en mercados extranjeros al emisor pero comprados por inversores cuya moneda local es la moneda de denominación del bono. Así, una empresa americana que emite en Japón bonos denominados en yenes estaría emitiendo bonos extranjeros. Este tipo de bonos suelen denominarse con nombres humorísticos o que hacen referencia a algún aspecto cultural del país de emisión. Por ejemplo, el bono anterior sería un bono ``samurái'', un bono emitido en Reino Unido en libras sería un bono ``bulldog'', o un bono extranjero emitido en España cuando la peseta era la moneda local era conocido como bono ``matador''. Las \textit{emisiones de deuda soberana} se realizan generalmente en moneda propia, tras las experiencias de crisis financieras en los años 80 y 90 por países en desarrollado que se endeudaban en moneda local. Aunque en el pasado el mercado internacional de este tipo de bonos era relativamente ilíquido y poco profundo, en la actualidad En Europa, este tipo de bonos conllevan el pago de un cupón anual, mientras que en Estados Unidos y Japón el pago de cupón es semestral. El mercado de \textit{titulizaciones de deuda} consiste en títulos de renta fija cuyos pagos están garantizados por masas de activos de deuda no directamente negociable. Este mercado fue particularmente relevante en la crisis financiera de 2007 y 2008, y ha alcanzado una internacionalización creciente con los años. Las titulizaciones de deuda o \textit{Asset Backed Securities} consisten empaquetar los pagos derivados de deuda no negociable y revender los derechos a esos pagos a terceras partes. La deuda no negociable que subyace a estos activos puede ser deuda de crédito al consumo, de compra de vehículos, de préstamos hipotecarios, etc... Esta última posibilidad es especialmente relevante y se denominan \textit{Mortgage Based Securities}. En general, en este tipo de deuda, el impago de la deuda subyacente no implica que el emisor se haga cargo de los pagos. En el mercado de \textit{covered bonds}, por el contrario, sí se produce este hecho y la masa de activos que representa el colateral no es sino una garantía adicional al pago de la deuda. Los \textit{pfandbriefe} de origen alemán son especialmente habituales en la Unión Europea y se han convertido en el estándar de emisión de deuda bancaria de máxima calidad. En estos instrumentos, los activos colaterales forman parte del balance del emisor. 

Por último, es relevante en el segmento de la renta fija por su dimensión puramente internacional el \textit{mercado de préstamos sindicados en divisas}. Este mercado consiste en la concesión de préstamos  denominados en monedas diferentes a las del prestatario, a través de un conjunto de bancos denominado sindicato bancario. Es posible la aparición de un mercado secundario para estos préstamos. Los prestamistas son grupos de bancos que acuerdan proveer los fondos y asumir los riesgos de forma conjunta. Las modalidades de emisión son fundamentalmente tres. En la modalidad \textit{best effort} el sindicato de bancos se compromete a tratar de colocar la máxima cantidad de deuda en el mercado secundario, pero sin comprometerse a cubrir la diferencia en caso de que no exista suficiente demanda. En la modalidad de grupo asegurador o underwriting, el miembro o miembros principales del sindicato bancario se comprometen a suscribir el préstamo para cubrir la parte que el mercado secundario no cubra, a cambio de un precio adicional. En la modalidad de club, las partes involucradas se comprometen a aportar los fondos, sin que uno o varios estén se comprometan a ejercer el papel de \textit{underwriter} si no es posible colocar en el mercado secundario la cantidad requerida.

El mercado internacional de \textbf{renta variable} engloba todo el tráfico internacional de títulos que confieren algún tipo de derecho de gestión sobre la sociedad emisora, que generalmente no dan lugar a derecho a recibir cantidades previamente determinadas de flujos futuros, y en el que intervienen agentes localizados en al menos dos jurisdicciones distintas. Aunque el mercado internacional de renta variable es fundamental para el funcionamiento de la economía mundial por ofrecer la posibilidad de participar en la gestión de proyectos de inversión más allá de las fronteras nacionales y aumentar las posibilidades de financiación sin restringirse al mercado doméstico o a la deuda, el volumen global es tan sólo una fracción del mercado de renta fija. 

Las \textit{euroacciones} son títulos de renta variable colocadas fuera del mercado doméstico del emisor y denominadas en moneda distinta a la del país donde el comprador está situado. Este tipo de emisiones se llevan a cabo generalmente por empresas consolidadas que quieren ampliar sus posibilidades de financiación y que se encuentran limitadas por el tamaño relativamente reducido de sus mercados domésticos. La colocación es over-the-counter, y es posible que se lleve a cabo a través de sindicatos bancarios. Los inversores en este tipo de acciones son casi siempre grandes inversores institucionales que tienen algún interés en el accionariado de una gran entidad particular. Las euroacciones toman formas específicas de todo tipo: acciones ordinarias, preferentes, sin voto, warrants... Estos mercados internacionales son de aparición relativamente reciente, desde mediados de los años 80, y dependen fuertemente de la coyuntura bursátil y económica general. Como aspecto negativo, es destacable que pueden implicar una dispersión excesiva del accionariado minoritario y un refuerzo del poder de control del grupo dominante. 

Las \textit{acciones extranjeras} son acciones emitidas y negociadas en mercados domésticos, de tal manera que están sujetas a la legislación local del país de emisión, pero que otorgan derechos sobre el capital de una empresa radicada en el extranjero. Confieren los mismos derechos y existen en las misma variedad de modalidades que las euroacciones o que la renta variable doméstica, pero permiten al inversor exponerse a mercados extranjeros y diversificar su cartera aun disfrutando de mayor liquidez derivada de la cotización en mercados organizados domésticos que generalmente impliquen menores costes de transacción. 

Los \textit{recibos de depósito} son instrumentos negociables que representan la titularidad de una acción emitida y cotizada en el extranjero, denominada en el extranjero, de la cual una entidad doméstica es depositaria. Así, este tipo de instrumentos permiten a un inversor negociar valores de renta variable extranjeros que no cotizan en su mercado doméstico ni al cual no podría tener acceso directo. Los \textit{American Depository Receipts} representan acciones extranjeras en Nueva York, y los \textit{International Depository Receipts} representan acciones americanas en bolsas extranjeras. Los \textit{European Depository Receipts}, por su parte, representan la titularidad de acciones extranjeras en la Bolsa de Luxemburgo. Los principales mercados de negociación de estos instrumentos son Nueva York, Londres, Frankfurt y Luxemburgo.

La \textbf{crisis financiera} que se inició en 2007 y 2008 empujó a las autoridades a implementar numerosos cambios regulatorios. Además, el fuerte impacto negativo sobre la actividad económica transformó algunos aspectos de los mercados financieros internacionales de forma duradera. En el mercado de \textit{deuda soberana} la crisis ha introducido una mayor presencia de la emisión mediante sindicación bancaria, incluso en emisiones en moneda propia, con el objetivo de mejorar la liquidez de las emisiones y aumentar la cuantía. Además, la sindicación permite diversificar los inversores a través de la actuación discrecional del banco colocador. Los TIPs y los bonos con cupón ligado a la evolución de un índice han aumentado su demanda por temor a un aumento de la inflación futura. Las emisiones en divisa extranjera también han mostrado una tendencia creciente, y generalmente están asociadas a la contratación de un swap de divisas. 

En el mercado de deuda de entidades financieras, el contexto internacional ha estado caracterizado por la presencia creciente de \textit{covered bonds} para obtener financiación de medio y largo plazo. Estos covered bonds, en los que el inversor tiene derecho a actuar contra el balance completo del emisor en caso de impago, han reducido la cuota de mercado de los \textit{residential mortgage backed securities} habituales en el periodo pre-crisis. Los programas de \textit{quantitative easing} han favorecido esta tendencia. Además, el carácter no ``bailinable'' de este tipo de bonos ha aumentado fuertemente su demanda internacional. En general, se ha producido una desincentivación del llamado ``originate to distribute''. Esta práctica consistía en otorgar el mayor volumen posible de créditos hipotecarios y de consumo para luego reempaquetarlos en y vender los derechos a los flujos de caja como activos negociables independientes cuyo posible impago no implica el recurso contra los activos de la entidad que inicialmente concede el préstamo al consumidor final. En cuanto a las \textit{entidades no financieras}, la crisis ha provocado en Europa un estado de opinión favorable a reemplazar o al menos reducir la financiación bancaria por la financiación directa a través de la emisión de títulos negociables. La crisis financiera mostró como los países fuertemente bancarizados son más vulnerables a las turbulencias financieras. Además, las políticas de relajación cuantitativa y otras medidas de política monetaria no ortodoxa son más fácilmente ejecutables en economías menos bancarizadas. Una excesiva bancarización puede también suponer un obstáculo a la internacionalización del sistema financiero y a la concentración de riesgos. Por ello, la Unión Europea trata de promover la creación de mercados de capital en los que las PYMES puedan financiarse sin recurrir al sistema bancario. El Mercado Alternativo de Renta Fija creado en España en 2013 es un ejemplo.

La \textit{normativa bancaria} ha sido el objeto de un impulso reformista a partir de la crisis bancaria. El acuerdo de Basilea III, cuya primera versión apareció en 2009, fue una respuesta de las autoridades regulatorias de las principales economías para mejorar la solvencia y la estabilidad del sistema bancario y tratar de evitar en el futuro las quiebras e inyecciones de capital que se habían producido en los primeros momentos de la crisis. El marco de Basilea III impone requisitos más estrictos de capital y se orienta de forma más explícita a las entidades con exposición internacional. En la actualidad se encuentra aún en proceso de implementación. En la Unión Europea, la directiva CRD IV y el reglamento CRR II son la manifestación de la implementación de Basilea III. Introducen normas adicionales sobre el gobierno corporativo y requisitos adicionales para entidades sistémicas. Además, la Unión Europea introdujo legislación comunitaria adicional para mejorar el proceso de resolución de entidades de crédito con problemas y tratar de reducir el feedback negativo entre rescates bancarios y deuda soberana que tanta volatilidad habían generado en los mercados financieros internacionales. En Estados Unidos fue especialmente significativa la Ley Dodd-Frank de 2010, que incorpora la llamada Regla Volcker para limitar las actividades especulativas con sus propios fondos que llevan a cabo los bancos comerciales. 

Las \textit{agencias de calificación} recibieron críticas por sus actuaciones en el periodo previo a la crisis. Los conflictos de interés derivados de la relación de las agencias con los sujetos de sus valoraciones de calidad crediticia, el poder de mercado derivado de la concentración en tres agencias principales, la lentitud a la hora de actualizar las valoraciones y la poca transparencia de los modelos internos fueron los factores principales de crítica. En el contexto europeo, un reglamento y una directiva fueron aprobados con el fin de tratar de reducir algunos de estos factores. 

El mercado de \textit{derivados} fue también objeto de gran controversia a raíz de la crisis. La capacidad de los derivados para aumentar el apalancamiento de una posición sin incurrir en costes elevados de transacción ha sido el principal factor. Los intentos de regulación han estado centrados en aumentar la transparencia de las posiciones a menudo fuera de balance, y a promover o obligar a las entidades a utilizar \textit{central counter parties} para liquidar sus posiciones y disminuir el riesgo de contrapartida y reducir la posibilidad de shocks sistémicos. Según datos del Banco Internacional de Pagos, los volúmenes de contratación de derivados en los mercados internacionales disminuyeron fuertemente en el periodo inmediatamente posterior a la crisis hasta 2011, y después repuntaron hasta 2014. En los tres últimos años se aprecia un ligero estancamiento o incluso disminución en algunos segmentos.

A lo largo de la exposición hemos examinado las características más relevantes de los agentes que operan en los mercados internacionales en general, las particularidades del mercado de renta fija y sus instrumentos, y de igual forma para el mercado de renta variable. Por último, hemos examinado algunos de cambios principales que se han producido en los mercados financieros internacionales como resultado de la crisis financiera. Como reflexión final, es necesario tener presente que los mercados financieros internacionales son sistemas muy complejos e interconectados, sujetos a constantes dińamicas de evolución que reflejan avances tecnológicos, factores políticos, cambios legales y fluctuaciones de la economía real. Por ello, para entender el funcionamiento y poder anticipar su evolución futura, es necesario prestar atención a las dinámicas de fondo y no sólo a su situación en un momento determinado en el tiempo.


\seccion{Preguntas clave}
\begin{itemize}
    \item ¿Qué son y para qué sirven los mercados financieros?
    \item ¿Cómo funcionan los mercados financieros?
    \item ¿Quién participa en los mercados financieros?
    \item ¿Qué activos se negocian en los mercados financieros?
    \item ¿Cómo han evolucionado los mercados financieros?
    \item ¿Qué papel juega la política económica en relación a los mercados financieros?
\end{itemize}

\esquemacorto

\begin{esquema}[enumerate]
	\1[] \marcar{Introducción}
		\2 Contextualización
			\3 Contrato financiero
			\3 Mercados financieros
			\3 Mercados financieros internacionales
		\2 Objeto
			\3 Que son y para qué sirven
			\3 Cómo funcionan
			\3 Quiénes participan
			\3 Qué activos se negocian
			\3 Cómo han evolucionado
		\2 Estructura
			\3 Agentes
			\3 Instrumentos de renta fija
			\3 Instrumentos y mercados de renta variable
			\3 Cambios post-crisis
	\1 \marcar{Agentes en los mercados financieros internacionales}
		\2 Instituciones financieras
			\3 Idea clave
			\3 Instituciones financieras del mercado monetario
			\3 Agencias financieras
			\3 Fondos de titulización
			\3 Fondos de inversión
			\3 Hedge funds
			\3 Private Equity Funds
		\2 Sociedades no financieras (corporates)
			\3 Emisiones de activos
			\3 Inversiones
			\3 Cobertura
		\2 Gobiernos
			\3 Financiación mediante deuda
			\3 Fondos soberanos
		\2 Organizaciones internacionales
			\3 Bancos de desarrollo
		\2 Entidades de estandarización
			\3 Contabilidad
			\3 Auditoría
			\3 Finanzas islámicas
	\1 \marcar{Características generales de los mercados financieros internacionales}
		\2 Elementos determinantes de la eficiencia
			\3 Amplitud
			\3 Profundidad
			\3 Libertad
			\3 Transparencia
			\3[$\then$] Implican que el mercado se acerca a eficiencia
		\2 Clasificación de mercados financieros
			\3 Plazo de las operaciones
			\3 Organización
			\3 Momento de emisión
			\3 Tipo de activo
			\3 Supervisores
			\3 Volúmenes de contratación
		\2 Características comunes de los mercados financieros internacionales
	\1 \marcar{Instrumentos y mercados de renta fija}
		\2 Mercado monetario
			\3 Depósitos y certificados de depósito
			\3 Repos o cesiones temporales de activos
			\3 Papel comercial
			\3 Letras del tesoro
		\2 Notas y bonos
			\3 Idea clave
			\3 Emisiones internacionales de bonos
			\3 EMTN -- Notas a medio plazo
			\3 Deuda negociable colateralizada
			\3 Bonos verdes
		\2 Préstamos sindicados en divisas
			\3 Características
			\3 Modalidades
		\2 Swaps de divisas entre bancos centrales
			\3 Características
			\3 Uso
		\2 Mercados de derechos de emisión -- ETS de la UE
			\3 Idea clave
			\3 Formulación
			\3 Valoración
		\2 Segmentos del mercado de bonos
			\3 Idea clave
			\3 Inversión
			\3 Especulativo/leverage/high yield
			\3 Segmento sin calificar
			\3 Otras
			\3 Valoración
	\1 \marcar{Instrumentos y mercados de renta variable}
		\2 Euroacciones
			\3 Instrumentos
			\3 Mercado primario
			\3 Mercado secundario
			\3 Valoración
		\2 Acciones extranjeras
			\3 Idea clave
			\3 Valoración
		\2 Depository receipts o recibos de depósito
			\3 Idea clave
			\3 Variantes
			\3 Principales mercados
			\3 Valoración
		\2 CDO -- Equity Tranche
			\3 Idea clave
			\3 Instrumentos
			\3 Emisores
			\3 Valoración
		\2 Bonos islámicos
			\3 Idea clave
			\3 Formulación
			\3 Valoración
		\2 Private equity
			\3 Idea clave
			\3 Agentes
			\3 Valoración
		\2 Venture capital
			\3 Idea clave
			\3 Agentes
			\3 Elevada concentración geográfica
			\3 Valoración
	\1 \marcar{Cambios post-crisis}
		\2 Deuda soberana
			\3 Sindicación
			\3 TIPs e index-linked-bonds
			\3 Emisiones divisa extranjera
		\2 Deuda entidades financieras
			\3 Covered bonds ganan importancia
			\3 Tipos bajos y liquidez abundante
		\2 Cláusulas de acción colectiva
			\3 Idea clave
			\3 Single-limb CAC / votación única
			\3 Double-limb CAC / votación con doble vuelta
			\3 EURO-CACs
		\2 Esfuerzos para desintermediar sistema financiero
			\3 Tipos de sistema financiero
			\3 Problemas de intermediación bancaria
			\3 Sustitución crédito bancario por bonos
		\2 Cambios en regulación financiera
			\3 Basilea III
			\3 CRD IV/CRR II
			\3 Dodd-Frank 2010
			\3 Normativa de resolución de entidades de crédito
			\3 Alternative Investment Funds Manager Directive (2011)
		\2 Agencias de calificación
			\3 Críticas post-crisis
			\3 Regulación europea
		\2 Derivados
			\3 Intentos de regulación
			\3 Cuantificación
		\2 Reducción de la cotización en bolsa
		\2 Private equity
		\2 Crisis Covid-19
			\3 Política monetaria
			\3 Evolución de índices de equity
			\3 Emisión de bonos
	\1[] \marcar{Conclusión}
		\2 Recapitulación
			\3 Agentes mercados financieros internacionales
			\3 Instrumentos y mercados
			\3 Cambios post-crisis
		\2 Idea final
			\3 Mercados reflejan
			\3 Evolución constante

\end{esquema}

\esquemalargo

\begin{esquemal}
	\1[] \marcar{Introducción}
		\2 Contextualización
			\3 Contrato financiero
				\4 Suavización de la senda de consumo
				\4[] Trasladar rentas presente futuro y vv.
				\4 Asegurar frente a riesgos
				\4[] Resultado de la aversión al riesgo de agentes
			\3 Mercados financieros
				\4[] Concepto abstracto
				\4[] $\to$ Mercados físicos
				\4[] $\to$ Mercados virtuales
				\4 Matching de partes de un contrato
				\4[] Partes de un contrato deben encontrarse
				\4[] Mercados centralizados simplifican proceso
				\4[] Pero no siempre posible
			\3 Mercados financieros internacionales
				\4 Amplían posibilidades de matching
				\4 Reducen costes y mejoran eficiencia:
				\4[] $\to$ Mayor variedad de contrapartes
				\4[] $\to$ Mayor variedad de contratos
				\4[] $\to$ Menores costes de información
				\4[] $\to$ Menores costes de intermediación
				\4 Apertura de mercados
				\4[] Desde años 80
				\4[] Explosión de flujos financieros internacionales
				\4 Impacto sobre economía mundial
				\4[] Suavización de shocks transitorios
				\4[] Pueden incentivar acumulación excesiva de deuda
				\4[] Pueden tener consecuencias sistémicas mundiales
		\2 Objeto
			\3 Que son y para qué sirven
			\3 Cómo funcionan
			\3 Quiénes participan
			\3 Qué activos se negocian
			\3 Cómo han evolucionado
				\4 Qué cambios tras la crisis financiera
		\2 Estructura
			\3 Agentes
			\3 Instrumentos de renta fija
			\3 Instrumentos y mercados de renta variable
			\3 Cambios post-crisis
	\1 \marcar{Agentes en los mercados financieros internacionales}
		\2 Instituciones financieras
			\3 Idea clave
				\4 Principales emisores de pasivos
				\4[] ~80\% del total
				\4 Principales tenedores de activos
				\4[] Inversión colectiva o particular
			\3 Instituciones financieras del mercado monetario
				\4 Ejemplos
				\4[] Bancos
				\4[] Fondos del mercado monetario
				\4 Actividad
				\4[] Emiten pasivos a corto plazo
				\4[] Equivalentes a dinero
				\4[] $\to$ Liquidez máxima
				\4[] $\to$ Interés cercano a cero
			\3 Agencias financieras
				\4 Instituciones públicas
				\4[] Pasivos cuentan con respaldo gubernamental
				\4 Concesión crédito o avales a sectores específicos
				\4 Fannie Mae, Ginnie Mae, Freddie Mac, Sallie Mae...
			\3 Fondos de titulización
				\4 Compran deuda no negociable
				\4[] Préstamos particulares
				\4[] Crédito comercial
				\4[] Créditos al consumo
				\4[] Préstamo hipotecario
				\4 Emiten bonos
				\4[] Ligados a deuda no negociable
				\4[] Inversores cobran si deudores originales pagan
			\3 Fondos de inversión
				\4 Instituciones de inversión colectiva
				\4[] Invierten fondos de múltiples orígenes
				\4[] Divididos en participaciones
				\4[] $\to$ Otorgan derechos sobre rendimientos
				\4 Ventajas
				\4[] Economías de escala por gestión profesional
				\4[] Reducción de costes de información
				\4[] Economías de escala por costes de transacción
				\4 Componentes
				\4[] Sociedad gerente
				\4[] $\to$ Decide objeto de inversión
				\4[] Sociedad depositaria
				\4[] $\to$ Custodia valores representativos de inversiones
				\4 Gestión
				\4[] Pasiva
				\4[] $\to$ Replicar rendimientos de un índice o cartera
				\4[] Activa
				\4[] $\to$ Elegir valores en que invertir
			\3 Hedge funds
				\4 Participación restringida y privada
				\4 Objetivo
				\4[] Generar retornos absolutos a c/p
				\4[] $\to$ Generalmente, estrategias activas
				\4 Apalancamiento
				\4[] Habitualmente elevado
				\4[] Papel en crisis financieras: Soros, crisis asiática, LTCM...
				\4[] Asumen riesgos importantes
				\4 Relacionados con bancos hasta crisis
				\4 Concentración de inversiones
				\4 Papel en gobierno corporativo
				\4[] Especialmente relevante para private equity funds
			\3 Private Equity Funds
				\4 Participación restringida y privada
				\4 Inversión en accionariado
				\4 Participación activa en gestión de la empresa
				\4 Equipo de expertos en gestión corporativa
				\4 Horizonte de inversión largo
				\4[] Habitualmente, participaciones no son líquidas
				\4[] Restricciones de 3,5, 10 años
				\4 Especialmente importantes en IDE
				\4[] Brownfield y Greenfield
		\2 Sociedades no financieras (corporates)
			\3 Emisiones de activos
				\4 Fija: crédito y títulos de deuda
				\4 Variable: acciones
			\3 Inversiones
				\4 Inversiones financieras de empresas
			\3 Cobertura
				\4 Operaciones en mercados de derivados
				\4 Transferir riesgo
				\4[] Cambiario, contrapartida...
		\2 Gobiernos
			\3 Financiación mediante deuda
				\4 Bancaria
				\4 Emisión de bonos
			\3 Fondos soberanos
				\4 Noruega (Government Pension Fund)
				\4 China Investment Corporation
				\4 Abu Dhabi Investment Authority
				\4 Kuwait Investment Authority
				\4 Arabia Saudí
				\4 Hong Kong
				\4 Singapur
				\4 Qatar
				\4 Inversiones en activos seguros
				\4[] Consideraciones políticas y de seguridad relevantes
		\2 Organizaciones internacionales
			\3 Bancos de desarrollo
				\4 Máxima calidad crediticia
				\4 Mercado secundario relativamente activo
				\4 Banco Europeo de Inversiones
				\4 Banco Asiático de Inversiones
				\4 Banco Mundial
				\4 Banco Asiático de Infraestructuras y Desarrollo (AIIB)
		\2 Entidades de estandarización
			\3 Contabilidad
				\4 International Accounting Standards Board
				\4 Elaboración de las NIIF
			\3 Auditoría
				\4 International Federation of Accountants (IFAC)
				\4[] Elaboración de International Standards of Accounting
				\4 IPIOB -- International Public Interest Oversight Board
				\4[] Sede en Madrid
			\3 Finanzas islámicas
				\4 Islamic Financial Services Board
				\4 Accounting and Auditing Organization for Islamic Financial Institutions
	\1 \marcar{Características generales de los mercados financieros internacionales}
		\2 Elementos determinantes de la eficiencia
			\3 Amplitud
				\4 Mayor gama de valores negociados
				\4[] $\to$ Mayor posibilidad de cubrir estados de naturaleza
			\3 Profundidad
				\4 Depende de volumen de negociación
				\4 Capacidad de absorción de órdenes de compra/venta
				\4[] Volumen que puede ser ejecutado
				\4[] $\to$ Sin afectar significativamente al precio
			\3 Libertad
				\4 Entrada y salida libre
				\4 Coste reducido de transacción
			\3 Transparencia
				\4 Obtención de información no es costosa
			\3[$\then$] Implican que el mercado se acerca a eficiencia
		\2 Clasificación de mercados financieros
			\3 Plazo de las operaciones
				\4 Mercados monetarios
				\4[] Deuda con vencimiento a corto plazo
				\4[] $\to$ Generalmente menos de 18 meses
				\4 Mercados de capitales
				\4[] Deuda con vencimiento a medio y largo plazo
				\4[] Acciones
			\3 Organización
				\4 Mercados organizados
				\4[] Reglas estandarizadas de transacción y liquidación
				\4[] Contratos estandarizados
				\4 Mercados no organizados
				\4[] $\to$ Over-the-counter
				\4[] Negociación, liquidación son ad-hoc entre partes
			\3 Momento de emisión
				\4 Mercados primarios
				\4[] Se emiten nuevos títulos
				\4 Mercados secundarios
				\4[] Títulos ya emitidos
			\3 Tipo de activo
				\4 Mercados de renta fija
				\4 Mercados de renta variable
			\3 Supervisores
				\4 Nacionales
				\4 Internacionales
				\4 No regulados
			\3 Volúmenes de contratación
				\4 Deuda pública anotada
				\4 Renta fija
				\4 Renta variable
		\2 Características comunes de los mercados financieros internacionales
	\1 \marcar{Instrumentos y mercados de renta fija}
		\2 Mercado monetario
			\3 Depósitos y certificados de depósito
				\4 Deuda a corto plazo
				\4 Generalmente 1 a 6 meses
				\4 Sin depósitos a la vista en euromercado
				\4 Mercado interbancario de depósitos
				\4 Certificados de depósitos negociables
			\3 Repos o cesiones temporales de activos
				\4 Compras y ventas a c/p y precio determinado
				\4 Financiación a corto plazo
				\4 Rentabilización picos de tesorería
			\3 Papel comercial
				\4 Renta fija corporativa a corto plazo
				\4 3, 6, 9, 12, 18 meses
				\4 Colocación
				\4[] Subasta
				\4[] sindicato bancario
				\4[] departamentos tesorería
			\3 Letras del tesoro
				\4 Hasta 12 meses
				\4 Orientación doméstica inicial
				\4 Tendencia a internacionalización en Europa
				\4 Sin cupón
		\2 Notas y bonos\footnote{Ver \href{https://ec.europa.eu/transparency/regexpert/index.cfm?do=groupDetail.groupDetailDoc&id=35768&no=1}{EC (2017): Analysis of European Corporate Bond Markets}.}
			\3 Idea clave
				\4 Vencimiento de 18 meses en adelante
				\4 Pago de cupón
				\4 Posible convertibilidad a equity
				\4 Posible warrants
				\4 Debentures
				\4[] Pago garantizado por un conjunto de activos
				\4[] Menor interés
				\4 Sindicación bancaria
				\4 Underwriters liderados por lead manager
			\3 Emisiones internacionales de bonos
				\4 Eurobonos:
				\4[] Emitidos por agente internacional
				\4[] Comprados por inversores
				\4[] $\to$ En países con moneda distinta del bono
				\4[] Ejemplo:
				\4[] $\to$ Empresa USA emite en Japón bonos en USD
				\4[] Generalmente listados en Londres o Luxemburgo
				\4 Bonos extranjeros
				\4[] Emitidos en mercado extranjero
				\4[] Comprados por inversores locales
				\4[] $\to$ Denominados en moneda local de inversor
				\4[] Ejemplo:
				\4[] $\to$ Empresa USA emite en Japón bonos en JPY
				\4[] $\to$ Bono samurai
				\4[] Yankee, bulldog, samurai, matador....
				\4 Emisiones soberanas en moneda propia
				\4[] Europa: cupón anual
				\4[] EEUU y Japón: cupón semestral
				\4 Bonos Brady\footnote{A iniciativa del Tesoro americano surgen en los 80 como alternativa para dar salida a los problemas de insolvencia de países en desarrollo. Emitidos por el país en desarrollo en cuestión tras reestructurar su deuda. Ver \href{https://www.investopedia.com/terms/b/bradybonds.asp}{Investopedia: Brady Bonds}.}
				\4[] Primera emisión en 1989
				\4[] Marco de reestructuración de PEDs tras crisis 80s
				\4[] Bancos comerciales intercambian deuda bancaria
				\4[] $\to$ Por bonos brady
				\4[] Bonos brady respaldados por bonos de tesoro americano
				\4[] $\to$ Bonos tesoro con cupon cero
				\4[] $\to$ Generalmente en dólares
				\4[] $\to$ Vencimiento de largo plazo (30 años)
				\4[] $\to$ Bonos de resplado en escrow en reserva federal
			\3 EMTN -- Notas a medio plazo
				\4 Euro Medium Term Notes
				\4 Explosión en últimas décadas
				\4 Vencimiento menos de 5 años
				\4 Acuerdos marco de emisión
				\4[] Emisiones de bonos tienen costes fijos elevados
				\4[] EMTN permiten emisiones continuas
				\4[] $\to$ Basadas en acuerdo marco
				\4[] $\to$ Pequeñas emisiones periódicas
				\4[] $\to$ A menudo, roll-over de emisiones
				\4 También euromercados
				\4 Tipo fijo o flexible
				\4 Posible rescatables antes de vencimiento
				\4 Intermediario coloca poco a poco
			\3 Deuda negociable colateralizada
				\4 Idea clave:
				\4[] Renta fija respaldada por activos
				\4 Asset Backed Securities -- ABS
				\4[] gran variedad de activos colaterales
				\4[] Mortgage Based Securities -- MBS
				\4[] $\to$ Hipotecas son colateral
				\4 CDOs
				\4 Covered bonds
				\4[] Deuda bancaria de máxima calidad
				\4[] Muy habitual en Unión Europea
				\4[] $\to$ Pfandbriefe alemanes estándar en UE
				\4[] Activos colaterales forman parte de balance
			\3 Bonos verdes
				\4 Idea clave
				\4[] Emisiones destinadas a financiar proyectos MA
				\4[] Banco Mundial pionero en 2009
				\4[] Fondos ligados a proyecto determinado
				\4 Incentivos
				\4[] Instrumento de política medioambiental
				\4[] Incentivos fiscales
				\4[] Proyectos de regulación bancaria favorable
				\4[] $\to$ ¿Requisitos de capital reducidos?
				\4 Valoración
				\4[] Especial crecimiento en últimos años
				\4[] Difícil certificar uso de fondos
				\4[] Mucho ruido en etiqueta ``verde''
				\4[] $\to$ Utilizado como instrumento de marketing

		\2 Préstamos sindicados en divisas
			\3 Características
				\4 Tipo variable generalmente
				\4 Roll-over: interés fijos cada 3 o 6 meses
				\4 Comisiones iniciales, compromiso por crédito no dispuesto, y agencia
				\4 Posible mercado secundario de préstamos sindicados.
			\3 Modalidades
				\4 Best effort
				\4[] Sin compromiso de suscripción por parte de bancos
				\4[] Si mercados secundarios no cubren
				\4[] $\to$ Bancos no se comprometen a cubrir
				\4 Grupo asegurador: underwriting
				\4 Club loan
		\2 Swaps de divisas entre bancos centrales\footnote{Ver \url{https://voxeu.org/article/central-bank-swap-lines}, y \href{https://www.imf.org/external/pubs/ft/bop/2018/pdf/Clarification0518.pdf}{FMI}.}
			\3 Características
				\4 Entre bancos centrales
				\4 Intercambio inicial de divisas
				\4[] A tipo spot
				\4[] P.ej:
				\4[] BCE transfiere euros a Fed
				\4[] Fed transfiere dólares a BCE
				\4 Préstamo a agentes privados que necesitan liquidez
				\4[] BCE presta con interés
				\4[] $\to$ Interés que paga a Fed+posible prima
				\4 Final del swap
				\4[] Partes devuelven cantidades con interés respectivo
			\3 Uso
				\4 Especialmente apropiado en crisis de liquidez
				\4 No es apropiado para financiar balanza de pagos
				\4 Alternativas:
				\4[] Bancos centrales prestan reservas de divisas
				\4[] $\to$ Sin acudir a emisor de divisa
				\4[] Bancos privados extranjeros se financian directamente
				\4[] $\to$ Tomando prestada divisa directamente de emisor
				\4[] Agentes toman prestado en moneda local
				\4[] $\to$ Y eliminan riesgo de TC vía derivados
				\4[] $\to$ En crisis, CIP puede no cumplirse\footnote{Tal y como sucedió en la crisis financiera de 2007-2009, en la que pedir prestado en euros y convertir a dólares vía un FX swap era más caro que pedir prestado en dólares.}
		\2 Mercados de derechos de emisión -- ETS de la UE
			\3 Idea clave
				\4 Contexto
				\4[] $\to$ Creado en 2005
				\4[] $\to$ Teorema de Coase
				\4[] $\to$ Internalización externalidad negativa
				\4[] $\to$ Sistema de precios como herramienta de transmisión de información
				\4 Objetivo
				\4[] $\to$ Emisores internalicen costes externos
				\4[] $\to$ Reducir distorsiones de mercado
				\4[] $\to$ Incentivar energías menos contaminantes
				\4 Resultado
				\4[] $\to$ Progresivos avances
				\4[] $\to$ Mercados de emisiones institucionalizados
			\3 Formulación
				\4 Asignación de derechos de emisión
				\4[] $\to$ Medidos en equivalentes de CO2
				\4[] $\to$ Sector industrial recibe derechos gratuitos
				\4[] $\to$ Asignación equitativa en toda UE
				\4[] $\to$ Más asignación a expuestos a fuga de carbono\footnote{Deslocalización de industrias contaminantes hacia jurisdicciones fuera de la UE con protección más laxa.}
				\4[] $\to$ Más asignación a plantas más eficientes
				\4[] $\to$ Más asignaciones a cumplidores de Kioto
				\4[] $\to$ Reducción anual de derechos asignados
				\4 Certificación de emisiones
				\4[] $\to$ Verificador acreditado
				\4 Compraventa de derechos
				\4[] $\to$ Vende derechos si emitió menos
				\4[] $\to$ Debe comprar derechos si emitió más
				\4[] $\to$ Derechos no reutilizables
				\4[] $\to$ Multas por incumplimiento superiores a precio de derechos
				\4 Implicaciones
				\4[] Recesiones globales hacen caer precio
				\4[] $\to$ Puede aumentar emisiones
				\4[] Necesario compensar exceso de derechos
				\4[] $\to$ Market Stability Reserve
				\4[] Market Stability Reserve
				\4[] $\to$ Intervención one-off
				\4[] $\to$ Reducción derechos si circulación excede límite
				\4[] $\to$ Aumento de derechos si circulación menor que límite
				\4[] $\to$ Controlar déficits/superávits estructurales
			\3 Valoración
				\4 Liquidez escasa en ocasiones
				\4 Precios demasiado bajos durante años
				\4 Revisiones periódicas
				\4 Fase 3: 2013
				\4 Fase 4: 2021-2030
				\4 Se espera aumento de precios
				\4[] $\to$ Máximo de 10 años en verano 2018
				\4 Precio de futuros Diciembre 2019
				\4[] $\to$ 21 € en febrero de 2019
		\2 Segmentos del mercado de bonos\footnote{Ver \href{https://www.vernimmen.net/Lire/Lettre_Vernimmen/Lettre_178.html}{Lettre Vernimmen 178}.}
			\3 Idea clave
				\4 Calificación crediticia
				\4[] Medida de posibilidad de impago
				\4 Continuo de calificaciones
				\4[] Desde AAA hasta C
				\4[] $\to$ Variaciones de nombres según agencia
				\4 Realmente, salto entre denominaciones
				\4[] Dos segmentos diferenciados
				\4[] $\to$ Especulativo
				\4[] $\to$ Inversión
				\4[] $\then$ Distinción más relevante
				\4[] $\then$ Diferencias de inversores
				\4[] Frontera teórica
				\4[] $\to$ Entre BBB- y BB+
				\4 División del trabajo para segmentos
				\4[] Diferentes equipos de análisis
				\4[] Diferentes inversores objetivo
			\3 Inversión
				\4 Entre AAA y BBB-
				\4[] Frontera variable
				\4[] En pre-crisis
				\4[] $\to$ Frontera ligeramente por debajo
				\4[] $\to$ Más hacia BB
				\4[] Tras crisis
				\4[] $\to$ Frontera sube a BBB
				\4 Documentación obligatoria
				\4[] Mucho más simple
				\4[] Estandarizada
				\4[] Proceso administrativo corto
				\4 Protección de inversores
				\4[] Default de otra emisión de la empresa
				\4[] $\to$ Implica default de esta emisión
				\4[] $\then$ Cross-default
				\4[] Mismo rango de seniority
				\4[] $\to$ Igual trato por emisor
				\4[] $\then$ Pari-passu
				\4[] Cambio de control
				\4[] $\to$ Inversores pueden exigir reembolso
				\4[] $\then$ En caso de cambio de control de empresa
				\4[] $\then$ Posible solo si cambio de calificación asociada
				\4 Programas EMTN
				\4[] Documentación actualizada anualmente
				\4[] Habitual en investment grade
			\3 Especulativo/leverage/high yield
				\4 De BB+ hacia abajo
				\4 Documentación obligatoria compleja
				\4[] Descripción detallada de empresa
				\4[] $\to$ Estrategia
				\4[] $\to$ Mercado
				\4[] $\to$ Información financiera
				\4 Roadshows
				\4[] Presentación a inversores
				\4 Inversores más sofisticados
				\4[] Verdaderos análisis de la situación
				\4[] Conocimiento del mercado
				\4[] $\to$ Análisis similar a introducción en bolsa
				\4 Protección de inversores
				\4[] Ratios financieros a cumplir
				\4[] $\to$ Deuda, solvencia, etc...
				\4[] Limitación de cesión de activos
				\4[] Limitación de adquisiciones y fusiones
				\4[] Limitación de ciertos pagos
				\4[] $\to$ Dividendos, buybacks, etc...
				\4[] Dedicación de dividendos de filiales
				\4[] $\to$ A repago de bonos
				\4[] ...
				\4 Garantías
				\4[] SSN -- Senior Secured Notes
				\4[] $\to$ Emisiones con garantías
				\4[] $\to$ Garantías no compartidas con otras emisiones
				\4[] $\to$ Habitualmente títulos de filiales
				\4[] SUN -- Senior Unsecured Notes
				\4[] $\to$ Emisiones sin garantías
				\4 Fuera de programas EMTN
				\4[] Emisiones individuales
				\4[] $\to$ Integran toda la documentación
			\3 Segmento sin calificar
				\4 Algunas empresas con características especiales
				\4[] Prefieren no pagar por calificación crediticia
				\4[] Mercado conoce bien sus características
				\4 Inversores atribuyen nota implícita
				\4 Interés ligeramente superior a calificadas
				\4 Vencimientos generalmente inferiores a 8 años
			\3 Otras
				\4 Cov-light
				\4[] Para leverage con buena calificación
				\4[] $\to$ Covenants reducidos
				\4 Fallen angels
				\4[] Investment grade que ha perdido calificación
				\4[] $\to$ Debe ahora emitir en segmento leverage
				\4[] $\then$ Posible sólo provisionalmente
			\3 Valoración
				\4 Fronteras relativamente fluidas entre segmentos
				\4[] Especialmente en momentos de crisis
				\4 Acceso a mercado cerrado en situaciones críticas
				\4[] Especialmente para high-yield
				\4 Investment grade en crisis covid
				\4[] Aumento enorme de emisión de bonos
				\4[] $\to$ Asegurarse liquidez
	\1 \marcar{Instrumentos y mercados de renta variable}
		\2 Euroacciones
			\3 Instrumentos
				\4 Acciones ordinarias
				\4 Acciones preferentes
				\4 Acciones sin voto
				\4 Warrants
				\4 Multidivisa...
			\3 Mercado primario
				\4 fuera de mercado doméstico
				\4 Generalmente OTC
				\4 Sindicatos bancarios
				\4 Emisores
				\4[] Principalmente americanos y europeos
			\3 Mercado secundario
				\4 Cotización extra bursátil
				\4 Agentes habituales
				\4[] Private equity funds
				\4[] Capital riesgo
				\4[] Venture capital
			\3 Valoración
				\4 Aparición reciente
				\4[] Mediados de los 80
				\4 Dependiente coyuntura bursátil
				\4 Ampliación de base de accionistas
				\4[] Especialmente atractivo empresas tecnológicas
				\4 Dispersión del accionariado
				\4[] $\rt$ refuerzo control grupo dominante
		\2 Acciones extranjeras\footnote{\href{https://www.investopedia.com/terms/c/cross-listing.asp}{Investopedia: cross-listing}.}
			\3 Idea clave
				\4 Acciones de empresas extranjeras
				\4 Cotizadas en mercado local directamente
				\4 ``Cross-listing''
				\4 Sujetas a legislación local en país de cotización
				\4[] $\to$ De país de emisión
				\4[] $\to$ No del país de origen de emisor
				\4 Representan capital empresas extranjeras
				\4[] = derechos que euroacciones o renta variable
			\3 Valoración
				\4 Aumento de liquidez
				\4[] Franja horaria de cotización más amplia
				\4[] Mayor variedad de monedas de denominación
				\4 Mejora de imagen
				\4[] Empresa aumenta exposición a público inversor
				\4[] Más atención en mercados extranjeros
				\4[] $\to$ Si también cotizadas en mercado extranjero
				\4 En Estados Unidos, requisitos más exigentes
				\4[] Habitualmente, Depositary Receipts preferidos
				\4 Mantenimiento de stock primario
		\2 Depository receipts o recibos de depósito
			\3 Idea clave
				\4 Certificado negociable emitido por banco
				\4 Representa acciones en empresa extranjera
				\4[] Cotizada en mercado extranjero
				\4 Permite exposición a valores extranjeros
				\4 Requisitos menos exigentes a cross-listing
				\4 Denominadas en moneda local de emisor de DRececeipt
				\4[] Inversores no necesitan negociar en divisas
			\3 Variantes
				\4 American Depository Receipts
				\4[] Acciones extranjeras en NY
				\4 International/Global Depository Receipts
				\4[] Acciones americanas en bolsas extranjeras
				\4 European Depository Receipts
				\4[] Acciones extranjeras en Bolsa de Luxemburgo
			\3 Principales mercados
				\4 Londres
				\4 Nueva York
				\4 Frankfurt
				\4 Luxemburgo
			\3 Valoración
				\4 No es posible acceder a todas las empresas
				\4[] Generalmente, sólo más importantes de mercados extranjeros
				\4 Liquidez habitualmente menor
				\4 Riesgo de divisa no se elimina
				\4[] Conversión automática sujeta a condiciones de banco emisor
				\4 Permite diversificación más allá de cross-listing
		\2 CDO -- Equity Tranche\footnote{Ver \href{http://www.afi.es/EO/CDOs.pdf}{AFI (2009): Collateralized debt obligations}.}
			\3 Idea clave
				\4 Derivados de crédito
				\4 Masa subyacente de activos que generan CFlow
				\4[] Revendidos a inversores en CDO
				\4[] $\to$ Diferentes órdenes de preferencia para recibir
				\4[] $\then$ Equity menor preferencia
				\4 Orden de preferencia en recepción de flujos
				\4[] ``Tranches'' o segmento
				\4[] Primer tranche: senior
				\4[] $\to$ Calificación de A a AAA
				\4[] $\to$ Prioridad para recibir flujos
				\4[] $\then$ Similar a deuda
				\4[] Segundo segmento: mezzanine
				\4[] $\to$ Siguiente preferencia para recibir
				\4[] $\to$ Suelen recibir calificación de deuda
				\4[] Último segmento: equity
				\4[] $\to$ Última preferencia
				\4[] $\to$ Cobran si todos cobran
				\4 Colocación internacional
			\3 Instrumentos
				\4 Depende de activo subyacente
				\4 CDO propriamente
				\4 CMO -- CMortgage Obligation
				\4 CLO -- CLoan Obligation
			\3 Emisores
				\4 Generalmente, sindicatos bancarios
				\4 A través de SPV
				\4 Habitual que bancos se queden con segmento equity
			\3 Valoración
				\4 No son propiamente equity
				\4 Realmente, derivado de crédito
				\4[] Con pagos similares a equity
				\4 Emisión masiva en EEUU
				\4[] Reducir exposición directa a créditos
				\4[] Originate-to-distribute
				\4 Papel en crisis de 2009
				\4[] Bancos mantienen segmento equity
				\4[] Colocan a terceros internacionalmente
				\4[] Amplificación de shocks
				\4 Transmisión de shocks a nivel internacional
		\2 Bonos islámicos
			\3 Idea clave
				\4 Corán: prohibido pago de interés
				\4 Financiación de proyectos
				\4[] Debe evitar interés
				\4 Sukuk
				\4[] relacionado con ``cheque''
				\4 A pesar de nombre ``bonos''
				\4[] $\to$ Cierta similitud con equity
				\4[] $\then$ Sujetos a fluctuación de valor de activos
			\3 Formulación
				\4 Empresa quiere financiar proyecto
				\4[] No posee fondos
				\4[] Posee activos fijos
				\4 Creación de SPV
				\4[] En jurisdicción extranjera
				\4[] Habitualmente, common law
				\4 SPV vende sukuk a inversores internacionales
				\4 Empresa vende activos a SPV
				\4 Empresa realiza pagos periódicos a SPV
				\4[] En concepto de uso de los activos
				\4 SPV  distribuye pagos a inversores
				\4 Pago final de empresa a SPV
				\4[] Recompra de activos
				\4 Revalorización de activos
				\4[] Puede aumentar valor de pago final
				\4[$\then$] Similitud con equity
			\3 Valoración
				\4 Pérdida popularidad últimos años
				\4[] Escándalo 1MD
				\4 Países europeos han emitido sukuk
				\4[] Reino Unido
				\4 Cotizados en bolsa de Luxemburgo
		\2 Private equity
			\3 Idea clave
				\4 Inversión en equity en compañías no cotizadas
				\4 Salir de inversión para realizar beneficio
				\4 Sin énfasis en crecimiento/desarrollo compañía
				\4 Énfasis en mejora de gestión
				\4[] Alineación incentivos accionistas--management
			\3 Agentes
				\4 Fondos de inversión
				\4[] Instituciones colectivas
				\4[] Búsqueda de rendimiento mayor a cotizado
				\4[] $\to$ Activismo
				\4[] $\to$ Búsqueda de alfa
				\4[] $\to$ Contrario a inversión pasiva
				\4 Hedge funds
				\4[] Inversión alternativa
				\4[] Sólo accesibles para inversores acreditados
				\4[] Muy activos en mercado internacional
				\4[] $\to$ Especialmente, si mayor tamaño
			\3 Valoración
				\4 Aumento de importancia en últimos años
				\4[] Paralelo a aumento de fondos pasivos en equity
				\4 Inversores activistas pueden influir en gestión
				\4 Generalmente, invierten en empresas consolidadas
		\2 Venture capital
			\3 Idea clave
				\4 Inversión en equity empresas nacientes
				\4 Elevado riesgo
				\4 Énfasis en crecimiento
			\3 Agentes
				\4 Menos importancia de inversores institucionales
				\4 Perfil de riesgo mucho mayor
				\4 Mucho más involucrados en producto y desarrollo
			\3 Elevada concentración geográfica
				\4 Emisores de equity
				\4[] Grandes centros tecnológicos globales
				\4[] $\to$ California, NY, Boston
				\4[] $\to$ Londres, Alemania, París
				\4[] $\to$ Centros regionales de innovación
				\4[] $\then$ Milán, Barcelona, Suiza...
				\4 Inversores en equity
				\4[] Distribución relativamente heterogénea
				\4[]
			\3 Valoración
				\4 Importancia cuantitativa reducida
				\4 Elevada importancia cualitativa en internacional
				\4 Depende fuertemente de lazos:
				\4[] $\to$ Históricos
				\4[] $\to$ Institucionales
				\4[] $\to$ Lingüisticos
				\4[] Necesario conocer mercado, instituciones, empresa...
	\1 \marcar{Cambios post-crisis}
		\2 Deuda soberana
			\3 Sindicación
				\4 Incluso en moneda propia
				\4 Mejora liquidez y cuantía
				\4 Diversificación de inversores vía banco colocador
			\3 TIPs e index-linked-bonds
				\4 Temor a inflación aumentada demanda
			\3 Emisiones divisa extranjera
				\4 Swap posterior
		\2 Deuda entidades financieras
			\3 Covered bonds ganan importancia
				\4 Sustitución RMBS
				\4 Favorecido por QE
				\4 Aumento demanda por status no bailinable
				\4 Desincentivación del \comillas{originate-to-distribute}
				\4[] Conceder préstamos para inmediatamente después venderlos
				\4[] En covered bonds, activos se mantienen en balance
			\3 Tipos bajos y liquidez abundante
				\4 Resultado de QE
				\4 En algunos mercados periféricos, problemas perduran
		\2 Cláusulas de acción colectiva\footnote{Ver \href{https://www.europarl.europa.eu/RegData/etudes/BRIE/2019/637974/EPRS_BRI(2019)637974_EN.pdf}{EPRS (2019) sobre CACs}}
			\3 Idea clave
				\4 Recomendaciones de FMI de 2014
				\4[] Mejorar comunicación entre tesoro e inversores
				\4[] Gestión y control de riesgos
				\4[] Test de estrés
				\4[] Provisión de liquidez en mercados secundarios
				\4[] Incluir cláusulas de acción colectiva (CACs)
				\4 Concepto
				\4[] Conjunto de reglas que definen mayorías
				\4[] $\to$ En caso de alteración de condiciones de deuda pública
				\4[] $\then$ Acciones que pueden tomar los acreedores
				\4 Objetivos de las CAC
				\4[] Ordenar procesos de reestructuración
				\4[] Aumentar seguridad jurídica
				\4 Evolución
				\4[] Inclusión en documentación a principios de s. XXI
				\4[] Reestructuración de deuda argentina
				\4[] $\to$ Determinados acreedores no aceptaron reestructuración
				\4[] $\then$ Justicia americana falló en su favor
				\4[] $\then$ Argentina no puede pagar reestructurados antes de los que no aceptaron
			\3 Single-limb CAC / votación única
				\4 Recomendadas por FMI
				\4 Votación que afecta a todos los acreedores
				\4[] Acreedores deciden si aceptar condiciones de emisor
				\4[] $\to$ Vinculan a todos los acreedores
				\4[] $\then$ También a los que votasen en contra
				\4 Reduce incentivos oportunistas
			\3 Double-limb CAC / votación con doble vuelta
				\4 Primero
				\4[] Aprobación por tenedores del saldo de deuda viva
				\4 Segundo
				\4[] Aprobación por tenedores de cada bono/instrumento
				\4[] $\to$ Votaciones separadas
				\4 Aumenta incentivos a holdout\footnote{Es decir, a negarse a aceptar reestructuración o cambio de condiciones.}
			\3 EURO-CACs
				\4 Inclusión obligatoria desde 2013
				\4 Vencimiento mayor a 1 año
				\4 Grecia introdujo retroactivamente
				\4[] Sólo algunos instrumentos fueron reestructurados
				\4[] $\to$ En primera votación, ganó aceptar
				\4[] $\to$ En segunda votación, sólo en algunos instrumentos
				\4 Euro-CAC desde 2012
				\4[] De tipo Double limb
				\4 Acuerdo del Eurogrupo de 2018
				\4[] EU deberá introducir single-limb a partir de 2022
				\4[] $\to$ Borrador para revisión del tratado del MEDE
		\2 Esfuerzos para desintermediar sistema financiero
			\3 Tipos de sistema financiero
			\3 Problemas de intermediación bancaria
			\3 Sustitución crédito bancario por bonos
				\4 Países bancarizados sufren más la crisis
				\4 Favorecido por políticas monetarias no convencional
				\4 Creación de mercados de capital para PYMES
				\4 Mercado Alternativo de Renta Fija (2013)
				\4 ICO
				\4 Reglamento STS
				\4 Poco éxito por el momento
		\2 Cambios en regulación financiera
			\3 Basilea III
				\4 Requisitos más estrictos de capital
				\4 Orientado a entidades con exposición internacional
				\4 En proceso de implementación
			\3 CRD IV/CRR II
				\4 Implementación Basilea III en UE
				\4 Requisitos añadidos para entidades sistémicas
				\4 Normas gobierno corporativo
			\3 Dodd-Frank 2010
				\4 Respuesta americana crisis 2007-2009
				\4 Regla Volcker
				\4[] Prohibición del propietary trading bancario
				\4[] Para entidades con acceso a garantía de depósitos
				\4[] Implementada con notable retraso
				\4[] $\to$ Bancos solicitan prórrogas
				\4[] $\to$ Desprenderse de activos ilíquidos
				\4 Reforma Trump de 2018
				\4[] Relajación de requisitos para SIP
			\3 Normativa de resolución de entidades de crédito
				\4 Feedback negativo rescates-deuda soberana
				\4 Emisiones deuda \comillas{bailinable}
				\4 BRRD
			\3 Alternative Investment Funds Manager Directive (2011)
				\4 AIFMD 2011
				\4 Regulación europea de:
				\4[] $\to$ Hedge funds
				\4[] $\to$ Private equity funds
				\4[] $\to$ Fondos de inversión inmobiliaria
				\4 Restricciones a perfil inversor
				\4 Restricciones estructura de remuneración
				\4[] $\to$ Evitar toma de riesgos excesiva
		\2 Agencias de calificación
			\3 Críticas post-crisis
				\4 Conflictos de interés
				\4 Poder de mercado de 3 principales
				\4 Lentitud
				\4 Poca transparencia de modelos
			\3 Regulación europea
				\4 Reglamento 426/2013
				\4 Directiva 2013/14
				\4 Obligación de registro con ESMA
				\4[] Si prestan servicios en UE
				\4 Canon regulatorio a agencias más grande
				\4[] Para las que facturan más de 10 millones de euros
				\4[] Objetivo
				\4[] $\to$ Incentivar agencias pequeñas
				\4[] $\then$ Favorecer competencia
		\2 Derivados
			\3 Intentos de regulación
				\4 Presión regulatoria para utilizar CCP
				\4[] Central counter parties
				\4 Aumentar transparencia de posiciones en derivados
				\4[] En muchos casos off-balance
				\4[] $\to$ Problemas para regular y supervisar
			\3 Cuantificación
				\4 Según datos BIS
				\4 Tendencia a ligera reducción-estancamiento en 2015, 2016, 2017
				\4 Disminución post-crisis inmediata
				\4[] repunte en 2012, 2013, 2014
		\2 Reducción de la cotización en bolsa
		\2 Private equity
		\2 Crisis Covid-19
			\3 Política monetaria
			\3 Evolución de índices de equity
			\3 Emisión de bonos
	\1[] \marcar{Conclusión}
		\2 Recapitulación
			\3 Agentes mercados financieros internacionales
			\3 Instrumentos y mercados
				\4 Renta fija
				\4 Renta variable
			\3 Cambios post-crisis
		\2 Idea final
			\3 Mercados reflejan
				\4 Avances tecnológicos
				\4 Factores políticos
				\4 Legales
				\4 Economía real
			\3 Evolución constante
				\4 Necesario entender procesos
				\4 No sólo situación en momento concreto
				\4 Grandes crisis tienen potencial de cambio
				\4[] $\to$ Transformaciones bruscas
                
\end{esquemal}









































\conceptos

\concepto{Ley Sarbanes-Oxley}

Ley aprobada en 2002 en Estados Unidos como resultado de una serie de escándalos que afectaron a compañías cotizadas en relación a la fidelidad de sus estados financieros. Incrementó los requisitos de auditoría y endureció la legislación sobre conflictos de interés.

\preguntas

\seccion{Test 2008}

\textbf{33.} Los principales emisores de renta fija en los mercados financieros internacionales son:

\begin{itemize}
	\item[a] Gobiernos
	\item[b] Organismos internacionales, incluidas las instituciones financieras multilaterales.
	\item[c] Instituciones financieras, tanto pública como privadas, y excluidas las instituciones financieras multilaterales.
	\item[d] Sociedades no financieras.
\end{itemize}

\textbf{41.} ¿A quién se considera la primera institución formal establecida para regularizar la cooperación entre bancos centrales?

\begin{itemize}
	\item[a] Fondo Monetario Internacional (FMI)
	\item[b] Banco Internacional de Pagos (BIP) de Basilea.
	\item[c] Sistema Europeo de Bancos Centrales (SEBC)
	\item[d] Acuerdo Monetario Europeo (AME)
\end{itemize}

\notas

\textbf{2008:} \textbf{33.} C \textbf{41.} B

El procedimiento de subasta en los mercados de renta fija pública en España es habitualmente el de subasta holandesa (competitiva)

Posible añadir una cuarta parte separada de la tercera que trate exclusivamente los cambios regulatorios post crisis, los dilemas de política económica (más regulación implica menos crédito? menos regulación implica más posibilidad de crisis?).


\bibliografia

Mirar en Palgrave:
\begin{itemize}
	\item accounting and economics
	\item assets and liabilities
\end{itemize}

Tema CECO Viejo para estructura general

Tema CECO Nuevo para post crisis

Cecchetti, S.; Schoenholtz, K. \textit{Money, Banking, and Financial Markets} (2014) Fourth Edition -- En carpeta Finanzas

Fabozzi, F. J. \textit{Handbook of Fixed Income Securities}. Ch. 19 International Bond Market and Instruments

Handbook FMI sobre emisión de securities

Vernimmen, P.; Quiry, P.; Dallocchio, M; Le Fur, Y.; Salvi, A. \textit{Corporate Finance. Theory and Practice} Ch. 15 The Financial Markets

Yellen, J. \textit{Financial Stability a Decade after the Onset of the Crisis}. 25 de agosto de 2017. \url{https://www.federalreserve.gov/newsevents/speech/yellen20170825a.htm}

Estadísticas BIS sobre derivados: http://www.bis.org/statistics/about\_derivatives\_stats.htm

Valdez Molyneux para post crisis

\end{document}
