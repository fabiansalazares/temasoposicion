\documentclass{nuevotema}

\tema{3B-6}
\titulo{Teoría del comercio internacional (II): Desarrollos recientes. Especial referencia a la competencia imperfecta y a los rendimientos crecientes.}

\begin{document}

\ideaclave

\seccion{Preguntas clave}
\begin{itemize}
	\item ¿Qué desarrollos en la teoría del comercio internacional han tenido lugar en las últimas décadas?
	\item ¿Qué anomalías han tratado de explicar?
	\item ¿Cómo explican las nuevas teorías la existencia de comercio internacional?
	\item ¿El comercio internacional es beneficioso según las nuevas teorías?
	\item ¿Qué papel juega la competencia imperfecta en las nuevas teorías?
	\item ¿Qué efecto tienen los rendimientos crecientes en los nuevos modelos de comercio internacional?
\end{itemize}

\esquemacorto

\begin{esquema}[enumerate]
	\1[] \marcar{Introducción}
		\2 Contextualización
			\3 Evolución del comercio internacional
			\3 Modelo neoclásico o tradicional
			\3 Anomalías del modelo tradicional
		\2 Objeto
			\3 ¿Qué modelos de las últimas décadas explican estas anomalías?
			\3 ¿Cómo explica la NTT
			\3 Según las nuevas teorías, ¿qué beneficios aporta el CI?
			\3 ¿Qué papel juega la comp. imperfecta y los rdtos. crecientes en la NTT?
		\2 Estructura
			\3 Precursores de la NTT
			\3 Competencia monopolística
			\3 Oligopolio
			\3 Otros modelos
	\1 \marcar{Precursores de la NTT}
		\2 Idea clave
			\3 Contexto
			\3 Elementos comunes
		\2 Disponibilidad -- Kravis (1956)
			\3 Idea clave
			\3 Valoración
		\2 Gaps tecnológicos -- Posner (1961)
			\3 Idea clave
			\3 Valoración
		\2 Ciclo de producto -- Hirsch (1967), Vernon (1966)
			\3 Idea clave
			\3 Formulación
			\3 Implicaciones
			\3 Valoración
		\2 Demanda y renta-- Linder (1961), Barker (1977)
			\3 Idea clave
			\3 Formulación
			\3 Implicaciones
			\3 Valoración
	\1 \marcar{Competencia monopolística}
		\2 Idea clave
			\3 Contexto
			\3 Objetivos
			\3 Resultados
		\2 Krugman (1979) y Krugman (1980)
			\3 Idea clave
			\3 Formulación
			\3 Implicaciones
		\2 Nueva Economía Geográfica
			\3 Idea clave
			\3 Formulación
			\3 Implicaciones
			\3 Valoración
		\2 Empresas heterogéneas: Melitz (2003)
			\3 Idea clave
			\3 Formulación
			\3 Implicaciones
		\2 IDE horizontal: Helpman, Melitz y Yeaple (2004)
			\3 Idea clave
			\3 Formulación
			\3 Implicaciones
			\3 Valoración
	\1 \marcar{Otros modelos}
		\2 Dumping recíproco -- Brander y Krugman (1983)
			\3 Idea clave
			\3 Formulación
			\3 Implicaciones
		\2 Economía política de los aranceles
			\3 Idea clave
			\3 Stolper-Samuelson
			\3 Redistribución de beneficios del comercio
			\3 Modelo de factores específicos
			\3 Aversión a la pérdida
			\3 Aversión a incertidumbre
			\3 Aversión a desigualdad
			\3 Concentración de intereses
			\3 Instituciones multilaterales pueden catalizar
			\3 Redistribución puede ser necesaria
			\3 Valoración
		\2 Diferenciación horizontal -- Eaton y Kierzkowski (1984)
			\3 Eaton y Kierzkowski (1984)
			\3 Baghwati (1982)
		\2 Diferenciación vertical: Shaked y Sutton(1983)
			\3 Idea clave
			\3 Formulación
			\3 Implicaciones
			\3 Valoración
		\2 Modelos neo-Heckscher-Ohlin
			\3 Idea clave
			\3 Formulación
			\3 Implicaciones
		\2 Política comercial estratégica
			\3 Idea clave
			\3 Modelos
			\3 Implicaciones
			\3 Valoración
	\1[] \marcar{Conclusión}
		\2 Recapitulación
			\3 Precursores
			\3 Competencia monopolística
			\3 Oligopolio
			\3 Otros modelos
		\2 Idea final
			\3 Cambios recientes en el comercio internacional
			\3 Influencia en otras áreas de la ciencia económica
			\3 Características del comercio mundial actual

\end{esquema}

\esquemalargo














\begin{esquemal}
	\1[] \marcar{Introducción}\footnote{Esta introducción hay que hacerla más larga que la media porque incluye anomalías y ecuación de gravedad.}
		\2 Contextualización
			\3 Evolución del comercio internacional
				\4 Explosión en últimos siglos y décadas
				\4[] CI ha crecido mucho más que PIB
				\4 Avance tecnológico:
				\4[] $\downarrow$ de costes de transporte
				\4[] $\downarrow$ de costes informacionales
				\4 Sujeto de estudio relativamente antiguo:
				\4[] $\to$ Hume, Smith, Ricardo, Mill, Torrens
				\4[] Ligado a la evolución de:
				\4[] $\to$ teoría económica
				\4[] $\to$ hallazgos empíricos
			\3 Modelo neoclásico o tradicional
				\4 Teoría neoclásica $\to$ modelo general
				\4[] Basado en EGWalrasiano 2x2x2x2
				\4[] Heckscher, Ohlin, Vanek, Haberler, Samuelson
				\4[] Dos economías 2x2x2x2
				\4[] Comparar:
				\4[] $\to$ Eq. por separado
				\4[] $\to$ Eq. conjunto
				\4[] $\then$ Comercio internacional depende de primitivas:
				\4[] -- Tecnologías
				\4[] -- Preferencias
				\4[] -- Dotaciones
			\3 Anomalías del modelo tradicional
				\4 Regularidades empíricas encontradas desde 40s
				\4[I] Presencia dominante de comercio intra-industrial
				\4[] Países importan ByS que también exportan
				\4[] Distinto a H-O
				\4[] $\to$ Predice comercio interindustrial dominante
				\4[] Índice Grubel-Lloyd
				\4[] $\to$ Medida de comercio interindustrial en sector $i$
				\4[] $G_i = 1 - \frac{\left| X_i - M_i \right|}{X_i + M_i}$
				\4[] Si $G_1 \to 0$
				\4[] $\then$ Nada de comercio interindustrial
				\4[] Si $G_1 \to 1$:
				\4[] $\to$ Importa tanto como exporta de $i$
				\4[] $\then$ Máximo comercio interindustrial
				\4[] Clases de diferenciación de Grubel y Lloyd
				\4[] i) Mismos inputs, baja sustituibilidad
				\4[] ii) Diferentes inputs, muy sustituibles
				\4[] iii) Mismos inputs y alta sustituibilidad
				\4[] $\to$ Tipos i) y ii) explicables con tradicional
				\4[] $\to$ Tipo iii) es el más relevante
				\4[] $\to$ Similares pero diferentes  económicamente
				\4[II] Ec. de gravedad predice muy bien los flujos de CI
				\4[] Relación muy fuerte entre volumen de CI y
				\4[] $\to$ Tamaño de economía doméstica
				\4[] $\to$ Tamaño de economía extranjera
				\4[] $\to$ Costes de transporte entre economías
				\4[] ``gravedad'' por similitud con ec. de Newton
				\4[] $\then$ Distinto a MVComparativa y H-O
				\4[] Tinbergen (1962)
				\4[] $\to$ Primera contrastación empírica
				\4[] ¿Cómo justificar teóricamente esta regularidad?
				\4[III] Asimetría de CI norte-norte y norte-sur
				\4[] Comercio norte-norte > norte-sur
				\4[] Contrario a predicciones de H-O
				\4[] $\to$ Diferentes dotaciones dan lugar a CI
				\4[IV] Paradoja de Leontieff\footnote{Esto puede contarse o no.}
				\4[] Análisis de patrón de CI de EE.UU.
				\4[] Exportaciones americanas intensivas en L
				\4[] Importaciones intensivas en K
				\4[] Si EE.UU. es relativamente abundante en K
				\4[] $\to$ Contrario a predicciones de H-O
				\4[V] Home-market effect
				\4[] Producción mundial de variedades tiende a concentrarse
				\4[] $\to$ En mercados más grandes
				\4[] Determinados mercados producen más que resto del mundo
				\4[] $\to$ \% sobre producción mundial > \% tamaño de economía
				\4[] Salarios mayores en mercados más grandes
				\4 Nueva teoría del comercio internacional
				\4[] Explicar estas anomalías a partir de:
				\4[] $\to$ Hallazgos empíricos
				\4[] $\to$ Estructura industrial del mercado
		\2 Objeto
			\3 ¿Qué modelos de las últimas décadas explican estas anomalías?
			\3 ¿Cómo explica la NTT\footnote{\textit{New Trade Theory}.} la existencia de CI?
			\3 Según las nuevas teorías, ¿qué beneficios aporta el CI?
			\3 ¿Qué papel juega la comp. imperfecta y los rdtos. crecientes en la NTT?
		\2 Estructura
			\3 Precursores de la NTT
			\3 Competencia monopolística
			\3 Oligopolio
			\3 Otros modelos
	\1 \marcar{Precursores de la NTT}
		\2 Idea clave
			\3 Contexto
				\4 Años 40, 50 y 60
				\4 Anomalías ya conocidas
				\4 Competencia monopolística sin formalizar
				\4 Ecuación de gravedad poco contrastada aún
			\3 Elementos comunes
				\4 Formalización inexistente o débil
				\4 Énfasis en tecnología y demanda
				\4 Introducen existencia de variedades
				\4 Poco análisis normativo
		\2 Disponibilidad -- Kravis (1956)
			\3 Idea clave
				\4 Kravis (1956)
				\4 Indisponibilidad da lugar a CI
				\4[] Los países comercian lo que no tienen
				\4[] Disponibilidad relativa y absoluta
				\4[] $\to$ Absoluta: no existe
				\4[] $\to$ Relativa: muy caro o dificil
				\4 Absoluta no es relevante
				\4[] Ventaja comparativa pueden explicar
				\4 Relativa es relevante
				\4[] Avances tecnológicos reducen coste
				\4[] $\to$ Pero en otros países, avances más lentos
				\4[] $\then$ Disponibilidad relativa donde más avance
				\4[] Demanda nacional de bienes ``exóticos''
				\4[] $\to$ Consumo de extranjeros aunque existan domésticos
				\4[] $\to$ Efecto demostración
				\4[] $\to$ Relacionado con Duesenberry (1949)
				\4[] $\then$ Diferenciación aumenta demanda
			\3 Valoración
				\4 Introduce gusto por variedad en CI
				\4 Influenciado por Chamberlin y Duesenberry
		\2 Gaps tecnológicos -- Posner (1961)
			\3 Idea clave
				\4 Posner (1961)
				\4 CI resultado de lags tecnológicos
				\4[] Tecnología se transmite despacio
				\4[] Hasta que se transmite
				\4[] $\to$ Oportunidades para la aparición del CI
				\4 Países con mayor dinamismo tecnológico
				\4[] Exportarán más
				\4 Grupos de países con gran dinamismo
				\4[] Pueden generar ciclos virtuosos de crecimiento
				\4[] $\then$ ``Edad de oro'' de CEE
			\3 Valoración
				\4 Relaciona crecimiento y tecnología con CI
				\4 Cierto impacto en teoría del desarrollo
				\4 Cierta influencia schumpeteriana
				\4[] Influencia posterior en modelos de crec. endógeno
		\2 Ciclo de producto -- Hirsch (1967), Vernon (1966)
			\3 Idea clave
				\4 Vernon (1966), Hirsch (1967), Hirsch (1975)
				\4 Ciclo de invención de productos induce CI
				\4 Productos tienen un ciclo con varias fases
				\4 Fase actual determina volumen de CI
			\3 Formulación
				\4 Fases en la vida de un producto
				\4[1] Introducción
				\4[] Consumidores aún no demandan
				\4[] No conocen producto
				\4[] Beneficios aumentan poco a poco
				\4[2] Crecimiento
				\4[] Aumenta demanda y ventas del producto
				\4[] Beneficios aumentan
				\4[] Se realizan economías de escala
				\4[] Competidores comienzan a entrar en mercado
				\4[3] Madurez
				\4[] Producto conocido ampliamente
				\4[] Ventas de cada empresa comienzan a caer
				\4[] Muchos consumidores
				\4[] Productor inicial reduce costes
				\4[] Caen cuotas de mercado
				\4[] Márgenes y costes caen
				\4[4] Saturación
				\4[] Competidores comienzan a ganar cuota de mercado
				\4[] Creciente diferenciación
				\4[5] Decadencia
				\4[] Tecnología de producción generalizada
				\4[] Productor inicial cada vez produce menos
				\4[] Variedades de competidores pueden ser superiores
				\4 Comercio internacional
				\4[] Países como empresas y mercados
				\4[] En primeras fases, CI internacional
				\4[] $\to$ Porque aún no hay productores locales
				\4[] $\to$ Porque costes de producción aún no han bajado
				\4[] $\to$ Porque tecnología aún restringida
				\4 Ejemplo
				\4[] Coches
				\4[] Inicialmente
				\4[] $\to$ Estados Unidos, Inglaterra, Alemania, Francia
				\4[] $\then$ Exportaciones a todo el mundo
				\4[] $\then$ Dda. extran. no suficiente para producción local
				\4[] Progresivamente:
				\4[] $\to$ Aparición de competidores extranjeros
				\4[] $\to$ Producción en extranjero
			\3 Implicaciones
				\4 Mayor invención de productos
				\4[] Mayores superávits de CC
			\3 Valoración
				\4 Buena explicación de muchos ciclos
				\4 Contrastación empírica compleja
				\4[] ¿Cómo delimitar fases?
				\4 Localización de productos es idea central
				\4[] Precede análisis de home-market
		\2 Demanda y renta-- Linder (1961), Barker (1977)
			\3 Idea clave\footnote{Ver Kemp (1965) en Economic Journal reseñando el libro de Linder}
				\4 Evidencia empírica muestra
				\4[] Rentas similares ligadas a CI elevado
				\4[] $\then$ CI Norte-norte > CI Norte-sur
				\4[$\then$] Teoría pura del CI no explica
				\4 Demanda es clave de CI en bienes manufacturados
				\4 Demanda induce producción de determinados bienes
				\4 Una vez se produce, posible exportar
				\4 Se exporta donde hay demandas similares
				\4 Linder (1961) plantea idea inicial
				\4 Barker (1977) formaliza
				\4[] En términos de demanda de características
				\4[] $\to$ Influencia de Lancaster (1971)
				\4[] Énfasis en renta como determinante demanda
				\4[] Presenta evidencia empírica
				\4[] $\to$ Elasticidad exportaciones a output > 1
			\3 Formulación
				\4 Demanda induce producción de bienes
				\4[] Empresarios sólo aceptan producir algo en un país
				\4[] $\to$ Si hay demanda nacional para ese bien
				\4 Países tienden a producir bienes para mercado doméstico
				\4 Economías de escala aumentan ventaja comparativa
				\4 Demanda extranjera de bienes nacionales induce CI
				\4[$\then$] Países con demandas similares comercian más
				\4[$\then$] Rentas similares relacionadas con demandas similares
				\4[$\then$] Niveles de renta similares inducen más comercio
			\3 Implicaciones
				\4 Demanda determina exportaciones
				\4[] Demanda de un bien estimula oferta nacional
				\4[] Aumento de oferta nacional realiza economías de escala
				\4[] $\to$ Ventaja comparativa en bien más demandado
				\4[] $\then$ Aumento de competitividad exterior
				\4[] $\then$ Aumento de exportaciones
				\4 Estructura de demanda depende de renta
				\4[] Rentas similares inducen demanda de bienes similares
				\4[] $\to$ Países con rentas similares comercian más
				\4[] $\then$ Flujos norte-norte mayores que norte-sur
				\4 Asimetría de flujos norte-norte y norte-sur
				\4[] Diferentes demandas representativas
				\4 Home-market effect posible
				\4[] Demanda representativa induce producción nacional
				\4[] Comercio aumenta economías de escala
				\4[] Economías de escala incentivan concentración
				\4[] $\then$ Aumento de cuota de mercado de exportador
				\4[] $\then$ Aumento de remuneración de exportador
			\3 Valoración
				\4 Influencia clave para Krugman 1979
				\4[] Primero, producción para mercado doméstico
				\4[] $\to$ Demanda de variedades
				\4[] Después, exportación donde también hay demanda
				\4 Énfasis en bienes manufacturados
				\4[] Autores admiten H-O para bienes homogéneos
	\1 \marcar{Competencia monopolística}
		\2 Idea clave
			\3 Contexto
				\4 Economías de escala / Rdtos. crecientes
				\4[] Se asume en general f.homotéticas: R $\uparrow$ E $\iff$ EE
				\4 Competencia imperfecta
				\4[] Empresas tienen poder de mercado
				\4[] Enfrentan demandas decrecientes
				\4 Diferenciación de productos
				\4[] Demanda positiva para cada variedad
				\4[] Aun teniendo distinto precio
				\4[] No hay interdependencia estratégica
			\3 Objetivos
				\4 Mostrar aparición de CI
				\4[] Con elementos anteriores
				\4[] Con dotaciones factoriales similares
				\4[] $\then$ Estructura de mercado determina CI
				\4 Mostrar rol de demanda doméstica
				\4[] En especialización y exportaciones
				\4 Mostrar ganancias del CI
				\4[] A partir de preferencias explícitas
				\4 Explicar comercio interindustrial
				\4[] Variedades de un bien
			\3 Resultados
				\4 Demanda de variedades induce ganancias del comercio
				\4 Economías de escala
				\4[] Induce entrada de nuevas empresas
				\4[] $\to$ Producción de variedades
				\4 Modelos fácilmente tratables
				\4[] Sin interdependencia estratégica
				\4[] $\to$ Modelos mucho más fáciles de computar
		\2 Krugman (1979) y Krugman (1980)
			\3 Idea clave
				\4 Contexto
				\4[] Home-market effect
				\4[] $\to$ Linder y Barker apuntan a explicación
				\4[] Competencia monopolística
				\4[] $\to$ Avances en modelización en años 70
				\4[] $\then$ Salop (1979)
				\4 Objetivos
				\4[] Explicar buen resultado empírico de EGravedad
				\4[] Explicar comercio intraindustrial
				\4[] Explicar home-market effect
				\4[] Explicar efecto de variedades en ganancias del comercio
				\4 Resultado
				\4[] Modelos seminales de NTT con CMonopolística
				\4[] Preferencias CES inspiradas en Dixit y Stiglit (1977)
				\4[] $\to$ Preferencia por la variedad
				\4[] $\to$  Sustituibilidad simétrica entre variedades
				\4[] Equilibrio de autarquía
				\4[] $\to$ Entrada de empresas hasta eliminar beneficios
				\4[] $\to$ Número de variedades depende de tamaño de mercado
				\4[] $\to$ Producción no depende de tamaño de mercado
				\4[] $\to$ Única forma de aumentar bienestar es $\uparrow$ mercado
				\4[] Equilibrio de libre comercio simple
				\4[] $\to$ Mismo número de empresas totales
				\4[] $\to$ Consumidores acceden a variedades extranjeras
				\4[] $\then$ Aumento de la utilidad por consumo de más variedades
				\4[] Número de empresas en equilibrio depende de:
				\4[] $\to$ (+) Tamaño del mercado
				\4[] $\to$ (--) Costes fijos de entrada
				\4[] $\to$ (--) Elasticidad de sustitución entre variedades
				\4[] Equilibrio con costes de transporte y tamaño asimétrico
				\4[] $\to$ Bienes de importación más caros
				\4[] $\to$ Empresas venden más en mercado local
				\4[] $\to$ Mayor mercado produce más variedades
				\4[] $\then$ Mayor tamaño de mercado induce exportaciones netas
				\4[] $\then$ \% de producción de clase de bienes > \% tamaño mercado
				\4[] $\then$ Dinámica de aglomeración si aumento de tamaño de mercado
				\4[] Movilidad de factores sin comercio
				\4[] $\to$ Tendencia a moverse donde mayor mercado
				\4[] $\to$ Consumidores quieren mayor número de variedades
				\4[] $\then$ Tendencia a aglomeración
			\3 Formulación
				\4 Demanda
				\4[] Preferencias CES à la Dixit-Stiglitz (1977)
				\4[] \fbox{$U(\vec{x})=\left( \sum_{i=1}^n \left( x_i \right) ^{\frac{\epsilon-1}{\epsilon}} \right)^{\frac{\epsilon}{\epsilon-1}}$}
				\4[] $\to$ Demanda decreciente de cada variedad
				\4[] $\to$ Menos $\rho$, Menos elast. de sustitución
				\4[] $\to$ Menos ESustitución, más poder de mercado
				\4[] $\then$ Menos $\rho$, demanda más inelástica
				\4 Oferta
				\4[] Economías de escala
				\4[] Costes totales:
				\4[] $\text{CT} = F + cq$
				\4[] $\to$ $w$: salario
				\4[] Costes medios:
				\4[] $\text{CM} = \frac{F}{q} + w$
				\4[] $\then$ Economías de escala\footnote{Recuérdese que rendimientos crecientes a escala implican siempre economías de escala. Sin embargo, economías de escala sólo implican rendimientos crecientes a escala cuando la función de producción es homotética.}
				\4[] Precio maximizador de beneficio:
				\4[] \fbox{$p^* = \mu \cdot c = \frac{\epsilon}{\epsilon-1} \cdot c$}
				\4[] $\to$ $c$: coste marginal
				\4[] $\to$ $\mu$: poder de mercado/(1 + inversa de EDemanda)
				\4 Equilibrio de autarquía
				\4[] Empresas entran hasta que se anulan beneficios
				\4[] Condición de equilibrio:
				\4[] $\pi = p^* q^* - c q^* - F = 0$
				\4[] $\then$ $q^* (p-c) = F$ $\then$ $q^*=\frac{F}{p-c}$
				\4[] $\then$ \fbox{$q^* = \frac{F}{c} \cdot \left( \epsilon - 1 \right)$}
				\4[] Número de empresas:
				\4[] $\to$ Eq. en mercado de L: $wL = n \cdot q^* = n \cdot \frac{F}{c} \cdot (\epsilon -1)$
				\4[] \fbox{$n^* = n(\underset{+}{\bar{L}}, \underset{-}{F}, \underset{+}{\mu})$}
				\4[] Donde: $\bar{L}$: oferta de trabajo/demanda total
				\4[] Más variedad de empresas cuanto:
				\4[] $\to$ Más oferta de trabajo/demanda
				\4[] $\to$ Menores costes fijos
				\4[] $\to$ Menor elast. de sust.\footnote{Porque cada empresa tendrá más poder de mercado y con ello obtendrá más beneficios, incentivándose la entrada.}
				\4 Equilibrio de libre comercio
				\4[] Empresas venden en nacional y exportan
				\4[] Trabajo inmóvil
				\4[] $\to$ Oferta de trabajo constante en cada país
				\4[] $\then$ Misma condición de equilibrio
				\4[] Cada consumidor accede a más variedades
				\4[] $\to$ Nacionales y extranjeras
				\4[] $\then$ Aumento del bienestar
				\4[] Más comercio cuanto más parecido sea el tamaño
				\4[] $\to$ Más variedades a intercambiar
				\4[] Volumen de importaciones en país 1:
				\4[] \fbox{$\text{IMP}_1 = \left[ \frac{n_2^*}{n_1^* + n_2^*} \right] w L_1$}
				\4[] $\to$ Mayores si mayor mercado doméstico ($L_1$)
				\4[] $\to$ Mayores si más variedades extranjeras ($n^*$)
			\3 Implicaciones
				\4 Ganancias del comercio
				\4[] Aumento de variedades disponibles
				\4[] Bajada de precios si mark-up variable
				\4 CI aún con países idénticos
				\4[] Mismas dotaciones, prefs., tecnologías
				\4[] $\to$ Estructura de mercado induce CI
				\4 Home-Market Effect con costes de transporte\footnote{Ver pág. 190 de Gandolfo.}
				\4[] Krugman (1980)
				\4[] Respecto a un bien $i$:
				\4[] $\to$ \% de prod. mundial de $i$ > \% dda. nacional de $i$
				\4[] $\to$ Salarios más altos donde mayor demanda relativa
				\4[] En presencia de costes de transporte
				\4[] $\to$ Consumidores demandan más bien local
				\4[] $\to$ EEscala en producción reducen coste
				\4[] $\then$ Más empresas donde más demanda local
				\4[] $\then$ Más producción de un bien en país más grande
				\4[] $\then$ \% de empresas de un mercado mayor que \% de mercado en mundo
				\4[] Dos efectos que se acumulan
				\4[] -- Mayor mercado para determinada industria
				\4[] $\to$ Más beneficios a obtener por potenciales entrantes
				\4[] $\to$ Aumenta número de variedades en ese mercado
				\4[] -- Costes de transporte
				\4[] $\to$ Desvían demanda hacia variedades producidas localmente
				\4[] Efecto conjunto
				\4[] $\to$ Más variedades donde mayor mercado
				\4[] $\to$ Más \% de variedades sobre mundo que \% tamaño mercado
				\4[] $\then$ Tendencia a concentración de industria en mayor mercado
				\4[] Salarios más altos en mayor mercado
				\4[] $\to$ Dados CdT, sería óptimo producir todo en un mercado
				\4[] $\then$ Necesario menores salarios en mercado pequeño para mantener empleo
				\4 Posible reducción del n. total de empresas
				\4[] Apertura al CI aumenta competencia\footnote{Esto es un supuesto adicional que no tiene porque imponerse.}
				\4[] $\to$ Se reduce poder de mercado
				\4[] $\to$ Baja el mark-up aplicado
				\4[] $\to$ Beneficios se reducen
				\4[] $\then$ Algunas empresas abandonan mercado
				\4[] Variedades post-apertura
				\4[] $\to$ Número total inferior a total de autarquía
				\4[] $\then$ Consumidores acceden a más variedades
				\4[] $\then$ Existen menos variedades que al principio
				\4 Ecuación de gravedad
				\4[] Implicaciones de modelo consistentes con EdeGravedad
				\4[] Importaciones dependen de:
				\4[] $\to$ Tamaño del mercado local
				\4[] $\to$ Tamaño del mercado extranjero (vía variedades extranjeras)
				\4[] $\to$ Costes de transporte (si hay)
				\4[] $\then$ Fundamento teórico de ec. de gravedad
				\4 Movilidad de factores
				\4[] Movilidad de factores puede sustituir al comercio
				\4[] Sin CI + Movilidad de factores
				\4[] $\to$ Trabajadores buscan mayor cantidad de variedades
				\4[] $\then$ Movimiento de factores puede sustituir CI
				\4[] Comparativa de costes
				\4[] $\to$ Movilidad de factores
				\4[] $\to$ Comercio internacional
				\4[] $\then$ Comercio internacional es más barato
				\4[] $\then$ Especialmente, que movilidad del trabajo
		\2 Nueva Economía Geográfica
			\3 Idea clave
				\4 Contexto
				\4[] Von Thünen
				\4[] $\to$ Análisis de localización óptima en ciudades
				\4[] $\to$ Pionero en economía espacial
				\4[] Marshall
				\4[] $\to$ Economías de escala tecnológicas con concentración
				\4[] i. Más facilidad para encontrar mano de obra
				\4[] ii. Spill-overs de información
				\4[] iii. Producción de inputs intermedios no comerciables
				\4[] $\then$ Difícil formalización
				\4[] $\then$ Análisis canónico hasta NEG
				\4[] Teoría clásica del CI
				\4[] $\to$ Economías son puntos sin dimensión espacial
				\4[] $\to$ Factores inmóviles entre países
				\4[] Hotelling
				\4[] $\to$ Análisis pionero
				\4[] $\to$ Formalización de decisión de localización empresas
				\4[] Salop (1979)
				\4[] $\to$ Entrada de empresas en contexto espacial
				\4[] Dixit y Stiglitz (1977)
				\4[] $\to$ Formalización de competencia monopolística
				\4[] $\to$ Análisis formal de equilibrio general
				\4[] $\then$ Agentes prefieren variedad
				\4[] $\then$ Incentivos a entrada de nuevas empresas/variedades
				\4[] Evidencia empírica
				\4[] $\to$ Aparición de núcleos y cinturones industriales
				\4[] $\to$ Concentración de población donde hay desarrollo industrial
				\4[] $\to$ Trabajo industrial móvil con patrones persistentes
				\4 Objetivos
				\4[] Explicar dinámicas de aglomeración
				\4[] $\to$ En contexto de desarrollo industrial
				\4[] $\to$ En contexto de bajada de precios de transporte
				\4[] Explicar patrón de comercio
				\4[] $\to$ En contexto de movilidad del trabajo
				\4[] $\to$ En contexto de costes de transporte y H-M-Effect
				\4 Resultados
				\4[] Dinámicas de aglomeración-dispersión
				\4[] $\to$ Dependen de parámetros clave
				\4[] Patrón de comercio endógeno
				\4[] $\to$ Depende de evolución de aglomeración-dispersión
				\4[] Integración comercial
				\4[] $\to$ Puede inducir aglomeración
				\4[] Movimientos de trabajadores
				\4[] $\to$ Endógenos
				\4[] $\to$ Posible concentración geográfica
			\3 Formulación
				\4 Dos bienes consumidos
				\4[] Agrícola homogéneo $C_A$
				\4[] Manufacturado compuesto $C_M = \left( C_i^{\frac{\epsilon-1}{\epsilon}} \right)^{\frac{\epsilon}{\epsilon-1}}$
				\4 Dos factores de producción
				\4[] Campesinos inmóviles entre países
				\4[] $\to$ Distribuidos entre los dos países de manera exógena
				\4[] Obreros móviles entre países
				\4 Dos países/regiones A y B
				\4[] Campesinos repartidos equitativamente entre países
				\4[] Obreros con distribución inicial arbitraria
				\4[] $\to$ Sujeto a variación endógena
				\4 Demanda de bienes
				\4[] Obreros y campesinos iguales demandas
				\4[] $\to$ Se distribuye entre agrícola y manufacturero
				\4 Coste de transporte
				\4[] Agrícola sin coste de transporte
				\4[] Manufacturero con costes tipo iceberg
				\4[] $\to$ Para que llegue 1 hace falta enviar $\tau > 1$
				\4 Decisión de localización de obreros
				\4[] Donde haya mayor salario
				\4[] $\to$ Donde salario nominal compre más salario real
				\4[] $\then$ Donde haya más variedades más baratas
				\4[] Dos efectos contrapuestos afectan localización
				\4[] $\to$ Efecto competencia
				\4[] $\to$ Efecto demanda
				\4 Efecto demanda
				\4[] Localización cerca de la demanda
				\4[] $\to$ Permite superar costes de transporte
				\4[] Si coste fijo superior a costes de transporte
				\4[] $\to$ Preferible concentrar producción
				\4[] Cuanta mayor población obrera
				\4[] $\to$ Más se retroalimenta el efecto demanda
				\4[] Producción de variedades manufactureras concentradas
				\4[] $\to$ Aumenta salario real de obreros en aglomeración
				\4[] $\then$ Tendencia hacia concentración donde ya se produce manufact.
				\4 Efecto competencia
				\4[] Costes de transporte
				\4[] $\to$ Reducen competencia con variedades en otro país
				\4[] $\then$ Permiten aumentar precios
				\4[] Cuanta más población campesina sobre total
				\4[] $\to$ Mayor es la demanda que no se mueve
				\4[] $\then$ Más incentivos a localizarse donde no se producen variedades
				\4 Parámetros iniciales determinan resultado
				\4[] Preferencia por la variedad $\epsilon$
				\4[] $\then$ Aumenta importancia de tener más variedades
				\4[] Costes de transporte
				\4[] $\to$ Reduce competencia con variedades en otro país
				\4[] $\to$ Actúa a favor de la aglomeración
				\4[] Peso del sector manufacturero en población
				\4[] $\to$ Aumenta efecto de movimiento de L sobre demanda
				\4[] $\then$ Actúa a favor de la aglomeración
				\4 Dinámica del movimiento de obreros y comercio
				\4[] Asumiendo
				\4[] $\to$ Preferencia suficiente por la variedad
				\4[] $\to$ Suficiente peso del sector manufacturero
				\4[] Dispersión en equilibrio
				\4[] $\to$ Costes de transporte elevados +  pob. obrera reducida
				\4[] $\to$ Elevado efecto competencia
				\4[] $\to$ Poco efecto demanda
				\4[] $\then$ Tendencia a dispersión
				\4[] $\then$ \grafica{krugman91dispersion}
				\4[] Múltiples equilibrios
				\4[] $\to$ Costes de transporte intermedios + pob. obrera moderada
				\4[] $\to$ Si población dispersa, tendencia a dispersión
				\4[] $\to$ Si población inicialmente aglomerada, tendencia aglom.
				\4[] $\then$ Equilibrio depende de shock/condición inicial
				\4[] $\then$ Múltiples equilibrios dispersos y aglomerados
				\4[] $\then$ \grafica{krugman91multiplesequilibrios}
				\4[] Aglomeración en equilibrio
				\4[] $\to$ Costes de transporte reducidos + elevada pob. obrera
				\4[] $\to$ Poco efecto competencia
				\4[] $\to$ Efecto demanda elevado
				\4[] $\then$ Tendencia a aglomeración
				\4[] $\then$ \grafica{krugman91aglomeracion}
				\4[] Sin costes de transporte y con costes de congestión
				\4[] $\to$ Posible dispersión de nuevo
				\4[] $\to$ Sin home-market effect
				\4[] $\to$ Costoso concentrarse
				\4[] $\to$ Sin costes de exportar
				\4[] $\then$ Dispersión máxima
			\3 Implicaciones
				\4 Patrón de comercio internacional
				\4[] Especialización asimétrica posible
				\4[] $\to$ Núcleo especializado en industria
				\4[] $\to$ Núcleo y periferia producen agrícolas
				\4[] $\to$ Mayor demanda en núcleo induce importación agrícola
				\4 Movimiento de factores
				\4[] Permite explicar patrón migratorio en siglo XIX y XX
				\4[] Campo a la ciudad
				\4[] $\to$ Al reducirse CdTransporte
				\4[] $\to$ Al aumentar demanda de bienes industriales
				\4 Núcleo y periferia
				\4[] Núcleo
				\4[] $\to$ Concentración de obreros
				\4[] $\to$ Concentración de variedades industriales
				\4[] $\to$ Salarios elevados
				\4[] $\to$ Exportación de producto manufacturado
				\4[] $\to$ Importación de productos agrícolas
				\4[] Periferia
				\4[] $\to$ Sin obreros
				\4[] $\to$ Sin variedades industriales
				\4[] $\to$ Salarios reducidos
				\4[] $\to$ Importación de producto manufacturado
				\4[] $\to$ Exportación de productos agrícolas
				\4 Integración comercial induce aglomeración
				\4[] Posible aumento desigualdades regionales
				\4[] Posibles tensiones de economía política
			\3 Valoración
				\4 Premio Nobel a Krugman en 2008
				\4[] Culmina programa de comp. monop. y EEscala en CI
				\4 Abre programa de investigación
				\4[] Geografía económica basada en
				\4[] $\to$ Externalidades pecunarias
				\4 De manera paradójica, mundo se vuelve más clásico\footnote{Ver conclusión de Krugman (2008) Nobel Prize Lecture.}
				\4[] En últimas décadas
				\4[] Aumenta comercio basado en VComparativa
				\4[] $\to$ Cadenas de valor global
				\4[] $\to$ Especialización
				\4[] $\to$ IDE vertical frente a horizontal
		\2 Empresas heterogéneas: Melitz (2003)
			\3 Idea clave
				\4 Contexto
				\4[] Evidencia empírica sobre empresas
				\4[] $\to$ Muy elevadas diferencias de productividad
				\4[] Fuerte correlación entre productividad y exportación
				\4[] $\to$ Evidencia empírica robusta
				\4[] $\then$ Más exportación ligada a más productividad
				\4 Objetivos
				\4[] Explicar relación entre productividad y exportaciones
				\4[] $\to$ ¿Dirección de la causalidad?
				\4[] Explicar efectos de comercio internacional
				\4[] $\to$ Sobre productividad de empresas
				\4[] $\to$ Sobre productividad agregada
				\4 Resultados
				\4[] Melitz (2003)
				\4[] Apertura al comercio causa aumento de productividad
				\4[] Apertura al comercio
				\4[] $\to$ Aumenta competencia en mercado doméstico
				\4[] $\then$ Empresas menos productivas desaparecen
				\4[] $\to$ Abre nuevos mercados a empresas nacionales
				\4[] $\then$ Nacionales más productivas exportan
				\4[] $\then$ Nacionales más productivas obtienen más beneficios
				\4[] Aumento de productividad general de la economía
				\4[] $\to$ Causado por exposición al comercio internacional
				\4[] Metodología de modelización basada en Krugman (1980)
				\4[] $\to$ Dixit y Stiglitz (1977)
				\4[] $\to$ Competencia monopolística
				\4[] $\to$ Economías de escala
				\4[] Elementos clave:
				\4[] $\to$ Costes fijos dependen de exportación
				\4[] $\then$ Si exporta, costes fijos más altos
				\4[] BOperativo depende positivamente de parámetro aleatorio
				\4[] $\to$ Caracteriza productividad
				\4[] Empresas exportan si:
				\4[] $\to$ BOperativo de exp. cubre CFijos de exp.
				\4[] Empresas producen sólo para MDoméstico si:
				\4[] $\to$ BOperativo de export. no cubre CFijos de export.
				\4[] $\to$ BOperativo de MDoméstico cubre CF domésticos
				\4 Empresas salen del mercado si:
				\4[] No pueden cubrir CF domésticos ni de exp.
				\4 Equilibrio de autarquía
				\4[] Producen empresas que cubren CF
				\4[] $\to$ Sólo los CF de vender en doméstico
				\4 Equilibrio de libre comercio
				\4[] BOperativos nacionales caen
				\4[] $\to$ Por mayor competencia
				\4[] $\to$ Empresas menos prod. no cubren CF domésticos
				\4[] $\then$ Empresas menos eficientes abandonan
				\4[] $\then$ Empresas más eficientes exportan y compiten
			\3 Formulación
				\4 Contexto Krugman (1980)
				\4[] Competencia monopolística
				\4[] Economías de escala
				\4 Precio de venta
				\4[] Mark-up sobre coste
				\4[] \fbox{$p = \frac{\epsilon}{\epsilon-1} \frac{c}{\phi} $}
				\4[] $\to$ $\phi$: productividad
				\4 Beneficios domésticos y de exportación
				\4[] Domésticos:
				\4[] $\to$ $\pi_d = \pi_d^o(\underset{+}{\phi}) - F$
				\4[] Exportación:
				\4[] $\to$ $\pi_x = \pi_x^o(\underset{+}{\phi}) - F_x$
				\4[] Exportar es costoso: $F < F_x$
				\4[] $\to$ Costes fijos de exportación más elevados
				\4[] Beneficio operativo crece con $\phi$
				\4[] $\to$ Crece más en doméstico: $\frac{d \, \pi_d^o}{d \, \phi} > \frac{d \, \pi_x^o}{d \, \phi}$
				\4 Equilibrio de autarquía
				\4[] Beneficios operativos cubren costes fijos $\pi_d^o (\phi^*) > F$
				\4[] Si $\phi > \phi^*$:
				\4[] $\to$ Produce si productividad suficiente
				\4[] $\to$ En caso contrario, abandona
				\4 Equilibrio de libre comercio
				\4[] Beneficio operativo de mercado doméstico cae
				\4[] $\to$ Por aumento de la competencia
				\4[] $\then$ $\uparrow$ Productividad mínima para mercado doméstico
				\4[] Empresas abandonan si beneficios negativos
				\4[] $\to$ Abandona mercado doméstico: $\pi_d^o < F$
				\4[] $\to$ Abandona exportación: $\pi_x^o < F_x$
				\4[] (Si abandona doméstico abandona extranjero)\footnote{Dado $F > F_x$ y $\frac{d \, \pi_d^o}{d \, \phi} > \frac{d \, \pi_x^o}{d \, \phi}$.}
				\4 Representación gráfica
				\4[] \grafica{melitz}
			\3 Implicaciones
				\4 Sólo empresas eficientes sobreviven apertura
				\4[] Por efecto de la mayor competencia
				\4[] $\to$ Con empresas exportadoras extranjeras
				\4 Exportadoras son más eficientes
				\4[] Coste de exportar es más elevado
				\4[] Necesario menor coste para exportar rentable
				\4[] $\to$ \underline{Efecto selección}
				\4 Productividad media aumenta con apertura
				\4[] Empresas menos eficientes abandonan mercado
				\4[] Empresa que quedan son más eficientes
				\4[] $\to$ Productividad media aumenta
				\4 Exportadoras producen más que domésticas
				\4[] Aprovechan economías de escala
				\4[] Quitan cuota de mercado a domésticas extranjeras
				\4[] $\to$ \underline{Efecto escala}
				\4 Variedades totales disminuyen
				\4[] En relación a total de autarquía
				\4[] Pero aparecen variedades importadas
				\4[] $\then$ Efecto ambiguo sobre variedades disponibles
		\2 IDE horizontal: Helpman, Melitz y Yeaple (2004)
			\3 Idea clave
				\4 Contexto
				\4[] Tendencia tras segunda guerra mundial
				\4[] Aumento de replicación de plantas en extranjero
				\4[] Desde años 80
				\4[] Aparición de multinacionales con IDE horizontal
				\4[] Melitz (2003): empresas heterogéneas en CI
				\4[] Helpman, Melitz y Yeaple (2004)
				\4[] Modelo de empresas heterógeneas aplicado a IDE
				\4 Objetivos
				\4[] Explicar IDE horizontal en vez de exportación
				\4 Resultados
				\4[] Trade off entre concentración y proximidad
				\4[] $\then$ Costes de transporte vs economías de escala
				\4[] Costes de transporte para todos
				\4[] Economías de escala más fáciles de realizar para + productivas
				\4[] $\then$ Más productivas invierten más en IDE horizontal
			\3 Formulación
				\4 Costes de transporte + protección
				\4[] Todas las empresas sufren por igual
				\4 Economías de escala por concentración
				\4[] Beneficio potencial a todas las empresas
				\4[] Empresas más productivas realizan con menos concentraciónº
				\4[] Empresas más productivas dependen menos de EEscala
				\4 Trade off concentración vs proximidad
				\4[] ¿Reducir costes de producción?
				\4[] ¿Realizar economías de escala?
				\4 Empresas más productivas
				\4[] Pueden permitirse menos economías de escala
				\4[] Puede competir replicando
			\3 Implicaciones
				\4 Empresas más productivas invierten más en IDE
				\4 Multinacionales tienden a ser más productivas
				\4 Multinacionales aumentan productividad de destino
				\4[] Son más productivas de país de origen
				\4[] Casi siempre más productivas que destino
			\3 Valoración
				\4[] Explicación bien ajustada empíricamente
				\4[] Relativamente menos importancia en la actualidad
				\4[] $\to$ Caída de coste de transporte
				\4[] $\to$ Evolución hacia GVCs
				\4[] $\then$ IDE vertical gana importancia
	\1 \marcar{Otros modelos}
		\2 Dumping recíproco -- Brander y Krugman (1983)
			\3 Idea clave
				\4 Contexto
				\4[] Fenómeno del dumping
				\4[] $\to$ Observación empírica habitual
				\4[] Empresa exporta bien al exterior
				\4[] $\to$ Fijando precio en extranjero menor que mercado doméstico
				\4[] Explicación básica basada en monopolio discriminador
				\4[] $\to$ Demandas distintas en diferentes mercados
				\4[] $\to$ Enfrenta demanda más elástica en mercado extranjero
				\4[] $\then$ Vende a menor precio donde demanda más elástica
				\4 Objetivos
				\4[] Explicar fenómeno de dumping
				\4[] $\to$ A partir de estructura de mercado oligopolística
				\4 Resultados
				\4[] Explicación alternativa a dumping
				\4[] Basada exclusivamente en:
				\4[] $\to$ Competencia oligopolística+CdTransporte
				\4[] Empresas venden más barato en mercado extranjero
				\4[] $\to$ Por costes de transporte tipo iceberg
				\4[] $\then$ Deben bajar precio para competir con empresas locales
				\4[] Supuestos básicos
				\4[] $\to$ Dos mercados
				\4[] $\to$ Costes de transporte iceberg
				\4[] $\to$ Demandas idénticas
				\4[] $\to$ Empresa en cada país
				\4[] $\to$ Competencia à la Cournot
			\3 Formulación
				\4 Demanda
				\4[] Simétricas en cada economía
				\4[] Decrecientes
				\4 Empresas
				\4[] Compiten à la Cournot
				\4[] Funciones de reacción en economía 1
				\4[] $q_{11} (q_{21})$, $q_{21}(q_{11})$
				\4[] $\to$ Se asume inducen equilibrio estable
				\4 Equilibrio
				\4[] Cantidades $q_1^* = q_2^*$ simétricas
				\4[] Precios $p_1^* = p_2^*$ simétricos
			\3 Implicaciones
				\4 CI sin ninguno de motivos habituales
				\4[] Sin diferencias de costes
				\4[] Sin economías de escala
				\4[] Sin diferentes demandas
				\4 Comercio intraindustrial
				\4[] Mismo producto exportado e importado
				\4[] Sin diferenciación de producto
				\4 \underline{Dumping recíproco}
				\4[] Dados:
				\4[] -- costes de transporte tipo iceberg
				\4[] -- precios idénticos en cada país
				\4[] $\to$ Para exportar $q$ hay que enviar $(1+c)q$
				\4[] $\to$ EDoméstica: $q$ a cambio de $p$
				\4[] $\to$ EExtranjera: $(1+c)q$ a cambio de $p$
				\4[] $\then$ Venden más caro doméstica que exportación
				\4[] $\then$ Se hacen dumping uno a otro
				\4 Beneficios del comercio
				\4[] Apertura comercial implica:
				\4[] -- Aumento de la competencia
				\4[] $\to$ Reducción de precios
				\4[] $\to$ Aumenta bienestar
				\4[] -- Aumento de costes de transporte
				\4[] $\to$ Reduce bienestar
				\4[] $\then$ Efecto ambiguo de apertura
				\4[] $\then$ Si CTransporte muy altos CI no mejora bienestar
		\2 Economía política de los aranceles
			\3 Idea clave
				\4 Contexto
				\4[] Economía política
				\4[] $\to$ Análisis de efectos de política económica
				\4[] $\then$ Sobre intereses de diferentes grupos sociales
				\4[] $\then$ Como resultado de intereses de diferentes grupos
				\4[] Efectos de política comercial
				\4[] $\to$ Afectan distinto a diferentes sectores
				\4 Objetivo
				\4[] Caracterizar efectos sobre diferentes sectores
				\4[] Entender impacto de estructura política sobre pol. arancelaria
				\4 Resultados
				\4[] Efectos de aranceles sobre diferentes grupos sociales
				\4[] $\to$ Beneficios y perjuicios
				\4[] $\to$ Diferentes grados de concentración
				\4[] $\to$ Diferente capacidad de respuesta
			\3 Stolper-Samuelson
				\4 En contexto Heckscher-Ohlin
				\4 Tras apertura comercial
				\4[] $\to$ Factor intensivo de sector de especialización
				\4[] $\then$ Aumenta pago al factor
				\4[] $\to$ Factor intensivo de sector que pierde producción
				\4[] $\then$ Coste de factores cae
				\4 Sector de factor intensivo en bien de especialización
				\4[] $\then$ Presión hacia reducción de aranceles
				\4 Sector de factor intensivo en bien que pierde producción
				\4[] $\then$ Presión hacia mantenimiento de aranceles
				\4 Países ricos
				\4[] Abundantes en capital
				\4[] $\to$ Capital gana con apertura
				\4[] Trabajo escaso
				\4[] $\to$ Compite con trabajo extranjero
				\4[] $\to$ Pierde con apertura
				\4[$\then$] Trabajo se opone a apertura
				\4 Países pobres
				\4[] Abundantes en trabajo
				\4[] $\to$ Con apertura venden al mundo
				\4[] Capital escaso
				\4[] $\to$ Compiten con capital extranjero
				\4[$\then$] Trabajo favorable a apertura
			\3 Redistribución de beneficios del comercio
				\4 Permite a perdedores aceptar reducción de aranceles
				\4 Pero costes de redistribución
				\4[] $\to$ Negociación entre sectores
				\4[] $\to$ Votaciones
				\4[] $\to$ Adquisición de información
				\4[] $\then$ Posible no sea rentable redistribuir
			\3 Modelo de factores específicos
				\4 Dos factores de capital inmóviles
				\4 Desarme arancelario mutuo
				\4[] $\to$ Aumenta beneficios nuevos exportadores
				\4[] $\to$ Reduce beneficio en sectores que ahora importan
				\4[] $\then$ Flujo de trabajo de un sector a otro
				\4[] $\then$ Caída de PMgK en sector perjudicado
				\4[] Diferentes intereses dentro de un mismo factor
				\4[] $\to$ Capital vs trabajo no siempre oposición homogénea
			\3 Aversión a la pérdida
				\4 Behavioral economics
				\4[] Empíricamente, aversión a pérdida mayor que ganancia
				\4 Apertura arancelaria
				\4[] $\to$ Induce beneficio en un sector
				\4[] $\to$ Aumenta pérdidas en otro
				\4 Si aversión a pérdida mayor que ganancia por beneficio
				\4[] $\then$ Oposición más fuerte
			\3 Aversión a incertidumbre
				\4 Apertura aumenta incertidumbre
				\4[] $\to$ ¿Efectos de equilibrio general serán positivos?
			\3 Aversión a desigualdad
				\4 Apertura al comercio puede aumentar desigualdad
				\4[] $\to$ Sector de especialización más rico
				\4[] $\to$ Sector que reduce producción más pobre
				\4 Seres humanos muestran cierta aversión a la desigualdad
				\4[] $\to$ Factor de oposición a apertura
			\3 Concentración de intereses
				\4 Efectos de reducción arancelaria
				\4[] $\to$ Difusos sobre consumidores
				\4[] $\to$ Muy concentrados sobre industria desprotegida
				\4 Perjuicio concentrado
				\4[] $\to$ Facilita coordinación entre perjudicados
				\4[] $\then$ Facilita oposición política a apertura
			\3 Instituciones multilaterales pueden catalizar
				\4 Commitment liberalizador
				\4[] Aumenta poder de negociación de liberalizadores
			\3 Redistribución puede ser necesaria
				\4 Mejora aceptación de apertura
				\4[] También es costosa
			\3 Valoración
				\4 Programa de investigación con muchas vertientes
				\4 Interacciones con sociología, ciencia política, demografía..
				\4 Ciencia económica no siempre ha examinado
				\4[] Supuestos demasiado fuertes
				\4[] $\to$ ¿Planificador social?
				\4[] $\to$ ¿Funciones de bienestar social?
				\4[] $\then$ ¿Realmente existen?
				\4[] $\then$ ¿Realmente consideradas en decisiones de PComercial?
		\2 Diferenciación horizontal -- Eaton y Kierzkowski (1984)
			\3 Eaton y Kierzkowski (1984)
				\4 Apertura comercial puede ser
				\4[] subóptima para un país importador
				\4[] óptima para país exportador
			\3 Baghwati (1982)
				\4[] Modelo ``biológico''
				\4[] Entorno económico determina características
				\4[] $\to$ Cada país unas variedades en autarquía
				\4[] Tras apertura:
				\4[] $\to$ Consumidores prefieren otras variedades
				\4[] $\then$ Comercio mejora bienestar
		\2 Diferenciación vertical: Shaked y Sutton(1983)
			\3 Idea clave
				\4 Contexto
				\4[] Diferenciación horizontal
				\4[] $\to$ Si dos variedades se ofrecen a mismo precio
				\4[] $\then$  Habrá demanda positiva de ambas
				\4[] $\then$  Porque hay distintas preferencias
				\4[] $\then$ De gustibus...
				\4[] Diferenciación vertical
				\4[] Si dos variedades se ofrecen al mismo precio
				\4[] $\to$ Sólo habrá demanda de una
				\4[] Productores extranjeros
				\4[] $\to$ Pueden estar especializados en determinadas calidades
				\4[] Aranceles sobre familias de productos dif. verticalmente
				\4[] $\to$ Pueden comprimir diferenciales de precios de calidades
				\4[] $\then$ Política comercial tiene efectos en diferenciación vertical
				\4 Objetivo
				\4[] Caracterizar relación costes y prod. de distinta calidad
				\4[] Caracterizar número de empresas en el mercado
				\4[] Predecir cuotas de mercado
				\4 Resultados
				\4[] Shaked y Sutton (1982)
				\4[] Variedades se distinguen por calidad
				\4[] Calidad y utilidad aportada
				\4[] $\to$ Relación monótona creciente
				\4[] Entrada depende de coste de mayor calidad
				\4[] $\to$ Si aumento de calidad implica poco $\uparrow$ coste
				\4[] $\then$ Consumidores prefieren variedades más caras
				\4[] Menor relación entre calidad y coste marginal
				\4[] $\to$ Menor número de variedades disponibles
			\3 Formulación
				\4 Consumidores
				\4[] Se distinguen por ingreso $t$
				\4[] \fbox{$U_j=u_i \cdot (t_j - p_i)$}
				\4[] $\to$ Utilidad a $j$ por consumir variedad $i$
				\4 Variedades
				\4[] Se distinguen por utilidad que aportan
				\4[] $\to$ $u_n > u_{n-1} > u_{n-2} > ...$
				\4[] Precios crecientes con utilidad aportada
				\4[] $\to$ $p_n > p_{n-1} > p_{n-2} > ...$
				\4 Empresas
				\4[] Coste marginal creciente con calidad $c(u_i)$
				\4 Equilibrio
				\4[] Asumiendo precio iguala coste marginal
				\4[] Dependerá de coste de producir más calidad
				\4[] Si $c(q_i)$ muy poco creciente
				\4[] $\to$ Todos consumidores preferirán más calidad
				\4[] $\to$ Empresas alta calidad desplazan a baja calidad
				\4[] $\then$ Aparece límite a número de empresas que entran
				\4[] $\then$ Cuota de mercado no se fragmenta
				\4[] Si $c(q_i)$ suficientemente creciente
				\4[] $\to$ Fenómeno contrario
				\4[] $\to$ No todos demandan alta calidad porque precio es alto
				\4[] $\to$ Entra una empresa para cada calidad
				\4[] $\to$ Cuota de mercado muy fragmentada
			\3 Implicaciones
				\4 Efecto de aranceles sobre calidades disponibles
				\4[] Aumentan precio de variedades extranjeras
				\4 País que impone arancel productor de calidades altas
				\4[] Arancel comprime estructura de costes marginales
				\4[] Aumento de precios de variedades peores extranjeras
				\4[] Pérdida de competitividad de variedades más baratas
				\4[] $\then$ Aumento de consumo de variedades mejores
				\4[] $\then$ Mejora de saldo comercial
				\4[] $\then$ Menor número de variedades consumidas
				\4[$\then$] Menor comercio intraindustrial
				\4[$\then$] Más consumo de variedades nacionales
				\4 País que impone arancel productor de calidades bajas
				\4[] Aumento de precios de variedades mejores extranjeras
				\4[] $\to$ Aumenta fragmentación del mercado
				\4[] Pérdida de competitividad de variedades extranjeras
				\4[] $\then$ Aumento de consumo de variedades peores nacionales
				\4[] $\then$ Mejora saldo comercial
				\4[] $\then$ Aumento número de variedades consumidas
				\4[] Menos importación de variedades extranjeras por mayor precio
				\4[$\then$] Menos comercio intraindustrial
				\4[$\then$] Más consumo de variedades nacionales
				\4 Buena calidad barata reduce variedad
				\4[] Competencia en precios será más intensa
				\4[] Más calidad a menos precio
				\4[] $\to$ Empresas de baja calidad expulsadas de mercado
				\4[] $\to$ Empresas de + calidad: + costes pero + demanda
			\3 Valoración
				\4 Estudio de cuotas de mercado
				\4[] ¿Más cuota debida a mejor producto?
				\4 Comercio internacional
				\4[] Explicar comercio interindustrial
				\4 Crecimiento económico
				\4[] Modelos de crecimiento endógeno
				\4 Análisis de la publicidad
				\4[] Permite informar de diferente calidad
		\2 Modelos neo-Heckscher-Ohlin
			\3 Idea clave
				\4 Falvey (1981)
				\4 Mínimos cambios en teoría tradicional
				\4 Explicar intra-industrial
				\4[] A partir de diferenciación vertical
				\4[] $\to$ Calidad diferencia variedades
				\4 Factores específicos
				\4[] K inmóvil entre industrias
				\4[] L móvil entre industrias
				\4 Analiza una sola industria y dos países
				\4[] Equilibrio parcial
				\4[] Se asumen otros precios constantes
				\4 Calidad depende de
				\4[] $\to$ Cantidad de capital utilizado
				\4 Precios competitivos
				\4[] Iguales a coste de:
				\4[] $\to$ Trabajo
				\4[] $\to$ Capital utilizado
				\4 Coste de L y K depende de:
				\4[] Abundancia relativa de factores
				\4 Países producen calidades
				\4[] Para las que tienen ventajas comparativas
				\4[] $\then$ Más capital, más calidad
			\3 Formulación
				\4 Países 1 y 2
				\4 Calidad: parámetro $\alpha \, \in \, [0,1]$
				\4 Coste de factores
				\4[] Trabajo: $w_1$ y $w_2$
				\4[] $\to$ $w_1 > w_2$
				\4[] Capital: $r_1$ y $r_2$
				\4[] $\to$ $r_2 > r_1$
				\4 Precio en cada país
				\4[] $p_1(\alpha) = w_1 + \alpha r_1$
				\4[] $p_2(\alpha) = w_2 + \alpha r_2$
				\4 Calidad de equilibrio $\alpha_0$
				\4[] Calidad que implica mismo coste
				\4[] A cada lado de $\alpha_0$
				\4[] $\to$ Un país distinto tiene VComparativa
				\4[] $\then$ Especialización en diferentes calidades
				\4[] $\then$ Especialización depende de dotaciones
				\4[] \grafica{neoho}
			\3 Implicaciones
				\4 Cada país se especializa en una calidad
				\4[] Según abundancia de capital específico
				\4 Explica comercio intraindustrial
				\4[] Como especialización en diferentes calidades
		\2 Política comercial estratégica
			\3 Idea clave
				\4 Contextualización
				\4 Objetivos
				\4[] Valorar efectos de política comercial
				\4[] $\to$ Aranceles
				\4[] $\to$ Subvenciones
				\4[] $\to$ Cuotas
				\4[] Sobre exportación e importación y entrada
				\4 Implicaciones
				\4[] Existen incentivos estratégicos a protección
				\4[] Equilibrio de Nash puede no ser óptimo
				\4[] $\to$ Puede desviarse de libre comercio
			\3 Modelos
				\4 Brander y Spencer (1981): aranceles y entrada Stackelberg
				\4[] Empresa incumbente extranjera
				\4[] $\to$ Produce para su mercado de origen extranjero
				\4[] $\to$ Produce para mercado doméstico
				\4[] Empresa nacional potencial entrante
				\4[] $\to$ Enfrenta costes fijos de entrada
				\4[] Equilibrio con entrada impedida
				\4[] $\to$ Incumbente produce suficiente para anular bfcios.
				\4[] Imposición de arancel
				\4[] $\to$ Reduce producción de incumbente extranjero
				\4[] $\to$ Abre posibilidad de entrada de empresa nacional
				\4 Spencer y Brander (1983): subvenciones a I+D
				\4[] Gobiernos actúan como líder de Stackelberg
				\4[] $\to$ Subvencionando I+d que reduce costes marginales
				\4[] Reducción de costes marginales en Cournot
				\4[] $\to$ Permite a empresas nacionales producir más
				\4[] $\then$ Más beneficio para empresas nacionales
				\4 Spencer y Brander (1985): subsidios y competencia Cournot
				\4[] En contexto de competencia à la Cournot
				\4[] $\to$ Subsidios pueden ser óptimos
				\4[] Impacto de subsidios
				\4[] $\to$ Aumentan producción y ventas de empresa nacional
				\4[] $\then$ Aumento de beneficio de empresa > subsidio
				\4 Eaton y Grossman (1986): generalización para Bertrand y Cournot
				\4[] Subsidio o arancel óptimo depende de variaciones conjeturales
				\4[] Con variaciones conjeturales de Cournot
				\4[] $\then$ Subsidios son óptimos
				\4[] Con variaciones conjeturales de Bertrand
				\4[] $\then$ Paradoja de Bertrand
				\4[] $\then$ Óptimo es no hacer nada
				\4[] Con variaciones conjeturales de Bertrand pero inexactas
				\4[] $\to$ Estiman menor reacción que reacción real
				\4[] $\to$ Equivalente a grado de diferenciación
				\4[] $\then$ Óptimo impuesto a exportación
				\4[] $\then$ Amenaza creíble de aumentar precio
				\4[] $\then$ Precio como complemento estratégico
				\4 Barrett (1994): medio ambiente y política comercial
				\4[] Regulación medioambiental puede usarse con fines de PCEstr
				\4[] Si competencia à la Bertrand con conjeturas inexactas
				\4[] $\to$ Aumento de estándares medioambientales exigidos
				\4[] $\then$ Amenaza creíble de aumentar precios
				\4[] $\then$ Paradoja de Bertrand no se cumple
				\4[] $\then$ Empresas pueden fijar precio por encima de coste
				\4[] Si competencia à la Cournot
				\4[] $\to$ Reducción de estándares medioambientales exigidos
				\4[] $\then$ Permite reducir costes marginales a empresas nacionales
				\4[] $\then$ Race-to-the-bottom
			\3 Implicaciones
				\4 Política comercial más allá de competencia perfecta
				\4 Política comercial
				\4[] altera patrón de comercio
				\4[] Efectos de retroalimentación con estructura industrial
			\3 Valoración
	\1[] \marcar{Conclusión}
		\2 Recapitulación
			\3 Precursores
			\3 Competencia monopolística
			\3 Oligopolio
			\3 Otros modelos
		\2 Idea final
			\3 Cambios recientes en el comercio internacional
				\4 Cambios tecnológicos
				\4[] Transformación de procesos industriales
				\4[] Nuevas variedades de producto
				\4[] Cambios en productividad relativa
				\4 Cambios en preferencias
				\4[] Nuevos productos generan nuevas demanda
				\4 Cambios en dotaciones
				\4[] Liberalización de movimientos de capital
				\4[] Flujos migratorios
				\4 Tensiones políticas
				\4[] Amenazas de guerra comercial
				\4[] Debilitamiento del multilateralismo
				\4[$\then$] Efectos profundos sobre CI
				\4 Tensiones
			\3 Influencia en otras áreas de la ciencia económica
				\4 Análisis de política comercial
				\4 Crecimiento económico y comercio
				\4 Geografía económica
				\4[] ¿Dónde y cómo se localizan ff.pp?
				\4[] ¿Dónde se produce qué?
				\4 Desigualdades inter- e intra-país
			\3 Características del comercio mundial actual
				\4 Norte-norte
				\4[] Domina comercio intraindustrial
				\4 Países grandes
				\4[] Más habitual el comercio intraindustrial
				\4[] $\to$ Más industrias
				\4[] $\to$ Más variedades producidas en cada industria
\end{esquemal}

\graficas


\begin{axis}{4}{Modelo de firmas heterogéneas de Melitz (2003): costes fijos de venta nacional y de exportación determinan resultado de apertura al comercio internacional.}{}{CF \\ $\pi$}{melitz}
	% extensión del eje de abscisas
	\draw[-] (4,0) -- (7.6,0);
	\draw[dotted] (7.6,0) -- (8,0);
	\node[below] at (7.6,0){$\phi$};
	
	% costes fijos de exportación
	\draw[dotted] (0,2) -- (5,2) -- (5,0);
	\node[left] at (0,2){$F_x$};
	
	% costes de fijos de venta en mercado doméstico
	\draw[dotted] (0,1.1) -- (1.27,1.1) -- (1.27,0);
	\draw[dotted] (1.1,1.1) -- (2.77,1.1) -- (2.77,0);
	\node[left] at (0,1.1){$F$};
	
	% Beneficio operativo de venta doméstica de autarquía
	\draw[dashed] (0,0) to [out=40, in=230](4,4);
	\node[above] at (4,4){$\pi_d^a$};
	
	% Beneficio operativo de venta doméstica de libre comercio
	\draw[-] (0,0) to [out=15, in=230](6,4);
	\node[above] at (6,4){$\pi_d^*$};
	
	% Beneficio operativo de exportación
	\draw[-] (0,0) to [out=5, in=230](6,3);
	\node[right] at (6,3){$\pi_x^*$};	
	
	% productividad mínima para vender en mercado doméstico
	% en autarquía
	\node[below] at (1.27,0){\small $\phi^*_A$};
	% en libre comercio
	\node[below] at (2.77,0){\small $\phi^*_\text{LC}$};
	
	\draw[decorate,decoration={brace, mirror,amplitude=3pt},xshift=0pt,yshift=-0.5cm] (0,0) -- (2.77,0) node[black,midway,xshift=2pt, yshift=-0.33cm] {\tiny Abandona};
	
	\draw[decorate,decoration={brace, mirror,amplitude=3pt},xshift=0pt,yshift=-0.5cm] (2.77,0) -- (5,0) node[black,midway,xshift=2pt, yshift=-0.33cm] {\tiny Sólo doméstico};
	
	% Productividad mínima para exportar
	\node[below] at (5,0){\small $\phi_x^*$};
	
	\draw[decorate,decoration={brace, mirror,amplitude=3pt},xshift=0pt,yshift=-0.5cm] (5,0) -- (8,0) node[black,midway,xshift=2pt, yshift=-0.33cm] {\tiny Exporta};

\end{axis}

\begin{axis}{4}{Modelo de Krugman (1991) del núcleo y la periferia. Diagrama de fase de la dinámica con costes de transporte elevados que resultan en un equilibrio con la población de obreros dispersada.}{}{$\dot{\lambda}$}{krugman91dispersion}
	% Extensión del eje de ordenadas hacia abajo y de abscisas hacia la derecha
	\draw[-] (0,0) -- (0,-4);
	\draw[-] (4,0) -- (8,0);
	\node[below] at (8.4,0){$\lambda$};	
	% Límite con lambda = 1
	\node[below] at (8,-0.2){$1$};
	\draw[-] (8,0.2) -- (8,-0.2);
	% Centro con lambda = 0.5
	\node[below] at (4,-0.3){0,5};	
	\draw[-] (4,0.2) -- (4,-0.2);

	% Dinámica de dot{lambda}
	\draw[-] (0,3) to [out=10, in=100](4,0) to [out=-80, in=200](8,-3);

	% Flechas de tendencia
	% Hacia la derecha desde origen
	\draw[-{Latex}] (0.5,0.5) -- (3.5,0.5);

	% Hacia la izquierda desde límite derecho
	\draw[-{Latex}] (7.5,-0.5) -- (4.5,-0.5);

\end{axis}

La variable $\lambda$ representa la concentración de la población de trabajadores obreros móviles en un uno de los países en cuestión.


\begin{axis}{4}{Modelo de Krugman (1991) del núcleo y la periferia. Diagrama de fase de la dinámica con costes de transporte intermedios que resultan en múltiples equilibrios posibles.}{}{$\dot{\lambda}$}{krugman91multiplesequilibrios}
	% Extensión del eje de ordenadas hacia abajo y de abscisas hacia la derecha
	\draw[-] (0,0) -- (0,-4);
	\draw[-] (4,0) -- (8,0);
	\node[right] at (8.4,0){$\lambda$};	
	% Límite con lambda = 1
	\node[below] at (8,-0.2){$1$};
	\draw[-] (8,0.2) -- (8,-0.2);
	% Centro con lambda = 0.5
	\node[below] at (4,-0.3){0,5};	
	\draw[-] (4,0.2) -- (4,-0.2);

	% Dinámica de dot{lambda}
	\draw[-] (0,-4) to [out=80,in=180](3,1) to [out=0, in=180](5,-1) to [out=0, in=260](8,4);

	% Flechas de tendencia
	% Hacia límite izquierdo, aglomeración
	\draw[-{Latex}] (1.4,0.5) -- (0.2,0.5);
	% Hacia centro desde izquierda, dispersión
	\draw[-{Latex}] (1.8,0.5) -- (3.6,0.5);
	% Hacia centro desde derecha, dispersión
	\draw[-{Latex}] (6.8,0.5) -- (4.4,0.5);
	% Hacia derecha, aglomeración	
	\draw[-{Latex}] (7.2,0.5) -- (8,0.5);

\end{axis}


\begin{axis}{4}{Modelo de Krugman (1991) del núcleo y la periferia. Diagrama de fase de la dinámica con costes de transporte reducidos pero presentes: el equilibrio tiende a la aglomeración.}{}{$\dot{\lambda}$}{krugman91aglomeracion}
	% Extensión del eje de ordenadas hacia abajo y de abscisas hacia la derecha
	\draw[-] (0,0) -- (0,-4);
	\draw[-] (4,0) -- (8,0);
	\node[right] at (8.4,0){$\lambda$};	
	% Límite con lambda = 1
	\node[below] at (8,-0.2){$1$};
	\draw[-] (8,0.2) -- (8,-0.2);
	% Centro con lambda = 0.5
	\node[below] at (4,-0.3){0,5};	
	\draw[-] (4,0.2) -- (4,-0.2);

	
	% Dinámica de dot{lambda}
	\draw[-] (0,-4) to [out=80, in=260](8,4) ;

	% Flechas de tendencia
	\draw[-{Latex}] (3.7,0.5) -- (0.3,0.5);
	\draw[-{Latex}] (4.3, -0.5) -- (7.7,-0.5);

\end{axis}



\begin{axis}{4}{Modelo neo-Heckscher-Ohlin de Falvey  (1981) para explicar comercio intraindustrial como resultado de la especialización en diferentes calidades.}{$\alpha$}{$p(\alpha)$}{neoho}
	% país con trabajo barato y capital caro
	\draw[-] (0,1) -- (4,3);
	\node[left] at (0,1){$w_2$};
	
	% país con trabajo caro y capital barato
	\draw[-] (0,1.75) -- (4,2.25);
	\node[left] at (0,1.75){$w_1$};
	
	% calidad de mismo coste
	\draw[dashed] (2,2) -- (2,0);
	\node[below] at (2,0){$\alpha_0$};
	
	% Especialización
	\draw[-{Latex}] (2,-0.5) -- (0,-0.5);
	\node[below] at (1,-0.8){\tiny País 2 produce};
	
	\draw[-{Latex}] (2,-0.5) -- (4,-0.5);
	\node[below] at (3,-0.8){\tiny País 1 produce};
\end{axis}

\conceptos

\concepto{Hipótesis de Linder}

Según esta conjetura formulada en 1961 por Burestam Linder, las diferencias en las preferencias de las economías constituyen una importante barrera al comercio. De esta forma, las economías con demandas similares tenderán a comerciar más entre sí. 

\concepto{Índice de Balassa}

También denominado \textit{índice de la ventaja comparativa revelada}, trata de medir la competitividad de un país en lo que respecta a las exportaciones de un producto determinado. Consiste en el cociente entre el porcentaje que representan las exportaciones de ese sector sobre el total de exportaciones del país, y el porcentaje que ese sector representa sobre el total mundial de exportaciones. Así, el índice de Balassa para un país $i$ en lo que respecta al sector $j$ se expresa formalmente como:

\begin{align*}
\text{BI}_i^j = \frac{X_i^j / X_i}{X_w^j / X_w}
\end{align*}


\preguntas

\seccion{Test 2017}
\textbf{27.} La ``Paradoja de Leontief'' se basa en la observación de que:

\begin{itemize}
	\item[a] EEUU es un país relativamente abundante en capital, pero las exportaciones suelen ser relativamente intensivas en capital.
	\item[b] EEUU es un país relativamente abundante en capital, pero las importaciones suelen ser relativamente intensivas en capital.
	\item[c] EEUU es un país relativamente abudante en trabajo, pero las importaciones suelen ser relativamente intensivas en capital.
	\item[d] EEUU es un país relativamente abundante en trabajo, pero las importaciones suelen ser relativamente intensivas en trabajo.
\end{itemize}

\textbf{29.} En comercio internacional, el ``modelo de gravedad'':

\begin{itemize}
	\item[a] Es un modelo que intenta explicar los flujos comerciales bilaterales.
	\item[b] Predice que un aumento del PIB, pero no del PIB per cápita, aumentará el comercio relativamente más que dicho aumento del PIB.
	\item[c] Predice que los países más pobres tienden a comerciar más entre ellos que los más ricos.
	\item[d] Predice que el comercio entre países depende linealmente de su distancia.
\end{itemize}

\seccion{Test 2016}

\textbf{30.} En relación a la Nueva Teoría del Comercio (NTT) y la Nueva Economía Geográfica (NEG), en presencia de economías de escala, competencia monopolística, costes de transporte, un bien diferenciado (manufacturas) y gusto por la variedad:

\begin{itemize}
	\item[a] El ``Home-Market Effect'' prevé que si un país tiene un mercado mayor para el bien diferenciado, acabará produciéndolo a escala y exportándolo al resto del mundo.
	\item[b] El ``Home-Market Effect'', bajo ciertas circunstancias, prevé el pago de un mayor salario a los trabajadores del sector que produce a escala ese bien diferenciado.
	\item[c] El ``Price Index Effect'' determinará que el índice de precios de las manufacturas en el mercado más grande para los productos comercializados será superior.
	\item[d] La respuesta a) y b) son verdaderas.
\end{itemize}

\seccion{Test 2015}

\textbf{27.} El comercio intraindustrial será dominante con respecto al comercio interindustrial (señale la respuesta verdadera):
\begin{itemize}
	\item[a] Si las relaciones capital-trabajo entre los países difieren sensiblemente y los sectores que intervienen en el comercio disfrutan de economías de escala.
	\item[b] Si las relaciones capital-trabajo entre los países son muy similares y los sectores que intervienen en el comercio disfrutan de economías de escala.
	\item[c] Es suficiente con que los sectores que intervienen en el comercio disfruten de economías de escala.
	\item[d] Si se produce entre países con un nivel de desarrollo económico desigual.
\end{itemize}

\textbf{28.} Señale la respuesta verdadera relativa al modelo de la brecha o desfase tecnológico de Posner (1961):

\begin{enumerate}
	\item[a] En esencia, explica el comercio por las diferencias en las preferencias de los consumidores.
	\item[b] Surge porque el modelo H-O consideró la tecnología dinámicamente.
	\item[c] Surge porque el modelo de Heckscher-Ohlin consideró la tecnología estáticamente.
	\item[d] Ninguna de las anteriores.
\end{enumerate}

\textbf{29.} Supongamos el caso de dos países, A y B, en cada uno de los cuales existe una única empresa que produce el bien X. Inicialmente ambos países están cerrados al comercio internacional, por lo que en cada país sólo opera su empresa nacional, en régimen de monopolio. Las preferencias y el poder adquisitivo de los consumidores son similares en los dos países, las empresas tienen la misma tecnología y los mismos costes de producción, que son constantes. Si se abren al comercio y existen costes de transporte (señale la verdadera):
\begin{itemize}
	\item[a] Dadas todas las similitudes enunciadas, no es posible que haya comercio internacional.
	\item[b] El resultado del comercio internacional es negativo en términos sociales debido a la existencia de los costes de transporte.
	\item[c] El resultado del comercio internacional puede ser beneficioso en términos sociales si las ganancias sociales derivadas del debilitamiento de las posiciones de poder de mercado son superiores al despilfarro derivado de la existencia de costes de transporte.
	\item[d] En ausencia de colusión entre las empresas, se llega a una situación en la que el precio es igual al coste marginal en ambos mercados.
\end{itemize}

\seccion{Test 2014}
\textbf{29.} La introducción de los costes de transporte en el modelo de dumping recíproco de Brander-Krugman implica:
\begin{itemize}
	\item[a] Una caída de las exportaciones y aumento del precio.
	\item[b] Una caída de las importaciones y una disminución del precio.
	\item[c] Un incremento de las exportaciones y de la producción total.
	\item[d] Una disminución del comercio internacional y un incremento de la producción total.
\end{itemize}

\seccion{Test 2011}
\textbf{27.} El modelo de Brander-Krugman explica existencia de comercio intraindustrial:
\begin{itemize}
	\item[a] Por la existencia de economías de escala en un mercado oligopolístico.
	\item[b] Por la existencia de preferencia por la variedad en un mercado oligopolístico.
	\item[c] Por la existencia de ventaja comparativa en un mercado oligopolístico.
	\item[d] Por el comportamiento de las empresas en un mercado oligopolístico.
\end{itemize}

\seccion{Test 2008}
\textbf{25.} Entre las explicaciones alternativas a la paradoja de Leontief, están:
\begin{itemize}
	\item[a] El patrón de especialización factorial.
	\item[b] La consideración del comercio intraindustrial y la posibilidad de existencia de reversión de factores.
	\item[c] Las condiciones similares de productividad de la mano de obra en países con productos competitivos en las importaciones.
	\item[d] Todas las respuestas son falsas.
\end{itemize}

\seccion{Test 2007}
\textbf{27.} Las modernas teorías del comercio internacional analizan el intercambio entre países de:
\begin{itemize}
	\item[a] Bienes homogéneos, producidos en condiciones de rendimientos constantes a escala, donde los consumidores de los distintos países prefieren disfrutar de aquellos bienes que sean más baratos. 
	\item[b] Bienes diferenciados, producidos en condiciones de rendimientos crecientes a escala, donde los consumidores de los distintos países prefieren disfrutar una gama amplia de variedades de los distintos bienes.
	\item[c] Bienes homogéneos, producidos en condiciones de rendimientos crecientes a escala, donde los consumidores de los distintos países prefieren disfrutar una gama amplia de variedades de los distintos bienes.
	\item[d] Bienes diferenciados, producidos en condiciones de rendimientos constantes a escala, donde los consumidores de los distintos países prefieren disfrutar de aquellos bienes que sean más baratos.
\end{itemize}

\seccion{Test 2006}
\textbf{25.} La ecuación conocida como ecuación gravitatoria (gravity equation), presente en la teoría del comercio internacional, relaciona:
\begin{itemize}
	\item[a] Los volúmenes de comercio bilateral entre dos países con una medida del tamaño del importador y del tamaño del exportador y algún índice de la distancia entre el exportador y el importador.
	\item[b] Los volumenes de comercio bilateral entre dos países con algún índice de la distancia entre el exportador y el importador.
	\item[c] Los volúmenes de comercio bilateral entre dos países con el tamaño relativo del importador en relación al exportador.
	\item[d] Los volúmenes de comercio bilateral entre dos países con una medida del tamaño importador y del tamaño del exportador y algún índice del grado de competencia relativo entre ambos mercados.
\end{itemize}

\textbf{26.} En los modelos de comercio internacional basados en competencia monopolística, puede señalarse en relación al tamaño de un país y su volumen de exportaciones lo siguiente (señale la respuesta VERDADERA):
\begin{itemize}
	\item[a] Los países de mayor tamaño exportan más porque exportan una mayor candidad de cada bien.
	\item[b] Los países de mayor tamaño exportan más porque sus dotaciones factoriales son mayores.
	\item[c] Los países de mayor tamaño exportan más porque exportan una mayor variedad de bienes.
	\item[d] No puede establecerse ningún tipo de relación entre tamaño de país y volumen de comercio en estos modelos.
\end{itemize}

\seccion{Test 2005}
\textbf{26.} De acuerdo con las modernas teorías del comercio internacional, el comercio intraindustrial consiste en el intercambio entre países de:
\begin{itemize}
	\item[a] Productos homogéneos, obtenidos en condiciones de competencia perfecta, en función de sus precios relativos.
	\item[b] Variedades de un bien diferenciado, obtenidas en condiciones de economías externas a las empresas, en función de sus precios relativos.
	\item[c] Productos homogéneos, obtenidos en condiciones de competencia imperfecta, en función de sus intensidades factoriales relativas.
	\item[d] Variedades de un bien diferenciado, obtenidas en condiciones de economías de escala internas a las empresas, debido a la existencia de demanda para las distintas variedades en los distintos países.
\end{itemize}

\seccion{Test 2004}
\textbf{25.} Los principales puntos de partida del modelo de comercio internacional en condiciones de competencia imperfecta, asociado con los nombres de E. Helpman y P. Krugman, son los siguientes:
\begin{itemize}
	\item[a] Las empresas diferencian el producto que elaboran, cada variedad se produce en condiciones de rendimientos decrecientes a escala en una estructura de mercado de competencia monopolística, y, al producirse cada variedad exclusivamente en un sólo país, lo que tiene lugar es el intercambio de distintas variedades entre distintos países.
	\item[b] Las empresas diferencian el producto que elaboran, cada variedad se produce en condiciones de rendimientos crecientes a escala en una estructura de mercado de duopolio a la Cournot, y, al producirse cada variedad exclusivamente en un sólo país, lo que tiene lugar es el intercambio de distintas variedades entre distintos países. 
	\item[c] Las empresas diferencian sólo parcialmente el producto que elaboran, cada variedad se produce en condiciones de rendimientos crecientes a escala en una estructura de mercado de competencia monopolística, y, al producirse cada variedad simultáneamente en varios países, lo que tiene lugar es el intercambio de las mismas variedades entre los mismos países.
	\item[d] Las empresas diferencian el producto que elaboran, en cada variedad se produce en condiciones de rendimientos crecientes a escala en una estructura de mercado de competencia monopolística, y, al producirse cada variedad exclusivamente en un sólo país, lo que tiene lugar es el intercambio de distintas variedades enre distintos países.
\end{itemize}

\notas

\textbf{2017:} \textbf{27.} B \textbf{29.} A

\textbf{2016:} \textbf{30.} D

\textbf{2015:} \textbf{27.} B \textbf{28}. C \textbf{29.} C 

\textbf{2014:} \textbf{29.} A 

\textbf{2011:} \textbf{27.} D

\textbf{2008:} \textbf{25.} D

\textbf{2007:} \textbf{27.} B

\textbf{2006:} \textbf{25.} A \textbf{26.} C

\textbf{2005:} \textbf{26.} D

\textbf{2004:} \textbf{25.} D

\bibliografia

Mirar en Palgrave:
\begin{itemize}
	\item factor prices in general equilibrium
	\item gravity equation
	\item gravity models
	\item international trade
	\item international trade and heterogeneous firms
	\item international trade, empirical approaches to
	\item international trade theory
	\item Leontief paradox
	\item terms of trade
	\item tradable and non-tradable commodities
	\item trade costs
	\item trade cycle
	\item trade policy, political economy of
\end{itemize}

Helpman, E.; Melitz, M.; Yeaple, S. (2004) \textit{Export Versus FDI with Heterogeneous Firms} American Economic Review Vol. 94. No.1 -- En carpeta del tema

Hirsch, S. (1975) \textit{The Product Cycle Model of International Trade--A Multi-Country Cross Section Analysis} -- En carpeta del tema

Kemp, M. C. (1965) \textit{Review: An Essay on Trade an Transformation by S. B. Linder} {The Economic Journal} Vol.75 pp. 200-201 -- En carpeta del tema

Krugman, P. R. (1979) \textit{Increasing returns, monopolistic competition, and international trade} Journal of International Economics 9 469-479 -- En carpeta del tema.

Krugman, P. R. (1980) \textit{Scale Economies, Product Differentiation, and the Pattern of Trade} The American Economic Review, Vol. 70. No. 5 pp. 950-959 -- En carpeta del tema

Krugman, P. R. (1991) \textit{Increasing Returns and Economic Geography} Journal of Political Economy, Vol. 99, No. 3 -- En carpeta del tema

Krugman, P. (2008) \textit{The Increasing Returns Revolution in Trade and Geography} Nobel Prize Lecture, December 8, 2008 -- En carpeta del tema

Posner, M. V. (1961) \textit{International Trade and Technical Change} Oxford Economic Papers. New Series. Vol. 13 -- En carpeta del tema


\end{document}
