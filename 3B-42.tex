\documentclass{nuevotema}

\tema{3B-42}
\titulo{La Unión Europea: las políticas de transporte, energía, medio ambiente, industria y tecnología, y sociedad de la información.}

\begin{document}

\ideaclave


Añadir Programa Cordis de Unión Europea: base de datos sobre proyectos de investigación \href{https://cordis.europa.eu/en}{Ver aquí}.

ENFATIZAR COMPETENCIAS EN CADA SECTOR -- APRENDER COMPETENCIAS EXCLUSIVAS Y COMPARTIDAS: Artículos 3 y 4 del TFUE

La Unión Europea es una unión monetaria y económica que trata de alcanzar sus objetivos primarios por medio de una serie de objetivos intermedios que conllevan la actuación en numerosas áreas. Las políticas de mayor importancia son aquellas con efectos pronunciados en todas las áreas de la economía y la sociedad, tales como la política monetaria, la política comercial o la política de competencia. Pero éstas áreas no son las únicas en las cuales la Unión ejerce su competencia. Existen una serie de materias menos generales pero con características comunes que son también objeto de actuación comunitaria. Los puntos en común son la incidencia sobre la competitividad, el carácter de industrias de red y la relación con el medio ambiente. Además, todas ellas son competencia compartida de la UE o competencia de apoyo. Son las siguientes: políticas de transporte, energía, medio ambiente, industria y tecnología y la sociedad de la información.

El objetivo de la Unión Europea al actuar en estas materias es mejorar la eficiencia de la economía europea y su crecimiento sostenible. Para ello, los fondos se destinan a sectores con fuertes externalidades y efectos sobre el mayor número posible de industrias o sectores relacionados. Se encuentran relacionados con este tema:

\begin{itemize}
    \item Tablas input-output: eslabonamientos sectores industriales
    \item Análisis teórico de la empresa
    \item Nuevos desarrollos del comercio internacional
    \item Economía del bienestar: externalidades
    \item Economía del medio ambiente
    \item Instituciones europeas
    \item Marco financiero y presupuestario de la UE
\end{itemize}

\seccion{Preguntas clave}

\begin{itemize}
	\item ¿Qué tienen en común los sectores objeto de políticas europeas específicas?
	\item ¿Por qué es necesaria la actuación de la UE en estos sectores?
	\item ¿En qué consisten las políticas europeas de cada uno de estos sectores?
	\item ¿Qué resultados han obtenido?
	\item ¿Qué retos futuros afrontan?
\end{itemize}


\esquemacorto

\begin{esquema}[enumerate]
	\1[] \marcar{Introducción}
		\2 Contextualización
			\3 Unión Europea
			\3 Competencias de la UE
			\3 Sectores con características especiales
			\3 Políticas sectoriales europeas
		\2 Objeto
			\3 ¿Qué tienen en común los sectores analizados?
			\3 ¿Por qué es necesaria la actuación de la Unión Europea?
			\3 ¿En qué consisten las políticas europeas en cada uno de estos sectores?
			\3 ¿Qué resultados han obtenido las políticas implementadas?
			\3 ¿Qué retos futuros afrontan?
		\2 Estructura
			\3 Transportes
			\3 Energía
			\3 Medio ambiente
			\3 Industria y tecnología
			\3 Sociedad de la información
	\1 \marcar{Transportes}
		\2 Justificación
			\3 Mercado único
			\3 Regiones periféricas y ultraperiféricas
			\3 Necesidades financieras elevadas
		\2 Objetivos
			\3 Mejorar conectividad a nivel europeo
			\3 Reducir impacto medioambiental
			\3 Aprovechar economías de escala europeas
		\2 Antecedentes
			\3 Tratado de París (1951)
			\3 Tratado de Roma (1957)
			\3 Caso 13/83 de Parlamento vs Consejo
			\3 Tratado de Maastricht (1992)
			\3 Tratado de Lisboa (2007)
		\2 Marco jurídico
			\3 TUE
			\3 TFUE
			\3 Caso 13/83 Parlamento vs. Consejo
			\3 Reglamento sobre RTE de 2013
			\3 Directivas sectoriales del transporte
		\2 Marco financiero
			\3 EFSI/FEIE
			\3 FEDER
			\3 Fondo de Cohesión
			\3 CEF -- Connecting Europe Facility
			\3 Programa Marco Polo
			\3 Horizonte 2020
		\2 Actuaciones
			\3 RTE-T/TEN-T
			\3 Mercado interior: liberalización de transporte
			\3 Seguridad
			\3 Acuerdos internacionales en sector aéreo
			\3 Derechos de consumidores y pasajeros
			\3 Reducción de emisiones
		\2 Valoración
			\3 Criterios generales de valoración
			\3 Carretera
			\3 Ferrocarril
			\3 Aéreo
			\3 Marítimo
		\2 Retos
			\3 Energía y medioambiente
			\3 Necesidades de capital
			\3 Identificación de proyectos prioritarios viables
	\1 \marcar{Energía}
		\2 Justificación
			\3 Energía es input esencial
			\3 Dependencia energética exterior
			\3 Proveedores poco diversificados
			\3 Externalidades de consumo energético
		\2 Objetivos
			\3 Garantizar seguridad energética
			\3 Diversificar proveedores
			\3 Diversificar fuentes de energía
			\3 Conectar mercados energéticos nacionales
			\3 Mantener precios competitivos
			\3 Fomentar investigación y competitividad
			\3 Promover uso de energías renovables
		\2 Antecedentes
			\3 Tratado de la CECA (1951)
			\3 Euratom y CEE (1957)
			\3 Crisis del petróleo de 1973 y 1979
			\3 Acta Única Europea de 1987
			\3 Tratado de Maastricht de 1991
			\3 Tratado de Lisboa
		\2 Marco jurídico
			\3 TFUE
			\3 Paquetes de directivas de 1996, 2003 y 2009
			\3 Estrategia Europa 2020
			\3 Paquete de energía limpia de 2019
		\2 Marco financiero
			\3 Fondo de Cohesión
			\3 FEDER
			\3 Connecting Europe Facility
			\3 EFSI/FEIE
			\3 Fondo Europeo de Eficiencia Energética
			\3 Fondo de Innovación
		\2 Actuaciones
			\3 SINCRONIZAR CON TEMAS DE 4A SOBRE POLÍTICA ENERGÉTICA Y MAMBIENTAL
			\3 RTE-Energía
			\3 Seguridad energética
			\3 Mercado Interior
			\3 Objetivos 2020 y 2030 en ámbito energético
			\3 Eficiencia energética
			\3 SET-Plan
			\3 Paquete de energía limpia de 2019
			\3 Metales para baterías
		\2 Valoración
			\3 Dependencia exterior muy elevada
			\3 Energía nuclear
		\2 Retos
			\3 Incertidumbre regulatoria e inversión
			\3 Unión de la Energía
			\3 Autosuficiencia energética
	\1 \marcar{Medio ambiente}
		\2 Justificación
			\3 Impacto sobre salud humana
			\3 Externalidades
		\2 Objetivos
			\3 Incentivar transición energética
			\3 Mantener competitividad
			\3 Actuación a nivel global
		\2 Antecedentes
			\3 Consejo Europeo de París (1972)
			\3 Acta Única Europea (1987)
			\3 ONU: Conferencia sobre el Cambio Climático
			\3 Tratado de Amsterdam (1997)
			\3 Proceso de Cardiff (1998)
			\3 Agenda de Gotemburgo (2001)
			\3 Tratado de Lisboa (2007)
		\2 Marco jurídico
			\3 Proceso de Cardiff (1998)
			\3 Acuerdo de París de 2015
			\3 TUE 3.3
			\3 TFUE.11, 191 y 193
			\3 Estrategia Europa 2020
			\3 Comunicación de CE (2018): visión estratégica de l/p
			\3 Directivas sectoriales
			\3 EAP
			\3 Paquete de Energía Limpia de 2019
		\2 Marco financiero
			\3 Programa LIFE
			\3 Fondo de Cohesión
			\3 Fondo de Innovación
			\3 Fondo Europeo de Eficiencia Energética
			\3 EFSI/FEIE
			\3 FEADER/FEMP
		\2 Actuaciones
			\3 Principios generales -- PPAC
			\3 Agencia Europea de Medio Ambiente
			\3 Objetivos 2020, 2030, 2050
			\3 VII Programa de Acción Medioambiental
			\3 VIII Programa de Acción Medioambiental
			\3 ETS -- Emissions Trading Scheme
			\3 Taxonomía de inversiones sostenibles
			\3 Planes nacionales de energía y clima para 2021-2030
			\3 Estrategias nacionales de largo plazo
			\3 Evaluaciones medioambientales
			\3 Informe SOER 2020
		\2 Valoración
			\3 Problemas del ETS
			\3 Objetivos de reducción de emisiones
		\2 Retos
			\3 Fuga de carbono
			\3 Política comercial
			\3 Exceso de derechos emitidos
			\3 Desarrollo y medio ambiente
			\3 Cumplimiento de objetivos
			\3 Reforma del sistema de emisiones 2021-2030
	\1 \marcar{Tecnología}
		\2 Justificación
			\3 Economías de escala en investigación
			\3 Externalidades positivas de la investigación
			\3 Investigación básica
		\2 Objetivos
			\3 Catalizar investigación básica
			\3 Promover transferencia a sector privado industrial
			\3 Incentivar investigaciones estratégicas
			\3 Comunicación de la CE 2017
		\2 Antecedentes
			\3 Agenda de Lisboa (2000)
			\3 Estrategia 2020 (2010)
			\3 Comunicación de la Comisión de 2017
		\2 Marco jurídico
			\3 TFUE 4
			\3 TFUE.173
			\3 Estrategia Europa 2020
			\3 Estrategia de Política Industrial 2017
		\2 Marco financiero
			\3 EFSI/FEIE
			\3 FEIE -- Fondos Estructurales y de Inversión Europea
			\3 Horizonte 2020
		\2 Actuaciones
			\3 Financiación de I+D e industria
			\3 Horizonte 2020
			\3 ERA -- European Research Area
			\3 ERA especial para Covid-19
			\3 Programa de respuesta global frente a COVID-19
			\3 Agencia Espacial Europea
			\3 Programa Galileo
			\3 EIP -- Acuerdos Europeos de Innovación
		\2 Valoración
			\3 Problemas de coherencia
			\3 Complejidad de programas
			\3 Éxitos
		\2 Retos
			\3 Reducir influencia de grupos de interés
			\3 Excesiva complejidad
			\3 Intereses de EEMM vs UE
			\3 Dificultades de implementación
	\1 \marcar{Política industrial}
		\2 Justificación
			\3 Economías de escala industriales
			\3 Peso de industria en PIB
			\3 Pérdida de competitividad relativa
		\2 Objetivos
			\3 Acelerar adaptación a cambios estructurales
			\3 Entorno favorable
			\3 Aumentar peso de industria en PIB
			\3 Explotar ventajas competitivas europeas
		\2 Antecedentes
			\3 50s a 70s
			\3 80s
			\3 Tratado de Maastricht (1992)
			\3 Informe Bangemann (1994)
		\2 Marco jurídico
			\3 TFUE 6
		\2 Marco financiero
			\3 EFSI/FEIE
			\3 FEIE -- Fondos Estructurales y de Inversión Europea
			\3 Programa COSME
		\2 Actuaciones
			\3 Programa COSME
			\3 Estrategia de Política Industrial 2017
			\3 Skills Agenda de 2016 -- Comunicación de la Comisión
			\3 Unión por la Innovación
			\3 Política comercial y rel. econ. exteriores
			\3 Competencia
		\2 Valoración
		\2 Retos
	\1 \marcar{Sociedad de la información}
		\2 Justificación
			\3 Economías de escala potenciales
			\3 Mejoras potenciales de productividad
			\3 Acceso aún incompleto a internet
			\3 Seguridad
			\3 Economías de red
		\2 Objetivos
			\3 Acceso universal a internet
			\3 Digitalización de empresas
			\3 Mercado único digital
			\3 Aprovechamiento de economías de escala
		\2 Antecedentes
			\3 Estrategia DSM
		\2 Marco jurídico
			\3 RTE-Te
			\3 Comunicación sobre Agenda Digital
			\3 Comunicación sobre política industrial de 2017
			\3 GDPR -- General Data Protection Regulation
			\3 Directivas del Mercado Único Digital
			\3 Directiva de Copyright aprobada por PE en 2019
			\3 Reglamento de libre circulación de datos no personales 2019
			\3 Regulación sobre 5G
		\2 Marco financiero
			\3 FEDER
			\3 Fondo de Cohesión
			\3 Horizonte 2020
			\3 Programa COSME
			\3 EFSI
			\3 Connecting Europe Facility -- Telecomunicaciones
		\2 Actuaciones
			\3 Ciberseguridad
			\3 Mercado Único Digital
			\3 Datos
			\3 Otras actuaciones
			\3 Plan de Acción 5G Europeo
		\2 Valoración
			\3 Éxitos
			\3 Controversias
		\2 Retos
			\3 Completar mercado único digital
			\3 Digitalización de empresas
			\3 Liberalización frente a regulación
			\3 Ciberseguridad
			\3 Inteligencia artificial
	\1[] \marcar{Conclusión}
		\2 Recapitulación
			\3 Transportes
			\3 Energía
			\3 Medio ambiente
			\3 Industria y tecnología
			\3 Sociedad de la información
		\2 Idea final
			\3 Liberalización y papel de la UE
			\3 Sensibilidad de políticas a nivel nacional
			\3 Papel de EEMM
			\3 Comparación con China o EEUU

\end{esquema}

\esquemalargo












\begin{esquemal}
	\1[] \marcar{Introducción}
		\2 Contextualización
			\3 Unión Europea
				\4 Institución supranacional ad-hoc
				\4[] Diferente de otras instituciones internacionales
				\4[] Medio camino entre:
				\4[] $\to$ Federación
				\4[] $\to$ Confederación
				\4[] $\to$ Alianza de estados-nación
				\4 Origen de la UE
				\4[] Tras dos guerras mundiales en tres décadas
				\4[] $\to$ Cientos de millones de muertos
				\4[] $\to$ Destrucción económica
				\4[] Marco de integración entre naciones y pueblos
				\4[] $\to$ Evitar nuevas guerras
				\4[] $\to$ Maximizar prosperidad económica
				\4[] $\to$ Frenar expansión soviética
				\4 Objetivos de la UE
				\4[] TUE -- Tratado de la Unión Europea
				\4[] $\to$ Primera versión: Maastricht 91 $\to$ 93
				\4[] $\to$ Última reforma: Lisboa 2007 $\to$ 2009
				\4[] Artículo 3
				\4[] $\to$ Promover la paz y el bienestar
				\4[] $\to$ Área de seguridad, paz y justicia s/ fronteras internas
				\4[] $\to$ Mercado interior
				\4[] $\to$ Crecimiento económico y estabilidad de precios
				\4[] $\to$ Economía social de mercado
				\4[] $\to$ Pleno empleo
				\4[] $\to$ Protección del medio ambiente
				\4[] $\to$ Diversidad cultural y lingüistica
				\4[] $\to$ Unión Económica y Monetaria con €
				\4[] $\to$ Promoción de valores europeos
				\4[$\to$] Objetivos de la UE
				\4[] Paz y bienestar a pueblos de Europa
			\3 Competencias de la UE
				\4 Tratado de la Unión Europea
				\4[] Atribución
				\4[] $\to$ Sólo las que estén atribuidas a la UE
				\4[] Subsidiariedad
				\4[] $\to$ Si no puede hacerse mejor por EEMM y regiones
				\4[] Proporcionalidad
				\4[] $\to$ Sólo en la medida de lo necesario para objetivos
				\4 Exclusivas
				\4[] i. Política comercial común
				\4[] ii. Política monetaria de la UEM
				\4[] iii. Unión Aduanera
				\4[] iv. Competencia para el mercado interior
				\4[] v. Conservación recursos biológicos en PPC
				\4 Compartidas
				\4[] i. Mercado interior
				\4[] ii. Política social
				\4[] iii. Cohesión económica, social y territorial
				\4[] iv. Agricultura y pesca \footnote{Salvo en lo relativo a la conservación de recursos biológicos marinos, que se trata de una competencia exclusiva de la UE}
				\4[] v. Medio ambiente
				\4[] vi. Protección del consumidor
				\4[] vii. Transporte
				\4[] viii. Redes Trans-Europeas
				\4[] ix. Energía
				\4[] x. Área de libertad, seguridad y justicia
				\4[] xi. Salud pública común en lo definido en TFUE
				\4 De apoyo
				\4[] Protección y mejora de la salud humana
				\4[] Industria
				\4[] Cultura
				\4[] Turismo
				\4[] Educación, formación profesional y juventud
				\4[] Protección civil
				\4[] Cooperación administrativa
				\4 Coordinación de políticas y competencias
				\4[] Política económica
				\4[] Políticas de empleo
				\4[] Política social
			\3 Sectores con características especiales
				\4 Sectores en cuestión
				\4[] $\to$ Transporte
				\4[] $\to$ Energía
				\4[] $\to$ Medio ambiente
				\4[] $\to$ Industria y tecnología
				\4[] $\to$ Sociedad de la información
				\4 Características comunes
				\4[] Incidencia sobre la competitividad
				\4[] $\to$ De muchos otros sectores
				\4[] $\to$ Influyen fuertemente sobre crecimiento
				\4[] Industrias de red
				\4[] $\to$ Número de usuarios reduce CMe y aumenta UMg
				\4[] $\to$ Fuertes economías de escala
				\4[] $\to$ Mayor tamaño de mercado aumenta eficiencia
				\4[] Efectos sobre el medio ambiente
				\4[] $\to$ Potencial impacto negativo
				\4[] $\to$ Spillovers entre estados miembros
				\4[] Impacto regional
				\4[] $\to$ Potencial para alterar dist. crecimiento
				\4[] $\to$ Efectos más allá de fronteras nacionales
				\4[] $\to$ Interacción con dinámicas de aglomeración
				\4[$\then$] Margen de actuación a nivel europeo
			\3 Políticas sectoriales europeas
				\4 Competencias compartidas UE--EEMM
				\4[] En todos los casos anteriores
				\4 UE como coordinador
				\4[] Explotar externalidades positivas
				\4[] Reducir externalidades negativas
				\4[] Coordinar actuaciones de EEMM
		\2 Objeto
			\3 ¿Qué tienen en común los sectores analizados?
			\3 ¿Por qué es necesaria la actuación de la Unión Europea?
			\3 ¿En qué consisten las políticas europeas en cada uno de estos sectores?
			\3 ¿Qué resultados han obtenido las políticas implementadas?
			\3 ¿Qué retos futuros afrontan?
		\2 Estructura
			\3 Transportes
			\3 Energía
			\3 Medio ambiente
			\3 Industria y tecnología
			\3 Sociedad de la información
	\1 \marcar{Transportes}
		\2 Justificación
			\3 Mercado único
				\4 Transporte de ByS consustancial a MInterior
				\4 No hay mercado único si los costes son muy altos
			\3 Regiones periféricas y ultraperiféricas
				\4 Transporte insuficiente o poco desarrollado
				\4 Costes de transporte más altos
				\4 Cohesión del territorio europeo
			\3 Necesidades financieras elevadas
				\4 Infraestructuras: altos periodos de maduración
				\4 Inversiones iniciales muy elevadas
				\4 Sector privado nacional
				\4[] Puede tener dificultades para financiar
				\4[] Puede no considerar rentables algunos proyectos
		\2 Objetivos
			\3 Mejorar conectividad a nivel europeo
				\4 Reducir tiempo y coste de tfransporte
				\4[] Regiones lejanas
				\4[] Regiones menos desarrolladas
				\4[] Regiones con conexiones insuficientes
				\4[] Infraestructuras obsoletas
				\4 Nuevas conexiones
			\3 Reducir impacto medioambiental
				\4 Transporte es importante emisor CO2
				\4[] 27\% de emisiones europeas
			\3 Aprovechar economías de escala europeas
				\4 Integrando mercados vía transporte
				\4 Financiando con recursos financieros europeos
		\2 Antecedentes
			\3 Tratado de París (1951)
				\4 Capítulo sobre transportes
			\3 Tratado de Roma (1957)
				\4 Transporte son competencia compartida
				\4 Cierta armonización de normas comunes
			\3 Caso 13/83 de Parlamento vs Consejo
				\4 PE y CE denuncian CdUE ante TJUE
				\4[] Acusan a CdUE:
				\4[] $\to$ No asegurar libertad de provisión de servicios
				\4[] $\to$ No implementar política común de transportes
				\4 TJUE falla a favor de Parlamento
				\4 Punto de inflexión
				\4[] Política de transporte debe desarrollarse
			\3 Tratado de Maastricht (1992)
				\4 Base legal para RTE de transporte
			\3 Tratado de Lisboa (2007)
				\4 Sectores aéreo y marítimo
				\4[] Requerían unanimidad hasta antes de Lisboa
				\4 Transporte por carretera, ferrocarril, canales
				\4[] Mayoría cualificada
				\4[$\then$] Escasos avances en aéreo y marítimo hasta entonces
				\4[$\then$] QMV para aéreo y marítimo tras Lisboa
		\2 Marco jurídico
			\3 TUE
				\4 3.3 sobre mercado interior
				\4 3.4 sobre UEM
			\3 TFUE
				\4 Art. 4 sobre competencias compartidas
				\4[] Transporte
				\4[] Redes transeuropeas
				\4 Transporte
				\4[] 90 y ss.
				\4 Redes Transeuropeas
				\4[] 170-172 y 194
			\3 Caso 13/83 Parlamento vs. Consejo
			\3 Reglamento sobre RTE de 2013
			\3 Directivas sectoriales del transporte
		\2 Marco financiero
			\3 EFSI/FEIE
				\4 European Fund for Strategic Investment
				\4 ``Plan Juncker''
				\4 CE+EIB
				\4[] Garantía presupuestaria de UE
				\4[] Capital de EIB
				\4[] $\to$ Movilizar 500.000 M de €
			\3 FEDER
				\4 ERDF -- European Regional Development Fund
				\4 Fondo Europeo de Desarrollo Regional
				\4 Especialmente en reducción de emisiones
			\3 Fondo de Cohesión
				\4 Principal objetivo es transporte
				\4 Canalizados a traves de Connecting Europe
			\3 CEF -- Connecting Europe Facility
				\4 Dentro de rúbrica 1a de MFP 2014-2020
				\4[] ``Crecimiento, competitividad y empleo''
				\4 Inversión en redes
				\4[] No sólo transporte
				\4[] $\to$ Energía
				\4[] $\to$ Telecomunicaciones
				\4[] $\to$ Transporte
				\4[] $\then$ Explotar sinergias entre los tres sectores
				\4 24.000 M de € para transporte\footnote{30.000 M de € en total}
				\4[] Implementar TEN-T
				\4 Transferencia a Fondo de Cohesión:
				\4[] CEF transfiere 11.3 M de €
				\4 Resto de fondos CEF
				\4[] $\to$ Disponible para todos los EEMM
				\4 Reducción en relación a MFP 2007-2013
				\4 CEF Deuda
				\4[] Garantías de deuda para proyectos
				\4 Distribución de financiación por modos\footnote{\url{https://ec.europa.eu/transport/themes/infrastructure/cef_en}}
				\4[] Ferrocarril: 16.000 M de €
				\4[] Carretera: 2.000 M de €
				\4[] Aéreo: 1.600 M de €
				\4[] Canales: 1.400 M de €
				\4[] Marítimo: 1.100 M de €
				\4[$\then$] Énfasis en transporte por ferrocarril
			\3 Programa Marco Polo
				\4 2007-2013
				\4 Transporte de mercancías que reduce emisiones
				\4 60.000 M de €
			\3 Horizonte 2020
				\4 Programa general de I+D para MFP 14-20
				\4[] 80.000 M de € disponibles en total
				\4 6.339 M de € para Transport Challenge
				\4[] MFP 14-20
		\2 Actuaciones
			\3 RTE-T/TEN-T\footnote{Redes Trans-Europeas/Trans-European Networks-Transportation}
				\4 Red Trans-Europea -- Transporte
				\4 Construcción y mejora de transportes clave
				\4[] Eliminar cuellos de botella
				\4[] Completar secciones transfronterizas clave
				\4[] Superar barreras geográficas naturales
				\4[] Interoperabilidad de rutas
				\4 Todos las modalidades de transporte
				\4 Gran parte gestionados por INEA
				\4[] Innovation and Networks Executive Agency
				\4 Dos capas
				\4[] $\to$ Core/Núcleo
				\4[] $\to$ Comprehensive/global
				\4 Proyectos ``core''
				\4[] A completar en 2030
				\4[] 10 corredores
				\4[] Cofinanciados con EEMM
				\4[] En España:
				\4[] $\to$ Mediterráneo
				\4[] $\to$ Atlántico
				\4 Proyectos ``Comprehensive''
				\4[] A completar en 2050
				\4 Objetivos generales
				\4[] Intermodalidad
				\4[] $\to$ Fácil paso de una red a otra
				\4[] Interoperabilidad
				\4[] $\to$ Diferentes sistemas circulen sobre misma red
				\4[] $\to$ Especialmente en aéreo y tren
				\4[] $\to$ Estándar ERTMS
				\4 Autopistas del mar/Motorways of the sea
				\4[] Mejorar conexión de puertos con interior
				\4[] Aumentar capacidad logística de puertos
			\3 Mercado interior: liberalización de transporte
				\4 Carretera
				\4[] Liberalización bastante avanzada
				\4[] Aún algunas restricciones como cabotaje
				\4[] $\to$ {Carga/descarga en EM distinto de residencia de transportista}
				\4 Ferrocarril
				\4[] ``paquetes ferroviarios''
				\4[] Separación gestor de red y operadores
				\4[] Apertura a transporte de mercancías
				\4[] Reformas reguladores ferroviarios nacionales
				\4 Aéreo
				\4[] Libre acceso a rutas intra comunitarias
				\4[] Asegurar neutralidad de asignación de \textit{slots}
				\4[] Cielo Único Europeo (2009): coordinación espacio aéreo
				\4[] Agencia Eurocontrol
				\4[] $\to$ Coordinar supervisores espacio aéreo
				\4[] Bloques de tráfico aéreo transfronterizos
				\4[] $\to$ Criterios operativos, no fronteras nacionales
				\4 Marítimo
				\4[] Transporte marítimo
				\4[] $\to$ Liberalizado
				\4[] Servicios portuarios
				\4[] $\to$ En proceso de liberalización
				\4[] Paquete Cinturón Azul
				\4[] $\to$ Reducción trámite aduanero para tráfico intra UE
				\4[] $\to$ Notificación electrónica de carga
			\3 Seguridad
				\4 Armonización de permisos de conducción
				\4 Normas europeas de aeronáutica
				\4[] Diseño y producción
				\4[] Inspección y mantenimiento
				\4[] Formación de pilotos
			\3 Acuerdos internacionales en sector aéreo
				\4 Negociación de acuerdos servicios aéreos
				\4[] Exclusividad de UE
				\4 Acuerdo con EEUU
				\4[] Vuelos punto a punto sin restricciones
			\3 Derechos de consumidores y pasajeros
				\4 Overbooking
				\4 Normas mínimas comunes de compensación
			\3 Reducción de emisiones
				\4 Fomento de intermodalidad
				\4[] Descongestionar modos más contaminantes
				\4 Coordinación con políticas medioambientales
				\4 Infraestructura para coches eléctricos
		\2 Valoración
			\3 Criterios generales de valoración
				\4 Relevancia
				\4 Efectividad y eficiencia
				\4 Contribución UE sobre EEMM
				\4 Coordinación con otras áreas de políticas UE
			\3 Carretera
				\4 Bastante liberalizado
				\4 Subsisten restricciones y oposición social
			\3 Ferrocarril
				\4 Redes necesitan reformas en algunos EEMM
				\4 Transporte de mercancías
				\4[] Relativamente poco desarrollado en comparación con USA
				\4 Geografía europea es adversa
			\3 Aéreo
				\4 Liberalización avanzada
				\4 Niveles muy elevados de seguridad
			\3 Marítimo
				\4 Liberalización y mejora notables
		\2 Retos
			\3 Energía y medioambiente
				\4 Transición a energía eléctrica
				\4 Reducir modos más contaminantes
			\3 Necesidades de capital
				\4 Reto permanente de infraestructuras
			\3 Identificación de proyectos prioritarios viables
				\4 Obstáculos políticos y financieros
				\4 Mejorar cooperación con sector privado
	\1 \marcar{Energía}
		\2 Justificación
			\3 Energía es input esencial
				\4 En toda producción
				\4 UE relativamente intensiva energéticamente
				\4 Precios eléctricos elevados en UE
				\4[] Mucho más altos que EEUU
			\3 Dependencia energética exterior
				\4 UE produce 50\% necesidades energéticas
			\3 Proveedores poco diversificados
				\4 RUS, NOR, KAZ, NIG, LIB, IRAQ
				\4[] $\to$ 70\% de energía total EU
				\4 6 EEMM depende de un proveedor
				\4 Inestabilidad política de proveedores
				\4[] Salvo NOR
				\4[] $\to$ Graves riesgos políticos y militares
			\3 Externalidades de consumo energético
				\4 Medio ambiente
				\4 Industrias de red
		\2 Objetivos
			\3 Garantizar seguridad energética
				\4 Suministro estable
			\3 Diversificar proveedores
				\4 Resistencia a shocks idiosincráticos al proveedor
			\3 Diversificar fuentes de energía
				\4 Resistencia a shocks en una fuente de energía
			\3 Conectar mercados energéticos nacionales
				\4 Mejor aprovechamiento de recursos energéticos
			\3 Mantener precios competitivos
				\4 Asegurar competitividad exportadora
			\3 Fomentar investigación y competitividad
				\4 Suministro futuro de energía
			\3 Promover uso de energías renovables
				\4 Reducir emisiones y dependencia exterior
				\4 Aprovechar recursos naturales de Europa
		\2 Antecedentes
			\3 Tratado de la CECA (1951)
				\4 Abolición de aranceles al carbón
				\4 Contexto de grandes necesidades energéticas
			\3 Euratom y CEE (1957)
				\4 Mercado europeo de energía nuclear
				\4[] Especial interés de Francia
				\4 Dudas sobre seguridad del suministro europeo
				\4[] Tras crisis del Canal de Suez
				\4 Programas comunes de investigación
				\4[] Energía nuclear
				\4[] Construcción de centrales
			\3 Crisis del petróleo de 1973 y 1979
				\4 Shock brusco sobre precio del petróleo
				\4 Creación de reservas estratégicas
				\4 Propuesto Mercado Común de la Energía
			\3 Acta Única Europea de 1987
				\4 Energía no explícitamente tratada
				\4 Comunicación de 1988 sobre energía
			\3 Tratado de Maastricht de 1991
				\4 RTE-E sobre energía
			\3 Tratado de Lisboa
				\4 Competencias compartidas de la UE
				\4 Abastecimiento energético es elemento central de UE
		\2 Marco jurídico
			\3 TFUE
				\4[] 4 sobre competencias compartidas
				\4[] 194 y ss.
			\3 Paquetes de directivas de 1996, 2003 y 2009
			\3 Estrategia Europa 2020
				\4 Objetivos 20-20-20
				\4[] 20\% de energías renovables
				\4[] 20\% de mejora de eficiencia energética
				\4[] 20\% de reducción de gases de invernadero
			\3 Paquete de energía limpia de 2019
				\4 Propuesta de la Comisión
				\4 Negociaciones concluidas en 2018
				\4 Prevista aprobación de 8 directivas y reglamentos
		\2 Marco financiero
			\3 Fondo de Cohesión
			\3 FEDER
			\3 Connecting Europe Facility\footnote{https://ec.europa.eu/inea/connecting-europe-facility/cef-energy}
				\4 5.350 M de € para energía
				\4 Mayor parte gestionada por INEA
			\3 EFSI/FEIE
			\3 Fondo Europeo de Eficiencia Energética
				\4 Apoyar objetivos 20/20/20
				\4 Iniciativa conjunta de:
				\4[] CE, EIB, Deutsche Bank y CDP italiano\footnote{Banco privado.}
			\3 Fondo de Innovación\footnote{https://ec.europa.eu/clima/policies/innovation-fund\_en}
				\4 Hasta 10.000 M de €
				\4 Financiado con ingresos del ETS
				\4 Primeros proyectos en 2020
				\4[] $\to$ Hasta 2030
				\4 Proyectos centrados en reducción de emisiones
		\2 Actuaciones
			\3 SINCRONIZAR CON TEMAS DE 4A SOBRE POLÍTICA ENERGÉTICA Y MAMBIENTAL
			\3 RTE-Energía\footnote{Ver \url{https://ec.europa.eu/energy/en/topics/infrastructure/trans-european-networks-energy}}
				\4 Concepto
				\4[] Estrategia para enlazar infraestructuras energéticas
				\4[] Nueve corredores prioritarios
				\4[] Tres áreas temáticas
				\4 Corredores prioritarios
				\4[] Eléctrico en mares del Norte
				\4[] Eléctrico norte-sur en Europa Occidental
				\4[] $\to$ Desde península ibérica  hasta norte
				\4[] Eléctrico norte-sur en Europa del Este
				\4[] Eléctrico en Mar Báltico
				\4[] Gas en Europa Occidental
				\4[] Gas en Europa del Este y Central
				\4[] Corredor del gas del Sur
				\4[] Gas en Mar Báltico
				\4[] $\to$ Especialmente relevante para repúblicas bálticas
				\4 Áreas temáticas prioritarias
				\4[] Smart grids
				\4[] $\to$ Integrar energía renovable
				\4[] $\to$ Mejorar gestión de picos/valles energéticos
				\4[] ``Autopistas eléctricas''
				\4[] $\to$ Aumentar conexiones generación y almacenamiento
				\4[] $\to$ Conexiones de larga distancia
				\4[] Red transfronteriza de intercambio de CO2
				\4 Interconexión eléctrica es objetivo principal
				\4[] Fundamental para electrificar transporte
				\4[] Lograr al menos 10\% de interconexión en 2020
				\4[] $\to$ Sobre capacidad instalada total
			\3 Seguridad energética
				\4 Mantenimiento de reservas
				\4[] Directiva de 2009 obliga a crear reservas
				\4[] Mantener niveles mínimos
				\4[] $\to$ 90 días de importaciones
				\4[] $\to$ 61 días de consumo
				\4 Proveedores
				\4[] Facilitar importaciones de proveedores distintos
				\4[] Mayor uso de Gas Natural Licuado
				\4[] Mejorar relaciones y conexiones
				\4[] $\to$ Asociación con Noruega
				\4[] $\to$ Nord Stream I y II de RUS a GER a través de Báltico
				\4 Informar a Comisión de acuerdos bilaterales
			\3 Mercado Interior
				\4 Reformas mercados nacionales
				\4[] Varios Paquetes de reformas
				\4[] $\to$ 1996, 1998, 2003, 2009
				\4[] Liberalización de acceso a redes de distribución
				\4[] Desintegración vertical
				\4[] $\to$ Generación
				\4[] $\to$ Transmisión
				\4[] $\to$ Distribución final
				\4[] Separación de propiedad de activos
				\4 Integración de nuevos EEMM
				\4[] Reducir dependencia Rusia
				\4[] Permitir paso de gas centroasiático
				\4 Interconexiones
				\4[] Aumentar seguridad de suministro
				\4[] Reducir interrupciones
				\4[] Mejor aprovechamiento de almacenamiento
				\4[] Eliminar islas energéticas
				\4[] $\to$ España una de ellas
				\4[] Objetivo mínimo de interconexión
				\4[] $\to$ > 10\% de capacidad instalada
				\4[] Reducir obstáculos a intercambios transfronterizos
			\3 Objetivos 2020 y 2030 en ámbito energético
				\4 Europa 2020
				\4[] $\to$ Fijados en 2010
				\4[] $\to$ Objetivos para 2020
				\4[] Reducción de emisiones en 20\% (respecto 1990)
				\4[] Aumento energías renovables hasta 20\%
				\4[] Eficiencia energética aumentada en 20\%
				\4 Objetivos para 2030
				\4[] Reducción de emisiones en 40\% (respecto (1990)
				\4[] Aumento energías renovables hasta 32\%
				\4[] Eficiencia energética aumentada en 32.5\%
			\3 Eficiencia energética
				\4 Inversión en nuevas tecnologías
				\4[] Reducir intensidad energética
				\4[] Redes inteligentes/Smart grids
				\4[] $\to$ Mejor control de picos de demanda/oferta
				\4 Transición a tecnologías con menos emisiones
				\4 Mejorar construcción de viviendas
				\4 Reducir dependencia de petróleo en transporte
				\4 Fondo Europeo de Eficiencia Energética
			\3 SET-Plan
				\4 Strategic Energy Technology
				\4[] Plan de largo plazo
				\4[] Innovación en transporte y producción
				\4[] Financiar investigación nuevas tec. de producción
				\4[] Coordinación de proyectos de investigación
			\3 Paquete de energía limpia de 2019\footnote{Ver \url{https://ec.europa.eu/energy/en/topics/energy-strategy-and-energy-union/clean-energy-all-europeans}}
				\4 Propuesta de la Comisión
				\4 Negociaciones concluidas en 2018
				\4 Prevista aprobación de 4 directivas y 4 reglamentos
				\4[1] Directiva sobre el Mercado Eléctrico
				\4[2] Reglamento sobre el Mercado Eléctrico
				\4[3] Reglamento sobre ACER\footnote{Agency for cooperation of Energy Regulators. Creada en el tercer paquete energético (2009). Sede en Liubliana.}
				\4[4] Reglamento sobre riesgos en energía eléctrica
				\4[5] Directiva de Energías Renovables
				\4[6] Directiva de Eficiencia Energética
				\4[7] Directiva de Eficiencia en Construcción
				\4[8] Reglamento de Gobernanza de la Unión Energética
			\3 Metales para baterías
				\4 Transición energética requiere baterías
				\4[] Especialmente, coches eléctricos
				\4[] Baterías requieren materias primas
				\4[] $\to$ Metales poco abundantes
				\4[] $\to$ Litio, praseodimio, estroncio...
				\4[] Europa relativamente poco abundante
				\4[] Elevado coste medioambiental de explotación
				\4[] $\to$ Problema principal
				\4[] $\then$ Trade off medioambiente-independencia
				\4 Elevada dependencia exterior de materias primas
				\4 Préstamos del BEI
				\4[] Desarrollo de fábricas de baterías
				\4 Programas de reutilización de residuos
				\4 Políticas de vecindad y diplomacia económica
				\4[] IP, ENI, IPA, IcSP...
		\2 Valoración
			\3 Dependencia exterior muy elevada
				\4 Problema persistente
				\4 Dificultades estructurales y geográficas
			\3 Energía nuclear
				\4 Muy elevados costes fijos
				\4 Percepción de riesgos distorsionada
				\4[] Desastres nucleares
		\2 Retos
			\3 Incertidumbre regulatoria e inversión
				\4 Reducir inseguridad jurídica
				\4 Horizontes de inversión más claros
			\3 Unión de la Energía
				\4 Regulación homogénea para UE
				\4 Estrategia marco para próximos años
				\4 Cumplimiento insuficiente
			\3 Autosuficiencia energética
				\4 Reducir peso de petróleo
				\4[] $\to$ 90\% importado
	\1 \marcar{Medio ambiente}\footnote{\url{https://www.europarl.europa.eu/factsheets/en/sheet/71/envir}onment-policy-general-principles-and-basic-framework}
		\2 Justificación
			\3 Impacto sobre salud humana
				\4 Emisiones de partículas
				\4 Contaminación del agua
				\4 Seguridad de energía nuclear
			\3 Externalidades
				\4 Especialmente problemático en MAmbiente
				\4 Difícil internalización de costes
		\2 Objetivos
			\3 Incentivar transición energética
				\4 Fuentes de energía con menos externalidades
			\3 Mantener competitividad
				\4 Evitar aumentos de costes excesivos
				\4 Desarrollar tecnologías competitivas
			\3 Actuación a nivel global
				\4 MAmbiente es bien público global
				\4 UE uno de los principales emisores
				\4 Ejercer presión en foros internacionales
		\2 Antecedentes
			\3 Consejo Europeo de París (1972)
				\4 Reconoce necesidad políticas a nivel europeo
			\3 Acta Única Europea (1987)
				\4 Introduce título sobre medio ambiente
			\3 ONU: Conferencia sobre el Cambio Climático
				\4 Primera en 1995
				\4 Protocolo de Kyoto en 1997
				\4 Conferencias sobre el Cambio Climático
				\4 Especialmente relevantes en opinión pública
				\4 Gobernanza global sobre acción climática
				\4 Objetivos de desarrollo sostenible
				\4 Madrid COP25
			\3 Tratado de Amsterdam (1997)
				\4 Obligación de integrar MA en políticas comunitarias
			\3 Proceso de Cardiff (1998)
				\4[] Integrar protección MA en políticas sectoriales
			\3 Agenda de Gotemburgo (2001)
				\4 Objetivos de sostenibilidad
				\4[] Complementan a Agenda 2000
				\4 Estrategia Europea de Desarrollo Sostenible (2005)
			\3 Tratado de Lisboa (2007)
				\4 Objetivo específico de la UE
		\2 Marco jurídico
			\3 Proceso de Cardiff (1998)
			\3 Acuerdo de París de 2015
				\4 195 miembros
				\4 Objetivos globales
				\4 Obligación de fijar objetivos nacionales
				\4 Sin mecanismos coercitivos
			\3 TUE 3.3
				\4 Protección de MA es objetivo específico
			\3 TFUE.11, 191 y 193
				\4 Obligación de incorporar MA en políticas
			\3 Estrategia Europa 2020
				\4 Objetivos 20-20-20
				\4[] 20\% de energías renovables
				\4[] 20\% de mejora de eficiencia energética
				\4[] 20\% de reducción de gases de invernadero
			\3 Comunicación de CE (2018): visión estratégica de l/p
			\3 Directivas sectoriales
				\4 Directiva de Responsabilidad Medioambiental
				\4 Directiva del Agua
				\4 Directiva del Sistema de Emisiones de Carbono de 2018
				\4 Directiva de EIA\footnote{Environmental Impact Assessment} de 1985, 2011, 2014
			\3 EAP\footnote{Environment Action Programme} VII -- Programa de Acción Medioambiental
			\3 Paquete de Energía Limpia de 2019
		\2 Marco financiero
			\3 Programa LIFE
				\4 Dedicación exclusiva a medio ambiente
				\4 Integrado en MFP 2014-2020
			\3 Fondo de Cohesión
			\3 Fondo de Innovación
			\3 Fondo Europeo de Eficiencia Energética
			\3 EFSI/FEIE
			\3 FEADER/FEMP
		\2 Actuaciones
			\3 Principios generales -- PPAC
				\4 \marcar{P}recaución
				\4[] Si se duda sobre carácter nocivo
				\4[] $\to$ Evitar liberación de contaminante
				\4[] $\to$ No permitir distribución
				\4 \marcar{P}revención
				\4[] Evitar potenciales daños de manera proactiva
				\4 \marcar{A}ctuación en origen
				\4[] Evitar contaminación desde la fuente
				\4[] $\to$ Más costoso evitar después
				\4 \marcar{C}ontaminador paga
			\3 Agencia Europea de Medio Ambiente
				\4 Creada en 1993
				\4 33 miembros
				\4 Programas de información y observación
				\4 Coordinación con agencias fuera-UE
			\3 Objetivos 2020, 2030, 2050
				\4 2020:
				\4[] 20\% energías renovables
				\4[] 20\% de reducción de emisiones respecto a 1990
				\4[] 20\% de eficiencia energética aumentada
				\4 2030
				\4[] 32,5\% de eficiencia energética aumentada
				\4[] 40\% de reducción de emisiones respecto a 1990
				\4[] 32,5\% de eficiencia energética aumentada
				\4 2050:
				\4[] Neutralidad climática
			\3 VII Programa de Acción Medioambiental
				\4 Asegurar implementación completa de directivas
				\4 Coordinar políticas europeas
				\4[] PAC, PPC, Energía, Transporte...
				\4[] Principios transversales
				\4[] Evaluaciones de Impacto Medioambiental
				\4 Innovación y avance científico
				\4[] Financiar investigación
				\4[] Facilitar intercambio tecnológico
				\4[] Sistema de Investigación Medioambiental
				\4 Implementar ETS -- Emissions Trading System
			\3 VIII Programa de Acción Medioambiental\footnote{Ver \href{https://www.consilium.europa.eu/es/press/press-releases/2019/10/04/8th-environmental-action-programme-council-adopts-conclusions/}{Comunicado de octubre de 2019 del Consejo.}}
				\4 En proceso de diseño
				\4[] A junio de 2020
				\4 Para periodo 2021-2027
				\4 Presentada Agenda Estratégica
				\4 Dotaciones presupuestarias trasversales
				\4[] Necesario dotar \% en FEDER, FCohesión, FEADER
				\4 Objetivos concretos:
				\4[] Generación de energías renovables
				\4[] Inversión nuevas tecnologías energéticas
				\4[] $\to$ ITER
				\4[] $\to$ Hidrógeno
				\4[] $\to$ Solar
				\4[] Gestión de redes eléctricas
				\4[] Gestión de residuos generados reutilizables
			\3 ETS -- Emissions Trading Scheme
				\4 Idea clave
				\4[] Contexto
				\4[] $\to$ Creado en 2005
				\4[] $\to$ Teorema de Coase
				\4[] $\to$ Internalización externalidad negativa
				\4[] $\to$ Sistema de precios como herramienta de transmisión de información
				\4[] Objetivo
				\4[] $\to$ Emisores internalicen costes externos
				\4[] $\to$ Reducir distorsiones de mercado
				\4[] $\to$ Incentivar energías menos contaminantes
				\4[] Resultado
				\4[] $\to$ Progresivos avances
				\4[] $\to$ Mercados de emisiones institucionalizados
				\4 Formulación
				\4[] Asignación de derechos de emisión
				\4[] $\to$ Medidos en equivalentes de CO2
				\4[] $\to$ Sector industrial recibe derechos gratuitos
				\4[] $\to$ Asignación equitativa en toda UE
				\4[] $\to$ Más asignación a expuestos a fuga de carbono\footnote{Deslocalización de industrias contaminantes hacia jurisdicciones fuera de la UE con protección más laxa.}
				\4[] $\to$ Más asignación a plantas más eficientes
				\4[] $\to$ Más asignaciones a cumplidores de Kioto
				\4[] $\to$ Reducción anual de derechos asignados
				\4[] Certificación de emisiones
				\4[] $\to$ Verificador acreditado
				\4[] Compraventa de derechos
				\4[] $\to$ Vende derechos si emitió menos
				\4[] $\to$ Debe comprar derechos si emitió más
				\4[] $\to$ Derechos no reutilizables
				\4[] $\to$ Multas por incumplimiento superiores a precio de derechos
				\4 Implicaciones
				\4[] Recesiones globales hacen caer precio
				\4[] $\to$ Puede aumentar emisiones
				\4[] Necesario compensar exceso de derechos
				\4[] $\to$ Market Stability Reserve
				\4[] Market Stability Reserve/Reserva de Estabilidad de Mercado
				\4[] $\to$ Intervención one-off
				\4[] $\to$ Reducción derechos si circulación excede límite
				\4[] $\to$ Aumento de derechos si circulación menor que límite
				\4[] $\to$ Controlar déficits/superávits estructurales
				\4 Valoración
				\4[] Liquidez escasa en ocasiones
				\4[] Precios demasiado bajos durante años
				\4[] Revisiones periódicas
				\4[] Fase 3: 2013
				\4[] Fase 4: 2021-2030
				\4[] Se espera aumento de precios
				\4[] $\to$ Máximo de 10 años en verano 2018
				\4[] Precio de futuros Junio 2020
				\4[] $\to$ En torno a 24 €
			\3 Taxonomía de inversiones sostenibles\footnote{Ver \href{http://www.revistasice.com/index.php/BICE/article/view/7077/7076}{ICE (2020) Reglamento de Taxonomía de la UE sobre Actividades Económicas Sostenibles.}}
				\4 Finanzas sostenibles
				\4[] Necesario financiar
				\4[] $\to$ Transformación energética
				\4[] $\to$ Descarbonización
				\4 Necesario implicar a sector financiero
				\4[] Canalizar ahorro hacia nuevas actividades
				\4[] $\to$ No sólo ahorro público
				\4[] $\then$ Ahorro privado papel esencial
				\4 Reglamento de Taxonomía aprobado en 2018
				\4 Metodología común para determinar sostenibildad
				\4 Actividad financiada debe:
				\4[] $\to$ Contribuir al menos a uno de siguientes objetivos
				\4[] $\to$ No causar perjuicio en ninguno de objetivos
				\4[] 1. Mitigación del CClimático
				\4[] 2. Adaptación al CClimático
				\4[] 3. Uso sostenible de agua y mares
				\4[] 4. Economía circular
				\4[] 5. Prevención de contaminación
				\4[] 6. Protección de diversidad y ecosistemas
				\4 Necesarios informes detallados periódicos
			\3 Planes nacionales de energía y clima para 2021-2030\footnote{Ver \href{https://ec.europa.eu/info/energy-climate-change-environment/overall-targets/national-energy-and-climate-plans-necps_en}{Comisión Europea sobre planes nacionales 2021-2030}}
				\4 Introducidos en 2018 en Paquete de Energía Limpia
				\4 Programas nacionales deben incluir medidas para 2021-2030
				\4[] Eficiencia energética
				\4[] Reducción de emisiones de gases ef. invernadero
				\4[] Renovables
				\4[] Interconexión energética
				\4[] Innovación e investigación
			\3 Estrategias nacionales de largo plazo\footnote{Ver \href{https://ec.europa.eu/info/energy-climate-change-environment/overall-targets/long-term-strategies_en}{Comisión Europea sobre estrategias nacionales de largo plazo para cumplir Acuerdo de París (2015).}}
				\4 Debe cubrir periodo hasta 30 años desde 2020
				\4 Deben prever planes para:
				\4[] Reducción total de emisiones
				\4[] Sumideros de carbono
				\4[] Planes sectoriales de reducción
				\4[] Progreso esperado en reducción de emisiones
				\4[] Efectos socio-económicos de los planes de reducción
				\4[] Conexiones con otros objetivos nacionales de largo plazo
				\4 Cada 10 años nueva estrategia
				\4 Puede renovarse a los 5 años si necesario
			\3 Evaluaciones medioambientales
				\4 Requeridas por varias directivas
				\4 Evaluación Medioambiental Estratégica
				\4[] Procedimiento de decisión sobre MA
				\4[] Marco para Evaluaciones de Impacto Medioambiental
				\4 Evaluación de Impacto Medioambiental
			\3 Informe SOER 2020\footnote{Ver \href{https://www.eea.europa.eu/soer/2020}{EEA (2020) sobre informe SOERF}.}
				\4 Publicado por Agencia Europea del Medioambiente
				\4 European Environment -- State and Outlook 2020
				\4 Evaluación ambiental completa a nivel Europeo
				\4 Cada 5 años nuevo informe
				\4 Áreas con graves problemas
				\4[] Protección de diversidad animal
				\4[] Ecosistemas marinos
				\4[] Vertidos de sustancias químicas
				\4[] Extracción de aguas subterráneas
				\4[] Ruidos
				\4[] Contaminación del agua
				\4 En general, desfavorable en relación PMA VII
		\2 Valoración
			\3 Problemas del ETS
				\4 Precios demasiado bajos durante años
			\3 Objetivos de reducción de emisiones
				\4 Incumplidos en algunos EEMM
				\4 Posible cumplimiento global
				\4 Bienes importados también implican emisiones
				\4[] No tenidos en cuenta
				\4[] $\to$ Problema principal
		\2 Retos
			\3 Fuga de carbono
				\4 Deslocalización a países menos regulados
				\4 Bienes importados también implican emisiones
			\3 Política comercial
				\4 Fuertes interacciones con política medioambiental
				\4 Competencia determina efecto de regulación medioambiental
				\4[] Barrett (1994)
				\4[] competencia à la Cournot
				\4[] $\to$ Incentivos a reducir protección
				\4[] Competencia à la Bertrand imperfecta
				\4[] $\to$ Incentivos a aumentar protección
			\3 Exceso de derechos emitidos
				\4 Precios demasiado bajos para desincentivar emisiones
			\3 Desarrollo y medio ambiente
				\4 EEMM con menos renta competitivos en industrias contaminantes
			\3 Cumplimiento de objetivos
				\4 Aparentemente, buenos resultados
				\4 Algunos sectores muy deficitarios
				\4[] Residencial
				\4[] Agricultura
				\4[] Transporte
				\4[] $\then$ Resultados peores que aparentemente
			\3 Reforma del sistema de emisiones 2021-2030
				\4 Objetivo: reducción del 43\%
				\4 Necesario aumentar ritmo de reducción
				\4 ETS mecanismo fundamental
				\4 Dos nuevos fondos
				\4[] Fondo de innovación
				\4[] Fondo de modernización
	\1 \marcar{Tecnología}
		\2 Justificación
			\3 Economías de escala en investigación
				\4 Recogida y acceso a datos
				\4 Transferencia de conocimiento
				\4 Coordinación de investigación
			\3 Externalidades positivas de la investigación
				\4 Transferencia investigación-sector privado
				\4 Oportunidades de desarrollo en sector industrial
				\4 Aumento de competitividad exterior
			\3 Investigación básica
				\4 Especial importancia de actuación del estado
				\4 Investigaciones a menudo no directamente rentables
				\4 Estado puede ejercer papel catalizador
				\4 Costes fijos elevados
		\2 Objetivos
			\3 Catalizar investigación básica
				\4 Evitar cuellos de botella de investigación
			\3 Promover transferencia a sector privado industrial
			\3 Incentivar investigaciones estratégicas
				\4 Energías renovables
				\4 Salud
				\4 Telecomunicaciones
				\4 Agricultura
			\3 Comunicación de la CE 2017
				\4 \url{https://eur-lex.europa.eu/resource.html?uri=cellar:c8b9aac5-9861-11e7-b92d-01aa75ed71a1.0001.02/DOC_1&format=PDF}
		\2 Antecedentes
			\3 Agenda de Lisboa (2000)
				\4 Unión por la Innovación
				\4 Objetivo:
				\4[] UE economía mundial más competitiva en 2010
				\4 Objetivo de inversión mínima en I+D
				\4 Relativo fracaso
				\4[] Estallido crisis
				\4[] Pérdida de competitividad USA, emergentes
				\4[] No vinculante
			\3 Estrategia 2020 (2010)
				\4 Marco de actuación en esta década
				\4 Continuación de Agenda 2020
				\4 Esfuerzo sobre seguimiento de políticas
				\4[] Scoreboard de Eurostat
			\3 Comunicación de la Comisión de 2017
		\2 Marco jurídico
			\3 TFUE 4
				\4 I+D y espacio es competencia compartida
			\3 TFUE.173
			\3 Estrategia Europa 2020
				\4 Inversión del 3\% en i+D para 2020
			\3 Estrategia de Política Industrial 2017
		\2 Marco financiero
			\3 EFSI/FEIE
				\4 Fondo Europeo para Inversiones Estratégicas
			\3 FEIE -- Fondos Estructurales y de Inversión Europea
				\4 FEDER
				\4 Fondo de Cohesión
				\4 Fondo Social Europeo
				\4 FEADER
				\4 FEMP
			\3 Horizonte 2020\footnote{\url{https://ec.europa.eu/programmes/horizon2020/sites/horizon2020/files/H2020_ES_KI0213413ESN.pdf}.}
				\4 80.000 millones de € en MFP 2014-2020
		\2 Actuaciones
			\3 Financiación de I+D e industria
			\3 Horizonte 2020
				\4[] Financiar proyectos de investigación
				\4[] $\to$ Básica
				\4[] $\to$ Aplicada
			\3 ERA -- European Research Area
				\4 Europa como área de investigación integrada
				\4 Permitir libre circulación de
				\4[] Investigadores
				\4[] Conocimiento científico
				\4[] Tecnología
				\4 Prioridades
				\4[] Aumentar efectividad de sistemas nacionales
				\4[] Óptima cooperación transacional y
				\4[] Aumentar grado de competencia de sistemas de investigación
				\4[] Libre mercado de trabajo de investigadores
				\4[] Igualdad de género en investigación
				\4[] Circulación libre de investigación
				\4[] Cooperación internacional
			\3 ERA especial para Covid-19
				\4 Esfuerzo paneuropeo
				\4 Impacto potencial enorme sobre economía
				\4[] Menor gasto en salud
				\4[] Movilidad de trabajo
				\4[] Evitar confinamientos
				\4 Movilización de fondos Horizonte 2020
			\3 Programa de respuesta global frente a COVID-19
			\3 Agencia Espacial Europea
			\3 Programa Galileo
			\3 EIP -- Acuerdos Europeos de Innovación
				\4 Coordinación de programas de investigación
				\4[] EEMM, UE, regiones
				\4[] Comisión Europea coordina
				\4 Salud y envejecimiento
				\4 Agricultura
				\4 Agua
				\4 Materias primas
				\4 Ciudades inteligentes
		\2 Valoración
			\3 Problemas de coherencia
				\4 Múltiples programas y solapamiento
			\3 Complejidad de programas
				\4 Dificultan aprovechamiento
				\4 Difícil identificar proyectos viables
			\3 Éxitos
				\4 Sectores específicos
				\4[] Aeronáutica y aeroespacial
				\4[] Defensa
				\4[] Biotecnología
				\4 Sectores con problemas de competitividad
				\4[] Tecnologías digitales
				\4[] Inteligencia artificial
				\4[] ...
		\2 Retos
			\3 Reducir influencia de grupos de interés
				\4 Captura de fondos en proyectos no rentables
				\4 Política industrial anticompetitiva
			\3 Excesiva complejidad
				\4 UE depende de EEMM para ejecución en muchos casos
				\4 Necesario control de gasto
				\4[$\then$] Aumento de complejidad
			\3 Intereses de EEMM vs UE
				\4 Solapados o contrapuestos
				\4 Competencia entre EEMM
				\4 EEMM contribuyentes netos a presupuesto
				\4[] Tratan de recuperar por vía de fondos i+d e industria
			\3 Dificultades de implementación
				\4 ¿Cómo identificar proyectos viables?
				\4 ¿Cómo evitar sustitución de financiación privada?
	\1 \marcar{Política industrial}
		\2 Justificación
			\3 Economías de escala industriales
				\4 Industrias con grandes costes fijos
				\4[] Aeronáutica
				\4[] Biotecnología
				\4[] Inteligencia artificial
			\3 Peso de industria en PIB
				\4 Industria es intensiva en K físico y humano
				\4 Productividad superior a otros sectores
				\4 Tendencia a reducir \% en PIB
			\3 Pérdida de competitividad relativa
				\4 Estados Unidos y Japón
				\4 Emergentes
		\2 Objetivos
			\3 Acelerar adaptación a cambios estructurales
				\4 Globalización
				\4 Tecnologías disruptivas
				\4 Competencia con emergentes
				\4 Nuevos modelos de negocio
			\3 Entorno favorable
				\4 Creación de empresas
				\4 Crecimiento de empresas
				\4 Cooperación entre empresas
				\4 Énfasis en PYMES
			\3 Aumentar peso de industria en PIB
				\4 Objetivo Europa 2020
				\4[] 20\% en 2020
			\3 Explotar ventajas competitivas europeas
				\4 Aumentar cuota de mercado en exportación
				\4 Aumentar valor añadido de exportaciones
				\4 Desarrollar clusters industriales
		\2 Antecedentes
			\3 50s a 70s
				\4 Campeones nacionales
				\4 Intervención vertical
			\3 80s
				\4 Problemas de competitividad
				\4 Adhesión de nuevos miembros
				\4 Reconversión industrias pesadas
			\3 Tratado de Maastricht (1992)
				\4 Incluye expresamente política industrial
				\4 Énfasis en I+D
			\3 Informe Bangemann (1994)
				\4 Cambio a enfoque horizontal
				\4 Rechaza prácticas tradicionales
				\4[] Promoción de sectores concretos
				\4[] Campeones nacionales
		\2 Marco jurídico
			\3 TFUE 6
				\4 Industria es competencia de apoyo
				\4[] UE puede complementar acción nacional
				\4 Tecnología
		\2 Marco financiero
			\3 EFSI/FEIE
				\4 Fondo Europeo para Inversiones Estratégicas
			\3 FEIE -- Fondos Estructurales y de Inversión Europea
				\4 FEDER
				\4 Fondo de Cohesión
				\4 Fondo Social Europeo
				\4 FEADER
				\4 FEMP
			\3 Programa COSME
				\4 Competitivenes of Entreprises and SMEntreprises
				\4 2300 millones de euros
		\2 Actuaciones
			\3 Programa COSME
				\4[] Financiación a PYMES
				\4[] Deuda
				\4[] Equity
				\4[] Garantías financieras
				\4[] $\to$ Canalizado a través de inst. locales
				\4[] Educación empresarial
				\4[] $\to$ Programas educativos
				\4 Reducir carga administrativa
				\4 Complemento a políticas nacionales
				\4 COSME Data Hub
				\4[] Database de proyectos financiados
			\3 Estrategia de Política Industrial 2017
				\4 Comunicación de la Comisión
				\4 Día de la Industria
				\4 Programa de Apoyo a Reformas Estructurales
				\4 Reciclaje de materiales
				\4 Estándares de emisiones mejorados
				\4 Modernizar marco de protección de IP
				\4 Extender Skills Agenda a nuevos sectores
			\3 Skills Agenda de 2016 -- Comunicación de la Comisión
				\4 Actuaciones en 10 sectores
				\4[] Aumentar capital humano
				\4[] Mejorar matching empresas--trabajadores
				\4 Formación de adultos
				\4[] Coordinado con EEMM
				\4 Marco de cualificaciones europeo
				\4 Formación digital
				\4 Sectores deficitarios en personal
				\4 Identificación de habilidades en inmigrantes
				\4 Formación profesional
				\4 Competencias clave en sistema educativo
				\4 Europass
				\4 Optimizar flujos de capital humano
				\4 Optimizar formación de posgrado
				\4[] Analizar rendimiento a nivel de grado
				\4[] Optimizar decisiones respecto posgrado
				\4[] $\to$ Educación
				\4[] $\to$ Experiencia
			\3 Unión por la Innovación
				\4 Programa hasta 2016
				\4 Promover colaboración público-privada
				\4[] A nivel transnacional
				\4 Reducir obstáculos a innovación
				\4[] Registro de patentes
				\4[] Fragmentación de mercado
				\4[] Fijación de estándares muy lenta
				\4[] Cuellos de botella en formación
			\3 Política comercial y rel. econ. exteriores
				\4 Defensa comercial en sectores específicos
				\4[] Acero
				\4[] Aerospacial
				\4[] Defensa
				\4[] Automóvil
				\4 Relaciones económicas exteriores
				\4[] Garantizar acceso a materias primas
			\3 Competencia
				\4 Fomentar competencia industrial
				\4[] Evitar concentraciones y abusos de posición dominante
		\2 Valoración
		\2 Retos
	\1 \marcar{Sociedad de la información}
		\2 Justificación
			\3 Economías de escala potenciales
				\4 Diseño y programación de software
				\4 Redes de telecomunicación
			\3 Mejoras potenciales de productividad
				\4 Costes de transmisión de información
				\4 Costes de procesamiento
				\4 Mercados digitales
			\3 Acceso aún incompleto a internet
				\4 Cercano al 90\%
			\3 Seguridad
				\4 Ciberterrorismo
				\4 Espionaje
				\4 Protección propiedad intelectual
			\3 Economías de red
				\4 Especialmente relevantes en estándares
		\2 Objetivos
			\3 Acceso universal a internet
				\4 Regiones rurales y periféricas especialmente
			\3 Digitalización de empresas
				\4 Presencia en mercados digitales
				\4 Relaciones con admón. vía redes
			\3 Mercado único digital
				\4 Eliminación de barreras al comercio por medios digitales
			\3 Aprovechamiento de economías de escala
		\2 Antecedentes
			\3 Estrategia DSM
				\4 2015
				\4 Parte de Estrategia 2020
		\2 Marco jurídico
			\3 RTE-Te
				\4 Redes trans-europeas de telecomunicaciones
			\3 Comunicación sobre Agenda Digital
			\3 Comunicación sobre política industrial de 2017
			\3 GDPR -- General Data Protection Regulation
			\3 Directivas del Mercado Único Digital
			\3 Directiva de Copyright aprobada por PE en 2019
				\4 Pendiente de aprobación en CdUE y transposición
			\3 Reglamento de libre circulación de datos no personales 2019\footnote{\url{https://eur-lex.europa.eu/legal-content/EN/TXT/?uri=CELEX:32018R1807}.}
			\3 Regulación sobre 5G
		\2 Marco financiero
			\3 FEDER
			\3 Fondo de Cohesión
			\3 Horizonte 2020
				\4 80.000 M de € en MFP 2014-2020
			\3 Programa COSME
			\3 EFSI
			\3 Connecting Europe Facility -- Telecomunicaciones
				\4 1000 M de € para 14-20
		\2 Actuaciones
			\3 Ciberseguridad
				\4 Centro Europeo de Ciberseguridad
				\4[] Desarrollar competencias
				\4 Directiva sobre Seguridad de Redes y Sistemas de Información
				\4 Certificaciones de Ciberseguridad
			\3 Mercado Único Digital
				\4 Prohibición de Geoblocking
				\4[] Discriminación basada en geolocalización IP
				\4 Tarifas de itinerancia
				\4[] Sin costes adicionales
				\4 Modernización de contratos comercio electrónico
				\4 Codificación derechos del consumidor
				\4 Regulación IVA en comercio electrónico
			\3 Datos
				\4 Directiva de protección de datos
				\4[] GDPR -- General Data Protection Regulation
				\4[] Entrada en vigor en 2018
				\4[] Misma regulación protección de datos para toda UE
				\4 Libre circulación de datos no personales
				\4 Reglamento 2019 de circulación de datos no personales
				\4[] Creación de espacio europeo común de datos
				\4[] Prohibición de obligación
				\4[] $\to$ De mantener datos en un EEMM
				\4[] Restringido a datos no personales
			\3 Otras actuaciones
				\4 Centros de Innovación Digital
				\4[] Ventanillas únicas para digitalización
				\4 Garantías a proyectos de inversión
				\4 Observatorio Europeo de Blockchain
				\4 Estrategia Europea de Digitalización
				\4 Plataformas de Especialización Europea
				\4[] Digitalización en Turismo
				\4[] Ciberseguridad
				\4[] Inteligencia Artificial
				\4 Área SEPA
			\3 Plan de Acción 5G Europeo\footnote{Ver \href{https://ec.europa.eu/digital-single-market/en/research-standards}{Comisión Europea: 5G Research and Standards}.}
				\4 Programa de inverstigación público-privado
				\4[] Definir estándares
				\4[] Identificar espectro disponible
				\4[] Visión general de 5G Europeo
				\4 Horizonte 2020
				\4[] Financiación de programas de investigación
				\4 Demostraciones de uso sectoriales
		\2 Valoración
			\3 Éxitos
				\4 Roaming europeo
				\4 Coordinación en uso de frecuencias de móviles
				\4 Asignación frecuencias 5G
				\4 Sistema de pagos europeo
				\4 Administración electrónica
			\3 Controversias
				\4 Regulación General de Protección de Datos
				\4[] Cargas administrativas adicionales
				\4[] Costes de adaptación
				\4[] Dificulta negocios basados en datos
				\4[] Pérdida de competitividad con China y EEUU
				\4 Directiva sobre Ciberseguridad
				\4[] Insuficiente
				\4 Excesivo énfasis en regulación
				\4[] Insuficiente liberalización
		\2 Retos
			\3 Completar mercado único digital
				\4 Persisten barreras nacionales
			\3 Digitalización de empresas
				\4 Relativamente débil
				\4[] En comparación con USA, Japón, China
			\3 Liberalización frente a regulación
				\4 En ocasiones exceso de carga administrativa
			\3 Ciberseguridad
			\3 Inteligencia artificial
	\1[] \marcar{Conclusión}
		\2 Recapitulación
			\3 Transportes
			\3 Energía
			\3 Medio ambiente
			\3 Industria y tecnología
			\3 Sociedad de la información
		\2 Idea final
			\3 Liberalización y papel de la UE
				\4 Objetivo general
				\4 En ocasiones, acciones contraproducentes
				\4[] Aumento de regulación y cargas administrativas
			\3 Sensibilidad de políticas a nivel nacional
				\4 Fuerte oposición a reformas en algunos sectores
				\4[] Especialmente industrial
				\4[] Intereses colectivos bien organizados
			\3 Papel de EEMM
				\4 Acción subsidiaria a UE
				\4 Colaboración necesaria
				\4 Difícil alcanzar acuerdos en algunas áreas
				\4[] Impuesto digital
				\4[] Industrias nacionales significativas
				\4[] Dependencia energética
			\3 Comparación con China o EEUU
				\4 Enfoque mucho más descentralizado
				\4 Inconvenientes y ventajas
				\4 Margen para avanzar en coherencia de políticas
\end{esquemal}



































\preguntas

\seccion{Test 2006}

\textbf{42.} Diga cuál de las siguientes políticas no estaba incluida en el texto del Tratado de Roma en 1957:

\begin{itemize}
	\item[a] La política de transportes
	\item[b] La política comercial
	\item[c] La política medioambiental
	\item[d] La política social
\end{itemize}

\seccion{Test 2004}
\textbf{45.} ¿Cuál de las siguientes características refleja mejor la situación de la política de Investigación y Desarrollo Tecnológico (IDT) de la UE?

\begin{itemize}
	\item[a] Supone un marco de apoyo a las políticas de investigación y desarrollo de los Estados miembros. El III Programa Marco, aplicable en el periodo 2000-2006, recoge las dotaciones financieras acordadas para las políticas de la UE en materia de IDT.
	\item[b] El actual Programa Marco de IDT fue aprobado por el Consejo en 2002 y regirá hasta 2006. Contiene las dotaciones financieras aplicables en los ámbitos de la CE y la CEEA (o Euratom).
	\item[c] La Agenda 2000 ha incluido la política común de IDT entre sus principales prioridades en materia de cohesión y, en consecuencia, ha duplicado las dotaciones asignadas por el presupuesto de la UE a su nueva política de IDT.
	\item[d] El objetivo que más recursos recibe en la política europea de IDT es la creación de un Espacio Europeo de Investigación.
\end{itemize}

\notas

\textbf{2006:} \textbf{42.} C

\textbf{2004:} \textbf{45.} B

\bibliografia

Mirar en Palgrave:
\begin{itemize}
	\item digital marketplaces
	\item energy economics
	\item economic development and the environment
	\item environmental economics
	\item European Union (EU) Research and Experimental Development (RandD) Policy
	\item Research and experimental development (R\&D) and technological innovation policy
	\item Research Joint Ventures
	\item trade and environmental regulations
	
	\item transport
\end{itemize}

European Parliament. \textit{Trans-European Networks -- Guidelines} (2019) \url{http://www.europarl.europa.eu/ftu/pdf/en/FTU_3.5.1.pdf} -- En carpeta del tema

European Parliament. \textit{Energy Policy: general principles} (2018) \url{http://www.europarl.europa.eu/factsheets/en/sheet/68/energy-policy-general-principles} -- En carpeta del tema

European Parliament. \textit{Environment policy: general principles and basic framework} (2018) \url{http://www.europarl.europa.eu/factsheets/en/sheet/71/environment-policy-general-principles-and-basic-framework} -- En carpeta del tema

Linklaters (2019) \textit{Clean Energy Package adopted} \url{https://lpscdn.linklaters.com/-/media/files/insights/2019/may/clean-energy-package-pdf.ashx?rev=adf1c7aa-ab7c-4a18-9f8d-5272b58dd370&hash=9C5139C1C389735C699935FCA78A5B85} -- En carpeta del tema

\end{document}
