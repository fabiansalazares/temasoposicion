\documentclass{nuevotema}

\tema{3B-33}
\titulo{Evolución y estructura sectorial y geográfica de los flujos comerciales y financieros internacionales. Los bloques comerciales y las nuevas áreas emergentes en el comercio internacional.}

\begin{document}

\ideaclave

Ver \href{https://www.wto.org/english/news_e/pres20_e/pr855_e.htm}{Previsiones y escenarios de OMC de abril (2020)}


Ver \href{https://voxeu.org/article/how-services-boost-goods-exports}{VOXEU (2020)} sobre exportación conjunta de servicios y bienes, y cómo la liberalización de servicios aumenta las exportaciones de bienes y viceversa. Se trata de un fenómeno global y puede servir de complemento en el segundo apartado.

Ver ESR 2019 de FMI en carpeta FMI de Coyuntura

Ver WTO (2019) Annual Report en carpeta del tema

\seccion{Preguntas clave}

\begin{itemize}
	\item ¿Cómo han evolucionado los flujos comerciales y financieros en la historia reciente?
	\item ¿Qué características sectoriales tienen los flujos comerciales y financieros en la actualidad?
	\item ¿Qué estructura geográfica caracteriza a los flujos comerciales y de inversión?
	\item ¿Qué bloques comerciales existen en la actualidad? 
	\item ¿Qué evolución puede esperarse de las áreas comerciales emergentes?
	\item ¿Qué causas y qué efectos tiene la guerra comercial actual?
\end{itemize}

\esquemacorto

\begin{esquema}[enumerate]
	\1[] \marcar{Introducción}
		\2 Contextualización
			\3 Evolución del comercio internacional
			\3 Globalización
			\3 Concepto de integración económica
			\3 Guerra comercial
		\2 Objeto
			\3 ¿Cómo han evolucionado los flujos comerciales y financieros en la historia reciente?
			\3 ¿Qué características sectoriales tienen los flujos comerciales y financieros en la actualidad?
			\3 ¿Qué estructura geográfica caracteriza a los flujos comerciales y de inversión?
			\3 ¿Qué bloques comerciales existen en la actualidad?
			\3 ¿Qué evolución puede esperarse de las áreas comerciales emergentes?
			\3 ¿Qué causas y qué efectos tiene la guerra comercial actual?
		\2 Estructura
			\3 Flujos comerciales y financieros
			\3 Bloques comerciales y áreas emergentes
	\1 \marcar{Evolución histórica}
		\2 Hasta la I Guerra Mundial
			\3 Contexto
			\3 Eventos
			\3 Consecuencias
		\2 Entreguerras
			\3 Contexto
			\3 Eventos
			\3 Consecuencias
		\2 Desde la II Guerra Mundial hasta la crisis del petróleo
			\3 Contexto
			\3 Eventos
			\3 Consecuencias
		\2 Desde la Crisis del petróleo hasta principios del S. XXI
			\3 Contexto
			\3 Eventos
			\3 Consecuencias
		\2 Desde principios del S. XXI hasta la Crisis Financiera
			\3 Contexto
			\3 Eventos
			\3 Consecuencias
		\2 Desde la Crisis Financiera Global hasta la actualidad
			\3 Contexto
			\3 Eventos
			\3 Consecuencias
		\2 COVID-19
			\3 Contexto
			\3 Eventos
			\3 Consecuencias
	\1 \marcar{Flujos de bienes y servicios}
		\2 Idea clave
			\3 Contexto
			\3 Objetivo
			\3 Resultados
		\2 Hechos estilizados
			\3[i] Crecimiento comercio > PIB hasta GCFinanciera
			\3[ii] Integración creciente de PEDs
			\3[iii] Comercio intraindustrial N-N mayor a N-S
			\3[iv] Ascenso de China heterogéneo
			\3[v] Servicios cada vez más importantes
			\3[vi] Impacto heterogéneo de covid
		\2 Mercancías
			\3 Idea clave
			\3 Estructura sectorial
			\3 Estructura geográfica
		\2 Servicios
			\3 Idea clave
			\3 Estructura sectorial
			\3 Estructura geográfica
		\2 Comercio intra-industrial
			\3 Idea clave
			\3 Hechos estilizados
		\2 Comercio intra-empresa
			\3 Idea clave
			\3 Hechos estilizados
		\2 CVGs -- Cadenas de valor global
			\3 Idea clave
			\3 Hechos estilizados
	\1 \marcar{Flujos financieros}
		\2 Idea clave
			\3 Contexto
			\3 Objetivo
			\3 Resultados
		\2 Hechos estilizados
			\3[i] Asimetría en últimas décadas
			\3[ii] Volumen relacionado con desarrollo financiero
			\3[iii] Tendencia creciente del \% a PEDs
			\3[iv] IDE menos volátiles que otros flujos de K
			\3[v] Flujos de ICartera a PEDs fuertemente concentrados
			\3[vi] Flujos a PMAs muy pequeños y sobre todo remesas
			\3[vii] Reducción de PIINs globales respecto  a pre-CFG
		\2 IDE
			\3 Idea clave
			\3 Tipos de IDE
			\3 Hechos estilizados
			\3 Estructura geográfica
			\3 Estructura sectorial
		\2 Cartera
			\3 Idea clave
			\3 Estructura geográfica
			\3 Estructura sectorial
		\2 Políticas implementadas
			\3 Idea clave
			\3 Liberalización vs restricción
			\3 IIAs -- Acuerdos de inversión internacional
			\3 Zonas Económicas Especiales
	\1 \marcar{Bloques comerciales y áreas emergentes}
		\2 Idea clave
			\3 Concepto
		\2 Europa
			\3 UE
			\3 EFTA
			\3 EEE
		\2 Norteamérica
			\3 NAFTA
			\3 USMCA -- US-México-Canada Agreement
		\2 Latinoamérica y Caribe
			\3 Idea clave
			\3 Mercosur
			\3 Alianza del Pacífico
		\2 Asia
			\3 Idea clave
			\3 ASEAN
			\3 AEC -- ASEAN Economic Community
			\3 TPP
			\3 CPTPP/TPP11
			\3 RCEP
		\2 África
			\3 Tripartite FTA
			\3 AfCFTA
	\1 \marcar{Guerra comercial}
		\2 Causas
			\3 Crecimiento de emergentes y problema ``del recién llegado''
			\3 Dólar como moneda de reserva global
			\3 Represalias
		\2 Consecuencias
			\3 Sistema multilateral de comercio internacional
			\3 Costes
			\3 Incertidumbre
		\2 Perspectivas
			\3 Propuestas de reforma del sistema multilateral
			\3 Bilateralismo
			\3 Regionalización de CVGs
			\3 Acuerdo EEUU--China de 2019
			\3 Trilema de Rodrik
			\3 ``Decoupling'' China-Estados
	\1[] \marcar{Conclusión}
		\2 Recapitulación
			\3 Flujos comerciales y financieros
			\3 Bloques comerciales y áreas emergentes
			\3 Guerra comercial
		\2 Idea final
			\3 Regionalización de CVGs
			\3 Incertidumbre global
			\3 Acuerdos cada vez más profundos
			\3 Sistema multilateral de comercio debilitado

\end{esquema}

\esquemalargo

\begin{esquemal}
	\1[] \marcar{Introducción}
		\2 Contextualización
			\3 Evolución del comercio internacional
				\4 Explosión en últimos siglos y décadas
				\4[] CI ha crecido mucho más que PIB
				\4 Avance tecnológico:
				\4[] $\downarrow$ de costes de transporte
				\4[] $\downarrow$ de costes informacionales
				\4 Sujeto de estudio relativamente antiguo:
				\4[] $\to$ Hume, Smith, Ricardo, Mill, Torrens
				\4[] Ligado a la evolución de:
				\4[] $\to$ teoría económica
				\4[] $\to$ hallazgos empíricos
			\3 Globalización
				\4 Concepto muy vago
				\4[] Múltiples acepciones según contexto
				\4 Sentido en ciencia económica
				\4[] Libre circulación mundial de
				\4[] $\to$ Bienes y servicios
				\4[] $\to$ Factores de producción
				\4[] $\to$ Tecnologías e ideas
				\4 Diferentes fases a lo largo de la historia
				\4[] Ruta de la seda, descubrimiento de América
				\4[] Antes de la I Guerra Mundial
				\4[] $\to$ Trabajo y capital muy liberalizados
				\4[] Entreguerras
				\4[] $\to$ Aislamiento de bloques económicos
				\4[] Posguerra
				\4[] $\to$ Progresiva liberalización comercial
				\4[] $\to$ Post BW: liberalización capitales
				\4[] $\to$ Nuevas tecnologías globalizan información
				\4 Últimas décadas de siglo XX
				\4[] $\to$ Aceleración del proceso
				\4 Hechos estilizados de la globalización
				\4[] Comercio internacional
				\4[] 30\% del PIB mundial en últimas décadas
				\4[] $\to$ Desde 10\% en años 70
				\4 Acciones extranjeras
				\4[] 20\% de carteras
				\4[] $\to$ Desde niveles insignificantes
				\4 Inversión Directa Extranjera
				\4[] alrededor del 15\% de FBK
				\4[] $\to$ Desde el 2\% en años 70
				\4 Población nacida en el extranjero
			\3 Concepto de integración económica
				\4 Economías/países/jurisdicciones acuerdan
				\4[] Reducir barreras a movimiento:
				\4[] $\to$ Bienes y servicios
				\4[] $\to$ Trabajo y capital
				\4 Intensidad variable de la integración
				\4[] ¿Qué movimiento se permite?
				\4[] ¿Hasta qué punto se eliminan las barreras?
				\4[] ¿Qué barreras se eliminan?
			\3 Guerra comercial
				\4 Proceso en pleno desarrollo actualmente (2019)
				\4[] Debilitamiento de sistema multilateral de comercio
				\4[] Presión sobre flujos de inversión
				\4 Insatisfacción creciente con estructura del CI
				\4[] Focalizada en determinadas regiones
				\4[] $\to$ Especialmente afectadas por liberalización comercial
				\4 Factores políticos y estratégicos
				\4[] Estados Unidos vs China
				\4[] Unión Europea vs EEMM
				\4[] Otros bloques comerciales
				\4 Efectos globales
				\4[] Mayores economías y bloques comerciales implicados
				\4[] Cadenas de valor internacional amplifican
		\2 Objeto
			\3 ¿Cómo han evolucionado los flujos comerciales y financieros en la historia reciente?
			\3 ¿Qué características sectoriales tienen los flujos comerciales y financieros en la actualidad?
			\3 ¿Qué estructura geográfica caracteriza a los flujos comerciales y de inversión?
			\3 ¿Qué bloques comerciales existen en la actualidad?
			\3 ¿Qué evolución puede esperarse de las áreas comerciales emergentes?
			\3 ¿Qué causas y qué efectos tiene la guerra comercial actual?
		\2 Estructura
			\3 Flujos comerciales y financieros
			\3 Bloques comerciales y áreas emergentes
	\1 \marcar{Evolución histórica}
		\2 Hasta la I Guerra Mundial
			\3 Contexto
				\4 Descubrimiento del Nuevo Mundo (1492)
				\4[] Enormes extensiones de tierra a explotar
				\4[] $\to$ Absorber exceso de mano de obra en Europa
				\4[] $\to$ Extraer materias primas
				\4[] $\to$ Incorporar mano de obra nativa
				\4 Avances tecnológicos
				\4[] Navegación
				\4[] Máquina de vapor
				\4[] Telecomunicaciones
				\4 Imperios coloniales
				\4[] Semejantes a bloques comerciales
				\4[] Comercio fuera de metrópoli muy restringido
			\3 Eventos
				\4 Industria textil crecientemente tecnificada
				\4 Abolición de las Leyes del Grano (1846)
				\4 Estabilidad cambiaria
				\4 Especialización colonial
				\4 Protección industrial
			\3 Consecuencias
				\4 División global del trabajo
				\4 Desarrollo de flujos financieros
				\4 Debate proteccionismo--liberalismo
				\4 Elevada integración comercial pre-IGM
				\4[] A priori, entendida como obstáculo a guerras
				\4[] $\to$ A pesar de ello, estalla IGM
				\4[] $\then$ Múltiples teorías más allá de economía
		\2 Entreguerras
			\3 Contexto
				\4 Suspensión de convertibilidad en IGM
				\4 Imperio Austrohúngaro desintegrado
				\4[] Dividido en varios países
				\4[] Nuevas barreras al comercio
				\4 Peso económico de EEUU en claro ascenso
				\4 Inflación durante la guerra
				\4[] Aumenta niveles de precios
				\4[] $\to$ Distorsión TCReal de economías europeas
				\4[] $\then$ ¿A qué TCN restablecer convertibilidad?
				\4 Nuevas teorías de comercio internacional
				\4[] Heckscher y Ohlin en años 20
				\4[] $\to$ Dotaciones relativas determinan comercio
			\3 Eventos
				\4 Programa de los 14 puntos de EEUU tras IGM
				\4[] Libre comercio
				\4[] Inicio de descolonización
				\4[] Autodeterminación
				\4 Restablecimiento de la convertibilidad
				\4[] Primero, Reino Unido
				\4[] $\to$ A tipo de pre-guerra
				\4[] Incentivo a restablecer después
				\4[] $\to$ Para fijar TCN menor
				\4[] $\then$ Mejorar TCR a costa del que restablece primero
				\4 Acumulación de desequilibrios exteriores
				\4[] Superávits en Francia y EEUU
				\4[] Déficits en Reino Unido
				\4[] Alemania imprime dinero para financiar reparaciones
				\4 IDE estadounidense en Europa
				\4 Créditos de reconstrucción
				\4 Reparaciones de guerra alemanas
				\4 Gran Depresión (1929)
				\4[] Subida de tipos de interés en EEUU
				\4[] $\to$ Tratar de frenar burbuja en mercados financieros
				\4[] Frenazo a IDE y repatriación
				\4[] Expansión de shock a periferia
				\4[] Fuerte caída del comercio
				\4 Proteccionismo e intervencionismo
				\4[] Imposición de aranceles ley Smoot-Hawley
				\4 Regionalización
				\4[] Búsqueda de espacios vitales
				\4 Importancia creciente del petróleo
				\4[] Mecanización de agricultura
				\4[] Aumento de comercio intra-regional
			\3 Consecuencias
				\4 Devaluaciones competitivas
				\4 Proteccionismo
				\4 Búsqueda de ``espacios vitales'' para comerciar
				\4 Productos primarios dominan comercio internacional
		\2 Desde la II Guerra Mundial hasta la crisis del petróleo
			\3 Contexto
				\4 Enorme destrucción K físico y humano
				\4 Clima favorable a cooperación internacional
				\4 EEUU y URSS grandes vencedores de la IIGM
				\4 Imperios coloniales debilitados
				\4 Anomalías respecto a modelo H-O
				\4[] Patrón exportador que no debería producirse
			\3 Eventos
				\4 Bretton Woods (1944)
				\4[] $\to$ Crear nueva arquitectura económica mundial
				\4[] URSS se desvincula
				\4[] Aparece FMI y BIRD
				\4[] OIC no llega a nacer
				\4[] GATT-47
				\4 Comercio mundial crece más que producción
				\4[] Especialmente en Europa Occidental
				\4 Comercio con PEDs apenas crece
				\4[] Con ellos
				\4[] Entre ellos
				\4 Plan Marshall
				\4[] Principal flujo financiero en 40s y primeros 50s
				\4[] 13.000 M de dólares
				\4 Tensión creciente con bloque comunista
				\4[] Bloqueo de Berlín occidental en 1947
				\4[] Rechazo de URSS a Plan Marshall
				\4 Creación del COMECON\footnote{Council for Mutual Economic Assistance.} en 1949
				\4[] Consejo de Ayuda Económica Mutua
				\4[] Intento por dividir internacionalmente el trabajo
				\4[] Relaciones bilaterales EEMM--URSS
				\4[] URSS >60\% de producción y comercio
				\4 Creación de la Comunidad Económica Europea en 1957
				\4 Manufacturas ganan peso frente a agrícolas
				\4 Petróleo y derivados aumentan importancia
				\4 Comercio agrícola sigue representando $\sim 50\%$
				\4 Convertibilidad restablecida progresivamente
				\4[] Cuenta corriente, no cuenta financiera
				\4[] Parte esencial de crecimiento de comercio
				\4[] En UE, plenamente a partir de 1958
				\4 Flujos financieros crecen lentamente
				\4 Aparición de euromercados
				\4[] Miedo a bloqueo de depósitos USD de URSS
				\4[] Q regulation en EEUU que limita interés depósitos
				\4 Crisis del petróleo
				\4[] En 1973 y 1979
				\4[] Inflaciones generalizadas
				\4[] Impulso a transformaciones en ecs. occidentales
				\4[] $\to$ Búsqueda de nuevos
			\3 Consecuencias
				\4 Flujos financieros creciente pero reducidos
				\4[] Aparecen en 50s
				\4[] No despegan hasta 60s
				\4 EEUU es poder económico hegemónico
				\4[] Comercial y financiero
				\4[] Emisor y receptor de IDE
				\4[] Posición acreedora en primeros 50s
				\4[] Comienza a deteriorarse en 60s
				\4 Configuración de tres grandes bloques comerciales
				\4[] Bloque occidental
				\4[] Bloque comunista
				\4[] $\to$ Flujos bilaterales con URSS como centro
				\4[] $\to$
				\4[] Resto del mundo
				\4[] $\to$ Pierde peso específico
				\4[] $\to$ Retraso en desarrollo
				\4 URSS aumenta dependencia de petróleo
				\4[] Compite con bloque OPEP de embargo
				\4 Aumento de la deuda en PEDs no petróleo
				\4[] Exceso de ahorro en expotadores de petróleo
				\4[] $\to$ Canalizado vía NYC hacia PEDs
				\4[] $\then$ Semilla de germen de crisis posteriores
		\2 Desde la Crisis del petróleo hasta principios del S. XXI
			\3 Contexto
				\4 Acumulación de desequilibrios
				\4[] EEUU fuerte déficit CC
				\4[] Dólares superan reservas de oro
				\4 Aumento de flujos de capital
				\4 Tensiones militares y políticas
				\4 Guerras proxy
				\4[] Occidente vs bloque soviético
				\4[] Enfrentados en terceros países
			\3 Eventos
				\4 Caída de Bretton Woods
				\4[] Nixon Shock en agosto de 1971
				\4[] Acuerdos de Smithsonian en 1971
				\4[] Caída del túnel en 1973
				\4[] Fin oficial de BW en Acuerdos de Jamaica (1976)
				\4 Expansión del comercio de mercancías
				\4[] Medias del 4\% anual hasta 1990
				\4 Aumento de comercio de servicios
				\4 EOI se muestra superior a ISI
				\4[] Tigres asiáticos rinden mejor que ARG, IND...
				\4 Regionalización comercial
				\4[] Unión Europea creciente integración
				\4[] $\to$ Acta Única 87
				\4[] $\to$ Maastricht 92
				\4[] Intentos en América Latina
				\4[] África sigue muy poco integrada
				\4 Integración de ex-URSS a partir de 1990
				\4[] Nuevas oportunidades de expansión de comercio
				\4[] Rusia como nuevo actor relevante
				\4[] $\to$ Exportación de energéticos
				\4[] Ex-bloque comunista en Europa del Este
				\4[] $\to$ Se acercan a UE
				\4[] $\to$ Fuerte expansión del comercio
				\4 Reunificación alemana
				\4[] Impacto sobre interés y ahorro en Alemania
				\4 OMC
				\4[] Nuevo marco multilateral de comercio
				\4[] Permite expansión en servicios
				\4[] Nuevo marco de resolución de disputas
				\4[] $\to$ Fuerte incentivo a cumplir reglas
				\4 Enorme crecimiento de flujos financieros
				\4[] Inversión en cartera se dispara
				\4[] IDE también crece fuertemente
				\4[] Impacto de digitalización incipiente en finanzas
			\3 Consecuencias
				\4 No-sistema financiero internacional
				\4[] Heterogeneidad de regímenes cambiarios
				\4[] Tendencia a bipolaridad tras crisis asiáticas
				\4[] $\to$ Hard peg
				\4[] $\to$ Libre flotación
				\4[] En realidad, miedo a flotar en muchos países
				\4[] $\to$ Nominalmente libre flotación
				\4[] $\to$ En la práctica, soft peg
				\4[] $\to$ También en algunos desarrollados
				\4 Tres grandes polos económicos
				\4[] EEUU
				\4[] Japón
				\4[] Unión Europea
				\4 Servicios es nuevo frente de expansión
				\4 Agricultura reduce peso en comercio mundial
				\4[] Apenas 10\%
				\4 ISI abandonado por EOI
				\4[] Tigres asiáticos convergen a pesar de crisis financiera
				\4[] Impulso adicional a comercio internacional
				\4 Reducción generalizada de aranceles
				\4[] Salvo agricultura
				\4[] Protección no-arancelaria es nuevo campo de batalla
		\2 Desde principios del S. XXI hasta la Crisis Financiera
			\3 Contexto
				\4 Fuerte crecimiento económico
				\4 Expansión del comercio > expansión de output
				\4 Emergentes asiáticos tienen miedo a crisis
				\4[] Elevadas tasas de ahorro
				\4[] Acumulación de divisas fuertes
			\3 Eventos
				\4 Estallido burbuja tecnológica en 2001
				\4[] Poco impacto sobre el comercio
				\4 Tipos bajos en int
				\4 Exceso de ahorro en emergentes
				\4[] ``savings glut''
				\4[] $\to$ Canalizado hacia desarrollados
				\4[] $\to$ Demanda de activos seguros
				\4 China se integra plenamente en comercio mundial
				\4[] Entrada en 2001
				\4 Consolidación de Cadenas Globales de Valor
				\4[] Productos ya no se producen en bloques regionales
				\4[] Diferentes fases repartidas globalmente
				\4[] Contenido importador de las exportaciones aumenta
				\4[] $\to$ Aranceles más disruptivos
				\4[] $\to$ Proteccionismo más difícil de aprovechar
			\3 Consecuencias
				\4 Acumulación de desequilibrios globales
				\4[] Enormes déficits CC en algunos desarrollados
				\4[] $\to$ EEUU
				\4[] $\to$ España
				\4[] $\to$ UK
				\4[] $\to$ ...
				\4[] Enormes acumulaciones de reservas en Japón+emergentes
				\4[] $\to$ Japón
				\4[] $\to$ China
				\4[] $\to$ Rusia
				\4[] $\to$ Tailandia
				\4[] $\to$ Arabia Saudí
				\4[] $\to$ ...
				\4 Saldos corrientes totales\footnote{Suma de saldos positivos y negativos en la cuenta corriente}
				\4[] Aumentan hasta máximos históricos\footnote{Ver IMF (2019) ESR.}
				\4 Flujos financieros en máximos históricos
				\4[] Préstamos interbancarios
				\4[] Inversión de cartera
				\4[] Flujos financieros transfronterizos:
				\4[] $\to$ Del 5\% PIB mundial en 90s
				\4[] $\to$ Al 20\% PIB mundial en 2007
				\4[] Pasivos externos totales
				\4[] $\to$ Del 60\% al 180\% en el mismo periodo
				\4 Política comercial mucho más compleja
				\4[] Más difícil conocer verdaderas rentas de exportación
				\4[] $\to$ ¿Cuánto corresponde a inputs importados?
				\4[] Debilitamiento de la política comercial
				\4 CVGs son vía para desarrollo
				\4[] Integración en CVGs es herramienta central
				\4[] Ascenso progresivo en escala de valor añadido
		\2 Desde la Crisis Financiera Global hasta la actualidad
			\3 Contexto
				\4 Fuertes desequilibrios de BP
				\4 Burbujas inmobiliarias en muchos desarrollados
				\4 Activos tóxicos en sistema financiero
			\3 Eventos
				\4 Estallido de la CFG en 2007
				\4 Lehman Brothers es wake-up call definitiva
				\4 Volatilidad en mercados de materias primas
				\4 Mercados financieros globales se secan
				\4[] Tipos de interés se disparan
				\4[] Enorme aumento de aversión al riesgo
				\4 Colapso de flujos comerciales 08-09
				\4[] ``Great Trade Collapse''
				\4[] Caída del 10\% del comercio mundial
				\4[] $\to$ Frente a caída del 1\% de PIB mundial
				\4[] De 16,2 billones (españoles) de dólares en 2008
				\4[] $\to$ A 12,5 billones en 2009
				\4[] Causas:
				\4[] $\to$ Fuerte caída de la demanda
				\4[] $\to$ Desaparición del crédito para exportación
				\4[] $\to$ Expansión de las CVG contagia desplome
				\4 Colapso de flujos financieros
				\4[] Caída cercana al 65\% entre 2007 y 2014-2015
				\4 Tipos bajos persistentes
				\4[] Caída en trampas de liquidez
				\4[] Especialmente en Europa
				\4[] $\to$ Aunque bajadas más tarde respecto a USA
				\4 Crisis de deuda en Europa
				\4[] Se extiende varios años
				\4[] Especialmente 2009-2011
				\4[] Excesivo endeudamiento privado
				\4[] Círculo vicioso del sistema bancario
				\4[] $\to$ Trilema de Pisany-Ferry
				\4 Quantitative Easing
				\4[] Compras masivas de activos financieros por BCs
				\4[] Caída de tipos de interés efectivos
				\4[] Flujo de capital hacia emergentes
				\4[] $\to$ Búsqueda de mayores rendimientos
			\3 Consecuencias
				\4 Emergentes resisten bien a crisis financiera
				\4[] Gracias a grandes reservas de divisas
				\4[] Cambio histórico
				\4 Sin respuesta proteccionista generalizada
				\4[] A pesar de:
				\4[] $\to$ turbulencias en comercio
				\4[] $\to$ Crisis
				\4 Guerra comercial 10 años después
				\4[] Factores no sólo económicos
				\4[] $\to$ Seguridad nacional
				\4[] $\to$ Geopolítica
				\4[] $\to$ Tensiones militares en Asia
				\4 Debilitamiento del sistema multilateral de comercio
				\4[] Órgano de Apelación ya no está operativo
				\4 Incipiente antagonismo China--EEUU
				\4[] Creciente desaparición de flujos comerciales
				\4 Aumento progresivo del peso de los servicios
				\4[] Nuevas tecnologías favorecen especialmente
				\4[] Turismo explota de China hacia fuera
		\2 COVID-19\footnote{Ver \href{https://www.wto.org/english/news_e/pres20_e/pr855_e.htm}{Escenarios de OMC abril 2020}}
			\3 Contexto
				\4 Fase más larga de crecimiento de historia
				\4[] Desde crisis financiera hasta marzo 2020
				\4 Crecimiento relativamente débil
				\4 Tensiones comerciales EEUU-China
				\4 Desvinculación incipiente
			\3 Eventos
				\4 Virus respiratorio desde diciembre 2019 en China
				\4 Expansión al resto del mundo desde febrero
				\4 Economías mundiales paralizan actividad
				\4 Shock de oferta y demanda al mismo tiempo
				\4[] Cadenas de producción paralizadas
				\4[] Desplome de la demanda
			\3 Consecuencias
				\4 Recesión global más grave desde Gran Depresión
				\4 Salida incierta
				\4 Reactivación asimétrica de economías
				\4 Descapitalización de empresas
				\4 Programas de estímulo monetario nunca vistos
	\1 \marcar{Flujos de bienes y servicios}\footnote{Ver WTO (2019) \textit{Annual Report}. y WTO (2019) \textit{World Trade Statistical Review}. \url{https://www.wto.org/english/res_e/statis_e/wts2019_e/wts19_toc_e.htm}. Ver también \url{https://data.wto.org/}}
		\2 Idea clave
			\3 Contexto
				\4 Reducción progresiva de costes de transporte
				\4 CVGs
				\4 Economías planificadas apenas relevantes
			\3 Objetivo
				\4 Caracterizar evolución de sectores
				\4 Representar relaciones entre sectores y regiones
				\4 Desarrollos recientes de política comercial
			\3 Resultados
				\4 Tendencias de largo plazo
				\4[] Integración de países en CVGs
				\4[] Crecimiento de los volúmenes de comercio
				\4[] Crecimiento de China
				\4[] Reducción de desequilibrios exteriores
				\4[] $\to$ Tras crisis financiera
				\4[] $\to$ Del 6\% al 3,5\% C/NF media
				\4[] Concentración geográfica de flujos comerciales
				\4[] Asia gana peso de manera constante
				\4[] Emergentes aumentan peso progresivamente
				\4 Últimos años
				\4[] EEUU cambia política comercial
				\4[] Renegociación de acuerdos comerciales
				\4[] Ralentización del crecimiento del comercio
				\4[] Servicios ganan peso progresivamente
				\4 Desarrollos recientes
				\4[] Acuerdo preliminar USA-China
				\4[] Bloqueo del órgano de apelación
				\4[] Crecimiento del comercio supera al PIB
				\4[] Entre 2014-2016, comercio crece más que PIB
				\4[] $\to$ Comercio > PIB a de nuevo desde 2017
		\2 Hechos estilizados
			\3[i] Crecimiento comercio > PIB hasta GCFinanciera
				\4 Desde BW hasta GFC
				\4[] Comercio internacional crece más que PIB
				\4 Integración de China, Rusia, India en CI
				\4[] Factor clave
				\4 Aumento de diversificación de bienes comerciados
			\3[ii] Integración creciente de PEDs
			\3[iii] Comercio intraindustrial N-N mayor a N-S
				\4 Países desarrollados comercian más intraindustria
				\4 Interindustria se reduce la diferencia
			\3[iv] Ascenso de China heterogéneo
				\4 Mayor exportador de mercancías
				\4 Tercer mayor importador de mercancías
				\4 Bajo peso en exportación de servicios
				\4 Creciente peso en importación de servicios
			\3[v] Servicios cada vez más importantes
				\4 Servitización de industria
				\4 Explosión de turismo
			\3[vi] Impacto heterogéneo de covid
				\4 Muy fuerte en viajes y transporte
				\4 Relativamente fuerte en bienes
				\4 Positivo en determinados servicios
				\4[] Al menos, inicialmente
		\2 Mercancías
			\3 Idea clave
				\4 19.5 USD billones en 2018\footnote{Ver \href{https://data.wto.org/}{Portal de datos de OMC}:indicators:International trade Statistics:merchandise value }
				\4[] Desde 17.3 USD billones en 2017
				\4[] $\then$ 12\% de crecimiento
				\4 Aumento acumulado desde 2008
				\4[] $\to$ Casi 20\% más que en 2008
				\4[] $\then$ Fuerte recuperación tras crisis financiera
				\4 Crecimiento del 3\% en 2018
				\4[] Aproximadamente igual al PIB (2.9\%)
				\4[] $\to$ Parcialmente por $\Delta$ precios del petróleo
				\4 UE es mayor exportador mundial en mayoría de sectores
				\4 Elevada concentración de exportadores
				\4[] 10 mayores exportadores concentran 72\% exportaciones
				\4 18.8 USD billones en 2019
				\4[] Ligera disminución
			\3 Estructura sectorial\footnote{Ver \href{https://data.wto.org/}{Portal de datos de OMC}:indicators:International trade Statistics:merchandise value }
				\4 Agrícolas
				\4[] 1,8 B de USD (2018)
				\4[] Fuerte aumento acumulado desde 2008
				\4[] $\to$ Casi 40\% más de lo exportado en 2008
				\4[] Caída entre 2014 y 2016
				\4[] $\to$ Correlativo a resto de sectores
				\4[] Soja aumenta peso total de exportaciones agrícolas
				\4[] $\to$ A pesar de caída en los precios desde 2008
				\4[] $\to$ Brasil es mayor exportador mundial
				\4[] $\to$ Sigue Estados Unidos
				\4[] China, UE y México mayor crecimiento
				\4[] Principales exportadores:
				\4[] 1. Unión Europea
				\4[] $\to$ 681.000 M USD en 2018
				\4[] 2. Estados Unidos
				\4[] $\to$ 172.000 M USD en 2018
				\4[] 3. Brasil
				\4[] 4. China
				\4[] 5. Canadá
				\4[] 6. Indonesia
				\4[] 7. Tailandia
				\4 Hidrocarburos y extractivos
				\4[] Segundo sector de exportación
				\4[] $\to$ 3.1 B españoles de USD
				\4[] Sector más volátil de los 3 principales
				\4[] $\to$ Más que agrícola
				\4[] $\to$ Más que manufacturas
				\4[] Caída acumulada desde 2008
				\4[] $\to$ Cerca del 90\% de lo exportado en 2008
				\4[] Toca fondo en 2016
				\4[] $\to$ Apenas 60\% de 2008
				\4[] Recuperación desde 2016 hasta 2018
				\4[] $\to$ Un 50\% más en 2018 respecto a 2016
				\4[] Causas generales:
				\4[] $\to$ Demanda más débil en desarrollados
				\4[] $\to$ Bajada de precios del petróleo
				\4[] Principales exportadores:
				\4[] 1. Unión Europea
				\4[] $\to$ Capacidad de refino elevada
				\4[] $\to$ Industria petroleoquímica
				\4[] 2. Rusia
				\4[] 3. Estados Unidos
				\4[] 4. Arabia Saudí
				\4[] 5. Australia
				\4[] 6. Canadá
				\4 Manufacturas
				\4[] Mayor sector de exportación
				\4[] $\to$ Más de 13,1 B de USD en 2018
				\4[] Aumento acumulado desde 2008
				\4[] $\to$ Casi 30\% más que en 2008
				\4[] Caída entre 2014 y 2016 en línea resto sectores
				\4[] UE, China mayores crecimientos en 2018
				\4[] USA, Corea, menores crecimientos (aunque >0\%)
				\4[] Fuerte concentración
				\4[] $\to$ 10 mayores exportan >80\% de exp. mundial
				\4[] Principales exportadores:
				\4[] 1. Unión Europea
				\4[] $\to$ 5 billones españoles USD
				\4[] 2. China
				\4[] $\to$ 2.3 billones españoles USD
				\4[] 3. Estados Unidos
				\4[] $\to$ 1.2 billones españoles USD
				\4[] 4. Japón
				\4[] 5. Corea
				\4[] 6. Hong Kong
				\4[] 7. México
				\4 Subsectores manufactureros
				\4[1.] Maquinaria y transportes
				\4[] $\to$ Casi 7 billones
				\4[] $\to$ Incluye automóviles, telecomunicaciones...
				\4[] $\to$ UE domina en automóviles, casi 50\%
				\4[] $\to$ China domina en telecom
				\4[2.] Otras manufacturas
				\4[] $\to$ Casi 2,8 B de USD
				\4[] $\to$ Artículos de consumo
				\4[] $\to$ Maquinaria de precisión
				\4[] $\to$ Otras manufacturas misceláneas
				\4[3.] Químicos
				\4[] $\to$ 2.2 billones USD
				\4[] $\to$ Incluye farmacéuticos
				\4[] $\to$ UE domina
				\4[4.] Acero y hierro
				\4[] $\to$ Casi 500.000 M USD
				\4[] $\to$ Europa domina
				\4[] $\to$ Sigue China
				\4[] $\to$ USA e India pierden peso
				\4[5.] Prendas de vestir
				\4[] $\to$ 500.000 M USD
				\4[] $\to$ China domina, más de 1/3
				\4[6.] Textil
				\4[] $\to$ 312.000 M USD
				\4[] $\to$ China domina
			\3 Estructura geográfica
				\4 Diez mayores economías por volúmenes M+X
				\4[] 1. China
				\4[] $\to$ Aumenta componentes electrónicos
				\4[] $\to$ Reduce superávit comercial
				\4[] $\to$ Aumenta valor añadido de exportaciones
				\4[] 2. Estados Unidos
				\4[] $\to$ Importaciones crecen fuertemente
				\4[] $\then$ 2.61 billones USD
				\4[] $\to$ Exportaciones aumentan al mismo \%
				\4[] $\then$ 1.66 billones USD
				\4[] 3. Alemania
				\4[] $\to$ Aumenta X de componentes de vehículos
				\4[] $\to$ Aumenta X de productos farmacéuticos
				\4[] 4. Japón
				\4[] $\to$ Caen componentes electrónicos
				\4[] $\to$ Cae maquinaria no eléctrica
				\4[] $\to$ Aumenta exportación de petróleo
				\4[] 5. Holanda
				\4[] $\to$ Aumenta hidrocarburos y medicamentos
				\4[] 6. Francia
				\4[] $\to$ Aumenta equipos de transporte
				\4[] $\to$ Aumenta aviación
				\4[] 7. Hong Kong
				\4[] $\to$ Caen metales preciosos y oro
				\4[] 8. Reino Unido
				\4[] 9. Corea del Sur
				\4[] 10. Italia
				\4 Estados Unidos es mayor importador mundial
				\4 Holanda es el exportador más dinámico en 2018
				\4[] Mayor crecimiento en \% de exportaciones
				\4 Fuerte concentración geográfica
				\4[] 10 mayores economías
				\4[] $\to$ 53\% del comercio mundial de mercancías
				\4 Crecimiento generalmente menor a pre-crisis
				\4[] En mayoría de principales exportadores
				\4[] $\to$ M y X crecen menos que antes de crisis
				\4 Por nivel de desarrollo: exportaciones\footnote{Pag. 19 de WTO (2019) WTS}
				\4[] PEDs+CIS mayor crecimiento en 2018
				\4[] Economías desarrolladas menor crecimiento en 2018
				\4 Por nivel de desarrollo: importaciones
				\4[] PEDs+CIS mayor crecimiento en últimos años
				\4[] $\to$ Caen en 2018
				\4[] Desarrollados crecimiento más lento en 2018
				\4 Por regiones: exportaciones
				\4[] Asia más de 30\% desde 2012
				\4[] Sudamérica apenas <10\% desde 2012
				\4[] Norteamérica superiores al 20\% desde 2012
				\4[] Europa cercano a 15\% desde 2012
				\4[] Otras regiones, similar a Europa
				\4 Por regiones: importaciones
				\4[] Asia más del 30\% desde 2012
				\4[] Norteamérica cerca del 30\% desde 2012
				\4[] Europa por encima del 10\% desde 2012
				\4[] Sudamérica fuerte caída entre 2014 y 2016
				\4[] $\to$ Recuperación a partir de 2016
				\4[] $\to$ Cercano a 0\% acumulado desde 2012
		\2 Servicios
			\3 Idea clave
				\4 5.85 USD billones en 2018
				\4[] Desde 5.4 USD billones en 2017
				\4 Crecimiento del 8\% en 2018
				\4[] Superior al de mercancías
				\4 Tendencia de largo plazo muy dinámica
				\4[] 65\% de crecimiento desde 2008
				\4[] Caída en 2009
				\4[] $\to$ Crecimiento ininterrumpido desde entonces
				\4 Cambios tecnológicos catalizador de crecimiento
				\4[] Modo 1 facilitado
				\4 6.03 USD billones en 2019\footnote{Actualizado en abril de 2020}
				\4[] $\to$ Ligero crecimiento (2\%) entre 2019 y 2018
				\4[] Crecimiento inferior a 2017-2018
				\4[] $\to$ 9\% entre 2017 y 2018
			\3 Estructura sectorial\footnote{Capítulo IV del WTO (2019). Ver \url{https://data.wto.org/}, seleccionando para años 2008-2018 y ``\textit{Trade in commercial services}'' en Indicators.}
				\4[1] Viajes
				\4[] Principal componente
				\4[] Más de 1,417 B USD españoles en 2019
				\4[] $\to$ +1,1\% respecto 2018
				\4[] $\to$ Débil crecimiento
				\4[] $\Delta$ más de 50\% desde 2008
				\4[2] Servicios a empresas
				\4[] Casi 1,2 B USD españoles en 2018
				\4[] $\Delta$ casi 40\% desde 2008
				\4[3] Transporte
				\4[] 1.215 B USD español en 2019
				\4[] Crecimiento de casi 25\% desde 2008
				\4[] Relativamente menos dinámico que otros
				\4[] Fuertemente ligado a comercio de mercancías
				\4[4] Telecomunicaciones\footnote{Sin datos para 2019 en abril de 2020.}
				\4[] 600.000 M USD en 2018
				\4[] Se dobla desde 2008
				\4[5] Servicios financieros y seguros\footnote{Sin datos para 2019 en abril de 2020.}
				\4[] Casi 500.000 M USD en 2018
				\4[] Crecimiento sostenido desde 2014
				\4[6] Propiedad intelectual\footnote{Sin datos para 2019 en abril de 2020.}
				\4[] Casi 400.000 M USD en 2018
				\4[] $\Delta$ de casi 60\% en 2018
				\4[7] Servicios manufactureros\footnote{Servicios de manufactura aplicados a factores de producción propiedad de terceros.}
				\4[] Cerca de 110.000 M USD en 2018
				\4[] Crecimiento débil desde 2008
				\4[] $\to$ Apenas 10\% acumulado
				\4[8] Construcción
				\4[] Cerca de 110.000 M USD en 2018
				\4[] Débil crecimiento desde 2008
				\4[9] Mantenimiento y reparación
				\4[] 100.000 M USD en 2018
				\4[] $\Delta$ 150\% desde 2008
			\3 Estructura geográfica\footnote{Capítulo V de WTO (2019) y \href{https://data.wto.org/}{Estadísticas de WTO}.}
				\4 Fuerte heterogeneidad
				\4 Elevada concentración de las exportaciones
				\4 G-20 casi 80\% de exportaciones totales
				\4 Desarrollados son principales exportadores
				\4[] Casi 70\% de exportaciones de servicios
				\4 PEDs y emergentes segunda mayor región exportadora
				\4[] Casi 2 B USD sobre 5.8 B USD total
				\4 Europa es principal región exportadora de servicios
				\4[] Casi 50\% de total en 2018
				\4 UE-27 es principal área aduanera exportadora
				\4[] 2,5 B USD sobre 5.8 B USD total
				\4 Asia sigue a Europa en exportaciones
				\4[] Casi 1.5 B USD en 2018
				\4 Economías asiáticas son las que más crecen
				\4[] Especialmente:
				\4[] $\to$ Servicios informáticos
				\4[] $\to$ Construcción
				\4[] $\to$ Servicios a empresas
				\4[] $\to$ Servicios relacionados por prop. intelectual
				\4 China e India + PEDs aumentan peso de servicios
				\4 Norteamérica (US, CAN, MEX) es tercera área exportadora
				\4[] Ligeramente inferior a 1 B USD en 2018
				\4 África tiene escaso peso en exportaciones mundiales
				\4[] Apenas 100.000 M USD
				\4 Estados Unidos es principal exportador mundial
				\4 Diez mayores economías por volúmenes X+M (2018)
				\4[] 1. Estados Unidos
				\4[] 2. China
				\4[] 3. Alemania
				\4[] 4. Reino Unido
				\4[] 5. Francia
				\4[] 6. Holanda
				\4[] 7. Irlanda
				\4[] 8. Japón
				\4[] 9. India
				\4[] 10. Singapur
				\4 Mayores exportadores mundiales en 2018
				\4[] 1. Estados Unidos
				\4[] 2. Reino Unido
				\4[] 3. Alemania
				\4[] 4. Francia
				\4[] 5. China
				\4[] 6. Holanda
				\4[] 7. Irlanda
				\4[] 8. India
				\4[] 9. Japón
				\4[] 10. Singapur
				\4[] 11. España
		\2 Comercio intra-industrial
			\3 Idea clave
				\4 Concepto
				\4[] Intercambio de variedades de una misma industria
				\4[] $\to$ Mutuamente exportadas e importadas
				\4[] Especial atención de estudios desde s. XX
				\4[] $\to$ Contrario a patrones de especialización
				\4 Tipos
				\4[] Horizontal
				\4[] $\to$ Intercambio de variedades de mismo producto
				\4[] $\then$ Móviles Sony por Samsung
				\4[] $\then$ Coches Mercedes por Cadillac
				\4[] Vertical
				\4[] $\to$ Misma industria pero distinta fase de procesamiento
				\4[] $\then$ Componentes de automóvil y automóviles terminados
				\4 Indicadores
				\4[] Grubel-LLoyd:
				\4[] $\to$ Comercio intraindustrial en industria $i$
				\4[] $\to$ $G_i = 1 - \frac{\left| X_i - M_i \right|}{X_i + M_i} = \frac{(X_i + M_i) - \left| X_i - M_i \right| }{X_i + M_i}$
			\3 Hechos estilizados\footnote{Ver OCDE (2010). Buscar estudios más recientes (escasos)}
				\4[i] Típicamente ocurre entre:
				\4[] Países ricos
				\4[] Estructura económica similar
				\4[] Grado de desarrollo equivalente
				\4[] Cercanía geográfica
				\4[ii] Inversión Extranjera Directa
				\4[] A menudo relacionada con CI intraindustrial
				\4[] Multinacionales establecen filiales en diferentes países
				\4[] $\to$ Comercian ByS entre filiales y matrices
				\4[] En emergentes:
				\4[] $\to$ Correlación CI i-industrial con flujos entrantes de IDE
				\4[iii] Mayor cuanto mayor contenido tecnológico
				\4[iv] Fuerte relación con comercio intra-empresa
				\4[v] Muy correlacionado con liberalización comercial
				\4[vi] Asimetría por regiones
				\4[] Niveles elevados de CI intra-industrial
				\4[] $\to$ Norteamérica, Europa, Australia, Sureste Asiático
				\4[] Niveles muy bajos de comercio intra-industrial
				\4[] $\to$ África
		\2 Comercio intra-empresa\footnote{Ver World Bank (2017).}
			\3 Idea clave
				\4 Concepto
				\4[] Comercio entre empresas relacionadas
				\4[] $\to$ Grado variable de partipación/control
				\4 Medición
				\4[] Muy difícil
				\4[] Precios de transferencia
				\4[] $\to$ Distorsionan fuertemente la medición
				\4[] Datos del Censo de los Estados Unidos
				\4 Fundamentos teóricos
				\4[] Internalización de procesos productivos
				\4[] $\to$ Si costes menores que a través de mercado\footnote{Coase (1934)}
				\4[] $\to$ Si cumplimiento de contratos incompletos costoso\footnote{Williamson (1985), Grossman y Hart (1986).}
				\4[] Razones para comerciar con empresas independientes\footnote{En inglés, habitualmente denominado \textit{arm's-length trade}.}
				\4[] $\to$ Acceder a mercados sin transferir tecnología
				\4[] $\to$ Mercados de exportación similares a domésticos
			\3 Hechos estilizados
				\4[i] Comercio entre vinculados es relevante
				\4[] Sin datos a nivel global
				\4[] Sí existen datos para Estados Unidos
				\4[] $\to$ $\sim 30\%$ de exportaciones son intraempresa
				\4[] $\to$ Más para exportaciones a economías avanzadas
				\4[] $\to$ Menos a emergentes y PEDs
				\4[] Datos dispersos para otros países
				\4[ii] Comercio intra-empresa menos volátil
				\4[] Empresas internacionalmente integradas
				\4[] $\to$ Más grandes
				\4[] $\to$ Más productivas
				\4[] $\to$ Más intensivas en capital físico y humano
				\4[] $\to$ Gestionan mejor los stocks
				\4[] Empresas exportadoras no internacionales
				\4[] $\to$ Mayor probabilidad de salir de mercados exteriores
				\4[] $\to$ Gestión deficiente de stocks
				\4[iii] Heterogeneidad entre sectores
				\4[] Sector manufacturero
				\4[] $\to$ Mayor peso de comercio intraempresa
				\4[] $\to$ Especialmente automóviles, medicamentos, transporte
				\4[iv] Comercio intra-empresa de servicios muy opaco
				\4[v] PEDs comercian menos intra-empresa
				\4[] Menor peso de multinacionales es determinante
				\4[vi] Correlación habitual con integración en CVG
		\2 CVGs -- Cadenas de valor global\footnote{Ver CECO Nuevo, WTO explanatory notes, Banco de España (2017), Timmer et al. (2014) en JEP.}
			\3 Idea clave
				\4 Concepto
				\4[] Estructura de producción fragmentada
				\4[] $\to$ En diferentes fases
				\4[] $\to$ Localizadas en diferentes países
				\4[] Conjunto de actividades o etapas de producción
				\4[] $\to$ Realizadas en distintos países
				\4[] $\to$ Requeridas para la elaboración de ByS finales
				\4 Justificación
				\4[] División

Trump’s reelection effort, including the Republican National Committee, has spent more than $800 million so far, while Biden and the Democrats have spent about $414 million through July, according to campaign spending reports. But Trump’s team has also gone dark on the airwaves for stretches as the general election has heated up, raising questions as to whether it was short on cash.del trabajo permite especialización
				\4[] Especialización mejora productividad
				\4 Dificultades de medición
				\4[] Estadísticas oficiales de comercio
				\4[] $\to$ Por defecto, en términos brutos
				\4[] $\to$ No descuentan valor de bienes intermedios
				\4[] $\to$ Contenido importador de exportaciones no cuantificado
				\4[] Tablas I-O globales\footnote{Especialmente relevante el proyecto WIOD / \textit{World Input-Output Database}, cuya última edición disponible data de 2016 y cubre el periodo 2000-2014 \url{wiod.org}}
				\4[] $\to$ Permiten identificar factores subyacentes
				\4[] $\to$ Estimar valor añadido importado
				\4[] $\to$ Estimar valor añadido exportado
				\4[] Base de datos TiVA de OCDE y WTO
			\3 Hechos estilizados
				\4[i] Participación forward mayor con PIBpc bajo
				\4[] Valor añadido producido nacionalmente
				\4[] $\to$ Enviado a terceras economías
				\4[ii]  Participación backward mayor con PIBpc alto
				\4[] Valor añadido importado de terceros
				\4[] $\to$ Utilizado en producción de nuevos bienes
				\4[] $\then$ Dedicados a consumo final
				\4[] $\then$ Reexportado a terceros
				\4[iii] Interconexión creciente entre comercio de ByS
				\4[iv] Especialización creciente en fases de producción
				\4[] Aumenta comercio intraindustrial
				\4[] Reduce valor añadido aportado en cada fase
				\4[v] Tendencia reciente a regionalización de CVGs
				\4[] Resultado de:
				\4[] $\to$ Mayor tamaño de bloques
				\4[] $\to$ Tensiones comerciales recientes
				\4[vi] Relacionadas con internacionalización de PYMES
				\4[vii] + \% industria en VAB, + participación en GVC
				\4[] Sectores industriales más participación forward
				\4[viii] Maduración de CVGs en últimos años
				\4[] In-shoring en otros
				\4[] $\to$ Vuelta a país de origen
				\4[] $\to$ Concentración
				\4[] Crisis covid plantea dudas
				\4[] $\to$ ¿Es óptimo CVG a nivel global?
				\4[] $\to$ ¿Producción JIT es demasiado arriesgada?
	\1 \marcar{Flujos financieros}\footnote{Ver Lane y Milesi-Ferretti (2017) en carpeta del tema y UNCTAD (2019) World Investment Report.}
		\2 Idea clave
			\3 Contexto
				\4 Desarrollo telecomunicaciones
				\4[] Posible obligarse internacionalmente de forma rápida
				\4 Crisis financieras recurrentes
				\4[] Ajustes bruscos de posiciones netas
				\4[] $\to$ Desaparece acceso a financiación
				\4[] $\to$ Aumento de costes de financiación
				\4 Alternancia expansiones--estancamiento
				\4[] Fases de gran aumento de flujos
				\4[] Seguidas de caídas de flujos
				\4[] $\to$ + Reducción de posiciones
			\3 Objetivo
				\4 Caracterizar tendencias históricas
				\4 Derivar hechos estilizados de flujos financieros
				\4 Plantear situación actual
			\3 Resultados
				\4 Aún presente reducción relativa a pre-crisis
				\4 Composición de flujos financieros
				\4[] Influenciada por cambios en estructura CVGs
				\4[] $\to$ IDE gana peso
				\4[] $\to$ Complejidad creciente de stocks
				\4 Peso creciente de centros financieros internacionales
				\4[] Dificulta estudio
				\4[] Camufla/distorsiona verdadera propiedad
		\2 Hechos estilizados
			\3[i] Asimetría en últimas décadas
				\4 Desarrollados
				\4[] $\to$ Largos en equity internacional
				\4[] $\to$ Cortos en deuda internacional
				\4 Emergentes/PEDs
				\4[] $\to$ Largos en deuda internacional
				\4[] $\to$ Cortos en equity
				\4[$\then$] Fuerte asimetría de la composición de flujos
			\3[ii] Volumen relacionado con desarrollo financiero
				\4 Demanda de activos seguros
				\4[] Difícil de suplir si poco desarrollo financiero
			\3[iii] Tendencia creciente del \% a PEDs
				\4 Sujeto a fluctuaciones de años/décadas
				\4 Tendencia de l/p es creciente
				\4 PEDs aumentan madurez de mercados
				\4 Creciente intensidad de K en PEDs
			\3[iv] IDE menos volátiles que otros flujos de K\footnote{Pág. 11 de UNCTAD (2019). }
				\4 ICartera y otra inversión son más volátiles
				\4 ICartera muy sensible a:
				\4[] $\to$  tensiones geopolíticas
				\4[] $\to$ Factores idiosincráticos al país
			\3[v] Flujos de ICartera a PEDs fuertemente concentrados
				\4 Asia recibe mayor parte de flujos
				\4 Resulta de mayor sensibilidad a incertidumbre
			\3[vi] Flujos a PMAs muy pequeños y sobre todo remesas
				\4 Apenas reciben IDE ni cartera
				\4 Sólo 3\% de flujos de ICartera y OInversión
				\4 Remesas son principal fuente de flujos
				\4[] Tendencia creciente desde crisis
				\4[] Casi doblan IDE recibida
			\3[vii] Reducción de PIINs globales respecto  a pre-CFG
				\4 Ligera reducción en términos absolutos
				\4 Fuerte reducción como \% de PIB mundial
				\4 EEUU es principal deudor neto
		\2 IDE\footnote{Ver \url{https://data.imf.org/?sk=B981B4E3-4E58-467E-9B90-9DE0C3367363} y WIR (2019) de UNCTAD.}
			\3 Idea clave
				\4 Concepto de IDE\footnote{Ver \href{https://www.bis.org/publ/cgfs22bde3.pdf}{BIS (2020): definitions of FDI: a methodological note}.}
				\4[] Adquisición de influencia/interés de largo plazo
				\4[] En empresa residente en otra economía
				\4[] Regla heurística de 5º Manual de BP
				\4[] $\to$ Control del 10\% del accionariado
				\4 Cifras globales
				\4[] 1.3 Billones USD en 2018
				\4[] Caída de 13\% respecto 2017
				\4[] Repatriación de capitales hacia Estados Unidos
				\4[] $\to$ por reformas fiscales
				\4[] $\then$ Flujos netos negativos
				\4 Difícil estimación
				\4[] Anuncios de IDE no siempre se materializan
				\4[] IDE planeada se ejecuta en varios periodos
				\4[] Desajustes entre estimación recibida y emitida
				\4[] $\to$ Significativos en la mayoría de los casos
				\4[] $\to$ Cercanos al 5\% habitualmente
				\4 Tendencia a volatilidad relativamente menor
				\4[] Respecto a otras categorías de inversión
				\4 Flujos muy afectados por incertidumbre
				\4 Stocks relativamente estables
				\4 Tendencia débil en última década
				\4[] Flujos inferiores a pre-crisis
				\4[] Claramente inferior a tendencia pre-2008
				\4 Fuerte crecimiento entre 2014 y 2015
				\4 Tendencia decreciente desde 2015 hasta 2018
				\4[] Especialmente en desarrollados
				\4[] Flujos negativos muy relevantes
				\4 Tensiones comerciales siguen siendo relevantes
				\4 Inflexión al alza en greenfield de manufacturas
				\4[] En 2018, greenfield manuf. crecen fuertemente
				\4 Sectores que más IDE reciben
				\4[] Servicios y manufacturas similares
				\4[] $\to$ Ligeras fluctuaciones
				\4[] Sector primario claramente inferior
			\3 Tipos de IDE
				\4 Brownfield vs greenfield
				\4[] Brownfield
				\4[] $\to$ Adquisición de planta ya existente
				\4[] $\to$ Planta existente continúa su actividad anterior
				\4[] Greenfield
				\4[] $\to$ Construcción de nueva planta
				\4[] $\to$ Planta antigua que cambia actividad tras inversión
				\4 Vertical vs horizontal
				\4[] Vertical
				\4[] $\to$ División del proceso productivo en fases
				\4[] $\to$ Plantas extranjeras llevan a cabo sólo una fase
				\4[] $\to$ Producción destinada a reexportación
				\4[] $\then$ Aprovechar ventajas competitivas en determinado segmento
				\4[] Horizontal
				\4[] $\to$ Replicación de proceso productivo en otro país
				\4[] $\to$ Evitar costes de transporte/aranceles
			\3 Hechos estilizados\footnote{Ver Antràs y Yeaple (2014) pág. 59 y ss..}
				\4[i] Bidireccional y norte-norte prevalece
				\4[] PEDs también reciben pero menor proporción
				\4[] Países más ricos:
				\4[] $\to$ Emiten más IDE
				\4[] $\to$ Reciben algo más de IDE que pobres
				\4[] Países más pobres:
				\4[] $\to$ Emiten menos IDE
				\4[] $\to$ Reciben más IDE que emiten
				\4[] $\to$ Más propensos a recibir que a emitir
				\4[] Mayor parte de la IDE es entre desarrollados
				\4[ii] Predominan IDE intraindustria intensiva en K e I+D
				\4[] Cuanto menos intensivo en K e I+D
				\4[] $\to$ Más probabilidad de no IDE
				\4[] $\then$ Comercio con empresas afiliadas, no filiales
				\4[] Sectores intensivos en K e I+D
				\4[] $\to$ Prefieren IDE a comerciar con afiliados
				\4[] $\then$ Probable debido a especificidad de activos
				\4[iii] Menor caída con distancia que con comercio
				\4[] Ec. de gravedad explica bien comercio e IDE
				\4[] $\to$ Relación entre comercio/IDE-distancia
				\4[] Caída de IDE respecto distancia
				\4[] $\to$ Menor a caída respecto distancia
				\4[iv] Multinacionales son más grandes que competidores no MN
				\4[] Más:
				\4[] $\to$ Grandes
				\4[] $\to$ Productivas
				\4[] $\to$ Intensivas en I+D
				\4[] $\to$ Exportadoras
				\4[v] MN matriz especializadas I+D, filiales en venta
				\4[] Matrices de MN se especializan en I+D
				\4[] Filiales se especializan sobre todo en venta
				\4[] $\to$ A mercado propio
				\4[] $\to$ A mercados extranjeros
				\4[vi] F\&A son \% elevado de IDE y entrada en desarrollados
				\4[] Gran parte de IDE corresponde a F\&A
				\4[] Especialmente importante para entrar en desarrollados
				\4[] $\to$ En PEDs, F\&A es \% menor
			\3 Estructura geográfica\footnote{Corresponde al último WIR disponible de la UNCTAD, publicado en junio de 2019.}
				\4 A desarrollados
				\4[] Menor volumen desde 2004
				\4[] Flujos negativos en muchos países
				\4[] $\to$ Por repatriación de capitales
				\4[] Caída en Reino Unido por incertidumbre Brexit
				\4[] Flujos a USA también caen
				\4[] Previsible rebote al alza en próximos años
				\4[] $\to$ Efecto de reformas fiscales decaiga
				\4[] $\to$ Flujos negativos se aminoren
				\4 A PEDs
				\4[] 54\% de IDE total
				\4[] $\to$ Récord histórico
				\4[] Relativamente estables
				\4[] Aumentan 2\% en 2018
				\4[] Heterogeneidad regional
				\4[] $\to$ Asia y África reciben récord histórico
				\4[] $\to$ Hispanoamérica y Caribe cae
				\4 Asia
				\4[] Mayor receptor global de IDE
				\4[] Aumentan 4\%
				\4[] Proyectos greenfield aumentan fuertemente
				\4[] Recuperación en greenfield tras caída en 2017
				\4[] Principales receptores
				\4[] $\to$ China, HK, Singapur, India,Indonesia
				\4[] Principales inversores en Asia:
				\4[] $\to$ China, HK, USA, Japón
				\4[] Menos dependiente de extractivas
				\4[] Manufacturas+servicios son motor de inversión
				\4[] Sensible a guerras comerciales
				\4 Norteamérica
				\4[] Relativamente estables
				\4[] Segunda región más receptora
				\4[] $\to$ Tras caída en Europa en 2018
				\4 Europa
				\4[] Fuerte caída en 2018 desde 2017
				\4[] Flujos negativos de gran tamaño en 2018
				\4[] Repatriaciones a EEUU
				\4[] Flujos a UK caen fuertemente
				\4 Hispanoamérica y Caribe
				\4[] Ligeramente inferior a 2017
				\4[] Efecto base en 2017
				\4[] Se mantiene inferior a peak commodities
				\4[] $\to$ Previo a crisis
				\4[] Principales receptores
				\4[] $\to$ Brasil, MÉX, ARG, COL, CHIL
				\4[] $\then$ BRA uno de principales a nivel mundial
				\4[] Principales inversores
				\4[] $\to$ USA
				\4[] $\to$ Holanda y España
				\4[] $\to$ Chile
				\4[] $\to$ Canadá
				\4[] Caribe relevante en offshore
				\4[] Sensible a evolución de China
				\4 África
				\4[] Aumento de $11\%$
				\4[] 46.000 M
				\4[] Heterogeneidad elevada
				\4[] Algunos países muestran flujos negativos
				\4[] IDE destinada a extracción de recursos
				\4[] Destacan Marruecos, Egipto, Sudáfrica, Etiopía
				\4[] Francia, Holanda, USA, principales inversores
				\4[] Dependiente de precios de commodities
				\4 Países: mayores receptores de flujos de IDE
				\4[] 1. Estados Unidos
				\4[] 2. China
				\4[] 3. Hong Kong
				\4[] 4. Singapur
				\4[] 5. Países Bajos
				\4[] 6. Reino Unido
				\4[] 7. Brasil
				\4[] 8. Australia
				\4[] 9. España
				\4[] $\to$ Fuerte aumento en ESP de 2017 a 2018
			\3 Estructura sectorial
				\4 Sector primario
				\4[] Algunos mercados restringen muy fuertemente
				\4 Sector secundario
				\4[] Muy concentrada regionalmente
				\4[] Sudeste asiático es centro global
				\4 Sector terciario
				\4[] Muy relacionada con capital humano
				\4[] Concentrada en grandes ciudades
				\4 Greenfield
				\4[] Ligeramente superior a 50\% de IDE
				\4[] $\to$ Supera a brownfield en 2018 respecto 2017
				\4[] Anuncios de proyectos aumentan fuertemente
				\4[] $\to$ Más en tamaño que en número de proyectos
				\4[] Especialmente en construcción y extractivas
				\4[$\to$] Principales sectores en greenfield
				\4[] 1.Químicos
				\4[] 2. Servicios a empresas
				\4[] 3. Servicios financieros y seguros
				\4[] 4. IT y telecomunicaciones
				\4[] 5. Alimentación, bebidas y tabaco
				\4 F\&A y brownfield
				\4[] Ligeramente superior al 60\% de total
				\4[] Proyectos de gran tamaño aumentan
				\4[] $\to$ Especialmente en química y servicios
				\4[] Número de proyectos cae ligeramente
				\4[$\to$] Principales subsectores en brownfield
				\4[] 1. Construcción
				\4[] 2. Utilities
				\4[] 3. Refino de petróleo y derivados
				\4[] 4. Servicios a empresas
				\4[] 5. Automóviles y transporte
		\2 Cartera\footnote{Ver págs. 11 y ss. de UNCTAD (2019).}
			\3 Idea clave
				\4 Más difícil medición que IDE
				\4 Volatilidad dentro del año
				\4[] ¿En qué momento contabilizar?
				\4[] ¿Contabilizar toda el capital que ha entrado?
				\4[] ¿Sólo el que no ha salido?
				\4[] $\to$ Diferentes criterios
				\4 Tendencia hacia desintermediación en deuda
				\4[] A lo largo de últimas décadas
				\4[] $\to$ Menor intermediación bancaria
				\4[] Cambio relativo progresivo
				\4[] $\to$ Menos deuda bancaria internacional
				\4[] $\to$ Más títulos de deuda internacional
				\4[$\then$] Aumentan stocks ICartera respecto OInversión
			\3 Estructura geográfica\footnote{Ver \url{https://data.worldbank.org/indicator/BN.KLT.PTXL.CD?most_recent_value_desc=true\&view=map}}
				\4 Mayores inversores netos positivos en 2018
				\4[] (Salidas de capital)
				\4[] 1. Italia
				\4[] 2. Alemania
				\4[] 3. Japón
				\4[] 4. Hong Kong
				\4[] 5. Holanda
				\4[] 6. Dinamarca
				\4[] 7. Corea del Sur
				\4[] 8. Noruega
				\4[] 9. España
				\4 Mayores inversores netos negativos en 2018
				\4[] (Entrada de capital)
				\4[] 1. Reino Unido
				\4[] 2. Luxemburgo
				\4[] 3. China
				\4[] 4. Finlandia
				\4[] 5. Nigeria
				\4[] 6. Suecia
				\4[] 7. Indonesia
			\3 Estructura sectorial
				\4 Sector financiero+deuda pública predomina
				\4 ICartera en equity relativamente pequeña
		\2 Políticas implementadas\footnote{Ver WIR de UNCTAD (2019)}
			\3 Idea clave
				\4 Flujos financieros muy sensibles a políticas
				\4 Potencial desestabilizador/nuevos desequilibrios
				\4 También catalizador de desarrollo
			\3 Liberalización vs restricción
			\3 IIAs -- Acuerdos de inversión internacional
				\4 Concepto
				\4[] Acuerdos internacionales
				\4[] Proteger inversores extranjeros
				\4[] $\to$ Expropiación
				\4[] $\to$ Trato discriminatorio
				\4[] Proveer seguridad jurídica
				\4[] Establecer marcos favorables a inversión
				\4[] 2300 acuerdos bilaterales en vigor
				\4 Variantes
				\4[] Acuerdos específicos de inversión
				\4[] Provisiones sobre inversión en otros acuerdos
				\4 Explosión de IIAs a mediados noventa
				\4 Tendencia decreciente desde entonces
				\4 Acuerdos bilaterales especialmente relevantes
				\4 Acuerdo ACP-UE expira en 2020
				\4[] Negociandose otro
				\4 ACAIA\footnote{ASEAN Comprehensive Investment Agreement} de ASEAN
				\4[] En vigor desde 2012
				\4[] Enmendado en 2019
				\4[] Prohibe exigir rendimientos a inversores
				\4 Unión Europea
				\4[] Acuerdos de inversión bilateral intra-UE
				\4[] $\to$ Derogados a partir de diciembre 2019
			\3 Zonas Económicas Especiales\footnote{Capítulo IV de UNCTAD (2019).}
				\4 Concepto
				\4[] Áreas geográficas definidas
				\4[] Sujetas a regímenes legales diferenciados
				\4[] $\to$ Facilitar inversión
				\4[] $\to$ Barreras al comercio reducidas
				\4[] $\then$ Fomentar desarrollo económico
				\4 Número
				\4[] 5.400 en la actualidad (2018-2019)
				\4[] Fuerte crecimiento en últimos 5 años
	\1 \marcar{Bloques comerciales y áreas emergentes}
		\2 Idea clave
			\3 Concepto
		\2 Europa
			\3 UE
			\3 EFTA
			\3 EEE
		\2 Norteamérica
			\3 NAFTA
				\4 Firmada en 1994
				\4 Mayor área de libre comercio del mundo
				\4 UE primer socio comercial
				\4 Déficit comercial con China persitente
				\4 Exportaciones agrícolas importantes
				\4 Desarrollo industrial en norte de México
				\4[] Importante catalizador
				\4[] Maquiladoras
				\4[] Provisión de componentes de automóviles
			\3 USMCA -- US-México-Canada Agreement
				\4 Renegociación de NAFTA aprobada en 2019
				\4 Reescritura de $\sim 60\%$ de NAFTA
				\4 Protección de prop. intelectual
				\4 Facilitación de comercio en fronteras
				\4 Nuevas normas de origen de importaciones
				\4[] Especialmente relevantes en automóviles
				\4[] $\to$ $>70\%$ de acero debe ser USMCA para trato favorable
				\4 Regulación laboral
				\4[] Mecanismo de resolución de disputas
				\4[] Especialmente relevante para México
				\4 Mecanismo de resolución de disputas
				\4[] Poco utilizado en NAFTA
				\4[] Fuertes debilidades
				\4[] MRDisputas de OMC prevalece
				\4[] Intento por mejorar y clarificar reglas
				\4 Farmacéuticos
				\4[] Reduce protección a industria
				\4[] Especialmente en términos de Prop. intelectual
				\4 Medio ambiente
				\4[] Gestión de residuos tratada explícitamente
				\4[] Sin referencias al cambio climático
				\4 Necesaria ratificación por 3 países
				\4 Entrada en vigor prevista en 2020
		\2 Latinoamérica y Caribe
			\3 Idea clave
				\4 Importante estabilización en últimos 20 años
				\4 Ciclo de commodities esencial
				\4 Factores a favor de la formación de bloques
				\4[] Cultura y lengua comunes
				\4[] Intereses ofensivos y defensivos similares
				\4 Factores en contra
				\4[] Grados de desarrollo heterogéneos
				\4[] Tamaño muy heterogéneo
				\4[] Inestabilidad política interna
				\4[] Dependencia de commodities
			\3 Mercosur
				\4 Establecido en 1991
				\4 Disputas internas frecuentes
				\4 Reducción de intercambios intrarregionales
			\3 Alianza del Pacífico
				\4 40\% del PIB de Latinoamérica
				\4 Hasta 2018, mejores perspectivas que Mercosur
				\4 Incertidumbre regional en últimos años
				\4[] Chile, Colombia, Perú, México
				\4[] Otras incorporaciones posibles
		\2 Asia
			\3 Idea clave
				\4 Región más dinámica del mundo
				\4 Mayor expansión prevista a medio plazo
				\4 Centro de gravedad del comercio mundial
				\4 Comercio intra-regional amplio margen de crecimiento
				\4 Integración intra-regional y global
			\3 ASEAN
				\4 Establecido en 1967
				\4 Bloque comercial del sureste asiático
				\4 Poca integración entre miembros
				\4 Muy abierto al exterior
				\4 PIB combinado 2.6 B USD españoles
				\4[] Mayor que India
				\4[] Tercer mayor mercado mundial
				\4 Potencia exportadora
				\4 Heterogeneidad entre miembros en integración
				\4[] Laos fortísimamente integrado
				\4[] Camboya apenas comercia con vecinos
				\4 Palanca de poder frente a China
				\4 Mayor crecimiento de comercio es intra-regional
				\4[] China y Europa le siguen (nuevo comercio)
			\3 AEC -- ASEAN Economic Community
				\4 Proyecto de mercado común en ASEAN
				\4 Propuesto en 2015
				\4 Objetivo para 2025
				\4 Libre movimiento de:
				\4[] Bienes
				\4[] Servicios
				\4[] Trabajo
				\4[] Inversiones
				\4[] Capital
				\4 Avances lentos y desiguales
				\4[] Aunque ciertos avances tangibles
				\4 ASEAN Single Window
				\4 Esencial para aumentar resistencia a shocks
				\4 Reducir barreras no arancelarias
			\3 TPP
				\4 Trans-Pacific Parnership
				\4 Reducir dependencia de China en Pacífico
				\4 Aumentar influencia de EEUU
				\4 Firmado en 2016
				\4 Sin efecto por no ratificación
				\4[] Sólo Japón y NZ ratifican
				\4[] Trump 2016 declara voluntad de no ratificar
			\3 CPTPP/TPP11
				\4 11 miembros
				\4[] AUS, BRU, CAN, CHI, JAP
				\4[] MAL, MEX, NZ, PER, SIN
				\4[] VIE
				\4 Firmado en 2018
				\4 Efectivo desde 2018
				\4 Ratificados a enero de 2020
				\4[] 7 de 11
				\4[] Faltan:
				\4[] $\to$ Brunei
				\4[] $\to$ Chile
				\4[] $\to$ Malasia
				\4[] $\to$ Peru
				\4 Sucesor de TPP sin Estados Unidos
				\4[] Idéntico en 2/3 a TPP original
				\4[] $\to$ En momento de abandono de EEUU
				\4 Elimina intereses ofensivos de EEUU
				\4[] Eliminación de protección añadida a copyright
				\4[] Reducción posibilidad de empresas que demandan gobiernos
				\4 En general, más avanzado que RCEP
				\4 Miembros
				\4[] BRU, MAL, SIN, VIE, AUS, CAN, JAP,
				\4[] MEX, NZ, PER
				\4 Estándares mínimos de protección MAmbiente
				\4 Mecanismos de resolución de disputas
				\4 Obligación de informar sobre empresas públicas
				\4[] Controlar ayudas vía empresas públicas
				\4 Comisión del CPTPP
				\4[] Órgano de decisión
				\4[] Creado en 2018
				\4[] Dos encuentros en 2019
				\4 Posibles miembros futuros
				\4[] Estados Unidos
				\4[] Taiwan
				\4[] Reino Unido
				\4[] Colombia
				\4[] Indonesia
				\4[] Corea del Sur
				\4[] Tailandia
			\3 RCEP\footnote{\url{https://www.csis.org/analysis/last-rcep-deal}}
				\4 Propuesto inicialmente por China en 2011
				\4[] Más inclusivo que TPP en términos de miembros
				\4 Prevista firma en 2020
				\4[] A finales de año\footnote{Ver \href{https://www.reuters.com/article/us-asia-trade-rcep/rcep-trade-pact-on-track-for-2020-signing-ministers-idUSKBN23U0U9}{Reuters}.}
				\4 Potencial mayor área de comercio mundial
				\4 15 miembros tras retirada de India
				\4[] ASEAN (10)
				\4[] China
				\4[] Japón
				\4[] Corea del Sur
				\4[] Australia
				\4[] Nueva Zelanda
				\4 Más probable tras retirada de TPP
				\4[] Y aprobación posterior de CPTPP/TPP11
				\4 Interés de China
				\4[] Reducir influencia americana
				\4 Regulación de:
				\4[] Agricultura
				\4[] Comercio industrial
				\4[] Comercio electrónico
				\4[] Datos digitales
				\4[] Medioambiente
				\4[] Propiedad intelectual
				\4 Posible MSDisputas en inversiones
				\4 No prohíbe aranceles en contenidos digitales
				\4[] A diferencia de CPTPP y USMCA
				\4[] $\to$ Interés ofensivo habitual de EEUU
				\4 Reducción de aranceles
				\4 Acceso a mercado incrementado
				\4 Menor profundidad que CPTPP y USMCA
				\4 Reglas de origen comunes para todo el bloque
				\4[] Un sólo certificado de origen común
		\2 África
			\3 Tripartite FTA
				\4 Propuesta en seno de UAfricana en 2015
				\4 FTA entre tres partes
				\4[] COMESA\footnote{Common Market for Eastern and Southern Africa}
				\4[] SADC\footnote{Southern African Development Community}
				\4[] EAC\footnote{East African Community}
			\3 AfCFTA
				\4 African Continental Free Trade Area
				\4 Iniciativa de Unión Africana
				\4 África como bloque comercial
				\4 Impulsar comercio intra-regional
				\4[] Actualmente, muy reducido
				\4[] Muy por debajo de resto del mundo
				\4 Mayor área en número de miembros
				\4[] 44 signatarios
				\4[] 22 han ratificado
				\4 Eliminar aranceles en 90\% de bienes
				\4 Efectivo desde 2019
				\4 Oposición de Nigeria
				\4[] Firma pero no ratifica
	\1 \marcar{Guerra comercial}
		\2 Causas
			\3 Crecimiento de emergentes y problema ``del recién llegado''\footnote{Latecomer's problem.}
			\3 Dólar como moneda de reserva global
			\3 Represalias
		\2 Consecuencias
			\3 Sistema multilateral de comercio internacional
			\3 Costes
			\3 Incertidumbre
		\2 Perspectivas
			\3 Propuestas de reforma del sistema multilateral
				\4 Ver ``Understanding trade wars'' en el libro de VoxEU.org
			\3 Bilateralismo
			\3 Regionalización de CVGs
			\3 Acuerdo EEUU--China de 2019
			\3 Trilema de Rodrik
				\4 Rodrik (2000)\footnote{JEP Winter 2000.}
				\4 Restricción empírica postulada
				\4 Economías abiertas deben elegir 2 de 3:
				\4[I] Integración económica
				\4[II] Democracia
				\4[III] Soberanía nacional
				\4 Tres alternativas:
				\4[A] Camisa de fuerza de oro
				\4[] Integración económica+Estado nación soberano
				\4[] Sin transferencias fiscales entre estados
				\4[] Flujos de capital y comerciales libres
				\4[] Mercados internacionales limitan PEconómica nacional
				\4[] Sólo se proveen BPúblicos compatibles con MFinancieros
				\4[] Necesarias políticas autoritarias/represivas
				\4[] $\to$ Ante crisis de deuda/balanza de pagos
				\4[B] Federalismo supranacional
				\4[] Integración económica+democracia
				\4[] Apertura comercial y financiera plena
				\4[] Estados nación pierden soberanía
				\4[] $\to$ Entidad supranacional asume soberanía
				\4[] Transferencias fiscales entre estados
				\4[] $\to$ Posibles déficits exteriores y fiscales
				\4[] Democracia a nivel supranacional
				\4[] $\to$ Entidad supranacional se convierte en nación
				\4[C] Compromiso à la Bretton Woods
				\4[] Democracia+soberanía nacional
				\4[] Sin plena integración comercial+financiera
				\4[] Barreras a movimiento de capital generalizados
				\4[] Estados pueden evitar endeudamiento exterior
				\4[] Posible provisión democrática de bienes públicos
				\4[] $\to$ En la medida en que permita cap. productiva nacional
				\4[] $\to$ Como lo decidan votantes/responsable soberano
				\4[] Sin transmisión de soberanía a ent. supranacional
			\3 ``Decoupling'' China-Estados
	\1[] \marcar{Conclusión}
		\2 Recapitulación
			\3 Flujos comerciales y financieros
			\3 Bloques comerciales y áreas emergentes
			\3 Guerra comercial
		\2 Idea final
			\3 Regionalización de CVGs
			\3 Incertidumbre global
			\3 Acuerdos cada vez más profundos
			\3 Sistema multilateral de comercio debilitado
\end{esquemal}

\preguntas

\seccion{Test 2017}

\textbf{38.} En relación a la inversión internacional:

\begin{itemize}
	\item[a] La principal economía receptora de inversión extranjera directa en 2016 fue China seguida de Estados Unidos.
	\item[b] En su Informe sobre las inversiones en el Mundo 2017, la UNCTAD propone políticas de inversión que fortalezcan las estrategias de desarrollo digital.
	\item[c] El Tribunal Multilateral de Inversiones con sede en Viena se creó para aplicar el Acuerdo sobre Medidas Comerciales relacionadas con la Inversión (TRIMS).
	\item[d] El Grupo de Trabajo de Comercio e Inversión creado en 1996 tras la Conferencia Ministerial de Singapur negocia actualmente nuevas reglas de aplicación multilateral.
\end{itemize}

\seccion{Test 2008}
\textbf{35.} La crisis de la Asociación Latinoamericana de Libre Comercio (ALALC) motivó la aparición de:
\begin{itemize}
	\item[a] La Asociación Latinoamericana de Integración (ALADI).
	\item[b] El Mercado Común Centroamericano (MCCA).
	\item[c] El Banco Interamericano de Desarrollo (BID).
	\item[d] Todas son verdaderas.
\end{itemize}

\seccion{Test 2005}
\textbf{38.} En el ámbito de la integración económica americana señale cuál de estas afirmaciones es \textbf{FALSA}:
\begin{itemize}
	\item[a] La Comunidad Andina (CAN) está constituida por Colombia, Perú, Ecuador y Venezuela.
	\item[b] Los países miembros fundadores del Mercado Común del Sur (MERCOSUR) son Argentina, Uruguay, Paraguay y Brasil.
	\item[c] El Área de Libre Comercio de las Américas (ALCA) o Free Trade Area of the Americas (FTAA) tiene como objetivo unir las economías del continente en una sola zona de libre comercio.
	\item[d] El Tratado de Libre Comercio de América del Norte (TLCAN) o North American Free Trade Area (NAFTA) es una zona de libre comercio constituida por Canadá, Estados Unidos y México.
\end{itemize}

\notas

\textbf{2017:} \textbf{38.} B

\textbf{2008:} \textbf{35.} A

\textbf{2005:} \textbf{38.} A

\bibliografia

Mirar en Palgrave:
\begin{itemize}
	\item
\end{itemize}

Calì, M. (2018) \textit{The impact of the US-China trade war on East Asia} VoxEU CEPR Policy Portal -- \url{https://voxeu.org/article/impact-us-china-trade-war-east-asia}

CEPR Press (2019) \textit{Trade War. The Clash of Economic Systems Endangering Global Prosperity} 

European Commission, DG Trade. (2019) \textit{Statistical Guide} July 2019 -- En carpeta del tema

IMF (2019) \textit{External Sector Report. The Dynamics of External Adjustment} Julio de 2019 \url{https://www.imf.org/en/Publications/SPROLLs/External-Sector-Reports} -- En carpeta del tema

OECD (2010) \textit{Measuring Globalisation: OECD Economic Globalisation Indicators 2010} OECD Economic Globalisation Indicators \url{https://www.oecd-ilibrary.org/docserver/9789264084360-en.pdf?expires=1579783272&id=id&accname=guest&checksum=1328BFE42330496EF9D23279264FA677} -- En carpeta del tema

Rodrik, D. (2000) \textit{How Far Will International Economoic Integration Go} Journal of Economic Perspectives. Winter 2000. -- En carpeta del tema

Timmer, M. P.; Erumban, A. A.; Los, B.; Stehrer, R.; de Vries, G. (2014) \textit{Slicing Up Global Value Chains} Journal of Economic Perspectives. Vol. 28. N. 2

UNCTAD (2019) \textit{World Investment Report} -- En carpeta del tema

World Bank (2017) \textit{Arm's-Length Trade: A Source of Post-Crisis Trade Weakness} Global Economic Prospects June 2017 Special Focus 2 -- En carpeta del tema

WTO (2019) \textit{Overview of developments in the international trading environment} Annual report by the Director-General -- En carpeta del tema

\end{document}
