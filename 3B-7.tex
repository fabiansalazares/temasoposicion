\documentclass{nuevotema}

\tema{3B-7}
\titulo{Política comercial (I): instrumentos y efectos. Barreras arancelarias y no arancelarias. Otros instrumentos tradicionales.}

\begin{document}

\ideaclave

FALTA Productor nacional es monopolista. ¿Efecto de arancel? Se analiza en II.2 Cuotas, pero hay que pensar si debe meterse en I.4 Competencia perfecta

HABLAR DE Aranceles de Trump en 2018, guerra comercial con China, EU, Canadá, México

\seccion{Preguntas clave}

\begin{itemize}
	\item ¿Qué es la política comercial?
	\item ¿Mediante qué instrumentos se implementa la política comercial?
	\item ¿Qué son los aranceles?
	\item ¿Cómo afectan al comercio?
	\item ¿Es posible mejorar el bienestar imponiendo aranceles?
	\item ¿Qué son las barreras no arancelarias?
	\item ¿Qué efectos tienen?
\end{itemize}

\esquemacorto

\begin{esquema}[enumerate]
	\1[] \marcar{Introducción}
		\2 Contextualización
			\3 Evolución del comercio internacional
			\3 Sujetos de análisis de la teoría pura del CI
			\3 Política comercial
		\2 Objeto
			\3 ¿Qué instrumentos conforman la política comercial?
			\3 ¿Qué son los aranceles?
			\3 ¿Cómo afectan al comercio internacional?
			\3 ¿Es posible mejorar el bienestar mediante aranceles?
			\3 ¿Qué son las barreras no arancelarias?
			\3 ¿Qué efectos tienen?
		\2 Estructura
			\3 Aranceles
			\3 Barreras no arancelarias
	\1 \marcar{Aranceles}
		\2 Idea clave
			\3 Contexto
			\3 Objetivos
			\3 Resultados
		\2 Equilibrio parcial
			\3 Idea clave
			\3 País pequeño
			\3 País grande
		\2 Equilibrio general
			\3 Idea clave
			\3 País pequeño
			\3 País grande
			\3 Arancel óptimo
		\2 Competencia imperfecta
			\3 Idea clave
			\3 Monopolio extranjero exportador: Brander y Spencer (1984)
			\3 Diferenciación vertical: Falvey (1981)
			\3 Competencia monopolística
			\3 Política comercial estratégica
		\2 Protección efectiva
			\3 Idea clave
			\3 Formulación
			\3 Implicaciones
			\3 Valoración
		\2 Economía política de los aranceles
			\3 Idea clave
			\3 Stolper-Samuelson
			\3 Redistribución de beneficios del comercio
			\3 Modelo de factores específicos
			\3 Aversión a la pérdida
			\3 Aversión a incertidumbre
			\3 Aversión a desigualdad
			\3 Concentración de intereses
			\3 Instituciones multilaterales pueden catalizar
			\3 Redistribución puede ser necesaria
			\3 Valoración
		\2 Negociación arancelaria
			\3 Idea clave
			\3 Reducción unilateral
			\3 Reducción recíproca
			\3 Reducción global de aranceles
			\3 Implicaciones
			\3 Valoración
	\1 \marcar{Barreras no arancelarias}
		\2 Subvenciones a la exportación
			\3 Idea clave
			\3 Efectos
		\2 Ayudas a I+D
			\3 Idea clave
			\3 Efectos
		\2 Ayudas públicas
			\3 Idea clave
			\3 Efectos
		\2 Cuotas de importación
			\3 Idea clave
			\3 Efectos
			\3 Monopolio nacional
		\2 Contingente arancelario
			\3 Idea clave
		\2 Impuesto a la exportación
			\3 Idea clave
		\2 Restricción voluntaria de exportaciones
			\3 Idea clave
			\3 Efectos
		\2 Ayuda ligada
			\3 Idea clave
			\3 Efectos
		\2 Ajustes fiscales en frontera
			\3 Idea clave
			\3 Efectos
		\2 Embargos
			\3 Idea clave
			\3 Efectos
	\1 \marcar{Conclusión}
		\2 Recapitulación
			\3 Aranceles
			\3 Barreras no arancelarias
		\2 Idea final
			\3 Economía política
			\3 Política comercial estratégica
			\3 Impacto macroeconómico
			\3 Política comercial en la UE

\end{esquema}

\esquemalargo






















\begin{esquemal}
	\1[] \marcar{Introducción}
		\2 Contextualización
			\3 Evolución del comercio internacional
				\4 Explosión en últimos siglos
				\4[] $\to$ Y más aún desde post 2GM
				\4 Avance tecnológico:
				\4[] $\downarrow$ de costes de transporte
				\4[] $\downarrow$ de costes informacionales
				\4 Sujeto de estudio relativamente antiguo:
				\4[] $\to$ Smith, Ricardo, Mill
				\4[] Ligado a la evolución de:
				\4[] $\to$ teoría económica
				\4[] $\to$ hallazgos empíricos
			\3 Sujetos de análisis de la teoría pura del CI
				\4 Patrón de comercio
				\4[] Qué ByS intercambian los países
				\4 Relación real de intercambio
				\4[] A qué precios intercambian los ByS
				\4 Intervención pública en el comercio internacional
				\4[] Qué efectos positivos y negativos tiene
				\4[] Cómo pueden aumentarse los beneficios del CI
				\4[] Cómo pueden mitigarse los costes del CI
			\3 Política comercial
				\4 Gobiernos regulan CI de distintas formas
				\4[] Interviniendo precios relativos
				\4[] Restringiendo volúmenes de comercio
				\4[] Regulando procedimiento de intercambio
				\4[] Regulando prácticas de empresas exportadoras
				\4[] Comerciando directamente
				\4[$\then$] Política comercial
		\2 Objeto
			\3 ¿Qué instrumentos conforman la política comercial?
			\3 ¿Qué son los aranceles?
			\3 ¿Cómo afectan al comercio internacional?
			\3 ¿Es posible mejorar el bienestar mediante aranceles?
			\3 ¿Qué son las barreras no arancelarias?
			\3 ¿Qué efectos tienen?
		\2 Estructura
			\3 Aranceles
			\3 Barreras no arancelarias
	\1 \marcar{Aranceles}
		\2 Idea clave
			\3 Contexto
				\4 Concepto de arancel
				\4[] Impuesto indirecto
				\4[]  Aplicado a intercambio de ByS en CI:
				\4[] $\to$ Exportaciones
				\4[] $\to$ Importaciones
				\4[] $\to$ Mercancías que transitan por la jurisdicción
			\3 Objetivos
				\4 Recaudar
				\4[] Ha reducido su importancia con el tiempo
				\4[] Actualmente, objetivo secundario
				\4[] Para UE, todavía importante
				\4[] $\to$ Recurso propio tradicional
				\4[] 12\% del presupuesto
				\4 Afectar precios relativos
				\4[] Aumentar precio de importaciones
				\4[] $\to$ Reducir consumo de variedades extranjeras
				\4[] $\to$ Aumentar demanda de productos domésticos
				\4[] Extraer renta de exportadores extranjeros
				\4[] $\to$ Si elast. de dda. internacional es alta
				\4 Aumentar producción nacional
				\4[] Aumentando demanda de productos domésticos
				\4[] Movilizando ff.pp. nacionales
				\4[] $\then$ Aumentando rentas nacionales
				\4 Proteger industria nacional
				\4[] Si extranjera tiene:
				\4[] $\to$ Poder de mercado
				\4[] $\to$ Ventaja comparativa
				\4 Redistribuir renta nacional
				\4[] De importadores a productores nacionales
				\4 Reducir dependencia exterior
				\4[] Razones de seguridad nacional
				\4[] Economía política internacional
			\3 Resultados
				\4 Arancel específico
				\4[] Cantidad fija a pagar por cada unidad importada
				\4[] $\to$ Independientemente del precio
				\4[] $\then$ No es necesario determinar precio de venta
				\4[] Suiza: sólo aranceles específicos
				\4[] $\to$ Famoso pesaje de aparatos electrónicos
				\4[] $\then$ Arancel al kg de ordenador
				\4 Arancel ad-valorem
				\4[] Porcentaje aplicado sobre precio
				\4[] $\then$ Necesario determinar precio de venta
				\4 Efectos dependientes de:
				\4[] -- Tipo de arancel
				\4[] -- Cuantía del arancel
				\4[] -- Elasticidades
				\4[] -- Costes administrativos
				\4 Enfoques de análisis:
				\4[] Equilibrio parcial y general
				\4[] $\to$ Parcial: otros mercados nacionales constantes
				\4[] $\to$ General: mercados nacionales interdependientes
				\4[] Tamaño del país
				\4[] $\to$ Pequeño: precios mundiales no se ven afectados
				\4[] $\to$ Grande: economía capaz de afectar precios mundiales
				\4 Competencia perfecta vs imperfecta
				\4[] Poder de mercado de productores nacionales
				\4[] $\to$ Oligopolio
				\4[] $\to$ Competencia monopolística
				\4[] $\to$ Competencia perfecta
		\2 Equilibrio parcial
			\3 Idea clave
				\4 Representar único sector
				\4 Precio internacional dado
				\4 Arancel ad-valorem aplicado:
				\4[] Añadido al precio como \%
				\4 Curvas de dda. y oferta
				\4[] Rectas y normales
				\4 Asunciones habituales de eq. parcial
				\4[] Sin efectos renta
				\4[] P.ej. prefs. cuasilineales
				\4 Imposición de un arancel ad-valorem
				\4[] Porcentaje sobre el precio
			\3 País pequeño
				\4 Precio nacional $\uparrow$ de $p$ a $p(1+t)$
				\4[] $\to$ Aumenta oferta nacional a nuevo precio $p(1+t)$
				\4[] $\to$ Cae demanda nacional a nuevo precio $p(1+t)$
				\4[] $\to$ Precio internacional constante a $p$
				\4[] $\to$ Gobierno recauda arancel por importaciones
				\4[] $\then$ $\downarrow$ cantidad importada
				\4[] $\then$ $\downarrow$ excedente de consumidor
				\4[] $\then$ $\uparrow$ beneficio de empresas
				\4[] $\then$ $\uparrow$ Coste de producción
				\4[] $\then$ $\uparrow$ recaudación
				\4[] $\then$ $\uparrow$ empleo nacional
				\4[] $\then$ Pérdida inequívoca de bienestar
				\4[] $\then$ Arancel óptimo es 0
				\4[] \grafica{parcialpequeno}
				\4 Otros efectos negativos externos al modelo
				\4[] Reducción de variedades
				\4[] Costes administrativos de recaudación
				\4 Arancel equivalente de autarquía
				\4[] Existe un arancel que elimina comercio
				\4[] Debe igualar precio nacional
				\4[] $\to$ Con precio que iguala D y S
			\3 País grande
				\4 Precio internacional $\downarrow$ a $p'$
				\4 Precio nacional $\uparrow$ de $p$ a $p'(1+t)$
				\4[] $\to$ Asumiendo que $p'(1+t) > p$
				\4[] $\then$ Oferta no es perfectamente inelástica\footnote{Si la oferta fuese perfectamente inelástica, un aumento de $t$ el precio que enfrentan los productores induciría una caída de $t$ en el precio, de tal manera que repercutiría totalmente en los productores del resto del mundo.}
				\4 Con respecto a arancel en país pequeño:
				\4[] $\then$ Precios nacional e internacional más bajos
				\4[] $\then$ Demanda más alta
				\4[] $\then$ Costes de producción menores
				\4[] $\then$ Recaudación más alta
				\4[] \grafica{parcialgrande}
				\4[$\then$] Extranjero sufre parte de la carga
				\4[$\then$] Posible mejorar bienestar social con arancel
				\4[$\then$] Puede existir arancel óptimo
		\2 Equilibrio general
			\3 Idea clave
				\4 Entender efecto de arancel
				\4[] Sobre conjunto de mercados
				\4 Asumiendo países grandes o pequeños
				\4[] Ambos enfoques posibles\footnote{Valga la pena repetir que la dicotomía entre equilibrio general y parcial es mucho más un gradiente que una línea divisoria.}
				\4 Dos bienes
				\4[] $\to$ X e Y
				\4 Dadas preferencias
				\4[] Importa Y
				\4[] Exporta X
				\4[$\to$] Arancel sobre importación de bien Y
			\3 País pequeño
				\4 Explicación gráfica
				\4[] Ejes representan cantidades de bienes x e y
				\4[] \grafica{generalsmall}
				\4 Precios internacionales constantes
				\4 Precio nacional
				\4[] $\uparrow$ precio relativo de Y
				\4[] $\downarrow$ precio relativo de X
				\4 Efectos
				\4[] Producción
				\4[] $\to$ $\uparrow$ de bien Y (importado)
				\4[] $\to$ $\downarrow$ de bien X (exportado)
				\4[] Consumo
				\4[] $\to$ $\downarrow$ de bien Y (importado)
				\4[] $\to$ $\uparrow$ de bien X (exportado)
				\4[] $\then$ $\downarrow$ Exportación de X
				\4[] $\then$ $\downarrow$ Importación de Y
				\4[] $\then$ Reducción general del comercio
				\4[] $\to$ Decisión tomada con precios nacionales
				\4[] $\then$ Tangente a RP con precios nacionales
				\4[] $\to$ Debe situarse sobre RP a precio mundial
				\4[] Factores de producción
				\4[] $\to$ F.p. intensivo de Y se beneficia
				\4[] $\to$ F.p. intensivo de X se ve perjudicado
			\3 País grande\footnote{Gandolfo, G. Págs. 226 y ss.}
				\4 País afecta a precios mundiales
				\4[] Caso general de eq. general anterior
				\4[] Dos países y dos bienes
				\4[] Explicación gráfica
				\4[] $\to$ Curvas de demandas recíprocas
				\4 Caso normal
				\4[] \grafica{generalgrande}
				\4[] Curvas de demanda recíproca normales
				\4[] $\to$ Estrictamente monótonas
				\4[] Arancel en la gráfica sobre curva cóncava
				\4[] $\to$ Desplazamiento hacia abajo de curva cóncava
				\4[] $\to$ Aceptan exportar menos por = importación
				\4 Efectos
				\4[] RRI
				\4[] $\to$ Mejora para el país que impone arancel
				\4[] $\to$ Importaciones más baratas en términos de bien exportado
				\4[] Precios
				\4[] $\to$ $\uparrow$ Bien importado gravado dentro del país
				\4[] $\then$ Industria nacional de gravado puede aplicar mayores precios
				\4[] $\then$ Industria nacional de gravado se beneficia
				\4[] $\to$ $\downarrow$ Precio mundial de bien importado
				\4[] $\then$ Exportadores mundiales perjudicados
				\4[] $\then$ Consumidores no sufren menor aumento que cuantía del arancel
				\4[] Factores de producción
				\4[] $\to$ $\uparrow$ F.P. del que bien importado es intensivo
				\4 Caso de Metzler\footnote{A veces, ``\textit{anomalía}''.}
				\4[] \grafica{metzler}
				\4[] Arancel mejora relación relativa de intercambio
				\4[] $\to$ Bien exportado se encarece
				\4[] $\to$ Bien importado se abarata
				\4[] Empresas nacionales que compiten con extranjeras pierden
				\4[] $\to$ Bien importado incluido arancel más barato que antes
				\4[] Cae precio nacional de bien gravado
				\4[] $\then$ Arancel perjudica a industria nacional protegida
				\4[] $\then$ Desplazamiento de recursos hacia sector exportador
				\4[] Reacción muy fuerte de oferta de bien importado
				\4[] $\to$ Tan inelástica, que precio cae mucho
				\4[] $\to$ Caída tan fuerte, que más barato aún con arancel
				\4[] Requiere curva de demanda anómala
				\4[] $\to$ En país extranjero que sufre arancel
				\4[] $\to$ Debe ser no-monótona
				\4[] $\to$ A partir de cierto punto, decreciente
				\4[] Efecto del arancel
				\4[] $\to$ Mejora relación real de intercambio
				\4[] $\then$ Precio relativo de exportaciones aumenta
				\4[] $\then$ Se importa más a cambio de menos exportación
				\4[] $\to$ Pero $\downarrow$ precio nacional de industria protegida
				\4[] $\then$ Industria nacional perjudicada por arancel
				\4[] Motivo de la paradoja
				\4[] $\to$ Industria extranjera muy inelástica
				\4[] $\to$ Precio responde fuertemente a $\downarrow$ demanda
				\4[] $\then$ Precios mundiales caen y compensan arancel
				\4 Caso de Lerner
				\4[] \grafica{lerner}
				\4[] Arancel empeora relación relativa de intercambio
				\4[] $\to$ Bien exportado se abarata
				\4[] $\to$ Bien importado se encarece
				\4[] Requiere curva de oferta anómala
				\4[] $\to$ En país que impone arancel
				\4[] $\to$ No monótona
				\4[] $\to$ Decreciente a partir de cierto punto
				\4[] Efecto del arancel
				\4[] $\to$ Empeoramiento de la RRI
				\4[] $\then$ País reduce renta
				\4[] $\then$ Se importa menos a cambio de más
				\4[] $\then$ Se exporta más a cambio de menos
				\4[] $\to$ Efecto ambiguo sobre industria nacional
				\4[] $\then$ Aumenta producción (y empleo)
				\4[] $\then$ Precio nacional depende de elasticidades
				\4[] Motivo de la paradoja
				\4[] $\to$ Demanda muy inelástica de bien importado
				\4[] $\to$ Ingreso de arancel se gasta en bien importado
				\4[] $\then$ Arancel aumenta demanda de importado en efecto neto
				\4[] $\then$ Empeoramiento de RRI en contra de país de arancel
			\3 Arancel óptimo
				\4[] ¿Existe arancel que maximice bienestar?
				\4[] $\to$ ¿Qué RRI maximiza bienestar?
				\4[] $\to$ Análisis normativo de los aranceles
				\4 Efectos contrapuestos de un arancel
				\4[] Reducción del comercio y el consumo
				\4[] $\to$ Negativo
				\4[] Cambio en RRI y protección a ind. nacional
				\4[] $\to$ Positivo
				\4[] ¿Cuál domina?
				\4[] $\to$ Determina signo del arancel óptimo
				\4 Representación gráfica
				\4[] Curvas de demanda recíproca:
				\4[] $\to$ Cuánto aceptan exportar dada importación
				\4[] \grafica{aranceloptimo}
				\4 País grande
				\4[] Bajo curvas de oferta normales
				\4[] $\to$ Aranceles mejoran RRI
				\4[] $\then$ Aranceles sobre importación mejoran bienestar
				\4[] $\then$ Incentivo a implantar arancel
				\4[] $\then$ Guerras comerciales son posibles
				\4 País pequeño
				\4[] Imposible alterar RRI
				\4[] $\to$ Demanda internacional perfectamente elástica
				\4[] $\then$ Curva de demanda recta
				\4[] Bajo curvas de dda. normales
				\4[] $\to$ Arancel protege industria nacional
				\4[] $\to$ Comercio se reduce
				\4[] Cambio en bienestar
				\4[] $\to$ Depende de curvas de indiferencia
				\4[] $\to$ Preferencia por comercio vs renta
				\4[] \grafica{aranceloptimosmallcountry}
		\2 Competencia imperfecta
			\3 Idea clave
				\4 Nuevas explicaciones del CI
				\4[] $\to$ Nuevos análisis del efecto de aranceles
				\4 Contextos distintos a la competencia perfecta
				\4[] Monopolios nacionales
				\4[] Oligopolios con producto homogéneo
				\4[] Diferenciación de producto
				\4[] $\to$ Localizada
				\4[] $\to$ Competencia monopolística
				\4 Conclusiones distintas a análisis tradicional
			\3 Monopolio extranjero exportador: Brander y Spencer (1984)
				\4 Brander y Spencer (1984)
				\4 Un sólo importador/productor
				\4[] ¿Arancel a importación es beneficiosa?
				\4 Resultado:
				\4[] Pendiente de IMg menor que demanda
				\4[] $\to$ Arancel puede mejorar bienestar
				\4 Explicación
				\4[] Arancel reduce demanda
				\4[] $\to$ Monopolista tiene menos poder de mercado
				\4[] $\then$ Extrae menos renta
				\4[] $\then$ Parte de la renta se captura a monopolista
				\4[] $\then$ Consumidores pierden bienestar inequívocamente
			\3 Diferenciación vertical: Falvey (1981)
				\4 Basado en Falvey (1981)
				\4[] VComparativa relativa a calidades
				\4[] Bienes con más calidad requieren f.p. escaso
				\4[] Países con más abundancia de f.p.
				\4[] $\to$ Producen calidades altas
				\4 Dos países:
				\4[] A: especializado en calidades altas
				\4[] B: especializado en calidades bajas
				\4 Arancel en país que produce calidad alta
				\4[] $\to$ Sobre importaciones de calidad baja
				\4[$\then$] A producirá más calidades bajas
				\4[$\then$] Calidad máx. producida en B será peor que sin arancel
				\4 \grafica{diferenciacionvertical}
			\3 Competencia monopolística
				\4 Aranceles afectan:
				\4[] Número de variedades
				\4[] Economías de escala
				\4[] $\to$ Vía índices de precios
				\4 Home-market-effect
				\4[] Aranceles en mercado principal amplifican efecto
				\4[] $\to$ Aumentan producción y exportación
			\3 Política comercial estratégica
				\4 Brander y Spencer (1983) y otros
				\4 Política comercial que afecta a
				\4[] $\to$ Resultado de interacción estratégica
				\4[] $\to$ Entre empresas de distintos países
				\4[$\to$] Tema 3B-8
		\2 Protección efectiva\footnote{Ver \href{http://qed.econ.queensu.ca/faculty/flatters/writings/ff_measuring_impacts_of_trade_policy.pdf}{Flatters (2004)}}
			\3 Idea clave
				\4 Contexto
				\4[] Economías compuestas de múltiples sectores
				\4[] Cuanto más avanzadas, más sectores diferentes
				\4[] Aranceles diferentes en diferentes sectors
				\4[] $\to$ Afectan utilización de inputs en unos y otros
				\4[] Tasa de protección nominal
				\4[] $\to$ Aumento \% de precio nominal
				\4[] Balassa (1965) y Corden (1971a)
				\4[] Bienes producidos nacionalmente
				\4[] $\to$ Requieren inputs extranjeros
				\4[] Inputs extranjeros importados
				\4[] $\to$ También pueden estar sujetos a aranceles
				\4[] $\then$ Industrias downstream sufren protección a upstream
				\4 Objetivos
				\4[] Caracterizar protección a sectores
				\4[] $\to$ Teniendo en cuenta protección a inputs
				\4[] Formular medida de protección real
				\4[] $\to$ Tasa de arancel es protección nominal
				\4[] $\then$ No protección real
				\4[] Considerar otros factores afectan a protección real
				\4[] $\to$ Coste de inputs
				\4[] $\to$ Valor añadido efectivamente creado
				\4[] $\to$ Aranceles aplicados a inputs
				\4 Resultados
				\4[] Concepto de protección efectiva
				\4[] $\to$ Tasa de protección sobre valor añadido producido
				\4[] $\then$ Teniendo en cuenta protección a output
				\4[] $\then$ Teniendo en cuenta protección a inputs
				\4[] $\then$ $\uparrow$ de VA resultado de protección arancelaria
				\4[] Cadenas de valor global
				\4[] $\to$ Mayor número de inputs
				\4[] $\to$ Mayor número de procesos export-import
				\4[] $\then$ Mayor potencial para afectar protección efectiva
				\4[] Relación con análisis input-output
				\4[] $\to$ Estimación de impacto general de arancel en industrias
			\3 Formulación
				\4 Protección nominal arancel ad-valorem
				\4[] Aumento \% sobre precio de venta
				\4[] $\text{P}_\text{Nominal} = \frac{P_D - P_W}{P_W} \cdot 100$
				\4[] Donde:
				\4[] $\to$ $P_D$: precio doméstico
				\4[] $\to$ $P_W$: precio mundial
				\4 Protección efectiva
				\4[] Protección en términos de VA
				\4[] $\to$ Considerando protección a inputs y outputs
				\4[] $\text{P}_\text{Efectiva} \text{en \%} = \frac{\text{VA}_D -  \text{VA}_W}{\text{VA}_W} \cdot 100 $
				\4[] Donde:
				\4[] $\to$ $\text{VA}_D$: VA en mercado doméstico con todos aranceles
				\4[] $\to$ $\text{VA}_D$: VA en mundial sin aranceles
				\4 Ejemplo
				\4[] Output: zapatos
				\4[] $\to$ Precio mundial: 150€
				\4[] Input: cuero
				\4[] $\to$ Coste mundial por unidad de zapato: 100 €
				\4[] Valor añadido de zapatos:
				\4[] $\to$ 150 - 100 = 50€
				\4[] Arancel a zapatos del 20\%
				\4[] $\to$ Precio doméstico: 180 €
				\4[] $\then$ Valor añadido 80 €
				\4[] $\then$ Protección efectiva: $\frac{80 - 50}{50} = 60\%$
				\4[] Arancel a cuero del 20\%
				\4[] $\to$ Coste doméstico: 120 €
				\4[] $\then$ Valor añadido: 150 - 120 = 30 €
				\4[] $\then$ Protección efectiva: $\frac{30 -50}{50} = -40\%$
				\4[] Aranceles a zapatos y a cuero
				\4[] $\to$ Precio doméstico: 180 €
				\4[] $\to$ Coste doméstico: 120 €
				\4[] $\then$ Valor añadido: 180 - 120 = 60 €
				\4[] $\then$ Protección efectiva: $\frac{60 -  50}{50}= 20\%$
			\3 Implicaciones
				\4 Venta en mercado nacional
				\4[] Protección a outputs posible
				\4[] Protección a inputs posible
				\4[] $\then$ Protección outputs compensa inputs
				\4[] $\then$ Posible protección efectiva positiva
				\4 Venta en mercado mundial
				\4[] Protección a outputs no es posible
				\4[] Protección a inputs posible
				\4[] $\to$ Reducción de valor añadido con prot. a inputs
				\4[] $\then$ Imposible protección efectiva positiva
				\4 Equilibrio general
				\4[] Economías compuestas de muchos sectores
				\4[] Interrelación entre inputs y outputs
				\4[] $\to$ Sujetos a aranceles diferentes
				\4[] $\then$ Necesarias tablas input-output para valorar
				\4 Efectos sustitución entre diferentes inputs
				\4[] Anteriormente, $a_{ij}$ asumido constante
				\4[] Realmente, puede no serlo
			\3 Valoración
				\4 Política arancelaria es realmente muy compleja
				\4 Efectos sectoriales no capturados por análisis simple

%				\4 Fórmula simplificada
%				\4[] Asumiendo:
%				\4[] $\to$ un input $i$
%				\4[] $\to$ \% de input no cambia con protección a input
%				\4[] Protección efectiva de bien $j$:
%				\4[] \fbox{$g_j = \frac{t_j-a_{ij} t_i}{1-a_{ij}}$}
%				\4[] $g_j$: protección efectiva de bien $j$
%				\4[] $t_j$: arancel a bien $j$ (+)
%				\4[] $t_i$: arancel a input $i$ (--)
%				\4[] $a_{ij}$: cantidad de $i$ para unidad de $j$ (+)
%				\4[] Estructura arancelaria más compleja
%				\4[] $\to$ Requiere uso de tablas input
%				\4 Ejemplo:
%				\4[] Precio internacional de bien X
%				\4[] $\to$ 100€/unidad
%				\4[] Protección nominal de bien X
%				\4[] $\to$ 50\%
%				\4[] Precio en mercado doméstico de bien X
%				\4[] $\to$ 150 € /unidad
%				\4[] Cantidad de input Y por unidad de bien:
%				\4[] $\to$ 0,5
%				\4[] Precio de input Y por unidad:
%				\4[] $\to$ 100 €/unidad
%				\4[] Coste de input Y por unidad de bien X:
%				\4[] $\to$ 50 €/unidad
%				\4[] Situación sin arancel a input Y
%				\4[] $\to$ Protección nominal:
%				\4[] $\then$ $50\%$
%				\4[] $\to$ Protección efectiva:
%				\4[] $\then$ $\frac{\text{VA}_\text{Proteccion} - \text{VA}}{\text{VA}} = \frac{150 -% 100}{50} = 100\% $
%				\4[] Situación con protección a input Y
%				\4[] $\to$ Protección nominal:
%				\4[] $\then$ $50\%$
%				\4[] $\to$ Protección efectiva:
%				\4[] $\then$ $\frac{\text{VA}_\text{Proteccion} - \text{VA}}{\text{VA}} = \frac{150 -% 100}{50} = 100\% $
%
%				\4[] Sector de telas
%				\4[] Precio de telas = 100
%				\4[] $\to$ Coste de inputs = 50
%				\4[] $\then$ Valor añadido = 40
%				\4[] Introducción de arancel del 20\%
%				\4[] $\to$ Precio post-arancel = 120
%				\4[] $\then$ Valor añadido = 60
%				\4[] Arancel vs protección sobre valor añadido
%				\4[] $\to$ Arancel: $20/100 = 20\%$
%				\4[] $\to$ Valor añadido: $20/40 = 50\%$
%				\4[] $\then$ Protección respecto VA $\uparrow$ + que nominal
%				\4[$\then$] Arancel a inputs determina protección efectiva-output
%			\3 Implicaciones
%				\4 VAñadido negativo y aranceles
%				\4[] VAñadido de producto nacional puede ser negativo
%				\4[] $\to$ Inputs más caros que output nacional
%				\4[] $\then$ Protección efectiva negativa
%				\4[] $\then$ Todo se importaría
%				\4[] $\then$ Sin producción nacional
%				\4[] Arancel a output importado
%				\4[] $\to$ Puede hacer protección positiva positivo
%				\4 EQuilibrio general
%				\4[] Economías compuestas de muchos sectores
%				\4[] Interrelación entre inputs y outputs
%				\4[] $\to$ Sujetos a aranceles diferentes
%				\4[] $\then$ Necesarias tablas input-output para valorar
%				\4 Efectos sustitución entre diferentes inputs
%				\4[] Anteriormente, $a_{ij}$ asumido constante
%				\4[] Realmente, puede no serlo
%			\3 Valoración
%				\4 Política arancelaria es realmente muy compleja
%				\4 Efectos sectoriales no capturados por análisis simple
		\2 Economía política de los aranceles
			\3 Idea clave
				\4 Contexto
				\4[] Economía política
				\4[] $\to$ Análisis de efectos de política económica
				\4[] $\then$ Sobre intereses de diferentes grupos sociales
				\4[] $\then$ Como resultado de intereses de diferentes grupos
				\4[] Efectos de política comercial
				\4[] $\to$ Afectan distinto a diferentes sectores
				\4 Objetivo
				\4[] Caracterizar efectos sobre diferentes sectores
				\4[] Entender impacto de estructura política sobre pol. arancelaria
				\4 Resultados
				\4[] Efectos de aranceles sobre diferentes grupos sociales
				\4[] $\to$ Beneficios y perjuicios
				\4[] $\to$ Diferentes grados de concentración
				\4[] $\to$ Diferente capacidad de respuesta
			\3 Stolper-Samuelson
				\4 En contexto Heckscher-Ohlin
				\4 Tras apertura comercial
				\4[] $\to$ Factor intensivo de sector de especialización
				\4[] $\then$ Aumenta pago al factor
				\4[] $\to$ Factor intensivo de sector que pierde producción
				\4[] $\then$ Coste de factores cae
				\4 Sector de factor intensivo en bien de especialización
				\4[] $\then$ Presión hacia reducción de aranceles
				\4 Sector de factor intensivo en bien que pierde producción
				\4[] $\then$ Presión hacia mantenimiento de aranceles
				\4 Países ricos
				\4[] Abundantes en capital
				\4[] $\to$ Capital gana con apertura
				\4[] Trabajo escaso
				\4[] $\to$ Compite con trabajo extranjero
				\4[] $\to$ Pierde con apertura
				\4[$\then$] Trabajo se opone a apertura
				\4 Países pobres
				\4[] Abundantes en trabajo
				\4[] $\to$ Con apertura venden al mundo
				\4[] Capital escaso
				\4[] $\to$ Compiten con capital extranjero
				\4[$\then$] Trabajo favorable a apertura
			\3 Redistribución de beneficios del comercio
				\4 Permite a perdedores aceptar reducción de aranceles
				\4 Pero costes de redistribución
				\4[] $\to$ Negociación entre sectores
				\4[] $\to$ Votaciones
				\4[] $\to$ Adquisición de información
				\4[] $\then$ Posible no sea rentable redistribuir
			\3 Modelo de factores específicos
				\4 Dos factores de capital inmóviles
				\4 Desarme arancelario mutuo
				\4[] $\to$ Aumenta beneficios nuevos exportadores
				\4[] $\to$ Reduce beneficio en sectores que ahora importan
				\4[] $\then$ Flujo de trabajo de un sector a otro
				\4[] $\then$ Caída de PMgK en sector perjudicado
				\4[] Diferentes intereses dentro de un mismo factor
				\4[] $\to$ Capital vs trabajo no siempre oposición homogénea
			\3 Aversión a la pérdida
				\4 Behavioral economics
				\4[] Empíricamente, aversión a pérdida mayor que ganancia
				\4 Apertura arancelaria
				\4[] $\to$ Induce beneficio en un sector
				\4[] $\to$ Aumenta pérdidas en otro
				\4 Si aversión a pérdida mayor que ganancia por beneficio
				\4[] $\then$ Oposición más fuerte
			\3 Aversión a incertidumbre
				\4 Apertura aumenta incertidumbre
				\4[] $\to$ ¿Efectos de equilibrio general serán positivos?
			\3 Aversión a desigualdad
				\4 Apertura al comercio puede aumentar desigualdad
				\4[] $\to$ Sector de especialización más rico
				\4[] $\to$ Sector que reduce producción más pobre
				\4 Seres humanos muestran cierta aversión a la desigualdad
				\4[] $\to$ Factor de oposición a apertura
			\3 Concentración de intereses
				\4 Efectos de reducción arancelaria
				\4[] $\to$ Difusos sobre consumidores
				\4[] $\to$ Muy concentrados sobre industria desprotegida
				\4 Perjuicio concentrado
				\4[] $\to$ Facilita coordinación entre perjudicados
				\4[] $\then$ Facilita oposición política a apertura
			\3 Instituciones multilaterales pueden catalizar
				\4 Commitment liberalizador
				\4[] Aumenta poder de negociación de liberalizadores
			\3 Redistribución puede ser necesaria
				\4 Mejora aceptación de apertura
				\4[] También es costosa
			\3 Valoración
				\4 Programa de investigación con muchas vertientes
				\4 Interacciones con sociología, ciencia política, demografía..
				\4 Ciencia económica no siempre ha examinado
				\4[] Supuestos demasiado fuertes
				\4[] $\to$ ¿Planificador social?
				\4[] $\to$ ¿Funciones de bienestar social?
				\4[] $\then$ ¿Realmente existen?
				\4[] $\then$ ¿Realmente consideradas en decisiones de PComercial?
		\2 Negociación arancelaria
			\3 Idea clave
				\4 Contexto
				\4[]
				\4 Objetivo
				\4 Resultados
			\3 Reducción unilateral
				\4 Mejora inequívoca en términos de precios de consumo
				\4[] $\to$ Productores más eficientes importan
				\4[] $\to$ Reducción de costes
				\4[] $\to$ Aumento de competencia
				\4[] $\then$ Mejora de capacidad de compra
				\4 Potencial pérdida para productores nacionales
				\4[] $\to$ Salvo casos de Metzler y Lerner
				\4 Potencial empeoramiento de RRI
				\4[] $\to$ Salvo caso de Lerner
				\4[$\then$] Incentivos a negociación sobre aranceles recíprocos
			\3 Reducción recíproca
				\4 Áreas de integración
				\4[] Tratado de preferencias comerciales
				\4[] Áreas de libre comercio
				\4[] Uniones aduaneras
				\4 Teoría de la negociación
				\4[] $\to$ Axiomática: Nash
				\4[] $\to$ Estratégica: Stahl y Rubinstein
				\4 Sujetas a problemas potenciales
				\4[] Desviación de comercio
				\4[] Represalias de no integrados
				\4[] Redistribución dentro de área de integración
			\3 Reducción global de aranceles
				\4 Razón de ser de OMC
				\4 Organizaciones multilaterales/regionales permiten:
				\4[] $\to$ Atarse al mástil frente a intereses domésticos
				\4[] $\to$ Reducir costes de negociación y transacción
			\3 Implicaciones
				\4 Estructura sectorial interna relevante
				\4[] Determina enfoque de reducción arancelaria
				\4[] $\to$ Unilateral o integración comercial
			\3 Valoración
				\4 Fundamento de OMC
				\4[] Marco de reducción generalizada
				\4 Diferentes
	\1 \marcar{Barreras no arancelarias}
		\2 Subvenciones a la exportación
			\3 Idea clave
				\4 Gobierno paga a productores
				\4[] Por cada unidad producida
				\4 Todo tipo de países
				\4[] Desarrollados y en desarrollo
				\4 Objetivos
				\4[] Aumentar producción de exportables
				\4[] Inducir ventaja competitiva
				\4[] Mitigar arancel sufrido por exportadores nacionales
				\4[] $\to$ Equivale a transferencia a extranjero
			\3 Efectos
				\4 Empresas siempre se benefician
				\4[] $\to$ ¿excedente social también es positivo?
				\4 País pequeño
				\4[] Precio que reciben empresas nacionales
				\4[] $\to$ Aumenta en la cuantía de la subvención
				\4[] $\to$ Precio nacional aumenta en cuantía de subvención\footnote{Las empresas nacionales reciben $p+s$ por vender en el extranjero. Si el precio nacional se mantuviese en $p$, preferirían vender toda la producción en el extranjero para recibir $s$ más por unidad vendida. Así, el precio nacional sube hasta $p+s$.}
				\4[] $\then$ Producen más
				\4[] $\then$ Lo venden a mayor precio dentro y fuera
				\4[] \grafica{subvencionexportacion}
				\4 País grande\footnote{Ver págs. 286 y 287 en Feenstra.}
				\4[] Aumento de ingresos
				\4[] $\then$ Aumento de la producción
				\4[] Exp. nacionales enfrentan dda. decreciente
				\4[] $\to$ $\downarrow$ Precio sin subvención
				\4[] $\then$ Empeora RRI nacional
				\4[] $\then$ Subvención no beneficia enteramente a productores
				\4[] A diferencia de aranceles en país grande:
				\4[] $\to$ Pérdida de eficiencia inequívoca
		\2 Ayudas a I+D
			\3 Idea clave
				\4 Contexto
				\4[] Subsidios a exportación
				\4[] $\to$ Prohibidos: acuerdo antisubvención
			\3 Efectos
				\4 Reducción de costes de producción
				\4 Amenaza creíble de aumentar producción
				\4 Coste de los fondos públicos
				\4[] Puede ser superior a beneficio aumentado
				\4 Economía política
				\4[] Grupos beneficiados tratan de aumentar subvención
		\2 Ayudas públicas
			\3 Idea clave
				\4 Contexto
				\4[] Empresas domésticas expuestas a:
				\4[] $\to$ Shocks de demanda
				\4[] $\to$ Competencia extranjera
				\4[] Liberalización de comercio de ByS
				\4[] $\to$ Aplicable a mayor parte de comercio internacional
				\4[] $\to$ Especialmente relevante en Unión Europea
				\4[] Ayudas estatales
				\4[] $\to$ Entrada en capital
				\4[] $\to$ Préstamos de largo plazo
				\4[] $\to$ Provisión de liquidez
				\4 Objetivos
				\4[] Mantener sector exportador en contexto de crisis
				\4[] Evitar descapitalización de sector exportador
				\4 Resultados
				\4[] Mixtos
				\4[] Falseamiento de competencia
				\4[] Permite evitar shock transitorios
			\3 Efectos
				\4 Mantenimiento de K y H
				\4[] Evitar descapitalización de empresas
				\4[] Histéresis en mercado de trabajo
				\4[] Supeditado a carácter transitorio de los shocks
				\4 Distorsión de competencia
				\4[] Países con mejores posiciones fiscales
				\4[] $\to$ Pero empresas menos eficientes
				\4[] $\then$ Pueden capturar cuota de mercado
				\4[] $\then$ Pérdida de eficiencia
		\2 Cuotas de importación
			\3 Idea clave
				\4 Restringir cantidad importada
				\4 Diferentes efectos dependen de estructura de mercado
				\4 Cuotas frente a aranceles
				\4[] En general, aranceles menos distorsión
				\4[] Problema de arancel
				\4[] $\to$ Control impreciso de cantidades
				\4[] $\to$ Necesario conocer elasticidades
			\3 Efectos
				\4 Competencia perfecta
				\4[] Existe cuota equivalente a arancel
				\4[] $\to$ Induce mismo precio de equilibrio
				\4 Competencia imperfecta en mercado doméstico
				\4[] No equivalencia
				\4[] Altera estructura del mercado
				\4[] $\to$ Aparece demanda residual para productores nacionales
				\4[] $\then$ Oportunidad de captura de rentas
				\4 Si estado no captura renta arancelaria
				\4[] $\to$ ¿cómo se reparte?
				\4[] i. Licencias de imp. a empresas nacionales
				\4[] $\to$ Extraen renta arancelaria
				\4[] ii. Licencias a cambio de lobbying
				\4[] $\to$ Llevan a cabo actividades ineficientes
				\4[] $\to$ Excedente se pierde total o parcialmente
				\4[] iii. Licencias subastadas por el gobierno
				\4[] $\to$ Vendidas por cuantía igual a renta
				\4[] $\to$ Gobierno extrae renta
				\4[] iv. País extranjero gestiona licencias\footnote{Habitualmente en el marco de los llamados \textit{Voluntary Export Restraint} o VERs.}
				\4[] $\to$ Acuerdos de restricción de exportaciones
				\4[] $\to$ Reducen incentivos a represaliar
				\4[] $\to$ Sólo se utilizan si industria amenazada
				\4[] $\to$ Acuerdo en Ronda Uruguay '94 para no implementar
			\3 Monopolio nacional
				\4 Situación:
				\4[] Industria nacional monopolística
				\4[] Industria extranjera compite con monopolista nacional
				\4 Introducción de arancel
				\4[] Similar a arancel de competencia perfecta
				\4[] $\to$ Monop. nacional aumenta precio al que vende
				\4[] $\to$ No puede subir más por competencia extranjera
				\4 Introducción de cuota
				\4[] Compite hasta cubrir cuota con extranjera
				\4[] Monopolista absoluto en resto de demanda
				\4[] $\to$ Aumenta poder de monopolio con cuota
		\2 Contingente arancelario\footnote{En inglés \textit{tariff-rate quotas}.}
			\3 Idea clave
				\4 Diferente tasa de arancel
				\4[] En diferentes tramos de importación
				\4 A partir de cierta cantidad importada
				\4[] $\to$ Rige determinado arancel
				\4 Habituales en la actualidad
				\4[] Ronda de Uruguay legitimó
				\4[] $\to$ Fomentar \textit{market-access} en mercados agrícolas
		\2 Impuesto a la exportación
			\3 Idea clave
				\4 Impuesto por unidad exportada
				\4 Objetivo
				\4[] Extraer excedente monopolista de otros mercados
				\4 Útil si mercado nacional competitivo
				\4[] $\to$ ``Coordina'' restricción monopolística
				\4[] $\to$ Extrae excedente en mercado extranjero
				\4 Si productor nacional es monopolista
				\4[] $\to$ Inútil y perjudicial para el país
		\2 Restricción voluntaria de exportaciones
			\3 Idea clave
				\4 Límites a la exportación a un país
				\4 Habituales en los 80:
				\4[] Japón restringe exportaciones a EEUU
				\4 Ventajas respecto a cuotas
				\4[] No incentivan represalias
				\4[] Sólo rentables para doméstico
				\4[] $\to$ Si industria nacional realmente amenazada
				\4 Acuerdo en Ronda de Uruguay para no implementar
			\3 Efectos
				\4 Transfieren renta a exportadores extranjeros
				\4[] Extranjeros venden a precio más alto
				\4[] $\to$ Extraen todo el excedente central
		\2 Ayuda ligada
			\3 Idea clave
				\4 Ayuda a países en desarrollo
				\4[] A cambio de importar de país desarrollados
			\3 Efectos
				\4 Distorsiones similares a subvenciones
				\4 País en desarrollo tiende a importar
				\4[] Bienes con peor relación calidad-precio
		\2 Ajustes fiscales en frontera
			\3 Idea clave
				\4 Impuestos sobre ventas o VAñadido
				\4[] Son arancel a exportación
				\4[] $\to$ Si no se exime a exportaciones
			\3 Efectos
				\4 Corrigen pérdida de competitividad de exp.
				\4 Ejemplo:
				\4[] IVA en destino en UE
				\4[] IVA exento a extracomunitarias
		\2 Embargos
			\3 Idea clave
				\4 Restricción a la exportación a un país
				\4 Motivos de:
				\4[] Seguridad nacional
				\4[] Sanciones a países enemigos
			\3 Efectos
				\4 Reducción de variedades en embargados
				\4 Salida de cadenas de valor globales
	\1 \marcar{Conclusión}
		\2 Recapitulación
			\3 Aranceles
			\3 Barreras no arancelarias
		\2 Idea final
			\3 Economía política
				\4 Krueger (1974)
			\3 Política comercial estratégica
				\4 No tratada en este tema
				\4 Aspecto esencial de la política comercial
				\4[] Políticas que afectan
				\4[] $\to$ A decisiones estratégicas de empresas
			\3 Impacto macroeconómico
				\4 Empleo
				\4[] Fluctuaciones de producción nacional
				\4[] $\to$ Afecta a empleo
				\4 Financiero
				\4[] Déficits y superávits por cuenta corriente
				\4[] $\to$ Requieren financiación o ajuste
				\4[] Política comercial es factor relevante
				\4[] $\to$ Pierde relevancia por integración
				\4 Crecimiento
				\4[] Modelos que relacionan CI y crecimiento
				\4[] $\to$ Presencia de variedades
				\4[] $\to$ Learning-by-doing
				\4[] $\to$ Análisis de industrias nacientes
				\4 Desigualdad
				\4[] Impacto desigual de política comercial
				\4[] $\to$ Diferentes sectores y ff.pp.
				\4[] $\to$ Muy distribuidos o muy concentrados
				\4[] $\then$ Análisis de economía política
			\3 Política comercial en la UE
\end{esquemal}















\graficas

\begin{axis}{4}{Representación del efecto de un arancel a la importación para un país pequeño en un contexto de equilibrio parcial}{}{$P$}{parcialpequeno}
	% eje de abscisas
	\draw[-] (4,0) -- (7,0);
	\node[below] at (7,0){Q};
	% oferta
	\draw[-] (0,0) -- (6,4);
	\node[right] at (6,4){$S$};
	
	% demanda
	\draw[-] (0,4) -- (6,0);
	\node[right] at (6,.5){$D$};
	
	% LIBRE COMERCIO
	% precio de libre comercio
	\draw[dashed] (0,0.75) -- (6,0.75);
	\node[left] at (0,0.75){$p$};
	
	% cantidad producida de libre comercio
	\draw[dashed] (1.12,0.75) -- (1.12,0);
	\node[below] at (1.12,0){\tiny $q_n$};
	
	% cantidad demandada de libre comercio
	\draw[dashed] (4.88,0.75) -- (4.88,0);
	\node[below] at (4.88,0){\tiny $q_m$};
	
	% exceso de demanda de libre comercio
	\draw[decorate,decoration={brace, mirror,amplitude=3pt},xshift=0pt,yshift=-1.1cm] (1.12,0) -- (4.88,0) node[black,midway,xshift=2pt, yshift=-0.33cm] {\tiny Importación de libre comercio};
	
	% POST ARANCEL
	% precio post arancel
	\draw[dashed] (0,1.5) -- (6,1.5);
	\node[left] at (0,1.5){$p \cdot (1+t)$};
	\draw[-{Latex}] (-0.5,0.75) -- (-0.5,1.4);
	
	% cantidad producida post arancel
	\draw[dashed] (2.24,1.5) -- (2.24,0);
	\node[below] at (2.24,0){\tiny $q_n'$};
	\draw[-{Latex}] (1.25,0.35) -- (2.1,0.35);
	
	% cantidad demandada post arancel
	\draw[dashed] (3.75,1.5) -- (3.75,0);
	\node[below] at (3.75,0){\tiny $q_m'$};
	\draw[-{Latex}] (4.8,0.35) -- (3.83,0.35);
	
	% exceso de demanda de libre comercio
	\draw[decorate,decoration={brace, mirror,amplitude=3pt},xshift=0pt,yshift=-0.5cm] (2.24,0) -- (3.75,0) node[black,midway,xshift=2pt, yshift=-0.33cm] {\tiny Importación post-arancel};
	
	% Aumento del excedente del productor
	\draw[white, fill=green, opacity=0.2] (0,0.75) -- (0,1.5) -- (2.24,1.5) -- (1.12,0.75);
	\node[] at (0.75,1.1){A};
	
	% Recaudación del gobierno
	\draw [white, fill=green, opacity=0.2] (2.24,1.5) -- (3.75,1.5) -- (3.75,0.75) -- (2.24,0.75);
	\node[] at (2.05,1.1){B};
	
	% Aumento de los costes de producción
	\draw[white, fill=red, opacity=0.2] (1.12,0.75) -- (2.24,1.5) -- (2.24,0.75);
	\node[] at (3,1.1){C};
	
	% Pérdida irrecuperable de eficiencia
	\draw[white, fill=red, opacity=0.2] (4.88,0.75) -- (3.75,0.75) -- (3.75,1.5);
	\node[] at (4,1.1){D};
\end{axis}

En verde se muestra el excedente del consumidor que no se destruye sino que simplemente cambia de manos. El área A representa el excedente del consumidor que se convierte en beneficio de las empresas nacionales. El área C corresponde a la recaudación que obtiene el gobierno como resultado de la implementación del arancel. Las áreas rojas representan las pérdidas irrecuperables de eficiencia. El área B representa la pérdida derivada del aumento de costes. El área D resulta de la reducción del consumo derivada del aumento del precio de oferta. Esta reducción del excedente del consumidor no resulta en un aumento de la recaudación ni en un aumento del excedente del productor por lo que se convierte en una pérdida irrecuperable de eficiencia.

\begin{axis}{4}{Representación del efecto de un arancel a la importación en un contexto de equilibrio parcial para un país grande cuya demanda afecta a los precios internacionales.}{x}{y}{parcialgrande}
	% eje de abscisas
	\draw[-] (4,0) -- (7,0);
	\node[below] at (7,0){Q};
	% oferta
	\draw[-] (0,0) -- (6,4);
	\node[right] at (6,4){$S$};
	
	% demanda
	\draw[-] (0,4) -- (6,0);
	\node[right] at (6,.5){$D$};
	
	% LIBRE COMERCIO
	% precio de libre comercio PRE-ajuste internacional
	\draw[dashed] (0,1) -- (6,1);
	\node[left] at (0,1){$p$};
	
	% precio de libre comercio POST-ajuste internacional
	\draw[-] (0,0.75) -- (6,0.75);
	\node[left] at (0,0.6){$p'$};
	\draw[-{Latex}] (-0.6,1.1) -- (-0.6,0.5);
	
	% cantidad producida de libre comercio
	\draw[dashed] (1.12,0.75) -- (1.12,0);
	\node[below] at (1.12,0){\tiny $q_n$};
	
	% cantidad demandada de libre comercio
	\draw[dashed] (4.88,0.75) -- (4.88,0);
	\node[below] at (4.88,0){\tiny $q_m$};
	
	% exceso de demanda de libre comercio
	\draw[decorate,decoration={brace, mirror,amplitude=3pt},xshift=0pt,yshift=-1.1cm] (1.12,0) -- (4.88,0) node[black,midway,xshift=2pt, yshift=-0.33cm] {\tiny Importación de libre comercio};
	
	% POST ARANCEL
	% precio post arancel PRE-ajuste internacional
	\draw[dashed] (0,1.75) -- (6,1.75);
	\node[left] at (0,1.85){$p \cdot (1+t)$};
	
	% precio post arancel POST-ajuste internacional
	\draw[-] (0,1.5) -- (6,1.5);
	\node[left] at (0,1.5){$p' \cdot (1+t)$};
	\draw[-{Latex}] (-1,0.5) -- (-1,1.3);
	
	% cantidad producida post arancel
	\draw[dashed] (2.24,1.5) -- (2.24,0);
	\node[below] at (2.24,0){\tiny $q_n'$};
	\draw[-{Latex}] (1.25,0.35) -- (2.1,0.35);
	
	% cantidad demandada post arancel
	\draw[dashed] (3.75,1.5) -- (3.75,0);
	\node[below] at (3.75,0){\tiny $q_m'$};
	\draw[-{Latex}] (4.8,0.35) -- (3.83,0.35);
	
	% exceso de demanda de libre comercio
	\draw[decorate,decoration={brace, mirror,amplitude=3pt},xshift=0pt,yshift=-0.5cm] (2.24,0) -- (3.75,0) node[black,midway,xshift=2pt, yshift=-0.33cm] {\tiny Importación post-arancel};
	
	% Aumento del excedente del productor
	\draw[white, fill=green, opacity=0.2] (0,0.75) -- (0,1.5) -- (2.24,1.5) -- (1.12,0.75);

	% Recaudación del gobierno
	\draw [white, fill=green, opacity=0.2] (2.24,1.5) -- (3.75,1.5) -- (3.75,0.75) -- (2.24,0.75);
	
	% Aumento de los costes de producción
	\draw[white, fill=red, opacity=0.2] (1.12,0.75) -- (2.24,1.5) -- (2.24,0.75);
	
	% Pérdida irrecuperable de eficiencia
	\draw[white, fill=red, opacity=0.2] (4.88,0.75) -- (3.75,0.75) -- (3.75,1.5);
\end{axis}

\begin{axis}{4}{Efecto de un arancel a la importación en un contexto de equilibrio general en un país pequeño}{X}{}{generalsmall}
	% Extensión del eje de ordenadas
	\draw[-] (0,4) -- (0,5);
	\node[left] at (0,5){Y};
	% FPP
	\draw[-] (0,1.8) to [out=-5, in=100](3.6,0);
	
	% RP inicial-precio mundial
	\draw[-] (4,0) -- (0.32,4);
	\node[above] at (0.32,4){{\tiny $-\frac{p_x}{p_y}$}};

	% Producción pre-arancel
	\node[circle, fill=black, inner sep=0pt, minimum size=3pt] (a) at (3.18,0.9) {};
	\node[left] at (3.18,0.9){\tiny P};
	
	% RP POST arancel-precio nacional
	\draw[-] (4,1) -- (0,2.04);
	\node[right] at (4,1){{\tiny $-\frac{p_x}{p_y+t}$}};
	
	% Producción post-arancel
	\node[circle, fill=black, inner sep=0pt, minimum size=3pt] (a) at (2,1.52) {};
	\node[below] at (2,1.5){\tiny P'};
	
	% RP POST arancel-precios mundiales
	\draw[dotted] (3.4,0) -- (-0.28,4);
	
	% RP POST arancel-precio nacional de consumo efectivo
	\draw[dashed] (4,1.2) -- (0,2.34);
	
	% Curva de indiferencia sobre RP inicial-precio mundial
	\draw[-] (0.87,3.87) to [out=280,in=170](2.37,2.37);
	
	% Consumo sobre RP inicial-precio mundial
	\node[circle, fill=black, inner sep=0pt, minimum size=3pt] (a) at (1.35,2.9) {};
	\node[right] at (1.35,2.9){\tiny C};
	
	% Curva de indiferencia sobre RP POST arancel-precio mundial 
	\draw[-] (0.56,3.56) to [out=280,in=170](2.06,2.06);
	
	% Consumo sobre RP POST arancel-precio mundial
	\node[circle, fill=black, inner sep=0pt, minimum size=3pt] (a) at (1.03,2.58) {};
	\node[right] at (1.03,2.58){\tiny C'};
	
	% Curva de indiferencia sobre RP POST arancel-precio nacional de consumo efectivo y sobre RP post arancel-precio mundial
	\draw[-, line width=1pt] (0.33,3.33) to [out=280,in=170](1.83,1.83);
	
	% Consumo sobre RP post arancel-precio nacional de consumo efectivo y sobre RP post arancel-precio mundial (CONSUMO FINAL EFECTIVO)
	\node[circle, fill=black, inner sep=0pt, minimum size=3pt] (a) at (1.7,1.85) {};
	\node[above] at (1.72,1.83){\tiny $\tiny C_e$};
	
	% Exportaciones de X
	\draw[decorate,decoration={brace, mirror,amplitude=3pt},xshift=0pt,yshift=-0.6cm] (1.35,0) -- (3.18,0) node[black,midway,xshift=2pt, yshift=-0.33cm] {\tiny Exportación inicial};
	\draw[decorate,decoration={brace, mirror,amplitude=3pt},xshift=0pt,yshift=-0.2cm] (1.72,0) -- (2,0) node[black,midway,xshift=0pt, yshift=-0.33cm] {\tiny Final};
	
	% Importaciones de Y
	\draw[decorate,decoration={brace, mirror,amplitude=3pt},xshift=-20pt,yshift=0cm] (0,2.9) -- (0,0.9) node[black,midway,xshift=-25pt, yshift=0cm] {\tiny Importación inicial};
	\draw[decorate,decoration={brace, ,amplitude=3pt},xshift=-3pt,yshift=0cm] (0,1.52) -- (0,1.83) node[black,midway,xshift=-10pt, yshift=0cm] {\tiny Final};
	
\end{axis}


El punto C muestra el equilibrio previo al arancel, en el que el país exporta netamente el bien X a cambio de importar netamente el bien Y. El punto $C_e$ muestra el equilibrio tras la imposición del arancel. Este equilibrio resulta de la tangencia entre la curva de indiferencia del consumidor representativo, y la recta discontinua que representa la restricción presupuestaria que enfrentan los consumidores, en la cual el precio relativo del bien $y$ ha aumentado por la introducción del arancel $t$ y por ello, ha reducido su pendiente en valor absoluto respecto a la recta presupuestaria inicial. Aunque los consumidores toman decisiones de consumo en relación a esta recta discontinua, la restricción presupuestaria real es la representada por la recta de puntos, cuya pendiente corresponde a los precios relativos mundiales exógenos $\frac{p_x}{p_y}$, ya que el país deberá realizar en todo caso sus intercambios comerciales a ese precio. Así, el punto $C_e$ no sólo representa la tangencia entre la recta discontinua y la curva de indiferencia, sino la intersección entre la recta discontinua y la recta de puntos. Al mismo tiempo, el punto $C_e$ viene determinado porque pertenece a la línea con pendiente de precios mundiales que pasa por el punto P', que será el que elegirán los productores nacionales sobre la FPP.

Secuencia para dibujar:
\begin{enumerate}
	\item FPP
	\item Recta con precios internacionales tangente a FPP + punto P
	\item Curva de indiferencia de libre comercio + punto C de tangencia con RP a precios internacionales
	\item Recta con precios tras arancel tangente a FPP + punto P'
	\item Recta con precios internacionales que pasa por P' (de puntos, en el dibujo)
	\item Recta que intersecciona a anterior, con pendiente de precios tras arancel
	\item Curva de indiferencia con tangencia exactamente en punto de intersección entre dos rectas anteriores + punto $C_e$
	\item Curva de indiferencia tangente a recta de puntos, situada entre primera curva de indiferencia dibujada y la dibujada en el paso anterior.
\end{enumerate}


\begin{axis}{4}{Representación del impacto de un arancel en un contexto de equilibrio general y dos países grandes.}{X}{Y}{generalgrande}
	% Curva de demanda recíproca del país B
	\draw[-, line width=1pt] (0,0) to [out=20, in=260] (3.5,4);
	\node[above] at (3.5,4){\tiny B};
	
	% Curva de demanda recíproca del país A
	\draw[-, line width=1pt] (0,0) to [out=80, in=190](4,3.5);
	\node[right] at (4,3.5){\tiny A};
	
	% Curva de demanda recíproca del país A POST-arancel
	\draw[-, line width=1pt] (0,0) to [out=80, in=185](4,2);
	\node[right] at (4,2){\tiny A'};
	
	% Relación relativa de intercambio PRE-arancel
	\draw[dashed] (0,0) -- (4,4);
	\node[circle, fill=black, inner sep=0pt, minimum size=3pt] (a) at (3.36,3.36) {};
	\node[above] at (3.25,3.32){\tiny E};
	
	% Relación relativa de intercambio POST-arancel
	\draw[dashed] (0,0) -- (4,2.82);
	\node[circle, fill=black, inner sep=0pt, minimum size=3pt] (a) at (2.64,1.86) {};
	\node[below] at (2.77,1.86){\tiny E'};
	
	% Recta vertical que pasa por equilibrio post-arancel y termina en curva de demanda inicial
	\draw[-] (2.64,0) -- (2.64,3.13);
\end{axis}

La curva B representa la cantidad de bien X que el país B estará dispuesto a exportar a cambio de importar bien Y. Con preferencias estrictamente convexas, la curva de demanda recíproca tiene forma convexa y creciente porque necesitará importar cantidades crecientes para compensar exportaciones adicionales.

La curva A representa la cantidad de bien Y que el país A estará dispuesto a exportar a cambio de importar bien X. Cuando el país A impone un arancel sobre la importación del bien X, éste resultará más caro para los agentes, de tal manera que estará dispuestos a exportar menor cantidad de bien Y a cambio de una misma cantidad de bien X. En términos de las curvas de demanda recíproca, la curva A se desplaza hacia abajo. El nuevo equilibrio implica un nivel menor de intercambio y una caída del precio relativo del bien X (sobre el que se impuso el arancel) y cuya demanda cayó como resultado en el país A. Así, los términos de intercambio del país A mejoran porque para una cantidad dada de importaciones de Y, necesita exportar menos cantidad de X. O equivalentemente, su relación real de intercambio mejora porque el bien que exporta (Y) se encarece en términos del bien que importa (X). 

El arancel tuvo un efecto positivo sobre la economía que lo impuso porque el aumento de precio del bien importado que sufrieron los consumidores nacionales fue inferior al importe del arancel, lo que permitió extraer una renta a los exportadores extranjeros. Esto sucedió porque mejoró la relación real de intercambio, de tal manera que el precio de las exportaciones se encareció en relación a las importaciones. Además, el precio en el mercado doméstico del bien importado aumenta, lo que permite a los productores nacionales del bien importable aumentar el precio al que venden su producto, protegiéndoles de la competencia exterior. 

\begin{axis}{4}{Representación gráfica de la paradoja de Metzler: un arancel positivo desprotege a la industria nacional que produce el bien sobre el que recae el arancel.}{X}{Y}{metzler}
	% Curva de oferta recíproca del país A
	\draw[-, line width=1pt] (0,0) to [out=80, in=190](4,3.5);
	\node[right] at (4,3.5){\tiny A};
	
	% Curva de oferta recíproca del país A POST-arancel
	\draw[-, line width=1pt] (0,0) to [out=80, in=185](4,2);
	\node[right] at (4,2){\tiny A'};
	
	% Curva de oferta recíproca del país B
	\draw[-, line width=1pt] (0,0) to [out=20, in=270] (3.2,2.2) to [out=90, in=-10](1.5,3.5);
	\node[above] at (1.4,3.35){\tiny B};
	
	% Relación relativa de intercambio PRE-arancel
	\draw[dashed] (0,0) -- (3.4,4);
	\node[circle, fill=black, inner sep=0pt, minimum size=3pt] (a) at (2.65,3.12) {};
	\node[above] at (2.63,3.15){\tiny E};
	
	% Relación relativa de intercambio POST-arancel
	\draw[dashed] (0,0) -- (4,2.42);
	\node[circle, fill=black, inner sep=0pt, minimum size=3pt] (a) at (3.18,1.92) {};
	\node[below] at (3.3,1.96){\tiny E'};
	
	% Cantidad constante de 1 sobre punto de equilibrio con arancel
	\draw[dotted] (3.18,0) -- (3.18,4);
	
	% Precios nacionales reales POST-arancel
	\draw[dotted] (0,0) -- (3.85,4);
	\node[circle, fill=black, inner sep=0pt, minimum size=3pt] (a) at (3.18,3.3) {};
	\node[right] at (3.15, 3.2){\tiny E''};
	
	% Curvas de indiferencia de país A
	%\draw[-] (1.27,1.23) to [out=60,in=190](4.27,3.63);
	%\node[right] at (4.27,3.63){\tiny I};
	%\draw[-] (1.83,0.68) to [out=60,in=190](4.83,3.08);
	%\node[right] at (4.83,3.08){\tiny I'};
\end{axis}

El punto E' representa el precio que pagarán en el mercado doméstico por importar X a cambio de Y. Esto es, la relación relativa de intercambio. Como se aprecia, E' representa un precio de las exportaciones relativas a las importaciones superior al de E, por lo que la RRI evoluciona favorablemente al país. Sin embargo, la industria nacional productora de bien 1 se ve perjudicada por el arancel. La línea entre el origen y el punto E'' caracteriza el precio nacional del bien X. Se aprecia que el precio reflejado por E'' es inferior al de libre comercio reflejado por la línea discontinua desde el origen hasta el punto E. Así, el arancel provoca una bajada de precio del bien 1 tras la introducción del arancel y por tanto, una pérdida para los productores nacionales del bien que el arancel pretendía proteger. 

\begin{axis}{4}{Representación gráfica de la paradoja de Lerner: un arancel positivo empeora la relación relativa de intercambio del país que lo impone.}{}{Y}{lerner}
	% Extension del eje de abscisas
	\draw[-] (4,0) -- (6,0);
	\node[below] at (6,0){X};
	
	% Curva de oferta recíproca del país A
	\draw[-, line width=1pt] (0,0) to [out=80, in=180](2,3.5) to [out=0,in=100](4.5,0.5);
	\node[right] at (4.5,0.5){\tiny A};
	
	% Curva de oferta recíproca del país A POST-arancel
	\draw[-, line width=1pt] (0,0) to [out=80, in=180](4,3.5) to [out=0,in=100](7,0.5);
	\node[right] at (7,0.5){\tiny A'};
	
	% Curva de oferta recíproca del país B
	\draw[-, line width=1pt] (0,0) to [out=20, in=260] (4.5,4);
	\node[above] at (4.5,4){\tiny B};
	
	% Relación relativa de intercambio POST-arancel
	\draw[dashed] (0,0) -- (5.03,4);
	\node[circle, fill=black, inner sep=0pt, minimum size=3pt] (a) at (4.39,3.47) {};
	\node[below] at (4.45,3.43){\tiny E'};
	
	% Relación relativa de intercambio PRE-arancel
	\draw[dashed] (0,0) -- (6,3.72);
	\node[circle, fill=black, inner sep=0pt, minimum size=3pt] (a) at (3.84,2.37) {};
	\node[right] at (3.84,2.35){\tiny E};
	
	% Demanda con arancel y RRI inicial
	\node[below] at (5.2,3.22){\tiny P};
	\node[circle, fill=black, inner sep=0pt, minimum size=3pt] (a) at (5.2,3.22) {};
	
	% Exceso de demanda de 1 a precios iniciales
	\draw[dotted] (5.2,3.22) -- (5.2,0);
	\draw[dotted] (3.84,2.35) -- (3.84,0);
	\draw[decorate,decoration={brace, mirror,amplitude=3pt},xshift=0pt,yshift=-0.1cm] (3.84,0) -- (5.2,0)  node[black,midway,xshift=0pt, yshift=-0.3cm] {\tiny Exceso de demanda};
	
	%Cantidad constante de 1 sobre punto de equilibrio con arancel
	%\draw[dotted] (3.18,0) -- (3.18,4);
	
	% Precios nacionales reales POST-arancel
	%\draw[dotted] (0,0) -- (3.85,4);
	%\node[circle, fill=black, inner sep=0pt, minimum size=3pt] (a) at (3.18,3.3) {};
	
	% Curvas de indiferencia de país A
	%\draw[-] (1.27,1.23) to [out=60,in=190](4.27,3.63);
	%\node[right] at (4.27,3.63){\tiny I};
	%\draw[-] (1.83,0.68) to [out=60,in=190](4.83,3.08);
	%\node[right] at (4.83,3.08){\tiny I'};
\end{axis}

La paradoja de Lerner implica un empeoramiento de la relación relativa de intercambio en contra del país que impone el arancel. Es decir, el precio de las exportaciones del país que impone el arancel se reduce en relación a las importaciones una vez introducido el arancel y en relación a la relación relativa de intercambio de libre comercio. La parodoja de Lerner se produce cuando la demanda del bien importado y gravado es muy elástica en el país importador. Ello se traduce en una curva de demanda recíproca anómala en el país que impone el arancel. Así, la imposición del arancel implica un exceso de demanda del bien importado respecto a lo que estaría dispuesto a exportar el otro país a los precios iniciales. Para eliminar este exceso de demanda, el precio relativo del bien importado debe aumentar, lo que resulta en un empeoramiento de la RRI para el país que introduce el arancel.

\begin{axis}{4}{Representación del concepto de arancel óptimo en un marco de equilibrio general con dos países, dos bienes y las curvas de oferta recíproca respectivas.}{X}{Y}{aranceloptimo}
	% Curva de oferta recíproca del país B
	\draw[-, line width=1pt] (0,0) to [out=20, in=260] (3.5,4);
	\node[above] at (3.5,4){\tiny B};
	
	% Curva de oferta recíproca del país A
	\draw[-, line width=1pt] (0,0) to [out=80, in=190](4,3.5);
	\node[right] at (4,3.5){\tiny A};
	
	% Curva de oferta recíproca del país A POST-arancel
	\draw[-, line width=1pt] (0,0) to [out=80, in=185](4,2);
	\node[right] at (4,2){\tiny A'};

	% Relación relativa de intercambio PRE-arancel
	\draw[dashed] (0,0) -- (4,4);
	\node[circle, fill=black, inner sep=0pt, minimum size=3pt] (a) at (3.36,3.36) {};
	\node[above] at (3.25,3.32){\tiny E};
	
	% Relación relativa de intercambio POST-arancel
	\draw[dashed] (0,0) -- (4,2.82);
	\node[circle, fill=black, inner sep=0pt, minimum size=3pt] (a) at (2.64,1.86) {};
	\node[below] at (2.7,1.86){\tiny E'};
	
	% Curvas de indiferencia de país A
	\draw[-] (1.27,1.23) to [out=60,in=190](4.27,3.63);
	\node[right] at (4.27,3.63){\tiny I};
	\draw[-] (1.83,0.68) to [out=60,in=190](4.83,3.08);
	\node[right] at (4.83,3.08){\tiny I'};
\end{axis}

El gráfico muestra las curvas de demanda recíproca de los países A y B. Un gráfico de demanda recíproca muestra las cantidades que un país está dispuesto a exportar a cambio de una cantidad dada de importaciones. Los ejes de coordenadas representan cantidades de bien X y Y, respectivamente. Dada una curva de demanda recíproca, la interpretación de un bien como exportado o importado es externa al gráfico y debe tomarse como dada. En este caso, se asume que el país B importa bien Y y exporta bien X, y que el país A exporta bien Y e importa bien X. El supuesto se basa en el hecho de que en presencia de FPP cóncavas, una economía exigirá cantidades crecientes de importaciones a cambio de una misma cantidad de importaciones. Así, la curva para el país A, que importa X y exporta Y será cóncava porque exige cada vez mayores cantidades de X para exportar una cantidad  dada de Y. En el caso del país B, la curva será convexa porque el país exigirá cantidades crecientes del bien Y a cambio de exportar una cantidad fija dada del bien Y. 

La intersección entre dos curvas de demanda recíproca determina el patrón de comercio y la relación de intercambio. La intersección de las curvas representa igualdad de oferta y demanda en ambos mercados como resultado de las importaciones y exportaciones de cada bien en cada país respectivo. Así, los puntos E y E' muestran qué cantidades de X el país A importa a cambio de exportar bien Y y de forma paralela, qué cantidades de bien Y el país B importa a cambio de qué cantidad de bien X exportado. La relación relativa de intercambio desde el punto de vista de uno de los países resulta del cociente entre la cantidad de bien importado y la cantidad de bien exportado ($\frac{M}{X}$). Esta cantidad es igual al cociente entre el precio del bien exportado y el bien importado ($\frac{P_x}{P_m}$).

En el gráfico se aprecia una relación monótona entre la relación relativa de intercambio y el arancel que el país A impone sobre las importaciones de bien X. El arancel impuesto sobre el bien X en A implica que los agentes del país estarán dispuestos a exportar menores cantidades de Y a cambio de importar una cantidad dada de X. O de forma equivalente, que aceptarán exportar una cantidad dada de Y a cambio de importar mayores cantidades de X. Tenemos, así, que debe aumentar el valor del cociente $\frac{M_X}{X_Y}$ y por ello, aumentar el valor de $\frac{P_X}{P_M}$ o la relación relativa de intercambio. Se aprecia por tanto, que un arancel sobre las importaciones resulta --dadas estas curvas de demanda recíproca- en una mejora de la relación relativa de intercambio para el país que lo introduce.

Los efectos del arancel sobre el bienestar pueden caracterizarse dibujando curvas de indiferencia social que representen las preferencias de los agentes del país en relación a la cantidad de exportaciones e importaciones. Asumiendo que el objetivo de los agentes es consumir la mayor cantidad posible de bienes, las curvas de indiferencia del país A (representadas en las curvas I y I') son cóncavas y crecientes: para mantener constante el bienestar aumentando la cantidad de bien Y exportado, hace falta compensar a los agentes del país con una cantidad creciente de bien X importado. En el gráfico se muestra como la aplicación de un arancel sobre las importaciones de bien X en el país A induce una mejora del bienestar y por tanto, permite inferir la existencia de un arancel óptimo.

\begin{axis}{4}{Representación del impacto de un arancel en un contexto de equilibrio general y un país pequeño que no afecta a los precios mundiales.}{X}{Y}{aranceloptimosmallcountry}
	% Curva de oferta recíproca del país B
	\draw[-, line width=1pt] (0,0) -- (4,4);
	\node[above] at (4,4){\tiny B};
	
	% Curva de oferta recíproca del país A
	\draw[-, line width=1pt] (0,0) to [out=80, in=190](4,3.5);
	\node[right] at (4,3.5){\tiny A};
	
	% Curva de oferta recíproca del país A POST-arancel
	\draw[-, line width=1pt] (0,0) to [out=80, in=185](4,2);
	\node[right] at (4,2){\tiny A'};
	
	% PRE arancel
	\node[circle, fill=black, inner sep=0pt, minimum size=3pt] (a) at (3.36,3.36) {};
	
	% POST arancel
	\node[circle, fill=black, inner sep=0pt, minimum size=3pt] (a) at (1.65,1.65) {};
	
	% Precio nacional de bienes nacionales POST-arancel
	\draw[dashed] (1.65,0) -- (1.65,2.6);
	\node[circle, fill=black, inner sep=0pt, minimum size=3pt] (a) at (1.65,2.6) {};
	\draw[dashed] (0,0) -- (2.55,4);
	\node[above] at (2.55,4){\tiny $\frac{p_1}{p_2}$};
	
	% Curvas de indiferencia de país A
	%\draw[-] (1.27,1.23) to [out=60,in=190](4.27,3.63);
	%\node[right] at (4.27,3.63){\tiny I};
	%\draw[-] (1.83,0.68) to [out=60,in=190](4.83,3.08);
	%\node[right] at (4.83,3.08){\tiny I'};
\end{axis}

\begin{axis}{4}{Efecto de un arancel sobre la importación de productos de calidades bajas}{$\alpha$}{$p(\alpha)$}{diferenciacionvertical}
	% País A productor de calidades altas
	\draw[-, line width=1pt] (0,1.75) -- (4,2.25);
	\node[right] at (4,2.25){$p_A(\alpha)$};
	\node[left] at (0,1.75){$W_1$};
	
	% País B productor de calidades bajas PRE-arancel
	\draw[-, line width=1pt] (0,0.75) -- (4,3.25);
	\node[right] at (4,3.25){$p_B(\alpha)$};
	\node[left] at (0,0.75){$W_2$};
	
	% País B productor de calidades bajas POST-arancel
	\draw[-] (0,1.25) -- (4,4);
	\node[right] at (4,4){$p_B'(\alpha)$};
	\node[left] at (0,1.25){$W_2'$};
\end{axis}

\begin{axis}{4}{Representación del efecto de una subvención a la exportación para un país pequeño en un contexto de equilibrio parcial}{}{$P$}{subvencionexportacion}
	% Extensión del eje de ordenadas
	\draw[-] (4,0) -- (6,0);
	\node[below] at (6,-0.1){$Q$};
	
	% Demanda nacional
	\draw[-,line width=1pt] (0,4) -- (4.5,0);
	\node[right] at (0.3,4){D};
	
	% Oferta nacional
	\draw[-,line width=1pt] (1.5,0) -- (6,4);
	\node[right] at (6,4){S};
	
	% Precio sin subvención
	\draw[-] (0,2) -- (6,2);
	\node[left] at (0,2){$p$};
	
	% Precio con subvención
	\draw[-] (0,3) -- (6,3);
	\node[left] at (0,3){$p+s$};
	
	% Exportaciones sin subvención
	\draw[dashed] (2.25,0) -- (2.25,2);
	\draw[dashed] (3.75,0) -- (3.75,2);
	\draw[decorate,decoration={brace, mirror,amplitude=3pt},xshift=0pt,yshift=-0.1cm] (2.25,0) -- (3.75,0)  node[black,midway,xshift=0pt, yshift=-0.3cm] {\tiny Sin subvención};
	
	% Exportaciones con subvención
	\draw[dashed] (1.12,0) -- (1.12,3);
	\draw[dashed] (4.88,0) -- (4.88,3);
	\draw[decorate,decoration={brace, mirror,amplitude=3pt},xshift=0pt,yshift=-0.6cm] (1.12,0) -- (4.88,0)  node[black,midway,xshift=0pt, yshift=-0.3cm] {\tiny Con subvención};
	
	% Pérdida irrecuperable de eficiencia de consumidores
	\draw[white, fill=red, opacity=0.2] (1.13,3) -- (1.13,2) -- (2.25,2);
	\node[left] at (1.6,2.4){B};
	
	% Aumento de coste de los productores
	\draw[white, fill=red, opacity=0.2] (4.88,3) -- (3.75,2) -- (4.88,2);
	\node[right] at (4.35,2.4){D};
	
	% Área de excedente inicial de consumidores
	\node[left] at (0.8,2.4){A};
	
	% Área central de excedente del producto
	\node[left] at (3.2,2.4){C};
	
	% Gasto en subvención
	\draw[-, color=yellow, thick] (0.02,2.02) -- (0.02,2.97) -- (4.86,2.97) -- (4.86,2.02) -- (0.02,2.02);
\end{axis}

El coste de la subvención para el gobierno es igual a la suma de las áreas A, B, C y D, o equivalentemente, el área dentro del rectángulo amarillo. El ingreso extra de los productores es igual al coste total de la subvención. El área D representa el aumento de costes de productor, por lo que el beneficio se reduce a la suma de A, B y C. Teniendo en cuenta que el ingreso extra se deriva del aumento del gasto del gobierno, el resultado global de la subvención en este contexto será una pérdida de eficiencia igual a D. 

\conceptos

\concepto{Hipótesis de Prebisch-Singer}

La hipotésis de Prebisch-Singer postula que la relación entre el precio de las materias primas y el precio de los bienes manufacturados tiende a decrecer con el tiempo, conduciendo a una reducción de los términos de comercio de los países en desarrollo que exportan fundamentalmente materias primas. La hipótesis de Prebisch-Singer fue utilizada como justificación de políticas de sustitución de importaciones.

\preguntas

\seccion{Test 2016}

\textbf{31.} Señale la respuesta correcta de entre las siguientes, relacionadas con la política comercial:

\begin{itemize}
	\item[a] La Tesis de Prebisch-Singer se refiere al desarrollo de los términos de intercambio de los países en desarrollo en beneficio de los países desarrollados.
	\item[b] En un  país grande, que influye en los precios internacionales, no podría darse el caso de que la apertura comercial conduzca a un deterioro de la relación real de intercambio y a una disminución del bienestar, debido a la capacidad de ajustar los precios.
	\item[c] La Ley de Engel es un ejemplo de como la promoción de un sector exportador puede frenar el crecimiento de un país por la apreciación de su moneda.
	\item[d] El índice de Grubel y Lloyd es una medida del grado de apertura de la economía de un país, considerando su comercio exterior en relación con el conjunto de su actividad económica global.
\end{itemize}

\seccion{Test 2013}

\textbf{30.} El coste de la protección de un arancel, en un contexto de país pequeño y equilibrio parcial, es igual a:

\begin{itemize}
	\item[a] La reducción en el excedente de los consumidores menos el incremento de la renta de los productores.
	\item[b] La reducción en el excedente de los consumidores menos el efecto ingreso del arancel.
	\item[c] La reducción en el excedente de los consumidores menos el incremento de la renta de los productores, más el incremento de la renta de los productores.
	\item[d] La reducción en el excedente de los consumidores menos el incremento de la renta de los productores y el efecto ingreso del arancel.
\end{itemize}

\textbf{31.} Cuando un país pequeño establece un arancel a las importaciones de un bien, en equilibrio general se producirá que:

\begin{itemize}
	\item[a] Los precios internacionales no permanecen constantes.
	\item[b] Los precios para los consumidores del país pequeño aumentan en la totalidad del arancel.
	\item[c] Los precios para el país pequeño como un todo no permanecen constante.
	\item[d] La producción del bien disminuye, el consumo y las importaciones aumentan en el país pequeño.
\end{itemize}

\seccion{Test 2011}

\textbf{33.} En una economía con dos factores productivos (capital y trabajo), perfectamente móviles entre sectores, un arancel a la importación sobre un producto intensivo en capital:

\begin{itemize}
	\item[a] Mejora la remuneración relativa del capital y empeora la del trabajo.
	\item[b] Empeora la remuneración relativa del capital y mejora la del trabajo.
	\item[c] Mejora la remuneración de los dos factores en términos absolutos.
	\item[d] No se producen efectos redistributivos.
\end{itemize}

\seccion{Test 2009}

\textbf{26.} Entre los efectos de la política comercial en mercados de competencia imperfecta, puede observarse que:

\begin{itemize}
 	\item[a] Una industria monopolista perderá poder de monopolio si la industria nacional se protege mediante una cuota de importación.
	\item[b] Una industria monopolista ganará poder de monopolio si la industria nacional se protege mediante una cuota de importación.
	\item[c] Una industria monopolista ganará poder de monopolio si la industria nacional se protege mediante un arancel.
	\item[d] Una industria monopolista no verá alterado su poder de monopolio si la industria nacional se protege tanto con un arancel como con una cuota que permitan un mismo nivel de importaciones.
\end{itemize}

\seccion{Test 2007}

\textbf{28.} Los principales efectos del establecimiento de un subsidio a la exportación en un país pequeño, en condiciones de competencia perfecta y equilibrio parcial, serían:

\begin{itemize}
	\item[a] Aumenta la producción, disminuye el consumo, aumentan las exportaciones, y el gasto en el subsidio se contrarresta con el ingreso arancelario resultante de la imposición de un arancel compensatorio. El efecto neto sobre el bienestar del país es positivo.
	\item[b] Aumentan la producción, el consumo y las exportaciones, y el subsidio supone un gasto para el gobierno. El efecto neto sobre el bienestar es ambiguo.
	\item[c] Aumenta la producción, disminuye el consumo, aumentan las exportaciones, y el gasto en el subsidio se contrarresta sólo parcialmente con el ingreso arancelario resultante de la imposición de un arancel compensatorio. El efecto neto sobre el bienestar del país es negativo.
	\item[d] Aumenta la producción, disminuye el consumo, aumentan las exportaciones, y el subsidio supone un gasto para el gobierno. El efecto neto sobre el bienestar del país es negativo.
\end{itemize}

\seccion{Test 2005}

\textbf{33.} Los principales efectos de la imposición de un arancel en un país pequeño, en condiciones de competencia perfecta y equilibrio parcial, serían:

\begin{itemize}
	\item[a] Aumentan la producción, disminuye el consumo, aumentan las importaciones, y el gobierno ve desaparecer el ingreso arancelario, pues el arancel se ve compensado por un aumento de las subvenciones a la exportación. El efecto neto sobre el bienestar del país es negativo.
	\item[b] Aumentan la producción y el consumo, se reducen las importaciones, y el gobierna recauda el ingreso arancelario. El efecto neto sobre el bienestar del país es positivo.
	\item[c] Aumenta la producción, disminuye el consumo, se reducen las importaciones, y el gobierno recauda el ingreso arancelario. El efecto neto sobre el bienestar del país es negativo.
	\item[d] Aumenta la producción, disminuye el consumo, se reducen las importaciones, y el gobierno ve desaparecer el ingreso arancelario, pues el arancel se ve compensado por un aumento de las subvenciones a la exportación. El efecto neto sobre el bienestar del país es ambiguo.
\end{itemize}

\notas

\textbf{2016:} \textbf{31.} A

\textbf{2013:} \textbf{30.} D \textbf{31.} Anulada

\textbf{2011:} \textbf{33.} A

\textbf{2009:} \textbf{26.} B

\textbf{2007:} \textbf{28.} C

\textbf{2005:} \textbf{33.} C

\bibliografia

Mirar en Palgrave:
\begin{itemize}
	\item antidumping
	\item countertrade
	\item effective protection
	\item European Union (EU) Trade Policy
	\item free trade and protection
	\item infant-industry protection
	\item international trade theory
	\item macroeconomic effects of international trade
	\item Metzler, Lloyd Appleton
	\item non-tariff barriers
	\item offer curve or reciprocal demand curve
	\item optimal tariffs
	\item strategic trade policy
	\item tariffs
	\item tariffs versus quota
	\item terms of trade
	\item tradable and non-tradable commodities
	\item trade costs
	\item trade policy, political economy of
\end{itemize}

Balassa, B. \textit{Tariff Protection in Industrial Countries: An Evaluation} (1965) Journal of Political Economy

Bown, C. Zhang, E. (2019) \textit{Will a US-China trade deal remove or just restructure the massive 2018 tariffs?} PIIE Trade \& Investment Watch, 24 april 2019 -- \url{https://voxeu.org/content/will-us-china-trade-deal-remove-or-just-restructure-massive-2018-tariffs}

Feenstra, R. C. \textit{Advanced International Trade} (2004) Ch. 7 y Ch. 8 -- En carpeta de Economía Internacional. 
Flatters, F. (2004) \textit{Measuring the Impacts of Trade Policies: Effective Rates of Protection} Queens University -- En carpeta del tema

Gandolfo, G. \textit{International Trade Theory and POlicy} (2014) 2nd Edition -- En carpeta de Economía Internacional

Krueger, A. O. (1974) \textit{The Political Economy of the Rent-Seeking Society} American Economic Review. Vol 64. No. 3

\end{document}
