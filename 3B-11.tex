\documentclass{nuevotema}

\tema{3B-11}
\titulo{Determinantes y efectos de los movimientos internacionales de factores productivos. Especial referencia a la inversión directa}

\begin{document}

\ideaclave

Ver UNCTAD (2019) World Investment Report. Capítulo C sobre producción internacional de empresas Multinacionales. 

Leer Borjas (1995) The Economic Benefics from Immigration de JEP (En carpeta del tema)

\seccion{Preguntas clave}

\begin{itemize}
	\item ¿Qué son los movimientos internacionales de factores productivos?
	\item ¿Por qué migran los trabajadores?
	\item ¿Qué efectos tienen las migraciones?
	\item ¿Por qué se producen los flujos internacionales de capital?
	\item ¿Qué efectos inducen los flujos internacionales de capital?
	\item ¿Por qué tiene lugar la inversión directa extranjera?
	\item ¿Qué efectos tiene sobre el país de destino?
\end{itemize}

\esquemacorto

\begin{esquema}[enumerate]
	\1[] \marcar{Introducción}
		\2 Contextualización
			\3 Globalización
			\3 Importancia de los movimientos internacionales de ff.pp.
			\3 Análisis económico de los movimientos factoriales
		\2 Objeto
			\3 ¿En qué consisten los mov. internacionales de ff.pp.?
			\3 ¿Por qué migran los trabajadores?
			\3 ¿Qué efectos tienen las migraciones?
			\3 ¿Por qué existen los flujos internacionales de capital?
			\3 ¿Qué efectos inducen los flujos internac. de capital?
			\3 ¿Por qué se produce la IDE?
			\3 ¿Qué efectos tiene?
		\2 Estructura
			\3 Movimientos de trabajo
			\3 Movimientos de capital
	\1 \marcar{Movimientos de trabajo}
		\2 Idea clave
			\3 Contexto
			\3 Objetivos
			\3 Resultados
		\2 Fenómenos a explicar
			\3 Desplazamientos de población activa entre economías
			\3 Cualificación relativa de migrantes tiende a caer
			\3 Menos movimiento de L que de K y de comercio
			\3 Incentivos a admitir o rechazar inmigración
			\3 Origen de migración ligada a redes en destino
			\3 Tendencia a permanecer en destino
		\2 Determinantes: modelo de Borjas (1987)
			\3 Idea clave
			\3 Formulación
			\3 Implicaciones
			\3 Valoración
		\2 Determinantes: dinámicas de aglomeración espacial
			\3 Idea clave
			\3 Formulación
			\3 Implicaciones
			\3 Valoración
		\2 Determinantes: redes de inmigración
			\3 Idea clave
			\3 Formulación
			\3 Implicaciones
			\3 Valoración
		\2 Otros determinantes de las migraciones
			\3 Sustitutivo del comercio internacional
			\3 Modelo de factores específicos Ricardo-Viner
		\2 Efectos
			\3 Modelo de MacDougall (1960): igualación de PMgL
			\3 Impacto sobre distribución de la renta
			\3 Brain drain vs brain gain vs brain waste
			\3 Análisis de redes
			\3 Inmigrantes como inputs diferenciados
			\3 Externalidades de la inmigración sobre población local
			\3 Variedad en lugar de origen
	\1 \marcar{Movimientos de capital}
		\2 Idea clave
			\3 Contexto
			\3 Objetivos
			\3 Resultados
		\2 Fenómenos a explicar
			\3 Hechos estilizados
			\3 Contradicción con modelo H-O-S
			\3 Puzzle de Feldstein y Horioka (1980)
			\3 Paradoja de Lucas (1990)
			\3 Ajuste intertemporal de la balanza de pagos
			\3 Allocation puzzle de Gourinchas (2007)
		\2 Determinantes
			\3 Soluciones a Lucas (1990) dentro de modelo neoclásico
			\3 Soluciones más allá de marco neoclásico
			\3 Soluciones a puzzle de la asignación
		\2 Efectos
			\3 Caída del coste de financiación
			\3 Aumento de inversión
			\3 Apreciación del tipo de cambio
			\3 Burbujas especulativas
			\3 Sudden stops y reversiones de flujos de capital
			\3 Teoría
			\3 Efecto composición
			\3 Efecto umbral
			\3 Hechos empíricos
		\2 Inversión directa extranjera
			\3 Idea clave
			\3 Hechos estilizados
			\3 Determinantes: marco OLI
			\3 Determinantes: proximidad vs concentración -- Helpman, Melitz y Yeaple (2004)
			\3 Determinantes: IDE vertical
			\3 Instituciones
			\3 Determinantes empíricos
			\3 Efectos
	\1[] \marcar{Conclusión}
		\2 Recapitulación
			\3 Movimientos de trabajo
			\3 Movimientos de capital
		\2 Idea final
			\3 Frontera de investigación
			\3 Beneficios de la globalización
			\3 Controversia sobre inmigración
			\3 Movimientos de ff.pp. en Unión Europea

\end{esquema}

\esquemalargo

\begin{esquemal}
	\1[] \marcar{Introducción}
		\2 Contextualización
			\3 Globalización
				\4 Concepto muy vago
				\4[] Múltiples acepciones según contexto
				\4 Sentido en ciencia económica
				\4[] Libre circulación mundial de
				\4[] $\to$ Bienes y servicios
				\4[] $\to$ Factores de producción
				\4[] $\to$ Tecnologías e ideas
				\4 Diferentes fases a lo largo de la historia
				\4[] Ruta de la seda, descubrimiento de América
				\4[] Antes de la I Guerra Mundial
				\4[] $\to$ Trabajo y capital muy liberalizados
				\4[] Entreguerras
				\4[] $\to$ Aislamiento de bloques económicos
				\4[] Posguerra
				\4[] $\to$ Progresiva liberalización comercial
				\4[] $\to$ Post BW: liberalización capitales
				\4[] $\to$ Nuevas tecnologías globalizan información
				\4 Últimas décadas de siglo XX
				\4[] $\to$ Aceleración del proceso
				\4 Hechos estilizados de la globalización
				\4[] Comercio internacional
				\4[] 30\% del PIB mundial en últimas décadas
				\4[] $\to$ Desde 10\% en años 70
				\4 Acciones extranjeras
				\4[] 20\% de carteras de inversión
				\4[] $\to$ Desde niveles insignificantes
				\4 Inversión Directa Extranjera
				\4[] alrededor del 15\% de FBK
				\4[] $\to$ Desde el 2\% en años 70
				\4 Inmigración en la UE
				\4[] 22,3 M sobre 512 M totales en 2017
				\4[] $\to$ 4,4\% de la población
				\4[] $\to$ Ciudadanos de países no-UE
				\4[] 2,4 M de inmigrantes no-UE entraron en la UE en 2017
			\3 Importancia de los movimientos internacionales de ff.pp.
				\4 Impacto económico
				\4[] Inversión
				\4[] Capital por trabajador
				\4[] Salarios y desempleo
				\4[] Financiación de déficits
				\4[] Productividad
				\4 Otros
				\4[] Economía política
				\4[] Factores culturales
			\3 Análisis económico de los movimientos factoriales
				\4 Marco tradicional de análisis
				\4[] Basado en equilibrio walrasiano y H-O-S
				\4[] Capital y trabajo tratados indistintamente
				\4 Modelos más recientes
				\4[] Características idiosincráticas del ff.pp.
				\4[] Nuevas familias de modelos
				\4[] Mayor énfasis en contrastación empírica
		\2 Objeto
			\3 ¿En qué consisten los mov. internacionales de ff.pp.?
			\3 ¿Por qué migran los trabajadores?
			\3 ¿Qué efectos tienen las migraciones?
			\3 ¿Por qué existen los flujos internacionales de capital?
			\3 ¿Qué efectos inducen los flujos internac. de capital?
			\3 ¿Por qué se produce la IDE?
			\3 ¿Qué efectos tiene?
		\2 Estructura
			\3 Movimientos de trabajo
			\3 Movimientos de capital
				\4 Inversión directa extranjera
	\1 \marcar{Movimientos de trabajo}
		\2 Idea clave
			\3 Contexto
				\4 Movimientos de población constante histórica
				\4 Muy a menudo, en olas
				\4[] Colonización del Nuevo Mundo
				\4[] Desplazamientos pre y posguerra mundial
				\4[] Bloque comunista: movimientos forzados
				\4[] ...
				\4 Efectos económicos de las migraciones
				\4[] Sobre población de origen
				\4[] Sobre población de destino
				\4[] $\then$ Salarios
				\4[] $\then$ Tasa de empleo
				\4 Gran crecimiento migratorio en las últimas décadas
				\4 Inmigración ha crecido menos que CI
				\4 Condiciones de trabajo no han sufrido race-to-the-bottom
				\4 Inmigración cualificada
				\4[] En términos absolutos, ha aumentado fuertemente
				\4[] En términos relativos, apenas no
				\4[] $\to$ cualificación ha aumentado en general en origen
			\3 Objetivos
				\4 Explicar patrón migratorio
				\4[] Cualificación relativa migrantes y nativos
				\4[] Cualificación relativa en país de origen
				\4 Explicar decisión de migración
				\4 Entender y predecir efectos de migración
				\4[] Sobre salarios
				\4[] Sobre mercado de trabajo
				\4 Decisión de inmigración
				\4[] ¿Por qué emigrar?
				\4 Impacto sobre país de destino
			\3 Resultados
				\4 Determinantes de las migraciones
				\4[] Modelos de autoselección
				\4[] Modelos de comercio internacional con restricción a CI
				\4 Efectos de la migración
				\4[] Sobre mercado local
				\4[] Sobre mercado de origen
				\4[] Sobre mercados de capital
				\4[] Sobre mercados de bienes y servicios en CInternacional
		\2 Fenómenos a explicar
			\3 Desplazamientos de población activa entre economías
				\4 Libre comercio debería eliminar necesidad
				\4[] De acuerdo con modelos basados en precios ff.pp.
			\3 Cualificación relativa de migrantes tiende a caer
				\4 En términos absolutos, no cae
				\4[] Inmigrantes cada vez mejor formados
				\4 En comparación con poblaciones de destino
				\4[] Tendencia hacia menor cualificación
			\3 Menos movimiento de L que de K y de comercio
				\4 Presión migratoria ha aumentado en términos absolutos
				\4[] En términos absolutos, mucho menos que K y CI
				\4 Fenómeno más relevante aún en últimos 30 años
			\3 Incentivos a admitir o rechazar inmigración
				\4 Economía política
				\4[] Grupos de interés en población de destino
				\4[] $\to$ ¿Se benefician de migración?
				\4[] $\to$ ¿Compiten con migrantes?
				\4[] $\then$ Explicar como migración beneficia/perjudica grupos
			\3 Origen de migración ligada a redes en destino
				\4 Fenómeno habitual
				\4 Más migración a destinos
				\4[] Donde ya existe comunidad de mismo origen
				\4 Evidencia empírica robusta
				\4 Mejores salarios donde hay otros con mismo origen
				\4[] Externalidades de red con comunidades grandes

			\3 Tendencia a permanecer en destino
				\4 Elevado porcentaje permanece
				\4 Inmigrantes tienen más posibilidad de re-emigrar
				\4[] En relación a población en país de destino
		\2 Determinantes: modelo de Borjas (1987)
			\3 Idea clave
				\4 Contexto
				\4[] Basado en Roy (1951)
				\4[] Modelo de autoselección
				\4[] Se observan dos distribuciones poblacionales
				\4[] $\to$ ¿Cómo se han generado?
				\4[] $\to$ ¿Agentes han elegido en cual situarse?
				\4 Variables observables:
				\4[] Años de educación de trabajadores
				\4[] $\to$ Cualificación respecto a población de destino
				\4[] $\to$ Cualificación respecto a población de origen
				\4[] Salario de migrantes en destino y origen
				\4[] $\to$ Diferencia con población de origen y destino
				\4 Objetivo
				\4[] ¿Por qué han emigrado los que han emigrado?
				\4[] $\to$ ¿Por qué se observa una cualificación relativa dada?
				\4 Resultado
				\4[] Quiénes deciden emigrar:
				\4[] $\to$ Los que obtienen beneficio neto de emigración
				\4[] $\to$ Ganan más tras emigrar
				\4[] $\then$ Descontados costes de emigrar
				\4[] Ganar más depende de:
				\4[] $\to$ Media de salarios
				\4[] $\to$ Pago por cualificación del agente concreto
				\4[] $\to$ Coste de emigrar
			\3 Formulación
				\4 Salarios tienen dos componentes
				\4[] $w_0 = \underbrace{\mu_0}_\text{general} + \underbrace{v_0}_\text{individual}$ $\to$ Origen
				\4[] $w_1 = \underbrace{\mu_1}_\text{general} + \underbrace{v_1}_\text{individual}$ $\to$ Destino
				\4[] Componente general
				\4[] $\to$ Uno en cada país para todos los migrantes
				\4[] Componente individual
				\4[] $\to$ De acuerdo a distribución aleatoria
				\4[] $\to$ Uno en cada país para cada migrante
				\4[] Correlación entre $v_0$ y $v_1$
				\4[] $\to$ Conocida por migrante
				\4[] $\to$ Relaciona pago a habilidades en destino y origen
				\4[] Varianza de retribuciones en destino
				\4[] $\to$ Conocida también por migrante
				\4 Coste de emigración
				\4[] $\pi = \left( \frac{C}{w_0} \right)$
				\4 Condición de emigración
				\4[] $w_1 - \frac{C}{w_0}- w_0 = \left( \mu_1 - \mu_0 - \frac{C}{w_0} \right) - (v_1 - v_0) > 0$
				\4 Asumiendo igual salario medio neto de coste
				\4[] Emigración si $(v_1 -v_0)>0$
				\4[] ¿Cuándo sucede? $\to$ Combinación de factores:
				\4[] i. Correlación entre $v_1$ y $v_0$
				\4[] $\to$ ¿Características retribuidas = en ambos países?
				\4[] ii. Varianza de la distribución de salarios
				\4[] $\to$ ¿Qué variabilidad retributiva en cada país?
			\3 Implicaciones
				\4 Asumiendo:
				\4[] Costes de migración no dependen de cualificación
				\4[] Salarios medios idénticos
				\4 Estructura de emigración observada
				\4[] Depende de:
				\4[] $\to$ Retribución a la cualificación
				\4[] $\to$ Relación entre retribuciones de origen y destino
				\4[] Depende de:
				\4[] $\to$ Salario medio
				\4[] $\to$ Coste de emigración
				\4[] $\then$ Sólo afectan a volumen total de inmigrantes
				\4 Tres estructuras posibles:
				\4[] $\to$ Selección positiva
				\4[] $\to$ Selección negativa
				\4[] $\to$ Refugiados
				\4 Selección positiva
				\4[] Salario en origen
				\4[] $\to$ Por encima de la media
				\4[] Salario en destino
				\4[] $\to$ Por encima de la media
				\4[] Razones para emigrar:
				\4[] $\to$ Componentes individuales relacionados
				\4[] $\to$ Mayor varianza salarial en destino
				\4[] $\then$ Cualificados ganan más en destino que origen
				\4[] $\then$ Inmigrantes relativamente cualificados
				\4 Selección negativa
				\4[] Salario en origen
				\4[] $\to$ Por debajo de la media
				\4[] Salario en destino
				\4[] $\to$ Por debajo de la media
				\4[] Razones para emigrar
				\4[] $\to$ Componentes individuales relacionados
				\4[] $\to$ Menor varianza salarial en destino
				\4[] $\then$ No cualificados ganan más en destino que origen
				\4[] $\then$ Inmigrantes relativamente poco cualificados
				\4 ``Refugiados''
				\4[] Ejemplo:
				\4[] $\to$ Refugiados políticos fuera de URSS
				\4[] $\to$ Emigración cubana a Miami
				\4[] $\to$ Judíos emigrando de Alemania nazi
				\4[] Salario en origen
				\4[] $\to$ Por debajo de la media
				\4[] Salario en destino
				\4[] $\to$ Por encima de la media
				\4[] Razones para emigrar
				\4[] $\to$ Relación negativa entre comp. individuales\footnote{De hecho no basta con que la relación sea negativa, ha de ser suficientemente negativa. Ver \textit{international migration} en Palgrave, pág. 6761 (3ª ed.) De hecho, la relación negativa ha de ser más negativa que la variabilidad relativa entre destino y origen más pequeña. En términos formales: $\rho_{01} < \min \left(  \frac{\sigma_1}{\sigma_0} , \frac{\sigma_0}{\sigma_1} \right)$.}
				\4[] $\to$ Variabilidad relativa suficientemente desigual
				\4[] $\then$ Diferente valoración de habilidades
				\4[] $\then$ Factores políticos, religiosos, étnicos...
			\3 Valoración
				\4 Enorme impacto en literatura
				\4[] Primera gran aplicación de Roy (1951)
				\4[] Trabajo seminal de autoselección migratoria
				\4[] Múltiples usos a partir de 90s
				\4 Difícil contrastación empírica
				\4[] Parámetro de costes de emigración
				\4[] $\to$ Cajón de sastre para resultados contradictorios
				\4 Inmigrantes no son muestras aleatorias
				\4[] Son resultado de optimización y autoselección
				\4 No tiene en cuenta efectos de equilibrio general
				\4[] Difícilmente aplicable en largo plazo
		\2 Determinantes: dinámicas de aglomeración espacial
			\3 Idea clave
				\4 Contexto
				\4[] Von Thünen
				\4[] $\to$ Análisis de localización óptima en ciudades
				\4[] $\to$ Pionero en economía espacial
				\4[] Marshall
				\4[] $\to$ Economías de escala tecnológicas con concentración
				\4[] i. Más facilidad para encontrar mano de obra
				\4[] ii. Spill-overs de información
				\4[] iii. Producción de inputs intermedios no comerciables
				\4[] $\then$ Difícil formalización
				\4[] $\then$ Análisis canónico hasta NEG
				\4[] Teoría clásica del CI
				\4[] $\to$ Economías son puntos sin dimensión espacial
				\4[] $\to$ Factores inmóviles entre países
				\4[] Hotelling
				\4[] $\to$ Análisis pionero
				\4[] $\to$ Formalización de decisión de localización empresas
				\4[] Salop (1979)
				\4[] $\to$ Entrada de empresas en contexto espacial
				\4[] Dixit y Stiglitz (1977)
				\4[] $\to$ Formalización de competencia monopolística
				\4[] $\to$ Análisis formal de equilibrio general
				\4[] $\then$ Agentes prefieren variedad
				\4[] $\then$ Incentivos a entrada de nuevas empresas/variedades
				\4[] Evidencia empírica
				\4[] $\to$ Aparición de núcleos y cinturones industriales
				\4[] $\to$ Concentración de población donde hay desarrollo industrial
				\4[] $\to$ Trabajo industrial móvil con patrones persistentes
				\4 Objetivos
				\4[] Explicar dinámicas de aglomeración de población
				\4[] $\to$ En contexto de desarrollo industrial
				\4[] $\to$ En contexto de bajada de precios de transporte
				\4 Resultados
				\4[] Dinámicas de aglomeración-dispersión
				\4[] $\to$ Dependen de parámetros clave
				\4[] Integración comercial
				\4[] $\to$ Puede inducir aglomeración
				\4[] Movimientos de trabajadores
				\4[] $\to$ Endógenos
				\4[] $\to$ Posible concentración geográfica
			\3 Formulación
				\4 Dos bienes consumidos
				\4[] Agrícola homogéneo $C_A$
				\4[] Manufacturado compuesto $C_M = \left( C_i^{\frac{\epsilon-1}{\epsilon}} \right)^{\frac{\epsilon}{\epsilon-1}}$
				\4 Dos factores de producción
				\4[] Campesinos inmóviles entre países
				\4[] $\to$ Distribuidos entre los dos países de manera exógena
				\4[] Obreros móviles entre países
				\4 Dos países/regiones A y B
				\4[] Campesinos repartidos equitativamente entre países
				\4[] Obreros con distribución inicial arbitraria
				\4[] $\to$ Sujeto a variación endógena
				\4 Demanda de bienes
				\4[] Obreros y campesinos iguales demandas
				\4[] $\to$ Se distribuye entre agrícola y manufacturero
				\4 Coste de transporte
				\4[] Agrícola sin coste de transporte
				\4[] Manufacturero con costes tipo iceberg
				\4[] $\to$ Para que llegue 1 hace falta enviar $\tau > 1$
				\4 Decisión de localización de obreros
				\4[] Donde haya mayor salario
				\4[] Dos efectos contrapuestos afectan localización
				\4[] $\to$ Efecto competencia
				\4[] $\to$ Efecto demanda
				\4 Efecto demanda
				\4[] Localización cerca de la demanda
				\4[] $\to$ Permite superar costes de transporte
				\4[] Si coste fijo superior a costes de transporte
				\4[] $\to$ Preferible concentrar producción
				\4[] Cuanta mayor población obrera
				\4[] $\to$ Más se retroalimenta el efecto demanda
				\4[] Producción de variedades manufactureras concentradas
				\4[] $\to$ Aumenta salario real de obreros en aglomeración
				\4[] $\then$ Tendencia hacia concentración donde ya se produce manufact.
				\4 Efecto competencia
				\4[] Costes de transporte
				\4[] $\to$ Reducen competencia con variedades en otro país
				\4[] $\then$ Permiten aumentar precios
				\4[] Cuanta más población campesina sobre total
				\4[] $\to$ Mayor es la demanda que no se mueve
				\4[] $\then$ Más incentivos a localizarse donde no se producen variedades
				\4 Parámetros iniciales determinan resultado
				\4[] Preferencia por la variedad $\epsilon$
				\4[] $\then$ Aumenta importancia de tener más variedades
				\4[] Costes de transporte
				\4[] $\to$ Reduce competencia con variedades en otro país
				\4[] $\to$ Actúa a favor de la aglomeración
				\4[] Peso del sector manufacturero en población
				\4[] $\to$ Aumenta efecto de movimiento de L sobre demanda
				\4[] $\then$ Actúa a favor de la aglomeración
				\4 Dinámica del movimiento de obreros y comercio
				\4[] Asumiendo
				\4[] $\to$ Preferencia suficiente por la variedad
				\4[] $\to$ Suficiente peso del sector manufacturero
				\4[] Dispersión en equilibrio
				\4[] $\to$ Costes de transporte elevados +  pob. obrera reducida
				\4[] $\to$ Elevado efecto competencia
				\4[] $\to$ Poco efecto demanda
				\4[] $\then$ Tendencia a dispersión
				\4[] $\then$ \grafica{krugman91dispersion}
				\4[] Múltiples equilibrios
				\4[] $\to$ Costes de transporte intermedios + pob. obrera moderada
				\4[] $\to$ Si población dispersa, tendencia a dispersión
				\4[] $\to$ Si población inicialmente aglomerada, tendencia aglom.
				\4[] $\then$ Equilibrio depende de shock/condición inicial
				\4[] $\then$ Múltiples equilibrios dispersos y aglomerados
				\4[] $\then$ \grafica{krugman91multiplesequilibrios}
				\4[] Aglomeración en equilibrio
				\4[] $\to$ Costes de transporte reducidos + elevada pob. obrera
				\4[] $\to$ Poco efecto competencia
				\4[] $\to$ Efecto demanda elevado
				\4[] $\then$ Tendencia a aglomeración
				\4[] $\then$ \grafica{krugman91aglomeracion}
				\4[] Sin costes de transporte y con costes de congestión
				\4[] $\to$ Posible dispersión de nuevo
				\4[] $\to$ Sin home-market effect
				\4[] $\to$ Costoso concentrarse
				\4[] $\to$ Sin costes de exportar
				\4[] $\then$ Dispersión máxima
			\3 Implicaciones
				\4 Movimiento de factores
				\4[] Permite explicar patrón migratorio en siglo XIX y XX
				\4[] Campo a la ciudad
				\4[] $\to$ Al reducirse CdTransporte
				\4[] $\to$ Al aumentar demanda de bienes industriales
				\4 Núcleo y periferia
				\4[] Núcleo
				\4[] $\to$ Concentración de obreros
				\4[] $\to$ Concentración de variedades industriales
				\4[] $\to$ Salarios elevados
				\4[] $\to$ Exportación de producto manufacturado
				\4[] $\to$ Importación de productos agrícolas
				\4[] Periferia
				\4[] $\to$ Sin obreros
				\4[] $\to$ Sin variedades industriales
				\4[] $\to$ Salarios reducidos
				\4[] $\to$ Importación de producto manufacturado
				\4[] $\to$ Exportación de productos agrícolas
				\4 Integración comercial induce aglomeración
				\4[] Posible aumento desigualdades regionales
				\4[] Posibles tensiones de economía política
			\3 Valoración
				\4 Premio Nobel a Krugman en 2008
				\4[] Culmina programa de comp. monop. y EEscala en CI
				\4 Abre programa de investigación
				\4[] Geografía económica basada en
				\4[] $\to$ Externalidades pecunarias
				\4 De manera paradójica, mundo se vuelve más clásico\footnote{Ver conclusión de Krugman (2008) Nobel Prize Lecture.}
				\4[] En últimas décadas
				\4[] Aumenta comercio basado en VComparativa
				\4[] $\to$ Cadenas de valor global
				\4[] $\to$ Especialización
				\4[] $\to$ IDE vertical frente a horizontal
		\2 Determinantes: redes de inmigración
			\3 Idea clave
				\4 Contexto
				\4[] Programa de investigación de redes
				\4[] $\to$ Origen en física y sociología
				\4[] $\to$ En economía de forma creciente
				\4[] Impacto de estructura de relaciones sociales
				\4[] $\to$ Sobre decisiones económicas de agentes
				\4[] Migración fuertemente influencidada por redes
				\4[] $\to$ Proveen a migrantes con información sobre mercado local
				\4 Objetivos
				\4[] Caracterizar efectos de distintas estructuras de RRSS
				\4[] $\to$ Sobre incentivos a migración
				\4[]
				\4 Resultados
				\4[] Programa de investigación incipiente
				\4[] Evidencia empírica muestra efecto de redes sobre migración
				\4[] $\to$ Redes más densas en destino aumentan no cualificada
			\3 Formulación
				\4 Redes sociales
				\4[] Grafos que caracterizan relaciones
				\4[] $\to$ Generalmente no direccionales
				\4[] Posible valorar conexión
				\4 Redes permiten reducir costes de emigración
				\4[] Encontrar trabajo más fácilmente
				\4[] Acceso a mercados de capital en mercado de destino
			\3 Implicaciones
				\4 Educación y densidad de las redes\footnote{\url{http://discovery.ucl.ac.uk/14286/1/14286.pdf}}
				\4[] Comunidades con redes débiles de migrantes
				\4[] $\to$ Sesgo hacia migrantes más cualificados
				\4[] Comunidades con redes fuertes de migrantes
				\4[] $\to$ Sesgo hacia migrantes menos cualificados
			\3 Valoración
				\4 Programa de investigación naciente
				\4 Mayor disponibilidad de datos en actualidad
				\4[] Puede facilitar estudio de migraciones y rr.ss.
		\2 Otros determinantes de las migraciones
			\3 Sustitutivo del comercio internacional
				\4 Movilidad de factores puede sustituir CI
				\4[] Aprovechamiento de ganancias similares
				\4 Planteado ya en Heckscher-Ohlin
				\4[] Si comercio internacional totalmente cerrado
				\4[] $\to$ Pero movilidad completa de ff.pp
				\4[] $\then$ Efecto idéntico
				\4 Problema de los costes
				\4[] Movilidad de factores tiene mismos costes que CI?
				\4[] $\to$ Generalmente, no
				\4[] $\to$ En migración especialmente no
				\4[] $\then$ Costes de aglomeración
				\4[] $\then$ Costes de adaptación a nuevo mercado
			\3 Modelo de factores específicos Ricardo-Viner
				\4 Con plena movilidad de factores entre industrias
				\4[] $\to$ Misma remuneración a factores en industrias
				\4[] $\then$ Salarios reales deben igualarse
				\4 Sin movilidad de K entre industrias pero plena mov. de L
				\4[] Aumento de productividad de industria Y
				\4[] $\to$ Aumento de PMgL en industria Y
				\4[] $\then$ Aumento de salario de industria Y
				\4[] $\then$ Movilidad de trabajo de industria X a Y
				\4 Trabajo nacional no logra igualar PMgL entre industrias
				\4[] Incentivo a movilidad del trabajo extranjero
				\4[] $\to$ Inmigración hasta igual PMgL entre sectores
		\2 Efectos
			\3 Modelo de MacDougall (1960): igualación de PMgL
				\4 Efecto de migración sobre productividad del trabajo
				\4[] Previo a Borjas (1987)
				\4[] $\to$ Sin tener en cuenta
				\4 Adaptación de MacDougall (1960) a laboral
				\4[] Modelo original examina también capital
				\4 Supuestos
				\4[] Dos economías
				\4[] F. de prod. con rdtos. constantes a escala
				\4[] PMg decreciente en L y K
				\4[] Capital constante
				\4[] Trabajo total constante
				\4[] Mercados competitivos de trabajo
				\4[] $\to$ Salario igual a productividad marginal
				\4[] Un país con más trabajo que otro
				\4[] $\to$ Diferentes
				\4 Apertura de movimiento de trabajo
				\4[] Trabajadores migran hacia mayores salarios
				\4[] $\to$ Productividades se igualan
				\4[] $\to$ Salarios se igualan
				\4[] País con productividad más alta
				\4[] $\to$ Recibe trabajadores
				\4[] $\to$ Salario y PMg bajan hasta igualarse
				\4 Representacion gráfica
				\4[] \grafica{macdougall}
			\3 Impacto sobre distribución de la renta
				\4 Efecto de migración sobre rentas de nativos
				\4[] Asumiendo:
				\4[] $\to$ Nativos son dueños del capital
				\4[] $\to$ Población fija de nativos
				\4[] $\to$ Trabajo es homogéneo
				\4[] ¿Cómo afecta emigración a sus rentas?
				\4 Dotación fija de capital
				\4[] $\uparrow$ L $\then$ $\downarrow$ $\frac{K}{L} \then \downarrow \text{PMg}_L, \uparrow \text{PMg}_K$
				\4[] Cae PMg de L
				\4[] $\to$ Bajan salarios
				\4[] Aumenta PMg de K
				\4[] $\to$ Aumenta rendimiento del capital
				\4[] $\then$ Aumenta renta del capital de nativos
				\4[] $\then$ Caen rentas salariales de nativos
				\4[] $\then$ Migración aumenta excedente
				\4[] \grafica{rentascapitalinelastico}
				\4 Oferta elástica de capital
				\4[] Inmigración no afecta a PMg ni de K ni de L
				\4[] $\to$ Aumento de PMgK atrae capital
				\4[] $\to$ K por trabajador se mantiene constante
				\4[] $\then$ Rentas de K de nativos constantes
				\4[] $\then$ Inmigración no altera mercado laboral a l/p
			\3 Brain drain vs brain gain vs brain waste\footnote{Ver \href{https://pubs.aeaweb.org/doi/pdfplus/10.1257/jep.25.3.107}{Gibson y McEnkenzie (2011) en JEP}.}
				\4 Brain drain
				\4[] Emigración de trabajadores de elevada cualificación
				\4[] $\to$ Que reduce stock de capital humano nacional
				\4[] ¿Tiene efectos negativos sobre mercado de origen?
				\4[] $\to$ No, si mercado laboral perfec. competitivo
				\4[] $\then$ Trabajadores remunerados a producto marginal
				\4[] $\then$ Sin externalidades de cualificación sobre otros
				\4[] $\then$ Sin complementariedades de trabajo cualificado
				\4[] $\to$ No, si transferencias fiscales origen/destino
				\4[] $\then$ Origen se beneficia de complemetariedades en destino
				\4[] $\to$ Sí, si extern+complemen. no compensadas
				\4[] $\then$ Sin transferencias fiscales a nivel internacional
				\4[] Hay realmente brain drain?
				\4[] $\to$ Emigración cualificada aumenta en términos absolutos
				\4[] $\to$ Pero cualificación en mercados de origen aumenta también
				\4[] $\then$ Efecto escaso sobre cualificación media
				\4[] $\then$ Brain drain puede incluso caído en últimos años
				\4 Brain gain
				\4[] Fenómeno postulado
				\4[] Aumento de emigración de cualificados
				\4[] $\to$ Acaba aumentando cualificación en mercado de origen
				\4[] ¿Por qué?
				\4[] $\to$ Posibilidad de emigrar incentiva inversión en KHumano
				\4[] $\then$ Finalmente, no todos emigran
				\4[] $\to$ Envío de remesas a país de origen
				\4[] $\to$ Inversión a país de origen
				\4 Brain waste
				\4[] Profesionales altamente cualificados que emigran
				\4[] $\to$ Trabajan en destino en profesiones menos cualificadas
				\4 Complementariedades de trabajo cualificado y no cualificado
				\4[] En la práctica, es difícil discriminar inmigración
				\4[] $\to$ Cualificada atrae no cualificada
				\4 Evidencia empírica al respecto
				\4[] Emigrantes no están generalmente más cualificados que media
				\4[] Brain drain afecta sobre todo a economías pequeñas y aisladas
				\4[] $\to$ Grandes mercados conectados sufren menos aunque PED
				\4[] Remesas compensan externalidades fiscales
				\4[] $\to$ En gran medida aunque no totalmente
				\4[] Otras externalidades son relevantes
				\4[] $\to$ Cambios institucionales tras retorno de emigrantes
				\4[] $\to$ Cambios en perfil demográfico de país de origen
			\3 Análisis de redes
				\4 Aparición de vínculos comerciales
				\4[] Resultado de redes de inmigración previas
				\4[] Ejemplo paradigmático:
				\4[] $\to$ Emigrantes chinos en regiones europeas
				\4 Transmisión de shocks macroeconómicos
				\4[] Redes de emigración pueden contribuir a amplificar
				\4[] $\to$ Aumento de vínculos comerciales
				\4[] Mercado de origen sufre shock adverso
				\4[] $\then$ Caída de exportaciones de destino a origen
				\4 Masa crítica y efectos de red
				\4[] Existen umbrales de emigración
				\4[] $\to$ A partir de los cuales efectos son relevantes
			\3 Inmigrantes como inputs diferenciados
				\4 Nivel educativo es clave de diferenciación
				\4[] ¿Cuánto debe afinarse el grado de diferenciación?
				\4[] $\to$ dos clases: universitaria o no
				\4[] $\to$ múltiples segmentos educativos
				\4 Diferenciar entre sectores es importante
				\4[] No sólo diferenciar por tipo de trabajador
				\4[] Por ejemplo, trabajo manual vs. no manual
			\3 Externalidades de la inmigración sobre población local
				\4 Respuesta de trabajadores y empresas
				\4[] Agentes nativos optimizan decisión
				\4[] $\to$ En función de emigración recibida
				\4[] $\then$ Otra cara de la moneda de Borjas (1987)
				\4[] Inmigrantes son sustitutos de nativos
				\4[] $\to$ Nativos abandonan industria sustituida
				\4[] $\to$ Nativos cambian a industria complementada
				\4[] Ejemplo:
				\4[] $\to$ Nativos aumentan tasas de graduación
				\4[] $\to$ Nativos abandonan industrias manuales
				\4[] Salarios relativos acentúan el proceso
				\4[] $\to$ Bajan allí donde reciben inmigrantes
				\4[] $\to$ Aumentan en industrias complementadas
				\4 Evidencia empírica
				\4[] Resultado habitual emigración sur-norte
				\4[] Trabajadores nativos en norte
				\4[] $\to$ Aumentan aprendizaje e innovación
				\4[] Se concentran en áreas urbanas
				\4[] $\to$ Aumentan densidad de actividad económica
				\4 Desplazamiento de población nativa
				\4[] Generalmente, hacia sectores con menos inmigrantes
				\4[] Inmigrantes menos cualificados
				\4[] $\to$ Nativos desplazados a más cualificación
				\4[] También posible lo contrario
				\4[] $\to$ Ej.: comerciantes en Asia
			\3 Variedad en lugar de origen
				\4 $\to$ Aumenta transferencia de ideas y tecnoloǵias
				\4[] Incremento de precio de ff.pp. fijos
				\4[] $\to$ Fundamentalmente precio de vivienda
				\4[] $\to$ Parcialmente compensado con $\uparrow$ salarios y empleo
	\1 \marcar{Movimientos de capital}
		\2 Idea clave
			\3 Contexto
				\4 Diferentes formas de capital
				\4[] Capital físico
				\4[] Capital financiero
				\4[] IDE
				\4[] Inversión de cartera
				\4[] Otra inversión
				\4 Fenómeno de las multinacionales
				\4[] Creciente presencia desde últimas décadas
				\4[] Empresas con presencia y centros de producción
				\4[] $\to$ En diferentes países
				\4[] En vez de fabricar en país de origen y exportar
				\4[] $\to$ Trasladan centros de producción

			\3 Objetivos
				\4 Definir hechos empíricos sobre capital a explicar
				\4 Explicar discordancias de hechos empíricos con otras teorías
				\4[] Modelo de H-O
				\4[] $\to$ CI depende de diferencias de factores
				\4[] $\to$ Con libre comercio, mov. de ff.pp no tiene sentido
				\4[] Modelo neoclásico
				\4[] $\to$ Capital fluye mayor remuneración
				\4[] $\to$ Capital remunerado a PMgK en equilibrio
				\4[] $\then$ De hecho, no fluye como sería de esperar
				\4[] Modelo intertemporal de la balanza de pagos
				\4[] $\to$ Contexto de mercados de capital integrados
				\4[] $\to$ Países consumen en función de renta permanente
				\4[] $\then$ Pero en la práctica, ahorro e inversión van unidos
				\4 Por qué fluye relativamente poco K
				\4[] Determinantes que resuelven Lucas (1990)
				\4 Por qué capital ha fluido de PEDs a desarrollados
				\4[] Determinantes que resuelven
				\4 Qué efecto tienen los flujos de K
				\4[] No hay relación clara entre flujo de K y crecimiento
				\4 Particularidades de IDE
			\3 Resultados
				\4 Explicaciones dentro de modelo neoclásico
				\4 Explicaciones superando marco neoclásico
				\4 Modelos de la IDE
		\2 Fenómenos a explicar
			\3 Hechos estilizados\footnote{Fundamentalmente extraído de Boz et al (2017). Esto debería sincronizarse con tema 3B-33.}
				\4 Crecimiento de los flujos de K
				\4[] Fuerte aumento desde 1980
				\4[] Alrededor de 1.2 billones USD en 80s
				\4[] Más de 6 billones USD a partir de 2000s
				\4 Dirección de los flujos de capital
				\4[] Periodo pre-crisis
				\4[] $\to$ Capital fluye hacia desarrollados
				\4[] $\to$ Más de 1\% de PIB mundial
				\4[] Periodo post-crisis
				\4[] $\to$ Sale capital de avanzados a partir de 2010
				\4[] $\to$ Emergentes reducen salidas de capital
				\4 Principales emisores de flujos de capital
				\4[] Exportadores de materias primas + China
				\4[] $\to$ Enorme crecimiento a partir de 2000s
				\4 Composición de los flujos
				\4[] IDE neta: emergentes + PEDs reciben en general
				\4[] No IDE y reservas: sale capital de emergentes\footnote{Es decir, los emergentes acumulan reservas y otros activos.}
				\4[] A partir de 2013, emergentes gastan reservas
				\4[] $\to$ Sigue saliendo capital no IDE ni reservas
				\4 En conjunto:
				\4[] Flujos de capital aumentan con el tiempo
				\4[] Capital sale de emergentes en 2000s hasta crisis
				\4[] Grandes flujos de capital hacia avanzados en 2000s
				\4[] $\to$ Sobre todo de China y exportadores materias primas
				\4[$\then$] Múltiples paradojas respecto a modelos teóricos
			\3 Contradicción con modelo H-O-S
				\4 Asumiendo:
				\4[] $\to$ marco de equilibrio general/H-O-S
				\4[] $\to$ apertura comercial muy avanzada en bienes
				\4 Teorema de la igualación de factores
				\4[] Apertura en bienes $\then$ igualación precio de ff.pp.
				\4[] $\then$ No habría motivos para flujos de K
				\4[] $\then$ Deberían ser mucho más pequeños que realmente
			\3 Puzzle de Feldstein y Horioka (1980)
				\4 Regresión de muestra de países OCDE:
				\4[] Tasa de inversión doméstica
				\4[] $\to$ Respecto de tasa de ahorro doméstico
				\4 Resultados
				\4[] Coeficiente cercano a 1
				\4[] $\to$ Relación entre ahorro e inversión
				\4 Puzzle
				\4[] Si mercados de capital estuviesen integrados
				\4[] $\to$ No debería haber relación entre S e I
				\4[] $\to$ Coeficiente debería acercarse a 0
			\3 Paradoja de Lucas (1990)
				\4 Principal paradoja respecto de teoría neoclásica
				\4 Asumiendo:
				\4[] Función de producción Cobb-Douglas
				\4[] Dos países
				\4[] Dos factores: trabajo y capital
				\4[] Distintas dotaciones de capital
				\4[] $\to$ Distinto capital por trabajador
				\4[] Misma dotación de trabajo
				\4[] Misma productividad total de los factores
				\4 Paradoja
				\4[] Mismo PTF y diferente PMg del trabajo
				\4[] $\then$ Diferente K por trabajador
				\4[] Tomando USA e India como ejemplo
				\4[] $\to$ $\text{PMg}_K$ de USA es casi 60 veces $\text{PMg}_K$ en India
				\4[$\to$] ¿Por qué no fluye K de USA a India hasta igualar?
				\4[$\to$] ¿Qué supuesto neoclásico es erróneo?
			\3 Ajuste intertemporal de la balanza de pagos
				\4 Enfoque intertemporal de la balanza de pagos
				\4[] Renta temporal menor a renta permanente
				\4[] $\to$ País es deudor
				\4[] $\to$ Entra capital
				\4[] Renta temporal mayor a renta permanente
				\4[] $\to$ País es acreedor
				\4[] $\to$ Sale capital del país
				\4[$\then$] Cabría esperar capital saliese de EEUU
				\4[] Sin embargo, los EEUU importan capital
			\3 Allocation puzzle de Gourinchas (2007)\footnote{En español ``\textit{rompecabezas de la asignación}''.}
				\4 PTF es importante determinante de $\text{PMg}_K$
				\4[] Crecimiento de PTF
				\4[] $\then$ Crecimiento de $\text{PMg}_K$
				\4[$\then$] Crecimiento de PTF debería atraer K
				\4 Hallazgo empírico de Gourinchas (2007)
				\4[] K fluye a países que invierten y crecen menos
		\2 Determinantes
			\3 Soluciones a Lucas (1990) dentro de modelo neoclásico
				\4 El propio Lucas (1990) plantea
				\4[i] L más productivo en desarrollados
				\4[] Cada trabajador equivale a varios trabajadores en PEDs
				\4[ii] Modelo no considera capital humano
				\4[] PEDs escasos en capital humano
				\4[] Capital humano complementario a capital
				\4[] $\to$ No tenido en cuenta por el modelo
				\4[] $\then$ PMgK es realmente más alta en desarrollados
				\4[iii] PTF más alta en países desarrollados
				\4[] (No propuesta por Lucas)
				\4[] (Generaliza anterior relativa a L más productivo)
				\4[] Múltiples factores:
				\4[] -- Calidad de instituciones
				\4[] -- Protección de derechos de propiedad
				\4[] -- Derechos de propiedad
				\4[] $\to$ Parcialmente fuera de prog. neoclásico
			\3 Soluciones más allá de marco neoclásico
				\4 Capital físico diferente de financiero
				\4[] Capital físico es muy rentable en PEDs
				\4[] Capital financiero no lo es tanto
				\4[] $\to$ Riesgo alto en PEDs
				\4[] $\to$ Sistemas financieros poco desarrollados
				\4[] $\to$ Costes de intermediación elevados
				\4[] $\to$ Inestabilidad política $\uparrow$ atractivo desarrollados
				\4[] $\then$ Capital físico fluye hacia PEDs
				\4[] $\then$ Capital financiero fluye hacia desarrollados
				\4 Firmas heterogéneas
				\4[] Ju y Wei (2006)
				\4[] Retornos K financiero y físico
				\4[] $\to$ Divergentes
				\4[] $\to$ Subdesarrollo financiero aumenta diferencia
				\4 Calidad de instituciones
				\4[] Reducen rentabilidad de capital físico en PEDs
				\4[] $\to$ Corrupción
				\4[] $\to$ Derechos de propiedad frágiles
			\3 Soluciones a puzzle de la asignación
				\4 Ahorro
				\4[] Paradoja muestra correlación positiva entre
				\4[] $\to$ Crecimiento
				\4[] $\to$ S -- I
				\4[] Equivalente a $\text{cov} (g,s) > \text{cov} (g,i)$
				\4[] Explicación:
				\4[] $\to$ Causalidad entre ahorro y crecimiento
				\4[] $\then$ PTF crece y sale capital
				\4 Comercio
				\4[] Sectores exportadores competitivos
				\4[] $\to$ Impulsan crecimiento de PTF
				\4[] Países en desarrollo con exportaciones competitivas
				\4[] $\to$ Tratan de mantener tipo real bajo
				\4[] Sectores no exportadores poco productivos
				\4[] $\to$ Demanda interna débil
				\4[] $\then$ Capital sale de los que mejoran PTF
		\2 Efectos
			\3 Caída del coste de financiación
				\4 Más capital disponible
				\4 Más liquidez disponible
				\4 Reduce riesgo moral
				\4[] Agentes deben tomar menos riesgos
				\4 Reduce selección adversa
				\4[] Menos proyectos rentables de inversión salen de mercado
			\3 Aumento de inversión
				\4 Teorías de demanda de inversión
				\4 Relación consistente
				\4[] Menor coste de financiación
				\4[] $\to$ Mayor inversión
			\3 Apreciación del tipo de cambio
				\4 Aumenta demanda de moneda nacional
				\4 Reduce exceso de demanda de divisas
				\4[$\then$] Apreciación
			\3 Burbujas especulativas
				\4 Entradas masivas de capital
				\4[] Reducen coste de financiación
				\4 Ausencia de oportunidades rentables de inversión
				\4[] Exceso de demanda de activos domésticos
				\4[] $\to$ Generalmente, activos inmobiliarios
				\4 Aumento constante de precios
				\4[] Activos domésticos se convierten en
				\4[] $\to$ Reserva de valor
				\4[] $\to$ Activo de inversión rentable
				\4 Entrada de capital mantiene proceso de burbuja
				\4[] Más capital para aumentar demanda y precios
				\4[] $\to$ Más aumento de los precios
				\4[] $\then$ Mayor entrada de capital
				\4 Fin de la burbuja
				\4[] Precios dejan de subir
				\4[] $\then$ Entrada de capital sufre parada brusca
			\3 Sudden stops y reversiones de flujos de capital
				\4 Reducción súbita de entradas de capital
				\4 Ocurren relativamente frecuentemente
				\4 Especialmente en países en desarrollo/emergentes
				\4 Persisten al menos un año, generalmente
				\4 Sudden stop y flow reversal al tiempo
				\4 Inducen depreciación del tipo de de cambio
				\4[] No quedan otras herramientas de ajuste disponibles
				\4 Inducen caídas fuertes del PIB via $\downarrow$ I
			\3 Teoría
				\4 Entrada de K tiene efectos positivos
				\4[] Canales directos e indirectos
				\4 Directos
				\4[] i. Aumenta ahorro doméstico
				\4[] $\to$ Vía aumento de la producción
				\4[] ii. Reduce coste del capital
				\4[] $\to$ Puede inducir mejor gestión del riesgo
				\4[] iii. Transfiere tecnología y know-how de gestión
				\4[] iv. Estimula desarrollo sector financiero doméstico
				\4 Indirectos
				\4[] i. Aumenta especialización
				\4[] $\to$ Posibilita economías de escala
				\4[] ii. Mejora calidad de política económica
				\4[] $\to$ Disciplina vía mercados
			\3 Efecto composición
				\4 No todos los flujos de capital son iguales
				\4[] Diferentes tipos de K, diferentes efectos
				\4[] IDE
				\4[] $\to$ Asociada robustamente a crecimiento
				\4 Inversión de cartera
				\4[] $\to$ Vínculo ambiguo con crecimiento
				\4 Deuda extranjera privada
				\4[] $\to$ Sin evidencia clara
				\4[] $\to$ Ocasionalmente negativa
				\4 Ayuda oficial
				\4[] $\to$ Sin vínculo robusto con crecimiento
				\4[] Crisis monetarias y financieras
				\4[] $\to$ Relacionadas con tipos de flujo de K
			\3 Efecto umbral
				\4 Efecto positivo de flujos de K
				\4[] $\to$ Requiere condiciones mínimas
				\4 Necesarias:
				\4[] $\to$ Instituciones suficientemente buenas
				\4[] $\to$ Control de la corrupción
				\4[] $\to$ Capital humano mínimo
				\4[] $\to$ Capacidad de gestión
				\4 Efecto umbral causa de efecto composición:
				\4[] $\to$ Umbral mínimo para atraer flujos IDE
				\4[] $\to$ Sin umbral, K capturado por élite
				\4[] $\to$ Hipótesis: resultados mixtos
			\3 Hechos empíricos
				\4 Resultados mixtos
				\4[] Objetivo:
				\4[] $\to$ relacionar crecimiento y entrada de K
				\4[] Difícil encontrar vinculo robusto y significativo\footnote{Eichengreen (2001), Prasad et al (2003).}
				\4 Difícil separar causalidades
				\4[] $\to$ ¿Por qué?
				\4 Crisis financieras e integración financiera
				\4[] Países que se integran a mercados de K
				\4[] $\to$ Sufren crisis financieras
				\4[] Pero ningún país prefiere aislarse después
				\4[] $\to$ ¿Por qué no?
				\4[] $\to$ ¿Por qué integración trae problemas?
		\2 Inversión directa extranjera
			\3 Idea clave
				\4 Definición de IDE
				\4[] Manual BP6 del FMI
				\4[] $\to$ control o grado significativo de influencia
				\4[] $\to$ >50\% control
				\4[] $\to$ >10\% grado significativo de influencia
				\4 Tipos de IDE
				\4[] \underline{Según uso de la IDE}
				\4[] Greenfield:
				\4[] $\to$ Inversor crea nueva planta/empresa
				\4[] Brownfield:
				\4[] $\to$ Inversor adquiere planta existente
				\4[] \underline{Según sentido de la integración}
				\4[] Vertical
				\4[] $\to$ Invierte en dif. etapas productivas
				\4[] Horizontal
				\4[] $\to$ Replica proceso productivo
				\4[] Conglomerado
				\4[] $\to$ Diferentes procesos y etapas
				\4 Hechos empíricos
				\4[] Crecimiento acelerado a final de s.XX
				\4[] $\to$ Más rápido que PIB y CI
				\4[] Desde 2000
				\4[] $\to$ Aumento de 50\% hasta hoy
				\4[] Últimos tres años
				\4[] $\to$ Desaceleración relativa de la IDE
				\4[] $\to$ Caída en proyectos greenfield
				\4[] $\to$ Especialmente en PEDs
				\4[] Países desarrollados
				\4[] $\to$ Reciben y emiten mayor parte de IDE
				\4[] $\to$ Han reducido peso relativo
				\4[] CI entre filiales de misma multinacional
				\4[] $\to$ Casi 50\% del CI
				\4[] $\then$ Multinacionales pieza clave de CI
			\3 Hechos estilizados\footnote{Ver Antràs y Yeaple (2014) pág. 59 y ss..}
				\4[i] Bidireccional y norte-norte prevalece
				\4[] PEDs también reciben pero menor proporción
				\4[] Países más ricos:
				\4[] $\to$ Emiten más IDE
				\4[] $\to$ Reciben algo más de IDE que pobres
				\4[] Países más pobres:
				\4[] $\to$ Emiten menos IDE
				\4[] $\to$ Reciben más IDE que emiten
				\4[] $\to$ Más propensos a recibir que a emitir
				\4[] Mayor parte de la IDE es entre desarrollados
				\4[ii] Predominan IDE intraindustria intensiva en K e I+D
				\4[] Cuanto menos intensivo en K e I+D
				\4[] $\to$ Más probabilidad de no IDE
				\4[] $\then$ Comercio con empresas afiliadas, no filiales
				\4[] Sectores intensivos en K e I+D
				\4[] $\to$ Prefieren IDE a comerciar con afiliados
				\4[] $\then$ Probable debido a especificidad de activos
				\4[iii] Menor caída con distancia que con comercio
				\4[] Ec. de gravedad explica bien comercio e IDE
				\4[] $\to$ Relación entre comercio/IDE-distancia
				\4[] Caída de IDE respecto distancia
				\4[] $\to$ Menor a caída respecto distancia
				\4[iv] Multinacionales son más grandes que competidores no MN
				\4[] Más:
				\4[] $\to$ Grandes
				\4[] $\to$ Productivas
				\4[] $\to$ Intensivas en I+D
				\4[] $\to$ Exportadoras
				\4[v] MN matriz especializadas I+D, filiales en venta
				\4[] Matrices de MN se especializan en I+D
				\4[] Filiales se especializan sobre todo en venta
				\4[] $\to$ A mercado propio
				\4[] $\to$ A mercados extranjeros
				\4[vi] F\&A son \% elevado de IDE y entrada en desarrollados
				\4[] Gran parte de IDE corresponde a F\&A
				\4[] Especialmente importante para entrar en desarrollados
				\4[] $\to$ En PEDs, F\&A es \% menor
			\3 Determinantes: marco OLI
				\4 No formalizado matemáticamente
				\4 Tres factores inducen empresa a ser multinacional
				\4 Influencia de Grossman y Hart
				\4[] Enfoque de derechos de propiedad
				\4[] $\to$ Como justificación existencia de empresa
				\4[]
				\4[] \textit{Ownership}
				\4 Posee determinado activo que genera valor suficiente
				\4[] $\to$ Si utilizado en varios plantas
				\4[] $\to$ Activos similares a bienes públicos
				\4[] $\to$ P.ej.: patentes, copyrights, patrones
				\4[] $\to$ P.ej.: equipos directivos, marcas...
				\4 Sin factor ownership/propiedad:
				\4[] $\to$ La firma no tiene razón de ser
				\4 Explica:
				\4[] $\to$ Multinacionales más grandes que no mnacionales
				\4[] $\then$ Concentran I+D
				\4[] $\then$ Ventaja por poseer activo
				\4[] \textit{Location}
				\4 Localización en extranjero aporta ventajas
				\4[] Por ejemplo:
				\4[] $\to$ Reducción de costes de transporte
				\4[] $\to$ Evitar aranceles
				\4[] $\to$ Aprovechar factores abundantes
				\4[] $\to$ Influye en IDE horizontal y vertical
				\4 Sin factor location/localización
				\4[] $\to$ La firma preferirá exportar
				\4[] \textit{Internationalization}
				\4 Expansión internacional es ventajosa
				\4[] Ejemplo:
				\4[] $\to$ Proceso productivo fácil de replicar
				\4[] $\to$ Producir dentro de la firma mejor que licenciar
				\4[] $\to$ Externalización difícil
				\4 Sin ventaja de internacionalización
				\4[] $\to$ Preferible licenciar a empresa extranjera
			\3 Determinantes: proximidad vs concentración -- Helpman, Melitz y Yeaple (2004)
				\4 Idea clave
				\4[] Contexto
				\4[] $\to$ Tendencia tras segunda guerra mundial
				\4[] $\to$ Aumento de replicación de plantas en extranjero
				\4[] $\to$ Desde años 80
				\4[] $\to$ Aparición de multinacionales con IDE horizontal
				\4[] $\to$ Melitz (2003): empresas heterogéneas en CI
				\4[] $\to$ Helpman, Melitz y Yeaple (2004)
				\4[] $\then$ Modelo de empresas heterógeneas aplicado a IDE
				\4[] Objetivos
				\4[] $\to$ Explicar IDE horizontal en vez de exportación
				\4[] Resultados
				\4[] $\to$ Trade off entre concentración y proximidad
				\4[] $\then$ Costes de transporte vs economías de escala
				\4[] $\to$ Costes de transporte para todos
				\4[] $\to$ Economías de escala más fáciles de realizar para + productivas
				\4[] $\then$ Más productivas invierten más en IDE horizontal
				\4 Formulación
				\4[] Costes de transporte + protección
				\4[] $\to$ Todas las empresas sufren por igual
				\4[] Economías de escala por concentración
				\4[] $\to$ Beneficio potencial a todas las empresas
				\4[] $\to$ Empresas más productivas realizan con menos concentraciónº
				\4[] $\to$ Empresas más productivas dependen menos de EEscala
				\4[] Trade off concentración vs proximidad
				\4[] $\to$ ¿Reducir costes de producción?
				\4[] $\to$ ¿Realizar economías de escala?
				\4[] Empresas más productivas
				\4[] $\to$ Pueden permitirse menos economías de escala
				\4[] $\then$ Puede competir replicando
				\4 Implicaciones
				\4[] Empresas más productivas invierten más en IDE
				\4[] Multinacionales tienden a ser más productivas
				\4[] Multinacionales aumentan productividad de destino
				\4[] $\to$ Son más productivas de país de origen
				\4[] $\to$ Casi siempre más productivas que destino
				\4 Valoración
				\4[] Explicación bien ajustada empíricamente
				\4[] Relativamente menos importancia en la actualidad
				\4[] $\to$ Caída de coste de transporte
				\4[] $\to$ Evolución hacia GVCs
				\4[] $\then$ IDE vertical gana importancia
				\4 Helpman (1984) y Markusen (1984)
				\4[] Modelos de equilibrio general
				\4[] Helpman (1984)
				\4[] $\to$ Algunos activos son bienes públicos
				\4[] Markusen (1984)
				\4[] $\to$ MNs multiplanta por costes de transporte
				\4 Markusen et al. (1996)
				\4[] Coexistencia de IDE vertical y horizontal
			\3 Determinantes: IDE vertical
				\4 Idea clave
				\4[] Modelos anteriores explican IDE por:
				\4[] $\to$ Evitar costes de transporte
				\4[] $\to$ Evitar aranceles
				\4[] Realmente, hay otro motivo fundamental
				\4[] $\to$ Producción menos costosa en otro lugar
				\4[] $\then$ Ventajas comparativas
				\4[] $\then$ Dinámicas de fragmentación de CVG
				\4 Formulación
				\4[] Multiples formulaciones de ventaja comparativa
				\4[] $\to$ Ricardiana
				\4[] $\to$ Dotación de factores
				\4 Implicaciones
				\4[] Especialización
				\4[] $\to$ aprovechamiento de ventaja comparativa
				\4[] $\to$ división internacional del trabajo
				\4[] Redistribución
				\4[] $\to$ Aumenta dda. L cualificado en desarrollados
				\4[] $\to$ Cae dda. L no cualificado en desarrollados
				\4[] $\to$ Aumenta dda. L no cualificado en PEDs
				\4[] $\then$ Puede explicar deslocalización de actividades
				\4[] $\then$ Puede explicar desigualdad
				\4[] $\then$ Pero evidencia empírica ambigua
				\4 Valoración
				\4[] Evidencia empírica abrumadora
				\4[] $\to$ A favor de IDE vertical como predominante
				\4[] Poca reexportación de sur hacia norte
				\4[] Elevada concentración de destinos de IDE
				\4[] $\to$ Matrices concentran destinos de filiales
			\3 Instituciones
				\4 Ver \href{https://core.ac.uk/download/pdf/35310841.pdf}{Benassy-Queré, Coupet y Mayer (2007)}
			\3 Determinantes empíricos
				\4 Tipo de cambio
				\4[] $\to$ Evidencia empírica confirma relación
				\4[] $\to$ Depreciaciones $\then$ Aumento de IDE
				\4[] Crisis monetarias
				\4[] $\to$ Aumentan IDE por oportunismo extranjero
				\4 Impuestos
				\4[] Relacionar sistemas tributarios e IDE
				\4[] $\to$ Elasticidad negativa impuestos e IDE
				\4 Otros factores
				\4[] $\to$ Instituciones
				\4[] $\to$ Política comercial
				\4[] $\to$ Aglomeración
				\4[] $\to$ Externalidades de información
			\3 Efectos
				\4 Salarios
				\4[] Hipótesis general:
				\4[] $\to$ IDE sube salarios
				\4[] Evidencia empírica
				\4[] $\to$ Multinacionales pagan sueldos más altos
				\4[] Spill-overs
				\4[] $\to$ Difíciles de valorar
				\4[] $\to$ + competencia: $\uparrow$ salario
				\4[] $\to$ -- competencia por trab. baja prod. $\to$ $\downarrow$ salario
				\4[] Variabilidad salarial
				\4[] $\to$ Resultados poco conclusivos
				\4[] $\to$ IDE aumenta diferencia entre cualificados y no
				\4 Productividad
				\4[] En corto plazo resultados mixtos
				\4[] $\to$ $\uparrow$ prod. de empresas receptoras de IDE
				\4[] $\to$ $\downarrow$ de competidoras locales
				\4[] En largo plazo, relación positiva
				\4[] $\to$ Tecnologías se filtran a empresas locales
				\4 Crecimiento
				\4[] Valoración sujeta a problemas habituales
				\4[] $\to$ Relación endógena crecimiento--inversión
				\4[] $\to$ Similares a modelos de CI y crecimiento
				\4 Crisis monetarias y financieras
				\4[] Estudios empíricos muestran relación negativa
				\4[] $\to$ Más IDE, menos probabilidad de crisis
				\4[] Pero determinadas crisis cambiarias aumentan IDE
				\4[] $\to$ Depreciación del tipo de cambio
				\4[] $\then$ Abaratamiento de inversión
				\4[] $\then$ Aumento de competitividad de IDE exportadora
	\1[] \marcar{Conclusión}
		\2 Recapitulación
			\3 Movimientos de trabajo
			\3 Movimientos de capital
		\2 Idea final
			\3 Frontera de investigación
				\4 Investigación sobre movimientos de ff.pp.
				\4[] $\to$ A caballo entre micro y macro
				\4 Relación con otras áreas
				\4[] Crecimiento económico
				\4[] Macroeconomía
				\4[] $\to$ Efectos sobre la macroeconomía
				\4[] Política monetaria
				\4[] $\to$ Efectos sobre tipos de interés
				\4[] Integración económica y monetaria
				\4[] $\to$ Libre mov. de ff.pp. es elemento clave
				\4[] Economía geográfica
				\4[] $\to$ Determinantes de localización
				\4[] $\to$ Aparición de multinacionales
			\3 Beneficios de la globalización
				\4 Teóricamente, generalizados
				\4[] Mejor asignación de factores
				\4[] Aumento especialización
				\4 En práctica, cuestión polémica
				\4[] Beneficios generales y dispersos
				\4[] Perjuicios relativamente concentrados
				\4[] $\to$ Conflictos economía política
				\4[] $\to$ Necesario compensar perdedores
			\3 Controversia sobre inmigración
				\4 Cada vez mayor actualidad
				\4 Sujeto a factores:
				\4[] $\to$ Climatológicos
				\4[] $\to$ Políticos
				\4 Inmigración como problema a resolver
			\3 Movimientos de ff.pp. en Unión Europea
				\4 Forman parte de 4 libertades
				\4 Aspecto clave de construcción europea
				\4 Necesario valorar
				\4[] Por qué K y L se mueven menos de esperado
				\4[] Efectos de movimiento K y L en Europa
\end{esquemal}






























\graficas

\begin{axis}{4}{Efecto de la migración en un modelo simple basado en MacDougall (1960): las productividades marginales se igualan}{}{}{macdougall}
	% Ampliación del eje de abscisas
	\draw[{Latex}-{Latex}] (-1,0) -- (5,0);
	
	% Segundo eje de ordenadas
	\draw[-] (4,-0.5) -- (4,4);
	\draw[-] (0,-0.5) -- (0,4);
	
	% Etiquetas de ejes de abscisas
%	\node[below] at (3.7,0){$L_A$};
%	\node[below] at (0.3,0){$L_B$};
	
	% Productividad marginal en A
	\draw[-] (0.2,4.2) to [out=280, in=160](4,0.5);
	\node[above] at (0.2,4.2){$\text{PMg}_A$};
	
	% Productividad marginal en B
	\draw[-] (3.8,4.2) to [out=260, in=20](0,0.5);
	\node[above] at (3.8,4.2){$\text{PMg}_B$};
	
	% Reparto inicial del trabajo
	\draw[dashed] (3,0) -- (3,4);
	\node[below] at (3,0){$\bar{L}$};
	
	% Salarios
	\draw[dashed] (0,2.39) -- (4,2.39); % Salario de B
	\node[right] at (4,2.39){$w_B$};
	\node[left] at (0,2.39){$w_B$};
	
	\draw[dashed] (0,0.9) -- (4,0.9); % Salario de A
	\node[right] at (4,0.9){$w_A$};
	\node[left] at (0,0.9){$w_A$};
	
	\draw[dashed] (0,1.46) -- (4,1.46); % Salario post-apertura
	\node[right] at (4,1.46){$w^*$};
	\node[left] at (0,1.46){$w^*$};
	
	% Reparto final del trabajo tras apertura
	\draw[dashed] (2,0) -- (2,4);
	\node[below] at (2,0){$L^*$};
	\draw[-{Latex}] (2.85,-0.25) -- (2.15,-0.25);
\end{axis}

\begin{axis}{4}{Efecto de la inmigración sobre las rentas del trabajo y el capital cuando la oferta del capital es inelástica.}{Trabajo}{Productividad marginal}{rentascapitalinelastico}
	% Productividad marginal
	\draw[-] (0,3.8) to [out=300, in=170](4,1);
	
	% Salario antes de inmigración
	\draw[dashed] (0,2.2) -- (1.27,2.2);
	\node[left] at (0,2.2){\scriptsize  $w_0$};
	
	% Salario después de inmigración
	\draw[dashed] (0,1.6) -- (2.15,1.6);
	\node[left] at (0,1.6){\scriptsize $w_1$};
	
	% Oferta de empleo antes de inmigración
	\draw[dashed] (1.3,2.2) -- (1.3,0);
	\node[below] at (1.3,0){\scriptsize$N$};
	
	% Oferta de empleo después de inmigración
	\draw[dashed] (2.15,1.6) -- (2.15,0);
	\node[below] at (2.15,0){\scriptsize $N+M$};
	
	\draw[white, fill=green, opacity=0.6] (1.29,2.2) to [out=310, in=147](2.15,1.6) -- (1.29,1.6);
	
	\draw[white, fill=green, opacity=0.2] (1.29,2.2) -- (1.29,1.6) -- (0,1.6) -- (0,2.2);
\end{axis}

El salario inicial previo a la inmigración es $w_0$, que corresponde a la productividad marginal de una oferta de trabajo de $N$ trabajadores. Cuando la oferta aumenta en $M$ trabajadores inmigrantes, el salario y la productividad pasan a ser $w_1$. Las rentas del capital --propiedad de los trabajadores nativos-- aumentan en la cuantía señalada en verde: el cuadrado de color claro muestra la renta que ganan a costa de las rentas salariales de los nativos; el triángulo verde oscuro muestra las rentas adicionales obtenidas como resultado de la mayor productividad del capital.

\conceptos


\preguntas

%\seccion{Test 2017}
%
%\textbf{38.} En relación a la inversión internacional:
%
%\begin{itemize}
%	\item[a] La principal economía receptora de inversión extranjera directa en 2016 fue China seguida de Estados Unidos.
%	\item[b] En su Informe sobre las inversiones en el Mundo 2017, la UNCTAD propone políticas de inversión que fortalezcan las estrategias de desarrollo digital.
%	\item[c] El Tribunal Multilateral de Inversiones con sede en Viena se creó para aplicar el Acuerdo sobre Medidas Comerciales relacionadas con la Inversión (TRIMS).
%	\item[d] El Grupo de Trabajo de Comercio e Inversión creado en 1996 tras la Conferencia Ministerial de Singapur negocia actualmente nuevas reglas de aplicación multilateral.
%\end{itemize}

\seccion{Test 2014}

\textbf{33.} De acuerdo con el modelo de autoselección de migrantes de Borjas (1987), señale la respuesta correcta:

\begin{itemize}
	\item[a] La mayor parte de los inmigrantes tendrá una elevada cualificación cuando el incremento de retribuciones por una mayor cualificación sea mayor en su país de origen que el país receptor.
	\item[b] La mayor parte de los inmigrantes tendrá una baja cualificación cuando el incremento de retribuciones por una mayor cualificación sea mayor en su país de origen que en el país receptor.
	\item[c] La mayor parte de los inmigrantes tendrá una baja cualificación cuando el incremento de retribuciones por una mayor cualificación sea mayor en su país receptor que el país de origen.
	\item[d] Ninguna de las anteriores son correctas.
\end{itemize}

\seccion{Test 2009}

\textbf{28.} El modelo de Borjas relativo a los flujos de trabajo:

\begin{itemize}
	\item[a] Proporciona microfundamentos a las teorías de los movimientos migratorios y concluye que éstos dependen del salario medio de cada país.
	\item[b] Proporciona microfundamentos a las teorías de los movimientos migratorios y concluye que éstos dependen del salario actual, de las expectativas de salario futuro y del coste de emigrar.
	\item[c] Concluye que la inmigración empobrece a los países que imponen aranceles a la inmigración de productos intensivos en factor trabajo.
	\item[d] Concluye que la autoselección positiva se corresponde con una situación en la que los emigrantes proceden de la parte inferior de la distribución de la renta en el país de origen y se sitúan por debajo de la renta media en el país de destino.
\end{itemize}

\seccion{Test 2007}

\textbf{29.} La principal contribución de S. Hymer a la hora de explicar la inversión extranjera directa y la aparición de la empresa multinacional, consiste en que fue el primer autor en destacar que:
\begin{itemize}
	\item[a] La empresa multinacional posee determinadas ventajas particulares en relación a las empresas locales, que explota internamente mediante el establecimiento de sucursales.
	\item[b] La empresa multinacional se aprovecha del menor coste laboral existente en el mercado local.
	\item[c] La empresa multinacional busca el mayor rendimiento del factor capital existente en el mercado local.
	\item[d] La empresa multinacional posee unas ventajas propias en relación a las empresas locales, que explota internamente mediante el establecimiento de sucursales; y ello en función de las ventajas de localización existentes en el mercado local respecto a su país de origen.
\end{itemize}

\seccion{Test 2006}

\textbf{24.} Considere las siguientes afirmaciones en relación a la Inversión Directa Extranjera (IDE) y la Empresa Multinacional (EMN):

\begin{itemize}
	\item[I] En los modelos teóricos de IDE horizontal, la EMN establece múltiples plantas productivas para atender los diferentes mercados.
	\item[II] En los modelos de IDE vertical, las EMN surgen cuando, en presencia de diferencias en los precios de los factores entre países, la empresa encuentra óptimo fragmentar su proceso productivo y desarrollar diferentes partes de este proceso en distintos países.
	\item[III] La presencia de elevadas barreras comerciales tiende a estimular la IDE de carácter horizontal y a impedir la de carácter vertical.	
\end{itemize}

Señale la opción correcta:

\begin{itemize}
	\item[a] Sólo la afirmación I es correcta.
	\item[b] Sólo la afirmación II es correcta.
	\item[c] Las tres afirmaciones son correctas.
	\item[d] Las tres afirmaciones son falsas.
\end{itemize}

\seccion{Test 2004}

\textbf{27.} A la hora de llevar a cabo la inversión extranjera directa:

\begin{itemize}
	\item[a] Las empresas multinacionales dan cada vez menor importancia al conocimiento y a la tecnología, dada su tendencia a invertir en los países más pobres. 
	\item[b] El elemento fundamental que rige la conducta de las multinacionales es la búsqueda de los menores costes laborales posibles.
	\item[c] Las empresas multinacionales dan cada vez mayor importancia a los activos basados en el conocimiento (con objeto de difundir las ideas y las tecnologías a través de las fronteras), y no tanto al capital físico y a los costes laborales.
	\item[d] Aunque parece que la tendencia está cambiando últimamente, las empresas multinacionales tienden a valorar particularmente la existencia de unos menores costes laborales en el país de destino, unido a la posibilidad de adquirir la tecnología local.
\end{itemize}

\notas

%\textbf{2017:} \textbf{38.} B

\textbf{2014:} \textbf{33.} B

\textbf{2009:} \textbf{28.} B

\textbf{2007:} \textbf{29.} A

\textbf{2006:} \textbf{24.} C

\textbf{2004:} \textbf{27.} C

\bibliografia

Mirar en Palgrave:
\begin{itemize}
	\item brain drain
	\item dutch disease and foreign aid
	\item factor prices in general equilibrium
	\item foreign direct investment *
	\item foreign investment
	\item globalization and labour *
	\item globalization and the welfare state *
	\item infant-industry protection *
	\item internal migration
	\item international capital flows *
	\item international coordination in asylum provision *
	\item international migration *
	\item international outsourcing
	\item labour mobility in the European Union *
	\item the political economy of unearned foreign income
	\item transfer problem
	\item uneven development *
\end{itemize}

Borjas, G. J. \textit{Self-Selection and the Earnings of Immigrants} (1987) American Economic Review -- En carpeta del tema

Borjas, G. J. (1995) \textit{The Economics Benefits from Immigration} (1995) Journal of Economic Perspectives -- En carpeta del tema

Boz, E.; Cubeddu, L.; Obstfeld, M. \textit{Revisiting the paradox of capital} (2017) VOX CEPR's Policy Portal -- En carpeta del tema

Duran, J. J.; Álvarez, I. (2019) \textit{Expansión internacional de las empresas multinacionales. Estructura y naturaleza institucional} ICE. Julio-Agosto 2019. Número 909 -- En carpeta del tema

Feenstra, R. \textit{Advanced International Trade} (2004) Princeton University Press - En carpeta del tema

Helpman, E.; Melitz, M.; Yeaple, S. (2004) \textit{Export Versus FDI with Heterogeneous Firms} American Economic Review Vol. 94. No.1 -- En carpeta del tema

gibson, J.; McKenzie, D. (2011) \textit{Eight Questions about Brain Drain} Journal of Economic Perspectives 2011 Vol. 25 N. 3 -- En carpeta del tema

Gourinchas, P. O.;  \textit{Capital Flows to Developing Countries: The Allocation Puzzle} (2013) Review of Economic Studies -- En carpeta del tema. Ver también NBER working paper de 2007

Krugman, P. R. (1991) \textit{Increasing Returns and Economic Geography} Journal of Political Economy, Vol. 99, No. 3 -- En carpeta del tema
Jovanović, M. N. \textit{The Economics of International Integration} (2006) Elgar Publishing -- En carpeta Economía Internacional

Lucas, R. \textit{Why Doesn't Capital Flow from Rich to Poor Countries?} (1990) The American Economic Review -- En carpeta del tema

MacDougall, G. D. \textit{The Benefits and Costs of Private Investment from Abroad: A Theoretical Approach} (1960) The Bulletin -- En carpeta del tema

Peri, G. \textit{Immigrants, Productivity, and Labour Markets} (2016) Journal of Economic Perspectives-- En carpeta del tema

Ruhl, K. J. \textit{The Ownership-Location-Internalization Framework} (2016) PennState. Multinationals and the Globalization of Production -- En carpeta del tema

OECD \textit{Foreing Direct Investment Statistics: Data, Analysis and Forecasts} \url{http://www.oecd.org/investment/statistics.htm} -- Descarga 

UNCTAD (2019) \textit{World Investment Report} -- En carpeta del tema

\end{document}
