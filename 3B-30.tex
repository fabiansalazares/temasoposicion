\documentclass{nuevotema}

\tema{3B-30}
\titulo{La financiación exterior del desarrollo económico. El problema de la deuda externa. La ayuda al desarrollo.}

\begin{document}

\ideaclave

Hay que añadir un apartado sobre el STDF -- Fondo para la Aplicación de Estándares y el Fomento del Comercio de la OMC. Ver \url{http://www.standardsfacility.org/es/qui\%C3\%A9nes-somos}

\seccion{Preguntas clave}

\begin{itemize}
	\item ¿Qué objetivos tiene la financiación exterior del desarrollo económico?
	\item ¿Por qué es necesaria?
	\item ¿Qué instrumentos se utilizan para financiar el desarrollo?
	\item ¿Qué es la deuda externa?
	\item ¿Por qué ha supuesto un problema para los países en desarrollo?
	\item ¿Qué soluciones se han implementado?
	\item ¿Quiénes son los agentes principales?
	\item ¿En qué consiste la ayuda al desarrollo?
	\item ¿Para qué sirve?
	\item ¿Qué efectos tiene?
	\item ¿Qué propuestas existen para mejorar su efectividad?
\end{itemize}

\esquemacorto

\begin{esquema}[enumerate]
	\1[] \marcar{Introducción}
		\2 Contextualización
			\3 Desigualdades de renta per cápita
			\3 Transferencia de ahorro y tecnología
			\3 Financiación exterior del desarrollo
		\2 Objeto
			\3 ¿Qué objetivos tiene la financiación exterior del desarrollo económico?
			\3 ¿Por qué es necesaria?
			\3 ¿Qué instrumentos para financiar el desarrollo?
			\3 ¿Qué es la deuda externa?
			\3 ¿Por qué ha supuesto un problema para los países en desarrollo?
			\3 ¿Qué soluciones se han implementado?
			\3 ¿Quiénes son los agentes principales?
			\3 ¿En qué consiste la ayuda al desarrollo?
			\3 ¿Para qué sirve?
			\3 ¿Qué efectos tiene?
			\3 ¿Qué propuestas existen para mejorar su efectividad?
		\2 Estructura
			\3 Financiación exterior del desarrollo
			\3 El problema de la deuda externa
	\1 \marcar{Financiación exterior del desarrollo}
		\2 Idea clave
			\3 Contexto
			\3 Objetivos
			\3 Resultados
		\2 Instrumentos no-AOD
			\3 Inversión extranjera directa
			\3 Inversión extranjera en equity en cartera
			\3 Deuda
		\2 Ayuda Oficial al Desarrollo
			\3 Idea clave
			\3 Concepto de AOD
			\3 Antecedentes
			\3 Comité de Ayuda al Desarrollo de OCDE
			\3 Tipos de AOD
			\3 Cuantificación de la AOD
			\3 Principales cifras de la AOD
			\3 Valoración
			\3 Retos
		\2 Otros instrumentos
			\3 Remesas
			\3 Alianzas público-privadas
			\3 Fondos de garantías
			\3 Fondos soberanos
			\3 Transformación de características temporales
			\3 Mitigación del riesgo
			\3 Captación de contribuciones privadas voluntarias
			\3 Modalidades impositivas internacionales
			\3 Propuestas
	\1 \marcar{El problema de la deuda externa}
		\2 Idea clave
			\3 Contexto
			\3 Objetivo
			\3 Resultados
		\2 Sostenibilidad de la deuda
			\3 Idea clave
			\3 Formulación
			\3 Factores principales que determinan crisis
			\3 Análisis de sensibilidad
			\3 Intolerancia a la deuda
			\3 Intolerancia a la deuda externa
			\3 Análisis de sostenibilidad de la deuda (DSA) del FMI
		\2 Agentes en escenario internacional de la deuda
			\3 Acreedores internacionales
			\3 Club de París
			\3 Club de Londres
			\3 Comités de bonistas
			\3 Deudores internacionales
			\3 Marco institucional
		\2 Crisis de los 80
			\3 Contexto
			\3 Eventos
			\3 Consecuencias
		\2 Crisis de los 90
			\3 Contexto
			\3 Eventos
			\3 Consecuencias
		\2 Cláusulas de acción colectiva
			\3 Idea clave
			\3 Single-limb CAC / votación única
			\3 Double-limb CAC / votación con doble vuelta
		\2 Reestructuración de deuda países más pobres
			\3 Idea clave
			\3 Tratamientos iniciales
			\3 Tratamiento Evian
			\3 Iniciativa HIPC
			\3 IADM -- Iniciativa para el Alivio de la Deuda Multilateral
			\3 Fondo fiduciario para contención y alivio de catástrofes
			\3 Valoración
	\1[] \marcar{Conclusión}
		\2 Recapitulación
			\3 Financiación exterior del desarrollo
			\3 El problema de la deuda externa
			\3 Probando
		\2 Idea final
			\3 Economía de los países en desarrollo
			\3 Modelos del desarrollo
			\3 Teorías del crecimiento económico
			\3 Cooperación al desarrollo UE y España

\end{esquema}

\esquemalargo

\begin{esquemal}
	\1[] \marcar{Introducción}
		\2 Contextualización
			\3 Desigualdades de renta per cápita
				\4 Enormes diferencias permanentes
				\4 Inversión, tecnología, como determinantes
			\3 Transferencia de ahorro y tecnología
				\4 Países desarrollados $\to$ PEDs
				\4 Fin del colonialismo
				\4 Motivos altruistas
				\4 Motivos interés nacional
			\3 Financiación exterior del desarrollo
				\4 Modelo neoclásico de ahorro e inversión
				\4[] Economías abiertas
				\4[] Agentes optimizadores
				\4[] $\to$ Capital debe fluir donde tiene mayor PMgK
				\4[] Países ricos
				\4[] $\to$ Mayores stocks de capital presente
				\4[] $\then$ PMgK menor
				\4[] Países pobres
				\4[] $\to$ Menores stocks de capital
				\4[] $\then$ PMgK mayor
				\4[] Sin fallos de mercado
				\4[] $\to$ Equilibrio competitivo óptimo
				\4[] $\then$ Capital debería fluir de ricos a pobres
				\4 En la práctica
				\4[] Fallos de mercado
				\4[] $\to$ Información asimétrica
				\4[] $\to$ Externalidades
				\4[] $\to$ Bienes públicos
				\4[] $\then$ Equilibrio competitivo no es óptimo
				\4[] Sistemas financieros poco desarrollados
				\4[] Insuficiente ahorro doméstico
				\4[] Cuellos de botella de economías en desarrollo
				\4[$\then$] Escasez de recursos en PEDs
				\4[$\then$] Persistencia de obstáculos a capitalización
				\4[$\then$] Tensiones de balanza de pagos
				\4[$\then$] Problemas específicos a financiación de PEDs
		\2 Objeto
			\3 ¿Qué objetivos tiene la financiación exterior del desarrollo económico?
			\3 ¿Por qué es necesaria?
			\3 ¿Qué instrumentos para financiar el desarrollo?
			\3 ¿Qué es la deuda externa?
			\3 ¿Por qué ha supuesto un problema para los países en desarrollo?
			\3 ¿Qué soluciones se han implementado?
			\3 ¿Quiénes son los agentes principales?
			\3 ¿En qué consiste la ayuda al desarrollo?
			\3 ¿Para qué sirve?
			\3 ¿Qué efectos tiene?
			\3 ¿Qué propuestas existen para mejorar su efectividad?
		\2 Estructura
			\3 Financiación exterior del desarrollo
			\3 El problema de la deuda externa
	\1 \marcar{Financiación exterior del desarrollo}
		\2 Idea clave
			\3 Contexto
				\4 Escasez de capital
				\4[] Infraestructuras deficientes
				\4[] Economías de escala sin realizar
				\4[] Tensiones de liquidez de divisas
				\4 Debilidad institucional
				\4[] Corrupción
				\4[] Inseguridad jurídica
				\4[] Guerras, disturbios
				\4 Formación de capital humano
				\4[] Cuello de botella habitual
				\4[] Grave escasez de alta cualificación
				\4 Inestabilidad macroeconómica
				\4[] Ciclos muy volátiles y profundos
				\4[] Sensibilidad a crisis financieras
				\4 Integración en economía mundial
				\4[] Reducida
				\4[] Muy poco comercio entre PEDs
			\3 Objetivos
				\4 Inducir crecimiento económico
				\4 Reducir pobreza
				\4 Canalizar conocimiento técnico
			\3 Resultados
				\4 Gama de instrumentos financieros
				\4 Ayuda Oficial al Desarrollo es elemento central
				\4 Marco institucional específico
				\4 Instrumentos alternativos
		\2 Instrumentos no-AOD
			\3 Inversión extranjera directa
				\4 Concepto
				\4[] Participación del 10\%
				\4[] Voluntad de ejercer control o participar en gestión
				\4 Brownfield vs greenfield
				\4[] Brownfield
				\4[] $\to$ Adquisición de planta ya existente
				\4[] $\to$ Planta existente continúa su actividad anterior
				\4[] Greenfield
				\4[] $\to$ Construcción de nueva planta
				\4[] $\to$ Planta antigua que cambia actividad tras inversión
				\4 Vertical vs horizontal
				\4[] Vertical
				\4[] $\to$ División del proceso productivo en fases
				\4[] $\to$ Plantas extranjeras llevan a cabo sólo una fase
				\4[] $\to$ Producción destinada a reexportación
				\4[] $\then$ Aprovechar ventajas competitivas en determinado segmento
				\4[] Horizontal
				\4[] $\to$ Replicación de proceso productivo en otro país
				\4[] $\to$ Evitar costes de transporte/aranceles
				\4 Problemas de oferta
				\4[] $\to$ Requiere seguridad jurídica y estabilidad
				\4 Problemas de demanda
				\4[] $\to$ Prejuicio respecto a poder extranjero/MNacionales
				\4 Ventajas
				\4[] $\to$ Estabilidad del capital
				\4[] $\to$ No sensible a crisis de deuda
				\4[] $\to$ Permite incorporación a cadenas globales de valor
				\4 Receptores principales
				\4[] PEDs más desarrollados o ricos en recursos
				\4[] Países más estables
				\4[] Destinos con elevado capital humano
				\4 Efecto atracción de IDE
				\4[] $\to$ IDE atrae más IDE
				\4 Actualidad
				\4[] Estructuras cada vez más complejas
				\4[] Relativamente menos IDE horizontal
				\4[] IED como parte de cadenas de valor
				\4 Fondos de capital riesgo
				\4[] Fondos gestionados por entidades financieras
				\4[] Sectores con alto potencial de crecimiento
				\4[] Horizonte temporal limitado
				\4[] Entrada de know-how e I+D en PEDs
			\3 Inversión extranjera en equity en cartera
				\4 Títulos negociables que no cumplen requisitos IED
				\4[] Participación inferior al 10\%
				\4 Evolución ligada a desarrollo de mercados financieros
				\4[] Aumento en últimos años
				\4 Principales receptores
				\4[] BRICS
				\4 Efectos
				\4[] Mejora de gestión corporativa
				\4[] $\to$ Vía coste de capital
				\4[] Incorporación de experiencia y tecnología
				\4[] No aumenta deuda externa
				\4[] Reduce vulnerabilidad a crisis de deuda
				\4 Inconvenientes
				\4[] Mayor volatilidad
				\4[] Sin gestión directa ni control
				\4[] $\to$ Más fácil salida del país/accionariado
			\3 Deuda
				\4 Préstamos públicos bilaterales
				\4[] Crédito directo entre estados
				\4[] Múltiples modalidades
				\4[] $\to$ Créditos a exportación
				\4[] $\to$ Deuda comercial garantizada
				\4[] Fuente relativamente estable de financiación
				\4[] Criterios políticos e históricos
				\4 Préstamos públicos multilaterales
				\4[] Instituciones financieras multilaterales
				\4[] $\to$ IMF, BM, Bancos Regionales
				\4[] Principal fuente de financiación para PMA
				\4[] Estatus de acreedor preferente
				\4[] $\to$ Salvo iniciativa HIPC
				\4 Préstamos sindicados
				\4[] Innovación financiera de los 70 para PEDs
				\4[] Grandes bancos internacionales
				\4[] Elemento fundamental de crisis de deuda de 80s
				\4 Bonos
				\4[] Impulso tras Plan Brady
				\4[] Permite aumentar base de inversores y prestamistas
				\4[] Modalidades novedosas de financiación
				\4[] $\to$ Bonos ligados a desastres naturales
				\4[] $\then$ Si desastre no se paga principal
				\4[] $\to$ Bonos indexados al PIB\footnote{Ver \url{https://voxeu.org/article/sovereign-gdp-linked-bonds-rationale-and-design}.}
				\4[] $\then$ Transferir riesgo de recesión
				\4[] $\then$ Aún muy poco utilizados
				\4[] $\then$ Sólo vinculación parcial hasta ahora
				\4[] $\then$ Pocos inversores dispuestos a asumir downside
				\4 Títulos de flujos futuros\footnote{Ver \url{https://blogs.iadb.org/bidinvest/es/financiamientos-de-flujos-futuros-y-sus-beneficios/}.}
				\4[] $\to$ Bonos ``verdes''
				\4[] Captación de capital a cambio de flujos futuros
				\4[] $\to$ Generados por masa de activos públicos
				\4[] Separación de propiedad
				\4[] $\to$ Permite aislar masa de activos de pres. público
				\4[] $\then$ Reduce riesgo de impago
				\4[] $\then$ Vincula flujos a calidad de activos subyacentes
				\4[] Facilita recurso a mercados de capital
				\4[] $\to$ A menor coste
				\4[] $\to$ Con mayores vencimientos
				\4[] $\then$ Debido a menor riesgo soberano
		\2 Ayuda Oficial al Desarrollo
			\3 Idea clave
				\4 Contexto
				\4[] Financiación de mercado no siempre disponible
				\4[] $\to$ Riesgo muy alto
				\4[] $\to$ Mercados poco desarrollados
				\4[] $\to$ Deuda muy elevada
				\4[] $\to$ Mecanismos de aseguramiento no disponibles
				\4[] Efecto marginal de la financiación
				\4[] $\to$ Importante en PEDs
				\4[] $\to$ Esencial para PMAs
				\4 Objetivos
				\4[] Ayuda altruista
				\4[] Apoyar países con cercanía ideológica/cultural
				\4[] Favorecer intereses comerciales nacionales
				\4[] Mitigar efectos de desastres
				\4[] Favorecer intereses geoestratégicos
				\4[] Evitar influencia de antagonistas
				\4[] Garantizar votos en inst. multilaterales
				\4 Resultados
				\4[] Marco institucional supranacional de AOD
				\4[] Development Assistance Committee de OCDE
				\4[] $\to$ También colaboran países no-OCDE
				\4[] Efectividad muy heterogénea
				\4[] Desde ``muy eficaz'' hasta ``contraproducente''
				\4[] $\to$ Y todos los intermedios
			\3 Concepto de AOD
				\4 Cooperación al Desarrollo
				\4[] Concepto amplio
				\4[] Conjunto de acciones de inst. públicas y privadas
				\4[] $\to$ Con objeto de aumentar nivel de desarrollo PEDs
				\4 Ayuda al Desarrollo
				\4[] Transferencia de recursos en términos concesionales
				\4[] Promover desarrollo de los PEDs
				\4[] Agentes gubernamentales y no gubernamentales
				\4 Ayuda Oficial al Desarrollo
				\4[] i. Provistos por organismos oficiales
				\4[] ii. Receptores en lista de CAD y instituciones multilat.
				\4[] iii. Promover desarrollo económico y bienestar
				\4[] $\to$ Política comercial y asist. militar excluidas
				\4[] iv. Carácter concesional
				\4[] $\to$ Elemento concesional superior a cierto umbral\footnote{Ver umbrales y cálculo en conceptos.}
				\4[] $\to$ Flujos no concesionales descontados ajustados al riesgo\footnote{Anteriormente se utilizaba una tasa de descuento general del 10\%.}
			\3 Antecedentes
				\4 Plan Marshall
				\4[] Primer programa de financiación exterior
				\4[] Objetivos:
				\4[] $\to$ Reconstituir stock de capital en Europa
				\4[] $\to$ Permitir liquidación de saldos bilaterales
				\4[] $\to$ Suplir escasez de dólares en Europa
				\4[] Entre 2\% y 3\% de RNB americana
				\4[] Éxito general:
				\4[] $\to$ Crecimiento muy fuerte en Europa
				\4[] $\to$ Estímulo a demanda externa en USA
				\4[] $\to$ Formación de Unión Europea de Pagos
				\4 Años 50 y 60
				\4[] Optimismo sobre financiación del desarrollo
				\4[] $\to$ Tras éxito del plan Marshall
				\4[] $\to$ Intervención y planificación vistas favorablemente
				\4[] Flujos bilaterales metrópolis/colonias
				\4[] $\to$ Antes y después de descolonizaciones
				\4[] Aparición progresiva de multilateralidad
				\4[] $\to$ BIRD (1944)
				\4[] $\to$ AIF (1960)
				\4[] $\to$ Expansión de NU y BM
				\4[] $\to$ Hasta 30\% de ayuda total
				\4[] Financiación de proyectos desde años 50
				\4[] $\to$ Modalidad dominante
				\4[] $\to$ Sustituye a fin. de BPagos de PMarshall
				\4[] $\to$ Énfasis en inversión de capital
				\4[] $\to$ Asistencia técnica
				\4[] Crecimiento es objetivo principal
				\4[] $\to$ Reducción de pobreza vía transmisión a sectores
				\4[] Fuerte crecimiento en PEDs en 50s y 60s
				\4 Años 70
				\4[] Énfasis en crecimiento cuestionado
				\4[] $\to$ Financiación de proyectos sigue predominando
				\4[] $\to$ Empleo, distribución de ingreso, pobreza..
				\4[] Aumento paralelo de críticas a ayuda al desarrollo
				\4[] $\to$ Bauer y otros
				\4[] Ligera caída de financiación de proyectos
				\4 Años 80
				\4[] Caída brusca del crecimiento
				\4[] $\to$ Aumento de tipos en USA
				\4[] $\to$ Crisis de deuda en LatAm y otros
				\4[] Cambios políticos en emisores de ayuda
				\4[] $\to$ Thatcher, Reagan
				\4[] $\to$ Anne Krueger en Banco Mundial
				\4[] Cambio de enfoque en Banco Mundial
				\4[] $\to$ Más sector privado
				\4[] $\to$ Más colaboración público-privada
				\4[] $\to$ Menos intervención pública
				\4[] Financiación de proyectos pierde peso
				\4[] $\to$ Más asistencia macroeconómica
				\4[] $\then$ Ayuda financiera + condicionalidad
				\4[] Paradoja micro-macro
				\4[] $\to$ Mosley (1983)
				\4[] $\to$ Evaluación ex-post de proyectos micro favorable
				\4[] $\to$ Regresiones macro sobre ayuda muy poco favorables
				\4[] $\to$ Efectos mucho mejores en Asia que África
				\4[] $\then$ Necesario redirigir a países que usan mejor
				\4[] Emerge Consenso de Washington
				\4[] i. Disciplina fiscal
				\4[] $\to$ Déficits persistentes en LatAm
				\4[] ii. Reestructuración del gasto público
				\4[] $\to$ Menos subsidios generales
				\4[] $\to$ Más énfasis en necesidades básicas
				\4[] iii. Reforma tributaria
				\4[] $\to$ Aumento de bases imponibles
				\4[] $\to$ Reducción de tipos marginales
				\4[] iv. Liberalización de tipos de interés
				\4[] $\to$ Privatizaciones bancarias
				\4[] $\to$ Liberalización del sector financiero
				\4[] v. Tipo de cambio
				\4[] $\to$ Flexible inicialmente
				\4[] $\to$ Modelo bipolar del FMI posteriormente
				\4[] vi. Liberalización comercial
				\4[] $\to$ Beneficios del comercio > costes
				\4[] vii. Liberalización de la IDE
				\4[] $\to$ Inicialmente, sin lib. completa de CFinanciera
				\4[] viii. Privatización
				\4[] $\to$ Thatcherismo, Escuela de Chicago
				\4[] ix. Desregulación
				\4[] $\to$ Menos barreras de entrada y salida
				\4[] $\to$ Énfasis en seguridad y medio ambiente
				\4[] x. Derechos de propiedad
				\4[] $\to$ Permitir al sector informal acceso a propiedad
				\4[] Crecimiento aumenta hasta 90s
				\4 Años 90
				\4[] Problemas de fin. desarrollo aumenta visibilidad
				\4[] $\to$ Corrupción
				\4[] $\to$ Mal aprovechamiento
				\4[] $\to$ Poca efectividad
				\4[] $\then$ ``Fatiga'' de la ayuda internacional
				\4[] Caída del bloque soviético
				\4[] $\to$ Fin de ayuda bilateral URSS-PEDs de órbita comunista
				\4[] $\to$ Occidente suple parcialmente
				\4[] Caída de flujos de ayuda en términos reales
				\4[] $\to$ Debilitamiento relaciones post-coloniales
				\4[] $\to$ Apoyo opinión pública menor
				\4[] $\to$ Visibilidad mayor de fenómeno de rent-seeking
				\4 Años 2000
				\4[] Conferencia de Monterrey en 2002
				\4[] $\to$ Organizada por ONU
				\4[] $\to$ Líderes países desarrollados+FMI+OMC
				\4[] Declaración de Doha
				\4[] $\to$ Adopción en Conferencia de Monterrey
				\4[] Énfasis en políticas nacionales
				\4[] $\to$ Recursos domésticos también deben movilizarse
				\4[] Comercio internacional
				\4[] $\to$ Motor de desarrollo
				\4[] $\to$ Necesario concluir Ronda de Doha
				\4[] Alivio excepcional de deuda
				\4[] $\to$ Nigeria e Irak
				\4[] $\then$ Incremento de volúmenes de AOD
				\4[] Deuda externa
				\4[] $\to$ Énfasis en prevención de nuevas crisis
			\3 Comité de Ayuda al Desarrollo de OCDE\footnote{Ver \href{https://www.oecd.org/dac/development-assistance-committee/}{OECD: Develpment Assistance Committee.}}
				\4 Foro de 30 países desarrollados
				\4 Peer reviews sobre ayuda de otros miembros
				\4 Emisión de recomendaciones
				\4[] Recoger evidencia empírica y experiencia
				\4 Monitorización
				\4[] Estadísticas
				\4[] Definiciones
				\4[] $\to$ AOD
				\4[] $\to$ HIPC
				\4 Países no-OCDE fuera de ámbito del Comité
				\4[] Algunos de ellos muy relevantes
				\4 Observadores
				\4[] Bancos de Desarrollo
				\4[] Instituciones de Bretton Woods
			\3 Tipos de AOD
				\4 Origen
				\4[] -- Multilateral
				\4[] $\to$ A través de IFIs
				\4[] $\to$ Generalmente, mayor grado de coordinación
				\4[] $\to$ Menos dependencia de vínculos históricos/políticos
				\4[] -- Bilateral
				\4[] $\to$ Concesión directa a PEDs
				\4[] $\to$ Más dependencia de vínculos históricos/políticos
				\4 Concesionalidad
				\4[] -- Donaciones
				\4[] $\to$ Concesionales al 100\%
				\4[] $\to$ Limitadas por efecto negativo presupuestario
				\4[] $\to$ No impactan sobre endeudamiento de receptor
				\4[] -- Financiación reembolsable concesional
				\4[] $\to$ Amplían activos exteriores del donante
				\4[] $\to$ Apropiado para proyectos financieramente sost.
				\4[] $\to$ Criterios de CAD para considerar AOD
				\4[] $\then$ \% mínimo de concesionalidad dependiente de receptor\footnote{Ver concepto \textit{Cómputo de elementos reembolsables como AOD.}}
				\4[] -- Financiación ``blended'' o combinada
				\4[] $\to$ Uso estratégico de fin. concesional+mercado
				\4[] $\to$ Donación para reducir riesgo, asist. técnica
				\4[] $\to$ Créditos para actividades sostenibles
				\4[] $\then$ Atraer fin. privada a partir de AOD
				\4[] $\then$ AOD como palanca de financiación
				\4 Destino
				\4[] -- Ligada/vinculada
				\4[] $\to$ Se exige fondos a empresas del país de origen
				\4[] $\then$ Internacionalización de empresas en emisor
				\4[] -- No ligada/desvinculada
				\4[] $\to$ Sin exigencia sobre nacionalidad de uso de fondos
				\4[] $\to$ Aumenta responsabilidad de PEDs
				\4[] $\to$ Algunos estudios apuntan a mayor eficiencia
				\4[] $\then$ Fuerte debate sobre conveniencia de desvincular
			\3 Cuantificación de la AOD
				\4 Reforma de 2014
				\4[] CAD moderniza sistema estadístico
				\4[] Cambios en criterios
				\4 Préstamos concesionales
				\4[] Sólo parte concesional computa como AOD
				\4[] Tasa de descuento ajustada al riesgo
				\4 Instrumentos del sector privado
				\4[] Considerando AOD como catalizador
				\4[] $\to$ Necesario cuantificar impulso a financiación privada
				\4 TOSSD -- Asistencia Oficial Total para el Desarrollo Sostenible
				\4[] Cuantificar totalidad de recursos públicos
				\4[] $\to$ Para desarrollo de PEDs
				\4[] Incluye AOD Y otros vehículos
				\4[] $\to$ Financiación combinada/blended
				\4[] $\to$ Mitigación del riesgo y similares
			\3 Principales cifras de la AOD\footnote{Extraído de OECD 2019}
				\4 Ayuda total
				\4[] Miembros del DAC de OCDE
				\4[] $\to$ $153.000$ M de dólares en 2018
				\4[] $\to$ $0,3\%$ de RNB
				\4 Mayores donantes mundiales en VAbsoluto
				\4[] Estados Unidos es mayor donante mundial\footnote{$\sim 35.000$ M.}
				\4[] $\to$ Seguido de GER, UK, JAP, FRA
				\4 Mayores donantes por $\%$ de renta
				\4[] Mayores donantes por $\%$ de RNB
				\4[] $\to$ LUX, NOR, DAN, UK, UAE, TUR
				\4 Grupos de origen
				\4[] Países UE casi $55\%$ del total
				\4[] G7 casi 3/4 del total
				\4 Principales destinos
				\4[] África mayor receptor mundial
				\4[] $\to$ Especialmente subsahariana
				\4[] Seguido de Asia y LatAm
				\4[] Fuerte heterogeneidad por origen de ayuda
				\4 Evolución reciente
				\4[] Crecimiento sostenido desde 2000
				\4[] $\to$ De menos de 80.000 a >140.000
				\4 Por tipo de AOD
				\4[] Ayuda bilateral es mayor parte
				\4[] $\to$ Casi 50\%
				\4[] $\to$ Aunque reduce peso relativo desde 2000
				\4[] Ayuda multilateral es segundo tipo
				\4[] $\to$ Mantiene peso relativo
				\4[] Gasto doméstico en refugiados
				\4[] $\to$ Aumenta fuertemente tras 2013
				\4[] Ayuda humanitaria
				\4[] $\to$ Casi superada por gasto doméstico
				\4 Objetivo del $0,7\%$
				\4[] De media, $0,3\%$ de RNB
				\4[] Entre 4 y 6 de países UE-28 cumplen objetivo
			\3 Valoración
				\4 Críticas
				\4[] Bauer (1975)
				\4[] $\to$ Transferencia de pobres en países ricos
				\4[] $\then$ A ricos en países pobres
				\4[] Incentivos perversos
				\4[] $\to$ Receptores incentivados a gastar más
				\4[] $\then$ Aunque destinos sean ineficientes
				\4[] $\to$ Aumento de corrupción en receptores
				\4[] $\to$ Persistencia de instituciones corruptas
				\4[] $\to$ Débiles incentivos a liberalización
				\4[] $\to$ ``Publicidad'' es medida principal\footnote{El apoyo en la opinión pública al gasto en desarrollo depende fuertemente de la publicidad que se de a los proyectos y a sus objetivos, pero en mucha menor medida a su efectividad real.}
				\4[] Introducción de distorsiones
				\4[] $\to$ Devolución de préstamos requiere presión fiscal
				\4[] $\to$ Condonaciones dificultan acceso a mercado
				\4[] $\to$ Ruptura de vínculo gasto--tributación
				\4[] Importancia cuantitativa reducida
				\4[] $\to$ Comercio
				\4[] $\to$ Inversión
				\4[] $\then$ Cuantías mucho más elevadas
				\4[] Evidencia empírica
				\4[] $\to$ Resultados favorables débiles y poco robustos
				\4 Argumentos a favor
				\4[] Esencial para países sin acceso a mercados
				\4[] $\to$ Puede no haber alternativas para BPagos
				\4[] Permite atraer inversiones directas y capital
				\4[] $\to$ Favorece aparición de marco legal y financiero
				\4[] Puede funcionar como palanca para reformas
				\4[] $\to$ Liberalización comercial
				\4[] $\to$ Reformas agrícolas
				\4[] $\to$ Administración pública
			\3 Retos\footnote{Ver Page y Pande (2018) en JEP.}
				\4 Mayor apropiación local de planes de desarrollo
				\4[] Considerado factor fundamental de éxito
				\4 Incorporar medidas cualitativas de impacto
				\4[] No sólo valorar elementos cuantitativos
				\4[] $\to$ Democratización
				\4[] $\to$ Seguridad jurídica
				\4[] $\to$ Derechos humanos
				\4 Incorporación de evidencia empírica
				\4[] Banerjee, Duflo, Kremer, List...
				\4[] Políticas de desarrollo deben basarse en ev. empírica
				\4[] $\to$ Experimentos aleatorizados
				\4[] $\to$ Énfasis general en diseño de experimentos
				\4[] $\then$ Extraer evidencia generalizable
		\2 Otros instrumentos
			\3 Remesas
				\4 Marco de balanza de pagos\footnote{Descrito en apartado 12.27}
				\4[] Transferencias personales a cobrar\footnote{Parte de la renta secundaria de la cuenta corriente.}
				\4[] + Beneficios sociales a recibir
				\4 Volumen
				\4[] Creciente en últimos años
				\4[] Duplica total de de AOD recibida
				\4[] Importante impulso a:
				\4[] $\to$ Desarrollo
				\4[] $\to$ Financiación de balanza de pagos
				\4 Distribución
				\4[] Emisores de emigración
				\4[] $\to$ Principales receptores
				\4[] Receptores de inmigración en últimos años
				\4[] $\to$ Principales emisores
				\4[] Menor concentración que otros tipos de financiación
				\4[] $\to$ 10 primeros receptores de IED:
				\4[] $\then$ 70\% de total de IDE
				\4[] $\then$ 59\% de remesas
				\4[] Países más pobres no son principales receptores
				\4[] $\to$ Países de renta media-baja mucho más renta baja
				\4 Evolución temporal
				\4[] Volatilidad
				\4[] $\to$ Mucho menor que IED y otros flujos privados
				\4[] $\to$ Mayor que AOD
				\4[] Recurso limitado en el tiempo
				\4[] $\to$ Migración temporal
				\4[] $\to$ Vínculos con país de origen se disipan
				\4 Efectos en receptores
				\4[] Relajar restricción externa al crecimiento
				\4[] Complemento a ahorro doméstico insuficiente
				\4[] Aumento del consumo privado
				\4[] $\to$ Sólo parcialmente dedicadas a inversión
				\4 Complementariedad con estrategias de desarrollo
				\4[] Herramienta útil si:
				\4[] $\to$ Marco legal puede canalizar a inversión en receptor
				\4[] $\to$ Coste de transferencia reducido
			\3 Alianzas público-privadas
				\4 Idea clave
				\4[] Financiación conjunta de proyectos de inversión
				\4[] $\to$ Fondos públicos + empresas privadas
				\4 Mitigación del riesgo privado
				\4[] Permite mayor atracción de capital
				\4[] $\to$ Efecto multiplicador de inversión pública
				\4 Esquema de financiación habitual
				\4[] Mínimo, 50\% privado generalmente
			\3 Fondos de garantías
				\4 Idea clave
				\4[] Mejorar acceso a crédito
				\4[] $\to$ De PYMES que comercian/invierten con PEDS
				\4[] $\to$ De PEDs que tratan de acceder a capital
				\4 Garantías de cartera
				\4[] Cobertura de carteras de créditos
				\4[] $\to$ Otorgados por IFinancieras a PYMES
				\4 Garantías institucionales
				\4[] Cobertura a riesgo de impago
				\4[] $\to$ De receptores de ayuda al desarrollo
				\4[] $\then$ Respecto de crédito
			\3 Fondos soberanos
				\4 Idea clave
				\4[] Capital controlado por sector público
				\4[] $\to$ Suavizar perfil intertemporal de ahorro
				\4[] $\then$ Invertir rentas y ahorro nacional
				\4[] Habitual en exportadores netos
				\4[] $\to$ Rentabilizar activos exteriores
				\4[] Posible canalizar hacia inversión en PEDs
				\4[] $\to$ Herramienta de desarrollo
				\4 Inversión soberana hacia PEDs
				\4[] Infraestructuras
				\4[] Energía
				\4 Implicaciones geopolíticos
				\4[] Palanca de influencia sobre PEDs
			\3 Transformación de características temporales
				\4 Idea clave
				\4[] Trasladar al presente recursos financieros
				\4[] $\to$ A partir de compromisos futuros de AOD
				\4 IFFIm -- Servicio Financiero Internacional para la Inmunización
				\4 Debt2Health
				\4 GAVI -- Global Alliance for Vaccines and Immunization
				\4 Implicaciones
				\4[] No movilizan financiación adicional
				\4[] Impacto modesto hasta la fecha
				\4[] Sí pueden aumentar eficacia de recursos disponibles
			\3 Mitigación del riesgo
				\4 Idea clave
				\4[] Aseguramiento de riesgos
				\4[] $\to$ Salud pública
				\4[] $\to$ Desastres naturales
				\4[] Garantizar niveles mínimos de demanda
				\4[] $\to$ Mercados de vacunas y medicamentos
				\4[] $\then$ Mantener mercado garantizado a productores
				\4 Implicaciones
				\4[] Reducen riesgo sistemático de inversión en PEDs
				\4[] Incentivo a provisión de bienes esenciales
			\3 Captación de contribuciones privadas voluntarias
				\4 Idea clave
				\4[] Informar a potenciales contribuyentes altruistas
				\4[] $\to$ Oportunidades de donación
				\4[] $\to$ Necesidades de PEDs
				\4 Implicaciones
				\4[] Éxito y volúmenes moderados
				\4[] Coste de administración elevado
				\4[] $\to$ Muchas pequeñas transacciones
				\4[] $\to$ Necesario gestionar
			\3 Modalidades impositivas internacionales
				\4 Idea clave
				\4[] Impuesto sobre determinada transacción
				\4[] $\to$ Destino específico
				\4 Gravamen sobre transporte aéreo
				\4[] Implementada tras Conf. Fuentes Innovadoras de 2006
				\4[] 13 países incluido FRA, UK, BRA, LUX
				\4[] Impuesto sobre billetes aéreos
				\4[] $\to$ Destinado a fondo global de salud
				\4 Ventajas
				\4[] $\to$ Fácil de recaudar
				\4[] $\to$ Costes de administración reducidos
				\4[] $\to$ Difícil evasión
				\4[] $\to$ Fácil implementación legal
				\4 Desventajas
				\4[] $\to$ Volumen limitado
				\4[] $\to$ Introducción de distorsiones
				\4[] $\to$ Impacto regresivo en países desarrollados
			\3 Propuestas
				\4 Apalancamiento de nuevos DEG
				\4[] Emitir nuevos DEG en FMI a favor de PEDs
				\4[] $\to$ Utilizarlos para garantizar bonos de desarrollo
				\4[] Requiere reforma de artículos de FMI
				\4[] Posibles riesgos sistémicos
				\4 Impuesto del carbono
				\4[] Propuesta de introducción a nivel global
				\4[] Recaudación destinada a mitigación y reducción
				\4[] $\to$ Especial impacto en PEDs
				\4 Impuesto sobre transacciones financieras
	\1 \marcar{El problema de la deuda externa}
		\2 Idea clave
			\3 Contexto
				\4 Economías abiertas
				\4[] Flujos comerciales y de capital
				\4[] $\to$ Ventajas del comercio
				\4[] $\to$ Financiación del desarrollo
				\4[] $\to$ Posibilidad de endeudamiento
				\4 Países en desarrollo
				\4[] Ciclos volátiles
				\4[] Diversificación relativamente baja
				\4 Historia reciente
				\4[] Fases de rápido endeudamiento
				\4[] Problemas de sostenibilidad
			\3 Objetivo
				\4 Analizar problemas de deuda externa
				\4[] En PEDs
				\4 Caracterizar agentes relevantes
				\4 Entender problemas de crisis de deuda de PEDs
				\4 Definir principales instrumentos de resolución
				\4[] Clubes de acreedores
				\4[] Iniciativas de reducción de deuda
			\3 Resultados
				\4 Sostenibilidad de la deuda
				\4[] Factores internos y externos a economía
				\4[] Puede requerir ajuste interno
				\4 PEDs especialmente vulnerables a crisis de deuda
				\4 Marco internacional para solución ordenada de crisis
				\4[] Relativamente ad-hoc
				\4[] IFIs multilaterales papel central
				\4[] Clubes de acreedores
				\4 ``Pecado original'' de PEDs
				\4[] Endeudamiento en moneda extranjera
				\4[] $\to$ Aversión a deuda tras 2000s
				\4[] $\to$ Emergentes y PEDs tienden a acumular reservas
		\2 Sostenibilidad de la deuda
			\3 Idea clave
				\4 Contexto
				\4[] Economía de los PEDs
				\4[] $\to$ Fuerte sensibilidad al ciclo
				\4[] Comercio exterior
				\4[] $\to$ Catalizador de desarrollo
				\4[] Endeudamiento exterior
				\4[] $\to$ No siempre sostenible
				\4[] $\to$ No siempre utilizado para inversión sostenible
				\4 Objetivo
				\4[] Caracterizar endeudamiento sostenible en el tiempo
				\4[] Identificar factores de insostenibilidad
				\4 Resultados
				\4[] Conjunto de factores determinantes
				\4[] $\to$ Interés y stock de deuda principales
				\4[] Necesario superávit comercial mínimo
				\4[] $\to$ Para mantener deuda estable
				\4[] $\then$ Necesaria financiación externa en caso contrario
			\3 Formulación
				\4 Valor presente de deuda debe converger
				\4[] Superávits comerciales descontados al presente
				\4[] $\to$ Deben igualar valor presente de la deuda
				\4[] $\sum_{t=1}^\infty \frac{1}{(1+r)^\infty} \left( C_t + I_t* \right) = (1+r) \cdot B_0 \sum_{t=1}^\infty (Y_t - G_t)$
				\4 Nivel de deuda estable sobre PIB
				\4[] PIB y deuda deben crecer a la misma tasa:
				\4[] $\to$ $Y_{t+1} = (1+g) Y_t$
				\4[] $\to$ $B_{t+1} = (1+g) B_t$
				\4[] (asumiendo no hay cuenta de K $\to$ CC = CF):
				\4[] $\text{CA}_t = B_{t+1} - B_t = g B_t$
				\4[] Definición de cuenta corriente:
				\4[] $\text{CA}_t = rB_t + Y_t - C_t - G_t - I_t = rB_t + \text{TB}_t$
				\4[] $\then$ $g B_t = r B_t + \text{TB}_t$
				\4[] $\then$ \fbox{$\frac{\text{TB}_t}{Y_t} = \frac{-(r-g) B_t}{Y_t}$}
				\4[] Asumiendo deuda positiva: $B_t < 0$
				\4[] $\to$ $r>g$ $\then$ $\text{TB}_t$ debe ser > 0
				\4[] $\then$ Necesario superávit ByS\footnote{En este análisis de la sostenibilidad de la deuda externa, el saldo de bienes y servicios se asemeja al déficit primario en el análisis de sostenibilidad de la deuda pública} para estabilizar deuda externa
				\4 Superávit ByS insuficiente
				\4[] Necesaria financiación exterior
				\4[] $\to$ ¿Hay acceso a mercado?
				\4[] $\then$ ¿Puede producirse crisis de deuda?
			\3 Factores principales que determinan crisis
				\4 Volumen de deuda
				\4[] En PEDs, denominada en moneda extranjera
				\4[] $\to$ Sensible a devaluaciones y depreciaciones
				\4[] Influye también sobre tipos de interés
				\4 Tasa de crecimiento
				\4[] Muy dependiente de uso de capital captado
				\4[] $\to$ Inversiones productivas vs consumo
				\4 Evolución de los tipos de interés
				\4[] Subidas de tipos de interés en divisas fuertes
				\4[] $\to$ Aumentan superávit necesario
				\4[] $\then$ Requieren políticas contractivas
				\4[] $\then$ Aumentan tensiones políticas internas
				\4[] PEDs especialmente sensibles a PM de Fed y BCE
				\4 Absorción interna
				\4[] Cuanto mayor absorción interna
				\4[] $\to$ Menores recursos reales libres
				\4[] Aparición de dilema
				\4[] $\to$ Ajuste interno vs financiación externa
				\4[] $\then$ Más ajuste interno, menos fin. exterior necesaria
				\4 Relación real de intercambio
				\4[] Aumenta compromiso de recursos reales necesarios
				\4 Acceso a la financiación internacional
				\4[] Aumenta deuda
				\4[] $\to$ Presión sobre sostenibilidad
				\4[] Pueden aliviar tensiones de liquidez a c/p
				\4[] $\to$ Ataque especulativo
				\4[] $\to$ Aumento brusco de tipos de interés
				\4 Expectativas
				\4[] Interés exigido a PEDs depende de expectativas
				\4[] $\to$ Interés futuro
				\4[] $\to$ Tipo de cambio
				\4[] $\to$ Relación real de intercambio
				\4[] $\to$ Demanda de materias primas
				\4 Actitud de las autoridades
				\4[] Impago de la deuda tiene costes y beneficios
				\4[] Costes de impago
				\4[] $\to$ Bienes en el exterior
				\4[] $\to$ Limitación del acceso comercial
				\4[] $\to$ Pérdida de calidad crediticia
				\4[] $\to$ Ajuste brusco de balanza de pagos
				\4[] Beneficios del impago
				\4[] $\to$ Corto plazo
				\4[] $\to$ Permite suavizar reducción de absorción
			\3 Análisis de sensibilidad
				\4 Variables sujetas a incertidumbre y endogeneidad
				\4 Habitual estimación de escenarios
				\4[] Pesimista, conservador, favorable...
				\4 Postular:
				\4[] -- Relaciones entre variables
				\4[] -- Distribuciones de probabilidad sobre variables
				\4 Caracterizar escenarios
				\4[] Dinámicas de deuda en diferentes contextos
			\3 Intolerancia a la deuda
				\4 Hecho empírico recurrente
			\3 Intolerancia a la deuda externa\footnote{Ver en \href{https://www.nber.org/papers/w9908}{Reinhart, Rogoff, Savastano (2003)} el concepto original.}
				\4 Reinhart, Rogoff y Savastano (2003)
				\4 Fenómeno recurrente en últimos dos siglos
				\4[] Países en desarrollo actualmente
				\4[] Otros países anteriormente
				\4[] Países impagan deuda de manera repetida
				\4[] $\to$ Doméstica y exterior
				\4[] $\then$ A pesar de niveles relativamente bajos
				\4[] Historia repetida de:
				\4[] $\to$ Niveles relativamente bajos de deuda
				\4[] $\to$ Procesos de ajuste muy costos en Y, L
				\4 Umbrales de intolerancia
				\4[] En países con intolerancia a la deuda
				\4[] $\to$ Mucho menores que en desarrollados
				\4[] A veces tan bajos como 15 o 20\%
				\4[] Empíricamente, existen umbrales seguros
				\4[] $\to$ Niveles máximos de endeudamiento
				\4[] $\then$ Que economía puede soportar sin pagar
				\4 Factores que determinan umbral de tolerancia
				\4[] Factores de economía política
				\4[] Path-dependency
				\4[] Sistemas fiscales débiles
				\4[] $\to$ Baja capacidad recaudatoria
				\4[] Sistemas financieros inestables
				\4[] $\to$ Riesgo moral endeudamiento exterior
				\4[] $\to$ Supervisión deficiente
				\4 En PEDs más bajo que en desarrollados
				\4[] $\to$ En algunos casos, apenas 15\% de PIB
				\4 Apertura de cuenta financiera
				\4[] Potencial aumento exterior
				\4[] $\to$ Aumenta probabilidad de superar umbral
				\4[] $\then$ Impago
				\4[] $\then$ Ajuste brusco
				\4[] $\then$ Recesión, desempleo, inflación
				\4[] $\then$ Mercados de capital
			\3 Análisis de sostenibilidad de la deuda (DSA) del FMI
				\4 Idea clave
				\4[] Marco estandarizado de análisis
				\4[] $\to$ Permitir comparabilidad
				\4[] Construir escenarios y comparar
				\4[] $\to$ A partir de información disponible
				\4[] $\to$ Previsiones sobre variables macro
				\4[] $\then$ Escenario base
				\4[] $\then$ Escenario muy optimista
				\4[] $\then$ Escenario conservador
				\4[] Dos tipos de análisis en función de acceso
				\4 Países con acceso a mercado
				\4[] Economías con poca dependencia de flujos oficiales
				\4[] Marco de 2011
				\4[] División de países en función de riesgo
				\4[] -- Bajo o moderado
				\4[] $\to$ Se aplica DSA básico
				\4[] -- Riesgo elevado
				\4[] Deuda sobre PIB
				\4[] $\to$ $>60\%$ en avanzadas
				\4[] $\to$ $>50\%$ en emergentes
				\4[] Necesidades de financiación altas presentes o futuras
				\4[] $\to$ $>15\%$ en avanzados
				\4[] $\to$ $>10\%$ en emergentes
				\4[] Tienen acceso excepcional a recursos del FMI
				\4[] $\then$ Se aplica DSA High Scrutiny
				\4 Países sin acceso a mercado
				\4[] Marco conjunto para FMI y GBM
				\4[] Países dependientes de flujos oficiales
				\4[] $\to$ Que desean evitar deuda excesiva
				\4[] Utilizado por deudores y acreedores
				\4[] Iniciativas de alivio de deuda HIPC, MDRI
				\4[] $\to$ Decisión basada en DSA
				\4[] $\then$ Determinar alivio necesario de deuda
		\2 Agentes en escenario internacional de la deuda
			\3 Acreedores internacionales
				\4 Acreedores públicos bilaterales
				\4[] Generalmente resultado de dos transacciones:
				\4[] -- Créditos de ayuda al desarrollo concesionales
				\4[] -- Seguros de crédito a la exportación
				\4 Acreedores privados
				\4[] Principalmente:
				\4[] $\to$ Banco comercial
				\4[] $\to$ Bonistas
				\4 Instituciones financieras multilaterales
				\4[] Peso especialmente relevante en PMAs
				\4[] Acreedores preferentes salvo programas especiales
				\4[] $\to$ Para mantener rating óptimo
			\3 Club de París\footnote{Ver Imam (2018) en carpeta del tema.}
				\4 Idea clave
				\4[] 22 principales acreedores bilaterales públicos
				\4[] $\to$ Principalmente OCDE+Brasil+Rusia+Sudáfrica
				\4[] Tratamiento de deuda pública de PEDs en manos de SPúblico
				\4[] $\to$ Deuda de media y largo plazo
				\4[] $\to$ Deuda con vencimiento original < 1 año excluida
				\4[] $\to$ También deuda privada de PEDs si garantizada por PEDs
				\4[] $\to$ En gran medida, agencias de crédito a la exportación
				\4[] En 2018, miembros poseen activos por 301.000 M
				\4[] $\to$ La mitad, pasivos resultado de AOD
				\4[] Observadores sin voto
				\4[] $\to$ FMI
				\4[] $\to$ Bancos de desarrollo
				\4[] Reuniones periódicas para tratar deuda externa
				\4[] Creación ad-hoc en 1956
				\4 Principios - SICCCC
				\4[] i) \marcar{S}olidaridad
				\4[] Acreedores actúan y negocian como grupo
				\4[] Posiciones coherentes entre sí
				\4[] ii) \marcar{I}ntercambio de información
				\4[] Acreedores comparten datos y previsiones
				\4[] iii) \marcar{C}ondicionalidad
				\4[] Exigen de deudor
				\4[] $\to$ tenga acuerdo con FMI
				\4[] $\to$ haya agotado oportunidades de financiación
				\4[] $\to$ no pueda realizar más ajuste interno
				\4[] $\then$ Club de París como último recurso
				\4[] iv) \marcar{C}aso por caso
				\4[] Tratamiento a medida de cada caso
				\4[] Influencia de enfoque Evian
				\4[] Análisis DSA sigue siendo punto de partida
				\4[] Establecimiento de fechas de corte
				\4[] $\to$ Deuda contraída a partir de fecha
				\4[] $\then$ No se renegocia
				\4[] v) \marcar{C}omparabilidad de trato
				\4[] Exigencia a otros acreedores no-Club de París
				\4[] $\to$ Realicen esfuerzo similar
				\4[] $\then$ Evitar condonación sólo sirva para que otros cobren
				\4[] vi) \marcar{C}onsenso
				\4[] Foro informal
				\4[] Miembros no pueden imponer decisiones recíprocamente
				\4[] Todos los acuerdos deben ser adoptado por cada país miembro
				\4 Actuaciones
				\4[] Reuniones mensuales generalmente
				\4[] $\to$ Salvo agosto y febrero
				\4[] Acreedores valoran situación exterior de deudores
				\4[] Apertura de negociaciones con deudor
				\4[] $\to$ Cuando ha alcanzado acuerdo con FMI
				\4[] $\to$ Cuando ha demostrado incapacidad para pagar
				\4[] $\then$ Tiene un financing gap susceptible de reestructurar
				\4[] Periodos de consolidación
				\4[] $\to$ Etapa en la que se aplica acuerdo
				\4[] $\to$ Suele coincidir con periodo de acuerdo con FMI
				\4[] Tratamiento de flujos
				\4[] $\to$ Reestructuración de los pagos debidos
				\4[] $\to$ Pagos debidos generalmente los que vencen en periodo de consolidación.
				\4[] Tratamiento de stocks
				\4[] $\to$ Solución final al problema de la deuda considerada
				\4[] $\to$ Reducción del stock de deuda
			\3 Club de Londres
				\4 Foro informal de coordinación de acreedores privados
				\4[] Primera reunión en Londres, en 70s
				\4[] Importancia creciente en 80s
				\4 Sin composición ni localización fija
				\4 Bank Advisory Committee
				\4[] Tratamiento de deuda de un país concreto
				\4[] $\to$ Ad-hoc para cada caso
				\4 Relación con Club de París
				\4[] A través de comparabilidad de trato
				\4 Decisiones por consenso
			\3 Comités de bonistas
				\4 Coordinación de bonistas acreedores
				\4 No se reúnen en Club de Londres
				\4 Comités ad-hoc para cada caso
			\3 Deudores internacionales
				\4 PEDs y PMA en este contexto
			\3 Marco institucional
				\4 Facilitar relaciones entre deudores y acreedores
				\4 Gestionar posibles crisis de manera ordenada
				\4 Coordinar acreedores
		\2 Crisis de los 80\footnote{Ver ``third world debt''.}
			\3 Contexto
				\4 Crecimiento elevado desde 50s
				\4[] En 70s, crecimiento se mantiene muy fuerte
				\4[] $\to$ Deuda/PIB a la baja
				\4[] $\then$ Deuda no se percibe como un problema
				\4 Relación real de intercambio favorable
				\4[] Balanza comercial favorable a PEDs
				\4[] Mejora percepción de sostenibilidad
				\4 Fuerte endeudamiento en los PEDs a partir de 1974
				\4[] Superávits por CC en exportadores de petróleo
				\4[] $\to$ Depositan en bancos comerciales
				\4[] Bancos canalizan ahorro hacia PEDs no exp. petróleo
				\4[] Desarrollados entran en recesión
				\4[] $\to$ Cae demanda de importaciones
				\4[] $\then$ Cae demanda de financiación de desarrollados
				\4[] Interés nominal bajo por exceso de ahorro en exportadores
				\4[] $\then$ PEDs no exportadores se endeudan
				\4[] $\then$ Deuda externa de PEDs en 1978 es $250\%$ la de 1973
				\4 Bancos comerciales cubren riesgo cambiario y de interés
				\4[] Denominados en dólares
				\4[] $\to$ Depósitos de exportadores de petróleo
				\4[] $\to$ Préstamos a PEDs
				\4[] $\then$ Riesgo de tipo de cambio cubierto
				\4[] Tipos de interés de préstamos
				\4[] $\to$ Referenciados a LIBOR
				\4[] $\then$ Protección frente a riesgo de interés
				\4[] Aumento del riesgo de default
				\4[] $\to$ Interés de deuda muy volátil
				\4[] $\to$ Riesgo de tipo de cambio transferido a deudores
				\4 Estanflación
				\4[] Países desarrollados sufren shock de oferta
				\4[] Crecimiento muy débil
				\4[] Tasas de inflación muy elevadas
			\3 Eventos
				\4 Segundo shock del petróleo en 1979
				\4[] Fuerte aumento de los precios
				\4[] Algunos PEDs pasan a ser exportadores netos
				\4[] $\to$ Especialmente México
				\4[] $\then$ MEX aumenta deuda esperando altos ingresos futuros
				\4 Subida de tipos de interés en EEUU
				\4[] Pagos de interés sobre deuda en dólares
				\4[] $\to$ Aumenta fuertemente en PEDs
				\4[] Salida de capital hacia EEUU
				\4 Apreciación del dolar
				\4[] Resultado de subida de tipos
				\4 Exportadores energéticos aumenta inversión doméstica
				\4[] Aumenta capacidad de absorción doméstica
				\4[] Petrodólares desaparecen de mercados financieros
				\4[] Menos capital disponible para PEDs deficitarios
				\4 Crisis en México
				\4[] Expectativa de devaluación
				\4[] $\to$ Aumenta salida de capital
				\4[] $\then$ Más presión devaluatoria
				\4[] Políticas fiscales expansivas
				\4[] $\to$ Agravan problema
				\4[] Reservas de divisas caen fuertemente
				\4[] $\to$ Suspensión de pagos en 1982
				\4 Contagio de la crisis
				\4[] En 1985, 15 identificados como altamente endeudados
				\4[] $\to$ Requieren asistencia coordinada
				\4[] A finales de 1986, más de 40 países
				\4[] $\to$ Problemas de deuda externa
				\4 Riesgo sistémico en acreedores
				\4[] Exposición muy elevada de principales bancos
				\4[] Especialmente a países LATAM
				\4[] Riesgos para estabilidad sistema bancario
				\4[] $\to$ En desarrollados
			\3 Consecuencias
				\4 Estrategia inicial: new money
				\4[] Problema se percibe como iliquidez
				\4[] $\to$ No como un problema de solvencia
				\4[] Actuación de Bancos comerciales (Club de Londres):
				\4[] $\to$ Alargaron vencimiento
				\4[] $\to$ Proporcionaron nueva financiación
				\4[] Actuación de gobiernos
				\4[] $\to$ Reestructuración en marco de Club de París
				\4[] $\to$ Condicionando a obtención de préstamos del FMI
				\4[] Actuación del FMI
				\4[] $\to$ Condicionalidad: ajuste interno
				\4[] $\to$ Programas de ajuste ad-hoc
				\4[] Énfasis en:
				\4[] $\to$ Reformas
				\4[] $\to$ Mejora de condiciones externas
				\4[] $\to$ Concesión de nueva financiación (``new money'')
				\4[] $\then$ Posible salida de crisis
				\4[] $\then$ Papel central de FMI
				\4[] Coincidencia de intereses de las tres partes
				\4[] -- Deudores
				\4[] $\to$ Evitar colapso definitivo
				\4[] -- Bancos comerciales
				\4[] $\to$ Evitar quiebra dada exposición
				\4[] -- Acreedores
				\4[] $\to$ Evitan crisis de sistema bancario
				\4[] Resultados
				\4[] $\to$ Decepcionantes
				\4[] $\to$ Oposición creciente a ajustes externos
				\4[] $\to$ Cada vez menor financiación de FMI
				\4[] $\to$ Bancos sanean cuentas y reducen exposición
				\4[] $\then$ Balance positivo para la banca
				\4[] $\then$ Situación de endeudados no mejora
				\4[] $\to$ Éxitos en CHILE, INDO, KOR, TUR
				\4[] $\then$ Compromisos sin nueva deuda
				\4 Plan Baker (1985)
				\4[] Énfasis en crecimiento de largo plazo
				\4[] Financiación conjunta privada+IFI multilaterales
				\4[] Condicionalidad para crecimiento de largo plazo
				\4[] $\to$ Liberalización comercial
				\4[] $\to$ Fomento de inversión nacional y extranjera
				\4[] $\to$ Menos intervencionismo
				\4[] Menos financiación de la esperada finalmente
				\4[] $\to$ Sin acuerdo con FMI en algunos países
				\4[] Bancos redujeron exposición aún más
				\4[] Brasil suspendió pagos en 1987
				\4 Plan Brady (1989-1994)
				\4[] Énfasis en reestructuración
				\4[] $\to$ ``Estrategia de salida''
				\4[] $\to$ Necesario sacrificio por acreedores
				\4[] $\then$ Objetivo de USD 70.000M reducción
				\4[] $\then$ Finalmente casi USD 60.000M para 1994
				\4[] Canje de deuda existente
				\4[] $\to$ Por bonos con menor nominal (Bonos brady)
				\4[] $\then$ Pero mayores garantías de pago
				\4[] Emisión de Bonos Brady
				\4[] $\to$ Sustituyen a préstamos bilaterales
				\4[] $\to$ Respaldados por bonos americanos
				\4[] $\then$ Permiten reestructuración
				\4[] $\then$ Aumentan liquidez de deuda
				\4[] Términos Houston del Club de París
				\4[] $\to$ Reprogramación de vencimientos para rentas medias
				\4[] $\to$ Acuerdos de conversión de deuda\footnote{La deuda se convierte en inversión en el país. }
				\4[] $\then$ Avance respecto a programas previos
				\4 Valoración
				\4[] Crecimiento económico a principios de 90
				\4[] $\to$ Recuperación general en la región
				\4[] Bancos reducen fuertemente su exposición
				\4[] Aparentemente, crisis solucionada
		\2 Crisis de los 90\footnote{Ver ``third world debt''.}
			\3 Contexto
				\4 Recuperación del crecimiento
				\4 Eliminación de restricciones de cuenta de capital
				\4 Emisión de bonos sustituyen a préstamos bilaterales
			\3 Eventos
				\4 Crisis de México en 1994
				\4[] Estabilización tras crisis de 80s
				\4[] Crecimiento relativamente alto a finales de 80s
				\4[] Aumento del déficit por cuenta corriente
				\4[] $\to$ 8\% en 1994
				\4[] Fuertes entradas de capital en México
				\4[] Financiación mediante tesobonos
				\4[] $\to$ Denominados en dólares
				\4[] $\to$ Vencimiento de corto plazo
				\4[] Asesinato de candidato presidencial en 1994
				\4[] $\to$ Comienza presión sobre reservas
				\4[] Cambio brusco de expectativas
				\4[] $\to$ Cambio en sentido de flujos de capital
				\4[] $\to$ Devaluación
				\4[] $\then$ Aumento explosivo del valor de la deuda
				\4[] $\then$ Crisis de pagos
				\4[] Programa de asistencia del FMI y EEUU en 1995
				\4[] $\to$ 50.000 millones de dólares
				\4[] $\to$ Reestructuración de deuda con acreedores
				\4[] $\to$ Medidas de política económica recomendadas
				\4[] $\then$ Control del déficit por CC en 1995
				\4[] $\then$ Acceso recuperado a mercados de capital en 1997
				\4 Crisis asiática de 1997
				\4[] Múltiples países afectados
				\4[] Deuda exterior denominada en dólares
				\4[] Tipos de cambio fijo
				\4[] Agentes privados tienen garantía implícita de gobierno
				\4[] $\to$ Se endeudan en dólares
				\4[] $\to$ Sistemas financieros relativamente desregulados
				\4[] $\to$ Déficits crecientes de cuenta corriente
				\4[] Incertidumbre política
				\4[] $\to$ dudas sobre continuidad de tipo fijo
				\4[] Tailandia detonante de la crisis
				\4[] Indonesia y Tailandia principales afectados
				\4[] Programas de ayuda de FMI y BM
				\4[] $\to$ Recuperaciones lentas en TAI, INDO
				\4 Crisis de Rusia de 1998
				\4[] Debilidad institucional desde fin de URSS
				\4[] Caída de precios de exportaciones energéticas
				\4[] $\to$ Principales fuentes de divisas para Rusia
				\4[] Salida de capitales de PEDs post crisis asiática
				\4[] Anuncio de moratoria de pago de deuda externa en 1998
				\4[] Flotación del rublo frente al dolar
				\4[] $\to$ Precio del dólar en rublos se triplica
				\4[] Reestructuración de todas las obligaciones exteriores
				\4[] $\to$ Incluida deuda FMI contraída un mes antes
				\4 Crisis de Argentina de 2001\footnote{Págs. 39 y 40 en CECO Nuevo. Buen resumen de la crisis.}
				\4[] Tipo fijo 1:1 con dólar desde 1991
				\4[] $\to$ Aparentemente currency board
				\4[] Contratos denominados en dólares
				\4[] Déficits fiscales persistentes
				\4[] $\to$ Gobierno se endeuda en dólares
				\4[] Depreciación del real brasileño en 1998
				\4[] $\to$ Argentina pierde competitividad relativa
				\4[] $\then$ Presión sobre cuenta corriente
				\4[] Bajos ingresos fiscales
				\4[] Presión sobre peso argentino crece
				\4[] Crédito del FMI en 2001
				\4[] Retiradas masivas de dólares
				\4[] $\to$ Corralito
				\4[] Protestas generalizadas
				\4[] $\to$ Dimisión de gobierno
				\4[] Enorme stock de deuda
				\4[] $\to$ Supera los 140.000 M de dólares
				\4[] Suspensión de pagos unilateral en 2002
				\4[] $\to$ Emisiones muy complejas y variadas
				\4[] $\to$ Muchas jurisdicciones y muchas modalidades
				\4[] $\then$ Procesos legales muy largos y complejos
				\4[] Acuerdo con FMI y BM en 2003
				\4[] $\to$ CRédito puente
				\4[] Canje de 2005
				\4[] $\to$ Gobierno ofreció bonos con haircut muy elevados
				\4[] $\then$ No todos los acreedores aceptan
				\4[] Caída del PIB 2002 respecto a 1999 supera 10\%
				\4[] $\to$ Muy fuerte aumento de pobreza
				\4 Crisis de Ecuador de 2008
				\4[] Fuerte aumento de deuda en 70s y 80s
				\4[] Pico del 115\% en 90s
				\4[] Reducción paulatina hasta mediados de 2000s
				\4[] $\to$ Con importante ajuste interno+oposición
			\3 Consecuencias
				\4 Debate sobre el ``sovereign debt restructuring mechanism''
				\4[] ¿Debe $\exists$ un marco de resolución de crisis deuda?
				\4[] $\to$ En vez de sistema ad-hoc FMI/CdParis/AcPrivados
				\4[] FMI propuso SDRM
				\4[] $\to$ Sovereign Debt Restructuring Mechanism
				\4[] $\then$ Integra todos los acreedores y FMI + deudores
				\4[] $\then$ Debate en 2002 y 2003
				\4[] $\then$ Sin éxito
				\4 Cláusulas de acción colectiva
				\4[] Reestructuración de deuda griega en 2012
				\4[] $\to$ Pérdida del 74\%
				\4[] Introducción retroactiva de CAC por gobierno griego
				\4[] $\to$ Para deuda emitida bajo legislación local
		\2 Cláusulas de acción colectiva\footnote{Ver \href{https://www.europarl.europa.eu/RegData/etudes/BRIE/2019/637974/EPRS_BRI(2019)637974_EN.pdf}{EPRS (2019) sobre CACs}}
			\3 Idea clave
				\4 Recomendaciones de FMI de 2014
				\4[] Mejorar comunicación entre tesoro e inversores
				\4[] Gestión y control de riesgos
				\4[] Test de estrés
				\4[] Provisión de liquidez en mercados secundarios
				\4[] Incluir cláusulas de acción colectiva (CACs)
				\4 Concepto
				\4[] Conjunto de reglas que definen mayorías
				\4[] $\to$ En caso de alteración de condiciones de deuda pública
				\4[] $\then$ Acciones que pueden tomar los acreedores
				\4 Objetivos de las CAC
				\4[] Ordenar procesos de reestructuración
				\4[] Aumentar seguridad jurídica
				\4 Evolución
				\4[] Inclusión en documentación a principios de s. XXI
				\4[] Reestructuración de deuda argentina
				\4[] $\to$ Determinados acreedores no aceptaron reestructuración
				\4[] $\then$ Justicia americana falló en su favor
				\4[] $\then$ Argentina no puede pagar reestructurados antes de los que no aceptaron
			\3 Single-limb CAC / votación única
				\4 Recomendadas por FMI
				\4 Votación que afecta a todos los acreedores
				\4[] Acreedores deciden si aceptar condiciones de emisor
				\4[] $\to$ Vinculan a todos los acreedores
				\4[] $\then$ También a los que votasen en contra
				\4 Reduce incentivos oportunistas
			\3 Double-limb CAC / votación con doble vuelta
				\4 Primero
				\4[] Aprobación por tenedores del saldo de deuda viva
				\4 Segundo
				\4[] Aprobación por tenedores de cada bono/instrumento
				\4[] $\to$ Votaciones separadas
				\4 Aumenta incentivos a holdout\footnote{Es decir, a negarse a aceptar reestructuración o cambio de condiciones.}
		\2 Reestructuración de deuda países más pobres
			\3 Idea clave
				\4 Contexto
				\4[] Estrategias de solución de 80s (Baker, Brady...)
				\4[] $\to$ Dirigidas sobre todo a países renta media
				\4[] Países de rentas bajas y débil crecimiento
				\4[] $\to$ También experimentan aumento deuda
				\4[] Muy poco impacto sobre sistema financiero internacional
				\4[] $\to$ Pero efecto sobre desarrollo local
				\4[] Debates sobre condonación de la deuda
				\4[] $\to$ Pérdida de acceso a mercados de capital
				\4[] $\to$ Problemas de riesgo moral
				\4[] $\to$ Contribuyente de país acreedor condona realmente
				\4[] $\to$ Menos fondos disponibles para AOD
				\4[] $\to$ Caída de rating de IFIs tras condonar
				\4[] $\to$ Condicionalidad necesaria
				\4 Objetivo
				\4[] Solucionar bloqueo de fin. a PMAs
				\4[] Establecer mecanismos coherentes de condonación
				\4 Resultados
				\4[] Endeudamiento persistente
				\4[] Acceso a fin. principalmente vía IFIs
				\4[] Casos de éxito son raros
			\3 Tratamientos iniciales
				\4 Concepto de tratamiento
				\4[] Combinación de:
				\4[] $\to$ Reestructuración
				\4[] $\to$ Condonación
				\4[] $\to$ Nuevos préstamos concesionales
				\4[] $\to$ Diferentes grados
				\4 Tratamiento clásico
				\4[] $\to$ Créditos incluyendo AOD reestructurado
				\4[] $\to$ Tipo de interés de mercado
				\4 Tratamiento ``Toronto'' de 1988
				\4[] $\to$ Para países más pobres
				\4[] $\to$ Reconoce 33\% de condonación
				\4[] $\then$ Reemplazado por términos Nápoles
				\4 Tratamiento ``Houston'' o tratamiento ``Trinidad''
				\4[] $\to$ Aumento de condonación al 50\%
				\4 Tratamiento ``Nápoles''
				\4[] $\to$ Para HIPC
				\4[] $\to$ 67\% de condonación no AOD
				\4[] $\to$ AOD reestructurada con gracia 16 años
				\4 Tratamiento Colonia
				\4[] $\to$ HIPC
				\4[] $\to$ 90\% de condonación no AOD
				\4[] $\to$ AOD reestructurada con gracia de 16 años
				\4 Valoración
				\4[] Poca adaptación a necesidades específicas
				\4[] $\to$ Condonación fija
				\4[] Condonación insuficiente
				\4[] $\to$ Referida a deuda previa a corte
				\4[] $\then$ No a stock de deuda total
				\4[] Deuda con IFIs queda fuera de condonaciones
			\3 Tratamiento Evian
				\4 Adopción por Club de París en 2003
				\4[] Dirigido a países de renta media
				\4 Marco coherente de reducción de deuda
				\4[] $\to$ Para países de renta media
				\4[] $\then$ Evitar recurso constante a Club de París
				\4 Fase 1 (consolidación)
				\4[] $\to$ Reprogramación de vencimientos de 1 a 3 años
				\4[] $\to$ Consolidación de trayectoria de pagos
				\4[] $\to$ Cumplimiento de programas de FMI
				\4 Fase 2
				\4[] $\to$ Tratamiento amplio para sostenibilidad
				\4[] $\to$ Puede incluir condonación
				\4[] De 2009 a 2014
				\4[] $\to$ Cuatro acuerdos en marco de Club de París
			\3 Iniciativa HIPC\footnote{Ver \href{https://www.imf.org/en/About/Factsheets/Sheets/2016/08/01/16/11/Debt-Relief-Under-the-Heavily-Indebted-Poor-Countries-Initiative}{FMI sobre HIPC.}}
				\4 Idea clave
				\4[] \textit{Heavily Indebted Poor Countries}
				\4[] FMI y BM en 1996
				\4[] Iniciativa sistemática de tratamiento
				\4[] Llevar países pobres a niveles sostenibles
				\4[] Liberar recursos para gasto social
				\4[] Dos tipos de medidas conjuntamente aplicadas
				\4[] $\to$ Reducción de deuda por acreedores
				\4[] $\to$ Medidas de reforma económica
				\4[] 39 países candidatos actualmente
				\4[] $\to$ 36 han recibido alivio irrevocable
				\4[] Tres fases
				\4[Etapa 0] Fase preliminar
				\4[] Implementación de programa de FMI y BM
				\4[] Recibe tratamiento Nápoles preliminar
				\4[Etapa I] Punto de decisión
				\4[] Necesario cumplir 4 condiciones:
				\4[] I - Requisitos de financiamiento de AIF de BM
				\4[] II - Endeudamiento insostenible
				\4[] III - Trayectoria de reforma previa
				\4[] IV -- DELP -- Documento de Estrategia de Lucha contra Pobreza
				\4[] Una vez alcanzados requisitos
				\4[] $\to$ FMI y BM deciden habilitación para condonación
				\4[] $\then$ Posible alivio provisional del servicio de la deuda
				\4[] $\then$ Hasta 90\% provisional
				\4[] Análisis de sostenibilidad por FMI
				\4[] Criterio general de no sostenibilidad
				\4[] $\to$ Si \fbox{$\frac{\text{VAN Deuda}}{\text{Exportaciones}} \geq 150\%$}: deuda NO sostenible
				\4[] Criterio fiscal para países muy exportadores:
				\4[] $\to$ $\frac{\text{Exportaciones}}{\text{PIB}} \geq 30\%$
				\4[] $\to$ $\frac{\text{Ingresos fiscales}}{\text{PIB}} \geq 15\%$
				\4[] $\then$ Necesario \fbox{$\frac{\text{VAN Deuda}}{\text{Exportaciones}} \geq 250\%$} en vez de 150\%
				\4[Etapa II] Punto de culminación
				\4[] Establecer trayectoria ulterior de reformas
				\4[] Aplicar reformas fundamentales de apartado anterior
				\4[] Adoptar DELP por al menos un año
				\4[] Si se cumple
				\4[] $\to$ Reducción de deuda irrevocable
				\4[] $\then$ Deuda se reduce hasta cumplir ratio 150\%/250\%
				\4 Valoración
				\4[] IFIs también participan
				\4[] $\to$ Abandonan estatus de acreedor preferente
				\4[] $\then$ Necesario crear HIPC trust funds\footnote{Asumir coste de condonaciones sobre capital de bancos de desarrollo y similares con donaciones de los países miembros.}
				\4[] En la medida en reducción de gasto en deuda
				\4[] $\to$ Se destine a reducción de pobreza
			\3 IADM -- Iniciativa para el Alivio de la Deuda Multilateral
				\4 Idea clave
				\4[] Suplemento a HIPC
				\4[] Condonación total de deuda para países
				\4[] $\to$ Que hayan alcanzado el punto de culminación de HIPC
				\4[] Deuda de tres instituciones multilaterales
				\4[] $\to$ FMI, AIF de BM y FAfD
				\4[] $\to$ BIAmD en 2007
				\4[] Sin alivio de acreedores privados o bilaterales
				\4 Objetivos del Milenio/Objetivos de Desarrollo Sostenible
				\4[] Liberar recursos para servicio de deuda
				\4[] $\to$ Aumenta viabilidad de ODM/ODS
				\4 Propuesto en el G8 de 2005
				\4[] BIAmericanoD ofrece programa similar en2007
				\4 Diferencia con iniciativa HIPC
				\4[] No es reducción hasta nivel sostenible
				\4[] $\to$ Se trata de reducción total
				\4 Punto de corte
				\4[] Finales de 2004
			\3 Fondo fiduciario para contención y alivio de catástrofes
				\4 Idea clave
				\4[] FMI, BM y BAfD
				\4[] Alivio de deuda tras
				\4[] $\to$ Desastres naturales
				\4[] $\to$ Emergencias de salud pública
				\4[] $\to$ Epidemias
				\4[] Proyectos de contención
				\4 Alivio post-catástrofe
				\4[] Alivio de flujos de deuda
				\4[] Alivio de saldo de deuda
				\4[] Características necesarias
				\4[] $\to$ Afecta a 1/3 de población
				\4[] $\to$ Destrucción de 1/4 de capacidad productiva
				\4[] $\to$ Perjuicios superiores a 100\% de PIB
				\4 Contención de catástrofes
				\4[] Dedicado a catástrofes sanitarias
				\4[] Reducción de stock y flujos de deuda
				\4[] Medidas de mitigación de impacto
				\4 Financiación
				\4[] Inicialmente, saldos restantes de otros fondos
				\4[] Posteriormente, donaciones bilaterales
				\4 Establecido en 2015
			\3 Valoración
				\4 Beneficiarios
				\4[] Gran mayoría han alcanzado punto de culminación
				\4[] $\to$ Más de 35 sobre 40
				\4[] Desafíos comunes de países que aún no:
				\4[] $\to$ Preservar la paz y estabilidad
				\4[] $\to$ Mejorar la gestión del gobierno
				\4[] $\to$ Prestación de servicios básicos
				\4 Resultado de la iniciativa
				\4[] Reducción notable del servicio de la deuda
				\4[] Mejora de gestión de la deuda pública
				\4[] Persisten:
				\4[] $\to$ Vulnerabilidad a shocks externos
				\4[] $\to$ Debilidad institucional
				\4[] $\then$ Reducción de deuda no es herramienta adecuada
				\4 Eficacia de la iniciativa
				\4[] Participación voluntaria
				\4[] $\to$ Puede inducir trato desigual
				\4[] $\then$ IFIs multilat. mayor parte del apoyo
				\4[] $\then$ Otros acreedores sólo han aliviado parcialmente
				\4[] $\then$ Incentivos a free-riding
				\4[] Críticas por falta de generosidad
				\4[] $\to$ Esfuerzo bilat. volunt. posible también
				\4[] Distorsión introducida
				\4[] $\to$ Preferible donar para que repagasen
				\4[] $\then$ Mantener disciplina crediticia
				\4[] $\then$ Explicitar contribución de contribuyentes
				\4[] $\then$ Pero fatiga del donante
				\4 Financiación
				\4[] Mayor parte procede del FMI:
				\4[] $\to$ Aportaciones bilaterales
				\4[] $\to$ Recursos propios
				\4[] $\then$ Fondo fiduciario SCLP-HIPC
	\1[] \marcar{Conclusión}
		\2 Recapitulación
			\3 Financiación exterior del desarrollo
			\3 El problema de la deuda externa
			\3 Probando
		\2 Idea final
			\3 Economía de los países en desarrollo
			\3 Modelos del desarrollo
			\3 Teorías del crecimiento económico
			\3 Cooperación al desarrollo UE y España
\end{esquemal}

\conceptos

\concepto{AOD Bruta y Neta}

La AOD bruta es la suma de todas las cantidades desembolsadas por un donante. La AOD neta es igual a la AOD bruta menos los ingresos recibidos resultantes de los pagos de amortización. La distinción entre AOD bruta y neta tiende a desaparecer en la nueva forma de contabilizar la ayuda reembolsable, que entra en vigor en 2019. Se contabilizará simplemente el elemento de concesional del préstamo, en lugar del valor nominal del desembolso.


\concepto{Cálculo del elemento concesional de un préstamo}

El elemento concesional de un préstamo se define a partir del cociente entre el valor actual de los flujos de reembolso y la cuantía del préstamo:

\begin{equation}
\text{Elemento concesional} = 1 - \frac{\text{VA}_\text{flujos de reembolso}}{\text{Préstamo}}
\end{equation}

La tasa de descuento de referencia a la que descontar los flujos de reembolso se determina mediante la suma de la tasa de descuento de referencia del FMI ($5\%$) y un diferencial de riesgo que depende del país receptor. Así, a los países menos avanzados y los de renta baja se les aplica una tasa del 9\%, a los países de renta media baja una tasa del 7\% y a los países de renta media-alta una tasa del 6\%. 

\concepto{Cómputo de préstamos reembolsables como AOD}\footnote{Ver \url{http://realidadayuda.org/glossary/elemento-de-donacion}.}

Para que un préstamo reembolsable compute como AOD, el elemento concesional debe superar un determinado umbral que depende del país receptor. Para los países menos avanzados y otros países de renta baja, el umbral es del 45\%. Para los países de renta media-baja es del 15\%. Por último, para los países de renta media-alta el umbral es del 10\%. Así, cuanto menor sea la renta de un país mayor debe ser el elemento concesional para que el préstamo sea considerado AOD. 

\concepto{Objetivos de Desarrollo Sostenible}

Los Objetivos de Desarrollo Sostenible constan de 17 objetivos y 169 metas intermedias, con un horizonte de cumplimiento que finaliza en 2030. Los objetivos son los siguientes:

\begin{enumerate}
	\item Poner fin a la pobreza en todas sus formas
	\item Fin al hambre, lograr la seguridad alimentaria y promover la agricultura sostenible.
	\item Garantizar una vida sana.
	\item Educación inclusiva, equitativa y de calidad.
	\item Igualdad de género.
	\item Garantizar la disponibilidad del agua.
	\item Garantizar el acceso a una energía asequible, segura, sostenible y moderna.
	\item Promover crecimiento económico y pleno empleo.
	\item Infraestructuras robustas e industrialización inclusiva y sostenible que fomente la innovación.
	\item Garantizar modalidades de consumo y producción sostenibles.
	\item Adoptar medidas urgentes para combatir el cambio climático.
	\item Conservar y utilizar de manera sostenible los océanos, mares y recursos marinos.
	\item Proteger, restablecer y promover el uso sostenible de ecosistemas terrestres.
	\item Promover sociedades pacíficas e inclusivas para el desarrollo sostenible, facilitar acceso a justicia.
	\item Fortalecer medios de ejecución y revitalizar Alianza Mundial para el Desarrollo Sostenible.
\end{enumerate}

\preguntas

\seccion{Test 2018}

\textbf{36.} Entre los instrumentos de financiación exterior del desarrollo económico; el que presenta el inconveniente de ser especialmente volátil, pues fluctúa en función de las expectativas, pudiendo provocar comportamientos de rebaño es:

\begin{itemize}
	\item[a] La inversión extranjera directa.
	\item[b] La inversión en cartera.
	\item[c] Los préstamos.
	\item[d] Las remesas.
\end{itemize}

\seccion{Test 2015}

\textbf{40.} Señale la respuesta correcta relativa al club de París:
\begin{itemize}
	\item[a] Es un organismo formal integrado por países acreedores y deudores cuya función es coordinar formas de pago y renegociación de deudas externas de los países e instituciones de préstamo con el fin de solucionar los problemas de balanza de pago.
	\item[b] Solo se reúne con un país deudor que necesita reestructurar su deuda externa y ha alcanzado previamente un acuerdo con el FMI para implementar un programa de reformas que le permiten resolver sus problemas de pago.
	\item[c] Actúa de acuerdo con los principios de solidaridad, mayoría simple, trato privilegiado, no condicionalidad, intercambio de información y análisis caso por caso.
	\item[d] Los acuerdos del Club de París se aplican a la deuda a corto, mediano y largo plazo del sector público, ya que el acuerdo es firmado con los gobiernos de los países deudores que no pueden cumplir con sus obligaciones externas.
\end{itemize}

\seccion{Test 2006}

\textbf{36.} En el Club de París:

\begin{itemize}
	\item[a] Sólo es objeto de renegociación la deuda contraída durante el ``periodo de consolidación'', cuyos vencimientos tienen lugar con posterioridad a la ``fecha de corte''.
	\item[b] Se exige al deudor que tenga un acuerdo con el FMI para garantizar la condicionalidad, pero en muchos casos se han renegociado deudas en este foro en ausencia de tal acuerdo.
	\item[c] La cláusula de ``comparabilidad de trato'' exige que todos los miembros del Club apliquen las mismas condiciones financieras a las deudas renegociadas.
	\item[d] El tratamiento ``Nápoles'' se palica a los países de renta baja, y permite la condonación de hasta el 67\% de los vencimientos de deuda objeto de renegociación.
\end{itemize}

\seccion{Test 2004}

\textbf{39.} La iniciativa Países Pobres Muy Endeudados (PPME) es:
\begin{itemize}
	\item[a] Una iniciativa de alivio de deuda para los países de ingreso mediano miembros del Banco Mundial.
	\item[b] Una iniciativa para otorgar un trato comercial preferente a los Países Menos Adelantados.
	\item[c] Una iniciativa de alivio de deuda para algunos países en desarrollo de ingreso bajo.
	\item[d] Una iniciativa de alivio de deuda para los países del África subsahariana.
\end{itemize}

\textbf{40.} Los cambios registrados en 1990-97 en la estructura de los flujos de capital privado en dirección a los países en desarrollo han consistido en:
\begin{itemize}
	\item[a] Una caída de la proporción de la inversión directa, como consecuencia del aumento de la proporción de los préstamos bancarios.
	\item[b] Un aumento de la proporción de la inversión directa y de la inversión en cartera.
	\item[c] Una caída de la proporción de la inversión directa y de los préstamos bancarios, en beneficio de la inversión en cartera.
	\item[d] Ninguna de las anteriores.
\end{itemize}

\seccion{9 de marzo de 2017}
\begin{itemize}
    \item ¿Qué efectos tiene para el deudor el cambio deuda-equity?
    \item ¿Qué es un fondo-buitre? ¿Qué es el Club de París?
    \item Se consideran ayudas al desarrollo: a) las transferencias destinadas a gasto militar; b) las trasferencias realizadas por ONGs?
    \item ¿Por qué los países no cumplen su compromiso del 0,7\% en ayuda al desarrollo?
    \item ¿En qué afecta la apertura de la economía al equilibrio en el mercado de fondos prestables?
    \item ¿Cuál es su opinión sobre la gestión de las ayudas al desarrollo por las ONGs españolas?
    \item ¿Qué paralelismos existen entre la crisis griega actual y la de México en los 80?
\end{itemize}

\seccion{21 de marzo de 2017}
\begin{itemize}
    \item ¿Qué iniciativas conoce en el ámbito comercial para ayudar al desarrollo?
    \item ¿Qué papel tienen los créditos a la exportación en relación a la financiación del desarrollo?
    \item ¿Cómo afectó la crisis de los años 80 al paso de un modelo de industrialización por sustitución de importaciones a un modelo basado en las exportaciones?
    \item La IDE es una fuente de financiación frente al llamado \textit{hot money}. ¿Cómo afectan a la disuasión del \textit{hot money} medidas como la Tasa Tobin, el encaje bancario o la ley vigente en Chile a este respecto?
    \item ¿Qué características debe cumplir una ayuda para ser considerada \comillas{ayuda oficial al desarrollo}?
    \item ¿La ayuda al sector militar sería ayuda oficial al desarrollo?
    \item ¿Corresponde al comité de ayuda al desarrollo definir el tipo de ayuda a ofrecer?
    \item ¿Por qué los países no cumplen el objetivo de 0,7\% de ayuda al desarrollo?
    \item ¿Cuál es la diferencia entre ayuda ligada y no ligada y qué relación tiene con el Consenso de la OCDE?
\end{itemize}

\seccion{4 de abril de 2017}
\begin{itemize}
    \item ¿Conoce las limitaciones para la ayuda ligada?
    \item ¿Solo influye el tipo de interes en la sostenibilidad de la deuda?
    \item ¿A donde deberia dirigirse la inversion?
    \item ¿Ha hablado del comite de ayuda al desarrollo de la OCDE. Le corresponde el desarrollo de la AOD?
    \item ¿La AOD va con un 25\% de liberalidad, también al sector militar?
    \item ¿Por qué los países no cumplen con el objetivo del 7\% (AOD)? ¿Es vinculante?
    \item ¿En qué consisten los programas de conversión de deuda?
\end{itemize}

\notas

\url{http://www.mineco.gob.es/portal/site/mineco/menuitem.b6c80362d9873d0a91b0240e026041a0/?vgnextoid=32977cb59784c310VgnVCM1000001d04140aRCRD}

\textbf{2018:} \textbf{36.} B

\textbf{2015:} \textbf{40.} Anulada

\textbf{2006:} \textbf{36.} D

\textbf{2004:} \textbf{39.} C \textbf{40.} B

\bibliografia

Mirar en Palgrave:

\begin{itemize}
	\item access to land and economic development
	\item agriculture and economic development
	\item burden of the debt
	\item city and economic development
	\item dependency
	\item development economics
	\item dual economics
	\item emerging markets
	\item endogenous growth theory
	\item extreme poverty
	\item factor misallocation and development
	\item \textbf{financial structure and economic development}
	\item fiscal and monetary policies in developing countries
	\item flypaper effect
	\item \textbf{foreign aid}
	\item globalization
	\item growth and institutions
	\item growth take-offs
	\item Indian economic development
	\item international indebtedness
	\item labour surplus economies
	\item Latin American economic development
	\item Lewis, W. Arthur
	\item linkages
	\item microcredit
	\item national debt
	\item nutrition and development
	\item poverty
	\item \textbf{poverty alleviation programmes}
	\item poverty lines
	\item poverty traps
	\item public debt
	\item regional development
	\item regional development, geography of
	\item religion and economic development
	\item research and experimental development (RandD) and technological innovation policy
	\item sovereign debt
	\item \textbf{terms of trade and economic development}
	\item taxation and poverty
	\item \textbf{Third World Debt}
	\item trade and poverty
	\item uneven development
	\item \textbf{Washington Consensus}
	\item \textbf{World Bank}
\end{itemize}


Easterly, W. (2003) \textit{Can Foreign Aid Buy Growth?} Journal of Economic Perspectives. Vol. 17. Number 3. Summer 2003 -- En carpeta del tema

Mosley, P. (1986) \textit{Aid-effectiveness: The Micro-Macro Paradox} IDS Bulletin vol. 17 no. 2 -- En carpeta del tema

Page, L.; Pande, R. (2018) \textit{Ending Global Poverty: Why Money Isn't Enough} Journal of Economic Perspectives. Vol. 32. Number 4 -- En carpeta del tema

Pilbeam, K. \textit{International Finance} (2006) 3rd Edition -- En carpeta Economía Internacional

Shleifer, A. (2009) \textit{Peter Bauer and the Failure of Foreign Aid} Conference in memory of Peter Bauer at the London School of Economics in 2006 -- En carpeta del tema

\end{document}
