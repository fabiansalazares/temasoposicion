
\documentclass{nuevotema}

\tema{3A-6}
\titulo{Teoría neoclásica de la demanda del consumidor. Otros desarrollos de la teoría de la demanda, en especial, la teoría de la preferencia revelada y la teoría de la demanda de características.}

\begin{document}

\ideaclave

Uno de los objetivos fundamentales de la economía --mas concretamente, de la microeconomía- es explicar y predecir el comportamiento de los agentes. Si nos centramos en los consumidores como agentes cuyo comportamiento debe ser descrito, sus decisiones pueden expresarse en términos de sus preferencias, de los precios de los bienes entre los que deben decidir, y de la renta total. Es decir, pueden definirse en términos de una relación de preferencias sobre un conjunto de decisión, abstrayéndonos del hecho de que la cantidad consumida de los bienes pueda expresarse en términos de números reales. El marco de análisis neoclásico establece una serie de relaciones entre las propiedades de la relación de preferencias y una función de utilidad que permite trasladar esas preferencias a la recta real, de tal manera que sea posible derivar una función de demanda de bienes en función de la renta y los precios. Este tema consiste en exponer esas relaciones, propiedades y resultados entre preferencias, utilidad, maximización de ésta y función de demanda, así como de plantear una serie de extensiones, variaciones, o puntos de vista alternativos a la teoría neoclásica de la demanda del consumidor, tales como la teoría de las preferencias reveladas, las aportaciones de Gary Becker, el enfoque cardinalista o la demanda de características.

La \marcar{teoría neoclásica de la demanda} parte del concepto de relación de preferencias. Este objeto matemático no es sino una ordenación de los elementos de un conjunto matemático que expresa las opciones disponibles para el agente decisor, denominado conjunto de decisión. En base a esta ordenación, se puede derivar una relación entre unos precios y una renta, por un lado, y una demanda de elementos del conjunto de decisión por otro. Es decir, se puede expresar la decisión del agente en cuanto a qué demandar en función de los precios relativos de cada decisión, y de una renta que caracteriza su restricción presupuestaria --es decir, el conjunto de decisiones que están a su alcance. 

Dada la dificultad práctica de derivar esa función general que exprese la demanda a partir de una relación de preferencias cuando el conjunto de decisión es un espacio real de dimensión $n$, resulta útil derivar una función de utilidad que exprese la ordenación implícita a la relación de preferencias como posiciones en el espacio real. Para que exista tal función de utilidad, la relación de preferencias debe satisfacer tres axiomas: completitud, transitividad y continuidad. Además, si la relación de preferencias satisface también otros axiomas añadidos, la función de utilidad resultante mostrará determinadas características adicionales. Entre ellos, la convexidad de la relación de preferencias, monotonía, no saturación, homoteticidad o cuasilinealidad, que implican respectivamente, cuasiconcavidad, no decrecimiento, curvas de indiferencia de grosor infinitesimal, homogeneidad, o cuasilinealidad de la función de utilidad. Además, gracias a este análisis a partir de una función de utilidad, es posible aplicar un análisis de estática comparativa para obtener relaciones entre cambios en la renta y los precios y cantidades demandadas de los bienes, así como la función de utilidad indirecta, que expresa la utilidad obtenida cuando se demandan cantidades óptimas dados unos precios y una renta.

El marco anterior aporta importantes predicciones respecto al comportamiento de los consumidores con diferentes ingresos a la hora de demandar bienes con precios variables. Sin embargo, adolece de ciertas limitaciones. Una de ellas, es que en la práctica resulta casi siempre imposible conocer las preferencias \comillas{verdaderas} del individuo. A menudo, todo lo que puede llegar a conocerse es la función de demanda por vía de sus decisiones observadas: dados unos precios y una renta, el consumidor decide consumir tales cantidades de los bienes, pero no sabemos en base a qué relación de preferencias, ni las características de ésta. Puede ser racional o no, convexa o no, etc... Por ello, resulta útil establecer algunos requisitos respecto a la función de demanda observada que impliquen la existencia de una relación de preferencias racional que induzca a su vez la función de demanda en cuestión. El enfoque de preferencias reveladas aborda este problema: si la demanda observada satisface el axioma WARP cuando el consumidor elige entre dos bienes, existirá una relación de preferencias racional que induzca tal función de demanda observada. Si elige entre más de dos bienes, deberá satisfacer el axioma SARP para que exista tal relación de preferencias. Si además, la función de demanda no es tal, sino que es una correspondencia entre el espacio de precios y renta y el conjunto presupuestario o de decisión, deberá satisfacer el axioma GARP. El enfoque de preferencias reveladas está conectado con diferentes facetas del análisis de la demanda de los agentes, incluido el problema de la agregación que examinamos a continuación.

El marco básico de la teoría neoclásica de la demanda describe el comportamiento de consumidores individuales. Pero en la mayoría de los casos, el economista se enfrenta al problema de describir y predecir el comportamiento de economías compuestas por miles o millones de agentes que toman decisiones individuales e interdependientes. Estas economías inducen también funciones de demanda en cuanto a que implican relaciones entre precios, rentas y demandas. En este punto surgen varias preguntas respecto a estas funciones de demanda agregada: ¿estas funciones pueden describirse en base a los preceptos de la teoría neoclásica de la demanda? ¿que requisitos deben cumplir las funciones de utilidad de agentes individuales para que la suma de los efectos de sus decisiones sea igual al efecto de la suma de sus rentas individuales y los precios? ¿las funciones de demanda agregada observada pueden ser racionalizadas como si se tratasen de funciones de demanda de un sólo consumidor que optimiza sus preferencias? ¿puede realizarse un análisis de bienestar de las funciones de demanda agregada observadas? ¿existe un consumidor representativo cuya optimización de las preferencias induzca el mismo resultado normativo que la optimización individual? ¿tiene sentido considerar las  demandas agregadas como funciones de demanda? La teoría de la agregación, de gran complejidad y que representa un área de investigación en sí misma, se examina brevemente en relación a la teoría de la demanda. Para ello, se introduce el concepto de la función de utilidad indirecta con forma polar de Gorman como requisito para que la función de demanda agregada pueda ser objeto de determinados análisis coherentes con la teoría neoclásica de la demanda. 

La teoría de la demanda de características propuesta por Lancaster supone una modificación relativamente simple del marco neoclásico estándar, pero que sin embargo tiene grandes repercusiones en determinadas áreas de la economía, especialmente en cuanto a aplicaciones empíricas. El modelo de Lancaster plantea las preferencias del consumidor en términos de características, cuya cuantía depende a su vez de los bienes demandados. Cada bien aporta una serie de características en proporciones variables y posiblemente relacionadas con otros bienes, de tal forma que éstos se convierten en una suerte de inputs de una tecnología de producción de características, que son el verdadero objeto de decisión de los agentes. Este marco permite racionalizar el grado de sustituibilidad entre los diferentes bienes, algo que la teoría de la demanda neoclásica no derivaba explícitamente. Las aplicaciones de la teoría de la demanda de las características se han centrado en la construcción de índices de precios y modelos de demanda hedónica. Este marco es, entre otros muchos ejemplos posibles, el fundamento de modelos econométricos de precios inmobiliarios que utilizan variables como ubicación, número de habitaciones, densidad, transporte público, etc... como variables independientes.

Por último, se examina el enfoque cardinalista de la teoría de la demanda por razones históricas, así como la aportación de Gary Becker en los años 60. El enfoque cardinalista, si bien superado por la irrelevancia de la utilidad como concepto mesurable a la hora de derivar funciones de demanda, fue el precursor del modelo neoclásico y por ende, de toda la microeconomía moderna, y conserva cierta relevancia por la aparición relativamente reciente de enfoques neocardinalistas. Sin embargo, es necesario remarcar que el gran resultado de la teoría de la demanda desarrollada en la primera mitad de siglo XX es, precisamente, lo innecesario de la cuantificación de la utilidad para derivar funciones de demanda a partir de renta y precios. 

El enfoque de Gary Becker es en gran medida resultado de las circunstancias del autor. En la tradición de la escuela de Chicago, Becker creía en la necesidad de fundamentar los hechos empíricos en principios teóricos, y viceversa, y dos artículos en los años 60 abrieron una nueva línea de investigación de gran importancia en el análisis de políticas posterior: la explicación del comportamiento familiar en el marco de la teoría neoclásica. Determinados estudios de sección cruzada mostraban una relación inversa entre renta y número de hijos, algo que se antojaba contradictorio a la intuición de que a más renta, mayor cantidad de descendencia resulta deseable. Gary Becker explica esta paradoja  modificando el marco de análisis neoclásico. En su teoría modificada de la demanda del consumidor, los agentes no obtienen utilidad de los bienes de forma directa sino de "mercancías" producidas a partir de los bienes y del tiempo, que no es sino un input más entre toda la variedad de bienes consumibles que el agente puede adquirir. Así, aumentos en la renta inducen --\textit{ceteris paribus-} aumentos en la demanda de hijos. Es decir, el efecto de la renta sobre la demanda de hijos es positivo si mantenemos todos las demás variables constantes. Sin embargo, existe un efecto sustitución que puede compensarlo. Supongamos que el aumento de la renta se produjo por un aumento del salario (abandonando así el supuesto anterior de \textit{ceteris paribus}). Dado que aumentos en el salario implican un encarecimiento relativo del tiempo dedicado a los hijos y los hijos son un bien intensivo en tiempo, aparece un efecto sustitución entre trabajo y tiempo dedicado a los hijos que redunda en una menor disposición a tener descendencia a medida que aumenta el salario. Sin embargo, los hijos siguen siendo un bien normal, ya que aumentos \comillas{puros} de la renta aumentan, \textit{ceteris paribus}, la demanda de los consumidores respecto a los hijos.

\seccion{Preguntas clave}
\begin{itemize}
	\item ¿Qué es la teoría de la demanda?
	\item ¿En qué consiste el modelo neoclásico?
	\item ¿Qué predicciones realiza?
	\item ¿Para qué sirve?
	\item ¿Qué otros modelos existen?
\end{itemize}

\esquemacorto

\begin{esquema}[enumerate]
	\1[] \marcar{Introducción}
		\2 Contextualización
			\3 Recursos escasos
			\3 Microeconomía
			\3 Teoría de la demanda
		\2 Objeto
			\3 Qué es la teoría de la demanda
			\3 Qué es la teoría neoclásica de la demanda
			\3 Qué predice
			\3 Para qué sirve
			\3 Qué extensiones son relevantes
			\3 Qué otros enfoques existen
		\2 Estructura
			\3 Teoría neoclásica
			\3 Otros desarrollos
	\1 \marcar{Teoría neoclásica de la demanda}
		\2 Idea clave
			\3 Decisiones del consumidor
			\3 Términos reales
			\3 Mínima información
		\2 Formulación
			\3 Conjunto de elección
			\3 Relación de preferencia
			\3 Función de utilidad
			\3 Problema de optimización
		\2 Implicaciones
			\3 Función de demanda
			\3 Función de utilidad indirecta
			\3 Agregación de Cournot
			\3 Agregación de Engel
			\3 Curva de Engel
			\3 Senda de expansión / Curva de oferta-renta
			\3 Curva de demanda inversa
			\3 Curva de precio-consumo
		\2 Aplicaciones
			\3 Comportamiento del consumidor
			\3 Análisis de bienestar
			\3 Teoría del valor y precios relativos
	\1 \marcar{Otros desarrollos}
		\2 Preferencias reveladas
			\3 Idea clave
			\3 Definición de relación de preferencia revelada
			\3 WARP
			\3 SARP
			\3 Generalized Axiom of Revealed Preferences
			\3 Relación con integrabilidad
		\2 Demanda de características
			\3 Idea clave
			\3 Formulación
			\3 Aplicaciones
		\2 Agregación
			\3 Idea clave
			\3 Formulación
		\2 Cardinalismo
			\3 Idea clave
			\3 Valoración
		\2 Gary Becker
			\3 Idea clave
			\3 Resultados
		\2 Dualidad
	\1[] \marcar{Conclusión}
		\2 Recapitulación
			\3 Teoría neoclásica de la demanda
			\3  Otros desarrollos
		\2 Idea final
			\3 Qué deciden los agentes
			\3 Impacto teoría económica
			\3 Nivel empírico

\end{esquema}

\esquemalargo












\begin{esquemal}
	\1[] \marcar{Introducción}
		\2 Contextualización
			\3 Recursos escasos
				\4 Razón de ser de la economía
				\4 Decidir entre alternativas
			\3 Microeconomía
				\4 Decisión de agentes individuales
				\4 Representar mediante modelos
				\4 Diferentes áreas
			\3 Teoría de la demanda
				\4 Decisiones de consumo
				\4 Consumidores
		\2 Objeto
			\3 Qué es la teoría de la demanda
			\3 Qué es la teoría neoclásica de la demanda
			\3 Qué predice
			\3 Para qué sirve
			\3 Qué extensiones son relevantes
			\3 Qué otros enfoques existen
		\2 Estructura
			\3 Teoría neoclásica
				\4 Idea clave
				\4 Formulación
				\4 Implicaciones
				\4 Valoración
			\3 Otros desarrollos
				\4 Agregación
				\4 Cardinalismo
				\4 Preferencias reveladas
				\4 Demanda de características
				\4 Gary Becker
	\1 \marcar{Teoría neoclásica de la demanda}
		\2 Idea clave
			\3 Decisiones del consumidor
				\4 Predecir
				\4 Entender
				\4 ¿Cuánto consumen de cada bien?
				\4 ¿Qué compran?
			\3 Términos reales
				\4 Precios absolutos/nominales irrelevantes
			\3 Mínima información
				\4 Preferencias
				\4 Restricción presupuestaria
				\4 Precios
		\2 Formulación
			\3 Conjunto de elección
				\4 Todos los elementos entre los que decidir
				\4 Dominio de la relación de preferencia
				\4 Ej. Platos en un menú, artículos en un supermercado...
			\3 Relación de preferencia
				\4 \underline{Definición}
				\4 Correspondencia entre conjunto de elección:
				\4[] Dados $x,y \in X$: $x \succsim y \iff x \in \succsim \left (y \right)$
				\4 Entendible como expresión de preferencias
				\4[] Si $x\succsim y$: $x$ es al menos tan preferida como $y$.
				\4 \underline{Relaciones de preferencia derivadas}
				\4[$\succ$] Si $x \succsim y$ pero \underline{no} $y \succsim x$
				\4[$\sim$] Si $x \succsim y$ \underline{y} $y \succsim x$
				\4 \underline{Propiedades}
				\4[(i)] \textit{Completa}
				\4[] Si $\forall x, y \in X: x \succsim y$ o $y \succsim x$
				\4[(ii)] \textit{Transitiva}
				\4[] Si $\forall x, y, z \in X: x \succ y, y \succ z \Rightarrow x \succ z$
				\4[(iii)] \textit{Continua}
				\4[] Si $\forall \, x \in X$:
				\4[] Son cerrados: $\left\lbrace y \in X: y \succsim x \right\rbrace $ y $\left\lbrace y \in X: y \precsim x \right\rbrace $
				\4[(iv)] \textit{No saturada}
				\4[] Si $\forall x \in X, \epsilon > 0:$
				\4[] $\exists y / ||y - x|| \leq \epsilon, y \succ x$
				\4[(v)] \textit{Monótona}
				\4[] Si $x,y \in X, x \succ y$ si $x >> y$\footnote{Es decir, si los elementos de $x$ y $y$ son expresables en términos numéricos y todos los elementos de $x$ son estrictamente mayores que los elementos respectivos de $y$.}
				\4[(vi)] \textit{Estrictamente monótona}
				\4[] Si $x,y \in X, x \succ y$ si $x \geq y, \exists \, i \, / \, x_i > y_i$\footnote{Es decir, si los elementos de $x$ y $y$ son expresables en términos numéricos y al menos un elemento de $x$ es mayor que un elemento respectivo de $y$, pudiendo ser el resto iguales o tambien superiores los de $x$ a los de $y$.}
				\4[(vii)] \textit{Convexa}
				\4[] $\forall x,y \in X, \alpha \in (0,1)$: $x \succsim y \Rightarrow \alpha x + (1 - \alpha) y \succsim y $
				\4[(viii)] \textit{Estrictamente convexa}
				\4[] $\forall x,y \in X, \alpha \in (0,1)$: $x \succsim y  \Rightarrow \alpha x + (1 - \alpha) y \succ y$
				\4[(ix)] \textit{Homotética}
				\4[] $\forall x, y \in X$, $x \sim y \Rightarrow \alpha  x \sim \alpha y \quad \forall \alpha$
				\4[(x)] \textit{Cuasilineal en bien n}
				\4[] $\forall x, y \in X, e_1 = (1,0,...,0) \in X$, $\succsim$ cuasilineal con respecto a bien 1:
				\4[] Si $x \sim y \iff x + \alpha e_1 \sim  y + \alpha e_1$ $\forall \, \alpha$
			\3 Función de utilidad
				\4 \underline{Representación de preferencias}
				\4 Teorema de representación de Debreu (1954):
				\4[] Si$\succsim$ completa, transitiva y continua
				\4[] $\then$ $\exists \, u: X \to \mathbb{R}$ continua que la representa
				\4 $U: X \to \mathbb{R}$ representa $\succsim$
				\4[$\iff$] $\left( x \succsim y \iff u(x) \geq u(y) \right)$
				\4 \underline{Propiedades}
				\4[(i)] \textit{Completa}
				\4[] El dominio de $u(\cdot)$ es X
				\4[(ii)] \textit{Curvas de indiferencia no se cruzan}
				\4[] Si la $\succeq$ es transitiva
				\4[(iii)] \textit{Continua}
				\4[(iv)] \textit{Curvas de indiferencia sin grosor}
				\4[] Si las preferencias no se saturan
				\4[(v)] \textit{Creciente}
				\4[] Si $\succsim$ es monótona
				\4[(vi)] \textit{Estrictamente creciente}
				\4[] Si $\succsim$ es estrictamente monótona
				\4[(vii)] \textit{Cuasicóncava}
				\4[] Si $\succsim$ es convexa.
				\4[] $(x)$ es cuasicóncava en $[a,b]$ si:
				\4[] $u(x) \geq \min \left\lbrace u(a), u(b) \right\rbrace$
				\4[(viii)] \textit{Estrictamente cuasicóncava}
				\4[] Si $\succsim$ es estrictamente convexa.
				\4[] Curvas de indiferencia nunca rectas
				\4[(ix)] \textit{Homogénea}
				\4[] Si $\succsim$ homotéticas.
				\4[] $\exists$ una transformación homogénea de grado 1
				\4[] Misma pendiente de curvas de indiferencia en rayos vectores
				\4[(x)] \textit{Cuasilineal en bien n}
				\4[] Si $\succsim$ cuasilineal en bien n
				\4[] $u(\vec{x}) = x_n +  g(\vec{x}_{-n})$
			\3 Problema de optimización
				\4 Representación de preguntas:
				\4[] ¿Qué demandar?
				\4[] ¿Cuánto de cada bien demandar?
				\4 Respuesta:
				\4[] Óptimo del problema
				\4[] $\to$ Caracteriza comportamiento individual
				\4[] $\underset{\vec{x}}{\max} \quad u(\vec{x})$
				\4[] $s.a:\quad \vec{p} \cdot \vec{x} \geq w$
				\4 Resolución por Método de Lagrange (2 bienes)
				\4[] $\mathcal{L} = u(x,y) - \lambda (p_x x + p_y y - w)$
				\4[] $\frac{d \mathcal{L}}{dx} = \pdv{u}{x} - \lambda p_x = 0 $
				\4[$\Rightarrow$] CPO: \fbox{$|\text{RMS}_{xy}| = \frac{p_x}{p_y} $}
				\4[] $\frac{dy}{dx} \equiv \text{RMS}_{xy} \equiv \frac{- u_x}{u_y}$
				\4[$\Rightarrow$] $x = f(y,p_x, p_y)$
				\4[$\Rightarrow$] $y = f(x, p_x, p_y)$
				\4 \textit{Sol. esquina}:
				\4[] $\text{RMS}_{xy} \neq \frac{p_x}{p_y}$ $\forall \, x, y$
				\4[] Si $|\text{RMS}_{xy}| >  \frac{p_x}{p_y} \Rightarrow y=0$
				\4[] Si $|\text{RMS}_{xy}| < \frac{p_x}{p_y} \Rightarrow x=0$
				\4 Dibujar gráfica
		\2 Implicaciones
			\3 Función de demanda
				\4[] $x = f(y, p_x, p_y, w)$
				\4[] $y = f(x, p_x, p_y, w)$
				\4[$\Rightarrow$] $x = f(y(x, p_x, p_y), p_x, p_y, w) =$
				\4[] \fbox{$x = f(p_x, p_y, w)$}
				\4[$\Rightarrow$] $y = f(x(y, p_x, p_y), p_x, p_y, w) = $
				\4[] \fbox{$y = f(p_x, p_y, w)$}
				\4 Propiedades
				\4[(i)] \textit{Homogénea de grado 0 en $\left( \vec{p}, w \right)$}
				\4[(ii)] \textit{Cumple ley de Walras: $\vec{p} \cdot \vec{x} = w$}
				\4[(iii)] \textit{Conjunto convexo}
				\4[] Si $\succsim$ es convexa, $x(\vec{p},w)$ es conjunto convexo
				\4[(iii')] \textit{Unicidad}
				\4[] Si $\succsim$ es estrictamente convexa,
				\4[] $x(\vec{p},w)$ tiene un sólo elemento.
			\3 Función de utilidad indirecta
				\4[] $u (x, y) = u\left( x(p_x,p_y, w), y(p_x, p_y, w) \right)$
				\4[] $ =v(p_x, p_y, w)) = v(\vec{p}, w)$
				\4[] \fbox{$u(\vec{x}) = v (\vec{p}, w)$}
				\4 Propiedades
				\4[(i)] \textit{Homogénea de grado 0 en $\left( \vec{p}, w \right)$}
				\4[(ii)] \textit{Estrictamente creciente en w}
				\4[(iii)] \textit{No creciente en todo $p_i \in \vec{p}$}
				\4[(iv)] \textit{Cuasiconvexa}
				\4[] $\left\lbrace (\vec{p}, w): v( \vec{p},w) \leq \bar{v} \right\rbrace$ es convexo
				\4[(v)] \textit{Continua en $\vec{p}$ y $w$.}
			\3 Agregación de Cournot
				\4[] $\varDelta$ de gastos ante $\varDelta$ precios con renta constante
				\4[] $\varDelta$ de gasto total suma 0
				\4[] Ley de Walras: $\vec{p} \cdot \vec{x} = \bar{w}$
				\4[] $\frac{d(\vec{p} \vec{x})}{dp_i} = \frac{d \bar{w}}{dp_i}$
				\4[] $\Rightarrow$ $x_i +\sum_j p_j \frac{dx_j}{dp_i} = 0$
				\4[] $\Rightarrow$ $ \sum_j \frac{p_j}{w} \frac{dx_j}{dp_i}= -\frac{x_i}{w}$
				\4[] $\Rightarrow$ $ \sum_j \frac{p_j}{w} \frac{x_j}{p_i} \epsilon_{ij} = -\frac{x_i}{w}$
				\4[] $\Rightarrow$ $\sum_j s_j \epsilon_{ij} = -\frac{p_i x_i}{w}$
				\4[] $\Rightarrow$ \fbox{$\sum_j s_j \epsilon_{ij} = -s_i$}
			\3 Agregación de Engel
				\4[] $\varDelta$ de gasto ante $\varDelta$ de renta con precios constantes
				\4[] Elasticidades ponderadas suman 1
				\4[] Ley de Walras: $\vec{p} \cdot \vec{x} = w$
				\4[] $\frac{d(\vec{p} \vec{x})}{dw} = \frac{dw}{dw}$
				\4[] $\Rightarrow$ $\vec{p} D_w (\vec{x}) = 1$
				\4[] $\Rightarrow$ $\sum p_i \frac{dx_i}{dw} = 1$
				\4[] $\Rightarrow$ $\sum p_i \frac{x_i}{w} \eta_i = 1$
				\4[] $\Rightarrow$ \fbox{$\sum s_i \eta_i=1$}
				\4[] Suma de $\varDelta$ de gasto iguales a $\varDelta$ de renta.
			\3 Curva de Engel
				\4 Relación entre renta y demanda de bien $x$
				\4 Dibujada en espacio $x$--$w$
				\4[$\to$] $w = f(x)$
				\4 \underline{Bien normal}
				\4[] $\frac{dw}{dx} > 0$
				\4 \underline{Bien inferior}
				\4[] $\frac{dw}{dx} < 0$
			\3 Senda de expansión / Curva de oferta-renta
				\4 Dados precios fijos
				\4[] $x = f(y, w)$
				\4[] Dibujada en espacio $x$--$y$
			\3 Curva de demanda inversa
				\4[] Relación entre precio y demanda de bien $x$
				\4[] Dada renta fija
				\4[] Dibujada en espacio $x$--$p_x$
			\3 Curva de precio-consumo
				\4[] Dado $w$
				\4[] $x = f(y, p_x)$
				\4[] Dibujada en espacio $x$--$y$ (2 bienes)
		\2 Aplicaciones
			\3 Comportamiento del consumidor
				\4 Estático
				\4 Dinámico
				\4 Incertidumbre
			\3 Análisis de bienestar
			\3 Teoría del valor y precios relativos
	\1 \marcar{Otros desarrollos}
		\2 Preferencias reveladas
			\3 Idea clave
				\4 Dadas decisiones de demanda observadas
				\4[] ¿existe $\succsim$ racional que las genere?
			\3 Definición de relación de preferencia revelada
				\4 Dadas: cestas $x_1$ y $x_0$, y $\succsim^*$
				\4 Suponemos que:
				\4[] Dados precios $p_1$
				\4[] $\to$ el agente elige la cesta 1.
				\4[] Dados precios $p_0$
				\4[] $\to$ el agente elige la cesta $x_0$
				\4 En tal caso:
				\4: $x_1 \succsim^* x_0 \iff p_1 x_1 \geq p_1 x_0$
				\4 I.e. si pudiendo elegir $x_1$ y $x_0$ el agente elige $x_1$
				\4[$\then$] entonces $x_1$ se revela preferido a $x_0$
			\3 WARP
				\4 Un conjunto de decisiones sobre elementos de X\footnote{El cual puede ser una función de demanda si el conjunto de decisión es un subconjunto de $\mathbb{R}^n$.}
				\4[] satisface WARP $\iff$
				\4[] $\forall \, x_0 \neq x_1 \in X:$
				\4[] $p_1 x_ 1 \geq p_1 x_0 \then p_0 x_1 > p_0 x_0$
				\4 En palabras:
				\4[] Si $x_1$ se revela preferido a $x_0$,
				\4[] $x_0$ no se revela preferido a $x_1$
				\4 {Relación con preferencias}
				\4[] Existe f. de utilidad $\then$ WARP
				\4[] $\left( \text{WARP} \, + \, \text{número de bienes}=2 \then \exists \text{f. de utilidad} \right)$
			\3 SARP\footnote{O axioma de preferencia revelada de Houthakker.}
				\4 Un conjunto de decisiones sobre elementos de X\footnote{El cual puede ser una función de demanda si el conjunto de decisión es un subconjunto de $\mathbb{R}^n$.}
				\4 cumple SARP $\iff$
				\4[] $\nexists$ ciclos de cestas reveladas preferidas a otras
				\4[] I.e: $\forall (x_n, x_{n-1}, ..., x_1, x_0) \in X$ tal que:
				\4[] $x_n \succsim^* x_{n-1} \succsim^* x_{n-2} \succsim^* ... x_1 \succsim^* x_0$
				\4[] No se cumple que $x_0 \succsim^* x_n$
				\4 Existencia de preferencias racionales
				\4[] Demanda observada cumple SARP $\iff$
				\4[] Existe $\succsim$ racional
				\4 Dos dimensiones (dos bienes)
				\4[] SARP $\iff$ WARP
				\4 Más de dos dimensiones
				\4[] WARP no tiene por qué implicar SARP
				\4[] Si no se cumple SARP
				\4[] $\to$ No hay $\succsim$ racional
				\4[] $\then$ SARP es cond. nec. y suf
				\4[] $\then$ WARP sólo caso particular
			\3 Generalized Axiom of Revealed Preferences
				\4 Similar a SARP
				\4 Aplicable si dda. observada es correspondencia\footnote{Algo que sucede cuando la función de utilidad no es estrictamente cuasicóncava, o equivalentemente, cuando la relación de preferencias no es estrictamente convexa.}
				\4 Misma formulación que SARP
				\4[] $\to$ pero si al menos un $\succ^*$ en vez de $\succsim^*$
			\3 Relación con integrabilidad
				\4 SARP $\iff$ Matriz de Slutsky semidef. negativa simétrica
		\2 Demanda de características\footnote{Lancaster, K. \textit{A new approach to consumer theory}. 1966. Journal of Political Economy.}
			\3 Idea clave
				\4 Agentes demandan características, no bienes
				\4 Bienes son inputs que producen características
			\3 Formulación
				\4 $\max \quad u(\vec{a})$
				\4 $s.a: \quad \vec{a} =  \vec{F}(\vec{x})$
				\4 $\quad \quad \quad \vec{p} \cdot \vec{x} \leq w$
				\4 \textbf{Casos particulares}
				\4[(i)] Sin combinaciones posibles
				\4[$\to$] Restricción presupuestaria:
				\4[] $p_i x_i \leq w$ o $p_j x_j \leq w$ o $p_k x_k \leq w$ o ...
				\4[] \grafica{sincombinaciones}
				\4[(ii)] Combinaciones posibles sin sinergias
				\4[] \grafica{combinacionessinsinergias}
				\4[(iii)] Combinaciones posibles con sinergias
				\4[] \grafica{combinacionesconsinergias}
				\4[(iv)] Modelo tradicional de demanda del consumidor
				\4[$\to$] Mismo número de bienes y características
				\4[$\to$] Cada bien aporta una sóla característica
				\4[$\to$] Sin sinergias
			\3 Aplicaciones
				\4 Modelos de demanda hedónica
				\4 Índices de precios hedónicos
				\4 Fundamenta sustituibilidad en características similares
		\2 Agregación
			\3 Idea clave
				\4 Anteriormente:
				\4[] $\to$ un consumidor
				\4 ¿Demanda agregada de acuerdo a teoría neoclásica anterior?
				\4 ¿Cuándo DA es $=f\left( \vec{p}, \sum_i w_i \right)$
				\4[] $\to$ Sin depender de distribución de $w_i$
				\4 ¿Cuándo existe consumidor representativo racional de la DA?
				\4[] $\to$ ¿Posible análisis agregado de bienestar?
			\3 Formulación
				\4 Función indirecta de utilidad FPG\footnote{Forma Polar de Gorman.}
				\4 Supuesto: cada consumidor $i$ tiene:
				\4[] $v_i^{\text{FPG}}(\vec{p},w_i) = a_i(\vec{p}) + b(\vec{p}) w_i$
				\4 \textbf{Propiedades:}
				\4[(i)] Efecto renta constante
				\4[(ii)] F. de u. homotética
				\4[(iii)] FPG con $w= \sum_i w_i$
				\4[] $\then$ Consumidor representativo
		\2 Cardinalismo
			\3 Idea clave
				\4 La utilidad se puede medir y comparar
				\4[$\Rightarrow$] $U(\cdot)$ definida hasta trans. afines positivas
				\4 Consumo de bienes hasta igual $\text{UMg}_i$
			\3 Valoración
				\4 Implica supuestos muy restrictivos
				\4 Utilidad cuasilineal en dinero o aditiva
				\4 No necesarios para derivar función de demanda
				\4 Interesante para comparar bienestar
		\2 Gary Becker
			\3 Idea clave
				\4 Bienes y tiempo como inputs
				\4 Aplicación teoría neoclásica a decisiones familiares
			\3 Resultados
				\4 ¿Son los hijos bienes normales?
				\4 Evidencia sección cruzada entre países:
				\4[] Son bienes inferiores $\to$ Más renta $\Rightarrow$ menos hijos
				\4 Becker: son bienes normales
				\4[] Aplicación teoría neoclásica con tiempo como input
				\4[] Hijos son bienes normales
				\4[] Pero cuestan mucho tiempo
				\4[] Subidas de sueldos/precio del trabajo
				\4[$\Rightarrow$] tiempo usado en hijos es más caro
		\2 Dualidad
	\1[] \marcar{Conclusión}
		\2 Recapitulación
			\3 Teoría neoclásica de la demanda
				\4 Idea clave
				\4 Definiciones
				\4 Problema de optimización
				\4 Función de demanda
				\4 Función de utilidad indirecta
				\4 Estática comparativa
			\3  Otros desarrollos
				\4 Agregación
				\4 Cardinalismo
				\4 Preferencias reveladas
				\4 Demanda de características
				\4 Gary Becker
		\2 Idea final
			\3 Qué deciden los agentes
				\4 Pregunta fundamental de la microeconomía
				\4 Teoría de la demanda: marco fundamental respecto a consumidores
				\4 Modelo neoclásico: cumple requisitos buenos modelos
				\4[$\to$] Robustez
				\4[$\to$] Gran validez externa
				\4[$\to$] Mínima información: preferencias, precios, renta
			\3 Impacto teoría económica
				\4 Enorme
				\4[$\to$] Casi todas las áreas reciben influencia
				\4[$\to$] O se basan en teoría de la demanda
				\4[] Desde macroeconomía
				\4[] Microeconomía aplicada
				\4[] Economía normativa
				\4[] Economía experimental...
			\3 Nivel empírico
				\4 Revolución de los datos
				\4 Estimación funciones de demanda
				\4 Demanda tradicional + prefs. reveladas + características
\end{esquemal}

\graficas

\begin{axis}{4}{Decisión del consumidor respecto a características sin posibilidad de combinar bienes.}{$a_1$}{$a_2$}{sincombinaciones}
	\draw[-] (0,0) -- (1,4);
	\draw[-] (0,0) -- (4,3);
	
	\draw[-] (1.23, 4) to [out=290, in=160](4.63,1);
	\draw[-] (0.4, 3.17) to [out=290, in=160](3.8,0.17);
	
	\node[circle,fill=black,inner sep=0pt,minimum size=4pt] (a) at (0.5,2) {};
	\node[circle,fill=black,inner sep=0pt,minimum size=4pt] (b) at (1.7,1.25) {};
	
\end{axis}

\begin{axis}{4}{Decisión del consumidor respecto a características con posibilidad de combinar bienes que no producen sinergias.}{$a_1$}{$a_2$}{combinacionessinsinergias}
	\draw[-] (0,0) -- (1,4);
	\draw[-] (0,0) -- (4,3);
	
	\draw[-] (1.13, 3.9) to [out=290, in=175](4.83,2);
	\draw[-] (0.1, 2.87) to [out=290, in=175](3.8,0.97);
	
	\node[circle,fill=black,inner sep=0pt,minimum size=4pt] (a) at (0.5,2) {};
	\node[circle,fill=black,inner sep=0pt,minimum size=4pt] (b) at (1.7,1.25) {};
	\draw[-] (0.5,2) -- (1.7,1.25);
\end{axis}

\begin{axis}{4}{Decisión del consumidor respecto a características con posiblidad de combinar bienes que producen sinergias}{$a_1$}{$a_2$}{combinacionesconsinergias}
	\draw[-] (0,0) -- (1,4);
	\draw[-] (0,0) -- (4,3);	
	
	\node[circle,fill=black,inner sep=0pt,minimum size=4pt] (a) at (0.5,2) {};
	\node[circle,fill=black,inner sep=0pt,minimum size=4pt] (b) at (1.7,1.25) {};
	
	\draw[-] (0.5,2) to [out=-10, in=120](1.7, 1.25);
	
	\draw[-] (0.1, 3.08) to [out=290, in=175](3.8,1.18);
\end{axis}


\conceptos

\concepto{Agregación de Cournot}

De forma similar a la agregación de Engel, la agregación de Cournot parte de la Ley de Walras o de la condición de equilibrio presupuestario. Sin embargo, en este caso se trata de mostrar lo que necesariamente sucede con el gasto cuando el precio de un bien varía manteniendo la renta constante. La agregación de Cournot puede expresarse en términos absolutos o de elasticidades de forma similar a la de Engel:

\begin{ecuacion}
    \sum_{i=1}^N p_i \pdv{x_i}{p_j} + x_j = 0 \Rightarrow \boxed{\sum_{i=1}^N s_i \cdot \epsilon_{x_i\text{--}p_j} = - s_j}
\end{ecuacion}

La agregación de Cournot muestra que dado una renta total constante, las variaciones del gasto en los diferentes bienes ante la variación del precio de un bien deben compensarse de tal manera que su suma sea igual a cero. 

\concepto{Agregación de Engel}

La llamada agregación de Engel es un resultado derivado de la Ley de Walras o condición de equilibrio presupuestario. Si la Ley de Walras impone la igualdad entre el valor de la oferta y el valor de la demanda, la condición de equilibrio presupuestario se utiliza habitualmente para imponer $\vec{p} \cdot \vec{x} = y$, siendo $y$ la renta disponible (o más generalmente, la oferta). Dada esta condición, la agregación de Engel resulta de derivar la condición anterior respecto a la renta disponible, mostrando lo que necesariamente debe suceder con la demanda ante una variación infinitesimal de la oferta:

\begin{ecuacion}
	\boxed{\sum_{i=1}^N s_i \cdot \eta_i = 1} = \sum_{i=1}^N p_i \cdot \pdv{x_i}{y} = \sum_{i=1}^N \underbrace{p_i \frac{x_i}{y}}_{s_i} \eta_i = 1 
\end{ecuacion}

La agregación de Engel puede expresarse tanto en términos de gasto absoluto como en términos de elasticidades y ponderaciones del gasto. Desde el punto de vista absoluto, afirma que la suma de los aumentos del gasto ante variaciones infinitesimales de la renta deberá ser igual a la unidad, que no es sino la relación entre el aumento de la renta y aumentos infinitesimales de la renta. Desde el punto de vista de las elasticidades, la expresión muestra como la suma de las elasticidades demanda-renta ponderadas por peso de la demanda de cada bien sobre el gasto total, debe ser igual a 1. O equivalentemente, que el gasto total debe aumentar tanto como aumente la renta.

\concepto{Cuasilinealidad de las preferencias}

Una relación de preferencias $\succsim$ es cuasilineal en el bien 1 si dados $\forall x, y \in X, \; e_1 = (1,0,...,0) \in X$, se cumple que: $x \sim y \iff x + \alpha e_1 \sim y + \alpha e_1 \forall \, \alpha \in \mathbb{R}$. Cuando esto sucede, y la $\succsim$ es además completa, transitiva y continua de manera que existe una función de utilidad $u: X \to \mathbb{R}$ que la representa, la función de utilidad es lineal en el bien 1, de tal manera que toma la forma $u(\vec{x}) =  g(\vec{x}_{-1}) + x_1$. Una función de utilidad con esta forma induce curvas de indiferencia que se pueden desplazar paralelas al eje del bien 1 simplemente aumentando la cuantía consumida de este bien.

En funciones de utilidad con esta característica de linealidad en el bien 1, si el bien 1 es el numerario, de manera que la renta está expresada en términos del bien 1, se cumple que los bienes no sufren efecto renta. Es decir, la demanda del resto de los bienes no depende de la renta siempre que la renta sea suficiente para adquirir la cantidad demandada total del resto de los bienes. Este resultado tiene implicaciones a la hora de calcular el excedente del consumidor, ya que en este caso, el excedente marshalliano sí será una medida válida del excedente porque coincidirá exactamente con la variación compensatoria y la variación equivalente. Asimismo, la ausencia de efectos renta tiene especial utilidad en los modelos de equilibrio parcial, ya que la demanda del bien no numerario no depende ni de la renta global, ni del consumo de numerario y es exclusivamente función del precio del bien en cuestión. Del problema de maximización de la utilidad:

\begin{align*}
	 \max \quad u(x,m) = \phi(x) + m & \\
	 s.a: \quad p \cdot x + m \leq w & \\
\end{align*}

Del lagrangiano:

\begin{align*}
	\mathcal{L} = \phi(x) + m - \lambda ( p \cdot x + m - w)
\end{align*}

Asumiendo que $\phi'(x) <0$, $\phi''(x) < 0$, derivamos las condiciones de primer orden:

\begin{align*}
	\text{CPO:} \quad & \pdv{\mathcal{L}}{x} = \phi ' (x) - \lambda p = 0 \\
	& \pdv{\mathcal{L}}{m} = 1 - \lambda = 0 \\
	& \pdv{\mathcal{L}}{m} = p \cdot x + m = 0 \\ \\
	& \Rightarrow \phi'(x^*) = p \Rightarrow x^* = f(p) \Rightarrow \frac{d \, x}{d \, m} = 0 \\ 
\end{align*}


\concepto{Función de utilidad de Cobb-Douglas}

Cuando la función de utilidad toma la forma $u(x,y) =  k x^\alpha \cdot y^{1 - \alpha}$, se cumple que el gasto en cada bien será una fracción constante de la renta $\alpha$ y $1-\alpha$, respectivamente. Este resultado está relacionado con la familia de funciones de elasticidad de sustitución constante a la que la función Cobb-Douglas pertenece.

\concepto{Ley de la Demanda}

Característica de algunas funciones de demanda en las cuales se cumple que --\textit{ceteris paribus}- si el precio del bien en consideración aumenta, disminuye la cantidad demandada.

\concepto{Modelo de demanda de características de Lancaster}

En este modelo, los consumidores no deciden entre bienes, sino entre características que los bienes poseen en proporciones variables. Así, la función de utilidad del consumidor no es una función de los bienes disponibles, sino de las características. La restricción presupuestaria, sin embargo, sigue expresándose en términos de los bienes consumidos, ya que las características no pueden adquirirse directamente. Esta situación puede formularse de forma general introduciendo una restricción adicional que relacione las cantidades de características obtenidas dada una cesta de bienes adquiridos:


\begin{align*}
	\max \quad u(\vec{a}) \\
	\quad s.a: &\vec{a} = \vec{F} (\vec{x}) \\
	&\vec{p} \cdot \vec{x} \leq w
\end{align*}


Si asumimos que el agente sólo puede gastar toda su renta en adquirir la máxima cantidad posible de un sólo bien, la representación gráfica de la demanda óptima será la intersección entre la curva de indiferencia más alta con alguno de los dos rayos vectores que expresan las características aportadas por cantidades variables de cada bien:

\begin{axis}{4}{Decisión del consumidor respecto a características sin posibilidad de combinar bienes.}{$a_1$}{$a_2$}{sincombinaciones}
	\draw[-] (0,0) -- (1,4);
	\draw[-] (0,0) -- (4,3);
	
	\draw[-] (1.23, 4) to [out=290, in=160](4.63,1);
	\draw[-] (0.4, 3.17) to [out=290, in=160](3.8,0.17);
	
	\node[circle,fill=black,inner sep=0pt,minimum size=4pt] (a) at (0.5,2) {};
	\node[circle,fill=black,inner sep=0pt,minimum size=4pt] (b) at (1.7,1.25) {};
	
\end{axis}

Cuando el consumidor puede adquirir cantidades positivas de ambos bienes, el conjunto de posibles características a consumir se convierte en una línea recta que une las dos cestas de características posibles cuandos sólo se permite adquirir un sólo bien:

\begin{axis}{4}{Decisión del consumidor respecto a características con posibilidad de combinar bienes que no producen sinergias.}{$a_1$}{$a_2$}{combinacionessinsinergias}
	\draw[-] (0,0) -- (1,4);
	\draw[-] (0,0) -- (4,3);
	
	\draw[-] (1.13, 3.9) to [out=290, in=175](4.83,2);
	\draw[-] (0.1, 2.87) to [out=290, in=175](3.8,0.97);
	
	\node[circle,fill=black,inner sep=0pt,minimum size=4pt] (a) at (0.5,2) {};
	\node[circle,fill=black,inner sep=0pt,minimum size=4pt] (b) at (1.7,1.25) {};
	\draw[-] (0.5,2) -- (1.7,1.25);
\end{axis}

Existe aún otra situación posible. En determinadas circunstancias, ciertos bienes presentan sinergias a la hora de dar lugar a características, de manera que dos bienes aporten más cantidad de una serie de características cuando se consumen juntos, que cuando lo hacen por separado. Como se puede apreciar, en este tipo de situaciones la recta de balance no es una línea recta:

\begin{axis}{4}{Decisión del consumidor respecto a características con posiblidad de combinar bienes que producen sinergias}{$a_1$}{$a_2$}{combinacionesconsinergias}
	\draw[-] (0,0) -- (1,4);
	\draw[-] (0,0) -- (4,3);	
	
	\node[circle,fill=black,inner sep=0pt,minimum size=4pt] (a) at (0.5,2) {};
	\node[circle,fill=black,inner sep=0pt,minimum size=4pt] (b) at (1.7,1.25) {};
	
	\draw[-] (0.5,2) to [out=-10, in=120](1.7, 1.25);
	
    \draw[-] (0.1, 3.08) to [out=290, in=175](3.8,1.18);
\end{axis}

\concepto{Monotonicidad y monotonicidad estricta}: (siguiendo a MWG) una relación de preferencias $\succsim$ es \textit{monótona} si dados $x, y \in X = \mathbb{R_+^n}$, $x >> y$ implica $x \succ y$. Es decir, es monótona si dados dos vectores $x$ e $y$ tales que $x$ ofrece más cantidad de todos los bienes que $y$, $x$ será estrictamente preferido a $y$ o $x \succ y$.

Por otro lado, una relación de preferencias $\succeq$ será estrictamente monótona si basta con que $x$ ofrezca una cantidad mayor de al menos un bien en relación a $y$ para que $x$ sea estrictamente preferido a $y$ o $x \succ y$.

Monotonicidad implica necesariamente monotonicidad fuerte, pero al revés. Cuando las preferencias son monotónicas, el consumidor necesita más de todos los bienes que componen la cesta para preferir una frente a otra. Cuando las preferencias son estrictamente monotónicas, le basta con más cantidad de alguno de los bienes que componen una cesta, aunque el resto sean iguales, para preferir a una cesta frente a otra. Así, en un contexto de preferencias con monotonicidad débil en el que $x$ es preferido a $y$, se cumplirá necesariamente que la cantidad de cada bien en $x$ es mayor que la cantidad respectiva en $y$. En esa situación, se cumple también en todo caso la condición necesaria para que con monotonicidad fuerte $x$ sea preferido a $y$.

\concepto{Relación entre preferencia revelada e integrabilidad} Kihlstrom, Mas-Colell, Sonnenschein (1974): <<\textit{In this paper we provide a statement of the relationship between the weak axiom of revealed preference (WA) and the negative semidefiniteness of the matrix of substitution terms (NSD). As a corollary we determine the relation between WA and the strong axiom of revealed preference (SA). The latter is equivalent to NSD and the symmetry of the matrix of substitution terms. The former, WA, implies NSD but is not implied by NSD. Also, WA is implied by the condition that the matrix of substitution terms is negative definite (ND), but it does not imply ND. Application of these results yield and infinity of demand functions which satisfy WA but not SA.}>>

\preguntas

\seccion{Test 2018}
\textbf{3.} Sea un consumidor cuyas preferencias se representan por la función de utilidad $U(x_1, x_2) = x_1 + x_2^{1/2}$. Si en el equilibrio $x_i > 0 (i=1,2)$, ante una subida del precio de ambos bienes del 5\%, para este consumidor en el nuevo equilibrio se observa que:

\begin{itemize}
	\item[a] La cantidad demandada de $x_1$ no varía.
	\item[b] La cantidad demandada de $x_2$ no varía.
	\item[c] Se reduce la cantidad demandada de ambos bienes un 5\%.
	\item[d] No varía la cantidad demandada de ninguno de los bienes. 
\end{itemize}

\seccion{Test 2017}
\textbf{3.} Observamos que un consumidor ha comprado este mes unas determinadas cantidades $(x_1^0, x_2^0)$ de dos bienes, gastando toda su renta mensual $m$ para maximizar su utilidad. Supongamos que el bien 2 es más caro que el bien 1 y que los precios de ambos bienes mantienen siempre una ratio relativa constante entre sí, siendo $p_2=2\cdot p_1$. Si, al mes siguiente, la renta del consumidor no cambia, pero el precio del bien 1 se reduce a la mitad, ¿cuál de estas afirmaciones sobre su nueva cesta de consumo $(x_1^1, x_2^1)$ es \underline{verdadera}?

\begin{itemize}
	\item[a] Las cantidades compradas de los dos bienes habrán aumentado, pero la cantidad $x_1$ seguirá duplicando la de $x_2$, es decir, $x^1_1 = 2 \cdot x_2^1$
	\item[b] La cantidad comprada de al menos uno de los dos bienes será mayor que en el mes anterior.
	\item[c] Las cantidades compradas de los dos bienes se habrán reducido, por tanto: $x_1^1 < x_1^0$, $x_2^1 < x_2^0$.
	\item[d] Las cantidades compradas de los dos bienes son exactamente las mismas que el mes anterior, $x_1^1 = x_1^0$, $x_2^1 = x_2^0$, ya que el cambio de precios deja la recta de balance inalterada.
\end{itemize}

\textbf{4.} Marta tiene la siguiente función de utilidad del consumo: $U(x_1,x_2) = \min (x_1, 3\cdot x_2)$. Si la renta de Marta es $m=40$ y los precios de ambos bienes son iguales, $p_1=p_2=1$, ¿cómo son las curvas de indiferencia de Marta y cuál es la cesta de consumo que maximiza su utilidad?

\begin{itemize}
	\item[a] Son líneas horizontales paralelas, por lo que la utilidad se maximiza en la solución de esquina $x_1 = 0$, $x_2 = 40$.
	\item[b] Son curvas en forma de L y la cesta óptima es $x_1 = 10$, $x_2=30$.
	\item[c] Son líneas verticales paralelas, por lo que la utilidad se maximiza en la solución de esquina $x_1 = 40$, $x_2 = 0$.
	\item[d] Son curvas en forma de L y la cesta óptima es: $x_1 = 30$, $x_2 = 10$.
\end{itemize}

\seccion{Test 2016}
\textbf{6.} Los datos muestran, sin lugar a dudas, que, en el último medio siglo, la renta per cápita ha aumentado en España de forma sustancial y las tasas de fecundidad, en cambio, se han reducido. Esto significa:
\begin{enumerate}
    \item[a] que la demanda de hijos no tiene nada que ver con la racionalidad económica.
    \item[b] que, en el comportamiento de los consumidores, el efecto de sustitución ha prevalecido sobre el efecto renta.
    \item[c] que, a medida que aumenta el nivel de renta, el porcentaje de personas casadas es menor.
    \item[d] que los hijos son considerados por los consumidores como bienes inferiores.
\end{enumerate}

\textbf{8.} Se sabe que la elasticidad de $y$ respecto a $x$ es 0,5. Por lo tanto, la elasticidad de $y$ respecto a $(1/x)$ será:
\begin{enumerate}
    \item[a] 0,5
    \item[b] 1
    \item[c] 2
    \item[d] Ninguna de las respuestas anteriores es correcta.
\end{enumerate}


\seccion{Test 2015}
\textbf{2.} Un consumidor, cuya renta es $M=150$, se enfrenta a los precios $(p_x, p_y) = (5,5)$ y maximiza su función de utilidad, que es estrictamente cóncava, adquiriendo la cesta $(x,y) = (10,20)$. Supongamos que los precios pasan a ser $(p_x, p_y) = (7,4)$. En la nueva situación el consumidor:
\begin{enumerate}
	\item[a] Aumentará su utilidad y aumentará el consumo del bien $x$.
	\item[b] Disminuirá su utilidad.
	\item[c] No se sabe si aumentará o disminuirá su utilidad, pues no se conoce la función de utilidad.
	\item[d] Aumentará su utilidad y aumentará el consumo del bien $y$.
\end{enumerate}

\textbf{5.} Considere un consumidor que demanda tres bienes y que en equilibrio se gasta la mitad de su renta en el bien 1 y se gasta la tercera parte de su renta en el bien 2. Si la elasticidad renta del bien 1 es $5/3$ y la elasticidad renta del bien 2 es 0, ¿cuál es la elasticidad renta del bien 3?

\begin{enumerate}
    \item[a] 0.
    \item[b] 3/5.
    \item[c] 1
    \item[d] -3
\end{enumerate}

\seccion{Test 2014}
\textbf{3.} Si las preferencias de un individuo sobre los bienes $x$ e $y$ son monótonas, entonces sus curvas de indiferencia.
\begin{enumerate}
    \item[a] No se cruzan.
    \item[b] Son crecientes.
    \item[c] Son decrecientes.
    \item[d] Son convexas.
\end{enumerate}

\textbf{4.} Un consumidor cuya renta monetaria es $I=4$ está considerando adquirir la cesta $(0,2)$. Si los precios de los bienes son $p_x = 1$ y $p_y = 2$ y su $\text{RMS}_{(0,2)} = 1$, entonces debería:
\begin{enumerate}
    \item[a] Disminuir su consumo de x y aumentar su consumo de y.
    \item[b] Aumentar su consumo de x y disminuir su consumo de y.
    \item[c] Aumentar su consumo de x y de y.
    \item[d] Mantener la cesta $(0,2)$.
\end{enumerate}

\seccion{Test 2011}

\textbf{6.} Suponga un mundo con dos bienes (bien A y bien B) y un consumidor con una función de utilidad creciente y diferenciable. Si la relación marginal de sustitución del bien A por el bien B es de 2, el precio del bien A es 4, el precio del bien B es 1, y el consumidor se halla en la frontera de su conjunto presupuestario, este puede aumentar su utilidad:
\begin{enumerate}
    \item[a] Aumentando el consumo del bien A y disminuyendo el consumo del bien B.
    \item[b] Aumentando el consumo del bien B y disminuyendo el consumo del bien A.
    \item[c] Aumentando el consumo de ambos bienes si fuera el caso que el consumidor no gastaba toda su renta.
    \item[d] No se puede saber si es más conveniente aumentar el consumo de A o el consumo de B pues dependerá de si sus preferencias son convexas o no.
\end{enumerate}

\seccion{Test 2009}

\textbf{3.} Un consumidor tiene la función indirecta de utilidad:  $V(p, m) = \frac{m}{ (p_1^2 + p_2^2)^{ 1/2 }  }$ siendo $m$ la renta y $p=(p_1, p_2)$ el vector de precios de los bienes 1 y 2. En ese caso, la demanda marshalliana del bien 1 es:
\begin{enumerate}
    \item[a] $x_1 (p,m) = \frac{p_1 m}{p_1^2 + p_2^2}$
    \item[b] $x_1 (p,m) = \frac{p_1 p_2^{1/2} m }{p_1^2 + p_2^2}$
    \item[c] $x_1 (p,m) = \frac{p_1 m}{p_2 (p_1^2 + p_2^2)}$
    \item[d] $x_1 (p,m) = \frac{p_1}{m}$
\end{enumerate}


\textbf{5}. Considere el caso de un consumidor cuya cesta de consumo es $x=(x_1, x_2, ..., x_n)$. Si las elasticidades-renta de los bienes $x_2, ..., x_n$ son todas ellas iguales a cero, entonces podemos afirmar que la elasticidad-renta del bien 1 debe ser igual a:

\begin{enumerate}
    \item[a] Cero.
    \item[b] Un valor positivo, inferior a 1.
    \item[c] La inversa de la suma de las proporciones de gasto $(p_i x_i / m)$ del consumidor en los bienes $x_2, ..., x_n$.
    \item[d] La inversa de la proporción de gasto en el bien 1.
\end{enumerate}

\seccion{Test 2007}
\textbf{3.} Un individuo con preferencias sobre los bienes x e y consume una combinación de bienes $(x_0, y_0)$ tal que la Relación Marginal de Sustitución del bien y por el x es mayor que los precios relativos de dichos bienes: $\left( | \text{RMS}_{y,x} | \right) > \frac{P_x}{P_y}$. Es falso que:
\begin{enumerate}
    \item[a] Si las preferencias son de buen comportamiento el consumidor mejorará siempre que consuma más unidades de x.
    \item[b] Si las preferencias no son de buen comportamiento puede que el consumidor esté en equilibrio.
    \item[c] Si las preferencias son estrictamente convexas, la condición de tangencia entre la recta de balance y la curva de indiferencia es condición necesaria y suficiente para alcanzar el óptimo del consumidor.
    \item[d] El consumidor nunca estará en equilibrio dado que no se cumple la condición de tangencia entre curva de indiferencia y recta presupuestaria.
\end{enumerate}

\textbf{6.} La ley fundamental de la demanda dice que:
\begin{enumerate}
    \item[a] Si el bien es tal que al aumentar la renta aumenta la demanda, cuando aumente el precio se reducirá la demanda.
    \item[b] Si el bien es tal que al aumentar la renta aumenta la demanda, cuando aumente el precio aumentará la demanda.
    \item[c] Si el bien es tal que al aumentar la renta aumenta la demanda, cuando aumente el precio no se puede prever cómo responderá la demanda.
    \item[d] Ninguna de las anteriores.
\end{enumerate}

\seccion{Test 2006}

\textbf{3.} Un individuo tiene unas preferencias representadas por la función de utilidad $U(x_1, x_2) = ax_1 + x_2^{1/2}$.
\begin{enumerate}
    \item[a] Si $a>0$ la ordenación subyacente verifica el axioma de monotonía, pero no el de estricta convexidad.
    \item[b] Si $a>0$ la ordenación subyacente verifica el axioma de estricta convexidad, pero no el de monotonía.
    \item[c] Si $a<0$ la ordenación subyacente verifica el axioma de monotonía, pero no el de estricta convexidad.
    \item[d] Si $a>0$ la ordenación subyacente verifica tanto el axioma de monotonía, como el de estricta convexidad.
\end{enumerate}


\textbf{7.} Que las funciones de demanda marshallianas sean homogéneas de grado 0 en los precios y la renta significa que:

\begin{enumerate}
    \item[a] Un cambio en la unidad monetaria no afecta a las decisiones del consumidor.
    \item[b] El consumidor experimenta ilusión monetaria.
    \item[c] El consumidor prefiere precios mayores siempre que la renta sea también mayor.
    \item[d] La demanda del consumidor no varía si aumenta la renta y los precios mantienen constantes.
\end{enumerate}

\textbf{8.} Suponga que un consumidor gasta toda su renta comprando, a precios $p^0 = (p_1^0, p_2^0)$ el vector $x^0 = (x_1^0, x_2^0)$ y, a precios $p^1=(p_1^1, p_2^1)$, el vector $x^1 = (x_1^1, x_2^1)$. El axioma débil de preferencia revelada indica que el vector $x^0$ se revela como preferido al vector $x^1$ cuando:
\begin{enumerate}
    \item[a] Si $p^0 x^1 \geq p^0 x^0 \Rightarrow p^1 x^1 > p^1 x^0$.
    \item[b] Si $p^0 x^0 \geq p^0 x^1 \Rightarrow p^1 x^1 < p^1 x^0$.
    \item[c] Si $p^0 x^0 > p^0 x^1 \Rightarrow p^1 x^1 > p^1 x^0 $.
    \item[d] Ninguna de las respuestas anteriores.
\end{enumerate}


\seccion{Test 2004}

\textbf{3.} Considere el siguiente gráfico, que ilustra el problema de elección de un consumidor competitivo, que elige la cesta que prefiere de su conjunto presupuestario.

\begin{axis}{4}{}{$x_1$}{$x_2$}{test2004p3}
	\draw[-] (0,3.5) -- (1.5,0);
	\draw[-to] (.7,3.8) -- (.3, 3.05);
	\node(above right) at (1, 4) {\small $\text{RB}_1$};
	
	\draw[-] (0,2) -- (4,0);
	\draw[-to] (3,1.2) -- (2.5,0.9);
	\node (above right) at (3.3,1.2) {\small $\text{RB}_2$};
	
	\node[circle,fill=black,inner sep=0pt,minimum size=4pt,label=below:{($z_1$,$z_2$)}] (a) at (2,0.5) {};
	
	\node[circle,fill=black,inner sep=0pt,minimum size=4pt,label=right:{($x_1$,$x_2$)}] (a) at (0.35,2.7) {};
	
	\node[circle,fill=black,inner sep=0pt,minimum size=4pt,label=above right:{($y_1$,$y_2$)}] (b) at (3.75,0.125) {};
	
\end{axis}

Suponga que, dada la recta de balance $\text{RB}_1$, elige la cesta $(x_1, x_2)$, y que dada la recta de balance $RB_2$, elige la cesta $(y_1, y_2)$. Según el AXIOMA DÉBIL DE PREFERENCIA REVELADA, debe cumplirse que:
\begin{enumerate}
    \item[a] $(x_1, x_2)$ sea preferido por el consumidor a $(y_1, y_2)$.
    \item[b] $(x_1, x_2)$ sea preferido por el consumidor a $(z_1, z_2)$.
    \item[c] $(y_1, y_2)$ sea preferido por el consumidor a $(z_1, z_2)$.
    \item[d] $(z_1, z_2)$ sea preferido por el consumidor a $(x_1, x_2)$.
\end{enumerate}

\notas

\textbf{2018.} \textbf{3.} B

\textbf{2017.} \textbf{3.} B \textbf{4.} D

\textbf{2016.} \textbf{6.} INVALIDADA. La pregunta se enmarca en el enfoque de Gary Becker de la modelización de los comportamientos de agentes en el seno de la familia. La respuesta correcta sería la B: el efecto renta es positivo porque los hijos son un bien normal, sin embargo requieren tiempo, y éste se ha convertido en un bien caro al aumentar la productividad del trabajo/los salarios. \textbf{8.} D: La respuesta correcta es $-0,5$. Si $\frac{dy}{y} \cdot \frac{x}{dx} = 0,5$, podemos sustituir $\frac{dy}{y}$ en la elasticidad respecto a $(1/x)$, y teniendo en cuenta que $\frac{d(\frac{1}{x})}{dx} = \frac{-1}{x^2}$ se obtiene el resultado de $-0,5$.

\textbf{2015.} \textbf{2.} D. Es una pregunta relativamente difícil. A priori parece imposible saber qué sucede con la utilidad, pero no es así. Tanto el nuevo consumo de $y$ como el aumento o disminución de la utilidad depende de la posición de la cesta óptima inicial respecto a las dos restricciones presupuestarias. Si la cesta inicial se situaba a la derecha de la intersección, entonces las curvas de indiferencia son tales que al aumentar el precio de $x$ y bajar el de $y$, el agente se situaría en una curva de indiferencia de menor utilidad. Inversamente sucedería si la cesta óptima se situase a la izquierda de la intersección. Tenemos datos suficientes para saber donde se sitúa la cesta óptima inicial, y vemos que sitúa precisamente en el punto de cruce entre las dos restricciones presupuestarias. En este punto, si la $|\text{RMS}|_{yx} = -\frac{dy}{dx}$ del óptimo aumenta, tal y como sucede cuando aumenta el cociente de precios $\frac{P_x}{P_y}$ aumenta, la nueva curva de indiferencia óptima se sitúa a la izquierda de la intersección, y por ello se trata de una curva de indiferencia que representa mayor utilidad y consumo de y.

\textbf{5.} C

\textbf{2014.} \textbf{3.} C. \textbf{4.} B

\textbf{2011.} \textbf{6.} B.

\textbf{2009.} \textbf{3.} A \textbf{5.} D

\textbf{2007.} \textbf{3.} D \textbf{6.} A

\textbf{2006.} \textbf{3.} D \textbf{7.} A  \textbf{8.} B

\textbf{2004.} \textbf{3.} C


En este tema falta hablar de la separabilidad de las preferencias y las funciones de utilidad. Hay unas notas de clase interesantes en la carpeta del tema.


\bibliografia

Mirar en Palgrave:
\begin{itemize}
	\item aggregation (theory)
	\item consumer expenditure
	\item demand theory
	\item Engel curve
	\item Engel's law
	\item integrability
    \item revealed preference theory
    \item utility
\end{itemize}


Jehle, Reny. \textit{Advanced Microeconomics}

MWG. \textit{Microeconomic Theory}. Ch. 1, 2, 3

Michael and Becker. \textit{On the new theory of consumer behavior}

Kreps. \textit{A course in microeconomic theory}. Ch. 2: The theory of consumer choice and demand.

Heckman. \textit{Gary Becker: model economic scientist}. \url{https://www.ncbi.nlm.nih.gov/pmc/articles/PMC4687489/}

Becker, G. \textit{Teoría económica}. (1977) Sign. 30586 (Biblioteca de Ávila)

Jackson, M. \textit{The Non-Existence of Representative Agents} \url{https://papers.ssrn.com/sol3/papers.cfm?abstract\_id=2684776}

Chiappori, P. A., Lewbel, A. \textit{Gary Becker's, <<A theory of the allocation of time>>}. \url{https://www2.bc.edu/arthur-lewbel/Becker5.pdf}

\end{document}
