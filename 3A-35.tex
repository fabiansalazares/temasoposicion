\documentclass{nuevotema}

\tema{3A-35}
\titulo{La política monetaria (I): el diseño y la instrumentación de la política monetaria.}

\begin{document}

\ideaclave

Leer Pfister y Cahuc 2020 para PM no convencionales.

\seccion{Preguntas clave}
\begin{itemize}
	\item ¿Qué es la política monetaria?
	\item ¿Por qué es importante?
	\item ¿Qué objetivos tiene la política monetaria?
	\item ¿Qué caracteriza a una política monetaria óptima?
	\item ¿Cómo se diseña una política monetaria óptima?
	\item ¿De qué instrumentos de política monetaria disponen los bancos centrales?
	\item ¿Qué es la política monetaria no convencional?
\end{itemize}

\esquemacorto

\begin{esquema}[enumerate]
	\1[] \marcar{Introducción}
		\2 Contextualización
			\3 Política monetaria
			\3 Evolución de la política monetaria
			\3 Instrumentación y diseño
		\2 Objeto
			\3 ¿Qué es la política monetaria? ¿Por qué es importante?
			\3 ¿Qué objetivos tiene?
			\3 ¿Qué caracteriza a una política monetaria óptima?
			\3 ¿Cómo se diseña una política monetaria óptima?
			\3 ¿De qué instrumentos disponen los bancos centrales?
			\3 ¿Qué es la política monetaria no convencional?
		\2 Estructura
			\3 Instrumentación de la PM
			\3 Diseño de la PM
	\1 \marcar{Instrumentación de la política monetaria}
		\2 Idea clave
			\3 Sistema financiero
			\3 Instrumentos de política monetaria
		\2 Dinero
			\3 Idea clave
			\3 Funciones
			\3 Dinero commodity o fiduciario
			\3 Dinero externo e interno
		\2 Estructura del sistema monetario
			\3 Banco central
			\3 Instituciones financieras
			\3 Sector no financiero
			\3 Agregados monetarios
			\3 Creación de dinero
		\2 Instrumentos convencionales
			\3 Idea clave
			\3 Política de crédito y redescuento
			\3 Operaciones de mercado abierto
			\3 Requisitos de reservas obligatorias mínimas
			\3 Otros instrumentos
		\2 Instrumentos no convencionales
			\3 Idea clave
			\3 Forward guidance
			\3 Quantitative easing y credit easing
			\3 Tipos de interés negativos
			\3 Helicopter money
			\3 Ampliación de colateral aceptable
			\3 Financiación directa de empresas no financieras
	\1 \marcar{Diseño de la política monetaria}
		\2 Idea clave
			\3 Contexto
			\3 Objetivos
			\3 Resultados
		\2 Evolución histórica
			\3 Idea clave
			\3 Currency School
			\3 Banking School
			\3 Gran depresión
			\3 Keynes
			\3 Monetarismo
			\3 Nueva Macroeconomía Clásica
			\3 Política monetaria actual
		\2 Implementación de la política monetaria
			\3 Idea clave
			\3 Instrumentos de política monetaria
			\3 Variables operativas
			\3 Objetivos intermedios
			\3 Objetivos finales de PM
		\2 Elección de objetivos de política monetaria
			\3 Idea clave
			\3 Agregado monetario vs interés nominal
			\3 Inflation targetting, price level-targetting o PIB nominal
		\2 Debates de política monetaria óptima
			\3 Divina coincidencia
			\3 Ruptura de la divina coincidencia
			\3 Discrecionalidad frente a reglas de PM
			\3 Independencia del banco central
			\3 Política monetaria en la ZLB
	\1[] \marcar{Conclusión}
		\2 Recapitulación
			\3 Instrumentación de la PM
			\3 Diseño de la PM
		\2 Idea final
			\3 Lecciones de diseño de PM no convencional hasta ahora
			\3 Efectos y mecanismos de transmisión de la PM
			\3 Arte y ciencia

\end{esquema}

\esquemalargo

\begin{esquemal}
	\1[] \marcar{Introducción}
		\2 Contextualización
			\3 Política monetaria
				\4 Decisiones tomadas por BCentral o AMonetaria
				\4[] Respecto a:
				\4[] $\to$ Cantidad de dinero en circulación
				\4[] $\to$ Condiciones de financiación
				\4 Para alcanzar determinados objetivos macro
				\4[] $\to$ Crecimiento sostenido de producto real
				\4[] $\to$ Tasa de paro
				\4[] $\to$ Estabilidad de precios
			\3 Evolución de la política monetaria
				\4 Ligada a evolución de varios factores:
				\4[] $\to$ Avances tecnológicos
				\4[] $\to$ Entorno macroeconómico
				\4[] $\to$ Contexto político
				\4[] $\to$ Teoría económica
				\4 Primera manifestación de PMonetaria moderna
				\4[] Descuento de deuda pública y privada
				\4[] $\to$ Banco de Inglaterra
				\4[] Fomentar mercado de deuda pública
				\4[] Prestamista de prestamistas
				\4 Segunda fase de política monetaria
				\4[] Suavizar fluctuaciones de tipos
				\4[] $\to$ Estacionales y cíclicas
				\4[] Mantener precio y reservas de oro
				\4 Política monetaria moderna
				\4[] Tres instrumentos convencionales
				\4[] $\to$ Tipo de descuento
				\4[] $\to$ Operaciones de mercado abierto
				\4[] $\to$ Requisitos de reservas bancarias
				\4[] Instrumentos no convencionales
				\4[] $\to$ Forward guidance
				\4[] $\to$ Quantitative Easing
				\4[] $\to$ Tipos negativos
			\3 Instrumentación y diseño
				\4 Objetivo último de PM
				\4[] Común a PEconómica general
				\4[] $\to$ Maximizar bienestar de ciudadanos
				\4 Actuaciones para lograr objetivo
				\4[] No son triviales
				\4[] Sujetos a controversias
				\4[] $\to$ Sobre resultados positivos
				\4[] $\to$ Sobre supuestos normativos
				\4 Instrumentos
				\4[] Decisiones a disposición de policy-maker
				\4 Diseño de política monetaria
				\4[] Utilización óptima de instrumentos
		\2 Objeto
			\3 ¿Qué es la política monetaria? ¿Por qué es importante?
			\3 ¿Qué objetivos tiene?
			\3 ¿Qué caracteriza a una política monetaria óptima?
			\3 ¿Cómo se diseña una política monetaria óptima?
			\3 ¿De qué instrumentos disponen los bancos centrales?
			\3 ¿Qué es la política monetaria no convencional?
		\2 Estructura
			\3 Instrumentación de la PM
			\3 Diseño de la PM
	\1 \marcar{Instrumentación de la política monetaria}
		\2 Idea clave
			\3 Sistema financiero
				\4 Entorno institucional
				\4[] En el que PM se lleva a cabo
				\4 Agentes del sistema financiero
				\4[] Comportamiento autónomo
				\4[] Decisiones interaccionan entre agentes
				\4 Relación con economía real
				\4 Papel de BCentrales y AMonetarias
				\4[] Alterar condiciones de sistema financiero
				\4[] $\to$ Afectar a economía real
				\4[] $\then$ Utilización de instrumentos disponibles
			\3 Instrumentos de política monetaria
				\4 Relaciones causa-efecto que BC puede utilizar
				\4[] BC provoca causas
				\4[] Esperando ocurra un efecto determinado
				\4 Elementos subyacentes a los instrumentos
				\4[] Dinero y similares
				\4[] Estructura del sistema monetario
		\2 Dinero
			\3 Idea clave
				\4 Activo real o financiero
				\4[] Tendencia histórica hacia financiero
				\4 Cumple una serie de funciones básicas
				\4 Presente en casi todas economías
				\4[] Posible concebir economía sin dinero
				\4[] $\to$ Difícil o imposible implementación
			\3 Funciones
				\4 Habitual en libros de texto
				\4[I] Medio de intercambio
				\4[] Todos los agentes aceptan a cambio de ByS
				\4[] $\to$ Muy líquido
				\4[] Evita doble coincidencia de demandas
				\4[] $\to$ Especialización del trabajo
				\4[] Requiere algunas características
				\4[] $\to$ Reconocible
				\4[] $\to$ Divisible
				\4[] $\to$ Transportable
				\4[II] Depósito de valor
				\4[] Permite diferir pago
				\4[] $\to$ Medio de ahorro
				\4[III] Unidad de cuenta
				\4[] Utilizado como numéraire
				\4[] Valor de ByS
				\4[] $\to$ En términos de dinero
				\4[] No es función esencial
			\3 Dinero commodity o fiduciario
				\4 Dinero commodity
				\4[] Demanda ajena a uso monetario
				\4[] $\to$ Valor intrínseco
				\4[] Poseen ciertas características físicas
				\4[] $\to$ Moldeable
				\4[] $\to$ Duradero
				\4[] $\to$ Composición constante
				\4 Dinero fiduciario
				\4[] Sin valor intrínseco
				\4[] Valor basado en confianza $\to$ \textit{fides}
			\3 Dinero externo e interno
				\4 ``inside'' vs ``outside'' money
				\4 Externo
				\4[] Riqueza neta de poseedores
				\4[] $\to$ No es deuda frente a nadie
				\4[] $\to$ Consolidación de balances no elimina
				\4[] Ejemplo:
				\4[] $\to$ Oro y plata
				\4[] $\to$ Dinero en efectivo dentro de país salvo BC
				\4 Interno
				\4[] No es riqueza neta de sectores considerados
				\4[] Consolidación de balances cancela saldo monetario
				\4[] Depósitos bancarios
		\2 Estructura del sistema monetario
			\3 Banco central
				\4 Funciones
				\4[] Ofertar base monetaria
				\4[] Prestamista de último recurso a IF
				\4[] Mantener reservas de divisas y oro
				\4[] Estabilizar canales de transmisión de PM
				\4[] Mantener sistema de pagos
				\4 Balance
				\4[] \underline{Activo}
				\4[] Activos Netos frente a Sector Exterior
				\4[] $\to$ Reservas de divisas
				\4[] $\to$ Deuda pública exterior
				\4[] Activos frente a Sector Público
				\4[] $\to$ Deuda pública
				\4[] Activos frente a Sector Privado
				\4[] $\to$ Instituciones financieras
				\4[] $\to$ Sector no financiero\footnote{Poco habitual, sujeto a restricciones legales y políticas.}
				\4[] Otros activos
				\4[] $\to$ Oro y otros
				\4[] $\to$ Propiedades inmobiliarias de BC
				\4[] \underline{Pasivo}
				\4[] Dinero en efectivo
				\4[] Reservas en efectivo
				\4[] Depósitos de IF
				\4[] Patrimonio Neto
			\3 Instituciones financieras
				\4 Funciones
				\4[] Transformación de vencimientos
				\4[] Transferencia de riesgos
				\4[] Mantener sistema de pagos
				\4 \underline{Activo}
				\4[] Préstamos al público
				\4[] Activos frente a Sector Público
				\4[] Reservas en efectivo
				\4[] Depósitos en banco central
				\4[] Activos físicos
				\4 \underline{Pasivo}
				\4[] Deuda con otras IFs
				\4[] Depósitos del público
				\4[] Patrimonio Neto
			\3 Sector no financiero
				\4 \underline{Activo}
				\4[] Dinero en efectivo
				\4[] Depósitos en IF
				\4[] Otros activos
				\4 \underline{Pasivo}
				\4[] Préstamos de IF
				\4[] Patrimonio neto
			\3 Agregados monetarios\footnote{Ver Manual on MFI Balance Sheet Statistics, pág. 110.}
				\4 Conjuntos de activos monetarios
				\4[] A partir de balances anteriores
				\4[] Agrupan diferentes características
				\4[] $\to$ Liquidez
				\4[] $\to$ Capacidad para generar más dinero
				\4[] Subconjuntos sucesivos
				\4 Diferentes definiciones
				\4[] Según área monetaria
				\4[] Aproximadamente equivalentes
				\4 Base monetaria
				\4[] También llamada ``high-powered money''
				\4[] $\to$ Monedas y billetes en circulación
				\4[] $\to$ Reservas de IFs en BC
				\4[] $\to$ Reservas de IFs en monedas y billetes
				\4 Agregados del Eurosistema
				\4[] Diferentes delimitaciones de ``oferta monetaria''
				\4[M0] Billetes y monedas en circulación
				\4[M1] = M0 + Depósitos a la vista
				\4[M2] = M1 + Dep. hasta 2 años +
				\4[] + Dep. redimibles con máximo de 3 meses antelación
				\4[$\to$] Gran variedad de modalidades
				\4[M3] = M2 + activos comercializables de c/p de IFs +
				\4[] + Deuda de hasta 2 años + repos + money market
				\4[] $\to$ Activos muy líquidos
				\4[] $\to$ Activos menos sustituibles
				\4[] $\to$ Relativamente más estable
				\4[] $\to$ Activos vencimiento mayor no incluidos\footnote{Aunque cuando están próximos al vencimiento, pueden ser tratados como sustitutos.}
			\3 Creación de dinero
				\4 Banco Central influye OMonetaria parcialmente
				\4[] BCentral determina base monetaria
				\4[] Creación de depósitos depende de:
				\4[] $\to$ Demanda de depósitos
				\4[] $\to$ Base monetaria
				\4[] $\to$ Requisitos de reservas
				\4[] $\to$ Políticas macroprudenciales
				\4[] $\then$ Control imperfecto sobre oferta monetaria
				\4 Multiplicador monetario
				\4[] Oferta monetaria vs base monetaria
				\4[] $\to$ ¿Qué relación entre ambas?
				\4[] $\then$ Indicador básico sistema monetario
				\4[] \fbox{$\frac{M}{B} = \frac{u + 1}{u + v} = \frac{C/D+1}{C/D + R/D}$}
				\4[] C: efectivo en manos del público
				\4[] D: saldos en depósitos al público\footnote{Incluyendo grados variables de liquidez en esta representación simplificada.}
				\4[] R: reservas de IF en BCentral
				\4[] B: base monetaria $\to$ $C+R$
				\4[] M: oferta monetaria $\to$ $C+D$
				\4[] u: coeficiente de efectivo = $C/D$
				\4[] $\to$ Cuánto desean mantener como efectivo
				\4[] v: coeficiente de caja o reservas = $R/D$
				\4[] $\to$ Cuántas reservas mantienen IFs dados depósitos
				\4[] Derivación:
				\4[] $\frac{M}{B} = \frac{C+D}{C+R} = \frac{C/D + 1}{C/D + R/D} = \frac{u+1}{u+v}$
				\4[] Si $v \to 1$:
				\4[] $\to$ Reservas del 100\% sobre depósitos ($R/D=1$)
				\4[] $\to$ No  hay reserva fraccionaria
				\4[] $\to$ Multiplicador monetario es 1
				\4[] $\then$ $\frac{M}{B} = \frac{u+1}{u+1}=1$
				\4[] $\then$ Sistema bancario no crea dinero
				\4 Multiplicador bancario
				\4[] Si $u \to 0$:
				\4[] $\to$ Relación entre oferta monetaria máxima y base
				\4[] \fbox{$\frac{M_\text{MÁX}}{B} = \frac{1}{v}$}
				\4[] $\to$ Agentes no demandan efectivo
				\4[] $\to$ Agentes convierten todo efectivo en depósitos
				\4[] $\to$ Creación de dinero alcanza máximo
				\4[] Cantidad máxima de depósitos dadas reservas
				\4[] $\to$ Si agentes sólo demandasen depósitos
				\4[] $\to$ Demanda nula de dinero en efectivo
				\4[] $\then$ Caso particular extremo de MMonetario
		\2 Instrumentos convencionales
			\3 Idea clave
				\4 Variación entre países
				\4[] Operativa e instrumentos disponibles
				\4[] $\to$ No son iguales en todos los países
				\4 Utilización variable
				\4[] En algunos países, algunos no se aplican
				\4[] Desarrollo de SFinanciero es relevante
			\3 Política de crédito y redescuento
				\4 Instrumento original de PM
				\4 IFs prestan/toman prestadas reservas
				\4[] Muy corto plazo
				\4[] IFs prestan reservas:
				\4[] $\to$ Fondos depositados en banco central
				\4[] IFs toman prestadas reservas:
				\4[] $\to$ Préstamos de corto plazo
				\4[] $\to$ Descuento de letras
				\4[] $\to$ Repos
				\4 Pasillo interbancario
				\4[] Banco central fija límites inferior y superior
				\4[] Límite superior
				\4[] $\to$ Tipo al que pedir prestado
				\4[] $\to$ BCentral ofrece fondos a ese tipo
				\4[] $\to$ Nadie aceptará pagar más que límite superior
				\4[] Límite inferior
				\4[] $\to$ Tipo al que se remuneran reservas
				\4[] $\to$ BCentral paga por mantener reservas
				\4[] $\to$ Nadie acepta menos por prestar reservas
				\4 En UE -- área del Euro
				\4[] ``Standing facilities''
				\4[] Marginal Lending Facility
				\4[] $\to$ Suelo del pasillo interbancario
				\4[] $\to$ En forma de repos
				\4[] Deposit Facility
				\4[] $\to$ Techo del pasillo interbancario
				\4[] EONIA
				\4[] $\to$ European OverNight Interest Average
				\4[] Transición a €STR
				\4[] $\to$ Ver \url{https://www.ecb.europa.eu/pub/economic-bulletin/focus/2019/html/ecb.ebbox201907_01~b4d59ec4ee.en.html}
				\4[] $\to$  Ver \url{https://blog.bankinter.com/economia/-/noticia/2019/8/8/que-estr-tipo-interes-corto-plazo-euro}
				\4[] $\to$  Ver \url{https://www.bde.es/f/webbde/INF/MenuHorizontal/Publicaciones/Boletines\%20y\%20revistas/InformedeEstabilidadFinanciera/ief_2019_1_Rec2_1.pdf}
			\3 Operaciones de mercado abierto
				\4 Compra/venta de deuda pública en mercado secundario
				\4[] Cambio de reservas de bancos comerciales
				\4[] $\to$ Abona cuenta de BComercial si BC compra deuda
				\4[] $\to$ Carga cuenta de BComercial si BC vende deuda
				\4[] $\then$ Aumento/caída de reservas no prestadas
				\4 Efecto sobre tipo de interés
				\4[] Afectan de manera directa e indirecta
				\4[] Directa
				\4[] $\to$ Presión sobre precio de activo afectado
				\4[] Indirecta
				\4[] $\to$ Aumento/caída de liquidez disponible
				\4[] $\to$ Ajuste de carteras y sustitución de activos
				\4 Esterilización
				\4[] Compra/venta implica $\Delta$ tamaño de balance
				\4[] Si BCentral no desea cambio en balance
				\4[] $\to$ debe realizar operación contraria
				\4[] $\to$ En otro activo o segmento de mercado
				\4[] $\to$ Compensar variación inicial del balance
				\4[] $\then$ Esterilización de operación
				\4 En UE -- área del Euro
				\4[] MROs
				\4[] LTROs
				\4[] TLTROs
				\4[] APPs
			\3 Requisitos de reservas obligatorias mínimas
				\4 Fijación de coeficiente de caja mínimo
				\4[] Ratio mínimo reservas--depósitos
				\4 Aplicable a determinados depósitos
				\4[] $\to$ Generalmente, los garantizados
				\4 Aumento de coeficiente de caja
				\4[] $\Delta$ necesidades estructurales de liquidez
				\4[] BComerciales deben obtener reservas
				\4[] $\to$ Reducción de depósitos
				\4[] $\to$ Captación de capital
				\4[] $\to$ Pedir prestado a BCentral y depositar
				\4[] Generalmente, piden prestado a BCentral
				\4[] $\to$ Presión sobre tipo de corto plazo
				\4[] $\to$ Spillover a otros tipos de interés
			\3 Otros instrumentos
				
				\4 Habituales en sistemas menos desarrollados
				\4[] Restricciones directas a préstamos
				\4[] Reservas respecto a activos bancarios
				\4[] $\to$ Más allá de depósitos
				\4 Regulación macroprudencial
		\2 Instrumentos no convencionales\footnote{Ver Pfister y Sahuc (2020) para complementar con efectos de las PM no convencionales.}
			\3 Idea clave
				\4 Causas de aparición de nuevos instrumentos
				\4[] Instrumentos convencionales inefectivos
				\4[] $\to$ Pérdida de credibilidad de BCentral
				\4[] $\to$ Límite inferior de interés nominal
				\4[] $\to$ Crisis financieras y canales de transmisión
				\4[] Nuevos desarrollos teóricos
				\4[] $\to$ Importancia de las expectativas
				\4[] $\to$ No linealidades si mercados desaparecen
				\4[] $\to$ Acelerador financiero
				\4[] $\to$ ...
				\4 Objetivo principal de PM no convencional
				\4[] Lograr reducciones de tipo de interés real
				\4[] $\to$ Cuando otros son inefectivos
			\3 Forward guidance
				\4 Comunicación de Bancos Centrales
				\4[] Desde 80s, habitual:
				\4[] $\to$ Condiciones económicas
				\4[] $\to$ Proyecciones utilizadas por BCentral
				\4[] $\to$ Análisis de políticas
				\4[] $\to$ Objetivos explícitos
				\4 Rasgo característico de forward-guidance
				\4[] Senda de tipos de interés futuro explícita
				\4[] Diferentes referencias posibles
				\4[] $\to$ Calendario
				\4[] $\to$ Cualitativa
				\4[] $\to$ Cuantitativa
				\4 Utilización en crisis financiera
				\4[] Fijar expectativas de agentes
				\4[] Tipos seguirán bajos más allá de recesión
				\4 Utilización por BC
				\4[] EE.UU.
				\4[] $\to$ Anuncio de diciembre de 2012
				\4[] UE
				\4[] $\to$ Julio de 2012: ``whatever it takes''
				\4[] $\then$ PM acomodaticia hasta que fuese necesario
				\4[] $\to$ Otros ejemplos tras crisis
				\4 Críticas
				\4[] Puede revelar pesimismo mayor del esperado
				\4[] Sensibilidad decreciente a shocks de información
			\3 Quantitative easing y credit easing\footnote{Ver palgrave: ``quantitative easing by the major western central banks during the Global Financial Crisis'' para etapas de crisis y programas de QE.}
				\4 Idea clave
				\4[] Compras de AFinancieros a cambio de reservas
				\4[] $\to$ Volumen relativamente muy elevado
				\4[] $\to$ Disposición a alterar balance de BC
				\4[] $\to$ Esterilizado y sin esterilizar
				\4[] $\to$ Reducir tipos de interés en general
				\4[] $\to$ Reducir tipos reales de largo plazo
				\4[] $\to$ Garantizar liquidez en mercados clave
				\4 Historia
				\4[] Banco de Japón en años 2000
				\4[] $\to$ Quantitative Easing
				\4[] Reserva Federal a partir de 2008
				\4[] $\to$ Inicialmente ``credit easing'' (Bernanke)
				\4[] $\to$ QE1, QE2 y QE3, MEP,
				\4[] Banco Central Europeo
				\4[] $\to$ Compras a partir de 2009
				\4[] $\to$ Varios programas posteriores
				\4 Mecanismos de transmisión
				\4[] Aumentar confianza de sector privado
				\4[] $\to$ Transmitir voluntad de hacer todo necesario
				\4[] $\to$ Frenar ``animal spirits'' negativos
				\4[] Señalizar política monetaria futura
				\4[] $\to$ Bajo interés aumenta duración
				\4[] $\to$ BC más expuesto a riesgo de duración
				\4[] $\then$ BC pierde si bajan tipos
				\4[] $\then$ Voluntad de mantener tipos bajos de BC
				\4[] Reajuste de carteras
				\4[] $\to$ Reservas y activos comprados sustitutos imperfectos
				\4[] $\then$ Agentes reajustan hacia otros activos
				\4[] $\then$ Reducción de interés en otros segmentos
				\4[] Liquidez del mercado
				\4[] $\to$ Evitar sudden stops de mercados
				\4[] $\to$ Prestamista de último recurso
				\4[] $\to$ Reducir prima de liquidez
				\4[] Base monetaria
				\4[] $\to$ Aumentar oferta de dinero
				\4[] $\to$ Aumentar creación de depósitos
				\4 Diferencia entre QE (Japón, 2000s) y QE (US,UE)
				\4[] Japón:
				\4[] $\to$ Énfasis en tamaño de balance y pasivo
				\4[] $\to$ Bonos comprados a bancos fueron de c/p
				\4[] $\to$ Activos comprados son sustitutos de dinero
				\4[] $\to$ Aumento de reservas es objetivo clave
				\4[] $\to$ Deuda del gobierno
				\4[] US, UE, UK (credit easing)
				\4[] $\to$ Énfasis en activo del balance
				\4[] $\to$ Activos de l/p poco sustitutivos de dinero
				\4[] $\to$ Expansión de balance es secundario
				\4[] $\to$ Deuda de agencias y MBS también
				\4[] $\to$ Fed aumentó su exposición a sector privado
			\3 Tipos de interés negativos
				\4 Existencia de límite inferior
				\4[] Si depósitos remunerados por debajo de 0\%
				\4[] $\to$ Agentes deberían preferir efectivo
				\4[] Pero efectivo genera costes adicionales
				\4[] $\to$ Posibilidad de pérdida
				\4[] $\to$ Almacenamiento
				\4[] Posible bajada por debajo de 0
				\4[] $\to$ Sin desplazamiento de dda. hacia billetes
				\4[] $\then$ ZLB no existe $\to$ Existe Effective Lower Bound
				\4 Reservas de IFs en bancos centrales
				\4[] Pagan interés a BC y no al revés
				\4[] $\then$ Interés negativo
				\4 Objetivo
				\4[] IFs no acumulen liquidez en BC
				\4[] Aumenten crédito y compren activos financieros
				\4[] $\to$ Aumenten precios de otros activos
				\4[] $\to$ Aumente oferta de crédito
				\4 Críticas
				\4[] Bancos no repercuten tipos negativos a clientes
				\4[] $\to$ Margen de beneficio cae
				\4[] $\to$ Reducción de beneficio
				\4[] $\then$ Puede reducir crédito
				\4[] Posible cambio masivo hacia efectivo
				\4[] $\to$ Sobre todo si INegativo percibido permanente
				\4[] Críticas por ``injusto''
				\4[] $\to$ Si clientes perciben tenencia de depósitos es costosa
				\4[] $\then$ Puede provocar rechazo social a medida
			\3 Helicopter money
				\4 En contexto de tipos negativos o cero
				\4[] Muy difícil bajarlos más aún
				\4 Milton Friedman (1969)
				\4[] Instrumento teórico
				\4[] Helicóptero que suelta billetes de 1000 \$
				\4[] Agentes piensan que es evento único que no se repetirá
				\4[] Implica aumento rápido y brusco de oferta monetaria
				\4[] $\to$ Agentes afirman precios aumentarán
				\4[] $\then$ Aumentarán demanda agregada rápidamente
				\4 Japón en crisis de 90s
				\4[] Consideración de idea para combatir deflación
				\4 Bernanke (2002)
				\4[] Financiar aumento rápido de oferta monetaria
				\4[] Con bajadas de impuestos financiadas monetariamente
				\4 Crisis de 2020
				\4[] Pagos monetarios indiscrimanados en varios países
				\4[] $\to$ Estados Unidos
				\4[] $\to$ Japón
				\4[] $\to$ Marruecos
				\4[] $\to$ ...
			\3 Ampliación de colateral aceptable
				\4 En contexto de operaciones de mercado abierto
				\4[] Generalmente a través de repos
				\4[] $\to$ Agente privado vende activo a banco central
				\4[] $\to$ Compromiso de recomprar en futuro
				\4 ¿Qué activos acepta banco central como colateral?
				\4[] Generalmente, activos sin riesgo
				\4[] Aplica haircut
				\4[] $\to$ Cantidad pagada por BCentral inferior a precio activo
				\4 Contexto de crisis financiera y PM no convencional
				\4[] Ampliación de elenco de activos
			\3 Financiación directa de empresas no financieras
				\4 Compra de bonos de empresas no financieras
				\4[] CSPP en UE
				\4 Préstamos de Banco Central directos a empresas no financieras
				\4 Compra de equity en mercados oficiales
	\1 \marcar{Diseño de la política monetaria}
		\2 Idea clave
			\3 Contexto
				\4 Política monetaria óptima enfrenta numerosos obstáculos
				\4 Transmisión imperfecta de decisiones de PM
				\4[] Shocks exógenos
				\4[] Conocimiento imperfecto de mecanismos
				\4 Incentivos de agentes
				\4[] Gobierno
				\4[] Autoridad monetaria/banco central
				\4[] $\to$ Tienen sus propios incentivos
				\4[] $\to$ Public choice
				\4 Consistencia temporal de la política monetaria
				\4[] Planes óptimos hoy
				\4[] $\to$ Pueden no serlo mañana
				\4[] $\then$ Agentes de economía conocen posible inconsistencia
				\4[] $\to$ Tiempo afecta optimalidad de los planes
				\4 Elección de instrumentos y variables intermedias
				\4[] Dados unos objetivos globales
				\4[] $\to$ Diferentes modelos de economía
				\4[] $\then$ Diferentes vías para alcanzar objetivos
				\4[] Necesario determinar:
				\4[] $\to$ Qué variables se quiere afectar
				\4[] $\to$ A través de qué variables se puede afectar
			\3 Objetivos
				\4 Garantizar crecimiento sostenible de output
				\4 Estabilizar precios
				\4 Permitir sistema monetario cumpla funciones
				\4[] Asignación de ahorro a inversión
				\4[] Transferir riesgos
				\4[] Conversión de vencimientos
			\3 Resultados
				\4 Evolución histórica de diseño de política monetaria
				\4[] Afectada por:
				\4[] $\to$ Coyuntura económica
				\4[] $\to$ Innovaciones financieras
				\4[] $\to$ Teoría económica
				\4 Marco de implementación
				\4[] BCentrales actúan sobre determinadas variables a su alcance
				\4[] $\to$ Esperan afectar otras que toman como objetivo
				\4 Elección de diferentes instrumentos
				\4 Debate de largo plazo sobre objetivo principal
				\4[] ¿Desempleo?
				\4[] ¿Inflación?
				\4[] ¿Nivel de precios?
				\4[] ¿PIB nominal?
		\2 Evolución histórica
			\3 Idea clave
				\4 Currency vs banking school
				\4[] Primer gran debate PM
				\4[] $\to$ Tras suspensión de convertibilidad a oro\footnote{Por Guerras Napoleónicas.}
			\3 Currency School
				\4 Gobierno debe controlar oferta monetaria
				\4[] Causalidad va de M $\to$ Y
				\4 Ricardo
				\4[] $\to$ M debe ligarse a reservas de oro
				\4[] $\to$ Oro depende de balanza de pagos
				\4[] $\then$ M para ajustar balanza de pagos
				\4[] $\then$ M exógena y determinable por BC
				\4 Fisher
				\4[] $\to$ Oferta monetaria ligada a nivel de precios
				\4 Friedman
				\4[] $\to$ Regla k-percent sobre crecimiento de M
				\4 Actualmente
				\4[] $\to$ Narrow-banking
				\4[] $\to$ CBDC\footnote{Central Bank Digital Currency.}
			\3 Banking School
				\4 Gobierno no puede controlar oferta monetaria
				\4[] $\to$ Innovaciones financieras hacen imposible
				\4[] $\then$ M se determina endógenamente
				\4 M no es exógena
				\4[] $\to$ Doctrina de ``real bills''
				\4[] $\to$ BC debe descontar letras reales
				\4[] $\then$ M se ajustará a necesidades de comercio
				\4[] $\then$ Oferta monetaria es pasiva a economía real
				\4 Críticas a doctrina de real-bills
				\4[] $\to$ Especialmente, Thornton
				\4[] $\then$ Difícil distinguir letras reales de no reales
				\4[] $\then$ Aunque posible distinguir, exceso de descuento
			\3 Gran depresión
				\4[] Adherencia a banking school y real bills
				\4[] $\to$ Factor importante en crisis
				\4[] Asumen que oferta monetaria se regulará sola
				\4[] $\to$ De hecho, enorme contracción de crédito y M
				\4[] $\then$ Deflación
				\4[] $\then$ Exceso de oferta en bienes
				\4[] $\then$ Exceso de oferta en empleo
				\4[] $\then$ Paro y depresión
			\3 Keynes
				\4 PM inefectiva en depresión
				\4[] Política fiscal es dominante
				\4[] PM pasiva y supeditada a PF
				\4[] $\to$ Mantener condiciones óptimas de interés
				\4 Síntesis neoclásica
				\4[] PM para afectar tipos de c/p
				\4[] $\to$ Y a tipos l/p por sustitución
				\4[] Sí es efectiva
				\4[] Puede aplicarse fine-tuning
				\4[] $\to$ Ligado a modelos macroeconométricos
			\3 Monetarismo
				\4 Reformulación de teoría cuantitativa del dinero
				\4[] $\to$ Demanda de dinero es estable
				\4[] $\to$ Expectativas adaptativas
				\4[] $\then$ Política monetaria tiene efectos reales
				\4 PM errónea causó Gran Depresión
				\4[] Oferta de dinero es variable intermedia clave
				\4 Lags e incertidumbre dificultan uso de PM
				\4[] $\to$ Fine tuning contraproducente
				\4[] $\to$ Curva de Phillips depende de expectativas
				\4[] $\then$ Curva de Phillips no es estacionaria
				\4[] $\then$ Agregado monetario mejor objetivo intermedio
			\3 Nueva Macroeconomía Clásica
				\4 Introducción de HER
				\4[] $\to$ Activar demanda sistemáticamente es imposible
				\4 Primera fase de NMC
				\4[] $\to$ Dinero sí es importante
				\4[] $\to$ PM efectiva por información incompleta
				\4[] $\to$ Análisis de discrecionalidad vs reglas
				\4[] $\to$ Crítica de Lucas a PM no estructural
				\4 Segunda fase de NMC: ciclo real
				\4[] $\to$ PM irrelevante
				\4[] $\to$ Dinero de hecho es irrelevante
				\4[] $\to$ ``Vuelta de tuerca'': Friedman no apoya
				\4[] $\to$ Delimitación negativa de la influencia de PM
				\4[$\then$] Paso de actuaciones PM a reglas PM
				\4[] Reglas determinan resultado
				\4[] Actuaciones discrecionales no son relevantes
				\4[] $\to$ Agentes con HER estiman act. discrecionales
				\4[] $\to$ Equilibrios son ENPS
				\4[] Commitment es elemento clave
				\4[] $\to$ Buscar instrumentos para hacer creíbles reglas
			\3 Política monetaria actual
				\4 Combinación de ambas currency vs banking
				\4[] Hasta cierto punto, BC controlan oferta monetaria
				\4[] Innovaciones financieras
				\4[] $\to$ Dificultan control y estabilidad de Dda. dinero
				\4 Elección de objetivos
				\4[] $\to$ Tiende a desvincularse de idea clara de oferta monetaria
				\4[] $\then$ Inflación
				\4[] $\then$ Tipo de interés
		\2 Implementación de la política monetaria
			\3 Idea clave
				\4 Esquema tradicional de implementación de PM
				\4 BC puede afectar determinadas variables
				\4[] De forma directa
				\4 Otras variables sólo
				\4[] $\to$ Indirectamente
				\4[] $\to$ A través de variables intermedias
				\4 Secuencia de variables
				\4[] 0. Instrumentos de política monetaria
				\4[] 1. Variables operativas
				\4[] 2. Objetivos intermedios
				\4[] 3. Objetivos finales
			\3 Instrumentos de política monetaria
				\4 Utilizables a discreción del BC
				\4[i] Política de crédito y redescuento
				\4[ii] Operaciones de mercado abierto
				\4[iii] Coeficiente de caja
			\3 Variables operativas
				\4[$\to$] BC determina a través de instrumentos
				\4 Interés a corto plazo
				\4 Base monetaria
				\4 Liquidez bancaria
				\4 Pasillo interbancario
			\3 Objetivos intermedios
				\4[$\to$] Variables operativas influyen fuertemente
				\4 Interés a largo plazo
				\4 Oferta monetaria
				\4 Crédito interbancario
				\4 Tipo de cambio
			\3 Objetivos finales de PM
				\4 Crecimiento del producto
				\4 Cuenta corriente
				\4 Desempleo
				\4 Inflación
				\4 Nivel de precios
				\4 PIB nominal
		\2 Elección de objetivos de política monetaria
			\3 Idea clave
				\4 Dados unos objetivos últimos
				\4[] Maximizar bienestar de agentes
				\4[] $\to$ ¿Cómo alcanzarlos?
				\4[] $\to$ ¿Qué variable intermedia debe ser objetivo?
				\4[] $\to$ ¿Qué objetivo final?
				\4 Dos grandes debates
				\4[] Oferta de dinero o interés nominal
				\4[] $\to$ Variables intermedias
				\4[] Objetivo de inflación, precios o PIB nominal
				\4[] $\to$ Objetivos finales de PM
				\4[] $\to$ inflation targeting vs price level targeting
			\3 Agregado monetario vs interés nominal
				\4 Poole (1970)
				\4[] Caracterizar respuestas a:
				\4[] $\to$ ¿Cuál debe ser la var. intermedia?
				\4[] $\to$ ¿Bajo qué condiciones?
				\4[] Objetivo:
				\4[] $\to$ Reducir volatilidad del output
				\4[] Dos posibles variables intermedias:
				\4[] -- Tipo de interés
				\4[] -- Oferta monetaria
				\4 Marco IS-LM
				\4[] $\to$ $Y = f(i)$, $f'(r) < 0$
				\4[] $\to$ $m = L(i, Y)$, $L_i < 0$, $L_Y > 0$
				\4[] Shocks exógenos dependen de varianzas de:
				\4[] $\to$ Demanda agregada
				\4[] $\to$ Demanda de dinero
				\4[] Si $\sigma^2_\text{DA} > \sigma_\text{MD}^2$:
				\4[] $\to$ Demanda agregada más volátil
				\4[] $\to$ LM creciente contribuye a suaviza
				\4[] $\then$ Óptimo: $M$ exógena, $i$ se ajusta
				\4[] $\then$ M debe ser variable intermedia
				\4[] \grafica{pooleshockda}
				\4[] Si $\sigma^2_\text{DA} < \sigma_\text{MD}^2$:
				\4[] $\to$ Demanda de dinero más volátil
				\4[] $\to$ Volatilidad de DDinero afecta a $Y$
				\4[] $\then$ Óptimo: $i$ exógena, $M$ se ajusta
				\4[] $\then$ $i$ debe ser variable intermedia
				\4[] \grafica{pooleshockmd}
				\4 Sargent y Wallace (1975)
				\4[] Basado en Poole (1970)
				\4[] Modelo ad-hoc de la economía
				\4[] $\to$ Sin microfundamentar
				\4[] $\then$ Función de pérdida ad-hoc
				\4[] $\then$ Basado en IS-LM
				\4[] Introduce:
				\4[] $\to$ Análisis dinámico
				\4[] $\to$ Expectativas racionales
				\4[] $\to$ Minimización de una función de pérdida
				\4[] Compara:
				\4[] $\to$ Regla de agregado monetario
				\4[] $\to$ Regla de interés
				\4[] Implicaciones de expectativas racionales:
				\4[] $\to$ Todas las reglas de M mismo efecto
				\4[] $\to$ Regla de $i$ pueden dejar P indeterminado
			\3 Inflation targetting, price level-targetting o PIB nominal\footnote{Basado en Bernanke y Mishkin (1997) y Hatcher y Minford (2014).}
				\4 Nuevo enfoque de implementación
				\4[] Énfasis en objetivos finales
				\4[] $\to$ No en variables intermedias
				\4[] Salvo relación directa y estable
				\4[] $\to$ Entre VIntermedia e inflación
				\4[] $\then$ Relación directa poco probable o inestable
				\4[] $\then$ VIntermedias a discreción de BCentral
				\4[] $\then$ Intermedias ajustan para objetivo final de inflación
				\4 A partir de 80s y 90s
				\4[] Implementado en numerosos países desarrollados
				\4[] Nueva Zelanda pionero
				\4[] Consolidación tras:
				\4[] $\to$ Crisis de los 90s
				\4[] $\to$ Caídas de sistemas de tipos fijos
				\4[] $\then$ TCN ya no es ancla nominal de expectativas
				\4[] Reserva Fed, BCE y muchos otros
				\4 Marco de PM, no regla fija
				\4[] Acomoda intervención discrecional
				\4[] Flexible a shocks temporales
				\4 Características básicas de inflation targetting
				\4[] (i) Se anuncia objetivo de inflación
				\4[] (ii) Voluntad explícita de inflación baja y estable
				\4[] (iii) Comunicación frecuente de planes y objetivos
				\4[] (iv) Responsabilidad y rendición de cuentas de BCentral
				\4[] $\to$ Algunos BC vinculan cumplimiento a contrato
				\4[] $\to$ Otros, vinculación implícita
				\4 Estabilización de corto plazo
				\4[] Posible a pesar de objetivo de m/p y l/p
				\4[] -- Índices de precios sin shocks de oferta
				\4[] $\to$ P.ej.: inflación subyacente
				\4[] -- Objetivos como rangos, no puntos
				\4[] -- Ajuste explícito de objetivos de c/p para estabilizar
				\4 Críticas a inflation-targetting
				\4[] ¿Qué indice de inflación tener en cuenta?
				\4[] $\to$ Problemas de medición
				\4[] $\to$ Problemas de definición
				\4[] $\to$ Sesgos de índices
				\4[] ¿Qué objetivo es óptimo?
				\4[] $\to$ Habitual por encima de 0\% pero no de 4\%
				\4[] $\to$ ¿Es siempre óptimo?
				\4[] Rigidez nominal a la baja de salarios
				\4[] $\to$ Inflación baja reduce flexibilidad de SReal
				\4[] ¿Realmente se puede controlar inflación?
				\4[] $\to$ Sujeta a lags y distorsiones
				\4[] $\to$ Justifica VIntermedia como complementario
				\4 Price-level targetting
				\4[] Objetivo no es inflación sino precios
				\4[] Desviaciones de inflación se compensan
				\4[] $\to$ Para mantener objetivo de precios
				\4[] Basado en modelos DSGE de NEK
				\4[] Compensación creíble desviaciones
				\4[] $\to$ Inflación más alta una vez pasada recesión
				\4[] $\then$ Amortigua recesiones mejor
				\4 PIB nominal como objetivo
				\4[] Propuesto como remplazo a inflation-targetting
				\4[] Crecimiento de PIB nominal tiene dos componentes
				\4[] $\to$ Inflación
				\4[] $\to$ Crecimiento del producto
				\4[] Si crecimiento del producto baja
				\4[] $\to$ Aumento de inflación compensa
				\4[] $\then$ Estímulo a PIB real
				\4[] Argumento a favor:
				\4[] $\to$ Más estabilizante de PIB que inflation-targetting
				\4[] En contra:
				\4[] $\to$ PIBNominal más difícil de entender por público
				\4[] $\to$ PIBNominal peor conocido que inflación
		\2 Debates de política monetaria óptima
			\3 Divina coincidencia
				\4 Idea clave
				\4[] Resultado normativo en modelo NEK simple
				\4[] Estabilización de precios es óptimo
				\4[] Dos fuentes de ineficiencia en modelo NEK
				\4[] -- Competencia monopolística
				\4[] $\to$ Precio superior a coste marginal
				\4[] $\to$ Producción inferior a la de óptimo
				\4[] $\then$ Solucionable con subsidio óptimo al empleo
				\4[] $\then$ Output natural es output eficiente
				\4[] -- Rigideces nominales
				\4[] $\to$ Impiden ajuste de precios nominales a empresas
				\4[] $\to$ Impiden implementación de planes óptimos de producción
				\4[] $\then$ Óptimo implica evitar toda variación de precios
				\4[] $\then$ Empresas no se desvían de planes óptimos de producción
				\4 Formulación
				\4[] Contexto modelo básico NEK
				\4[] $\to$ Sólo rigideces nominales, no reales
				\4[DIS] IS dinámica
				\4[] \fbox{$\tilde{y}_t = \textrm{E}_t \left\lbrace \tilde{y}_{t+1} \right\rbrace - \frac{1}{\sigma} \left( \underbrace{i_t - \textrm{E}_t \left\lbrace \pi_{t+1} \right\rbrace}_{r_t} - r^n_t \right) $}
				\4[NKPC] Curva de Phillips Neo-Keynesiana
				\4[] \fbox{$\pi_t = \text{E}_t \left\lbrace \pi_{t+1} \right\rbrace + \textsc{k} \tilde{y}_t $}
				\4[WS] Mercado de trabajo
				\4[] $w_t - p_t = \sigma c_t + \phi n_t$
				\4[MP] Mercado de dinero
				\4[] \fbox{$m_t - p_t = y_t - \eta i_t$}
				\4[TR] Regla de Taylor simple
				\4[] \fbox{$i_t = \rho + \phi_\pi \pi_t + \phi_y \tilde{y}_t + v_t $}
				\4 Diferencia constante entre:
				\4[] -- Output natural
				\4[] -- Output óptimo
				\4[] $\to$ Output natural es first-best
				\4[] $\then$ Output gap nulo es óptimo
				\4[] $\then$ Output debe moverse con output natural
				\4 Costes de inflación por rigidez de precios
				\4[] Mark-up se desvía de óptimo
				\4[] $\to$ Para las que no ajustan precios
				\4[] Precios relativos de bienes se distorsionan
				\4[] $\to$ Algunas empresas cambian, otras no
				\4 Implicaciones
				\4[] Inflación + rigidez nominal
				\4[] $\to$ Desvía economía de eq. con precios flexibles
				\4[] $\to$ Firmas quieren cambiar precios pero no pueden
				\4[] $\to$ $\pi_t = 0$ $\iff$ $y_t = y_t^n$
				\4[] $\then$ Óptimo si y solo si inflación nula
				\4[] $\to$ $\pi_t = 0$
				\4[] $\to$ $i_t = r_t^n$
				\4[] $\then$ \fbox{Estabilidad de precios $\iff$ output eficiente}
			\3 Ruptura de la divina coincidencia
				\4 Idea clave
				\4[] Subsidio óptimo imposible de implementar
				\4[] $\to$ Output óptimo mayor a output natural
				\4[] $\then$ Mantener output óptimo implica inflación
				\4[] $\then$ Estabilidad de precios ya no es PM óptima
				\4[] Trade-off output vs inflación
				\4[] $\to$ El que enfrentan todos los bancos centrales
				\4[] $\to$ Diferentes formas de resolverlo
				\4[] $\to$ Modelo subyacente asumido determina
				\4[] $\then$ Minimización de función de pérdida inflación-output
				\4 Imperfecciones reales
				\4[] Eq. de precios flexibles
				\4[] $\to$ Ya no es eficiente
				\4[] $\to$ $y_t^n \neq y_t^e$
				\4[] $\to$ múltiples fuentes posibles de rigidez real
				\4[NKPC'] CPhillips con output eficiente $\neq$ natural
				\4[] $\pi_t = \beta E\left\lbrace \pi_{t+1} \right\rbrace + \textsc{k} \underbrace{(y_t - y_t^e)}_{x_t}+ \textsc{k} \underbrace{(y_t^e - y_t^n)}_{u_t}$
				\4[] Donde:
				\4[] $\to$ $u_t$: cost-push shocks
				\4[] $\to$ $x_t$: output gap relevante a efectos de bienestar
				\4[] Inflación sigue dependiendo de output gap $y_t - y_t^n$
				\4[] $\to$ Pero separamos $x_t$ y shocks cost-push
				\4[] $\to$ Importante para PM es $x_t \equiv y_t - y_t^e$
				\4[] $\to$ Pero desviaciones de $y_t^n$ aumentan inflación
				\4 Trade-off inflación vs output eficiente
				\4[] PM debe minimizar función de pérdida
				\4[] $\to$ Inflación
				\4[] $\to$ Desviación respecto output eficiente
				\4[] $E_0 \sum_{t=0}^\infty \left( \beta^t \pi_t^2 + \lambda x_t^2) \right)$
				\4[] Sujeto a NKPC'
				\4[] Si output natural es eficiente:
				\4[] $\to$ Minimizar inflación minimiza $x_t$
				\4[] Si output natural es menor que eficiente:
				\4[] $\to$ Minimizar $x_t$ implica inflación
				\4[] $\then$ Trade-off
				\4[] $\then$ PM óptima implica ponderar ambas
				\4[] $\to$ ¿PM óptima discrecional o reglada?
			\3 Discrecionalidad frente a reglas de PM
				\4 Kydland y Prescott (1977)
				\4[] Análisis de consistencia temporal de políticas
				\4[] $\to$ Aplicación pionera a trade-off empleo--inflación
				\4[] Política consistente:
				\4[] $\to$ En cada periodo, pasado está dado
				\4[] $\to$ Acepta que pasado ya no se puede cambiar
				\4[] Política óptima:
				\4[] $\to$ Optimiza globalmente considerando reacciones
				\4[] $\to$ No hay pasado exógeno
				\4[] Problema de la inconsistencia:
				\4[] Plan óptimo a priori
				\4[] $\to$ Ya no es óptimo cuando tiempo avanza
				\4[] $\to$ Cuando tiempo avanza, es óptimo desviarse
				\4[] $\then$ Planes óptimos son inconsistentes
				\4[] $\then$ Agentes saben que óptimo hoy no lo será mañana
				\4[] $\then$ Falta de compromiso lleva a subóptimos
				\4[] Discrecionalidad
				\4[] $\to$ Capacidad para cambiar planes
				\4[] $\to$ Agentes conocen disposición a cambiar planes
				\4[] $\to$ Cambio de planes implica resultados subóptimos
				\4[] $\then$ Resultados subóptimos si no hay forma de ``obligarse''
				\4[] Reglas
				\4[] $\to$ Sujección a planes definidos
				\4[] $\to$ Potencial para alcanzar óptimos
				\4[] $\to$ Necesario definir reglas óptimas
				\4[] \grafica{kydlandprescott1977}
				\4 Barro y Gordon (1983a) y (1983b)
				\4[] Desarrollan Kydland y Prescott (1977)
				\4[] Se centran en trade-off empleo-inflación
				\4[] Análisis positivo
				\4[] $\to$ resultados de política discrecional
				\4[] BCentral minimizador de inflación--desempleo
				\4[] $\to$ Puede crear sorpresas de inflación
				\4[] $\then$ Sorpresas reducen desempleo
				\4[] $\to$ Si poca inflación esperada, fuerte efecto
				\4[] $\then$ Incentivo para aumentar inflación
				\4[] Agentes con HER conocen incentivos
				\4[] $\to$ Saben que plan óptimo no es consistente sin commitment
				\4[] $\then$ Esperan inflación alta en el futuro
				\4[] $\then$ Resultado subóptimo
				\4[] i. Crecimiento excesivo de inflación y M
				\4[] ii. Pendiente de curva de Phillips es importante
				\4[] iii. Autoridad monetaria actúa contracíclicamente
				\4[] iv. Desempleo acaba siendo independiente de PM
				\4[] Compromiso con regla de PM
				\4[] $\to$ Afecta positivamente a resultados
				\4[] Prestigio de BCentral
				\4[] $\to$ Potencial para saltarse commitment
				\4[] Descuento subjetivo alto
				\4[] $\to$ Dificulta uso de prestigio
				\4 Consenso actual
				\4[] Reglas son preferibles a discrecionalidad
				\4[] $\to$ Cumplimiento de reglas debe ser creíble
				\4[] En situaciones excepcionales
				\4[] $\to$ Aceptables decisiones discrecionales
				\4[] Prestigio de instituciones es muy importante para que:
				\4[] $\to$ Discrecionalidad excepcional sea efectiva
				\4[] $\to$ Efectividad de PM se mantenga en futuro
				\4[] $\to$ Abaratar la señalización del commitment
			\3 Independencia del banco central\footnote{Palgrave: ``central bank independence'' por Walsh, C. E.}
				\4 Tipos de independencia
				\4[] Independencia de objetivos
				\4[] $\to$ Fijan libremente su política
				\4[] Independencia de instrumentos
				\4[] $\to$ Gobierno/PLegislativo fija objetivos
				\4[] $\to$ Fijan libremente como alcanzar objetivos
				\4[$\to$] Diversas maneras de ponderar independencia
				\4 Análisis teórico
				\4[] Sesgo inflacionario de AMonetaria discrecional
				\4[] $\to$ Quiere reducir paro por debajo de natural
				\4[] Agentes con HER tienen en cuenta incentivos
				\4[] $\to$ Equilibrio con sesgo inflacionario
				\4 Elección de miembros de Banco Central
				\4[] Tiene importancia decisiva en PM
				\4[] $\to$ ¿Estabilidad de precios o output?
				\4[] $\to$ ¿Inflación a cambio de más output?
				\4[] ``Hawks'' y ``doves''
				\4 Rogoff (1985)
				\4[] Comparación entre objetivos BC y sociedad
				\4[] ¿Deben tener mismos objetivos?
				\4[] ¿BC debe preocuparse más por inflación?
				\4[] Marco de formulación
				\4[] $\to$ IS-LM dinámico
				\4[] $\to$ HER
				\4[] $\to$ PM efectiva a c/p por salarios rígidos
				\4[] Resultados
				\4[] $\to$ BC más hawk puede ser mejor que BC con FBS de sociedad
				\4[] $\to$ BC más hawk implica más volatilidad de output
				\4[] $\to$ Óptimo es más hawk que sociedad pero ``hawkness'' finita
				\4[] $\to$ Más hawks en BC puede ser óptimo
				\4 Evidencia empírica
				\4[] Correlación robusta independencia--baja inflación
				\4[] Independencia no desestabiliza output
				\4[] $\to$ Alesina y Summers (1993)
				\4[] $\to$ No parece haber trade-off
				\4 Críticas
				\4[] Independencia excesiva evita rendición de cuentas
				\4[] Inflación baja y estable no es único objetivo
				\4[] $\to$ Otros objetivos en PE
			\3 Política monetaria en la ZLB
				\4 Asumiendo divina coincidencia s.p.g.
				\4[] $y_t^n = y_t^e \then x_t \equiv y_t - y_t^e = \tilde{y}_t \equiv y_t - y_t^n$
				\4 Óptimo implica:
				\4[] $\to$ Inflación nula
				\4[] $\to$ Interés nominal igual a real natural $\to$ $i_t = r_t^n$
				\4 Shock negativo a interés real natural
				\4[] Cae por debajo de 0
				\4[] $\to$ Interés nominal por debajo de 0
				\4[] $\then$ Equilibrio óptimo
				\4[] ¿Qué sucede si restricción tal que $i_t \geq 0$?
				\4[] $\to$ Zero-lower bound
				\4[] $\then$ Óptimo de first-best inalcanzable
				\4[] $\then$ Necesario encontrar óptimos de second-best
				\4 Regla con commitment vs discrecionalidad
				\4[] Prometer inflación futura más allá de shock
				\4[] $\to$ Aumento del output gap futuro
				\4[] $\to$ Caída interés real futuro
				\4[] Si promesa es creíble por commitment
				\4[] $\to$ Agentes estiman output futuro más alto
				\4[] $\then$ Output aumenta en el presente
				\4[] $\then$ Impacto de shock se reduce
				\4[] $\then$ Óptimo de second best con commitment
				\4[] $\then$ Fundamento teórico de forward-guidance
				\4[] Sin commitment
				\4[] $\to$ Compromiso no es creíble
				\4[] $\to$ Agentes no anticipan inflación futura
				\4[] $\then$ Resultado subóptimo
	\1[] \marcar{Conclusión}
		\2 Recapitulación
			\3 Instrumentación de la PM
			\3 Diseño de la PM
		\2 Idea final
			\3 Lecciones de diseño de PM no convencional hasta ahora
				\4[I] En lo posible, aplicar en base a reglas
				\4[] No es nuevo de PM no convencional
				\4[II] Priorizar cambios en balance
				\4[] No expansión del balance
				\4[] Aumentar oferta monetaria
				\4[] $\to$ Objetivo independiente en sí mismo
				\4[III] Forward guidance y QE no son sustitutos
				\4[] Operan en diferentes canales
				\4[] Tienen diferentes efectos
				\4[] Forward guidance
				\4[] $\to$ Expectativas de tipos futuros
				\4[] Quantitative easing
				\4[] $\to$ Reasignación de carteras y tipos
				\4[IV] Compra de activos aumenta exposición de BCentral
				\4[] Riesgo de duración aumenta fuertemente para BCentral
				\4[] Si suben los tipos de interés
				\4[] $\to$ Balance del BC se deteriora
				\4[] $\then$ Contribuyentes asumen pérdidas
				\4[] Puede aumentar compromiso con expansión monetaria
				\4[V] Elección de activos depende de contexto
				\4[] Valorar segmentación de mercados
				\4[] Compra de activos aparte de deuda pública
				\4[] $\to$ En ocasiones planteada compra de equity
				\4[] $\to$ Problemas legales y estatutarios
				\4[] $\to$ ¿Puede BCentral asignar crédito a sector privado?
				\4[] $\to$ ¿Sector público participando equity es capitalismo/libre mercado?
				\4[VI] Balance de BCentrales es instrumento de PM
				\4[] Se consolida como instrumento disponible
				\4[] Puede permitir aumentar número de objetivos
				\4[] Necesario investigar más
			\3 Efectos y mecanismos de transmisión de la PM
				\4 Hemos asumido efectos y mecanismos conocidos
				\4[] No son cuestiones triviales
				\4 Sujetos a evolución
				\4[] Economías y SFinanciero cambian constantemente
				\4[] Teoría económica caracteriza nuevos mecanismos
				\4 Sujetos a controversia
				\4[] Qué efectos tiene PM
				\4[] Qué mecanismos de transmisión de PM
				\4 Análisis empírico
				\4[] Muy difícil
				\4[] $\to$ Difícil/imposible construcción contrafactuales
			\3 Arte y ciencia
				\4 Blinder: central banker y académico
				\4 Cita famosa: PM es arte y ciencia
				\4[] << Habiendo trabajado en ambos lados
				\4[] $\to$ Policy making
				\4[] $\to$ Academia
				\4[] Puedo afirmar que PM es arte
				\4[] $\to$ Pero ciencia muy útil para practicar arte >>
\end{esquemal}























\graficas

\begin{axis}{4}{Poole (1970): comparación de interés y oferta monetaria como variables intermedias cuando la demanda agregada es más volátil que la demanda de dinero.}{}{$i$}{pooleshockda}
	% Extensión de eje de abscisas
	\draw[-] (4,0) -- (6,0);
	\node[below] at (5.5,0){$Y$};
	
	% IS A
	\draw[-] (0.5,4) -- (4,0.5);
	\node[above] at (0.5,4){\tiny $\text{IS}_A$};
	
	% IS B
	\draw[-] (2,4) -- (5.5,0.5);
	\node[above] at (2,4){\tiny $\text{IS}_B$};
	
	% LM
	\draw[-, color=red] (1.5,0.5) -- (5,4);
	\node[above] at (5,4){\tiny $\text{LM}$};
	
	% Interés nominal fijo
	\draw[-, color=red] (0,1.75) -- (6,1.75);
	\node[right] at (6,1.75){\tiny $i$};
	
	% Intersección IS-A con LM e i fijo
	\draw[dashed] (2.75,1.75) -- (2.75,0);
	\node[below] at (2.75,0){ \tiny $Y_I$};

	% Intersección IS-B con LM e i fijo
	\draw[dashed] (3.5,2.5) -- (3.5,0);
	\node[below] at (3.5,0){ \tiny $Y_M$};
	
	% Intersección IS-B con i fijo
	\draw[dashed] (4.25,1.75) -- (4.25,0);
	\node[below] at (4.25,0){ \tiny $Y_I'$};
	
	% Volatilidad de output con i fijo
	\draw[decorate,decoration={brace, mirror,amplitude=3pt},xshift=0pt, yshift=-0cm] (2.75,-0.5) -- (3.5,-0.5) node[black,midway,xshift=0pt, yshift=-10pt] {\tiny M fijo};
	
	% Volatilidad de output con M fijo
	\draw[decorate,decoration={brace, mirror,amplitude=3pt},xshift=0pt, yshift=-0cm] (2.75,-1) -- (4.25,-1) node[black,midway,xshift=0pt, yshift=-10pt] {\tiny $i$ fijo};
\end{axis}

Se aprecia en el gráfico como la utilización del interés como variable intermedia cuyo valor se establece fijo induce una volatilidad mayor que cuando la variable intermedia fijada es la oferta monetaria. Este resultado depende crucialmente de la pendiente creciente de la curva LM: desplazamientos de la curva IS se ven amortiguados por el efecto de un aumento del interés.

\begin{axis}{4}{Poole (1970): comparación de interés y oferta monetaria como variables intermedias cuando la demanda de dinero es más volátil que la demanda agregada.}{}{$i$}{pooleshockmd}
	% Extensión de eje de abscisas
	\draw[-] (4,0) -- (6,0);
	\node[below] at (5.5,0){$Y$};
	
	% IS 
	\draw[-] (0.5,4) -- (4,0.5);
	\node[above] at (0.5,4){\tiny $\text{IS}_A$};
	
	% LM A
	\draw[-, color=red] (0.5,0.5) -- (4,4);
	\node[above] at (4,4){\tiny $\text{LM}_A$};

	% LM B
	\draw[-, color=red] (2,0.5) -- (5.5,4);
	\node[above] at (5.5,4){\tiny $\text{LM}_B$};
	
	% Interés nominal fijo
	\draw[-, color=red] (0,1.82) -- (6,1.82);
	\node[right] at (6,1.82){\tiny $i$};
	
	% Intersección IS con i fijo
	\draw[dashed] (2.68,1.82) -- (2.68,0);
	\node[below] at (2.68,0){ \tiny $Y_I$};
	
	% Intersección IS con LM A
	\draw[dashed] (2.25,2.25) -- (2.25,0);
	\node[below] at (2.25,0){ \tiny $Y_M$};
	
	% Intersección IS con LM B
	\draw[dashed] (3,1.5) -- (3,0);
	\node[below] at (3,0){ \tiny $Y_M'$};
	
	% Volatilidad de output con i fijo
	%\draw[decorate,decoration={brace, mirror,amplitude=3pt},xshift=0pt, yshift=-0cm] (2.75,-0.5) -- (3.5,-0.5) node[black,midway,xshift=0pt, yshift=-10pt] {\tiny M fijo};
	
	% Volatilidad de output con M fijo
	\draw[decorate,decoration={brace, mirror,amplitude=3pt},xshift=0pt, yshift=-0cm] (2.25,-0.5) -- (3,-0.5) node[black,midway,xshift=0pt, yshift=-10pt] {\tiny $M$ fijo};
	
\end{axis}

Cuando los shocks afectan a la demanda de dinero, el resultado es contrario al del gráfico anterior. La volatilidad del output es máxima cuando la oferta monetaria es la variable intermedia. Cuando se establece el interés como variable intermedia, la volatilidad del output se reduce al mínimo porque el punto de la recta IS es constante.

\begin{axis}{4}{Kydland y Prescott (1977): inconsistencia de la política monetaria óptima y sesgo inflacionario resultante.}{$u_t - u^*$}{$\pi_t$}{kydlandprescott1977}
	% Extensión de ejes
	\draw[-] (-3,0) -- (0,0); % abscisas
	\draw[-] (0,0) -- (0,-3); % ordenadas

	% Curvas de Phillips
	\draw[-] (-3,4) -- (3,-4);
	\draw[-] (-3,5.7) -- (3,-2.3);
	
	% Curvas de indiferencia de función de pérdida
	\draw[-] (-3,3) to [out=-20, in=90](0,0) to [out=270, in=20](-3,-3);
	\draw[-] (-3,3.53) to [out=-20, in=90](0.53,0) to [out=270, in=20](-3,-3.53);
	
	% Óptimo
	\node[circle,fill=black,inner sep=0pt,minimum size=4pt] (a) at (0,0) {};	
	\node[above] at (-0.45,0){O};
	
	% Equilibrio
	\node[circle,fill=black,inner sep=0pt,minimum size=4pt] (a) at (0,1.8) {};
	\node[right] at (0,1.8){B};
	
\end{axis}

El punto A muestra el óptimo alcanzable en presencia de commitment. En ausencia de commitment, el equilibrio es el punto B, en el que el desempleo es el mismo que en el óptimo pero la inflación es mayor.

\preguntas

\seccion{Test 2017}
\textbf{19.} ¿Cuál de las siguientes afirmaciones es correcta?

\begin{itemize}
	\item[a] Los depósitos bancarios son títulos de deuda (IOUs) de los bancos comerciales con los hogares y las empresas y constituyen uno de los componentes de la base monetaria.
	\item[b] Las reservas bancarias son títulos de deuda (IOUs) del banco central con los bancos comerciales y no son uno de los componentes de la base monetaria.
	\item[c] El dinero en efectivo es un título de deuda (IOUs) del banco central con hogares, empresas y bancos comerciales y un componente de la base monetaria.
	\item[d] Todas las anteriores.
\end{itemize}

\textbf{20.} Según el modelo convencional de política monetaria:

\begin{itemize}
	\item[a] La política monetaria es neutral en el corto y en el largo plazo, es decir, sólo afecta al nivel de precios pero no a la actividad económica.
	\item[b] La tasa de inflación está determinada únicamente por la respuesta de la política monetaria a las perturbaciones de oferta.
	\item[c] Dado que existen rigideces nominales (es decir, los precios no se ajustan instantáneamente a cambios en los tipos de interés nominales), existe una diferencia entre el nivel de producción y su valor de equilibrio que solo depende de las decisiones del banco central.
	\item[d] Ninguna de las anteriores.
\end{itemize}

\seccion{Test 2008}

\textbf{21.} Siendo la delegación de la política monetaria a un banquero central conservador una de las soluciones viables al problema de la inconsistencia dinámica (Rogoff, 1985), es \underline{falso} que:

\begin{itemize}
	\item[a] La delegación es siempre una solución pareto superior a la no delegación, aunque el banquero central no comparta los objetivos de los agentes económicos en términos de inflación y producción deseada.
	\item[b] El banquero central deberá ser más averso a la inflación que los agentes económicos para que la tasa de inflación dinámicamente consistente sea inferior.
	\item[c] Si el banquero central es más averso a la inflación que los agentes económicos, la delegación tiene peor resultado que la no delegación en lo referido a la estabilización del output.
	\item[d] Si el banquero central es más averso a la inflación que los agentes económicos, la delegación tiene mejor resultado que la no delegación en lo referido a la inflación media.
\end{itemize}


\seccion{Test 2007}
\textbf{20.} La aplicación del problema de la inconsistencia dinámica a la política monetaria sugiere que, sin alguna forma de asegurar un compromiso (commitment):
\begin{itemize}
	\item[a] La tasa de inflación será más alta de lo que sería en presencia de un compromiso.
	\item[b] El nivel de producción será inferior de lo que sería en presencia de un compromiso.
	\item[c] La tasa de inflación será mayor y el nivel de producción inferior de lo que serían en presencia de un compromiso.
	\item[d] La tasa de inflación y el nivel de producción serán superiores a lo que serían en presencia de un compromiso.
\end{itemize}

\notas

\textbf{2017:} \textbf{19.} C \textbf{20.} D

\textbf{2008:} \textbf{21.} A

\textbf{2007:} \textbf{20.} A


\bibliografia


Mirar en Palgrave:

\begin{itemize}
	\item bank rate
	\item banking crises
	\item banking industry
	\item capital, credit and money markets *
	\item credit
	\item Credit Crunch chronology: april 2007-september 2009
	\item central bank communication
	\item \textbf{central banking} *
	\item central bank independence
	\item cheap money
	\item dear money
	\item demand for money: empirical studies
	\item demand for money: theoretical studies
	\item endogenous and exogenous money
	\item European Central Bank and Monetary Policy in the Euro Area
	\item European Monetary Integration
	\item European Monetary Union
	\item Euro Zone Crisis 2010
	\item fiat money
	\item financial intermediaries *
	\item financial intermediation *
	\item Friedman, Milton
	\item high-powered money and the monetary base *
	\item inflation
	\item inflation and growth
	\item inflation dynamics
	\item \textbf{inflation targeting} *	
	\item inflationary gap
	\item inside and outside money
	\item international monetary policy
	\item \textbf{liquidity trap} *
	\item monetarism
	\item monetary aggregation
	\item monetary and fiscal policy overview
	\item \textbf{monetary business cycle models (sticky prices and wages})
	\item monetary business cycle (imperfect information)
	\item monetary disequilibrium and market clearing
	\item monetary economics, history of
	\item monetary equilibrium
	\item monetary overhang
	\item \textbf{monetary policy}
	\item \textbf{monetary policy, history of}
	\item \textbf{monetary transmission mechanism}
	\item money
	\item money and general equilibrium
	\item money illusion
	\item money in economic activity
	\item money supply
	\item money, classical theory of
	\item neutrality of money
	\item \textbf{optimal fiscal and monetary policy (with commitment)}
	\item \textbf{optimal fiscal and monetary policy (without commitment)}
	\item optimum quantity of money
	\item \textbf{quantitative easing by the major western central banks during the global financial crisis}
	\item quantity theory of money
	\item search-and-matching models of monetary exchange
	\item sound money
	\item targets and instruments
	\item \textbf{Taylor rules}
	\item tight money
	\item time consistency of monetary and fiscal policy
	\item wages, real and money
\end{itemize}

Arrow, K. et al \textit{Handbook of Monetary Economics} (2011) En carpeta de Macroeconomía

Barro, R.; Gordon, D. \textit{A Positive Theory of Monetary Policy in a Natural Rate Model} (1983) Journal of Political Economy -- En carpeta del tema

Barro, R.; Gordon, D. \textit{Rules, discretion and reputation in a model of monetary policy} (1983) Journal of Monetary Economics -- En carpeta del tema

Bernanke, B. S.; Mishkin, F. S. \textit{Inflation Targeting: A New Framework for Monetary Policy?} (1997) Journal of Economic Perspectives -- En carpeta del tema

Bordo, M. D.; Cochrane, J. H.; Seru, A. \textit{The Structural Foundations of Monetary Policy} \url{https://www.hoover.org/research/structural-foundations-of-monetary-policy} -- En carpeta macro

Cecchetti, S.; Schoenholtz, K. \textit{Money, Banking, and Financial Markets} (2014) Fourth Edition -- En carpeta Finanzas

Clarida, R.; Galí, J.; Gertler, M. \textit{The Science of Monetary Policy: A New Keynesian Perspective} (1999) Journal of Economic Literature -- En carpeta del tema

Dell'Ariccia, G.; Rabanal, P.; Sandri, D. \textit{Unconventional Monetary Policies in the Euro Area, Japan, and the United Kingdom} (2018) Journal of Economic Perspectives, Fall -- En carpeta del tema

European Central Bank \textit{Manual of MFI Balance Sheet Statistics} (2012) -- En carpeta Contabilidad Nacional

European Parliament. \textit{Monetary policy of the European Central Bank} (2015) -- En carpeta del tema

Fabozzi, F. J. \textit{Handbook of Fixed Income Securities} Ch. 5(Macro-Economic dynamics and the corporate bond market)

Friedman, B. M. \textit{Targets and Instruments of Monetary Policy} (1990) Handbook of Monetary Economics. Ch. 16 -- En carpeta del tema

Fuhrer, J.; Kodrzycki, Y.; Little, J. S.; Olivei, G. \textit{Understanding Inflation and Implications for Monetary Policy: A Phillips Curve Retrospective} (2009) MIT Press -- En carpeta del tema

Goodhard, C.; Jensen, M. (2015) \textit{Currency school versus Banking School: an ongoing confrontation} Economic Thought, 4 (2) pp. 20-31 \href{http://eprints.lse.ac.uk/64068/1/Currency\%20School\%20versus\%20Banking\%20School.pdf}{Enlace} -- En carpeta del tema

Gomes, S.; Jacquinot, P.; Pisani, M. (2010) \textit{The EAGLE. A model for policy analysis of macroeconomic interdependence in the Euro Area} ECB Working Paper-- En carpeta del tema

Hatcher, M.; Minford, P. \textit{Inflation targeting vs price-level targeting: a new survey of theory and empirics} (2014) VoxEU -- \url{https://voxeu.org/article/inflation-targeting-vs-price-level-targeting}

Kuttner, K. N. \textit{Outside the Box: Unconventional Monetary Policy in the Great Recession and Beyond} (2018) Journal of Economic Perspectives, Fall 2018 -- En carpeta del tema

Pfister, C.; Sahuc, J-G. (2020) \textit{ Unconventional Monetary Policies: A Stock-Taking Exercise} Banque de France. Working Paper. \href{https://publications.banque-france.fr/sites/default/files/medias/documents/wp761.pdf}{Disponible aquí} -- En carpeta del tema.

Poole, W. \textit{Optimal Choice of Monetary Policy Instruments in a Simple Stochastic Macro Model} (1970) The Quarterly Journal of Economics -- En carpeta del tema

Rogoff, K. \textit{The Optimal Degree of Commitment to an Intermediate Monetary Target} (1985) The Quarterly Journal of Economics -- En carpeta del tema

Sumner, S. (2013) \textit{A market-driven nominal GDP targeting regime} Mercatus Research --  En carpeta del tema

Veronesi, P. \textit{Handbook of Fixed Income Securities} (2016) Ch. 5 Bond Markets and Monetary Policy. Ch. 6 Bond Markets and Unconventional Monetary Policy

\end{document}
