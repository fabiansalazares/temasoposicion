\documentclass{nuevotema}

\tema{3B-18}
\titulo{Teorías explicativas de las crisis monetarias y financieras internacionales.}

\begin{document}

\ideaclave

\seccion{Preguntas clave}

\begin{itemize}
	\item ¿Por qué se producen las crisis financieras?
	\item ¿Qué modelos las explican?
	\item ¿Qué implicaciones de política económica se derivan?
	\item ¿Pueden preveerse?
	\item ¿En qué consiste el contagio financiero?
\end{itemize}

AÑADIR SECCIÓN SOBRE TEORÍAS DEL ACELERADOR FINANCIERO. Ver:

\href{https://www.nber.org/papers/w6455.pdf}{Bernanke, Gertler, Gilchrist (1998)}


VER ARTÍCULO DE BDE 2019 SOBRE MODELOS DE ALERTA TEMPRANA -- EN CARPETA DEL TEMA

Releer \textit{currency crisis models} de Palgrave por Burnside, Eichenbaum y Rebelo.

\esquemacorto

\begin{esquema}[enumerate]
	\1[] \marcar{Introducción}
		\2 Contextualización
			\3 Macroeconomía
			\3 Sistema monetario internacional
			\3 Crisis monetarias y financieras internacionales
		\2 Objeto
			\3 ¿Por qué se producen las crisis financieras?
			\3 ¿Qué modelos las explican?
			\3 ¿Qué implicaciones se derivan de los modelos?
			\3 ¿Pueden preverse las crisis financieras?
			\3 ¿En qué consiste el contagio financiero?
		\2 Estructura
			\3 Modelos de crisis de primera generación
			\3 Modelos de crisis de segunda generación
			\3 Modelos de crisis de tercera generación
			\3 Otros modelos de crisis
			\3 Contagio
			\3 Predicción
	\1 \marcar{Modelos de primera generación}
		\2 Idea clave
			\3 Contexto
			\3 Objetivo
			\3 Resultados
		\2 Formulación
			\3 Supuestos
			\3 Equilibrio monetario
			\3 Tipo de cambio sombra
			\3 Ataque especulativo
			\3 Representación gráfica
		\2 Implicaciones
			\3 Objetivos inconsistentes
			\3 Momento del ataque
			\3 Determinantes de la sostenibilidad
		\2 Valoración
			\3 Capacidad explicativa
			\3 Críticas
			\3 Extensiones
	\1 \marcar{Crisis de segunda generación}
		\2 Idea clave
			\3 Contexto
			\3 Objetivo
			\3 Resultados
		\2 Formulación
			\3 Supuestos
			\3 Shock exógeno con expectativas bajas de inflación
			\3 Shock exógeno negativo con expectativas altas de inflación
		\2 Implicaciones
			\3 Devaluación asegurada si déficit elevado
			\3 Tipo fijo creíble
			\3 Equilibrios múltiples
		\2 Valoración
			\3 Capacidad explicativa
			\3 Críticas
			\3 Extensiones y variaciones
	\1 \marcar{Crisis de tercera generación}
		\2 Idea clave
			\3 Contexto
			\3 Objetivo
			\3 Resultados
		\2 Formulación
			\3 Diferentes mecanismos
			\3 Riesgo moral
			\3 Fragilidad financiera
			\3 Balance comercial deficitario
			\3 Hot money
		\2 Implicaciones
			\3 Regulación financiera para evitar
			\3 Intervención del FMI
			\3 Controles de capital
		\2 Valoración
			\3 Capacidad explicativa
			\3 Críticas
			\3 Extensiones
	\1 \marcar{Otros modelos de crisis}
		\2 Crisis de cuarta generación
			\3 Idea clave
			\3 Implicaciones
			\3 Valoración
		\2 Crisis bancarias
			\3 Idea clave
			\3 Formulación
			\3 Implicaciones
			\3 Valoración
		\2 Crisis de Minsky
			\3 Idea clave
			\3 Implicaciones
			\3 Valoración
		\2 La Gran Recesión
			\3 Idea clave
			\3 Causas
			\3 Implicaciones
		\2 Crisis en países en desarrollo
			\3 Idea clave
			\3 Causas
			\3 Implicaciones
		\2 Acelerador financiero
			\3 Idea clave
			\3 Formulación
			\3 Implicaciones
			\3 Valoración
	\1 \marcar{Contagio}
		\2 Idea clave
			\3 Concepto
			\3 Enfoques de estudio
		\2 Canales de contagio
			\3 Comercio
			\3 Bancos
			\3 Inversión en cartera
			\3 Wake-up calls
	\1 \marcar{Predicción de crisis}
		\2 Idea clave
			\3 Concepto
			\3 Objetivos
			\3 Resultado
		\2 Modelos
			\3 Indicadores
			\3 Expectativas de especuladores
			\3 Early Warning Systems
			\3 MIP Scoreboard
			\3 FSSA y FSAP
	\1[] \marcar{Conclusión}
		\2 Recapitulación
			\3 Modelos de 1ª generación
			\3 Modelos de 2ª generación
			\3 Modelos de 3ª generación
			\3 Otros modelos de crisis
			\3 Contagio
			\3 Predicción
		\2 Idea final
			\3 Lecciones principales
			\3 ¿Las crisis son necesariamente malas?
			\3 Regímenes cambiarios
			\3 Instituciones internacionales y coordinación

\end{esquema}

\esquemalargo












\begin{esquemal}
	\1[] \marcar{Introducción}
		\2 Contextualización
			\3 Macroeconomía
				\4 Análisis de fenómenos económicos a gran escala
				\4 Énfasis sobre variables agregadas
			\3 Sistema monetario internacional
				\4 Conjunto de interrelaciones
				\4[] Económicas y financieras
				\4[] Institucionales y legales
				\4[] $\to$ Entre diferentes economías
				\4 Sistema dinámico
				\4[] Proceso de evolución constante
				\4[] Dinámicas endógenas y exógenas
				\4 Mercados financieros internacionales
				\4[] Canal de transmisión
				\4[] $\to$ Obligaciones y derechos internacionales
			\3 Crisis monetarias y financieras internacionales
				\4 Crisis financiera
				\4[] Venta masiva de activos en un mercado financiero
				\4 Crisis monetaria internacional
				\4[] Venta masiva de una divisa
				\4[] $\to$ Compra de otra divisa considerada segura
				\4 Consecuencias:
				\4[] $\to$ Devaluaciones y depreciaciones bruscas
				\4[] $\to$ Desaparición de crédito
				\4[] $\to$ Contagio a otras economías
				\4[] $\then$ Costes de ajuste en economía real
				\4[] $\then$ Recesión
				\4[] $\then$ Desempleo
				\4 Frecuencia de crisis
				\4[] Aumenta tras caída de BW
				\4[] $\to$ Aunque presentes desde antigüedad
		\2 Objeto
			\3 ¿Por qué se producen las crisis financieras?
			\3 ¿Qué modelos las explican?
			\3 ¿Qué implicaciones se derivan de los modelos?
			\3 ¿Pueden preverse las crisis financieras?
			\3 ¿En qué consiste el contagio financiero?
		\2 Estructura
			\3 Modelos de crisis de primera generación
			\3 Modelos de crisis de segunda generación
			\3 Modelos de crisis de tercera generación
			\3 Otros modelos de crisis
			\3 Contagio
			\3 Predicción
	\1 \marcar{Modelos de primera generación}
		\2 Idea clave
			\3 Contexto
				\4 Caída de Bretton Woods en 70s
				\4[] Excesos fiscales en USA
				\4 Tipos fijos predominan
				\4[] Al menos, como objetivo de muchos BC
				\4 Déficit fiscal
				\4[] Capacidad tributaria poco desarrollada
				\4[] Gasto público excesivo
				\4[] Tensiones políticas
				\4[] $\then$ Financiación vía señoreaje
				\4 Movimientos de capital
				\4[] Crecen fuertemente en 60 y 70
				\4[] Controles de capital más difíciles de implementar
				\4 Crisis monetarias en Latinoamérica
				\4[] Años 60 y 70
				\4[] México 1973-1982
				\4[] Argentina 1978-1981
				\4[] Devaluaciones en Latinoamérica en 1982
				\4 Ataques especulativos
				\4[] Ventas masivas de moneda nacional
				\4[] Desestabilizan tipo de cambio
				\4 Trabajos principales
				\4[] Henderson y Salant (1978)
				\4[] Krugman (1979)
				\4[] Flood y Garber (1984)
			\3 Objetivo
				\4 Caracterizar abandono brusco de TCFijo
				\4 Explicar causas de insostenibilidad de TCFijo
				\4 Caracterizar momento de abandono de TCFijo
			\3 Resultados
				\4 Crisis resulta de objetivos inconsistentes
				\4 Tipo de cambio sombra determina crisis
				\4[] TCSombra es TC que prevalecería si TCFlexible
				\4 Déficit excesivo causa abandono de TC
				\4 Con HER, crisis cuando TCSombra igual a TCFijo
		\2 Formulación
			\3 Supuestos
				\4 Demanda de dinero
				\4[] Forma logarítmica
				\4[] Asumimos $y$ constante y normalizada a 1
				\4[] \fbox{$m^d \equiv p -\lambda i$}
				\4 Oferta de dinero
				\4[] \fbox{$m^s \equiv d + r$}
				\4[] $\to$ $d$: crédito doméstico $\to$ deuda pública
				\4[] $\to$ $r$: reservas
				\4 UIP
				\4[] \fbox{$\dot{s} = i - i^*$}
				\4 PPA
				\4[] \fbox{$s= p-p^*$}
				\4[] Asumiendo:
				\4[] $\to$ $p^*$ constante
				\4[] $\to$ $i = i^*$ por apertura de CF
				\4[] $\then$ $s=p$
				\4[] $\then$ Tipo de cambio depende de precios domésticos
				\4 Tipo de cambio fijo
				\4[] BCentral se compromete a $s = \bar{s}$
				\4[] $\then$ Precios constantes por PPA
				\4[] $\then$ Interés igual a interés extranjero por UIP
			\3 Equilibrio monetario
				\4 Oferta monetaria igual a demanda
				\4[] $m^s \equiv d + r = p - \lambda i \equiv m^d$
				\4[] $\to$ $d + r = s - \lambda i$
				\4[] $\then$ \fbox{$d+r = s$}
				\4 Aumento de oferta monetaria
				\4[] Por compra de crédito doméstico
				\4[] $\to$ Asumiendo no es posible esterilizar
				\4 Reducción de reservas
				\4[] Dos motivos concurrentes
				\4[] --Déficit fiscal financiado monetariamente
				\4[] $\to$ Aumenta crédito doméstico
				\4[] $\to$ Aparece exceso de demanda de divisas
				\4[] $\to$ Necesario reducir balance
				\4[] $\then$ Venta de divisas por moneda nacional para cubrir ED
				\4[] $\to$ Para compensar $\uparrow$ crédito doméstico comprado
				\4[] $\to$ Misma cuantía, signo opuesto a aumento de $d$
				\4[] $\to$ Financiado por BC que compra deuda pública
				\4[] $\to$ $\dot{r}(t) = -\dot{d}(t) = -\mu$
				\4[] --Déficit por cuenta corriente
				\4[] $\to$ Déficit fiscal estimula absorción
				\4[] $\to$ Moneda local demasiado apreciada
				\4[] $\to$ Exceso de oferta de moneda nacional
				\4[] $\then$ Compra de divisas a cambio de moneda local
			\3 Tipo de cambio sombra
				\4 TC que prevalecería
				\4[] $\to$ Si precios no se igualasen a mundiales
				\4[] $\to$ Si interés no se igualase a mundial
				\4[] $\to$ Con reservas agotadas
				\4[] $\then$ Si no se defendiese tipo de cambio
				\4 Sustituyendo precio nacional dada PPA
				\4[] $d + \underbrace{r}_{=0} = p - \lambda i = s + p^* - \lambda i$
				\4[] $\then$ \fbox{$\hat{s} = d $}
				\4 Dinámica del TCSombra
				\4[] Dado que déficit público monetizado constante
				\4[] $\to$ $d$ aumenta a tasa constante $\mu$
				\4[] $\then$ TCSombra deprecia proporcionalmente a déficit
			\3 Ataque especulativo
				\4 Estrategia de ataque
				\4[] Venta de moneda nacional
				\4[] $\to$ Posiblemente apalancada
				\4[] $\to$ Esperando depreciación de moneda nacional
				\4[] $\to$ Tratando de agotar reservas de BC
				\4[] $\to$ Recomprar moneda nacional con divisa
				\4[] $\then$ Ganar diferencia tras depreciación
				\4 Depreciación esperada
				\4[] Depende de:
				\4[] $\to$ Tipo de cambio sombra
				\4[] $\to$ Mantenimiento o no de TCFijo
				\4 $\hat{s} > \bar{s}$
				\4[] Crisis cambiaria beneficiaría a especuladores
				\4[] Incentivos a lanzar ataque especulativo
				\4[] Si ataque falla:
				\4[] $\to$ TC se mantiene constante
				\4[] $\then$ Sólo pierden costes de transacción
				\4[] Si ataque tiene éxito:
				\4[] $\to$ TC se deprecia
				\4[] $\then$ Divisas ganan valor frente a moneda local
				\4 $\hat{s} < \bar{s}$
				\4[] Sin incentivos a lanzar ataque especulativo
				\4[] Agotamiento de reservas induce TCN apreciado
				\4[] $\then$ Pierden dinero habiendo comprado divisas
			\3 Representación gráfica
				\4 Eje de abscisas
				\4[] Tiempo
				\4 Eje de ordenadas
				\4[] Tipo de cambio fijo
				\4[] Tipo de cambio sombra
				\4 Ataque especulativo
				\4[] Momento temporal en el que TC sombra mayor que fijo
				\4[] $\to$ A partir de ahí, TC sombra menor que fijo
				\4[] $\then$ Rentable cambiar moneda local por divisa
				\4 \grafica{primerageneracion}
		\2 Implicaciones
			\3 Objetivos inconsistentes
				\4 Mantener:
				\4[] $\to$ Gasto público elevado
				\4[] $\to$ Tipo de cambio fijo
				\4 Acaba siendo necesario renunciar a uno
			\3 Momento del ataque
				\4 Especuladores conocen:
				\4[] Reservas disponibles
				\4[] Déficit fiscal
				\4[] $\then$ Evolución de reservas disponibles
				\4 Prevén efecto de ataque especulativo sobre TC
				\4[] Lanzan cuando consideran será rentable
				\4[] $\to$ Puede ser antes de agotar reservas por déficit
				\4[] $\then$ Ataque agota reservas bruscamente
				\4 Hipótesis de expectativas racionales
				\4[] Lanzan ataque inmediatamente cuando $\hat{s}=\bar{s}$
				\4[] $\then$ Beneficio nulo para todos
			\3 Determinantes de la sostenibilidad
				\4 Déficit público
				\4[] Superávits fiscales evitan financiación monetaria
				\4[] Es posible mantener constante oferta monetaria
				\4[] $\to$ Sin aumentar crédito doméstico en balance BC
				\4 Financiación del déficit
				\4[] Financiación monetaria es determinante
				\4[] $\to$ Subir impuestos reduce probabilidad de crisis
				\4[] $\to$ Endeudarse en divisa retrasa/evita crisis
				\4 Volumen de reservas
				\4[] Mayores volúmenes retrasan crisis
				\4 Cuenta corriente
				\4[] Superávit en CC implica aumento de reservas
				\4[] $\to$ Posible vender DPública para drenar liquidez/esterilizar
				\4[] $\then$ No se tiende a devaluación
				\4[] $\then$ Mucho más fácil frenar revaluación que devaluación
				\4[] Déficit en CC implica EDemanda de divisas
				\4[] $\to$ BC vende reservas para cubrir EDemanda
				\4[] $\to$ BC aumenta crédito doméstico para mantener precios
				\4[] $\then$ Caída de reservas
				\4 Tipo de cambio fijo
				\4[] $\bar{s}$ demasiado alto
				\4[] $\to$ Déficit en cuenta corriente
				\4[] $\then$ Más presión sobre divisas
				\4[] $\then$ Mayor probabilidad de ajuste
		\2 Valoración
			\3 Capacidad explicativa
				\4 Buena explicación de crisis de BP
			\3 Críticas
				\4 Modelos básicos son determinísticos
				\4[] Momento de crisis es perfectamente conocido
				\4 Sin devaluación en momento de ataque
				\4[] Porque se produce cuando $\tilde{s}=\bar{s}$
				\4 Crisis también con buenos fundamentales
				\4[] Modelos de siguientes generaciones
			\3 Extensiones
				\4 Información imperfecta de especuladores
				\4[] Sobre:
				\4[] $\to$ Volumen de reservas
				\4[] $\to$ Voluntad de mantener TCFijo del BC
				\4 Régimen post-crisis no es TCFlexible
				\4[] BC devalúa a TCFijo
				\4 Esterilización de intervención
				\4[] BC monetiza déficit
				\4[] Vende otros activos domésticos
				\4 Sustituibilidad imperfecta
				\4[] Entre activos domésticos y reservas
				\4 Interacción con sector bancario
				\4[] Bancos tienen problemas de solvencia
				\4[] Sector público garantiza implícitamente
				\4[] Especuladores saben que déficit aumentará
				\4[] $\to$ Para rescatar bancos
				\4[] $\then$ Tienen en cuenta para lanzar ataque
				\4[] $\then$ Conexión con tercera generación
	\1 \marcar{Crisis de segunda generación}
		\2 Idea clave
			\3 Contexto
				\4 Años 90
				\4[] Sistema Monetario Europeo
				\4[] Ataques a lira, libra, peseta, escudo..
				\4[] $\to$ Necesario aumento margen de fluctuación
				\4[] $\to$ Abandono de TCFijo
				\4[] Sin embargo, fundamentales no implican crisis
				\4[] $\to$ Reservas suficientemente grandes
				\4[] $\to$ Déficits públicos bajo control
				\4[] $\to$ Posible financiar déficit por otras vías
				\4 Ataques especulativos
				\4[] Exitosos a pesar de fundamentales
				\4[] Modelos de primera generación no explican
				\4 Sachs et al. (1996)
			\3 Objetivo
				\4 Explicar crisis de TCFijo
				\4[] $\to$ En presencia de fundamentales sólidos
				\4[] $\to$ Aunque reservas sean suficientes
				\4[] $\to$ Aunque no haya objetivos inconsistentes
				\4 Tener en cuenta trade-off de aut. monetaria
				\4[] Devaluación vs desempleo
				\4[] $\to$ Elegir entre desprestigio o descontento
			\3 Resultados
				\4 Crisis posible aunque objetivos consistentes
				\4 Múltiples equilibrios
				\4 Expectativas auto-cumplidas
				\4 Crisis cambiarias aunque reservas suficientes
		\2 Formulación\footnote{Sacado de Goldstein y Razin (2015).}
			\3 Supuestos
				\4 Gobierno minimiza función de pérdida
				\4[] $L(y_t, \pi) = (y-y^*)^2 + \pi^2 + I_\pi \cdot \pi$
				\4[] $L_{y}, L_\pi > 0$, $L_{yy}, L_{\pi \pi} > 0$ $\then$ Crecimiento convexo
				\4[] $I_\pi = 0$, $I_\pi = F$ si $\pi \neq 0$  $\then$ $I_\pi \cdot \pi$: coste devaluación
				\4[] Existe output eficiente $y^*$
				\4[] $\to$ Desviaciones de $y^*$ generan desutilidad
				\4[] Asumiendo PPA, inflación $\epsilon$ implica devaluación
				\4[] Inflación genera:
				\4[] $\to$ Desutilidad creciente en sí misma $\pi^2$
				\4[] $\to$ Devaluación con coste fijo $I_\pi \cdot \pi$
				\4 Curva de Phillips con expectativas
				\4[] $y =  \hat{y} + \alpha (\pi -  \pi^e) + u$
				\4[] Sorpresa de inflación ($\pi - \pi^e$) aumenta output
				\4[] Shocks exógenos $u$ alteran output
			\3 Shock exógeno con expectativas bajas de inflación
				\4 Si gobierno mantiene inflación cero
				\4[] Función de pérdida crece por desviación de output
				\4 Comparación entre:
				\4[] -- F. de pérdida sin inflación
				\4[] -- F. de pérdida con inflación/devaluando
				\4[] $\to$ Asumiendo coste de devaluación $I_\pi \cdot \pi$ lo suficientemente alto
				\4[] $\then$ Probablemente gob. prefiere no devaluar
				\4[] $\then$ TC se mantiene fijo
			\3 Shock exógeno negativo con expectativas altas de inflación
				\4 Si gobierno mantiene inflación cero:
				\4[] Función de pérdida crece doblemente:
				\4[] $\to$ Desviación de output por shock exógeno
				\4[] $\to$ Desviación de output por desviación respecto inf. esperada
				\4[] $\then$ Muy costoso mantener tipo fijo
				\4 Comparación entre:
				\4[] -- F. de pérdida sin inflación
				\4[] -- F. de pérdida con inflación/devaluando
				\4[] $\to$ Aunque coste de devaluar $I_\epsilon \epsilon$
				\4[] $\then$ Devaluación permite amortiguar shock sobre output
				\4[] $\then$ Probablemente gob. prefiera devaluar
%			\3 Problema de maximización de la autoridad monetaria
%				\4 Función de pérdida a minimizar
%				\4[] $L=L(x_t, \pi_t)$
%				\4[] $L_x, L_\pi > 0$, $L_{xx}, L_{\pi \pi} > 0$
%				\4[] $\then$ Convexas en $x_t$ y $\pi_t$
%				\4[] $\then$ Prefiere distribuir entre ambas
%				\4[] $x_t$: ingresos tributarios
%				\4[] $\pi_t$: inflación
%				\4 Restricción presupuestaria
%				\4[] $D = f(\underset{+}{x_t}, \underset{+}{\pi_t-\pi^e_t})$
%				\4[] $\to$ $D$: déficit financiable sin emitir deuda
%				\4[] + ingresos $x_t$ y + inflación inesperada $\pi_t$
%				\4[] $\to$ Permiten financiar más déficit
%			\3 Comparación de problemas de maximización
%				\4 Devaluación no permitida
%				\4[] Precios constantes para mantener PPA
%				\4[] $\to$ Asumiendo no hay inflación en la divisa ancla
%				\4[] $\then$ No puede financiar por vía monetaria
%				\4[] $\then$ Déficit financiado por impuestos
%				\4[] Función de pérdida
%				\4[] $\to$ $L_f=L(x_t,0)$ %=\frac{1}{2} x_t^2=\frac{1}{2}\left( Rb_t + \theta \pi_t^e \right)^2$
%				\4 Devaluación permitida
%				\4[] Precios pueden variar
%				\4[] $\to$ Puede financiar por vía monetaria
%				\4[] $\then$ Distribuye entre impuestos y monetización
%				\4[] $\then$ Inflación positiva
%				\4[] Función de pérdida
%				\4[] $\to$ $L_d = L(x_t,\pi_t) $
%				%\4[] $\then$ $L_d (b_t, \pi_t^e) = \frac{1}{2} \lambda \left( Rb_t + \theta \pi_t^e \right)^2 $
%				\4[$\then$] Gobierno compara $L_d$ y $L_f$
%				\4[] $L_d + c < L_f$
%				\4[] $\to$ Donde $c$ es coste de devaluación
%				\4[] Devaluación rentable si:
%				\4[] $\to$ Deuda muy alta
%				\4[] $\to$ Interés muy alto
%				\4[] $\to$ Inflación esperada alta
%				\4[] $\to$ Coste de devaluación bajo
%			\3 Credibilidad del gobierno
%				\4 Agentes privados estiman inflación
%				\4[] Con HER, conocen incentivos del gobierno
%				\4[] $\to$ Estiman $\pi^e$ basándose en incentivos a devaluar
		\2 Implicaciones
			\3 Devaluación asegurada si déficit elevado
				\4 Déficit elevado a financiar
				\4[] Mantiendo TCFijo, sólo consolidación fiscal para financiar
				\4[] $\to$ Output cae fuertemente por consolidación
				\4[] $\to$ Función de pérdida crece demasiado por convexidad
				\4[] $\to$ Devaluación implicaría menor coste
				\4[] $\then$ Inevitable monetizar y devaluar
			\3 Tipo fijo creíble
				\4 Déficit pequeño a financiar
				\4[] Con TCFijo, sólo consolidación fiscal para financiar
				\4[] $\to$ F. de pérdida crece poco aunque sea convexa
				\4[] $\to$ Coste de devaluar elevado compensa consolidación
				\4[] $\then$ Devaluación siempre más costosa que impuestos
				\4[] $\then$ Agentes lo saben y esperan inflación 0
				\4[] $\then$ No hay devaluación
			\3 Equilibrios múltiples
				\4 Déficit intermedio
				\4 Expectativas de inflación determinan equilibrio
				\4 Inflación esperada nula
				\4[] Posible mantener inflación efectiva nula
				\4[] Pérdida por financiación con impuestos
				\4[] $\to$ Menor que devaluación + coste
				\4[] $\then$ Preferible no devaluar
				\4 Inflación esperada alta
				\4[] Financiación sólo con consolidación
				\4[] $\to$ F. de pérdida aumenta por consolidación
				\4[] $\to$ F. de pérdida aumenta
				\4[] $\then$ Gobierno no está dispuesto a asumir coste
				\4[] Gobierno obligado a aumentar inflación
				\4[] $\to$ Para financiar déficit
				\4[] $\then$ Compensa devaluar a pesar de coste de devaluación
				\4[$\then$] Ambos equilibrios son posibles y estables
				\4[] Expectativa de inflación determina equilibrio
				\4[] Determinación de expectativa al margen del modelo
				\4[] $\to$ P.ej.: animal spirits
				\4[] $\then$ Expectativas auto-cumplidas
				\4[] $\then$ Nivel de deuda posibilita exp. auto-cumplidas
		\2 Valoración
			\3 Capacidad explicativa
				\4 Buena explicación de crisis de primeros 90s
				\4 Crisis asociadas a desempleo, inflación y déficit
				\4[] A pesar de reservas suficientes
				\4[] $\to$ Economías acaban devaluando
				\4 Turquía en 2018
				\4[] Conflicto entre:
				\4[] $\to$ Permitir depreciación e inflación
				\4[] $\to$ Aumentar tipos y enfriar economía
			\3 Críticas
				\4 Fundamentales son conocimiento público
				\4[] Supuesto muy restrictivo
				\4[] Especialmente difícil con deuda moderada
				\4[] $\to$ Poco incentivo a prestar atención
				\4[] $\then$ Múltiples equilibrios más difícil aún
			\3 Extensiones y variaciones
				\4 Output gap y desempleo
				\4[] En vez de impuestos
				\4 Agentes heterogéneos
				\4[] Diferente información
				\4[] $\then$ Diferentes expectativas
				\4 Expectativas endógenas
				\4[] Equilibrios únicos
	\1 \marcar{Crisis de tercera generación}
		\2 Idea clave
			\3 Contexto
				\4 Crisis en Asia 1997
				\4[] Buenos fundamentales
				\4[] Deuda y déficit reducido
				\4[] Reservas suficientes
				\4[] Crecimiento alto y desempleo bajo
				\4[] $\then$ Caídas muy fuertes del tipo de cambio
				\4[] $\then$ Quiebras generalizadas
				\4[] $\then$ Contagio regional
				\4[] $\then$ Recesión muy fuerte
				\4 Bancos inician crisis
				\4[] Quiebra de banco en Tailandia
				\4 Contagio rápido a países de la región
				\4[] Corea, Indonesia, Malasia, Hong Kong...
				\4 Trabajos principales
				\4[] Diaz Alejandro (1985),Krugman (1998), (1999)
				\4[] Kaminsky y Reinhart (1999), Chang y Velasco (2001)
			\3 Objetivo
				\4 Explicar crisis a pesar de buenos fundamentales
				\4 Caracterizar papel de varios mecanismos
				\4[] Balance de bancos
				\4[] Riesgo moral
				\4[] Reversiones de flujos de capital
			\3 Resultados
				\4 Interacción bancos-mercados cambiarios
				\4[] $\to$ Causante de crisis
				\4 Buenos fundamentales no bastan
				\4[] Déficit público
				\4[] Reservas
				\4[] Inflación
				\4[] $\to$ No son suficientes para evitar crisis
				\4[] $\then$ Deseq. del mercado financiero son catalizador
		\2 Formulación
			\3 Diferentes mecanismos
				\4 Diferentes modelos cambian énfasis
				\4 Elementos comunes
				\4[] Tipos de cambio fijo
				\4[] Endeudamiento en dólares
				\4[] Entradas masivas de capital
				\4[] Sudden stop
				\4[] Quiebras bancarias y economía real
			\3 Riesgo moral
				\4 Exceso de inversión es causa de crisis
				\4 Aseguramiento implícito de sector privado por gobierno
				\4[] Empresas domésticas estiman
				\4[] $\to$ Protegidas por gobierno frente a $\Delta$ TC
				\4[] $\to$ Serán rescatadas por gobierno si quiebran
				\4[] $\then$ Se endeudan en divisa y excesivamente
				\4[] $\then$ No asumen riesgo personal alguno
				\4[] $\then$ Riesgo moral
				\4[] Sector financiero doméstico poco regulado
				\4[] $\to$ Pide prestado en mercados internacionales
				\4[] $\to$ Se endeudan en exceso
				\4[] Prestamistas extranjeros
				\4[] $\to$ Creen también aseguramiento público
				\4[] $\then$ Aceptan prestar en divisa
				\4 Inflación de activos
				\4[] Provocada por exceso de crédito
				\4[] $\to$ Activos parecen más seguros de lo que son realmente
				\4 Deflación de activos
				\4[] Activos financieros caen inesperadamente
				\4[] Intermediarios financieros quiebran
				\4[] $\to$ Gobierno acude al rescate
				\4[] $\to$ Deuda divisas presiona reservas
				\4[] $\then$ Se desencadena ciclo perverso
			\3 Fragilidad financiera
				\4 Pánico sobre activos domésticos
				\4[] Prestamistas extranjeros o intermed. domésticos
				\4[] $\to$ Liquidan activos domésticos
				\4[] $\to$ Activos pierden valor bruscamente
				\4[] $\then$ Quiebras y rescates bancarios
				\4[] $\then$ Se desencadena proceso
				\4 Dificultades de financiación
				\4[] Capital sale del país rápidamente
				\4[] BCentral forzado a subir tipos
				\4[] $\to$ Bancos no pueden refinanciarse
				\4[] $\then$ Quiebras
			\3 Balance comercial deficitario
				\4 Déficit comerciales excesivos
				\4[] Hasta 10\% en Tailandia
				\4 Entrada de capital se frena bruscamente
				\4[] Enormes flujos de salida de capital
				\4 Necesaria devaluación para aumentar exportaciones netas
				\4 Deuda en divisa
				\4[] Aumenta rápidamente volumen por devaluación
				\4[] $\to$ Quiebras generalizadas
				\4[] $\then$ Proceso de realimentación
			\3 Hot money
				\4 Flujos de capital de gran cuantía bajo TCFijo
				\4[] $\to$ Previsión de devaluación inminente
				\4[] $\to$ Diferenciales de interés por encima de riesgo cambiario
				\4 Potencial desestabilizador
				\4[] Demostrado especialmente en 3ª gen.
				\4 Desincentivar hot money
				\4[] Exit taxes
				\4[] Restricciones temporales de horizonte de inversión
		\2 Implicaciones
			\3 Regulación financiera para evitar
				\4 Crisis bancarias tienden a preceder
				\4[] Crisis monetarias ocurren después
				\4[] $\then$ Crisis monetaria agrava crisis bancaria
				\4[$\then$] Necesario evitar desequilibrios en sistema bancario
				\4 Énfasis de reformas sobre regulación bancaria
				\4[] En 2000s
				\4 Influencia sobre Basilea II
			\3 Intervención del FMI
				\4 Objetivo:
				\4[] Estabilizar tipo de cambio
				\4[] Evitar explosión de valor de deuda en divisa
				\4 Recomendación principal
				\4[] Aumentar tipos de interés
				\4[] $\to$ Incentivar entrada de capital
				\4[] $\to$ Frenar salida de capital
				\4[] Utilizar líneas de crédito
				\4[] $\to$ Disuadir especuladores
				\4 Fracaso
				\4[] Aumento de tipos de interés contrae financiación
				\4[] $\to$ Deflación
				\4[] $\to$ Output gap
				\4[] $\then$ Quiebras en sector privado
				\4[] $\then$ Presión ulterior sobre reservas y TC
			\3 Controles de capital
				\4 Introducidos por Malasia en 1998
				\4[] Contra recomendación del FMI
				\4 Permiten reestruct. ordenada sector financiero
				\4 FMI empezó a considerar tras crisis
				\4[] Cambio frente a postura previa
				\4[] $\to$ Muy lentamente
				\4[] En 2012 matiza postura institucional
				\4[] $\to$ Acepta ciertos controles en circunstancias concretas
				\4[] $\to$ Introduce Capital Flow Management Measures
		\2 Valoración
			\3 Capacidad explicativa
				\4 Gran complejidad de la crisis
				\4[] Muchos fenómenos en uno
				\4 Dificil formular un modelo general
			\3 Críticas
				\4 Carácter relativamente ad-hoc de modelos
				\4 Sin modelo general
			\3 Extensiones
				\4 Muy numerosas
				\4 También aplicando 1ª y 2ª generación
	\1 \marcar{Otros modelos de crisis}
		\2 Crisis de cuarta generación
			\3 Idea clave
				\4 Contexto
				\4[] Tras crisis asiáticas
				\4[] Modelos de tercera generación aún inestables
				\4[] $\to$ Muchas variedades
				\4[] $\to$ Múltiples mecanismos
				\4 Malas políticas y equilibrios institucionales
				\4[] Comunes a todos los países en crisis
				\4 Objetivo
				\4[] Caracterizar factores institucionales de crisis
				\4[] Determinantes profundos de factores superficiales
				\4[] $\to$ Déficits por cuenta corriente
				\4[] $\to$ Hiperinflación
				\4[] $\to$ Expectativas auto-cumplidas
				\4[] $\to$ Endeudamiento excesivo
				\4[] $\then$ ¿Por qué?
				\4 Resultados
				\4[] Instituciones son importantes
				\4[] Entorno legal y político también
				\4[] Aún en proceso de consolidación
			\3 Implicaciones
				\4 Factores que aumentan probabilidad de crisis
				\4[] Marco contractual
				\4[] Derechos de propiedad y protección de accionistas
				\4 Flujos de capital
				\4[] De c/p si mal cumplimiento de contratos
			\3 Valoración
				\4 Muy conectados con nuevo institucionalismo
				\4 Aun poco definido
				\4 Importantes sinergias con:
				\4[] Teoría de regulación
				\4[] Behavioral economics
				\4[] Teoría de contratos
				\4[] Economía de la gobernanza y organizaciones
		\2 Crisis bancarias\footnote{Ver primeros capítulos de Kling (2011).}
			\3 Idea clave
				\4 Contexto
				\4 Objetivos
				\4 Resultados
			\3 Formulación
				\4 Malas apuestas
				\4 Exceso de riesgo
				\4 Equilibrios múltiples
				\4 Externalidades
			\3 Implicaciones
			\3 Valoración
		\2 Crisis de Minsky
			\3 Idea clave
				\4 Estructuras financieras de empresas
				\4[] Concepto clave
				\4[] Paso de uno a otro induce crisis económicas
				\4[i] Estructura de cobertura
				\4[] Flujos de caja mayores que compromisos
				\4[] $\to$ En todos los periodos futuros
				\4[] $\then$ No tiene que refinanciar nada en futuro
				\4[ii] Estructura especulativa
				\4[] Flujos de caja mayores que interés
				\4[] $\to$ Pero menores que principal
				\4[] $\then$ Tiene que refinanciar principal
				\4[] $\then$ Vulnerable a problemas de refinanciación
				\4[iii] Estructura Ponzi
				\4[] Flujos de caja no cubren pagos de interés
				\4[] $\to$ Ganancias de capital o refinanciación necesaria
				\4[] $\then$ Cualquier contracción de crédito induce quiebra
				\4 Tendencia del sistema
				\4[] Inmediatamente tras crisis
				\4[] $\to$ Miedo a financiar inversión l/p con deuda c/p
				\4[] $\then$ Empresas mantienen estructura de cobertura
				\4[] Periodo de estabilidad
				\4[] $\to$ Precauciones anteriores parecen excesivas
				\4[] $\to$ Empresas ``caen en tentación'' y financian a c/p
				\4[] $\then$ Financian l/p con deuda a c/p
				\4[] Periodo de crisis
				\4[] $\to$ Márgenes de seguridad insuficientes
				\4[] $\then$ Empresa y bancos incumplen compromisos
				\4[] $\then$ Crisis tiene lugar
			\3 Implicaciones
				\4 Estabilidad es desestabilizante
				\4[] Periodos de estabilidad son germen de crisis
			\3 Valoración
				\4 Descripción simple de muchas crisis financieras
				\4 Problemas de falsabilidad
				\4[] No excluye un estado de la naturaleza
				\4[] $\to$ ``Si no se cumple, se cumplirá pronto''
		\2 La Gran Recesión
			\3 Idea clave
				\4 Contexto
				\4 Objetivo
				\4 Resultado
			\3 Causas
			\3 Implicaciones
		\2 Crisis en países en desarrollo
			\3 Idea clave
			\3 Causas
			\3 Implicaciones
		\2 Acelerador financiero
			\3 Idea clave
			\3 Formulación
			\3 Implicaciones
			\3 Valoración
	\1 \marcar{Contagio}
		\2 Idea clave
			\3 Concepto
				\4 Regularidad empírica
				\4 Diferentes definiciones
				\4 Correlación temporal positiva
				\4[] Entre ataques especulativos sobre
				\4[] $\to$ Divisas
				\4[] $\to$ Deuda soberana
				\4[] $\to$ Mercados de activos
				\4 Transmisión entre países de shocks
				\4[] Más allá de links fundamentales
				\4[] $\to$ Exceso de correlación entre shocks
				\4 Contagio impropio vs puro
				\4[] Impropio
				\4[] $\to$ Aumento de vínculos fundamentales tras shock
				\4[] Puro
				\4[] $\to$ Todos los demás casos de contago
				\4[] $\then$ Sin relación con fundamentales
			\3 Enfoques de estudio
				\4 Causas fundamentales vs comportamiento de inversores
				\4[] Se centran en una u otra causa
				\4[] $\to$ Implican diferentes interpretaciones del concepto
				\4 Canales de coordinación
				\4[] Qué factores concretos inducen contagio
				\4[] $\to$ Entendiendo contagio en sentido amplio
		\2 Canales de contagio
			\3 Comercio
				\4 Estudios
				\4[] Eichengreen et al. (1996)
				\4[] Gerlach y Smets (1995)
				\4 Análisis de datos de panel
				\4[] Comercio bilateral entre industrializados
				\4 Eslabonamientos de comercio internacional
				\4[] $\to$ Cuanto mayores, mayor probabilidad de contagio
				\4[] $\to$ Condiciones macro similares a veces son menos importantes
				\4[] $\then$ Comercio más importante que fundamentales
				\4 Creación excesiva de crédito en un país
				\4[] $\to$ Aumenta probabilidad de crisis monetaria en país
				\4[] $\then$ Aumento de competitividad frente a socio
				\4[] $\then$ Crisis monetaria más probable en socio
				\4 Contagio a monedas fuertes
				\4[] Regímenes de TCFijo sostenibles también pueden contagiarse
				\4[] $\to$ Sostenibilidad endógena a sostenibilidad de socio
				\4[] Ejemplos:
				\4[] $\to$ Depreciación RU $\to$ IRL en 1992
				\4[] $\to$ Devaluación FIN $\to$ UK en 1992
				\4[] $\to$ Devaluación ESP $\to$ POR en 1992-1993
				\4[] $\to$ Efectos globales de crisis asiática sobre PEDs
			\3 Bancos
				\4 Vínculos comerciales no son suficientes
				\4[] Crisis Tequila
				\4[] $\to$ Presión en LATAM y Asia tras crisis MEX 1994
				\4[] $\to$ THA a otros países Asia con pocos vínculos
				\4[] $\to$ Crisis asiática a RUSIA
				\4 Contagio vía spillovers
				\4[] Prestamista común a varios países en crisis
				\4[] $\then$ USA en crisis tequila
				\4[] $\then$ Japón en crisis asiática
				\4[] $\then$ Alemania en crisis rusa
				\4[] Prestamista ajusta cartera tras primera crisis
				\4[] $\to$ Restaurar ratios de capital
				\4[] $\to$ Recalibrar exposición a región
				\4[] $\to$ Margin calls
			\3 Inversión en cartera
				\4 Sobrereacción de inversores de cartera
				\4 Cronología de crisis
				\4[]Crisis en un país reduce riqueza
				\4[] Aumenta aversión al riesgo
				\4[] $\to$ Colapso de confianza
				\4[] $\to$ Incertidumbre sobre inf. privada de otros
				\4[] $\then$ Ataques sucesivos sin motivos fundamentales
				\4 Similitudes institucionales
				\4[] Aumenta probabilidad de contagio
			\3 Wake-up calls
				\4 ``Alarma del despertador''
				\4 Desatención racional
				\4[] Es racional no prestar atención a fundamentales
				\4[] $\to$ Porque resulta costoso
				\4 Factores de riesgo común
				\4[] Inversores estiman probabilidad de impago
				\4[] $\to$ Depende de factor común a varios países
				\4[] $\to$ No prestan atención a condiciones idiosincráticas
				\4 Crisis en un país
				\4[] Desencadena percepción de riesgo respecto a otros
				\4[] $\to$ Dinero sale de países con factores comunes
				\4[] $\then$ Crisis se transmite sin atender a idiosincrasias
	\1 \marcar{Predicción de crisis}
		\2 Idea clave
			\3 Concepto
				\4 ¿Es deseable predecir las crisis?
				\4[] Si es deseable:
				\4 ¿Pueden predecirse las crisis?
				\4 ¿Cómo hacerlo?
				\4 ¿Cómo se intenta predecir en la práctica?
			\3 Objetivos
				\4 Estimar probabilidad de crisis
				\4 Caracterizar ámbitos de actuación
				\4[] $\to$ Para reducir probabilidad y daño
			\3 Resultado
				\4 Múltiples modelos de predicción
				\4 Enfoques cuantitativo y calitativo
				\4 Más complejidad no siempre aumenta efectividad
		\2 Modelos
			\3 Indicadores
				\4 Crisis en términos de umbrales
				\4 Múltiples grupos de indicadores
				\4[] Liberalización financiera
				\4[] Otros indicadores financieros
				\4[] Cuenta corriente
				\4[] Cuenta financiera
				\4[] Variables fiscales
				\4 Explicación vs predicción
				\4[] Muy difícil elegir indicadores correctos
			\3 Expectativas de especuladores
				\4 Tratar de capturar sentimiento de especuladores
				\4[] Ratios compraventa
				\4[] Opciones
				\4[] CDS
				\4[] ...
				\4 Efectividad variable frente a indicadores
			\3 Early Warning Systems
				\4 Modelos generalmente econométricos
				\4 Umbrales estadísticos de crisis
				\4[] A partir de indicadores
			\3 MIP Scoreboard
				\4 Conjunto de indicadores MIP
				\4 Macroeconomic Imbalance Procedure
			\3 FSSA y FSAP\footnote{Diferencia entre FSSA y FSAP es difícil de encontrar explicada. Ver \url{https://www.fsa.go.jp/en/faq/others/others_a_2.html}.}
				\4 Marco de predicción conjunto FMI-GBM
				\4 Financial Sector Assessment Program (FSAPs)
				\4[] Estabilidad del sistema financiero en conjunto
				\4[] $\to$ No de instituciones financieras concretas
				\4[] Incluye FSSA
				\4[] Específico de cada país
				\4[] Incluye recomendaciones
				\4[] Parte de artículo IV para países sistémicos
				\4 Incluyen Financial System Stability Assessment (FSSAs)
				\4[] No siempre públicos
				\4[] $\to$ \textit{stress tests}
				\4[] Análisis individualizados por instituciones
	\1[] \marcar{Conclusión}
		\2 Recapitulación
			\3 Modelos de 1ª generación
			\3 Modelos de 2ª generación
			\3 Modelos de 3ª generación
			\3 Otros modelos de crisis
			\3 Contagio
			\3 Predicción
		\2 Idea final
			\3 Lecciones principales
				\4[I] Crisis casi siempre dependen de vulnerabilidades
				\4[II] Reversión súbita de CC no suele ser culpable
				\4[III] Emergentes acumulan más vulnerabilidades
				\4[IV] Algunas crisis son más costosas que otras
			\3 ¿Las crisis son necesariamente malas?
				\4 Milton Friedman
				\4[] No necesariamente
				\4[] Fuerza ajuste a equilibrio
				\4 Crisis de primera y segunda generación
				\4[] Generalmente, poco dañinas para economía real
				\4[] Fuerzan ajuste
				\4[] En algunos casos inducen mejoras económicas
				\4[] $\to$ Aunque coste político
				\4[] $\to$ Coste reputacional
			\3 Regímenes cambiarios
				\4 Papel clave en crisis financieras
				\4 En últimas décadas
				\4[] Recomendación bipolar
				\4[] Miedo a flotar
				\4[] Opacidad sobre regímenes cambiarios
			\3 Instituciones internacionales y coordinación
				\4 Evitar equilibrios desfavorables
				\4 Coordinar respuesta a crisis regionales/mundiales
\end{esquemal}
























\graficas

\begin{axis}{4}{Crisis de primera generación: momento del ataque especulativo}{$t$}{$\tilde{s}$\\ $\bar{s}$}{primerageneracion}
	% TC fijo
	\draw[thick] (0,1.5) -- (4,1.5);
	\node[right] at (4,1.5){$\bar{s}$};
	
	
	% TC sombra
	\draw[-] (0,0) -- (4,3);
	\node[right] at (0.5,0.75){$\tilde{s}$};
	
	
	% Momento temporal del ataque especulativo
	\draw[dashed] (1.97,1.5) -- (1.97,0);
	\node[below] at (1.97,0){\tiny Ataque};
	
	% Zona en la que ataque no es rentable
	\draw[decorate,decoration={brace, mirror,amplitude=3pt},xshift=0pt,yshift=-0.3cm] (0,0) -- (1.97,0) node[black,midway,xshift=2pt, yshift=-0.33cm] {\tiny $\tilde{s} < \bar{s}$ \quad Ataque no rentable};
	
	
	% Zona en la que ataque es rentable
	\draw[decorate,decoration={brace, mirror,amplitude=3pt},xshift=0pt,yshift=-0.3cm] (1.98,0) -- (4,0) node[black,midway,xshift=2pt, yshift=-0.33cm] {\tiny $\tilde{s} > \bar{s}$ Ataque rentable};
	
	
\end{axis}


\preguntas

\seccion{Test 2018}

\textbf{32.} La crisis del Sistema Monetario Europeo que tuvo lugar en el bienio 1992--1993, puede ser descrita mediante un modelo de crisis cambiaria de:

\begin{itemize}
	\item[a] Primera Generación, explicado por la evolución desfavorable de los fundamentos o condiciones económicas fundamentales.
	\item[b] Segunda Generación, explicado por la presencia de expectativas que se autoconfirman.
	\item[c] Tercera Generación, explicado por la interacción de crisis cambiarias y crisis financieras que se retroalimentan.
	\item[d] Cuarta Generación, explicado por la no sostenibilidad fiscal que anticipa la crisis cambiaria.
\end{itemize}


\seccion{Test 2011}

\textbf{29.} En un modelo de crisis de primera generación en la que hay una economía con agentes que disponen de información perfecta, se cumple la HER y la PPA, el Gobierno monetiza el déficit, y existe compromiso de mantener un tipo de cambio fijo:
\begin{itemize}
	\item[a] El compromiso se romperá tras el agotamiento gradual de las reservas internacionales.
	\item[b] Un ataque especulativo que precipite la crisis es inevitable.
	\item[c] Si el gobierno mantiene un déficit reducido, el compromiso de fijación del tipo de cambio será creíble.
	\item[d] El nivel de reservas internacionales de partida no afecta al momento en el que se produce la crisis.
\end{itemize}

\textbf{35.} La crisis mejicana desatada en la década de los ochenta fue:
\begin{itemize}
	\item[a] Una crisis bancaria.
	\item[b] Una crisis de deuda externa.
	\item[c] Una crisis cambiaria. 
	\item[d] Ninguna respuesta es correcta.
\end{itemize}

\seccion{Test 2006}
\textbf{28.} Señale la afirmación FALSA en relación a los denominados modelos de 1ª generación de crisis monetarias:
\begin{itemize}
	\item[a] El tipo de cambio sombra es el tipo de cambio que prevalecería en el mercado si no existiera intervención en el mercado de divisas por parte del banco central.
	\item[b] La causa principal de la crisis en estos modelos es una inadecuada política económica; en general, persistentes déficits públicos en un régimen de tipos de cambio fijos.
	\item[c] La crisis, aunque inevitable, es un suceso aleatorio. No se puede fijar con exactitud el momento en que ésta se origina.
	\item[d] Lo importante para determinar la crisis son las políticas económicas esperadas por los inversores, pero no las pasadas.
\end{itemize}

\notas


\textbf{2018:} \textbf{32.} B

\textbf{2011:} \textbf{29.} B \textbf{35.} B

\textbf{2006:} \textbf{28.} C

\bibliografia

\begin{itemize}
	\item banking crises
	\item capital flight
	\item credit cycle
	\item Credit Crunch chronology: april 2007–september 2009
	\item currencies
	\item currency boards
	\item currency crises
	\item \textbf{currency crises models}
	\item currency unions
	\item Euro Zone crisis 2010
	\item exchange control
	\item financial accelerator
	\item financial crisis
	\item financial market contagion
	\item Great Depression
	\item hot money
	\item international capital flows
	\item Minsky crisis
	\item rational inattention
\end{itemize}

Aliber, R. Z.; Kindleberger, C. P. \textit{Manias, Panics, and Crashes} (2015) Palgrave MacMillan -- En carpeta del tema

Breuer, J. \textit{An Exegesis on Currency and Banking Crises} (2004) Journal of Economic Surveys -- En carpeta del tema

Cecchetti, S. Schoenholt, K. (2018) \textit{Sudden stops: A primer on balance-of-payments crises} Voxeu.org \href{https://voxeu.org/content/sudden-stops-primer-balance-payments-crises}{Enlace}

Chang, R.; Velasco, A. \textit{A Model of Financial Crises in Emergin Markets} (2001) Quarterly Journal of Economics -- En carpeta del tema

Das, D. K. \textit{Asian Crisis: Distilling Critical Lessons} (2000) UNCTAD Discussion Papers -- En carpeta del tema

Diaz Alejandro, C. \textit{Good-bye Financial Repression, Hello Financial Crash} (1985) Journal of Development Economics -- En carpeta del tema

Dornbusch, R. \textit{Expectations and Exchange Rate Dynamics} (1976) Journal of Political Economy -- En carpeta del tema

Eichengreen, B. Rose, A. Wyplosz, C. \textit{Contagious Currency Crises} (1996) NBER Working Paper Series -- En carpeta del tema

Gandolfo, G. \textit{International Finance and Open Economy Macroeconomics} (2016) Springer Verlag. Ch 16 Capital Movements, Speculation and Currency Crises

Goldstein, I.; Razin, A. \textit{Three Branches of Theories of Financial Crises} (2015) Foundations and Trends in Finance -- En carpeta del tema

James, J.; Warsh, I. W.; Sarno, L. \textit{Handbook of Exchange Rates} (2012) Ch. 25. Wiley Publications -- En carpeta del tema y Economía Internacional (Libro completo)

Kaminsky, G.; Reinhart, C. \textit{The Twin Crises: The Causes of Banking and Balance-of-Payments Problems} (1999) American Economic Review -- En carpeta del tema

Krugman, P. \textit{Crises: the Next Generation} (2001) Conference Honoring Assaf Razin, Tel Aviv -- En carpeta del tema

Krugman, P. \textit{What Happened to Asia?} (1998) \url{http://web.mit.edu/krugman/www/DISINTER.html}

IMF. \textit{The IMF's Institutional View on Capital Flows in Practice} (2018) -- En carpeta del tema

Obstfeld, M. \textit{Models of Currency Crises with Self-Fulfilling Features} (1995) NBER Working Paper Series -- En carpeta del tema

Obstfeld, M.; Rogoff, K. \textit{The Mirage of Fixed Exchange Rates} (1995) Journal of Economic Perspectives -- En carpeta del tema

Obstfeld, M.; Shambaugh, J.; Taylor, A. M. \textit{The Trilemma in History: Tradeoffs among Exchange Rates, Monetary Policies, and Capital Mobility} (2004) NBER Working Paper Series -- En carpeta del tema

Obstfeld, M.; Shambaugh, J.; Taylor, A. \textit{Financial Stability, the Trilemma, and International Reserves} (2008) NBER Working Paper Series -- En carpeta del tema

Sachs, J. Tornell, A. Velasco, A. \textit{The Mexican Peso Crisis: Sudden Death or Death Foretold?} (1996) NBER Working Paper Series -- En carpeta del tema

Sachs, J. Tornell, A. Velasco, A. \textit{Financial Crises in Emergin Markets: The Lessons from 1995} (1996) NBER Working Paper Series -- En carpeta del tema

Sarno, L.; Taylor, M. \textit{The economics of exchange rates} (2002) Cambridge University Press -- En carpeta Economía internacional

Taylor, M. P. (1995) \textit{The Economics of Exchange Rates} Journal of Economic Literature Vol. XXXIII -- En carpeta del tema

\end{document}
