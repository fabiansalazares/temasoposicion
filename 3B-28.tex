\documentclass{nuevotema}

\tema{3B-28}
\titulo{Regulación bancaria y financiera. Fundamentos económicos y evidencia empírica.}

\begin{document}

\ideaclave

Añadir apartado sobre regulación financiera, política monetaria, bancos centrales y medio ambiente. Ver \href{https://voxeu.org/article/central-banks-and-climate-change}{VOXEU (2020)}.

NOTA. AÑADIR IFRS 9 EN CAMBIOS EN REGULACIÓN RECIENTE EN CAPITULO III

NOTA. AÑADIR apartado sobre Shadow Banking

NOTA. Añadir reformas de Basilea III acordadas en diciembre de 2017 y a implementar a partir de 2022

La crisis financiera que estalló en 2007 y que alcanzó su punto álgido en otoño de 2008 tuvo un impacto global en términos de caídas generalizadas del producto, aumentos del déficit público y un deterioro generalizado de la confianza en el sistema financiero y los bancos. Aunque la actuación de las autoridades públicas logró atajar la crisis en la mayoría de los países desarrollados y evitó un colapso generalizado, las turbulencias financieras tuvieron efectos graves sobre la economía real y por ende, sobre el bienestar de la población. La íntima interrelación entre la economía financiera y la economía real recordó la necesidad de contar con una regulación adecuada del sistema financiero que reduzca el riesgo de nuevas crisis, reduzca su gravedad cuando estas se produzcan, y trate de evitar la introducción de distorsiones adicionales. Aunque los objetivos últimos están claros, los obstáculos al diseño de políticas adecuadas son notables: el sistema financiero tiene una enorme complejidad, con un altísimo número de interconexiones internacionales y con la economía real, conflictos de interés y problemas de carácter político y legal. Por todo ello, resulta necesario examinar el problema de la regulación financiera y bancaria desde un punto de vista económico que atienda a sus fundamentos teóricos y a la realidad empírica. Así, el \textbf{objeto} de esta exposición consiste en dar respuesta a una serie de preguntas tales como: ¿quiénes son los sujetos de la regulación bancaria y financiera? ¿qué agentes actúan en el sistema financiero? ¿por qué es necesario regular el sistema financiero? ¿qué efectos positivos y negativos se pueden derivar de la regulación? ¿de qué instrumentos regulatorios disponen los reguladores? ¿cuáles emplean habitualmente? ¿cómo se regula en la práctica a nivel global y europeo? ¿qué regulación está por venir? La \textbf{estructura} de la exposición se divide en tres partes. En la primera, presentamos los rasgos generales del sistema financiero y bancario. En la segunda, examinamos los fundamentos teóricos de la regulación. En la tercera y última, recorremos la regulación en vigor en la actualidad a nivel global, en Europa y Estados Unidos.

El \marcar{sistema financiero} tiene tres \textbf{funciones} básicas: transformar plazos, transferir riesgos y transferir fondos. La \underline{transformación de plazos} se lleva a cabo por los intermediarios financieros, especialmente los bancos, que toman prestado a corto plazo y prestan a largo plazo. Así, canalizan el ahorro hacia inversiones productivas compatibilizan el deseo de los ahorradores de disponer de su ahorro en el corto plazo, con la preferencia de los inversores por el crédito a largo plazo. Esta diferencia de liquidez del ahorro y la inversión da lugar al llamado \textit{maturity mismatch} y debe ser gestionado por el banco y su gestión, a su vez, supervisada por el regulado para que reducir los riesgos de pánico bancario. La función de \underline{transferencia de riesgos} consiste en la transmisión del riesgo de las inversiones desde los ahorradores a los intermediarios financieros y otros agentes menos aversos al riesgo, de manera que el ahorro efectivamente se canalice hacia proyectos de inversión arriesgados. Los intermediarios asumen el riesgo de contrapartida a cambio de un diferencial. Además, la especialización de los intermediarios financieros en la valoración del riesgo de default asociado a los deudores reduce los costes de información: conocer la solvencia de un prestatario potencial es costoso y está sujeto a múltiples asimetrías de información. Por último, la función de \underline{sistema de pagos} hace referencia a la canalización de transferencias financieras que el sistema permite. Así, los bancos y otras instituciones financieras permiten a partes alejadas físicamente obligarse y transferir capacidad de compra, asegurando la ejecución efectiva de las transferencias y con ello, permitiendo el tráfico económico.

En el sistema financiero participan una gran variedad de \textbf{agentes}, clasificables atendiendo en cuanto a sus características institucionales, sus objetivos y sus funciones. Los \underline{clientes} del sistema financiero son la razón última de su existencia. Los clientes del sistema financiero son aquellos inversores particulares y empresas no financieras que necesitan realizar pagos y recibir cobros de otros agentes, empresas no financieras que necesitan financiar su operativa corriente y sus inversiones de capital, y consumidores que desean suavizar su perfil de consumo intertemporal a través del ahorro y la deuda. Los \underline{bancos} son generalmente el agente del sistema financiero con el que los clientes finales tienen un contacto directo. La actividad de los bancos se extiende a las tres funciones anteriores: canalizan pagos, transforman plazos y transmiten riesgos. Existen diferentes clases de bancos de acuerdo con sus actividades concretas, sus propietarios y su marco regulatorio: bancos comerciales, bancos de inversión, bancos hipotecarias, cajas de ahorros, cooperativas, banca islámica... Las empresas \underline{aseguradoras} se especializan en la transferencia de riesgos, asumiendo los riesgos de sus clientes a cambio de un prima y transfiriendo parte de esos riesgos asumidos en los mercados financieros. Las \underline{cámaras de compensación} se interponen entre las partes de una transacción financiera con el objetivo de eliminar el riesgo de contrapartida derivada de la liquidación de los contratos, a cambio también de una prima. Las \underline{agencias de calificación} valoran el riesgo de impago de instrumentos de deuda y similares a cambio de un precio. Su actividad es útil en la medida en que permite señalizar la calidad crediticia de activos determinados, realizando economías de escala informativas y permitiendo una valoración adecuada de los activos financieros. Además de estas instituciones, existen \underline{otros participantes} en el mercado financiero tales como las sociedades de inversión colectiva tales como fondos de inversión y SICAVs y similares o participantes en el llamado ``\textit{shadow banking}''. El concepto de shadow banking hace referencia a la prestación de servicios bancarios en entornos no regulados que escapan a la supervisión de las autoridades competentes. Esta actividad supone un reto de especial dificultad para los reguladores. 

La regulación bancaria y financiera se basa en una serie de \marcar{aspectos teóricos} que sirven para justificar su existencia e informar su implementación. Los tres fenómenos principales que dan lugar a necesidad de regular son las asimetrías de información, los problemas a la hora de gestionar el riesgo por los agentes participantes, y las externalidades. En las situaciones de \textbf{asimetría de información}, una de las partes de un contrato posee una ventaja informacional cuyo aprovechamiento reduce el bienestar de la otra parte. La parte en desventaja tiende a reducir su participación en el mercado o a abstenerse de contratar de forma absoluta. Este comportamiento resulta en una desaparición del mercado para un determinado contrato financiero. La selección adversa y el riesgo moral son dos fenómenos habituales en el contexto financiero con consecuencias negativas. La \underline{selección adversa} consiste en la desaparición de un mercado cuando el comprador no puede conocer ex-ante las características del bien y los vendedores tienen incentivos a vender bienes de calidad inferior a la esperada. En el mercado de crédito, se trata de problema habitual. Los prestamistas no tienen toda la información necesaria para valorar la solvencia del prestatario, y aumentan el interés exigido. Este aumento del interés desincentiva a los prestatarios más solventes, pero mantiene la demanda de financiación para proyectos muy arriesgados, iniciando un ciclo de restricciones crecientes de liquidez y riesgos cada vez mayores. El \underline{riesgo moral} consiste en la existencia de asimetrías de información que se manifiestan una vez las partes se han obligado contractualmente. La parte que debe llevar a cabo una determinada acción a cambio de un precio, denominada agente, tiene incentivos a actuar de tal manera que aumentando su beneficio, provoque un perjuicio al agente principal. En el contexto financiero y bancario, el riesgo moral se manifiesta generalmente cuando los deudores actúan en contra de los intereses de los acreedores en su propio beneficio, utilizando información que los acreedores no tienen. Ello da lugar a restricciones al crédito, ya que los acreedores no confían en que los deudores no tomarán medidas que les perjudicarán. En estas situaciones, el equilibrio resultante es subóptimo, y existe margen para la introducción de regulación.

El problema de la \textbf{gestión deficiente del riesgo} se manifiesta como concentración excesiva de las inversiones en determinados activos, y en la realización de inversiones demasiado arriesgadas dado el perfil del inversor. La concentración excesiva de riesgos es la otra cara de la diversificación insuficiente. Las causas de este fenómeno son políticas, macroeconómicas o microeconómicas. Presiones políticas pueden inducir inversiones en proyectos de dudoso valor añadido, u obligar a entidades de crédito a invertir en deuda soberana nacional. Las causas macroeconómicas pueden deberse a una gran variedad de fenómenos, pero en general tienen que ver con distorsiones en los precios de los activos que estimulan la demanda de grupos de activos y con ello, aumentan la concentración del riesgo. Las razones microeconómicas de la concentración tienen que ver con la estructura de incentivos de los gestores de la empresa. La regulación financiera puede potencialmente introducir distorsiones adicionales que agraven el problema en éste ámbito. La toma de riesgos excesivos supone la inversión en activos que aumentan el riesgo global de un agente el mercado financiero, aumentando la posibilidad de daños a otros agentes expuestos al agente original y potencialmente, a todo el sistema financiero.

El concepto de \textbf{externalidad} hace referencia a los efectos positivos o negativos que la actuación de un agente tiene sobre otros agentes que no tienen poder de decisión sobre el fenómeno causante del efecto. Las externalidades negativas son aquellos costes que exceden los costes privados que en que un agente incurre como resultado de sus actividades. Cuando los agentes no internalizan esos costes adicionales en su proceso de decisión, sus actividades inducen resultados subóptimos y se abre un espacio para la actuación regulatoria. La toma de riesgos sistémicos es un ejemplo de externalidad negativa en el sistema financiero. Determinados agentes pueden verse incentivados a generar exposiciones cruzadas de sus balances de tal manera que shocks idiosincráticos tengan consecuencias sistémicas. El carácter procíclico del crédito es otro ejemplo de externalidad. La reducción del crédito tiene un efecto externo sobre la confianza de los agentes en la liquidez de sus contrapartes, y puede degenerar en una desaparición súbita y generalizada de la liquidez del sistema financiero. Los rescates bancarios tienen un coste directo sobre las finanzas públicas, pero pueden tener también un efecto de externalidad negativo sobre otra entidades de crédito si el precio de la deuda soberana se reduce bruscamente como consecuencia del rescate, y si la deuda soberana del estado que rescata tiene un peso elevado en su balance. Esta externalidad puede dar lugar a un círculo vicioso: el deterioro de la posición de una entidad de crédito obliga al estado a rescatarlo, lo que a su vez reduce el precio de la deuda pública, deteriorando el activo de otros bancos y provocando nuevos rescates que retroalimentan el proceso. 

Los \textbf{instrumentos regulatorios} consisten, \textit{grosso modo}, en la imposición de condiciones a la estructura del balance, requisitos de liquidez mínima, la inspección y supervisión de la información financiera, restricciones relativas a la composición de las carteras de inversión, regulación de la propiedad de las entidades de crédito e implementación de mecanismos de seguridad para contener el daño en situación de crisis. Las condiciones sobre la \underline{estructura del balance} son habituales en la regulación bancaria. Las regulaciones basadas en los \textit{ratios de capital} imponen un valor mínimo al cociente entre el capital y el volumen total de activos ponderados en función del riesgo estimado. Este tipo de instrumentos regulatorios sufren la complejidad de medir y calificar el capital, así como el problema de la ponderación de los activos en función del riesgo. Para ello se utilizan criterios fijos o modelos de cuantificación del riesgo. Los \textit{ratios de apalancamiento} establecen una relación entre el capital y el tamaño total del balance sin ponderar por riesgo, añadiendo en ocasiones cuantificaciones de la exposición total. Los \textit{ratios de liquidez} obligan a las entidades a mantener una determinada cantidad de activos líquidos suficiente para hacer frente a salidas de caja durante un periodo dado. En esta categoría de instrumentos regulatorio se encuentran también los ratios de liquidity mismatch, que imponen restricciones sobre la relación entre el vencimiento de activos y pasivos. Las medidas regulatorias relativas a la \underline{inspección y supervisión} incluyen la fijación de estándares sobre presentación de datos, la periodicidad de los informes, la transparencia en relación a los pasivos fuera de balance y los requisitos de información al público. Este tipo de instrumentos regulatorios tienen por objetivo permitir a los mercados una valoración del capital de las entidades más ajustada al riesgo subyacente. Las \underline{restricciones de cartera} tratan de evitar una concentración excesiva de la exposición a determinados sectores con un doble fin: inducir diversificación para la entidad en caso de evolución desfavorable en el sector de exposición, y reducir la dependencia del sector de una determinada entidad o grupo de entidades. Las \underline{restricciones de propiedad de las instituciones financieras} imponen restricciones a las fusiones y adquisiciones de entidades del sistema financiero con el fin de evitar crecimiento y concentraciones excesivas que pueden aumentar el riesgo sistémico. \underline{Los mecanismos de seguridad} pretenden evitar la aparición de crisis financieras interviniendo directamente sobre las causas inmediatas. Los \textit{seguros de depósitos} son garantías sobre las cantidades mínimas que los titulares de depósitos bancarios pueden retirar en caso de iliquidez o insolvencia de una entidad de crédito. La mera existencia del mecanismo desincentiva las retiradas masivas de efectivo y generan un círculo virtuoso que reduce la probabilidad de pánicos bancarios generalizados. La hipótesis del \textit{too-big-to-fail} caracteriza una determinada política regulatoria según la cual algunos bancos son demasiado importantes a nivel sistémico como para quebrar. Si los mercados entienden que existe una garantía implícita de la autoridad pública respecto determinadas entidades que reduce la posibilidad de quiebra, el precio de los activos emitidos por esa entidad sufrirá distorsiones y se incentivará el crecimiento excesivo de entidades. Mecanismos regulatorios destinados a hacer creíble la inexistencia del too-big-to-fail pueden contribuir a desincentivar crecimientos excesivos de los balances bancarios. Por último, la regulación de las \textit{inyecciones de liquidez} en momentos de crisis son un instrumento regulatorio destinado a generar confianza en los mercados. Por otro lado, son susceptibles de provocar similares distorsiones en los precios de los activos.

La \marcar{regulación bancaria y financiera en la práctica} está compuesta por normas jurídicas de gran complejidad y sofisticación, sujetas a innumerables excepciones, implementaciones particulares e interpretaciones. En el \underline{siglo XIX} los grandes debates regulatorios se centraban en la necesidad de contar con un banco central, la doctrina de las \textit{real bills} y los debates sobre la prerrogativa de creación de dinero. Dada la internacionalización de los mercados financieros y la actividad bancaria, la regulación superó progresivamente el ámbito exclusivamente nacional y se internacionalizó. En la \underline{actualidad} existen una serie de instituciones o cuasi-instituciones internacionales con un papel central a la hora de definir las líneas generales de la regulación financiera. El G-20 es una reunión anual de las principales economías del mundo en la cual se debaten las actuaciones de política económica a nivel global y de manera creciente, la coordinación de la regulación financiera internacional. El Banco Internacional de Pagos sigue jugando un papel muy importante en la coordinación de los bancos centrales y sirve de sede al Financial Stability Board creado en 1999 con el objetivo de coordinar la respuesta regulatoria mundial a situaciones de crisis y emitir recomendaciones relativas a la regulación global de los mercados financieros tales como las listas de bancos de importancia sistémica objeto de especiales medidas de precaución según Basilea III. El Comité de Basilea fue creado en 1975 para servir de foro en el que debatir el diseño y la mejora de la regulación bancaria. Aunque no emite legislación vinculante, ha sido el creador de los Acuerdos de Basilea que examinaremos posteriormente. Estos tres acuerdos han sido determinantes en la evolución de la regulación bancaria en las principales economías del mundo. En la Unión Europea, el proceso de Lamfalussy iniciado en 2001 tiene por objetivo diseñar e implementar una regulación común a nivel europeo en todas las áreas de la actividad financiera. 

Los \textbf{Acuerdos de Basilea} son un conjunto de recomendaciones y guías de implementación que tratan de mejorar la regulación bancaria para reducir el impacto de las crisis financieras y disminuir la probabilidad de que se produzcan. El Acuerdo de \underline{Basilea I} fue aprobado en 1988. Introdujo el ratio de capital y la ponderación por riesgo de los activos. El ratio de capital mínimo es del 8\% del volumen de activos ponderados por riesgo, y de ese 8\%, la mitad debe ser capital social y reservas desembolsadas. En cuanto a la ponderación de activos, se establecen cinco categorías basadas en el riesgo de crédito que van desde el efectivo y la deuda pública en moneda nacional hasta la deuda corporativa, la deuda soberana y los demás activos de manera subsidiaria. Aunque el sistema de ponderación tenía por objetivo ser sencillo y no dar demasiado margen al incumplimiento, su relativa rigidez acaba provocando la aparición de distorsiones. En todo caso, el acuerdo fue un éxito en la medida en que mostró que era posible establecer un marco regulatorio bancario y la mayoría de los bancos alcanzaron la ratio mínima establecida en el periodo de transición.

En 2004 se culmina el \underline{Acuerdo de Basilea II}. Este acuerdo introduce cambios de gran calado respecto al acuerdo anterior, y establece un marco más integrado de regulación. El Acuerdo define tres pilares de actuación: i) \textit{restricciones y mínimos de capital} con el objetivo de controlar los riesgos de forma directa (englobando las medidas del Acuerdo de Basilea I), ii) herramientas de \textit{supervisión por el regulador} y iii) herramientas de \textit{disciplina a través de transparencia} mediante un aumento de la información disponible al público. En la ponderación de activos, se introduce ahora la valoración del riesgo operacional y del riesgo de mercado además del riesgo de crédito considerado anteriormente, y se establece la posibilidad de que los bancos valoren el riesgo de sus activos mediante modelos internos personalizados. Mientras que el riesgo de crédito hace referencia a la posibilidad de que los obligados con el banco incumplan su compromiso de pago, el riesgo de mercado hace referencia a movimientos adversos en el precio de los activos y el riesgo operacional hace referencia a la posibilidad de sufrir pérdidas derivadas de la operativa corriente del banco tales como cambios en la demanda de mercado, cambios legales, inestabilidad política, etc... El capital mínimo se mantiene en un 8\%. 

El \underline{Acuerdo de Basilea III} se aprueba en diferentes episodios en 2010 y 2011 y se espera su implementación completa a finales de 2019, tras varias extensiones de un plazo original desde 2013 y a 2015. Los ratios de capital cambian en esta tercera versión. El ratio de capital mínimo se sitúa en el 4,5\% en forma de Common Equity Tier 1. A esta cuantía se añade un 2,5\% en concepto de bufer de conservación de capital, de tal manera que en condiciones normales, una entidad debe contar con al menos un 7\% de los activos ponderados por riesgo como CET1. Cuando el CET1 se sitúe entre el 4,5\% y el 7\%, deben aplicarse restricciones a la distribución de dividendos. Además, se introduce la posiblidad de aplicar un margen anticíclico adicional del 0 al 2,5\%. Además de lo anterior, el Tier 1\% debe ser superior al 6\% en todo momento, y la suma del Tier 1 y el Tier 2 superior al 8\% en todo momento. Se introduce un ratio de apalancamiento mínimo del 3\% que toma como denominador los activos consolidados totales. Además, se plantean también dos ratios de liquidez. El ratio de cobertura de liquidez o \textit{Liquidity Coverage Ratio} define la obligación de mantener activos líquidos de alta calidad que garanticen las salidas netas de efectivo en un plazo de 30 días. El ratio de financiación estable neta o \textit{Net Stable Funding Ratio} define la obligación de mantener unas fuentes de financiación estable disponibles superiores a las necesarias durante un año en un contexto de stress en los mercados financieros. Los bancos de importancia sistémica o G-SIBs están sujetos a requisitos adicionales de capital en términos de CET1 sobre el APR, y en términos de CET1 sobre el Tier 1. En estos momentos, un nuevo paquete de reformas en ocasiones denominado Basilea IV se encuentra en fase de debate. Previsiblemente, el ratio de capital será aumentado y se introducirán modificaciones sobre los modelos de valoración del riesgo permitidos.

Si los Acuerdos de Basilea establecen el marco regulatorio a nivel global, es necesario tener en cuenta que la regulación vinculante corresponde a la potestad legislativa de los estados. En \textit{Estados Unidos}, la ley Glass-Steagal de 1933 estableció el principio de separación entre banca comercial y banca de inversión hasta su derogación en 1999 bajo críticas de excesiva rigidez y posibilidad de explotación de numerosos agujeros legales. La derogación ha sido a su vez objeto de críticas que le atribuyen el aumento de los riesgos incurridos por la banca comercial en el periodo 1999-2007. En 2010 se aprueba la Ley Dodd-Frank, en gran medida como respuesta a la crisis financiera. Se establece el Financial Stability Oversight Council para monitorizar los riesgos del sistema financiero, se introducen cambios en la regulación del seguro de depósitos americanos (FDIC) y se introduce la llamada Volcker Rule. La Ley Glass-Stegall había establecido una separación clara entre banca comercial y banca de inversión: la banca de inversión tenía prohibido captar depósitos y otorgar préstamos, y la banca comercial tenía prohibido suscribir emisiones de activos y negociar instrumentos financieros. La regla de Volcker es en cierto modo comparable, al prohibir a los bancos beneficiarios del sistema de garantía de depósitos el negociado de instrumentos financieros en su propio nombre y riesgo conocido como \textit{propietary trading}.

En el contexto Europeo, el informe Larosière de 2009 propuso la creación del \textit{Sistema Europeo de Supervisión Financiera}. Se trata de un conjunto de instituciones destinadas a monitorizar las condiciones del sistema financiero europeo y coordinar las actuaciones. El \textit{European Systemic Risk Board (ESRB)} es la institución encargada de la supervisión macroprudencial y depende del BCE. La regulación a nivel microeconómico depende tres instituciones: la European Banking Authority, la European Securities and Market Authority y la European Insurance and Occupational Pensions, encargadas respectivamente de regulación bancaria, la regulación del mercado de valores y la regulación de los fondos de pensiones. La implementación del Acuerdo de Basilea III se ha llevado a cabo mediante la Capital Requirements Directive IV y la Capital Requirements Regulation II. Estas normativas introducen las reglas prudenciales comunes a todos los sistemas bancarios europeos, establecen restricciones a la remuneración de directivos y el gobierno corporativo, establecen un colchón de capital para instituciones sistémicas de entre 1\% y el 3,5\% CET1, e introducen la posibilidad de implementar a nivel nacional un colchón sistémico facultativo. La normativas MiFID II y MiFIR II establecen el marco legal europeo en relación al negociado de instrumentos financieros. Aunque aprobados en 2014, entraron en vigor en enero de 2018. La regulación comunitaria de los mercados de instrumentos financieros se extiende así más allá del segmento equity, regulando también los bonos, las commodities y los derivados. El objetivo último es la consecución de un mercado único europeo de capitales.

La \textit{Unión Bancaria} es uno de los proyectos más ambiciosos de la regulación comunitaria. El objetivo consiste en establecer un mercado de financiación bancaria único a nivel europeo, reduciendo la atomización de los mercados e incentivando la diversificación de las carteras bancarias. La Unión Bancaria se plantea en torno a tres pilares, de los cuales dos han sido implementados y un tercero se encuentra en fase de discusión, con un futuro aún incierto. Los dos primeros son el Mecanismo Único de Supervisión y el Mecanismo Único de Resolución. El Mecanismo Único de Supervisión fue creado en 2014 tras la conclusión de una batería de test de stress a los bancos de la UE. En el marco del SSM, el BCE es el encargado de supervisar las entidades de mayor tamaño, y a los bancos nacionales la supervisión de las entidades menores, aunque el BCE mantiene la potestad de supervisar cualquier entidad en un momento dado y también delega en muchos casos la supervisión de sus entidades a los bancos nacionales. El Mecanismo Único de Resolución es un marco institucional que trata de reducir el impacto de las quiebras bancarias sobre los contribuyentes y la economía de la Unión a través de una Junta Única de Resolución y un Fondo Único de Resolución financiado por las entidades bancarias. Además de la legislación anteriormente referida, es especialmente importante en el contexto de la resolución bancaria la Banking Resolution and Recovery Directive (BRRD) de 2014. El Seguro de Depósitos Común es el tercer pilar de este sistema, pero se encuentra aún en fase de debate entre los estados miembros. Sí se encuentra en vigor una directiva europea que armoniza las prestaciones de los seguros de depósito nacionales. 

Más allá del ámbito bancario y de los mercados financieros, la crisis financiera puso de manifiesto la necesidad de renovar la regulación de las agencias de calificación. La ESMA es la entidad competente para proponer regulación en esta materia. La directiva 2013/14 establece requisitos sobre los modelos utilizados para valorar la solvencia y trata de atajar los conflictos de interés que dañan las valoraciones de las agencias de calificación. Los mercados de activos financieros basados en tecnologías \textit{blockchain} son aun terra incognita para el regulador. Las autoridades europeas y algunas voces dentro del G-20 han expresado su disposición a regular las criptomonedas en el futuro si continúan creciendo en valor de mercado y volumen de transacción, pero aún no se ha concretado ninguna regulación a nivel internacional. 

A lo largo de la exposición hemos expuesto los elementos principales de la regulación bancaria y financiera: quiénes son los sujetos involucrados, en qué principios teóricos se fundamenta la regulación y qué regulación existe en la práctica. Es preciso tener en cuenta que el sistema financiero se transforma constantemente y la imposición de regulación es uno de los motores de ese cambio. Otros factores son los tipos de interés bajos que han predominado durante la última década y tras la crisis financiera, la emergencia de competidores no bancarios en el negocio de la intermediación financiera o las nuevas tecnologías tales como el blockchain o las transacciones a través de móviles. En la Unión Europea, la regulación financiera y bancaria será en el futuro uno de los factores determinantes para la consolidación del mercado único y la estabilidad económica.



%El sector financiero de las economías actuales lleva a cabo la intermediación financiera y la gestión de los flujos monetarios entre agentes económicos de diversa índole. Por su presencia central en el emparejamiento de ahorradores e inversores, y compradores y vendedores de toda clase de activos financieros y reales, el sistema bancario y las instituciones financieras tienen la capacidad de provocar enormes fluctuaciones de la renta y la riqueza en situaciones de crisis, como ningún otro sector de la economía. Además, el elevado grado de interconexión e interdependencia de los agentes que lo constituyen hacen al sistema financiero especialmente vulnerable a los llamados shocks sistémicos. Los eventos de carácter sistémico son aquellos fenómenos que afectan a la posición crediticia de una entidad y que pueden desencadenar un proceso de deterioro de la posición de otras entidades, provocando en suma una retirada generalizada del crédito y la confianza en el sistema financiero que acaba por ahogar la economía real. Estas situaciones de ahogamiento de la economía real como resultado de crisis financieras y bancarias es resultado de la paralización de las dos funciones básicas del sistema financiero.
%
%Por ello, los estados regulan el sistema financiero tratando de eliminar los factores que contribuyen a este tipo de crisis. Dos factores íntimamente relacionados se encuentran detrás de las crisis: asimetrías de información y asunción de riesgos excesivos. La regulación trata de reducir tanto la fuerza con la que actúan estos factores como el impacto que provocan en forma de crisis. Sin embargo, la regulación genera también respuestas inesperadas y distorsión de los incentivos que pueden a su vez incrementar las asimetrías de información. Por ejemplo, los seguros de depósitos pueden generar problemas añadidos de riesgo moral, así como los rescates bancarios a entidades \comillas{too-big-to-fail}.
%
%La crisis financiera de 2007 a 2009 supuso un fuerte impacto para la relativa estabilidad financiera que había predominado durante las décadas posteriores a la Segunda Guerra Mundial. Generó una respuesta regulatoria en las principales economías mundiales, así como a nivel internacional con las conclusiones del Comité de Basilea.

\seccion{Preguntas clave}
\begin{description}
    \item[Sistema financiero] ¿Cómo funciona el sistema bancario y financiero?
    \item[Sistema financiero] ¿Por qué se producen las crisis bancarias?
    
    \item[Teoría de la regulación] ¿Qué es la regulación financiera y bancaria?
    \item[Teoría de la regulación] ¿Por qué regular?
    \item[Teoría de la regulación] ¿Qué efectos positivos tiene la regulación?
    \item[Teoría de la regulación] ¿Qué efectos negativos tiene la regulación?
    
    \item[Práctica de la regulación] ¿Qué instrumentos de regulación existen?
    \item[Práctica de la regulación] ¿Cómo se regula?
    \item[Práctica de la regulación] ¿Qué trayectoria ha seguido la regulación?
    \item[Práctica de la regulación] ¿Qué respuesta regulatoria ha tenido lugar ante la crisis?
\end{description}

\esquemacorto

\begin{esquema}[enumerate]
	\1[] \marcar{Introducción}
		\2 Contextualización
			\3 Crisis 2007 y otras
			\3 Regulación
		\2 Objeto
			\3 Sistema financiero
			\3 Aspectos teóricos
			\3 Cómo se regula
		\2 Estructura
			\3 Sistema financiero y bancario
			\3 Aspectos teóricos
			\3 Regulacion en la práctica
	\1 \marcar{Sistema financiero y bancario}
		\2 Funciones del sistema financiero
			\3 Transformar plazos
			\3 Transferencia de riesgos
			\3 Sistema de pagos
		\2 Agentes relevantes para el regulador financiero
			\3 Clientes
			\3 Bancos
			\3 Aseguradoras
			\3 Cámaras de compensación
			\3 Agencias de calificación
			\3 Otras instituciones
	\1 \marcar{Aspectos teóricos de la regulación}
		\2 Asimetrías de información
			\3 Selección adversa
			\3 Riesgo moral
		\2 Gestión del riesgo
			\3 Concentración de riesgos
			\3 Riesgos excesivos
		\2 Externalidades
			\3 Idea clave
			\3 Riesgo sistémico
			\3 Prociclicidad del crédito
			\3 Sector público
		\2 Equilibrios múltiples
			\3 Idea clave
			\3 Formulación
			\3 Soluciones a equilibrio de pánico bancario
		\2 Instrumentos de regulación
			\3 \underline{Microprudencial}
			\3 Idea clave
			\3 Estructura del balance
			\3 Requisitos de liquidez
			\3 Inspección y supervisión por regulador
			\3 Restricciones de cartera
			\3 Utilización de CCP
			\3 \underline{Macroprudenciales}
			\3 Idea clave
			\3 Restricciones de propiedad de las instituciones
			\3 G-SIBs: requisitos de capital aumentados
			\3 LTV -- Loan-to-value Ratios
			\3 Búferes contracíclicos de capital
			\3 Reservas mínimas en divisa
			\3 Provisiones dinámicas
			\3 \underline{Mecanismos de seguridad}
			\3 Idea clave
			\3 Test de estrés
			\3 Seguro de depósitos
			\3 TLAC y MREL
			\3 Evitar too-big-to-fail
			\3 Inyecciones de liquidez
			\3 Garantías de depósitos
			\3 Corralito
	\1 \marcar{Regulación en la práctica}
		\2 Evolución
			\3 Siglo XIX
			\3 Siglo XX
			\3 Actualidad
		\2 Acuerdos de Basilea
			\3 Basilea I (1988)
			\3 Basilea II (2004)
			\3 Basilea III (2010-2012)
			\3 Reformas de 2017/Basilea IV
		\2 Estados Unidos
			\3 Glass-Steagall (1933)
			\3 Dodd-Frank (2010)
			\3 Reforma de Dodd-Frank (2018)
		\2 Unión Europea
			\3 Sistema Europeo de Supervisión Financiera (ESFS)
			\3 \underline{Unión Bancaria}
			\3 Mecanismo Único de Supervisión
			\3 Mecanismo Único de Resolución
			\3 \underline{Single rulebook}
			\3 Directiva de Fondos de Garantía de Depósitos (DGDS) 
			\3 BRRD y SRMR
			\3 CRR I -- Capital Requirements Regulation I (2013)
			\3 CRD IV -- Capital Requirements Directive IV (2013)
			\3 CRR II/CRD V (2019)
			\3 IFR/IFD (2019)
			\3 MiFID II y MiFIR (Markets in Financial Instruments)
			\3 EMIR -- European Market Infrastructure Regulation
			\3 Transición Eonia a ESTR
		\2 Otras
			\3 Medio ambiente y regulación financiera
			\3 Asia
			\3 Agencias de calificación
			\3 Blockchain
		\2 Retos de la regulación
			\3 Shadow banking
			\3 Criptomonedas
			\3 Narrow banking
			\3 Central Bank Digital Currency
			\3 Consecuencias indeseadas de la regulación
			\3 Regulator shopping
			\3 Forum shopping
	\1[] \marcar{Conclusión}
		\2 Recapitulación
			\3 Sistema financiero
			\3 Aspectos teóricos de la regulación
			\3 Regulación en la práctica
		\2 Idea final
			\3 Sistema financiero y bancario como proceso endógeno
			\3 Transformaciones sector bancario
			\3 Unión Europea

\end{esquema}

\esquemalargo

\begin{esquemal}
	\1[] \marcar{Introducción}
		\2 Contextualización
			\3 Crisis 2007 y otras
				\4 Impacto global
				\4 Costes fiscales
				\4 Bajadas PIB
				\4 Deterioro general de la confianza
			\3 Regulación
				\4 Objetivos:
				\4[] reducir frecuencia
				\4[] reducir efectos
				\4[] reducir distorsiones o no agravarlas
				\4 Obstáculos
				\4[] muy elevada complejidad
				\4[] interrelación ec. financiera y real
				\4[] conflictos de interés
				\4[] problemas políticos
		\2 Objeto
			\3 Sistema financiero
				\4 Quienes son los sujetos de la regulación
			\3 Aspectos teóricos
				\4 Por qué regular
				\4 Qué efectos positivos y negativos
				\4 De qué instrumentos regulatorios disponen los reguladores
			\3 Cómo se regula
				\4 Global
				\4 Europa
				\4 Qué regulación en el futuro
		\2 Estructura
			\3 Sistema financiero y bancario
			\3 Aspectos teóricos
			\3 Regulacion en la práctica
	\1 \marcar{Sistema financiero y bancario}
		\2 Funciones del sistema financiero
			\3 Transformar plazos
				\4 Ahorradores quieren disponibilidad a c/p
				\4 Inversores quieren financiarse a l/p
				\4[$\then$] Sistema financiero transforma plazos
				\4[] Proveen financiación a largo plazo
				\4[] $\to$ A proyectos de inversión
				\4[] Ofrecen activos líquidos de corto plazo
				\4[] $\to$ A depositantes
				\4 Bancos intermedian entre ahorradores e inversores
				\4[] Toman prestado a corto plazo
				\4[] Prestan a largo plazo
				\4[] $\then$ Liquidity mismatch es fuente de tensiones
			\3 Transferencia de riesgos
				\4 Reducción de riesgos
				\4[] Int. finan. asumen riesgo de contrapartida
				\4[] Entre prestamista y prestatario
				\4 Economías de escala informativas
				\4[] Intermediarios reducen los costes de información
				\4[] $\to$ Conocer solvencia de prestatario es costoso
				\4[] $\to$ Potenciales economías de escala
			\3 Sistema de pagos
				\4 Agentes necesitan:
				\4[] Financiarse
				\4[] Ahorrar de forma segura
				\4[] Intercambiar pagos por transacciones presentes
				\4 Sistema financiero:
				\4[] permite canalizar transferencias
				\4[] Asegura ejecución de transferencias
				\4[$\then$] Sistema financiero es ``sistema circulatorio''
		\2 Agentes relevantes para el regulador financiero
			\3 Clientes
				\4 Inversores no financieros
				\4 Empresas no financieras
				\4[] Necesitan sistema de pagos
				\4[] $\to$ Para llevar a cabo transacciones
				\4[] $\to$ Obtener financiación
				\4 Personas físicas
				\4[] $\to$ Suavizar senda de consumo intertemporal
			\3 Bancos
				\4 Bancos centrales
				\4 Bancos comerciales
				\4 Bancos de inversión
				\4 Bancos hipotecarios
				\4 Otros
				\4[] Cajas de ahorros, mutual banks, bancos cooperativos, banca islámica...
			\3 Aseguradoras
				\4 Transferencia del riesgo
			\3 Cámaras de compensación
				\4 Reducción de riesgo de contrapartida
				\4 Se interponen entre partes
				\4[] Para liquidar un contrato financiero
			\3 Agencias de calificación
				\4 Economías de escala en valoración del riesgo
				\4 Negocio consiste en señalizar bien la solvencia
			\3 Otras instituciones
				\4 Fondos de inversión
				\4[] Sociedades de inversión colectiva
				\4[] Invierten en activos financieros
				\4[] Búsqueda de rendimiento para participantes
				\4 Shadow banking
				\4[] Intermediación bancaria y servicios similares
				\4[] Al margen del sistema bancario regulado
	\1 \marcar{Aspectos teóricos de la regulación}
		\2 Asimetrías de información
			\3 Selección adversa
				\4 Información asimétrica sobre crédito
				\4[] Prestamista ignora tipo de prestatario
				\4[] $\to$ Dificil discriminar entre tipos de prestatario
				\4[] $\to$ Aumenta precio del crédito
				\4[] Crédito demasiado caro para prestatarios de calidad
				\4[] $\to$ Salen de mercado
				\4[] $\to$ Prestamistas prefieren no prestar
				\4[] $\to$ No son capaces de distinguir solvencia
				\4[$\then$] Deterioro de condiciones y calidad de crédito
			\3 Riesgo moral
				\4 Deudores actúan en contra de acreedores
				\4[] Poseen información que acreedores no tienen
				\4[] Utilizan para maximizar sus preferencias
				\4[] De forma incompatible con preferencias de acreedores
				\4[$\then$] Restricciones al crédito por desconfianza
				\4[] $\to$ Acreedores no pueden confiar en deudores
				\4[] $\to$ Desconfianza impide inversión
				\4[] $\to$ Riesgo moral induce resultados subóptimos
		\2 Gestión del riesgo
			\3 Concentración de riesgos
				\4 Diversificación insuficiente
				\4 Razones políticas:
				\4[] deuda soberana
				\4[] Bancos nacionales forzados a invertir nacional
				\4 Razones microeconómicas:
				\4[] incentivos de dirección
				\4[] regulación introduce distorsiones
			\3 Riesgos excesivos
				\4 Disminución de rentabilidad
				\4 Políticas de incentivos
				\4 Gestión deficiente
				\4 Riesgo de crisis sistémicas
		\2 Externalidades
			\3 Idea clave
				\4 Los costes sociales exceden los costes privados
				\4[] Los agentes no internalizan el coste de sus acciones
				\4[$\Rightarrow$] Margen de actuación para regulador
				\4 Diferentes problemas de falta de internalización
			\3 Riesgo sistémico
				\4 Exposiciones cruzadas de los balances
				\4 Shock idiosincrático
				\4[] $\to$ Tiene consecuencias sistémicas
			\3 Prociclicidad del crédito
				\4 Crisis de confianza reducen crédito
				\4 Reducción de crédito reduce confianza
				\4[$\then$] Ciclo perverso de falta de liquidez
			\3 Sector público
				\4 Rescates bancarios reducen margen del SP
				\4 Deuda del SP para rescatar bancos
				\4[] $\to$ Deterioro finanzas públicas
				\4[] $\to$ Deuda pública en balance bancario
				\4[] $\then$ Deterioro del activo bancario
				\4[] $\then$ Estados empujados a rescatar bancos
				\4[] $\then$ Ciclo perverso deuda pública-balance
		\2 Equilibrios múltiples\footnote{Basado en Diamond (2007) que explica Diamond y Dybvig (1983).}
			\3 Idea clave
				\4 Agentes demandan activos líquidos
				\4 Bancos:
				\4[] Proveen activos líquidos
				\4[] Adquieren activos ilíquidos
				\4[] $\to$ Mismatch entre liquidez de oferta y demanda
			\3 Formulación
				\4 Bancos:
				\4[] Se endeudan a c/p con depositantes
				\4[] $\to$ Depositantes pueden retirar depósitos en todo momento
				\4[] Invierten en proyectos de l/p
				\4[] $\to$ No pueden recuperar VAN en cualquier momento
				\4[] Mantienen fracción en forma de activos de c/p
				\4[] $\to$ Para cubrir retiradas de depositantes
				\4 Depositantes
				\4[] Adquieren activos líquidos de c/p a bancos
				\4[] $\to$ Depósitos bancarios
				\4[] Liquidan fracción de activos de c/p en cada periodo
				\4 Equilibrio I: funcionamiento normal
				\4[] Depositantes liquidan fracción habitual
				\4[] Banco puede cubrir retiradas con reservas
				\4[] $\to$ Nadie percibe riesgo de no poder liquidar
				\4[] $\then$ No hay pánico bancario
				\4[] $\then$ Banco mantiene funcionamiento
				\4 Equilibrio II: pánico bancario
				\4[] Depositantes estiman banco no puede devolver depósitos
				\4[] Aumenta retirada de depósitos por encima de habitual
				\4[] Banco no puede hacer frente a retiradas
				\4[] $\to$ Se vuelve cierto que banco no puede devolver depósitos
				\4[] $\then$ Aumenta retirada de depósitos de forma masiva
				\4[] $\then$ Banco ilíquido debe cerrar aunque solvente a l/p
				\4[] $\then$ Posible contagio a otros bancos también solventes
			\3 Soluciones a equilibrio de pánico bancario
				\4 Seguros de depósito
				\4 Promesa de dejar quebrar
				\4 Provisión de liquidez de emergencia
		\2 Instrumentos de regulación
			\3 \underline{Microprudencial}
			\3 Idea clave
				\4 Asegurar supervivencia de instituciones individuales
				\4 Evitar quiebras y suspensiones de pagos
				\4 Énfasis en entidades de manera indivudal
			\3 Estructura del balance
				\4 Ratio de capital
				\4[] Ratio entre capital y activos ponderados por riesgo
				\4[] $\text{RC} = \frac{\text{Capital}}{\text{Activos ponderados por riesgo}}$
				\4[] Diferentes formas de:
				\4[] $\to$ Medir y calificar capital
				\4[] $\to$ Ponderar activos por riesgo en denominador
				\4[] Diferentes formas de ponderar
				\4[] $\to$ Uso de modelos internos
				\4[] $\to$ Criterios fijos
				\4[] A mayor riesgo, mayor ponderación
				\4 Ratio de apalancamiento
				\4[] $\text{RA} = \frac{\text{Capital Tier 1}}{\text{Exposición}} = \frac{\text{Capital Tier 1}}{\text{Activo total + Exposición off-balance}}$
			\3 Requisitos de liquidez
				\4 Ratios de liquidez
				\4[] Relación entre:
				\4[] $\to$ activos líquidos
				\4[] $\to$ necesidades de fondos en periodo dado
				\4[] $\then$ Mantener liquidez bajo stress
				\4 Ratios de liquidity mismatch
				\4[] Medida del liquidity mismatch
				\4[] Relación entre vencimientos activos y pasivos
			\3 Inspección y supervisión por regulador
				\4 Estándares de presentación de información
				\4 Obligación de presentar informes periódicos
				\4 Presentación de información pasivos off-balance
			\3 Restricciones de cartera
				\4 Restricciones a la concentración sectorial
				\4[] $\to$ Aumentar diversificación
				\4[] $\then$ Reducir riesgo de cola de la entidad
			\3 Utilización de CCP
				\4 Derivados financieros
				\4 Central Counterparty
				\4 Se interpone entre las dos contrapartes
				\4 Contrata con ambas
				\4 Toma garantías de todos los actores
				\4[] Exigen adicionales si aumenta posición negativa
				\4 Posible consecuencias macro desfavorables
				\4[] Aumento de márgenes exigidos en momentos de estrés
				\4[] $\to$ Presión sobre tipos de interés
				\4[] $\then$ Sequía de mercado
				\4[] $\then$ Cash crunch
				\4[] $\then$ ...
			\3 \underline{Macroprudenciales}
			\3 Idea clave\footnote{Ver \href{https://www.imf.org/external/pubs/ft/wp/2014/wp14214.pdf}{Claessens (2014): An Overview of Macroprudential Policy Tools} y \href{https://www.brookings.edu/blog/up-front/2020/02/11/what-are-macroprudential-tools/\#cancel}{Brookings (2020): What are macropudential tools?}.}
				\4 Evitar prociclicidad
				\4[] De expansión del crédito
				\4[] De crisis financieras
				\4 Enfoque de SFin en su conjunto
				\4[] Asegurar estabilidad
				\4[] $\to$ Evitar ajustes bruscos de balances
				\4[] Reducir riesgos de conjunto de balances
				\4 Amortiguar efectos de shocks sistémicos
				\4[] Afecta por igual a todas las entidades del SFIn
				\4 Reducir externalides negativas de medidas microprudencial
				\4[] Medidas para aumentar supervivencia de una entidad
				\4[] $\to$ Pueden aumentar riesgos para resto de entidades
				\4[] -- Ejemplo: aumento de mora bancaria
				\4[] $\to$ Regulación micro: necesario aumentar capital
				\4[] $\then$ Captar capital en mercados
				\4[] $\then$ Reducir crédito
				\4[] Reducción de crédito
				\4[] $\to$ Aumenta quiebras
				\4[] $\then$ Aumento de mora
				\4[] Resultado:
				\4[] $\to$ Microprudencial por sí sólo aumenta inestabilidad
				\4[] -- Ejemplo: fondos de capital de CCParty
				\4[] $\to$ Alimentado con aportaciones de bancos
				\4[] $\to$ Amortigua shocks micro a un CCParty individual
				\4[] Shock a varios bancos
				\4[] $\to$ CCParty deben aumentar fondos de capital
				\4[] $\then$ Reducen fondos disponibles para otros CCParty
				\4[] $\then$ Retroalimentación del proceso
				\4[$\then$] Necesarias medidas macroprudenciales
				\4 Hasta GFC, macroprudencial habitual en PEDs
				\4[] Posteriormente, también en desarrollados
			\3 Restricciones de propiedad de las instituciones
				\4 Prohibiciones de compra de instituciones
				\4[] Evitar tamaño excesivo
				\4[] $\to$ Reducir riesgo sistémico
			\3 G-SIBs: requisitos de capital aumentados
				\4 Definido en base a criterios medibles
				\4[] Tamaño
				\4[] Centralidad en mercado financiero
				\4[] Complejidad de relaciones
				\4[] Actividad transfronteriza
				\4[] Difícil sustituibilidad de actuaciones
				\4[] Más capital para absorber pérdidas
				\4[] $\to$ Reduce probabilidad de quiebra
				\4[] $\then$ Reduce probabilidad de inestabilidad SFin
			\3 LTV -- Loan-to-value Ratios
				\4 Límite a préstamo sobre valor
				\4[] ``Loan-to-value''
				\4 Fijado para conjunto de sistema financiero
				\4 Evitar expansión excesiva de crédito
				\4 Reducir prociclicidad
				\4[] Si LTV muy pequeños
				\4[] $\to$ Shock sistémico aumenta mucho las quiebras
				\4[] $\then$ Aumentan mucho las ventas de vivienda
			\3 Búferes contracíclicos de capital
				\4 Evitar contracción de crédito en recesión
				\4[] Que agrave recesiones
				\4 Requisitos adicionales según ciclo
				\4 Más capital exigido en fase de crecimiento
				\4 Reducción en recesión
			\3 Reservas mínimas en divisa
				\4 Mantenimiento mínimo en divisa de reservas
				\4 Evitar descalce de divisas
				\4[] En SFins endeudados en divisas
				\4 Evitar crisis cambiaria de 3a generación
			\3 Provisiones dinámicas
				\4 España pionero
				\4 En fase expansiva
				\4[] Poca mora
				\4[] $\to$ Pocas provisiones por impago
				\4 En fase recesiva
				\4[] Aumento de mora
				\4[] $\to$ Aumento de provisiones por impago
				\4 Problema
				\4[] Deterioro de resultados de bancos en crisis
				\4[] $\to$ Porque aumentan provisiones por créditos dudosos
				\4[] $\then$ Contracción ulterior del crédito
				\4 Solución vía provisiones dinámicas
				\4[] Provisiones permitidas a banco se ajustan dinámicamente
				\4[] $\to$ Más provisiones admisibles en fase de crecimiento
				\4[] $\to$ Menos en fase de recesión
				\4[] $\then$ Provisiones contracíclicas
				\4[]
			\3 \underline{Mecanismos de seguridad}
			\3 Idea clave
				\4 Minimizar impacto de crisis si se produce
				\4 Predecir cercanía de crisis
			\3 Test de estrés
				\4 Simular efectos de shocks sobre solvencia
				\4 Valorar cuantitativamente resistencia
				\4 Doble enfoque micro y macro
				\4[] Micro
				\4[] $\to$ Resistencia de entidades individuales
				\4[] Macro
				\4[] $\to$ Resistencia sistema en agregado
				\4[] $\to$ Considerar externalidades entre entidades
			\3 Seguro de depósitos
				\4 Desincentivar retiradas de fondos
				\4 $\to$ Evitar pánicos bancarios
			\3 TLAC y MREL
				\4 TLAC
				\4[] Total Loss-Absorbing Capacity
				\4[] Desarrollados por FSB
				\4[] Exigible a entidades sistémicas G-SIB
				\4[] Cantidad mínima de pasivos bailinables
				\4[] $\to$ En caso de resolución
				\4[] $\then$ Tenedores asumen pérdidas
				\4 MREL
				\4[] Minimum Requirement for own funds and Eligible Liabilities
				\4[] Contexto de Unión Europea
				\4[] Paquete Bancario de 2019
				\4[] $\to$ CRR II/CRD V/BRRD II/SRM II
				\4[] Aplicable en general a todas las entidades
				\4[] $\to$ Salvo si no desempeñan función crítica
				\4 Diferencias\footnote{Ver \href{https://home.kpmg/xx/en/home/insights/2016/12/mrel-tlac-cut-from-the-same-cloth-fs.html}{KPMG: MREL \& TLAC -- Cut from the same cloth.}}
				\4[] Ambas son mismo concepto fundamental
				\4[] $\to$ Capital mínimo para absorber pérdidas en resolución
				\4[] Ambas se establecen como requisitos mínimos
				\4[] $\to$ Desde Paquete de 2019
				\4[] Diferente rango de entidades sujetas
				\4[] $\to$ TLAC centrado en G-SIBs
				\4[] Diferente origen de obligación
				\4[] $\to$ TLAC por FSB y acuerdo G-20
				\4[] $\to$ MREL por Unión Europea
				\4[] TLAC restringe más instrumentos elegibles
				\4[] MREL admite muchos más instrumentos
				\4[] $\to$ Parte de amplia variedad de activos de bancos
				\4[] Denominador del cálculo
				\4[] $\to$ TLAC: APRs
				\4[] $\then$ Similar a ratio de capital
				\4[] $\to$ MREL: \% de pasivos totales
				\4[] $\then$ Cercano a ratio apalancamiento
			\3 Evitar too-big-to-fail
				\4 Hipótesis que caracteriza política regulatoria
				\4 ``algunos bancos son demasiado grandes para caer''
				\4[] $\Rightarrow$ Distorsiones valoración de activos
				\4[] $\Rightarrow$ Garantía implícita del sector público
				\4 Objetivo de regulación:
				\4[] $\to$ Evitar too-big-to-fail
				\4[] $\to$ Desincentivar crecimiento excesivo
			\3 Inyecciones de liquidez
				\4 Préstamos a entidades financieras
				\4 Evitar problema de liquidez $\then$ problema de solvencia
			\3 Garantías de depósitos
				\4 Estado garantiza valor de depósitos
				\4[] $\to$ No hay razón para retirar
				\4[] $\then$ Equilibrio de pánico bancario no tiene sentido
				\4 Depósitos contigentes a pánico
				\4[] Depositantes aceptan retirada puede suspenderse
				\4[] $\to$ Banco mantiene liquidez en caso de pánico
			\3 Corralito
				\4[] Gobierno decreta restricciones a retiradas
				\4[] $\to$ Sólo retiradas parciales son posibles
				\4[] $\then$ Equilibrio desfavorable se evita ``por la fuerza''
	\1 \marcar{Regulación en la práctica}
		\2 Evolución
			\3 Siglo XIX
				\4 Doctrina de las letras reales
				\4 Debates currency vs banking vs free-banking schools
				\4 Walter Bagehot a finales de siglo
			\3 Siglo XX
				\4 Bretton Woods
				\4 G-7
			\3 Actualidad
				\4 Banco Internacional de Pagos (1930)
				\4[] Coordinación de Bancos Centrales
				\4 G-20
				\4 Comité de Basilea
				\4[] Creado en 1974
				\4[] Marco de deliberación para Acuerdos de Basilea
				\4 Financial Stability Board (2009)
				\4[] Sucesor del Financial Stability Forum (1999)
				\4[] Sede en Basilea junto a BIS
				\4 Comité de Basilea (1975)
				\4 IOSCO (1985)
				\4[] Sede en Madrid
				\4[] International Organization of Securities Commission
				\4[] Asociación de reguladores de mercados de valores
				\4[] $\to$ CNMV española
				\4 IAIS (1994)
				\4[] International Association of Insurance Supervisors
				\4[] Supervisores del mercado de seguros
				\4 IOPS (2004)
				\4[] International Organization of Pension Supervisors
				\4 Proceso Lamfalussy (2001)\footnote{ Ver \href{https://ec.europa.eu/info/business-economy-euro/banking-and-finance/financial-reforms-and-their-progress/regulatory-process-financial-services/regulatory-process-financial-services_en}{Comisión Europea sobre Proceso de Lamfalussy} como marco de diseño e implementación de regulación financiera en la UE}
				\4[] Enfoque de aprobación de nueva regulación
				\4[] $\to$ regulación comunitaria de servicios financieros
				\4[] $\to$ división en 4 niveles del proceso de regulación
				\4[] Nivel 1
				\4[] $\to$ PE y CdUE adoptan nuevas directivas y reglamentos
				\4[] $\to$ Alcanzar acuerdos políticos clave
				\4[] $\to$ Proceso costos en tiempo y recursos
				\4[] $\to$ Restringir uso a grandes decisiones políticas
				\4[] Nivel 2
				\4[] $\to$ Comisión adopta, adapta y actualiza reg. técnica
				\4[] $\to$ Consulta con otros órganos de EEMM
				\4[] $\to$ Detalles concretos de implementación
				\4[] Nivel 3
				\4[] $\to$ Comités de supervisores nacionales
				\4[] $\to$ Sustituidos por EIOPA, ESMA, EBA
				\4[] Nivel 4
				\4[] $\to$ Comisión supervisa cumplimiento de reglas
		\2 Acuerdos de Basilea
			\3 Basilea I (1988)
				\4 Aprobación en 1988
				\4 Entrada en vigor en 1992
				\4 Introduce ratio de capital
				\4 Ponderación por riesgo del activo
				\4 8\% de capital sobre APR
				\4 Mitad del 8\% debe ser Tier 1
			\3 Basilea II (2004)
				\4 Aprobado en 2004
				\4[I] \underline{1er pilar:}
				\4[] $\to$ Requisitos mínimos de capital
				\4[] $\to$ Control de riesgos
				\4 Riesgo de crédito
				\4[] 8\% de capital sobre APR
				\4[] Valoración riesgo de crédito:
				\4[] Estándar
				\4[] Foundation Internal Ratings-Based
				\4[] $\to$ Banco fija ponderación por riesgo
				\4[] Advanced IRB
				\4[] $\to$ Banco fija parámetros del modelo
				\4[] $\to$ Incluyendo exposición y pérdidas por impago
				\4 Riesgo operativo
				\4[] Riesgo derivado de pérdidas superiores a las esperadas
				\4[] Valoración del riesgo operativo
				\4[] $\to$ Básico
				\4[] $\to$ Estándar
				\4[] $\to$ Avanzado
				\4 Riesgo de mercado
				\4[] Estimación VaR del riesgo del trading book
				\4[II] \underline{2do pilar: }
				\4[] $\to$ supervisión regulatoria
				\4 Promover mejor gestión interna de riesgos
				\4[] $\to$ Bancos deben poder conocer sus riesgos
				\4[] $\to$ Deben diseñar herramientas de control adecuadas
				\4 Implementación de herramientas de control
				\4[] $\to$ Para facilitar control por regulador
				\4 Diseñar herramientas de intervención rápida
				\4[III] \underline{3er pilar: }
				\4[] $\to$ disciplina vía transparencia
				\4 Requisitos de diseminación de información
				\4[] Para que mercados evalúen riesgo
				\4 Estándares de diseminación
				\4 Información pública obligatoria
			\3 Basilea III (2010-2012)
				\4 Acuerdo entre 2010 y 2012
				\4 Ratio de capital
				\4[] Capital mínimo:
				\4[] $\to$ 4,5\% Common Equity Tier 1
				\4[] Búfer de conservación de capital
				\4[] $\to$ 2,5\% Common Equity Tier 1
				\4[$\then$] 7\% CET1 en condiciones normales
				\4[$\then$] CET1 entre 4,5\% y 7\%:
				\4[] $\to$ Restricciones sobre repartos beneficios
				\4 Margen anticíclico
				\4[] 0-2,5\%
				\4[] A discreción de reguladores nacionales
				\4[$\to$] Tier 1 debe ser > 6\% en todo momento\footnote{Tier 1 es una categoría más amplia que CET1. Incluye ``Additional Tier 1'' Capital (AT1) como contingent convertible y otros.}
				\4[$\to$] Tier 1 + Tier 2 debe ser > 8\%
				\4 Identificación de SIBs (systemically important banks)
				\4[] Requisitos adicionales de capital para estos bancos
				\4 Ratio de apalancamiento
				\4[] mínimo de 3\%\footnote{Cociente entre capital Tier 1 y activos consolidados totales.}
				\4[] $\to$ desincentivar crecimiento excesivo de balances
				\4 Ratio de cobertura de liquidez
				\4[] Liquidity Coverage Ratio
				\4[] Garantizar salidas netas de efectivo
				\4[] $\to$ 30 días
				\4 Ratio de financiación estable neta
				\4[] Net Stable Funding Ratio:
				\4[] garantizar financiación estable superior a necesario
				\4[] $\to$ En contexto de fuerte tensión durante 1 año
				\4 Implementación gradual hasta 2019
			\3 Reformas de 2017/Basilea IV
				\4 Limitaciones a reducción de capital necesario
				\4[] Derivadas de estimación vía IRB
				\4 Reducción de ponderación de préstamos hipotecarios de bajo riesgo
				\4 Tasa de apalancamiento adicional para G-SIBs
				\4[] Mínimo del 50\% del ratio de capital necesario
		\2 Estados Unidos
			\3 Glass-Steagall (1933)
				\4 Separación banca comercial e inversión
				\4 Explotación de vacíos legales crecientes
				\4 Derogación en 1999
				\4 Sin evidencia de derogación $\then$ crisis 2007
			\3 Dodd-Frank (2010)
				\4 Establece Financial Stability Oversight Council
				\4 Cambian regulación del FDIC
				\4 Volcker Rule\footnote{En líneas generales, se prohíbe que los bancos tomen posiciones con sus propias cuentas, lo que se conoce habitualmente como \textit{propietary trading}. De esta forma, se pretende evitar que las entidades asuman riesgos en perjuicio de sus clientes.}
			\3 Reforma de Dodd-Frank (2018)
				\4 Aumenta umbral de too-big-to-fail
				\4[] Requisitos extra para entidades grandes
				\4[] Antes, activo > 50.000 M \$
				\4[] Ahora, activo > 250.000 M \$
		\2 Unión Europea
			\3 Sistema Europeo de Supervisión Financiera (ESFS)
				\4 Creado a partir de Informe Larosière (2009)
				\4 Macro
				\4[] European Systemic Risk Board (ESRB)
				\4[] $\to$ Supervisión macroprudencial del sistema financiero
				\4[] $\to$ Dependiente del BCE
				\4 Micro
				\4[] $\to$ Mediar entre supervisores nacionales
				\4[] $\to$ Adopción de estándares nacionales
				\4[] $\to$ Adopción de decisiones técnicas
				\4[] $\to$ Supervisión de supervisores
				\4[] $\to$ Coordinación en caso de crisis
				\4[] European Banking Authority (EBA)
				\4[] European Securities and Market Authority (ESMA)
				\4[] European Insurance and Occupational Pensions Authority (EIOPA)
			\3 \underline{Unión Bancaria}
				\4 Tres pilares:
				\4[] i. Mecanismo único de supervisión
				\4[] ii. Mecanismo único de resolución
				\4[] iii. Single rulebook
			\3 Mecanismo Único de Supervisión
				\4 Single Supervisory Mechanism (2014)
				\4 Creado tras batería de test de stress
				\4 Supervisión de bancos más grandes por BCE
				\4 Supervisión de resto de bancos:
				\4[] $\to$ Responsable nacional
				\4 BCE delega a menudo
				\4 Potestad de supervisar en cualquier momento
			\3 Mecanismo Único de Resolución
				\4 Single Resolution Mechanism (2014)
				\4 Reducir impacto quiebras sobre contribuyentes
				\4 Reducir contagio en Unión Europea
				\4 Resolución predecible
				\4 Autoridad Única de Resolución (SRB)
				\4 Fondo Único de Resolución (SRF)
			\3 \underline{Single rulebook}
			\3 Directiva de Fondos de Garantía de Depósitos (DGDS) \footnote{En 2013 se alcanzó un acuerdo que armoniza la cobertura a 100.000 € por depositante y banco. La financiación de los fondos de garantía se obtiene mediante un porcentaje de los depósitos, mediante contribuciones de los bancos si la anterior es insuficiente, y mediante financiación alternativa si esta no es suficiente. Los bancos contribuyen según su perfil de riesgo.}
				\4[] Armonizar garantías de depósitos
				\4[] Reducir competencia intra-UE
				\4[] Garantizar cobertura mínima
				\4[] Regular financiación de fondos de garantía
			\3 BRRD y SRMR
				\4 Directiva de Resolución y Recuperación Bancaria
				\4 Single Resolution Mechanism Regulation
				\4 BRRD I (2014) $\to$ BRRD II (2019)
				\4 Incorporada a ordenamiento de EEMM máximo en 2016
				\4[] Regulación del proceso de resolución
				\4[] A implementar por países
				\4 Seguro de depósitos común\footnote{En proyecto. Con la última propuesta de la Comisión de mayo de 2017 y el clima político actual parece aumentar la probabilidad de que se lleve a cabo en el futuro próximo. Existen también propuestas de creación de un sistema de reaseguros de depósitos europeo que permita hacer frente a shocks en el conjunto de un sistema bancario nacional aun manteniendo el carácter nacional de la gestión de los sistemas de seguro de depósitos.}
				\4 Énfasis en evitar too-big-to-fail
				\4 Bancos y firmas de inversión
				\4 Facultades de intervención de la gestión
				\4 Planes para resolver filiales en terceros países
				\4 Explicita jerarquía de resolución
				\4 Fondos nacionales de resolución
				\4[] FROB en España
				\4[] Con contribuciones de entidades bancarias
				\4 BRRD II aprobada en 2019\footnote{Que sienta las bases regulatorias del bail-in como instrumento de resolución. Ver \url{https://eba.europa.eu/documents/10180/1899764/Andrea+Enria+-+Hearing+at+the+Italian+Senate+\%28English+translation\%29.pdf}}
				\4 Reforma de 2019 (BRRD II/SRMR II)\footnote{Ver \url{https://www.moodysanalytics.com/regulatory-news/jun-07-19-eu-publishes-brrd-ii-and-srmr-ii-in-the-official-journal}}
				\4[] Aplicables desde inicio 2021
				\4[] SRMR II:
				\4[] $\to$ Armonización de MREL a TLAC de FSB\footnote{TLAC: Total Loss-Absorbing Capacity. MREL: Minimum Requirements of Eligible Assets. El FSB estableció el concepto de TLAC en 2014. En la regulación europea a partir de 2014/2015, el TLAC se implementó como MREL, aunque no exactamente con el mismo tenor. El MREL exigido es diferente al ratio de capital mínimo en el marco de Basilea III y CRD II/CRD V. Ver \url{https://srb.europa.eu/sites/srbsite/files/list_of_public_qas_on_mrel_-_clean.pdf}}
			\3 CRR I -- Capital Requirements Regulation I (2013)
				\4 2013
				\4 Reglas prudenciales comunes
				\4 Implementación requisitos Basilea III
			\3 CRD IV -- Capital Requirements Directive IV (2013)
				\4 2013
				\4 Remuneración de directivos
				\4[] Desincentivar toma excesiva de riesgos
				\4[] Limitar remuneración variable
				\4 Gobierno corporativo
				\4 Colchón de capital para instituciones sistémicas: 1\%-3,5\% CET1
				\4 Colchón sistémico facultativo nacional\footnote{Superiores al 3\% obligan a notificar al ESRB, la EBA y la Comisión. Más del 5\% debe ser aprobado por la Comisión.}
			\3 CRR II/CRD V (2019)
				\4 Aprobadas en 2019
				\4 Entrada en vigor de mayoría de medidas en 2021
				\4 Implementar reformas de 2017 de Basilea III
			\3 IFR/IFD (2019)\footnote{\url{https://www.nortonrosefulbright.com/en/knowledge/publications/f6b2e0a7/the-new-prudential-regime-for-investment-firms}}
				\4 Entrada en vigor de mayoría de medidas en 2021
				\4 Regulación de empresas de inversión
				\4 Firmas de inversión implican menos riesgos...
				\4[] ...que instituciones de crédito reguladas
				\4[] $\to$ Menores requisitos prudenciales
			\3 MiFID II y MiFIR (Markets in Financial Instruments)
				\4 Aprobado en 2014
				\4 A implementar en 2017
				\4 Regulación dealers, brokers
				\4[] $\to$ Más allá de equities
				\4[] $\to$ Bonos, commodities, derivados...
				\4 Mejorar protección al cliente
				\4 Mercado único europeo en finanzas
				\4 Regulación trading de alta frecuencia
				\4 Continúa
				\4[] MiFiD I (2007)
				\4[] Investment Services Directive (1992)
				\4 Entrada en vigor enero de 2018
			\3 EMIR -- European Market Infrastructure Regulation
				\4 Aprobada en 2012
				\4 Transparencia
				\4[] Obligación de proveer información a repositorios
				\4[] $\to$ Registros de derivados
				\4[] Publicación de posiciones agregadas de derivados
				\4[] $\to$ OTC
				\4[] $\to$ Cotizados en mercados oficiales
				\4 Reducción de riesgo de crédito
				\4[] Contratos OTC estandarizados
				\4[] $\to$ Deben compensarse y liquidarse en CCP
				\4[] Contratos fuera de CCP
				\4[] $\to$ Técnicas de mitigación del riesgo
				\4 Regulación de CCP
				\4[] Riesgo operativo
				\4[] Capital
				\4[] Transparencia
				\4 Acreditación de equivalencia
				\4[] Reconocimiento de CCP y repositorios fuera de UE
				\4[] Confirmar requisitos mínimos de fuera-UE
				\4[] Reconocimiento permite:
				\4[] $\to$ Utilización de CCP por residentes en UE
			\3 Transición Eonia a ESTR\footnote{Pronunciado ``ester''}
				\4 Ver \url{https://www.ecb.europa.eu/pub/economic-bulletin/focus/2019/html/ecb.ebbox201907_01~b4d59ec4ee.en.html}
				\4 Ver \url{https://blog.bankinter.com/economia/-/noticia/2019/8/8/que-estr-tipo-interes-corto-plazo-euro}
				\4 Ver \url{https://www.bde.es/f/webbde/INF/MenuHorizontal/Publicaciones/Boletines\%20y\%20revistas/InformedeEstabilidadFinanciera/ief_2019_1_Rec2_1.pdf}
		\2 Otras
			\3 Medio ambiente y regulación financiera
				\4 Ver \href{https://voxeu.org/article/central-banks-and-climate-change}{VOXEU (2020)}
			\3 Asia
			\3 Agencias de calificación
				\4 Regulación competencia de la ESMA
				\4 Reglamento 426/2013
				\4 Directiva 2013/14
				\4 Reducir abuso de ratings como herramienta de decisión
				\4 Transparencia modelos
				\4 Reducción de conflictos de interés
			\3 Blockchain
				\4 Proyectos de regulación
				\4 Por el momento, incertidumbre
		\2 Retos de la regulación
			\3 Shadow banking
				\4 Regulación desincentiva actividad bancaria
				\4 Agentes no oficialmente bancos actúan como tales
				\4 Desviación de actividad bancaria a sector no regulado
			\3 Criptomonedas
				\4 Conjunto de algoritmos matemáticos
				\4 Permiten mantenimiento de balances sin autoridad central
				\4[] Evitando double-spending
				\4[] Garantizando seguridad de balances
				\4 Ausencia de todo regulador
				\4 Creación de oferta monetaria
				\4[] Diferentes reglas posibles
				\4[] -- Regla de crecimiento fija
				\4[] -- Autoridad fija crecimiento de moneda
				\4[] ...
				\4 En sistemas descentralizados
				\4[] imposible regular
				\4[] imposible intervenir (por el momento)
				\4[] $\to$ Sin incurrir en enormes costes
				\4[] $\then$ Necesario controlar tráfico de internet
				\4 Potencial sector de shadow banking masivo
			\3 Narrow banking
				\4 Modalidad especial de bancos comerciales
				\4 Respaldar 100\% de depósitos a la vista
				\4[] Con cuentas en el banco central
				\4 Eliminar riesgo para depositantes
				\4 Eliminar seguro de depósitos
				\4 Eliminar crisis bancarias
				\4 Propuesto como solución a problema inestabilidad
				\4 Problemas potenciales
				\4[] Contracción del crédito en crisis
				\4[] $\to$ Todos los depósitos a narrow banks
				\4[] $\then$ Credit crunch
				\4 Central Bank Digital Currency para público
				\4[] Posible alternativa
				\4[] Dinero de BC para público
				\4[] $\to$ Sin intermediación vía banco comercial
			\3 Central Bank Digital Currency
				\4 Sustitución de efectivo
				\4 Dinero electrónico en cuentas de banco central
				\4 Disponibles para el público
				\4 Eliminan necesidad de intermediarios bancarios
				\4 Necesaria nueva regulación bancaria
				\4[] Bancos pasarían a ser financiados por equity
				\4[] Bancos centrales deberían poder prestar
				\4[] $\to$ Para evitar contracción de crédito en recesión
				\4[] $\then$ Si todos los agentes liquidan créditos por saldo CBDC
			\3 Consecuencias indeseadas de la regulación
			\3 Regulator shopping
				\4 Contextos con varios reguladores
				\4 Posible elegir regulador
				\4 Elección de regulador más laxo o favorable
				\4[$\then$] Distorsiona decisiones
				\4[$\then$] Aumenta posible inestabilidad
			\3 Forum shopping
				\4 Globalización creciente
				\4 Costes de información reducidos
				\4 Deslocalizaciones cada vez más fáciles
				\4 Operar desde el extranjero
				\4[] $\then$ Evitar regulación
	\1[] \marcar{Conclusión}
		\2 Recapitulación
			\3 Sistema financiero
			\3 Aspectos teóricos de la regulación
			\3 Regulación en la práctica
		\2 Idea final
			\3 Sistema financiero y bancario como proceso endógeno
				\4 Evolución responde a regulación
			\3 Transformaciones sector bancario
				\4 Tipos de interés bajos
				\4 Competidores no bancarios
				\4 Nuevas tecnologías: blockchain, móvil, etc...
			\3 Unión Europea
				\4 Estabilidad bancaria y financiera y futuro de la UE
				\4 Muy fuerte vínculo
\end{esquemal}














































\conceptos

\concepto{APR}

Activos Ponderados por Riesgo. Cuando se trata de evaluar la capacidad de un banco para absorber pérdidas, se pone en relación la estructura financiera capaz de absorber pérdidas (basicamente, el equity), y los activos susceptibles de ver su precio alterado por las fluctuaciones del mercado.

\concepto{Banking book}

Activos del balance que la entidad pretende conservar hasta vencimiento. Por ello, no se valoran mark-to-market y a la hora de calcular el VaR se estima con un nivel de confianza del 99.5\%. Contrapuesto al trading book.

\concepto{Trading book}

Activos que el banco posee con el fin de venderlos y obtener una rentabilidad antes del vencimiento. Se valoran mark-to-market y se aplican intervalos de confianza más amplios a la hora de estimar pérdidas en un contexto VaR.

\concepto{Tier 1 capital}

Capital capaz de absorber pérdidas sin comprometer el funcionamiento de la entidad. Básicamente se compone de capital autorizado y reservas declaradas.

\concepto{Tier 2 capital}

Capital capaz de absorber pérdidas en caso de liquidación de la entidad. Compuesto de reservas no declaradas, instrumentos híbridos de deuda, convertibles, etc...

\concepto{EBA}

La European Banking Authority se crea en 2011 y e establece en Londres. Su institución se debe en gran medida a la inestabilidad financiera, que puso de manifiesto la necesidad de establecer una agencia encargada de establecer de forma independiente la regulación del sector bancario, así como de realizar los llamados test de stress. Tiene capacidad para establecer regulación a nivel nacional si el regulador nacional no actúa de acuerdo con sus obligaciones.

\concepto{Informe Lamfalussy}

Aprobado en 2001 por un comité presidido por Alexandre Lamfalussy, el informe recomienda un procedimiento de aprobación de regulación financiera a nivel comunitario. Los objetivos últimos son en definitiva la convergencia de la supervisión y la interpretación consistente de la normativa. El proceso se divide en cuatro fases: 

\concepto{Bail-in}

Financiación de las recapitalizaciones de las entidades mediante la reestructuración de la deuda y/o la conversión de esta en recursos propios. Este tipo de recapitalizaciones requiere un esquema regulatorio bien definido en relación al orden de prelación de las obligaciones del banco reestructurado, así como del recurso a fondos de resolución o recursos públicos cuando el bail-in no sea suficiente o su uso para la totalidad de la reestructuración se estime excesivo.

\begin{enumroman}
    \item Adopción por parte del PE y el Consejo de la UE
    \item Asesoramiento técnico por comités sectoriales
    \item Coordinación entre reguladores nacionales sobre nuevas normas
    \item Aplicación
\end{enumroman}

\concepto{Tier y Tier 2 en Basilea IIi}: 

Las definiciones de Tier 1 y Tier 2 se establecen en el artículo 49 del Acuerdo de Basilea III. El capital Tier 1 es aquel capaz de absorber pérdidas aun manteniéndose la empresa en funcionamiento (going concern). El capital Tier 1 se divide a su vez en \textit{common equity} y Tier 1 adicional.

El capital Tier 2 es aquel capaz de absorberlas en caso de quiebra sin romper compromisos contractuales. 

Los \textit{mínimos} se establecen en el artículo 50 del Acuerdo de Basilea III:

<<All elements above are net of the associated regulatory adjustments and are subject
to the following restrictions (see also Annex 1): 

\begin{itemize}
    \item Common Equity Tier 1 must be at least 4.5\% of risk-weighted assets at all times.
    \item Tier 1 Capital must be at least 6.0\% of risk-weighted assets at all times.
    \item Total Capital (Tier 1 Capital plus Tier 2 Capital) must be at least 8.0\% of risk weighted assets at all times.
\end{itemize}
>>


\preguntas

\seccion{Test 2019}

\textbf{38.} A raíz de los sucesivos acuerdos de Basilea se han ido estableciendo diferentes requerimientos de capital a las entidades financieras para garantizar que las mismas cuentan con recursos propios suficientes para hacer frente a sus deudos o pasivos. Señale la afirmación correcta:

\begin{itemize}
	\item[a] Recientemente el Comité de Basilea acordó una reducción en los requerimientos de capital para aquellas entidades que mantengan productos verdes (medioambientalmente sostenibles) en su cartera.
	\item[b] Como medida macroprudencial, el acuerdo de Basilea III establece un colchón de capital adicional: el colchón de conservación del capital, cuya constitución queda sujeta a periodos en que el crecimiento del crédito esté acelerándose.
	\item[c] El acuerdo Basilea III incluye unas revisiones de los métodos estándar para calcular: el riesgo de crédito, el riesgo de mercado, el riesgo de ajuste de valoración del crédito, y el riesgo operacional; a fin de mejorar la sensibilidad al riesgo y la comparabilidad.
	\item[d] Actualmente los bancos europeos deben mantener un colchón de conservación del capital consistente en capital de la máxima calidad (capital de nivel 1) igual a $4,5\%$ de los activos ponderados por el riesgo.
\end{itemize}

\seccion{Test 2016}

\textbf{18.} La crisis económica ha cambiado el diseño de las políticas macroeconómicas, dando pie al desarrollo de las políticas macroprudenciales, que están ocupando un lugar cada vez más protagonista. Cuál de los siguientes no es un instrumento característico de la política macroprudencial:

\begin{itemize}
	\item[a] Los requisitos de capital a los bancos
	\item[b] Los límites al Loan-to-Value (LTV) para los préstamos hipotecarios.
	\item[c] Los límites al nivel de gasto público.
	\item[d] Los límites al crédito en moneda extranjera.
\end{itemize}

\textbf{40.} La Directiva sobre Requisitos de Capital IV incorpora Basilea III al ordenamiento jurídico comunitario. Es falso que:

\begin{itemize}
	\item[a] Los órganos de dirección de las entidades deben ser lo suficientemente diversos en términos de edad y sexo, entre otros factores.
	\item[b] El componente variable no será superior al 100 \% del componente fijo de la remuneración total en ningún caso. 
	\item[c] Los bancos siguen teniendo que mantener un capital de al menos el 8 \% de sus activos ponderados por riesgo. Sin embargo, respecto Basilea II, cambia la composición y la definición de capital.
	\item[d] La ratio de capital estructural Tier 1 se fija en el 6 \% de sus activos ponderados por riesgo.
\end{itemize}

\seccion{Test 2013}

\textbf{45.} Seleccione la respuesta correcta. El Acuerdo de Basilea III:

\begin{itemize}
	\item[a] Es el acuerdo alcanzado por los países de la Zona euro en 2009 con el fin de crear un supervisor bancaria único en la región que se integrará en el BCE.
	\item[b] Es un acuerdo impulsado por el G20 en 2009 y que supuso la puesta en marcha de un plan especial para luchar contra los paraísos fiscales y la financiación del terrorismo.
	\item[c] Está orientado a aumentar los requerimientos y la calidad del capital bancario y, además, introduce requerimientos de apalancamiento y liquidez para las entidades bancarias.
	\item[d] Está orientado a aumentar los requerimientos y la calidad del capital bancario, pero no aborda requerimientos de apalancamiento ni liquidez.
\end{itemize}

\seccion{Test 2008}

\textbf{40.} El Acuerdo sobre Requerimientos de Capital, también denominado Acuerdo de Basilea II, trata, entre otros, de:

\begin{itemize}
	\item[a] El riesgo de mercado.
	\item[b] La variación de los tipos de interés o de tipos de cambio.
	\item[c] Los riesgos de crédito.
	\item[d] Todas son verdaderas.
\end{itemize}

\notas

Hablar sobre la exposición soberana de los sistemas bancarios. Para comenzar, por ejemplo: \url{https://www.bbva.com/es/sabes-por-que-se-incrementan-las-exposiciones-soberanas/}

\textbf{2019:} \textbf{38.} C 

\textbf{2016:} \textbf{18.} C \textbf{40.} B

\textbf{2013:} \textbf{45.} C

\textbf{2008:} \textbf{40.} Anulada

Leer Symposium del JEP de Winter 2019 \url{https://www.aeaweb.org/issues/538}


En la carpeta del tema hay guardado un documento de EY que resume las (nuevas) CRR II y CRD V propuestas por la Comisión Europea a finales de 2016. \textbf{Mantenerse al tanto de estos aspectos es fundamental.}

Hay que tener presente a lo largo del tema que la regulación también tiene efectos negativos y la exposición no se puede plantear como un relato de las ventajas de regular. Por ejemplo, Slovik (2012):

\textit{<<Bank regulation might have contributed to or even reinforced adverse systemic shocks that materialised during the financial crisis. Capital regulation based on risk-weighted assets encourages innovation designed to circumvent regulatory requirements and shifts banks’ focus away from their core economic functions. Tighter capital requirements based on risk-weighted assets may further contribute to these skewed incentives. The estimated macroeconomic costs of redirecting banks’ attention away from such unconventional business practices are low. During a medium-term adjustment period, for each percentage point of bank equity, regulation that is not based on risk-weighted assets would affect annual GDP growth by -0.02 percentage point more than under the risk-weighted assets framework. Refocusing banks’ attention toward their main economic functions is a core requirement for durable financial stability and sustainable economic growth.>>}


\bibliografia

Mirar en Palgrave:

\begin{itemize}
	\item assets and liabilities
	\item banking crises
	\item banking industry
	\item deposit insurance
	\item finance
	\item finance (new developments)
	\item financial intermediation
	\item financial market contagion
\end{itemize}

Mirar en Palgrave Money \& Finance:

\begin{itemize}
	\item asset and liability management
	\item bank charters
	\item banking crises
	\item banking deregulation and monetary policy
	\item banking firm
	\item banking output
	\item banking panics
	\item banking structure and competition
	\item banking supervision
	\item bank mergers
	\item Bank rate
	\item bank runs
\end{itemize}

Mirar en Banking Crises: Perspectives from the New Palgrave Dictionary of Economics todo pero especialmente los siguientes artículos
\begin{itemize}
	\item credit rating agencies
	\item euro zone crisis of 2010
	\item Minsky crisis
	\item shadow banking
\end{itemize}

Valdez, Molyneux

Eurozone political spring (eBook VOXeu)

Un resumen de las aportaciones de CRD IV/CRR II: http://europa.eu/rapid/press-release\_MEMO-13-272\_en.htm

Akerlof, Blanchard, Romer, Stiglitz. \textit{The contours of banking and the future of its regulation} What have we learned (2014)

Banco Internacional de Pagos. \textit{Basilea III: Marco del coeficiente de apalancamiento y sus requisitos de divulgación} (2014) Comité de Supervisión Bancaria de Basilea -- En carpeta del tema

Cecchetti, S.; Schoenholtz, K. \textit{Money, Banking, and Financial Markets} (2014) Fourth Edition -- En carpeta Finanzas

Diamond, D. W. (2007) \textit{Banks and Liquidity Creation: A Simple Exposition of the Diamond-Dybvig Model} Economic Quarterly - Volume 93, Number 2. -- En carpeta del tema 

Enria, A. \textit{The 'banking reform package': CRD 5/CRR 2/BRRD 2} (2017) European Banking Authority -- En carpeta del tema

Goodhard, C.; Jensen, M. (2015) \textit{Currency school versus Banking School: an ongoing confrontation} Economic Thought, 4 (2) pp. 20-31 \href{http://eprints.lse.ac.uk/64068/1/Currency\%20School\%20versus\%20Banking\%20School.pdf}{Enlace} -- En carpeta del tema

Hellwig, M. \textit{``Total assets'' versus ``Risk-weighted assets'': does it matter for MREL requirements?} European Parliament In-Depth Analysis (2016) -- En carpeta del tema

Unicredit (2017) \textit{The best way to track ECB rate-hike expectations} Rates Perspectives, No. 26 -- En carpeta del tema

Various Authors. \textit{Symposium: Financial Stability Regulation} (2019) Journal of Economic Perspectives: Winter 2019 -- \url{https://www.aeaweb.org/issues/538} 

Vernimmen, P.; Quiry, P.; Dallocchio, M; Le Fur, Y.; Salvi, A. \textit{Corporate Finance. Theory and Practice} Ch. 15 The Financial Markets

\end{document}
