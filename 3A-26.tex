\documentclass{nuevotema}

\tema{3A-26}
\titulo{Magnitudes macroeconómicas y contabilidad nacional según el SEC-2010}

\begin{document}

\ideaclave

Ver Nota Metodológica del IGAE en relación al SEC-2010 en carpeta del tema.

Ver Muriel de la Riva (2016) sobre cambios metodológicos en SEC-2010 y para hacer apartado. 

\seccion{Preguntas clave}

\begin{itemize}
	\item ¿Qué son las magnitudes macroeconómicas?
	\item ¿Qué es la contabilidad nacional?
	\item ¿Cómo se estructura el sistema de cuentas nacionales según el SEC-2010?
	\item ¿Para qué sirve?
	\item ¿Qué retos y qué problemas enfrena la contabilidad nacional?
\end{itemize}

\esquemacorto

\begin{esquema}[enumerate]
	\1[] \marcar{Introducción}
		\2 Contextualización
			\3 Definición de economía y macroeconomía
			\3 Variables agregadas y contabilidad nacional
			\3 Desarrollo de la contabilidad nacional
		\2 Objeto
			\3 ¿Qué es una macromagnitud?
			\3 ¿Cuáles son las principales macromagnitudes?
			\3 ¿Qué es la contabilidad nacional?
			\3 ¿Cómo se registra la información macroeconómica según el SEC 2010?
			\3 ¿Para qué sirve?
			\3 ¿Qué retos y qué problemas enfrenta la contabilidad nacional?
		\2 Estructura
			\3 Macromagnitudes
			\3 Cuentas del SEC-2010
	\1 \marcar{Macromagnitudes}
		\2 Idea clave
			\3 Agregados
			\3 Cocientes
		\2 Flujos y stocks
			\3 Stocks
			\3 Flujos
		\2 Agregados principales
			\3 VAB
			\3 PIB
			\3 PIN
			\3 RNB
			\3 RNN
			\3 RNBD
			\3 RNND
			\3 Ahorro
			\3 Ahorro neto
			\3 Saldo de Operaciones Corrientes con el Exterior
			\3 Capacidad/Necesidad de financiación
			\3 Riqueza neta
			\3 Rentas empresariales
	\1 \marcar{Cuentas del SEC-2010}
		\2 Idea clave
			\3 Concepto
			\3 Objetivo
			\3 Resultados
		\2 Reglas y criterios contables
			\3 Marco jurídico
			\3 Residencia
			\3 Criterios de valoración
			\3 Volumen y valor
			\3 Valoraciones de productos
			\3 Periodo temporal y momento de registro
			\3 Contabilidad por partida doble/cuádruple
			\3 Saldos contables
		\2 Unidades estadísticas y sectores institucionales
			\3 Unidades estadísticas
			\3 Sectores institucionales
		\2 Análisis funcional
			\3 Idea clave
			\3 UAE-- Unidades de Actividad Económica a Nivel Local
			\3 Clasificación Nacional de Actividades Económicas
			\3 Ramas de actividad
		\2 Fuentes de información
			\3 INE
			\3 Banco de España
			\3 AEAT
			\3 Proveedores de servicios de pago
			\3 RIE-- Registro de inversiones exteriores de SEComercio
		\2 Cambios metodológicos en SEC-2010
			\3 Idea clave
			\3 I+D capitalizada
			\3 Gasto militar capitalizada
			\3 Administración pública
			\3 Procesamiento de bienes en el exterior
			\3 Pensiones
			\3 Sector de seguros
			\3 Instituciones financieras
			\3 Impacto sobre series de PIB
		\2 Cuentas corrientes
			\3 \underline{Bienes y servicios} (0)
			\3 \underline{Producción} (I)
			\3 \underline{Explotación} (II.1.1)
			\3 \underline{Asignación de la renta primaria} (II.1.2)
			\3 \underline{Distribución secundaria de la renta} (II.2)
			\3 \underline{Distribución secundaria de la renta en especie} (II.3)
			\3 \underline{Utilización de la renta} (II.4)
		\2 Cuentas de acumulación (III)
			\3 \underline{Cuenta de capital} (III.1)
			\3 VPNDAYTK (III.1.1) 
			\3 Cuenta de Adq. de Activos No Financieros (III.1.2)
			\3 \underline{Cuenta Financiera} (III.2)
			\3 \underline{Cuenta de otras variaciones de los activos} (III.3)
			\3 Cuenta de otras variaciones del \underline{volumen} de activos (III.3.1)
			\3 Cuenta de revalorización (III.3.2)
		\2 Balances (IV)
			\3 Idea clave
			\3 \underline{Balance inicial} (IV.1)
			\3 \underline{Variaciones del balance} (IV.2)
			\3 \underline{Balance final} (IV.3)
		\2 Cuentas de resto del mundo (V)
			\3 Idea clave
			\3 \underline{Cuenta de intercambios exteriores de bienes y servicios} (V.1)
			\3 \underline{Cuenta exterior de ingreso primario y secundario} (V.2)
			\3[] \underline{Cuenta exterior de acumulación} (V.3)
			\3 \underline{Cuenta exterior de capital} (V.3.1)
			\3 VPNDSOCEYTK (V.3.1.1)
			\3 Cuenta de adquisición de activos no financieros (V.3.1.2)
			\3 \underline{Cuenta financiera} (V.3.2)
		\2 Otras cuentas
			\3 Input-Output
			\3 Cuentas satélite
		\2 Relaciones con otros conceptos
			\3 Balanza de pagos
			\3 Tablas input-output
	\1[] \marcar{Conclusión}
		\2 Recapitulación
			\3 Cuentas del SEC-2010
			\3 Macromagnitudes
		\2 Idea final
			\3 Dificultades de medición
			\3 Cuantificación del bienestar: más alla del PIB

\end{esquema}

\esquemalargo

\begin{esquemal}
	\1[] \marcar{Introducción}
		\2 Contextualización
			\3 Definición de economía y macroeconomía
				\4 Definición de Robbins
				\4[] Economía es estudio de comportamiento humano
				\4[] $\to$ Gestionando recursos escasos con usos alternativos
				\4[] $\to$ Para satisfacer una serie de necesidades humanas
				\4 Macroeconomía
				\4[] Estudio de economías formadas por muchos agentes
				\4[] $\to$ Miles, cientos de miles, millones de agentes
				\4[] Necesaria agregación de información
			\3 Variables agregadas y contabilidad nacional
				\4 Resumen información de muchos agentes
				\4[] $\to$ En un sólo indicador
				\4 Pérdida de información pero más tratable
				\4 Contabilidad Nacional
				\4[] Sistema de vars. agregadas
				\4[] $\to$ Recopilar inf. micro en variables agregadas
				\4[] $\to$ Presentar relaciones entre variables
				\4[] $\then$ No se puede gestionar lo que no se conoce
				\4[] $\then$ Posibilita contrastar modelos agregados
				\4[] $\then$ Ilumina relaciones causales
			\3 Desarrollo de la contabilidad nacional
				\4 Avances previos
				\4[] Tableau Économique s. XVIII
				\4[] Equilibrio general walrasiano
				\4 Keynesianismo
				\4[] Nacimiento de la macroeconomía como tal
				\4[] $\to$ Estudio directo de vars. agregadas
				\4 Leontieff, Kuznets, Keynes, Tinbergen
				\4[] Introducen conceptos básicos
				\4[] Paralelo y conectado con Keynesianismo
				\4 Diseño del primer sistema de contabilidad nacional
				\4[] 1947 en Estados Unidos
				\4[] 1952: Naciones Unidas establece líneas generales
		\2 Objeto
			\3 ¿Qué es una macromagnitud?
			\3 ¿Cuáles son las principales macromagnitudes?
			\3 ¿Qué es la contabilidad nacional?
			\3 ¿Cómo se registra la información macroeconómica según el SEC 2010?
			\3 ¿Para qué sirve?
			\3 ¿Qué retos y qué problemas enfrenta la contabilidad nacional?
		\2 Estructura
			\3 Macromagnitudes
			\3 Cuentas del SEC-2010
	\1 \marcar{Macromagnitudes}\footnote{Ver ``Agregados'' en SEC-2010 en Español, pág. 315.}
		\2 Idea clave
			\3 Agregados
				\4 Indicadores derivados de sistema de cuentas derivados de:
				\4[] Transacciones directas del sistema de cuentas
				\4[] $\to$ Ej.: producción total, consumo final, FBK...
				\4[] Saldos contables
				\4[] $\to$ Ej.: CNF, PIB, VAB, Riqueza neta...
				\4[$\to$] Agregando en una cifra diferentes sectores institucionales
				\4 Resumen resultado clave sobre actividad económica
				\4 Permiten análisis y comparación entre economías
			\3 Cocientes
				\4 Algunos agregados en relación a población
				\4[] $\to$ Agregado por habitante
				\4[] $\to$ Agregado por hogar
				\4[] $\to$ ...
		\2 Flujos y stocks
			\3 Stocks
				\4 Tenencia de activos y pasivos
				\4[] $\to$ En un momento determinado
				\4 Resultado de acumulación de transacciones
				\4[] $\to$ En momentos pasados
			\3 Flujos
				\4 Variaciones de valor económico
				\4[] Entre mediciones de valor de stocks
				\4[] $\to$ En distintos momentos temporales
				\4 Flujos de transacción
				\4[] Resultado de interacción entre uds. institucionales
				\4[] Transacciones de mutuo acuerdo entre uds.
				\4 Otros flujos
				\4[] No pueden clasificarse como transacciones
				\4[] Cambios de valor de activos y pasivos
				\4[] $\to$ que no resultan de transacciones
				\4[] P.ej.: pérdidas por desastres, variaciones de precios
		\2 Agregados principales
			\3 VAB\footnote{Ver \href{https://www.eustat.eus/documentos/opt_0/tema_155/elem_2531/definicion.html}{Eustat: valor añadido a precios básicos}}
				\4 Valor Añadido Bruto creado
				\4[] Un periodo dado
				\4[] En un sector determinado
				\4[] En un territorio determinado
				\4 Producción a precios básicos
				\4[] Precio de venta al comprador
				\4[] $\to$ Sin IVA
				\4[] $\to$ Sin IIEE
				\4[] $\to$ Sin otros impuestos sobre los productos
				\4[] $\to$ Con subvenciones a los productos
				\4[-] Consumo intermedio a precios de adquisición
				\4[=] Medida del valor añadido creado
				\4[] Sin introducir distorsión por impuestos al VA
				\4[] $\to$ Diferentes tipos por sectores
				\4 Útil en análisis sectorial
			\3 PIB
				\4 Agregado fundamental
				\4 En cuentas nacionales, a precio de mercado
				\4[] $\to$ En periodo en cuestión
				\4 Tres formas de obtener:
				\4[] Producción: $\text{PIB} \equiv P - \text{CI} + \text{Tn/Po}$
				\4[] $\to$ Posible descomponer por ramas de actividad
				\4[] $\to$ Posible calcular como suma de ramas
				\4[] Gasto: $\text{PIB} \equiv \text{GCF} + \text{FBK} + \text{X} - \text{M}$
				\4[] $\to$ Posible descomponer por sectores institucionales
				\4[] Ingreso: $\text{PIB} \equiv \text{RA} + \text{EBE} + \text{RMB} + \text{Tn/P+M}$
				\4[] $\to$ Posible descomponer por tipo de renta (L o K)
				\4 Discrepancias estadísticas
				\4[] En la práctica, tres métodos difieren ligeramente
				\4[] Necesario calcular todos y valorar
			\3 PIN
				\4 PIB neto de consumo de capital fijo/depreciación K
				\4[] $\text{PIN} \equiv \text{PIB} - \text{CKF}$
			\3 RNB
				\4 Rentas obtenidas por los agentes residentes
				\4[] Preferentemente relacionado a PIB vía ingreso
				\4[] rentas de L y K, impuestos netos pagados y cobrados
				\4[] $\text{RNB} \equiv \text{PIB} + \text{RA}_{\to \text{N}} - \text{RA}_{\to \text{RM}} + \text{RP}_{\to \text{N}} - \text{RP}_{\to \text{RM}} + \text{Tn/P+M}_{\to \text{N}} - \text{Tn/P+M}_{\to \text{RM}}$
			\3 RNN
				\4 RNB neto de consumo de capital fijo
				\4[] $\text{RNN} \equiv \text{RNB} - \text{CKF}$
			\3 RNBD
				\4 Rentas brutas de residentes disponibles para consumo
				\4[] RNB neto de transferencias corrientes netas
				\4[] $\text{RNBD} \equiv \text{RNB} + \text{TRC}_{\to N} - \text{TRC}_{\to RM}$
			\3 RNND
				\4 RNBD neto de consumo de capital fijo
				\4[] Más apropiada que RNBD
				\4[] $\to$ CKF debe reponerse para mantener producción
				\4[] $\text{RNND} \equiv \text{RNBD} - \text{CKF}$
			\3 Ahorro
				\4 Proporción de la RNBD no consumida
				\4[] $\text{S} = \text{RNBD} - \text{GCF}$
			\3 Ahorro neto
				\4 Proporción de la RNND no consumida
			\3 Saldo de Operaciones Corrientes con el Exterior
				\4 Diferencia entre
				\4[] Ingresos y gastos corrientes en el exterior
				\4[] $\text{SOCE} = \text{X} - \text{M} + \text{RA}_{N \to RM} - \text{RA}_{RM \to N}+ \text{RP}_{N \to RM} + \text{Tn/P+M}_{N \to RM} - \text{TRC}_{N \to RM}$
				\4 Equivale a saldo de Cuenta Corriente de BP
			\3 Capacidad/Necesidad de financiación
				\4 Suma de fin. prestada/obtenida por sectores inst.
				\4[] Recursos netos que economía pone a disposición del mundo
				\4 Saldo público
				\4[] Cap./Nec. de las administraciones públicas
				\4[] $\to$ Especialmente relevante
				\4[] $\Rightarrow$ Influye en sostenibilidad de déficit público
			\3 Riqueza neta
				\4 Suma del valor neto de los sectores institucionales
				\4 Valor neto:
				\4[]  Valor de activos no financieros
				\4[] + Valor neto de activos y pasivos financieros
				\4[] $\text{Riqueza neta} = \text{ANF} + \text{AF} - \text{PF}$
			\3 Rentas empresariales
				\4 Suma de todos los ingresos empresariales de los sectores
	\1 \marcar{Cuentas del SEC-2010}
		\2 Idea clave
			\3 Concepto
				\4 Conjunto de reglas y recomendaciones
				\4[] Para cuantificar stocks y flujos económicos
				\4[] En un sistema contable
				\4 Sistema de flujos y stocks
				\4[] Recogen actividad económica
				\4[] $\to$ En un periodo determinado
			\3 Objetivo
				\4 Caracterizar estado de una economía
				\4[] Estructura sectorial y regional
				\4[] Valor añadido por industria y sector
				\4[] Desarrollo a lo largo del tiempo
				\4[] Relación con otras economías
				\4[] Prever estados futuros de una economía
				\4 Supervisar cumplimiento de obligaciones
				\4[] P.ej.: Pacto de Estabilidad y Crecimiento
				\4 Ayudar al diseño de políticas económicas
				\4[] Diagnosticar problemas
				\4[] Comprender efectos de políticas
				\4[] Implementar soluciones
			\3 Resultados
				\4 Sistema estandarizado de cuentas
				\4 Conjunto de macromagnitudes comparables
				\4 Componentes
				\4[] Cuentas de los sectores institucionales
				\4[] Descripción sistemática de las fases de la producción
				\4[] Variaciones de la riqueza entre periodos
				\4[] $\to$ De los diferentes sectores de la economía
				\4 Marco input-output y contabilidad por industrias
				\4[] Flujos de bienes y servicios entre industrias
				\4[] Rentas generadas por una industria
				\4[] $\to$ Igual a valor añadido
				\4[] Oferta de bienes en una industria
				\4[] $\to$ Igual a demanda de bienes
		\2 Reglas y criterios contables
			\3 Marco jurídico
				\4 Sistema de Cuentas Nacionales de 2008 (SCN-2008)
				\4[] ONU, Comisión Europea, FMI, OCDE
				\4[] Armonizar contabilidad nacional a nivel global
				\4[] Describir economías con criterios homogéneos
				\4 Sistema Europeo de Cuentas de 2010 (SEC-2010)
				\4[] Aprobado por Reglamento de UE en 2013
				\4[] Entrada en vigor en 2014
				\4[] Remplaza a SEC-95
				\4[] Basado en SCN-2008
				\4[] Adaptado a contexto europeo
				\4[] Definiciones más específicas, idiosincráticas a UE
			\3 Residencia
				\4 Territorio económico
				\4[] Donde u. inst. tiene relación + fuerte
				\4 Factores:
				\4[] Presencia física
				\4[] Sujección a leyes
			\3 Criterios de valoración
				\4 Valores monetarios efectivos cuando sea posible
				\4[] $\to$ Resultado de valor de transacción
				\4 En otros casos:
				\4[] Estimaciones
				\4[] Referencia a precios de mercado
				\4[] Costes de producción
			\3 Volumen y valor
				\4 Términos de volumen
				\4[] Valoración a precios constantes
				\4[] Precios vigentes en ejercicio anterior de base
				\4 Términos de valor
				\4[] Valoración a precios corrientes
				\4[] Precios vigentes en periodo actual
			\3 Valoraciones de productos
				\4 Precio de vendedor y comprador
				\4[] Habitualmente son diferentes
				\4[] $\to$ Impuestos, gastos transporte, márgenes..
				\4[] $\Rightarrow$ Diferencias
				\4 Empleos
				\4[] A precios del comprador
				\4[] $\to$ Incluye gasto de transporte
				\4 Recursos
				\4[] A precios básicos
				\4 Precios básicos
				\4[] Precio a cobrar a comprador
				\4[] -- Impuestos sobre producción\footnote{Incluyendo IVA, IIEE y otros impuestos sobre la producción}
				\4[] + Subvención a producción
				\4 Precio de productor
				\4[] Precio a cobrar a comprador
				\4[] -- IVA o similar
				\4[] Precio de productor utilizable
				\4[] $\to$ Si precio básico no disponible
				\4 Importaciones
				\4[] Precio CIF (Cost, Insurance, Fright)
				\4[] En contexto de balanza de pagos
				\4[] $\to$ Importaciones a FOB
				\4[] $\then$ Necesario desagregar bienes y servicios\footnote{Como es el caso de los seguros y el transporte.}
				\4 Exportaciones
				\4[] FOB
				\4[] $\to$ Free on board
				\4[] $\to$ Precio en frontera del exportador
			\3 Periodo temporal y momento de registro
				\4 Cualquier periodo puede ser aceptable
				\4 Generalmente años, trimestres
				\4 Preferentemente, criterio de devengo
				\4[] $\to$ Salvo impuestos
				\4 Principio de devengo
			\3 Contabilidad por partida doble/cuádruple
				\4 Transacción da origen a dos asientos
				\4[] En cada parte implicada
				\4 Cuentas en forma de T
				\4[] Dos áreas: empleos y recursos
				\4[] $\to$ Equivalentes a debe y haber
				\4 Contabilidad vertical por partida doble
				\4[] Igualdad de asientos acreedores y deudores
				\4[] Para todas las transacciones
				\4[] $\then$ Cuentas coherentes de cada ud. individual
				\4[] $\then$ Activos -- pasivos =  neto
				\4 Contabilidad horizontal por partida doble
				\4[] Si A suministra algo a B
				\4[] $\to$ Idéntico monto en ambos agentes
				\4[] Coherencia de categorías entre uds. inst.
				\4[$\then$] Partida cuádruple
			\3 Saldos contables
				\4 Cada cuenta se equilibra
				\4[] Diferencia entre recursos y empleos
				\4 Saldos contables como primera partida
				\4[] Cuenta siguiente
				\4[] $\to$ Secuencia de cuentas es un todo articulado
		\2 Unidades estadísticas y sectores institucionales
			\3 Unidades estadísticas
				\4 Agentes económicos capaces de:
				\4[] Ser propietarios de bienes y activos
				\4[] Contraer pasivos
				\4[] Realizar en nombre propio:
				\4[] $\to$ Actividades económicas
				\4[] $\to$ Transacciones con otras unidades
				\4 Centro de decisión económica
				\4[] Uniformidad de comportamiento
				\4[] $\to$ Consumo
				\4[] $\to$ Producción
				\4[] Autonomía de decisión
				\4 Agrupables en sectores institucionales
			\3 Sectores institucionales
				\4 Agrupaciones de unidades estadísticas
				\4[] En función de:
				\4[] $\to$ Funciones principales
				\4[] $\to$ Objetivos
				\4[] $\to$ Comportamiento
				\4[S.11] Sociedades no financieras
				\4[] Productores de mercado de bienes y servicios
				\4[] $\to$ No financieros
				\4[] Recursos proceden de la venta de su producción
				\4[S.12] Instituciones financieras
				\4[] Actividad principal es intermediación financiera
				\4[] $\to$ y auxiliares
				\4[S.13] Administraciones públicas
				\4[] Productores no de mercado
				\4[] Financiados mediante pagos obligatorios
				\4[] Producción
				\4[] $\to$ Consumo público e individual
				\4[] Redistribución
				\4[] Cuatro subsectores (España)
				\4[] $\to$ Admón. central
				\4[] $\to$ Admón. regional
				\4[] $\to$ Admón. local
				\4[] $\to$ Fondos de la SS
				\4[S.14] Hogares
				\4[] Todos los hogares residentes
				\4[] Incluidos hogares institucionales
				\4[] $\to$ Prisiones
				\4[] $\to$ Hospitales
				\4[] $\to$ Residencias
				\4[] ..
				\4[S.15] ISFLSH
				\4[] Inst. Sin Fines de Lucro al Servicio de los Hogares
				\4[] Venden a precios no significativos económicamente
				\4[] No están bajo el control del gobierno
				\4[S.2] Resto del mundo
		\2 Análisis funcional
			\3 Idea clave
				\4 Contexto
				\4[] Análisis institucional basado en:
				\4[] $\to$ Comportamiento similar
				\4[] $\to$ Similares objetivos
				\4[] $\to$ Similares características legales y sociales
				\4[] Diferentes sectores de la economía
				\4[] $\to$ Producen diferentes bienes y servicios
				\4[] $\to$ Relaciones de interdependencia
				\4 Objetivos
				\4[] Entender interrelaciones entre actividades
				\4[] Análisis de equilibrio general entre actividades
				\4 Resultados
				\4[] Clasificación de actividades
				\4[] Cuentas nacionales tienen en cuenta
				\4[] Aplicaciones específicas en otros ámbitos de
			\3 UAE-- Unidades de Actividad Económica a Nivel Local
				\4 Unidades productoras o establecimientos
				\4 Realizan ejercicio en una actividad CNAE de clase 4
				\4 Ubicadas en lugares delimitado geográficamente
			\3 Clasificación Nacional de Actividades Económicas
				\4 21 Secciones generales $\to$ Letras
				\4 Códigos XXXX de 4 dígitos para clasificar actividad
				\4[] División: 2 dígitos
				\4[] Grupo: 3 dígitos
				\4[] Clases: 4 dígitos
				\4 Asignado a establecimientos con actividad similar
			\3 Ramas de actividad
				\4 Agrupaciones de actividades similares a divisiones
				\4 Esencial en análisis input-output
				\4 Actividades aproximadamente similares
				\4[] Entendibles como sectores en ``sectores de actividad''
		\2 Fuentes de información
			\3 INE
				\4 Operaciones con bienes y servicios
				\4[] Destino y origen
				\4 Operaciones de distribución
				\4[] Cómo se reparte el valor añadido
				\4 Operaciones no incluidas en anteriores
				\4[] Adquisiciones menos cesiones de ANFNP
				\4 Auxiliarmente, información de AEAT y otros
			\3 Banco de España
				\4 Operaciones financieras
				\4[] $\to$ Acumulación financiera
				\4 Variaciones de activos y pasivos financieros
			\3 AEAT
				\4 Datos de importación y exportación
			\3 Proveedores de servicios de pago
				\4 Información sobre transacciones
			\3 RIE-- Registro de inversiones exteriores de SEComercio
				\4 Base de datos DATAINVEX
				\4 Diferencias significativas con datos BP
				\4[] BP criterio de caja
				\4[] $\to$ Flujos contabilizados a medida que se desembolsan
				\4[] RIE criterio de devengo
				\4[] $\to$ Flujos contabilizados de una sola vez
				\4[] Distinto alcance de medición
				\4[] $\to$ RIE no incluye préstamos entre filiales
				\4[] $\to$ RIE no incluye inversiones en inmuebles
				\4[] BP publica sólo datos de inversión neta
				\4[] $\to$ RIE bruta y neta
				\4[] RIE desagrega entre ETVE y no ETVE
				\4[] Diferente desglose
				\4[] $\to$ BP sólo sectores institucionales residentes
				\4[] $\to$ RIE tipo de operación, país, CCAA
		\2 Cambios metodológicos en SEC-2010\footnote{Ver \url{https://ec.europa.eu/commission/presscorner/detail/en/MEMO_14_21} y Muriel de la Riva (2015) en carpeta del tema.}
			\3 Idea clave
				\4 SCN-2008
				\4[] Nueva edición de SCN de ONU
				\4[] $\to$ Sustituye a SNA 1993
				\4[] $\then$ Aún presente en algunos países
				\4[] $\then$ Necesario conocer cambios
				\4 Implementada por UE en 2014
				\4 Estados Unidos implementa su versión en 2013
				\4 Mayor consistencia con 6Manual de BP
				\4[] También adoptada a nivel europeo
			\3 I+D capitalizada
				\4 Considerada como FBK en SEC-2010
				\4 Deja de considerarse gasto corriente
				\4 Efecto sobre del PIB anual\footnote{Ver Muriel de la Riva (2015), pág. 10.}
				\4[] No tiene impacto en el año en que se compra
				\4[] $\to$ Sobre producción o valor añadido
				\4[] Sí tiene en años posteriores
				\4[] $\to$ Vía coste de amortización
				\4 Relevante para objetivos Europa 2020
			\3 Gasto militar capitalizada
				\4 Considerada como FBK en SEC-2010
				\4 Efecto sobre del PIB anual\footnote{Ver Muriel de la Riva (2015), pág. 10.}
				\4[] No tiene impacto en el año en que se compra
				\4[] $\to$ Sobre producción o valor añadido
				\4[] Sí tiene en años posteriores
				\4[] $\to$ Vía coste de amortización
				\4 Valoración en SEC-95
				\4[] Considerada como GCF o CI
			\3 Administración pública
				\4 Redefinición del perímetro del sector
				\4 Se mantienen criterios tradicionales
				\4[] No deben ser unidades separadas de Admón. Pública
				\4[] Regla del 50\%
				\4[] $\to$ Venta cubre 50\% de costes de producción
				\4[] $\to$ Necesario incluir intereses en costes de prod.
				\4 Criterios cualitativos adicionales
				\4[] Productores no deben tener incentivos:
				\4[] $\to$ A ajustar producción para que sea viable
				\4[] No son capaces de operar
				\4[] $\to$ En condiciones de mercado
				\4[] No pueden contraer obligaciones financieras
				\4[] $\to$ De manera autónoma
				\4 Poco impacto sobre déficit y deuda pública
			\3 Procesamiento de bienes en el exterior
				\4 Valoración en SEC-95
				\4[] Considerada como importación y exportación
				\4 Consideración en SEC-2010
				\4[] No entra dentro de X e I
				\4[] $\to$ Enfoque basado en cambio de propiedad
				\4[] $\to$ No en desplazamiento físico
				\4[] Sí se anota un servicio de manufactura
				\4 Sin impacto sobre saldo de cuenta corriente
				\4 Sí reduce nivel total de importaciones y exportaciones
			\3 Pensiones
				\4 Análisis más detallado obligatorio
				\4 Tabla suplementaria
				\4[] Obligaciones de todos los sistemas de pensiones
			\3 Sector de seguros
				\4 Valoración en SEC-95
				\4[] Diferencia entre primas e indemnizaciones
				\4[] $\to$ Excesiva volatilidad
				\4 Cambios en SEC-2010
				\4[] Cambio en fórmula de cálculo
				\4[] $\to$ Reducir volatilidad de output
			\3 Instituciones financieras
				\4 Mayor desagregación de subsectores
			\3 Impacto sobre series de PIB
				\4 Previsto ligero aumento
				\4[] Cercano a 2.5\% del PIB
				\4 Capitalización de i+D mayor parte
		\2 Cuentas corrientes\footnote{Ver pág. 389 de SCN-2008.}
			\3 \underline{Bienes y servicios} (0)
				\4 Identidad
				\4[] $\to$ Sin saldo contable
				\4 Muestra origen y destino de ByS
				\4[] Ofrecidos a la economía
				\4 Relacionada con tablas input/output
				\4 Recursos a izquierda, empleo a derecha
				\4[] $\to$ Al revés que el resto
				\4 \textbf{Recursos}
				\4[] Producción
				\4[] Importaciones de bienes y servicios
				\4[] Impuestos netos de subvenciones a productos
				\4 \textbf{Empleos}
				\4[] Consumos intermedios
				\4[] Gasto en consumo final
				\4[] Formación bruta de capital fijo
				\4[] Variación de existencias
				\4[] Exportaciones
			\3 \underline{Producción} (I)
				\4 \textbf{Recursos}
				\4[] Producción (precios básicos)
				\4[] Tn/Po\footnote{Impuestos netos de subvenciones (Tn) sobre los productos incluido el IVA pero sin importaciones y sin otros impuestos a la producción que se consideran ya incluidos en la \textit{producción}. Es posible que la producción esté a precios de productor de tal manera que Tn/Po sólo incluya el IVA.}
				\4 \textbf{Empleos}
				\4[] Consumo Intermedio (CI)
				\4[] \fbox{\textit{PIB/VAB (sectores)}}
				\4[] --Consumo de Capital Fijo (CKF)
				\4[] \fbox{\textit{PIN/VAN (sectores)}}
			\3 \underline{Explotación} (II.1.1)
				\4 Idea clave
				\4[] Explicar la generación de rentas
				\4[] $\to$ Cómo se distribuye entre trabajo y capital
				\4 \textbf{Recursos}
				\4[] PIB
				\4 \textbf{Empleos}
				\4[] Remuneración de Asalariados (RA)
				\4[] Tn/P+M (Impuestos netos sobre producción e importaciones)
				\4[] \fbox{\textit{Excedente Bruto de Explotación (EBE)}}
				\4[] \fbox{\textit{Renta Mixta Bruta (RMB)}}
				\4[] --CKF
				\4[] ENE
				\4[] RMN
			\3 \underline{Asignación de la renta primaria} (II.1.2)
				\4 Idea clave
				\4[] Explicar la percepción de rentas
				\4[] $\to$ Qué rentas primarias reciben los residentes
				\4 \textbf{Recursos}
				\4[] EBE
				\4[] RMB
				\4[] RA
				\4[] $\text{RA}_{\text{RM}\to \text{N}}$
				\4[] $\text{RP}_{\text{RM}\to \text{N}}$
				\4[] Tn/P+M (AA.PP Nacionales)
				\4 \textbf{Empleos}
				\4[] $\text{RA}_{\text{N} \to \text{RM}}$
				\4[] $\text{RP}_{\text{N} \to \text{RM}}$
				\4[] \fbox{\textit{RNB}}
			\3 \underline{Distribución secundaria de la renta} (II.2)
				\4 Idea clave
				\4[] $\to$ ¿Qué rentas secundarias reciben los residentes?
				\4[] Transformación del ingreso primario
				\4[] $\to$ En ingreso disponible
				\4[] $\to$ Tras pagos y cobros de transferencias
				\4[] Distribución secundaria de la renta en especie (II.3)
				\4[] $\to$ Subcuenta
				\4[] $\to$ Mismo saldo que RNBD
				\4[] $\to$ A nivel de sectores es relevante:
				\4[] $\to$ Tener en cuenta prestaciones sociales en especie y gasto colectivo
				\4 \textbf{Recursos}
				\4[] RNB
				\4[] $\text{Transferencias corrientes}_{\text{RM} \to \text{N}}$
				\4[] $\to$ Impuestos sobre ingreso y riqueza
				\4[] $\to$ Contribuciones sociales
				\4[] $\to$ Beneficios en especie
				\4[] $\to$ Transferencias personales
				\4[] $\to$ Otras transferencias
				\4 \textbf{Empleos}
				\4[] $\text{Transferencias corrientes}_{\text{N} \to \text{RM}}$
				\4[] $\to$ Impuestos sobre ingreso y riqueza
				\4[] $\to$ Contribuciones sociales
				\4[] $\to$ Beneficios en especie
				\4[] $\to$ Transferencias personales
				\4[] $\to$ Otras transferencias
				\4[] \fbox{\textit{RNB Disponible}}
			\3 \underline{Distribución secundaria de la renta en especie} (II.3)
				\4 Idea clave
				\4[] Redistribución vía prestaciones en especie
				\4[] $\to$ Entre sectores institucionales
				\4[] $\then$ Se compensan entre sectores
				\4[] $\then$ Sin saldos
				\4 \textbf{Recursos}
				\4[] RNBD
				\4[] Transferencias sociales en especie
				\4 \textbf{Empleos}
				\4[] RNBD
				\4[] Transferencias sociales en especie
			\3 \underline{Utilización de la renta} (II.4)
				\4 Idea clave
				\4[] Cuánta renta se consume
				\4[] Cuánta renta se ahorro
				\4 \textbf{Recursos}
				\4[] RNB Disponible
				\4 \textbf{Empleos}
				\4[] GCF Individual
				\4[] GCF Colectivo
				\4[] \fbox{\textit{Ahorro Nacional Bruto}}
				\4[] -- CKF (Consumo de capital fijo)
				\4[] \textit{ANN}
		\2 Cuentas de acumulación (III)
			\3 \underline{Cuenta de capital} (III.1)
				\4 Idea clave
				\4[] Financiación de Act. no financieros
				\4[] Hasta qué punto ahorro y TRK
				\4[] $\to$ Financian adq. y ces. de ANF
				\4 Una cuenta dividida en dos:
				\4[] 1. Ahorro y transferencias de K: VPNDAYTK (III.1.1)
				\4[] 2. Formación de capital y adq. Ces. ANFNP (III.1.2)
				\4[] $\to$  Combinando ambas, el saldo es la CNF\footnote{Capacidad/Necesidad de Financiación)}
			\3 VPNDAYTK (III.1.1) \footnote{Variación del Patrimonio Neto Debida a Ahorro y Transferencias de Capital}
				\4 Idea clave
				\4[] Contribución de ahorro neto y TRK
				\4[] $\to$ A financiar variación de riqueza neta
				\4 \textbf{Variación de Pasivos y del Patrimonio Neto}
				\4[] Ahorro Nacional Neto
				\4[] $\text{Transferencias de capital}_{\text{RM} \to \text{N}}$
				\4[] -- $\text{Transferencias de capital}_{\text{N}\to \text{RM}}$
				\4 \textbf{Variación de Activos}
				\4[] VPNDAYTK
			\3 Cuenta de Adq. de Activos No Financieros (III.1.2)
				\4 Idea clave
				\4[] Dada FBK y adq. neta de activos NFNP
				\4[] $\to$ ¿Cuánta financiación es necesaria?
				\4 \textbf{Variación de Pasivos y Patrimonio Neto}
				\4[] VPNDAYTK
				\4 \textbf{Variación de Activos}
				\4[] Formación Bruta de Capital
				\4[] $\to$ FBK Fijo
				\4[] $\to$ Variación de existencias (VE)
				\4[]  -- CKF
				\4[] Adq. menos ces. ANFNP\footnote{Activos no financieros no producidos.}
				\4[] \fbox{\textit{Capacidad (+)/Necesidad de financiación (-)}}
			\3 \underline{Cuenta Financiera} (III.2)
				\4 Idea clave
				\4[] Cómo cambian activos y pasivos financieros
				\4[] Cómo se utiliza/utiliza la cap./nec. de financiación
				\4 \textbf{Variación de pasivos y Patrimonio Neto}
				\4[] Capacidad (+)/Necesidad de financiación (-)
				\4[] Incremento neto de pasivos
				\4 \textbf{Variación de activos}
				\4[] Adquisición neta de activos financieros
				\4[] $\to$ Oro monetario y DEG
				\4[] $\to$ Dinero legal y depósitos
				\4[] $\to$ Títulos de deuda
				\4[] $\to$ Préstamos
				\4[] $\to$ Participaciones de capital y fondos de inversión
				\4[] $\to$ Sistemas de seguros, pensiones y garantías
				\4[] $\to$ Derivados y OCE
				\4[] $\to$ Otras cuentas por cobrar
			\3 \underline{Cuenta de otras variaciones de los activos} (III.3)
				\4 Idea clave
				\4[] $\Delta$ de A y P no debidos al ahorro y TK debidas a:
				\4[] $\to$ $\Delta$ de volumen
				\4[] $\to$ $\Delta$ de valor
				\4[] Dividida en dos cuentas
				\4[] $\to$ Cuenta de otras variaciones del volumen de activos
				\4[] $\to$ Cuenta de revalorización
			\3 Cuenta de otras variaciones del \underline{volumen} de activos (III.3.1)
				\4 \textbf{Variación de pasivos y Patrimonio Neto}
				\4 \textbf{Variación de activos}
				\4[] Activos producidos (neta)
				\4[] Activos no producidos (neta)
				\4[] Activos financieros (neta)
				\4[] \fbox{\textit{Var. del v. neto debidas a otras var. en vol. de activos}}
			\3 Cuenta de revalorización (III.3.2)
				\4 \textbf{Variación de pasivos y Patrimonio Neto}
				\4 \textbf{Variación de activos}
				\4[] Activos no financieros
				\4[] Activos financieros/pasivos
				\4[] \fbox{\textit{Var. del v. neto debidas a ganancias y pérdidas}}
		\2 Balances (IV)
			\3 Idea clave
				\4 Cuantificar activos, pasivos y valor neto
				\4[] Antes y después de variaciones
				\4 Tres componentes:
				\4[] Balance inicial (IV.1)
				\4[] Variaciones del balance (IV.2)
				\4[] Balance final (IV.3)
			\3 \underline{Balance inicial} (IV.1)
				\4 \textbf{Activos}
				\4[] Activos no financieros
				\4[] Activos financieros
				\4 \textbf{Pasivos}
				\4[] Pasivos financieros
				\4[] \fbox{\textit{Valor neto}}
			\3 \underline{Variaciones del balance} (IV.2)
				\4 \textbf{Activos}
				\4[] Activos no financieros
				\4[] Activos financieros
				\4 \textbf{Pasivos}
				\4[] Pasivos financieros
				\4[] \fbox{\textit{Variación total del valor neto}}
			\3 \underline{Balance final} (IV.3)
				\4 \textbf{Activos}
				\4[] Activos no financieros
				\4[] Activos financieros
				\4 \textbf{Pasivos}
				\4[] Pasivos financieros
				\4[] \fbox{\textit{Valor neto}}
		\2 Cuentas de resto del mundo (V)
			\3 Idea clave
				\4 Representar operaciones entre residentes y no residentes
				\4[] Sectores institucionales interaccionan con resto del mundo
				\4 Perspectiva del resto del mundo
				\4[] Recursos son ingresos para el resto del mundo
				\4[] Exportaciones son gastos para el resto del mundo
				\4[] $\to$ Importaciones del país son recursos del RM
				\4[] $\to$ Exportaciones del país son usos del RM
			\3 \underline{Cuenta de intercambios exteriores de bienes y servicios} (V.1)
				\4 Idea clave
				\4[] Cuánto contribuyen las exportaciones netas al ahorro
				\4[] Cuál es el saldo de la balanza de bienes y servicios
				\4 \textbf{Recursos}
				\4[] Importaciones de bienes y servicios
				\4 \textbf{Empleos}
				\4[] Exportaciones de bienes y servicios
				\4[] \fbox{Saldo de intercambios exteriores de ByS}
			\3 \underline{Cuenta exterior de ingreso primario y secundario} (V.2)
				\4 Idea clave
				\4[] Relación entre ahorro nacional y RM
				\4[] ¿Cómo se ahorra con respecto al RM?
				\4[] $\to$ Bienes y servicios
				\4[] $\to$ Rentas primarias
				\4[] $\to$ Rentas secundarias
				\4 \textbf{Recursos}
				\4[] Saldo de intercambios exteriores de ByS
				\4[] $\text{RA}_{\text{N} \to \text{RM}}$
				\4[] $\text{RP}_{\text{N} \to \text{RM}}$
				\4[] Tn/P+M pagados al RM
				\4[] Impuestos sobre renta y patrimonio al RM
				\4[] Cotizaciones y prestaciones sociales al RM
				\4[] Otras transferencias corrientes al RM
				\4 \textbf{Empleos}
				\4[] $\text{RA}_{\text{RM} \to \text{N}}$
				\4[] $\text{RP}_{\text{RM}\to \text{N}}$
				\4[] Tn/P+M pagados a N
				\4[] Impuestos sobre renta y patrimonio a N
				\4[] Cotizaciones y prestaciones sociales a N
				\4[] Otras transferencias corrientes a N
				\4[] \fbox{\textit{Saldo de operaciones corrientes con RM} (SOCE)}
			\3[] \underline{Cuenta exterior de acumulación} (V.3)
			\3 \underline{Cuenta exterior de capital} (V.3.1)
				\4 Idea clave
				\4[] Qué saldo tiene RM respecto de N resultado de:
				\4[] $\to$ Ahorro
				\4[] $\to$ Adq. menos ces. act. no producidos
				\4[] $\to$ Transferencias de capital
				\4 Divido en dos cuentas:
				\4[] 1. Variación del PN debida a SOCE y TK
				\4[] 2. Cuenta de adquisición de activos no financieros
			\3 VPNDSOCEYTK (V.3.1.1)
				\4 Idea clave
				\4[] Variación del patrimonio neto del RM debido a:
				\4[] $\to$ Ahorro del resto del mundo
				\4[] $\to$ Transferencias de K del país al RM
				\4 \textbf{Variación de Pasivos y Patrimonio Neto}
				\4[] SOCE
				\4[] $\text{Transferencias de capital}_{\text{N} \to \text{RM}}$
				\4[] -- $\text{Transferencias de capital}_{\text{RM}\to \text{N}}$
				\4 \textbf{Variación de Activos}
				\4[] \fbox{\textit{VPNDSOCEYTK}}
			\3 Cuenta de adquisición de activos no financieros (V.3.1.2)
				\4 Idea clave
				\4[] Cuánta financiación aporta el RM a N
				\4 \textbf{Variación de Pasivos y Patrimonio Neto}
				\4[] VPNDSOCEYTK
				\4 \textbf{Variación de Activos}
				\4[] Adq. menos ces. ANFNP
				\4[] \fbox{\textit{Capacidad (+)/Necesidad de financiación (-)}}
			\3 \underline{Cuenta financiera} (V.3.2)
				\4 Idea clave
				\4[] Coincide con CFinanciera (III.2) de ec. nacional
				\4[] $\to$ Signo contrario de partidas
				\4 \textbf{Variación de pasivos y Patrimonio Neto}
				\4[] Capacidad (+)/Necesidad de financiación (-)
				\4[] Incremento neto de pasivos
				\4 \textbf{Variación de activos}
				\4[] Adquisición neta de activos financieros
		\2 Otras cuentas
			\3 Input-Output
			\3 Cuentas satélite
				\4 Objetivos:
				\4[] Extender información de otras cuentas
				\4[] Cuantificar indicadores no monetarios relevantes
				\4 A partir de
				\4[] Información de cuentas nacionales
				\4[] Información externa a cuentas
				\4 Matrices de contabilidad social
				\4 Cuenta satélite del turismo
				\4[] Turismo implica gasto en muchos sectores
				\4[] $\to$ Hostelería
				\4[] $\to$ Transporte
				\4[] $\to$ Alimentación
				\4[] $\to$ ...
				\4[] A priori contabilidad nacional no cuantifica
				\4[] $\Rightarrow$ Necesaria estimación del impacto del turismo
				\4 Análisis del sistema monetario
				\4 Capital humano
				\4 Análisis micro del gasto de los hogares
				\4 Producción de hogares
				\4 Estimaciones fiscales
				\4 Cuentas medioambientales
		\2 Relaciones con otros conceptos
			\3 Balanza de pagos
			\3 Tablas input-output
	\1[] \marcar{Conclusión}
		\2 Recapitulación
			\3 Cuentas del SEC-2010
			\3 Macromagnitudes
		\2 Idea final
			\3 Dificultades de medición
				\4 Cuantificación de realidad económica es compleja
				\4[] No todas transacciones son monetarias
				\4[] No todas transacciones monetarias se registran
				\4[] Métodos de estimación son imperfectos
			\3 Cuantificación del bienestar: más alla del PIB
				\4 Contabilidad nacional permite
				\4[] $\to$ Comparar con pasado
				\4[] $\to$ Comparar con similares
				\4[] $\to$ Identificar problemas de economías
				\4[] $\to$ Diseñar soluciones
				\4 ¿Permite cuantificar el bienestar en sí mismo?
				\4[] PIB es indicador más utilizado
				\4[] Pero no captura todas dimensiones
				\4[] $\to$ Reparto de renta
				\4[] $\to$ Preferencia por desigualdad
				\4[] $\to$ Provisión de necesidades básicas
				\4[] $\to$ Explotación de recursos no renovables
				\4[] $\to$ Sostenibilidad de crecimiento
				\4 Beyond GDP
				\4[] Programa de investigación
				\4[] $\to$ Fleurbay, Blanchet
				\4[] Iniciativa de Comisión Europea
				\4[] $\to$ Construir nuevos indicadores
				\4[] $\to$ Cuantificación de más dimensiones
\end{esquemal}


































\conceptos

\concepto{Rentas de la propiedad}

\concepto{Impuestos sobre los productos, la producción y las importaciones}

Los impuestos sobre los productos netos de subvenciones (Tn/Po) son aquellos impuestos pagados en función de la cantidad o el valor de los bienes y servicios producidos o vendidos, netos de las subvenciones por los mismos conceptos. No incluyen impuestos sobre el beneficio u otro ingreso recibido por las empresas. Incluyen el IVA.

Los otros impuestos sobre la producción (Tn/O) incluyen aquellos impuestos aplicables al uso de factores de producción así como a licencias y tasas para obtener permiso para llevar a cabo alguna actividad determinada o para producir en términos generales. 

Los impuestos sobre la producción (Tn/P) incluyen los impuestos sobre los productos y los otros impuestos sobre la producción.

Los impuestos sobre la producción y las importaciones (Tn/P+M) incluyen los impuestos sobre la producción y los impuestos sobre las importaciones.

\concepto{Precios básicos y precios de productor}

Los \textit{precios básicos} son las cantidades que retienen los productores por cada unidad vendida de producto. No incluye los impuestos recibidos del comprador y reintegrados a la administración pública, pero sí que incluye las subvenciones recibidas por la administración pública para abaratar el coste de producción. Si al precio básico se le añaden los impuestos sobre los productos salvo el IVA y se descuentan las subvenciones a los productos, tenemos el \textit{precio de productor}. El \textit{precio de adquisición} es el resultado de sumar al precio de producción el IVA no deducible, los gastos de transporte y los márgenes comerciales de mayoristas y minoristas.

\begin{align*}
\text{P}_\text{ADQUISICION} = \text{P}_\text{PRODUCCIÓN} + \text{IVA} + \text{Transporte} + \text{Márgen comercial de minoristas/mayoristas} \\
\text{P}_\text{PRODUCCION} = \text{P}_\text{BASICO} + \text{Impuestos}_\text{Productos sin IVA} - \text{Subvención}_\text{Productos}
\end{align*}


\preguntas

\seccion{Test 2017}
\textbf{16.} Un país con un déficit exterior del 10\% del PIB y un superávit público del 1\% del PIB:

\begin{itemize}
	\item[a] Tiene una tasa de inversión inferior a la tasa de ahorro privado.
	\item[b] Tiene una tasa de inversión igual a la tasa de ahorro privado.
	\item[c] Tiene una tasa de inversión superior a la tasa de ahorro privado.
	\item[d] Con esta información no se puede saber nada sobre la tasa de ahorro privado y de inversión.
\end{itemize}

\notas
\textbf{2017}: \textbf{16.} C

Mirar diferencia entre impuestos sobre producción e impuestos sobre productos

\bibliografia

Mirar en Palgrave:

\begin{itemize}
	\item national income
\end{itemize}

Banco de España. \textit{Cuentas Financieras de la Economía Española} (2018) -- En carpeta del tema (ZIP)

Eurostat/Comisión Europea. \textit{European System of Accounts} (2010) -- En carpeta del tema

IGAE (2014) \textit{Nota sobre los cambios metodológicos de aplicación del nuevo SEC 2010 que afectan a las Cuentas de las Administraciones Públicas} -- En carpeta del tema

Ledo Arias, R.; Frutos Vivar, R. \textit{Sistema de cuentas nacionales.} Diapositivas curso TECO.

Lequiller, F.; Blades, D. \textit{Understanding national accounts} (2014) OCDE -- En carpeta del tema

Muñoz-Cidad, C. \textit{Las cuentas nacionales. Introducción a la economía aplicada.} Partes I, II y III

Muriel de la Riva, S. (2015) \textit{Las novedades metodológicas del SEC 2010 desde el enfoque institucional de la Contabilidad Nacional} Revista Índice. Octubre de 2015 -- En carpeta del tema. 

Naciones Unidas, Comisión Europea, OCDE, FMI, Grupo del Banco Mundial. \textit{Sistema de Cuentas Nacionales} (2008) -- En carpeta del tema

\end{document}
