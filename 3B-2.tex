\documentclass{nuevotema}

\tema{3B-2}
\titulo{La empresa y las decisiones de inversión. Diferentes criterios de valoración de proyectos. Rentabilidad, riesgo y coste del capital.}

\begin{document}

\ideaclave

Existen múltiples formas de definir el concepto de empresa, así varias justificaciones de su existencia. De esta forma, una empresa puede entenderse como un conjunto de inputs organizado para producir un conjunto de outputs, un conjunto de contratos que configuran una entidad social con carácter diferenciado, una institución que permite reducir los costes de transacción y gestión... Los objetivos de la empresa son también varios en función de la perspectiva que se considere y de la empresa en cuestión. Así, el objetivo de la empresa puede consistir en la producción de la máxima cantidad de producto, en la maximización de los beneficios de la empresa, en la utilización de la máxima cantidad de factores productivos, la creación de valor para el accionista de la empresa... En esta exposición asumimos que el objetivo último de las empresas es la creación de valor para el accionista, dadas las potestades que la legislación confiere a éstos en las economías capitalistas para decidir en última instancia como se gestiona la empresa. La creación o destrucción de valor tiene origen en una actividad muy determinada: la inversión. Así, la inversión entendida como sacrificio en un momento temporal determinado para obtener un beneficio en otro, se concreta en el seno de la empresa en la organización de los recursos disponibles de tal manera que estos generen recursos adicionales que aumenten el patrimonio de los accionistas. La inversión creadora de valor es, o debería ser, el objetivo último de los gestores de la empresa. Dada la conjunción de múltiples factores a la hora de determinar el resultado de una inversión, la toma de decisiones respecto a la inversión es un proceso complejo construido en torno a múltiples conceptos, entre los cuales destacan la rentabilidad, el riesgo o el coste del capital. El \textbf{objeto} de esta exposición consiste en dar respuesta a preguntas tales cómo: ¿qué es la inversión? ¿qué es la rentabilidad? ¿qué es el riesgo? ¿qué es el coste del capital? ¿qué criterios existen para decidir entre inversiones? ¿en qué invierten las empresas? ¿qué particularidades tiene la inversión en activo fijo y activo corriente? La \textbf{estructura} de la exposición se divide en tres partes. En primer lugar examinamos los conceptos de rentabilidad, riesgo y coste del capital. Posteriormente analizamos la valoración de proyectos de inversión, de forma general y sin entrar en las particularidades de los diferentes tipos de proyectos. Por último, presentamos los aspectos concretos de la inversión en activo fijo y en activo corriente.

Los conceptos de \marcar{rentabilidad, riesgo y coste del capital} son el punto de partida de la decisión de inversión en el seno de las empresas. Como se ha dicho ya, una \textbf{inversión} es una renuncia a una cantidad de riqueza en un momento temporal dado para obtener un aumento en otro momento diferente, con el objetivo de que la ganancia compense la pérdida. Esa diferencia entre beneficio y sacrificio se traduce a priori en un aumento del capital de la empresa o en términos contables, en un aumento del patrimonio neto. Una inversión puede caracterizarse así como una secuencia de flujos de caja asociados a momentos temporales determinados. La rentabilidad es el concepto genérico que relaciona los sacrificios y los beneficios en una medida cuantitativa, con el objetivo de resumir en una sola cantidad lo deseable de una inversión concreta. En su forma más simple y genérica, la \textbf{rentabilidad ex-post} resulta de dividir el beneficio presente entre el sacrificio pasado. Es una medida ex-post de la rentabilidad porque ambos valores se conocen sin incertidumbre alguna. En la práctica, sin embargo, apenas existen inversiones cuya ganancia futura se conozca con certeza absoluta, y conocer el beneficio presente dado un sacrificio pasado sirve para valorar el resultado de una inversión pasada pero no sirve para valorar la oportunidad de nuevas inversiones. En estos casos, la rentabilidad de una inversión puede caracterizarse como el cociente entre la esperanza matemática de la ganancia y el sacrificio presente. Hablamos en estos casos de \textbf{rentabilidad ex-ante}. Por supuesto, cuando las ganancias y los sacrificios se producen en más de dos periodos, la cuantificación de la rentabilidad se hace más compleja, pero se basa en el mismo concepto: relacionar sacrificio y ganancia en una variable que permita comparar con otras inversiones. Cabe señalar que cuando los agentes se comportan de manera racional y aprovechan toda la información de la que disponen, en un mercado sin imperfecciones, el sacrificio requerido presente requerido para llevar a cabo la inversión tenderá a ajustarse de tal manera que la rentabilidad ex-ante de una inversión sea igual a otras con características similares. Así, cabe preguntarse: ¿\textbf{qué factores determinan la rentabilidad} de una inversión? Los principales son la aversión al riesgo de los inversores, la preferencia que éstos tengan por la liquidez, la tasa de inflación, los impuestos que graven la rentabilidad o las ganancias, y el ruido estadístico que generan en la práctica los sesgos cognitivos de los inversores y las asimetrías de información. La aversión al riesgo resulta de la preferencia por ganancias ciertas frente a ganancias inciertas en la mayoría de los agentes económicos. Así, los inversores aversos al riesgo demandan una prima por un mayor riesgo asumido que se traduce en mayores rentabilidades de la inversión. Este factor es el principal determinante de la rentabilidad. La preferencia por la liquidez hace referencia al hecho de que una inversión cuya ganancia se obtiene en un momento más distante en el futuro exige del inversor un grado de abstinencia adicional y e impone un coste de oportunidad por no poder utilizar el capital invertido en otras inversiones potencialmente más rentables. La inflación reduce el valor real de la ganancia cuando ésta toma forma de ganancia monetaria. Dado que afecta por igual a todas las inversiones cuya ganancia se obtiene en la misma moneda, no es generalmente un factor a considerar en la valoración de proyectos de inversión. 

Dado que el \textbf{riesgo} es el principal determinante de la rentabilidad, cabe examinarlo con más detalle. Como se ha dicho anteriormente, las ganancias a percibir en el futuro raramente son conocidas con certeza. En la práctica, las ganancias percibidas son realizaciones de distribuciones de probabilidad, cuya forma concreta afecta a la preferencia de los inversores por unas inversiones u otras. La varianza es una medida de la dispersión de las ganancias en las diferentes realizaciones posibles. Se calcula como la suma de las desviaciones respecto de la media al cuadrado. La desviación estándar es la raíz cuadrada de la varianza. Ambas medidas, junto con la esperanza estadística de las ganancias, forman la base del análisis de la rentabilidad de las inversiones desde el punto de vista financiero. Si el riesgo puede entenderse de forma abstracta como la posibilidad de obtener diferentes ganancias a partir de una misma inversión, en la práctica existen muy variadas causas de esta incertidumbre. El riesgo industrial es una categoría muy amplia que agrupa todos aquellos riesgos relacionados con el ciclo operativo de la empresa que lleva a cabo una inversión. El riesgo de liquidez concierne la posibilidad de que el inversor no pueda liquidar su inversión de forma satisfactoria antes de lo esperado inicialmente. El riesgo de tipo de cambio surge de la volatilidad de las ganancias cuando fluctúa el precio de las divisas. El riesgo de interés hace referencia a los cambios en la rentabilidad predominante para inversiones similares que pueden variar el atractivo de una inversión ya comprometida en términos relativos. El riesgo de solvencia se refiere a la posibilidad de que alguna de las partes implicadas en la generación de la ganancia no pueda honrar sus compromisos y ello reduzca el beneficio. Otros riesgos tales como el riesgo de inflación, los riesgos sistémicos, el riesgo político o los riesgos de catástrofes naturales son también muy relevantes a la hora de valorar la posible dispersión de las ganancias a obtener. Cuando un inversor puede \textit{diversificar} su inversión tiene en cuenta el \underline{riesgo sistemático} y obvia el \underline{riesgo específico} de la inversión a la hora de valorar una inversión. El riesgo sistemático es aquel que una inversión tiene en común con todos los demás inversiones en su misma categoría, mercado, región geográfica, etc... El riesgo específico es aquel exclusivo a la empresa y que no está relacionado con otras inversiones. En la medida en que un capital sea invertido de forma suficientemente diversificada, el peso del riesgo específico en una cartera tenderá a desaparecer y por ello, cada inversión será relevante en la medida en que aumente o disminuya la sensibilidad a factores de riesgo comunes al conjunto de las inversiones. 

El \textbf{coste del capital} es una magnitud de carácter financiero --no contable- que caracteriza el rendimiento que los inversores exigen por invertir en un activo determinado. Cuando se evalúan proyectos de inversión, el coste del capital se utiliza como tasa de descuento, de tal manera que la cuantía de las ganancias y los desembolsos se corrige de acuerdo con la tasa de descuento y la distancia en el tiempo en la que se producirá los ingresos o los desembolsos. La estimación del coste del capital asociado a una inversión se puede llevar a cabo de \underline{forma directa}, a partir de modelos de valoración de activos tales como el modelo CAPM o el modelo APT. Estos modelos predicen el valor de inversiones a partir de otras magnitudes tales como la $\beta$ o sensibilidad del precio del activo respecto del mercado, en el caso del CAPM, y a partir de ese valor predicho se estima un coste del capital para la inversión en cuestión. La ventaja de éstos métodos a la hora de calcular el coste del capital de una empresa en su conjunto es que permiten abstraerse de la estructura del capital, calculando directamente el coste del capital del conjunto de los activos, de tal manera que cambios en la estructura del capital no introduzcan sesgos en la estimación del coste del capital. Sin embargo, requieren de supuestos muy fuertes y de difícil cumplimiento en la práctica. Además, el cálculo de las betas no es sino una estimación de muy dificil cálculo preciso. El método de \underline{cálculo indirecto} de estimación del coste del capital utilizado por una empresa consiste en valorar por separado la deuda y el equity, y calcular posteriormente una media ponderada de ambos denominada \textit{Weighted Average Cost of Capital} (WACC). Modigliani y Miller (1958) demostraron que en condiciones de ausencia de fricciones tal como costes de insolvencia o transacción y mercados  financieros perfectos, el WACC no depende de la estructura financiera de la empresa. Otra forma de estimar el coste del capital de una inversión es el llamado \underline{cálculo implícito} a partir del precio de mercado de la inversión y la secuencia de flujos de caja. Aunque en teoría este método debería ser muy preciso y sencillo, en la práctica tiene poca utilidad para la mayoría de las inversiones porque es muy difícil o imposible (por inexistente) conocer la secuencia de flujos de caja que los agentes participantes en la formación de precios estiman de forma consensuada. 

Los inversores y por ende, los gestores de las empresas, tienen por objetivo fundamental crear valor a partir de sus decisiones de inversión y para ello, aplican diferentes métodos de \marcar{valoración de proyectos}. Algunos \textbf{principios fundamentales de valoración} son aplicables a todos los proyectos de inversión. Se deben tener en cuenta los \underline{flujos financieros y no los flujos contables}, pues estos últimos no tienen en cuenta la importancia del capital circulante o las necesidades operativas de fondos e introducen aspectos no relevantes a la valoración como correcciones por amortización o depreciación. Los \underline{flujos considerados} en la valoración de proyectos deben ser aquellos cuya realización depende de la decisión de inversión, de tal manera que los costes hundidos no son relevantes. Debe razonarse en todo momento en \underline{términos de oportunidad}, considerando lo que aporta una decisión en relación a las otras alternativas posibles y no como inversión por separado. Así, por ejemplo, si posee la propiedad de un terreno y se está valorando construir un hotel, debe compararse con la posibilidad de construir un garaje, alquilar o vender el terreno, o no tomar acción alguna en previsión de un aumento del valor y una venta posterior. Los \underline{costes de financiar un proyecto deben valorarse por separado a los desembolsos propios a la inversión}. Los flujos de financiación son relevantes para valorar el coste de la financiación pero no para estimar la capacidad de una inversión para crear valor para los aportadores del capital. Sin embargo, si se está tratando de valorar la participación en un proyecto como aportador de equity, es necesario tener en cuenta que mayores cantidades de deuda aumentarán el riesgo de la inversión y por ello, el coste del capital que se utiliza como tasa de descuento. Por último es necesario tener en cuenta el \underline{tratamiento fiscal} de las ganancias y la rentabilidad, así como ser constante en todo momento en cuanto a los criterios utilizados para cuantificar los flujos, como por ejemplo la divisa utilizada o los supuestos aplicados. En la práctica, la valoración de proyectos consiste en la estimación de una magnitud a partir de la información disponible sobre cada proyecto, de tal manera que sea posible su comparación inequívoca. Los métodos más utilizados son el Valor Actual Neto, la Tasa Interna de Rentabilidad y el Payback period, así como la simple intuición. Graham y Harvey (2001) analiza las diferencias en los \underline{métodos aplicados} en función del tamaño de la empresa y la formación de los managers y encuentra que el método del VAN es el más habitual en managers con MBAs, que los managers de mayor edad tienden a utilizar más el método del payback period, y que las PYMES utilizan sobre todo la intuición.

El método del \textbf{Valor Actual Neto} consiste en descontar los desembolsos e ingresos de un proyecto de inversión a una tasa de descuento subjetiva que corresponde con el coste del capital estimado, y sumarlos para obtener el VAN propiamente. En la medida en que el VAN de un proyecto de inversión sea positivo, estará generando valor para el inversor y destruyéndolo si es negativo. Un inversor que no estuviese sujeto a límites respecto al capital que puede utilizar debería de esta forma llevar a cabo todos los proyectos para los que estimase un VAN positivo. Sin embargo, el capital disponible para cualquier inversor es siempre limitado y por ello resulta necesario decidir entre inversiones. De acuerdo con el criterio del VAN, entre dos inversiones con VAN positivo, deberán llevarse a cabo aquellas con un valor mayor. 

El método de la \textbf{Tasa Interna de Rentabilidad} consiste en llevar a cabo aquellas inversiones cuya TIR sea superior a una tasa de rentabilidad dada. La Tasa de Rentabilidad Interna es la otra cara de la moneda del Valor Actual Neto y se calcula como la tasa de descuento que iguala a 0 el Valor Actual Neto del proyecto de inversión. A la hora de decidir entre dos proyectos de inversión cuando el capital es limitado, el criterio del TIR propone llevar a cabo el proyecto con un valor mayor. Cabe preguntarse si los criterios del VAN y el TIR deciden siempre de la misma forma entre dos proyectos dados. Si dos proyectos implican el mismo desembolso y la duración de la secuencia de flujos es la misma, ambos métodos son perfectamente equivalentes. Sin embargo, cuando ésto no sucede, pueden proponer decisiones diferentes. Los llamados \textit{puntos de Fisher} son aquellas tasas de rentabilidad para las cuales se prefiere un proyecto A a uno B cuando la tasa es mayor al punto de Fisher, y viceversa cuando la tasa de rentabilidad es menor. Además, determinados proyectos de inversión con desembolsos en varios periodos o desembolsos posteriores a las ganancias implican la existencia de múltiples valores del TIR o incluso inexistencia de una tasa de descuento que iguale el VAN a 0. Por ello, el VAN se considera el único método plenamente consistente y conceptualmente correcto. El TIR es un criterio atractivo para las empresas porque permite repartir tareas en la toma decisiones de inversión, a pesar de sus problemas. Los directivos de alto nivel fijan el nivel mínimo de rentabilidad y los empleados de nivel más bajo estiman los flujos de caja, de tal manera que se llevan a cabo los proyectos cuyo TIR está por encima. La \textbf{TIR Modificada (MIRR)} ofrece una alternativa a la TIR que solventa la posible multiplicidad de resultados y la incompatibilidad con el criterio del VAN. La TIRM se calcula sumando todos los flujos positivos actualizados al coste del capital hasta el periodo final y hallando la tasa de descuento que aplicada a esa suma iguala su valor con el desembolso inicial. Se lleva a cabo el proyecto con mayor MIRR.

\textbf{Otros criterios de inversión}, algunos ya mencionados anteriormente, son el \underline{payback period} o plazo de recuperación, el rendimiento contable o ROCE o el índice del valor actual neto. El criterio del plazo de recuperación de la inversión consiste simplemente en dividir la inversión entre el flujo de caja periódico, obteniendo los periodos necesarios para recuperar la inversión inicial. Las limitaciones de este criterio son evidentes: no tiene en cuenta el coste de oportunidad de la inversión, no tiene en cuenta la variabilidad en los flujos ni los ingresos obtenibles una vez recuperado el capital aportado. El rendimiento contable o ROCE representa la relación entre el ingreso operativo después de impuestos (EBITDA menos impuestos) y el activo contable medio utilizado medio en un periodo determinado. Los ajustes por amortización y depreciación reducen el valor del activo fijo, lo que introduce un sesgo creciente al alza sobre el ROCE. Además, el método de amortización aplicado puede también cambiar el ROCE medio a lo largo de un intervalo de tiempo. El \underline{Índice del Valor Actual Neto} es especialmente útil cuando el horizonte temporal de inversión no es relevante, los gestores tienen una restricción de capital y existen múltiples alternativas de inversión que requerirían un desembolso total más alto que la restricción en caso de llevarse todos a cabo. El IVAN se calcula como el cociente entre el VAN de los flujos positivos y el VAN de los flujos negativos. Así, el IVAN no expresa sino la cantidad de riqueza actual obtenible por cada unidad desembolsada. El objetivo del manager es hallar la combinación de proyectos con mayor IVAN medio ponderado, de tal manera que se extraiga la máxima cantidad de riqueza por unidad de capital utilizado.

La decisión de inversión no consiste únicamente en la aplicación de un criterio a una secuencia de flujos de caja a percibir o desembolsar. Concierne también la \textbf{estimación de esos flujos de caja} futuros cuyo valor concreto está generalmente sujeto a incertidumbre. Diferentes métodos de estimación tratan de sistematizar el proceso para hacer un uso eficiente de la información de que se dispone. El método más general consiste en la comprensión del modelo de negocio de la empresa y posteriormente la \underline{formulación de diferentes escenarios} futuros en función de diferentes supuestos. Por ejemplo, es habitual construir escenarios optimista, realista y pesimista y posteriormente realizar un análisis de sensibilidad para valorar la robustez de los escenarios a cambios en los supuestos. El \underline{método de Montecarlo} muestra un grado de sofisticación adicional y es utilizado menos habitualmente, aunque puede ser de gran utilidad en determinados contextos. Consiste en formular un modelo matemático que relacione los flujos de caja con los valores de una serie de variables aleatorias. Posteriormente, se simulan realizaciones de las variables aleatorias y y con ellas, se obtiene una estimación de la distribución de probabilidad de las ganancias. Los métodos de valoración de \underline{opciones reales y los árboles de decisión} permiten cuantificar el valor que tienen las diferentes alternativas una vez iniciado el proyecto de inversión. No es igual de valioso un proyecto cuyo desembolso inicial no puede recuperarse bajo ningún concepto, que un proyecto que permite su liquidación y recuperación de parte del desembolso inicial una vez que se conoce determinada información que apunta a una evolución desfavorable. 

El \marcar{activo fijo y el activo corriente} tienen algunas diferencias que son relevantes a la hora de gestionar la inversión en uno u otro y que permiten al manager optimizar el rendimiento. En el caso del \textbf{activo fijo}, la vida útil y la vida óptima son dos conceptos centrales. El primero es un concepto de carácter técnico que hace referencia al periodo temporal en el que el activo fijo continúa funcionado de forma normal. La vida óptima es el periodo de utilización de un activo fijo que maximiza el valor actual neto. En contextos en los que el remplazo del activo fijo tiene idénticas características y no hay obsolescencia técnica, el problema para el manager es básicamente una maximización del VAN respecto del periodo hasta el remplazo. Cuando existe obsolescencia del activo fijo, la decisión de remplazo se convierte en un trade-off entre el coste del remplazo y el coste de oportunidad derivado de la posibilidad de utilizar un activo fijo con cualidades técnicas superiores. Los \underline{activos inmobiliarios} son una masa de activo fijo de especial importancia en la mayoría de las empresas. Los criterios de inversión en \textit{activo fijo inmobiliario} dependen en gran medida de la empresa concreta. Empresas de reciente creación y gran crecimiento potencial tienden a preferir el alquiler o el lease por permitir mayor escalabilidad y liberar capital para otros fines. Empresas consolidadas que esperan aumentos de la renta inmobiliaria pueden preferir comprar los activos inmobiliarios para ahorrar incrementos de renta y obtener ganancias ligadas al ciclo de inversión. Empresas que actúan en mercados maduros suelen contar con patrimonio inmobiliario propio pero pueden preferir vender y arrendar para financiar entrada en nuevos mercados. Para las empresas cuyo mercado es precisamente el inmobiliario, las propiedades inmobiliarias objeto de su ciclo de explotación no son propiamente activo fijo aunque a nivel contable puedan en ocasiones considerarse como tales. La inversión en \textbf{activo corriente} es, en términos abstractos y generales, un problema de gestión de inventario cuya caracterización teórica parte de los modelos S-s. Estos modelos caracterizan la decisión como resultado de ponderar dos costes: el coste de oportunidad por mantener una inversión en capital circulante o stock de inventario, y un coste por ajuste del capital circulante o de reposición del inventario. La llamada fórmula de Wilson o de Cantidad Económica de Pedido caracteriza la cantidad óptima de capital circulante a reponer en un periodo dado. Por supuesto, la gestión de inventarios en la práctica se complica con numerosos factores adicionales, pero el problema sigue teniendo los mismos rasgos básicos.

A lo largo de la exposición se ha analizado el problema de decisión de inversión tanto de forma general como en el seno de la empresa, atiendo a los conceptos básicos como rentabilidad, riesgo y coste del capital, examinando los criterios de decisión más habituales y valorando algunas particularidades de la inversión en activos concretos. La capacidad de los managers para tomar decisiones correctas de inversión en el día a día de las empresa tiene consecuencias que van mucho más allá de la propia entidad. Las economías modernas son el resultado de millones de decisiones de inversión. Cuando las empresas cuentan con mejores procesos de decisión, managers más formados y mayor información disponible para valorar lo apropiado de llevar a cabo una inversión, el crecimiento agregado de la economía, la competitividad exterior y por todo ello, el bienestar de los ciudadanos se ve positivamente afectado. Este hecho interpela al policy-maker y al sector público en su conjunto, subrayando la necesidad de fomentar la acumulación de capital humano y social de los gestores empresariales.


\seccion{Preguntas clave}

\begin{itemize} 
    \item ¿Qué es la inversión?
    \item ¿Por qué invierten las empresas?
    \item ¿Qué es la rentabilidad?
    \item ¿Qué es el riesgo de una inversión?
    \item ¿Qué es el coste del capital?
    \item ¿Cómo invierten las empresas?
    \item ¿En qué invierten las empresas?
    \item ¿Qué criterios se utilizan para decidir entre inversiones?
\end{itemize}


\esquemacorto

\begin{esquema}[enumerate]
	\1[] \marcar{Introducción 2-2}
		\2 Contextualización
			\3 Razón de ser de la empresa
			\3 Creación de valor
			\3 Inversión
		\2 Objeto
			\3 Qué es la inversión
			\3 Qué es el riesgo
			\3 Qué es la rentabilidad
			\3 Qué es el coste del capital
			\3 Qué criterios existen para decidir entre inversiones
			\3 En qué invierten las empresas
			\3 Particularidades de la inversión en activo fijo y corriente
		\2 Estructura
			\3 Rentabilidad, riesgo y coste del capital
			\3 Valoración de proyectos de inversión
			\3 Inversión en activo fijo y corriente
	\1 \marcar{Rentabilidad, riesgo y coste del capital 10-12}
		\2 Definición de inversión
			\3 Sacrificio y beneficio
			\3 Aumento del capital
			\3 Flujos de caja
		\2 Rentabilidad ex-post
			\3 Idea clave
			\3 Formulación genérica
			\3 Variantes
		\2 Rentabilidad ex-ante
			\3 Idea clave
			\3 Formulación
			\3 Mercado eficiente
		\2 Determinantes de la rentabilidad
			\3 Aversión al riesgo
			\3 Preferencia por liquidez
			\3 Inflación
			\3 Impuestos
			\3 Ruido
		\2 Riesgo
			\3 Idea clave
			\3 Medición del riesgo
			\3 Fuentes de riesgo
			\3 Riesgo específico
			\3 Riesgo sistemático
			\3 Diversificación
		\2 Coste del capital
			\3 Idea clave
			\3 Estimación del coste del capital
			\3 Coste del capital en la práctica
	\1 \marcar{Valoración y decisión de inversión 12-24}
		\2 Principios fundamentales de valoración
			\3 Tener en cuenta flujos financieros
			\3 Flujos añadidos
			\3 Razonar en términos de oportunidad
			\3 Rentabilidad y coste de financiación por separado
			\3 Impuestos son relevantes
			\3 Consistencia
			\3 Métodos utilizados en la práctica
		\2 Valor Actual Neto
			\3 Idea clave
			\3 Formulación:
			\3 Criterio de decisión
		\2 Tasa Interna de Rentabilidad (TIR)
			\3 Idea clave
			\3 Formulación
			\3 Criterio de decisión
		\2 TIR modificada
			\3 Idea clave
			\3 Formulación
			\3 Criterio de decisión
		\2 Otros métodos
			\3 Plazo de recuperación o payback
			\3 Rendimiento contable (ROCE)
			\3 Índice del valor actual neto
		\2 Estimación de flujos de caja
			\3 Idea clave
			\3 Planes de negocio y escenarios
			\3 Montecarlo
			\3 Flexibilidad, opciones reales y árboles de decisión
	\1 \marcar{Inversión en activo fijo y corriente 24-27}
		\2 Activo fijo
			\3 Maquinaria
			\3 Activos inmobiliarios
		\2 Activo corriente
			\3 Idea clave
			\3 Minimización de costes totales
			\3 Extensible
	\1[] \marcar{Conclusión 27-30}
		\2 Recapitulación
			\3 Análisis general de las decisiones de inversión
			\3 Determinantes de la inversión
			\3 Elección entre inversiones alternativas
			\3 Inversión en activo fijo y corriente
		\2 Idea final
			\3 Inversión
			\3 Economía
			\3 Agregación

\end{esquema}

\esquemalargo













\begin{esquemal}
	\1[] \marcar{Introducción 2-2}
		\2 Contextualización
			\3 Razón de ser de la empresa
				\4 Múltiples concepciones de empresa
				\4[] Empresa como como conjunto de inputs
				\4[] Empresa como conjunto de contratos
				\4[] Empresa como agente social
				\4[] ...
				\4 Por qué existe
				\4[] Producir output
				\4[] Vender output
				\4[] Utilizar factores productivos
				\4[] Crear valor para el accionista
			\3 Creación de valor
				\4 Objetivo último de la empresa
				\4 Inversión:
				\4[] Herramienta de creación de valor
				\4[] Caracteriza actividad principal de empresas
			\3 Inversión
				\4 Organizar recursos disponibles
				\4[] Para generar valor económico
				\4[] $\to$ Aumentar patrimonio de accionistas
				\4 Decisión principal de gestores
				\4 Consecuencias macro y micro
		\2 Objeto
			\3 Qué es la inversión
			\3 Qué es el riesgo
			\3 Qué es la rentabilidad
			\3 Qué es el coste del capital
			\3 Qué criterios existen para decidir entre inversiones
			\3 En qué invierten las empresas
			\3 Particularidades de la inversión en activo fijo y corriente
		\2 Estructura
			\3 Rentabilidad, riesgo y coste del capital
			\3 Valoración de proyectos de inversión
			\3 Inversión en activo fijo y corriente
	\1 \marcar{Rentabilidad, riesgo y coste del capital 10-12}
		\2 Definición de inversión
			\3 Sacrificio y beneficio
				\4 Renuncia a una cantidad de riqueza
				\4 Previsión de un ingreso futuro
			\3 Aumento del capital
				\4 Por la diferencia entre:
				\4[] Sacrificio y beneficio
				\4 En términos de empresa:
				\4[] Aumento del patrimonio neto
			\3 Flujos de caja
				\4 Cuantifican sacrificios y beneficios
				\4[] $\to$ Cuantías positivas y negativas
				\4[] $\to$ Caracterizan una inversión determinada
		\2 Rentabilidad ex-post
			\3 Idea clave
				\4 Concepto genérico
				\4 Relación entre sacrificio y beneficio
				\4[] Medida de aumento del patrimonio
				\4[] $\to$ Derivado de inversión
			\3 Formulación genérica
				\4 $R_t \equiv \frac{P_{t}}{P_{t-1}}$
				\4[] Donde $P_{t}$ y $P_{t-1}$
				\4[] $\to$ Precios de inversiones en $t$ y $t-1$
			\3 Variantes
				\4 Múltiples
				\4[] ¿Se conoce beneficio futuro?
				\4[] ¿Cómo definimos beneficio?
				\4[] $\to$ Contable
				\4[] $\to$ Económico
				\4[] ¿Qué sacrificio tenemos en cuenta?
				\4[] $\to$ Total: rentabilidad económica
				\4[] $\to$ Fondos propios aportados: r. financiera
				\4[] $\to$ ...
		\2 Rentabilidad ex-ante
			\3 Idea clave
				\4 Medida de rentabilidad
				\4[] $\to$ Antes de obtener beneficio
				\4[] $\then$ Beneficio no se conoce con certeza
				\4[] $\then$ Sujeto a incertidumbre
			\3 Formulación
				\4 $E(R_t) \equiv \frac{E(P_{t+1})}{P_{t}}$
				\4[$\to$] Suponiendo un sólo periodo
				\4[$\to$] Suponiendo $P_{t+1}$ es todo beneficio
			\3 Mercado eficiente
				\4 $P_t$ se ajusta para variar rentabilidad
				\4[] Hasta que se iguale a inversiones similares
				\4[$\then$] Rentabilidad refleja toda información disponible
		\2 Determinantes de la rentabilidad
			\3 Aversión al riesgo
				\4 Beneficio futuro sujeto a incertidumbre
				\4[] $\to$ Rentabilidad a obtener es arriesgada
				\4 En general, agentes muestran aversión al riesgo
				\4[] Exigen prima por riesgo
				\4[] $\to$ Inversiones más arriesgadas son más rentables
				\4 Principal determinante de rentabilidad
			\3 Preferencia por liquidez
				\4 Sacrificio durante periodo más largo tiene costes
				\4[] Agentes pueden necesitar fondos
				\4[] Coste de oportunidad de fondos invertidos
				\4[] $\to$ Exigen rentabilidad por esperar
			\3 Inflación
				\4 Aumento generalizado de los precios
				\4[] $\to$ Reducción del valor real del beneficio
				\4 Inflación más elevada
				\4[] $\then$ Exigencia de rentabilidad más alta
				\4 Afecta a todas las inversiones
				\4[] $\to$ No es relevante para comparar si = divisa
			\3 Impuestos
				\4 Diferentes tipos impositivos
				\4[] Para diferentes inversiones
			\3 Ruido
				\4 Sesgos cognitivos + problemas de información
				\4[] Rentabilidad no refleja información
		\2 Riesgo
			\3 Idea clave
				\4 Inversores no conocen con certeza beneficio
				\4[] $\to$ Rentabilidad queda sujeta a incertidumbre
				\4 Diferentes factores determinan rentabilidad
				\4 Caracterizar riesgo de una inversión
				\4[] $\to$ Describir distribución de beneficios-sacrificios
				\4 Medidas del riesgo
				\4[] Variables que describen segundo momento de distribución
			\3 Medición del riesgo
				\4 Varianza
				\4[] Suma de las desviaciones al cuadrado frente a media
				\4[] $\sigma(r)^2 = \sum_{i} p_i \cdot (r_i - \bar{r})^2$
				\4 Desviación estándar
				\4[] Raíz cuadrada de varianza
				\4 Otros momentos de dist. de prob. de inversión
			\3 Fuentes de riesgo
				\4 Agrupables en muy variadas categorías
				\4 Riesgo industrial
				\4[] Categoría muy amplia
				\4[] Riesgos relacionados con actividad de inversión
				\4[] Afecta a capacidad para generar flujos positivos
				\4 Riesgo de liquidez
				\4[] Posibilidad de no poder liquidar inversión
				\4 Solvencia
				\4[] Contrapartidas de la actividad de inversión
				\4[] $\to$ No pueden hacer frente a sus obligaciones
				\4 Riesgo de tipo de cambio
				\4[] Flujos de caja en divisas diferentes
				\4[] $\to$ Posibles variaciones de valor del flujo
				\4 Riesgo de tipo de interés
				\4[] Variaciones del tipo de interés
				\4[] $\to$ Cambian coste de oportunidad de inversión
				\4[] $\then$ Afectan al valor de la inversión
				\4 Riesgo de inflación
				\4 Riesgo sistémico
				\4 Riesgo político
				\4 Riesgo de catástrofe natural
			\3 Riesgo específico
				\4 Riesgo idiosincrático a una inversión determinada
				\4 Riesgos que afectan exclusivamente a la empresa
			\3 Riesgo sistemático
				\4 Riesgo que afecta a inversiones en un mercado
				\4[] P.ej.:
				\4[] $\to$ riesgo sistemático de empresas españolas
			\3 Diversificación
				\4 Mediante diferentes instrumentos
				\4[] Posible diversificar fondos
				\4 Diversificación permite eliminar riesgos específicos
				\4[] Un sólo proyecto de inversión
				\4[] $\to$ Exposición a riesgo sistemático + específico
				\4[] Suficiente diversificación
				\4[] $\to$ Exposición a riesgo sistemático
				\4[] $\to$ Realizaciones específicas tienden a compensarse
		\2 Coste del capital
			\3 Idea clave
				\4 Rendimiento exigido por inversores
				\4[] A cambio de invertir en una serie de activos
				\4[] $\to$ Coste de oportunidad de inversión similar
				\4 Utilizado como tasa de descuento
				\4[] Para evaluar proyectos de inversión
				\4 Valoración de inversiones
				\4[] Implica descuento de flujos
				\4[] $\to$ ¿A qué tasa descontar?
				\4[] $\to$ ¿Qué rentabilidad generan inversiones similares?
				\4[] $\then$ Coste del capital
			\3 Estimación del coste del capital
				\4 A tener en cuenta
				\4[] Coste del capital es una variable financiera
				\4[] $\to$ No contable
				\4[] $\then$ Debe utilizarse información financiera
				\4[] Coste del capital es subjetivo
				\4[] $\to$ Cada agente toma un coste del capital
				\4[] $\then$ Coste de capital ``de mercado'' es agregación
				\4 Cálculo directo
				\4[] A partir de modelos de valoración de activos
				\4[] P.ej.: CAPM
				\4[] $\to$ ¿Rentabilidad exigida a conjunto de activos?
				\4[] ¿Qué $\beta$ tienen activos y deuda+equity?
				\4[] $\to$ Posible derivar relación entre betas
				\4[] $\to$ $\beta_A = \beta_E \cdot \frac{V_E}{V_D+V_E} + \beta_D \cdot \frac{V_D}{V_D+V_E}$
				\4[] $\to$ Asumiendo $\beta_D=0$ y escudo fiscal:
				\4[] \quad $\beta_A = \frac{\beta_E}{1+(1-T_C) \cdot \frac{V_D}{V_E}}$
				\4[] Beta de activos menor o igual a beta de equity
				\4[] Ventaja:
				\4[] $\to$ Estructura del capital no induce a error
				\4[] $\to$ Cambios en estructura se reflejan en valoración
				\4[] Desventaja:
				\4[] $\to$ Implica supuestos fuertes ligados a modelos
				\4[] $\to$ Cálculo de betas es poco fiable
				\4 Cálculo indirecto
				\4[] Cálculo por separado de valores de la empresa
				\4[] $\to$ Deuda
				\4[] $\to$ Equity
				\4[] Coste de financiación via deuda
				\4[] $\to$ Coste de refinanciar en condiciones actuales
				\4[] Coste de financiación via equity
				\4[] $\to$ Modelo de valoración
				\4[] $\then$ Ponderación de costes del capital
				\4[] \fbox{$\text{WACC} = k = k_E \cdot \frac{V_E}{V_E + V_D} + k_D \cdot (1-t) \cdot \frac{V_D}{V_E + V_D}$}
				\4[] Modigliani-Miller:
				\4[] $\to$ WACC no depende de estructura financiera
				\4[] $\to$ Dados determinados supuestos
				\4 Cálculo implícito
				\4[] A partir de:
				\4[] $\to$ Valor de mercado de la empresa
				\4[] $\to$ Flujos de caja libres a recibir
				\4[] $V = V_E + V_D = \sum_{t=0}^\infty \frac{\text{FCF}_t}{(1+k)^t}$
				\4[] Dado V y estimados FCF
				\4[] $\to$ Hallar $k$
				\4[] Problema:
				\4[] $\to$ ¿Qué FCF estima el mercado?
				\4[] $\to$ ¿Hay consenso en el mercado?
			\3 Coste del capital en la práctica
				\4 Diferencia entre costes de k de empresa y proyecto
				\4[] Tener presente que no tienen porqué ser iguales
				\4[] Coste de capital de proyecto
				\4[] $\to$ Depende de riesgo del proyecto en cuestión
				\4[] Coste de capital de la empresa
				\4[] $\to$ Depende de riesgo de conjunto de empresa
				\4 Reducción del coste del capital de empresa
				\4[] Empresa no puede reducir coste usando deuda
				\4[] $\to$ Aumenta el riesgo para accionistas
				\4[] $\then$ Aumenta coste del equity
				\4 Valoración de empresas
				\4[] Para valorar empresa es necesario conocer $k$
				\4[] Para conocer $k$ es necesario conocer valor de empresa
				\4[] Posibles soluciones
				\4[] $\to$ Asumir estructura de capital objetivo
				\4[] $\to$ Estimación iterativa aproximada
				\4[] $\to$ Utilizar método directo
	\1 \marcar{Valoración y decisión de inversión 12-24}
		\2 Principios fundamentales de valoración
			\3 Tener en cuenta flujos financieros
				\4 No flujos contables
				\4 Contables no tienen en cuenta working capital
				\4 Contables incluyen depreciación y amortización
			\3 Flujos añadidos
				\4 Los costes hundidos no se consideran
				\4 Se deben considerar las contribuciones marginales
				\4[] $\to$ Acciones que implicarán cambios en flujos futuros
				\4[] $\to$ Exclusivamente
				\4[] Absurdo continuar proyecto por costes hundidos
				\4[] $\to$ El único objetivo es añadir valor
			\3 Razonar en términos de oportunidad
				\4 ¿Qué aporta un proyecto respecto alternativa?
				\4 Ejemplo:
				\4[] Valorar proyecto de inversión
				\4[] $\to$ Ejecutable en propiedad inmobiliaria
				\4[] Empresa ya posee la propiedad
				\4[] Valoración de inversión implica comparar:
				\4[] $\to$ Con posibilidad de alquilar propiedad
				\4[] $\to$ Con posibilidad de vender propiedad
			\3 Rentabilidad y coste de financiación por separado
				\4 Sólo flujos operativos y de inversión
				\4 Flujos de financiación
				\4[] $\to$ Relevantes para valorar coste de financiación
				\4[] $\to$ No para rentabilidad del proyecto
				\4[] Si aumenta deuda
				\4[] $\to$ Aumentan flujos de financiación
				\4[] $\then$ Cambiará tasa de descuento
				\4[] $\then$ Necesario tener en cuenta
			\3 Impuestos son relevantes
				\4 Si se valoran flujos antes de impuestos
				\4[] Necesario descontarlos a $k$ antes de impuestos
			\3 Consistencia
				\4 Mantener criterios de forma constante
				\4[$\to$] Impuestos
				\4[$\to$] Divisas
				\4[$\to$] Inflación
				\4[$\to$] ...
			\3 Métodos utilizados en la práctica
				\4 Más utilizado VAN
				\4 TIR: menos realista, asume reinversión
				\4 Payback period: empresas pequeñas
				\4 Graham \& Harvey (2001)
				\4[] MBAs utilizan VAN
				\4[] Managers más viejos utilizan pay-back
				\4[] PYMES utilizan sobre todo intuición
		\2 Valor Actual Neto
			\3 Idea clave
				\4 Suma de corriente de flujos de caja
				\4[$\to$] Descontados a rentabilidad exigida
			\3 Formulación:
				\4[] $\text{VAN} = \sum_i^n \frac{\text{FC}_i}{(1+r)^i}$
			\3 Criterio de decisión
				\4 Si decisión entre llevar a cabo o no un proyecto:
				\4[] Se lleva a cabo si $\text{VAN} > 0$
				\4[] $\then$ proyectos se llevan a cabo si crean valor
				\4 Si decisión entre dos proyectos:
				\4[] Se lleva a cabo proyecto con mayor VAN
				\4 Puntos de Fisher:
				\4[] Tasas de descuento para las que:
				\4[] Si tasa de descuento > PFisher:
				\4[] $\to$ preferible proyecto A
				\4[] Si tasa de descuento < PFisher:
				\4[] $\to$ preferible proyecto B
		\2 Tasa Interna de Rentabilidad (TIR)
			\3 Idea clave
				\4 Tasa de descuento $\rho$
				\4[] $\to$ que iguala a 0 el VAN del proyecto
				\4 Asume que flujos pueden reinvertirse a tasa $\rho$
				\4 Posibles inconsistencias como criterio de decisión
				\4[] $\to$ No siempre coincidente con VAN
				\4[] $\to$ Múltiples valores posibles
			\3 Formulación
				\4 $\rho \in \mathbb{R}: 0 = \sum_i^n \frac{\text{FC}_i}{(1+\rho)^i}$
			\3 Criterio de decisión
				\4 Si decisión entre llevar a cabo o no un proyecto:
				\4 Si $\rho > k$ proyecto debe ser realizado
				\4[] Si TIR > coste de capital, debe realizarse
				\4[] $\to$ Si TIR/tasa de descuento necesaria para $\text{VAN}=0$
				\4[] $\to$ es mayor que rentabilidad exigida
				\4[] $\then$ Se lleva a cabo
				\4[] $\then$ VAN será positivo
				\4 Si decisión entre dos proyectos:
				\4[] $\to$ Se lleva a cabo proyecto con mayor TIR
				\4 Posibles inconsistencias con VAN
				\4[] ¿TIR y VAN ordenan proyectos de igual forma?
				\4[] Sí, si:
				\4[] $\to$ mismo desembolso inicial
				\4[] $\to$ misma duración de secuencia de flujos
				\4[] No necesariamente en resto de casos
				\4[] Ej.:
				\4[] $\to$ $\text{VAN}_A > \text{VAN}_B$
				\4[] $\to$ $\text{TIR}_B > \text{TIR}_A$
				\4[] $\then$ Incoherencia de criterios
				\4[] Ej.:
				\4[] $\to$ Proyecto A tiene múltiples TIR positivas
				\4[] $\to$ Cuál elegir?
		\2 TIR modificada
			\3 Idea clave
				\4 Solventar problemas de TIR
				\4[] $\to$ TIR modificada toma un solo valor
				\4[] $\to$ Consistente con VAN como criterio de decisión
				\4 Tasa a la que habría que invertir el desembolso inicial
				\4[] para igualar valor futuro en $n$ de flujos positivos
				\4[] actualizados a tasa requerida de retorno
				\4 Equivalentemente:
				\4[] TIR de la inversión si
				\4[] $\to$ flujos positivos reinvertidos a retorno requerido
			\3 Formulación
				\4[i] Actualizar flujos positivos hasta periodo $n$
				\4[] aplicando tasa de descuento requerida
				\4[ii] Hallar tasa de descuento que iguala
				\4[] Desembolso inicial actualizado a $n$
				\4[] y flujos actualizados de \textit{i}
				\4 \fbox{$1+\text{MIRR} = \sqrt[n]{\frac{\text{FV}}{\text{P}}}$}
				\4[] $\then$ $P\cdot(1+\text{MIRR})^n = \text{FV}$
				\4[] Donde:
				\4[] $\to$ FV: flujos positivos actualizados a periodo n a coste de k
				\4[] $\to$ P: desembolso inicial
			\3 Criterio de decisión
				\4 Dados proyectos A y B:
				\4[] se lleva a cabo proyecto con mayor MIRR
		\2 Otros métodos
			\3 Plazo de recuperación o payback
				\4 ¿Cuánto tiempo tardan la suma de los flujos positivos
				\4[] ...en igualar el desembolso inicial?
				\4 Payback descontado
				\4[] Versión del payback standard
				\4[] Descontar flujos positivos a tasa de descuento
				\4[] $\to$ Sumar
				\4[] $\then$ Siempre mayor a payback standard para $\rho > 1$
			\3 Rendimiento contable (ROCE)
				\4 Ingreso operativo después de impuestos
				\4[] por unidad de capital utilizado
				\4 $\text{ROCE} = \frac{\text{Ingreso operativo después de impuestos}}{\text{Valor contable del activo total}}$
				\4[] Ingreso operativo después de impuestos:
				\4[] $\to$ EBITDA - Impuestos
			\3 Índice del valor actual neto
				\4 $\frac{\text{VAN de flujos positivos}}{\text{VAN de flujos negativos}}$
				\4 Situaciones con:
				\4[] Restricciones de financiación
				\4[] Horizonte temporal no relevante
				\4 Objetivo:
				\4[] Encontrar combinanción de proyectos
				\4[] $\to$ Con máximo IVAN medio ponderado.
		\2 Estimación de flujos de caja
			\3 Idea clave
				\4 Métodos de VAN y TIR
				\4[] Basados en descuento de flujos
				\4[] $\to$ ¿Cuáles son esos flujos?
				\4[] $\to$ ¿Qué cuantías?
				\4 Necesario estimar
				\4 Sistematizar estimación
				\4[] Diferentes métodos de estimación
				\4[] Más precisos requieren más información
			\3 Planes de negocio y escenarios
				\4 Tratar de estimar futuro probable
				\4[$\to$] Identificar parámetros que afectan flujos
				\4[$\to$] Cuantificar flujos de caja futuros
				\4 Análisis de sensibilidad
				\4[] Cuantificar sensibilidad a shocks
				\4[] $\to$ Inversiones muy sensibles son más arriesgadas
				\4 Construcción de escenarios
				\4[] Caracterizan diferentes estados futuros
				\4[] P. ej.: pesimista, optimista, realista..
			\3 Montecarlo
				\4 Formular modelo:
				\4[] Flujos de caja dependen de vars. aleatorias
				\4 Postular distribución de vars. aleatorias
				\4 Simular realizaciones de vars. aleatorias
				\4[] $\to$ Hallar flujos resultantes
			\3 Flexibilidad, opciones reales y árboles de decisión
				\4 Método del VAN asume irreversibilidad
				\4[] $\to$ Pero no necesariamente
				\4[] $\to$ Managers pueden responder a eventos
				\4 Opciones reales
				\4[] Cuantificación de opciones de inversión
				\4[] Opciones reales no son títulos separados
				\4[] $\to$ Pero añaden valor a proyecto
				\4[] Posibilidad de cancelar o aumentar inversión
				\4[] $\to$ Es valiosa para manager
	\1 \marcar{Inversión en activo fijo y corriente 24-27}
		\2 Activo fijo
			\3 Maquinaria\footnote{Ver Hartman y Hon Tan (2014).}
				\4 Vida útil
				\4[] ¿Cuánto tiempo puede utilizarse?
				\4 Vida óptima
				\4[] ¿Cuánto tiempo debe utilizarse?
				\4[] $\to$ Para optimizar creación de valor
				\4 Reemplazos idénticos
				\4[] La tecnología no avanza
				\4[] $\to$ Reemplazos de activo fijo son iguales
				\4[] Problema de maximización del VAN
				\4[] $\to$ maximizando periodo de utilización
				\4[] $\then$ Vida óptima
				\4[] $\underset{n}{\max} \quad \text{VAN}_n = -A_0 + \sum_{t=1}^n \frac{Q_t}{(1+r)^t} +  \frac{\text{VR}_n}{(1+r)^n}$
				\4 Métodos MAPI y similares
				\4[] Tener en cuenta obsolescencia
				\4[] Decisión de reemplazo es un trade-off
				\4[] Reducción de coste por mantener:
				\4[] $\to$ Coste fijo se diluye en más producción
				\4[] $\to$ Coste fijo medio cada vez menor
				\4[] Aumento de coste por mantener:
				\4[] $\to$ Costes de mantenimiento y reparación
				\4[] $\to$ Obsolescencia
			\3 Activos inmobiliarios
				\4 Activo de especial importancia
				\4[] Muy dependiente de contexto concreto
				\4 Nuevas empresas
				\4[] Preferible alquiler o lease
				\4[] Deseable escalabilidad
				\4[] Deseable máxima liquidez
				\4 Empresas consolidadas con potencial crecimiento
				\4[] Puede ser preferible comprar
				\4[] $\to$ Ahorrar incremento de rentas
				\4[] O leases de largo plazo
				\4 Empresas maduras
				\4[] A menudo cuentan con patrimonio inmob.
				\4[] Pueden vender y tomar lease
				\4[] $\to$ Para financiar entrada nuevos mercados
				\4 Empresas del sector inmobiliario
				\4[] Activos inmob. forman parte de ciclo operativo
				\4[] $\to$ Métodos similares a inventarios
		\2 Activo corriente
			\3 Idea clave
				\4 Problema abstracto:
				\4[] Activo corriente genera costes:
				\4[] $\to$ de oportunidad por tenencia (inacción)
				\4[] $\to$ de reposición (ajuste)
				\4[] $\then$ Innacción vs. ajuste
			\3 Minimización de costes totales
				\4 Problema simplificado
				\4[] Asumiendo precio unitario constante
				\4[] Asumiendo coste fijo por pedido
				\4[] $k$ $\to$ coste de un pedido
				\4[] $V$ $\to$ volumen de activo corriente en periodo
				\4[] $S$ $\to$ Cantidad de un pedido
				\4[] $g$ $\to$ Coste de tenencia media
				\4[] $\frac{S}{2}$ $\to$ stock medio
				\4[] $\underset{S}{\min} \quad \frac{V}{S}k + g \frac{S}{2}$
				\4 Solución
				\4[] $S^* = \sqrt{\frac{2kV}{g}}$
				\4[$\to$] Fórmula de Wilson/Cantidad Económica de Pedido
			\3 Extensible
				\4 Coste por pedido variable
				\4[] P.ej: dependiente de cantidad
				\4 Descuento por cantidad
				\4 Diferente función de stock medio
	\1[] \marcar{Conclusión 27-30}
		\2 Recapitulación
			\3 Análisis general de las decisiones de inversión
			\3 Determinantes de la inversión
				\4 Riesgo
				\4 Rentabilidad
				\4 Coste del capital
			\3 Elección entre inversiones alternativas
				\4 Criterios de decisión
			\3 Inversión en activo fijo y corriente
				\4 Especificidades
		\2 Idea final
			\3 Inversión
				\4 Motor de crecimiento
				\4 Generación de valor
			\3 Economía
				\4 Resultado de múltiples decisiones de inversión
				\4 Crecimiento y bienestar resultado de decisiones
			\3 Agregación
				\4 Empresas con mejores procesos de decisión
				\4[] Influyen en toda la economía
				\4 Inversiones que crean valor
				\4[] Aumentan stock de capital de economía
				\4[] Aumenta PIB
				\4[] $\to$ Aumentan bienestar
				\4[$\then$] Sector público debe prestar atención
				\4[$\then$] Fomento de la transparencia, información, educación
				\4[$\then$] Sector público también toma decisiones de inversión
\end{esquemal}























\graficas

\conceptos

\preguntas


\seccion{Test 2018}

\textbf{23.} Los criterios del Valor Actual Neto y de la Tasa Interna de Retorno para valorar proyectos de inversión:

\begin{itemize}
	\item[a] Conducen a las mismas decisiones de aceptación y rechazo de inversiones simples.
	\item[b] Conducen a las mismas decisiones de aceptación y rechazo de inversiones simples sólo si asumimos la no reinversión de los flujos intermedios de caja.
	\item[c] Conducen a las mismas decisiones de aceptación y rechazo tanto de inversiones simples como de inversiones no simples mixtas.
	\item[d] Ninguna de las opciones anteriores es correcta.
\end{itemize}

\seccion{Test 2015}

\textbf{26.} Señale la respuesta correcta referida a los distintos métodos de valoración de proyectos de inversión suponiendo una tasa de descuento positiva y proyectos de inversión simples:

\begin{enumerate}
    \item[a] El \textit{adjusted payback} (o plazo de recuperación actualizado o descontado) será menor que el \textit{payback} (o plazo de recuperación sin coregir).
    \item[b] El VAN (valor actual neto) decrece cada vez más despacio conforme aumenta la tasa de descuento.
    \item[c] Una condición suficiente para que la jerarquización de los proyectos de inversión de acuerdo al criterio del VAN y de la TIR (tasa interna de retorno) coincidan es que no exista un punto de Fisher en el conjunto definido por todos los proyectos de inversión con VAN y TIR positivos.
    \item[d] El criterio del VAN supone implícitamente que los flujos positivos del proyecto de inversión se reinvierten a un tipo de interés superior a la tasa de descuento.
\end{enumerate}

\seccion{Test 2009}
\textbf{23.} Se entiende por método de valoración de las opciones reales:

\begin{enumerate}
    \item[a] La técnica que valora la empresa en función de los flujos de caja que genera, ya que es la única forma real y tangible de conocer su valor.
    \item[b] La técnica que busca cuantificar el valor inmaterial de la empresa o fondo de comercio.
    \item[c] La técnica que no valora la empresa sólo por el flujo de caja que genera sino también por las posibilidades de inversiones futuras.
    \item[d] La técnica que cuantifica el valor de la empresa estando ésta en circunstancias normales.
\end{enumerate}

\seccion{Test 2004}

\textbf{23.} Indique cuál de las siguientes afirmaciones referentes a los métodos de valoración de empresas es INCORRECTA:

\begin{enumerate}
    \item[a] Los métodos del VALOR ACTUAL NETO (VAN) y de la TASA DE INTERÉS INTERNO (TII) pueden considerarse métodos dinámicos de valoración de empresas al tener en cuenta el valor del dinero en el tiempo.
    \item[b] El método del VAN tiene el inconveniente respecto al método de la TII de tener que especificar un tipo de actualización o descuento.
    \item[c] El resultado de ambos métodos puede ser o bien un número imaginario o bien un número real (positivo o negativo).
    \item[d] Un inconveniente común a ambos métodos es la denominada \comillas{hipótesis de reinversión de los flujos intermedios de caja}.
\end{enumerate}

\notas

\textbf{2018:} \textbf{23.} A

\textbf{2015:} \textbf{26}. Anulada, parece que hay más de una correcta.

\textbf{2009:} \textbf{23}. C.

\textbf{2004:} \textbf{23}. C. El VAN nunca es un número imaginario.


\bibliografia

Vernimmen: caps. 28, 29, 30

Suárez Suárez está en la biblioteca y es la base de todos las versiones disponibles de este tema. Es un libro muy anticuado, con citas de textos de años 40, 50 y 60 de autores fundamentales europeos. Sin embargo, el tipo era catedrático en la complutense y todos los tecos parecen haber usado su libro para este tema. Quizás haya sido miembro de algún tribunal, y es previsible que las preguntas del tipo test vayan en la dirección de este libro. El apartado en relación a simulaciones de Monte Carlo es útil para entender el concepto. La sección de métodos bayesianos puede usarse también.

Mirar en Palgrave:
\begin{itemize}
    \item amortization
    \item depreciation
    \item internal rate of return
    \item investment decision criteria
    \item options
    \item present value
    \item S-s models
\end{itemize}




Fernández, Pablo. \textit{Papers en SSRN.}

Fernández, Pablo. \textit{WACC: definición, interpretaciones equivocadas y errores (WACC: Definition and Errors)}. \url{https://papers.ssrn.com/sol3/papers.cfm?abstract_id=1633408}

Fernández, P. \textit{Cash flow is cash and is a fact: net income is just an opinion} (2006) Working Paper IESE -- En carpeta del tema

Hartman, J. C.; Hon Tan, C. \textit{Equipment replacement analysis: a literature review and directions for future research} (2014) -- En carpeta del tema

Hull cap. 35 para opciones reales.

Vernimmen, caps. 28, 29, 30

\end{document}
