\documentclass{nuevotema}

\tema{3B-19}
\titulo{La liberalización del sector exterior y sus implicaciones sobre la estabilidad macroeconómica. Opciones de política económica.}

\begin{document}

\ideaclave

\seccion{Preguntas clave}
\begin{itemize}
	\item ¿Qué es liberalizar el sector exterior?
	\item ¿Por qué liberalizar el comercio exterior de mercancías?
	\item ¿Por qué liberalizar los flujos de capital?
	\item ¿Qué efectos tiene?
	\item ¿En qué consiste el proceso de liberalización?
	\item ¿Qué evidencia empírica existe al respecto?
	\item ¿Qué implicaciones de economía política se derivan?
\end{itemize}

\esquemacorto

\begin{esquema}[enumerate]
	\1[] \marcar{Introducción}
		\2 Contextualización
			\3 Globalización
			\3 Crecimiento del comercio
			\3 Librecambismo vs protección
			\3 Crecimiento de los flujos de capital
			\3 Liberalización exterior
		\2 Objeto
			\3 ¿Qué es liberalizar el sector exterior?
			\3 ¿Por qué liberalizar el comercio exterior de mercancías?
			\3 ¿Por qué liberalizar los flujos de capital?
			\3 ¿Qué efectos tiene?
			\3 qué consiste el proceso de liberalización?
			\3 ¿Qué evidencia empírica existe al respecto?
			\3 ¿Qué implicaciones de economía política se derivan?
		\2 Estructura
			\3 Liberalización de la cuenta corriente
			\3 Liberalización de la cuenta financiera
			\3 Economía política del sector exterior
	\1 \marcar{Liberalización de la cuenta corriente}
		\2 Idea clave
			\3 Contexto
			\3 Objetivo
			\3 Resultado
		\2 Beneficios de la liberalización
			\3 Teorías neoclásica del comercio
			\3 Competencia monopolística
			\3 Productividad de empresas
			\3 Crecimiento endógeno: spill-overs vía comercio
			\3 Diversificación de proveedores
			\3 Reducción de desigualdades en PEDs y PEDs vs desarrollados
			\3 Aparición de lobbies pro-liberalización en otros países
		\2 Argumentos en contra
			\3 Relación relativa de intercambio
			\3 Estructuralismo
			\3 Home-Market Effect
			\3 Dependencia de proveedores exteriores
			\3 Deslocalización de empresas en desarrollados
			\3 Aumento de desigualdades
			\3 Aumento del déficit por CC
			\3 Costes de reasignación de recursos
		\2 Proceso de liberalización
			\3 Liberalización interna
			\3 Política fiscal contractiva
			\3 Política monetaria no inflacionaria
			\3 Arancelización
			\3 Reducción arancelaria
			\3 Gradualidad del ajuste
			\3 Apertura unilateral vs integración comercial
		\2 Evidencia empírica
			\3 Medidas de liberalización
			\3 Bretton Woods
			\3 OMC
			\3 Bloque comunista
			\3 China
			\3 UE: mercado interior y euro
	\1 \marcar{Liberalización de la cuenta financiera}
		\2 Idea clave
			\3 Contexto
			\3 Objetivo
			\3 Resultado
		\2 Justificación de la liberalización
			\3 Mejora de eficiencia asignativa del capital
			\3 Financiar inversión por encima de ahorro
			\3 Financiación de shocks adversos transitorios
			\3 Costes administrativos de controles de capital
			\3 Efecto disciplina
			\3 Aumento del comercio
			\3 Aumento de productividad media
		\2 Argumentos en contra
			\3 Trilema de Mundell y Fleming
			\3 Exposición a ciclo financiero global
			\3 Sudden stops y reversiones de flujos de capital
			\3 Imposible represión financiera
			\3 Hot money
			\3 Intolerancia a la deuda externa
		\2 Proceso de liberalización
			\3 Liberalización de cuenta comercial
			\3 Regulación del sector financiero nacional
			\3 Liberalización del sistema financiero doméstico
			\3 Enfoque gradualista
			\3 Enfoque de shock
			\3 Tipo de cambio
			\3 Instituciones
			\3 Problemas de implementación
		\2 Análisis empírico
			\3 Medidas de liberalización
			\3 Problemas de medición
			\3 Estabilidad monetaria
			\3 Crecimiento
			\3 Hot money
			\3 Crisis financieras
			\3 Controles a entradas de flujos de corto plazo
			\3 Liberalización previa al euro: SME duro
			\3 Crisis sudeste asiático
			\3 Postura del FMI
	\1 \marcar{Economía política del sector exterior}
		\2 Idea clave
		\2 Liberalización y democracia
		\2 Economía política de los aranceles
			\3 Idea clave
			\3 Stolper-Samuelson
			\3 Redistribución de beneficios del comercio
			\3 Modelo de factores específicos
			\3 Aversión a la pérdida
			\3 Aversión a incertidumbre
			\3 Aversión a desigualdad
			\3 Concentración de intereses
			\3 Instituciones multilaterales pueden catalizar
			\3 Redistribución puede ser necesaria
			\3 Valoración
		\2 Trilema de Rodrik
			\3 Idea clave
			\3 Formulación
			\3 Implicaciones
			\3 Valoración
	\1[] \marcar{Conclusión}
		\2 Recapitulación
			\3 Liberalización de la cuenta corriente
			\3 Liberalización de la cuenta financiera
			\3 Economía política del sector exterior
		\2 Idea final

\end{esquema}

\esquemalargo

\begin{esquemal}
	\1[] \marcar{Introducción}
		\2 Contextualización
			\3 Globalización
				\4 Proceso generalizado en últimos siglos
				\4 Aceleración en últimas décadas
				\4 Varias oleadas y retrocesos
				\4 Definición relativamente difusa
				\4[] Aumento de flujos comerciales y de servicios
				\4[] Difusión rápida de información y tecnologías
			\3 Crecimiento del comercio
				\4 Diferentes oleadas
				\4[] Aumentos mayores y menores que crecimiento económico
				\4 Tendencia de muy largo plazo creciento
				\4[] Reducción de coste de transporte
				\4[] Mejor en comunicaciones
				\4[] Crecimiento de la población
			\3 Librecambismo vs protección
				\4 Debate de largo plazo
				\4 Relativo consenso entre economistas
				\4[] Libre comercio es first best
				\4 Pero existen dificultades
				\4[] Problemas de economía política
				\4[] Incentivos unilaterales a proteger
				\4[] Liberalizaciones parciales
				\4[] Costes de ajuste
			\3 Crecimiento de los flujos de capital
				\4 Muy complejos determinantes
				\4[] Varias teorías
				\4[] Objeto en otros temas
				\4 Desde años 60 y especialmente desde 80s
				\4[] Fuerte aumento de flujos de K
				\4 Desarrollo de sector financiero internacional
				\4 Crisis monetarias y financieras
				\4 Raramente se producen vueltas atrás en liberalización
				\4[] A pesar de problemas ligados a liberalización
			\3 Liberalización exterior
				\4 Proceso complejo
				\4 Raramente inmediato
				\4 Sujeto a fuertes debates
		\2 Objeto
			\3 ¿Qué es liberalizar el sector exterior?
			\3 ¿Por qué liberalizar el comercio exterior de mercancías?
			\3 ¿Por qué liberalizar los flujos de capital?
			\3 ¿Qué efectos tiene?
			\3 qué consiste el proceso de liberalización?
			\3 ¿Qué evidencia empírica existe al respecto?
			\3 ¿Qué implicaciones de economía política se derivan?
		\2 Estructura
			\3 Liberalización de la cuenta corriente
			\3 Liberalización de la cuenta financiera
			\3 Economía política del sector exterior
	\1 \marcar{Liberalización de la cuenta corriente}
		\2 Idea clave
			\3 Contexto
				\4 Fenómeno casi omnipresente desde IIGM
				\4[] Reducción de restricciones a movimientos de CC
				\4[] Especialmente:
				\4[] $\to$ Comercio de mercancías
				\4[] $\to$ Algunos servicios
				\4[] $\to$ Rendimientos de inversión
				\4 Proceso asimétrico y asíncrono
				\4[] Diferentes intensidades
				\4[] Diferentes evoluciones temporales
				\4 Efectos de liberalización muy variados
				\4[] Beneficios
				\4[] Costes
				\4[] Diferentes bfcios. y costes para distintos grupos
			\3 Objetivo
				\4 Justificar liberalización de la cuenta corriente
				\4 Valorar inconvenientes de la liberalización
				\4 Caracterizar proceso de liberalización
				\4 Extraer conclusiones de evidencia empírica
			\3 Resultado
				\4 Justificaciones de la liberalización
				\4[] Diferentes fuentes de beneficio
				\4 Problemas de la liberalización
				\4[] Pérdida de poder adquisitivo
				\4[] Dependencia exterior
				\4[] Dinámicas de empobrecimiento
				\4 Proceso de liberalización
				\4[] Influye en resultados
		\2 Beneficios de la liberalización
			\3 Teorías neoclásica del comercio
				\4 Idea clave
				\4[] Liberalización de cuenta corriente deseable porque
				\4[] $\to$ Permite reducir costes de producción
				\4[] $\then$ Permite aumentar renta real
				\4[] $\then$ Más consumo de bienes y servicios
				\4 Ventaja comparativa
				\4[] Apertura comercial permite aprovechar ventaja comparativa
				\4[] $\to$ Ventaja comparativa por diferentes tecnologías
				\4[] $\then$ Diferentes costes de oportunidad de prod. de bienes
				\4[] Países se especializan
				\4[] $\to$ En producción de bien con menor coste de oportunidad
				\4[] $\then$ Menores costes totales de producción
				\4[$\then$] Liberalización reduce costes de producción
				\4[$\then$] Liberalización aumenta renta real
				\4 Heckscher-Ohlin
				\4[] Países disponen de diferentes dotaciones de ff.pp.
				\4[] Mismas tecnologías de prod. disponibles
				\4[] Mismas demandas
				\4[] En autarquía
				\4[] $\to$ Países producen ambos bienes
				\4[] En libre comercio
				\4[] $\to$ Países se especializan en bien producido a menor coste
				\4[] $\to$ Bien intensivo en factor abundante producido a menor coste
				\4[] $\then$ Especialización en bienes producibles a menor coste
				\4[$\then$] Liberalización reduce costes de producción
				\4[$\then$] Liberalización aumenta renta real
			\3 Competencia monopolística
				\4 Idea clave
				\4[] Consumidores prefieren variedad de productos
				\4[] $\to$ Demanda de servicios diferenciados
				\4[] Economías de escala
				\4[] $\to$ Inducen entrada limitada de empresas diferenciadas
				\4[] Libre comercio
				\4[] $\to$ Permite acceso a más variedades
				\4[] $\then$ Aumento de utilidad de consumidores
				\4 Formulación
				\4[] Función de utilidad
				\4[] $\to$ Bien homogéneo
				\4[] $\to$ Bien diferenciado compuesto CES
				\4[] Demanda de variedades
				\4[] $\to$ Ingreso dedicado a bien diferenciado
				\4[] $\to$ Nivel de precios
				\4[] $\to$ Precio de variedad dada
				\4[] $\to$ Elasticidad de sustitución entre variedades
				\4[] Empresas de variedades homogéneas
				\4[] $\to$ Mark-up sobre coste marginal
				\4[] $\to$ Entrada hasta eliminar beneficios
				\4 Implicaciones
				\4[] Liberalización beneficia más a más pequeños
				\4[] $\to$ Les permite acceder a más variedades extranjeras
				\4[] Home-market effect
				\4[] $\to$ Presencia de costes de transporte
				\4[] $\to$ Preferible producir más cerca de la demanda
				\4[] $\then$ Liberalización puede aumentar producción en mercado grande
				\4[] $\then$ Puede aumentar bfcio. del comercio para grandes
			\3 Productividad de empresas
				\4 Melitz (2003)
				\4[] Modelos previos de competencia monopolística
				\4[] $\to$ Empresas diferenciadas pero igual productividad
				\4[] Evidencia empírica robusta
				\4[] $\to$ Empresas muy diferentes productividades
				\4[] $\to$ Liberalización tiene efectos sobre estructura empresas
				\4[] Costes fijos de entrada
				\4[] $\to$ Obligan a mantener beneficio operativo mínimo
				\4[] $\then$ Salida del mercado en caso contrario
				\4[] Liberalización comercial
				\4[] $\to$ Abre mercado a competencia extranjera
				\4[] $\to$ Permite empresas nacionales exportar
				\4[] $\then$ Caída de beneficio operativo en mercado doméstico
				\4[] $\then$ Coste fijo adicional por exportar
				\4[] $\then$ Beneficio operativo adicional por exportación
				\4[] Resultados
				\4[] $\to$ Empresas menos competitivas salen del mercado
				\4[] $\to$ Empresas más competitivas ganan más
				\4[] $\then$ Aumento del tamaño medio de empresas
				\4[] $\then$ Aumento de la productividad media
			\3 Crecimiento endógeno: spill-overs vía comercio
				\4 Comercio induce transferencia de conocimiento
				\4[] Aumento de productividad
				\4 Grossman y Helpman (1991)
				\4[] Trade, knowledge spilloves and growth
				\4[] Producción nacional necesita inputs no comerciables
				\4[] $\to$ Más productividad cuanta mayor variedad
				\4[] Coste de introducir nueva variedad depende de:
				\4[] $\to$ stock de capital tecnológico
				\4[] Stock de capital tecnológico crece con:
				\4[] $\to$ Variedades presentes
				\4[] $\to$ Comercio internacional
				\4[] Capital tecnológico disponible para todos
				\4[] $\to$ Cualquiera puede utilizar
				\4[] Beneficio de crear nueva variedad
				\4[] $\to$ Se iguala con coste en libre entrada
				\4[] Aumento de output tras aumentar variedades de no comerciable
				\4[] $\to$ Aumenta comercio internacional
				\4[] $\then$ Aumenta stock de capital tecnológico
				\4[] Implicaciones
				\4[] $\to$ Comercio aumenta exposición a nuevas ideas
				\4[] $\to$ Nuevas ideas aumentan creación de variedades
				\4[] $\then$ Crecimiento endógeno
			\3 Diversificación de proveedores
				\4 Idea clave
				\4[] Intervención gubernamental en comercio exterior
				\4[] $\to$ Integración económica con socios políticos
				\4[] $\to$ Comercio de estado
				\4[] $\to$ Discriminación de importaciones
				\4[] $\to$ ...
				\4[] Puede alterar elenco de proveedores de bienes
				\4[] Posible incentivación a concentración de proveedores
				\4[] Liberalización puede tener efecto diversificador
				\4[] $\to$ Acceso a más variedades
				\4[] $\to$ Menor dependencia de proveedor exterior
				\4 Shocks de oferta y conflicto
				\4[] Múltiples ejemplos históricos
				\4[] Crisis del Petróleo de 1973 y 1979
				\4[] $\to$ Shock de oferta generalizado en Occidente
				\4[] Embargo a Huawei
				\4[] $\to$ China debe acumular stock de microprocesadores
				\4[] Mascarillas en crisis de 2020
				\4[] $\to$ Escasez inicial
				\4[] Hidroxicloroquina en crisis de 2020
				\4[] $\to$ Escasez inicial
				\4 Implicaciones
				\4[] Liberalización permite mayor adaptación de suministros
				\4[] Apertura comercial para aumentar resiliencia
				\4[] Efecto puede de hecho ser contrario
				\4[] $\to$ Con un proveedor muy eficiente
				\4[] $\then$  Aumento de concentración con liberalización
			\3 Reducción de desigualdades en PEDs y PEDs vs desarrollados
				\4 Aumento de variedades en PEDs
				\4[] Aumenta productividad
				\4 Integración en CVGs
				\4[] Transfiere capital tecnológico a PEDs
				\4[] Know-how a PEDs
				\4 Especialización en desarrollados
				\4[] Más trabajo dedicado a I+D y frontera tec.
				\4[] Nuevas variedades
				\4[] Más innovación
				\4[$\then$] Especialización mundial del trabajo
				\4[$\then$] Crecimiento global
			\3 Aparición de lobbies pro-liberalización en otros países
				\4 Otros países exportan a mercado nacional
				\4[] Aumento de exportadores extranjeros
				\4 Aparición de relaciones comerciales con empresas nacionales
				\4 Exposición a nuevas variedades y tecnologías
				\4 Presión para liberalizar en otros países
				\4[$\then$] Mejora potencial de RRI
		\2 Argumentos en contra
			\3 Relación relativa de intercambio
				\4 Liberalización aumenta demanda de importaciones
				\4 Aumento de demanda de bienes importados
				\4[] Con oferta no perfectamente elástica
				\4[] $\to$ Aumenta precio de bien importado
				\4[$\then$] Empeoramiento de RRI
				\4[] Necesario exportar más
				\4[] $\to$ Para importar igual cantidad
			\3 Estructuralismo
				\4 Idea clave
				\4[] Determinados bienes tienen demandas inelásticas
				\4[] $\to$ Ley de Engel
				\4[] $\then$ Demanda
				\4[] Países en desarrollo
				\4[] $\to$ Competitivos en determinados bienes
				\4[] $\to$ Tienden a ser los de demandas inelásticas
				\4[] Prebisch, Singer y otros
				\4 ISI
				\4[] Sustituir importaciones extranjeras
				\4[] $\to$ Por producción doméstica
				\4[] Fomentar aparición de industria nacional
				\4[] $\to$ Evitar caer en producción de inelásticos
				\4 EOI
				\4[] Orientar industria a exportación
				\4[] Fomentar aparición de industrias exportadoras
				\4[] Requiere apertura en otros ámbitos
				\4[] $\to$ Bienes de capital
				\4[] $\to$ IDE
				\4[] $\to$ Transferencia de capital
			\3 Home-Market Effect
				\4 Idea clave
				\4[] Empresas se localizan en mercados más grandes
				\4[] $\to$ Evitar costes de transporte
				\4[] Mercados más pequeños
				\4[] $\to$ Pierden productores locales
				\4[] $\to$ Aumentan importaciones
				\4 Implicaciones
				\4[] Países relativamente pequeños
				\4[] $\to$ Pueden sufrir deslocalización
				\4[] Empresas deslocalizan hacia mercados grandes
				\4[] $\to$ Caída de sueldos
				\4[] $\to$ Caída de renta real en países pequeños
			\3 Dependencia de proveedores exteriores
				\4 Seguridad nacional
				\4[] Algunos inputs y bienes de consumo estratégicos
				\4[] $\to$ Esenciales para mantener actividad económica
				\4[] $\to$ Carencias pueden provocar inestabilidad política
				\4[] $\to$ Pérdida de acceso aumenta riesgo de conflicto
				\4 Liberalización comercial
				\4[] Puede desincentivar producción nacional de estratégicos
				\4[] $\to$ Si doméstico no es competitivo en su producción
				\4 Protección en industrias estratégicas
				\4[] Captura demanda hacia productores nacionales
				\4[] Evita trampas de dependencia
			\3 Deslocalización de empresas en desarrollados
				\4 Liberalización permite importación de más baratos
				\4 Países en desarrollo tienen ventajas comparativas
				\4[] En bienes intensivos en mano de obra no cualificada
				\4 Reducción de costes de trasporte + trans. tecnológicas
				\4[] Hacen viable producción en PEDs
				\4 Liberalización comercial permite translado e importación
				\4[$\then$] Aumento de desempleo en regiones industriales
				\4[$\then$] Conflictos de economía política
			\3 Aumento de desigualdades
				\4 En contexto Heckscher-Ohlin
				\4[] Stolper-Samuelson
				\4[] $\to$ Factor intensivo de bien de especialización beneficiado
				\4[] $\then$ Países desarrollados: trabajo cualificado
				\4[] $\then$ PEDs: trabajo no cualificado
				\4[] En países desarrollados
				\4[] $\to$ Trabajo cualificado gana más tras liberalización
				\4[] $\to$ Trabajo cualificado ya ganaba más antes
				\4[] $\then$ Aumento de desigualdades
				\4[] En PEDs
				\4[] $\to$ Trabajo no cualificado gana más tras liberalización
				\4[] $\to$ Acceso a bienes intensivos en cualificado más barato
				\4[] $\then$ Caída de desigualdad
				\4 Efecto superstar\footnote{Ver \footnote{https://economics.mit.edu/files/10480}{Acemoglu (2015)}}
				\4[] Fenómeno empíricamente contrastado
				\4[] Ligado a cambios tecnológicos recientes
				\4[] Con mercados abiertos y globales
				\4[] $\to$ Personas con ligera ventaja comparativa
				\4[] $\then$ Capturan cuota de mercado mucho mayor
			\3 Aumento del déficit por CC
				\4 Cuenta corriente desde la óptica ahorro-inversión
				\4[] $\text{Saldo CC} = S_\text{Priv} - I_\text{Priv}+T - G$
				\4 Aumento del consumo privado
				\4[] Por acceso a bienes importados más competitivos
				\4 Aumento de la inversión privada
				\4[] Por aumento del tamaño de la economía nacional
				\4 Caída de la recaudación tributaria
				\4[] Por reducción de aranceles
				\4 Aumento del gasto público
				\4[] Costes de compensación a perdedores
				\4[$\then$] Potencial aumento del déficit por CC
			\3 Costes de reasignación de recursos
				\4 Sectores expuestos a economía internacional
				\4[] Aumenta variación de precios relativos
				\4[] Fluctuaciones de oferta y demanda
				\4 Reasignación de recursos productivos
				\4 Paro friccional
				\4[] Búsqueda de empleo entre reasignación
				\4 Paro estructural
				\4[] Capital humano insuficiente/específico
				\4[] $\to$ Reasignación imposible en corto plazo
				\4[] $\then$ Aumento del desempleo
				\4[] $\then$ Posible aparición de histéresis
		\2 Proceso de liberalización
			\3 Liberalización interna
				\4 Liberalización exterior
				\4[] Induce
				\4[] $\to$ Reasignación ff.pp
				\4[] $\to$ Especialización regional
				\4[] $\to$ Comercio interregional
				\4 Barreras internas a movilidad
			\3 Política fiscal contractiva
				\4 Idea clave
				\4[] Evitar caída del ahorro público
				\4[] $\to$ Presión hacia déficit de CC
				\4 Contención de gasto discrecional
				\4 Evitar crecimiento excesivo estab. automáticos
				\4 Focalizar gasto en sectores afectados
				\4 Incentivar reorganización de ff.pp.
			\3 Política monetaria no inflacionaria
				\4 Tipo de cambio real indirecto
				\4[] Unidades de bien extranjero
				\4[] $\to$ Para comprar una de nacional
				\4[] $\text{TCR} = S \cdot \frac{P}{P^*} $
				\4[] $\then$ $\uparrow$ TCR $\then$ Apreciación bien nacional
				\4 Inflación
				\4[] Provoca aumento de TCR
				\4[] $\to$ Provoca pérdida de competitividad bien nacional
				\4[] $\then$ Aumento de déficit CC
				\4[] $\then$ Deseable evitar política inflacionaria
				\4 Política monetaria que evite inflación
				\4[] Banco Central Independiente
				\4[] Comunicación de políticas futuras estable
				\4[] Sesgo conservador de composición BCN
				\4[] Control de oferta monetaria
				\4[] Regla de Taylor
				\4[] ...
				\4[$\then$] Evitar apreciación del TCER
			\3 Arancelización
				\4 Cuotas de importación son fuertemente distorsionantes
				\4[] Aparición de rentas a capturar
				\4 Cuotas inducen rent-seeking
				\4[] Licencias de importación permiten extracción de rentas
				\4[] $\to$ ¿Quién las captura?
				\4[] Estado vende licencias
				\4[] $\to$ Incentivos a fraude y corrupción
				\4 Cuotas limitan acceso a importaciones
				\4[] Introducen cuellos de botella industria nacional
				\4[] Ante aumento de la demanda de bien
				\4[] $\to$ Reducen aún más el excedente del consumidor
				\4 Permiten mantenimiento de monopolios nacionales
				\4[] Demanda no cubierta por cuota
				\4[] $\to$ A disposición del monopolista nacional
				\4[] $\then$ Rentas de monopolio
				\4 Aumentan distorsiones frente a aranceles
				\4[] Aranceles permiten cierto grado de adaptación
				\4[] $\to$ Ante aumentos de oferta y dda. nacional
				\4[] Cuotas aumentan inelasticidad de oferta
				\4[] $\to$ Menos capacidad de adaptación oferta nacional
			\3 Reducción arancelaria
				\4 Medidas de salvaguardia
				\4[] Evitar perjuicio grave con liberalización rápida
				\4[] Acuerdo forma parte de OMC
				\4 Aumentar exposición a competencia exterior
				\4 Aumentar demanda nacional de exportaciones
				\4 Suavizar perfil arancelario
				\4[] Evitar picos elevados en determinados sectores
			\3 Gradualidad del ajuste\footnote{Ver \href{https://www.econstor.eu/bitstream/10419/140734/1/505623722.pdf}{Popov (2006)}.}
				\4 Debate de largo plazo
				\4[] ¿Liberalización progresiva o rápida?
				\4 Economías planificadas
				\4[] Elevada intervención del estado en precios relativos
				\4[] Apertura exterior e interior a la vez
				\4[] Potencial para aumento de distorsiones
				\4[] $\to$ Algunos precios se mantienen administrados
				\4[] $\to$ Otros dependen de oferta y demanda
				\4[] $\then$  Aumento de distorsiones
				\4 Terapia de shock
				\4[] Liberalización rápida
				\4[] Anclar expectativas rápidamente
				\4[] $\to$ No tiene vuelta atrás
				\4[] $\to$ En el futuro, liberalización se mantendrá
				\4[] Recesión transformacional
				\4[] $\to$ Reasignación rápida de factores
				\4[] $\to$ Colapso institucional posible
				\4[] $\then$ Factores en contra de terapia de shock
				\4 Apertura gradual
				\4[] Evitar recesión transformacional
				\4[] Reducir obstáculos a comercio en determinadas industrias
				\4[] Zonas económicas especiales
				\4[] Expectativas mal ancladas
				\4[] $\to$ Aumento de incertidumbre
				\4[] $\to$ Liberalización puede percibirse como reversible
			\3 Apertura unilateral vs integración comercial
				\4 Contexto
				\4[] Análisis de Viner, Meade, Lipsey
				\4[] $\to$ Integración reduce distorsión de arancel propio
				\4[] $\to$ Aumenta bienestar del consumidor nacional
				\4[] $\to$ Reduce costes de producción nacional
				\4[] $\then$ ¿Reducción compensa otros efectos?
				\4[] Simplemente con esos efectos
				\4[] $\to$ Cooper y Massel (1965)
				\4[] $\to$ Apertura unilat. $\Rightarrow$ creación máx. de comercio
				\4[] $\to$ Apertura unilat. evita desviación de comercio
				\4[] $\then$ Apertura unilateral debería ser óptima
				\4[] Asume supuesto implícito
				\4[] $\to$ Liberalizador unilat. exporta sin aranceles
				\4[] Realmente, exportaciones sufren aranceles:
				\4[] $\to$ Aranceles son herramienta de negociación
				\4[] $\to$ Cuanto mayor UA, otros más incentivados a gravar
				\4[] Otros efectos de integración económica
				\4[] $\to$ Posibles contrapartidas sobre exportaciones
				\4[] $\then$ Reducción a aranceles nacionales
				\4[] $\to$ Otros objetivos no económicos
				\4[] $\then$ Diversificación de exportadores
				\4[] $\then$ Relaciones diplomáticas, militares
				\4[] Posible comparar apertura unilat. vs integración
				\4 Objetivos
				\4[] Comparar apertura unilateral vs integración
				\4[] $\to$ Liberalización vía apertura unilateral
				\4[] $\to$ Liberalización vía integración
				\4 Resultado
				\4[] Debate de largo plazo
				\4[] Regionalismo vs multilateralismo
				\4[] Aparición de argumentos a favor de integración
				\4[] $\to$ Explican por qué no simple apertura unilateral
				\4[] $\to$ Explicar incentivos de integración vs apertura
				\4 Cooper y Massel (1975)
				\4[] Apertura unilateral mejor que integración
				\4[] $\to$ Aumenta bienestar de consumidores
				\4[] $\to$ Mantiene cierto poder recaudatorio vía aranceles
				\4[] $\to$ Evita desviación de comercio
				\4 Johnson (1965)
				\4[] Algunos países quieren mantener industria nacional
				\4[] $\to$ Nacionalismo
				\4[] $\to$ Seguridad nacional
				\4[] $\to$ Economía política
				\4[] Liberalización vía integración
				\4[] $\to$ Permite mantener industria dentro de bloque
				\4[] $\then$ Sustitutivo imperfecto de industria nacional
				\4[] $\then$ Puede compensar
				\4 Wonnacott y Wonnacott (1981)
				\4[] Preferencia por industria nacional innecesaria
				\4[] $\then$ Justificable con incentivos puramente económicos
				\4[] Lib. vía integración
				\4[] $\to$ permite reducir aranceles a exportaciones
				\4[] Mayores aranceles previos a integración
				\4[] $\then$ Más poder de negociación frente a potenciales socios
				\4[] $\then$ Permiten obtener más reducción para exportaciones
				\4[] Cuanto mayor sea la UA resultante
				\4[] $\to$ Más probable sufra aranceles exteriores
				\4[] $\to$ Más efecto sobre RRI favorable a UA
				\4[] $\to$ Otros países tratan de obtener arancel óptimo
				\4[] $\then$ UA aumenta poder de negociación frente a terceros
				\4 Implicaciones
				\4[] $\to$ Existen razones económicas para lib. vía int.
				\4[] $\then$ Capacidad para alterar RRI frente a no integrados
				\4[] $\then$ Posibilidad de reducir aranceles sufridos por exportadores
				\4[] Decisión entre lib unilateral e integración
				\4[] $\to$ Depende de aranceles que enfrentan exportadores
				\4[] $\then$ Si aranc. elevados, preferible integración
				\4[] $\then$ Aumentar poder de negociación
				\4[] Depende de tamaño de UA
				\4[] $\to$ Si grande, puede afectar RRI
				\4[] $\to$ Si grande, puede ganar poder de negociación
		\2 Evidencia empírica
			\3 Medidas de liberalización
				\4 Índice de apertura exterior
				\4[] Habitualmente, suma de X y M como \% de PIB
				\4 Perfil arancelario
				\4[] Códigos arancelarios establecen aranceles
				\4[] $\to$ Para categorías de producto
				\4 Volumen de comercio sujeto a cuotas
				\4 Comercio intraindustrial
				\4[] Índice de Gruber y Lloyd
			\3 Bretton Woods
				\4 Liberalización progresiva desde fin IIGM
				\4 Plan Marshall fue elemento clave en inicios
				\4[] Saldos en dólares a países europeos
				\4[] Permiten saldar cuentas corrientes
				\4[] $\to$ A países que sufrían escasez reservas internacionales
				\4[] $\then$ Progresiva expansión de comercio intra-europeo
				\4 Cuentas corrientes apertura completa finales 50
				\4[] Elevados aranceles
				\4 Rondas sucesivas de GATT
				\4[] Centradas en bajada de aranceles hasta Dillon 60
			\3 OMC
				\4 Liberalización multilateral
				\4[] Principio de NMF
				\4[] $\to$ Liberalización con un país
				\4[] $\then$ Obliga a liberalizar para todos
				\4[] $\then$ Efecto acumulativo
				\4[] Permite utilizar a OMC como excusa
				\4[] $\to$ Atarse al mástil de liberalización
				\4[] $\then$ Palanca para evitar problemas ec. política interna
				\4 Liberalización más allá de aranceles
				\4[] GATS, Trips, TRIMS
				\4 Liberalización frente a China y Rusia
				\4[] China en 2001
				\4[] Rusia en 2011
				\4[] Debate sobre carácter de economía de mercado de China
			\3 Bloque comunista
				\4 Colapso institucional
				\4[] Reducción de capacidad del estado
				\4[] $\to$ Hacer cumplir sus propias normas
				\4 Colapso institucional generalizado en antigua URSS
				\4[] Especialmente Rusia y Asia Central
				\4 Europa del Este
				\4[] Instituciones más robustas
				\4[] Mayor resistencia a concentración de intereses
				\4[] $\to$ Rent-seeking menos efectivo
				\4[] $\then$ Posible mayor liberalización
			\3 China
				\4 Liberalización progresiva en 80s y 90s
				\4 Culminación de proceso en 2000s
				\4 Mercado interior muy grande
				\4 Intervención del estado se mantiene en muchos sectores
			\3 UE: mercado interior y euro
				\4 Liberalización rápida pero en fases
				\4 Primera fase: Unión Aduanera
				\4[] Del 57 al 68
				\4[] Eliminación de aranceles internos
				\4[] Persisten barreras no arancelarias
				\4[] $\to$ Dassonvile de 1974: eliminación de exacciones
				\4[] $\to$ Cassis de Dijon 1979: reconocimiento mutuo
	\1 \marcar{Liberalización de la cuenta financiera}
		\2 Idea clave
			\3 Contexto
				\4 Concepto de liberalización financiera
				\4[] Desregulación de operaciones financieras domésticas
				\4[] Liberalización de cuenta financiera exterior
				\4 Proceso complejo
				\4[] Posible desestabilización financiera
				\4[] Problemas de economía política
				\4[] Necesarios cambios institucionales
				\4 Ventajas e inconvenientes
			\3 Objetivo
				\4 Valorar optimalidad de liberalización
				\4 Entender argumentos a favor y en contra
				\4 Evidencia empírica al respecto
			\3 Resultado
				\4 Amplia literatura teórica
				\4[] Argumentos a favor
				\4[] $\to$ Principalmente, aumento productividad e inversión
				\4[] Argumentos en contra
				\4[] $\to$ Principalmente, sobre inestabilidad financiera
				\4 Evidencia empírica muestra beneficios generales
				\4 Dependen de tipo de mercado
				\4 PEDs se benefician más que PMAs
				\4 Liberalización tiende a aumentar inestabilidad financiera
		\2 Justificación de la liberalización
			\3 Mejora de eficiencia asignativa del capital
				\4 Capital fluye hacia proyectos más productivos
				\4 Si países tenía exceso de ahorro antes de abrir
				\4[] Puede obtener más rentabilidad por ahorro
				\4[] $\to$ Sacando capital e invirtiendo en exterior
				\4 Si país tenía insuficiencia de ahorro pre-apertura
				\4[] Puede reducir coste de financiación
				\4[] $\to$ Permitiendo entrada de capital
			\3 Financiar inversión por encima de ahorro
				\4 Realización de planes de inversión
				\4[] Evitan restricciones por ahorro doméstico
			\3 Financiación de shocks adversos transitorios
				\4 Shocks de oferta exteriores
				\4[] Fuerte aumento de precio de input importado
				\4[] Caída temporal de exportaciones nacionales
				\4[] $\then$ Presión sobre saldo de cuenta corriente
				\4 Flujos de capital permiten amortiguar
				\4[] Financiación de déficit en CC hasta amortiguar shock
				\4 Alternativa es ajuste de CC
				\4[] Reducción de importaciones
				\4[] Reducción de precios de exportación
			\3 Costes administrativos de controles de capital
				\4 Costes pueden ser significativos
				\4 Autorizaciones para sacar capitales
				\4 Impuestos a entrada de capitales
				\4 Obtención de licencias de inversión
				\4 Aumento de corrupción
				\4 Capitales acaban fluyendo de todas formas
				\4 Controles de capital sólo sirve para aumentar costes
				\4 Ejemplo:
				\4[] Bitcoin en 2010s en China y Venezuela
				\4[] Salida de capital vía facturación falsa
			\3 Efecto disciplina
				\4 Salida de capital de c/p tienen efectos inmediatos
				\4[] Aumento de coste de financiación deuda pública
				\4[] Depreciación de TCN
				\4[] Caída de reservas
				\4 Efectos adversos potenciales
				\4[] Aumenta coste de financiación
				\4[] Aumenta precio de bienes importados
				\4[$\then$] Incentivos a políticas macro estables
				\4[] Contención del gasto en fase expansiva
				\4[] Evitar financiación monetaria de déficits
				\4[] Evitar endeudamiento excesivo
			\3 Aumento del comercio
				\4 Posible financiar déficits y superávits
				\4 Saldos de CC $\neq 0$ son ahora más baratos
				\4[$\then$] Incentivo a libre comercio
			\3 Aumento de productividad media
				\4 Helpman, Melitz y Yeaple (2004)
				\4 Costes de transporte y aranceles
				\4[] Afectan a todas las empresas por igual
				\4 Economías de escala
				\4[] Permiten compensar costes de transporte y protección
				\4[] $\to$ Favorecen exportación
				\4[] No todas las empresas realizan EEscala hasta rentabilidad
				\4 Inversión directa extranjera horizontal
				\4[] Replicación de procesos productivos en mercado de destino
				\4[] No realiza EEscala
				\4[] Evita costes de transporte y aranceles
				\4 Empresas más productivas
				\4[] Pueden permitirse replicación e IDE horizontal
				\4[] $\to$ Liberalización CF permite entrada de IDE
				\4[] $\then$ IDE origen en empresas más productivas
				\4[] $\then$ Aumento de productividad media
		\2 Argumentos en contra
			\3 Trilema de Mundell y Fleming
				\4 PM independiente y liberalización CF
				\4[] $\to$ Sólo posible con TCN flexible
				\4 Con TCN Fijo
				\4[] Eventualmente insostenible si PM exógena
			\3 Exposición a ciclo financiero global
				\4 Rey (2015)
				\4 No hay trilema sino dilema
				\4 Opciones:
				\4[] CF abierta + PM dependiente de EEUU
				\4[] CF cerrada + PM independiente
				\4 CF abierta
				\4[] Estados Unidos fija tipos de interés
				\4[] Inicia ciclo financiero global
				\4[] Si baja tipos de interés
				\4[] $\to$ Capital fluye hacia emergentes
				\4[] $\to$ Aumento de inversión en emergentes
				\4[] $\then$ Apreciación de TCN
				\4[] $\then$ Caída de interés en emergentes
				\4[] $\then$ Aumento de déficits CC en emergentes
				\4[] PM se hace dependiente de americana
				\4[] $\to$ Si más contractiva
				\4[] $\then$ Fuerte apreciación del TCN
				\4[] $\then$ Pérdida de competitividad
				\4[] $\to$ Si desea evitar impacto de $\Delta i$ en EEUU
				\4[] $\then$ Pierde independencia
				\4 Única opción para mantener PM independiente
				\4[] Cuenta de capital cerrada
			\3 Sudden stops y reversiones de flujos de capital
				\4 Ocurren relativamente frecuentemente
				\4 Especialmente en países en desarrollo/emergentes
				\4 Persisten al menos un año, generalmente
				\4 Sudden stop y flow reversal al tiempo
				\4 Inducen depreciación del tipo de de cambio
				\4[] No quedan otras herramientas de ajuste disponibles
				\4 Inducen caídas fuertes del PIB via $\downarrow$ I
				\4 Libre movimiento de capital
				\4[] Venta de pasivos nacionales es menos costosa
				\4 Préstamos de corto plazo
				\4[] Prestamistas pueden inducir sudden-stop
				\4[] $\to$ Simplemente evitando renovación de préstamos
				\4 Endeudamiento en moneda extranjera
				\4[] Banco central
				\4[] $\to$ No puede proveer liquidez
				\4[] $\to$ No puede monetizar deuda
				\4 Pequeño sector exportador
				\4[] Si flujos de capital se revierten
				\4[] $\to$ Necesario aumentar exportaciones
				\4[] Si sector exportador es pequeño
				\4[] $\to$ Necesario reorganizar producción
				\4[] $\then$ Muy costoso
				\4 Aumento de percepciones globales del riesgo
				\4[] Capital se desplaza hacia activos percibidos como seguros
				\4 TCN fijo + libre movimiento de K
				\4[] Vulnerabilidad clásica
				\4[] Incentiva ataques especulativos de primera generación
				\4 Stock de reservas pequeño
				\4[] Asiáticos aprenden lección tras crisis de 90s
			\3 Imposible represión financiera
				\4 Represión financiera
				\4[] Fijación administrativa de tipos de interés
				\4[] Represión de inversión fuera de cauces determinados
				\4[] Canalización de ahorro privado hacia SPúblico
				\4[] $\to$ Reducción de coste de financiación
				\4[] $\to$ Pago de deuda
				\4 Con liberalización de cuenta financiera
				\4[] Posible sacar capital de economía
				\4[] $\to$ Para obtener mayores rendimientos
				\4[] $\then$ Imposible canalizar ahorro
			\3 Hot money\footnote{Ver hot money en Palgrave.}
				\4 Flujos de capital de gran cuantía bajo TCFijo
				\4[] $\to$ Previsión de devaluación inminente
				\4[] $\to$ Diferenciales de interés por encima de riesgo cambiario
				\4 Potencial desestabilizador
				\4[] Demostrado especialmente en 3ª gen.
				\4 Desincentivar hot money
				\4[] Exit taxes
				\4[] Restricciones temporales de horizonte de inversión
			\3 Intolerancia a la deuda externa\footnote{Ver en \href{https://www.nber.org/papers/w9908}{Reinhart, Rogoff, Savastano (2003)} el concepto original.}
				\4 Reinhart, Rogoff y Savastano (2003)
				\4 Fenómeno recurrente en últimos dos siglos
				\4[] Países impagan deuda de manera repetida
				\4[] $\to$ Doméstica y exterior
				\4[] $\then$ A pesar de niveles relativamente bajos
				\4 Empíricamente, existen umbrales seguros
				\4[] Niveles máximos de endeudamiento
				\4[] $\to$ Que economía puede soportar sin pagar
				\4 Factores que determinan umbral de tolerancia
				\4[] Factores de economía política
				\4[] Path-dependency
				\4[] Sistemas fiscales débiles
				\4[] $\to$ Baja capacidad recaudatoria
				\4[] Sistemas financieros inestables
				\4[] $\to$ Riesgo moral endeudamiento exterior
				\4[] $\to$ Supervisión deficiente
				\4 En PEDs más bajo que en desarrollados
				\4[] $\to$ En algunos casos, apenas 15\% de PIB
				\4 Apertura de cuenta financiera
				\4[] Potencial aumento exterior
				\4[] $\to$ Aumenta probabilidad de superar umbral
				\4[] $\then$ Impago
				\4[] $\then$ Ajuste brusco
				\4[] $\then$ Recesión, desempleo, inflación
				\4[] $\then$ Mercados de capital
		\2 Proceso de liberalización
			\3 Liberalización de cuenta comercial
				\4 Generalmente preferible antes de lib. CF
				\4 Efecto Brecher-Díaz Alejandro
				\4[] Cuenta comercial cerrada+libre CF
				\4[] Posible entrada de capitales
				\4[] $\to$ Presión hacia apreciación del TCN
				\4[] Aranceles elevados en industrias protegidas
				\4[] $\to$ Aumento de K exterior hacia industria protegida
				\4[] $\to$ Fuerte aumento de producción
				\4[] $\then$ Distorsión de producción nacional
				\4[] $\then$ Exceso de producción de bien protegido
				\4[] Remuneración a capital extranjero
				\4[] $\to$ Por venta de bien protegido en mercado nacional
				\4[] $\then$ Posible retirada de beneficios
				\4[] $\then$ Salida de capital masiva posterior
				\4[] $\then$ Enorme efecto empobrecedor
			\3 Regulación del sector financiero nacional
				\4 Evitar fallos de mercado habituales en sist.fin
				\4[] i. Exceso de riesgo
				\4[] ii. Malas inversiones
				\4[] iii. Externalidades
				\4[] iv. Múltiples equilibrios
				\4[] v. Selección adversa y riesgo moral
				\4 Mercados oficiales transparentes
				\4 Marco de resolución
				\4 Mecanismos de seguridad ante crisis
				\4 Regulación macroprudencial
				\4[] Restricciones LTV-ratio
				\4[] Reservas mínimas en divisas
				\4[] Provisiones dinámicas
				\4[] Capital pro-cíclico
				\4[] Capital para SIB
			\3 Liberalización del sistema financiero doméstico
				\4 Permitir asignación de capital entrantes
				\4 Aumento de competencia
				\4 Fusión y desaparición entidades inviables
				\4 Realización de economías de escala
				\4 Facilitar reasignación de recursos entre sectores
				\4 Marco monetario estable
				\4[] Permitir transmisión de política monetario
				\4[] Contener inflación y tipos de interés
				\4[] $\to$ Evitar movimientos bruscos al abrir CF
				\4[] $\to$ Deseable independencia del Banco Central
			\3 Enfoque gradualista
				\4 Evitar:
				\4[] Ajustes bruscos
				\4[] Presión sobre tipo de cambio
				\4[] Hot money
				\4[] Endeudamiento de sector no comerciables
				\4 Experiencia de liberalización en América Lat 80
				\4[] Aperturas bruscas
				\4[] Exceso de endeudamiento
				\4[] Enorme entrada de capitales
				\4[] Burbujas especulativas en muchos sectores
				\4 Flujos de capital de largo plazo
				\4[] IED principalmente
				\4[] Posible limitar a equity
				\4 Flujos de capital de corto plazo
				\4[] Impuestos
				\4[] Restricciones
				\4[] Coste de control sobre c/p aumenta
			\3 Enfoque de shock
				\4 Liberalización rápida
				\4 Evitar:
				\4[] Dudas sobre liberalización futura
				\4[] Inversores extranjeros esperen vuelta atrás
				\4[] $\to$ Capitales queden atrapados
				\4 Aprovechar ventanas de oportunidad ec. política
			\3 Tipo de cambio
				\4 Receta ortodoxa
				\4[] O hard peg
				\4[] $\to$ Unión monetaria
				\4[] $\to$ Dolarización
				\4[] $\to$ Junta de conversión
				\4[] O tipo flexible
				\4 Evitar
				\4[] Crisis cambiarias de 1, 2, 3 generación
				\4 Movimientos de capital aumentan posibilidad
				\4 Trilema de Tinbergen
				\4[] Necesario un instrumento para cada objetivo
				\4[] Un instrumento para dos objetivos distintos
				\4[] $\to$ Acaba provocando resultados indeseables
				\4[] Política monetaria para mantener TCN estable
				\4[] $\to$ No permite utilizar para equilibrio interno
			\3 Instituciones
				\4 Independencia de banco central
				\4[] Tender a inflación estable
				\4[] Evitar shocks de interés bruscos
				\4[] Reducir influencias de ciclo político
			\3 Problemas de implementación
				\4 Cuellos de botella en no comerciables
				\4[] Generalmente, sin acceso a financiación exterior directa
				\4[] Obtienen financiación intermediada vía bancos
				\4 Economía política
				\4[] Diferentes grupos se benefician de diferentes políticas
				\4 Ciclo financiero internacional
				\4[] Tipos de interés internacionales
				\4[] $\to$ Determinarán signo de flujos netos
		\2 Análisis empírico
			\3 Medidas de liberalización
				\4 De jure
				\4[] Basadas en:
				\4[] $\to$ cambios legislativos
				\4[] $\to$ Fecha efectiva de posibilidad de inversión
				\4[] Problema:
				\4[] $\to$ Lags en implementación efectiva
				\4[] $\to$ Ocasionalmente, sólo de iure sin implementación
				\4 De facto
				\4[] Medidas basadas en consecuencias efectivas de lib.
				\4[] Pro ejemplo:
				\4[] $\to$ Flujos de capital vs GDP
				\4[] Problema:
				\4[] $\to$ Sujetas a fluctuaciones cíclicas
			\3 Problemas de medición
				\4 Cómo cuantificar liberalización
				\4[] Variables dummy arbitrarias
				\4[] Índices sintéticos
				\4[] Equity disponible para inversores extranjeros
				\4[] Correlación entre precios de activos
				\4[] ...
				\4 Causalidad de la liberalización
				\4[] Crecimiento induce liberalización?
				\4[] Liberalización induce crecimiento
				\4[] Necesarias variables instrumentales
				\4[] $\to$ No siempre fáciles de encontrar
			\3 Estabilidad monetaria
				\4 Correlación estabilidad-liberalización
				\4 Países con menor inflación y menos señoreaje
				\4[] Más probable liberalicen cuenta de capital
				\4 Difícil establecimiento de causalidad
			\3 Crecimiento
				\4 Evidencia empírica mixta
				\4[] Depende de tipo de país
				\4 Países de ingreso medio
				\4[] Liberalización aumenta crecimiento
				\4[] $\to$ Productividad
				\4[] $\to$ Inversión
				\4[] $\then$ Más crecimiento que sin liberalización
				\4 Países de ingreso bajo
				\4[] Liberalización financiera apenas tiene efecto
				\4[] Sistema financiero poco desarrollado
				\4[] No es capaz de captar financiación exterior
			\3 Hot money\footnote{Ver tema 3B-16, II. Evidencia empírica}
				\4 Idea clave
				\4[] Movimientos de capital bruscos y grandes
				\4[] Expectativa de apreciación/devaluación
				\4[] Contexto de tipo fijo
				\4 Implicaciones
				\4[] Regímenes soft peg
				\4[] $\to$ Sujetos a flujos de capital
				\4[] $\then$ Expectativas de relevantes
				\4[] Efectos sobre economía real doméstica
				\4[] $\to$ Inestabilidad de precios
				\4[] $\to$ Cambios bruscos en tipos de interés
				\4[] $\to$ Inestabilidad financiera
				\4 Valoración
				\4[] Relativamente raros en patrón oro
				\4[] $\to$ Confianza en estabilidad de tipo de cambio
				\4[] Entreguerras más frecuentes
				\4[] $\to$ Enorme entrada de K en Francia 26-28
				\4[] $\then$ Expectativa de revaluación
				\4[] $\to$ Salidas de oro desde bloque oro en años 30
				\4[] $\then$ Expectativa de ruptura de paridad con oro
				\4[] Bretton-Woods, años finales
				\4[] $\to$ Aparecen de nuevo los movimientos hot money
				\4[] $\to$ Preceden devaluaciones/revaluaciones
				\4[] 1973, rumores fin de Smithsonian Agreement
				\4[] $\to$ Estados Unidos: control de rentas y tipos bajos
				\4[] $\to$ Bundesbank: contracción monetaria
				\4[] $\then$ Hot money flows salen de EEUU hacia Alemania
				\4[] $\then$ Expectativa de devaluación de dolar
			\3 Crisis financieras
				\4 Evidencia empírica muestra relación clara
				\4[] Liberalización financiera
				\4[] $\to$ Aumenta frecuencia de crisis
			\3 Controles a entradas de flujos de corto plazo
				\4 Ejemplo paradigmático: Chile
			\3 Liberalización previa al euro: SME duro
				\4 Resultados
				\4[] Crisis de segunda generación
				\4[] Múltiples equilibrios posibles
				\4[] Expectativas determinan
			\3 Crisis sudeste asiático
				\4 Liberalización+TCN fijo+problemas sist. bancario
			\3 Postura del FMI
				\4 Postura tradicional
				\4[] Evitar controles de capital
				\4[] Recomienda liberalización
				\4 Cambio en últimas décadas
				\4[] Especialmente tras experiencia crisis asiática
				\4[] Controles de capital pueden ser útiles
				\4[] $\to$ Crisis cambiarias
				\4[] $\to$ Crisis financieras
				\4[] Pero deben ser transitorios
	\1 \marcar{Economía política del sector exterior}
		\2 Idea clave
		\2 Liberalización y democracia
		\2 Economía política de los aranceles
			\3 Idea clave
				\4 Contexto
				\4[] Economía política
				\4[] $\to$ Análisis de efectos de política económica
				\4[] $\then$ Sobre intereses de diferentes grupos sociales
				\4[] $\then$ Como resultado de intereses de diferentes grupos
				\4[] Efectos de política comercial
				\4[] $\to$ Afectan distinto a diferentes sectores
				\4 Objetivo
				\4[] Caracterizar efectos sobre diferentes sectores
				\4[] Entender impacto de estructura política sobre pol. arancelaria
				\4 Resultados
				\4[] Efectos de aranceles sobre diferentes grupos sociales
				\4[] $\to$ Beneficios y perjuicios
				\4[] $\to$ Diferentes grados de concentración
				\4[] $\to$ Diferente capacidad de respuesta
			\3 Stolper-Samuelson
				\4 En contexto Heckscher-Ohlin
				\4 Tras apertura comercial
				\4[] $\to$ Factor intensivo de sector de especialización
				\4[] $\then$ Aumenta pago al factor
				\4[] $\to$ Factor intensivo de sector que pierde producción
				\4[] $\then$ Coste de factores cae
				\4 Sector de factor intensivo en bien de especialización
				\4[] $\then$ Presión hacia reducción de aranceles
				\4 Sector de factor intensivo en bien que pierde producción
				\4[] $\then$ Presión hacia mantenimiento de aranceles
				\4 Países ricos
				\4[] Abundantes en capital
				\4[] $\to$ Capital gana con apertura
				\4[] Trabajo escaso
				\4[] $\to$ Compite con trabajo extranjero
				\4[] $\to$ Pierde con apertura
				\4[$\then$] Trabajo se opone a apertura
				\4 Países pobres
				\4[] Abundantes en trabajo
				\4[] $\to$ Con apertura venden al mundo
				\4[] Capital escaso
				\4[] $\to$ Compiten con capital extranjero
				\4[$\then$] Trabajo favorable a apertura

			\3 Redistribución de beneficios del comercio
				\4 Permite a perdedores aceptar reducción de aranceles
				\4 Pero costes de redistribución
				\4[] $\to$ Negociación entre sectores
				\4[] $\to$ Votaciones
				\4[] $\to$ Adquisición de información
				\4[] $\then$ Posible no sea rentable redistribuir
			\3 Modelo de factores específicos
				\4 Dos factores de capital inmóviles
				\4 Desarme arancelario mutuo
				\4[] $\to$ Aumenta beneficios nuevos exportadores
				\4[] $\to$ Reduce beneficio en sectores que ahora importan
				\4[] $\then$ Flujo de trabajo de un sector a otro
				\4[] $\then$ Caída de PMgK en sector perjudicado
				\4[] Diferentes intereses dentro de un mismo factor
				\4[] $\to$ Capital vs trabajo no siempre oposición homogénea
			\3 Aversión a la pérdida
				\4 Behavioral economics
				\4[] Empíricamente, aversión a pérdida mayor que ganancia
				\4 Apertura arancelaria
				\4[] $\to$ Induce beneficio en un sector
				\4[] $\to$ Aumenta pérdidas en otro
				\4 Si aversión a pérdida mayor que ganancia por beneficio
				\4[] $\then$ Oposición más fuerte
			\3 Aversión a incertidumbre
				\4 Apertura aumenta incertidumbre
				\4[] $\to$ ¿Efectos de equilibrio general serán positivos?
			\3 Aversión a desigualdad
				\4 Apertura al comercio puede aumentar desigualdad
				\4[] $\to$ Sector de especialización más rico
				\4[] $\to$ Sector que reduce producción más pobre
				\4 Seres humanos muestran cierta aversión a la desigualdad
				\4[] $\to$ Factor de oposición a apertura
			\3 Concentración de intereses
				\4 Efectos de reducción arancelaria
				\4[] $\to$ Difusos sobre consumidores
				\4[] $\to$ Muy concentrados sobre industria desprotegida
				\4 Perjuicio concentrado
				\4[] $\to$ Facilita coordinación entre perjudicados
				\4[] $\then$ Facilita oposición política a apertura
			\3 Instituciones multilaterales pueden catalizar
				\4 Commitment liberalizador
				\4[] Aumenta poder de negociación de liberalizadores
			\3 Redistribución puede ser necesaria
				\4 Mejora aceptación de apertura
				\4[] También es costosa
			\3 Valoración
				\4 Programa de investigación con muchas vertientes
				\4 Interacciones con sociología, ciencia política, demografía..
				\4 Ciencia económica no siempre ha examinado
				\4[] Supuestos demasiado fuertes
				\4[] $\to$ ¿Planificador social?
				\4[] $\to$ ¿Funciones de bienestar social?
				\4[] $\then$ ¿Realmente existen?
				\4[] $\then$ ¿Realmente consideradas en decisiones de PComercial?
		\2 Trilema de Rodrik
			\3 Idea clave
				\4 Contexto
				\4[] Sistema de Bretton Woods
				\4[] $\to$ Relativa estabilidad política en occidente
				\4[] $\to$ Relativa estabilidad financiera
				\4[] $\to$ Políticas keynesianas en muchos países
				\4[] Caída de Bretton Woods
				\4[] Liberalización progresiva/rápida de flujos de capital
				\4[] Globalización de comercio y capitales
				\4[] $\to$ Aumento de sincronización cíclica
				\4[] $\to$ Aumento exposición a shocks externos
				\4 Objetivos
				\4[]
				\4 Resultados
				\4[] Imposible tres características al mismo tiempo
				\4[] $\to$ Apertura plena al exterior
				\4[] $\to$ Soberanía nacional
				\4[] $\to$ Democracia representativa nacional
			\3 Formulación
				\4 Imposible compatibilizar tres características
				\4[i] Democracia representativa
				\4[] Mayorías que acaban exigiendo mayor redistribución de la renta
				\4[] Trabajo organizado
				\4[ii] Soberanía nacional
				\4[] Capaz de decidir sus propios mix de políticas
				\4[iii] Libre comercio y mov. de K con exterior
				\4 Con (i) y (ii), imposible añadir (iii):
				\4[] Apertura al exterior obliga a:
				\4[] $\to$ Flexibilizar economía para mantener competitividad
				\4[] $\to$ Acomodar efectos de flujos de K sobre TCR
				\4[] $\then$ Aumento de tensiones sociales, sindicatos, etc...
				\4[] $\then$ Aparecen presiones proteccionistas
				\4[] $\then$ Presión exterior para tomar medidas de ajuste
				\4[] $\then$ Cae democracia
				\4 Con (i) y (iii), imposible añadir (ii):
				\4[] Soberanía nacional permite:
				\4[] $\to$ Aumento de programas de gasto
				\4[] $\to$ Medidas que aumentan poder de sindicatos e insiders
				\4[] $\to$ Medidas de protección de mercado nacional
				\4[] $\then$ Fin del libre comercio
				\4[] $\then$ Fugas de capital inducen controles de capital
				\4[] $\then$ Presiones para aumentar protección de mercado nacional
				\4 Con (ii) y (iii), imposible añadir (i)
				\4[] Democracia representativa permite:
				\4[] $\to$ Mayor presión de grupos de población
				\4[] $\to$ Mayor representación de grupos perjudicados por apertura
				\4[] $\to$ Mayores presiones para redistribución rentas
				\4[] $\then$ Aumento del gasto público
				\4[] $\then$ Salidas de capital si deterioro sostenib. exterior
			\3 Implicaciones
				\4
			\3 Valoración
	\1[] \marcar{Conclusión}
		\2 Recapitulación
			\3 Liberalización de la cuenta corriente
			\3 Liberalización de la cuenta financiera
			\3 Economía política del sector exterior
		\2 Idea final
\end{esquemal}










































\preguntas

\notas

\bibliografia

Mirar en Palgrave:
\begin{itemize}
	\item capital controls
	\item dual track liberalization
	\item financial liberalization
	\item foreign direct investment
	\item international capital flows
	\item international migration
\end{itemize}

Calvo, G. A.; Leiderman, L.; Reinhart, C. \textit{Inflows of Capital to Developing Countries in the 1990s} (1996) Journal of Economic Perspectives -- En carpeta del tema

Dornbusch, R. \textit{Expectations and Exchange Rate Dynamics} (1976) Journal of Political Economy -- En carpeta del tema

Eichengreen, B. (2001) \textit{Capital Account Liberalization: What Do Cross-Country Studies Tell Us?} The World Bank Economic Review, Vol. 15, No. 3 341-365 -- En carpeta del tema

Grossman, G. M.; Helpman, E. (1989) \textit{Comparative Advantage and Long-Run Growth} American Economic Review -- En carpeta del tema.

IMF (2000) \textit{Capital Controls: Country Experiences with Their Use and Liberalization} IMF Occasional Paper 190 \href{https://www.imf.org/external/pubs/ft/op/op190/index.htm}{Disponible aquí} -- En carpeta del tema

Kose, M. A.; Prasad, E.; Rogoff, K.; Wie, S-J. (2006) \textit{Financial Globalization: A Reappraisal} NBER Working Paper Series -- En carpeta del tema

Popov, V. (2006) \textit{Shock therapy versus gradualism reconsidered: lessons from transition economies after 15 years of reforms} TIGER Working Paper Series, No. 82 -- En carpeta del tema


Rey, H. (2018) \textit{Dilemma not trilemma: the global financial cycle and monetary policy independence} NBER Working Paper Series \href{https://www.nber.org/papers/w21162.pdf}{Disponible aquí} -- En carpeta del tema

Taylor, M. P. (1995) \textit{The Economics of Exchange Rates} Journal of Economic Literature Vol. XXXIII -- En carpeta del tema

\end{document}
