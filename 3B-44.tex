\documentclass{nuevotema}

\tema{3B-44}
\titulo{Las relaciones económicas exteriores de la Unión Europea. La política de cooperación al desarrollo de la Unión}

\begin{document}

\ideaclave

Ver \href{https://www.consilium.europa.eu/en/press/press-releases/2020/04/09/report-on-the-comprehensive-economic-policy-response-to-the-covid-19-pandemic/}{Comunicado del Eurogrupo del 10 de abril de 2020} sobre medidas a adoptar al respecto de la crisis del COVID-19.

Ver págs. 23-24 de IMF (2019) ESR sobre ajuste de desequilibrios externos e intra-zona euro tras la Crisis Financiera Global.

ESTE TEMA HAY QUE COMPLEMENTARLO CON EL 4A-22, CONCRETAMENTE APARTADO 6.2.2 DE CECO NUEVO.

\seccion{Preguntas clave}
\begin{itemize}
	\item ¿Qué marco institucional configura las relaciones económicas exteriores de la UE?
	\item ¿Qué estructura tiene el comercio exterior de la UE?
	\item ¿Qué estructura tiene la inversión exterior de la UE?
	\item ¿En qué consiste la política de cooperación al desarrollo de la UE?
\end{itemize}

\esquemacorto

\begin{esquema}[enumerate]
	\1[] \marcar{Introducción}
		\2 Contextualización
			\3 Unión Europea
			\3 Competencias de la UE
			\3 Relaciones económicas exteriores
			\3 Factores determinantes de rel. ec. exteriores
			\3 Importancia de UE en economía mundial
			\3 Política exterior de UE en materia económica
			\3 Cooperación al desarrollo
		\2 Objeto
			\3 ¿Qué marco institucional configura rel. econ. exteriores?
			\3 ¿Qué estructura tiene el comercio exterior europeo?
			\3 ¿Qué estructura tiene la inversión exterior de la UE?
			\3 ¿En qué consiste la política de cooperación al desarrollo de la UE?
		\2 Estructura
			\3 Marco institucional
			\3 Estructura del sector exterior de la UE
			\3 Política de cooperación al desarrollo
	\1 \marcar{Marco institucional}
		\2 Organizaciones multilaterales
			\3 Idea clave
			\3 G20
			\3 G8
			\3 OECD
			\3 ONU
			\3 FMI
			\3 Banco Mundial
			\3 Otros
		\2 Fondos de Actuaciones Exteriores 2014-2020
			\3 Idea clave
			\3 DCI -- Instrumento de Cooperación al Desarrollo
			\3 ENI -- Instrumento Europeo de Vecindad
			\3 IPA -- Instrumento de Ayuda a la Preadhesión II
			\3 IcSP -- Instrumento para la Estabilidad y la Paz
			\3 EIDHR -- Instrumento Europeo para Defensa de Derechos Humanos
			\3 IP -- Instrumento de Asociación
		\2 PESC -- Política Exterior y de Seguridad Común
			\3 Idea clave
			\3 Actuaciones exteriores
		\2 Desarrollo regional y vecindad
			\3 Idea clave
			\3 Política Europea de Vecindad
			\3 Instrumento de Ayuda a la Preadhesión -- IPA
			\3 Unión por el Mediterraneo
			\3 EaP -- Eastern Partnership
		\2 Sanciones y embargos de la UE
			\3 Idea clave
			\3 Actuaciones
		\2 PAC -- Política Agrícola Común
			\3 Idea clave
			\3 Actuaciones exteriores
		\2 PPC -- Política de Pesca Común
			\3 Idea clave
			\3 Actuaciones exteriores
		\2 Política comercial autónoma
			\3 Idea clave
			\3 Actuaciones exteriores
			\3 Reglamento de obstáculos al comercio
			\3 Comité de Medidas de Defensa comercial
		\2 Política comercial convencional
			\3 GATT-47 y 94
			\3 GATS
			\3 TRIPS
			\3 Ronda de Doha
			\3 Conferencias ministeriales
			\3 ITA -- Tecnologías de la Información
			\3 EGA -- Bienes medioambientales
			\3 TiSA -- Comercio de servicios
			\3 GPA -- Contratación pública
			\3 Europa
			\3 MENA
			\3 África
			\3 América
			\3 Asia
			\3 Pacífico
	\1 \marcar{Estructura del sector exterior de la UE}
		\2 Idea clave
			\3 Economía de la UE
			\3 Tensiones internacionales
		\2 Comercio
			\3 Evolución
			\3 Bienes
			\3 Servicios
			\3 Retos
		\2 Inversión
			\3 Stocks (2017)
			\3 Flujos (2017)
			\3 Empresas extranjeras en UE
	\1 \marcar{Política de cooperación al desarrollo}
		\2 Justificación
			\3 Motivos altruistas y humanitarios
			\3 Acceso a mercados
			\3 Geoestratégicos
			\3 Desarrollo económico de socios
		\2 Antecedentes
			\3 Post-descolonización
			\3 Años 70 y 80
			\3 Institucionalización en 90s
			\3 Tratado de Lisboa
		\2 Objetivos
			\3 Coordinar EEMM
			\3 Fomentar comercio y desarrollo
			\3 Mejorar diseño de políticas en PEDs
			\3 Aliviar crisis humanitarias
		\2 Marco jurídico
			\3 Programas de coop. al desarrollo
			\3 Instrumentos unilaterales
			\3 Acuerdos económicos bilaterales
			\3 EuropeAid -- DG Cooperación Internacional y Desarrollo
		\2 Marco financiero
			\3 Instrumento de Financiación de la Cooperación al Desarrollo
			\3 Fondo Europeo de Desarrollo
			\3 Banco Europeo de Inversiones
		\2 Actuaciones
			\3 \underline{Preferencias generalizadas}
			\3[] SPG general
			\3 SPG+
			\3 EBA -- Everything But Arms
			\3 Waiver de servicios
			\3 \underline{Acuerdos EPA con ACP}
			\3[] \underline{Ayuda Oficial al Desarrollo}
			\3 Idea clave
			\3 Formas de ayuda
			\3 Distribución geográfica
			\3 Distribución sectorial
			\3 FED -- Fondo Europeo de Desarrollo
			\3 DCI -- Instrumento de Cooperación al Desarrollo
		\2 Valoración
			\3 Paradoja de la ayuda exterior
			\3 Sistema de Preferencias generalizadas
		\2 Retos
			\3 Incentivos perversos
			\3 Mala asignación del capital
	\1[] \marcar{Conclusión}
		\2 Recapitulación
			\3 Marco institucional
			\3 Relaciones económicas exteriores
			\3 Política de cooperación al desarrollo
		\2 Idea final
			\3 Valoración
			\3 Retos

\end{esquema}

\esquemalargo















\begin{esquemal}
	\1[] \marcar{Introducción}
		\2 Contextualización
			\3 Unión Europea
				\4 Institución supranacional ad-hoc
				\4[] Diferente de otras instituciones internacionales
				\4[] Medio camino entre:
				\4[] $\to$ Federación
				\4[] $\to$ Confederación
				\4[] $\to$ Alianza de estados-nación
				\4 Origen de la UE
				\4[] Tras dos guerras mundiales en tres décadas
				\4[] $\to$ Cientos de millones de muertos
				\4[] $\to$ Destrucción económica
				\4[] Marco de integración entre naciones y pueblos
				\4[] $\to$ Evitar nuevas guerras
				\4[] $\to$ Maximizar prosperidad económica
				\4[] $\to$ Frenar expansión soviética
				\4 Objetivos de la UE
				\4[] TUE -- Tratado de la Unión Europea
				\4[] $\to$ Primera versión: Maastricht 91 $\to$ 93
				\4[] $\to$ Última reforma: Lisboa 2007 $\to$ 2009
				\4[] Artículo 3
				\4[] $\to$ Promover la paz y el bienestar
				\4[] $\to$ Área de seguridad, paz y justicia s/ fronteras internas
				\4[] $\to$ Mercado interior
				\4[] $\to$ Crecimiento económico y estabilidad de precios
				\4[] $\to$ Economía social de mercado
				\4[] $\to$ Pleno empleo
				\4[] $\to$ Protección del medio ambiente
				\4[] $\to$ Diversidad cultural y lingüistica
				\4[] $\to$ Unión Económica y Monetaria con €
				\4[] $\to$ Promoción de valores europeos
				\4[$\to$] Objetivos de la UE
				\4[] Paz y bienestar a pueblos de Europa
			\3 Competencias de la UE
				\4 Tratado de la Unión Europea
				\4[] Atribución
				\4[] $\to$ Sólo las que estén atribuidas a la UE
				\4[] Subsidiariedad
				\4[] $\to$ Si no puede hacerse mejor por EEMM y regiones
				\4[] Proporcionalidad
				\4[] $\to$ Sólo en la medida de lo necesario para objetivos
				\4 Exclusivas
				\4[] i. Política comercial común
				\4[] ii. Política monetaria de la UEM
				\4[] iii. Unión Aduanera
				\4[] iv. Competencia para el mercado interior
				\4[] v. Conservación recursos biológicos en PPC
				\4 Compartidas
				\4[] i. Mercado interior
				\4[] ii. Política social
				\4[] iii. Cohesión económica, social y territorial
				\4[] iv. Agricultura y pesca \footnote{Salvo en lo relativo a la conservación de recursos biológicos marinos, que se trata de una competencia exclusiva de la UE}
				\4[] v. Medio ambiente
				\4[] vi. Protección del consumidor
				\4[] vii. Transporte
				\4[] viii. Redes Trans-Europeas
				\4[] ix. Energía
				\4[] x. Área de libertad, seguridad y justicia
				\4[] xi. Salud pública común en lo definido en TFUE
				\4 De apoyo
				\4[] i. Protección y mejora de la salud humana
				\4[] ii. Industria
				\4[] iii. Cultura
				\4[] iv. Turismo
				\4[] v. Educación, formación profesional y juventud
				\4[] vi. Protección civil
				\4[] vii. Cooperación administrativa
				\4 Coordinación de políticas y competencias
				\4[] Política económica
				\4[] Políticas de empleo
				\4[] Política social
			\3 Relaciones económicas exteriores
				\4 Concepto
				\4[] Conjunto de:
				\4[] $\to$ Acuerdos internacionales
				\4[] $\to$ Flujos comerciales y financieros
				\4[] $\to$ Stocks de inversión inward y outward
				\4[] $\to$ Ayuda al desarrollo de PEDs
				\4[] $\then$ Entre miembros UE y exterior
				\4 Grado de apertura
				\4[] $\uparrow$ importancia de relaciones ec. ext.
			\3 Factores determinantes de rel. ec. exteriores
				\4 Coyuntura económica mundial
				\4 Crecimiento de países socios
				\4 Contexto institucional
				\4 Marco jurídico
				\4 Expectativas futuras
			\3 Importancia de UE en economía mundial
				\4 Mayor exportador e importador mundial
				\4 Mayor economía junto con EEUU
				\4 Mercado de 445 millones de personas
			\3 Política exterior de UE en materia económica
				\4 Instrumento fundamental del soberanía
				\4[] Pero parte de política de UE
				\4 UE asume parte relevante de PExterior de EEMM
				\4[] A pesar de no ser entidad soberana
				\4 Múltiples facetas
				\4[] Presencia en organismos
				\4[] Política comercial
				\4[] Seguridad
				\4 Influencia política
				\4 Diplomacia económica
			\3 Cooperación al desarrollo
				\4 Faceta particular de política económica exterior
				\4 Múltiples motivos para llevar a cabo
				\4 UE actor fundamental a nivel mundial
		\2 Objeto
			\3 ¿Qué marco institucional configura rel. econ. exteriores?
			\3 ¿Qué estructura tiene el comercio exterior europeo?
			\3 ¿Qué estructura tiene la inversión exterior de la UE?
			\3 ¿En qué consiste la política de cooperación al desarrollo de la UE?
		\2 Estructura
			\3 Marco institucional
			\3 Estructura del sector exterior de la UE
			\3 Política de cooperación al desarrollo
	\1 \marcar{Marco institucional}
		\2 Organizaciones multilaterales
			\3 Idea clave
				\4 UE representada en principales organismos
				\4 Tipo de representación
				\4[] Con voto
				\4[] $\to$ Áreas de competencia UE
				\4[] Como observadora
				\4[] $\to$ EEMM participan por separado
				\4[] $\to$ Áreas de competencia no atribuida a UE
			\3 G20
				\4 Representación por CE, BCE y Eurogrupo
				\4[] $\to$ Máximo nivel de representación
				\4 Presidente de CE y BCE
			\3 G8
				\4 Cede protagonismo a G20
				\4 Presidente de CE + Presidente Consejo Europeo
				\4[] $\to$ Con voz
			\3 OECD
				\4 Cuasimiembro
				\4[] $\to$ Delegación permanente
				\4[] $\to$ Diseño de trabajos y estrategias
				\4 Sin obligaciones presupuestarias
				\4 Sin voto
			\3 ONU
				\4 Carácter de observador
				\4 EEMM mantienen protagonismo
			\3 FMI
				\4 Coordinación de voto de dirs. ejecutivos
				\4 Participación de BCE como observador
				\4[] También en Directorio Ejecutivo
				\4[] $\to$ En algunas materias excepcionales
			\3 Banco Mundial
				\4 Coordinación de decisiones
				\4 Enfoque uniforme de estrategias desarrollo
			\3 Otros
				\4 FSB -- Financial Stability Board
				\4[] $\to$ BCE
				\4 BIS -- Bank of International Settlements
				\4[] $\to$ BCE y EEMM
		\2 Fondos de Actuaciones Exteriores 2014-2020
			\3 Idea clave
				\4 Instrumentos financieros de MFP 2014-2020
				\4 Objetivo común
				\4[] Proyectar influencia de UE
				\4[] Mejorar vínculos con vecinos
				\4[] Transmitir valores UE extranjero
				\4 Reglamento 236/2014
			\3 DCI -- Instrumento de Cooperación al Desarrollo
				\4 20.000 M de €
				\4 Reducción de pobreza
				\4 Geográficos, sectoriales y pan-africanos
				\4 Complemento FED
				\4[] Complemento
				\4 Más abajo explicación completa
			\3 ENI -- Instrumento Europeo de Vecindad
				\4 15.000 M de € MFP 2014-2020
				\4 Financiación de Política Europea de Vecindad
				\4 Cooperación transfronteriza
				\4 Extensión de programas UE más allá de fronteras
				\4[] Horizonte 2020
				\4[] Erasmus
				\4[] COSME
			\3 IPA -- Instrumento de Ayuda a la Preadhesión II
				\4 12.000 M de €
				\4 Asistencia técnica para adhesión
				\4 Adopción de acervo comunitario y normas técnicas
			\3 IcSP -- Instrumento para la Estabilidad y la Paz
				\4 2.000 M de € en MFP 2014-2020
				\4 Países que han sufrido crisis o guerras
				\4 Prevención de conflictos
			\3 EIDHR -- Instrumento Europeo para Defensa de Derechos Humanos
				\4 1.300 M de €
				\4 Democratización
				\4 Instituciones
				\4 Estado de Derecho
				\4 Especialmente en países en desarrollo
				\4[] Pero no exclusivamente
			\3 IP -- Instrumento de Asociación\footnote{\href{https://ec.europa.eu/fpi/what-we-do/partnership-instrument-advancing-eus-core-interests_en}{CE}}
				\4 1.000 M de €
				\4 Objetivos estratégicos de UE
				\4 Asuntos relacionados con PESC
				\4 Diplomacia europea
				\4 No es necesario acuerdo con país receptor
		\2 PESC -- Política Exterior y de Seguridad Común
			\3 Idea clave
				\4 Miembros UE tienen papel en seguridad
				\4 Unión Europea comienza a desarrollar rol propio
				\4 Tratado de Lisboa
				\4[] Desaparecen tres pilares tradicionales
				\4[] Configurar voz única de UE en el mundo
				\4 Objetivos
				\4[] Mantener paz en zonas de conflicto
				\4[] Reforzar seguridad internacional
				\4[] Fomentar respeto DDHH
			\3 Actuaciones exteriores
				\4 Negociaciones UE en zonas de conflicto
				\4 Política Común de Seguridad y Defensa
				\4[] Elemento integral de la PESC
				\4 Fomento de cooperación económica internacional
				\4 Alto Representante de Asuntos Exteriores
				\4[] Representa UE internacionalmente
				\4 Servicio Europeo de Acción Exterior
				\4[] Dependiente de ARPE
				\4[$\then$] Contenido económico transversal
				\4[$\then$] Especialmente en ámbito de diplomacia económica
		\2 Desarrollo regional y vecindad
			\3 Idea clave
				\4 Complemento a integración económica
				\4[] Dirigida a EEMM a priori
				\4[] Reducir diferencias de renta
				\4[] $\to$ Derivadas de integración
				\4 Extensión a países fuera de UE
				\4[] Preparar adhesión
				\4[] Mejorar relaciones de vecindad
			\3 Política Europea de Vecindad
				\4 Apoyo a procesos de reforma política y económica
				\4[] $\to$ En vecinos UE
				\4[] $\to$ ARG, ARM, AZE, BIE, EGY, GEO, ISR
				\4[] $\to$ JOR, LIB, LYB, MOL, MAR, PAL, SYR, TUN, UKR
				\4 Planes de acción IEV
				\4[] $\to$ Cada país acuerda con UE
				\4 15.400 millones de € para MFP 2014-2020
				\4 Condicionalidad
				\4[] Cumplimiento de programas de reforma
				\4[] Acceso a mercado
				\4 Críticas a PEV
				\4[] Enfoque top-down
				\4[] $\to$ UE dicta, socios cumplen
				\4[]
			\3 Instrumento de Ayuda a la Preadhesión -- IPA
				\4 Cinco aspectos fundamentales
				\4[] Cooperación transfronteriza
				\4[] Fortalecimiento de instituciones
				\4[] Desarrollo regional: transportes, MA
				\4[] Recursos humanos
				\4[] Desarrollo rural
				\4[$\then$] Reducir coste de adaptación tras adhesión
			\3 Unión por el Mediterraneo
				\4 Herededada de poceso de Barcelona de 90s
				\4 Proyectos conjuntos de obra pública
				\4 Movilización de fondos para préstamos privados
				\4 Facilitación de comercio
				\4 Movilidad trabajadores cualificados
				\4 CBC -- Cross Border Coperation
			\3 EaP -- Eastern Partnership
				\4 Países de Europa del Este
				\4[] Fuera de UE
				\4[] $\to$ Moldavia
				\4[] $\to$ Ucrania
				\4[] $\to$ Georgia
				\4[] $\to$ Bielorusia
				\4[] $\to$ Azerbaiyán
				\4[] $\to$ Armenia
				\4 Apoyo a adaptación a DCFTA
				\4[] En marcha para GEO, UKR, MOL
				\4 Sujetas a condicionalidad
				\4 Parte de política europea de vecindad
				\4 Apoyo a reforma política
				\4 Pendiente de renovación post-2020
%				\4 Documento de Estrategia-País
%				\4[] $\to$ Agenda de reformas y desarrollo
%				\4[] Documento de Estrategia-Multipaís
%				\4[] $\to$ Prioridades regionales
		\2 Sanciones y embargos de la UE
			\3 Idea clave
				\4 Comercio y flujos financieros
				\4[] Aumentan dependencia de países pequeños no-UE
				\4 Estados terceros incumplen derecho internacional público
				\4[] Ius cogens
				\4[] Convención Naciones Unidas
				\4[] Acuerdos Derechos Humanos
				\4[] Otros convenios
				\4[] Crímenes de guerra
				\4[] ...
				\4 Vetar vínculos comerciales y financieros
				\4[] Medida de presión a país infractor
				\4[] Reducir margen de acción de personas concretas
			\3 Actuaciones
				\4 Restricciones de entrada en UE
				\4 Prohibición de comerciar con empresas
				\4 Embargo de bienes en UE
				\4 Prohibición de operar con entidades financieras
		\2 PAC -- Política Agrícola Común
			\3 Idea clave
				\4 Política fundamental en UE
				\4[] Regular mercados agrícolas internos
				\4[] Fomentar desarrollo rural
				\4 Fuente de controversias y distorsiones
				\4[] Rondas de WTO
				\4[] Guerras comerciales y tensiones con EEUU
				\4[] Quejas de países en desarrollo
			\3 Actuaciones exteriores
				\4 Principio de preferencia comunitaria
				\4[] Trato favorable a agricultura UE
				\4[] Importante contencioso en negociaciones WTO
				\4 Apoyo a productores europeos
				\4[] Ayudas ligadas a producción
				\4[] $\to$ Excepcionales actualmente
				\4[] Ayudas desacopladas
				\4[] $\to$ No dependen de producción
				\4 Acuerdo agrícola de WTO
		\2 PPC -- Política de Pesca Común
			\3 Idea clave
				\4 UE es importador neto de pescado
				\4 Trata de acceder a recursos de pesca
			\3 Actuaciones exteriores
				\4 Acuerdos de pesca con terceros
				\4[] Permitir acceso de flota de EEMM
				\4 Numerosos países africanos
				\4[] Especialmente importantes:
				\4[] $\to$ Mauritania
				\4[] $\to$ Marruecos
				\4[] $\to$ Groenlandia
		\2 Política comercial autónoma
			\3 Idea clave
				\4 Medidas unilaterales de política comercial
				\4 Competencia exclusiva de UE
				\4[] Comisión propone y negocia acuerdos
			\3 Actuaciones exteriores
				\4 Arancel Aduanero Común
				\4 Regímenes económicos
				\4[] Franquicia arancelaria
				\4[] i. Perfeccionamiento activo
				\4[] ii. Perfeccionamiento activo
				\4[] iii. Importación temporal
				\4[] iv. Depósito aduanero
				\4[] v. Transformación bajo control aduanero
				\4 Regímenes comerciales
				\4[] Vigilancia
				\4[] Certificación
				\4[] Autorización
				\4 Exenciones y franquicias arancelarias
			\3 Reglamento de obstáculos al comercio
				\4 Reglamento de la UE 3286/94
				\4 Objetivo
				\4[] Denunciar ante instituciones UE
				\4[] Obstáculos comerciales en 3os países
				\4 Procedimiento
				\4[] 1. Empresas/sector/EMiembro denuncia ante CE
				\4[] 2. CE inicia consultas bilaterales
				\4[] $\to$ Revisión de medidas proteccionistas
				\4[] 3. Aprobación de medidas de retorsión
				\4[] $\to$ Posible denuncia ante OSD
			\3 Comité de Medidas de Defensa comercial
				\4 Defensa comercial
				\4[] Salvaguardias
				\4[] Antidumping
				\4[] Antisubvención
				\4 Asiste CE en implementación de medidas de defensa comercial
				\4 Presidido por representante de Comisión
				\4 Representantes de todos los EEMM
				\4 Medidas sobre las que opina:
				\4[] Imponer o medidas definitivas o provisionales
				\4[] Iniciar revisiones de caducidad
				\4[] Modificación de medidas existentes
		\2 Política comercial convencional
			\3 GATT-47 y 94
				\4 Acuerdo multilateral de OMC
				\4[] $\to$ UE es parte del acuerdo
			\3 GATS
				\4 Acuerdo multilateral de OMC
				\4[] $\to$ UE es parte del acuerdo
				\4 Importante impulsor del GATS
			\3 TRIPS
				\4 Acuerdo Multilateral de OMC
				\4[] $\to$ UE es parte del acuerdo
			\3 Ronda de Doha
				\4 Inicio en 2001
				\4 Rige single undertaking
				\4 Temas principales
				\4[] Agricultura
				\4[] NAMA
				\4[] Servicios
				\4[] Propiedad intelectual
				\4[] Facilitación del comercio
				\4[] Normas
				\4[] Solución de diferencias
				\4 Paralización actual
				\4[] Numerosos fracasos anteriores
				\4[] Agricultura es principal bloqueo
				\4[] $\to$ PAC es muy importante en UE
				\4[] $\to$ Intereses agrícolas muy organizados
				\4[] $\to$ Exportadores agrícolas exigen apertura
				\4[] $\then$ Muy difícil acuerdo
				\4[] UE ha tratado de mantener ronda viva
				\4[] Actualmente, negociación ``durmiente''
				\4[] $\to$ Esfuerzos trasladados a plurilaterales
				\4[] $\to$ Trump/USA presionan hacia bilaterales
			\3 Conferencias ministeriales
				\4 CM Bali 2013
				\4[] Paquete de Bali
				\4[] $\to$ Acuerdo de Facilitación de Comercio\footnote{Que ya ha entrado en vigor tras su firma por 2/3 de los miembros de la OMC.}
				\4[] $\to$ Seguridad alimentaria
				\4[] $\to$ Programa para futuras negociaciones
				\4[] Relativo éxito de UE
				\4 CM Nairobi 2015
				\4[] Paquete de Nairobi
				\4[] $\to$ Salvaguardias seguridad alimentaria
				\4[] $\to$ Restricciones a subvención de ex. agrícolas
				\4[] $\to$ Medidas de apoyo a PMAs
				\4[] Sin acuerdo para continuar Ronda de Doha
				\4 CM Buenos Aires 2017
				\4[] Sin nuevos acuerdos
				\4[] Constata estancamiento
				\4 CM Astaná 2020
				\4[] Previsible choque de posturas
				\4[] $\to$ Sobre Órgano de Apelación
				\4[] $\to$ Reformas propuestas
			\3 ITA -- Tecnologías de la Información
				\4 Firmado en 1996
				\4 Eliminación recíproca de aranceles
				\4[] Bienes relacionados con TI
				\4 Éxito del acuerdo
				\4[] Más de 50 países han firmado
				\4[] Más del 97\% del comercio mundial
				\4 Ampliado en Nairobi 2015
			\3 EGA -- Bienes medioambientales
				\4 Negociación comienza en 2014
				\4[] Sin avances desde 2016
				\4[] Principales economías exp./imp.
				\4 Liberalización comercio bienes MAmb.
				\4[] Ejemplos:
				\4[] $\to$ Turbinas solares
				\4[] $\to$ Filtros de agua
				\4[] $\to$ Equipos de medición
			\3 TiSA -- Comercio de servicios
				\4 Negociación comienza en 2013
				\4[] Sin avances desde final 2016
				\4 Ampliar GATS
				\4[] Más áreas de liberalización
				\4[] Más países
				\4 23 miembros negociando
				\4 Objetivo l/p es multilateralización
			\3 GPA -- Contratación pública
				\4 Entrada en vigor en 1996
				\4[] Ampliado y revisado en 2014
				\4 Liberalizar licitaciones públicas
				\4[] Igualdad de condiciones entre partes
				\4[] Licitaciones que superen un mínimo
				\4 Especialmente importante para la UE
				\4[] VC en provisión a Admón. Pública
			\3 Europa
				\4 EEE -- Espacio Económico Europeo
				\4[] Firmado en 1994
				\4[] UE+EFTA -- Suiza
				\4[] $\to$ UE + ISL + NOR + LIE
				\4[] Cuatro libertades del mercado único
				\4[] Acuerdos sectoriales
				\4[] Cooperación en otras áreas
				\4[] Áreas excluidas:
				\4[] $\to$ Unión aduanera y política comercial
				\4[] $\to$ PAC
				\4[] $\then$ Mayor mercado interior del mundo
				\4 Suiza
				\4[] Rechazo por referéndum al EEE
				\4[] Equiparar con EEE
				\4[] 7 acuerdos sectoriales
				\4[] UE es principal socio comercial de UE
				\4 Turquía
				\4[] Unión Aduanera
				\4[] Productores industriales con excepciones
				\4[] $\to$ Siderurgia no incluida
				\4[] $\to$ Agricultura sólo ciertas concesiones recíprocas
				\4[] En revisión
				\4 Ucrania
				\4[] Acuerdo de Asociación general en 2014
				\4[] $\to$ Contiene proyecto de DCFTA
				\4[] DCFTA entra en vigor en 2017
				\4[] $\to$ Liberalización de comercio
				\4[] $\to$ Armonización de regulación
				\4[] $\to$ Facilitación de comercio
				\4[] $\to$ Contratación pública
				\4[] $\to$ Seguridad alimentaria
				\4[] $\to$ Competencia
				\4[] $\to$ ...
				\4[] $\to$ Protección de propiedad intelectuales
				\4[] $\to$ Prohibidas importaciones de Crimea y Sebastopol
				\4[] $\then$ Acceso selectivo al mercado interior
				\4[] Préstamos a Ucrania
				\4[] Socio estratégico en el este
				\4[] Fuerte crecimiento de exportaciones a Ucrania
				\4[] UE es mayor socio comercial de Ucrania
				\4[] Ucrania exporta materias primas y maquinaria
				\4[] UE exporta manufacturas, maquinaria, químicos
				\4[] Elevado stock de IDE de Europa hacia Ucrania
				\4 Balcanes
				\4[] Acuerdos con ALB, SER, BOS, MON, MAC
				\4 Política Europea de Vecindad
				\4[] Marco de acuerdos de asociación:
				\4[] $\to$ UCR, MOL
			\3 MENA
				\4 Política Europea de Vecindad
				\4[] Marco de acuerdos de asociación:
				\4[] $\to$ MAR, TUN, ARG, EGI, ISR, JOR, LIB, PAL
				\4[] ARM, GEO, AZE
				\4 Euro-Mediterranean partnership (Euromed)
				\4[] 4,4\% del comercio extra-UE total
				\4[] MAR, ARG, TUN, LIB, EGY, ISR, JOR, TUR,
				\4[] $\to$ SYR y LIB no están en vigor
				\4[] Objetivo final
				\4[] $\then$ Crear área DCFTA\footnote{Deep and Comprehensive Free-Trade Agreement.}
 euro-mediterránea
				\4 Marruecos
				\4[] Acuerdo de asociación de 2000
				\4[] Mecanismo de resolución de disputas
				\4[] Acuerdo de liberalización agrícola
				\4[] Explorando DCFTA
				\4 Túnez
				\4[] Acuerdo de asociación en 1998
				\4[] Negociando DCFTA
			\3 África
				\4 Sudáfrica
				\4[] Acuerdo de libre comercio de 1999
				\4[] $\to$ Completada implementación en 2012
				\4[] $\to$ 90\% comercio
				\4 Comunidad Sudafricana para el Desarrollo
				\4[] Parte comercial de un EPA
				\4 ECOWAS\footnote{Economic Community of Western African States.}
				\4[] Acuerdo EPA
				\4[] $\to$ Libre entrada para importaciones de ECOWAS
				\4[] $\to$ Liberalización progresiva a entrada de productos UE
				\4 África Central
				\4[] Camerún
				\4[] $\to$ EPA interino
				\4[] Negociaciones con el resto de la región
				\4[] $\to$ Incluido Camerún
				\4[] Acceso al mercado vía SPG y EBA
				\4[] $\to$ Al margen de EPA
				\4 EAC -- Comunidad de África del Este
				\4[] EPA regional concluido
				\4[] Kenia y Ruanda tienen EPA con EU
				\4 África del Este y del Sur -- ESA\footnote{Eastern and Southern Africa.}
				\4[] EPAs sobre parte comercial
				\4[] $\to$ con Zimbabue, Madagascar, Seych. y Mauricio
				\4[] Negociando EPA completo
			\3 América
				\4 Estados Unidos
				\4[] Negociación de TTIP
				\4[] $\to$ 2013--2016
				\4[] Paralizado tras elección de Trump
				\4[] Muy ambicioso pero difícil de negociar
				\4[] Negociación sobre gravamen a empresas digitales
				\4[] $\to$ Sujeto contencioso
				\4[] $\to$ Exportaciones netas americanas
				\4[] $\to$ UE pretende establecer gravamen general
				\4[] $\then$ Americanas principales afectadas
				\4[] $\then$ Negociaciones rotas 17 junio 2020
				\4 Canadá
				\4[] CETA en 2016
				\4[] Aplicado provisionalmente en algunas áreas
				\4[] $\to$ Pendiente de ratificaciones nacionales
				\4[] $\to$ Pendiente de fallo de ECJ sobre resolución de disputas
				\4 México
				\4[] En vigor TLC desde 2000
				\4[] $\to$ Gran éxito
				\4[] $\to$ Enorme aumento de flujos comerciales
				\4[] Renegociado y modernizado recientemente
				\4[] $\to$ Última ronda de negociación concluida
				\4[] $\to$ Acuerdo político concluido y firmado
				\4[] $\to$ Actualmente, revisión legal de acuerdos
				\4 Centroamérica
				\4[] Costa Rica, El Salvador, Guatemala,
				\4[] Honduras, Nicaragua y Panamá
				\4[] Pequeña importancia para UE
				\4[] Enorme importancia de UE para Centroamérica
				\4[] Acuerdo de Asociación
				\4[] $\to$ En vigor desde 2013
				\4[] $\to$ Barreras técnicas, servicios, comercial
				\4[] $\to$ Favorecer integración entre ellas
				\4[] $\to$ Cooperación al desarrollo
				\4[] $\to$ Fomentar paz seguridad, inst. democráticas
				\4 Caribe
				\4[] EPA con Cariforum (15 países)
				\4[] $\to$ Apertura comercial recíproca gradual
				\4[] $\to$ Comercio de servicios e IDE
				\4 Comunidad Andina
				\4[] Colombia, Ecuador, Perú, Bolivia
				\4[] UE es socio muy importante
				\4[] $\to$ Segundo socio comercial
				\4[] $\to$ Uno de principales inversores
				\4[] $\to$ Mercado de exportación muy grande
				\4[] Notable potencial de crecimiento
				\4[] $\to$ Economías en plena expansión
				\4[] $\to$ Estabilización tras décadas de crisis
				\4[] $\to$ Sobretodo Colombia, Ecuador, Perú
				\4[] $\to$ Bolivia exportador de materias primas clave
				\4[] Acuerdo Multipartes
				\4[] $\to$ Con COL, ECU, PER
				\4[] $\to$ Liberalización comercial y barreras técnicas
				\4[] $\to$ Mejorar acceso a inversiones
				\4[] $\to$ Fomentar integración entre economías
				\4 Mercosur
				\4[] Negociación muy larga
				\4[] $\to$ Desde 1999
				\4[] Paralizada varios años
				\4[] Reanudada en 2018--Concluida en 2019
				\4[] Sexto socio comercial de Unión Europea
				\4 Chile
				\4[] UE es socio importante para Chile
				\4[] $\to$ Segunda fuente de importaciones
				\4[] $\to$ Tercer mercado de exportación
				\4[] Acuerdo de Asociación en 2002
				\4[] Acuerdo de Libre Comercio en 2003
			\3 Asia
				\4 Japón
				\4[] EU-Japan Economic Partnership Agreement
				\4[] Entrada en vigor en 2019
				\4[] Muy ambicioso:
				\4[] $\to$ Agricultura
				\4[] $\to$ Estándares laborales y medio ambiente
				\4[] $\to$ Protección al consumidor
				\4[] $\to$ Protección de datos
				\4[] $\to$ Salvaguardias sobre servicios públicos
				\4[] $\to$ Desarrollo sostenible
				\4[] $\to$ Cumplimiento de Acuerdo de París
				\4 Singapur
				\4[] Dos acuerdos firmados:
				\4[] $\to$ Acuerdo de Libre Comercio
				\4[] $\to$ Acuerdo de Protección de Inversiones
				\4[] Aprobados en 2019 por PE
				\4[] ALC en vigor
				\4[] Protección de inversiones pendiente de EEMM
				\4 Vietnam
				\4[] Dos acuerdos firmados no ratificados:
				\4[] $\to$ Acuerdo de Libre Comercio
				\4[] $\to$ Acuerdo de Protección de Inversiones
				\4[] Pendiente de aprobación por Parlamento de Vietnam
				\4[] Acuerdo de Protección de Inversiones
				\4[] $\to$ Pendiente de ratificación por EEMM (julio 2020)\footnote{\href{https://ec.europa.eu/trade/policy/countries-and-regions/countries/vietnam/}{EC (2020): Vietnam and trade policy}.}
				\4 China
				\4[] Inversiones
				\4[] $\to$ En negociación en feb. 2019
				\4[] $\to$ Eliminar barreras a la inversión
				\4[] $\to$ Aumentar protección a inversores
				\4[] $\to$ Reemplazar acuerdos bilaterales EEMM--China
				\4[] Servicios
				\4[] $\to$ Marco del TiSA
				\4[] $\to$ Sin avances desde 2016
				\4[] Bienes medioambientales
				\4[] $\to$ Marco del EGA
				\4[] $\to$ Sin avances desde 2016
				\4 India
				\4[] Inicio de negociaciones en 2007
				\4[] UE más ambiciosa que India
				\4[] $\to$ País tradicionalmente defensivo
				\4[] Negociaciones paralizadas desde 2013
				\4[] $\to$ Posible reanudación
				\4[] $\to$ Sin fecha de reanudación
				\4 Indonesia
				\4[] En negociación para FTA
				\4 Filipinas
				\4[] En negociación
				\4[] Sin acuerdos hasta la fecha
				\4 Tailandia
				\4[] Negociaciones en 2013 para FTA
				\4[] Paralizadas desde 2014
				\4 Myanmar/Birmania
				\4[] Conversaciones técnicas
			\3 Pacífico
				\4 Australia
				\4[] En negociación desde 2018
				\4[] $\to$ Objetivo es acuerdo amplio
				\4 Nueva Zelanda
				\4[] En negociación desde 2017
				\4[] $\to$ Objetivo es acuerdo amplio
				\4 Otros
				\4[] Papúa Nueva Guinea
				\4[] $\to$ EPA desde 2011
				\4[] Fiji
				\4[] $\to$ EPA desde 2014
	\1 \marcar{Estructura del sector exterior de la UE}
		\2 Idea clave
			\3 Economía de la UE
				\4 Enorme importancia en economía mundial
				\4 UE tradicional centro de teoría económica
				\4 Ventajas del comercio
				\4[] Conocidas y aprovechadas por EEMM
				\4 Grado de apertura de la UE
				\4[] Economía muy abierta
				\4[] Bloque comercial más abierto del mundo
				\4 Principal exportador e importador de servicios\footnote{Ver ICE (2019) pág. 48.}
				\4 Principal exportador e importador de mercancías\footnote{Ver ICE (2019) pág. 39.}
			\3 Tensiones internacionales
				\4 Conflictos comerciales USA
				\4[] Con China
				\4[] Con UE
				\4 Sanciones internacionales
				\4[] UE sanciona varios países y organizaciones
				\4[] SYR, IRA, NK, RUS, VEN...
				\4[] Especialmente relevantes a Rusia
				\4[] $\to$ Anexión de Crimea
				\4[] $\to$ Sanciones posteriores
		\2 Comercio\footnote{Ver DG Trade Statistical Guide 2019.}
			\3 Evolución
				\4 Ligera pérdida de peso en PIB en últimos 10 años
				\4[] Del $20\%$ al $\sim 17\%$
				\4[] Sigue siendo mayor bloque comercial del mundo
				\4[] $\to$ Estados Unidos es segundo bloque comercial
				\4[] $\to$ China cada vez se acerca más
				\4 Aumento de apertura al comercio fuera-UE
				\4[] De menos del $30\%$ a más del $35\%$
				\4 UE como principal socio comercial
				\4[] Para todos los países que bordean UE
				\4 Crecimiento de exportaciones agrícolas
				\4[] $\uparrow 78\%$ en últimos 10 años
				\4 Crecimiento de exportaciones no-agrícolas
				\4[] $\uparrow 47\%$ en últimos 10 años
				\4 Superávit en bienes y servicios
				\4[] Casi 166.000 M de € en 2018
				\4[] $\to$ Mayor del mundo
				\4[] $\to$ Supera a China y Rusia
				\4[] Cambio desde déficit >200.000 M de € en 2008
				\4[] Pequeño déficit en bienes
				\4[] Elevado superávit en servicios
				\4[] $\then$ Saldo neto superavitario
				\4 Mayores superávits de ByS\footnote{Ver pág. 52 de DG TRADE 2019}
				\4[] 1 - USA
				\4[] 2 - Suiza
				\4[] 3 - Australia
				\4[] 4 - UAE
				\4[] 5 - Arabia Saudí
				\4 Mayores déficits de ByS
				\4[] 1 - China
				\4[] 2 - Rusia
				\4[] 3 - Vietnam
				\4[] 4 - Tailandia
				\4[] 5 - Malasia
			\3 Bienes
				\4 Peso de la UE
				\4[] $15\%$ del comercio mundial de mercancías
				\4[] $\to$ 3.9 billones (españoles) de €
				\4[] $\to$ +3 veces el PIB de España
				\4[] Segundo exportador mundial
				\4[] $\to$ China es el primero
				\4[] Segundo importador mundial
				\4[] $\to$ USA es el primero
				\4 Saldo total
				\4[] Ligero déficit en 2018
				\4[] $\to$ Tras superávits 2013-2017
				\4 Comercio según acuerdos comerciales
				\4[] FTA implementados
				\4[] $\to$ Casi 1/3 del comercio total de bienes
				\4[] $\then$ 1/3 del total acuerdos de nueva generación
				\4[] $\then$ 2/3 del total son acuerdos tradicionales
				\4[] FTA concluidos
				\4[] $\to$ 6,4\% del comercio total de bienes
				\4[] Futuros FTA en negociación
				\4[] $\to$ 11,1\% del comercio total de bienes
				\4 Principales socios
				\4[] 1 - USA
				\4[] 2 - China
				\4[] 3 - Suiza
				\4[] 4 - Rusia
				\4[] 5 - Turquía
				\4[] 6 - Noruega
				\4[] 7 - Japón
				\4[] 8 - Corea del sur
				\4 Origen de importaciones
				\4[] 1 - China
				\4[] 2 - USA
				\4[] 3 - Rusia
				\4[] 4 - Suiza
				\4[] 5 - Noruega, Turquía, Japón, Corea...
				\4 Destino de exportaciones
				\4[] 1 - USA
				\4[] 2 - China
				\4[] 3 - Suiza
				\4[] 4 - Rusia
				\4[] 5 - Turquía, Japón, Noruega
				\4 Mayores desequilibrios
				\4[] Superávits
				\4[] 1 - USA
				\4[] 2 - Suiza
				\4[] 3 - Hong Kong
				\4[] 4 - Emiratos Árabes Unidos
				\4[] 5 - Australia
				\4[] Déficits
				\4[] 1 - China
				\4[] 2 - Rusia
				\4[] 3 - Vietnam
				\4[] 4 - Malasia
				\4[] 5 - Tailandia
			\3 Servicios
				\4 Comercio total
				\4 Peso de la UE
				\4[] Más de $1/5$ del comercio mundial
				\4[] Mayor exportador mundial
				\4 Principales socios
				\4[] Con mucha diferencia, USA
				\4 Balanza de servicios
				\4[] Superávit exportador a favor de UE
				\4 Exportación de servicios por modos
				\4[] (por orden de importancia)
				\4[] i. Modo 3
				\4[] $\to$ Vía presencia comercial
				\4[] $\to$ Ej.: hotel extranjero con personal local
				\4[] ii. Modo 1
				\4[] $\to$ Provisión trans-fronteriza
				\4[] iii. Modo 2
				\4[] $\to$ Consumidor se desplaza
				\4[] iv. Modo 4
				\4[] $\to$ Desplazamiento de trabajadores
				\4[] $\to$ Ej.: constructora trae trabajadores extranjeros
				\4 Destino de exportaciones de servicios
				\4[] 1 - USA
				\4[] 2 - Suiza
				\4[] 3 - China
				\4[] 4 - Japón
				\4[] 5 - Rusia
				\4[] 6 - Singapur
				\4[] 7 - Noruega
				\4 Origen de importaciones de servicios
				\4[] 1 - USA
				\4[] 2 - Suiza
				\4[] 3 - China
				\4[] 4 - Bermudas
				\4[] 5 - Singapur
				\4[] 6 - Japón
				\4[] 7 - India
				\4 Desequilibrios\footnote{Pág. 49 de informe DG Trade 2019. Utilizados los datos de 2017 porque 2018 aún incompletos.}
				\4[] Superávits
				\4[] 1. Suiza
				\4[] 2. Rusia
				\4[] 3. Japón
				\4[] 4. China
				\4[] 5. Australia
				\4[] 6. USA
				\4[] 7. Arabia Saudí
				\4[] Déficits
				\4[] 1. Tailandia
				\4[] 2. Turquía
				\4[] 3. India
				\4 Principales sectores de exportación
				\4[] 1. Servicios a empresas
				\4[] 2. Transporte
				\4[] 3. Telecomunicaciones e IT
				\4[] 4. Viajes
				\4[] 5. Servicios financieros
			\3 Retos
				\4 Reducir déficit ByS con China
				\4[] $\to$ Casi 200.000 M de €
				\4 Aumentar apertura
				\4 Implementar acuerdos nueva generación
				\4 Aumentar diversificación de socios
				\4[] No depende sólo de UE
				\4[] China + USA peso muy elevado
				\4[] Desarrollo de socios es factor clave
				\4 Guerra comercial
				\4[] Mantener operativo sistema multilateral
		\2 Inversión\footnote{Pág. 34 de DG Trade 2019}
			\3 Stocks (2017)
				\4 Entrante
				\4[] 1 - USA
				\4[] 2 - Suiza
				\4[] 3 - Canadá
				\4[] 4 - Japón
				\4[] 5 - Noruega
				\4[] 6 - Rusia
				\4 Saliente
				\4[] 1 - USA
				\4[] 2 - Suiza
				\4[] 3 - Brasil
				\4[] 4 - Canadá
				\4[] 5 - Rusia
				\4[] 6 - China
				\4[] 7 - México
			\3 Flujos (2017)
				\4[] Offshores
				\4[] $\to$ Segundo mayor componente en ambos
				\4 Entrantes
				\4[] 1 - Canadá
				\4[] 2 - Suiza
				\4[] 3 - China
				\4[] 4 - Japón
				\4[] 5 - Brasil
				\4[] 6 - Noruega
				\4[] 7 - Corea del Sur
				\4 Salientes
				\4[] 1 - USA
				\4[] 2 - Suiza
				\4[] 3 - Méjico
				\4[] 4 - Islandia
				\4[] 5 - China
				\4[] 6 - Brasil
				\4[] 7 - India
			\3 Empresas extranjeras en UE
				\4 Según país que controla
				\4[] 1 - USA
				\4[] 2 - Suiza
				\4[] 3 - Offshore
				\4[] 4 - Noruega
				\4[] 5 - China
				\4 Según empleados en empresas controladas
				\4[] 1 - USA
				\4[] 2 - Suiza
				\4[] 3 - Offshore
				\4[] 4 - Japón
				\4[] 5 - Canadá
	\1 \marcar{Política de cooperación al desarrollo}
		\2 Justificación
			\3 Motivos altruistas y humanitarios
				\4 Opinión pública europea
				\4[] Rechazo de desigualdades extremas
				\4[] Altruismo frente a PEDs y pobreza
			\3 Acceso a mercados
				\4 Mercados de exportación
				\4[] Más desarrollo implica más demanda
				\4[] $\to$ Menos paro y más renta en UE
			\3 Geoestratégicos
				\4 Mantener influencia política y cultural UE
				\4 Frenar avance de competidores
				\4[] China, Rusia principalmente
				\4 Mantener apoyo de socios en PEDs
				\4[] Muy poblados
				\4[] Controlan materias primas clave
				\4[] Potencial de crecimiento elevado
				\4 Reducir tensiones migratorias
				\4[] Generan inestabilidad en:
				\4[] $\to$ EEMM receptores
				\4[] $\to$ Fronteras
				\4[] $\to$ Países de tránsito
			\3 Desarrollo económico de socios
				\4 Especialización del trabajo
				\4[] Mejora eficiencia
				\4[] Ventaja fundamental del comercio
				\4[] $\to$ Mayor variedad de inputs
				\4[] $\to$ Mejor utilización recursos productivos
				\4[] $\then$ Aumentar renta también en UE
				\4 Crecimiento potencial de PEDs
				\4[] Tecnologías y productos futuros
				\4[] $\then$ También beneficia UE
		\2 Antecedentes
			\3 Post-descolonización
				\4 Ayuda centrada en antiguas colonias
				\4 Poca coordinación entre EEMM
			\3 Años 70 y 80
				\4 Crisis financieras y globalización
				\4 Expansión de ámbito geográfico de ayudas
			\3 Institucionalización en 90s
				\4 Tratado de Maastricht recoge expresamente
				\4[] Por primera vez
			\3 Tratado de Lisboa
				\4 Define como competencia compartida
				\4[$\then$] Necesidad de coordinar actuaciones
		\2 Objetivos
			\3 Coordinar EEMM
				\4 Ayuda al desarrollo es competencia compartida
				\4[] EEMM y UE desarrollan
				\4 Fomentar sinergias y aumentar eficiencia
			\3 Fomentar comercio y desarrollo
				\4 Integrar PEDs en comercio mundial
				\4 Aprovechar ventajas del comercio
				\4 Aumentar competitividad y renta de PEDs
				\4[$\then$] Aumentar renta de UE
			\3 Mejorar diseño de políticas en PEDs
				\4 Reducir corrupción
				\4 Aumentar efectividad de políticas públicas
				\4 Facilitar acceso a exportadores e inversores UE
				\4[$\then$] Mejorar bienestar en PEDs
				\4[$\then$] Mejorar acceso a mercados de PEDs
			\3 Aliviar crisis humanitarias
				\4 Reducir pérdidas humanas en crisis
				\4 Reducir riesgo de nuevos desastres
		\2 Marco jurídico
			\3 Programas de coop. al desarrollo
				\4 Programas geográficos
				\4 Programas horizontales o temáticos
				\4[] Ejemplo:
				\4[] $\to$ Promoción de la democracia
				\4[] $\to$ Derechos Humanos
				\4[] $\to$ Prevención de crisis
				\4[] $\to$ ...
			\3 Instrumentos unilaterales
				\4 Sistema de Preferencias Generalizadas
				\4[] Reglamento de 2012
			\3 Acuerdos económicos bilaterales
				\4 Acuerdos comerciales con preferencias
				\4 Acuerdos de colaboración diseño de políticas
				\4 Acuerdos de asistencia financiera
				\4 Ayuda pre-adhesión
			\3 EuropeAid -- DG Cooperación Internacional y Desarrollo
				\4 Fundada en 2011
				\4[] Fusión de EuropeAid y DG para ACP
				\4 Delegaciones en países receptores de ayuda
				\4 Coordinación con SEAcción Exterior
		\2 Marco financiero
			\3 Instrumento de Financiación de la Cooperación al Desarrollo
				\4 Parte del presupuesto de la UE
				\4 20.000 M de € para periodo 2014-2020
				\4 Cubre casi todos los programas
				\4[] Geográficos y temáticos
				\4[] $\to$ Salvo ACP
			\3 Fondo Europeo de Desarrollo
				\4 Creado por Tratado de Roma (1957)
				\4 Fuera del presupuesto UE
				\4[] Contribuciones voluntarias de EEMM
				\4[] Recursos cedidos por BEI
				\4 30.000 M de € para periodo 2014-2020
				\4 Acuerdo de Cotonú para ACP y EPAs
				\4[] Principal soporte financiero
			\3 Banco Europeo de Inversiones
				\4 Inst. financiera internacional
				\4 Misión inicial
				\4[] Desarrollar mercado interior
				\4[] Financiar inversión
				\4 Actualidad
				\4[] También proyectos fuera de UE (10\%)
				\4[] $\to$ Balcanes
				\4[] $\to$ Política Europea de Vecindad
				\4[] $\to$ Otros PEDs
				\4 Préstamos a PEDs
				\4[] Financiación difícil de obtener para PEDs
				\4[] Tipos de interés concesionales
		\2 Actuaciones
			\3 \underline{Preferencias generalizadas}
			\3[] SPG general
				\4 Historia
				\4[] Conferencia UNCTAD New Delhi 1968
				\4[] Waiver en el GATT: 1979
				\4[] $\to$ Permite excepción a NMF
				\4[] CEE ya implementan desde 1971
				\4[] Tendencias de l/p de UE
				\4[] $\to$ Concentración hacia países más débiles
				\4[] $\to$ Simplificación de regímenes
				\4[] $\to$ Más graduación de las preferencias
				\4[] Última reforma aprobada en 2012
				\4[] $\to$ Para 2016--2025
				\4 Exención arancelaria
				\4[] Productos no sensibles
				\4 Productos sensibles
				\4[] 3.5\% respecto a tipo NMF
				\4[] 20\% para textil y confección
				\4 Sensibilidad de los productos
				\4[] Existencia de producto similar en UE
				\4[] Incidencia de importación en UE
				\4[] $\then$ Sobre todo, productos industriales
				\4 Retirada de preferencias
				\4[] 17.5\% del total de importaciones de rama
				\4 Eligibilidad
				\4[] Renta per cápita baja
			\3 SPG+
				\4 Exención arancelaria
				\4[] Lista de productos cubiertos
				\4 Cumplimiento de dos requisitos
				\4[] i. Vulnerabilidad
				\4[] ii. Compromiso con convenciones internacionales
				\4[] $\to$ DDHH, laborales, medioambiente, gobernanza
				\4 Sin cancelar por volumen de exportación
			\3 EBA -- Everything But Arms
				\4 Exención total a todos los productos de PMA
				\4[] Según criterios de ONU
				\4 Mecanismo de retirada
				\4[] Si incumplen obligaciones
				\4[] Si perjuicio a productores comunitarios
				\4 Interacción con ALCs de la UE
				\4[] Firma de un ALC no supone retirada de prefs. EBA
			\3 Waiver de servicios
				\4 GATS 1994
				\4[] Obligaciones generales:
				\4[] $\to$ Nación Más Favorecida
				\4[] $\to$ Transparencia respecto a cambios regulatorios
				\4[] $\to$ Normativa nacional transparente y proporcionada
				\4[] $\to$ Monopolios permitidos pero respetando NMF
				\4 WTO 2011 Génova:
				\4[] Aprobación de waiver de servicios
				\4[] $\to$ Exención a NMF de GATS
				\4[] Permite acceso preferencial a servicios
				\4[] $\to$ PMAs
				\4[] $\to$ 15 años
				\4 Nairobi: 2015 UE presenta compromisos de waiver\footnote{Ver \href{https://trade.ec.europa.eu/doclib/press/index.cfm?id=1256&title=EU-offers-Least-Developed-Countries-preferential-market-access-for-services}{Comisión Europea (2015): Nota de prensa -- EU offers Least Developed Countries preferential market access for services.}}
				\4[] Lista de sectores
				\4[] Amplia variedad de actividades
				\4[] No incluye transporte
				\4[] Evita intereses defensivos
				\4[] Permite servicios en que UE deficitario
				\4[] $\to$ Enfermeras y matronas
				\4[] $\to$ Veterinarios
				\4[] Incluye otros irrelevantes para PMAs
				\4[] $\to$ Intensivos en cap. humano inexistente
			\3 \underline{Acuerdos EPA con ACP}
				\4 Historia
				\4[] Acuerdo de Cotonú en 2000
				\4[] $\to$ Relaciones UE--ACP
				\4[] $\to$ Hasta año 2007
				\4[] $\to$ Creación de áreas de libre comercio
				\4[] $\to$ Integración regional entre destinatarios
				\4[] $\to$ Reformas democráticas
				\4[] $\to$ Ayuda financiera de FED y BEI
				\4[] $\to$ Migraciones
				\4[] $\then$ Prorrogada validez hasta 2020
				\4 Acuerdos de Asociación Económica -- EPA\footnote{Economic Partnership Agreement.}
				\4[] Superpuestos a Acuerdo de Cotonú
				\4[] Específicos para regiones
				\4[] Enfoque asimétrico
				\4[] $\to$ Acuerdos adaptados a destinatarios
				\4[] Incluyen compromisos de:
				\4[] $\to$ Eliminación de aranceles y cuotas para entrada en UE
				\4[] $\to$ Apertura progresiva a exportaciones de UE
				\4 Siete configuraciones regionales de EPAs
				\4[] África Occidental
				\4[] África Central
				\4[] África Oriental y Meridional
				\4[] Comunidad de Desarrollo de África Austral
				\4[] Comunidad de África del Este
				\4[] Caribe
				\4[] Pacífico
			\3[] \underline{Ayuda Oficial al Desarrollo}
			\3 Idea clave
				\4 Concepto
				\4[] Definición del CAD
				\4[] $\to$ Comité de Ayuda al Desarrollo de la OCDE
				\4[] Recursos económicos dirigidos a PEDs
				\4[] $\to$ Fomento de desarrollo socioeconómico
				\4[] Requisitos mínimos de concesionalidad
				\4[] $\to$ > 25\% de recursos transferidos son donación
				\4 Importancia
				\4[] Más de 130.000 M de dólares anuales
				\4[] $\to$ A nivel mundial
				\4[] UE+EEMM aporta más de la mitad
				\4[] $\to$ Casi 75.000 M de € en 2019
				\4[] Recuperación post-crisis
				\4[] Índice de esfuerzo
				\4[] $\to$ Proporción sobre RNB
				\4[] $\to$ Objetivo mínimo del 0.7\% de RNB
				\4[] $\to$ Varios países europeos superan
				\4 Programas de objetivos
				\4[] Objetivos de Desarrollo Sostenible 2030
				\4[] Agenda 2030
			\3 Formas de ayuda
				\4 Financiación de proyectos
				\4 Financiación de programas
				\4 Ayuda presupuestaria
				\4 Enfoque SWAp
				\4[] Ayuda al desarrollo
				\4[] $\to$ Considerando todos los actores
				\4[] $\then$ Gobiernos
				\4[] $\then$ Donantes
				\4[] $\then$ Empresas privadas
				\4[] $\then$ Población civil
				\4[] Conjunto de principios sobre políticas
				\4[] $\to$ No sólo programas concretos
				\4 Ayuda humanitaria
				\4 Asistencia técnica
				\4 Contribución a fondos temáticos
			\3 Distribución geográfica
				\4 África Subsahariana
				\4[] 50\% de la ayuda
				\4 Asia Central y Sureste Asiático
				\4[] 15\%
				\4[] Especial importancia estratégica
				\4 Iberoamérica y Caribe
				\4[] 12\%
				\4 Resto
				\4[] 20\%
			\3 Distribución sectorial
				\4 Infraestructura económica (32\%)
				\4[] Comunicaciones, transporte, energía...
				\4 Infraestructura social (22\%)
				\4[] Gobernanza, paz, seguridad...
				\4 Producción (11\%)
				\4[] Mejorar capacidad productiva
				\4 Multisector (11\%)
				\4[] Medioambiente, desarrollo urbano, igualdad...
				\4 Ayuda humanitaria (8\%)
				\4[] Catástrofes naturales, conflictos armados
				\4[] Ayuda alimentaria, reconstrucción...
				\4 Educación, sanidad, población (7\%)
			\3 FED -- Fondo Europeo de Desarrollo
				\4 Principal fuente de financiación para ACP
				\4 30.500 M de € para MFP 2014-2020
				\4[] De los cuales 1.000 a gastos administrativos
				\4 No es parte de MFP UE propiamente
				\4 Parte de MFP específico
				\4[] Incluye:
				\4[] $\to$ FED
				\4[] $\to$ Fondos del BEI
			\3 DCI -- Instrumento de Cooperación al Desarrollo\footnote{\href{https://www.europarl.europa.eu/RegData/etudes/BRIE/2017/608764/EPRS_BRI(2017)608764_EN.pdf}{EPRS (2017): Development Cooperation Instrument}.}
				\4 Parte del presupuesto UE
				\4 20.000 M de € para MFP 14-20
				\4 Método de implementación
				\4[] Directo e indirecto
				\4[] $\to$ Sin gestión compartida
				\4 Reducción de pobreza como objetivo central
				\4 Objetivos del Milenio y ODS
				\4 Complemento a FED
				\4[] Países receptores de FED
				\4[] $\to$ No reciben programas geográficos de DCI
				\4 Tres tipos de programa
				\4[] Geográfico
				\4[] Sectorial
				\4[] Pan-Africano
		\2 Valoración
			\3 Paradoja de la ayuda exterior
				\4 Países que reciben poca ayuda
				\4[] Se desarrollan con instituciones adecuadas
				\4 Países que han recibido ingentes ayudas
				\4[] No han salido de pobreza extrema
				\4[$\then$] ¿Realmente coop. tiene efecto causal positivo?
			\3 Sistema de Preferencias generalizadas
				\4 Críticas
				\4[] No contribuye realmente a desarrollo de PMAs
				\4[] Complejidad técnica dificulta utilización
				\4[] Tratamiento no suficientemente favorable
				\4[] Productos ``sensibles'' son los más competitivos
				\4[] Posibilidad de salvaguardias
				\4[] Aplicación a países que no lo necesitan
				\4 UE como importador de PMAs
				\4[] Mayor importador del mundo
				\4[] 76.000 millones de €
				\4 Última reforma
				\4[] Trata de corregir críticas
				\4[] Menor número de países
				\4[] Más productos no sensibles
				\4[] Más países beneficiarios de SPG+
				\4[] $\to$ Más supervisión de cumplimiento convenciones
		\2 Retos
			\3 Incentivos perversos
				\4 Cita de Peter Bauer
				\4[] Premio Milton Friedman 2002
				\4[] Ayuda exterior es:
				\4[] ``...excelente método para transferir dinero
				\4[] ...de gente pobre en países ricos...
				\4[] ...a gente rica en países pobres.''
				\4 Cooperación puede incentivar:
				\4[] $\to$ Rent-seeking en emisor y destinatarios
				\4[] $\to$ Aumento de burocracias
				\4[] $\to$ Dependencia de ayuda exterior
			\3 Mala asignación del capital
				\4 ¿Quién recibe y quién no?
				\4[] Emisor decide con información imperfecta
				\4[] $\to$ Difícil acertar
				\4 Recepción de inversiones
				\4[] Resultado de equilibrar rentabilidad y riesgo
				\4[] $\to$ Cooperación al desarrollo tiene otros criterios
				\4[] $\then$ ¿Son los más apropiados?
				\4[] $\then$ ¿Tienen ``skin-in-the-game''?
	\1[] \marcar{Conclusión}
		\2 Recapitulación
			\3 Marco institucional
			\3 Relaciones económicas exteriores
			\3 Política de cooperación al desarrollo
		\2 Idea final
			\3 Valoración
				\4 Economía muy abierta
				\4 Actor fundamental en el mundo
				\4 Defensor del sistema multilateral
				\4[] Basado en reglas
			\3 Retos
				\4 Mantener sistema multilateral
				\4[] Frenar tendencia bilateralismo
				\4 Mantener peso en el mundo
				\4[] Necesario crecimiento econ. y población
\end{esquemal}





























%\begin{esquemal}
%    \1[] Introducción
%    \1 Marco institucional
%        \2 PESC (Política Exterior y de Seguridad Común)
%            \3 ¿Qué es?
%                \4 Actuaciones UE
%                \4 Paz, seguridad internacional, cooperación, democracia, DDHH
%                \4 Carácter político, geoestratégico y de seguridad
%                \4 Impacto económico
%                \4 Consejo Europeo: directrices políticas generales
%                \4 Consejo de la UE: seguimiento efectivo
%            \3 Efectos economía internacional
%                \4 Misiones África: Sahel (Níger, Mali)
%                \4 Lucha piratería
%                \4 Oriente Medio
%        \2 PAC
%            \3 ¿Qué es?
%                \4 Política común optimización prod. agrícola
%                \4 Pilar política de mercados
%                \4 Pilar desarrollo rural
%            \3 Efectos economía internacional
%                \4 UE: primer bloque exportador e importador de prod. agricolas
%                \4 Primer importador productos agrícolas PMA
%                \4 Principios mercado interior y preferencia comunitaria
%        \2 Política de desarrollo regional y apoyo a la ampliación
%            \3 ¿Qué es?
%                \4 Complemento europeo integración económica
%                \4 Reducir desigualdades regionales
%                \4 Aminorar efectos shocks asimétricos
%                \4 Objetivos más allá de fronteras UE
%            \3 Efectos economía internacional
%                \4 Candidatos a la adhesión
%                \4 Instrumentos de Ayuda a la Pre-Adhesión (IPA II 2014-2020)
%                \4 Country Strategy Paper
%                \4 Multi-Country Paper
%        \2 Política comercial: marco institucional
%            \3 ¿Qué es?
%                \4 Decisiones autoridades afectan flujos com. ext.
%            \3 Efectos economía internacional
%                \4 Mayor bloque comercial del mundo
%                \4 Mayor exportador e importador mundial
%                \4 Enorme impacto decisiones
%    \1 Relaciones económicas comerciales
%        \2 Política comercial autónoma
%            \3 Política arancelaria
%                \4 Arancel aduanero común
%                \4 Regímenes comerciales de importación y exportación
%                \4 Medidas de defensa comercial
%            \3 Regímenes comerciales de importación y exportación
%                \4 Régimen general
%                \4 Autorización
%                \4 Vigilancia
%                \4 Certificación
%            \3 Defensa comercial
%                \4 Código barreras no arancelarias OMC
%                \4 Órgano de Solución de Diferencias
%                \4 Antidumping
%                \4 Antisubvención
%                \4 Salvaguardia
%                \4 Reglamento de obstáculos al comercio
%        \2 Acuerdos multilaterales
%            \3 OMC
%                \4 UE y EEMM parte de OMC
%                \4 Todos los acuerdos OMC: GATT-94, GATS, TRIPS
%            \3 IX Conferencia Ministerial de Bali 2013
%                \4 Acuerdo de Facilitación de Comercio
%            \3 X Conferencia Ministerial de Nairobi 2015
%                \4 Paquete de Nairobi
%                \4 Algodón
%                \4 Productos agrícolas
%                \4 Seguridad alimentaria
%        \2 Acuerdos plurilaterales
%            \3 ITA 1996
%            \3 EGA (negociaciones 2014)
%            \3 TiSA (negociaciones 2013)
%            \3 GPA 1994 - Revisión 2014
%        \2 Acuerdos bilaterales
%            \3 Espacio Económico Europeo
%            \3 Suiza
%            \3 Turquía
%            \3 Balcanes
%            \3 Política Europea de Vecindad
%            \3 Rusia
%            \3 África
%            \3 Estados Unidos
%            \3 Canadá
%            \3 México
%                \4 UE importante emisor de IDE en México
%                \4 ALC 2000
%            \3 Centroamérica
%                \4 Acuerdo de asociación 2013
%            \3 Perú, Colombia, Ecuador
%                \4 Fuertes tasas de crecimiento
%                \4 Acuerdo multipartes 2013 (Ecuador 2016)
%            \3 Chile
%            \3 China
%                \4 Flujos IDE reducidos
%                \4 Diálogo Económico y Comercial de Alto Nivel (2008)
%            \3 India
%                \4 Fragmentación mercado interior indio
%                \4 Sin acuerdo
%            \3 Corea del Sur
%                \4 ALC 2011 (nueva generación)
%            \3 ASEAN
%                \4 Prioridad comercial
%                \4 Acuerdos por separado
%            \3 Acuerdos en negociación
%                \4 Deep and Comprehensive Free Trade Areas
%                \4 Rusia
%                \4 EPA (Economic Partnership Agreement) en África Subsahariana
%                \4 TTIP
%                \4 Mercosur
%                \4 Japón
%                \4 ASEAN: Tailandia y Filipinas
%                \4 India: paralizadas
%                \4 China: negociación acuerdo de inversiones
%        \2 Brexit
%            \3 Importancia Reino Unido
%                \4 Volumen comercio
%                \4 Principales socios
%                \4 Principales sectores
%            \3 Posibles escenarios
%                \4 OMC
%                \4 Acuerdo de libre comercio
%                \4 Unión Aduanera
%        \2 Relaciones económicas multilaterales no comerciales de la UE
%            \3 G8
%            \3 G20
%                \4 Presidente Comisión, BCE, Eurogrupo
%            \3 FMI
%            \3 OCDE
%            \3 Naciones Unidas
%    \1 Política de cooperación al desarrollo de la UE
%        \2 Justificación
%            \3 Razones éticas y humanitarias
%            \3 Beneficio mutuo
%            \3 Apertura mercados
%            \3 Contribución estabilidad 
%        \2 Objetivos
%            \3 Erradicar pobreza
%            \3 Desarrollo sostenible
%            \3 Integración economía mundial
%            \3 Mejora marco institucional, gobernanza y seguridad jurídica
%        \2 Marco jurídico
%            \3 Tratado de Maastricht (1992)
%                \4 Institucionalización
%            \3 Tratado de Lisboa (2009)
%                \4 Competencia compartida
%                \4 Exigencia de coordinación
%        \2 Marco financiero
%            \3 Instrumento de Financiación de la Cooperación al Desarrollo
%                \4 20 MM de euros para periodo 2014-2020
%            \3 Fondo Europeo de Desarrollo
%                \4 ACP y ultramar
%                \4 No presupuestario
%                \4 Contribuciones voluntarias y ocasionalmente recurso BEI
%            \3 Banco Europeo de Inversiones
%                \4 10\% financiación total
%            \3 Ayuda Oficial al Desarrollo
%                \4 14.000 millones Unión Europea
%                \4 60.000 millones ayuda bilateral
%                \4 Índice de esfuerzo muy desigual
%        \2 Actuaciones
%            \3 Programas
%                \4 Geográficos
%                \4 Horizontales
%            \3 Instrumentos
%                \4 Normativos
%                \4 Financieros
%            \3 Sistema de Preferencias Generalizadas
%                \4 Régimen SPG General: -3,5\% tipo MFN, importaciones máx. 17,5\%
%                \4 Régimen SPG+: vulnerables, exp. concentradas, imp. reducidas, ratificación convenios DDHH...
%                \4 Everything But Arms: PMA - exención total
%            \3 Acuerdos con países ACP
%                \4 Red de EPA (Economic Partnership Agreements)
%                \4 ECOWAS
%                \4 Camerún
%                \4 África Oriental y Meridional: negociaciones para EPA completo
%                \4 Comunidad de Desarrollo del África Austral - efectiva en 2012
%                \4 Comunidad de África del Este: liberalización progresiva
%                \4 Caribe: EPA con 15 países caribeños
%                \4 Pacífico: Papúa Nueva Guinea, Fiji
%            \3 Ayuda Oficial al desarrollo
%                \4 Componente de donación del 25\%
%                \4 Cooperación financiera
%                \4 Cooperación económica
%                \4 Ayuda humanitaria y de emergencia
%                \4 Asistencia técnica
%                \4 África Subsahariana: 47\%
%                \4 Asia Central y Sureste Asiático: 15\%
%                \4 Iberoamérica y Caribe: 12\%
%                \4 Diversificación geográfica elevada
%                \4 Destinos: infraestructura económica, social, producción, multisector, ayuda humanitaria
%        \2 Valoración y retos
%            \3 SPG y SPG+
%                \4 Complejidad técnica
%                \4 Países receptores no PEDs
%                \4 SPG+ soluciona países destinatarios
%                \4 Refuerzo supervisión en SPG+
%            \3 Ayuda Oficial al Desarrollo
%                \4 Crisis migratorias
%                \4 Conflicto Siria-Iraq
%                \4 Marruecos, Libia, Senegal
%                \4 Restricciones presupuestarias
%    \1[] Conclusión
%\end{esquemal}

\preguntas

\seccion{Test 2014}

\textbf{43.} Los acuerdos de Asociación Económica son los firmados por la Unión Europea:

\begin{itemize}
	\item[a] Con algunos países de África, Caribe y Pacífico (ACP) dentro de los acuerdos de Lomé.
	\item[b] Con algunos países de África, Caribe y Pacífico (ACP) dentro de los Acuerdos de Cotonú
	\item[c] Con la Asociación Euro-Mediterránea.
	\item[d] Con Mercosur
\end{itemize}

\notas

\textbf{2014:} \textbf{43.} B

\bibliografia

\begin{itemize}
	\item European Union Budget
\end{itemize}

European Commission, DG Trade. \textit{Statistical Guide} (2018) Junio -- En carpeta del tema

European Commission, DG Trade. \textit{Statistical Guide} (2019) Julio -- En carpeta del tema

European Commission. \textit{European Development Fund (EDF)} \url{https://ec.europa.eu/europeaid/funding/funding-instruments-programming/funding-instruments/european-development-fund\_en}

European Commission. \textit{Overview of Economic Partnership Agreement} (2018) Actualización de Noviembre de 2018. -- En carpeta del tema

European Commission. \textit{Statistics} (2019) Statistics -- \url{http://ec.europa.eu/trade/policy/countries-and-regions/statistics/}

Eurostat. \textit{The EU in the world} (2018) Eurostat Statistical Books -- En carpeta del tema

IMF (2019) \textit{External Sector Report. The Dynamics of External Adjustment} Julio de 2019 \url{https://www.imf.org/en/Publications/SPROLLs/External-Sector-Reports} -- En carpeta del tema

\end{document}
