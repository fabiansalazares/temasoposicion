\documentclass{nuevotema}

\tema{3A-25}
\titulo{Teoría de la Regulación. Implicación sobre la liberalización de sectores.}

\begin{document}

\ideaclave

FALTA apartado sobre Yardstick competition

\seccion{Preguntas clave}

\begin{itemize}
	\item ¿Qué es la regulación?
	\item ¿Por qué es necesaria la regulación?
	\item ¿Cuáles son los objetivos de la regulación?
	\item ¿Qué problemas enfrenta el regulador?
	\item ¿En qué consiste la teoría tradicional de la regulación?
	\item ¿Qué es la nueva teoría de la regulación?
	\item ¿En qué consiste la liberalización de mercados regulados?
	\item ¿Qué formas de liberalización pueden aplicarse?
	\item ¿En qué marco institucional se llevan a cabo los procesos de liberalización?
	\item ¿Qué opciones de liberalización se pueden aplicar a una industria de red?
\end{itemize}


\esquemacorto

\begin{esquema}[enumerate]
	\1[] \marcar{Introducción}
		\2 Contextualización
			\3 Economía y microeconomía
			\3 Justificación de la intervención pública
			\3 Instrumentos de actuación
			\3 Regulación
			\3 Liberalización
		\2 Objeto
			\3 ¿Qué es la regulación?
			\3 ¿Por qué es necesaria la regulación?
			\3 ¿Cuáles son los objetivos de la regulación?
			\3 ¿Qué problemas enfrenta el regulador?
			\3 ¿En qué consiste la teoría tradicional de la regulación?
			\3 ¿Qué es la nueva teoría de la regulación?
			\3 ¿En qué consiste la liberalización de mercados regulados?
			\3 ¿Qué formas de liberalización pueden aplicarse?
			\3 ¿En qué marco institucional se llevan a cabo los procesos de liberalización?
			\3 ¿Qué opciones de liberalización se pueden aplicar a una industria de red?
		\2 Estructura
			\3 Análisis teórico de la regulación
			\3 Liberalización de sectores regulados
	\1 \marcar{Análisis teórico de la regulación}
		\2 Bienes públicos
			\3 Idea clave
			\3 Formulación
			\3 Implicaciones
		\2 Externalidades
			\3 Idea clave
			\3 Formulación
			\3 Implicaciones
		\2 Información imperfecta
			\3 Idea clave
			\3 Formulación
			\3 Implicaciones
		\2 Monopolio natural con información perfecta
			\3 Idea clave
			\3 Formulación
			\3 Implicaciones
			\3 Óptimo de primer orden
			\3 Óptimo de segundo orden
			\3 Multiproducto y precios lineales
		\2 Monopolio con información imperfecta
			\3 Idea clave
			\3 Coste del servicio o tasa de retorno
			\3 Reglas Price Cap
			\3 Subasta de explotación
			\3 Yardstick competition/Competencia referencial
			\3 Diseño de mecanismos
		\2 Problemas de la regulación
			\3 Idea clave
			\3 Costes de la regulación
			\3 Efectos colaterales no deseados
			\3 Rent-seeking
			\3 Captura del regulador
			\3 Asimetrías informativas en el proceso regulador
		\2 Contexto institucional de la regulación
			\3 Idea clave
			\3 Legislación
			\3 Contratos de franquicia
			\3 Comisiones independientes de regulación
			\3 Provisión pública
			\3 Credibilidad y rendición de cuentas
			\3 Política de defensa de la competencia
	\1 \marcar{Liberalización de sectores regulados}
		\2 Idea clave
			\3 Contexto
			\3 Objetivos
			\3 Resultado
		\2 Introducción de mecanismos de mercado
			\3 Contratación externa
			\3 Concesión de derechos de propiedad: cuotas
			\3 Competencia entre unidades públicas
			\3 Vales, bonos o cheques
			\3 Modalidades de copago: precios públicos, tasas
		\2 Contestabilidad
			\3 Idea clave
			\3 Formulación
			\3 Implicaciones
			\3 Valoración
		\2 Proceso de liberalización
			\3 Condicionantes a cumplir
			\3 Problemas identificados en la práctica
		\2 Industrias de red
			\3 Idea clave
			\3 Aspectos generales de industrias de red
			\3 Modelos de separación de segmentos
			\3 Regulación de acceso al input no competitivo
			\3 Competencia por el segmento no competitivo
	\1[] \marcar{Conclusión}
		\2 Recapitulación
			\3 Análisis teórico de la regulación
			\3 Liberalización de sectores regulados
		\2 Idea final
			\3 Medio ambiente
			\3 Cambio tecnológico
			\3 Unión Europea
			\3 España

\end{esquema}

\esquemalargo

\begin{esquemal}
	\1[] \marcar{Introducción}
		\2 Contextualización
			\3 Economía y microeconomía
				\4 Definición de Robbins
				\4[] Decisiones respecto a bienes escasos
				\4[] $\to$ Con usos alternativos
				\4[] $\to$ Para satisfacer necesidades humanas
				\4 Microeconomía
				\4[] Estudio de decisiones a nivel individual
				\4[] $\to$ Empresas
				\4[] $\to$ Consumidores
				\4[] $\to$ Gobiernos
			\3 Justificación de la intervención pública
				\4 Marco básico de funcionamiento
				\4[] Marco legal de actuación
				\4[] Reducir incertidumbre de agentes económicos
				\4[] Garantizar derechos de propiedad
				\4 Eficiencia
				\4[] Presencia de fallos de mercado
				\4[] $\to$ Asignaciones ineficientes en sentido de Pareto
				\4 Equidad
				\4[] Sociedad realiza juicios de valor
				\4[] sobre deseable de asignaciones
				\4[] $\to$ Actúa para cambiarlas
				\4 Estabilización
				\4[] Suavizar fluctuaciones cíclicas
				\4[] Reducir impacto de shocks sobre bienestar
			\3 Instrumentos de actuación
				\4 Empresas públicas
				\4[] Ordenación de factores productivos
				\4[] directamente por el Estado
				\4[] $\to$ Proveer bienes y servicios
				\4 Presupuesto público
				\4[] Recaudar fondos mediante ingresos públicos
				\4[] Distribuirlos mediante gasto público
				\4 Regulación
				\4[] Disposiciones legales y reglamentarias
				\4[] Cumplimiento mediante poder coactivo
				\4[] Determinación de:
				\4[] $\to$ Precios
				\4[] $\to$ Esquemas tarifarios
				\4[] $\to$ Cantidades a proveer
				\4[] $\to$ Condiciones de provisión
				\4[] $\to$ Esquemas de incentivos
				\4[] $\to$ Características de ByS provistos
				\4[] $\to$ ...
			\3 Regulación
				\4 Disposiciones legales y reglamentarias
				\4 Cumplimiento mediante poder coactivo
				\4 Determinación de:
				\4[] $\to$ Precios
				\4[] $\to$ Esquemas tarifarios
				\4[] $\to$ Cantidades a proveer
				\4[] $\to$ Condiciones de provisión
				\4[] $\to$ Esquemas de incentivos
			\3 Liberalización
				\4 Regulación también tiene inconvenientes
				\4 Mercado puede inducir resultados Pareto-superiores
				\4[] Bajo determinadas condiciones
				\4 Liberalización de sectores
				\4[] En términos generales
				\4[] $\to$ Reducción de la intervención del estado
				\4[] $\then$ Eliminación de regulación
				\4[] $\then$ Provisión privada de bienes y servicios
				\4[] $\then$ Introducción de mecanismos de mercados
				\4 Particularidades ligadas a industria concreta
				\4[] Industrias contestables
				\4[] Industrias de red
		\2 Objeto
			\3 ¿Qué es la regulación?
			\3 ¿Por qué es necesaria la regulación?
			\3 ¿Cuáles son los objetivos de la regulación?
			\3 ¿Qué problemas enfrenta el regulador?
			\3 ¿En qué consiste la teoría tradicional de la regulación?
			\3 ¿Qué es la nueva teoría de la regulación?
			\3 ¿En qué consiste la liberalización de mercados regulados?
			\3 ¿Qué formas de liberalización pueden aplicarse?
			\3 ¿En qué marco institucional se llevan a cabo los procesos de liberalización?
			\3 ¿Qué opciones de liberalización se pueden aplicar a una industria de red?
		\2 Estructura
			\3 Análisis teórico de la regulación
			\3 Liberalización de sectores regulados
	\1 \marcar{Análisis teórico de la regulación}
		\2 Bienes públicos
			\3 Idea clave
				\4 No excluibles
				\4[] Excluir a determinados agentes del consumo
				\4[] $\to$ No es posible
				\4 Bienes no rivales
				\4[] Consumo por un agente
				\4[] $\to$ No reduce oferta disponible para otro
				\4 Realmente pueden subsumirse en externalidades
				\4[] Bienes públicos como externalidades positivas
				\4[] $\then$ Insuficiente incentivo a provisión privada
			\3 Formulación
				\4 Dado:
				\4[] $\to$ Acceso libre al bien producido
				\4[] $\to$ Sin competencia por su consumo
				\4 Incentivos a provisión subóptima
				\4[] $\to$ Agentes no tienen incentivos a pagar por bien
				\4[] $\to$ Free-riding en el consumo del bien público
				\4[] $\then$ Financiación privada insuficiente
				\4[] $\then$ Provisión subóptima
			\3 Implicaciones
				\4 Dos grandes métodos de solución
				\4[] Provisión pública directa del bien
				\4[] $\to$ Financiada a través de ingresos públicos
				\4[] Provisión privada regulada/subvencionada
				\4[] $\to$ Empresa privada siguiendo directivas públicas
				\4[] $\then$ Provee bien de manera compatible con óptimo
				\4 Regulación de la provisión privada
				\4[] Sector público fija un precio de provisión
				\4[] Subvenciona consumo/producción de bien
				\4 Regulación del consumo privado
				\4[] Racionamiento
				\4[] Límites a capturas máximas
		\2 Externalidades
			\3 Idea clave
				\4 Producción y/o consumo de un bien
				\4[] $\to$ Afecta a terceros que no participan
				\4[] $\then$ Prod/cons. no internalizan efectos sobre terceros
				\4 Regulación de externalidades
				\4[] $\to$ Internalización de coste público
			\3 Formulación
				\4 Precios no reflejan coste social
				\4[] $\to$ No reflejan costes externos de consumo/producción
				\4 Decisiones de agentes individuales
				\4[] $\to$ Consistentes con precios
				\4[] $\then$ Efe
				\4[] $\to$ Decisión de agentes privados no induce óptimo
			\3 Implicaciones
				\4 Definición de derechos de propiedad
				\4[] $\to$ Sobre efectos externos
				\4[] $\then$ Mercado privado sobre efectos externos
				\4[] $\then$ Agentes pueden alcanzar óptimo de Pareto
				\4 Impuestos pigouvianos
				\4[] $\to$ Alteran precios relativos
				\4[] $\to$ Distorsionan decisión de agentes privados
				\4[] $\then$ Distorsión induce decisión óptima
				\4 Provisión directa por sector público
				\4[] $\to$ Sector público conoce producción/consumo óptimo
				\4[] $\then$ Emp. públicas implementan decisión óptima
				\4 Regulación de provisión por sector privado
				\4[] Obligaciones de provisión
				\4[] $\to$ Limitaciones a la producción
				\4[] $\then$ Externalidades negativas en producción
				\4[] $\to$ Provisión obligatoria
				\4[] $\then$ Externalidades positivas en producción
				\4[] $\to$ Limitaciones del consumo privado
				\4[] $\then$ Externalidades negativas en consumo
				\4[] $\to$ Consumo mínimo obligatorio
				\4[] $\then$ Externalidades positivas en conusmo
				\4[$\then$] Requiere supervisión y coacción pública
		\2 Información imperfecta
			\3 Idea clave
				\4 Información imperfecta
				\4[] En contexto de aversión al riesgo:
				\4[] $\to$ Aumenta costes de aseguramiento
				\4[] $\then$ Eleva costes de transacción
				\4[] $\then$ Desaparición de determinados mercados
				\4 Información asimétrica
				\4[] Comportamiento estratégico de agentes
				\4[] $\to$ Poseedores de información privada
				\4[] Puede inducir equilibrios subóptimos
				\4 Soluciones
				\4[] En ocasiones, mecanismos privados pueden mitigar
				\4[] Intervención pública vía regulación
				\4[] $\to$ Puede permitir mitigar más efectivamente
			\3 Formulación
				\4 Información asimétrica
				\4[] Induce ineficiencias
				\4[] -- Selección adversa
				\4[] $\then$ Reducción de la calidad del bien intercambiado
				\4[] $\then$ Desaparición de un mercado en último término
				\4[] -- Riesgo moral
				\4[] $\then$ Principal y agente diferentes objetivos
				\4[] $\then$ Desincentiva actividades socialmente benef.
				\4 Ineficiencia
				\4[] Ag. privados no pueden implementar soluciones
				\4[] $\to$ Carecen de poder coactivo
				\4[] $\to$ Dificultades de coordinación
				\4[] $\to$ Restricciones de liquidez
				\4[] Equilibrios subóptimos
				\4[] $\to$ Mercados desaparecen
				\4[] $\to$ Calidad inferior a óptima
				\4[] $\to$ Oportunidades de mejoras paretianas imposibles
			\3 Implicaciones
				\4 Información imperfecta
				\4[] $\to$ Risk-pooling vía figuras jurídicas
				\4[] $\then$ Sociedades anónimas, reaseguros, créditos sindicados
				\4[] $\to$ Diversificación de riesgos
				\4[] $\then$ Mercados financieros organizados y regulados
				\4 Información asimétrica
				\4[] Participación forzosa en seguros
				\4[] $\then$ Evitar SAdversa
				\4[] Provisión pública de seguros y crédito
				\4[] $\to$ Inexistencia de incentivos privados a proveer
				\4[] $\then$ Sanidad pública universal
				\4[] $\then$ Consorcio de seguros
				\4[] $\then$ Crédito a emprendedores...
				\4[] $\to$ Provisión de información
				\4[] $\then$ Empresas públicas de homologación
				\4[] $\then$ Protección de patentes y marcas
				\4[] $\then$ ...
		\2 Monopolio natural con información perfecta
			\3 Idea clave
				\4 Un sólo productor es más eficiente que varios
				\4[] Menores costes de producción totales
				\4[] $\to$ Producción por monopolista es más eficiente
				\4 Caracterización matemática
				\4[] Función de costes es subaditiva
				\4[] $c(q_1 + q_2) < c(q_1) + c(q_2)$
				\4[] $\then$ Monopolio puede ser socialmente preferible
				\4[] $\then$ Necesarias = condiciones que contestabilidad
				\4 Sostenibilidad de monop. natural monoproducto
				\4[] Economías de escala es condición suficiente
				\4[] $\to$ No es condición necesaria
				\4[] $\then$ Posible monopolio natural con desec. escala
				\4 Monopolio natural multiproducto
				\4[] Complementariedad de costes es condición suficiente
				\4[] $\to$ No necesaria
				\4[] Condición de sostenibilidad
				\4[] $\to$ Precios no dan lugar a subsidios cruzados
				\4[] $\then$ Venta de algunos bienes a $P<\text{CMe}$
				\4[] $\then$ Otros bienes a $P>\text{CMe}$
				\4[] $\then$ Financia algunas pérdidas con otros beneficios
			\3 Formulación
				\4 Requisitos del monopolio natural
				\4[] Único producto
				\4[] $\to$ Cond. suficiente: economías de escala
				\4[] $\then$ Cond. suficiente: CMe decrecientes
				\4[] $\then$ No es condición necesaria
				\4[] \grafica{monopolionatural}
				\4[] Multiples productos
				\4[] $\to$ Múltiples conjuntos de condiciones nec./suf.
				\4[] $\to$ Uno de ellos:
				\4[] Complementariedad de costes:
				\4[] $\uparrow$ de producción de un bien $\then$ $\downarrow$ CMg de resto
				\4 Sostenibilidad\footnote{Ver Conceptos.}
				\4[] Posibilidad de mon. natural
				\4[] $\to$ No implica mon. natural tenga lugar
				\4[] Bajo determinados supuestos, puede ser inestable
				\4[] $\to$ Existencia de beneficio para entrante potencial
				\4[] $\to$ Libre entrada
				\4[] $\then$ Destruye monopolio natural
			\3 Implicaciones
				\4 Monopolio natural inestable
				\4[] Si produce en segmento creciente de CMe\footnote{Es decir, si produce en el segmento en el que la función de costes muestra deseconomías de escala.}
				\4[] $\to$ Entrante puede bajar precio y cantidad
				\4[] $\then$ Capturando parte del mercado
				\4[] $\then$ Siempre hay beneficio positivo potencial
				\4[] $\then$ Siempre hay incentivos a entrar
				\4 Monopolio natural estable:
				\4[] Si produce en segmento decreciente de CMe
				\4[] $\to$ El incumbente puede aumentar producción ante entrada potencial
				\4[] Si aumenta hasta CMe, no habrá incentivos a entrada
				\4[] $\then$ Entrante que baje precios pierde dinero
				\4 Necesidad de regulación
				\4[] Si equilibrio es ineficiente
				\4[] Gobierno puede incentivar una sóla empresa
				\4[] $\to$ Para asegurar monopolio natural
				\4[] $\then$ Pero deberá regular precios
			\3 Óptimo de primer orden
				\4 Monopolio natural es óptimo
				\4[] $\to$ Desde el punto de vista de minimización de costes
				\4[] Cuando es:
				\4[] $\to$ Sostenible, se mantiene sólo
				\4[] $\to$ Insostenible, necesario prohibir entrada
				\4 Monopolios que fijan precio libremente
				\4[] $\to$ Imponen precio superior a coste marginal
				\4[] $\then$ Equilibrio subóptimo frente a CPerfecta
				\4 Asumiendo monopolio natural es deseable
				\4[] ¿Cómo evitar rentas de monopolio subóptimas?
				\4[] $\then$ ¿Cómo regular monopolio natural?
				\4 Precio igual a coste marginal
				\4[] Maximiza ESocial bajo supuestos generales
				\4 Costes medios decrecientes
				\4[] Por ejemplo, por CFijo $>0$ y CMg constante
				\4[] $\to$ $\text{CMe} = \frac{\text{CF}}{q} + \text{CMe}$
				\4[] $\then$ Coste medio decrece asintóticamente
				\4[] $\then$ Coste marginal siempre inferior a coste medio
				\4 $P=\text{CMg}$ y $\text{CMe} < \text{CMg}$
				\4[] Implica pérdidas para el monopolista
				\4[] $\to$ En caso contrario, nadie provee
				\4 Posibles opciones regulatorias
				\4[] Provisión directa por sector público
				\4[] Subvención al monopolista
				\4[] $\to$ Financiación potencialmente distorsionadora
				\4[] Regulación de precio para cubrir pérdidas
			\3 Óptimo de segundo orden
				\4 Asumiendo $P=\text{CMg}$ no disponible
				\4[] $\max \quad W$
				\4[] $\text{s.a:} \quad \pi \geq 0$
				\4 Precio igual a coste medio
				\4[] Permite provisión del bien por monopolio privado
				\4[] No implica subvención
				\4[] Requiere:
				\4[] $\to$ Conocer verdadera estructura de costes
				\4[] $\to$ Incentivar revelación de costes
				\4[] Implica:
				\4[] $\to$ Menor cantidad de óptimo de 1er orden
				\4[] $\to$ Precio mayor a coste marginal
				\4 Discriminación de segundo grado
				\4[] Fijación de precios no lineales
				\4[] $\to$ Distintos precios para distintas unidades
				\4[] Permite:
				\4[] $\to$ Cubrir costes mínimos por monopolista
				\4[] $\to$ Tarifas para que consumidores se autoseleccionen
				\4[] $\to$ Aumentar bienestar social
				\4 Problemas
				\4[] Requisitos de información
				\4[] $\to$ Insuperables en práctica
				\4[] Estructuras de costes
				\4[] $\to$ Incentivos a no revelar
				\4[] $\to$ Dificultades de medición aún revelando
			\3 Multiproducto y precios lineales
				\4 También entendible como discriminación de 3er grado
				\4 Ramsey (1927), Boiteux (1956)
				\4 Un sólo mercado
				\4[] Posible fijar P=CMe
				\4 Varios mercados
				\4[] Posible fijar P=CMe en todos los mercados
				\4 ¿Es posible mejorar P=CMe en todos mercados?
				\4[] $\to$ Modular precios en función de elasticidades
				\4[] $\to$ Demandas + elásticas $\then$ + pérdida eficiencia
				\4[] $\then$ Discriminación de 3er grado puede aumentar eficiencia
				\4[] $\then$ Precio más alto donde menos afecta cantidad
				\4 Problema a resolver:
				\4[] (i) Dado un beneficio mínimo del monopolista
				\4[] ¿qué precios debe asignar para equilibrio Pareto-eficiente?
				\4[] ¿qué precios maximizan excedente social?
				\4[] (ii) Dado un excedente social mínimo
				\4[] ¿qué precios asignar para maximizar beneficio de monopolista?
				\4 Bienes con demanda inelástica
				\4[] $\to$ subir precios
				\4 Bienes con demanda elástica
				\4[] $\to$ Subidas menores
				\4 Formulación
				\4[] \fbox{$\frac{p_i - c_i}{p_i} = \frac{\lambda}{1+\lambda} \cdot \frac{1}{\left| \epsilon_i \right| - \sum_{k \neq i}^K \epsilon_{k-i}}$}
				\4[] \fbox{$\lambda = \left| \frac{\Delta \textrm{Beneficio social}}{\Delta \textrm{Beneficio del monopolista}} \right| $}: PSombra de FPúblicos
				\4[] $\eta_{ki} = \frac{d q_k}{q_k} \cdot \frac{p_i}{d p_i}$
				\4[] $\to$ Elasticidad cruzada de bien $k$ y precio de $i$
				\4[$\then$] Sustitutivos ($\epsilon_{ki} > 0$) aumentan precio de $i$.
				\4[$\then$] Complementarios ($\epsilon_{ki} <0$) bajan precio de $i$.
				\4[] Evitar el \textit{cream skimming}
				\4[] $\to$ Entrada de competidores en segmentos con beneficios
				\4[] $\then$ Necesarias regulaciones de entrada si precios
		\2 Monopolio con información imperfecta
			\3 Idea clave
				\4 Contexto
				\4[] Intervenciones anteriores
				\4[] Grandes requisitos informacionales
				\4[] En la práctica, información muy limitada
				\4 Objetivo
				\4[] Superar problema de información asimétrica
				\4[] Considerar incentivos de empresas y consumidores
				\4[] $\to$ Para inducir comportamiento óptimo
				\4[] Mitigar selección adversa y riesgo moral
				\4[] Evitar captura de rentas informacionales
				\4 Resultado
				\4[] Aproximación práctica a fijación de precios
				\4[] Sujeto a desviaciones de optim. racional
				\4[] Efectos indeseados de actuaciones
			\3 Coste del servicio o tasa de retorno
				\4 Fijar un ingreso máximo
				\4[] Empresa puede extraer ese retorno como máximo
				\4 $R = E + sK$
				\4[] $R \to$ ingreso (depende de dda. y precios fijados)
				\4[] $E \to$ gastos operativos
				\4[] $s \to$ tasa de retorno del capital permitida
				\4[] $K \to$ capital utilizado
				\4 Maximización de beneficio sujeta a restricción
				\4[] $\underset{K,L}{\max} \quad \Pi = R(K,L) - wL - rK$
				\4[] $\text{s.a:} \quad \frac{R(K,L) - wL}{K} \leq s \then \Pi \leq (s-r) \cdot K$
				\4 Revisión periódica
				\4[] $\to$ Precios
				\4[] $\to$ Valoración del capital
				\4 Varios inconvenientes
				\4[] $\to$ ¿Cómo determinar $s$ óptimo?
				\4[] $\to$ Reduce incentivos para aumentar eficiencia
				\4[] $\to$ Requisitos informativos aumentan captura de regulador
				\4[] $\to$ ¿Cómo valorar el capital?\footnote{¿Precio marginal, histórico, reposición...?}
				\4[$\to$] Lags de revisión pueden ser perjudiciales
				\4[] Dañan a consumidores si costes bajan
				\4[] Dañan a empresa si costes suben
				\4[$\to$] Efecto de Averch-Johnson\footnote{Ver pág. 1298 de Joskow (Handbook of Law and Economics).}
				\4[] Regulador fija $s$
				\4[] $\to$ menor que $r_m$ en ausencia de regulación
				\4[] $\to$ mayor que coste de capital $r$
				\4[] Incentivos a invertir en exceso en K:
				\4[] $\frac{d L}{d K} = \left| \text{RMST}_{lk} \right| = \frac{ \partial f(k) / \partial k}{\partial f(l) / \partial l} = \frac{r}{w} + \mu(r-s)$, $r < s$
				\4[] $\then$ Empresa invierte en K más que en ausencia de regulación
				\4[] Representación gráfica
				\4[] \grafica{averchjohnson}
				\4[] (Puede estimular innovación por sust. de L y K)
			\3 Reglas Price Cap
				\4 Propuesto por primera vez
				\4[] Regular British Telecom (1983)
				\4 Funcionamiento
				\4[] Fijación de una senda de precios
				\4[] Empresa decide el resto
				\4 Ventajas
				\4[] $\to$ Fuerza mejoras de eficiencia periódicas
				\4[] $\to$ Sencillez de regla dificulta captura regulatoria
				\4 Lags en revisión de precios
				\4[] $\Rightarrow$ Incentivan mayor reducción de costes
				\4 CPI-X
				\4[] Stephen Littlechild $\to$ Fórmula de Littlechild
				\4[] Senda de precios ligada a IPC menos tasa X
				\4[] X ligado a mejoras de productividad del sector
				\4 Normativa de desindexación de economía española
				\4[] Ley 2/2015 de 30 de marzo
				\4[] Desvincula revisiones de precios públicos de IPC
				\4[] $\to$ Evitar inflación de segunda ronda
				\4[] $\to$ Corregir presiones inflacionarias persistentes
			\3 Subasta de explotación
				\4 Demsetz (1968) y otros
				\4 Competencia ex-ante
				\4[] Anterior a provisión del bien o servicio
				\4 Formulación
				\4[] Participantes en subasta pujan ofreciendo:
				\4[] $\to$ Precio al que prestarán el servicio
				\4[] Administración concede explotación exclusiva
				\4[] $\to$ Quien ofrezca menor precio
				\4[] $\to$ Quien cumpla requisitos de solvencia, calidad...
				\4 Implicaciones
				\4[] Precio ganad
			\3 Yardstick competition/Competencia referencial
				\4 Shleifer (1985) inicia literatura
				\4 Ver \url{http://www.staff.city.ac.uk/~sj355/yardstick.pdf}
				\4 Comparar rentabilidad de industrias similares
				\4 Vincular rentabilidad máxima permitida
				\4[] A rentabilidad de competidores comparables
			\3 Diseño de mecanismos
				\4 Asumir información asimétrica
				\4[] Regulador tiene menos información que regulado
				\4 Tener en cuenta incentivos de regulado
				\4[] $\to$ Diseñar mecanismos que alineen objetivos
				\4 Mecanismo de Loeb-Magat
				\4[] Firma y regulador conocen demanda
				\4[] Sólo firma conoce costes
				\4[] Empresa anuncia precio
				\4[] Empresa recibe subsidio decreciente con precio
				\4[] $\then$ Firma tiende a fijar precio marginal
				\4 Campo muy fértil de investigación
				\4[] Regulación de monopolios sólo una aplicación
		\2 Problemas de la regulación
			\3 Idea clave
				\4 Intervención regulatoria
				\4[] Existencia de fallos de mercado
				\4[] $\to$ Condición necesaria pero no suficiente
				\4 Regulación también sujeta a fallos
				\4[] Aparición de trade-offs
				\4[] $\to$ Problema a solucionar
				\4[] $\to$ Problemas que introduce la regulación
			\3 Costes de la regulación
				\4 Diseño de regulación requiere
				\4[] Recogida de información
				\4[] Estimación de demandas y costes
				\4 Recursos administrativos dedicados
				\4[] No disponibles para otras actividades
			\3 Efectos colaterales no deseados
				\4 Endogeneización de la calidad
				\4[] Al establecer límites de precios máximos
				\4[] $\to$ Incentivo a ahorro de costes vía menor calidad
				\4 Ineficiencia X
				\4[] Sin presión competitiva
				\4[] $\to$ Desincentivo a maximizar eficiencia productiva
				\4 Sobrecapitalización o efecto Averch-Johnson
				\4[] Explicado más arriba
				\4[] Limitación a rendimiento máximo del capital
				\4[] $\to$ Induce inversión excesiva en capital
				\4 Aumento del gasto público
				\4[] En contexto de subvenciones y soportes de precios
				\4[] $\to$ Introducción de impuestos distorsionantes
				\4[] $\to$ Efectos macroeconómicos
				\4[] $\to$ Tensiones de liquidez y solvencia
				\4 Problemas de las subastas
				\4[] Aparición de corrupción y captura regulatoria
				\4[] Contratos incompletos y cautividad posterior
				\4 Ineficiencia dinámica
				\4[] Protección de monopolio inhibe competencia
				\4[] $\to$ Desincentivo a la innovación
			\3 Rent-seeking
				\4 Regulación puede camuflar barreras de entrada
				\4[] BdEntrada para preservar rentas de monopolio
				\4 Agentes dedican recursos a capturar rentas
				\4[] Lobbying, presiones políticas, MdComunicación...
				\4[] $\then$ Aumento del gasto improductivo
			\3 Captura del regulador
				\4 Relación habitual regulados-reguladores
				\4[] Necesaria para
				\4 Desarrollo de relaciones personales
				\4[] Inevitables a medio plazo tras contacto repetido
				\4 Circulación de trabajadores
				\4[] Regulados $\to$ Reguladores y viceversa
			\3 Asimetrías informativas en el proceso regulador
				\4 Constante en todo proceso regulatorio
				\4 Regulados adoptan comportamiento estratégico
				\4[] Proveen información
				\4[] $\to$ Tendente a infraestimar sus beneficios
				\4[] Ocultan información
				\4[] $\to$ Que muestre verdadera cuantíade rentas
		\2 Contexto institucional de la regulación
			\3 Idea clave
				\4 Posible regular p y q con herramientas variadas
				\4 Estado debe elegir herramienta óptima
				\4 En la práctica, combinaciones de herramientas
			\3 Legislación
				\4 Establecimiento directo de regulaciones
				\4 Problemas:
				\4[] Falta de cualificación técnica de legisladores
				\4[] Falta de información actualizada
				\4[] Vulnerable a grupos de interés organizados
			\3 Contratos de franquicia
				\4 Competencia \textit{por} el mercado
				\4[] En vez de competencia \textit{dentro} el mercado
				\4 Empresas compiten por derecho a monopolizar
				\4[] $\to$ Estado trata de extraer máx. de renta a monop.
				\4 Requisitos
				\4[] Número suficiente de competidores
				\4[] Actúen de forma independiente
				\4[] Subasta diseñada adecuadamente
				\4 Diseño de mecanismos aplicado a contratos de franquicia
				\4 En la práctica:
				\4[] Contratos con horizontes temporales cortos
				\4[] Mejorar credibilidad de contratos
				\4[] $\to$ Si l/p, cambios econ. hacen insostenibles
			\3 Comisiones independientes de regulación
				\4 Formadas por número reducido de miembros
				\4[] Votan sí o no regulaciones impuestas
				\4[] Apoyadas por plantilla de técnicos
				\4 Estructura cuasi-judicial
				\4 Capacidad de aprobar regulación
				\4 Independencia deseable
			\3 Provisión pública
				\4 Estado controla medios de producción
				\4[] $\to$ Provee directamente el servicio o bien
				\4 Habitual en Europa pre-privatizaciones
				\4[] Telefonía, eléctricas, ferrocarril...
			\3 Credibilidad y rendición de cuentas
				\4 Independencia determina credibilidad de regulador
				\4[] Autonomía de decisión
				\4[] $\to$ Aumenta credibilidad frente a regulados
				\4[] $\then$ Reglas más transparentes
				\4[] $\then$ Mayor seguridad jurídica
				\4[] $\then$ Más presión competitiva
				\4 Rendición de cuentas
				\4[] Regulador debe ser regulado también
				\4[] Evitar fallos del sector público
				\4[] $\to$ Captura de regulador
				\4[] $\to$ Corrupción
				\4[] $\to$ Agendas propias
				\4[] $\to$ ...
				\4 Trade-off independencia y rendición de cuentas
				\4[] Maskin y Tirole (2004)
				\4[] Necesario ponderar ambas
			\3 Política de defensa de la competencia
				\4 Complemento de liberalización
				\4 Carácter posterior y complementario a liberalización
				\4 Objetivos
				\4[] Seguimiento de actividad regulada
				\4[] Detección de incumplimientos
				\4[] Corrección de conductas ilícitas
	\1 \marcar{Liberalización de sectores regulados}
		\2 Idea clave
			\3 Contexto
				\4 Regulación trata de máx. bienestar social
				\4[] Aumentar consumo
				\4[] Reducción de rentas
				\4[] Evitar ineficiencias
				\4 Años 70
				\4[] Análisis de fallos del sector público
				\4[] Críticas crecientes a regulación
				\4[] Cambios en monopolios naturales
				\4[] $\to$ Aumento de demanda reduce ``naturalidad''
				\4[] $\to$ Análisis de contestabilidad cambia prescripciones
				\4[] $\to$ Avances tecnológicos desplazan costes
				\4 Regulación induce trade-offs
				\4[] $\to$ Fallos del sector público
				\4[] $\to$ Costes de la regulación
				\4 Eficiencia y equidad
				\4[] Mayor equidad afecta dinámicamente a eficiencia
				\4[] Coste de equidad a priori
				\4[] $\to$ Puede mejorar equidad en último término
			\3 Objetivos
				\4 Optimizar intervención regulatoria
				\4 Mejorar bienestar social
				\4 Aprovechar ventajas de fuerzas competitivas
				\4[] En provisión de bienes públicas
			\3 Resultado
				\4 Proceso generalizado en países desarrollados
				\4 Introducción de mecanismos de mercado
				\4[] En sectores previamente sujetos a regulación
				\4 Privatización de sectores con presencia pública
				\4 segregación de actividades en industrias de red
				\4 Liberalización no implica necesariamente desregulación
				\4[] Liberalización puede implicar mejor regulación
		\2 Introducción de mecanismos de mercado
			\3 Contratación externa
				\4 Idea clave
				\4[] Contratación con sector privado
				\4[] $\to$ Para la producción de bienes y servicios
				\4[] Sector público regula/ejecuta provisión directa al ciudadano
				\4 Formulación
				\4[] Sector público
				\4[] $\to$ Determina cantidad a proveer
				\4[] $\to$ Contrata con sector privado
				\4[] $\to$ Canaliza recursos hacia sector privado
				\4[] Sector privado
				\4[] $\to$ Produce bien o servicio
				\4[] $\to$ Pone a disposición de sector público
				\4 Implicaciones
				\4[] Mecanismos de mercado aplicados a producción
				\4[] $\to$ Obtención de inputs en mercados privados
				\4[] $\to$ Gestión privada del proceso productivo
				\4[] Necesaria competencia en sector privado
				\4[] En caso contrario privatización de monop. público
				\4 Características de bien deben ser especificables
				\4[] Sujeto a dificultades habituales de contratos
				\4[] $\to$ Información asimétrica
				\4[] $\to$ Incertidumbre
				\4[] $\to$ Complejidad del contrato
				\4[] $\to$ Imposible especificar todos los estados
				\4 Ejemplos
				\4[] Hospitales públicos de gestión privada
				\4[] $\to$ Carteras de servicios exigibles
				\4[] $\to$ Gestión privada de la prestación de servicios
				\4[] Externalización de actividades no esenciales
				\4[] $\to$ Servicios de limpieza, cátering...
				\4[] $\then$ De hospitales o establecimientos públicos
				\4[] Infraestucturas con elevados costes fijos
				\4[] $\to$ Empresas privadas construyen y financian
				\4[] $\to$ Estado otorga a cambio explotación de l/p
				\4[] $\then$ Autopistas son ejemplo típico
			\3 Concesión de derechos de propiedad: cuotas
				\4 Idea clave
				\4[] Bien común
				\4[] $\to$ No excluible
				\4[] $\to$ Rival
				\4[] Derechos de propiedad mal definidos
				\4[] $\to$ Implican sobreexplotación y congestión
				\4[] Gobierno concede derechos de propiedad
				\4[] $\to$ Agente privado gestiona acceso a mercado
				\4[] $\then$ Extrae beneficio sujeto a disciplina de mercado
				\4[] Gobierno extrae renta de agente privado
				\4[] $\to$ Vía coste de concesión de derechos
				\4 Formulación
				\4[] Concesión de cuota/derecho de explotación
				\4[] Diferentes métodos de concesión
				\4[] $\to$ Subasta
				\4[] $\to$ Decisión administrativa
				\4[] $\to$ Concurso de méritos
				\4[] $\to$ ...
				\4 Implicaciones
				\4[] En teoría:
				\4[] $\to$ Eficiencia final no depende de asignación inicial
				\4[] $\to$ Asignación inicial sí tiene efectos redistributivos
				\4[] Asignación de DPropiedad a quienes sufren externalidad
				\4[] $\to$ Implica los que contaminan deben pagar
				\4[] $\then$ Incentiva reducción de costes vía menos externalidad
				\4[] $\then$ Incentivos a reducción de externalidades
				\4 Ejemplos
				\4[] Mercados de derechos de emisiones
				\4[] Emissions Trading System de Unión Europea
				\4[] Subastas de espectro radioeléctrico
				\4[] Derechos de pesca y caza
			\3 Competencia entre unidades públicas
				\4 Idea clave
				\4[] Creación de mercados internos a la Admón. pública
				\4[] Diferentes unidades compitan por recursos
				\4[] Basando competencia en indicador cuantificable
				\4 Formulación
				\4[] Fijar indicador de valoración
				\4[] Definir estructura de incentivos
				\4[] Recursos a recibir en función de indicador
				\4 Implicaciones
				\4[] Competencia virtuosa entre administraciones
				\4[] $\to$ Transparencia en asignación de recursos
				\4[] $\to$ Incentivos a mejorar indicador considerado deseable
				\4[] Degeneración de incentivos
				\4[] $\to$ Actuación de Admón. encaminada a indicador
				\4[] $\to$ Indicador mal definido
				\4[] $\then$ Resultados no deseables
				\4 Ejemplo
				\4[] -- Políticas activas de empleo
				\4[] Más fondos en función de resultados anterireos
				\4[] -- Universidades de excelencia
				\4[] Reciben más fondos en función de
				\4[] $\to$ Objetivos de investigación
				\4[] $\to$ Publicaciones
				\4[] $\to$ Alumnado internacional
				\4[] $\to$ ...
			\3 Vales, bonos o cheques
				\4 Idea clave
				\4[] Asignación de crédito a agentes privados
				\4[] Crédito disponible para bien con externalidad positiva
				\4[] Agente decide uso óptimo bien
				\4[] Consumo genera externalidad positiva
				\4 Formulación
				\4[] Administración asigna crédito para consumo
				\4[] Unidades de consumo por agente
				\4[] $\to$ Dadas características subjetivas
				\4[] Agente decide donde consumir dada oferta
				\4[] Administración asigna recursos a oferta
				\4[] $\to$ Según demanda de agentes que recibieron crédito
				\4 Implicaciones
				\4[] Estímulos a eficiencia de proveedores
				\4[] $\to$ Consumidores votan con los pies
				\4[] $\then$ Mejor servicio a agentes demandantes
				\4[] $\then$ Más recursos disponibles para la administración
				\4[] Elección de agentes privados con inform. privada
				\4[] $\to$ Determina reparto de recursos
				\4[] Posibles problemas de equidad
				\4[] $\to$ Proveedores pueden exigir aportaciones complementarias
				\4[] $\then$ Sólo rentas altas elegirán mejores proveedores
				\4[] $\then$ Desigualdad en calidad y cantidad
				\4[] Gastos de gestión pueden aumentar
				\4[] $\to$ Acreditación de proveedores
				\4[] $\then$ Garantizar calidad mínima ex-ante
				\4[] Seguimiento de la actividad efectivamente realizada
				\4[] $\to$ Supervisión de calidad ex-post
				\4[] Uso adecuado de cheques a familias
				\4[] $\to$ Evitar aparición de mercados secundarios
				\4 Ejemplo
				\4[] Cheque escolar
				\4[] Libre elección de centro hospitalario
			\3 Modalidades de copago: precios públicos, tasas
				\4 Idea clave
				\4[] Consumidores de servicios públicos
				\4[] Financian parcialmente el coste del servicio
				\4 Implicaciones
				\4[] Racionalización de la demanda de servicio
				\4[] Usuarios toman conciencia del verdadero coste
				\4[] Transmisión de señales útiles
				\4[] Muestran disposición a pagar del usuario
				\4[] Instrumento de financiación
				\4[] Menos presión sobre cuentas públicas
				\4 Ejemplos
				\4[] Matrículas universitarias
				\4[] Copago farmacéutico y sanitario
				\4[] Precios públicos en general
		\2 Contestabilidad
			\3 Idea clave
				\4 Contexto
				\4[] Baumol (1982)
				\4[] Baumol, Panzar y Willig (1982)
				\4[] Bailey y Baumol (1984)
				\4[] Regulación de monopolios naturales
				\4[] $\to$ Considerada casi axiomática
				\4[] Presión pro-desregulación
				\4[] $\to$ Friedman, Hayek
				\4[] $\to$ Escuela de Chicago en 70s
				\4 Objetivo
				\4[] Mostrar condiciones bajo las cuales
				\4[] $\to$ Monopolios naturales pueden alcanzar eq. óptimo
				\4[] $\to$ Ec. de escala no necesariamente anticompetitivas
				\4 Resultado
				\4[] Requisitos de contestabilidad
				\4[] No se prescribe libre mercado incondicionalmente
			\3 Formulación
				\4 Mercado perfectamente contestable
				\4[] No hay precio de equilibrio
				\4[] $\to$ Que permita entrante potencial rebajar precio
				\4[] $\then$ Y obtener un beneficio positivo
				\4[] Luego precio de equilibrio es tal que
				\4[] $\to$ No hay incentivos a entrar en mercado
				\4[] $\then$ Precio iguala coste medio
				\4[] $\then$ Una sola empresa produce
				\4 Requisitos
				\4[] Ausencia de barreras de entrada
				\4[] Sin costes hundidos
				\4[$\then$] No hay precios que induzcan beneficio persistente
				\4[] Siempre puede entrar un nuevo competidor y eliminarlos
			\3 Implicaciones
				\4 Mercados perfectamente contestables
				\4[] Benchmark de comparación
				\4[] $\to$ Para estructuras alejadas de competencia perfecta
				\4[] $\then$ Monopolios y oligopolios
				\4[] A diferencia de competencia perfecta
				\4[] $\to$ Sí pueden alcanzarse
				\4 Amenaza de hit-and-run induce $P=\text{CMe}$
				\4[] Producir cantidad $q_c$ tal que $\text{CMe}(q_c) = P(q_c)$
				\4[] $\then$ Beneficios iguales a cero
				\4[] $\then$ Eliminación de beneficio económico
				\4[] $\then$ Mejora eficiencia frente a monopolio sin entrada
				\4[] $\then$ Deseable que monopolios sean contestables
				\4[] $\then$ Si no hay CMe mínimo, $P>\text{CMg}$
				\4 Regulaciones y retrasos en entrada
				\4[] Permiten a incumbente mantener precios altos
				\4[] $\then$ Obtener beneficios de monopolio
			\3 Valoración
				\4 Fuerte influencia en teoría regulatoria
				\4 Competencia perfecta pierde peso como benchmark
				\4[] Carece de sentido en muchos mercados
				\4 Contestabilidad como objetivo más realista
		\2 Proceso de liberalización
			\3 Condicionantes a cumplir
				\4 Evitar concentración excesiva
				\4[] Vertical y horizontal
				\4[] Poder de mercado
				\4[] $\to$ No necesaria e unívocamente ineficiente
				\4[] $\to$ Si excesivo, seguramente ineficiente
				\4 Diseño de mercados transparentes
				\4[] Productores y consumidores
				\4[] $\to$ Deben poder acceder a información disponible
				\4[] Prohibición de prácticas anticopmetitivas
				\4[] $\to$ Publicidad engañosa
				\4[] $\to$ Abusos de posición dominante
				\4[] $\to$ Colusión
				\4[] Promoción de estándares y normas técnicas
				\4[] $\to$ Puede contribuir a transparencia
				\4[] $\to$ Puede aumentar también la carga regulatoria
				\4 Desarrollo institucional suficiente
				\4[] Órgano regulador debe poder promover
				\4[] $\to$ Funcionamiento efectivo del mercado
				\4[] Suficientes:
				\4[] $\to$ Independencia de presión política y empresas
				\4[] $\to$ Recursos técnicos y financieros
			\3 Problemas identificados en la práctica
				\4 Costes de transición a la competencia
				\4[] Costes varados
				\4[] $\to$ Costes de difícil o imposible recuperación
				\4[] $\to$ Empresas incumbentes incurrieron
				\4[] $\to$ Potenciales entrantes no deben incurrir
				\4[] $\then$ Red distribución eléctrica
				\4[] $\then$ Plantas de generación
				\4[] $\then$ Contratos a l/p con expectativas de más ingresos
				\4[] Necesidad de compensar a empresas que operaban
				\4[] $\to$ Que realizaron inv. específicas/activos varados
				\4[] $\then$ Enfrentan nuevo contexto jurídico
		\2 Industrias de red
			\3 Idea clave
				\4 Contexto
				\4[] Liberalización en sentido general
				\4[] $\to$ Promoción de la competencia
				\4[] $\to$ Eliminación de barreras de entrada
				\4[] Industrias de red
				\4[] $\to$ Activo fijo diferenciado esencial de enorme coste
				\4[] $\to$ Activos no esenciales de pequeño coste
				\4[] Avances tecnológicos
				\4[] $\to$ Mercados organizados electrónicos
				\4[] Aumento de demanda
				\4[] $\to$ Explosión demográfica
				\4[] $\then$ Monopolios naturales dejan de serlo
				\4 Objetivo
				\4[] Incorporar mecanismos de mercado
				\4[] Desvincular monopolio natural de segmento competitivo
				\4[] Implementar presión competitiva en sectores regulados
				\4 Resultado
				\4[] Relativamente exitosa introducción
				\4[] Monopolios públicos en países desarrollados
				\4[] $\to$ Disgregados
				\4[] $\to$ Privatizados
				\4[] $\to$ Liberalizados
			\3 Aspectos generales de industrias de red
				\4 Aspecto común a industrias de red
				\4[] $\to$ Verticalmente integrados
				\4 Dos componentes diferenciables:
				\4[--] Red
				\4[] $\to$ Economías de escala
				\4[] $\then$ Potencialmente no competitivo
				\4[--] Segmento competitivo
				\4[] $\to$ Economías de escala menor tamaño
				\4[] $\then$ Barreras de entrada naturales reducidas
				\4 Factores de monopolización de la red
				\4[--] Razones de costes
				\4[] $\to$ Economías de escala y alcance
				\4[] $\then$ Evitar duplicar elevados costes fijos
				\4[--] Razones de demanda
				\4[] $\to$ Externalidades de red
				\4[] $\then$ Demanda crece cuando aumenta su consumo
				\4[--] Razones históricas
				\4[] $\to$ Primera empresa tiene ventaja en costes
				\4[] $\then$ Capital humano y técnico acumulado
			\3 Modelos de separación de segmentos
				\4 Separación de la propiedad
				\4[] Separación vertical
				\4[] $\to$ Competitivo
				\4[] $\to$ No competitivo
				\4[] Implicaciones
				\4[] $\to$ Evita discriminación entre entrantes
				\4[] $\to$ Impide ecs. de escala en integración vertical
				\4[] Ejemplo clásico
				\4[] $\to$ Mercado eléctrico español
				\4[] $\then$ REE: gestión de la red
				\4[] $\then$ Endesa, Iberdrola...: generación y comercialización
				\4 Propiedad compartida de operador no competitivo
				\4[] Participantes en segmento competitivo
				\4[] $\to$ Propietarios compartidos de red
				\4[] Implicaciones
				\4[] $\to$ Relación más flexible entre segmentos
				\4[] $\to$ Falta de incentivos a discriminar
				\4[] $\to$ Colusión más fácil de implementar
				\4[] $\to$ Relación compleja si hay muchos socios
				\4[] Ejemplo clásico
				\4[] $\to$ Propiedad de slots deaeropuertos
				\4[] $\then$ Compartida entre principales aerolíneas
				\4 Separación operativa
				\4[] Gestión de la red separada de competitivos
				\4[] $\to$ Tercero ajeno a competitivo gestiona red
				\4[] $\to$ Competitivos participan parcialmente en propiedad
				\4[] $\then$ Propiedad y control de red separados
				\4[] $\to$ Múltiples formas societarias posibles
				\4[] $\then$ Posible entidad sin ánimo de lucro
				\4[] Híbrido entre las dos formas anteriores
				\4[] Implicaciones
				\4[] $\to$ Impide colusión
				\4[] $\to$ Posible captura del gestor por competitivos
				\4[] $\to$ Necesario esquema de incentivos para gestor
				\4[] $\then$ Deseable participación en beneficios por gestor
				\4 Separación en partes recíprocas
				\4[] No competitivo separado en sectores pequeños
				\4[] Dentro de cada sector
				\4[] $\to$ Competitivo y competitivo juntos
				\4[] Sectores interconectados recíprocamente
				\4[] Implicaciones
				\4[] $\to$ Cierto grado de competencia en no competitivo
				\4[] $\to$ Competencia referencial por comparación
				\4 Límites a la competencia
				\4[] Límites a habilidad de integrada vertical
				\4[] $\to$ Para competir en segmento competitivo
				\4 Ejemplo:
				\4[] Restricción de capacidad
				\4[] $\to$ En su actividad en segmento competitivo
				\4 Otras vías de separación
				\4[] -- Separación contable
				\4[] -- Separación de funciones
				\4[] -- Separación corporativa
				\4[] No promueven ni protegen competencia
				\4[] Sí pueden facilitar
				\4[] $\to$ Determinación de precios de acceso
				\4[] $\to$ Detección de subsidios cruzados
			\3 Regulación de acceso al input no competitivo
				\4 Dueño de red/segmento no competitivo
				\4[] $\to$ Compite también en segmento competitivo
				\4 Acceso a input esencial por entrantes a competitivo
				\4[] $\to$ Determinado por regulador
				\4[] $\then$ Precio
				\4[] $\then$ Volúmenes
				\4 Propiedad no necesariamente pública
				\4[] $\to$ Sí necesario regular acceso
				\4[] Posible competencia de incumbente en competitivo
				\4[] $\to$ Posee red no competitiva
				\4[] $\to$ Compite en segmento competitivo
				\4[] Objetivos
				\4[] $\to$ Regla de determinación de tarifa
				\4[] $\to$ Lograr uso eficiente de instalaciones
				\4[] $\to$ Evitar entrada excesiva
				\4[] $\to$ No desincentivar inversiones
				\4[] $\to$ Incentivar mejoras futuras de la red
				\4[] $\to$ Evitar duplicaciones
				\4[] Resultados
				\4[] $\to$ Trade-off entre precio alto y bajo
				\4[] Tarifas elevadas
				\4[] $\to$ Barreras de entrada a competidores
				\4[] $\then$ En segmento competitivo
				\4[] $\to$ Refuerza poder de mercado de propietario
				\4[] $\to$ Si precios muy elevados
				\4[] $\then$ Incentiva duplicación ineficiente de red
				\4[] Tarifas baratas
				\4[] $\to$ Entrada excesiva e ineficiente
				\4[] $\to$ Desincentiva mantenimiento de red
			\3 Competencia por el segmento no competitivo
				\4 Idea clave
				\4[] Licitación de acceso al componente monopolístico
				\4[] Agentes compiten por explotación de no competitivo
				\4[] Estado extrae rentas de monopolio
				\4 Formulación
				\4[] Subasta de primer precio
				\4[] $\to$ Participantes entregan oferta en sobre cerrado
				\4[] $\then$ Admón. adjudica a mayor precio ofrecido
				\4[] Subasta inglesa
				\4[] $\to$ Se parte de precio inicial
				\4[] $\to$ Subastador va aumentando
				\4[] $\then$ Adjudicado cuando nadie supera
				\4[] Subasta holandesa
				\4[] $\to$ Se parte de precio inicial
				\4[] $\to$ Subastador va reduciendo
				\4[] $\then$ Adjudicado cuando alguien supera
				\4 Implicaciones
				\4[] Captura de rentas de monopolio a favor de estado
				\4[] Incentivo a eficiencia
				\4[] Aumento de la transparencia
				\4[] $\to$ Si subasta competitiva e informada
				\4[] Problemas cuando otros factores son relevantes
				\4[] $\to$ Proyecto tecnológico
				\4[] $\to$ Solvencia financiera
				\4[] $\to$ Cobertura de población
				\4[] $\then$ ¿Cómo ponderar diferentes factores?
				\4[] $\then$ ¿Qué criterio es relevante?
				\4[] Posible colusión de licitadores
				\4[] $\to$ ¿Cómo evitar acuerdo colisivo?
				\4[] Periodo de concesión debe ser largo
				\4[] $\to$ Incentivar inversiones y cuidado de equipos
				\4[] Periodo de concesión debe ser corto
				\4[] $\to$ Evitar desviaciones de condiciones de mercado
	\1[] \marcar{Conclusión}
		\2 Recapitulación
			\3 Análisis teórico de la regulación
			\3 Liberalización de sectores regulados
		\2 Idea final
			\3 Medio ambiente
			\3 Cambio tecnológico
				\4 Economía de plataformas
				\4 Redes sociales
			\3 Unión Europea
				\4 Competencias regulatorias
			\3 España
				\4 Marco regulatorio
				\4 CNMC
\end{esquemal}


\graficas


\begin{axis}{4}{Representación gráfica del efecto Averch-Johnson: la restricción sobre la rentabilidad del capital induce una intensidad de capital superior a la que maximiza el beneficio.}{K}{$\pi$}{averchjohnson}
	% Beneficio en función de capital
	\draw[-] (0,0) to [out=70,in=180](2.2,3) to [out=0,in=100](3.75,0);
	\node[left] at (2,3.1){\tiny $\pi(K)$};
	
	% Restricción sobre la rentabilidad
	\draw[-] (0,0) -- (4,2.5);
	\node[right] at (4,2.5){\tiny $(s-r)K)$};
	
\end{axis}


\conceptos



\preguntas

\seccion{4 de abril de 2017}
\begin{itemize}
    \item ¿Qué características tiene el modelo de Akerloff?
    \item ¿Qué fallos de mercado hay en el mercado de arrendamientos de vivienda?
    \item ¿Ha oído alguna vez el término paternalismo libertario?
    \item ¿Sabe qué son los modelos optim in optim out?
    \item Desarrolle la teoría de la captura del regulador.
    \item ¿Qué relación hay entre las políticas regulatorias y las TIC?
    \item Métodos de valoración de las políticas regulatorias.
\end{itemize}

\seccion{Test 2013}

\textbf{8.} Cuando el regulador fija precios utilizan mecanismos de precios máximos como la regla IPC-X quiere garantizar:
\begin{itemize}
	\item[a] Eficiencia asignativa porque maximiza el excedente del consumidor.
	\item[b] Eficiencia productiva porque la empresa tiene incentivos a invertir en mejoras tecnológicas que reduzcan sus costes.
	\item[c] Fijarle precios a la empresa para que no actúe como un monopolista.
	\item[d] Una correcta distribución de la renta.
\end{itemize}

\notas

\textbf{2013:} \textbf{8.} B

\bibliografia

Mirar en Palgrave:
\begin{itemize}
	\item advertising
	\item anti-trust enforcement
	\item Averch-Johnson effect
	\item communications
	\item \textbf{competition}
	\item competition and selection
	\item contract theory
	\item contracting in firms

	\item corporations
	\item economic organization and transaction costs
	\item energy economics
	\item entrepreneurship
	\item firm boundaries
	\item firm, theory of the
	\item ideal output
	\item \textbf{industrial organization}
	\item industrial relations
	\item information sharing among firms
	\item international coordination of regulation
	\item labour-managed firms
	\item \textbf{marginal and average cost pricing}
	\item \textbf{market structure}
	\item monopoly
	\item \textbf{natural monopoly}
	\item oligopoly
	\item predatory pricing
	\item price discrimination (theory)
	\item profit and profit theory
	\item \textbf{public utility pricing and finance}
	\item \textbf{regulation}
	\item \textbf{regulation and deregulation}
	\item trade and environmental regulations
\end{itemize}

Joskow, P. L; \textit{Ch. 16 Regulation of Natural Monopoly} Handbook of Law and Economics vol. 2 (2007) En carpeta del tema

Ricketts, M. (2002) \textit{The Economics of Business Entreprise. An Introduction to Economic Organisation and the Theory of the Firm} Third Edition. Edward Elgar Publishing Limited -- En carpeta del Organización Industrial

Schleifer, A. \textit{State versus Private Ownership} (1998) Journal of Economic Perspectives -- En carpeta del tema

\end{document}
