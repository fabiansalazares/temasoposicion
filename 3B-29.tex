\documentclass{nuevotema}

\tema{3B-29}
\titulo{Economía de los países en desarrollo. Teoría del desarrollo.}

\begin{document}

\ideaclave

Añadir \href{https://scholar.google.es/scholar?hl=es&as_sdt=0\%2C5&q=The+O-Ring+Theory+of+Economic+Development&btnG=}{Teoría O-Ring de Kremer (1992)} en Análisis agregado: análisis teórico

\seccion{Preguntas clave}

\begin{itemize}
	\item ¿Qué características comunes tienen los países en desarrollo?
	\item ¿Qué aporta una teoría del desarrollo diferenciada de la teoría del crecimiento?
	\item ¿Qué modelos teóricos tratan de explicar los diferentes niveles de desarrollo?
	\item ¿Qué implicaciones de política económica se derivan?
	\item ¿Qué evidencia empírica existe sobre las políticas de desarrollo?
	\item ¿En qué situaciones son efectivas?
	\item ¿Qué factores determinan su efectividad?
\end{itemize}

\esquemacorto

\begin{esquema}[enumerate]
	\1[] \marcar{Introducción}
		\2 Contextualización
			\3 Desigualdades de renta per cápita
			\3 Transferencia de ahorro y tecnología
			\3 Teoría del crecimiento económico
			\3 Teoría del desarrollo
		\2 Objeto
			\3 ¿Qué características comunes tienen los países en desarrollo?
			\3 ¿Qué aporta una teoría del desarrollo diferenciada de la teoría del crecimiento?
			\3 ¿Qué modelos teóricos tratan de explicar los diferentes niveles de desarrollo?
			\3 ¿Qué implicaciones de política económica se derivan?
			\3 ¿Qué evidencia empírica existe sobre las políticas de desarrollo?
			\3 ¿En qué situaciones son efectivas?
			\3 ¿Qué factores determinan su efectividad?
		\2 Estructura
			\3 Economías en desarrollo
			\3 Enfoque macroeconómico
			\3 Microeconomía del desarrollo
	\1 \marcar{Economías en desarrollo}
		\2 Idea clave
			\3 Contexto
			\3 Objetivo
			\3 Resultados
		\2 Hechos estilizados
			\3 Stock de capital per cápita muy pequeño
			\3 Bajo nivel de industrialización
			\3 Baja productividad total de los factores
			\3 Ahorro insuficiente
			\3 Exportaciones muy poco diversificadas
			\3 Especialización en bienes de bajo valor añadido
			\3 Elevada desigualdad
			\3 Dualidad sectorial
			\3 Sector agrícola tiene mucho peso
			\3 Fuerte crecimiento de la población
			\3 Sector financiero formal poco desarrollado
		\2 Evolución hacia el desarrollo
			\3 Esperanza de vida
			\3 Transiciones demográficas
			\3 Acumulación de capital
			\3 Productividad total de los factores
			\3 Sendas de crecimiento heterogéneas
			\3 Curva de Kuznets
	\1 \marcar{Análisis macroeconómico}
		\2 Idea clave
			\3 Contexto
			\3 Objetivo
			\3 Resultados
		\2 Modelos de crecimiento económico
			\3 Idea clave
			\3 Modelo de Harrod-Domar
			\3 Modelo de Solow-Swan/neoclásico
			\3 Crecimiento endógeno
			\3 Implicaciones
		\2 Sectores múltiples
			\3 Idea clave
			\3 Modelo dual de Lewis
		\2 Efectos acumulativos
			\3 Idea clave
			\3 Formulación
			\3 Implicaciones
			\3 Valoración
		\2 Equilibrios múltiples
			\3 Idea clave
			\3 Formulación
			\3 Implicaciones
			\3 Valoración
		\2 Sector exterior y restricciones
			\3 Idea clave
			\3 Estructuralismo y centro-periferia
			\3 EOI vs ISI
			\3 Teoría de la restricción externa
		\2 Modelos de complementariedad tecnológica/O-Ring
			\3 Idea clave
			\3 Formulación
			\3 Implicaciones
			\3 Valoración
		\2 Estabilidad macroeconómica
			\3 Idea clave
			\3 Formulación
			\3 Implicaciones
			\3 Valoración
		\2 Instituciones
			\3 Idea clave
			\3 Formulación
			\3 Implicaciones
			\3 Valoración
		\2 Financiación del desarrollo
			\3 Idea clave
			\3 Instrumentos
			\3 Implicaciones
			\3 Valoración
	\1 \marcar{Microeconomía del desarrollo}
		\2 Idea clave
			\3 Contexto
			\3 Objetivos
			\3 Resultados
		\2 Experimentos aleatorizados
			\3 Idea clave
			\3 Formulación
			\3 Implicaciones
			\3 Valoración
		\2 Alternativas a RCTs
			\3 Corrección de Heckman
			\3 Variables instrumentales
		\2 Ejemplos de aplicación
			\3 Mercados de crédito
			\3 Conflicto
			\3 Educación
		\2 Valoración
			\3 Ética de la experimentación en políticas
			\3 Oposición de opinión pública a aleatorización
			\3 Estudios observacionales frente a aleatorizados
			\3 Evaluación de impacto
			\3 Propuestas en investigación micro-desarrollo
	\1[] \marcar{Conclusión}
		\2 Recapitulación
			\3 Economías en desarrollo
			\3 Análisis macroeconómico
			\3 Microeconomía del desarrollo
		\2 Idea final
			\3 Contribución de España y UE
			\3 Instituciones internacionales
			\3 Objetivos de desarrollo

\end{esquema}

\esquemalargo












\begin{esquemal}
	\1[] \marcar{Introducción}
		\2 Contextualización
			\3 Desigualdades de renta per cápita
				\4 Enormes diferencias permanentes
				\4 Inversión, tecnología, instituciones...
				\4[] Como determinantes
			\3 Transferencia de ahorro y tecnología
				\4 Países desarrollados $\to$ PEDs
				\4 Fin del colonialismo
				\4 Motivos altruistas
				\4 Motivos interés nacional
			\3 Teoría del crecimiento económico
				\4 Énfasis en diferentes niveles de renta pc
				\4 Crecimiento PIBpc como objetivo último
				\4[] Asume implícitamente crecimiento mejora bienestar
				\4 Poco énfasis en cuestiones distributivas
			\3 Teoría del desarrollo
				\4 Estatus peculiar en la ciencia económica
				\4 Concepto de desarrollo como punto de partida
				\4[] Concepto multidimensional
				\4[] Sin definición canónica
				\4[] $\to$ Ingreso per cápita
				\4[] $\to$ Distribución del ingreso
				\4[] $\to$ Esperanza de vida
				\4[] $\to$ Educación
				\4[] $\to$ Libertades
				\4[] $\to$ Medio ambiente
				\4 Amartya Sen
				\4[] Desarrollo es medida de oportunidades de acceso
				\4[] $\to$ A bienes que es razonable valorar
				\4 Toma conceptos y modelos de muchas áreas
				\4[] $\to$ Economía laboral
				\4[] $\to$ Crecimiento económico
				\4[] $\to$ Economía de la salud
				\4[] $\to$ Economía política
				\4[] $\to$ ...
		\2 Objeto
			\3 ¿Qué características comunes tienen los países en desarrollo?
			\3 ¿Qué aporta una teoría del desarrollo diferenciada de la teoría del crecimiento?
			\3 ¿Qué modelos teóricos tratan de explicar los diferentes niveles de desarrollo?
			\3 ¿Qué implicaciones de política económica se derivan?
			\3 ¿Qué evidencia empírica existe sobre las políticas de desarrollo?
			\3 ¿En qué situaciones son efectivas?
			\3 ¿Qué factores determinan su efectividad?
		\2 Estructura
			\3 Economías en desarrollo
			\3 Enfoque macroeconómico
			\3 Microeconomía del desarrollo
	\1 \marcar{Economías en desarrollo}
		\2 Idea clave
			\3 Contexto
				\4 Definir desarrollo
				\4[] Puede hacerse definiendo subdesarrollo
				\4 Subdesarrollo
				\4[] Niveles muy bajos en determinados indicadores
				\4[] $\to$ Mortalidad infantil
				\4[] $\to$ Esperanza de vida
				\4[] $\to$ Educación
				\4[] $\to$ Desigualdad
				\4[] $\to$ PIBpc en PPA
				\4[] Otros indicadores no definen subdesarrollo
				\4[] $\to$ Stock de capital
				\4[] $\to$ Crecimiento de la población
				\4[] $\to$ Tasa de ahorro
				\4[] $\to$ Tasa de inversión
				\4[] $\to$ ...
				\4[] $\then$ Instrumental para explicar subdesarrollo
				\4 Características de PEDs
				\4[] Muy heterogéneos
				\4[] No hay característica definitoria
				\4[] Sí hay algunos rasgos habituales
				\4 Transición a desarrollo
				\4[] Diferentes vías
				\4[] Heterogeneidad regional
				\4[] Algunos elementos comunes
			\3 Objetivo
				\4 Caracterizar rasgos habituales en PEDs
				\4 Describir evolución hacia desarrollo
			\3 Resultados
				\4 Evolución heterogénea
				\4[] Entre países aparentemente similares
				\4 Algunas características comunes o frecuentes
				\4 Outliers en casi todas las áreas
		\2 Hechos estilizados
			\3 Stock de capital per cápita muy pequeño
				\4 Capital físico
				\4[] Muy escaso
				\4[] Obsoleto
				\4 Capital humano
				\4[] Baja tasa de acumulación
				\4[] Escolarización débil
			\3 Bajo nivel de industrialización
				\4 Poca capacidad productiva en bienes de inversión
				\4 Industria poco intensiva en capital
			\3 Baja productividad total de los factores
				\4 Asignaciones ineficientes de factores productivos
				\4 Muy escasa capacidad de reasignación de inputs
			\3 Ahorro insuficiente
				\4 Muy poca capacidad para financiar inversión
				\4 Ahorro exterior debe suplir
				\4[] Exposición a crecimiento de deuda externa
			\3 Exportaciones muy poco diversificadas
				\4 Especialización en sectores de bajo VAñadido
				\4 Materias primas tienen elevado peso en export.
				\4 Exposición a $\uparrow$ $\downarrow$ en precios de commodities
			\3 Especialización en bienes de bajo valor añadido
				\4 Materias primas
				\4 Exportaciones agrícolas
				\4 Industria pesada
				\4 Textil
			\3 Elevada desigualdad
				\4 Sectores con rentas propias de desarrollados
				\4 Gran mayoría con rentas muy bajas
				\4 Propensión a conflictos y rent-seeking
			\3 Dualidad sectorial
				\4 Característica habitual
				\4[] Agricultura-industria es dualidad paradigmática
				\4[] $\to$ No exclusiva
				\4[] $\then$ También dualidad en contexto urbano
				\4[] $\then$ Industria y servicios alto VA
				\4[] $\then$ Servicios urbanos de muy bajo VAñadido
				\4 Sector con muy baja productividad
				\4[] Excedentario en trabajo
				\4[] Muy bajo capital por trabajador
				\4[] Absorbe exceso de empleo
				\4[] Sector agrícola como ejemplo paragdimático
				\4[] $\to$ En ciudades, servicios de bajo VA
				\4[] $\then$ Limpiacristales, comida callejera, mendicidad...
			\3 Sector agrícola tiene mucho peso
				\4 Absorbe exceso de empleo
				\4 Muy baja capitalización de producción agrícola
			\3 Fuerte crecimiento de la población
				\4 Tasas de natalidad se mantiene elevada
				\4 Mortalidad baja fuertemente a nivel global
				\4[$\then$] Exceso de mano de obra
				\4[$\then$] Reducción del capital por trabajador
			\3 Sector financiero formal poco desarrollado
				\4 Baja participación en sistema financiero
				\4 Tipos de interés muy elevados en sistema formal
				\4 Restricciones a cantidades prestadas
				\4 Selección adversa
				\4[] Tipos altos expulsan a prestatarios solventes
				\4[] $\to$ Peor calidad media de proyectos a financiar
				\4 Riesgo moral
				\4[] Riesgos excesivos en proyectos financiados por sector formal
				\4[] Aumenta pérdidas de sector formal
				\4[] $\to$ Reducción el crédito
		\2 Evolución hacia el desarrollo
			\3 Esperanza de vida
				\4 Aumento fuertemente correlacionado con $\uparrow$ PIBpc
				\4[] Especialmente gen fases iniciales
				\4 Relación discutida entre PIB y esperanza de vida
				\4[] ¿Más PIB porque más esperanza de vida?
				\4[] ¿Más esperanza de vida porque más PIB?
			\3 Transiciones demográficas
				\4 Inicialmente
				\4[] Altas tasas de mortalidad y natalidad
				\4[] $\to$ Estacionarias
				\4[] $\then$ Población apenas varía
				\4 Transición es periodo en el que:
				\4[] Tasa de natalidad cae con cierto retraso
				\4[] $\to$ Mortalidad cae antes que natalidad
				\4[] $\then$ Crecimiento de población durante transición
				\4 Cada vez más cortas
				\4[] Primeras transiciones demográficas
				\4[] $\to$ Europa Occidental, Estados Unidos
				\4[] $\then$ Mucho más largas que actuales
				\4[] $\then$ Menor desfase entre $\downarrow$ mortalidad y natalidad
			\3 Acumulación de capital
				\4 Stocks de capital aumenta de forma persistente
				\4 Capital se concentra geográficamente
			\3 Productividad total de los factores
				\4 Explica parte relevante de crecimiento de PIBpc
			\3 Sendas de crecimiento heterogéneas
				\4 Trayectorias de crecimiento de l/p muy variables
				\4 Equilibrios múltiples
				\4[] Fases alternas de:
				\4[] $\to$ Fuerte crecimiento
				\4[] $\to$ Estancamiento
				\4[] $\to$ Recesiones largas y profundas
				\4 Crecimiento sostenido
				\4[] Estados Unidos, Reino Unido, noroeste de Europa
			\3 Curva de Kuznets
				\4 Relación entre desigualdad de ingreso y PIBpc
				\4[] Muestra U invertida
				\4[] $\to$ Aumento inicial de la desigualdad
				\4[] $\to$ Caída a partir de cierto nivel de renta
				\4 Evidencia empírica favorable
				\4[] Barro (2000)
				\4[] Barro (2008)
				\4 Pero no explica diferencias en desigualdad
	\1 \marcar{Análisis macroeconómico}
		\2 Idea clave
			\3 Contexto
				\4 Variables agregadas
				\4[] PIB, empleo, inflación, VAB sectorial...
				\4 Disponibilidad de datos estadísticos macro
				\4[] A partir de años 40
				\4 Apertura exterior
				\4[] Tras IIGM, proceso progresivo
				\4 Capacidad de procesamiento
				\4[] Limitada hasta 80s
				\4[] Conjuntos de microdatos muy poco disponibles
			\3 Objetivo
				\4 Explicar desarrollo a partir de variables agregadas
				\4 Formular políticas macroeconómicas óptimas para desarrollo
			\3 Resultados
				\4 Primera aproximación
				\4[] Modelos de crecimiento económico
				\4 Diferentes familias de modelos posteriores
				\4[] Idiosincráticos a EDesarrollo
				\4 Influencias heterogéneas y heterodoxas
		\2 Modelos de crecimiento económico
			\3 Idea clave
				\4 Contexto
				\4[] Macroeconomías representadas como un todo
				\4[] Heterogeneidad abstraída en un sólo sector
				\4[] Énfasis en:
				\4[] $\to$ Acumulación de factores de producción
				\4[] $\to$ Crecimiento tecnológico
				\4 Objetivos
				\4[] Entender efecto de ff.pp. sobre crecimiento
				\4[] $\to$ Hasta qué punto es responsable
				\4[] $\to$ Qué parte no puede explicar
				\4 Resultados
				\4[] Punto de partida de análisis
				\4[] Análisis de convergencia
				\4[] $\to$ En qué contexto es posible
				\4[] $\to$ Qué factores causan
			\3 Modelo de Harrod-Domar
				\4 Idea clave
				\4[] Crecimiento equilibrado
				\4[] $\to$ Relación óptima constante entre capital y output
				\4[] $\then$ Inversión para ajustar a relación óptima
				\4[] Población crece a tasa exógena
				\4[] Crecimiento del PIB depende de
				\4[] $\to$ Aumento de la inversión
				\4[] $\to$ Tasa de ahorro $\to$ Multiplicador del gasto
				\4 Formulación
				\4[] Crecimiento de la población
				\4[] $\to$ $L(t) = L(0) \cdot e^{nt}$
				\4[] $\then$ $\frac{d \, L}{L} = n \cdot d \, t$
				\4[] Relación óptima capital-output
				\4[] $\to$ $\frac{K}{Y} = v$
				\4[] Inversión óptima
				\4[] $\to$ $I=v \cdot \frac{d \, Y}{d \, t}$
				\4[] Crecimiento del output
				\4[] $\to$ $\frac{d \, Y}{d \, t} = \frac{d \, I}{s}$
				\4[] Crecimiento equilibrado del output
				\4[] $\to$ $I = v\cdot \frac{d \, I}{s}$ $\to$ $\frac{d\, I}{I} = \frac{s}{v}$
				\4[] Economía en senda de crecimiento equilibrado si:
				\4[] $\frac{d \, I}{I} = \frac{d \, K}{K} = \frac{d \, Y}{Y} = \frac{s}{v} = n$
				\4 Implicaciones\footnote{Ver Easterly (1997).}
				\4[] Economía siempre en ``el filo de la navaja''
				\4[] $\to$ Si shock de output positivo $\then$ Inflación
				\4[] $\to$ Si shock de output negativo $\then$ Desempleo
				\4[] $\to$ Si crecimiento de población se desvía
				\4[] Inversión óptima necesaria para desarrollo
				\4[] $\to$ ¿Cuánto hay que aumentar stock de capital?
				\4[] $\then$ Fin. exterior cubre necesidades
				\4[] Limitar crecimiento de población
				\4[] $\to$ Para evitar dinámica inestable
				\4[] Inversión para satisfacer objetivo de crecimiento
				\4 Valoración
				\4[] Ampliamente utilizado en desarrollo
				\4[] Definir inversión necesaria
				\4[] $\to$ A partir de objetivo de crecimiento\footnote{Ver Easterly (1997)}
				\4[] Brecha de financiación
				\4[] $\to$ Inversión que PED no puede financiar
				\4[] $\then$ Ayuda exterior debe cubrir
				\4[] Fuertes críticas posteriores
				\4[] $\to$ Proporción fija capital--output no tiene sentido
				\4[] $\to$ TFP más importante que stock de capital
				\4[] $\to$ K humano y organizativo más importantes que físico
				\4[] $\to$ Exceso de endeudamiento exterior
			\3 Modelo de Solow-Swan/neoclásico
				\4 Idea clave
				\4[] Función de producción
				\4[] $\to$ Rendimientos decrecientes en factores
				\4[] $\to$ Rendimientos constantes a escala
				\4[] Factores de producción
				\4[] $\to$ Trabajo: crece a tasa exógena
				\4[] $\to$ Capita: acumulable
				\4[] Rendimiento decreciente del capital
				\4[] $\to$ Convergencia a capital per cápita fijo
				\4[] $\then$ Estado estacionario
				\4[] Crecimiento per cápita en estado estacionario
				\4[] $\to$ Atribuible a cambio tecnológico exógeno
				\4[] Convergencia condicional
				\4[] $\to$ En la medida en que parámetros sean similares
				\4 Formulación
				\4[] Función de producción
				\4[] $\to$ $F(K,AL) = A(t) K(t)^\alpha L(t)^{1-\alpha}$
				\4[] Función de producción por trabajador efectivo
				\4[] $\to$ $\frac{1}{AL}F(K, AL) = F(\frac{K}{AL},1) = f(k)$
				\4[] $\to$ $f'(k) > 0, \; f''(k)<0$
				\4[] $\to$ Condiciones de Inada en $f'(k)$
				\4[] Crecimiento del trabajo
				\4[] $L(t) = L(0)\cdot e^{nt} \rightarrow \frac{\dot{L}}{L} = n$
				\4[] Efectividad del trabajo
				\4[] $A(t) = L(0)\cdot e^{gt} \rightarrow \frac{\dot{A}}{A} = g$
				\4[] Dinámica del capital
				\4[] $\dot{K} = S\cdot F(K, AL) - \delta K$
				\4[] \fbox{$\dot{k} = \frac{\dot{K}}{AL} = sf(k) - (n+g+\delta)k$}
				\4[] Capital por trabajador efectivo en EEstacionario
				\4[] $\to$ $sf(k^*) = (n+g+\delta)k^*$
				\4[] $\to$ \fbox{$k^* = \left( \frac{s}{n+g+\delta} \right)^{ \frac{1}{1-\alpha} }$}
				\4[] $\to$ \fbox{$f(k^*) = \left( \frac{s}{n+g+\delta} \right)^{ \frac{\alpha}{1-\alpha} }$}
				\4 Implicaciones
				\4[] Convergencia condicional
				\4[] $\to$ PEDs eventualmente alcanzan a desarrollados
				\4[] $\to$ Economías deben tomar parámetros similares
				\4[] Diferencias en PIBpc son temporales
				\4[] $\to$ Economías convergen a PIBpc similares
				\4[] $\to$ Shocks aleatorios pueden desviar temporalmente
				\4[] Crecimiento tecnológico
				\4[] $\to$ Permite ``escapar'' de estado estacionario
				\4 Valoración
				\4[] Evidencia empírica favorable a conv. condicional
				\4[] Punto de partida de análisis
				\4[] $\to$ Contabilidad de crecimiento
				\4[] $\to$ Crecimiento endógeno
				\4[] Trampas de crecimiento
				\4[] $\to$ Fáciles de formular
				\4[] $\to$ F. de prod. convexa a tramos
			\3 Crecimiento endógeno
				\4 Idea clave
				\4[] Progreso técnico resultado de dinámicas internas
				\4[] $\to$ No es proceso exógeno
				\4[] Economías no tienen por qué converger
				\4[] $\to$ Procesos endógenos desvían de EEstacionario
				\4 Formulación
				\4[] AK básico
				\4[] AK con convergencia
				\4[] Capital humano
				\4[] $\to$ Induce R=E, R $\uparrow$ E
				\4[] Learning-by-doing
				\4[] $\to$ Acumulación de capital mejora técnica
				\4[] Variedad de producto
				\4[] $\to$ Variedad de inputs aumenta productividad
				\4[] Modelos schumpeterianos
				\4[] $\to$ Innovación técnica genera cuasirrentas
				\4[] $\to$ Cuasirrentas inducen nuevas innovaciones
				\4 Implicaciones
				\4[] Políticas públicas afectan tasas de crecimiento
				\4[] Rendimientos en factor reproducible
				\4[] $\to$ No dependen negativamente de stock de factor
				\4[] Rendimientos en total de factores de producción
				\4[] $\to$ No son constantes(=1)
				\4[] Convergencia no tiene lugar
				\4[] $\to$ Economías no tienden a EEstacionario
			\3 Implicaciones
				\4 Acumulación de factores productivos
				\4[] Base del crecimiento
				\4 Convergencia o ausencia de ella
				\4[] Debate fundamental
				\4 Macroeconomías como cajas negras
				\4[] Escasa atención a estructura interna
				\4[] $\to$ Un gran ``sector'' productivo
		\2 Sectores múltiples
			\3 Idea clave
				\4 Contexto
				\4[] Economías son múltiples sectores en interacción
				\4[] Modelos de crecimiento
				\4[] $\to$ No consideran generalmente estructura sectorial
				\4[] $\to$ Cómo máximo, diferentes funciones de producción
				\4[] Economías en desarrollo
				\4[] $\to$ Diferencias sectoriales muy acentuadas
				\4[] Evidencia empírica sobre evolución sectorial
				\4[] $\to$ Crecimiento inicial del sector industrial
				\4[] $\to$ Sector agrario reduce \% en VAB y trabajo
				\4[] $\to$ Transición posterior de industria a servicios
				\4[] $\then$ Participación industrial forma de U invertida
				\4[] Evidencia empírica sobre demanda agregada
				\4[] $\to$ Ley de Engel
				\4[] $\to$ Aumento del consumo público
				\4[] $\to$ Disminución de bienes agrícolas como \% de gasto
				\4[] $\to$ Aumenta demanda de importaciones
				\4 Objetivos
				\4[] Entender impacto de estructura sectorial
				\4[] $\to$ Sobre desarrollo económico
				\4[] Formular política económica sobre sectores
				\4[] $\to$ Que tenga impacto favorable sobre desarrollo
				\4 Resultados
				\4[] Interacción recíproca
				\4[] $\to$ Desarrollo económico
				\4[] $\to$ Cambio de estructura sectorial
				\4[] Diferencias en crecimiento de diferentes sectores
				\4[] $\to$ Relevante sobre proceso de desarrollo
				\4[] Dualidad sectorial como problema
				\4[] $\to$ Hirschmann, Leibestein, Lewis, Myrdal, Prebisch...
				\4[] $\to$ Economías nacionales y mundial bisectoriales
				\4[] Inspiración de estructuralismo posterior
			\3 Modelo dual de Lewis
				\4 Idea clave
				\4[] Economías divididas en dos sectores
				\4[] Trasferencia de ff.pp. de un sector a otro
				\4[] $\to$ Potencial aumento de output
				\4[] $\to$ Desencadena dinámica de crecimiento
				\4 Formulación
				\4[] Sector con excedente de trabajo (agrícola)
				\4[] $\to$ Muy baja productividad media
				\4[] $\to$ Productividad marginal próxima a cero
				\4[] $\to$ Salarios agrícolas por encima de PMgL
				\4[] $\to$ Propietarios agrícolas demandan bienes industriales
				\4[] $\to$ Menos trabajadores, más excedente para capital
				\4[] $\then$ Menos trabajadores no implica menor output
				\4[] Sector con necesidades de capital (industrial)
				\4[] $\to$ Productividad marginal del trabajo elevada
				\4[] $\to$ Precios por encima de coste
				\4[] $\then$ Dueños del K obtienen excedente
				\4[] $\then$ Excedente se reinvierte
				\4[] Agotamiento de trabajadores agrarios
				\4[] $\to$ Inicia fase de aumento de salarios
				\4 Implicaciones
				\4[] Transición agrícola a industria
				\4[] $\to$ No tiene por qué reducir output
				\4[] Crecimiento de output como mejora paretiana
				\4[] $\to$ Hasta agotar
				\4[] Inversión y reinversión de excedente industrial
				\4[] $\to$ Es elemento de crecimiento
				\4[] Desarrollo como transformación sectorial
				\4[] $\to$ Eliminar excedentes de trabajo disponible
				\4[] Fases del desarrollo
				\4[] $\to$ En la primera, ff.pp. se desplazan
				\4[] $\to$ En la segunda, aumentan salarios y consumo
				\4 Valoración
				\4[] Predicción explica fenómeno recurrente
				\4[] $\to$ Transición agrícola-industria
				\4[] Reinversión de excedentes parcialmente
				\4[] $\to$ Dinámicas de reinversión
		\2 Efectos acumulativos
			\3 Idea clave
				\4 Myrdal (1957), Kaldor (1957), Arrow (1962)
				\4 Productividad genera aumento del output
				\4[] Aumento del output mejora productividad
				\4 Diferentes mecanismos:
				\4[] $\to$ Innovación tecnológica
				\4[] $\to$ Economías de escala dinámicas
				\4 Fenómenos de refuerzo y feedback induce:
				\4[] $\to$ Aceleración del crecimiento
				\4[] $\to$ Trampas de pobreza
			\3 Formulación
				\4 Simplificación con dos ecuaciones
				\4 Output:
				\4[] $\to$ \fbox{$\dot{y} = \alpha + \beta \dot{x}$}
				\4 Productividad:
				\4[] $\to$ \fbox{$\dot{x} = \gamma + \delta \dot{y}$}
				\4 Crecimiento del output:
				\4[] $\to$ \fbox{$\dot{y} = \alpha + \beta \gamma + \beta \delta \dot{y}$}
				\4[] $\to$ $\dot{y} = \frac{\alpha + \beta \gamma}{1-\beta \delta}$
			\3 Implicaciones
				\4 Tasa de crecimiento indeterminada
				\4[] $\to$ $\alpha=\gamma=0$ y $\beta \delta = 1$
				\4[] $\then$ $\dot{y} = 0 + 0 + 0 \dot{y}$
				\4[] $\then$ Compatible con cualquier tasa de crecimiento
				\4[] $\then$ Trayectoria pasada determina crecimiento
				\4 Senda de crecimiento explosivo o crecimiento nulo
				\4[] $\to$ $\alpha>0$ o $\gamma>0$ y $\beta\delta = 1$
				\4[] $\then$ Crecimiento aumenta cada vez más
				\4[] $\then$ Senda con $\dot{y}=$ también es equilibrio
				\4[] $\then$ Posible crecimiento o estancamiento
				\4 Si $\alpha>0$ o $\gamma >0$ y $\beta \delta < 1$
				\4[] $\to$ Una única senda de crecimiento única y estable
				\4 Otras sendas de crecimiento posibles
				\4[] Impulsos iniciales son necesarios
				\4 Eslabonamientos son importantes
				\4[] $\to$ Promocionar industrias con mayores eslabonamientos
			\3 Valoración
				\4 Concepto relevante en múltiples modelos
				\4 Desarrollo económico fruto de impulso clave
				\4[] $\to$ ``Desata'' dinámica de acumulación y progreso
		\2 Equilibrios múltiples
			\3 Idea clave
				\4 Múltiples equilibrios locales estables
				\4[] $\to$ Algunos de ellos, trampas de pobreza
				\4 Necesario impulsos para superar estabilidad
				\4[] $\to$ Alcanzar equilibrios superiores
				\4 Desarrollo como superación de equilibrios locales
				\4[] $\to$ Trampas de pobreza
			\3 Formulación
				\4 Impaciencia ante factores propios a subdesarrollo
				\4[] $\to$ Corrupción
				\4[] $\to$ Ineficiencia del estado
				\4[] $\to$ Limitada esperanza de vida
				\4[] $\to$ Inseguridad frente a riqueza futura
				\4[] $\to$ Tasas de consumo elevadas
				\4[] $\then$ Ahorro insuficiente
				\4[] $\then$ Inversión insuficiente
				\4 Imperfecciones en mercados financieros
				\4[] $\to$ Límite a ahorro disponible para inversión
				\4[] $\to$ Uso ineficiente del ahorro
				\4 Crecimiento demográfico elevado
				\4[] $\to$ Aumento de renta induce aumento de población
				\4[] $\to$ Fertilidad endógena
				\4[] $\then$ Capital per cápita se mantiene bajo
				\4[] $\then$ PIBpc no aumenta
				\4 Baja elast. sust. técnica K y L con bajo desarrollo
				\4[] $\to$ Reduce flexibilidad de economía
				\4[] $\to$ Impide adopción de tecnologías más productivas
				\4 Rendimientos de escala dinámicos
				\4[] $\to$ Pequeño tamaño de economía impide realización
				\4 Especialización insuficiente
				\4[] $\to$ Economías de escala
			\3 Implicaciones
				\4 Múltiples equilibrios posibles
				\4[] Desarrollo no es fenómeno inevitable en l/p
				\4 Necesario big-push para salir de eq. de pobreza
				\4[] $\to$ Impulso inicial a crecimiento
				\4[] $\then$ Rosestein-Rodan: Macro-programa de inversión
				\4[] $\then$ Nurkse: Activad demanda intersectorial
				\4[] $\then$ Hirschman: activad dda. sectores con eslabonamientos
				\4 Necesaria intervención pública
				\4[] Para estimular demanda
				\4[] Para realizar economías de escala
				\4[] Para eliminar barreras a ``big-push'
			\3 Valoración
				\4 Compatible con Solow
				\4 Medidas de pol. económica muy heterogéneas
				\4[] Justificables muchas medidas
		\2 Sector exterior y restricciones
			\3 Idea clave
				\4 Contexto
				\4[] Harrod-Domar y modelo neoclásico
				\4[] $\to$ Poca consideración por sector exterior
				\4[] Concepción keynesiana de la demanda
				\4[] $\to$ Autónoma
				\4[] $\to$ Inducida
				\4[] Demanda como motor de crecimiento
				\4[] $\to$ Sector agrícola demanda productos industriales
				\4[] $\then$ Economías de escala estáticas y dinámicas
				\4[] Pequeñas economías domésticas
				\4[] $\to$ Limitan demanda de prod. industriales nacionales
				\4[] $\then$ Demanda exterior potencialmente mucho mayor
				\4[] $\then$ Importancia de sector exterior en desarrollo
				\4 Objetivos
				\4[] Caracterizar relación sector exterior y estructura
				\4[] Entender relación entre desarrollo y subdesarrollo
				\4[] Entender efecto de especialización comercial
				\4[] $\to$ Sobre desarrollo
				\4 Resultados
				\4[] Sector exterior como determinante de desarrollo
				\4[] Estructura interna de economía
				\4[] $\to$ Determina relaciones económicas exteriores
				\4[] $\then$ Determinan desarrollo
				\4[] Análisis dinámico de comercio exterior
				\4[] $\to$ Especialización
				\4[] $\to$ Cambios en estructura
				\4[] Comercio exterior puede causar subdesarrollo
				\4[] $\to$ Dinámicas de especialización
				\4[] $\to$ Elasticidad-renta de dda. exportaciones de PEDs
				\4[] $\to$ Bajo crecimiento de productividad en sect.
				\4[] PEDs deben especializarse en sector industrial
				\4[] $\to$ Empresas industriales tienen poder de mercado
				\4[] $\to$ Elasticidad-renta favorable a industriales
				\4[] $\then$ Espec. agrícola mantiene subdesarrollo
				\4[] Políticas de sustitución de importaciones
				\4[] $\to$ Estimular industria nacional
			\3 Estructuralismo y centro-periferia
				\4 Idea clave
				\4[] Análisis neoclásico del sector exterior no considera
				\4[] $\to$ Efectos de incertidumbre sobre exportaciones
				\4[] $\to$ Dificultades para reorganizar ff.pp.
				\4[] $\then$ Exagera ventajas libre CI en PEDs
				\4[] Análisis neo-marxista
				\4[] $\to$ Exagera costes de dependencia exterior
				\4[] $\to$ Minimiza beneficios de transferencia tecnológica
				\4[] Estructuralismo
				\4[] $\to$ Restricciones internas y CInternacional interaccionan
				\4[] $\to$ Efectos recíprocos
				\4[] CEPAL, Prebisch, Singer, Furtado, Pinto, Chenery...
				\4 Formulación
				\4[] Análisis de equilibrio general
				\4[] $\to$ Desarrollo afecta precios mundiales
				\4[] Economía mundial dividida en dos
				\4[] Centro
				\4[] $\to$ Especializada en productos industriales
				\4[] $\to$ Alta productividad industrial
				\4[] $\to$ Aprovecha bajo coste de materias primas
				\4[] $\to$ Capital muy rentable si invertido
				\4[] Periferia
				\4[] $\to$ Especializada en agricultura y materias primas
				\4[] $\to$ Demandan productos industriales
				\4[] Deterioro progresivo de relación real de intercambio
				\4[] $\to$ Ley de Engel reduce precio de exportaciones
				\4[] $\to$ Demanda creciente de productos industriales
				\4 Implicaciones
				\4[] Necesario diversificar estructura exportadora
				\4[] $\to$ Menor impacto de incert. en CInternacional
				\4[] Necesario atraer capital y aumentar exportaciones
				\4[] $\to$ Cuellos de botella para sector industrial
				\4 Valoración
			\3 EOI vs ISI
				\4 Idea clave
				\4[] Políticas económicas para lograr desarrollo
				\4[] $\to$ En contexto de economías abiertas
				\4[] Fuerte influencia estructuralista y dependencia
				\4 ISI
				\4[] Import-substitution industrialization
				\4[] Énfasis en producir nacionalmente
				\4[] $\to$ Productos industriales a priori importados
				\4[] Cuotas y aranceles a importaciones industriales
				\4[] $\to$ Protección de sustitutivos domésticos
				\4[] Latinoamérica, India
				\4 EOI
				\4[] Énfasis en desarrollo de industria exportadora
				\4[] Protección arancelaria/cuotas a potenciales exportadores
				\4[] Sudeste asiático
				\4 Valoración
			\3 Teoría de la restricción externa
				\4 Idea clave
				\4[] Modelo keynesiano de economía cerrada
				\4[] $\to$ Output determinado por demanda interna
				\4[] Harrod (1933) y otros
				\4[] $\to$ Demanda externa es determinante de output
				\4[] $\then$ Saldo exterior determina crecimiento de output
				\4[] Thirwall (1977)
				\4[] $\to$ Formalización de Harrod
				\4 Formulación
				\4[] Equilibrio de cuenta corriente en l/p
				\4[] $\to$ $XP = MP^*$
				\4[] $\to$ Posibles desequilibrios en el corto plazo
				\4[] $\to$ Asumidas imposibles en medio/largo plazo
				\4[] $\then$
				\4[] Demanda de exportaciones por el RM
				\4[] $X=A \cdot \left( \frac{P^*}{P} \right)^\gamma (Y^*)^\epsilon$
				\4[] Demanda de importaciones por economía nacional
				\4[] $M=B \cdot \left( \frac{P}{P^*} \right)^\eta Y^\pi$
				\4[] Crecimiento de output compatible con equilibrio
				\4[] $\to$ \fbox{$\dot{y}=\frac{(1+\gamma+\eta)(\dot{p} - \dot{p}^*)+\epsilon \dot{y}^*+ \epsilon \dot{y}^*}{\pi}$}
				\4[] Si relación relativa de intercambio constante
				\4[] $\to$ \fbox{$\dot{y} = \frac{\epsilon \dot{y}^*}{\pi}$}
				\4 Implicaciones
				\4[] Mayor output aumenta importaciones
				\4[] $\to$ Necesario aumentar exportaciones también
				\4[] Mayor elasticidad de exportaciones a renta mundial
				\4[] $\to$ Mayor crecimiento sostenible
				\4[] Mayor relación relativa de intercambio ($\dot{p} - \dot{p^*}$)
				\4[] $\to$ Permite más crecimiento
				\4[] Mayor crecimiento mundial
				\4[] $\to$ Permite mayor crecimiento
				\4[] Con especialización en agricultura
				\4[] $\to$ Elast. de exp. $\epsilon$ es baja
				\4[] $\then$ Ley de Engel
				\4[] $\then$ Desarrollo restringido por especialización
				\4[] Elast. relativa $\epsilon/\pi$ mayor en PD que PEDs
				\4[] $\to$ Exportaciones más elásticas que importaciones
				\4[] $\to$ No existe convergencia
				\4[] $\then$ Divergencia
				\4[] $\then$ Solución requiere intervención pública
				\4 Valoración
				\4[] Menos énfasis en factores de producción
				\4[] $\to$ Sector exterior es verdadero determinante
				\4[] Argumentación favorable a estructuralismo
				\4[] $\to$ RRI determina desarrollo
				\4[] $\to$ Centro--periferia como obstáculo a desarrollo
				\4[] $\then$ Necesario cambio en estructura
				\4[] No considera flujos de capital
				\4[] $\to$ Importante en determinados PEDs
				\4[] Contrastación empírica relativamente favorable
		\2 Modelos de complementariedad tecnológica/O-Ring
			\3 Idea clave
				\4 Contexto
				\4[] División de trabajo es tendencia de l/p
				\4[] Muchas actividades deben hacerse bien
				\4[] Si una falla, resto no funciona bien
				\4[] Mejoras en una etapa del proceso
				\4[] $\then$ Mejoran todo el resultado global
				\4 Objetivos
				\4[] Efecto sobre desarrollo de países
				\4[] $\to$ Cuellos de botella en determinadas actividades
				\4[] $\to$ Heterogeneidad de cualificación, habilidades
				\4[] $\to$ Disrupción de procesos productivos
				\4[] Explicar flujos descapitalizadores de K humano
				\4 Resultados
				\4[] Kremer (1993)
				\4[] Microfundamentación de desarrollo/subdesarrollo
				\4[] Explicación de subdesarrollo
				\4[] $\to$ Basada en complementariedades tecnológicas
				\4[] Funciones de producción complementarias
				\4[] Assortative matching
				\4[] $\to$ Similares se juntan con similares
			\3 Formulación
				\4 Función de producción
				\4[] $y=F(q_1, ..., q_n) = n \cdot q_1 \cdot ... \cdot q_n$
				\4[] $n$: número de tareas
				\4[] $q_i$: representa calidad de trabajo en proceso $i$
				\4 Ejemplo: 4 trabajadores y dos niveles de habilidad
				\4[] 2 trabajadores con habilidad $q_h$
				\4[] 2 trabajadores con habilidad $q_l$
				\4[] $q_h > q_l$
				\4[] Función de producción requiere dos tareas
				\4[] $\to$ Dos trabajadores
				\4 Juntando iguales niveles de habilidad
				\4[] Producción total: $y_1 + y_2 = q_h^2 + q_l^2$
				\4 Juntando distintos niveles de habilidad
				\4[] Producción total: $y_1 + y_2 = 2q_h q_l$
			\3 Implicaciones
				\4 Empresas/países buscan trabajadores de calidad similar
				\4[] País que dispone de trabs. con habilidad alta
				\4[] $\to$ Puede pagar salario alto a otro de hab. alta
				\4[] País que dispone de trabs. con habilidad baja
				\4[] $\to$ No puede pagar salario alto a otro de hab. alta
				\4[] $\then$ Países con trab. cualificado crecen más que subdesarrollados
				\4 Efectos de cuello de botella
				\4[] Escasa cualificación del trabajo en un sector
				\4[] $\to$ Reduce productividad de todo el resto
				\4[] $\then$ Subdesarrollo
				\4 Salarios más altos si trabajo complementario de más calidad
				\4[] Salarios más altos en desarrollados
				\4 Círculos virtuosos inducen desarrollo
				\4[] Más productividad aumenta productividad del resto
				\4[] Otros tienen más incentivo a aumentar su productividad
				\4[] $\to$ Dinámica virtuosa
				\4 Trabajo de alta cualificación se junta
				\4 Fuga de cerebros
				\4[] De subdesarrollados a desarrollados
				\4[] Contrario a implicaciones neoclásicas típicas
				\4[] $\to$ Asumiendo retornos decrecientes a factores
				\4[] $\to$ Capital fluye donde es más productivo
				\4[] $\then$ Cualificados deberían ir a subdesarrollados
				\4[] $\then$ En la práctica, sucede al contrario
			\3 Valoración
		\2 Estabilidad macroeconómica
			\3 Idea clave
				\4 Contexto
				\4[] Años 70
				\4[] $\to$ Aumento de precio de materias primas
				\4[] $\to$ Rentas elevadas para PEDs
				\4[] $\to$ Estrategia ISI
				\4[] $\then$ Inversiones en industria pesada
				\4[] Primero shock del petróleo
				\4[] $\to$ PEDs no exp-petróleo necesitan estimular
				\4[] $\then$ Aumento de inversión
				\4[] $\then$ Endeudamiento externo
				\4[] $\then$ Baja competitividad tras fracaso de ISI
				\4[] Crisis de deuda en los 80
				\4[] $\to$ Tras aumento de tipos
				\4[] $\to$ Crisis cambiarias
				\4 Objetivos
				\4[] Caracterizar efecto de inestabilidad
				\4[] $\to$ Sobre crecimiento en PEDs
				\4[] Definir buenas prácticas macroeconómicas en PEDs
				\4 Resultados
				\4[] Formulación en marco neoclásico
				\4[] Consenso de Washington
				\4[] $\to$ Diferentes versiones e implementaciones
			\3 Formulación
				\4[i] Disciplina presupuestaria
				\4[] Evitar inversiones ineficientes
				\4[] Reducir endeudamiento y déficits gemelos
				\4[ii] Cambios en prioridades de gasto público
				\4[] De áreas menos productivas a servicios básicos
				\4[] $\to$ Sanidad
				\4[] $\to$ Educación
				\4[] $\to$ Infraestructuras
				\4[iii] Reforma fiscal
				\4[] Aumentar tamaño de bases imponibles
				\4[] Reducir tipos marginales
				\4[iv] Tipo de cambio competitivo
				\4[] Preferencia por tipos flexibles inicialmente
				\4[] Transición a modelo bipolar posterior
				\4[] $\to$ O flexible o hard-peg
				\4[v] Apertura a IDE
				\4[] Transferencia de tecnología asociada
				\4[] Flujos de entrada menos volátiles
				\4[vi] Privatizaciones, desregulación, derechos de propiedad
				\4[] Aumentar productividad del capital
				\4[] Entorno favorable a inversión extranjera
			\3 Implicaciones
				\4 Tres lineas de énfasis
				\4[] Equilibrio macroeconómico
				\4[] Reducción del tamaño del sector público
				\4[] Mayor competitividad del sector privado
				\4 Doble ámbito micro-macro
			\3 Valoración
				\4 Resultados mixtos
				\4 Críticas
				\4[] Inestabilidad financiera por entrada de K volátil
				\4[] $\to$ Problemas de balanza de pagos
				\4[] $\to$ Ajustes bruscos dolorosos
				\4[] Mejores resultados cuando SPúblico intervino
				\4[] $\to$ Incentivos a innovación
				\4[] $\to$ Políticas sectoriales
				\4[] $\to$ Cronogramas de inversión
				\4[] $\to$ Planes de objetivos de exportaciones
				\4[] $\then$ Especialmente en Asia y Chile
				\4 Implementación generalizada
				\4[] A pesar de críticas
				\4[] $\to$ Punto de partida de políticas de desarrollo
		\2 Instituciones
			\3 Idea clave
				\4 Contexto
				\4[] Institucionalismo clásico
				\4[] $\to$ Veblen, Mitchell, Commons, Galbraith...
				\4[] $\to$ Más allá de preferencias, tecnología, dotaciones
				\4[] $\then$ Marco legal
				\4[] $\then$ Hábitos y costumbres a nivel social
				\4[] $\then$ Motivaciones en contexto social
				\4[] Nueva Economía Institucional (NIE)
				\4[] $\to$ Coase, North, Williamson,Olson...
				\4[] $\then$ Formular institucionalismo en marco neoclásico
				\4[] $\then$ Modelización matemática de instituciones
				\4[] Definición de instituciones de North (1990)
				\4[] $\to$ Reglas del juego en una sociedad
				\4[] $\to$ Restricciones del comportamiento fijadas por humanos
				\4[] $\then$ Que regulan las interacciones entre humanos
				\4[] Teoría del crecimiento
				\4[] $\to$ Acumulación de ff.pp. y tecnología
				\4[] $\then$ Causas próximas del crecimiento
				\4[] $\to$ Instituciones, geografía, naturaleza...
				\4[] $\then$ Causas de causas próximas
				\4[] $\then$ Causas profundas de crecimiento
				\4[] Teoría del desarrollo
				\4[] $\to$ Muy amplia variedad institucional en PEDs
				\4[] $\to$ Mucha heterogeneidad en desarrollo
				\4 Objetivos
				\4[] Entender relación entre instituciones y desarrollo
				\4[] Diseñar instituciones para $\uparrow$ desarrollo
				\4[] Explicar consolidación de instituciones ineficients
				\4 Resultados
				\4[] Muy difícil contrastación empírica
				\4[] $\to$ ¿Cómo experimentar con instituciones?
				\4[] $\then$ Necesarios métodos alternativos
				\4[] Experimentos naturales
				\4[] $\to$ Contrastar efectos de instituciones
				\4[] Marco de análisis general
				\4[] $\to$ Herramientas formales neoclásicas
				\4[] $\to$ Técnicas econométricas
				\4[] Instituciones importan para desarrollo
				\4[] $\to$ Reparto de la renta
				\4[] $\to$ Incentivos a acumulación de factores
			\3 Formulación
				\4 Manifestación de instituciones
				\4[] Leyes positivas
				\4[] Normas consuetudinarias
				\4[] Hábitos de agentes
				\4[] Preferencias sociales
				\4 Razón de ser de instituciones
				\4[] Economizar recursos
				\4[] $\to$ Realizar economías de escala
				\4[] $\to$ Mejorar transmisión de información
				\4[] $\to$ Mitigar fallos de mercado
				\4[] Redistribución de renta
				\4[] Resolución de conflictos
				\4[] $\to$ Canalizar conflictos por renta
				\4[] Reducir el riesgo
				\4[] $\to$ Distribuir riesgos entre agentes
				\4[] $\to$ Diversificar riesgo idiosincrático
				\4 Estructura institucional
				\4[] Conjunto de instituciones interconectadas
				\4 Instituciones políticas
				\4[] Determinan distribución de la renta futura
				\4 Distribución de la renta
				\4[] Otorgan poder político
				\4[] Poder político sirve para configurar instituciones
			\3 Implicaciones
				\4 Instituciones son importantes
				\4[] Especialmente en PEDs
				\4 Acemoglu, Johnson y Robinson (2001)
				\4[] Colonias con instituciones coloniales extractivas
				\4[] $\to$ Mucho menor desarrollo en presente
				\4[] Colonias con instituciones europeas
				\4[] $\to$ Mayor desarrollo en presente
				\4[] Mortalidad de colonos como instrumento
				\4[] $\to$ Determina tipo de instituciones coloniales
				\4 Derechos de propiedad
				\4[] Resultados robustos sobre importancia
				\4[] $\to$ Relación habitual con desarrollo
				\4[] Pero difícil desentrañar relación causal
				\4[] $\to$ ¿Desarrollo causa de buena protección?
				\4 Instituciones son endógenas a desarrollo
				\4[] Distribución de renta
				\4[] $\to$ Determina instituciones
				\4[] Poder político de facto y de iure
				\4[] $\to$ Modulan evolución institucional
				\4 Experimentos naturales
				\4[] Permiten establecer causalidad
				\4[] Ejemplos:
				\4[] $\to$ Corea del Norte y del Sur
				\4[] $\to$ Ex-colonias en latitudes similares
				\4 Instrumentos\footnote{Ver \href{http://www.econ.nyu.edu/user/benhabib/Acemoglu-Johnson-Robinson3.pdf}{NYU sobre Acemoglu, Johnson y Robinson} (en carpeta del tema)}
				\4[] Herramienta econométrica
				\4[] Correlación entre dos variables
				\4[] $\to$ No implica causalidad
				\4[] $\to$ Puede resultar de causalidad inversa
				\4[] $\then$ Explicada causa también explicativa
				\4[] $\to$ Puede existir causa común a dos variables
				\4[] Encontrar variable instrumental
				\4[] $\to$ Causa variable explicativa postulada
				\4[] $\to$ Variable explicada no afecta a instrumento
				\4[] $\then$ Regresar explicada frente a instrumento
				\4[] $\then$ ¿Hay relación entre instrumento y explicada?
				\4[] Ampliamente utilizados en análisis de desarrollo
				\4[] -- Ejemplo: Acemoglu, Johnson y Robinson (2001)
				\4[] Instituciones son determinantes de desarrollo
				\4[] $\to$ Protección de derechos de propiedad
				\4[] $\to$ Checks and balances
				\4[] $\to$ Control de poder estatal
				\4[] $\to$ ...
				\4[] D$\to$ Pero desarrollo actual podría causar instituciones actuales
				\4[] Mortalidad de colonos como instrumento de instituciones
				\4[] $\to$ Mortalidad elevada causa instituciones extractivas
				\4[] $\to$ Mortalidad en etapa colonial no debería afectar output actual
				\4[] Entorno de mortalidad elevada en periodo colonial
				\4[] $\to$ Colonos establecen instituciones extractivas
				\4[] $\then$ Instituciones estimadas a partir de instrumento
				\4[] $\then$ Más mortalidad, instituciones extractivas
				\4[] $\then$ Menos mortalidad, instituciones de más calidad
				\4[] Fase final
				\4[] $\to$ Estimar desarrollo a partir de instit. instrumentadas
				\4[] $\then$ No a partir de instituciones actuales
			\3 Valoración
				\4 Programa de investigación muy influyente
				\4[] Análisis empírico de desarrollo
				\4[] Estímulo a nuevos métodos econométricos
				\4[] $\to$ Cuasi-experimentos
				\4[] $\to$ Variables instrumentales
				\4 Marco formal de estática comparativa general
				\4[] Inexistente
				\4[] Modelos específicos
				\4[] $\then$ Difícil caracterizar qué y qué no funciona
		\2 Financiación del desarrollo
			\3 Idea clave
				\4 Contexto
				\4 Objetivos
				\4 Resultados
			\3 Instrumentos
				\4 Inversión extranjera
				\4[] Directa
				\4[] $\to$ Brownfield vs greenfield
				\4[] $\to$ Vertical vs horizontal
				\4[] Cartera
				\4 Deuda
				\4[] Préstamos bilaterales públicos
				\4[] Préstamos bilaterales privados
				\4[] Préstamos privados
				\4[] Bonos
				\4 Ayuda Oficial al Desarrollo
				\4[] Definición de OCDE
				\4[] Concesionalidad mínima necesaria
				\4[] Múltiples instrumentos
				\4[] $\to$ Donaciones
				\4[] $\to$ Reembolsable
				\4[] $\to$ Blended
				\4[] $\to$ Vinculada/desvinculada
				\4 Otros instrumentos
				\4[] Remesas
				\4[] Alianzas público-privadas
				\4[] Fondos de garantías
				\4[] Fondos soberanos
				\4[] Transformación de características temporales
				\4[] Mitigación del riesgo
			\3 Implicaciones
				\4 IDE transmite capital humano y tecnolóǵico
				\4 Flujos de cartera tienden a ser volátiles
				\4[] Pueden provocar inestabilidad financiera en destino
				\4[] Críticas a apertura exterior al capital
				\4 AOD -- Ayuda Oficial al Desarrollo
				\4[] Puede atraer proyectos privados
			\3 Valoración
				\4 Evidencia empírica muy heterogénea
				\4 Fuertemente dependiente de:
				\4[] Instituciones y coyuntura de receptor
				\4[] Tipo de instrumento utilizado
				\4 Críticas a AOD
				\4[] Incentivos perversos en donante y receptor
				\4[] Introducción de distorsiones fiscales
				\4[] Pequeña importancia cuantitativa
				\4[] $\to$ En relación a otras fuentes
	\1 \marcar{Microeconomía del desarrollo}
		\2 Idea clave
			\3 Contexto
				\4 Políticas macroeconómicas de desarrollo
				\4[] Efectos causales difíciles de valorar
				\4[] Enorme número de factores a considerar
				\4[] Interacciones muy complejas
				\4 Ayuda al desarrollo
				\4[] Opinión pública generalmente favorable
				\4[] Proporción importante de proyectos de desarrollo
				\4 Proyectos de desarrollo
				\4[] Ámbito circunscrito a área/grupo/actividad determinada
				\4[]
			\3 Objetivos
				\4 Explicar mecanismos microeconómicos
				\4[] Que inducen efectos de políticas de desarrollo
				\4 Valorar efectos de políticas sobre desarrollo
				\4[] Al nivel de agentes individuales
				\4 Fundamentar proyectos de desarrollo
			\3 Resultados
				\4 Amplia variedad de técnicas de evaluación
				\4 Enorme aumento de RCTs
		\2 Experimentos aleatorizados
			\3 Idea clave
				\4 Contexto
				\4[] Economía laboral y finanzas públicas
				\4[] $\to$ Énfasis en cuasi experimentos
				\4[] Medicina y ciencias de la salud
				\4[] $\to$ Larga tradición de experimentos aleatorizados
				\4[] $\to$ Doble ciego\footnote{Experimentos en los cuales ni el experimentador ni el sujeto de experimentación saben que quién está recibiendo el tratamiento y quién no.}
				\4[] Construcción de contrafactuales en política económica?
				\4[] $\to$ ¿Qué sucede con agente si no se le aplica política?
				\4[] $\then$ Contrafactual
				\4[] Efecto causal como diferencia entre:
				\4[] $\to$ Estado tras aplicar política/tratamiento
				\4[] $\to$ Estado si no se hubiese tratado
				\4 Objetivos
				\4[] Identificar efecto causal de políticas públicas
				\4[] $\to$ ¿Qué proyectos de desarrollo funcionan y cuales no?
				\4[] $\then$ Ej.: ¿efecto de más profesores
				\4[] Evitar causalidad inversa y autoselección
				\4 Resultados
				\4[] Enorme impacto sobre programas de desarrollo
				\4[] Dificultades para generalizar resultados
			\3 Formulación
				\4 Efecto causal de un tratamiento
				\4[] Efecto = $Y_{i1} - Y_{i0}$
				\4[] $\to$ Dif. entre $i$ tratado (1) y sin tratar (0)
				\4[] Problema:
				\4[] $\to$ Sólo es posible observar tratado o no tratado
				\4[] $\then$ No puede observarse tratamiento y no tratamiento
				\4[] $\then$ ``Contrafactual'' no es observable
				\4[] Alternativa de estimación de efecto causal:
				\4[] $\to$ Estimar efecto medio sobre dos grupos de agentes
				\4[] $\to$ Un grupo ha sido tratado
				\4[] $\to$ Otro grupo no
				\4[] $\to$ Características similares salvo estatus de tratamiento
				\4[] $\then$ Construcción de contrafactual
				\4[] $\then$ Posible aproximar evolución de no tratados sin tratamiento
				\4 Construcción de contrafactuales
				\4[] Partiendo de población base
				\4[] Dividir aleatoriamente en dos grupos
				\4[] $\to$ 0. A los que no se aplica política
				\4[] $\to$ 1. A los que se aplica política
				\4 Efecto medio local del tratamiento (LATE)
				\4[] $E(Y_i | T_i = 1) - E(Y_i | T_i = 0)$
				\4[] Diferencia entre:
				\4[] $\to$ Media en grupo de tratados
				\4[] $\to$ Media en grupo de no tratados
			\3 Implicaciones
				\4 Similares características en tratados y no tratados
				\4[] Mayor cuanto mayor tamaño de muestra
				\4 Muy variados ejemplos de aplicación
				\4[] Medicamentos antiparasitarios
				\4[] $\to$ Efecto sobre educación $\to$ Capital humano
				\4[] Microcréditos
				\4[] $\to$ Efecto positivo sobre beneficios\footnote{En empresas en percentiles altos de rentabilidad.}
				\4[] $\to$ Poco efecto de l/p sobre desarrollo humano
				\4 Experimentos naturales
				\4[] Factor exógeno determina a quién afecta política
				\4[] $\to$ No tiene relación con características de tratados
			\3 Valoración
				\4 Heterogeneidad de resultados
				\4[] Efectos pueden ser muy diferentes entre sujetos
				\4[] Catástrofe general + algunos éxitos
				\4[] $\to$ Pueden resultar en efecto medio positivo
				\4[] $\then$ Necesario examinar
				\4 Aleatorización puede ser sólo aparente
				\4[] Agentes pueden:
				\4[] $\to$ Obtener tratamiento aunque no seleccionados
				\4[] $\to$ No recibir tratamiento aunque seleccionados
				\4 Requiere herramientas complementarias
				\4[] Meta-análisis de resultados
				\4[] Artículos de replicación
				\4[] Juntas de revisión de resultados
				\4[] Registros centralizados de experimentos
				\4 Efectos de equilibrio general
				\4[] RCTs aplicados a subconjunto de población relevante
				\4[] Implementación sobre grupo más amplio
				\4[] $\to$ Tiene efectos no valorados en RCT
				\4 Tamaño de los grupos es relevante
				\4[] En esperanza matemática, RCTs muestra
				\4 Extrapolación de resultados
				\4[] RCTs muestran qué ha funcionado en ese contexto
				\4[] Pero otros factores en otros contextos
				\4[] $\to$ Pueden alterar efectos
				\4[] $\then$ Extrapolación espuria
		\2 Alternativas a RCTs
			\3 Corrección de Heckman
				\4 En contexto de muestras no aleatorias
				\4[] Ejemplo:
				\4[] $\to$ Estimar determinantes de salarios ofrecidos
				\4[] $\then$ Sólo se considera a los que trabajan
				\4[] $\then$ Muestra sesgada hacia personas con características
				\4 Dos etapas
				\4[1ª etapa] Estimar probabilidad de selección
				\4[] A partir de conjunto de variables
				\4[2ª etapa] Estimar efecto de política
				\4[] Corrigiendo por probabilidad estimada de autoselección
			\3 Variables instrumentales
				\4 En contexto de muestras no aleatorias
				\4 Estatus de tratamiento y efectos
				\4[] Pueden estar correlacionados
				\4[] $\then$ Estimación sesgada del efecto
				\4 Variable instrumental
				\4[] No relacionada con efecto de tratamiento
				\4[] Sí fuertemente relacionada con estatus
				\4 Regresión en dos etapas
				\4[] 1. Estimar estatus de tratamiento
				\4[] $\to$ A partir de instrumento
				\4[] 2. Estimar efecto de tratamiento
				\4[] $\to$ A partir de estatus estimado en 1ª fase
		\2 Ejemplos de aplicación
			\3 Mercados de crédito
				\4 Provisión de microcréditos en PEDs
			\3 Conflicto
				\4 Obligar comunidades a elegir repres. femeninos
			\3 Educación
				\4 Profesores sustitutos
				\4 Provisión de libros de texto
				\4 Subsidios condicionados a asistencia
		\2 Valoración
			\3 Ética de la experimentación en políticas
				\4 ¿Es aceptable privar a alguien de política?
				\4[] Si esa política se prevé efectiva y beneficiosa
			\3 Oposición de opinión pública a aleatorización
				\4 Difícil aceptar métodos aleatorios
			\3 Estudios observacionales frente a aleatorizados
				\4 Sesgo de autoselección
				\4[] Existe
				\4[] Puede ser corregido
				\4[] ¿Es mayor que problemas de aleatorización?
				\4 Fuerte debate metodológico
				\4[] ¿RCT son ``gold standard?
				\4[] $\to$ ¿Deben primarse frente a otros métodos?
				\4[] $\to$ ¿Evidencia derivada de RCT más valiosa que otros?
			\3 Evaluación de impacto
				\4 Necesaria aplicación de técnicas microeconométricas
				\4 En ocasiones, choca con teoría
				\4[] Teorías pueden explicar fenómeno dado
				\4[] $\to$ Pero muchos otros fenómenos se solapan
			\3 Propuestas en investigación micro-desarrollo
				\4[i] Evitar jerarquización de métodos
				\4[] Diferente adecuación a distintas circunstancias
				\4[ii] Justificar aleatorización ex-ante
				\4[] Ponderar beneficios y costes
				\4[] Valorar consideraciones éticas
				\4[iii] Explicitar supuestos subyacentes a RCTs
				\4[iv] Ir más allá de efectos causales medios
				\4[] Valorar distribución de efectos
				\4[v] Evitar dependencia excesiva de RCTs
				\4[] Considerar también otros métodos
				\4[] En muchas ocasiones, alternativas son preferibles
	\1[] \marcar{Conclusión}
		\2 Recapitulación
			\3 Economías en desarrollo
			\3 Análisis macroeconómico
			\3 Microeconomía del desarrollo
		\2 Idea final
			\3 Contribución de España y UE
			\3 Instituciones internacionales
				\4 Grupo del BM
				\4 Bancos regionales
				\4 FMI
			\3 Objetivos de desarrollo
				\4 ODS--Objetivos de Desarrollo del Milenio
				\4 ODM--Objetivos de Desarrollo Sostenible
\end{esquemal}























\preguntas

\seccion{Test 2016}
\textbf{42.} En relación a la economía del desarrollo, señale cuál de las siguientes afirmaciones es falsa:
\begin{enumerate}
	\item[a] El modelo del sector dual de Lewis asume que la oferta de mano de obra no calificada para el sector capitalista es limitada.
	\item[b] La teoría de Liebenstein sirve para justificar una clara intervención del Estado en el proceso inversor de los países en desarrollo a través de un aumento del gasto público.
	\item[c] El Club de París es un foro informal de los principales acreedores públicos bilaterales cuyas decisiones se toman por consenso y que reclama de los demás acreedores fuera del Club un trato comparable.
	\item[d] En el marco de sostenibilidad de la deuda establecido por el Banco Mundial y el FMI, los umbrales indicativos de la carga de la deuda correspondientes al desempeño firme son los más bajos.
\end{enumerate}

\seccion{Test 2005}
\textbf{46.} El Índice de Desarrollo Humano establecido por Naciones Unidas ha supuesto un paso adelante en la medición del crecimiento y del bienestar de los países. En síntesis, el índice combina en su cálculo:
\begin{enumerate}
	\item[a] El PIB, el nivel de deuda externa y la tasa de inflación.
	\item[b] El PIB, la tasa de inflación y la tasa de desempleo.
	\item[c] El PIB, la tasa de desempleo y la tasa de alfabetización.
	\item[d] El PIB, la esperanza de vida y la tasa de alfabetización.
\end{enumerate}


\notas

\textbf{2016:} \textbf{42.}

\textbf{2005:} \textbf{46.}

\bibliografia

Mirar en Palgrave:

\begin{itemize}
	\item \textbf{access to land and economic development}
	\item \textbf{agriculture and economic development}
	\item burden of the debt
	\item \textbf{city and economic development}
	\item dependency
	\item \textbf{development economics}
	\item \textbf{dual economics}
	\item \textbf{emerging markets}
	\item endogenous growth theory
	\item extreme poverty
	\item \textbf{factor misallocation and development}
	\item \textbf{financial structure and economic development}
	\item fiscal and monetary policies in developing countries
	\item flypaper effect
	\item foreign aid
	\item globalization
	\item \textbf{growth and institutions}
	\item growth take-offs
	\item \textbf{Indian economic development}
	\item international indebtedness
	\item labour surplus economies*
	\item \textbf{Latin American economic development}
	\item \textbf{Lewis, W. Arthur}
	\item \textbf{microcredit}
	\item national debt
	\item nutrition and development
	\item \textbf{poverty}
	\item poverty alleviation programmes
	\item poverty lines
	\item poverty traps
	\item public debt
	\item \textbf{regional development}
	\item regional development, geography of
	\item \textbf{religion and economic development}
	\item research and experimental development (RandD) and technological innovation policy
	\item sovereign debt
	\item \textbf{terms of trade and economic development}
	\item taxation and poverty
	\item Third World debt
	\item trade and poverty
	\item \textbf{uneven development}
	\item \textbf{Washington Consensus}
	\item World Bank
\end{itemize}

Acemoglu, D.; Johnson, S.; Robinson, J. A. (2001) \textit{The Colonial Origins of Comparative Development: An Empirical Investigation} American Economic Review. Vol. 91. No. 5 -- En carpeta del tema

Acemoglu, D.; Robinson, J. (2002) \textit{The Political Economy of the Kuznets Curve} Review of Development Economics

Acemoglu, D. (2010) \textit{Theory, General Equilibrium, and Political Economy in Development Economics} Journal of Economic Perspectives. Vol. 24. No. 3 Summer 2010 -- En carpeta del tema

Banerjee, A.; Duflo, E. (2010) \textit{Giving Credit Where It Is Due} Journal of Economic Perspectives. Vol. 24. No. 3 Summer 2010 -- En carpeta del tema

Calvo, G. A.; Mishkin, F. \textit{The Mirage of Exchange Rate Regimes for Emerging Market Countries} (2003) Journal of Economic Perspectives: Fall 2003 -- En carpeta del tema

Chenery, H. B. (1975) \textit{The Structuralist Approach to Development Policy} American Economic Review, Vol. 65. No. 2 -- En carpeta del tema

Deaton, A. (2009) \textit{Instruments of Development: Randomization in the tropics, and the search for the elusive keys to economic development} The Keynes Lecture, British Academy. October 9th 2008 -- En carpeta del tema

Deaton, A. (2010) \textit{Understanding the Mechanisms of Economic Development} Journal of Economic Perspectives. Vol. 24. No. 3 Summer 2010 -- En carpeta del tema

Easterly, W. (1997) \textit{The Ghost of Financing Gap. How the Harrod-Domar Growth Model Still Haunts Development Economics} World Bank. Policy Research Working Paper 1807 -- En carpeta del tema

Easterly, W.; Levine,R. (2001) \textit{It's Not Factor Accumulation: Stylized Factsand Growth Models} The World Bank Economic Review, Vol. 15. No. 2 -- En carpeta del tema

Golin, D. (2014) \textit{The Lewis Model: a 60-Year Retrospective} Journal of Economic Perspectives. Vol. 28. Number 3 - \url{https://www.aeaweb.org/issues/343} -- En carpeta del tema

Hesse, H. (2008) \textit{Export Diversification and Economic Growth} Commission on Growth and Development - Working Paper No. 21 -- En carpeta del tema

Kremer, M. (1993) \textit{The O-Ring Theory of Economic Development} The Quarterly Journal of Economics. Vol.108. No. 3 -- En carpeta del tema.

Krueger, A. O. (1974) \textit{The Political Economy of the Rent-Seeking Society} American Economic Review. Vol 64. No. 3

Lewis, A. (1954) \textit{Economic Development with Unlimited Supplies of Labour} The Manchester School

Ravalion, M. (2018) \textit{Should the Randomistas (Continue to) Rule?} Center for Global Development. Working Paper 492 August 2018 -- En carpeta del tema

Ray, D. (2010) \textit{Uneven Growth: A Framework for Research in Development Economics} Journal of Economic Perspectives. Vol. 24. No. 3 Summer 2010 -- En carpeta del tema

Ray, D. (2013) \textit{Notes for a Course in Development Economics} Graduate development notes en \url{https://debrajray.com/teaching-material/} -- En carpeta Desarrollo Económico

Rodrik, D. (2010) \textit{Diagnostics before Prescription} Journal of Economic Perspectives. Vol. 24. No. 3 Summer 2010 -- En carpeta del tema

Rodrik D.; Rosenzweig, M. (2010) \textit{Development Economics} Handbook in Economics. Vol. 5 -- En carpeta del tema

Rosenzweig, M. R. (2010) \textit{Microeconomics Approaches to Development: Schooling, Learning and Growth} Journal of Economic Perspectives. Vol. 24. No. 3 Summer 2010 -- En carpeta del tema

Varios autores (2010) \textit{Symposium: the Agenda for Development Economics} Journal of Economic Perspectives. Vol. 24. No. 3 Summer 2010 -- \url{https://www.aeaweb.org/issues/149}

\end{document}
