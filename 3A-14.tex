\documentclass{nuevotema}

\tema{3A-14}
\titulo{El modelo de competencia perfecta}

\begin{document}

\ideaclave

De forma general, es habitual describir la microeconomía como la rama de la economía que trata de entender y predecir el comportamiento de agentes económicos a través de la formulación de modelos teóricos y el análisis empírico. Esta comprensión del comportamiento de los agentes se articula inicialmente en torno a la teoría de la demanda del consumidor y la teoría de la empresa, que respectivamente tratan de explicar el comportamiento y los procesos de decisión de consumidores y empresas por separado. Sin embargo, tanto consumidores como empresas --y otros agentes cuyo comportamiento tiene contenido económico como autoridades monetarias, sector público, votantes, y un largo etc...- toman decisiones en un contexto de interacción con otros agentes que también toman decisiones. Si bien es posible y necesario modelizar su comportamiento como si sus decisiones se tomasen de forma aislada al resto de agentes --de tal modo que, salvo su propia decisión todo fuese exógeno al agente en cuestión- resulta también necesario y posible caracterizar el resultado de esa interacción entre agentes.

El \marcar{modelo de competencia perfecta} es uno de los modelos que tratan de caracterizar el resultado de las interacciones entre agentes en un contexto de mercado. Su objetivo consiste en describir el resultado de la competencia entre compradores y vendedores que tienen como objetivo maximizar sus preferencias dados unos recursos de los que inicialmente disponen, y que toman los precios de los bienes intercambiados como dados --es decir, como exógenos a su proceso de decisión-. La importancia del modelo de competencia perfecta y su impacto en casi todas las áreas de la economía teórica radica en el hecho de que establece una serie de condiciones bajo las cuales un equilibrio descentralizado es un óptimo de Pareto. Esto es, de manera que la búsqueda del propio interés de cada agente resulta en un equilibrio en el que no sea posible mejorar a ninguno de los agentes participantes sin perjudicar a al menos uno de los demás. Este resultado no es sino una actualización formalizada de la famosa mano invisible de Adam Smith.

La \textbf{formulación} del modelo de competencia perfecta se sustenta en una serie de definiciones previas y un conjunto de supuestos y postulados de los cuales se acaban derivando los Teoremas del Bienestar, resultados centrales de la microeconomía actual. Las \underline{definiciones} planteadas corresponden a los simples conceptos de numerario, asignación, dotación y factibilidad de una asignación. Este último concepto describe la propiedad de una asignación en la cual los totales de los bienes consumidos son iguales o inferiores a las cantidades de bienes de las que se dispone inicialmente. Los \underline{supuestos} fundamentales del modelo de competencia perfecta son \textit{i)} la ausencia de incentivos de los agentes a comportarse como si sus acciones influyesen el precio y la cantidad de equilibrio y \textit{ii)}, la información perfecta respecto a precios y cantidades. Dado el objetivo de caracterizar el modelo de competencia perfecta y los resultados que le son exclusivamente propios, resulta suficiente un modelo de equilibrio parcial que reduzca la complejidad añadida propia de los modelos de equilibrio general. Para ello, se plantean algunos supuestos adicionales que configuran el análisis como propio del equilibrio parcial y no general. Así, se asume una función de utilidad cuasilineal para obviar los efectos renta y se agrupan todos los bienes cuyos mercados no desean examinarse en un bien compuesto que a su vez actúa como numerario. Además, se impone el supuesto de carácter eminentemente técnico de la imposibilidad de controlar la reventa de los bienes adquiridos --de manera que se acabe imponiendo un precio único en el mercado.

Tras presentar estos supuestos, se introducen los tres condiciones que caracterizan el equilibrio competitivo. En primer lugar, el \underline{problema de optimización de los consumidores}, que consiste en maximizar la función de utilidad cuasilineal y que da lugar a una condición de óptimo tal que el precio habrá de ser igual a la utilidad marginal que induce el consumo del bien. En segundo lugar, el \underline{problema de maximización del beneficio de las empresas}, cuya condición de óptimo dado su carácter de precio aceptantes consiste en la igualación del precio y el coste marginal de producción. Por último, la condición de \underline{vaciado de mercados} implica que la demanda y la oferta del bien considerado se igualan eliminando cualquier exceso de demanda existente. De estas condiciones se extrae la definición de equilibrio competitivo, que no es sino un precio del bien y una asignación que las satisface. 

El concepto de \underline{equilibrio competitivo} no está indisolublemente ligado al modelo de competencia perfecta, ya que existen otros equilibrios competitivos que exceden los límites del modelo tales como aquellos en los que están presentes externalidades. Así, en la medida en que una asignación y un vector de precios satisfaga las tres condiciones, será un equilibrio competitivo. Sin embargo, dados los supuestos de \textit{i)} ausencia de externalidades, \textit{ii)}información perfecta y \textit{iii)} ausencia de poder de mercado, es posible deducir un resultado fundamental de enorme importancia en casi todas las áreas de la economía, los llamados Teoremas del Bienestar. El \underline{Primer Teorema del Bienestar} constituye la primera parte del resultado: dados los supuestos anteriores más el supuesto de no saturación de las preferencias, cualquier equilibrio competitivo será un \underline{óptimo de Pareto}. Desde otro punto de vista, el Teorema indica que en el contexto del modelo el comportamiento maximizador de los agentes permite alcanzar una situación óptima en el sentido de que no se podría mejorar el bienestar de ningún agente sin perjudicar al menos a otro. Así, la toma descentralizada de decisiones de producción y consumo arroja un resultado tan satisfactorio como el que podría alcanzar un planificador centralizado que dispusiese a voluntad las asignaciones de los consumidores. El \underline{Segundo Teorema del Bienestar} complementa el primero mostrando como --con algunos supuestos adicionales de carácter técnico- es posible alcanzar cualquier óptimo de Pareto dado simplemente reasignando las dotaciones iniciales de numerario de que disponen los agentes. El Primer Teorema puede interpretarse en términos de \underline{maximización del excedente} cuando la función de utilidad es cuasilineal, dadas la propiedad de este tipo de funciones de igualar variación equivalente y compensatoria. Así, se presenta cómo un equilibrio competitivo maximiza el espacio entre la curva de demanda y la curva de oferta que corresponde al llamado excedente social.

Tras presentar estos conceptos centrales al tema y a la microeconomía en general, se analizan brevemente las \textbf{estáticas comparativas} de dos fenómenos habituales: variaciones en la oferta y la demanda, y la introducción de impuestos de carácter específico. Ambos fenómenos tienen impacto sobre los precios y las cantidades de equilibrio y afectan de forma desigual al bienestar de consumidores y el beneficio de empresas en función de parámetros exógenos al modelo. Tras este breve análisis de estática comparativa, se enumeran brevemente las aplicaciones más relevantes del modelo de competencia perfecta tanto a nivel teórico como empírico.

De los conceptos presentados surgen dos preguntas que completan los resultados del modelo. En primer lugar, {¿qué sucede con el equilibrio si endogeneizamos el número de empresas?} Es decir, si las empresas pueden decidir no sólo qué cuantía producen sino si entran o no en el mercado de tal manera que el número de éstas no sea una cantidad exógena, {¿cuál será el equilibrio competitivo?} Para responder, se presenta el llamado \marcar{análisis de entrada o de largo plazo}, cuyo objetivo es caracterizar el equilibrio competitivo cuando las empresas son libres de entrar o salir del mercado en cuestión. En este marco de equilibrio competitivo de largo plazo, el número de empresas aparece como variable endógena y no exógena como en el análisis previo. En este contexto, el número de empresas productoras es una variable endógena que se determina teniendo en cuenta ese proceso de decisión adicional respecto a la entrada o salida del mercado. Partiendo de un modelo simple en el que no existen costes fijos ni hundidos, es posible derivar conclusiones respecto a la existencia y unicidad de un equilibrio competitivo de largo plazo. Así, la presencia de funciones de coste estrictamente convexas y estrictamente crecientes redundan en la inexistencia de equilibrios competitivos a corto plazo, mientras que los costes marginales constantes resultan en la existencia de infinitos equilibrios competitivos de largo plazo. La situación de existencia y unicidad de equilibrio competitivo de largo plazo se produce exclusivamente cuando la función de costes toma un valor de escala mínima eficiente único. Esto es, cuando el coste medio toma un valor mínimo único. Las implicaciones de este modelo se presentan muy someramente en términos de la estática comparativa de corto plazo a partir de un equilibrio de largo plazo, y del efecto de un aumento de la demanda sobre el número de empresas.

La segunda gran pregunta que completa el modelo básico de competencia perfecta concierne al proceso mediante el que se alcanza o no el equilibrio: {¿tienden los mercados descentralizados al equilibrio competitivo?} {¿cómo lo hacen?} Si el análisis estático sólo atiende a la determinación de los valores que satisfacen una serie de condiciones, el \marcar{análisis dinámico} analiza el proceso por el que se alcanzan esos equilibrios, si es que se llegan a alcanzar. Así, el análisis dinámico se fundamenta en la caracterización del proceso de ajuste temporal y su caracterización como sistema estable o inestable. Cuando partiendo de una situación alejada del equilibrio competitivo el sistema converge hacia éste con el tiempo, hablamos de un equilibrio estable. Y de forma similar, de un equilibrio inestable cuando el sistema se aleja progresivamente del equilibrio competitivo. Los posibles mecanismos de ajuste son infinitos en su variedad y complejidad, pero presentamos tres ejemplos básicos que permiten su exposición sencilla en un contexto de equilibrio parcial y que dan una idea de los supuestos habituales a la hora de analizar el problema de la estabilidad. El mecanismo de \textbf{ajuste walrasiano} consiste en una variación de precio proporcional al exceso de demanda: a mayor exceso de demanda, mayor del precio. De igual modo, a menor exceso de demanda, mayor disminución del precio. Dado que la caracterización del equilibrio como estable o inestable dado el mecanismo implica la resolución de una ecuación diferencial, se presentan de forma gráfica y para abreviar tiempo dos ejemplos en los cuales, respectivamente, el equilibrio es estable e inestable. Posteriormente se describe brevemente el ajuste \textbf{marshalliano o en cantidades}. En este tipo de mecanismo, la cantidad producida y demandada se ajusta en función de la diferencia entre el precio máximo de demanda y el precio mínimo de oferta. Por último, se presenta el \textbf{modelo de Cobweb} como representación de un fenómeno habitual en ciertos mercados: la fluctuación cíclica en torno a un precio de equilibrio, ya sea de forma convergente, divergente, o cíclicamente estable. El supuesto fundamental de este modelo es la determinación de la oferta en base al precio en un periodo anterior, de forma que existe un desajuste crónico entre oferta y demanda por la incapacidad de la primera para ajustarse de acuerdo con el desajuste de cada periodo. El modelo de Cobweb describe así un mecanismo que racionaliza el comportamiento de ciertos mercados que muestran variaciones cíclicas habituales en torno a un precio de equilibrio.

La presentación concluye con una recapitulación de lo presentado y una reflexión final acerca de la utilidad del modelo de competencia perfecta. Si bien ningún mercado o economía real satisfacen los supuestos y condiciones del modelo de competencia perfecta, la utilidad de éste trasciende el plano empírico y se manifiesta sobre todo en el plano teórico. En la medida en que es el \textit{benchmark} frente al que se comparan otros mercados en cuanto a su eficiencia y al bienestar que generan, el modelo de competencia perfecta es una de las piedras angulares de la modelización económica actual.

\seccion{Preguntas clave}
\begin{itemize}
	\item ¿Qué es el modelo de competencia perfecta?
	\item ¿En qué supuestos se basa?
	\item ¿Qué implicaciones se derivan?
	\item ¿Para qué sirve?
\end{itemize}

\esquemacorto

\begin{esquema}[enumerate]
	\1[] \marcar{Introducción} 2'
		\2 Contextualización
			\3 Microeconomía
			\3 Interacción de agentes
			\3 Competencia perfecta: modelo benchmark
		\2 Objeto
			\3 Qué es el modelo de competencia perfecta
			\3 Qué supuestos lo fundamentan
			\3 Qué implicaciones
			\3 Para qué sirve
			\3 Qué extensiones
		\2 Estructura
			\3 Modelo de competencia perfecta
			\3 Entrada y largo plazo
			\3 Ajuste dinámico
	\1 \marcar{Modelo de competencia perfecta} 12'
		\2 Idea clave
			\3 Modelo básico de interacción
			\3 Ausencia de fricciones y externalidades
			\3 Resultado: equilibrio eficiente
			\3 Equilibrio parcial y general
		\2 Formulación
			\3 Definiciones
			\3 Supuestos
			\3 Problema del consumidor
			\3 Problema de la empresa j
			\3 Vaciado de mercados
			\3 {Equilibrio competitivo}
		\2 Implicaciones
			\3 Primer Teorema Fundamental del Bienestar
			\3 {Segundo Teorema del Bienestar}
		\2 Estática comparativa
			\3 Desplazamiento de curvas de oferta y demanda
			\3 Impuesto específico
		\2 Aplicaciones
			\3 Benchmark de eficiencia
			\3 Modelización de mercados
	\1 \marcar{Entrada y largo plazo} 7'
		\2 Idea clave
			\3 Endogeneizar número de empresas
			\3 Diferentes modelos de entrada
			\3 Equilibrio competitivo de largo plazo
		\2 Formulación
			\3 Equilibrio de largo plazo
			\3 Condiciones de equilibrio
			\3 Existencia de equilibrio
		\2 Implicaciones
			\3 Aumentos demanda
			\3 Estática comparativa a corto plazo
	\1 \marcar{Ajuste dinámico} 7'
		\2 Idea clave
			\3 Análisis estático
			\3 Análisis dinámico
			\3 Estabilidad vs inestabilidad
			\3 Mecanismos de estabilidad
			\3 Relación entre estabilidad walrasiana y marshaliana
		\2 Estabilidad walrasiana
			\3 Idea clave
			\3 Formulación
			\3 Implicaciones
		\2 Ajuste marshalliano en cantidades
			\3 Idea clave
			\3 Formulación
			\3 Implicaciones
		\2 Modelo de cobweb/tela de araña
			\3 Idea clave
			\3 Implicaciones
	\1[] \marcar{Conclusión} 2'
		\2 Recapitulación
			\3 Modelo de competencia perfecta
			\3 Entrada y largo plazo
			\3 Ajuste dinámico
		\2 Idea Final
			\3 Mercados reales
			\3 Impacto modelo de competencia perfecta

\end{esquema}

\esquemalargo















\begin{esquemal}
	\1[] \marcar{Introducción} 2'
		\2 Contextualización
			\3 Microeconomía
				\4 Explicar y predecir comportamiento
				\4 Decisiones individuales
				\4 Interacción de las decisiones de agentes
			\3 Interacción de agentes
				\4 Mercados: marco de interacción entre agentes
				\4 Objetivo:
				\4[] Caracterizar resultado de interacción
				\4[] Caracterizar propiedades del resultado
				\4 ¿Qué precios y cantidades se intercambian?
			\3 Competencia perfecta: modelo benchmark
				\4 Símil física: movimiento sin fricciones
				\4 Agentes toman precios como dados
				\4 Sin costes de transacción
				\4 Competencia perfecta: objeto de comparaciones
				\4 Eje central teoría económica neoclásica
				\4 Walras, Marshall, Hicks, Samuelson, Arrow, Uzawa
		\2 Objeto
			\3 Qué es el modelo de competencia perfecta
			\3 Qué supuestos lo fundamentan
			\3 Qué implicaciones
			\3 Para qué sirve
			\3 Qué extensiones
		\2 Estructura
			\3 Modelo de competencia perfecta
				\4 Idea clave
				\4 Formulación
				\4 Implicaciones
				\4 Aplicaciones
			\3 Entrada y largo plazo
				\4 Idea clave
				\4 Formulación
				\4 Implicaciones
			\3 Ajuste dinámico
				\4 Tatônnement walrasiano
				\4 Modelo de Cobweb
	\1 \marcar{Modelo de competencia perfecta} 12'
		\2 Idea clave
			\3 Modelo básico de interacción
				\4 Agentes toman acciones de otros como dadas.
				\4[$\to$] Como si su acción no tuviese efecto
				\4[$\to$] Agente es punto en una línea: existe sin longitud
			\3 Ausencia de fricciones y externalidades
				\4 Sin costes de transacción.
				\4 Información perfecta
				\4 Sin externalidades
				\4[$\to$] Decisiones no afectan directamente utilidad otros agentes
			\3 Resultado: equilibrio eficiente
				\4 Minimiza costes sociales
				\4 Maximiza utilidades individuales
				\4 Es óptimo de Pareto
			\3 Equilibrio parcial y general
				\4 Posible caracterización con ambos
				\4[$\to$] Eq. parcial suficiente para ilustrar
				\4 Equilibrio parcial:
				\4[] $\to$ Un sólo mercado aislado
		\2 Formulación
			\3 Definiciones
				\4 \textit{Numerario}
				\4[] Bien que sirve de unidad de medida
				\4[] $\to$ Precios expresados en unidades de este bien.
				\4[] Eq. parcial: todos bienes salvo $x$ agrupados en $m$
				\4 \textit{Asignación}
				\4[] Vector $(x_1, \ldots ,x_I, q_1, \ldots, q_J)$
				\4[] $\vec{x}_i$ demanda consumidor $i$, $\vec{q}_j$ oferta empresa $j$
				\4 \textit{Dotación}
				\4[] Numerario inicial de agentes $(\omega_1, \ldots, \omega_I)$
				\4 \textit{Asignaciones factibles}
				\4[] Asignaciones tal que:
				\4[] demanda $\leq$ oferta
				\4[] $ \sum_{i=1}^{I} x_{ik} \leq \sum_{i=1}^{I} \omega_{ik} + \sum_{j=1}^{J} q_{jk} \quad \forall k$
			\3 Supuestos
				\4 Equilibrio parcial
				\4[$\to$] Suficiente para ilustrar idea
				\4[$\to$] Relativamente fácil extensión a $n$ mercados
				\4[(i)] \textit{Dos bienes}
				\4[] Numerario y bien $x$ homogéneo
				\4[(ii)] \textit{Información perfecta}
				\4[] Compradores y vendedores conocen todos precios
				\4[(iii)] \textit{Utilidad cuasilineal}
				\4[] Forma general: $u(x,m) = \phi(x) + m$
				\4[(iv)] \textit{Reventa no controlada}
				\4[$\to$] Supuestos (i)-(iv) $\Rightarrow$ \textit{precio único}
				\4[(v)] \textit{Consumidores y empresas son precio aceptantes}
				\4[] Toman precios como dados
				\4[] No consideran impacto de sus decisiones
			\3 Problema del consumidor
				\4 Distribuir renta entre numerario y bien $x$
				\4[] $\underset{x,m}{\max} \quad u(x,m) = \phi(x) + m$
				\4[] $s.a: \quad p\cdot x + m \leq w$
				\4[] \fbox{CPO: $\; \; \phi'(x) = p$}
				\4[] $\Rightarrow x_i=x_i(p)$, $x'_i(p) < 0$
				\4 \textit{Demanda agregada}
				\4[] $X(p) = \sum_{i=1}^I x_i(p)$
				\4[] \grafica{demandaagregada}
			\3 Problema de la empresa j
				\4 Maximizar beneficio en términos de numerario
				\4[] $\underset{q \in Q_j}{\max} \quad p \cdot q - c(q)$
				\4[] $\text{s.a:} \quad \quad c'(q) > 0, c''(q) >0$
				\4[] \fbox{CPO: $p=\text{CMg}(q)$}
				\4[] $\then q_j = q_j(p)$:
				\4[] $\quad$ Si $\text{CMe}>p$: $q_j(p)=0$
				\4[] $\quad$ Si $\text{CMe} \leq p$: $q_j=q$ tal que $\text{CMg}(q) = p$
				\4[] \grafica{ofertaindividual}
				\4 \textit{Oferta agregada}
				\4[] Suma ofertas de las $j$ empresas dado $p$
				\4[] $Q(p) = \sum_{j=1}^J q_j(p)$
				\4[] \grafica{ofertaagregada}
			\3 Vaciado de mercados
				\4 Demanda = Oferta = Producción + Dotación
				\4[] $ \sum_{i=1}^I x_i = \sum_{i=1} \omega_i + \sum_{j=1} q_j $
				\4[] i.e.: demanda iguala dotación + producción
				\4[] \grafica{vaciadodemercados}
			\3 {Equilibrio competitivo}
				\4 Asignación y precios
				\4[] $(x_0, ..., x_I, q_0, ..., q_J)$, $p$
				\4 Que cumplen:
				\4 Precio $p^*$ de bien $x$ y asignación factible que:
				\4[(i)] Maximiza utilidad de consumidores
				\4[] \fbox{CPO: $\; \; \phi'(x) = p$}
				\4[(ii)] Maximiza beneficios de empresas
				\4[] \fbox{CPO: $p=\text{CMg}(q)$}
				\4[(iii)] Vacía mercados
				\4[] $\then$ $p=\phi'(x) = \text{CMg}(q)$
		\2 Implicaciones
			\3 Primer Teorema Fundamental del Bienestar
				\4 Optimalidad de Pareto\footnote{Más específicamente, optimalidad fuerte de Pareto. La optimalidad débil expresa la cualidad de una asignación dada la cual no existen otras asignaciones capaces de aumentar el bienestar de todos los demás agentes.}
				\4[] No existe asignación que $\uparrow$ utilidad de un agente
				\4[] Y al menos $=$ la del resto
				\4 FPU: niveles de utilidad de óptimos de Pareto.
				\4 Utilidad cuasilineal y dos consumidores
				\4[] $\to$ FPU recta, y decreciente.
				\4[] \grafica{fpu}
				\4 Enunciado del teorema:
				\4[] Si preferencias no saturadas y supuestos anteriores:
				\4[] $\then$ cualquier equilibrio competitivo $p^*$ y $(x^*_1, \ldots, x^*_I, y^*_1, \ldots, q^*_J)$
				\4[] $\then$ es óptimo de Pareto
				\4 Demostración del PTFB en competencia perfecta
				\4[] Excedente social = EC + EP
				\4[] Problema maximización ES:\footnote{Siendo $\omega_m$ el numerario total,y $\mu$ el multiplicador lagrangiano del problema de optimización.}
				\4[] $\underset{(x_1, \ldots , x_I), (q_1, \ldots , q_J)}{\max} \sum_{i=1}^I \left( \phi_i(x_i) + w - p x_i \right)$
				\4[] \quad \quad \quad \quad \quad \quad \quad  $+ \sum_{j=1}^J \left( p q_j - c_j (q_j) \right)$
				\4[] $\quad \; \quad s.a: \quad \quad \; \; \quad \sum_{i=1}^I x_i - \sum_{j=1}^J q_j = 0$
				\4[] CPO\footnote{A partir de $\mathcal{L} = \sum_{i=1}^I \phi_i(x_i) - \sum_{j=1}^J c_j (q_j) + \omega_m + \mu \left( \sum_{i=1}^I x_i - \sum_{j=1}^J q_j \right)$.}: $\mu \leq c'_j(q_j^*) \quad \forall \, j=1, \ldots, J$
				\4[] $\quad \quad \quad \phi'(x_i^*) \leq \mu \quad \forall \, i=1, \ldots, I$
				\4[] $\then$ \fbox{$\phi'(x) = p = \text{CMg}(x)$}
				\4[] $\then$ $p=\mu$ = c. social producción = c. social consumo
				\4[] $\then$ Misma condición de óptimo que para EC
				\4[] $\then$ EC maximizan excedente social
				\4[] Espacio entre curvas de oferta y demanda
				\4[] \grafica{excedentesocial}
			\3 {Segundo Teorema del Bienestar}\footnote{\textit{<<out of all possible Pareto optimal outcomes, one can achieve any particular one by enacting a lump-sum wealth redistribution and then letting the market take over.>>} \url{https://en.wikipedia.org/wiki/Fundamental_theorems_of_welfare_economics}. La prueba de este Teorema hace uso del Teorema del Hiperplano Separador.}
				\4 Asumiendo conjunto de producción y preferencias convexas
				\4 Dado cualquier óptimo de Pareto $(u_1, \ldots, u_I)$
				\4[] existen transferencias $(T_1, \ldots, T_I)$, $\sum_{i=1}^I T_i = 0$
				\4[] que permiten alcanzar el óptimo
				\4[] $\to$ Como un equilibrio competitivo
				\4 Posible alcanzar cualquier óptimo
				\4[] Reasignando numerario
				\4[] $\then$ Vía transferencias/impuestos de suma fija
				\4[$\then$] Fundamento teórico de impuesto de capitación\footnote{En inglés \textit{poll tax}. En este tipo de impuestos, el gravamen no depende del comportamiento de los agentes.}
				\4 Extensión a eq. general:
				\4 [] Existen precios y trans. de dotaciones iniciales
				\4 [] Para alcanzar cualquier óptimo de Pareto

		\2 Estática comparativa
			\3 Desplazamiento de curvas de oferta y demanda
				\4 {Aumento de empresas}
				\4[] Dadas $n$ empresas idénticas: $Q(p) = n \cdot q(p)$
				\4[] $\uparrow n \then$ inclina $Q(p)$ hacia abajo
				\4[] $\then$ $\uparrow X*$, $\downarrow p^*$
				\4[] \grafica{aumentoempresas}
				\4 {Aumento de demanda}
				\4[] Motivos posibles: $\uparrow$ consumidores, cambio prefs., \textellipsis
				\4[] Si aumenta utilidad individual por consumir bien:
				\4[] $\then$ curva demanda se desplaza a derecha
				\4[] \grafica{aumentodemanda}
				\4[] Si aumenta número de consumidores:
				\4[] $\then$ Curva de dda. más elástica (menos pendiente)
				\4[] Especial importancia si consideramos entrada y l/p
			\3 Impuesto específico
				\4 Concepto
				\4[] Pagado por unidad de bien
				\4[] Consumidores abonan
				\4[] $\to$ Sin pérdida de generalidad
				\4[] $\to$ Incidencia es otra cuestión
				\4 Dos precios:
				\4[] $p^*$ recibido por empresas
				\4[] $p^* + t$ pagado por consumidores
				\4 ¿Cómo reacciona $p^*$ ante $\varDelta$ de t?
				\4 Analíticamente:\footnote{V. pág. 323 MWG.}
				\4[] $\frac{d \, S\left( p^*(t) \right)}{d \, t} =  \frac{d \, D \left( p^*(t) + t \right)}{d \, t}$
				\4[] $\then S'(p^*(t)) \cdot {p^*}'(t) = D' \cdot \left({p^*}'(t) + 1 \right) $
				\4[] $\then \frac{d \, {p^*}(t)}{d t} =  \frac{D'}{S' - D'}$
				\4[] $D'<0$, $S'>0 \then \frac{d \, {p^*}(t)}{d t} < 0 $
				\4[] $\then$ precio baja si aumenta impuesto
				\4[] $\then$ $\left| \frac{d \, p^*(t)}{d \, t} \right|$ es $1$ como máximo
				\4 Gráficamente
				\4[] Impacto depende de elasticidad de curvas
				\4[] Lado + elástico: menor impacto en exced. y precio
				\4[] \grafica{impuestoofertaelastica}
				\4 Implicaciones
				\4[] $p^*$ es precio de productor
				\4[] ¿Cuándo cae el precio que reciben productores?
				\4[] $\to$ Cuando $S'$ muy pequeña
				\4[] $\then$ Cuando oferta muy inelástica
				\4[] $\to$ Cuando $D'$ muy grande
				\4[] $\then$ Cuando oferta muy elástica
		\2 Aplicaciones
			\3 Benchmark de eficiencia
				\4 Comparación con otras estructuras de mercado
				\4 ¿Qué pérdidas de eficiencia con otras estructuras?
				\4 ¿Qué efecto tienen impuestos?
				\4 ¿Impuestos específicos y ad-valorem equivalentes?
				\4 ¿Cuotas y aranceles equivalentes?
			\3 Modelización de mercados
				\4 ¿Existen mercados cuyo equilibrio se acerque a CP?
				\4 ¿Sirve CP para predecir y entender mercados reales?
				\4 Relativo acercamiento a realidad si:
				\4[] Muchos compradores y vendedores
				\4[] Producto muy homogéneo
				\4[] Sin o bajos costes de transporte
				\4[] Libertad de entrada y salida
				\4[] Sin restricciones legales de precios y/o oferta
				\4[] Dificil coordinación oferentes/demandantes
				\4 Ejs.:
				\4[] Mercados financieros muy líquidos
				\4[] Ciertos mercados agrícolas
				\4[] Mercado mundial de petróleo
	\1 \marcar{Entrada y largo plazo} 7'
		\2 Idea clave
			\3 Endogeneizar número de empresas
				\4 Hasta ahora, número de empresas exógeno.
				\4 Manteniendo demás supuestos, responder a:
				\4[] $\to$ ¿de qué depende número de empresas?
				\4[] $\to$ ¿cómo afecta endogeneización a EC?
				\4[] $\to$ ¿existe equilibrio si nº empresas es endógeno?
			\3 Diferentes modelos de entrada
				\4 Modelo simple:
				\4[$\to$] Infinitas empresas pueden entrar
				\4[$\to$] Idénticas funciones de coste
				\4[$\to$] Entran si beneficio es $>0$
				\4[$\then$] Entran si $\text{P} > \text{CMe}$
				\4[$\to$] Sin costes hundidos
			\3 Equilibrio competitivo de largo plazo
				\4 Asignación, precio y número de empresas tal que:
				\4[] Consumidores maximizan utilidad
				\4[] Empresas maximizan beneficios
				\4[] Oferta iguala a demanda
				\4[] Empresas no tienen incentivos a salir o entrar
		\2 Formulación
			\3 Equilibrio de largo plazo
				\4 Precio $p^*$
				\4 Cantidad $q^*$
				\4 Número de empresas $J^*$
			\3 Condiciones de equilibrio
				\4[(i)] Maximización de utilidad
				\4[] $x^* = \arg \underset{x,m}{\max} \quad u(x,m) = \phi(x) + m$
				\4[] $\quad \quad \quad \text{s.a}: \quad p \cdot x + m \leq w $
				\4[(ii)] Maximización de beneficios
				\4[] $q^* = \arg \underset{q \geq 0}{\max}\quad  p^* - c(q)$
				\4[(iii)] Vaciado de mercados
				\4[] $\sum_i x_i(p) \equiv  D(p^*) \equiv S(p^*) \equiv J^* \cdot q^*$
				\4[(iv)] Condición de libre entrada
				\4[] $\pi = p^* q^* - c( q^* ) = 0 $
				\4[] $\to$ Si beneficio negativo, salen del mercado
				\4[] $\to$ Si beneficio positivo, nuevas empresas entran
				\4[] $\then$ Empresas obtienen beneficio nulo
			\3 Existencia de equilibrio\footnote{Seccion 10.F de MWG.}
				\4 \textit{Costes totales estrictamente convexos y crecientes}
				\4[] No existe CMe mínimo $\neq \text{CMe}(0)$
				\4[] $\to$ P = CMg > CMe
				\4[] $\then$ Siempre pueden obtenerse beneficios entrando
				\4[] $\then$ No existe $J^*$ que anule beneficios
				\4[] $\then$ Cada vez hay más empresas que producen cada vez menos
				\4[] $\then$ No existe EC de largo plazo
				\4[] \grafica{eqlpcostesestrictamenteconvexos}
				\4 \textit{Costes totales lineales}
				\4[] Costes marginales constantes
				\4[] Sin costes fijos
				\4[] $\text{CMe}(q)$ mínimo para todo $q$
				\4[] $\then$ Todo $J^*$ que vacíe mercados es EC de largo plazo
				\4[] $\then$ Producción de cada empresa indeterminada.
				\4[] $\then$ Existen inf. eq. competitivos con libre entrada
				\4[] \grafica{eqlprendimientosconstantes}
				\4 \textit{Escala eficiente única}
				\4[] Si existe $\bar{q}$ única tal que $\text{CMe}(\bar{q})$ es mínimo
				\4[] Entrarán/saldrán empresas hasta que $q=\bar{q}$
				\4[] $q^*=\bar{q} \then \, J^*$ único que $\then D(p^*) = J^* \cdot q^*$
				\4[] $\then$ existe $(J^*, p^*, q^*)$ único de EC a largo plazo
				\4[] \grafica{eqlpescalaeficienteunica}
		\2 Implicaciones
			\3 Aumentos demanda
				\4 Si escala eficiente única
				\4[] $\then$ Aumenta número de empresas en mercado
				\4[] $\then$ Producción individual constante
			\3 Estática comparativa a corto plazo\footnote{Ver pág. 340 MWG}
				\4 Debe partir de EC a largo plazo
				\4 Debe considerar f. de costes de c/p
	\1 \marcar{Ajuste dinámico} 7'
		\2 Idea clave
			\3 Análisis estático
				\4 Contestar a:
				\4[] Existe EC?
				\4[] Qué valores toma el EC?
			\3 Análisis dinámico
				\4 ¿Es posible alcanzar EC?
				\4 ¿Qué hace falta para que EC sea alcanzable?
				\4 ¿Cómo se alcanza EC?
			\3 Estabilidad vs inestabilidad
				\4 Estabilidad:
				\4[] Sistema converge a equilibrio de análisis estático
				\4[] Dada perturbación desde equilibrio
				\4[] $\to$ Sistema tiende de nuevo a equilibrio
				\4 Inestabilidad:
				\4[] Sistema no tiende a ningún valor determinado
				\4[] Dada perturbación desde equilibrio
				\4[] $\to$ Sistema se desvía cada vez más
				\4[] \grafica{estabilidadequilibrio}
				\4 Diferentes variables a tener en cuenta
				\4[] Estabilidad walrasiana
				\4[] $\to$ Precio es variable de ajuste
				\4[] Estabilidad marshalliana
				\4[] $\to$ cantidades son variable de ajuste
			\3 Mecanismos de estabilidad
				\4 Reglas de cambio de precios o cantidad
				\4 Dadas asignaciones y precios
				\4[] ¿Cómo varían cantidades y precios
				\4[$\to$] Diferentes supuestos posibles
				\4[$\to$] Diferentes resultados sobre estabilidad
			\3 Relación entre estabilidad walrasiana y marshaliana
				\4 Realmente, son conceptos distintos
				\4 En Walras
				\4[] Oferta es:
				\4[] $\to$ o simultánea a intercambio
				\4[] $\to$ o previa a intercambio
				\4[] Precio es variable de ajuste
				\4[] $\to$ Equilibra oferta y demanda
				\4 En Marshall
				\4[] Oferta determinada antes de precio
				\4[] Dimensión temporal
				\4[] Producción se ajusta en el tiempo
				\4[] $\then$ En función de precios anteriores
		\2 Estabilidad walrasiana\footnote{Ver Takama (1974), Chapter 3, pág. 300 y ss.}
			\3 Idea clave
				\4 Mecanismo de ajuste\footnote{P.ej. el famoso subastador walrasiano que pregunta repetidamente a los agentes qué cantidades ofrecen y demandan y, a qué precios están dispuestos a vender y comprar hasta que encuentran una combinación de precios y cantidades que vacían el mercado. El término ``subastador'' resulta del inglés ``\textit{auctioneer}'' de la primera traducción de 1954. En el original en francés, el término utilizado es ``\textit{crieurs}'' en plural, con una connotación distinta a la interpretación habitual de una especie de superagente capaz de conocer todos los precios y todas las demandas y ofertas al tiempo, y ajustarse en consecuencia.}
				\4[] $\varDelta$ precio depende de exceso de demanda
				\4[] $D > S \Rightarrow \varDelta P > 0 $
				\4[] ¿Precios y cantidades convergen a EC?
			\3 Formulación
				\4 Dadas $S(p)$ y $D(p)$ tal que $D(p^*) = S(p^*)$
				\4 $\dot{p}(t) = \lambda \left( D \left(p (t) \right) - S \left( p(t) \right) \right)$
				\4 Estabilidad:
				\4[] $t \to \infty \Rightarrow p(t) \to p^*$
				\4[] \grafica{walrasestable}
				\4 Inestabilidad:
				\4[] Sistema tal que $t \to \infty \Rightarrow p(t) \to \pm \infty$
				\4[] \grafica{walrasinestable}
			\3 Implicaciones
				\4 Estabilidad de EC no garantizada
				\4 Instituciones de mercado/parámetros determinan resultado
		\2 Ajuste marshalliano en cantidades
			\3 Idea clave
				\4 Existe mecanismo ajuste cantidades
				\4[$\to$] Cantidad varía en función de precios aceptables para cantidad dada
				\4[] ¿Precios y cantidades convergen a EC?
			\3 Formulación
				\4 $\dot{q} = \mu \left( P_D \left( q (t) \right) - P_S \left( q(t) \right) \right)$
				\4 Estabilidad:
				\4[] $t \to \infty \Rightarrow q(t) \to q^*$
				\4[] \grafica{marshallestable}
				\4 Periodo de mercado
				\4[] Bienes perecederos
				\4[] $\to$ Demanda determina precio totalmente
				\4[] $\to$ Cantidad producida es fija
				\4[] Bienes duraderos
				\4[] $\to$ Posible aumentar oferta gastando stock
				\4[] $\to$ Vendedor tiene precio de reserva tal que $Q=0$
				\4[] $\to$ Cantidad máxima que puede vender
				\4[] $\then$ A partir de c. máxima, = que bienes perecederos
				\4 Periodo de corto plazo
				\4[] Existe cierto margen para ajustar oferta
				\4[] Demanda no determina totalmente el precio
				\4[] Oferta relativamente inelástica pero no totalmente
				\4[] Cantidad se ajusta hasta igualar oferta y demanda
				\4 Periodo normal o de largo plazo
				\4[] Todos los factores son variables
				\4[] Equilibrio de oferta y demanda determina precio
				\4[] Precio iguala coste marginal y medio de l/p
				\4[] $\to$ Escala eficiente
				\4 Periodo secular
				\4[] Periodos muy largos
				\4[] No es posible estimar cambios en demanda y oferta
				\4[] Mercado puede ser totalmente diferente
			\3 Implicaciones
				\4 Debate Hicks-Newman:
				\4[] Estabilidad marshall vs walras son equivalentes?
				\4[] $\to$ Hicks: son diferentes
				\4[] $\to$ Newman: son diferentes problemas
				\4 Asociación habitual:
				\4[] Hicks: periodo de mercado, producción fija
				\4[] Marshall: corto plazo, producción ajustable \footnote{También se asocia en ocasiones el ajuste walrasiano a contextos de monopolio.}
		\2 Modelo de cobweb/tela de araña
			\3 Idea clave
				\4 Empresas deciden producción respecto a $p_{t-1}$
				\4 Elast. determinan trayectoria precios, cantidad
				\4[] Ej.: ciertos mercados agrícolas
				\4 Secuencia:
				\4[1] Shock de oferta: se produce $q_1 = \bar{S} < q^*$
				\4[] $\to$ Se vende $q_1$ a $p_1$ tal que $q_1=D(p_1)$
				\4[] $\Rightarrow$ $p_1 > p^*$
				\4[2] Oferta se determina en función de precio en t=1
				\4[] $\to$ Se vende $q_2 > q_1$ a precio tal que $q_2 = D(p_2)$
				\4[] $\Rightarrow$ $\downarrow$ precio
				\4[3] \ldots
				\4 Múltiples trayectorias posibles:
				\4[] Convergente: \grafica{cobwebestable}
				\4[] Divergente: \grafica{cobwebstable}
				\4[] Cíclica
			\3 Implicaciones
				\4 Shocks generan oscilaciones
				\4 Posible persistencia en el largo plazo
	\1[] \marcar{Conclusión} 2'
		\2 Recapitulación
			\3 Modelo de competencia perfecta
			\3 Entrada y largo plazo
			\3 Ajuste dinámico
		\2 Idea Final
			\3 Mercados reales
				\4 Gran mayoría, muy alejados de comp. perfecta
				\4[] $\to$ O no se cumplen requisitos,
				\4[] $\to$ o no se converge a equilibrio competitivo
				\4 Algunas excepciones
			\3 Impacto modelo de competencia perfecta
				\4 Enorme impacto teórico
				\4 Walras + Marshall (finales siglo XIX)
				\4[$\to$] Base microeconomía neoclásica
				\4 Eq. general
				\4 Nueva Macro Clásica
				\4 \ldots
\end{esquemal}





























\graficas

\begin{axis}{4}{Derivación gráfica de la función de demanda agregada a partir de demandas invidivuales simétricas.}{$ $}{$P$}{demandaagregada}
	\draw[-] (0,3) -- (1,0);
	\node[below] at (1,0){$\bar{q}$};
	
	\draw[-] (0,3) -- (2,0);
	\node[below] at (2,0){$2 \bar{q}$};
	
	\draw[-] (0,3) -- (3,0);
	\node[below] at (3,0){$3\bar{q}$};
	
	\draw[-] (0,3) -- (4,0);
	\node[below] at (4,0){$4 \bar{q}$};
\end{axis}

Se aprecia que la demanda agregada inversa dado un precio es la suma horizontal de las demandas individuales. En el ejemplo, las demandas individuales son simétricas. 

\begin{axis}{4}{Función de oferta individual y su relación con costes marginales y medios.}{$y$}{ $\text{CMe}(y)$ \\ $\text{CMg}(y)$ \\ $\text{P}(y)$ }{ofertaindividual}
	% costes medios
	\draw[-] (0.9,4) to [out=280, in=180](2.65,2) to [out=0, in=260](4.4,4);
	\node[right] at (4.5,4){CMe};
	
	% costes marginales
	\draw[-] (0,0) to [out=15, in=235](2.65,2) to [out=55,in=260](3.4,4);
	\node[above] at (3.5,4){CMg};	
	
	% oferta
	\draw[thick,color=red] (2.63,2) to [out=55,in=260](3.38,4);
	\draw[thick, color=red] (0,0) -- (0,2);
\end{axis}

En rojo se muestra la función inversa de oferta. Se puede apreciar que cuando el coste medio es superior al coste marginal, la oferta de la empresa es igual a 0. Esto es así porque si la empresa produciese una cantidad de output que igualase coste marginal y precio, obtendría beneficios negativos y en el contexto del ejemplo, siempre es posible obtener beneficios nulos a partir de una producción nula.

\begin{axis}{4}{Derivación gráfica de la función de oferta agregada.}{$ $}{$p$}{ofertaagregada}
	\draw[-] (0,0) to [out=60, in=265](1,4);
	\draw[dotted] (1,4) -- (1,0);
	\node[below] at (1,0){$\bar{y}$};
	
	\draw[-] (0,0) to [out=30, in=265](2,4);
	\draw[dashed] (2,4) -- (2,0);
	\node[below] at (2,0){$2\bar{y}$};
	
	\draw[-] (0,0) to [out=15, in=265](3,4);
	\draw[dashed] (3,4) -- (3,0);
	\node[below] at (3,0){$3\bar{y}$};

	\draw[-] (0,0) to [out=7, in=265](4,4);
	\draw[dashed] (4,4) -- (4,0);
	\node[below] at (4,0){$4\bar{y}$};
\end{axis}

\begin{axis}{4}{Vaciado de mercados.}{$P$}{$Q$}{vaciadodemercados}
	\draw[-] (0,4) -- (4,0);
	
	\draw[-] (0,0) -- (4,4);
	
	% Cantidad que vacía mercado
	\draw[dotted] (2,2) -- (2,0);
	\node[below] at (2,0){$D=S$};
	
	% EQUILIBRIO
	\node[circle, fill=black, inner sep=0pt, minimum size=6pt] (a) at (2,2) {};
\end{axis}

\begin{axis}{4}{Frontera de posibilidades de utilidad con utilidad cuasilineal.}{$u_x$}{$u_y$}{fpu}
	\draw[-] (0,3.5) -- (3.5,0);
\end{axis}

\begin{axis}{4}{Maximización del excedente social.}{$Q$}{$P$}{excedentesocial}
	
	% DEMANDA
	\draw[-] (0,3.5) -- (4,1);
	\node[right] at (4,1){$D$};
	
	% OFERTA
	\draw[-] (0,0.5) -- (4,4);
	\node[right] at (4,4){$S$};
	
	% CANTIDAD DE EQUILIBRIO
	\draw[dotted] (2,2.25) -- (2,0);
	\node[below] at (2,0){$q^*$};
	
	% PRECIO DE EQUILIBRIO
	\draw[dotted] (2,2.25) -- (0,2.25);
	\node[left] at (0,2.25){$p^*$};
	
	% EQUILIBRIO
	\node[circle, fill=black, inner sep=0pt, minimum size=6pt] (a) at (2,2.25) {};
	
	% COSTE
	\draw [white, fill=red, opacity=0.2] (0,0.5) -- (4,4) -- (4,0) -- (0,0);
	
	% EXCEDENTE PRODUCTOR
	\draw [white, fill=blue, opacity=0.2] (0,0.5) -- (0,2.25) -- (2,2.25);
	\node[below] at (0.7, 2){EP};
	
	% EXCEDENTE CONSUMIDOR
	\draw [white, fill=green, opacity=0.2] (0,3.5) -- (2,2.25) -- (0,2.25);
	\node[below] at (0.7, 2.8){EC};
	
\end{axis}

\begin{axis}{4}{Efecto sobre la oferta de un aumento del número de empresas en el mercado.}{$Q$}{$P$}{aumentoempresas}
	% demanda
	\draw[-] (0,4) -- (4,0);
	
	% oferta
	\draw[-] (0,0) -- (4,4);
	
	% oferta post aumento
	\draw[-] (0,0) -- (4,2.5);
	
	% Equilibrio 1
	\node[circle, fill=black, inner sep=0pt, minimum size=3pt] (a) at (2,2) {};
	
	% Equilibrio 2
	\node[circle, fill=black, inner sep=0pt, minimum size=3pt] (a) at (2.45,1.55) {};
	
	% Precio equilibrio 1
	\draw[dotted] (2,2) -- (0,2);
	\node[left] at (0,2){$p^*$};
	
	% Cantidad equilibrio 1
	\draw[dotted] (2,2) -- (2,0);
	\node[below] at (2,0){$q^*$};
	
	% Precio equilibrio 2
	\draw[dotted] (2.45,1.55) -- (0,1.55);
	\node[left] at (0,1.55){$p'$};
	
	% Cantidad equilibrio 2
	\draw[dotted] (2.45,0) -- (2.45,1.55);
	\node[below] at (2.45,0){$q'$};
	
\end{axis}

\begin{axis}{4}{Efecto de un aumento de la demanda sobre equilibro competitivo.}{$Q$}{$P$}{aumentodemanda}
	% extensión eje abscisas
	\draw[-] (4,0) -- (6,0);
	
	% demanda
	\draw[-] (0,4) -- (4,0);
	
	% demanda 2
	\draw[-] (0,4) -- (5.5,0);
	
	% oferta
	\draw[-] (0,0) -- (4,4);
		
	% Equilibrio 1
	\node[circle, fill=black, inner sep=0pt, minimum size=3pt] (a) at (2,2) {};
	
	% Precio equilibrio 1
	\draw[dotted] (2,2) -- (0,2);
	\node[left] at (0,2){$p^*$};
	
	% Cantidad equilibrio 1
	\draw[dotted] (2,2) -- (2,0);
	\node[below] at (2,0){$q^*$};
	
	% Equilibrio 2
	\node[circle, fill=black, inner sep=0pt, minimum size=3pt] (a) at (2.31,2.31) {};
	
	% Precio equilibrio 2
	\draw[dotted] (2.31,2.31) -- (0,2.31);
	\node[left] at (0,2.31){$p'$};
	
	% Cantidad equilibrio 2
	\draw[dotted] (2.31,2.31) -- (2.31,0);
	\node[below] at (2.31,0){$q'$};
	
\end{axis}

\begin{axis}{4}{Impacto de un impuesto en un contexto de oferta más elástica que demanda.}{$Q$}{$P$}{impuestoofertaelastica}
	% demanda
	\draw[-] (1.1,4) -- (2.5,0);
	
	% oferta
	\draw[-] (0,1.5) -- (4,2.5);
	
	% equilibrio sin impuesto
	\node[circle, fill=black, inner sep=0pt, minimum size=3pt] (a) at (1.82,1.93) {};
	
	% área de recaudación
	\draw[dashed] (0,2.7) -- (1.54,2.7) -- (1.54,1.88) -- (0,1.88);
	\draw [white, fill=yellow, opacity=0.2] (0,2.7) -- (1.54,2.7) -- (1.54, 1.88) -- (0,1.88) -- (0,2.7);
	
	% precio consumidor con impuesto
	\node[left] at (0,1.88){$p^*$};
	
	% precio productor con impuesto
	\node[left] at (0,2.7){$p^*+t$};
	
\end{axis}


%cambiar -> convexidad estricta
\begin{axis}{4}{Beneficios positivos de una empresa con costes estrictamente convexos.}{$Q$}{P \\ CMg \\ CMe}{eqlpcostesestrictamenteconvexos}
	% precio
	\draw[-] (0,2.5) -- (4,2.5);
	\node[left] at (0,2.5){$p$};
	
	% coste marginal
	\draw[-] (0,0) to [out=40, in=250](3.5,4);
	\node[above] at (3.5,4){CMg};
	% coste medio
	\draw[-] (0,0) to [out=20, in=250](4.5,4);
	\node[above] at (4.5,4){CMe};
	
	% equilibrio
	\node[circle, fill=black, inner sep=0pt, minimum size=3pt] (a) at (2.72,2.51) {};
	
	% excedente del productor
	\draw[dotted] (2.72,2.51) -- (2.72, 1.42) -- (0,1.42);
	\draw [white, fill=blue, opacity=0.2] (0,2.51) -- (2.72,2.51) -- (2.72, 1.42) -- (0,1.42);
	\node[] at (1.2,1.9){EP};
	
\end{axis}

El gráfico muestra como para cualquier $p$ y cualquier cantidad, la empresa obtiene beneficios positivos. En este contexto, cualquier empresa fuera del mercado podría entrar y obtener beneficios positivos, aunque cada empresa produjese una cantidad infinitesimal y los beneficios se redujesen también infinitesimalmente. Por tanto, no existirá un número de empresas que iguale a 0 los beneficios individuales y no existirá un equilibrio competitivo de largo plazo.

\begin{axis}{4}{Equilibrio a largo plazo con funciones de producción con rendimientos constantes a escala.}{$Q$}{$P$}{eqlprendimientosconstantes}
	% demanda
	\draw[-] (0,4) -- (4,0);
	
	% oferta
	\draw[-] (0,2) -- (4,2);
	\node[right] at (4,2){CMg};
	
	% cantidad de equilibrio
	\draw[dashed] (2,2) -- (2,0);
	\node[below] at (2,0){$q$};

\end{axis}

En un contexto de rendimientos constantes a escala, las empresas obtienen beneficios nulos para cualquier cantidad producida, y por tanto existen infinitos equilibrios competitivos de largo plazo.


\begin{axis}{4}{Equilibrio de una empresa individual a largo plazo con funciones de producción con escala eficiente única.}{$q$}{p \\ CMg \\ CMe}{eqlpescalaeficienteunica}
	% costes medios
	\draw[-] (0.9,4) to [out=280, in=180](2.65,2) to [out=0, in=260](4.4,4);
	\node[right] at (4.5,4){CMe};
	
	% costes marginales
	\draw[-] (0,0) to [out=15, in=235](2.65,2) to [out=55,in=260](3.4,4);
	\node[above] at (3.5,4){CMg};	
	
	% oferta
	\draw[thick,color=red] (2.63,2) to [out=55,in=260](3.38,4);
	\draw[thick, color=red] (0,0) -- (0,2);
	
	% precios
	\draw[-] (0,2) -- (2.63,2);
	\node[left] at (0,2){$p^*$};
	
	% precio que implica beneficio positivo
	\draw[dashed] (0,2.4) -- (2.91,2.4);
	\node[left] at (0,2.4){$p'$};
	\draw [white, fill=blue, opacity=0.2] (0,2.4) -- (2.91,2.4) -- (2.91,2.03) -- (0,2.03);
\end{axis}

La gráfica muestra como en tanto que el precio no sea igual a $p^*$, de tal manera que se igualen costes medios y marginales, las empresas obtendrán entrantes obtendrán beneficios. El proceso de entrada de nuevas empresas continuará hasta que el precio alcance el equilibrio de escala eficiente, alcanzándose un equilibrio competitivo de largo plazo.

\begin{axis}{4}{Convergencia de un sistema dinámico hacia un equilibrio estable.}{$t$}{$p(t)$}{estabilidadequilibrio}
	% equilibrio 
	\draw[dotted] (0,2) -- (4,2);
	
	% exceso de demanda positivo
	\draw[-] (0,0) to [out=75, in=180](4,1.9);
	
	% exceso de demanda negativo
	\draw[-] (0,4) to [out=285, in=180](4,2.1);
\end{axis}

\begin{axis}{4}{Ajuste walrasiano estable.}{$Q$}{$P$}{walrasestable}
	% equilibrio
	\node[circle, fill=black, inner sep=0pt, minimum size=3pt] (a) at (1.79,2.27) {};
	\node[left] at (0,2.27){$p^*$};
	\draw[dashed] (0,2.27) -- (4,2.27);

	% DEMANDA
	\draw[-] (0.5,4) -- (3.5,0);
	\node[above] at (0.5,4){D};
	
	% exceso de demanda negativo
	\draw[-{Latex}] (1.24,3.01) -- (1.74,2.347);
	\draw[-{Latex}] (0.74,3.68) -- (1.24,3.01);

	% exceso de oferta negativo
	\draw[-{Latex}] (2.35,1.53) -- (1.85, 2.2);
	\draw[-{Latex}] (2.85,0.867) -- (2.35,1.53);
	
	% OFERTA
	\draw[-] (0,0.5) -- (3.5,4);
	\node[above] at (3.5,4){S};
	
	% exceso de demanda positivo
	%\draw[-{Latex}] (0,0.5) -- (.7,1.2);
	\draw[-{Latex}] (0.7,1.2) -- (1.2,1.7);
	\draw[-{Latex}] (1.2,1.7) -- (1.7,2.2);
	
	% exceso de demanda negativo
	\draw[-{Latex}] (2.85, 3.35)-- (2.35,2.85);
	\draw[-{Latex}] (2.35,2.85) -- (1.85,2.35);
		
	% precio alto
	\draw[dashed] (0,3.5) -- (4,3.5);
	\node[left] at (0,3.5){$p'$};
	
	% precio bajo
	\draw[dashed] (0,1) -- (4,1);
	\node[left] at (0,1){$p'$};
	
	% exceso de demanda positivo con precio bajo
	\draw[decoration={brace,mirror,raise=5pt},decorate]
	(0.45,1) -- node[below=9pt] {$\text{ED}>0$} (2.81,1);
	
	% exceso de demanda negativo con precio alto
	\draw[decoration={brace,raise=9pt},decorate]
	(0.97,3.27) -- node[above=12pt] {$\text{ED}<0$} (2.95,3.27);	
\end{axis}

\begin{axis}{4}{Ajuste walrasiano inestable.}{$Q$}{$P$}{walrasinestable}
	% EQUILIBRIO
	\node[circle, fill=black, inner sep=0pt, minimum size=5pt] (a) at (2.4,1.2) {};
	\draw[dashed] (0,1.2) -- (4,1.2);
	\node[left] at (0,1.2){$p^*$};
	
	% DEMANDA
	\draw[-] (1,4) -- (3,0);
	\node[above] at (1,4){D};
	
	% precio superior a equilibrio
	\draw[-{Latex}] (2.4,1.2) -- (2.2,1.6);
	\draw[-{Latex}] (2.2,1.6) -- (2,2);
	\draw[-{Latex}] (2,2) -- (1.8,2.4);
	\draw[-{Latex}] (1.8,2.4) -- (1.6, 2.8);
	\draw[-{Latex}] (1.6,2.8) -- (1.4, 3.2);
	
	% precio inferior a equilibrio
	\draw[-{Latex}] (2.4,1.2) -- (2.6, 0.8);
	\draw[-{Latex}] (2.6,0.8) -- (2.8, 0.4);
	\draw[-{Latex}] (2.8,0.4) -- (3, 0);
	
	% OFERTA
	\draw[-] (0,3) -- (4,0);
	\node[above] at (1,1.5){S};
	
	% precio superior a equilibrio
	\draw[-{Latex}] (2.4,1.2) --  (2.1, 1.425);
	\draw[-{Latex}] (2.1, 1.425) -- (1.8, 1.65);
	\draw[-{Latex}] (1.8, 1.65) -- (1.5, 1.875);
	\draw[-{Latex}] (1.5, 1.875) -- (1.2, 2.1);
	\draw[-{Latex}] (1.2, 2.1) -- (0.9, 2.325);
	\draw[-{Latex}] (.9, 2.325) -- (0.6, 2.55);
	\draw[-{Latex}] (.6, 2.55) -- (.3, 2.775);
	
	% precio inferior a equilibrio
	\draw[-{Latex}] (2.4,1.2) -- (2.7, 0.975);
	\draw[-{Latex}] (2.7,0.975) -- (3, 0.75);
	\draw[-{Latex}] (3,0.75) -- (3.3, 0.525);
	\draw[-{Latex}] (3.3, 0.525) -- (3.6, 0.3);
	\draw[-{Latex}] (3.6,0.3) -- (3.9, 0.075);
	
	% precio bajo
	\draw[dashed] (0,.5) -- (4,.5);
	\node[left] at (0,.5){$p'$};
	
	% precio alto
	\draw[dashed] (0,2.7) -- (4,2.7);
	\node[left] at (0,2.7){$p''$};
	
	% exceso de demanda positivo con precio alto
	\draw[decoration={brace,raise=9pt},decorate]
	(0.35,2.42) -- node[above=12pt] {\small $\text{ED}>0$} (1.55,2.42);
	
	% exceso de demanda negativo con precio bajo
	\draw[decoration={brace,mirror,raise=2pt},decorate]
	(2.82,.5) -- node[below=6pt] {} (3.29,.5);
	\draw[-{Latex}] (3.02,.44) -- (4,1.7);
	\node[above] at (4,1.7) {\small $\text{ED}<0$};
	
\end{axis}

\begin{axis}{4}{Ajuste marshalliano estable.}{$Q$}{$P$}{marshallestable}
	% DEMANDA
	\draw[-] (0,4) -- (4,0);
	\node[right] at (3,1.2){D};
	
	\draw[-{Latex}] (1,3) -- (1.5,2.5);
	\draw[-{Latex}] (1.5,2.5) -- (1.96,2.04);
	
	% OFERTA
	\draw[-] (0,0) -- (4,4);
	\node[above] at (4,4){S};
	
	\draw[-{Latex}] (1,1) -- (1.5,1.5);
	\draw[-{Latex}] (1.5,1.5) -- (1.96,1.96);
	
	% EQUILIBRIO
	\node[circle, fill=black, inner sep=0pt, minimum size=5pt] (a) at (2,2) {};
	
	% exceso de cantidad
	\draw[dashed] (1,0) -- (1,4);
	\draw[decoration={brace,raise=2pt},decorate]
	(1,1) -- node[left=3pt] {$P_d(q) - P_s(q)$} (1,3);
\end{axis}

\pagebreak

\begin{axis}{4}{Ejemplo de modelo de Cobweb convergente.}{$Q$}{$P$}{cobwebejemplo}
	% DEMANDA
	\draw[-] (0,3) -- (4,1);
	\node[right] at (4,1){D};
	
	% OFERTA
	\draw[-] (0,0) -- (4,3);
	\node[right] at (4,3){S};
	
	% SHOCK
	\draw[dotted] (1,0) -- (1,4);
	
	% EQUILIBRIO
	\node[circle, fill=black, inner sep=0pt, minimum size=3pt] (a) at (2.4,1.8) {};
	
	% TRAYECTORIA
	% primer periodo: origen
	\node[circle, fill=black, inner sep=0pt, minimum size=3pt] (a) at (1,2.5) {}; 
	\node[below] at (0.94,2.5){\scriptsize  1};
	\draw[dashed, -{Latex}] (1,2.5) -- (3.35,2.5) -- (3.33,1.32);
	
	% segundo periodo
	\node[circle, fill=black, inner sep=0pt, minimum size=3pt] (a) at (3.33,1.32) {};
	\node[below] at (3.33,1.32){\scriptsize 2};
	\draw[dashed, -{Latex}] (3.33,1.32) -- (1.78,1.32) -- (1.78,2.11);
	
	
	% tercer periodo
	\node[circle, fill=black, inner sep=0pt, minimum size=3pt] (a) at (1.78,2.11) {};
	\node[above] at (1.78,2.11){\scriptsize 3};
	\draw[dashed, -{Latex}] (1.78,2.11) -- (2.8,2.11) -- (2.8, 1.6);
	
	% cuarto periodo
	\node[circle, fill=black, inner sep=0pt, minimum size=3pt] (a) at (2.8,1.6) {};
	\node[right] at (2.8,1.6){\scriptsize 4};
	\draw[dashed, -{Latex}] (2.8, 1.6) -- (2.15,1.6) -- (2.15,1.92);
	
	% quinto periodo
	\node[circle, fill=black, inner sep=0pt, minimum size=3pt] (a) at (2.15,1.92) {};
	\node[left] at (2.15,1.85){\scriptsize 5};
	\draw[dashed] (2.15,1.92) -- (2.52,1.92) -- (2.52,1.81);
\end{axis}

En este ejemplo de modelo de Cobweb, el precio y la cantidad convergen al equilibrio competitivo. La línea de puntos muestra la oferta en el shock inicial que desvía el sistema del equilibrio competitivo

\conceptos

\concepto{Diferencia entre impuesto de capitación e impuesto de suma fija}

Los impuesto de capitación (\text{poll taxes}) son un caso particular de los impuestos de suma fija. En los impuestos de capitación, cada sujeto tributario debe pagar una cantidad \textit{igual} al resto de los obligados y determinada independientemente de sus circunstancias particulares. Los impuestos de suma fija (\textit{lump-sum taxes}) son una categoría más general de impuestos que engloban a los impuestos de capitación. En éstos últimos, se grava a los sujetos con cantidades fijas que no dependen de su comportamiento o decisiones. En este tipo de impuestos, la cantidad sí puede ser diferente entre individuos y depender de variables que no dependen de la voluntad del individuo. 

\concepto{Costes variables y salida de empresas a corto plazo}

En las representaciones habituales de los modelos de competencia perfecta, el análisis de corto plazo implica la existencia de costes fijos y variables. Los primeros no dependen de la cantidad producida, y se incurren por el mero hecho de producir una cantidad positiva. Los costes variables, por el contrario, sí dependen positivamente de la producción. Una vez que una una empresa ha tomado la decisión de producir, incurre en un coste fijo que se asume irreversible en el plazo inmediato. En este punto, el objetivo de la firma no consiste en minimizar su coste fijo, que ya ha sido incurrido, sino en cubrir sus costes variables. En la medida en que pueda vender su producción a un precio medio por encima del coste variable medio, estará extrayendo un beneficio de cada unidad vendida y cuantas más unidades venda, mayor será su beneficio total. Esto contribuirá a su vez a reducir las pérdidas derivadas del coste fijo incurrido. En conclusión, \textit{en el corto plazo las empresas competitivas tratarán de cubrir sus costes variables y abandonarán el mercado si no lo consiguen}.


\preguntas

\seccion{Test 2017}
\textbf{8.} En el contexto de un mercado perfectamente competitivo, con muchas pequeñas empresas que venden un producto totalmente homogéneo, señale cuál de estas afirmaciones es \textbf{FALSA}:

\begin{itemize}
	\item[a] Cada empresa escoge su nivel óptimo de producción $q_i^*$ para tratar de obtener el máximo beneficio posible, pero esta producción puede cambiar si varía el precio de equilibrio del mercado.
	\item[b] Una empresa en competencia perfecta puede tener pérdidas aunque esté produciendo su nivel óptimo $q_i^*$, pero aun así se mantendría en el mercado a corto plazo.
	\item[c] Una empresa en un mercado de competencia perfecta tiene siempre una curva de coste marginal constante, ya que su producción es muy pequeña en comparación con el total del mercado.
	\item[d] A corto plazo, una empresa en competencia perfecta estará dispuesta a mantenerse en el mercado a pesar de obtener pérdidas siempre que recupere al menos sus costes fijos.
\end{itemize}

\seccion{Test 2015}

\textbf{11.} Considere una economía perfectamente competitiva con libertad de entrada y salida y con costes medios a largo plazo en forma de U.

\begin{itemize}
	\item[a] La oferta de la industria a largo plazo es creciente en el tramo de rendimientos decrecientes.
	\item[b] La oferta de la industria a largo plazo es creciente.
	\item[c] La oferta de la industria a largo plazo es tal que el precio es constante en todos los tramos.
	\item[d] La oferta de la industria a largo plazo es creciente a partir del mínimo de los costes variables. 
\end{itemize}

\seccion{Test 2013}

\textbf{7.} Una empresa pública deberá cerrar el negocio cuando:

\begin{itemize}
	\item[a] Los costes fijos sean muy altos.
	\item[b] El excedente social sea menor que las pérdidas.
	\item[c] Los costes fijos se conviertan en costes hundidos.
	\item[d] Los costes medios empiecen a ser crecientes.
\end{itemize}

\textbf{10.} Un mercado está formado por 100 consumidores idénticos cada uno de ellos con una función de demanda igual a $D=30-2P$ y 100 oferentes idénticos cuya función de oferta individual es $S=P$. Si el sector público decide imponer un precio mínimo de 12, entonces:

\begin{itemize}
	\item[a] La pérdida de bienestar causada será de 900.
	\item[b] El precio efectivo de mercado será el de equilibrio.
	\item[c] La pérdida de bienestar causada será de 1.500.
	\item[d] La pérdida de bienestar causada será de 1.200.
\end{itemize}

\seccion{Test 2007}

\textbf{9.} Una empresa que produce el bien $x$ en un mercado de competencia perfecta, con unos costes $C(x)$ es \textbf{FALSO} que:
\begin{enumerate}
	\item[a] Si la empresa está en equilibrio producirá donde $P=\text{CMg}$, independientemente de que está operando en el corto o el largo plazo.
	\item[b] Si la empresa está en equilibrio a corto plazo producirá una cantidad sobre la rama creciente de los costes marginales.
	\item[c] Si la empresa está en equilibrio a largo plazo producirá una cantidad sobre la rama creciente de los costes marginales.
	\item[d] Si la empresa está en equilibrio a largo plazo producirá una cantidad igual al tamaño óptimo de planta o escala mínima eficiente.
\end{enumerate}

\seccion{Test 2006}

\textbf{4.} Suponga una industria perfectamente competitiva con libertad de entrada, compuesta por empresas con idénticas funciones de coste a largo plazo. Un impuesto sobre la cantidad producida causa los siguientes efectos en el equilibrio:

\begin{itemize}
	\item[a] Deja invariable la cantidad producida en la industria, el precio y el número de empresas, y reduce el beneficio de cada productor.
	\item[b] Deja invariable la cantidad producida por cada empresa, pero aumenta el precio y disminuye el número de empresas.
	\item[c] Deja invariable el precio, pero disminuye el número de empresas y el beneficio de cada productor.
	\item[d] Deja invariable el precio, disminuye el beneficio de cada productor y aumenta el excedente del consumidor.
\end{itemize}

\seccion{Test 2005}

\textbf{12.} Suponga un mercado competitivo en equilibrio a largo plazo con libertad de entrada y salida de empresas en la industria, formado por empresas idénticas. Suponga, además, que cada empresa produce con beneficios nulos, la demanda de mercado es $X = A - P$, con $A > 15$, y el precio de equilibrio es $P=10$. A partir del equilibrio inicial, el gobierno introduce un impuesto de 5 euros por unidad producida. Si en el nuevo equilibrio tras el impuesto las empresas siguen obteniendo beneficios nulos, entonces en el nuevo equilibrio, es FALSO que:

\begin{itemize}
	\item[a] El número de empresas que operan en el mercado disminuirá.
	\item[b] Las empresas que operen no modificarán su nivel de producción.
	\item[c] El precio que pagan los consumidores será $P= 15 €$.
	\item[d] La producción de mercado no se verá alterada.
\end{itemize}

\notas

\textbf{2017:} \textbf{8.} ANULADA

\textbf{2015:} \textbf{11.} C

\textbf{2013:} \textbf{7.} B \textbf{10.} D

\textbf{2007} \textbf{9.} D

\textbf{2006:} \textbf{4.} B

\textbf{2005:} \textbf{12.} D

Al contar el tema, es importante tener en cuenta que hasta que no llegamos al apartado del largo plazo, estamos suponiendo que el número de empresas es fijo y por tanto la curva de oferta no es perfectamente elástica. 

Notar la descentralización del equilibrio competitivo como elemento clave, y expresar ventajas frente a proceso de decisión centralizado en cuanto a disminución de costes de recogida de información.

\bibliografia

Mirar en Palgrave:
\begin{itemize}
	\item competition
	\item large economies
	\item marginal utility of money
	\item market structure
	\item measure theory
	\item perfect competition
\end{itemize}

Blaug, M. \textit{Fundamental Theorems of Modern Welfare Economics, Historically Contemplated} (2007) History of Political Economy -- En carpeta del tema

Chiang, A; Wainwright, K. \textit{Fundamental Methods of Mathematical Economics} Sect. 15.2 The Dynamics of Market Price, 17.4 The Cobweb Model

Kreps, D. \textit{A Course in Microeconomic Theory}. Ch. 8 The Competitive Firm and Perfect Competition

Mas-Colell, A.; Green, J.; Whinston, M. \textit{Microeconomic Theory}. Ch. 10 Competitive Markets

Purcell, T.; Beard, R. \textit{Walrasian and Marshallian stability: An application to the Australian pig industry} (1999) \url{https://ageconsearch.umn.edu/bitstream/124531/2/Purcell1.pdf}

Takayama, A. \textit{Mathematical Economics} (1974)

Varian, H. \textit{Microeconomic Analysis}. Ch. 13 Competitive Markets


\end{document}
