\documentclass{nuevotema}

\tema{3A-3}
\titulo{Los economistas neoclásicos.}

\begin{document}

\ideaclave

La evolución del pensamiento económico es un proceso complejo que resulta de interacción de varios factores. Entre ellos, destacan el contexto económico del momento, el pensamiento económico precedente y los avances en otras disciplinas como la filosofía, las matemáticas, la física o la biología. El conocimiento de la historia del pensamiento económico permite entender las raíces intelectuales del pensamiento actual, aproximarse al análisis de los fenómenos económicos pasados y actuales, y valorar la importancia de los diferentes programas de investigación. Además, muchos de los conceptos económicos de la actualidad no son sino destilaciones y generalizaciones de teorías descritas por primera vez hace siglos. 

El término economía neoclásica aparece por primera vez en Veblen (1900) para hacer referencia al conjunto de teorías y herramientas de la obra de Alfred Marshall. Posteriormente, Hicks y Stigler utilizan el término para hacer referencia a un sistema teórico del cual la obra de Marshall es una parte y que agrupa un conjunto más amplio de conceptos y teorías. Aunque el término neoclásico fue posteriormente utilizado de manera idiosincrática por autores como Keynes, y en la actualidad en ocasiones para hacer referencia a la ciencia económica no heterodoxa, en esta exposición asumimos la definición de Stigler y Hicks, por ser la más habitual en la historia del pensamiento económico. Así, la revolución neoclásica se caracteriza por tres factores: el análisis marginal o relativo al efecto de unidades adicionales sobre una caracterización de las preferencias de un agente, el uso de las matemáticas para representar el comportamiento económico, y la optimización del bienestar que realizan los agentes económicos individuales como origen último de los fenómenos económicos. Así, la revolución neoclásica da lugar a un nuevo paradigma caracterizado por una creciente formalización del proceso de decisión y el individualismo metodológico. Este nuevo paradigma sentaría las bases de la modelización económica de la actualidad. El objeto de la exposición consiste así en dar respuesta a preguntas tales como: ¿quiénes fueron los economistas neoclásicos? ¿cuáles fueron sus aportaciones principales? ¿qué trayectoria intelectual siguieron? ¿quiénes los influenciaron? ¿a quiénes influenciaron? La estructura de la exposición se divide en dos partes: marginalistas, neoclásicos principales y otros nombres. Aunque la distinción entre ambos grupos es en gran medida arbitraria, es habitual enfatizar el principio marginal como componente destacado de la obra de Jevons, Menger y Walras. En el apartado de neoclásicos principales examinamos la trayectoria y obra de Marshall, J. B. Clark, Wicksell y Fisher. Por último, recorremos brevemente a otros economistas relevantes del mismo periodo tales como Wicksteed, Edgeworth, Pigou o Pareto, cuyos programas de investigación se encuentran ligados al neoclasicismo aunque en algunas áreas bien avanzan el programa e investigación hacia otros temas, bien divergen ligeramente de los temas principales del paradigma. 

El llamado marginalismo aparece en la segunda mitad del siglo XIX en un contexto de relativa estabilidad del pensamiento clásico combinada con la aparición de conceptos novedosos con potencial para transformar el método de formalización. El racionamiento deductivo y la modelización abstracta explotados con tanto éxito por Ricardo, la búsqueda de proporciones óptimas habitual en la obra de Malthus, los avances matemáticos, el principio marginal aplicado a la productividad y la renta y el utilitarismo tendente a caracterizar las decisiones adecuadas como resultado de una ponderación entre placer y dolor, formaron el caldo de cultivo apropiado para la aparición del llamado marginalismo de Jevons, Menger y Walras. Es necesario tener en cuenta, sin embargo, el trabajo de los predecesores del marginalismo. \textbf{Thünen} es habitualmente nombrado como el primer economista moderno en utilizar el cálculo diferencial en la modelización económica. Formaliza el concepto de productividad marginal y lo aplica a la modelización espacial para definir el concepto de renta y caracterizar el reparto óptimo de la tierra, precediendo en casi dos siglos a los modelos modernos de economía espacial y geográfica. \textbf{Cournot} define el concepto de función de demanda decreciente aunque sin relacionarlo con la idea de utilidad decreciente. Además, caracteriza el equilibrio competitivo en contextos de monopolio, duopolio y competencia perfecta, abriendo el camino para lo que sería un aspecto central de toda la microeconomía posterior. \textbf{Dupuit} y \textbf{Bertrand} critican y extienden a Cournot. El primero relaciona la función decreciente de demanda con la idea de utilidad decreciente. El segundo critica el monopolio de Cournot y define un nuevo modelo en el que la variable de decisión es el precio. En una obra de oscura circulación y que estuvo a punto de caer en el olvido, \textbf{Gossen} formuló las llamadas tres leyes de Gossen --que serían posteriormente redescubiertas por Jevons-: i) la utilidad del consumo decrece con cantidad del bien, ii) la utilidad marginal debe ser igual para cada bien en el óptimo, iii) un bien tiene valor si la demanda es mayor que la oferta, de tal manera que es de la escasez de donde se deriva el valor de un bien. 

En 1871 aparecen tres obras que habrían de cambiar la trayectoria del pensamiento económico por su caracterización explícita de la utilidad marginal decreciente como determinante principal de la decisión de los consumidores. William Stanley \textbf{Jevons} (1835--1882) fue el autor de la primera de ellas. \textit{Teoría de la Economía Política} (1871) formula una teoría de la utilidad basada en el pensamiento de Bentham. Según Jevons, los agentes buscan maximizar su utilidad, entendiendo ésta como una cuantificación del bienestar individual. Así, el valor y el precio de las cosas depende únicamente, segun Jevons, de la utilidad que genera. Ésta concepción del valor es una ruptura frontal con la concepción clásica del valor que atribuía éste a la cantidad de trabajo aplicada en la producción de un bien. Es preciso notar que Jevons rechazaba la posibilidad de realizar comparaciones de utilidad entre diferentes individuos. En este marco teórico, Jevons analiza también la oferta de trabajo de los agentes individuales en términos de utilidad. Según Jevons, el trabajo ofertado será aquél que iguale la utilidad marginal de la remuneración con la desutilidad que genera el trabajo a partir de cierta cantidad. Este análisis es precursor de todo el análisis neoclásico posterior del mercado de trabajo. Además, Jevons niega la existencia de una relación inversa entre beneficios y salarios, en una crítica directamente dirigida a la idea de la lucha de clases. La llamada paradoja de Jevons es otra de las aportaciones más duraderas del autor a la ciencia económica. La paradoja describe la observación empírica de que aumentos en la productividad de un factor resultan en aumentos de la demanda y el uso del factor, de manera contraria a como cabría esperar dado que resulta necesaria una menor cantidad de factor productivo para producir una misma cantidad de bien. Los equilibrios de manchas solares (``\textit{sunspot equilibria}'') son una idea original a Jevons que tendría continuidad en el siglo XX con los modelos de equilibrios múltiples. El concepto hace referencia a la posibilidad de que existan múltiples equilibrios en un sistema y que la realización de uno u otro dependa de factores ajenos a los fenómenos representados por la teoría económica. La aparición de manchas solares es uno de esos hipotéticos factores de determinación del equilibrio. Jevons intentó relacionar las manchas solares con la aparición de fluctuaciones cíclicas debidas a la variación de la producción agrícola, dando nombre al concepto.

Carl \textbf{Menger} (1840--1921) no sólo formuló la teoría de la utilidad y el principio equimarginal de forma independiente a Jevons, sino que dio lugar a escuela económica de pleno derecho --la escuela austriaca- y defendió las virtudes de la deducción y el individualismo metodológico. \textit{Principios de Economía} (1871) es la obra en la que Menger enuncia el principio equimarginal. Menger evita la formalización matemática y las influencias benthamitas, formulando la llamada ``tabla de Menger'' que caracteriza la utilidad marginal que aporta cada bien hasta que las utilidades marginales se igualan. Aunque es inevitable aplicar el término, Menger rechaza la idea de que los agentes reciben unidades de utilidad que cuantifican su bienestar a la manera de los utilitaristas, y se limita a conceder un mero valor expositivo a la cuantificación del principio equimarginal. Menger rechaza categóricamente las teorías del valor basadas en el trabajo y resuelve así la paradoja del agua y los diamantes a partir de la demanda como factor determinante. La teoría de la imputación de Menger permite caracterizar el valor de los factores de producción o los bienes intermedios: son valiosos porque son capaces de generar bienes de consumo final. La teoría del valor de Menger es así una \textit{teoría subjetiva} que muestra el intercambio como un proceso capaz de generar valor para todos los intervinientes. Menger analizó también la aparición del dinero como sustitutivo del trueque. Caracterizó la aparición de dinero como un fenómeno espontáneo (``orgánico'') que aporta bienestar a la sociedad por la posibilidad de realizar intercambios sin tener que encontrar contrapartidas que demanden exactamente el bien que son capaces de ofrecer. Así, el dinero es según Menger una institución que permite intermediar entre agentes con diferentes capacidades y necesidades reduciendo sustancialmente los costes de transacción. 

La obra de Menger dio lugar a la llamada escuela austriaca. En Alemania, el historicismo trataba de inferir leyes con las que describir la evolución de la economía a partir del examen de la historia. En un contexto de confrontación metodológica, los historicistas alemanes nombraron el pensamiento de Menger y sus discípulos tales como \textbf{Böhm-Bawerk} (1851--1914) o \textbf{von Wieser} (1851--1926) como ``escuela austriaca''. La escuela austriaca influyó fuertemente en la economía neoclásica tanto por sus ideas centrales como por contribuciones más específicas de sus autores. Los austriacos defienden el individualismo metodológico como principio orientador de toda la ciencia económica. La economía es para ellos un sistema dinámico en constante transformación a medida que los agentes acceden a nueva información que deben gestionar y respecto de la que deben reaccionar. Así, rechazan categóricamente el estudio de los fenómenos económicos desde un punto de vista estático y macroscópico. La obra de Böhm-Bawerk examinó con especial detalle el proceso productivo y el papel del tiempo sobre precio, valor e ingresos. Según el autor, el interés surge de tres elementos: la impaciencia o la preferencia por el presente que muestran los seres humanos, la expectativa de mayor riqueza en un futuro que genera la inversión y el papel de la abstinencia como factor de producción que aumenta el valor de los bienes. Los dos primeros factores dan lugar a la demanda de un pago de interés. El tercer factor hace posible el pago del interés. Además de este análisis del pago al capital, Böhm-Bawerk fue uno de los primeros economistas en comentar y criticar la obra de Marx. Von Wieser fue el otro gran discípulo de Menger. Introdujo por primera vez los términos ``utilidad marginal'' y ``coste de oportunidad''. Puso gran énfasis en la subjetividad de la utilidad para justificar el hecho de que un mismo bien pueda aportar diferente grado de bienestar a diferentes individuos. La influencia de la escuela austriaca fue más allá del periodo propiamente neoclásico, de la mano de autores como Schumpeter y Hayek (premio Nobel 1974).

Léon \textbf{Walras} (1834--1910) es uno de los autores más importantes no ya de la revolución neoclásica, sino de toda la ciencia económica hasta nuestros días. Su definición y subsecuente formulación del concepto del equilibrio general en lenguaje matemático tuvieron una recepción controvertida cuando fueron presentadas al público en su obra \textit{Elementos de Economía Política Pura} (1871), pero transformaron radicalmente la modelización de fenómenos económicos a partir de los años 70 del siglo XX y unos años después de su primera traducción al inglés en 1954. Toda la microeconomía, la macroeconomía moderna, la economía matemática, los neo-austriacos, Pareto, Wicksell, Arrow, Debreu, Leontief, Friedman... están influenciados por el concepto walrasiano del equilibrio general. Vilfredo Pareto fue su sucesor en la cátedra de economía política de Lausana y continuó algunos de sus temas. Además, Walras fue el otro autor que formuló el principio equimarginal en 1871. En lo metodológico, Walras consolida plenamente tanto el concepto de modelo abstracto como el uso de las matemáticas en economía. El autor, que conocía la obra de Cournot, Dupuit y Bertrand por vía de su padre y también economista Auguste Walras, analiza la economía en un nivel de abstracción desconocido hasta la fecha, en el que el comportamiento económico y la decisión de los agentes se representa mediante funciones y conjuntos matemáticos. Con ese lenguaje, Walras introduce el modelo de equilibrio general como una representación abstracta de todas las interacciones que afectan a una economía y resultan relevantes para caracterizar un fenómeno. En ese contexto, una economía es un conjunto de mercados interconectados a través de las restricciones presupuestarias a las que están sujetos los agentes que participan en la economía. El equilibrio walrasiano se define como el conjunto de cantidades y precios que maximizan las preferencias de todos los participantes, que respetan todas las restricciones presupuestarias y que eliminan todas las diferencias entre oferta y demanda en diferentes mercados. Walras caracterizó las condiciones de existencia y estabilidad del equilibrio. Aunque la caracterización de la existencia de equilibrio de Walras es errónea y una formulación correcta no apareció hasta el siglo XX, su examen del problema sentó las bases de los desarrollos posteriores. En cuanto a la estabilidad del equilibrio, Walras trataba de describir la tendencia (o falta de ella) espontánea de una economía hacia ese equilibrio cuando los agentes intercambian voluntaria y libremente los bienes en cuestión. Walras describió un mecanismo que podría llevar a ese equilibrio y lo denominó ``\textit{tâtonnement}''. Éste consiste en el intercambio de cantidades a precios que varían en función de los excesos de demanda: excesos de demanda positivos inducen subidas de precios y viceversa, hasta que se obtiene una asignación de equilibrio. El modelo de equilibrio general walrasiano dio lugar a un gran número de modelos relacionados que caracterizan las decisiones de producción, la maximización de utilidad, el crédito o aspectos monetarios. Walras también desarrolló una teoría del capital que entiende éste factor de producción como una fuente potencial de flujos de servicios que permiten llevar a cabo un proceso productivo, y que determinan el precio del capital.

Alfred \textbf{Marshall} (1842--1924) es, junto con Walras, el economista asociado al neoclasicismo con un impacto más duradero y directo sobre la ciencia económica posterior. La obra de Marshall se convirtió en la referencia obligada para todos los estudiantes de economía de la época. Así, los \textit{Principios de Economía Política} de Mill que habían sido la referencia desde su aparición en 1848, cedieron el paso a los \textit{Principios de Economía} (1890) de Marshall. \textit{Teoría Pura del Comercio Internacional} (1879) es su otra gran obra. El pensamiento de Marshall estuvo fuertemente influenciado por J. S. Mill, Bentham, Ricardo y los autores marginalistas. Su influencia es evidente en Keynes y todos los economistas posteriores asociados a la universidad de Cambrigde, así como a la microeconomía, la organización industrial o la teoría del comercio internacional. En lo metodológico, Marshall se caracteriza por aplicar el llamado enfoque de equilibrio parcial a los mercados de bienes, y el enfoque de equilibrio general para los mercados de factores aunque no desarrolla plenamente esta idea. Se opone a los ataques frontales a la economía clásica por considerar que fragilizan el estatus de la disciplina. Las matemáticas juegan un papel mixto en la obra de Marshall. Aunque las utiliza a menudo para expresar conceptos especialmente relevantes, prefiere en general los métodos verbales y recomienda utilizar las matemáticas sólo en la medida en que sean superiores a la palabra para transmitir un concepto. En general, Marshall trata de ser pragmático a la hora de modelizar la economía y evita los modelos excesivamente complejos o las abstracciones demasiado alejadas de la realidad que caracterizaban a Walras. Las diferencias con este último autor son básicamente i) el enfoque parcial frente al general ii) el ajuste hacia equilibrio en competencia perfecta, que Marshall divide explícitamente en periodos temporales en los cuales en el corto plazo es más fácil variar precios y en el largo plazo cantidades, mientras que Walras tiende a contemplar el ajuste de precios de manera predominante y iii) la importancia de factores ajenos a la pura teoría económica: mientras que Walras formula modelos que tienden a abstraerse hasta el punto de considerar sólo los fenómenos económicos más esenciales, Marshall prefiere prestar al menos cierta atención a factores institucionales, sociales o culturales en sus modelos. 

La teoría de la demanda de Marshall caracteriza el equilibrio parcial característico de su obra. El autor trata de representar el comportamiento del mercado de un bien dado manteniendo constantes los valores que afectan al resto de bienes. Para ello, introduce la función de utilidad separable cuasilineal y define el proceso de decisión racional como una optimización de esa función de utilidad separable sujeta a una restricción presupuestaria. Este marco se concreta en una función de utilidad que toma la forma de una suma de dos elementos: una función de la cantidad consumida del bien cuyo mercado se pretende analizar, y la cantidad de dinero o de resto de bienes. De esta forma, la utilidad marginal del dinero es constante. Como medida del bienestar, Marshall define el excedente del consumidor como la suma de las utilidades marginales aportadas por cada unidad del bien hasta que la utilidad marginal del bien se iguala con la utilidad marginal del dinero. 

La teoría de la utilidad marginal de Marshall sirve también como fundamento de su teoría de los periodos de mercado y del valor ``normal'' de los bienes. En el periodo presente, todos los costes son hundidos y el único objetivo de la empresa es vender una producción cuya cuantía es fija. Por ello, en los mercados de bienes perecederos es la demanda el factor que determina el precio. En los mercados de bienes duraderos, entran en juego consideraciones intertemporales ya que la empresa puede efectivamente restringir la demanda. En el corto plazo, la producción está restringida pero es variable dados unos factores de producción aplicables en cuantías fijas tan sólo de forma parcial. En e largo plazo, todos los factores son variables y nuevas empresas pueden entrar en el mercado, por lo que las empresas minimizarán costes y los precios de venta se verán determinados por el coste de producción. Marshall define así el valor de mercado como el precio en un momento determinado y el valor normal como el precio cuando todas las fuerzas de mercado y todos los procesos de optimización han tenido lugar. El beneficio normal corresponde en estos casos con el coste de oportunidad y se iguala al beneficio mínimo necesario para mantener la actividad. Hasta que se alcanza este nivel de beneficio normal, existe la posibilidad de extraer beneficios extraordinarios que Marshall denomina cuasirrentas por su carácter temporal. El concepto de renta es para Marshall idéntico al concepto de renta ricardiana en tanto que remuneración por encima de la productividad marginal que corresponde al beneficio extraíble de la escasez. 

Marshall realiza también algunas aportaciones originales a la teoría del comercio internacional con la introducción de las curvas de oferta recíproca que desarrollan a J. S. Mill. Las curvas de oferta recíproca se aplican al análisis arancelario para determinar que un país puede mejorar si aplica un arancel cuando se encuentra en el segmento inelástico de la curva de demanda extranjera. En general, se muestra a favor del libre comercio e incluso de la liberalización unilateral, pero reconoce que la estructura del mercado determina la posibilidad de mejorar aplicando medidas proteccionistas.

John \textbf{Bates Clark} (1847--1938) fue el introductor del neoclasicismo en Estados Unidos, a pesar de estar influenciado por los historicistas alemanes. Su obra influenció fuertemente a Fisher, a la incipiente teoría de la organización industrial y a los economistas americanos de la época. Su obra \textit{La Distribución de la Riqueza} (1899) recoge sus principales aportaciones al área que da título al libro. Bates Clark examina los determinantes de la distribución del producto y concluye que todo ingreso puede reducirse a un pago por trabajo. El beneficio es el pago al trabajo del empresario, el retorno del capital es debido al ahorro derivado del trabajo. La renta, sin embargo, es un ingreso espurio derivado de la escasez. Por todo ello, Bates Clark se opone a la concepción marxista de que la organización social en clases determina la distribución. Formaliza la idea que la remuneración depende de la productividad marginal. Por otro lado, Bates Clark examina la importancia de la estructura de los mercados de factores sobre la distribución. Así, concluye que la competencia y la movilidad de factores garantizan que los retornos del capital y el trabajo tenderán a igualarse.

Knut \textbf{Wicksell} (1851-1926) fue un economista escandinavo fuertemente influenciado por Menger que examinó una gran variedad de temas asociados a la economía neoclásica y contribuyó a estabilizar el programa de investigación. Influenció a la escuela austriaca, a Hicks, a Keynes, al incipiente análisis dinámico y al análisis en torno a la idea del desequilibrio centra en Keynes y los neokeynesianos de los años 60, además de dar lugar a la llamada Escuela de Estocolmo de Economía. Además de sintetizar las aportaciones de otros neoclásicos y exponer los puntos débiles de otras teorías, su aportación más destacada es su análisis del tipo de interés. Wicksell definió el ``tipo de interés natural'' como aquella tasa que implica una inflación estable. Cuando el tipo predominante es más alto que el natural, la actividad económica se contrae por debajo de su capacidad. Cuando éste es más bajo, la actividad económica recibe un estímulo que impulsa el producto por encima de su valor normal. Esta forma de entender el interés es el origen de los usos posteriores del adjetivo ``natural'' que subyacen a la idea de la NAIRU o el output gap.

Irving \textbf{Fisher} (1867--1947) fue uno de los economistas más eclécticos de la etapa neoclásica. Sus conocimientos matemáticos, su prosa fluida, su capacidad para examinar temas dispares y la exploración de temas novedosos en el momento le situaron como una influencia clave en toda la macroeconomía posterior, la econometría, la teoría de números índices, la economía austríaca y especialmente la obra de Schumpeter, el monetarismo, la modelización de la demanda del consumo. La gran obra de Irving Fisher es \textit{La Teoría del Interés} (1907) en sus diferentes ediciones. Fisher introduce la modelización explícita y formalizada del tiempo como dimensión relevante de la decisión de agentes económicos. Esperar para consumir es costoso y por ello los agentes exigen una compensación por ahorrar. Así, el interés surge de la impaciencia y no de la abstinencia como factor de producción postulada por Böhm-Bawerk. Además, el interés está determinado por el coste de oportunidad del capital, de tal manera que las inversiones tienden a remunerarse a la tasa correspondiente a inversiones similares. Con esta base, Fisher introduce el descuento de flujos de caja y la valoración de inversiones como diferencia entre flujos positivos y negativos descontados al presente. El \textit{teorema de la separación} de Fisher precede al teorema de Modigliani-Miller. El teorema muestra que si los mercados de capitales son perfectos, las decisiones de inversión y financiación de una empresa son independientes, de tal manera que una empresa debe centrarse en maximizar el valor presente de sus inversiones de forma independiente de los instrumentos que utilice para financiarlas, en ese contexto idealizado.

Fisher realiza también aportaciones duraderas en el ámbito de la teoría monetaria. El autor otorga un papel fundamental al dinero en la determinación de las condiciones económicas. Para ello, reformula la ecuación cuantitativa del dinero. Explicita su carácter de condición de equilibrio frente a la mera identidad, afirmando que los precios son el elemento pasivo de la ecuación pero que tardan en ajustarse y ello provoca fluctuaciones en la economía real. Fisher pone énfasis en la dinámica como ajuste hacia el equilibrio: afirma que las economías están en desequilibrio de forma natural y el objetivo de la economía debe consistir en entender ese ajuste hacia el equilibrio. El poder adquisitivo del dinero es un tema de especial interés para el autor, y es en esa materia donde desarrolla la teoría de números índices para poder cuantificar las variaciones de poder adquisitivo.

La ecuación de Fisher es una de sus contribuciones más duraderas. La ecuación relaciona el tipo de interés nominal con el tipo real y la inflación. Aunque ex-post se trate de una identidad que define el tipo de interés real, desde un punto de vista ex-ante puede entenderse como una condición de equilibrio a la que tienden las economías. Fisher afirma que el ajuste del tipo de interés nominal es lento e imperfecto en el corto plazo. Así, un estímulo nominal que aumente la inflación supone una bajada del tipo de interés real y con ello, un estímulo a la inversión. Fisher utiliza este resultado para teorizar sobre las fluctuaciones cíclicas de la economía. Por ejemplo, afirma que un factor determinante de la Gran Depresión fue la subida brusca del tipo de interés real. 

Más allá de los autores anteriores, las contribuciones que realizaron \textbf{otros nombres} a la ciencia económica en el periodo neoclásico contribuyeron a estabilizar e iniciar fructíferos programas de investigación que llegan hasta nuestros días. \textit{Wicksteed} formalizó la idea del agotamiento del producto de Menger e introdujo la idea en Inglaterra. Para ello, caracterizó la función de producción como una función homogénea de grado 1. En este contexto, los factores remunerados a producto marginal agotan efectivamente el producto de la economía. Estudió también las condiciones necesarias para la eficiencia económica desde un punto de vista moral, introduciendo la idea del no-tuísmo frente al egoísmo como requisito para la eficiencia. 

Francis Ysidro \textbf{Edgeworth} (1845--1926) es habitualmente considerado como el verdadero fundador de la economía matemática. Aunque Walras ya había utilizado las matemáticas para expresar su modelo del equilibrio general, es Edgeworth quien introduce plenamente técnicas avanzadas, fuertemente influenciado por los avances de la época y en especial la mecánica clásica de Hamilton. Avanza el concepto de óptimo de Pareto, que éste último autor desarrollaría y bautizaría. Edgeworth critica la representación de Walras del ajuste hacia el equilibrio, señalando que cuando se producen las transacciones, cambian también las dotaciones de los agentes y el equilibrio puede volverse inestable. Esta crítica forzó a Walras a reformular su concepto del \textit{tâtonnement} e introducir la idea de precios virtuales, a los cuales no llegan a producirse transacciones pero que sí generan excesos de demanda respecto de los cuáles varían los precios hasta el equilibrio. Edgeworth propuso un nuevo mecanismo de equilibrio basado en procesos de negociación sucesiva entre agentes, precursor del enfoque del núcleo y la teoría de juegos.

Arthur Cecil \textbf{Pigou} (1877--1959) fue el sucesor de Marshall en Cambridge. Destacó por sus contribuciones pioneras a la economía del bienestar, formalizando las bases sentadas por Marshall en cuanto a la valoración del bienestar. Introdujo los costes sociales frente a los costes privados, y postuló la posibilidad de aplicar impuestos correctores para corregir las externalidades negativas. Entre otros muchos temas, examinó también el equilibrio competitivo de largo plazo, postulando la curva de costes medios en forma de U y el equilibrio en la escala eficiente. La síntesis que Pigou hizo del neoclasicismo económico sirvió a Keynes como objetivo de su crítica frontal a la existencia de fuerzas autoestabilizadoras que empujaban a las economías hacia equilibrios de plena utilización de su capacidad productiva.

Vilfredo \textbf{Pareto} (1848--1923) fue el sucesor de Walras en Lausana. Desarrolló algunos de sus temas gracias a su formación matemática, y su obra fue fundamental para estabilizar el modelo neoclásico de la demanda. Demostró que la cardinalidad de las utilidad no es necesaria, y que se puede representar la demanda de un agente racional a partir de relaciones de preferencia y funciones de utilidad ordinales. En el campo de la sociología realizó también contribuciones relevantes, relacionando los métodos cuantitativos cada vez más habituales en economía con el estudio de la organización de la sociedad.

A lo largo de la exposición hemos examinado la trayectoria y contribuciones de los principales autores que dieron forma a la revolución neoclásica: Jevons, Menger, Walras, Marshall, Bates Clark, Wicksell, Fisher, Wicksteed, Edgeworth, Pigou y Marshall, además de algunos predecesores relevantes. Como es habitual en la historia del pensamiento económico, muchas de las contribuciones de estos autores no tuvieron continuidad como tales. Sin embargo, la economía neoclásica estabilizó una forma de entender y representar los fenómenos económicos que perdura hasta hoy: no es posible entender un artículo académico, un libro de texto o un informe económico sin ser partícipe consciente o inconsciente de las ideas centrales del pensamiento económico neoclásico. 


\seccion{Preguntas clave}
\begin{itemize}
	\item ¿Quiénes fueron los economistas neoclásicos?
	\item ¿Cuáles fueron sus aportaciones?
	\item ¿Qué trayectoria intelectual siguieron?
	\item ¿Quiénes los influenciaron? ¿A quienes influenciaron?
\end{itemize}

\esquemacorto

\begin{esquema}[enumerate]
	\1[] \marcar{Introducción}
		\2 Contextualización
			\3 Evolución de la ciencia económica
			\3 Historia del pensamiento económico
			\3 Economía neoclásica
		\2 Objeto
			\3 ¿Quiénes fueron los economistas neoclásicos?
			\3 ¿Cuáles fueron sus aportaciones principales?
			\3 ¿Qué trayectoria siguieron?
			\3 ¿Quienes los influenciaron y a quiénes influenciaron?
			\3 ¿Qué implicaciones se derivan de sus teorías?
		\2 Estructura
			\3 Los marginalistas
			\3 Neoclásicos
			\3 Otros autores
	\1 \marcar{Los marginalistas}
		\2 Idea clave
			\3 Método deductivo
			\3 Comportamiento racional
			\3 Énfasis microeconómico
			\3 Demanda como determinante de valor
			\3 Principio equimarginal
			\3 Enfoque de equilibrio estático
		\2 Predecesores
			\3 Thünen
			\3 Cournot
			\3 Dupuit
			\3 Gossen
		\2 Jevons, William Stanley
			\3 Vida
			\3 Obras
			\3 Teoría de la utilidad
			\3 Mercado de trabajo
			\3 Gestión de recursos finitos
			\3 Equilibrio de manchas solares (sunspot equilibria)
		\2 Walras, Léon
			\3 Vida
			\3 Influenciado por
			\3 Influenció a
			\3 Obras
			\3 Metodología
			\3 Equilibrio general
			\3 Teoría de la distribución
		\2 Menger, Carl
			\3 Vida
			\3 Obras
			\3 Teoría del valor
			\3 Teoría de la distribución
			\3 Methodenstreit
		\2 Escuela austriaca
			\3 Idea clave
			\3 Böhm-Bawerk
			\3 Von Wieser
			\3 Von Mises
			\3 Von Hayek
			\3 Schumpeter
			\3 Teoría austriaca del ciclo
	\1 \marcar{Neoclásicos}
		\2 Alfred Marshall
			\3 Vida
			\3 Influenciado por
			\3 Influenció a
			\3 Obras
			\3 Diferencias con Walras
			\3 Metodología
			\3 Teoría de la demanda con utilidad cuasilineal
			\3 Bienestar
			\3 Periodos de mercado y valor ``normal''
			\3 Rentas y cuasi-rentas
			\3 Comercio internacional
			\3 Economías de escala externas
		\2 John Bates Clark
			\3 Vida
			\3 Influenciado por
			\3 Influenció a
			\3 Obras
			\3 Metodología
			\3 Teoría de la distribución
			\3 Capitalismo vs socialismo
			\3 Distribución del producto
			\3 Organización industrial
		\2 Irving Fisher
			\3 Vida
			\3 Influenciado por
			\3 Influenció a
			\3 Obras
			\3 Interés y capital
			\3 Teoría de la inversión
			\3 Teoría monetaria
		\2 Knut Wicksell
			\3 Vida
			\3 Influenciado por
			\3 Influenció a
			\3 Obras
			\3 Síntesis del neoclasicismo
			\3 Interés
			\3 Proceso acumulativo
			\3 Efectos de Wicksell
	\1 \marcar{Otros nombres}
		\2 Philip Henry Wicksteed
			\3 Vida
			\3 Distribución del producto
			\3 Egoísmo frente a no-tuismo
		\2 Francis Ysidro Edgeworth
			\3 Vida
			\3 Obras
			\3 Economía matemática
			\3 Ajuste hacia equilibrio
		\2 Arthur Cecil Pigou
			\3 Vida
			\3 Economía del bienestar
			\3 Equilibrio competitivo de largo plazo
			\3 Externalidades
			\3 Keynes
		\2 Vilfredo Pareto
			\3 Vida
			\3 Utilidad ordinal
			\3 Optimización
			\3 Producción
			\3 Sociología
	\1[] \marcar{Conclusión}
		\2 Recapitulación
			\3 Marginalistas
			\3 Neoclásicos
			\3 Otros autores
		\2 Idea final
			\3 Programas de investigación sin continuidad
			\3 Contribuciones duraderas y germen de programas

\end{esquema}

\esquemalargo

\begin{esquemal}
	\1[] \marcar{Introducción}
		\2 Contextualización
			\3 Evolución de la ciencia económica
				\4 Conjunción de múltiples factores
				\4 Contexto económico
				\4 Contexto teórico previo en economía
				\4 Avances en otras discIplinas
				\4[] Filosofía
				\4[] Matemáticas
				\4[] Biología
			\3 Historia del pensamiento económico
				\4 Permite entender origen de pensamiento actual
				\4 Permite entender problemas históricos
				\4 Permite valorar programas de investigación
			\3 Economía neoclásica
				\4 Término ``neoclásico''
				\4[] Thorstein Veblen (1900)
				\4[] $\to$ Primer uso
				\4[] $\to$ Programa de investigación de Marshall
				\4[] Hicks, Stigler
				\4[] $\to$ Extienden a revolución marginal
				\4[] $\to$ Individualismo metodológico
				\4[] $\to$ Productividad marginal
				\4 Nuevas herramientas teóricas
				\4[] Análisis marginal
				\4[] $\to$ Énfasis sobre efectos de unidades adicionales
				\4[] Matemáticas
				\4[] $\to$ Aplicación de avances matemáticos
				\4[] $\to$ Al diseño de modelos que expliquen fenómenos económicos
				\4[] Optimización individual
				\4[] $\to$ Agentes racionales
				\4[] $\to$ Información completa sobre consecuencias actos
				\4[] $\to$ Maximizan unas preferencias o beneficios
				\4 Nuevo paradigma
				\4[] Conjunto de nuevas herramientas compatibles
				\4[] Conjunto de modelos relativamente coherentes
				\4[] Cuerpo general de implicaciones de política económica
		\2 Objeto
			\3 ¿Quiénes fueron los economistas neoclásicos?
			\3 ¿Cuáles fueron sus aportaciones principales?
			\3 ¿Qué trayectoria siguieron?
			\3 ¿Quienes los influenciaron y a quiénes influenciaron?
			\3 ¿Qué implicaciones se derivan de sus teorías?
		\2 Estructura
			\3 Los marginalistas
			\3 Neoclásicos
			\3 Otros autores
	\1 \marcar{Los marginalistas}
		\2 Idea clave
			\3 Método deductivo
				\4 Modelos abstractos
				\4 Creciente matematización
				\4[] $\to$ A pesar de carencias técnicas
			\3 Comportamiento racional
				\4 Influencia utilitarista
				\4 Agentes económicos ponderan placer y dolor
				\4 Tienen en cuenta utilidad que aportan los bienes
			\3 Énfasis microeconómico
				\4 Individualismo metodológico
				\4 Comportamiento de agentes individualizados
			\3 Demanda como determinante de valor
				\4 Fuerza primaria en determinación de precios
				\4[] $\to$ Contrapuesto a teoría del valor-trabajo
				\4 Utilidad subjetiva determina demanda
			\3 Principio equimarginal
				\4 Extensión del principio marginal de Ricardo
				\4 Aplicación a toda la teoría
				\4[] Utilidad
				\4[] $\to$ Consumir bienes hasta igualar utilidad marginal
				\4[] Empresas
				\4[] $\to$ Aplicar ff.pp. hasta igualar beneficio marginal
			\3 Enfoque de equilibrio estático
				\4 Tendencia hacia el equilibrio de las economías
				\4 Aceptan Ley de Say
				\4 En general, laissez-faire
		\2 Predecesores
			\3 Thünen
				\4 1783-1850
				\4 Primer economista moderno
				\4 Uso del cálculo diferencial
				\4 Enuncia principio equimarginal
				\4[] Aplica al análisis de la utilidad
				\4 Análisis del uso óptimo de la tierra
				\4[] $\to$ Motivo de pobreza es falta de tierra libre
				\4 Análisis de la productividad marginal
				\4 Precursor de los modelos espaciales
			\3 Cournot
				\4 1801-1877
				\4 Función de demanda decreciente
				\4[] $\to$ Pero no relaciona con UMg decreciente
				\4 Teoría del monopolio
				\4[] $\to$ Maximización de función de beneficio
				\4[] $\to$ Partiendo de función de demanda
				\4 Teoría del duopolio de Cournot
				\4[] $\to$ Empresas deciden producción
				\4[] $\to$ Asumen producción de competidor no reacciona
				\4 Competencia perfecta
				\4[] $\to$ Caso extremo de número infinito de vendedores
				\4 Crítica de Bertrand en 1880s
				\4[] $\to$ Es precio lo que las empresas mantienen fijo
			\3 Dupuit
				\4 1804-1866
				\4 Semejanzas con Cournot
				\4[] École des Ponts et des Chaussées
				\4[] Formación de ingeniero
				\4[] Función de demanda decreciente en precio
				\4 Relación demanda y utilidad
				\4[] Utilidad decreciente $\then$ Demanda decreciente
				\4 Primitiva medida del excedente del consumidor
			\3 Gossen
				\4 1810-1858
				\4 Leyes de Gossen
				\4[i] UMg decrece con la cantidad de bien
				\4[ii] UMg es igual para cada bien en el óptimo
				\4[iii] Un bien tiene valor si demanda es mayor que oferta
				\4[] $\then$ El valor se deriva de la escasez
				\4 Obra original apenas tiene repercusión
				\4[] $\to$ Redescubierto por Jevons
		\2 Jevons, William Stanley
			\3 Vida
				\4 1835-1882
				\4 Formación ciencias naturales
				\4 Termodinámica, física, electromagnetismo
				\4[] Influencias centrales
			\3 Obras
				\4 La Teoría de la Economía Política (1873)
			\3 Teoría de la utilidad
				\4 Influencia benthamita declarada
				\4 Agentes buscan maximización de la utilidad
				\4 Utilidad marginal decrece con cantidad consumida
				\4 Relación entre utilidad marginal y precio de bien
				\4[] Si es igual para todos los bienes consumidos
				\4[] $\to$ Agente está en equilibrio
				\4[] Si algunos bienes tienen mayor UMg/P
				\4[] $\to$ Posible consumir más de esos bienes y $\uparrow$ bienestar
				\4 Equilibrio
				\4[] Agentes consumen bienes hasta igualar UMg/P
				\4[] Precios dependen de oferta y demanda
				\4[] Oferta es creciente con oferta
				\4[$\then$] Valor depende de demanda
				\4[$\then$] Utilidad marginal determina precios relativos
				\4[$\then$] Valor/precio depende únicamente de utilidad
				\4[] Ruptura definitiva con modelo clásico TVT
				\4 Aplicación a paradoja del agua y los diamantes
				\4[] Una muy pequeña cantidad de diamantes
				\4[] $\to$ Utilidad marginal elevada
				\4[] Una cantidad muy grande de agua
				\4[] $\to$ Utilidad marginal muy pequeña
				\4[] Para equilibrar relación UMg y precios
				\4[] $\to$ Necesario mucha agua por un diamante
				\4[] $\then$ Diamantes valen mucho más que agua
				\4 Contrario a comparaciones interpersonales
			\3 Mercado de trabajo
				\4 Análisis de oferta en términos de utilidad
				\4[] Precursor análisis neoclásico del mercado de trabajo
				\4 Oferta es cantidad de trabajo que iguala:
				\4[] Utilidad marginal de remuneración
				\4[] Desutilidad de trabajo
				\4 Niega relación inversa beneficios-salarios
				\4[] Salario no es salario de subsistencia
				\4[] $\to$ No es cantidad necesaria para reproducir bienes
				\4[] $\to$ Ataque a base de la lucha de clases
				\4 Crítica a teoría del fondo de salarios
				\4[] Salarios vienen determinados por capital acumulado
				\4[] $\to$ Salarios = $\frac{\text{Capital}}{\text{Trabajadores}}$
				\4[] $\then$ Tautología
				\4[] $\then$ No explica cuánto capital se destina a salarios
				\4[] $\then$ A posteriori, siempre cierta
			\3 Gestión de recursos finitos
				\4 Carbón en Inglaterra
				\4[] Cada vez más caro de extraer
				\4[] Incentivos a extraer de peores yacimientos
				\4[] Aumento de coste de bienes ingleses
				\4[] $\to$ Tendencia a cada vez menor productividad
				\4[] $\then$ Inglaterra pierde poder económico
				\4[] $\then$ Estados Unidos será futura potencia mundial
				\4 Paradoja de Jevons
				\4[] Observación empírica
				\4[] Aumento de productividad de un factor
				\4[] $\to$ Aumenta su demanda y su uso
			\3 Equilibrio de manchas solares (sunspot equilibria)
				\4 En contexto de múltiples equilibrios
				\4 ¿Qué equilibrio acaba prevaleciendo?
				\4[] $\to$ Depende de fenómeno exógeno que coordina
				\4 Manchas solares
				\4[] Fenómeno completamente exógeno a economía
				\4 Intento de Jevons
				\4[] Relacionar ciclo de negocios con manchas solares
				\4[] $\to$ Causan variaciones en tiempo y producción agrícola
				\4 Evidencia empírica posterior
				\4[] Rechaza relación
				\4 Utilización teórica del concepto
				\4[] Pigou y otros
				\4[] Keynes: ``animal spirits'' es heredero
				\4[$\then$] Caracterizar determinación de equilibrio
				\4[] $\to$ En función de fenómeno exógeno al modelo
		\2 Walras, Léon
			\3 Vida
				\4 1834-1910
			\3 Influenciado por
				\4 Cournot
				\4 Auguste Walras
			\3 Influenció a
				\4 Escuela de Lausana
				\4 Toda microeconomía posterior
				\4[] Kenneth Arrow
				\4[] Gerard Debreu
				\4[] Maurice Allais
				\4 Macroeconomía
				\4 Economía matemática
				\4 Austriacos
				\4 Pareto
				\4 Wicksell
				\4 ...
			\3 Obras
				\4 Elementos de Economía Política Pura (1871)
			\3 Metodología
				\4 Formación matemática
				\4 Consolida uso de matemáticas en economía
				\4 Formalización matemática de modelos
			\3 Equilibrio general
				\4 Modelo general de las interacciones económicas
				\4[] Interrelaciones e interacciones entre mercados
				\4 Existencia
				\4[] Si número de incógnitas (precio y cantidad)
				\4[] = número de ecuaciones independientes
				\4[] $\to$ Erróneo: no tiene porqué
				\4 Estabilidad
				\4[] Cómo se llega al equilibrio?
				\4[] Ajuste hacia equilibrio vía tâtonnement
				\4[] $\to$ Precio sube si exceso de demanda
				\4[] $\to$ Hasta alcanzar equilibrio
				\4[] $\then$ Crieurs publicitan precios de intercambio
				\4[] $\then$ Traducido como auctioneer
				\4 Ley de Walras
				\4[] Generalización de la restricción presupuestaria
				\4[] Suma de valor de excesos de dda. = 0
				\4 Sub-modelos
				\4[] Producción
				\4[] Maximización de utilidad
				\4[] Producción
				\4[] $\to$ No tiene en cuenta bienes duraderos
				\4[] Crédito
				\4[] Aspectos monetarios
			\3 Teoría de la distribución
				\4 Contribución ya en Elementos de Economía Política  (1875)
				\4 Precede análisis de Wicksell y Wicksteed
				\4 En equilibrio competitivo con libre entrada
				\4[] Empresas producen en escala eficiente
				\4[] $\to$ Minimizan costes medios
				\4[] $\to$ Entran nuevas empresas con costes medios mínimos
				\4[] $\then$ Rendimientos constantes a escala
				\4 Agotamiento del producto con R=E
				\4[] Suma de productos marginales por cantidad de factor
				\4[] $\to$ Igualan output total
				\4[] $\then$ Agotamiento del producto
		\2 Menger, Carl
			\3 Vida
				\4 1840-1921
			\3 Obras
				\4 Principios de Economía (1871)
				\4 ``\textit{Grundsätze der Volkswirtschaftslehre}''
			\3 Teoría del valor
				\4 Rechazo categórico de teoría del valor-trabajo
				\4[] Base del valor de intercambio
				\4 Teoría subjetiva del valor
				\4[] Valor depende de satisfacción subjetiva de deseos
				\4[] Aplicable a:
				\4[] $\to$ Bienes consumidos directamente
				\4[] $\to$ Bienes intermedios por imputación
				\4 Principio equimarginal
				\4[] Agentes igualan beneficio marginal de bienes
				\4[] Ilustra con ejemplo numérico
				\4[] $\to$ Tabla de Menger
				\4 No asume cardinalidad
				\4[] Sólo utiliza comparaciones numéricas para ilustrar
				\4 Teoría de la imputación
				\4[] Utilidad indirecta $\to$ Valor de bienes intermedios
				\4[] Valor de todos bienes depende $\leftarrow$ que generarán
				\4[] $\Rightarrow$ Análisis de oferta a partir de utilidad
				\4[] Totalmente contrario a clásicos
				\4[] $\to$ Trabajo incorporado a bienes
				\4[] $\to$ Valor depende de trabajo directo e incorporado
				\4[] $\then$ Orden inverso de principio del valor
				\4 Evita implicaciones benthamitas/utilitaristas
				\4[] Mayor utilidad para mayor número de personas
				\4[] $\to$ Imposible técnicamente
			\3 Teoría de la distribución
				\4 Teorema del Agotamiento del Producto
				\4[] Cada factor recibe su contribución a la producción
				\4[] $\to$ Remuneración de factores agota producto
				\4 Evita formalizar matemáticamente
			\3 Methodenstreit
				\4 Preocupación por cuestiones metodológicas
				\4 Disputa con historicistas alemanes
				\4 Economía es ciencia libre de juicios de valor
				\4 Individualismo metodológico
				\4[] Economía reducible a comportamientos individuales
				\4[] Rechazo de historicismo, marxismo
				\4 Deducción + contrastación empírica
				\4[] Frente a historicisimo:
				\4[] $\to$ Inferencia de leyes a partir de historia
		\2 Escuela austriaca
			\3 Idea clave
				\4 Seguidores de Menger
				\4 Metodológicamente opuestos a historicismo
				\4 Economía como sistema dinámico
				\4[] $\to$ Proceso constante de transformación
				\4[] $\to$ Erróneo estudiar en términos estáticos
				\4[] $\to$ Herramientas ciencias naturales no son apropiadas
			\3 Böhm-Bawerk
				\4 Vida
				\4[] 1851-1914
				\4[] Énfasis sobre el papel del tiempo sobre:
				\4[] $\to$ Precio
				\4[] $\to$ Valor
				\4[] $\to$ Ingresos
				\4 Capital es tiempo
				\4[] Para mantenerse hasta que inversión da fruto
				\4[] $\to$ Necesario capital
				\4[] Cuanto más tiempo sea necesario para producir
				\4[] $\to$ Más capital necesario
				\4[] Capital equivale en definitiva a tiempo necesario
				\4[] Cuanto menos tiempo sea necesario
				\4[] $\to$ Más avanzado será un proceso productivo
				\4[] Cuanto menos capital requiera
				\4[] $\to$ Más avanzado un proceso productivo
				\4 Periodo medio de producción
				\4[] Concepto similar a duración
				\4[] ¿Cuánto tarda en media una inversión
				\4[] ...en rendir sus frutos?
				\4[] $\to$ Ponderación de output de inversión
				\4[] $\to$ En función de tiempo
				\4[] Inversiones con menor periodo medio de produccion
				\4[] $\then$ Superiores técnicamente
				\4 Interés es remuneración por factor de producción
				\4[] Factor de producción es el tiempo en el capital
				\4[] Interés depende de tres factores:
				\4[] $\to$ i. Impaciencia / preferencia por el presente
				\4[] $\then$ UMg de bien futuro menor a presente
				\4[] $\to$ ii. Expectativa de mayor riqueza en futuro
				\4[] $\then$ UMg de renta cae con el tiempo
				\4[] $\to$ iii. Abstinencia: factor de producción
				\4[] $\then$ Bienes presentes son técnicamente superiores a futuros
				\4[] Por i y ii, interés se exige por agentes
				\4[] Gracias a iii,
				\4[] $\to$ Hay producto con que pagar interés
				\4[] La abstinencia es un factor de producción
				\4[] Abstinencia permite aumentar producto
				\4[] $\then$ $\exists$ hoy implica mejor técnica que $\exists$ mañana
				\4[] Carl Menger rechaza apoyar modelo de Bohm-Bawerk
				\4[] $\to$ Pero Von Mises y otros discípulos aceptan
				\4 Explicación de paradoja del valor
				\4[] Granjero tiene una serie de sacos de trigo
				\4[] 1er saco:
				\4[] $\to$ Fabrica pan para comer
				\4[] 2o saco:
				\4[] $\to$ Fabrica pan para fortalecerse
				\4[] 3er saco:
				\4[] $\to$ Alimenta a sus animales
				\4[] 4o saco
				\4[] $\to$ Hace whisky
				\4[] 5o saco:
				\4[] $\to$ Alimenta a sus palomas
				\4[] ¿Cuánto estará dispuesto a pagar por el quinto saco?
				\4[] $\to$ El valor que le genere alimentar a palomas
				\4[] Valor no depende de utilidad total sino marginal
				\4[] ¿Por qué agua más barata que diamantes?
				\4[] $\to$ Utilidad total mucho mayor que diamantes
				\4[] $\to$ Utilidad marginal muy pequeña
				\4[] $\then$ Diamantes tienen más valor
				\4 Análisis y crítica de Marx
				\4[] Señala contradicciones
				\4[] Critica teoría de explotación
				\4[] $\to$ Ignora tiempo y papel del capital
				\4[] $\to$ Problema de la transformación
				\4[] Papel del empresario en producción
				\4[] $\to$ Necesario remunerar
				\4[] Remuneración al capital necesaria
				\4[] $\to$ Por factores anteriores que dan lugar a interés
			\3 Von Wieser
				\4 Vida
				\4[] 1851-1926
				\4 Introduce término ``utilidad marginal''
				\4 Utilidad es concepto subjetivo
				\4[] $\to$ Mismo bien puede aportar diferente utilidad
				\4 Paradoja del valor
				\4[] Valor como concepto subjetivo permite resolver
				\4 Valor de bienes de capital
				\4[] Contribución al bien final que sí tiene valor subjetivo
				\4 Introduce ``coste de oportunidad''
				\4[] Valor obtenible si bien aplicado a uso alternativo
				\4[] Mejora teoría de la imputación
			\3 Von Mises
				\4 Praxeología
				\4[] $\to$ Análisis de decisión humana consciente
				\4 Sistema de precios
				\4[] Describe valor subjetivo
				\4[] Permite asignación eficiente de recursos
				\4 Cálculo de precios relativos en sociedades socialistas
				\4[] Tarea imposible en la práctica
				\4[] Bienes de capital no tienen precio
				\4[] $\to$ Porque son propiedad del estado
				\4[] $\then$ Imposible asignar eficientemente
			\3 Von Hayek
				\4 Premio Nobel en 1974
				\4 Interacción política y economía
				\4 Papel de información en precios
				\4 Defensa de libertad económica
				\4 Free-banking
				\4 Rechazo a métodos cuantitativos de ciencias físicas
				\4[] No utilizables en ciencias sociales
				\4[] $\to$ Pretence of Knowledge (1974)
			\3 Schumpeter
				\4 Relación compleja con escuela austriaca
				\4[] Alumno de Bohm-Bawerk y Von Wieser
				\4[] Desarrolla temas de autores
				\4 Rechazo de equilibrio walrasiano para crecimiento
				\4 Teoría del ciclo y del crecimiento
				\4[] Emprendedores son elemento central
				\4[] Interacción con sistema monetario y financiero
				\4[] $\to$ Elemento esencial
				\4[] $\to$ Permite acceder a recursos
			\3 Teoría austriaca del ciclo
				\4 Exceso de crédito es causa de ciclos
				\4 Política monetaria demasiado laxa
				\4[] Bancos centrales compran mucho papel
				\4[] Oferta monetaria aumenta
				\4[] Interés nominal por debajo de natural
				\4[] $\to$ Influencia Wickselliana
				\4 Inversión excesiva
				\4[] Por interés nominal inferior a natural
				\4[] Proyectos poco rentables se llevan a cabo
				\4[] Mala asignación de recursos
				\4 Inversores ajustan carteras
				\4[] Venden proyectos poco rentables
				\4[] Reacción en cadena
				\4[] Cae precio de activos
				\4[] Suben tipos de interés
				\4[] Demanda agregada se desploma
				\4[$\then$] Recesión
				\4 Ciclo comienza de nuevo
	\1 \marcar{Neoclásicos}
		\2 Alfred Marshall
			\3 Vida
				\4 1842-1924
			\3 Influenciado por
				\4 J. S. Mill
				\4 Bentham
				\4 Ricardo
				\4 Marginalistas
			\3 Influenció a
				\4 Keynes
				\4 Pigou
				\4 Toda microeconomía posterior
				\4 Organización industrial
				\4 Macroeconomía internacional
			\3 Obras
				\4 Teoría Pura del Comercio Internacional (1879)
				\4 Principios de Economía (1890-1920)
			\3 Diferencias con Walras
				\4 Comparación habitual en historiografía
				\4[] Realmente, poco fundamento
				\4[] No desarrollaron obra en oposición al otro
				\4[] Diferentes objetivos
				\4 Enfoque de equilibrio parcial
				\4[] Énfasis en mercado particular
				\4[] Se asume mercados no interaccionan
				\4 Ajuste hacia equilibrio en competencia perfecta
				\4[] Empresas prefieren variar cantidades
				\4[] Ajuste de precios dificil en comp. perfecta
				\4[] Si precio de oferta > precio de demanda
				\4[] $\to$ Aumentan inventarios $\Rightarrow$ Baja producción
				\4[] Si precio de demanda > precio de oferta
				\4[] $\to$ Aumenta producción
				\4[] Variaciones de producción $\Rightarrow$ Variacion precios
				\4 Mayor importancia de factores no económicos
				\4[] Walras abstrae al extremo
				\4[] Supuestos muy fuertes
				\4[] Marshall tienen en cuenta economía, sociedad, cultura
			\3 Metodología
				\4 Generaliza uso de análisis de eq. parcial
				\4[] Aplicación a mercados de bienes
				\4[] $\to$ No a mercados de factores
				\4[] Para mercados de factores prefiere eq. general
				\4 Opuesto a ataques a clásicos
				\4[] Bases de ciencia económica no deben atacarse
				\4[] $\to$ Evitar fragilizar estatus de la disciplina
				\4 Matemáticas
				\4[] Prefiere métodos verbales salvo ideas principales
				\4[] Uso moderado a pesar de formación matemática
				\4 Elasticidades
				\4[] Introduce análisis de elasticidad
				\4[] Medidas de variación
				\4[] $\to$ De unas variables sobre otras
			\3 Teoría de la demanda con utilidad cuasilineal
				\4 Análisis de equilibrio parcial paradigmático
				\4[] Un mercado es relevante
				\4[] Resto de mercados agrupados en un sólo bien compuesto
				\4 Dos bienes relevantes:
				\4[] $x$ -- mercado a examinar
				\4[] $m$ -- compuesto del resto de bienes
				\4 Función de utilidad
				\4[] Separable
				\4[] $U(x,y) = u(x) + m$
				\4 Restricción presupuestaria
				\4[] $px + m = w$
				\4[$\Rightarrow$] Derivación de demanda
				\4[$\Rightarrow$] Derivación de excedente del consumidor
			\3 Bienestar
				\4 Objetivo:
				\4[] Medir retorno por consumir un bien
				\4[] En términos de utilidad
				\4 Método:
				\4[] $\to$ Calcular excedente del consumidor
				\4[] Asumir utilidad constante del dinero
				\4[] Restar a cantidad total que aceptaría pagar
				\4[] la cantidad efectivamente pagada
				\4 Bienestar agregado a partir de excedente del cons.
				\4[] Examina varios escenarios
			\3 Periodos de mercado y valor ``normal''
				\4 Idea clave
				\4[] Análisis temporal basado en eqs. estáticos
				\4[] $\to$ Periodos de mercado
				\4[] Convergencia hacia valor y beneficio normales
				\4[] $\to$ Cuando restricciones desaparecen
				\4[] $\Rightarrow$ Largo plazo
				\4 Periodo presente o equilibrio temporal
				\4[] Costes hundidos $\to$ coste irrelevante
				\4[] Objetivo de empresa: vender producción
				\4[] Producción invariable
				\4[] Bienes perecederos
				\4[] $\to$ Demanda determina precio
				\4[] Bienes duraderos
				\4[] $\to$ Consideraciones dinámicas
				\4[] $\to$ Posible restringir oferta
				\4 Corto plazo
				\4[] Producción restringida pero variable
				\4[] Factores de producción parcialmente fijos
				\4 Largo plazo
				\4[] Todos los factores de producción son variables
				\4[] $\to$ Nuevas empresas pueden entrar
				\4[] Empresas minimizan costes
				\4[] Costes determinan precio de venta
				\4 Valor normal y valor de mercado
				\4[] Valor normal
				\4[] $\to$ Fuerzas de mercado despliegan todos efectos
				\4[] Valor de mercado
				\4[] $\to$ Valor en un momento determinado
				\4 Beneficio normal
				\4[] Remuneración a propietario si valor normal
				\4[] Beneficio normal =  coste de oportunidad
				\4[] $\to$ Beneficio mínimo para mantener actividad
			\3 Rentas y cuasi-rentas
				\4 Rentas son beneficio por escasez
				\4[] Remuneración por encima de beneficio normal
				\4[] Similar a idea de Ricardo
				\4 Cuasi-rentas
				\4[] Beneficio económico tendente a desaparición
				\4[] $\to$ Por encima de coste de oportunidad de ff.pp.
				\4[] Rentas temporales
				\4[] $\to$ Desaparecen cuando oferta puede aumentar
				\4[] $\to$ Competencia permite aumento de oferta
				\4[] Oferta restringida por periodo de mercado
			\3 Comercio internacional
				\4 Curvas de oferta recíproca
				\4[] Desarrollo de programa de Mill
				\4[] Aplicación al análisis de aranceles
				\4 Análisis arancelario
				\4[] ¿Cuando puede un país mejorar aplicando arancel?
				\4[] $\to$ Equilibrio en segmento inelástico de curva extranjera
				\4 Libre comercio
				\4[] En general a favor
				\4[] Incluso libre comercio unilateral
				\4[] Pero reconoce posibilidades de proteccionismo
				\4[] $\to$ Estructura del mercado determina beneficios
			\3 Economías de escala externas
				\4 Observación empírica
				\4[] En l/p, con aumento de demanda y CPerfecta
				\4[] $\to$ Costes tienden a disminuir
				\4 Economías de escala externas a la firma
				\4[] Coste medio de una empresa dada
				\4[] $\to$ Decreciente con producción de otras
				\4 Marshall introduce como explicación de CMe decreciente
				\4 Explicaciones de economías de escala externas
				\4[] Mercado de trabajo cualificado
				\4[] $\to$ Aparece por concentración de industria
				\4[] Incentivo a industrias subsidiarias
				\4[] $\to$ Abaratan coste de inputs
				\4[] Mejoras en trasporte e infraestructura
				\4[] $\to$ Atraidos por mayor actividad
				\4[] Especialización en formación de capital humano
				\4 Precede programas de investigación futuros
				\4[] Learning-by-Doing de Arrow y Lucas
				\4[] Crecimiento endógeno de Romer (1986)
				\4[] Análisis espacial del crecimiento de Krugman
				\4[] Desarrollo económico
				\4[] $\to$ Efectos recíprocos
				\4[] $\to$ Múltiples equilibrios
				\4[] Análisis input-output
				\4[] Eslabonamientos hacia delante y hacia atrás
		\2 John Bates Clark
			\3 Vida
				\4 1847-1938
				\4 Introductor de neoclasicismo en América
			\3 Influenciado por
				\4 Historicistas alemanes
				\4 Socialismo cristiano
				\4 Austriacos
			\3 Influenció a
				\4 Fisher
				\4 Economistas americanos
				\4 Organización industrial
			\3 Obras
				\4 La Distribución de la Riqueza (1899)
			\3 Metodología
				\4 Rechazo del individualismo metodológico
				\4 Sociedad determina comportamiento del individuo
				\4 Contexto de decisiones económicas es relevante
				\4[$\then$] Economía no es sólo optimización individual
				\4 Leyes de la sociedad dan lugar a valor
				\4[] No existiría riqueza
				\4[] $\to$ Sin leyes que garantizasen propiedad privada
			\3 Teoría de la distribución
				\4 Concepción clásica de la distribución de la renta
				\4[] A trabajo $\to$ Salario de subsistencia
				\4[] A tierra $\to$ Renta Por cantidad fija
				\4[] A empresario $\to$ Cantidad residual
				\4 Concepción marxista de la distribución
				\4[] A trabajo $\to$ Cantidad necesaria para reproducir
				\4[] A capital $\to$ Diferencia entre precio y salario
				\4[] $\then$ Plusvalía
				\4 Marginalismo
				\4[] Utilidad de últimas unidades consumidas
				\4[] $\to$ Determinan precio que consumidores aceptan pagar
				\4[] Utilidad marginal decreciente
				\4[] $\to$ Demanda inversa decreciente
				\4[] Cuando precio iguala utilidad marginal
				\4[] $\to$ Demanda de equilibrio
				\4[] Aplicación de marginalismo a teoría de producción
				\4[] $\to$
				\4 Ff.pp. remunerados a productividad marginal
				\4[] En la medida en que competencia funcione correctamente
				\4[] Sindicatos presionan para remuneración = PMg
				\4[] $\to$ Cuando empresas tienen poder monopsonio en trabajo
				\4[] En contexto de monopolio bilateral
				\4[] $\to$ Empresas y sindicatos son dañinos para sociedad
				\4 Agotamiento del producto con remuneración
				\4[] En la medida en que:
				\4[] $\to$ Todos los factores se remuneren a PMg
				\4[] $\to$ Rendimientos constantes a escala
			\3 Capitalismo vs socialismo
				\4 Inicialmente, cercano a socialismo cristiano
				\4[] Justicia social como medida de deseabilidad
				\4 Rechazo posterior del socialismo redistributivo
				\4[] Evita se remunere a cada uno según contribución
				\4[] Destruye incentivos
				\4 Capitalismo como mejor forma de socialismo
				\4[] Verdadero socialismo es remunerar según esfuerzo
				\4[] Capitalismo es mejor aproximación a socialismo verdadero
				\4[] $\to$ Porque tiende a remunerar según esfuerzo
			\3 Distribución del producto
				\4 Todo ingreso puede reducirse a trabajo
				\4[] Beneficio $\to$ trabajo del empresario
				\4[] Retorno del capital $\to$ Ahorro derivado de trabajo
				\4[] Renta $\to$ ingreso espurio derivado de escasez
				\4 Distribución no depende de clases sociales
				\4[] Rechazo del marxismo
				\4 Remuneración es productividad marginal
			\3 Organización industrial
				\4 Estudio de estructura de los mercados factoriales
				\4[] $\to$ Para entender distribución del producto
				\4 Retornos de K y L
				\4[] Tienden a igualarse
				\4[] $\to$ Entre diferentes mercados
				\4[] Competencia y movilidad de factores garantizan
				\4 Defensa de la competencia
				\4[] ``Si nada impide competencia...
				\4[] $\to$ ...progreso continuará eternamente''
		\2 Irving Fisher
			\3 Vida
				\4 1867-1947
			\3 Influenciado por
				\4 Jevons
				\4 Termodinámica
				\4 Bates Clark
			\3 Influenció a
				\4 Econometría
				\4 Números índice
				\4 Austriacos
				\4[] Especialmente Schumpeter
				\4 Macroeconomía posterior
				\4 Keynes
				\4[] Interés y eficiencia marginal del capital
				\4 Monetarismo
				\4 Demanda de consumo
				\4 Samuelson
			\3 Obras
				\4 La Naturaleza del Capital y el Ingreso (1906)
				\4 \underline{La Teoría del Interés (1907)}
				\4 El Poder de Compra del Dinero (1911)
				\4 La Teoría del Interés (1930)
			\3 Interés y capital
				\4 Tiempo relevante para agentes económicos
				\4 Esperar es costoso
				\4[] Individuos exigen compensación por ahorro
				\4[] Consumo futuro planeado mayor que presente
				\4[] Debido a impaciencia
				\4[] $\to$ No a hipotético factor de 'abstinencia'
				\4[] $\to$ Tiempo no es factor de producción
				\4[] $\then$ Opuesto a Böhm-Bawerk
				\4 Modelo de dos periodos
				\4 Interés es precio del tiempo
				\4 Basado en:
				\4[] $\to$ Impaciencia
				\4[] $\to$ Oportunidad
				\4 Oportunidad:
				\4[] Retorno sobre coste de oportunidad
				\4[] Retorno y coste son secuencias de flujos de caja
				\4 Descuento de flujos de caja
				\4[] Introduce concepto
			\3 Teoría de la inversión
				\4 La Teoría del Interés (1930)
				\4 ¿Cómo distribuir intertemporalmente dotación?
				\4[] $\to$ Para maximizar utilidad
				\4 Modelo dinámico formal
				\4[] Dos periodos
				\4[] Términos matemáticos
				\4 Precursor de modelos dinámicos formales
				\4 Dotación exógena
				\4[] Cantidad $w_1$ en el periodo 1
				\4[] Cantidad 0 en e periodo 2
				\4 Inversión
				\4[] Cantidad de $w_1$ no consumida en 1
				\4[] $\to$ Destinada a capital
				\4 Rentas en periodo 2
				\4[] Inversión aplicado a f. de prod
				\4[] Ahorro a tipo de interés $r$
				\4 Problema de optimización
				\4[] Maximizar utilidad
				\4[] $\to$ Eligiendo consumo en periodo 1 y 2
				\4 Realmente, dos problemas separados
				\4[I] Problema de optimización de inversión
				\4[II] Problema de optimización de consumo
				\4 Problema de optimización de inversión
				\4[] Maximizar valor presente de flujos netos
				\4[] $\underset{k}{\max} \quad w-k + \frac{f(k)}{1+r} \quad \text{s.a:} k \leq w$
				\4[] $\text{CPO:} \quad f'(k) = 1+r$
				\4[] $\then$ Invertir dotación hasta que $f'(k) = 1+r$
				\4[] $\then$ Invertir hasta eficiencia marginal de K sea interés
				\4[] Representación gráfica
				\4[] \grafica{fisherproblemainversion}
				\4 Problema de optimización del consumo
				\4[] Maximizar utilidad dada restricción intertemporal
				\4[] $\underset{c_1, c_2}{\max} \quad u(c_1) + v(c_2)$
				\4[] $\text{s.a:} \quad c_1 + \frac{c_2}{1+r} = w -k^* + \frac{f(k^*)}{1+r}$
				\4[] Representación gráfica
				\4[] \grafica{fisherproblemaconsumidor}
				\4 Optimización intertemporal de empresa y consumidor
				\4[$\then$] Teorema de la separación de Fisher
				\4[] Objetivo de la empresa es maximizar valor presente
				\4[] $\to$ Independientemente de preferencias de accionistas
				\4[] Si mercados de capital perfectos:
				\4[] $\to$ Financiación independiente de inversión
				\4[] Aplicación microeconómica
				\4[] $\to$ Punto en FPP independiente de dda. de consumo óptimo
				\4[] Precursor de Modigliani-Miller
			\3 Teoría monetaria
				\4 Dinero como mecanismo transmisor
				\4[] De los cambios en la realidad económica
				\4 Teoría cuantitativa del dinero
				\4[] Variaciones de la oferta monetaria
				\4[] $\to$ Causan variaciones proporcionales del nivel de precios
				\4 Formulación de la Teoría Cuantitativa del Dinero
				\4[] Original: $MV = PT$
				\4[] Matiza: $MV + M'V' = PT$
				\4[] $\to$ $M'$: saldos en depósitos
				\4[] $\to$ $V'$: velocidad del dinero en depósitos
				\4[] P como elemento pasivo de la ecuación
				\4[] Condición de equilibrio
				\4[] Énfasis sobre dinámica:
				\4[] $\to$ En general, economías están en desequilibrio
				\4[] $\to$ Necesario entender proceso de ajuste
				\4 Poder adquisitivo del dinero
				\4[] Necesario mantener
				\4[] Inflación cero
				\4[] Desarrolla teoría de números índice
				\4[] $\to$ Para cuantificar $\Delta$ del poder adquisitivo
				\4 Ecuación de Fisher/Efecto fisher
				\4[] Forma general: $i = r + \pi^e$
				\4[] En eq. estacionario $\pi = \pi^e$
				\4[] Utilizable como identidad
				\4[] $\to$ calcular tipo de interés real ex-post
				\4 Efecto Fisher
				\4[] A c/p, interés nominal apenas se ajusta
				\4[] $\to$ Ajuste de interés nominal lento e imperfecto
				\4[] $\then$ $\uparrow \pi^e$ $\to$ $\downarrow r$
				\4[] $\then$ Estímulo monetario $\downarrow$ r inicialmente
				\4[] A l/p, interés nominal se ajusta 1:1 a inflación
				\4[] $\to$  $\frac{d \, i}{d \, \pi} = 1$
				\4[] $\then$ Efecto Fisher\footnote{Barsky (1987): ``The Fisher hypothesis, which states that nominal interest rates rise point-for-point with expect inflation, leaving the real rate unaffected, is one of the cornerstones of neoclassical monetary theory.''}
				\4 Efecto de variaciones del tipo real
				\4[] Capaces de generar fluctuaciones
				\4[] Factor determinante de Gran Depresión
				\4[] $\to$ Precursor de monetarismo
		\2 Knut Wicksell
			\3 Vida
				\4 1851-1926
			\3 Influenciado por
				\4 Carl Menger
			\3 Influenció a
				\4 Escuela austriaca
				\4 Escuela de Estocolmo
				\4 Hicks
				\4 Análisis del desequilibrio
				\4 Análisis dinámico
			\3 Obras
				\4 Interés y precios (1898)
				\4 Valor, capital y renta (1894)
				\4 Lecciones de Economía Política (1901)
			\3 Síntesis del neoclasicismo
				\4 Sintetiza y cohesiona obras de otros neoclásicos
				\4 Integra teoría del capital austriaca y Fisher
				\4 Refina definiciones y resultados de otros autores
				\4 Expone puntos débiles de teorías
			\3 Interés
				\4 Rechaza abstinencia como factor de producción
				\4 De acuerdo con Fisher: basta impaciencia
				\4 Interés natural $r$
				\4[] $\to$ Productividad marginal del capital
				\4 Interés monetario $i$
				\4[] Tipo de interés nominal de préstamos bancarios
			\3 Proceso acumulativo\footnote{Ver \href{https://www.hetwebsite.net/het/essays/money/cumulative.htm}{HET Website}.}
				\4 Contribución más conocida de Wicksell
				\4 Compatibilizar dos puntos en conflicto:
				\4[] $\to$ Teoría cuantitativa del dinero
				\4[] $\to$ Ley de Say
				\4[] Fisher afirma teoría cuantitativa del dinero
				\4 Teoría cuantitativa del dinero
				\4[] Aumentos exógenos de la oferta monetaria
				\4[] $\to$ Resultan en aumentos proporcionales de nivel de precios
				\4 Ley de Say
				\4[] No existen excesos agregados de demanda de bienes
				\4[] $\to$ Oferta ``crea'' demanda en términos de Mill
				\4[] $\to$ Demanda racionada por oferta
				\4[] Independientemente de todo fenómeno monetario
				\4[$\then$] ¿Cómo tienen entonces lugar los $\uparrow$ P de la TCD?
				\4 Wicksell trata de explicar
				\4[] Como resultado de interés monetario menor que natural
				\4 Si $r>i$
				\4[] Posible pedir prestado a tipo $i$
				\4[] Invertir y obtener $r$
				\4[] $\to$ Extraer beneficio de inversión
				\4[] $\then$ Demanda de dinero aumenta
				\4 Pero inversión debe ser igual a ahorro
				\4[] En contexto de pleno empleo
				\4 Sin embargo, cuando $r>i$, $I > S$
				\4[] Aumento incompleto de oferta porque pleno empleo
				\4[] Presión sobre precios de bienes de capital
				\4[] $\to$ Repercutidos a bienes de consumo
				\4[] $\then$ Aumento de nivel de precios
				\4[$\then$] Explicación de TCD
				\4[$\then$] Dicotomía clásica no se cumple en corto plazo
				\4[$\then$] Dicotomía clásica se mantiene en corto plazo
				\4[$\then$] Inflación es fenómeno \textit{real}
				\4 Proceso acumulativo
				\4[] Si interés monetario por debajo de natural
				\4[] $\to$ Proceso inflacionario acumulativo
				\4 ¿Cuando acaba el proceso acumulativo?
				\4[] Bancos no tienen reservas para seguir prestando
				\4[] Dos opciones:
				\4[] $\to$ Dejan de prestar
				\4[] $\then$ Se reduce presión $I>S$
				\4[] $\to$ Piden prestados fondos en mercado monetario
				\4[] $\then$ Aumenta interés monetario
				\4 Origen de conceptos 'naturales' posteriores
				\4[] Tasa de desempleo natural
				\4[] Output tendencial
				\4[] Output gap
			\3 Efectos de Wicksell
				\4 Efecto de precio
				\4 Efecto real
	\1 \marcar{Otros nombres}
		\2 Philip Henry Wicksteed
			\3 Vida
				\4 1844-1927
			\3 Distribución del producto
				\4 Examen sistemático del problema
				\4 Formaliza agotamiento del producto de Menger
				\4[] Introduce idea en Inglaterra
				\4 Supuestos:
				\4[] Función de producción homogénea de grado 1
				\4[] Factores remunerados a producto marginal
				\4[$\Rightarrow$] Remuneración de factores agota producto
			\3 Egoísmo frente a no-tuismo
				\4 Niega que egoísmo sea principio necesario de economía
				\4 No-tuísmo sí es necesario
				\4[] Que A no esté interesado en las preferencias de B
				\4[] No implica que no le importe la felicidad de otros
				\4[] $\to$ Sólo necesario que no le importe felicidad de B
		\2 Francis Ysidro Edgeworth
			\3 Vida
				\4 1845-1926
			\3 Obras
				\4 Psicología Matemática (1881)
			\3 Economía matemática
				\4 Verdadero fundador de la disciplina
				\4 Primer uso de matemáticas avanzadas en economía
				\4 Influenciado por Hamilton
				\4 Uso esencialmente cualitativo de la matemática
				\4[] $\to$ Uso de funciones sin definir explícitamente
				\4 Introduce concepto de óptimo de Pareto
				\4[] $\to$ Pareto desarrollaría y daría nombre
			\3 Ajuste hacia equilibrio
				\4 Diferente a Marshall y tâtonnement
				\4 Basado en negociaciones sucesivas entre agentes
				\4[] Sin coordinador central como el subastador Walrasiano
				\4 Precursor de:
				\4[] enfoque del núcleo en eq. general
				\4[] teoría de juegos
		\2 Arthur Cecil Pigou
			\3 Vida
				\4 1877-1959
				\4 Sucesor de Marshall en Cambridge
			\3 Economía del bienestar
				\4 Pionero
				\4 Formaliza bases sentadas por Marshall
				\4 Distinción entre costes sociales y privados
				\4 intervención estatal puede mejorar bienestar
			\3 Equilibrio competitivo de largo plazo
				\4 Naturaleza $\to$ Rendimientos decrecientes
				\4 Ingenio humano $\to$ Rendimientos crecientes
				\4[$\Rightarrow$] Rendimientos constantes
				\4 Idea de costes de largo plazo en forma de U
				\4[] Equilibrio en escala eficiente
			\3 Externalidades
				\4 Formaliza las externalidades negativas
				\4 Propone impuestos correctores
			\3 Keynes
				\4 Defensor de modelo neoclásico frente a Keynes
				\4[] $\to$ Existen fuerzas autoestabilizadoras
		\2 Vilfredo Pareto
			\3 Vida
				\4 1848-1923
				\4 Sucesor de Walras en Lausana
			\3 Utilidad ordinal
				\4 Demuestra que cardinalidad no es necesaria
			\3 Optimización
				\4 Trabaja ideas de Edgeworth sobre optimización
				\4 Introduce caja de Edgeworth
			\3 Producción
				\4 Crítica a Wicksteed
				\4 Análisis de leyes de producción
			\3 Sociología
				\4 Contribuciones duraderas
				\4 Distribución de Pareto
				\4 Teoría de las élites
	\1[] \marcar{Conclusión}
		\2 Recapitulación
			\3 Marginalistas
				\4 Predecesores
				\4 Jevons
				\4 Menger
				\4 Walras
			\3 Neoclásicos
				\4 Alfred Marshall
				\4 John Bates Clark
				\4 Knut Wicksell
				\4 Irving Fisher
			\3 Otros autores
				\4 Philip Henry Wicksteed
				\4 Edgeworth
				\4 Pigou
				\4 Pareto
		\2 Idea final
			\3 Programas de investigación sin continuidad
				\4 Como es habitual en todas las escuelas
				\4[] Algunos programas no tienen continuidad
			\3 Contribuciones duraderas y germen de programas
				\4 Toda la economía actual es heredera de neoclásicos
				\4[] Modelización basada en lógica simbólica
				\4[] Importancia del margen
				\4[] Importancia de la demanda
				\4[] Simplificación de factores de producción
				\4[] $\to$ Permite análisis macroeconómico
				\4 Desarrollan temas principales de ec. clásica
				\4[] De forma más sistemática y organizada
				\4[] Construyen bases de economía moderna
				\4[] Hacen posible discusión en términos modernos
\end{esquemal}































\graficas

\begin{axis}{4}{Representación gráfica del problema de la inversión en el modelo de inversión de dos periodos de Fisher.}{$Y_1$}{$Y_2$}{fisherproblemainversion}
	% Dotación inicial
	\node[below] at (3,0){$E_1$};
	
	% FPP
	\draw[-] (3,0) to [out=95, in=-5](0,3);
	
	% Recta presupuestaria
	\draw[-] (0.5,4) -- (3.8,0);
	
	% Inversión y producción de óptimo
	\draw[dashed] (0,1.9) -- (2.2,1.9) -- (2.2,0);
	\node[left] at (0,1.9){$Y_2^*$};
	\node[below] at (2.2,0){$Y_1^*$};
\end{axis}


\begin{axis}{4}{Representación gráfica del problema de la optimización del consumo en el modelo de inversión de dos periodos de Fisher.}{x}{y}{fisherproblemaconsumidor}
	% Dotación inicial
	\node[below] at (3,0){$E_1$};
	
	% FPP
	\draw[-] (3,0) to [out=95, in=-5](0,3);
	
	% Recta presupuestaria
	\draw[-] (0.5,4) -- (3.8,0);
	
	% Inversión y producción de óptimo
	\draw[dashed] (0,1.9) -- (2.2,1.9) -- (2.2,0);
	\node[left] at (0,1.9){$Y_2^*$};
	\node[below] at (2.2,0){$Y_1^*$};
	
	% CI 1
	\draw[-] (0.5, 4.52) to [out=275, in=175](3.2,2.52);
	
	% CI 2
	\draw[-] (2.5,2.1) to [out=275, in=175](5.2,0.1);
\end{axis}

El gráfico muestra como la decisión de consumo óptimo es independiente de la decisión de inversión óptima. La inversión óptima maximiza el valor presente del ingreso total (dotación y producción tras inversión). Posteriormente, la decisión de consumo optimiza el reparto de ese ingreso en función de las preferencias intertemporales del individuo.

\conceptos

\concepto{Distribución y agotamiento del producto}

Un problema al que los economistas clásicos trataron de dar respuesta es el el problema de la distribución del producto. Para producir una cantidad determinada de producto, la empresa requiere una cantidad de factores de producción dada. Esos factores se compran a cambio de una remuneración, que en último término se extrae del producto. ¿Cómo se reparte el producto entre los factores de producción aplicados? Ricardo respondió a la pregunta considerando el beneficio del empresario como el residuo del producto una vez deducidos los salarios y la renta de la tierra. Los salarios reciben una remuneración acorde con el salario de subsistencia y la tierra recibe la renta determinada por el diferencial entre la productividad de la tierra en cuestión y la tierra libre de renta.

En el contexto neoclásico, el énfasis sobre el margen y la utilidad dio al traste con la teoría clásica de la distribución. Menger aplicó el concepto de la utilidad marginal a la producción, y postuló que los factores de producción son útiles en la medida en que con ellos se pueden obtener otros bienes de los que derivar utilidad de forma directa. Este razonamiento implica la posibilidad de valorar la contribución de los factores de producción en términos marginales. Sin embargo, Menger consideraba que todas las unidades de bienes son ``igualmente marginales'', y que por ello la utilidad que proporcionaban debía valorarse multiplicando la utilidad marginal por el número de unidades consumidas. De forma análoga, introdujo la idea de remunerar los factores de producción en relación a su producto marginal. Sin embargo, esta idea requiere de supuestos adicionales para no caer en inconsistencias lógicas. Si los rendimientos marginales de un factor de producción son decrecientes y el factor es remunerado con su producto marginal, existirá un residuo que no corresponderá a ningún factor de producción. Si los rendimientos fuesen crecientes, sucedería al contrario y la remuneración de los factores superaría al producto. La cuestión de la distribución del producto se convierte así en una pregunta: ¿qué condiciones son necesarias para que la remuneración de factores agote el producto?

Wicksteed abordó el problema en términos matemáticos y llegó a la conclusión de que la remuneración agotaría el producto si la función de producción tuviese rendimientos constantes a escala. O equivalentemente, si fuese homogénea de grado uno. A pesar de lo restrictivo de esta condición, Wicksteed estaba plenamente convencido de su aplicabilidad práctica. Aunque incuestionable desde el punto de vista formal, desde el punto de vista empírico la solución dejaba ciertamente que desear. Pareto lo atacó afirmando que es un hecho empírico que numerosos procesos de producción muestran rendimientos decrecientes o crecientes a escala, y que es imposible cuantificar la productividad marginal de factores en cantidades fijas. Además, la solución de Wicksteed se basa en un supuesto que no tienen porqué cumplirse y es que efectivamente las fuerzas del mercado remuneran de acuerdo con la productividad marginal.



\concepto{Ley de las proporciones variables}

Enunciada por primera vez en términos generales por Thünen, la ley de las proporciones variables postula que, manteniendo fija la cantidad del resto de los factores, la productividad marginal de un factor terminará por decrecer a medida que se aumenta su cantidad.

\concepto{Leyes de Gossen}

Las leyes de Gossen son tres postulados relativos a la utilidad que los agentes derivan del consumo de un bien. 
\begin{itemize} 
\item La \textit{Primera Ley de Gossen} establece que que la utilidad que se deriva del consumo de un bien decrece a medida que se incrementa la cantidad consumida. 
\item La \textit{Segunda Ley de Gossen} establece que para alcanzar el óptimo, la utilidad marginal que se deriva de cada bien consumido debe igualarse. 
\item La \textit{Tercera Ley de Gossen} establece como requisito necesario para que un bien tenga valor que la demanda sea mayor que la oferta. A pesar de ser menos conocida, la Tercera Ley es indicativa del avanzado tratamiento que Gossen fue capaz de dar al concepto de renta y cómo consiguió relacionarlo con el concepto de utilidad.
\end{itemize}

\preguntas


\seccion{Test 2019}

\textbf{2.} Indique cuál de las siguientes contribuciones teóricas de Irving Fisher (1867-1947) \underline{\textbf{no}} fue resultado de su obra:

\begin{itemize}
	\item[a] La distinción entre el tipo de interés nominal y el tipo de interés real.
	\item[b] Su interpretación del tipo de interés, que fue precursora de la teoría monetaria del interés basada en la Preferencia por la liquidez.
	\item[c] El concepto de ``tasa de rendimiento sobre el coste'', en tanto que posteriormente pasaría a denominarse ``tasa interna de rendimientos de una inversión''.
	\item[d] El concepto de ``tasa de rendimientos sobre el coste'', en tanto que fue precursor del concepto de la eficiencia marginal del capital de Keynes.
\end{itemize}

\seccion{Test 2017}
\textbf{1.} Indique cuál de las siguientes afirmaciones es \underline{\textbf{INCORRECTA}} en relación con la \textit{Teoría de la utilidad} de W. S. Jevons (1835-82):

\begin{itemize}
	\item[a] No distinguió explícitamente entre mediciones cardinales u ordinales de la utilidad, pero fue más bien precursor del método ordinal.
	\item[b] Se niega a hacer comparaciones de utilidad interpersonales, algo que considera imposible.
	\item[c] Su utilitarismo entendido como maximización de la utilidad era absoluto, supeditando siempre el cálculo del bien o del mal moral al cálculo de la utilidad.
	\item[d] Distinguió entre utilidad total y utilidad marginal, utilizando ecuaciones y diagramas que representaban curvas de utilidad marginal decreciente.
\end{itemize}

\seccion{Test 2016}
\textbf{1.} En la teoría de Marshall una curva de oferta decreciente
\begin{itemize}
	\item[a] es imposible.
	\item[b] es posible y tiene como efecto una industria no competitiva.
	\item[c] es posible si el coste marginal social no coincide con el coste marginal privado.
	\item[d] es posible sólo en sectores regulados por el estado.
\end{itemize}

\textbf{2.} La teoría del interés de Wicksell
\begin{itemize}
	\item[a] Influyó de forma importante en la teoría monetaria de Keynes, pero no en los economistas de la escuela austríaca.
	\item[b] Influyó de forma importante en la teoría monetaria de Hayek, pero no en la teoría monetaria de los economistas de Cambridge.
	\item[c] Influyó de forma importante en las teorías de Hayek y de Keynes.
	\item[d] No tuvo relevancia a partir de la década de 1930.
\end{itemize}

\seccion{Test 2014}
\textbf{2.} La explicación de los beneficios empresariales como una renta temporal resultante de los cambios dinámicos de la eoconomía se encuentra en la obra de:
\begin{itemize}
	\item[a] J. B. Clark
	\item[b] A. Marshall
	\item[c] J. A. Schumpeter
	\item[d] Todas las respuestas son correctas.
\end{itemize}

\textbf{19.} El efecto Fisher muestra que el cambio en la inflación que tiene su origen en un cambio en el crecimiento de la cantidad de dinero se traduce en:
\begin{itemize}
	\item[a] Una variación del tipo de interés.
	\item[b] Una variación del tipo de interés real en igual proporción.
	\item[c] Una variación del tipo de interés nominal en mayor proporción.
	\item[d] Una variación del tipo de interés real en menor proporción.
\end{itemize}

\seccion{Test 2011}
\textbf{1.} ¿Cuál de estas proposiciones es acorde a la teoría de la preferencia temporal positiva del economista austriaco Eugen Böhm-Bawerk?
\begin{itemize}
	\item[a] los bienes presentes son técnicamente superiores a los bienes futuros, por lo que tienen menor valor.
	\item[b] los bienes presentes son técnicamente superiores a los bienes futuros, por lo que tienen mayor valor.
	\item[c] los bienes futuros son técnicamente superiores a los bienes presentes, por lo que tienen mayor valor.
	\item[d] los agentes subvaloran las necesidades presentes, por lo que los bienes futuros tienen mayor valor.
\end{itemize}

\seccion{Test 2008}
\textbf{1.} Es falso que los economistas marginalistas:
\begin{itemize}
	\item[a] Postularon la ley de la utilidad marginal decreciente.
	\item[b] Definieron utilidad marginal como el incremento de utilidad derivado del consumo de una unidad adicional del mismo.
	\item[c] Plantearon la teoría de la preferencia revelada como justificación analítica a la ley de la utilidad marginal decreciente.
	\item[d] Consideraban la existencia de una medida de satisfacción, la utilidad.
\end{itemize}

\seccion{Test 2005}
\textbf{2.} El modelo clásico determina un punto de equilibrio donde se cruzan la demanda y oferta agregada en el que:
\begin{itemize}
	\item[a] Existe alto nivel de empleo y elevada inflación, pero puede haber desequilibrios en balanza de pagos.
	\item[b] Existe pleno empleo.
	\item[c] La cantidad de dinero es constante y precios y salarios muestran rigidez a la baja.
	\item[d] Se produce inestabilidad porque la curva de demanda agregada es vertical para el nivel de producción de pleno empleo.
\end{itemize}

\notas

\textbf{2019:} \textbf{2.} B

\textbf{2017:} \textbf{1.} C

\textbf{2016:} \textbf{1.} B \textbf{2.} C

\textbf{2014:} \textbf{2.} D \textbf{19.} A

\textbf{2011:} \textbf{1.} B

\textbf{2008:} \textbf{1.} C

\textbf{2005:} \textbf{2.} B

\bibliografia

Mirar en Palgrave:
\begin{itemize}
	\item austrian economics
	\item ceteris paribus
	\item Böhm-Bawerk, Eugen von
	\item Clark, John Bates
	\item consumer surplus
	\item demand price
	\item external economies
	\item Fisher, Irving *
	\item Gossen, Hermann Heinrich
	\item Jenkin, Henry Charles Fleeming
	\item Jevons, William Stanley
	\item law of demand
	\item marginal productivity theory
	\item marginal revolution
	\item marginal utility of money
	\item Marshall, Alfred *
	\item market period
	\item markets
	\item marketplaces
	\item Menger, Carl
	\item Methodenstreit
	\item natural rate and market rate of interest
	\item 'neoclassical'
	\item offer curve or reciprocal demand curve
	\item Pigou, Arthur Cecil
	\item profit and profit theory
	\item rent
	\item Robinson Crusoe
	\item Schumpeter, Joseph Alois
	\item supply and demand
	\item Walras, Léon *
	\item Walras' Law
	\item Wicksell, Johan Gustaf Knut *
	\item Wicksell effects *
	\item Wicksteed, Philip Henry *
	\item Wiesner, Friedrich Freiherr, (Baron) von
\end{itemize}

Argandoña, A; \textit{Irving Fisher: un gran economista} (2013) IESE Working Paper -- En carpeta del tema

Blaug, M. \textit{Economic Theory in Retrospect} (1996) 5th edition - En carpeta \textit{Historia del Pensamiento Económico}

Economist, the. \textit{Schumpeter Centenary}. Nov 19th 1983 (En carpeta del tema)

Robbins, L. \textit{A History of Economic Thought. The LSE Lectures} (1998) -- En carpeta \textit{Historia del Pensamiento Económico}

Samuels, W. J; Biddle, J. E.; Davis, J. B. \textit{A Companion to the History of Economic Thought} (2003)

Screpanti, E; Zamagni, S. \textit{An Outline of the History of Economic Thought} (2005) -- En carpeta \textit{Historia del Pensamiento Económico}

Weintraub, E. R. \textit{Neoclassical Economics} The Concise Encyclopedia of Economics -- \url{http://www.econlib.org/library/Enc1/NeoclassicalEconomics.html}

\end{document}
