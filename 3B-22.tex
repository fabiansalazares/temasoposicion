\documentclass{nuevotema}

\tema{3B-22}
\titulo{El Sistema Económico Internacional desde la desaparición del sistema de Bretton-Woods hasta la actualidad. Propuestas de reforma.}

\begin{document}

\ideaclave

Ver \href{https://www.nber.org/papers/w21813.pdf}{Frankel (2015) sobre Acuerdos del Plaza}

Hablar del ``fear of floating''. Ver Calvo y Reinhart (2000).

PARA Crisis de 2008: \url{https://www.econlib.org/cee/2008FinancialCrisis/}

PARA Propuestas de reforma
https://voxeu.org/article/imf-75-reforming-global-reserve-system

PARA Propuestas de reforma (zona euro)
https://voxeu.org/article/euro-area-architecture-what-reforms-are-still-needed-and-why

PARA Propuestas de reforma: sistema monetario internacional
\url{https://voxeu.org/article/monetary-policy-world-cryptocurrencies} Las criptomonedas como competencia al dinero de bancos centrales. ¿Es posible que las criptomonedas acaben sustituyendo al dinero de bancos centrales y con ello desaparezca el control monopolístico del gobierno en la provisión de liquidez?

PARA Propuestas de reforma: cuotas en el FMI y reforma del sistema, bloqueo americano en 2019, NABs que expiran en 2020s  https://piie.com/system/files/documents/pb18-9.pdf 

\seccion{Preguntas clave}

\begin{itemize}
	\item ¿Qué es el sistema económico internacional?
	\item ¿Cómo evolucionó tras la desaparición del sistema de Bretton Woods?
	\item ¿Qué eventos principales tuvieron lugar?
	\item ¿En qué contexto se produjeron los grandes cambios?
	\item ¿Qué consecuencias se derivaron?
	\item ¿En qué situación se encuentra en la actualidad?
	\item ¿Qué propuestas de reforma se plantean?
\end{itemize}

\esquemacorto

\begin{esquema}[enumerate]
	\1[] \marcar{Introducción}
		\2 Contextualización
			\3 Sistema económico y monetario internacional
			\3 Mercados internacionales de capital
			\3 Tendencias de l/p tras caída de Bretton Woods
		\2 Objeto
			\3 ¿Qué es el sistema económico internacional?
			\3 ¿Cómo evolucionó tras la desaparición del sistema de Bretton Woods?
			\3 ¿Qué eventos principales tuvieron lugar?
			\3 ¿En qué contexto se produjeron los grandes cambios?
			\3 ¿Qué consecuencias se derivaron?
			\3 ¿En qué situación se encuentra en la actualidad?
			\3 ¿Qué propuestas de reforma se plantean?
		\2 Estructura
			\3 Caída de Bretton Woods
			\3 Gran Moderación
			\3 Gran Recesión
			\3 Actualidad
	\1 \marcar{Caída de Bretton Woods}
		\2 Contexto
			\3 Económico
			\3 Político
			\3 Teórico
		\2 Eventos
			\3 Desequilibrios previos
			\3 Ventas masivas de dólares en 1971
			\3 Nixon Shock (agosto 1971)
			\3 Acuerdos de Smithsonian en Washington (diciembre 1971)
			\3 Crisis del petróleo del 73
			\3 Acuerdos de Jamaica (1976)
		\2 Consecuencias
			\3 Política monetaria como instrumento de PEconómica
			\3 Regímenes cambiarios polares
			\3 Volatilidad de tipos de cambio
			\3 Impulso a integración europea
	\1 \marcar{Gran Moderación}
		\2 Contexto
			\3 Económico
			\3 Político
			\3 Teórico
		\2 Eventos
			\3 Inestabilidad inicial
			\3 Acuerdo de Bonn en 1978
			\3 Sistema Monetario Europeo en 1979
			\3 Nueva política monetaria en EEUU desde 1979
			\3 Acuerdos del Hotel Plaza (1985)
			\3 Acuerdos del Louvre (1987)
			\3 Crisis del ERM
			\3 Crisis asiática
			\3 Crisis en Rusia, Turquía y Brasil
			\3 Crisis argentina
			\3 Crisis de las puntocom
		\2 Consecuencias
			\3 Crecimiento y baja inflación
			\3 Miedo a flotar
			\3 Savings glut
			\3 Nuevos polos de crecimiento
	\1 \marcar{Gran Recesión}
		\2 Contexto
			\3 Económico
			\3 Político
			\3 Teórico
		\2 Eventos
			\3 Ascenso de China
			\3 Acumulación de desequilibrios
			\3 Crisis financiera en Estados Unidos
			\3 Recesión global
			\3 Crisis del euro
		\2 Consecuencias
			\3 Contracción global del output
			\3 Reformas
			\3 Inestabilidad política
			\3 Desequilibrios persisten
	\1 \marcar{Propuestas de reforma}
		\2 Contexto
			\3 Económico
			\3 Político
			\3 Teórico
		\2 Problemas del sistema global de reservas
			\3 Ajuste asimétrico de la cuenta corriente
			\3 Dilema de Triffin
			\3 Acumulación de reservas por emergentes
			\3 Propuesta I: múltiples divisas de reserva
			\3 Propuesta II: aumentar papel de DEGs
		\2 Gobernanza mundial
			\3 Trilema de Rodrik
			\3 Organización Mundial del Medioambiente
			\3 Integración monetaria y económica en África
		\2 Sistema multilateral de comercio
			\3 Situación actual
			\3 Propuestas de desbloqueo
		\2 Criptomonedas
			\3 Situación actual
			\3 Perspectivas futuras
			\3 Propuestas de solución
	\1[] \marcar{Conclusión}
		\2 Recapitulación
			\3 Caída de Bretton Woods
			\3 Gran Moderación
			\3 Gran Recesión
			\3 Propuestas de reforma
		\2 Idea final
			\3 Interacción economía y política
			\3 Cambio climático
			\3 Integración de emergentes

\end{esquema}

\esquemalargo















\begin{esquemal}
	\1[] \marcar{Introducción}
		\2 Contextualización
			\3 Sistema económico y monetario internacional
				\4 Sistema económico internacional
				\4[] Conjunto de:
				\4[] $\to$ Relaciones comerciales
				\4[] $\to$ Marco institucional y legal
				\4[] $\to$ Flujos financieros
				\4[] $\then$ Entre economías mundiales
				\4 Sistema monetario
				\4[] Componente elemental de sistema económico internacional
				\4[] Conjunto de instituciones y flujos financieros que permiten
				\4[] $\to$ Solucionar desequilibrios de balanza de pagos
				\4[] $\to$ Acceso a crédito
				\4[] $\to$ Canalizar pagos en divisas
				\4[] $\then$ Para aprovechar ganancias del comercio int.
				\4[] $\then$ Para suavizar patrón de consumo intertemporal
			\3 Mercados internacionales de capital
				\4 Pilar central de sistema monetario
				\4 Historia de sistema monetario y económico
				\4[] $\to$ Íntimamente ligado a hª mercados de capital
				\4 Diferentes regímenes de mercados de capital
				\4[] Ayudan a distinguir fases de evolución SEInternacional
				\4[] $\to$ Exposición dividida en fases
			\3 Tendencias de l/p tras caída de Bretton Woods
				\4 Avances tecnológicos
				\4[] Salud
				\4[] $\to$ Esperanza de vida
				\4[] Comunicaciones
				\4[] $\to$ Internet, móviles
				\4[] Inteligencia artificial
				\4[] Fuentes de energía
				\4 Globalización
				\4[] Tendencia en todo el mundo hacia:
				\4[] $\to$ Integración de mercados
				\4[] $\to$ Homogeneización de preferencias y reglas
				\4 Cambio demográfico
				\4[] Aumento esp. de vida + menor natalidad
				\4[] $\to$ Envejecimiento acelerado de
				\4 Liberalización comercial y financiera
				\4[] Crecimiento persistente de comercio > PIB
				\4[] $\to$ A pesar de freno reciente
				\4[] Crecimiento de flujos financieros
				\4[] Explosión de flujos de información
				\4 Multilateralismo
				\4[] Derecho internacional público basado en reglas
				\4[] $\to$ Aumentan certidumbre de intercambios
				\4[] $\to$ Menor recurso a uso de la fuerza
				\4[] Especialmente relevante en comercio
				\4[] $\to$ Culmina en creación OMC
				\4 Cambio climático
		\2 Objeto
			\3 ¿Qué es el sistema económico internacional?
			\3 ¿Cómo evolucionó tras la desaparición del sistema de Bretton Woods?
			\3 ¿Qué eventos principales tuvieron lugar?
			\3 ¿En qué contexto se produjeron los grandes cambios?
			\3 ¿Qué consecuencias se derivaron?
			\3 ¿En qué situación se encuentra en la actualidad?
			\3 ¿Qué propuestas de reforma se plantean?
		\2 Estructura
			\3 Caída de Bretton Woods
			\3 Gran Moderación
			\3 Gran Recesión
			\3 Actualidad
	\1 \marcar{Caída de Bretton Woods}
		\2 Contexto
			\3 Económico
				\4 Presión creciente sobre dólar
				\4[] Salidas de capital constantes
				\4[] Medidas de control insuficientes
				\4[] Mercados de capital cada vez más porosos
				\4[] Reservas insuficientes
				\4[] Precio de dólar en mercados privados supera precio Fed
				\4[] $\to$ Francia amenaza con liquidar dólares
				\4 Miedo a inflación en Alemania
				\4[] Rechazo a compra masiva de dólares
				\4 Libra esterlina bajo presión
				\4[] Salidas de capital constantes
				\4 Inflación en EEUU
				\4[] No muy alta en términos absolutos
				\4[] Sí demasiado alta para mantener competitividad
				\4 Europa y Japón se acercan a convergencia
			\3 Político
				\4 Guerra de Vietnam
				\4[] Expansión fiscal en Estados Unidos
				\4 Sindicatos en Reino Unido
				\4[] Presión a subida de salarios
				\4 Guerra fría
				\4[] EEUU, Europa, Japón socios frente a URSS
				\4[] Incentiva cooperación a pesar de desequilibrios
				\4 Guerra del Yom Kippur
				\4[] Tensión en Oriente Próximo
				\4[] Estados Unidos apoya Israel
				\4[] $\to$ OPEC introduce embargo en 1973
			\3 Teórico
				\4 Agotamiento de SNC
				\4 Monetarismo
				\4[] Dinero es importante
				\4[] $\to$ Tiene fuertes efectos sobre ciclo
				\4[] $\then$ PM pasiva a PF causa inflación
				\4 Críticas crecientes a PF activista
				\4 Robert Lucas
				\4[] Dinero afecta output
				\4[] Efectos se agotan si muy utilizados
		\2 Eventos
			\3 Desequilibrios previos
				\4 Déficits fiscales crecientes por Vietnam
				\4 Aumento de flujos de capital
				\4 Aumento del precio del oro en mercados privados
				\4 Inflación en Estados Unidos
			\3 Ventas masivas de dólares en 1971
				\4 Compras de marcos alemanes y florines holandeses
				\4[] Alemania y Holanda dejan apreciar su moneda en mayo
				\4[] $\to$ Otros países europeos también
				\4 Ventas de dólares no se frenan
				\4 Suiza redime dólares por oro
				\4 Rumores de liquidación dólares por oro
				\4[] De Francia e Inglaterra
			\3 Nixon Shock (agosto 1971)
				\4 Fin de semana del 13 de agosto de 1971
				\4[] $\to$ Nixon toma decisión final
				\4 Cierre de ventanilla de oro
				\4[] Suspensión de convertibilidad a $\$35$ la onza
				\4 Impuesto del $10\%$ a las importaciones
				\4 Presión a otros países para que revalúen
				\4[] Evitar pérdida de prestigio de revaluar
			\3 Acuerdos de Smithsonian en Washington (diciembre 1971)
				\4 Diciembre del 71
				\4 Devaluación del dólar respecto al oro
				\4[] Pero Sin reapertura de la ventanilla de oro
				\4[] Mejora ligera de competitividad americana
				\4[] $\to$ Aunque no compensa otras políticas
				\4[] $\to$ Política fiscal y monetaria expansiva pre-elecciones
				\4 Bandas de flotación del $\pm 2.25$ respecto al dólar
				\4[] $\to$ ``tunel'
				\4[] $\to$ Permite $4,5\%$ entre monedas\footnote{Si una moneda europea A se deprecia $2,25\%$ frente al dólar y otra moneda B se aprecia otro $2,25\%$, en la práctica hay un margen de flotación del $4.5\%$. Si la moneda A se aprecia un $4,5\%$ y la moneda B se deprecia otro $4,5\%$, resulta posible una variación del $9\%$ aun manteniéndose dentro del sistema. Considerando este margen de fluctuación como excesivo, los países de la CEE establecieron la ``serpiente en el túnel''.}
				\4 Serpiente en el túnel
				\4[] Margen de flotación entre monedas europeas
				\4[] $\to$ Inferior al margen del túnel
				\4[] $\then$ Margen bilateral del $2,25\%$ respecto al dólar.
				\4 Libra sale del acuerdo en junio del 1972
				\4 Intento de negociación de un nuevo acuerdo
				\4[] $\to$ Fracaso
				\4 Japón sale del acuerdo
				\4[] $\to$ Libre flotación
				\4 Flotación conjunta de europeos en el 73
				\4[$\then$] Abandono de Acuerdo de Smithsonian en 1973
			\3 Crisis del petróleo del 73
				\4 Embargo de envíos de petróleo
				\4[] A aliados de Israel en Guerra del Yom Kippur
				\4 Aumento de inflación en Estados Unidos
				\4 Empeora presión devaluatoria del dólar
			\3 Acuerdos de Jamaica (1976)
				\4 Abandono definitivo del marco de BW
				\4 Segunda enmienda a artículos del FMI (1978)
				\4 Se permite todo régimen cambiario
				\4[] $\to$ Incluida libre flotación
				\4[] $\then$ Salvo fijación de TCN respecto oro
				\4[] Entrada en vigor en 1978
		\2 Consecuencias
			\3 Política monetaria como instrumento de PEconómica
				\4 PM ya no está supeditada a mantener TC
				\4 Gobiernos pueden utilizar libremente para eq. interno
				\4 Financiación monetaria de déficits impulsa inflación
			\3 Regímenes cambiarios polares
				\4 Fijos ajustables no parecen viables
				\4[] $\to$ Aumento de flujos de capital
				\4[] $\then$ Necesarias enormes intervenciones para compensar
				\4 Economías grandes y relativamente aisladas
				\4[] $\to$ Libre flotación
				\4[] $\then$ Estados Unidos y Japón
				\4 Economías pequeñas y dependientes de comercio internacional
				\4[] $\to$ Fijación fuerte del TC
				\4[] Miembros de CEE
				\4[] $\to$ Mantener paridades muy importante dada PAC
				\4[] $\to$ Serpiente Europea
				\4[] $\then$ Sistema Monetario Europeo
				\4[] Pequeñas economías abiertas
				\4[] $\to$ Juntas de conversión
				\4[] $\to$ Regímenes fijos
			\3 Volatilidad de tipos de cambio
				\4 Fluctuaciones constantes
				\4 Alrededor del $3\%$ mensual
				\4 TCReal mucho más volátiles
				\4 Intervención pública habitual en mercados cambiarios
			\3 Impulso a integración europea
				\4 Políticas europeas ya implementadas
				\4[] Unión aduanera
				\4[] Política agrícola común
				\4 Estabilidad cambiaria necesaria
				\4[] Aprovechar ventajas de integración comercial
				\4[$\then$] Impulso a integración monetaria
				\4[] Informe Werner en 1970
				\4[] Serpiente en el túnel
				\4[] Sistema Monetario Europeo en 1979
	\1 \marcar{Gran Moderación}
		\2 Contexto
			\3 Económico
				\4 Inestabilidad cambiaria tras caída BW
				\4 Inflación elevada
			\3 Político
				\4 Victoria de Jimmy Carter en 1977
				\4 Políticas expansivas
			\3 Teórico
				\4 Influencia de monetarismo
				\4 Reglas de crecimiento de M aparecen como óptimas
				\4 Modelo de overshooting de Dornbusch
		\2 Eventos
			\3 Inestabilidad inicial
				\4 Dólar sufre fuerte depreciación inicial
				\4[] Posterior apreciación
				\4[] $\to$ De nuevo depreciación en 1977 con Carter
				\4 Volatilidad general aumenta
				\4 Intervención en monedas aparte del dólar
				\4[] Frenar apreciaciones
				\4[] $\to$ Especialmente marco, yen
				\4[] Mantener nivel frente a depreciaciones
				\4[] $\to$ Resto de monedas
				\4 Búsqueda de equilibrio interno predomina en EEUU
				\4[] Inflación y expansión fiscal
			\3 Acuerdo de Bonn en 1978
				\4 EEUU acepta:
				\4[] reducir déficit fiscal
				\4[] subir precio del petróleo doméstico
				\4[] contener salarios
				\4 Japón y Europa aceptan
				\4[] Expansión fiscal y monetaria
				\4[] Limitar apreciación de monedas
				\4[] Contener entradas de capital
				\4[] $\to$ Para limitar depreciación del dólar
			\3 Sistema Monetario Europeo en 1979
				\4 Acuerdo firmado en 1978
				\4[] $\to$ Entrada en vigor en 1979
				\4 Iniciativa francesa
				\4 Superar problemas de la serpiente
				\4[] Desaparición del túnel tras Jamaica 1976
				\4[] $\to$ ``Serpiente en el lago''
				\4[] Deficiente coordinación de política fiscal
				\4[] Fondo Europeo de Cooperación Monetario insuficiente
				\4[] $\to$ BCentrales no aceptan cesión de poder
				\4[] $\then$ Apenas coordinan intervención cambiaria
				\4[] $\then$ No existe equivalente regional de FMI
				\4[] $\then$ Marco alemán se convierte en líder
				\4 Objetivos franceses
				\4[] Marco de políticas y apoyo monetario
				\4[] $\to$ Para facilitar paridad marco-franco
				\4 Objetivos alemanes
				\4[] Federalismo europeo
				\4[] Mitigar efectos de depreciación del dólar
				\4 ERM --  Exchange Rate Mecanism
				\4[] Bandas del $2,25\%$ alrededor de TCN central
				\4[] Todos los EEMM salvo:
				\4[] $\to$ Italia con banda del $6\%$
				\4[] $\to$ Reino Unido fuera
				\4 Ajustes de TCN central
				\4[] Cada $8 \sim 12$ meses
				\4[] Estabilidad notable a lo largo de los 80
				\4[] $\to$ Aunque tensión sobre franco primeros 80s
				\4[] $\to$ Reajustes no compensaron diferencial de inflación
			\3 Nueva política monetaria en EEUU desde 1979
				\4 Paul Volcker nuevo presidente de la Fed en 1979
				\4 Ronald Reagan gana elecciones en 1980
				\4 Contracción monetaria
				\4[] Subida de interés
				\4[] Caída de crecimiento de oferta monetaria
				\4[] $\to$ Whatever it takes para reducir inflación
				\4[$\then$] Fuerte apreciación del TCN y TCR en 80-82
				\4 Expansión fiscal de Reagan
				\4[] Reducción de impuestos
				\4[] Aumento de gasto militar
				\4[] $\then$ Tipos de interés aumenta aún más
				\4[] $\then$ Interés de dólar aumenta
				\4 Apreciación continúa entre 1983 y 1985
				\4[] Regla de $\Delta M$ estable se mantiene
				\4[] Déficit continúa aumentando
				\4[] Cada vez más apreciación real
				\4[] Tesoro americano atribuye a baja inflación
				\4[] $\to$ Hasta cierto punto, explicación cierta
				\4 Europeos implementan EMS
				\4 Japón aumenta superávit comercial
				\4 Presiones proteccionistas en congreso americano
			\3 Acuerdos del Hotel Plaza (1985)\footnote{Ver \href{https://www.nber.org/papers/w21813.pdf}{Frankel (2015)}}
				\4 Apreciación adicional en 1985
				\4[] Atribuida a burbuja especulativa
				\4 Reunión de G-5 en Hotel Plaza de Nueva York
				\4[] Deseo de evitar legislación proteccionista
				\4[] $\to$ Contrario a agenda de Reagan
				\4 Comunicado conjunto:
				\4[] Iniciar apreciación ordenada frente a dolar
				\4[$\then$] Dólar comienza a caer
				\4[$\then$] Sin cambios en política fiscal y monetaria
				\4 Depreciación rápida
				\4[] EEUU había agotado activos exteriores
				\4[] Rentas primarias escasas
				\4[] Pocas exportaciones
				\4[] $\then$ Necesario TCN menor que pre-apreciación
				\4 Europa y Japón sufren por apreciación
				\4[] Dólar se deprecia hasta $40\%$ respecto pico
			\3 Acuerdos del Louvre (1987)
				\4 Estabilización del dólar
				\4[] Evitar depreciación ulterior
				\4 Japón acepta medidas de estímulo
				\4 Alemania acepta bajadas ligeras de impuestos
				\4 EEUU promete ajustes de PM
				\4 Papel pasivo de FMI
				\4[] Apenas influye en política económica
				\4[] Poco incentivo a negociar en marco FMI
			\3 Crisis del ERM\footnote{Eichengreen, págs. 149 y ss.}
				\4 Finales de los 80
				\4[] Crecimiento bajo
				\4[] Desempleo alto
				\4[] Tipos estables
				\4[] $\to$ Euroesclerosis
				\4[] $\to$ Impulso integrador
				\4 Acta Única de 1987
				\4[] Integración de mercados europeos
				\4[] Eliminación de controles de capital
				\4[] $\to$ Dificultan reajustes del TC
				\4[] $\then$ Sin reajustes del ERM después de 1987
				\4[] Tipos de cambio son obstáculo para integración completa
				\4 Informe Delors de 1989
				\4[] Propone integración económica y monetaria
				\4[] $\to$ Moneda única en una década\footnote{A semejanza del Informe Werner.}
				\4[] $\to$ Unión económica
				\4[] Base de Tratado de Maastricht
				\4[] Fase I -- 1990
				\4[] $\to$ Eliminar todos los controles de capital
				\4[] Fase II -- 1994
				\4[] $\to$ Convergencia de política económica
				\4[] $\to$ Objetivos de convergencia nominal
				\4[] $\to$ Creación del EMI
				\4[] Fase III -- 1997
				\4[] $\to$ Para los que cumplan requisitos
				\4[] $\to$ Fijación irrevocable de TCN
				\4 Recesión global en 1990
				\4[] Dólar se deprecia
				\4[] Europa pierde competitividad
				\4 Tratado de Maastricht de 1991
				\4[] Preparar transición a Unión Económica y Monetaria
				\4 Colapso de Unión Soviética y reunificación alemana
				\4[] Déficits por CC en Finlandia
				\4[] expansión fiscal en Alemania
				\4[] Apreciación del marco frente al dolar
				\4[] Pérdida de competitividad generalizada añadida
				\4 Dudas sobre compromiso con Maastricht
				\4[] Mantener TCN se vuelve muy costoso
				\4[] Controlar inflación para mantener TCN
				\4[] $\to$ En contexto de expectativas de inflación alta
				\4[] $\then$ Mantenerla baja es muy costoso
				\4[] $\then$ Recesión dificulta más aún
				\4 Crisis en 1992
				\4[] Dinamarca rechaza Maastricht
				\4[] Lira, libra, peseta, escudo caen
				\4[] $\to$ Se acercan a límite inferior del ERM
				\4[] $\to$ Finlandia se sale de peg
				\4[] $\to$ Presión sobre Suecia y otros
				\4[] $\to$ Libra sale del ERM (Soros)
				\4[] Gobiernos no están dispuestos a defender monedas
				\4[] $\to$ Aunque en general, reservas suficientes
				\4[] Ampliación de bandas de fluctuación
				\4[] $\to$ Al 15\%
				\4 Explicaciones de la crisis
				\4[] I. Armonización inadecuada de políticas pasadas
				\4[] $\to$ Inflación pasada demasiado elevada
				\4[] $\to$ Desequilibrio de TCReal
				\4[] II. Armonización inadecauda de poĺíticas futuras
				\4[] $\to$ Compromiso dudoso con Maastricht
				\4[] $\to$ No tiene sentido defender TC sin Maastricht
				\4[] III. Ataques especulativos/múltiples equilibrios
				\4[] $\to$ Modelos de crisis de tercera generación
			\3 Crisis asiática
				\4 Gobiernos fuertes en tigres asiáticos
				\4[] Tailandia, Indonesia, Corea, Singapur, Malasia, Filipinas
				\4[] $\to$ Semiautoritarios o dictaduras
				\4[] Resistentes a demandas de transferencias
				\4[] $\to$ Control de inflación
				\4 Controles de capital en funcionamiento
				\4 Crecimiento rápido vía exportaciones
				\4 Tipos de cambio estables
				\4[] Confianza generalizada
				\4[] Credibilidad de bancos centrales
				\4 Inversión en asiáticos emergentes
				\4[] Rendimientos bajos en Japón y EEUU
				\4[] $\to$ Depresión en Japón
				\4[] $\to$ Renta variable muy cara en EEUU
				\4[] $\then$ Búsqueda de rendimiento en tigres asiáticos
				\4[] $\then$ Entrada de capitales en tigres asiáticos
				\4 Carry trade
				\4[] Tomar prestado en yenes y dólares
				\4[] Invertir en asiáticos
				\4[] $\to$ Búsqueda de yield
				\4 Conexión bancos--industria
				\4[] Gobierno canaliza inversión crediticia hacia industria
				\4[] Creencia generalizada que gobiernos no dejarán caer
				\4 Mala gestión de cuentas de capital
				\4[] Apertura de cuenta de capital
				\4[] $\to$ Antes de flotar TCN
				\4[] Liberalización de endeudamiento offshore en divisas
				\4[] Restricciones a inversión entrante de IDE
				\4 Tailandia
				\4[] Primeros problemas en 1996
				\4[] Ralentización de exportaciones
				\4[] $\to$ Competencia china
				\4[] $\to$ Más inventarios en electrónica
				\4[] Exceso de inversión improductiva
				\4[] $\to$ Mercado inmobiliario en Bangkok
				\4[] Caída de la bolsa en 1996
				\4[] Primer colapso bancario
				\4[] Baht sufre presión devaluatoria
				\4[] Autoridades no devalúan ni frenan expansión
				\4[] $\to$ Miedo a reducir confianza
				\4[] Reservas se agotan en julio de 1997
				\4[] $\then$ Gobierno forzado a devaluar y flotar
				\4[] $\then$ Quiebras generalizadas de deudores en divisa
				\4 Transmisión del shock tailandés
				\4[] Similar dinámica en Filipinas
				\4[] $\to$ Flotación poco después que baht
				\4[] Malasia e Indonesia caen poco después
				\4[] Hong Kong resiste
				\4[] Taiwan flota divisa
				\4 Iniciativa de Chiang Mai
				\4[] Soporte mutuo para estabilizar divisas
				\4[] Evitar recurso al FMI
				\4[] Poca condicionalidad asociada
				\4[] $\to$ Poca disposición real a apoyar monedas
				\4[] $\then$ Relativo fracaso
				\4[] $\then$ Éxito como marco de discusión y cooperación
				\4 Reversión de cuenta corriente
				\4[] Inversión se desploma
				\4[] $\to$ Proporcionalmente más que ahorro
				\4[] Bancos centrales comienzan a acumular divisas
			\3 Crisis en Rusia, Turquía y Brasil
				\4 Contexto similar a crisis asiáticas
				\4[] Con particularidades nacionales
				\4[] Problemas estructurales comunes
				\4[] $\to$ Evasión fiscal
				\4[] $\to$ Gasto público elevado
				\4[] $\to$ Controles de precios en importaciones de consumo
				\4[] $\to$ Desigualdades excesivas
				\4[] $\to$ Inflación elevada
				\4 Estabilizar inflación vía fijación del TCN
				\4[] Brasil, Rusia, Turquía, México, Argentina
				\4[] Soluciona síntoma (inflación)
				\4[] $\to$ Causas estructurales persisten
				\4[] $\then$ Salarios aumentan fuertemente
				\4[] $\then$ Bienes no comerciables se encarecen
				\4[] $\then$ Perdida de competitividad
				\4 Ataques sucesivos tras crisis asiática
				\4[] $\to$ Brasil, Turquía, Rusia
				\4[] $\then$ Devaluaciones a finales de los 90
				\4 Recuperación rápida
			\3 Crisis argentina
				\4 Junta de conversión desde finales de los 80
				\4 Bancos son filiales de grandes bancos extranjeros
				\4 Deuda extranjera denominada en dólares
				\4 Shocks adversos a finales de los 90
				\4[] Crisis asiática + rusa + brasileña + apreciación dolar
				\4[] $\to$ Devaluaciones perjudican competitividad argentina
				\4[] $\to$ Apreciación dólar empeora
				\4 Déficits públicos persistentes
				\4 Sindicatos presionan salarios al alza
				\4 Suspensión de convertibilidad en 2001
				\4 Quiebra del sistema bancario
				\4 Corralito
				\4[$\then$] Depresión generalizada
				\4 Quiebra de LCTM
			\3 Crisis de las puntocom
				\4 Enorme aumento de cotización bursátil
				\4[] Empresas relacionadas con internet
				\4[] Muchas de ellas sin verdadero modelo de negocio
				\4[] $\to$ Nasdaq crece +400\% entre 95 y marzo 2000
				\4[] $\to$ Compra de stocks ``por si acaso''
				\4[] Salida a bolsa de muchas empresas tecnológicas
				\4 Crash de 2000
				\4[] Caída brusca de bolsa tecnológica
				\4[] Muchas empresas desaparecen
				\4[] $\to$ Sin modelo de negocio real
				\4 Consecuencias sobre conjunto de economía
				\4[] Sin consecuencias reales sobre economía real
				\4[] Relativamente contenido
				\4[] Sector tecnológico tiene aún muy poco peso
				\4 Bajada de tipos de interés
				\4[] Cercanos a cero tras crisis
				\4[] $\to$ Influencia en crisis posterior
				\4[] Subida a partir de 2005
		\2 Consecuencias
			\3 Crecimiento y baja inflación
				\4 Dos décadas de fuerte crecimiento
				\4[] $\to$ Mayor etapa de crecimiento desde IIGM
				\4[] $\to$ Generalizada en gran parte del mundo
				\4[] $\then$ ``Gran Moderación''
				\4 Varias teorías al respecto
				\4[I] Cambios en la estructura de las economías
				\4[] Mejoras en gestión de inventarios
				\4[] $\to$ Menor volatilidad
				\4[] Avances en tecnologías de la información
				\4[] $\to$ Producción más eficiente y menos riesgos
				\4[] Desregulación y flexibilidad
				\4[] $\to$ Más fácil ajuste a shocks
				\4[] Economías más abiertas
				\4[] $\to$ Mayor diversificación de comercio
				\4[] $\to$ Más estabilidad de financiación
				\4[II] Buena suerte respecto a shocks
				\4[] Simplemente, shocks estocásticos menos fuertes
				\4[] Ejemplo:
				\4[] $\to$ Sin shocks energéticos adversos
				\4[] $\to$ 1973 y 1979 no se repiten
				\4[] Crítica:
				\4[] $\to$ Shocks sólo se miden una vez ocurren
				\4[] $\to$ Buenas políticas reducen efecto de shock
				\4[] $\then$ Shocks más débiles sólo en apariencia
				\4[III] Buenas políticas
				\4[] Especialmente política monetaria
				\4[] Respuestas sistemáticas a inflación y output
				\4[] Aumento de independencia de bancos centrales
				\4[] $\to$ Reglas de Taylor explican nueva política monetaria
				\4[] $\then$ Menos inestabilidad de política monetaria
				\4[] Mejoras en comunicación y transparencia
				\4[] $\to$ Mejor fijación de expectativas
				\4[] $\to$ Menos efecto de rumores y problemas de infor.
			\3 Miedo a flotar
				\4 Crisis cambiarias de distintos tipos
				\4[] Cambian consenso sobre regímenes cambiarios
				\4 A finales de los 90
				\4[] Recomendación bipolar
				\4[] $\to$ O fijo fuerte, o totalmente libre
				\4 Tras crisis de los 90
				\4[] Miedo general a flotación libre
				\4[] De iure, libre flotación
				\4[] $\to$ Realmente, intervenciones frecuentes
				\4[] $\to$ Opacidad respecto a régimen real
			\3 Savings glut
				\4 Bernanke (2005)
				\4 Emergentes acumulan enormes reservas
				\4[] China
				\4[] Exportadores asiáticos que sufren crisis del 97
				\4[] Exportadores de energía
				\4[] Estabilidad en Latinoamérica
				\4 Atractivo de la inversión en dólares
				\4[] No hay activos seguros suficientes
				\4[] Economía americana crece
				\4[] Liderazgo tecnológico de EEUU se mantiene
				\4[] Gasto militar crece a buen ritmo
				\4[] $\then$ Déficits fiscales en EEUU
				\4[] $\then$ Dólar sigue gozando de ``privilegio exorbitante''
				\4 Mercados mundiales inundados de liquidez
				\4[$\then$] Tipos de interés caen fuertemente
				\4 Formación de burbujas en desarrollados
			\3 Nuevos polos de crecimiento
				\4 Grandes economías emergentes
				\4[] China, India, Brasil, Rusia, Sudáfrica, México
				\4[] $\to$ Muy elevadas tasas de crecimiento
				\4[] $\to$ Tamaño de economías alcanza masa crítica
				\4[] $\then$ Crecimiento en términos absolutos es significativo
				\4 Efectos globales
				\4[] Expansión y contracción de emergentes
				\4[] $\to$ Efectos también sobre desarrollados
				\4[] $\then$ Volatilidad global no sólo depende de desarrollados
	\1 \marcar{Gran Recesión}
		\2 Contexto
			\3 Económico
				\4 Desequilibrios globales
				\4[] Enormes déficit por CC en EEUU, España, RU...
				\4[] Emergentes grandes superávits CC
				\4 Savings glut
				\4[] Centro inundado de liquidez
				\4[] Tipos bajos tras crisis de puntocom
				\4 Reservas
			\3 Político
				\4 Guerras Iraq, Afganistán
				\4 Constitución Europea, Tratado de Lisboa
				\4 Rusia perfil bajo
				\4 EEUU rol hegemónico
			\3 Teórico
				\4 Modelos DSGE predominantes
				\4 Poca atención a imperfecciones del s. financiero
				\4 Behavioral economics
		\2 Eventos
			\3 Ascenso de China
				\4 Crecimiento rápido de la productividad china
				\4 Tipo de cambio nominal fijo
				\4 Exceso de demanda de renminbi
				\4 Controles de capital en China
				\4[] Notablemente efectivos
				\4[] Agentes chinos ``obligados'' no acceden a activos extranjeros
				\4 Esterilización del gobierno chino
				\4[] Exceso de demanda de renminbi provoca:
				\4[] $\to$ Presión sobre apreciación
				\4[] $\to$ Expansión de M en china e inflación
				\4[] Para evitarlo:
				\4[] $\to$ Compra ilimitada de divisa extranjera
				\4[] $\to$ Venta de deuda pública\footnote{Los llamados ``bonos de esterilización''.}
				\4[] $\then$ Oferta monetaria constante en China
				\4[] $\then$ Inflación bajo control
				\4[] $\then$ Compra de deuda pública americana
				\4 Codependencia financiera
				\4[] Estados Unidos depende de ahorro chino
				\4[] $\to$ Financiar déficit de CC
				\4[] China depende de mercados financieros americanos
				\4[] $\to$ Oferta de activo libre de riesgo muy líquido
				\4 Replicación del modelo
				\4[] Otros países en desarrollo replican estrategia de China
				\4[$\then$] Savings glut
			\3 Acumulación de desequilibrios
				\4 Precios inmobiliarios
				\4[] Aumentan desde finales de los 90
				\4[] En 2003, ratios alcanzan máximo desde IIGM:
				\4[] $\to$ Precios inmobiliarios reales
				\4[] $\to$ Precio de venta vs coste de construcción
				\4[] Siguen aumentando hasta 2007
				\4[] $\to$ Superando mayoría de máximos anteriores
				\4[] Aumento de la demanda de crédito
				\4[] $\to$ Compradores confían en $\uparrow$ precios inm.
				\4[] $\to$ Reducen capital aportado para compra
				\4[] $\then$ Aumento del apalancamiento
				\4[] $\then$ Aumento del riesgo si precios bajan
				\4 Expansión de la oferta de crédito
				\4[] Mercados inundados de liquidez
				\4[] IFs conceden crédito barato y fácil
				\4[] $\to$ Acreedores con muy poco equity
				\4[] $\to$ Acreedores con pocas rentas
				\4[] $\to$ Acreedores con poco colateral
				\4[] $\then$ Aumento del riesgo
				\4 Titulización en forma de CDOs
				\4[] Generación de hipotecas arriesgadas
				\4[] $\to$ Asumiendo riesgo hasta venta
				\4[] Empaquetamiento de hipotecas arriesgadas
				\4[] $\to$ En títulos valor
				\4[] $\to$ Cupones ligados a pago de hipotecas subyacentes
				\4[] $\to$ Divididos en \textit{tranches}
				\4 Colocación de CDOs
				\4[] Ultima fase de ``originate-to-distribute''
				\4[] Todo tipo de inversores internacionales
				\4[] Intermediarios mantienen hasta colocar
				\4[] $\to$ Algunos toman posiciones demasiado grandes
			\3 Crisis financiera en Estados Unidos
				\4 Aumento de la tasa de morosidad
				\4[] Ya en 2005 y 2006
				\4[] Varios prestamistas subprime quiebran
				\4[] $\to$ Situación relativamente contenida
				\4 Verano de 2007
				\4[] Caída de precios inmobiliarios
				\4[] Bear Stearns rescata fondo que invierte en CDOs
				\4[] Otro fondo de Bear Stearns quiebra
				\4[] BNP Paribas suspende uno de sus fondos
				\4[] $\to$ Porque mercado de subprime virtualmente desaparecido
				\4 Finales de 2007
				\4[] Bancos de todo el mundo se anotan pérdidas
				\4[] $\to$ Por deterioro de inversiones en inmobiliario
				\4[] Inyecciones de liquidez de bancos centrales
				\4[] Caída de precios inmobiliarios en
				\4 Verano de 2008
				\4[] Quiebras generalizadas de agencias hipotecarias
				\4[] Bancos anuncian pérdidas generalizadas
				\4[] 15 de septiembre:
				\4[] $\to$ colapso de Lehman Brothers
				\4[] $\to$ BoA compra Merrill Lynch
				\4[] $\to$ Fed rescata AIG
				\4[] Inyecciones masivas de liquidez
			\3 Recesión global
				\4 Contracción del crédito
				\4 Inversiones se frenan en todo el mundo
				\4 Desarrollados entran en recesión
				\4[$\then$] Caída muy fuerte del PIB mundial
				\4[$\then$] Desempleo aumenta a ritmo muy fuerte
				\4[$\then$] Inflación se desploma
			\3 Crisis del euro
				\4 Crisis griega
				\4[] Crisis inicialmente fiscal
				\4[] $\to$ Recaudación muy baja
				\4[] $\to$ Evasión fiscal elevada
				\4[] $\to$ Poco gasto productivo
				\4[] Principios de 2010
				\4[] Datos reales de PIB y deuda
				\4[] $\to$ Falseados por gobierno durante varios años
				\4[] Aumento rápido de prima de riesgo
				\4[] $\then$ Primer rescate en abril de 2010
				\4[] Bancos europeos expuestos a Grecia
				\4[] $\to$ Opacidad
				\4[] $\to$ Contagio a periferia y centro
				\4[] Creación de EFSF y EFSM en 2010
				\4[] $\to$ Mercados no confían en capacidad de rescatar
				\4[] $\then$ Deterioro de financiación a periferia
				\4[] Nueva crisis en 2015
				\4[] $\to$ Tras elección de Syriza
				\4[] $\to$ Salida del euro posible
				\4[] $\then$ Tensión en toda la zona euro
				\4[] $\then$ Tercer rescate aprobado en verano de 2015
				\4 Crisis en Irlanda
				\4[] Crisis inicialmente bancaria
				\4[] Bancos sobreexpuestos a inmobiliario
				\4[] $\to$ Caída de precios a lo largo de 2010
				\4[] $\then$ Bancos no reciben financiación
				\4[] Gobierno forzado a rescatar bancos en nov. 2010
				\4[] $\to$ Déficit público se dispara
				\4[] $\to$ Financiación prácticamente desaparece
				\4[] $\then$ Crisis bancaria se convierte en fiscal
				\4[] $\then$ Rescate a finales de 2010: EFSF+EFSM+FMI
				\4 Crisis en Portugal
				\4[] Mezcla de bancaria y fiscal
				\4 Crisis en España
				\4[] Crisis bancaria + cuenta corriente + fiscal
				\4[] Bancos fuertemente expuestos a inmobiliario
				\4[] $\to$ Burbuja desde primeros 2000
				\4[] $\to$ Exceso de inversión
				\4[] Expansión fiscal tras quiebra de Lehman Brothers
				\4[] $\to$ Empeora posición fiscal
				\4[] $\to$ Déficit alcanza 11\%
				\4[] Deterioro de cuentas públicas continúa en 2010 y 2011
				\4[] Desempleo se dispara
				\4[] Reforma del 135 de la Constitución en 2011
				\4[] Segundo rescate griego a principios de 2012
				\4[] $\to$ Incluye reestructuración
				\4[] Verano de 2012
				\4[] $\to$ Prima de riesgo supera 600 puntos
				\4[] $\to$ Financiación casi desaparece para España
				\4[] $\then$ ``Whatever it takes''
				\4[] $\then$ Programa OMT
				\4[] Rescate del ESM
				\4[] $\to$ $\sim 40000$ M € en 2012
				\4[] $\then$ Recapitalizar sistema bancario
		\2 Consecuencias
			\3 Contracción global del output
				\4 2007 a 2009
				\4[] En todo el mundo
				\4 2011 a 2013
				\4[] En zona euro también
				\4 Recuperación de niveles de PIB pre-crisis
				\4[] Heterogeneidad intra e interregional
				\4[] Estados Unidos recupera rápidamente
				\4[] España recupera en 2017
				\4 Caída generalizada de precios materias primas
				\4[] Inestabilidad en emergentes y PEDs
				\4 Taper Tantrum de 2013
				\4[] Fed anuncia reducción progresiva de QE
				\4[] $\then$ $\uparrow$ brusco del interés de bonos del Tesoro
				\4[] Interpretado como síntoma de adicción a QE
				\4[] Renta variable apenas cayó
				\4[] $\to$ En comparación con renta fija
				\4[] Se asumía que QE estaba inflacionando renta variable
				\4[] $\to$ Por exceso de liquidez
				\4[] $\to$ Por caída de tipo de descuento
				\4[] $\then$ Múltiples interpretaciones
				\4[] $\then$ Reducción QE también señaliza optimismo
			\3 Reformas
				\4 Zona Euro
				\4[] Creación de mecanismos de asistencia financiera
				\4[] $\to$ EFSF, EFSM, OMT
				\4[] Fiscal Compact
				\4[] $\to$ En TEGC (2012)
				\4 EEUU
				\4[] Dodd-Frank
				\4[] Política monetaria
				\4 G-20
				\4[] Se consolida como principal foro económico mundial
				\4[] Sustituye a G-7
			\3 Inestabilidad política
				\4 Causalidad muy dificil de establecer
				\4[] Pero habitualmente relacionada
				\4 Brexit
				\4 Guerras en MENA
				\4 Tensiones con China, Rusia
			\3 Desequilibrios persisten
				\4 Dólar sigue siendo moneda de reserva
				\4 Déficits comerciales americanos continúan
	\1 \marcar{Propuestas de reforma}
		\2 Contexto
			\3 Económico
				\4 Desequilibrios persisten
				\4 EEUU incurre en déficits crecientes
				\4 Aumento de tensiones comerciales
				\4 Críticas a gobernanza multilateral
				\4[] Emergentes critican reparto poder
				\4 Innovaciones financieras
				\4[] Bitcoin
				\4[] Shadow banking
				\4[] Fintech
				\4 Mercados de capital
				\4[] Caída de financiación vía equity
			\3 Político
				\4 Asertividad de emergentes
				\4 Elección de Trump
				\4 Tensiones en Unión Europea
				\4 Brexit
			\3 Teórico
				\4 Monetarismo de mercado
				\4[] Propuesta de NGDP como ancla nominal
				\4 DSGE
				\4[] Fuertemente criticados pero consolidación
				\4 Behavioral economics
				\4[] Aplicado también a mercados financieros
		\2 Problemas del sistema global de reservas
			\3 Ajuste asimétrico de la cuenta corriente
				\4 Ya planteado por Keynes (1942-1943)
				\4 Diferente presión para equilibrar cuentas corrientes
				\4[] Cuando ciclo entra en fase recesiva
				\4 Superávit de cuenta comercial
				\4[] Sin apenas presión para:
				\4[] $\to$ Aumentar absorción o inflar oferta monetaria
				\4[] $\then$ Equilibrar cuenta corriente
				\4 Déficit en cuenta comercial
				\4[] Fuerte presión para equilibrar
				\4[$\then$] Efecto recesivo global durante crisis
				\4[] Países con déficit contraen demanda
				\4[] Países con superávit no expanden demanda
			\3 Dilema de Triffin
				\4 Ya planteado por Triffin en 1960
				\4 Contexto:
				\4[] Moneda de reserva global es moneda de país
				\4 Demanda de liquidez global requiere déficits CC
				\4[] En país emisor de divisa de reserva
				\4 Déficits CC y endeudamiento excesivo
				\4[] Acaban erosionando confianza en moneda de reserva
				\4[$\then$] Incompatibilidad objetivos PM--estabilidad sistema
				\4[$\then$] Sistema inestable
				\4[] Valor de la moneda central tiende a depreciación
			\3 Acumulación de reservas por emergentes
				\4 PEDs tratan de reducir riesgo de financiación
				\4[] Sudden-stops y otros problemas potenciales
				\4 Acumulan reservas de divisas como seguro
				\4 Reservas invertidas en países desarrollados
				\4[] Activos seguros de países industriales
				\4[$\then$] PEDs prestan a ricos a bajo interés
				\4[$\then$] Desequilibrios globales se profundizan
			\3 Propuesta I: múltiples divisas de reserva
				\4 Solución inercial
				\4[] Sistema puede tender a ella automáticamente
				\4 Algunas voces afirma ya está en marcha
				\4[] Caída de reservas en dólares
				\4[] $\to$ En términos de dólares\footnote{Es decir, considerando la cantidad de dólares mantenida como activo de reserva, sin considerar efectos del tipo de cambio.}
				\4 Permite diversificación de reservas
				\4[] Reduce inestabilidad frente a shocks del dólar
				\4 Externalidades de red
				\4[] Dificultan transición a multidivisas
				\4[] Dólar mantiene superioridad
			\3 Propuesta II: aumentar papel de DEGs
				\4 Actualmente:
				\4[] Derechos Especiales de Giro (DEGs)
				\4[] $\to$ Derecho potencial de adquisición de divisas
				\4[] $\to$ Divisas: USD, EUR, RMB, JPY, GBP
				\4[] Creado en 1969
				\4[] Valor del DEG:\footnote{\url{http://www.imf.org/external/np/fin/data/rms\_sdrv.aspx}.}
				\4[] DEG compra cantidad fija $\vec{x}$ de divisas
				\4[] ¿Cuántos USD para comprar $\vec{x}$ de divisas?
				\4[] $\to$ Cantidad necesaria = valor en USD de 1 DEG
				\4[] Cantidades fijas consultables
				\4[] Emitidos
				\4[] $\to$ Hasta 2017, ~204.000 M de DEGs
				\4[] $\to$ Menos del 4\% global de activos de reserva
				\4 Convertir DEGs en verdadero activo de reserva mundial
				\4[] Emisiones mucho más grandes
				\4[] Sustituir dólar como medio de transacción
				\4[] Proveer PEDs con activo de reserva
				\4 Financiar todo préstamo con DEGs
				\4[] Actualmente, papel muy reducido
				\4[] Stand-by y otros programas se prestan divisas
				\4[] $\to$ No DEGs
				\4 Creación contracíclica de DEGs
				\4[] FMI actuando como banco central mundial
				\4[] Asignar DEGs contracíclicamente
				\4 Países en desarrollo
				\4[] Asignar más a países en desarrollo
				\4[] Permitir uso en mercados de capital
				\4[] $\to$ Financiación a PEDs con DEGs
				\4[] $\to$ Acumulación de reservas en DEGs
				\4 Reducir dependencia de financiación bilateral de FMI
				\4[] Actualmente, líneas de crédito bilateral
				\4[] $\to$ Para complementar financiación vía cuotas
				\4[] $\to$ Elimina también necesidad de aumentos de la cuota
		\2 Gobernanza mundial
			\3 Trilema de Rodrik
				\4 Rodrik (2000)\footnote{JEP Winter 2000.}
				\4 Restricción empírica postulada
				\4 Economías abiertas deben elegir 2 de 3:
				\4[I] Integración económica
				\4[II] Democracia
				\4[III] Soberanía nacional
				\4 Tres alternativas:
				\4[A] Camisa de fuerza de oro
				\4[] Integración económica+Estado nación soberano
				\4[] Sin transferencias fiscales entre estados
				\4[] Flujos de capital y comerciales libres
				\4[] Mercados internacionales limitan PEconómica nacional
				\4[] Sólo se proveen BPúblicos compatibles con MFinancieros
				\4[] Necesarias políticas autoritarias/represivas
				\4[] $\to$ Ante crisis de deuda/balanza de pagos
				\4[B] Federalismo supranacional
				\4[] Integración económica+democracia
				\4[] Apertura comercial y financiera plena
				\4[] Estados nación pierden soberanía
				\4[] $\to$ Entidad supranacional asume soberanía
				\4[] Transferencias fiscales entre estados
				\4[] $\to$ Posibles déficits exteriores y fiscales
				\4[] Democracia a nivel supranacional
				\4[] $\to$ Entidad supranacional se convierte en nación
				\4[C] Compromiso à la Bretton Woods
				\4[] Democracia+soberanía nacional
				\4[] Sin plena integración comercial+financiera
				\4[] Barreras a movimiento de capital generalizados
				\4[] Estados pueden evitar endeudamiento exterior
				\4[] Posible provisión democrática de bienes públicos
				\4[] $\to$ En la medida en que permita cap. productiva nacional
				\4[] $\to$ Como lo decidan votantes/responsable soberano
				\4[] Sin transmisión de soberanía a ent. supranacional
			\3 Organización Mundial del Medioambiente
				\4 Cambio climático parece acelerarse
				\4 Varios problemas de gobernanza impiden acción
				\4[I] Outsourcing a jurisdicciones menos restrictivas
				\4[] Implementación de medidas a nivel nacional
				\4[] $\to$ Cuotas de emisión
				\4[] $\to$ Impuestos al carbono
				\4[] $\to$ Mercados de derechos de emisión
				\4[] $\then$ Incentivan outsourcing a países sin medidas
				\4[II] Externalidades de red y economías de escala
				\4[] Industrias contaminantes a menudo son de red
				\4[] $\to$ Transporte terrestre, aeronaves, trenes..
				\4[] Economías de escala en tecnologías
				\4[III] Free-riding
				\4[] Incentivos a no actuar si otros actúan
				\4[$\then$] Necesario marco de gobernanza global
				\4[] UNEP -- Programa de Naciones Unidas para el Medio Ambiente
			\3 Integración monetaria y económica en África
				\4 Muy poco comercio interafricano
				\4 África será tercio de población mundial
				\4 Integración comercial en el continente
				\4[] $\to$ Potenciales beneficios
				\4[] $\to$ Spill-overs a todo el mundo
		\2 Sistema multilateral de comercio
			\3 Situación actual
				\4 EEUU tiene enormes déficits CC
				\4 Emergentes y UE tienen superávit
				\4 EEUU bloquea renovación de Órgano de Apelación
				\4[] Sólo tres miembros en la actualidad
				\4[] $\to$ Mínimo necesario para seguir funcionando
				\4[] $\then$ Aumento de plazos de decisión
				\4[] $\then$ Posible bloqueo a medio plazo
				\4 Razones para bloquear de EEUU
				\4[] Contractualismo vs constitucionalismo
				\4[] $\to$ EEUU prefiere soluciones vía negociación
				\4[] $\to$ MSD\footnote{Mecanismo de solución de disputas.} sustituye negociación
				\4[] Plazos largos
				\4[] $\to$ Superiores a 90 días
				\4[] Poca transparencia del dictamen de apelación
				\4[] $\to$ EEUU quiere partes tengan más información
				\4[] $\to$ Facilitar solución flexible por vía diplomática
			\3 Propuestas de desbloqueo
				\4 Sustitución de MSD por mecanismo de arbitraje
				\4[] Existe ya en protocolos de WTO
				\4 Abandono de OMC por EEUU
				\4[] No depende de resto de miembros
				\4[] Posibilidad no totalmente descartable
				\4[] Dificil tras elecciones legislativas americanas
				\4 Extensión del mandato de miembros del OApelación
				\4 Aprobación de reforma por mayoría, no consenso
				\4[] Opción ``nuclear''
				\4[] Economías grandes no quiere sentar precedente
				\4[] $\to$ Países pequeños aumentarían poder enormemente
		\2 Criptomonedas
			\3 Situación actual
			\3 Perspectivas futuras
			\3 Propuestas de solución
	\1[] \marcar{Conclusión}
		\2 Recapitulación
			\3 Caída de Bretton Woods
			\3 Gran Moderación
			\3 Gran Recesión
			\3 Propuestas de reforma
		\2 Idea final
			\3 Interacción economía y política
			\3 Cambio climático
			\3 Integración de emergentes
\end{esquemal}




























\conceptos

\concepto{Q regulation}

\concepto{Interest equalization tax}

\concepto{Sudden stop}

\concepto{Financial trilemma}

(Schoenmaker, 2013) plantea el trilema financiero parafraseando al conocido trilema atribuido a Mundell-Fleming. En el contexto del trilema financiero, las economías del mundo contemporáneo deben elegir dos de tres políticas posibles:
\begin{itemize}
	\item Integración con mercados globales de capital.
	\item Estabilidad financiera interna
	\item Soberanía respecto a la regulación financiera
\end{itemize}

Así, una economía puede obtener estabilidad financiera en la medida en que esté dispuesta a aislarse de los mercados financieros globales o en la medida en que permita una cesión de soberanía a  instituciones externas que implementen una política de estabilidad financiera coordinada.

\preguntas

\seccion{Test 2017}
\textbf{36.} Cuando se desencadena la crisis económica y financiera en 2008/2009, no existen entidades multilaterales con competencias efectivas sobre las políticas económicas y financieras de las economías afectadas, de ahí que el G-20,

\begin{itemize}
	\item[a] del que España forma parte como país miembro, reconociese explícitamente las debilidades económico-financieras del orden multilateral.
	\item[b] con gran peso en la economía y finanzas mundiales, pasase definitivamente a sustituir y asumir el total de los cometidos del G-7, que ya no realiza actualmente las cumbres habituales.
	\item[c] que representa más del 95\% del PIB mundial, decidiese reunirse, a partir de la Cumbre de Toronto, al menos, dos veces al año para tratar temas mundiales de gran trascendencia.
	\item[d] alentó unas políticas macroeconómicas de corte Keynesiano.
\end{itemize}

\seccion{Test 2011}

\textbf{30.} En los Acuerdos del Louvre de 1987:

\begin{itemize}
	\item[a] Los países participantes acordaron depreciar el dólar respecto al yen y el marco alemán.
	\item[b] No participó el Reino Unido.
	\item[c] Los países participantes buscaban frenar la depreciación del dólar.
	\item[d] Se acordó una nueva emisión de DEG.
\end{itemize}


\seccion{Test 2004}

\textbf{34.} La desaparición del sistema de tipos de cambio fijos de Bretton Woods permitió:
\begin{itemize}
	\item[a] Aumentar la confianza internacional en el dólar estadounidense.
	\item[b] Mantener la simetría del sistema monetario internacional, que era una característica básica del sistema de Bretton Woods.
	\item[c] Dar mayor autonomía a las políticas monetarias nacionales con miras a alcanzar los equilibrios interno y externo de las economías.
	\item[d] Que los tipos de cambio dejaran de actuar como estabilizadores automáticos y empezaran a depender de las decisiones de política económica.
\end{itemize}

\notas

\textbf{2017:} \textbf{36.} D

\textbf{2011:} \textbf{30.} C

\textbf{2004:} \textbf{34.} C


\bibliografia

International Monetary Fund, Palgrave: \textit{<<Keynes once helpfully remarked that in order to comprehend the Bretton Woods institutions one has to understand that the Fund is a bank, and the Bank is a fund.>>} Interesante frase para explicar instituciones.

Mirar en Palgrave:
\begin{itemize}
	\item Bretton Woods system
	\item central banks during the Global Financial crisis
	\item debt mutualisation in the ongoing eurozone crisis -- a tale of 'North and the 'South
	\item economics in Soviet Union
	\item Eurozone crisis 2010
	\item Fannie Mae, Freddie Mac and the crisis in US mortgage finance
	\item financial crisis
	\item Germany in the Euro Area crisis
	\item Greek crisis in perspective: causes, illusions and failures
	\item Greek crisis in perspective: origins, effects and ways-out
	\item international financial institutions
	\item International Monetary Fund
	\item Irish crisis: origins and resolution
	\item Japan, economics in
	\item Japanese economy
	\item Korea, economics in
	\item LIBOR: origins, economics, crisis, scandal and reform
	\item Minsky crisis
	\item regulatory responses to the financial crisis: an interim assessment
	\item sovereign debt
	\item Soviet Economic Reform
	\item stagflation
	\item subprime mortgage crisis
\end{itemize} 

Akinci, O. Beck, R. Donati, P. Goldberg, L. Stracca, L. \textit{A decade after Lehman, puzzles and challenges in the international monetary system} (2019) -- \url{https://voxeu.org/article/decade-after-lehman-puzzles-and-challenges-international-monetary-system}

Baldwin, R. (2016) \textit{The World Trade Organization and the Future of Multilateralism} Journal of Economic Perspectives -- En carpeta del tema

Bordo, M. D.; Schwartz, A. J. (1984) \textit{A Retrospective on the Classical Gold Standard, 1821--1931} University of Chicago Press -- En carpeta Historia Económica

Brunnermeir, M. \textit{The Euro Crisis} (2018) Ch. 7 of The Structural Foundations of Monetary Policy -- En carpeta Macro

Calvo, G. A.; Reinhart, C. M. (2000) \textit{Fear of floating} NBER Working Paper Series -- En carpeta del tema 

Dellas, H.; Tavlas, G. S. \textit{Milton Friedman and the case for flexible exchange rates and monetary rules} (2017) Bank of Greece Working Paper -- En carpeta del tema

Frankel, J. F. (2015) \textit{The Plaza Accord, 30 years later} NBER Working Paper Series \href{https://www.nber.org/papers/w21813.pdf}{Disponible aquí} -- En carpeta del tema

Gertler, M.; Gilchrist, S. (2018) \textit{What Happened: Financial Factors in the Great Recession} Journal of Economic Perspectives: summer 2018 -- En carpeta del tema

Hull, J. C. \textit{Options, Futures and Other Derivatives} (2018) 10th Edition. Chapter 8. Securitization and the Credit Crisis of 2007

IMF (2019) \textit{External Sector Report. The Dynamics of External Adjustment} Julio de 2019 \url{https://www.imf.org/en/Publications/SPROLLs/External-Sector-Reports} -- En carpeta del tema

Obstfeld, M; Taylor, A. \textit{International Monetary Relations: Taking Finance Seriously} Journal of Economic Perspectives, Summer 2017

\textbf{Parker, R. E.; Whaples. W. (2013) \textit{Routledge handbook of major events in economic history} Routledge -- En carpeta del tema}

Peterson Institute for International Economics. (2018) \textit{IMF Quota and Governance Reform Once Again} Edwin M. Truman. Policy Brief -- En carpeta del tema

Pilbeam, K. \textit{International Finance} (2006) 3rd Edition -- En carpeta Economía Internacional

Rey, H. (2018) \textit{Dilemma not trilemma: the global financial cycle and monetary policy independence} NBER Working Paper Series \href{https://www.nber.org/papers/w21162.pdf}{Disponible aquí} -- En carpeta del tema

Rodrik, D. (2000) \textit{How Far Will International Economoic Integration Go} Journal of Economic Perspectives. Winter 2000. -- En carpeta del tema

\end{document}
