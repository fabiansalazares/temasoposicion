\documentclass{nuevotema}

\tema{3B-8}
\titulo{La política comercial (II): La política comercial estratégica. La política de promoción exterior: justificación, instrumentos y objetivos.}

\begin{document}

\ideaclave

\seccion{Preguntas clave}
\begin{itemize}
	\item ¿Qué es la política comercial estratégica?
	\item ¿Qué modelos teóricos explican las decisiones de política comercial estratégica?
	\item ¿Qué evidencia empírica existe sobre la pol. comercial estratégica?
	\item ¿Qué relevancia tiene la política comercial estratégica en el contexto internacional actual?
	\item ¿Qué es la política de promoción exterior?
	\item ¿Por qué se lleva a cabo la promoción exterior?
	\item ¿Qué objetivos tienen las políticas de promoción exterior?
	\item ¿Qué instrumentos se utilizan para llevar a cabo la política de promoción exterior?
\end{itemize}


\esquemacorto

\begin{esquema}[enumerate]
	\1[] \marcar{Introducción}
		\2 Contextualización
			\3 Evolución del comercio internacional
			\3 Sujetos de análisis de la teoría pura del CI
			\3 Beneficios del libre comercio
			\3 Política comercial estratégica
			\3 Promoción exterior
		\2 Objeto
			\3 ¿Qué es la política comercial estratégica?
			\3 ¿Qué modelos teóricos explican las decisiones de política comercial estratégica?
			\3 ¿Qué evidencia empírica existe sobre la pol. comercial estratégica?
			\3 ¿Qué es la política de promoción exterior?
			\3 ¿Por qué se lleva a cabo la promoción exterior?
			\3 ¿Qué objetivos tienen las políticas de promoción exterior?
			\3 ¿Qué instrumentos se utilizan para llevar a cabo la política de promoción exterior?
		\2 Estructura
			\3 La política comercial estratégica
			\3 La política de promoción exterior
	\1 \marcar{La política comercial estratégica}
		\2 Idea clave
			\3 Contexto
			\3 Objetivo
			\3 Resultados
		\2 Representación básica en teoría de juegos
			\3 Desarme arancelario: dilema del prisionero
			\3 Un gobierno subvenciona entrada en 3er mercado
			\3 Dos gobiernos subvencionan entrada en 3er mercado
			\3 Implicaciones generales
		\2 Industria naciente
			\3 Idea clave
			\3 Formulación
			\3 Implicaciones
			\3 Valoración
		\2 Brander y Spencer (1981): entrada y Stackelberg
			\3 Idea clave
			\3 Formulación
			\3 Implicaciones
			\3 Valoración
		\2 Spencer y Brander (1983) -- Subsidios a la I+D
			\3 Idea clave
			\3 Formulación
			\3 Implicaciones
			\3 Valoración
		\2 Brander y Spencer (1985) -- Subsidios a la exportación en Cournot
			\3 Idea clave
			\3 Formulación
			\3 Implicaciones
			\3 Valoración
		\2 Eaton y Grossman (1986) -- Competencia à la Bertrand
			\3 Idea clave
			\3 Formulación
			\3 Implicaciones
			\3 Valoración
		\2 Oligopolio verticalmente diferenciado -- Shaked y Sutton (1983)
			\3 Idea clave
			\3 Formulación
			\3 Implicaciones
			\3 Valoración
		\2 Barret (1994): pol. medioambiental estratégica
			\3 Idea clave
			\3 Formulación
			\3 Implicaciones
	\1 \marcar{La política de promoción exterior}
		\2 Idea clave
			\3 Contexto
			\3 Objetivos
			\3 Resultados
		\2 Justificación de las políticas de promoción exterior
			\3 Economías de escala
			\3 Resistencia al fluctuaciones cíclicas
			\3 Externalidades
			\3 Costes hundidos
			\3 Redes sociales
			\3 Barreras de entrada
			\3 Path-dependency/Dependencia de senda
			\3 Escasez de capital humano en internacionalización
		\2 Instrumentos no financieros
			\3 Regulatorios
			\3 Formación
			\3 Información
			\3 Promoción
			\3 Diplomacia económica
			\3 Provisión de consultoría y asesoramiento
		\2 Instrumentos financieros de promoción exterior
			\3 Idea clave
			\3 Créditos directos a exportación
			\3 Garantías para créditos a exportación
			\3 Avales
			\3 Capital para inversión
			\3 Determinantes de volumen de financiación
			\3 Determinantes de condiciones de financiación
	\1 \marcar{Indicadores de competitividad}
		\2 Idea clave
			\3 Concepto de competitividad
			\3 Medición de la competitividad
		\2 Indicadores de competitividad-precio/coste
			\3 Idea clave
			\3 Metodología
			\3 Índice de Tendencia de Competitividad
			\3 Ventajas e inconvenientes
		\2 Indicadores de competitividad revelada
			\3 Idea clave
			\3 Cuota de España en exportaciones mundiales de bienes
			\3 Cuota de España en exportaciones de bienes de UE
			\3 Cuota de España en exportaciones de servicios
		\2 Otros indicadores
			\3 Calidad institucional
			\3 Doing business
			\3 Otros
	\1[] \marcar{Conclusión}
		\2 Recapitulación
			\3 Política comercial estratégica
			\3 Política de promoción exterior
		\2 Idea final
			\3 Contexto internacional
			\3 Unión Europea
			\3 Relación con otras áreas

\end{esquema}

\esquemalargo

\begin{esquemal}
	\1[] \marcar{Introducción}
		\2 Contextualización
			\3 Evolución del comercio internacional
				\4 Explosión en últimos siglos
				\4[] $\to$ Y más aún desde post 2GM
				\4 Avance tecnológico:
				\4[] $\downarrow$ de costes de transporte
				\4[] $\downarrow$ de costes informacionales
				\4 Sujeto de estudio relativamente antiguo:
				\4[] $\to$ Smith, Ricardo, Mill
				\4[] Ligado a la evolución de:
				\4[] $\to$ teoría económica
				\4[] $\to$ hallazgos empíricos
			\3 Sujetos de análisis de la teoría pura del CI
				\4 Patrón de comercio
				\4[] Qué ByS intercambian los países
				\4 Relación real de intercambio
				\4[] A qué precios intercambian los ByS
				\4 Intervención pública en el comercio internacional
				\4[] Qué efectos positivos y negativos tiene
				\4[] Cómo pueden aumentarse los beneficios del CI
				\4[] Cómo pueden mitigarse los costes del CI
			\3 Beneficios del libre comercio
				\4 Modelos clásicos de CI concluyen
				\4[] Librecambio es preferible en principio
				\4 Existencia de distorsiones
				\4[] Puede hacer deseables intervenciones
				\4 Competencia imperfecta
				\4[] Oligopolio
				\4[] $\to$ Interdependencia estratégica
				\4[] Competencia monopolística
				\4[] $\to$ Variedad de producto
			\3 Política comercial estratégica
				\4 Gobiernos regulan CI de distintas formas
				\4[] Interviniendo precios relativos
				\4[] Restringiendo volúmenes de comercio
				\4[] Regulando procedimiento de intercambio
				\4[] Regulando prácticas de empresas exportadoras
				\4[] Comerciando directamente
				\4 Política comercial estratégica
				\4[] Decisión interactiva de política comercial
				\4[] Gobiernos toman decisiones y responden mutuamente
			\3 Promoción exterior
				\4 Exportación no depende sólo de precio
				\4[] Muchos otros factores son relevantes
				\4[] $\to$ Relaciones de largo plazo
				\4[] $\to$ Marketing
				\4[] $\to$ Publicidad
				\4[] $\to$ Calidad del producto
				\4[] $\to$ Cuestiones políticas y geoestratégicas
				\4 Actuación vía precios puede no ser posible
				\4[] Por acuerdos internacionales
				\4[] Por inefectividad de impuestos o subvenciones
				\4 Gobiernos pueden actuar en dimensión no-precio
				\4[] Aumentar atractivo de exportaciones nacionales
				\4[] Facilitar compra-venta de exportación nacional
		\2 Objeto
			\3 ¿Qué es la política comercial estratégica?
			\3 ¿Qué modelos teóricos explican las decisiones de política comercial estratégica?
			\3 ¿Qué evidencia empírica existe sobre la pol. comercial estratégica?
			\3 ¿Qué es la política de promoción exterior?
			\3 ¿Por qué se lleva a cabo la promoción exterior?
			\3 ¿Qué objetivos tienen las políticas de promoción exterior?
			\3 ¿Qué instrumentos se utilizan para llevar a cabo la política de promoción exterior?
		\2 Estructura
			\3 La política comercial estratégica
			\3 La política de promoción exterior
	\1 \marcar{La política comercial estratégica}
		\2 Idea clave
			\3 Contexto
				\4 Política comercial
				\4[] Actuaciones del poder público en relación a CI
				\4[] $\to$ Maximizar bienestar nacional
				\4 Sujeto muy antiguo de política económica
				\4[] Irwin (1996) sobre historia del libre comercio
				\4[] $\to$ Platón y Aristóteles ambiguos sobre comercio internacional
				\4[] $\to$ Política comercial  en Grecia y Roma
				\4[] ESalamanca, mercantilismo, fisiócratas, Smith, Ricardo, Malthus...
				\4[] $\to$ Impuestos y subvenciones a exportadores e importadores
				\4[] $\then$ Valorar oportunidad y efectos
				\4 Años 80
				\4[] Problema formulado en términos modernos
				\4[] Interdependencia estratégica como elemento central
				\4[] $\to$ Tras desarrollo de TJuegos
				\4[] Política comercial tiene efectos dentro y fuera
			\3 Objetivo
				\4 Enfoque positivo
				\4[] Explicar efectos de subvenciones e impuestos
				\4[] $\to$ Sobre patrón de comercio
				\4[] $\to$ Sobre precios relativos
				\4 Enfoque normativo
				\4[] Caracterizar políticas comerciales óptimas
				\4[] $\to$ Dado contexto institucional legal
				\4[] Valorar regulación comercial internacional
			\3 Resultados
				\4 Marco de análisis microeconómico
				\4[] Herramientas de teoría pura del CI tradicionales
				\4[] Teoría de juegos
				\4[] Interdependencia estratégica
				\4[] Amenazas creíbles son elemento importante
				\4 Dilemas del prisionero
				\4[] Resultados posibles en múltiples contextos
				\4 Acuerdos internacionales pueden mejorar equilibrio
				\4[] Política comercial estratégica es elemento central
		\2 Representación básica en teoría de juegos
			\3 Desarme arancelario: dilema del prisionero
				\4 Matriz de pagos
				\4 \grafica{prisionerodesarme}
				\4 Si un gobierno desarma
				\4[] País que desarma deteriora su RRI
				\4[] País que no desarma mejora RRI
				\4[] $\to$ Asumiendo caso estándar de ofertas recíprocas
				\4[] País que no desarma estimula output
				\4 Si dos gobiernos desarman
				\4[] Beneficios del comercio
				\4[] Menor coste de importaciones para consumidors
				\4[] Creación de comercio
				\4 Si ningún gobierno desarma
				\4[] Sin ganancias del comercio
			\3 Un gobierno subvenciona entrada en 3er mercado
				\4 Matriz de pagos
				\4[] \grafica{ungobierno}
				\4 Si gobierno europeo no interviene:
				\4[] Dos equilibrios de Nash
				\4[] $\then$ Boeing entra, Airbus no entra
				\4[] $\then$ Boeing no entra, Airbus entra
				\4 Si gobierno europeo subvenciona:
				\4[] Un sólo equilibrio de Nash
				\4[] $\then$ Boeing no entra, Airbus entra
			\3 Dos gobiernos subvencionan entrada en 3er mercado
				\4 Matriz de pagos
				\4[] \grafica{dosgobiernos}
				\4 Dilema del prisionero
				\4[] Ambos gobiernos tienen incentivos unilaterales
				\4[] $\to$ A desviarse de óptimos de Pareto
				\4[] Equilibrio de Nash subóptimo:
				\4[] $\to$ Ambos gobiernos subvencionan entradas
			\3 Implicaciones generales
				\4 Gobiernos pueden tener incentivos a subvencionar
				\4[] Subvención actúa como amenaza creíble
				\4 Cuando sólo un gobierno puede actuar:
				\4[] Empresa doméstica incentivada a entrar a toda costa
				\4[] Empresa extranjera percibe entrada segura de doméstica
				\4[] $\then$ Empresa extranjera no entra
				\4[] $\then$ Mejora de bienestar para país subvencionador
				\4[] $\then$ Gobiernos subvencionan exportadores nacionales
				\4[] $\then$ Subvención toma múltiples formas
				\4 Cuando todos los gobiernos pueden actuar
				\4[] Todos subvencionan empresas nacionales
				\4[] $\then$ Gasto innecesario en subvenciones y ayudas
				\4[] $\then$ Dilemas del prisionero
				\4[] $\then$ Margen para intervención coordinada si posible
		\2 Industria naciente
			\3 Idea clave
				\4 Contexto
				\4[] Muy antiguo debate
				\4[] $\to$ ¿Industrias requieren protección para desarrollarse?
				\4[] $\to$ ¿Posible desviar beneficios futuros aunque coste presente?
				\4[] Economías de escala externas
				\4[] $\to$ Mayor número de productores
				\4[] $\then$ Reduce coste medio
				\4[] Economías de escala internas
				\4[] $\to$ Mayor producción
				\4[] $\then$ Reduce coste medio
				\4[] Economías de escala dinámicas
				\4[] $\to$ Producción acumulada en el pasado
				\4[] $\then$ Reduce coste medio
				\4 Objetivo
				\4[] Justificar protección de industria naciente
				\4[] Valorar efectividad de protección
				\4 Resultado
				\4[] Debate de largo alcance
				\4[] Argumentos a favor y en contra
				\4[] Ejemplos de éxito y fracaso
			\3 Formulación
				\4 Situación inicial
				\4[] Industria nacional inexistente
				\4[] Potenciales economías de escala
				\4 Fase de intervención
				\4[] Restricciones cuantitativas
				\4[] Aranceles suficientemente altos
				\4[] Producción nacional aumenta
				\4[] Realización progresiva de economías de escala
				\4 Fase de libre comercio
				\4[] Eliminación de restricciones y aranceles
				\4[] Empresas nacionales son competitivas
			\3 Implicaciones
				\4 Test de Mill
				\4[] Una vez retirado el apoyo
				\4[] $\to$ Sector protegido debe ser rentable por sí solo
				\4 Test de Bastable
				\4[] Suma descontada de beneficios en libre comercio
				\4[] $\to$ Superior a costes en fase de intervención
			\3 Valoración
				\4 Crítica de los mercados de capital
				\4[] Asumiendo mercados financieros que se aproximan a perfección
				\4[] Pérdidas iniciales pueden financiarse con bfcios. futuros
				\4 Heterogeneidad de resultados
				\4 Coste de los fondos públicos
				\4[] Coste difícil de estimar
				\4[] En qué medida se induce crowding-out de mejores usos
				\4 ISI vs EOI
				\4[] Import-substitution Industrialization
				\4[] Export-oriented industrialization
				\4[] Consenso sobre inefectividad de ISI
				\4[] Países con EOI aparentemente mejores resultados
				\4[] Insuficiente tamaño de mercado
				\4[] $\to$ Una de las causas del fracaso de ISI
				\4 Reacción de mercados de importación
		\2 Brander y Spencer (1981): entrada y Stackelberg
			\3 Idea clave
				\4 Contextualización
				\4[] Oligopolio
				\4[] $\to$ Varias empresas
				\4[] $\to$ Interdependencia estratégica
				\4[] $\then$ Toman decisiones considerando respuesta de otros
				\4[] Número de empresas competidores
				\4[] $\to$ No necesariamente fijo
				\4[] $\to$ Posible entrada o no entrada
				\4[] Monopolio extranjero
				\4[] Política arancelaria
				\4[] $\to$ Permite extraer renta de empresas extranjeras
				\4[] $\to$ Implica menor producción y consumo
				\4[] $\to$ Puede afectar entrada de nuevas empresas
				\4[] $\then$ Arancel puede tener efectos sobre bienestar
				\4[] $\then$ Posible utilización para incentivar entrada empresas nacionales
				\4 Objetivos
				\4[] Caracterizar efectos de aranceles sobre entrada
				\4[] Caracterizar optimalidad de aranceles que afectan entrada
				\4 Resultados
				\4[] Aranceles pueden incentivar entrada de empresas domésticas
				\4[] Aranceles pueden aumentar bienestar nacional
				\4[] $\to$ Capturando parte de renta de monopolista
				\4[] $\to$ Entrada compensa reducción de producción
			\3 Formulación
				\4 Empresa extranjera
				\4[] Produce para mercado su mercado de origen extranjero
				\4[] Produce y exporta a mercado nacional
				\4[] Enfrenta costes fijos
				\4[] Decide primero cuando producir para mercado nacional
				\4[] $\to$ Se comporta como líder de Stackelberg
				\4 Empresa nacional
				\4[] Decide después de incumbente:
				\4[] $\to$ Entrar o no
				\4[] $\to$ Cuánto producir si entrar
				\4[] Enfrenta costes fijos
				\4[] $\to$ Posible equilibrio de Stackelberg bfcio < 0
				\4[] $\then$ Entrante nacional no entraría
				\4 Representación gráfica
				\4[] \grafica{branderspencer81}
			\3 Implicaciones
				\4 Entrada más probable con aranceles
				\4[] Menor producción de incumbente
				\4[] Mayor demanda residual para potencial entrante
				\4[] Isobeneficios de incumbente se desplazan hacia izquierda
				\4[] Equilibrio de Stackelberg más a la izquierda
				\4[] $\to$ Más probable entrante en isobeneficio > 0
				\4 Arancel puede aumentar bienestar nacional
				\4[] Arancel permite extraer renta de incumbente extranjero
				\4[] Arancel implica también disminución de consumo
				\4[] Entrada de productor nacional aumenta consumo
				\4[] $\then$ Neto puede ser positivo
				\4 Posibles incentivos unilaterales a establecer aranceles
				\4[] Especialmente, en contexto de monopolistas internacionales
				\4 Aranceles pueden ser incentivo a IDE
				\4[] Incumbente crea sus propios entrantes potenciales
				\4[] Reduce incentivo de gobierno a imponer aranceles
			\3 Valoración
				\4 Abre programa de investigación
				\4[] Política comercial en contexto de interdep. estratégica
		\2 Spencer y Brander (1983) -- Subsidios a la I+D
			\3 Idea clave
				\4 Contexto
				\4[] Años 80
				\4[] Subsidios a la exportación prohibidos
				\4[] Subsidios a I+D generalizados
				\4[] $\to$ Especialmente en industrias exportadoras
				\4 Objetivos
				\4[] Teoría positiva de los subsidios a la exportación
				\4[] Valorar papel de:
				\4[] $\to$ Secuencialidad de decisión
				\4[] $\to$ Papel de subsidios a exportación y a I+D
				\4 Resultado
				\4[] I. Subsidios positivos a I+D pueden ser óptimos
				\4[] $\to$ Cuando sólo un país los impone
				\4[] $\to$ Cuando los bienes son sustitutos estratégicos
				\4[] $\then$ Maximización de renta nacional
				\4[] II. Empobrecimiento del vecino posible
				\4[] $\to$ Si ambos países pueden imponer subsidios I+D
				\4[] $\to$ Si asimetría de demanda no es muy fuerte
				\4[] III. Subsidios a exportación reducen subsidios a I+D
				\4[] $\to$ Tienen efecto generalmente equivalente
				\4[] $\to$ Pero subsidios a exportación muy restringidos
				\4[] $\then$ Gobiernos ``escapan'' vía subsidios a I+D
			\3 Formulación
				\4 Juego de dos etapas
				\4[] 2 agentes
				\4[] $\to$ Empresa nacional
				\4[] $\to$ Empresa extranjera
				\4[] $\then$ Deciden cuánto invertir y cuanto producir
				\4[] Beneficios dependen de:
				\4[] $\to$ Producción propia (+)
				\4[] $\to$ Producción de la otra empresa (-)
				\4[] $\then$ Costes y demanda
				\4[] $\to$ Inversión propia en I+D (+)
				\4 Etapas del juego
				\4[] 1. Empresas deciden cuanto invertir en I+D
				\4[] 2. Empresas deciden cuanto producir
				\4[] Resultado:
				\4[] $\to$ Funciones de reacción de cada empresa
				\4[] $\to$ Equilibrio de Nash bajo supuestos habituales sobre demanda
				\4 Subsidio a I+D en un país
				\4[] Gobierno actúa antes que empresas
				\4[] $\to$ Amenaza totalmente creíble de actuar
				\4[] Estructura de costes de las empresas cambia
				\4[] $\to$ Inversión en I+D más barata
				\4[] $\then$ Invierten más en I+D
				\4[] Más inversión en I+D
				\4[] $\to$ Menores costes
				\4[] $\then$ Más producción resulta creíble
				\4[] Competidor reacciona reduciendo producción
				\4[] $\then$ Más beneficios para empresa subvencionada
				\4[] Equilibrio de Stackelberg
				\4[] $\to$ Equilibrio se asemeja a Stackelberg
				\4[] $\to$ Empresa subvencionada actúa como líder en I+D
				\4[] $\then$ Subvención le permite comprometerse a más I+D
				\4[] \grafica{spencerbrancersubsidiounpais}
				\4 Subsidios a I+D en ambos países
				\4[] Un sólo gobierno subvencionando
				\4[] $\to$ Empobrecimiento del país que no subvenciona
				\4[] $\to$ Ganancia de bienestar del que subvenciona
				\4[] Dos países que pueden subvencionar y no cooperan
				\4[] $\to$ Intentarán ganar a costa del vecino
				\4[] Equilibrio de Nash bajo FReacción y dda. estándar
				\4[] $\to$ Subvención a I+D positiva en ambos
				\4[] $\to$ Producción superior a eqs. anteriores
				\4[] $\then$ Precio inferior
				\4[] $\then$ Beneficios inferiores
				\4[] $\then$ Sobreproducción
				\4[] Dilema del prisionero para productores
				\4[] $\to$ Óptimo individual dado comportamiento del otro
				\4[] $\to$ Subóptimo global
				\4[] Óptimo para consumidores
				\4[] $\to$ Precio inferior respecto a ausencia de subvención
				\4 Subvenciones a la exportación y a I+D
				\4[] Subsidios anunciados al mismo tiempo
				\4[] $\to$ Antes de que empresa decida I+D
				\4[] Un sólo país puede subvencionar ambos
				\4[] $\to$ Producción e I+D
				\4[] Equilibrio óptimo
				\4[] $\to$ Subvención a la exportación
				\4[] $\then$ Reducción de costes de producción
				\4[] $\then$ Empresa actúa como líder de Stackelberg en output
				\4[] $\to$ Impuesto al I+D
				\4[] $\then$ Menor gasto improductivo en I+D
				\4[] $\then$ Costes ya se reducen por subvención a exportación
			\3 Implicaciones
				\4 Acuerdos antisubvención de exportaciones
				\4[] Potencialmente aumentan subsidios a I+D
				\4[] Fenómeno apreciado en últimas décadas
				\4 Coste de los fondos públicos no analizados
				\4[] Asumiendo subvención financiada con impuestos
				\4[] $\to$ Sin exceso de gravamen asociado
				\4[] $\then$ $\uparrow$ de beneficios induce mejora de bienestar
				\4[] En la práctica, no sería así
				\4 Bloques comerciales de gran tamaño
				\4[] En la práctica, múltiples bloques pueden subvencionar
				\4 Incentivos a negociación
				\4[] Gobiernos conocen problemas de comportamiento no cooperativo
				\4[] $\to$ Incentivos a negociar acuerdos antisubvención
				\4[] $\then$ I+D más difícil de evitar y gravar vía acuerdos
			\3 Valoración
				\4 Artículo seminal
				\4 Inicia programa de investigación
				\4[] PC estratégica en oligopolio
		\2 Brander y Spencer (1985) -- Subsidios a la exportación en Cournot
			\3 Idea clave
				\4 Contexto
				\4[] Spencer y Brander (1983)
				\4[] $\to$ Artículo seminal
				\4[] $\to$ Formulación relativamente compleja
				\4[] Teoría neoclásica del CI
				\4[] $\to$ Subsidios a la exportación son poco efectivos
				\4[] $\to$ Aranceles contrarrestan efecto
				\4[] $\to$ Sólo sirven para subvencionar consumidores extranjeros
				\4[] Sin embargo, subvenciones son habituales
				\4[] $\to$ Por vías indirectas
				\4[] $\to$ Cuando no están prohibidos
				\4 Objetivos
				\4[] Análisis de subsidios a exportación
				\4[] Simplificación de Spencer y Brander (1983)
				\4[] Explicar desviación de beneficios con subvenciones
				\4 Resultados
				\4[] Competencia imperfecta puede explicar
				\4[] Competencia à la Cournot induce subvención
				\4[] $\to$ Como decisión óptima
				\4[] Intervención pública puede ser second-best
				\4[] $\to$ Competencia imperfecta introduce restricción
				\4[] $\then$ Subvención es óptima
			\3 Formulación
				\4 Dos empresas en dos países distintos
				\4[] Representadas por $x$ e $y$
				\4 Maximización de beneficio de empresa $x$
				\4[] $\underset{x}{\max} \quad \pi(x,y;s) = p(x+y) x - c(x) + s \cdot x$
				\4[] CPO: \quad $\pi_x  = p' x + p - c_x +s = 0$
				\4[] CSO: \quad $\pi_{xx} = p'' x  + 2p' - c_{xx} < 0$
				\4 Supuestos sobre demanda:
				\4[] I. $\pi_{xy} \equiv p'' x + p' < 0$
				\4[] $\to$ BMg decrece con producción de otra empresa
				\4[] $\then$ Curvas de reacción decrecientes
				\4[] II. $\pi_{xx} < \pi_{xy}$, $\pi^*_{yy} < \pi^*_{yx}$
				\4[] $\to$ Efectos de $\uparrow$ prod. propia más fuertes que $\uparrow$ prod. ajena
				\4[$\then$] $D \equiv \pi_{xx} \pi_{yy}^* - \pi_{xy} \pi_{xy}^* >0$
				\4[$\then$] Equilibrio es único global y estable
				\4 Solución simultánea a problemas de máx. de $x$ e $y$
				\4[] Equilibrio no cooperativo de Cournot
				\4 Aumento del subsidio nacional
				\4[] Reduce costes para cualquier producción
				\4[] Aumenta producción nacional:
				\4[] $\pdv{x}{s} \equiv y_s \equiv \frac{-\pi_{yy}^*}{D} > 0$
				\4[] $\to$ Asumido $\pi_{yy} < 0$, $D>0$
				\4[] $\then$ $\pdv{x}{s} > 0$
				\4[] Cae producción extranjera:
				\4[] $\pdv{y}{s} \equiv y_s \equiv \frac{\pi_{yx}^*}{D} < 0$
				\4[] Aumenta beneficio para cualquier producción
				\4[] $\to$ Permite ``compromiso'' de mayor producción
				\4[] $\then$ Desplaza curva de reacción hacia fuera
				\4[] $\then$ Como en Spencer y Brander (1983)
			\3 Implicaciones
				\4 Incentivo unilateral a subsidio positivo
				\4[] País doméstico
				\4 Caída del precio del bien
				\4[] Más producción implica menor previo de venta
				\4 Aumento del beneficio nacional
				\4[] Relativamente sorprendente
				\4[] Subsidio aumenta beneficio de la empresa
				\4[] $\to$ Pero cuesta dinero al gobierno
				\4[] Beneficio total nacional:\footnote{Entendiendo como beneficio la simple suma del beneficio de la empresa y la cantidad subvencionada, sin considerar el bienestar de los consumidores.}
				\4[] $\underset{s}{\max} \quad G(s) = \pi(x,y; s) - sx$
				\4[] $\pdv{G}{s} = \pi_s - x - s x_ s = 0$
				\4[] Puede demostrarse\footnote{Ver Brander y Spencer (1985), pág. 89 y 90, ecuación (13) y anteriores.} que $s^* > 0 $
				\4[] $\to$ Si función de reacción de $y$ es decreciente
				\4[] $\to$ Si beneficio de $x$ cae con aumento de $y$
				\4[] $\then$ Subsidio positivo es óptimo
				\4[] $\then$ Equivale a amenaza creíble de aumentar producción
				\4 Caída del beneficio extranjero
				\4[] Subvención a doméstica reduce ventas extranjera
				\4[] $\to$ Cae beneficio extranjero
				\4 Relación relativa de intercambio
				\4[] En contexto de CPerfecta
				\4[] $\to$ Aumento de la producción deteriora RRI
				\4[] $\then$ Caen rentas de factores
				\4[] $\then$ Cae renta nacional
				\4[] En contexto de competencia imperfecta
				\4[] $\to$ Cae precio pero es mayor que CMg
				\4[] $\to$ Aumentan ventas
				\4[] $\then$ Puede inducir aumento neto del beneficio
				\4 Mercado nacional también relevante
				\4[] Asumiendo monopolio en mercado nacional
				\4[] Si costes marginales constantes
				\4[] $\to$ Sin efectos relevantes
				\4[] Si costes marginales decrecientes
				\4[] $\to$ Subsidio aumenta producción
				\4[] $\to$ Más producción reduce costes
				\4[] $\then$ Más probable que el subsidio sea beneficioso
			\3 Valoración
				\4 Simplificación de Spencer y Brander (1983)
				\4 Justificación sencilla de incentivos a exportación
		\2 Eaton y Grossman (1986) -- Competencia à la Bertrand
			\3 Idea clave
				\4 Contexto
				\4[] Similar a anteriores
				\4[] $\to$ Empresas exportadoras
				\4[] $\to$ Interdependencia estratégica
				\4[] Gobiernos gravan positiva y negativamente:
				\4[] $\to$ I+D
				\4[] $\to$ Exportaciones
				\4[] $\to$ Producción
				\4[] Gobiernos tratan de provocar:
				\4[] $\to$ Desviación de rentas
				\4[] $\to$ Variaciones de la RRI
				\4[] Spencer y Brander (1983), Brander y Spencer (1985)
				\4[] $\to$ Conjeturas consistentes y de Cournot
				\4[] $\to$ Subvenciones para liderar à la Stackelberg
				\4[] Pero conjeturas pueden ser:
				\4[] $\to$ No consistentes\footnote{Una conjetura no consistente es aquella que no coincide con la reacción real del agente sobre cuya reacción se realiza la conjetura.}
				\4[] $\to$ Diferentes de la conjetura de Cournot\footnote{Es decir, se puede conjeturar una reacción diferente a una variación de 0 en la cantidad producida ante un aumento en la cantidad producida por la otra empresa.}
				\4 Objetivos
				\4[] Generalizar el análisis
				\4[] $\to$ Conjeturas no consistentes
				\4[] $\to$ Conjeturas de Bertrand
				\4[] $\to$ Oligopolio multifirma
				\4[] $\to$ Entrada y salida endógena
				\4[] $\to$ Consumo doméstico de bienes
				\4 Resultado
				\4[] Subsidios no necesariamente óptimos
				\4[] Impuestos a la exportación pueden serlo
				\4[] Conjeturas consistentes y a la Bertrand
				\4[] $\to$ Laissez-faire es óptimo
			\3 Formulación
				\4 Similar contexto a Brander y Spencer (1985)
				\4[] Dos empresas en dos países
				\4[] Bienes sustitutivos
				\4[] Venden en tercer mercado
				\4[] $\to$ Pero en este caso, fijan precios
				\4 Reacción real diferente a conjeturada
				\4[] Requisito esencial
				\4[] $\to$ Para que intervención sea deseable
				\4[] Si reacción real menor a conjeturada
				\4[] $\to$ Impuesto negativo/subvención
				\4[] Si reacción real mayor a conjeturada
				\4[] $\to$ Impuesto positivo
				\4 En Cournot simple y habitual
				\4[] $\to$ Reacción conjeturada: 0
				\4[] $\to$ Reacción real:\footnote{Supongamos una función de demanda $D(x+y) = 1 -(x +y(x))$. La empresa $x$ maximiza la función de beneficio $\pi(x,y) = \left( 1 - (x+y(x)) \right) x - c(x) $. La condición de primer orden es $1 - 2x - y - y'(x) x - c' = 0$. Reordenando, tenemos que la función de reacción de $x$ es $x = \frac{1-y-c'}{2+y'}$. En un contexto de Cournot, la conjetura de $x$ sobre la respuesta de $y$ a variaciones en su producción es tal que $y'(x)=0$, y de forma análoga respecto a la conjetura de $y$ en relación a la respuesta de $x$ tal que $x'(y)=0$. Es evidente que esta conjetura no coincide con la función de reacción, que implica que $\frac{d \, x}{d \, y} = \frac{d \, y}{d \, x} = -\frac{1}{2}$. Así, tenemos que la reacción real ($-1/2$) es menor a la conjeturada ($0$) y será óptimo introducir un impuesto negativo a las exportaciones o una subvención, de manera concordante con Spencer y Brancer (1983) y Brander y Spencer (1985).} $-\frac{1}{2}$
				\4[] Reacción real es menor a conjeturada
				\4[] $\to$ Subvención es óptima
				\4[] $\then$ Brander y Spencer (1985)
				\4 Conjeturas consistentes
				\4[] Reacción real es igual a reacción conjeturada
				\4[] Por ejemplo, en Bertrand básico
				\4[] $\to$ Reacción conjeturada igual a -1
				\4[] $\to$ Reacción real igual a -1
				\4[] Impuesto a exportación $\uparrow$ costes domésticos
				\4[] Competidor podría fijar precio arbitrariamente menor
				\4[] $\to$ Y llevarse todo el mercado
				\4[] $\then$ Laissez-faire preferible
				\4 En Bertrand con conjeturas inconsistentes
				\4[] Reacción conjeturada igual a -1
				\4[] Reacción real mayor a -1
				\4[] $\then$ Conjeturas inconsistentes
				\4[] $\then$ Reacción real mayor a conjeturada
				\4[] Impuesto a exportación es óptimo
				\4[] $\to$ Fijado antes de que empresas decidan
				\4[] Empresa doméstica puede fija precio más alto
				\4[] $\to$ Amenaza creíble de fijar precios altos
				\4[] $\to$ Gobierno extrae parte de la renta
				\4[] $\then$ Impuesto a exportación es óptimo
				\4[] Representación gráfica
				\4[] \grafica{eatongrossmanbertrand}
			\3 Implicaciones
				\4 Consistencia de conjeturas es fundamental
				\4[] Determina optimalidad de intervención
				\4 Varias empresas en un país\footnote{Ver Eaton y Grossman (1986), pág. 397.}
				\4[] Contexto de múltiples empresas que no coluden
				\4[] $\to$ En un sólo país
				\4[] $\to$ Sin competencia extranjera
				\4[] $\to$ Tratan de vender en tercer mercado
				\4[] Cuando una aumenta producción
				\4[] $\to$ Impone externalidad pecuniaria sobre las otras
				\4[] Gobierno puede forzar equilibrio cooperativo doméstico
				\4[] $\to$ Introduciendo un impuesto a las exportaciones
				\4[] $\then$ Todas se ven forzadas a aumentar precio
				\4[] $\then$ Gobierno extrae renta
			\3 Valoración
				\4 Política comercial vs política industrial
				\4[] Política comercial
				\4[] $\to$ Actuaciones sobre RRI
				\4[] Política industrial
				\4[] $\to$ Actuaciones sobre estructura de costes
				\4[] Ambas pueden desviar beneficios
				\4[] $\to$ Elección depende de conjeturas e instituciones
				\4 Competencia en mercado nacional
				\4[] Relevante sobre efectos de políticas en exterior
				\4 Análisis de entrada y salida
				\4[] También analizado en artículo
				\4[] Libre entrada puede:
				\4[] $\to$ Eliminar beneficios positivos de domésticas
				\4[] $\to$ Especialmente si CFijos suficientemente bajos
				\4[] Costes fijos de entrada suficientemente altos
				\4[] $\to$ Pueden mantener bfcios. positivos de empresas domésticas
				\4[] $\to$ Pueden hacer viable pol. industrial estratégica
				\4 Generalización de resultados de Brander y Spencer
				\4 Análisis superficial de demanda
				\4 Sin caracterización de diferenciación
		\2 Oligopolio verticalmente diferenciado -- Shaked y Sutton (1983)
			\3 Idea clave
				\4 Contexto
				\4[] De gustibus non est disputandum
				\4[] $\to$ No siempre se cumple
				\4[] En ocasiones, calidad es parámetro objetivo
				\4[] $\to$ Todos los agentes están de acuerdo
				\4[] En contexto de comercio internacional
				\4[] $\to$ Posible especialización en determinadas calidades
				\4[] Política arancelaria
				\4[] $\to$ Puede tener efectos sobre calidades producidas
			\3 Formulación
				\4 Utilidad de los consumidores
				\4[] $U_i = u_j \cdot \left( t_i - p_j \right)$
				\4 Empresas
				\4[] Coste de variedad aumenta con calidad $u_j$
				\4[] Producen variedades à la Bertrand
				\4[] $\to$ Precio igual a coste
				\4 Equilibrio
				\4[] Cuotas de mercado de cada variedad
				\4[] $\to$ Depende de relación entre coste marginal y calidad
				\4[] Más calidad aumenta muy poco el coste marginal
				\4[] $\to$ Consumidores prefieren calidades altas
				\4[] $\then$ Menos dispersión de cuotas de mercado
				\4 Especialización de producción de variedades
				\4[] Depende de tecnología disponible
				\4[] Determinados países, producir mejor aumenta coste más
				\4[] $\to$ Especialización en variedades con menos calidad
				\4 Política arancelaria estratégica
				\4[] Puede alterar especialización en diferentes calidades
			\3 Implicaciones
				\4 Efecto de arancel depende de calidad producida
				\4 Economía doméstica produce calidades bajas
				\4[] Arancel aumenta coste de variedades mejores extranjeras
				\4[] $\to$ Mayor consumo de variedades nacionales
				\4[] $\to$ Cae calidad media consumida localmente
				\4[] $\to$ Calidad máxima producida localmente aumenta
				\4 Economía doméstica produce calidades altas
				\4[] Arancel aumenta coste de variedades peores extranjeras
				\4[] $\to$ Mayor consumo de variedadesnacionales
				\4[] $\to$ Calidad media consumida localmente aumenta
				\4[] $\to$ Calidad mínima producida localmente aumenta
			\3 Valoración
		\2 Barret (1994): pol. medioambiental estratégica
			\3 Idea clave
				\4[] Reg. medioambiental
				\4[] $\to$ Puede ser instrumento de PComercial
			\3 Formulación
				\4 Incentivos de PComercial a regulación débil/fuerte
				\4[] Regulación débil entendida como:
				\4[] $\to$ Coste de mitigación menor que daño marginal
				\4[] Depende de:
				\4[] $\to$ Estructura competitiva nacional e internacional
				\4 Competencia à la Cournot
				\4[] Incentivos a establecer reg. medioambiental débil
				\4[] Mantener amenaza de cantidad producida elevada
				\4 Competencia à la Bertrand
				\4[] Incentivos a establecer reg. medioambiental fuerte
				\4[] Reg. medioambiental es amenaza creíble de $\uparrow$ precio
			\3 Implicaciones
				\4 Protección medioambiental tiene consecuencias de CI
				\4 Política MA puede ser instrumento de PComercial
				\4 Estructura del mercado determina PolMA
	\1 \marcar{La política de promoción exterior}
		\2 Idea clave
			\3 Contexto
				\4 Política comercial
				\4[] Alterar relación relativa de intercambio
				\4[] $\to$ Instrumentos arancelarios y no arancelarios
				\4 Políticas de oferta
				\4[] Alterar contexto microeconómico
				\4[] $\to$ Mejorar productividad de empresas nacionales
				\4[] $\to$ Reducir costes de producción
				\4 Política industrial
				\4[] Alterar estructura industrial
				\4 Restricciones de tratados internacionales
				\4[] Políticas anteriores tienen efectos indeseados
				\4[] Dilemas del prisionero
				\4[] Otros equilibrios subóptimos posibles
				\4[] Políticas anteriores muy restringidas
				\4[] $\to$ Aranceles muy bajos
				\4[] $\to$ Cuotas prohibidas en GATT
				\4[] $\to$ Subvenciones a exportación prohibidas
				\4[] $\to$ Ayudas a industria nacional restringidas
				\4[] $\to$ Política industrial restringida en contexto UE
			\3 Objetivos
				\4 Aumentar renta nacional
				\4 Aumentar base exportadora
				\4[] Conjunto de empresas que exportan
				\4 Aumentar propensión a exportar
				\4[] Proporción de empresas nacionales que exportan
				\4 Aumentar valor añadido de exportaciones
				\4 Aumentar diversificación de destinos de exportación
				\4 Mejorar relación de intercambio
				\4 Aumentar IDE recibida
				\4[$\then$] Internacionalizar economía
			\3 Resultados
				\4 Entorno de inversión
				\4[] Aumento de IDE recibida
				\4[] Transferencia tecnológica
				\4[] Introducción de economía en cadenas de valor global
				\4 Competitividad ex-post y no precio
				\4[] Mejora de cuotas de exportación
				\4[] Mejora en índices de competitividad
		\2 Justificación de las políticas de promoción exterior
			\3 Economías de escala
				\4 Mayor tamaño de mercado potencial
				\4[] Aumento de producción y ventas
				\4[] $\then$ Realización de economías de escala
				\4 Capital insuficiente
				\4[] Empresas no pueden afrontar fase inicial
				\4[] $\to$ Son poco competitivas al principio
				\4[] Mercados financieros imperfectos no proveen capital
				\4[] $\to$ Riesgo elevado
				\4[] $\to$ Insuficiente información
				\4 Sin intervención pública
				\4[] Empresas domésticas tienen costes altos
				\4[] $\to$ No realizan economías de escala
				\4[] $\then$ No compiten a nivel internacional
				\4 Intervención del sector público
				\4[] Proveer financiación
			\3 Resistencia al fluctuaciones cíclicas
				\4 Acceso a mercados de exportación diversificados
				\4[] Amortiguar efecto de shocks idiosincráticos
				\4 Evidencia empírica en crisis
				\4[] Empresas exportadoras soportan mejor la crisis
				\4[] $\to$ Ingresos de exportación más resistentes
				\4 Intervención pública pro-internacionalización
				\4[] Suavizar fluctuaciones macroeconómicas
			\3 Externalidades
				\4 Exportación e internacionalización tiene spill-overs
				\4[] Sobre otras empresas domésticas
				\4[] $\to$ Contactos e información
				\4[] $\to$ Eslabonamientos
				\4[] $\to$ Transferencia tecnológica
				\4 Empresas individuales pueden no internalizar externalidad
				\4[] No invierten en internacionalización
				\4[] $\to$ Beneficio total subóptimo
			\3 Costes hundidos
				\4 Exportación requiere costes fijos
				\4[] Cuando actividad exportadora se consolida
				\4[] $\to$ Se convierten en hundidos
				\4 Incumbentes ya han superado costes hundidos
				\4[] Menores costes
				\4 Potenciales exportadores no pueden competir
				\4 SP puede facilitar financiación
				\4[] Superar costes hundidos de competidores
			\3 Redes sociales
				\4 CI requiere a menudo de contactos en destinos
				\4 Insuficientes incentivos al establecimiento de contactos
				\4 Intermediación necesaria para emparejar exportado-importador
				\4 Sector público puede jugar papel de intermediador
			\3 Barreras de entrada
				\4 Explícitas
				\4[] Aranceles, cuotas, restricciones
				\4[] $\to$ Objeto de negociación comercial bi/pluri/multilateral
				\4 Implícitas
				\4[] Regulaciones complejas
				\4[] Estándares de calidad
				\4 SPúblico puede informar sobre regulación y estándares
				\4[] Más beneficiarios reduce coste
				\4[] Embajadas y oficinas consulares como punto de partida
			\3 Path-dependency/Dependencia de senda
				\4 Factores históricos afectan a presente
				\4[] Redes sociales (in)existentes
				\4[] Imagen de marca
				\4[] Propensión a asociar producto a país
				\4[] Implantación de competidores
				\4[] $\then$ Impacto negativo sobre beneficios esperados
				\4 SP puede proveer
			\3 Escasez de capital humano en internacionalización
				\4 Exportación requiere conocimiento específico
				\4[] Legislación internacional y país de destino
				\4[] Técnicas comerciales
				\4[] Idioma local
				\4 Insuficientes incentivos para formar en empresa
				\4[] Capital humano no suficientemente específico a la empresa
				\4[] Cuantos más exportadores, menos especificidad
				\4 Sector público puede contribuir a formación de capital
		\2 Instrumentos no financieros
			\3 Regulatorios
				\4 Reducir carga burocrática de IDE
				\4 Simplificar trámites de exportación
				\4[] Especialmente vía acuerdos con país de destino
			\3 Formación
				\4 Programas de formación de capital humano
				\4 Ejemplo en España:
				\4[] Becas ICEX
			\3 Información
				\4 Proporcionar información sobre destino
				\4[] Regulación
				\4[] Costumbres
				\4[] Estudios de mercado
				\4[] Análisis macroeconómico
				\4[] Riesgos políticos
			\3 Promoción
				\4 Organización de ferias comerciales
				\4 Marketing y publicidad
				\4 Construcción de marca-país
				\4 Subvención de misiones y misiones inversas
				\4 Organización de seminarios y foros de inversión
			\3 Diplomacia económica
				\4 Utilización de capacidad de influencia del estado
				\4[] A nivel político, militar y económico
				\4[] $\to$ Apoyo de intereses económicos
				\4 Presión mediante diplomacia comercial
				\4[] Concesión de contratos y preferencias
				\4[] $\to$ A cambio de contrapartidas en otras áreas
				\4 Influencia de factores:
				\4[] Históricos
				\4[] Militares
				\4[] Geoestratégicos
				\4 Participación del SP en economía de destino
				\4[] Aumenta efectividad de diplomacia económica
			\3 Provisión de consultoría y asesoramiento
				\4 De forma gratuita o subvencionada
				\4 Aprovechamiento de economías de escala
				\4[] A nivel de proveedor de consultoría y asesoramiento
				\4 También en contexto de disputas comerciales
				\4 Colaboración con proveedores privados
				\4[] ``Business intelligence''
				\4 Ejemplo en España:
				\4[] ICEX
		\2 Instrumentos financieros de promoción exterior
			\3 Idea clave
				\4 Componentes de una oferta de exportación
				\4[] Oferta técnica
				\4[] $\to$ Características esenciales de lo exportado
				\4[] Oferta comercial
				\4[] $\to$ Precio de lo exportado
				\4[] Oferta financiera
				\4[] $\to$ Condiciones de financiación de la compra
				\4 Permitir a exportador mejorar oferta financiera
				\4 Especialmente relevante para PYMES
				\4[] Más difícil acceso a mercados financieros
				\4[] Menor solvencia y mayores riesgos
			\3 Créditos directos a exportación
				\4 SP provee financiación a empresa
				\4 Empresa financia venta a importador en destino
				\4 Liquidez tras financiar exportación
				\4 Habitualmente canalizados por intermediario
				\4[] Fondos públicos
				\4[] IFs públicas y privadas
				\4 Ejemplo en España:
				\4[] Línea ICO Internacional
			\3 Garantías para créditos a exportación
				\4 Sector financiero financia a empresa
				\4 Financiación privada exige garantía de pago
				\4 Sector público provee financiación
				\4 ECAs -- Export Credit Agencies
				\4 Ejemplo en España:
				\4[] CESCE
			\3 Avales
				\4 Determinados proyectos exigen aval
				\4[] Licitador debe aportar aval
				\4[] $\to$ Garantizar cumplimiento
				\4[] $\to$ Asegurar posibles imprevistos
				\4 SP facilita avales en proyectos internacionales
				\4 Ejemplo en España:
				\4[] Avales de CESCE
			\3 Capital para inversión
				\4 Empresas requieren capital de l/p
				\4[] Establecerse en país de destino
				\4[] Plantas de producción para exportar
				\4 SP puede participar en capital de empresas
				\4[] Participación temporal en accionariado
				\4[] Préstamos de largo plazo
			\3 Determinantes de volumen de financiación
				\4 Dotación de fondos concesionales
				\4 Límites a seguro de crédito a exportación
				\4[] Techos-país
				\4[] $\to$ Limitar exposición a un país determinado
				\4[] Techo-operación
				\4[] $\to$ Limitar exposición en una operación concreta
			\3 Determinantes de condiciones de financiación
				\4 \% de fondos disponibles destinada a concesional
				\4 Condiciones en las que se otorga crédito concesional
				\4 Condiciones de apoyo a la inversión
	\1 \marcar{Indicadores de competitividad}
		\2 Idea clave
			\3 Concepto de competitividad
				\4 Definición genérica
				\4[] Capacidad para competir en mercados internacionales
				\4 Diferentes concepciones de competitividad
				\4[] Medida de la ventaja comparativa
				\4[] Capacidad para generar niveles altos
				\4[] $\to$ De ingresos y empleo elevados de forma sostenida
				\4[] $\then$ Compitiendo internacionalmente
				\4 Factor determinante de cuenta corriente
				\4[] Capacidad para colocar ByS en exterior
				\4[] Demanda de ByS extranjeros frente a nacionales
			\3 Medición de la competitividad
				\4 Dificultades de medición
				\4[] Dependencia de la definición
				\4[] Problemas de datos
				\4[$\then$] Uso de indicadores cuantitativos
				\4[] Agregar variables en una
		\2 Indicadores de competitividad-precio/coste
			\3 Idea clave
				\4 Carácter ex-ante
				\4[] Valorar competitividad antes de intercambio
				\4 Evolución comparativa de costes y precios
				\4[] Respecto a competidores o resto del mundo
				\4 Instituciones que compilan
				\4[] Banco Central Europeo
				\4[] $\to$ Indicadores de competitividad armonizados
				\4[] Informe trimestral sobre precios y costes
				\4[] $\to$ Comisión Europea
				\4[] Banco de España
				\4[] $\to$ Índices de competitividad
				\4[] Secretaria de Estado de Comercio
				\4[] $\to$ Índices de Tendencia de Competitividad
				\4 Tipo de cambio efectivo real
				\4[] En lo fundamental, son equivalentes
				\4 Valoración indirecta de competitividad
				\4[] Precios y costes pueden aumentar
				\4[] $\to$ Pero que aumente calidad y sofisticación
				\4[] $\then$ Y mejore comp. aunque aumente precio
				\4[] $\then$ Necesario completar con otros indicadores
			\3 Metodología
				\4 En lo fundamental, TCER
				\4 Elementos de variación
				\4[] Grupo de socios comerciales
				\4[] $\to$ Deseable represente a competidores
				\4[] Ponderación de los tipos de cambio
				\4[] $\to$ Peso a asignar a TC de cada país
				\4[] Deflactores
				\4[] $\to$ Muy amplias posibilidades
				\4[] $\then$ IPC, IValorUnitarioExportación
				\4[] $\then$ CLUs, deflactor de PIB, IPIndustriales...
				\4[] Fórmula de cálculo
				\4[] $\to$ Multiplicación de índices de precios relativos usados
				\4[] $\then$ Ejemplo: $\text{ITC} = \frac{\text{IPR} \cdot \text{IPX}}{100}$
			\3 Índice de Tendencia de Competitividad
				\4 Formulación
				\4[] \fbox{$\text{ITC}_t^a = \frac{\text{IPX}_t^a \cdot \text{IPR}_t^a}{100}$}
				\4 IPX -- Índice de Tipo de Cambio
				\4[] \fbox{$\text{IPX}_t^a = 100 \cdot \Pi_{i=1}^I \left( \frac{1}{\text{tc}_{it}} \right)^{n_i}$}
				\4[] $\to$ $n_i$: ponderación normalizada de cada moneda, $\sum_i n_i = 1$
				\4[] $\to$ $\text{tc}_{it}$: tipo de cambio directo
				\4[] $\then$ Media geométrica ponderada de TC bilaterales
				\4[] $\then$ Aumento (caída) de IPX indica apreciación (depreciación)
				\4 IPR -- Índice de Precios Relativos
				\4[] \fbox{$\text{IPR}_t^a = 100 \cdot \frac{\text{IPC}_\text{España,t}^a}{\Pi_{i=1}^I \left( \text{IPC}_{i,t}^a \right)^{n_i}}$}
				\4[] $\then$ Relación entre IPC de España...
				\4[] $\then$ ...y media geométrica ponderada de IPCs de referencia
			\3 Ventajas e inconvenientes
				\4 Dependientes de deflactor usado
				\4 Índices de precios al consumo
				\4[] Mejor calidad de datos
				\4[] Buena comparabilidad
				\4[] Alta frecuencia
				\4[] Pocas revisiones posteriores
				\4[] Problemas
				\4[] $\to$ Excluyen bienes comercializables de capital
				\4[] $\to$ Sensibles a impuestos indirectos
				\4[] $\to$ Sólo indirectamente relacionados con costes de producción
				\4 Costes laborales unitarios
				\4[] Sujetos a volatilidad
				\4[] Cambios más significativos:
				\4[] $\to$ Especificidades en estadísticas de salarios y empleo
				\4[] No recogen:
				\4[] $\to$ Evolución de costes asociados a producción
				\4[] $\to$ Precio final también afectado por margen empresarial
				\4 Indicadores basados en IVUs\footnote{Índices de Valor Unitario.}
				\4[] Aproximación de precios de operaciones de X y M
				\4[] Fáciles de obtener con datos aduaneros
				\4[] No recogen precios directamente
				\4[] Sujetos a efecto composición
				\4[] $\to$ Cuando cambia estructura relativa del XyM
				\4 Deflactores del PIB
				\4[] Muy fáciles de calcular
				\4[] Sujetos a volatilidad de series trimestrales
				\4[] $\to$ Especialmente en economías pequeñas
				\4 Desventaja común salvo IVUs
				\4[] Inclusión de ByS no comercializables internacionalmente
				\4[] $\to$ Que sólo afectan indirectamente a competitividad
				\4 Efectividad de indicadores
				\4[] Índices basados en IPC
				\4[] $\to$ Aproximan cerca del 80\% de $\Delta$ de X
				\4[] Ligeramente inferior para CLUs
		\2 Indicadores de competitividad revelada
			\3 Idea clave
				\4 Carácter ex-post
				\4[] Una vez se ha producido el intercambio
				\4 Evolución comparativa de cuota y volúmenes
				\4[] En un periodo determinado
				\4 Resultado de interacción de múltiples factores
				\4[] Todos los que influyen en los intercmabios
			\3 Cuota de España en exportaciones mundiales de bienes
				\4 Cuota mantenida estable desde 2000
				\4[] $\to$ A pesar de aumento de emergentes
				\4[] $\to$ A pesar de China
				\4[] $\to$ A pesar de deterioro de costes relativos
				\4 Sólo Alemania también logra mantener
				\4[] RU, FRA, ITA pierden cuota claramente
				\4 Explicaciones a ``paradoja española''
				\4[] Cambios en la composición de la oferta
				\4[] $\to$ Buena adaptación a demanda mundial
				\4[] Bienes de buena calidad-precio
				\4[] Diversificación de la oferta
				\4[] Diversificación geográfica
				\4[] Aumento de la base exportadora
				\4[] $\to$ Margen extensivo
				\4[] Aumento de las ventas de exportación
				\4[] $\to$ Margen intensivo
			\3 Cuota de España en exportaciones de bienes de UE
				\4 Senda más favorable que exportaciones mundiales
				\4[] Cuota ha crecido hasta el 5,3\%
				\4[] $\to$ No sólo se ha mantenido estable
			\3 Cuota de España en exportaciones de servicios
				\4 Pérdida en últimas dos décadas
				\4 Exportaciones han crecido
				\4[] Pero menos que exportaciones mundiales
				\4 España es potencia mundial en servicios
				\4[] Pero emergentes han crecido mucho
		\2 Otros indicadores
			\3 Calidad institucional
				\4 Índice de competitividad global
				\4 Publicado por WEForum
				\4 Agregación de 114 indicadores
				\4[] Agrupados en 12 pilares
				\4[] Tres tipos de economías
				\4[] $\to$ Economías centradas en ff.pp.
				\4[] $\to$ Economías centradas en eficiencia
				\4[] $\to$ Economías centradas en innovación
				\4[] Ponderados según grado de desarrollo de economía
			\3 Doing business
				\4 Publicado por el Banco Mundial
				\4 Informe anual
				\4 Agregación de indicadores sobre
				\4[] Actividades de PYMES locales
				\4[] $\to$ ¿Qué normas les afectan en su actividad?
				\4 Mejora apreciable en últimos 10 años
			\3 Otros
				\4 World Competitiveness Ranking
				\4[] PUblicado por escuela de negocios IMD
				\4 Complejidad económica
				\4[] Know-how acumulado
				\4[] Catálogo de productos y servicios
				\4[] $\to$ Que un país es capaz de exportar
				\4 Indicadores mundiales de gobernanza
				\4 Competitividad del sector turístico
				\4[] Publicado por WEF
				\4[] Encabeza el ranking actualmente
				\4[] Ascenso en los últimos años
	\1[] \marcar{Conclusión}
		\2 Recapitulación
			\3 Política comercial estratégica
			\3 Política de promoción exterior
		\2 Idea final
			\3 Contexto internacional
				\4 Consolidación progresiva de grandes bloques
				\4 Guerra comercial posible
			\3 Unión Europea
				\4 Política comercial competencia exclusiva
				\4
			\3 Relación con otras áreas
\end{esquemal}




\graficas 

\begin{tabla}{Europa puede subvencionar la entrada de Airbus en un tercer mercado. Estados Unidos no puede subvencionar la entrada de Boeing. }{prisionerodesarme}

	\begin{tabular}{c || l | c | c}
		& \multicolumn{3}{c}{País B} \\ 
		& & \textbf{Desarme} & \textbf{Mantiene} \\ \hline \hline
	\multirow{2}{*}{País A} &\textbf{Desarme} & $(a,a)$ & $(b,c)$  \\ 
			& \textbf{Mantiene} & $(c,b)$ & \textbf{$(d,d)$} \\ \hline
	\end{tabular} 
	Con $c>a>d>b$, el equilibrio de Nash
\medskip
\end{tabla}

\begin{tabla}{Europa puede subvencionar la entrada de Airbus en un tercer mercado. Estados Unidos no puede subvencionar la entrada de Boeing. }{ungobierno}
	\begin{tabular}{c || l | c | c}
		& \multicolumn{3}{c}{Airbus} \\ 
		& & \textbf{C} & \textbf{NC} \\ \hline \hline
	\multirow{2}{*}{Boeing} &\textbf{C} & $(-5,-5)$ & $(50,0)$  \\ 
		& \textbf{NC} & $(0,50)$ & $(0,0)$ \\ \hline
	\end{tabular} 
\medskip

Europa no subvenciona entrada de Airbus.

\bigskip

	\begin{tabular}{c || l | c | c}
	& \multicolumn{3}{c}{Airbus} \\ 
	& & \textbf{C} & \textbf{NC} \\ \hline \hline
	\multirow{2}{*}{Boeing} &\textbf{C} & $(-5,1)$ & $(50,0)$  \\ 
	& \textbf{NC} & $(0,56)$ & $(0,0)$ \\ \hline
\end{tabular} 
\medskip 

Europa subvenciona entrada de Airbus.
\end{tabla}

Cuando Europa no subvenciona la entrada de Airbus, existen dos ENEP: una empresa entra y la otra no lo hace. Sin embargo, cuando Europa subvenciona la entrada de Airbus, existe un sólo ENEP tal que Airbus entra pero Boeing se mantiene fuera del mercado. 


\begin{tabla}{Ambos gobiernos --Europa y Estados Unidos- pueden subvencionar la entrada de Airbus y Boeing --respectivamente- en un tercer mercado de aeronaves. }{dosgobiernos}
	\begin{tabular}{c || l | c | c}
		& \multicolumn{3}{c}{Airbus} \\ 
		& & \textbf{C} & \textbf{NC} \\ \hline \hline
		\multirow{2}{*}{Boeing} &\textbf{C} & $(10,10)$ & $(35,5)$  \\ 
		& \textbf{NC} & $(5,35)$ & $(25,25))$ \\ \hline
	\end{tabular} 
	\medskip
\end{tabla}

Cuando ambos gobiernos pueden subvencionar a sus empresas respectivas, existe un sólo ENEP: aquel en el que ambas empresas compiten. El resultado es compatible con un contexto de dilema del prisionero en el que el ENEP es subóptimo de Pareto.


\begin{axis}{4}{Efecto de un arancel a las importaciones en un contexto de Brander y Spencer (1981): el incumbente reduce su producción y permite la entrada al entrante potencial doméstico.}{}{}{branderspencer81}
	% Expansión de eje de abscisas
	\draw[-] (4,0) -- (8,0);
	\node[below]  at (8,0){$q_L$};
	\draw[-] (0,4) -- (0,8);

	\node[left]  at (0,8){$q_S$};

	% Curvas de indiferencia de líder/incumbente
	%\draw[-] (0.6,2.3) to [out=40, in=140](2.6,2.3);
	\draw[-] (1.05,1.25) to [out=40, in=140](3.38,1.25);

	% Curvas de reacción de líder/incumbente
	\draw[-] (0,6) -- (3,0);
	\node[right] at (0.5,5.5){\tiny $q_L(q_S)$};
	\node[left] at (0,6){\tiny $A-c$};

	% Curvas de indiferencia de seguidora/entrante
	%\draw[-] (0.5,3.5) to [out=-50, in=50](0.5,1.5);
	
	% Curva de indiferencia con beneficio 0
	\draw[-] (2.65,2.5) to [out=-50, in=50](2.65,0.5);
	\node[above] at (2.65,2.5){\tiny $\pi_s < 0$};
	
	% Curva de indiferencia con beneficio positivo
	\draw[-] (1.15,3.25) to [out=-50, in=50](1.15,1.25);
	\node[above] at (0.85,3.2){\tiny $\pi_s > 0$};
	
	% Curva de reacción de seguidora/entrante
	\draw[-] (0,3) -- (6,0);
	\node[right] at (5.5,0.5){\tiny $q_S(q_L)$};

	% Curva de indiferencia de líder/entrante tras arancel
	\draw[dashed] (-0.1,1.98) to [out=40, in=140](1.9,1.98);

	% Curva de reacción de incumbente tras arancel
	\draw[dashed] (0,4) -- (2,0); 
	\node[left] at (0,4){\tiny $A-c-t$};

	% Equilibrio pre-arancel
	\node[circle, fill=black, inner sep=0pt, minimum size=5pt] (a) at (3,1.5) {};
	
	% Equilibrio pos-arancel
	\node[circle, fill=black, inner sep=0pt, minimum size=5pt] (a) at (1.5,2.2) {};
\end{axis}

\begin{axis}{4}{Equilibrios en un contexto de Spencer y Brancer (1983) cuando un país puede aplicar una subvención al I+D o las exportaciones.}{$q_1$}{$q_2$}{spencerbrancersubsidiounpais}
	% Función de reacción de 1 antes del subsidio
	\draw[-] (0,3.5) -- (2,0);
	\node[right] at (0.05,2.2){\tiny $q_1(q_2)$};
	
	% Función de reacción de 1 después del subsidio
	\draw[dashed] (0,4.5) -- (2.57,0);
	\node[right] at (1.3,2.2){\tiny $q_1^S(q_2)$};
	
	% Función de reacción de 2
	\draw[-] (0,2) -- (3.5,0);
	\node[right] at (3,0.4){\tiny $q_2(q_1)$};
	
	% Equilibrio inicial
	\node[circle, fill=black, inner sep=0pt, minimum size=5pt] (a) at (1.27,1.27) {};
	\node[left] at (1.27,1.2){E};
	
	% Equilibrio tras la aplicación de la subvención
	\node[circle, fill=black, inner sep=0pt, minimum size=5pt] (a) at (2.1,0.8) {};
	\node[right] at (2.1,0.9){S};
	
\end{axis}

La línea discontinua muestra la función de reacción de las empresas ante cambios en la producción de las otras empresas respectivas. La línea discontinua muestra la función de reacción de la empresa 1 tras las implementación de una subvención a la I+D o a la exportación. En el nuevo equilibrio, la empresa que se beneficia de la subvención produce más y la empresa que no lo recibe, menos.

\begin{axis}{4}{Política óptima en un contexto de conjeturas inconsistentes à la Bertrand en las que la reacción real es mayor que la conjeturada: un impuesto a la exportación aumenta el bienestar.}{$p_x$}{$p_y$}{eatongrossmanbertrand}
	% Curvas de indiferencia de empresa x
	\draw[-] (0.5, 1.5) to [out=-45, in=180](2,1) to [out=0, in=230](3.6,1.6);
	\node[left] at (0.5,1.5){\tiny $u_x^0$};

	\draw[-] (0.6, 2.25) to [out=-45, in=180](2,1.75) to [out=0, in=230](3.6,2.35);
	\node[left] at (0.6,2.25){\tiny $u_x^1$};
	
	\draw[-] (0.7, 3) to [out=-45, in=180](2,2.5) to [out=0, in=230](3.7,3.1);
	\node[left] at (0.7,3){\tiny $u_x^2$};
	
	% Reacción de empresa x
	\draw[-] (1.7,0.2) -- (2.7,3.9);
	\node[above] at (2.7,3.9){$x$};
	
	% Reacción de empresa x tras impuesto a la exportación
	\draw[dashed] (2.7,0.2) -- (3.7,3.9);
	\node[above] at (3.4,3.9){$x'$};
	
	% Reacción de empresa y
	\draw[-] (0.5,0.45) -- (4.2,3.45);
	\node[right] at (4.2,3.45){$y$};
	
	% Equilibrio inicial
	\node[circle, fill=black, inner sep=0pt, minimum size=3pt] (a) at (2.12,1.75) {};
	\node[left] at (2.12,1.85){\tiny E};
	
	% Equilibrio final
	\node[circle, fill=black, inner sep=0pt, minimum size=3pt] (a) at (3.41,2.82) {};
	\node[left] at (3.41,2.84){\tiny S};
	
\end{axis}


\preguntas

\seccion{Test 2014}

\textbf{30.} En un modelo oligopolístico de bienes homogéneos, la política comercial estratégica óptima desde el punto de vista del país que lo aplica es:
\begin{itemize}
	\item[a] Introducir una subvención a la exportación si la competencia es según el modelo de Bertrand.
	\item[b] Introducir una subvención a la exportación si la variación conjetural es cero.
	\item[c] Introducir un impuesto sobre las exportaciones si las empresas compiten según el modelo de Cournot.
	\item[d] No intervenir el mecanismo de precios.
\end{itemize}

\seccion{Test 2005}

\textbf{27.} Las políticas comerciales estratégicas se han tendido a aplicar por lo general:
\begin{itemize}
	\item[a] En los países avanzados, en sectores de elevado valor añadido y productividad, donde las empresas pueden tener problemas a la hora de apropiarse de los resultados de su inversión en I+D.
	\item[b] En los países emergentes, en sectores tradicionales de escasa productividad y sometidos a una creciente competencia salarial por parte de los países pobres.
	\item[c] En los países en desarrollo, en sectores de elevado valor añadido y productividad, con objeto de desarrollar dichos sectores hasta que sean capaces de hacer frente a la competencia internacional.
	\item[d] En los países avanzados, en sectores tradicionales de escasa productividad y sometidos a una creciente competencia salarial por parte de los países en desarrollo.
\end{itemize}

\notas

\textbf{2014:} \textbf{30.} B

\textbf{2005:} \textbf{27.} A

\bibliografia

Mirar en Palgrave:
\begin{itemize}
	\item antidumping *
	\item countertrade *
	\item foreign trade
	\item globalization and the welfare state *
	\item infant-industry protection *
	\item international outsourcing *
	\item international trade theory
	\item non-price competition *
	\item optimal tariffs
	\item regional and preferential trade agreements
	\item non-tariff barriers
	\item sanctions and export deflation
	\item strategic trade policy *
	\item tariff versus quota
	\item trade and poverty
	\item trade cycle
	\item trade costs
\end{itemize}

Barrett, S. (1994) \textit{Strategic environmental policy and international trade} Journal of Public Economics -- En carpeta del tema

Brander, J.; Brander, J. (1983) \textit{International R \& D Rivalry and Industrial Strategy} \textit{Review of Economic Studies} -- En carpeta del tema

Brander, J.; Spencer, B. (1985) \textit{Export subsidies and international market share rivalry} Journal of International Economics  -- En carpeta del tema


Eaton, J.; Grossman, G. (1986) \textit{Optimal Trade and Industrial POlicy under Oligopoly} The Quarterly Journal of Economics -- En carpeta del tema

Eaton, J.; Kierzkowski. H. (1984) \textit{Oligopolistic Competition, Product Variety, Entry Deterrence, and Technology Transfer} The RAND Journal of Economics -- En carpeta del tema

\end{document}
