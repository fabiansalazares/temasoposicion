\documentclass{nuevotema}

\tema{3B-13}
\titulo{Mecanismos de ajuste de la balanza de pagos. Especial referencia al enfoque intertemporal de balanza de pagos. Análisis de sostenibilidad del déficit y de la deuda exterior.}

\begin{document}

\ideaclave

\seccion{Preguntas clave}
\begin{itemize}
	\item ¿Qué es la balanza de pagos?
	\item ¿Cómo se ajusta la balanza de pagos a un equilibrio determinado?
	\item ¿Qué mecanismos de ajuste existen?
	\item ¿Qué modelos representan el funcionamiento de esos mecanismos?
	\item ¿Qué es el enfoque intertemporal de la balanza de pagos?
	\item ¿Quiénes prestan y quienes piden prestado?
	\item ¿Qué conclusiones normativas se derivan?
	\item ¿Qué condiciones son necesarias para que la deuda exterior sea sostenible?
	\item ¿Qué papel juega el déficit en la sostenibilidad de la deuda exterior?
\end{itemize}

\esquemacorto

\begin{esquema}[enumerate]
	\1[] \marcar{Introducción}
		\2 Contextualización
			\3 Flujos econ. y fin. internacionales
			\3 Balanza de pagos
			\3 Ajuste de la balanza de pagos
		\2 Objeto
			\3 ¿Cómo se alcanza un determinado eq. de BP?
			\3 ¿Qué mecanismos de ajuste hacia el eq. son relevantes?
			\3 ¿Qué modelos representan el proceso de ajuste?
			\3 ¿En qué consiste el enfoque intertemporal de la BP?
			\3 ¿Qué es la sostenibilidad de la deuda exterior?
			\3 ¿Qué efecto tiene el déficit en la sostenibilidad de la deuda?
			\3 ¿Qué condiciones deben cumplirse para que la deuda sea sostenible?
		\2 Estructura
			\3 Enfoque de flujos
			\3 Enfoque de stocks
			\3 Enfoque intertemporal
			\3 Análisis de sostenibilidad
	\1 \marcar{Enfoque de flujos}
		\2 Idea clave
			\3 Equilibrio de flujos
			\3 Sólo cuenta comercial
			\3 Contexto teórico
		\2 Elasticidades
			\3 Idea clave
			\3 Formulación
			\3 Implicaciones
			\3 Valoración
		\2 Enfoque de absorción
			\3 Idea clave
			\3 Formulación
			\3 Implicaciones
			\3 Valoración
	\1 \marcar{Enfoque de stocks}
		\2 Idea clave
			\3 Equilibrio de stocks
			\3 Estabilidad
			\3 Contexto teórico
		\2 Modelo de flujo-especie
			\3 Idea clave
			\3 Formulación
			\3 Implicaciones
			\3 Valoración
		\2 Enfoque monetario
			\3 Idea clave
			\3 Formulación
			\3 Implicaciones
			\3 Valoración
		\2 Enfoque de cartera
			\3 Idea clave
			\3 Formulación
			\3 Implicaciones
			\3 Valoración
	\1 \marcar{Enfoque intertemporal y sostenibilidad}
		\2 Idea clave
			\3 Equilibrio intertemporal
			\3 Microfundamentación
			\3 Contexto teórico
		\2 Modelo simple de dos periodos
			\3 Idea clave
			\3 Formulación
			\3 Implicaciones
		\2 Modelo de horizonte infinito estocástico
			\3 Idea clave
			\3 Formulación
			\3 Implicaciones
		\2 Sostenibilidad de la deuda exterior
			\3 Idea clave
			\3 Formulación
			\3 Implicaciones
	\1 \marcar{Crisis de balanza de pagos}
		\2 Idea clave
			\3 Contexto
			\3 Objetivos
			\3 Resultados
		\2 Formulación
			\3 Identidades del ahorro, la inversión y la entrada de capital
			\3 Sudden stops y reversiones de flujos de capital
			\3 Factores de riesgo de sudden stops
		\2 Implicaciones
			\3 Flujos de capital pueden ser desestabilizantes
			\3 Liberalización de CF puede tener inconvenientes
			\3 Sistema financiero doméstico es factor clave
			\3 Uniones monetarias requieren mecanismos emergencia
		\2 Valoraciones
			\3 Suceso recurrente
			\3 Papel clave del FMI
	\1[] \marcar{Conclusión}
		\2 Recapitulación
			\3 Enfoque de flujos
			\3 Enfoque de stocks
			\3 Enfoque intertemporal y sostenibilidad
		\2 Idea final
			\3 Impacto de los modelos analizados
			\3 Impacto de análisis de BP en policy-making
			\3 Economía política

\end{esquema}

\esquemalargo












\begin{esquemal}
	\1[] \marcar{Introducción}
		\2 Contextualización
			\3 Flujos econ. y fin. internacionales
				\4 Cantidades crecientes
				\4 Diferentes características
				\4 Comercio de bienes y servicios
				\4 Rentas primarias
				\4 Transferencias corrientes, permantes..
				\4 Flujos financieros
			\3 Balanza de pagos
				\4 Estado estadístico que resume flujos
				\4[] Reales y de carácter financiero
				\4 Entre residentes y no residentes
				\4[] Desde el punto de vista de un país determinado
				\4 En un periodo temporal
				\4[] La balanza de pagos mide flujos
				\4 Expresada en términos de cuentas
				\4[] Sujeto a métodos contables
				\4 Historia
				\4[] Primera edición del manual en 1948
				\4[] Sexta y última edición: 2008
				\4[] $\to$ Revisiones menores respecto a 5a edición
				\4 Estructura funcional básica
				\4[] Cuenta corriente
				\4[] $\to$ Bienes y servicios
				\4[] $\to$ Rentas primarias
				\4[] $\to$ Rentas secundarias
				\4[] Cuenta de capital
				\4[] $\to$ Transferencias de capital
				\4[] $\to$ Adq. ces. ANFNP
				\4[] Cuenta financiera
				\4[] $\to$ Inversión directa
				\4[] $\to$ Inversión directa
				\4[] $\to$ Derivados financieros
				\4[] $\to$ Otra inversión
			\3 Ajuste de la balanza de pagos
				\4 Equilibrio contable
				\4[] Siempre se alcanza
				\4[] Resulta de la propia definición de sus elementos
				\4[] $\to$ Útil como descripción de transacciones
				\4[] $\to$ No permite explicar ni predecir
				\4 Equilibrio económico
				\4[] Existen:
				\4[] $\to$ restricciones cuantitativas algunos flujos
				\4[] $\to$ relaciones entre flujos y stocks
				\4[] $\to$ expectativas y decisiones de agentes
				\4[] $\then$ Eq. de BP que alcanzados no son arbitrarios
				\4[] $\then$ Dependen de mecanismos de ajuste
				\4 Mecanismos de ajuste
				\4[] Resultan de varios factores
				\4[] $\to$ Nivel de precios
				\4[] $\to$ Tipo de cambio
				\4[] $\to$ Demanda agregada
				\4[] $\to$ Absorción
				\4[] $\to$ Elasticidades de demanda y oferta
				\4[] $\to$ Oferta monetaria
				\4[] $\to$ Stock de deuda
				\4[] $\to$ Tipos de interés
				\4[] $\to$ Restricciones de capital y cuenta corriente
				\4[] $\to$ ...
		\2 Objeto
			\3 ¿Cómo se alcanza un determinado eq. de BP?
			\3 ¿Qué mecanismos de ajuste hacia el eq. son relevantes?
			\3 ¿Qué modelos representan el proceso de ajuste?
			\3 ¿En qué consiste el enfoque intertemporal de la BP?
			\3 ¿Qué es la sostenibilidad de la deuda exterior?
			\3 ¿Qué efecto tiene el déficit en la sostenibilidad de la deuda?
			\3 ¿Qué condiciones deben cumplirse para que la deuda sea sostenible?
		\2 Estructura
			\3 Enfoque de flujos
			\3 Enfoque de stocks
			\3 Enfoque intertemporal
			\3 Análisis de sostenibilidad
	\1 \marcar{Enfoque de flujos}
		\2 Idea clave
			\3 Equilibrio de flujos
				\4 Equilibrio de BP depende de vars. flujo
				\4[] Flujos deben tomar determinado valor
				\4[] $\to$ Para que BP tienda a equilibrio
				\4[] $\then$ Flujos se ajustan para inducir eq. de BPAj
				\4[] Se ven afectadas por diferentes factores
				\4[] $\to$ Demanda nacional/absorción
				\4[] $\to$ Demanda de importaciones
				\4[] $\to$ Oferta de exportaciones
				\4[] $\to$ Excesos de demanda de divisas
				\4[] ...
				\4 Stocks no se tienen en cuenta
				\4[] Dinero
				\4[] Reservas
				\4[] Activos financieros
				\4[] ...
			\3 Sólo cuenta comercial
				\4 Composición de CF no se tiene en cuenta
				\4 En general, CC con saldo 0 es equilibrio
				\4[] $\then$ Cuenta financiera también estará en equilibrio
				\4[] Posibles otros equilibrios
				\4[] $\to$ Eq. represente $\Delta$ de reservas fija
				\4 Bienes y servicios para simplificar
				\4[] No consideramos rentas ni CCapital
				\4[] Sin pérdida de generalidad
			\3 Contexto teórico
				\4 Primeras décadas del siglo XX
				\4[] Inestabilidad del sistema monetario
				\4[] Patrón oro post-I GM
				\4[] Reservas en oro y divisas internacionales
				\4[] $\to$ Otros activos de deuda escasos
				\4 Bretton Woods
				\4[] Restricciones de la cuenta financiera
				\4[] Devaluaciones controladas en marco FMI
				\4[] $\to$ ``desequilibrio fundamental de BP''
				\4[] $\then$ Tipo de cambio como objeto de intervención
				\4[] $\then$ Tipo de cambio para alcanzar equilibrio
				\4 Equilibrio interno y externo
				\4[] Objetivos macroeconómicos centrales
				\4[] Equilibrio externo
				\4[] $\to$ Determinado saldo de cuenta corriente
				\4[] $\to$ CC con saldo nulo es caso particular
				\4[] Equilibrio interno
				\4[] $\to$ Pleno empleo
				\4 Autores relevantes
				\4[] Bickerdicke (1920), Robinson (1937), Keynes (1933)
				\4[] Lerner (1944), Machlup (1943), Alexander (1952)...
				\4 Mundell-Fleming
				\4[] Elementos fundamentales de M-F
				\4[] Sientan bases de macroeconomía de ec. abierta
		\2 Elasticidades
			\3 Idea clave
				\4 Asumimos $p_x$ y $p_m$ fijos
				\4 Caracterizar condiciones para que:
				\4[] i) $\Delta$ de TCN mejoren saldo de CC
				\4[] $\to$ Si $\text{CC} = B(E)$, ¿$\frac{d \, B}{d \, E} > 0$?
				\4[] $\to$ ¿Depreciación de moneda local mejora CC?
				\4[] ii) Tipo de cambio nominal converja
				\4[] $\to$ No se produzcan deprec./aprec. explosivas
				\4[] $\to$ ¿Cómo evolucionan TCN y CC ante desequilibrio?
			\3 Formulación
				\4[(i)] $\Delta$ de TCN mejoren saldo de CC
				\4[] BP moneda local: $B \equiv P_x \cdot X(E) - E \cdot P_M \cdot M(E)$
				\4[] Efecto de un cambio en TCN (E):
				\4[] $\frac{dB}{dE}=P_X \frac{dX}{dE} - \left( P_M M + P_M E \frac{dM}{dE} \right) > 0$
				\4[] $P_X \frac{dX}{dE} - P_M \left( M(E) + E \frac{dM}{dE} \right) > 0$
				\4[] $P_M \left( \frac{P_X}{P_M} \frac{dX}{dE} - M(E) - E \frac{dM}{dE} \right) > 0$
				\4[] $P_M M(E) \underbrace{\left( \frac{P_X \cdot X}{E P_M M} \underbrace{\frac{dX}{dE} \frac{E}{X}}_{\eta_X} - \underbrace{\frac{dM}{dE} \frac{E}{M}}_{\eta_M} - 1 \right)}_{\iff > 0} > 0$
				\4[(I)] Marshall-Lerner\footnote{O Bickerdicke-Robinson.} \fbox{$\frac{P_X X}{E P_M M} \eta_x + \left| \eta_m \right| > 1$}
				\4[(ii)] Estabilidad del tipo de cambio
				\4[] X y M implican oferta y demanda de divisas
				\4[] $\to$ Oferta y demanda de divisas determinan TC
				\4[] Exceso de dda. de divisas determina $\Delta$ TC
				\4[] Supuesto general:
				\4[] $\to$ Si exceso de dda. > 0 $\then$ $\uparrow$ TC
				\4[] $\to$ Si exceso de dda. < 0 $\then$ $\downarrow$ TC
				\4[] $\then$ Exceso de dda. causa depreciación moneda local
				\4[] ¿Cómo varía el exceso de dda. ante $\Delta$ de TC?
				\4[] Exceso de demanda:
				\4[] $\to$ $Z(E) \equiv D(E) - S(E)$
				\4[] Demanda de divisas:
				\4[] $\to$ $D(E) = P_M \cdot M(E)$
				\4[] Oferta de divisas:
				\4[] $\to$ $S(E) = P_X \cdot \frac{X(E)}{E}$
				\4[] Cambio en $Z(E)$ ante $\Delta$ de TC
				\4[] $\to$ $\dv{Z(E)}{E} = \dv{D(E)}{E} - \dv{S(E)}{E}$
				\4[] Cambio en demanda de divisas ante $\Delta$ de TC
				\4[] $\to$ $\dv{D(E)}{E} = P_M \cdot \dv{M(E)}{E}$
				\4[] $\then$ $\dv{M}{E} < 0$ $\then$ $\dv{D(E)}{E} < 0$
				\4[] $\to$ M decrecen con depreciación moneda local
				\4[] $\then$ Siempre relación negativa entre TC y demanda
				\4[] Cambio en oferta de divisas ante $\Delta$ de TC
				\4[] $\to$ $\dv{S}{E} = P_X \left( \frac{\frac{d X}{d E}\cdot E - X(E)}{E^2} \right) = \frac{P_X \cdot X(E)}{E^2} \left( \eta_X - 1 \right)$
				\4[] $\then$ $\eta_X > 1 $ $\then$ $\dv{S(E)}{E} > 0$ $\then$ $\dv{Z(E)}{E} < 0$
				\4[] $\then$ $\eta_X < 1 $ $\then$ $\dv{S(E)}{E} < 0$ $\then$ $\dv{Z(E)}{E} \lesseqgtr 0$
				\4[] $\to$ X aumentan con depreciación (efecto volumen)
				\4[] $\to$ Valor de X $\frac{P_X}{E}$ en divisas $\downarrow$ (efecto valor)
				\4[] $\then$ $S(E)$ $\uparrow$ con $E$ si $\eta_X > 1$
				\4[] $\then$ Si $\eta_X > 1$, EVolumen domina EValor
				\4[] $\then$ Si $\eta_X < 1$, hay que mirar $\eta_M$
			\3 Implicaciones
				\4 Si $\eta_X > 1$ para todo E:
				\4[] EVolumen domina siempre a EValor
				\4[] $\dv{Z(E)}{E} < 0$ para todo E
				\4[] $\then$ Sistema tiende a equilibrio estable
				\4[] $\then$ Un equilibrio único
				\4[] $\then$ \grafica{unicoequilibrio}
				\4 Si $\eta_X < 1$:
				\4[] EValor domina a EVolumen en exportaciones
				\4[] Posible aparición de equilibrio inestable
				\4[] Depende de pendientes relativas
				\4[] $\to$ Determinan signo de $\dv{Z(E)}{E} < 0$
				\4[] Si $\dv{Z(E)}{E} < 0$ para intervalo
				\4[] $\to$ EDemanda > 0 decrece con depreciación
				\4[] $\to$ EDemanda < 0 crece con apreciación
				\4[] $\then$ Sistema tiende a equilibrio estable
				\4[] Si $\dv{Z(E)}{E} > 0 $ para otro intervalo
				\4[] $\to$ Exceso dda. crece cada vez más
				\4[] $\to$ Moneda se deprecia cada vez más
				\4[] $\then$ Sistema diverge, no es estable
				\4[] $\then$ \grafica{multiplesequilibrios}
				\4 Pass-through
				\4[] Análisis anterior asume:
				\4[] $\to$ Precios de M iguales en divisa en F y H\footnote{``F'' por \textit{foreign}, ``H'' por \textit{home}. Lo cual sucede si el precio se fijan en la moneda del productor y se importan pagando ese precio en divisas.}
				\4[] $\to$ $M(E)$ y $X(E)$ caracterizan demandas respectivas
				\4[] $\then$ Para aislar efecto de $\Delta$ E sobre CC
				\4[] De hecho, bienes exportables tienen 2 precios
				\4[] $\to$ Precio en moneda local de productor
				\4[] $\to$ Precio en moneda local de importador
				\4[] $\then$ Pueden no coincidir (en misma moneda)
				\4[] $\then$ Márgenes empresariales afectados
				\4[] Elasticidad-TCN de precios en moneda local
				\4[] $\then$ Pass-through del tipo de cambio
				\4[] $\text{Pass-through de M} = \frac{d \ln P_M^H}{d \ln E}$
				\4[] $\text{Pass-through de X} = \frac{d \ln P_X^F}{d \ln E}$
				\4[] Especialmente pass-through de importaciones
				\4[] $\to$ Demanda de M depende de $P_M^H$
				\4[] $\to$ CC depende de $P_M^F$ y $E$
				\4[] Si pass-through de M muy pequeño:
				\4[] $\to$ Demanda de M estable en H aunque $\uparrow$ E
				\4[] $\to$ Márgenes empresariales deteriorados
				\4[] $\to$ Demanda de importaciones poco sensible a depreciación
				\4[] $\then$ Peor saldo de CC ante depreciación
				\4[] Resultados empíricos habituales
				\4[] $\to$ Respuesta parcial de P locales ante $\Delta$ E
				\4[] $\to$ Depende de industrias y poder de mercado
				\4[] $\then$ Pass-through incompleto
				\4 Curva J
				\4[] Descripción de fenómeno empírico relativo a $\Delta$ E
				\4[] Efectos de devaluación dependen de tiempo
				\4[] 1. Inicialmente, deterioro de saldo CC
				\4[] 2. A continuación, mejora
				\4[] 3. Estabilización con saldo mejor que inicial
				\4[] Potencial inestabilidad
				\4[] $\to$ Si fase 1 se retroalimenta
				\4[] $\to$ Si agentes no estiman futura mejora de CC
				\4[] Causas del fenómeno:
				\4[] $\to$ Contratos pre-existentes
				\4[] $\to$ Monedas de denominación de los contratos
				\4[] $\to$ Lags en ajustes de precios
				\4[] $\to$ Lags en ajustes de cantidades
				\4[] $\to$ Problemas de reasignación de contratos
				\4[] $\to$ En l/p, condición de M-S tiende a cumplirse
				\4[] Evidencia sobre curva J
				\4[] $\to$ No concluyente
				\4[] $\to$ Resultados mixtos
				\4[] $\to$ Algunos estudios muestran curva S, no J
				\4[] \grafica{curvaj}
			\3 Valoración
				\4 Importancia empírica
				\4[] Muy elevada
				\4[] X y M reaccionan fuertemente a $\Delta$ E
				\4[] Devaluaciones competitivas importantes
				\4[] $\to$ Ampliamente utilizadas a lo largo de historia
				\4 Argumentos pro y contra TC flexibles
				\4[] Dadas determinadas elasticidades y estabilidad
				\4[] $\to$ TC flexibles inducen equilibrio de CC
				\4[] $\then$ TC flexibles inducen equilibrio externo
				\4[] Argumentos en contra
				\4[] $\to$ Curva J existe y puede desestabilizar
				\4[] $\to$ $\Delta E$ afectan cuantía de DA
				\4[] $\to$ Países grandes provocan 2nd round-effects\footnote{Supongamos un mundo formado por dos economías grandes que comercian entre sí. Una depreciación reduce las importaciones de un país. A priori esto debería causar una mejora del saldo de la CC (con Marshall-Lerner). Sin embargo, esa reducción de las importaciones provoca una caída de la demanda agregada del otro país, y una reducción de las exportaciones del primer país que puede potencialmente neutralizar la mejora de la CC.}
				\4[] $\to$ Flujos de K financiero afectan ajuste de $Z(E)$\footnote{En el análisis de estabilidad anterior, se asume identidad entre exceso de demanda de divisas y balanza comercial. Sin embargo, en presencia de transferencias de capital, pueden aparecer excesos de demanda no nulos en el mercado de divisas que provoquen apreciaciones/depreciaciones aun encontrándose en equilibrio la balanza comercial/cuenta corriente.}
				\4 Enfoque relevante pero no suficiente
				\4[] Análisis de equilibrio parcial
				\4[] Capacidad explicativa general muy reducida
				\4[] Necesario integrar en modelos con:
				\4[] $\to$ Efectos demanda agregada
				\4[] $\to$ Efectos de variables monetarias
				\4[] $\to$ Efectos de transferencia intertemporal
		\2 Enfoque de absorción\footnote{De Palgrave, formulación simplificada.}
			\3 Idea clave
				\4 Saldo de CC depende de:
				\4[] Diferencia entre absorción y output
				\4 Déficit comercial se reduce si:
				\4[] $\to$ Output aumenta más que absorción
				\4[] $\to$ Absorción cae más que output
				\4 Concepto de absorción
				\4[] Consumo doméstico de bienes y servicios
				\4[] $\to$ Independiente de origen nacional o importado
				\4 Ajuste de Balanza de Pagos
				\4[] Resultado de variación del gasto autónomo
				\4 Contexto teórico
				\4[] Años 40 y 50
				\4[] Análisis keynesiano predominante
				\4[] $\to$ Multiplicadores de gasto
				\4[] $\to$ Demanda agregada y demanda efectiva
				\4[] Machlup (1943)
				\4[] $\to$ Multiplicadores y balanza comercial
				\4[] Alexander (1952), (1959)
				\4[] $\to$ Comparación con enfoque de elasticidades
				\4[] $\to$ Efectos conjuntos de ambos
				\4[] $\to$ Formalización del marco de análisis
			\3 Formulación
				\4 Supuestos keynesianos habituales
				\4 Identidad del gasto y la renta
				\4[] $Y = \underbrace{C + I + G}_{\text{absorción}} + X - M$
				\4[] $\then$ $B \equiv X-M = Y - A$
				\4 Supuestos sobre componentes de la DA
				\4[] Simplificación máxima
				\4[] $\to$ Aislar efecto de multiplicadores y absorción
				\4[] $C = C_0 + c Y$
				\4[] $C_0 = I + G$
				\4[] $\to$ Sin sector público
				\4[] $\to$ Demanda de inversión exógena
				\4[] $\then$ Incluidos dentro de consumo autónomo
				\4[] $\then$ PMgConsumo < 1
				\4[] Importaciones
				\4[] $\to$ Componente autónomo
				\4[] $\to$ Componente endógeno depende de renta
				\4[] $M = M_0 - mY$
				\4 Equilibrio en mercado de bienes
				\4[] $Y = C_0 + cY + X - (M_0 + mY)$
				\4[] $(1-c+m)Y = C_0 + \underbrace{X - M_0}_{\text{NX}}$
				\4[] $\to$ $Y=k \cdot (C_0 + \text{NX}) = \frac{1}{1-c+m} \cdot (C_0 + \text{NX})$
				\4[] $\to$ $k$ es multiplicador del gasto autónomo
				\4 Saldo CC en términos de componentes autónomos
				\4[] \fbox{$B=\underbrace{k (C_0 + X_0 - M_0)}_Y - \underbrace{\left( C_0 + ck \left( C_0 + X_0 - M_0 \right) \right)}_A $}
				\4 Cambios en vars. exógenas sobre CC es suma de:
				\4[] $\to$ Efecto sobre output
				\4[] $\to$ Efecto sobre absorción Absorción
				\4 Efecto de $\Delta$ de $C_0$ sobre CC
				\4[] $d C_0$ captura múltiples fenómenos s.p.g.
				\4[] $\to$ $\Delta$ de inversión por $\Delta$ de interés
				\4[] $\to$ $\Delta$ de gasto público discrecional
				\4[] $\to$ $\Delta$ de consumo autónomo
				\4[] $\to$ ...
				\4[] $\then$ Expenditure changing: $\Delta$ gasto total
				\4[] Efecto sobre output ($d Y$):
				\4[] $\to$ $d Y = k \cdot d C_0$
				\4[] Efecto sobre absorción ($dA$):
				\4[] $\to$ $d A = dC_0 + c \cdot k \cdot d C_0 = (1+ck)d  C_0$
				\4[] Efecto total\footnote{Ver concepto ``\textit{Derivación del efecto de un cambio en el gasto autónomo sobre la cuenta comercial}''.} ($dB = dY - dA$):
				\4[] $\to$ \fbox{$dB = k -1 - ck = -mk \, dC_0$}
				\4[] $\then$ CC empeora si $d C_0 > 0$
				\4 Efecto de $\Delta$ de $\text{NX}$ sobre CC
				\4[] $d \, \text{NX}$ captura múltiples fenómenos:
				\4[] $\to$ Efecto de depreciación/apreciación de TCN
				\4[] $\to$ $\Delta$ exportaciones autónomas
				\4[] $\to$ $\Delta$ importaciones autónomas
				\4[] $\then$ Expenditure switching: $\Delta$ composición de gasto
				\4[] Efecto sobre output ($d Y$):
				\4[] $\to$ $d Y = k \cdot d \text{NX}$
				\4[] Efecto sobre absorción ($d A$):
				\4[] $\to$ $d A = c \cdot k \cdot d \text{NX}$
				\4[] Efecto total:
				\4[] \fbox{$\to$ $d B = k \cdot d \text{NX} - ck \cdot d \text{NX} =  (1-c)k \cdot d \text{NX}$}
				\4[] $\then$ CC mejora si $d \, \text{NX} > 0$
			\3 Implicaciones
				\4 Mejora de saldo de CC en pleno empleo
				\4[] Dado pleno empleo y mejora de CC dada:
				\4[] $\to$ ¿Cuánto ESwitching y EChanging necesario?
				\4[] $\alpha$: expenditure changing
				\4[] $\beta$: expenditure switching
				\4[] Pleno empleo (output constante):
				\4[] $\to$ $dY = 0 = \underbrace{k\cdot d C_0 \cdot \alpha}_{\text{EChanging sobre y}} + \underbrace{k\cdot d n}_{\text{ESwitching sobre y}}$
				\4[] $\then$ $d C_0 = - d N \frac{\beta}{\alpha}$
				\4[] Mejora de CC ($dB$):
				\4[] $\to$ $d B = -mk \cdot d C_0 \cdot \alpha + (1-c)k \cdot d N \cdot \beta$
				\4[] $\then$ $d B = d N \beta$
				\4[] $\then$ $d C_0 = \frac{-d B}{\alpha}$
				\4[] En pleno empleo, mejorar CC implica:
				\4[] $\to$ ESwitching igual a mejora de CC
				\4[] $\then$ Para mejorar CC
				\4[] $\to$ EChanging negativa proporcional a mejora de CC
				\4[] $\then$ Para mantener pleno empleo
				\4 Política económica y equilibrio interno y externo
				\4[] Prescripción general de PE de SNClásica
				\4[] Equilibrio interno
				\4[] $\to$ Política fiscal para afectar DA
				\4[] Equilibrio externo
				\4[] $\to$ Política monetaria para afectar TCN
				\4[] Basado en dos asunciones:
				\4[] $\to$ TCN más efectivo sobre NX
				\4[] $\to$ Dda. autónoma más efectiva sobre DA
				\4 Efecto Harberger-Laursen-Metzler
				\4[] ¿Cómo afecta TCN a absorción?
				\4[] $\to$ Formulación anterior no afirma nada
				\4[] Modelos que integran elasticidades y absorción
				\4[] $\to$ Deben postular relación entre TCN y A
				\4[] Harberger (1950), Laursen y Metzler (1950)
				\4[] 1. $\uparrow$ E o $P_M$ deteriora RRI ($\frac{P_X}{E \cdot P_M}$)
				\4[] 2. $\downarrow$ precio de M implica $\downarrow$ Y
				\4[] 3. $\downarrow$ Y implica $\downarrow$ A
				\4[] 4. Dada $PMgC < 1$, A cae menos que Y
				\4[] 5. $dY < dA$ implica $dB <0$, luego $\uparrow$ déficit CC
				\4[] $\to$ Deterioro RRI aumenta absorción, dado Y
				\4[] $\to$ Deterioro RRI empeora saldo CC, dado Y
				\4[] Formalmente:
				\4[] $\to$ $d E > 0 \then d \left( \frac{P_X}{P_M \cdot E} \right) < 0 \then dA >0$
				\4[] $\to$ $dY = 0, (dE > 0 \then dA > 0) \then d B < 0$
			\3 Valoración
				\4 Limitaciones del análisis
				\4[] 1. No considera efecto de $\uparrow$ E sobre $\pi$
				\4[] $\to$ Devaluaciones aumenta presión inflacionaria
				\4[] $\to$ Puede general espirales inflacionarias
				\4[] $\then$ Efectos de segunda ronda
				\4[] 2. No considera efectos de EChanging en PEmpleo
				\4[] $\to$ Output no puede aumentar vía multiplicador
				\4[] $\to$ Inflación o importaciones deben absorber
				\4[] 3. Condiciones monetarias no consideradas
				\4[] $\to$ Efecto de CC sobre oferta monetaria
				\4 Enfoque relevante pero no suficiente
				\4[] Análisis de equilibrio parcial
				\4[] Capacidad explicativa general muy reducida
				\4[] Necesario integrar en modelos con:
				\4[] $\to$ Efectos de TCN sobre ddas.
				\4[] $\to$ Efectos de variables monetarias
				\4[] $\to$ Efectos de transferencia intertemporal
	\1 \marcar{Enfoque de stocks}
		\2 Idea clave
			\3 Equilibrio de stocks
				\4 Desequilibrios en BP
				\4[] $\to$ Resultado de desequilibrios de stocks
				\4 Stocks relevantes
				\4[] Oro
				\4[] Oferta monetaria
				\4[] Activos financieros
			\3 Estabilidad
				\4 Análisis clásico/neoclásico
				\4 Estabilidad de la economía
				\4[] Tienden a plena utilización de recursos
				\4 Dinero como velo
				\4[] Dicotomía clásica
				\4[] $\to$ Vars. reales y nom. determinadas por separado
				\4[] $\then$ Output independiente de dinero
				\4 Ajuste hacia equilibrio
				\4[] Se admiten desviaciones de equilibrio
				\4[] Tendencia a no examinar
			\3 Contexto teórico
				\4 Corriente muy antigua
				\4[] Escuela de Salamanca, Hume, Currency school
				\4 Demanda estable de dinero
				\4[] $\to$ Teoría Cuantitativa del Dinero
				\4[] $\then$ Monetarismo y otros modelos
		\2 Modelo de flujo-especie
			\3 Idea clave
				\4 Hume (1752): ``Sobre la balanza comercial''
				\4[] Predecesores ya conocían
				\4[] Presentación analítica
				\4[] $\to$ Cantillon, Azpilcueta...
				\4 Dicotomía clásica
				\4[] Output determinado independientemente
				\4[] Hume admitía variaciones en c/p
				\4[] $\to$ No consideradas en análisis
				\4 Relación oro-precios-flujos para equilibrar BP
				\4[] Expansión de oro aumenta precios
				\4[] $\to$ Caen exportaciones netas y sale oro
				\4[] $\then$ Contracción monetaria hasta ajustar precios
			\3 Formulación
				\4 Cuatro proposiciones básicas
				\4[I] Cantidad de oro determina oferta monetaria
				\4[] Multiplicador monetario constante
				\4[] $\to$ Caso particular: sin reserva fraccionaria
				\4[II] Oferta monetaria determina nivel de precios
				\4[] Teoría cuantitativa de dinero
				\4[III] Nivel de precios determina exportaciones netas
				\4[] PPA se cumple
				\4[] Oro vale igual en todas partes
				\4[] $\to$ TCN constante
				\4[] $\then$ Precios son único determinante de TCR
				\4[] Condición de Marshall-Lerner se cumple
				\4[] $\to$ Aumento de precios nacionales reduce NX
				\4[IV] Exportaciones netas determinan variación de oro
				\4[] Transacciones internacionales liquidadas en oro
				\4[] No existe otro medio de transacción
				\4 Extensión de Mill\footnote{Ver extracto de Friedman (1953) en concepto \textit{Extensión de Mill al mecanismo de flujo-especie}.}
				\4[] Tipo de interés es mecanismo de ajuste adicional
				\4[] Salida de oro reduce stock de oro
				\4[] Reducción de stock de oro reduce oferta monetaria
				\4[] $\to$ Aumenta tipo de interés
				\4[] Aumento de tipo de interés
				\4[] $\to$ Entrada de capital en país
				\4[] $\to$ Caída de demanda agregada
				\4[] $\then$ Acelera ajuste del déficit de CC
			\3 Implicaciones
				\4 Stock de oro se ajusta hasta CC=0
				\4[] Shock aumenta stock de oro:
				\4[] 1. Suben precios y apreciación de TCReal
				\4[] 2. Aumentan importaciones, caen exportaciones
				\4[] 3. Aparece déficit por cuenta corriente
				\4[] 4. Salida de oro para liquidar saldo CC
				\4[] 5. Cae stock de oro
				\4[] 6. Caen precios y depreciación de TCReal
				\4[] 7. Caen importaciones, aumentan exportaciones
				\4[] 8. Se reduce déficit por cuenta corriente
			\3 Valoración
				\4 Escasa aplicabilidad actual
				\4[] No hay dinero universal y con oferta exógena
				\4[] Reserva fraccionaria + PM discrecional
				\4[] $\then$ Oferta monetaria inestable
				\4[] Demanda de dinero
				\4[] $\then$ Errática en últimas décadas
				\4[] Rigidez de precios
				\4[] $\then$ Significativa
				\4 Impacto teórico
				\4[] Primera exposición analítica de ajuste BP
				\4[] Marco conceptual de econ. internacional
				\4[] $\to$ Hasta aplicación de Keynes a BP
				\4[] Aceptación generalizada
				\4[] $\to$ Lo que Keynes llamaba ``clásicos''
		\2 Enfoque monetario\footnote{Ver Gandolfo pág. 247 pero sobre todo Palgrave ``\textit{monetary approach to the balance of payments}''.}
			\3 Idea clave
				\4 BPagos resulta de mercado de dinero
				\4[] EDemanda de dinero deben eliminarse
				\4[] $\to$ Oferta monetaria como variable stock determinante
				\4[] $\to$ TCN ajusta oferta monetaria
				\4 Economía cerrada
				\4[] Oferta de dinero depende de:
				\4[] $\to$ Crédito interno
				\4[] Variables de ajuste:
				\4[] $\to$ Renta real
				\4[] $\to$ Nivel de precios
				\4[] $\to$ Tipo de interés
				\4 Economía abierta
				\4[] Oferta de dinero depende de:
				\4[] $\to$ Crédito interno
				\4[] $\to$ Operaciones con divisas del banco central
				\4[] Aparecen otro canales de ajuste de EDemanda
				\4[] $\to$ Balanza de pagos
				\4[] $\then$ Concretamente, stock de reservas de divisas
			\3 Formulación
				\4 Supuestos básicos
				\4[] País pequeño
				\4[] $\to$ Interés exógeno
				\4[] $\to$ Flujos de capital infinitamente elásticos
				\4[] Pleno empleo
				\4[] $\to$ Output no depende de EDemanda de dinero
				\4[] Tipo de cambio fijo
				\4[] $\to$ TCN no es variable de ajuste
				\4[] $\then$ Centrar análisis sobre reservas
				\4[] Integración perfecta con economía mundial
				\4[] $\to$ Sin costes de transporte
				\4[] $\to$ Arbitraje determina precios domésticos
				\4[] $\then$ Paridad de poder adquisitivo
				\4[] Multiplicador monetario constante
				\4[] $\to$ Ratio constante demanda-efectivo
				\4[] $\then$ Relación base-oferta monetaria constante
				\4 Oferta de dinero
				\4[] $M^S = m M_0$
				\4[] $\to$ Asumimos $m=1$ s.p.g.
				\4[] $M_0 = eR + D$
				\4[] $\to$ $R$: reservas de divisas
				\4[] $\to$ $D$: crédito interno
				\4[] $\to$ $e$: tipo de cambio nominal directo
				\4 Demanda de dinero
				\4[] $M^D = P \cdot L(Y,i)$, $L_Y > 0$, $L_i<0$
				\4[] $\to$ $P$: precios mundiales (por arbitraje)
				\4[] $\to$ $Y$: renta real
				\4[] $\to$ $i$: interés nominal mundial (por arbitraje)
				\4 Equilibrio
				\4[] $M^D = M^S \then eR + D = P \cdot f(Y,i)$
				\4[] $\then$ \fbox{$R=\frac{P \cdot L(Y,i)-D}{e} = f\left( \underset{-}{i}, \underset{-}{e}, \underset{+}{Y}, \underset{-}{D}, \underset{+}{P} \right)$}
				\4[] $\then$ \fbox{$e = \frac{L(i,Y) - D}{R} = g(\underset{-}{i}, \underset{+}{Y}, \underset{-}{D}, \underset{-}{R}, \underset{+}{P})$}
				\4[] Equilibrio de largo plazo
				\4[] $\to$ Puede no ser así a corto plazo
				\4[] $\then$ Distribución variable de $\Delta Y$ y $\Delta P$
			\3 Implicaciones
				\4 Expansión de crédito doméstico: largo plazo
				\4[] Vía compra de crédito doméstico (D)
				\4[] Y, e, P, i son exógenos
				\4[] $\to$ Demanda de dinero es exógena
				\4[] Sólo reservas pueden ajustarse
				\4[] $\to$ $\uparrow$ D $\to$ $\downarrow$ R
				\4[] $\to$ Expansión D reduce reservas proporcionalmente
				\4[] $\then$ Estímulo ineficaz y empeora stock de reservas
				\4 Expansión de crédito doméstico: corto plazo
				\4[] Y, i, P acomodan aumento de oferta
				\4[] $\to$ Aumenta output nominal
				\4[] $\to$ Cae interés doméstico
				\4[] Ajuste hacia equilibrio de l/p progresivo
				\4[] 1. Aumento de precio nacional $\to$ Déficit CC
				\4[] 2. Caída de interés nacional $\to$ Salida de capitales
				\4[] 3. Exceso de demanda de divisas
				\4[] 4. Banco central vende reservas para mantener TCN
				\4[] 5. Reducción de oferta monetaria
				\4[] 6. Restablecimiento de equilibrio
				\4 Entrada de capital exógena
				\4[] P. ej.: flight to safety
				\4[] Entrada de capital a pesar de:
				\4[] $\to$ Mismo interés
				\4[] $\to$ Mismo nivel de precios
				\4[] Aparece exceso de oferta de divisas
				\4[] $\to$ BCentral compra divisas para fijar TCN
				\4[] $\to$ BCentral aumenta stock de reservas
				\4[] $\then$ Aumenta oferta monetaria
				\4[] Esterilización de expansión monetaria
				\4[] $\to$ Reducir presión inflacionaria
				\4[] $\to$ Vendiendo crédito doméstico
				\4[] $\then$ Mantiene oferta monetaria
				\4[] $\then$ Causalidad no es sólo $\uparrow D \to \downarrow R$
				\4[] $\then$ También $\uparrow R \to \downarrow D$
				\4 Integración con economía internacional
				\4[] Integración determina absorción vía precios
				\4[] $\to$ Proporción de bienes no comerciables
				\4[] $\to$ costes de transporte
				\4[] Integración perfecta
				\4[] $\to$ Arbitraje en mercado de bienes
				\4[] $\then$ Nivel de precios es exógeno a economía
				\4[] Integración muy pequeña
				\4[] $\to$ No es posible arbitrar bienes
				\4[] $\then$ Diferentes precios mundial--doméstico
				\4[] $\then$ Precios nacionales absorben exp. monetaria
				\4 Política comercial
				\4[] Aranceles aumentan nivel de precios
				\4[] $\to$ Reducen valor real de oferta monetaria
				\4[] $\then$ Exceso de demanda de dinero
				\4[] Ajuste hacia equilibrio
				\4[] $\to$ Superávit comercial
				\4[] $\to$ Aumento de reservas
				\4[] $\to$ Aumento de oferta monetaria
				\4[] $\then$ Equilibrio monetario
				\4 Crecimiento del producto
				\4[] Aumento exógeno de Y induce demanda de dinero
				\4[] $\to$ Necesario aumentar reservas
				\4[] $\then$ Superávit de cuenta corriente
				\4 País grande
				\4[] Economía doméstica afecta:
				\4[] $\to$ Precio mundial de bienes
				\4[] $\to$ Interés mundial
				\4[] Proceso de ajuste más complejo
				\4[] $\to$ Resultados cualitativamente similares
			\3 Valoración
				\4 Relación con otros enfoques
				\4[] Resultados esencialmente similares a absorción
				\4[] Balanza de pagos es ventana al mundo
				\4[] Absorción:
				\4[] $\to$ Equilibrio de demanda y oferta de bienes
				\4[] $\then$ Desequilibrio induce acumulación de reservas
				\4[] Monetario:
				\4[] $\to$ Equilibrio de demanda y oferta de dinero
				\4[] $\then$ Flujos derivados de reservas acumuladas
				\4 No sólo flujos son importantes
				\4[] Enfatiza resultado de mecanismo flujo-especie
				\4[] $\to$ Para economías monetarias en general
				\4 Importancia de las reservas
				\4[] Énfasis en reservas sobre otros activos
				\4[] $\to$ Es relevante por múltiples razones
				\4[] 1. $\uparrow$ de activos extranjeros no implica $\uparrow$ M
				\4[] $\to$ P.ej.: $\downarrow$ $i$ y salida de capital $\to$ $\downarrow$ R
				\4[] 2. Stocks de R es necesario para estabilizar TCN
				\4[] $\to$ Potenciales crisis monetarias
				\4[] 3. Oferta monetaria depende de R y CInterno
				\4[] $\to$ Autoridad monetaria controla CInterno
				\4[] $\to$ Debe prestar atención a $\Delta R$ para determinar M
		\2 Enfoque de cartera
			\3 Idea clave
				\4 Análisis de carteras
				\4[] Tobin y Markowitz inician análisis eq. parcial
				\4[] $\to$ Media-varianza
				\4[] $\to$ ¿Cómo mantener determinada riqueza?
				\4[] $\then$ ¿Cuánto demandar de cada activo financiero?
				\4[] McKinnon y Oates (1966), Branson (1974)...
				\4[] $\to$ Simplificación y extensión eco. abierta
				\4[] $\then$ ¿Cuánto de cada activo internacional?
				\4 Balanza de pagos
				\4[] Desequilibrio de stocks de activos financieros
				\4[] $\to$ Determina saldo de cuenta financiera
				\4[] $\then$ Determina saldo de cuenta corriente
				\4[] Política monetaria
				\4[] $\to$ Genera desequilibrios de stocks de activos
			\3 Formulación
				\4 Modelo simplificado de eq. parcial
				\4[] De Grauwe (1983)
				\4[] Riqueza constante
				\4[] $\to$ $W = \bar{W}$
				\4[] Sólo tres activos
				\4[] $\to$ Dinero
				\4[] $\to$ Bono nacional
				\4[] $\to$ Bono extranjero
				\4[] País pequeño
				\4[] $\to$ $i^*$ exógeno y constante
				\4[] $\to$ Oferta perf. elástica de bonos extranjeros
				\4[] Sustituibilidad variable
				\4[] $\to$ Elasticidades de ddas. capturan
				\4 Demanda de activos
				\4[] Dinero:
				\4[] $\frac{L}{W} = \cdot l(Y,i,i^*)$, $l_Y >0$, $l_i, l_{i^*} <0$
				\4[] Bonos nacionales:
				\4[] $\frac{H}{W} = h(Y,i,i^*)$, $h_Y < 0$, $h_i > 0$, $h_{i^*} <0$
				\4[] Bonos extranjeros:
				\4[] $\frac{F}{W} =  f(Y,i,i^*)$, $f_Y < 0$, $f_i < 0$,
				\4[] $\to$ $W = W \cdot l(Y,i,i^*) + W \cdot h(Y,i,i^*) + W \cdot f(Y,i,i^*)$
				\4 Oferta de activos
				\4[] Dinero: $M^S$ $\to$ exógena y determinable
				\4[] Bonos nacionales: $H^S$ $\to$ exógena y determinable
				\4[] Bonos extranjeros: $F^S$ $\to$ infinitamente elástica
				\4[] $\to$ $W = W \cdot M^S + W \cdot H^S + W \cdot F^S$
				\4 Equilibrio en todos los mercados
				\4[] $[M^S - l(Y,i,i^*)] + [N^S - n(Y,i,i^*)] + [F^S - f(Y,i,i^*)] = 0$
				\4 Representación gráfica
				\4[] Espacio $F$--$i$
				\4[] $\to$ ¿Qué relación entre cuenta financiera e interés?
				\4[] Mercado de dinero
				\4[] 1. Más demanda de F implica compra de divisas
				\4[] 2. Compra de divisas implica reducción oferta de M
				\4[] 3. Eq. en mercado de dinero implica menor dda. de M
				\4[] 4. Menor demanda de M implica mayor interés
				\4[] $\then$ $i$ de equilibrio aumenta con F
				\4[] Mercado de bonos nacionales
				\4[] 1. Y, $i^*$, oferta de bonos constantes
				\4[] 2. Sólo un interés nacional equilibra mercado
				\4[] $\then$ $i$ de equilibrio constante en F
				\4[] Mercado de bonos extranjeros
				\4[] 1. Más interés nacional reduce dda. de F
				\4[] $\then$ Curva decreciente
				\4[] \grafica{carteraequilibrio}
			\3 Implicaciones
				\4 Política monetaria expansiva
				\4[] Vía compra de bonos
				\4[] Mercado de bonos
				\4[] $\to$ Reduce oferta de bonos
				\4[] $\then$ Cae interés nacional
				\4[] $\then$ Curva N hacia abajo
				\4[] Mercado de dinero
				\4[] $\to$ Aumenta oferta monetaria
				\4[] $\then$ A igual F, interés nacional debe ser menor
				\4[] $\then$ Curva L hacia derecha
				\4[] Nuevo equilibrio
				\4[] $\to$ Menor interés nacional
				\4[] $\to$ Más activos extranjeros
				\4[] $\to$ Más dinero
				\4[] $\to$ Menos bonos nacionales
				\4[] $\then$ Salen capitales
				\4[] $\then$ Reducción del déficit comercial
				\4[] \grafica{carteraexpansionmonetaria}
			\3 Valoración
				\4 Similar estructura a modelo monetario
				\4 Extensiones aumentan capacidad explicativa
				\4[] Tipo flexible
				\4[] $\to$ Modelización de overshooting
				\4[] Gasto público
				\4[] $\to$ Dinámica de deuda y riqueza neta
				\4[] ...
	\1 \marcar{Enfoque intertemporal y sostenibilidad}\footnote{Ver Palgrave:``\textit{international finance}'' y Obstfeld y Rogoff caps. 1 y 2.}
		\2 Idea clave
			\3 Equilibrio intertemporal
				\4 Enfoques anteriores
				\4[] Análisis de estática comparativa
				\4[] $\to$ ¿Qué equilibrios existen?
				\4[] $\then$ ¿Cómo llegar a ellos?
				\4 Análisis intertemporal
				\4[] Equilibrios son secuencias temporales
				\4[] $\to$ Valores interdependientes
				\4[] Balanza de pagos en un periodo resulta de:
				\4[] $\to$ Balanza de pagos pasada
				\4[] $\to$ Balanza de pagos futura
				\4[] $\to$ Demanda de consumo intertemporal
				\4 Preguntas clave
				\4[] ¿Qué relación entre balanza de pagos presente y futuro?
				\4[] ¿Cómo afecta coste de consumo intertemporal a balanza de pagos?
			\3 Microfundamentación
				\4 Enfoque intertemporal no necesita microfund.
				\4[] Posible análisis positivo de trayectorias
				\4[] $\to$ ¿Qué dinámicas de BP son posibles?
				\4[] $\to$ ¿Cómo afectan $\Delta$ $r$ y $\Delta$ $e$ a balanza de pagos
				\4 Combinación con análisis microfundamentado
				\4[] Agentes deciden consumo e inversión
				\4[] $\to$ Sujetos a restricción intertemporal
				\4[] Agentes representativos como macroeconomías
				\4[] $\to$ Endeudamiento es balanza de pagos
				\4[] Posible análisis normativo
				\4[] $\to$ Caracterizar balanzas de pagos óptimas
				\4[] $\to$ Comparación normativa de política económica
			\3 Contexto teórico
				\4 Análisis cuantitativo general
				\4[] Simulación de dinámica de deuda externa
				\4 Enfoque neoclásico
				\4[] Marco de optimización de utilidad restringida
				\4[] Análisis intertemporal de Fisher (1930)
				\4[] $\to$ Ahorro e inversión resultado de optimización
				\4[] Equilibrio general
				\4[] $\to$ Intercambio bienes y activos financieros
				\4[] $\then$ Muy similar a modelo de comercio internacional
				\4 Crítica de Lucas (1976)
				\4[] Modelos anteriores no resisten cambios de política
				\4[] Modelos deben resultar de estructura ``profunda''
				\4[] $\to$ Tecnología
				\4[] $\to$ Preferencias
				\4[] $\to$ Dotaciones
				\4[] Análisis microfundamentado intertemporal de la BP
				\4[] $\to$ Derivados de decisiones óptimas de oferta y demanda
				\4[] $\to$ Sí es robusto a cambios en política
				\4[] $\to$ Modelos más generales
				\4 Modelos DSGE
				\4[] Elemento básico de modelos de ec. abierta
		\2 Modelo simple de dos periodos
			\3 Idea clave
				\4 Dos periodos
				\4 País pequeño
				\4[] Interés exógeno
				\4 Tecnología de inversión
				\4[] Posible incorporar
				\4[] Rendimientos decrecientes a escala
				\4[] Modelos simples
				\4[] $\to$ Dotaciones dadas en cada periodo
				\4[] $\to$ Transferencia intertemporal vía bono con rdto. constante
			\3 Formulación
				\4 Maximización de la utilidad
				\4[] $\underset{c_t,c_{t+1}}{\max} \quad U = u(c_t, c_{t+1}) $
				\4 Restricción intertemporal
				\4[] $\text{s.a:} \quad \, c_t + b_t = y_t$
				\4[] \quad \quad \, \, $c_{t+1} = y_{t+1} + (1+r) b_t$
				\4[] $\then$ \quad \, $c_t + \frac{c_{t+1}}{1+r} = y_t + \frac{y_{t+1}}{1+r}$
				\4 Saldo de cuenta corriente
				\4[] Diferencia entre $y_t$ y $c_t$
				\4 Equilibrio intertemporal
				\4[] Déficit/superávit en un periodo
				\4[] $\to$ Han de compensarse en el otro periodo
				\4[] Déficit de cuenta corriente en $t$
				\4[] $\to$ País vende activos por misma cuantía
				\4[] $\to$ Recompra activo financiero en siguiente periodo
				\4 Representación gráfica
				\4[] \grafica{dosperiodos}
			\3 Implicaciones
				\4 Mejora de bienestar con apertura
				\4[] En autarquía, saldo nulo de CC
				\4[] $\to$ No se venden/compran activos financieros
				\4[] $\to$ Interés implícito: tangencia CI-dotación
				\4[] Apertura beneficiosa si interés mundial distinto
				\4[] $\to$ País tendrá ventaja comparativa en $t$ o $t+1$
				\4[] Si interés de autarquía menor que interés mundial
				\4[] $\to$ Consumo futuro más caro en doméstica
				\4[] $\to$ Consumo presente más barato en doméstica
				\4[] $\then$ Tras apertura, superávit CC en presente
				\4[] Si interés de autarquía mayor que interés mundial
				\4[] $\to$ Consumo futuro más barato en doméstica
				\4[] $\to$ Consumo presente más caro en doméstica
				\4[] $\then$ Tras apertura, déficit en CC en presente
				\4[] \grafica{aperturacuentafinanciera}
				\4 Restricciones financieras
				\4[] Economías pueden tener cerrada la financiación
				\4[] $\to$ Riesgo político
				\4[] $\to$ Impagos anteriores
				\4[] $\to$ Guerras, embargos...
				\4[] No pueden incurrir en déficits por cuenta corriente
				\4[] Sí pueden mostrar superávits
				\4[] $\to$ Sí optimización requiere CC<0 presente
				\4[] $\then$ Restricción induce pérdida de bienestar
				\4[] $\then$ Restricciones vinculantes reducen bienestar
				\4[] \grafica{restriccionendeudamiento}
				\4 Efectos de variación del interés
				\4[] Enfoque intertemporal permite distinguir ES y ER
				\4[] $\to$ Entre consumo presente y futuro
				\4[] $\to$ Ante variación del interés/precio cons. presente
				\4[] \grafica{variacioninteres}
				\4 Separación inversión y consumo
				\4[] Dado interés exógeno, BP resulta de:
				\4[] $\to$ Decisiones independientes de C e I
				\4[] $\then$ Dos etapas
				\4[] 1. Decisión de inversión
				\4[] $\to$ Maximizar valor presente de producto
				\4[] 2. Decisión de consumo
				\4[] $\to$ Maximizar utilidad
				\4[] $\then$ Balanza de pagos óptima
				\4[] \grafica{separacioninversionconsumo}
				\4 Flujos de capital hacia países en desarrollo
				\4[] Asumiendo:
				\4[] $\to$ PMg decreciente del capital
				\4[] $\to$ PEDs tienen menor dotación de K
				\4[] $\to$ Interés mundial exógeno
				\4[] $\then$ FPPIntertemporal muy sesgada hacia arriba
				\4[] Apertura al capital exterior
				\4[] $\to$ RP tangente muy a la izquierda
				\4[] $\to$ Induce déficits iniciales muy elevados
				\4[] $\to$ Superávits posteriores
				\4[] Paradoja de Lucas (1990)
				\4[] $\to$ PEDs tienen poco déficit en relación a K
				\4[] $\to$ Desarrollados reciben capital
				\4 Gasto público
				\4[] Asumiendo:
				\4[] $\to$ Gasto público financiado con impuestos
				\4[] $\to$ CC en equilibrio en $t$ y $t+1$ en ambos periodos
				\4[] $\Delta G_t > 0$, $\Delta G_{t+1} =0$
				\4[] $\to$ Reducción de Y disponible en t
				\4[] $\to$ Reducción de K disponible en $t+1$, $\downarrow$ en $t+1$
				\4[] $\then$ Desplazamiento sesgado hacia la izquierda
				\4[] $\then$ Intersección con eje de abscisas cae en $\Delta G_t$
				\4[] $\then$ Intersección con ordenadas cae $> \Delta G_{t+1}$
				\4[] Efecto sobre CC:
				\4[] $\to$ Aparece superávit en $t$
				\4[] $\to$ Aparece déficit en $t+1$
				\4[] $\then$ Resultado de preferencias convexas\footnote{En cierta medida esto es contraintuitivo. El impuesto detrae renta del periodo $t$, y sin embargo aumenta el superávit en $t$. Para entenderlo, supongamos que el agente ahorra toda la renta en $t$ para invertirlo en capital y aumentar la producción en $t+1$. Antes del impuesto en $t$, tendrá disponible para el consumo en $t+1$ una cantidad igual al producto más el capital ahorrado. Al introducir el impuesto en $t$, el producto en $t$ se ve detraído en una cantidad igual a la cuantía del impuesto. En el periodo $t+1$, el consumo máximo se ve detraído doblemente: por un lado, porque el capital ahorrado es necesariamente menor; por otro, porque la producción ha sido menor al haber sido menor el capital disponible.
				
				Renta disponible antes de impuesto: $Y_t = F(K_t) + K_t$, $Y_{t+1} = F(K_t + F(K_t) - C_t) + K_t + F(K_t) - C_t$
				Después del impuesto: $Y_t = F(K_t) + K_t - G_t$, $Y_\text{t+1} = F(K_t + F(K_t) - C_t - G_t) + K_t + F(K_t) 
				- C_t - G_t$. Como vemos, en $t+1$ el impuesto $G_t$ aparece y lo hace doblemente. }
				\4[] \grafica{expansionfiscal}
				\4 País grande
				\4[] Ahorro afecta a tipo de interés
				\4[] Recta presupuestaria deja de ser recta
		\2 Modelo de horizonte infinito estocástico
			\3 Idea clave
				\4 Análisis de dos periodos caracteriza:
				\4[] $\to$ Interacción balanza de pagos presente y futura
				\4[] $\to$ Efectos de interés
				\4[] $\to$ Análisis normativo
				\4 Horizonte infinito estocástico permite:
				\4[] Sostenibilidad de:
				\4[] $\to$ Déficits CC permanentes
				\4[] $\to$ Roll-over de stocks de deuda
				\4[] Efectos específicos a incertidumbre
				\4[] $\to$ Actitud frente al riesgo
				\4[] $\then$ Introducir composición diferenciada de CF
				\4[] Papel de expectativas sobre interés futuro
				\4[] Contrastación empírica con series largas
				\4[] Caracterizar resultados en relación a tasas de crecimiento
				\4[] Costes de ajuste del capital
			\3 Formulación
				\4 Maximización de la utilidad
				\4[] $\underset{\left\lbrace C_t \right\rbrace}{\max} \quad U = \sum_0^\infty \quad \beta^t u(c_t)$
				\4 Restricción intertemporal
				\4[] $\sum_{t=0}^\infty \left( \frac{1}{1+r} \right)^t \cdot E_t \left\lbrace C_t \right\rbrace = \sum_{t=0}^\infty \left( \frac{1}{1+r} \right)^t E_t \left\lbrace Y_t \right\rbrace + (1+r)B_0 $
				\4[] Incluye condición de no juego de Ponzi + transversalidad
				\4[] $\to$ $\lim_{t \to \infty} \frac{1}{(1+r)^t} \cdot B_t = 0$
				\4[] $\to$ Valor presente de deuda converge a cero
				\4[] $\to$ Economía se mantiene solvente
				\4[] $\to$ Economía no financia al resto del mundo en el infinito
				\4 Óptimo
				\4[] $ u'(C_t) = \beta (1+r) E_t \left(  u'(C_{t+1}) \right)$
				\4 Utilidad cuadrática
				\4[] Consumo sigue paseo aleatorio
				\4[] $\to$ Shocks inesperados a la renta afectan CC
			\3 Implicaciones
				\4 Suavización del patrón de consumo y déficit de CC
				\4[] CC caracterizable en relación a vars. permanentes
				\4[] Valor permanente de secuencia $\left\lbrace X_t \right\rbrace_0^T$
				\4[] $\to$ $\tilde{X}$ cte. que induce = valor presente que $\left\lbrace X_t \right\rbrace_0^T$
				\4[] Asignando valores permanentes:
				\4[] $\to$ $\tilde{Y}_t$, $\tilde{I}_t$, $\tilde{G}_t$
				\4[] CCorriente en relación a variables permanentes
				\4[] $\to$ Derivable de prob. de optimización
				\4[] $\text{CC}_t = B_{t+1} - B_t = (Y_t - \tilde{Y}_t) - (I_t - \tilde{I}_t) - (G_t - \tilde{G}_t)$
				\4[] $\then$ Superávit cuando producción por encima de tendencia
				\4[] $\then$ Déficit cuando inversión por encima de tendencia
				\4[] $\then$ Déficit cuando aumenta gasto discreccional
				\4 Consistencia dinámica
				\4[] Senda de BP óptima es consistente si:
				\4[] $\to$ Sigue siendo óptima en el futuro
				\4[] Ejemplo:
				\4[] Secuencia $\text{BP}_t$, $\text{BP}_{t+1}$, $BP_{t+2}$ óptimos en $t$
				\4[] Si en $t+1$ $\text{BP}_{t+1}$ sigue siendo óptima
				\4[] $\to$ Secuencia de BP es consistente
				\4[] Función de utilidad U determina inconsistencia
				\4[] $\to$ En la práctica, decisiones inconsistentes
				\4[] Ej. decisión inconsistente:
				\4[] $\to$ Superávits de CC futuros son óptimos en presente
				\4[] $\to$ En futuro, superávits presentes ya no son óptimos
				\4[] $\then$ Acreedores consideran inconsistencia
				\4 Impacto de shocks sobre renta
				\4[] Asumiendo:
				\4[] $\to$ Shocks serialmente correlados $Y_t = \rho(Y_{t-1} - \bar{Y}_t) + \epsilon_t$
				\4[] $\to$ Utilidad cuadrática
				\4[] Shock temporal ($\epsilon_t > 0$, $\epsilon_{t+1} = 0$, $\rho<1$)
				\4[] $\to$ Consumo no aumenta tanto como output
				\4[] $\then$ Superávit de CC temporal
				\4[] Shock permanente ($\epsilon_t > 0$, $\epsilon_{t+1} = 0$, $\rho=1$)
				\4[] $\to$ Consumo aumenta igual que output
				\4[] $\then$ CC inalterada
				\4 Paradoja de Feldstein-Horioka
				\4[] Resultados de modelo intertemporal de la BP
				\4[] $\to$ En ec. cerrada, CC=CF=0, ahorro igual inversión
				\4[] $\to$ En ec. abierta, posibles déficit de CC
				\4[] $\then$ Ahorro no tiene por qué igualar inversión
				\4[] $\then$ Ahorro no necesariamente correlado con inversión
				\4[] Feldstein y Horioka (1980)
				\4[] $\to$ Regresión ahorro-inversión sección cruzada, OLS
				\4[] $\then$ Ahorro explica inversión en gran medida
				\4[] $\then$ Poca movilidad del capital
				\4[] Trabajos posteriores tras caída Bretton-Woods
				\4[] $\to$ Menor poder explicativo de ahorro sobre inversión
				\4[] $\to$ Poder significativo en todo caso
				\4[] Posibles explicaciones
				\4[] $\to$ PDesarrollados están cerca de EE y no necesitan K
				\4[] $\to$ Cointegración temporal entre S e I
				\4[] $\to$ Ciclo vital y evolución demográfica relacionan S e I
		\2 Sostenibilidad de la deuda exterior
			\3 Idea clave
				\4 ¿La senda de la deuda exterior es sostenible?
				\4[] Deuda exterior respecto a PIB nominal
				\4[] $\to$ Generalmente, deuda no indexada a producto
				\4[] $\to$ Caracteriza capacidad de pago de deuda
				\4[] ¿Converge a una cantidad determinada?
				\4[] ¿Tiende a crecer al infinito?
				\4 Necesario análisis de sostenibilidad
				\4[] Para inversores
				\4[] Para gestores de deuda
				\4[] Para policy makers
				\4[] $\to$ ¿Hay incentivos a devolver la deuda?
				\4[] $\to$ ¿Es razonable esperar suficiente ahorro?
				\4[] Si diverge, devolución será imposible
				\4[] Si no hay incentivos a pagar, devolución
				\4[] $\Rightarrow$ Crisis de deuda
				\4 Concepto de PIIN
				\4[] Posición de Inversión Internacional Neta
				\4[] Neto entre activos y pasivos exteriores
				\4 Variación de la PII respecto a PIB nominal depende:
				\4[] -- Efectos valoración y volumen (OC)
				\4[] -- Cap./Nec. de financiación (CNF)
				\4[] $\to$ Adquisición de obligaciones netas con exterior
				\4[] $\to$ Depende de competitividad exterior
				\4[] $\to$ Depende de rentas recibidas, incluido interés
				\4[] -- Crecimiento del PIB nominal
				\4 Capacidad de financiación depende:
				\4[] -- Factores autónomos
				\4[] $\to$ Bienes y servicios
				\4[] $\to$ Rentas secundarias
				\4[] $\to$ Cuenta de capital
				\4[] $\then$ $\text{CNF}^*$
				\4[] -- Factores endógenos a la PII
				\4[] $\to$ Rentas primarias
				\4[] $\then$ RPI
			\3 Formulación
				\4 $\Delta \text{PII} = \text{PII}_t - \text{PII}_{t-1} = \text{CNF}^*_t + \text{RPI}_t + OC_t$
				\4[] $\to$ $\text{RPI} =i_t^A \cdot A_{t-1} - i_t^P\cdot P_{t-1}$
				\4[] $\to$ $i_t^A = i_t^P \Rightarrow i \cdot A_{t-1} - i\cdot P_{t-1} = i \cdot \text{PII}_t$
				\4[] $\text{PII}_t - \text{PII}_{t-1} = \text{CNF}^*_t + i_t^A \cdot A_{t-1} - i_t^P\cdot P_{t-1} + OC_t $
				\4[] $\text{PII}_t = \text{CNF}^*_t + (1+i) \text{PII}_{t-1} + OC_t$
				\4[] $\to$ Dividiendo entre $y_t = (1+g) y_{t-1}$
				\4[] \fbox{$\text{pii}_t = \text{cnf}^*_t + \frac{1+r}{1+g}\text{pii}_{t-1} + \text{oc}_t$}
				\4[] $\to$ Estado estacionario: $\text{pii}_t = \text{pii}_{t+1} = \text{pii}_{t-1}$
				\4[] \fbox{$\text{cnf}^* = \frac{g-r}{1+g}\text{pii} + \text{oc}_t$}
			\3 Implicaciones
				\4[] Asumiendo:
				\4[] $\to$ Sin efectos de valoración y volumen ($\text{oc}_t=0$)
				\4[] $\to$ sin rentas secundarias ni transferencias de K
				\4[] $\then$ $\text{CNF}^*$ es balanza comercial
				\4 Crecimiento del PIBn mayor a interés
				\4[] $\text{CNF}^*$ puede ser < 0
				\4[] $\to$ Hasta igualar $-\text{pii} \cdot \frac{g-r}{1+g}$
				\4[] $\to$ Deuda sostenible aun con déficit comercial
				\4[] $\to$ Interpretable como I más rentable que deuda
				\4 Crecimiento del PIB menor que interés
				\4[] $\text{CNF}^*$ <0 sólo sostenible si:
				\4[] $\to$ $\text{pii}_t > 0$
				\4[] $\then$ Déficit sólo sostenible si economía acreedora
				\4[] $\then$ Déficit sostenible hasta límite $\text{pii}\cdot \frac{g-r}{1+g}$
				\4[] $\to$ Interpretable como I menos rentable que deuda
				\4[$\then$] Deseable $\text{CNF}^*$ superior a umbral para evitar ajustes
				\4 Efectos del volumen de deuda
				\4[] Si $g<r$ y deuda muy elevada:
				\4[] $\to$ $\text{cnf}^*$ deberá ser muy grande
				\4[] $\then$ Aumenta esfuerzo exigido
				\4[] $\then$ Más probable impago de la deuda
				\4[] Impacto sobre interés
				\4[] $\to$ Deuda elevada aumenta interés
				\4[] $\then$ Aumenta riesgo de impago
				\4[] $\then$ Espiral de insolvencia
				\4 Factores de inestabilidad
				\4[] (Al margen de análisis anterior)
				\4[] Estructura del pasivo
				\4[] $\to$ Vencimientos cercanos aumentan tensiones
				\4[] $\to$ Entrada de IDE mejora sostenibilidad
				\4[] $\then$ Atracción de equity preferible
				\4[] Aumento de tipos de interés reales
				\4[] $\to$ Si estimados permanentes, $\uparrow$ carga de deuda
				\4[] $\then$ $\uparrow$ Probabilidad de impago
				\4[] $\then$ Posible corte de financiación
				\4[] Cortes de acceso a financiación
				\4[] $\to$ Pánico financiero
				\4[] $\to$ Guerras, problemas jurídicos/políticos
				\4[] $\to$ Profecía autocumplida (múltiples equilibrios)
				\4[] Caída del crecimiento/recesión

	\1 \marcar{Crisis de balanza de pagos}\footnote{Ver \href{https://voxeu.org/content/sudden-stops-primer-balance-payments-crises}{Cecchetti y Schoenholtz en VOXEU (2018)}, \href{https://mpra.ub.uni-muenchen.de/6982/1/MPRA_paper_6982.pdf}{Reinhart y Calvo (2000)} y \href{https://ideas.repec.org/p/wbk/wbrwps/7639.html} .}
		\2 Idea clave
			\3 Contexto
				\4 Cita de Dornbusch
				\4[] ``It is not speed that kills, but the sudden stop''
				\4[] Viejo dicho de mercados financieros
				\4 Ajustes bruscos son costosos
				\4 A lo largo de la exposición
				\4[] Componentes de la balanza de pagos
				\4[] Vía de ajuste para igualar sumas de saldos CC, CK, CF
				\4 Ahorro e inversión
				\4[] Exceso de ahorro sobre inversión
				\4[] $\to$ Superávit por cuenta corriente
				\4[] $\to$ Salida de capital
				\4[] $\then$ Necesario encontrar activos de inversión
				\4[] $\then$ Activos aumentan más que pasivos
				\4[] Exceso de inversión sobre ahorro
				\4[] $\to$ Déficit por cuenta corriente
				\4[] $\to$ Entrada de capital
				\4[] $\then$ Necesario encontrar contrapartes para pasivo nacional
				\4[] $\then$ Pasivos aumentan más que activos
				\4 Problemas de balanza de pagos
				\4[] Desequilibrios graves en cuenta corriente
				\4[] Coste de financiación elevado
				\4[] Cambios bruscos en flujos de capital
				\4[] $\to$ De entrantes a salientes rápidamentemente
				\4[] Reacciones excesivas a shocks de información
				\4[] $\to$ Pánicos de inversores
				\4[] $\to$ Wake-up calls
				\4[] $\to$ Contagio
				\4[] $\to$ ...
			\3 Objetivos
				\4 ¿Qué sucede cuando no es posible financiar déficit?
				\4 ¿Cómo se ajusta la balanza de pagos?
				\4 ¿Qué implicaciones de política económica?
			\3 Resultados
				\4 Ajustes bruscos son mucho más costosos que graduales
				\4 Crisis de BP ligadas a recesiones y crisis bancarias
				\4 Instrumentos de ajuste son importantes para evitar crisis
				\4 Programas de ajuste del FMI para suavizar ajuste
				\4 Reducción de factores de riesgo importante
				\4 Liberalizaciones cuenta financiera deben evitar $\uparrow$ riesgos
		\2 Formulación
			\3 Identidades del ahorro, la inversión y la entrada de capital
				\4 Renta Nacional Bruta Disponible
				\4[] $\text{RNBD} = C+G+I+\text{NX}+ \text{RP}+\text{RS}$
				\4 Ahorro Nacional
				\4[] $S= \text{RNBD} - C -G = I+ \text{NX} + \text{RP}+ \text{RS}$
				\4 Exceso de ahorro nacional, CCorriente y CFinanciera
				\4[] $S-I+\text{CK} = \underbrace{\text{NX}+\text{RP} + \text{RS}}_{\text{CC}} +\text{CK} = \text{VNA} - \text{VNP}$
				\4 Ahorro insuficiente para cubrir inversión
				\4[] Necesario aumentar pasivos netos
				\4[] $\to$ ¿Quién los acepta?
				\4[] $\to$ ¿Quién provee el capital?
				\4[] $\to$ ¿A qué coste?
				\4[] $\to$ ¿Es posible en todas circunstancias encontrar financiación?
			\3 Sudden stops y reversiones de flujos de capital
				\4 Ocurren relativamente frecuentemente
				\4 Especialmente en países en desarrollo/emergentes
				\4 Persisten al menos un año, generalmente
				\4 Sudden stop y flow reversal al tiempo
				\4 Inducen depreciación del tipo de de cambio
				\4[] No quedan otras herramientas de ajuste disponibles
				\4 Inducen caídas fuertes del PIB via $\downarrow$ I
			\3 Factores de riesgo de sudden stops
				\4 Libre movimiento de capital
				\4[] Venta de pasivos nacionales es menos costosa
				\4 Préstamos de corto plazo
				\4[] Prestamistas pueden inducir sudden-stop
				\4[] $\to$ Simplemente evitando renovación de préstamos
				\4 Endeudamiento en moneda extranjera
				\4[] Banco central
				\4[] $\to$ No puede proveer liquidez
				\4[] $\to$ No puede monetizar deuda
				\4 Pequeño sector exportador
				\4[] Si flujos de capital se revierten
				\4[] $\to$ Necesario aumentar exportaciones
				\4[] Si sector exportador es pequeño
				\4[] $\to$ Necesario reorganizar producción
				\4[] $\then$ Muy costoso
				\4 Aumento de percepciones globales del riesgo
				\4[] Capital se desplaza hacia activos percibidos como seguros
				\4 TCN fijo + libre movimiento de K
				\4[] Vulnerabilidad clásica
				\4[] Incentiva ataques especulativos de primera generación
				\4 Stock de reservas pequeño
				\4[] Asiáticos aprenden lección tras crisis de 90s
		\2 Implicaciones
			\3 Flujos de capital pueden ser desestabilizantes
				\4 Pueden alimentar inversión excesiva
				\4 Presiones especulativas sobre tipo de cambio fijo
			\3 Liberalización de CF puede tener inconvenientes
				\4 Exceso de inversión
				\4 Apreciación del tipo de cambio
				\4 Sudden stops y reversiones del flujo de K
				\4 Crisis financieras
			\3 Sistema financiero doméstico es factor clave
				\4 Relaciones con proveedores de capital extranjeros
				\4 Estructura de incentivos de bancos nacionales
				\4[] Determina dependencia de flujos de capital extranjeros
				\4[] $\to$
			\3 Uniones monetarias requieren mecanismos emergencia
				\4 Target 2 en eurozona
				\4[] Ante fuga de capitales
				\4[] $\to$ De bancos privados en un país en crisis
				\4[] $\to$ Hacia países centrales
				\4[] BCNs incurren saldos acreedores con BCE
				\4[] $\to$ BCE provee liquidez automáticamente
		\2 Valoraciones
			\3 Suceso recurrente
				\4 Crisis latinoamericanas de los 80
				\4 Crisis asiática de los 90
				\4 Crisis de la eurozona de 2010s
			\3 Papel clave del FMI
				\4 Programas de asistencia financiera
				\4[]
				\4 Programas de asistencia concesional
				\4[] Especial importancia en crisis actual
				\4 Programas de reforma
				\4[]
				\4 Fracaso en los 90
				\4[] Propone aumento fuerte de tipos de interés
				\4[] $\to$ Incentivar entrada de capital
				\4[] Acabo provocando recesión
				\4[] $\to$ Agudizó crisis
	\1[] \marcar{Conclusión}
		\2 Recapitulación
			\3 Enfoque de flujos
			\3 Enfoque de stocks
			\3 Enfoque intertemporal y sostenibilidad
		\2 Idea final
			\3 Impacto de los modelos analizados
				\4 Enfoque intertemporal predominante
				\4[] Buena explicación de crisis de deuda
				\4[] Fácil extensión e incorporación de otros modelos
				\4[] Posible análisis normativo
				\4 Contrastación empírico de enf. intertemporal
				\4[] Numerosas anomalías
				\4[] $\to$ Feldstein y Horioka, Lucas,
			\3 Impacto de análisis de BP en policy-making
				\4 Generalizado
				\4 Patrón oro
				\4 Bretton Woods
				\4 Caída de Bretton Woods
				\4 Crisis monetarias y financieras
				\4 Elección de régimen cambiario
				\4 Unión Europea
				\4[] Procedimiento de Deseq. Macroeconómicos
				\4[] $\to$ Saldo por cuenta corriente
				\4[] $\to$ PIIN
				\4[] Debate sobre:
				\4[] $\to$ superávits alemanes
				\4[] $\to$ necesidad de transferencias fiscales
				\4[] ...
			\3 Economía política
				\4 Reducción de desequilibrios de BP
				\4[] Requiere de políticas de ajuste
				\4[] $\to$ Ganadores y perdedores
				\4 Análisis de economía política
				\4[] ¿Quién gana y quién pierde con ajuste de BP?
				\4[] ¿Qué ajustes de BP son viables políticamente?
\end{esquemal}























\graficas

\begin{axis}{4}{Enfoque de elasticidades. Demanda y oferta de divisas dando lugar a un sólo equilibrio estable: $\eta_x > 1 \forall \, r$}{$D,S$}{$E$}{unicoequilibrio}
	% CURVA DE DEMANDA
	\draw[-] (1,4) -- (3.5,0);
	\node[above] at (1,4){\small D};
	
	% CURVA DE OFERTA
	\draw[-] (0.5,0.5) -- (4,4);
	\node[above] at (4,4){\small S};
	
	% Equilibrio estable
	\node[circle, fill=black, inner sep=0pt, minimum size=5pt] (a) at (2.15,2.15) {};

	% Exceso de demanda negativo
	\draw[decorate,decoration={brace,amplitude=3pt, mirror},xshift=0pt,yshift=-0.5cm] (2.9,3.5) -- (1.7,3.5) node[black,midway,xshift=2pt, yshift=0.33cm] {\tiny $\text{ES}<0$ $\then$ $\downarrow E$ };
	
	% Exceso de demanda positivo
	\draw[decorate,decoration={brace,amplitude=3pt, },xshift=0pt,yshift=-0.5cm] (2.8,1.5) -- (1.1,1.5) node[black,midway,xshift=2pt, yshift=-0.33cm] {\tiny $\text{ES}>0$ $\then$ $\uparrow E$ };
	
\end{axis}


\begin{axis}{4}{Demanda y oferta de divisas dando lugar a dos equilibrios: uno estable y otro inestable}{$D,S$}{$E$}{multiplesequilibrios}
	% CURVA DE DEMANDA
	\draw[-] (1,4) -- (3.5,0);
	\node[right] at (1,4){\small D};
	
	% CURVA DE OFERTA: SEGMENTO CRECIENTE
	\draw[-] (1,0.2) to [out=30, in=270](3.5,1.5); 
	\draw[-] (3.5,1.5) to [out=90, in=-20](0.5,3); 
	\draw[-] (0.5,3) to [out=160, in=280](0.1,4);
	\node[right] at (3.5,1.5){S};
	
	% Equilibrio inestable
	\node[circle, fill=black, inner sep=0pt, minimum size=5pt] (a) at (1.8,2.73) {};
	\draw[-{Latex}] (2.6, 3.5) -- (1.85,2.8);
	\node[right] at (2.6,3.5){\tiny Equilibrio inestable};

	% Equilibrio estable
	\node[circle, fill=black, inner sep=0pt, minimum size=5pt] (a) at (3,0.75) {};
	\draw[-{Latex}] (3.9,0.75) -- (3.1,0.75);
	\node[right] at (3.9,0.75){\tiny Equilibrio estable};
	
\end{axis}

\begin{axis}{4}{Curva J: depreciación del tipo de cambio nominal que induce primero un empeoramiento del saldo comercial y una posterior mejora respecto del saldo inicial.}{$t$}{$B$}{curvaj}
	% Saldo de cuenta comercial
	\draw[-] (0,1.5) -- (2,1.5) to [out=280, in=180](2.5,1) to [out=0,in=270](3.2,2) to [out=90,in=180](4.5,3.5);
	\node[right] at (4.5,3.5){\tiny Saldo de CC};
	
	% Instante de la devaluación
	\draw[dashed] (2,1.5) -- (2,-0.1);
	\node[below] at (2,-0.2){\tiny Depreciación};
	
\end{axis}

\begin{axis}{4}{Modelo de cartera: equilibrio parcial con dinero, bonos nacionales y bonos extranjeros.}{$F$}{$i$}{carteraequilibrio}
	% Equilibrio en mercado de dinero
	\draw[-] (0.5,0.5) -- (3.5,3.5);
	\node[right] at (3.5,3.5){\small L};
	
	% Equilibrio en mercado de bonos nacionales
	\draw[-] (0,2) -- (4,2);
	\node[right] at (4,2){\small N};
	
	% Equilibrio en mercado de bonos extranjeros
	\draw[-] (0.5,3.5) -- (3.5,0.5);
	\node[right] at (3.5,0.5){\small F};
	
\end{axis}

\begin{axis}{4}{Modelo de cartera: equilibrio parcial con dinero, bonos nacionales y bonos extranjeros. Efecto de una expansión monetaria}{$F$}{$i$}{carteraexpansionmonetaria}
	% Equilibrio en mercado de dinero
	\draw[-] (0.5,0.5) -- (3.5,3.5);
	\node[right] at (3.5,3.5){\small L};
	
	% Equilibrio en mercado de bonos nacionales
	\draw[-] (0,2) -- (4,2);
	\node[right] at (4,2){\small N};
	
	% Equilibrio en mercado de bonos extranjeros
	\draw[-] (0.5,3.5) -- (3.5,0.5);
	\node[right] at (3.5,0.5){\small F};
	
	% Equilibrio inicial
	\node[circle, fill=black, inner sep=0pt, minimum size=5pt] (a) at (2,2) {};
	
	% Equilibrio en mercado de dinero tras expansión
	\draw[dashed] (1.5,0.5) -- (4.5,3.5);
	\node[right] at (4.5,3.5){\small L};
	
	% Equilibrio en mercado de bonos nacionales tras expansión
	\draw[dashed] (0,1.5) -- (4,1.5);
	\node[right] at (4,1.5){\small N'};
	
\end{axis}


\begin{axis}{4}{Representación de la balanza de pagos en dos periodos como resultado de una optimización de la utilidad.}{$c_t$}{$c_{t+1}$}{dosperiodos}
	% Recta presupuesta con pendiente tg alpha = 1+r
	\draw[-] (0,3.5) -- (3.5,0);
	
	% Dotación inicial
	\node[circle, fill=black, inner sep=0pt, minimum size=3pt] (a) at (1.75,1.75) {};
	\draw[dashed] (0,1.75) -- (1.75,1.75) -- (1.75,0);
	\node[below] at (1.75,0){$y_t$};
	\node[left] at (0,1.75){$y_{t+1}$};
	
	% Curva de indiferencia
	\draw[-] (1.55,3) to [out=270,in=180](4.05,0.5);
	
	% Equilibrio intertemporal de balanza de pagos
	\node[circle, fill=black, inner sep=0pt, minimum size=3pt] (a) at (2.25,1.25) {};
	\draw[dashed] (0,1.25) -- (2.25,1.25) -- (2.25,0);
	\node[left] at (0,1.25){$c_{t+1}$};
	\node[below] at (2.25,0){$c_t$};
	
	% Superávit de CC en t+1
	\draw[decorate,decoration={brace,amplitude=3pt, mirror},xshift=-0.75cm,yshift=0cm] (0,1.75) -- (0,1.25) node[black,midway,xshift=-15pt, yshift=0cm] {\tiny $\text{CC} > 0$};
	
	% Superávit de CC en t
	\draw[decorate,decoration={brace,amplitude=3pt, mirror},xshift=0pt,yshift=-0.5cm] (1.75,0) -- (2.25,0) node[black,midway,xshift=0pt, yshift=-10pt] {\tiny $\text{CC} < 0$ };
	
\end{axis}

\begin{axis}{4}{Representación gráfica del aumento de utilidad que resulta de una apertura de la cuenta financiera.}{$ $}{$c_{t+1}$}{aperturacuentafinanciera}
	% Extensión del eje de abscisas
	\draw[-] (4,0) -- (6,0);
	\node[below] at (6,0){$c_t$};

	% Recta presupuesta de autarquía con pendiente tg alpha = 1+r
	\draw[dashed] (0,3.5) -- (3.5,0);
	
	% Dotación inicial
	\node[circle, fill=black, inner sep=0pt, minimum size=3pt] (a) at (1.75,1.75) {};
	\draw[dashed] (0,1.75) -- (1.75,1.75) -- (1.75,0);
	\node[below] at (1.75,0){$y_t$};
	\node[left] at (0,1.75){$y_{t+1}$};
	
	% Curva de indiferencia de autarquía
	\draw[dashed] (1.17,3.17) to [out=270,in=180](3.17,1.17);
	
	% Recta tras apertura de la cuenta financiera
	\draw[-] (0,2.45) -- (6,0);
	
	% Curva de indiferencia tras apertura de la cuenta financiera
	\draw[-] (1.28,3.28) to [out=270,in=180](3.28,1.28);
	
	% Equilibrio de balanza de pagos tras apertura
	\node[circle, fill=black, inner sep=0pt, minimum size=3pt] (a) at (2.5,1.45) {};
	\draw[dashed] (0,1.45) -- (2.5,1.45) -- (2.5,0);
	\node[left] at (0,1.45){$c_{t+1}$};
	\node[below] at (2.5,0){$c_t$};
\end{axis}

La recta y la curva de indiferencia discontinuas muestran el bienestar y el tipo de interés en situación de autarquía previa a la apertura de la cuenta financiera. La línea continua muestra la restricción intertemporal de la balanza de pagos en contexto de apertura, y la curva de indiferencia continua representa la utilidad alcanzada, que supera a la de la situación de autarquía.

\begin{axis}{4}{Representación gráfica de la pérdida de bienestar que resulta de una restricción al endeudamiento cuando el equilibrio óptimo de la balanza de pagos implica incurrir en déficits por cuenta corriente en el presente y superávits en el futuro.}{$c_t$}{$c_{t+1}$}{restriccionendeudamiento}
	% Recta presupuestaria
	\draw[-] (0,3) -- (1.75,1.25) -- (1.75,0);
	\draw[dashed] (1.5,1.5) -- (3,0);
	
	% Curva de indiferencia de óptimo sin restricción al endeudamiento
	\draw[dashed] (1.72,2.13) to [out=270,in=180](3.72,0.13);
	
	% Óptimo sin restricción al endeudamiento
	\node[circle, fill=black, inner sep=0pt, minimum size=3pt] (a) at (2.3,0.7) {};
	
	% Curva de indiferencia con restricción al endeudamiento
	\draw[-] (1.6,2.01) to [out=270,in=180](3.6,0.01);
	
	% Óptimo con restricción al endeudamiento
	\node[circle, fill=black, inner sep=0pt, minimum size=3pt] (a) at (1.75,1.25) {};
\end{axis}

\begin{axis}{4}{Representación gráfica del efecto de un aumento del tipo de interés.}{$c_t$}{$c_{t+1}$}{variaciondeinteres}
	% Recta presupuestaria intertemporal con interés inicial
	\draw[dashed] (0,2.25) -- (3.75,0);
	
	% Curva de indiferencia de interés inicial
	\draw[dashed] (1.31,2.6) to [out=270,in=180](3.31,0.6);

	% Óptimo con interés inicial
	\node[circle, fill=black, inner sep=0pt, minimum size=3pt] (a) at (2.1,1) {};
	\draw[dashed, line width=0.1pt] (0,1) -- (2.1,1) --(2.1,0);
	\node[below] at (2.1,0){\tiny $c_t$};
	\node[left] at (0,1){\tiny $c_{t+1}$};
	
	% Recta presupuestaria tras subida de interés, descontando efecto renta, sólo efecto sustitución
	\draw[dotted] (0,4.22) -- (2.25,0.47);
	
	% Óptimo manteniendo curva de utilidad - sólo con efecto sustitución
	\node[circle, fill=black, inner sep=0pt, minimum size=3pt] (a) at (1.57,1.6) {};
	
	% Resta presupuestaria intertemporal tras subida de interés
	\draw[-] (0,3.75) -- (2.25,0);
	
	% Curva de indiferencia tras subida de interés
	\draw[-] (1.13,2.43) to [out=270,in=180](3.13,0.43);

	% Óptimo tras subida de interés
	\node[circle, fill=black, inner sep=0pt, minimum size=3pt] (a) at (1.4,1.4) {};
	\draw[-,line width=0.1pt] (0,1.4) -- (1.4,1.4) -- (1.4,0);
	\node[below] at (1.4,0){\tiny $c_t'$};
	\node[left] at (0,1.4){\tiny $c_{t+1}'$};
	
	% Diferencia de consumo presente tras aumento del interés
	\draw[decorate,decoration={brace,amplitude=3pt},xshift=-0.6cm,yshift=0cm] (0,1) -- (0,1.4) node[black,midway,xshift=-15pt, yshift=0cm] {\tiny $\Delta Y > 0$};

	% Diferencia de consumo futuro tras aumento del interés
	\draw[decorate,decoration={brace,amplitude=3pt},xshift=0cm,yshift=-0.4cm] (2.1,0) -- (1.4,0) node[black,midway,xshift=0pt, yshift=-0.25cm] {\tiny $\Delta Y < 0$};
\end{axis}

El gráfico muestra como un aumento del interés --que representa un encarecimiento del consumo presente en términos del consumo futuro- induce una reducción del consumo presente y un aumento del consumo futuro, independientemente del equilibrio inicial. Estas variaciones son la suma de los efectos renta y sustitución. La recta de puntos representa el efecto sustitución.

\begin{axis}{4}{Representación gráfica del teorema de separación de las decisiones de inversión y consumo y su aplicación a la balanza de pagos: un país puede incurrir en déficit por cuenta corriente el presente para financiar consumo e inversión que permitirán un superávit futuro.}{}{$c_{t+1}$}{separacioninversionconsumo}
	% Extensión del eje de abscisas
	\draw[-] (4,0) -- (6,0);
	\node[below] at (6,0){$c_t$};
	
	% Frontera intertemporal de posibilidades de producción
	\draw[-] (0,3.5) to [out=-10, in=100](3.5,0);
	\node[left] at (0,3.5){\tiny $y\left( k_t + y(k_t) \right)$};
	\node[below] at (3.5,0){\tiny $y(k_t) + k_t$};
	
	% Dotación inicial
	\node[circle, fill=black, inner sep=0pt, minimum size=3pt] (a) at (2.34,2.34) {};
	\draw[dashed] (0,2.34) -- (2.34,2.34) -- (2.34,0);
	\node[left] at (0,2.34){\tiny $y(k_t)$};
	\node[below] at (2.34,0){\tiny $y(k_t)$};
	
	% Recta presupuestaria
	\draw[-] (0,3.75) -- (5.5,0.95);
	
	% Equilibrio sobre FPP
	\node[circle, fill=black, inner sep=0pt, minimum size=3pt] (a) at (1.38,3.05) {};
	\draw[dashed] (0,3.05) -- (1.38,3.05) -- (1.38,0);
	\node[left] at (0,3.05){\tiny $y(k_t + i_t)$};
	\node[below] at (1.38,0){\tiny $y(k_t) - i_t$};
	
	% Curva de indiferencia
	\draw[-] (3.5,3.2) to [out=270,in=180](5.5,1.2);
	
	% Equilibrio de balanza de pagos
	\node[circle, fill=black, inner sep=0pt, minimum size=3pt] (a) at (4.61,1.39) {};
	\draw[dashed] (0,1.39) -- (4.61,1.39) -- (4.61,0);
	\node[left] at (0,1.39){\tiny $c_{t+1}$};
	\node[below] at (4.61,0){\tiny $c_t$};
	
	% Superávit por cuenta corriente en el periodo t
	\draw[decorate,decoration={brace,amplitude=3pt, mirror},xshift=0cm,yshift=-0.4cm] (1.38,0) -- (4.61,0) node[black,midway,xshift=0pt, yshift=-0.3cm] {\tiny Déficit};
	
	% Déficit por cuenta corriente en t+1
	\draw[decorate,decoration={brace,amplitude=3pt, mirror},xshift=-1.1cm,yshift=0cm] (0,3.05) -- (0,1.39) node[black,midway,xshift=-15pt, yshift=0cm] {\tiny Superávit};
\end{axis}

\begin{axis}{4}{Efecto de una expansión fiscal que reduce el output disponible en un contexto de modelo intertemporal de la balanza de pagos con producción.}{}{$c_{t+1}$}{expansionfiscal}
	% FPP pre impuesto
	\draw[-] (0,3.9) to [out=-10, in=100](3.9,0);
	
	% FPP post-impuesto
	\draw[-] (0,3) to [out=-5, in=95](3.5,0);
	
	% Recta presupuestaria pre-impuesto
	\draw[-] (0.75,4) -- (4,1.7);
	
	% Curva de indiferencia pre impuesto
	\draw[-] (1.1,4.65) to [out=280, in=170](3.1,2.65);
	
	% Equilibrio pre-impuesto
	\node[circle, fill=black, inner sep=0pt, minimum size=3pt] (a) at (2,3.1) {};
	\node[right] at (2,3.12){\tiny A};
	
	% Recta presupuestaria post-impuesto
	\draw[-] (0.91,3.15) -- (4.16,0.85);
	
	% Curva de indiferencia post impuesto
	\draw[-] (0.68,4.23) to [out=280, in=170](2.68,2.23);

	% Producción pre-impuesto
	\node[circle, fill=black, inner sep=0pt, minimum size=3pt] (a) at (2.2,2.22) {};
	\node[below] at (2.2,2.2){\tiny B};
	
	% Equilibrio post-impuesto
	\node[circle, fill=black, inner sep=0pt, minimum size=3pt] (a) at (1.61,2.67) {};
	\node[right] at (1.61,2.67){\tiny C};
	
	% Curva de indiferencia pre-impuesto
%	
%%	% Extensión del eje de abscisas
%%	\draw[-] (4,0) -- (6,0);
%%	\node[below] at (6,0){$c_t$};
%%	
%%	% Frontera intertemporal de posibilidades de producción
%%	\draw[dashed] (0,3.5) to [out=-10, in=100](3.5,0);
%%	%\node[left] at (0,3.5){\tiny $y\left( k_t + y(k_t) \right) + k_t + y(k_t)$};
%%	%\node[below] at (3.8,0){\tiny $y(k_t) + k_t$};
%%	
%%	% Dotación inicial
%%	\node[circle, fill=black, inner sep=0pt, minimum size=3pt] (a) at (2.34,2.34) {};
%%	%\draw[dashed] (0,2.34) -- (2.34,2.34) -- (2.34,0);
%%	%\node[left] at (0,2.34){\tiny $y(k_t)$};
%%	%\node[below] at (2.34,0){\tiny $y(k_t)$};
%%	
%%	% Recta presupuestaria inicial con cuenta corriente equilibrada
%%	\draw[dashed] (0.7,4) -- (4.7,0);
%%	
%%	% Curva de indiferencia de óptimo inicial
%%	\draw[dashed] (1.8,3.1) to [out=300,in=180](4.1,1.6);
%%
%%	% Frontera intertemporal de posibilidades de producción con impuesto
%%	\draw[-] (0,2.8) to [out=-10, in=100](3.1,0);
%%	\node[left] at (0,3.2){\tiny $y\left( k_t + y(k_t) -g_t \right) + k_t + y(k_t) - g_t$};
%%	\node[below] at (2.7,0){\tiny $y(k_t) + k_t-g_t$};
%%	
%%	% Recta presupuestaria con impuesto
%%	\draw[-] (0.20,3.75) -- (4.20,-0.25);
%%	
%%	% Producción óptima tras impuesto
%%	\node[circle, fill=black, inner sep=0pt, minimum size=3pt] (a) at (1.65,2.34) {};
%%	
%%	% Curva de indiferencia de óptimo tras impuesto
%%	\draw[-] (1.45,2.75) to [out=300,in=180](3.75,1.25);
%%	
%%	% Consumo óptimo tras impuesto
%%	\node[circle, fill=black, inner sep=0pt, minimum size=3pt] (a) at (2,2) {};
%	
%	% Recta presupuestaria
%	%\draw[-] (0,3.75) -- (5.5,0.95);
%	
%	% Equilibrio sobre FPP
%	%\node[circle, fill=black, inner sep=0pt, minimum size=3pt] (a) at (1.38,3.05) {};
%	%\draw[dashed] (0,3.05) -- (1.38,3.05) -- (1.38,0);
%	%\node[left] at (0,3.05){\tiny $y(k_t + i_t)$};
%	%\node[below] at (1.38,0){\tiny $y(k_t) - i_t$};
%	
%	% Curva de indiferencia
%	%\draw[-] (3.5,3.2) to [out=270,in=180](5.5,1.2);
%	
%	% Equilibrio de balanza de pagos
%	%\node[circle, fill=black, inner sep=0pt, minimum size=3pt] (a) at (4.61,1.39) {};
%	%\draw[dashed] (0,1.39) -- (4.61,1.39) -- (4.61,0);
%	%\node[left] at (0,1.39){\tiny $c_{t+1}$};
%	%\node[below] at (4.61,0){\tiny $c_t$};
%	
%	% Superávit por cuenta corriente en el periodo t
%	%\draw[decorate,decoration={brace,amplitude=3pt, mirror},xshift=0cm,yshift=-0.4cm] (1.38,0) -- (4.61,0) node[black,midway,xshift=0pt, yshift=-0.3cm] {\tiny Déficit};
%	
%	% Déficit por cuenta corriente en t+1
%	%\draw[decorate,decoration={brace,amplitude=3pt, mirror},xshift=-1.1cm,yshift=0cm] (0,3.05) -- (0,1.39) node[black,midway,xshift=-15pt, yshift=0cm] {\tiny Superávit};
\end{axis}

\conceptos

\concepto{Derivación del efecto de un cambio en el gasto autónomo sobre la cuenta comercial}

El efecto del cambio en el gasto autónomo $d C_0$ sobre la cuenta comercial es la diferencia entre el efecto sobre el output y el efecto sobre la absorción:

\begin{align*}
d B = d Y - d A = k \cdot d C_0 - (1+ck) \cdot d C_0 = (k - 1 - ck) d C_0 \\
= \left( \frac{1}{1-c+m} - 1 - \frac{c}{1-c+m}\right) d C_0 = \left( \frac{1-1+c-m - c}{1-c+m}\right) d C_0 = \left( -m \cdot \frac{1}{1-c+m} \right) d C_0 = \\
d B = -m k \cdot d C_0
\end{align*}

\concepto{Extensión de Mill al mecanismo de flujo-especie}

La mejor explicación es la de Friedman (1953), pág. 166:

<< Changes in interest rates are perhaps best classified under
this heading of changes in internal prices. Interest-rate changes
have in the past played a particularly important role in adjust-
ment to external changes, partly because they have been sus-
ceptible to direct influence by the monetary authorities, and
partly because, under a gold standard, the initial impact of a
tendency toward a deficit or surplus was a loss or gain of gold
and a consequent tightening or ease in the money market. The
rise in the interest rate produced in this way by an incipient
deficit increased the demand for the currency for capital purposes
and so offset part or all of the deficit. This reduced the rate at
which the deficit had to be met by a decline in internal prices,
which was itself set in motion by the loss of gold and associated
decrease in the stock of money responsible for the rise in in-
terest rates. Conversely, an incipient surplus increased the stock
of gold and eased the money market. The resulting decline in
the interest rate reduced the demand for the currency for capital
purposes and so offset part or all of the surplus, reducing the
rate at which the surplus had to be met by the rise in internal
prices set in motion by the gain of gold and associated rise in
the stock of money. >>

\preguntas

\seccion{Test 2019}

\textbf{31.} De acuerdo al enfoque monetario de Balanza de Pagos (Johnson, 1972), ¿cuáles son los supuestos fundamentales en los que se basa el modelo para concluir que los desajustes en el mercado monetario son la causa fundamental de los desequilibrios de la Balanza de Pagos?

\begin{itemize}
	\item[a] Tipo de cambio fijo: perfecta movilidad de capitales, pero sustituibilidad imperfecta entre activos nacionales y extranjero; y pleno empleo de recursos, entre otros.
	\item[b] Tipo de cambio flexible; movilidad imperfecta de capitales; y ley del precio único, y entre otros.
	\item[c] Tipo de cambio fijo; perfecta movilidad de capitales y perfecta sustituibilidad entre activos nacionales e internacionales; y pleno empleo de recursos, entre otros.
	\item[d] Tipo de cambio fijo, movilidad imperfecta de capitales y ley del precio único, entre otros.
\end{itemize}


\seccion{Test 2018}

\textbf{30.} El concepto de ``efecto curva J'' señala que, en una economía con déficit comercial, la devaluación de la moneda nacional, ceteris paribus:

\begin{itemize}
	\item[a] Puede agravar el déficit comercial a largo plazo, al adaptarse gradualmente la demanda de exportaciones a los precios resultantes del nuevo tipo de cambio.
	\item[b] Puede agravar el déficit comercial a corto plazo, al encarecer las importaciones durante un periodo de tiempo en el que la demanda de exportaciones no se ha adaptado a los precios resultantes del nuevo tipo de cambio.
	\item[c] Puede aliviar el déficit comercial a corto plazo, pues la demanda de exportaciones se adapta rápidamente a los precios resultantes del nuevo tipo de cambio.
	\item[d] Ninguna de las opciones anteriores es correcta.
\end{itemize}

\seccion{Test 2017}
\textbf{30.} Si se dan las condiciones para que se verifique la llamada ``curva en J'', en ese caso una devaluación provocará:

\begin{itemize}
	\item[a] Un empeoramiento inicial de la balanza comercial y un empeoramiento aún mayor a largo plazo.
	\item[b] Un empeoramiento a corto plazo de la balanza comercial y una mejora a largo plazo.
	\item[c] Un empeoramiento inicial de la balanza comercial y un empeoramiento a largo plazo.
	\item[d] Una mejora a corto plazo de la balanza comercial y una mejora aún mayor a largo plazo.
\end{itemize}

\seccion{Test 2014}
\textbf{32.} De acuerdo con el modelo presentado por Obstfeld y Rogoff (1995) de enfoque intertemporal de determinación de la balanza de pagos, señale la respuesta correcta:

\begin{itemize}
	\item[a] El déficit por cuenta corriente es insostenible cuando el tipo de interés real es menor que el crecimiento del país.
	\item[b] La evolución negativa de la Relación Real de Intercambio deteriora la sostenibilidad del déficit por cuenta corriente.
	\item[c] Una bajada de los tipos de interés puede poner en peligro la sostenibilidad del déficit por cuenta corriente.
	\item[d] En ningún caso, el déficit por cuenta corriente es sostenible a largo plazo.
\end{itemize}


\seccion{Test 2009}

\textbf{30}. Una economía abierta, que mantiene un sistema de tipo de cambio fijo, sufre una perturbación que afecta al mercado monetario de forma contractiva. El consiguiente ajuste de la balanza de pagos se llevaría a cabo mediante:
\begin{enumerate}
	\item[a] Un aumento de las reservas de divisas.
	\item[b] Una disminución de las reservas de divisas.
	\item[c] Una devaluación del tipo de cambio.
	\item[d] Una apreciación del tipo de cambio.
\end{enumerate}

\seccion{Test 2008}
\textbf{28.} Si el tipo de cambio real se deprecia en un 8\%, ¿cuando se cumpliría la condición Marshall-Lerner?

\begin{itemize}
	\item[a] Las exportaciones aumentan en un 4\% y las importaciones se reducen en un 5\%.
	\item[b] Las exportaciones aumentan en un 2\% y las importaciones se reducen en un 4\%.
	\item[c] Las exportaciones aumentan en un 3\% y las importaciones se reducen en un 2\%.
	\item[d] Las exportaciones aumentan en un 2,5\% y las importaciones se reducen en un 3,5\%.
\end{itemize}

\textbf{30.} En el primer tramo de la curva J significa:

\begin{itemize}
	\item[a] Que no se ha tenido en cuenta la condición Marshall-Lerner.
	\item[b] Que se ha practicado una política revaluatoria ineficiente.
	\item[c] Que se importa menos productos.
	\item[d] Que se anula cualquier incidencia positiva de la devaluación sobre las exportaciones netas.
\end{itemize}

\textbf{31.} Un aumento de los precios extranjeros, con tipos de cambio flexibles, según la teoría keynesiana de ajuste de la balanza de pagos de los tipos de cambio:
\begin{enumerate}
	\item[a] Mejorarán las exportaciones.
	\item[b] Mejorará el saldo de la Balanza de Pagos.
	\item[c] Mejorará la balanza por cuenta corriente y conducirá a una apreciación.
	\item[d] Empeorará la balanza por cuenta corriente, conduciendo a una depreciación.
\end{enumerate}

\seccion{Test 2007}

\textbf{31.} La presentación habitual de la condición de Marshall-Lerner, que establece las condiciones para que una devaluación mejore la balanza de cuenta corriente, exige que la suma de los valores absolutos de las elasticidades de exportaciones e importaciones con respecto al tipo de cambio sea superior a la unidad. Una correcta aplicación de la condición así formulada requiere:
\begin{itemize}
	\item[a] Una situación inicial de equilibrio de la balanza de bienes y servicios.
	\item[b] Que las curvas de oferta de exportaciones e importaciones sean perfectamente elásticas.
	\item[c] Una situación inicial de equilibrio de la balanza de bienes y servicios y que las curvas de oferta de exportaciones e importaciones sean perfectamente elásticas.
	\item[d] Ninguna de las anteriores.
\end{itemize}


\seccion{Test 2004}
\textbf{28.} De acuerdo con el enfoque monetario de la balanza de pagos, un déficit de la balanza de pagos en un país significaría que:
\begin{itemize}
	\item[a] Existe un exceso de demanda de dinero en dicho país, y un exceso de oferta de dinero en el resto del mundo.
	\item[b] Existe un exceso de oferta de dinero, tanto en dicho país como en el resto del mundo.
	\item[c] Existe un exceso de oferta de dinero en dicho país, acompañado de equilibrio en el mercado de dinero del resto del mundo, siempre y cuando el país en cuestión sea pequeño.
	\item[d] Existe un exceso de oferta de dinero en dicho país, y un exceso de demanda de dinero en el resto del mundo.
\end{itemize}

\notas

\textbf{2019}: \textbf{13.} C

\textbf{2018}: \textbf{30.} B

\textbf{2017}: \textbf{13.} B

\textbf{2014}: \textbf{32.} B

\textbf{2009}: \textbf{30.} A

\textbf{2008}: \textbf{28.} A \textbf{30.} D \textbf{31.} C

\textbf{2007}: \textbf{31.} C

\textbf{2004}: \textbf{28.} D

\bibliografia

Mirar en Palgrave:
\begin{itemize}
	\item \textbf{absorption approach to the balance of payments} *
	\item balance of trade, history of *
	\item \textbf{elasticities approach to the balance of payments} *
	\item external debt
	\item \textbf{J-curve} *
	\item international capital flows
	\item \textbf{international finance} *
	\item international indebtedness
	\item international liquidity
	\item \textbf{international monetary institutions} *
	\item international monetary policy
	\item international real business cycles
	\item \textbf{macroeconomic effects of international trade} *
	\item \textbf{Marshall-Lerner condition} *
	\item \textbf{monetary approach to the balance of payments} *
	\item new open economy macroeconomics *
	\item overshooting *
	\item purchasing power parity *
	\item \textbf{specie-flow mechanism} *
\end{itemize}

Cecchetti, S. Schoenholt, K. (2018) \textit{Sudden stops: A primer on balance-of-payments crises} Voxeu.org \href{https://voxeu.org/content/sudden-stops-primer-balance-payments-crises}{Enlace}
Gandolfo, G. \textit{International Finance and Open-Economy Macroeconomics}. (2016)

Eichengreen, B.; Gupta, P. (2016) \textit{Managing Sudden Stops} World Bank Group. Policy Research Working Paper -- En carpeta del tema

Goldberg, P. K.; Knetter, M. M. \textit{Goods Prices and Exchange Rates: What Have We Learned?} (1997) Journal of Economic Literature

Reinhart, C.; Calvo, G. (2000) \textit{When Capital Inflows Come to a Sudden Stop: Consequences and Policy Option} Reforming the International Monetary and Financial System: IMF -- En carpeta del tema 

Wang, P. \textit{The Economics of Foreign Exchange and Global Finance} (2005) 2nd Edition -- En carpeta de economía internacional


\end{document}
