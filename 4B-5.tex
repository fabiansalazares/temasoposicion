\documentclass{nuevotema}

\tema{4B-5}
\titulo{El presupuesto como elemento compensador de la actividad económica. Efectos discreccionales y automáticos del presupuesto. La medición del efecto macroeconómico del presupuesto.}

\begin{document}

\ideaclave

VER JEP DE PRIMAVERA 2019: \url{https://www.aeaweb.org/issues/547}

\seccion{Preguntas clave}

\begin{itemize}
	\item ¿Qué es la estabilización del ciclo económico?
	\item ¿Qué papel juega el presupuesto público en la estabilización del ciclo?
	\item ¿Qué efectos tiene el gasto discrecional sobre el ciclo económico?
	\item ¿Qué son los estabilizadores automáticos?
	\item ¿Cómo se puede cuantificar el efecto estabilizador la política fiscal?
\end{itemize}

\esquemacorto

\begin{esquema}[enumerate]
	\1[] \marcar{Introducción}
		\2 Contextualización
			\3 Objeto de la economía pública
			\3 Importancia del sector público
			\3 Justificación de la intervención pública
			\3 Instrumentos de actuación
			\3 Función de estabilización
			\3 Vías de estabilización del presupuesto
		\2 Objeto
			\3 ¿Cómo se justifica la función estabilizadora del presupuesto?
			\3 ¿Cómo funcionan las medidas estabilizadores discrecionales?
			\3 ¿Cómo funcionan los estabilizadores automáticos del presupuesto?
			\3 ¿Cómo se cuantifican los efectos macroeconómico del presupuesto?
		\2 Estructura
			\3 Justificación de la PM estabilizadora
			\3 Los estabilizadores automáticos
			\3 Política fiscal discreccional
			\3 Medición del efecto macroeconómico del presupuesto
	\1 \marcar{Medición del efecto macroeconómico del presupuesto}
		\2 Idea clave
			\3 Contexto
			\3 Objetivo
			\3 Resultados
		\2 Formulación
			\3 Saldo público
			\3 Output gap
			\3 Saldo cíclico
			\3 Saldo estructural
		\2 Valoración del tono de la política fiscal
			\3 Idea clave
			\3 Política fiscal expansiva
			\3 Política fiscal contractiva
			\3 Representación gráfica
		\2 Indicador utilizado por España
			\3 Marco normativo
			\3 Metodología estilizada
	\1 \marcar{Justificación de la PF estabilizadora}
		\2 Teoría neoclásica
			\3 Idea clave
			\3 Formulación
			\3 Implicaciones
			\3 Valoración
		\2 Teoría keynesiana
			\3 Idea clave
			\3 Formulación
			\3 Implicaciones
			\3 Valoración
		\2 Síntesis neoclásica
			\3 Idea clave
			\3 Formulación
			\3 Implicaciones
			\3 Valoración
		\2 Monetarismo
			\3 Idea clave
			\3 Formulación
			\3 Implicaciones
			\3 Valoración
		\2 Nueva Macroeconomía Clásica
			\3 Idea clave
			\3 Formulación
			\3 Implicaciones
			\3 Valoración
		\2 Modelo del ciclo real
			\3 Idea clave
			\3 Formulación
			\3 Implicaciones
			\3 Valoración
		\2 Nueva Economía Keynesiana
			\3 Idea clave
			\3 Formulación
			\3 Implicaciones
			\3 Valoración
		\2 Implicaciones globales
			\3 Presupuesto tiene efectos reales
			\3 Agentes consideran futuro
			\3 Presupuesto interacciona con política monetaria
			\3 Rigideces nominales y reales modulan efectividad de PF
	\1 \marcar{Política fiscal discreccional en marco IS-LM}
		\2 Idea clave
			\3 Contexto
			\3 Objetivos
			\3 Resultados
		\2 Formulación
			\3 Marco IS-LM
		\2 Implicaciones
			\3 Interacción con demanda de dinero
			\3 Interacción con interés
			\3 Aumento del gasto manteniendo equilibrio presupuestario
			\3 Economía abierta
			\3 Equivalencia ricardiana
			\3 Consolidaciones expansivas
			\3 Retardos
			\3 Gastos comprometidos
	\1 \marcar{Política fiscal discrecional en marco RBC}
		\2 Idea clave
			\3 Contexto
			\3 Objetivos
			\3 Resultados
		\2 Formulación
			\3 Empresas
			\3 Consumidores
			\3 Gobierno
			\3 Resolución
			\3 Dinámica del equilibrio
			\3 Estimación de shocks tecnológicos
		\2 Implicaciones
			\3 Shock transitorio de gasto público
			\3 Shock permanente del gasto público
			\3 Comparación transitorio-permanente en gasto público
		\2 Extensiones
			\3 Estimación de shocks
			\3 Mercado de trabajo
			\3 Impuestos distorsionantes
			\3 Sectores múltiples
			\3 Dinero
			\3 Ciclos reales endógenos
		\2 Valoración
			\3 Relación con otros modelos
			\3 Cómo valorar capacidad de replicación
			\3 Resultados habituales
			\3 Capacidad de predicción
			\3 Simplificación general ampliable
	\1 \marcar{Moderna política fiscal estabilizadora}
		\2 Modelos neo-keynesianos de segunda generación
			\3 Idea clave
			\3 Formulación
			\3 Implicaciones
			\3 Retardos
			\3 Gastos comprometidos
		\2 Evidencia empírica
			\3 Métodos de estimación
			\3 Multiplicadores
			\3 Factores coyunturales
			\3 Importancia de la previsión
			\3 Consolidaciones expansivas
			\3 Estabilizadores automáticos
	\1 \marcar{Los estabilizadores automáticos}
		\2 Idea clave
			\3 Contexto
			\3 Objetivo
			\3 Resultados
		\2 Estabilizadores automáticos por el lado de los ingresos
			\3 Idea clave
			\3 Factores determinantes del efecto estabilizador
			\3 Medición del efecto estabilizador
			\3 Impuestos según efecto estabilizador
		\2 Estabilizadores automáticos por el lado de los gastos
			\3 Idea clave
			\3 Seguros de desempleo
			\3 Programas de sustitución de rentas
			\3 Programas de reducción de la pobreza
			\3 Reglas de gasto
			\3 Medición del efecto estabilizador
		\2 Conclusiones
			\3 Efecto estabilizador doble
			\3 Desempleo afecta a estabilizadores por ambos lados
			\3 Importancia relativa de gasto e ingresos
			\3 Balance de los estabilizadores automáticos
	\1[] \marcar{Conclusión}
		\2 Recapitulación
			\3 Justificación de la PM estabilizadora
			\3 Los estabilizadores automáticos
			\3 Política fiscal discreccional
			\3 Medición del efecto macroeconómico del presupuesto
		\2 Idea final
			\3 Gasto público en España
			\3 Medidas de estabilidad presupuestaria y sostenibilidad
			\3 Federalismo fiscal

\end{esquema}

\esquemalargo

\begin{esquemal}
	\1[] \marcar{Introducción}
		\2 Contextualización
			\3 Objeto de la economía pública
				\4 Rama de la economía
				\4[$\to$] Cómo interviene el estado en la economía
				\4[$\to$] ¿Qué efectos tiene la intervención
				\4[$\to$] ¿Qué procesos de decisión existen en el sector público?
			\3 Importancia del sector público
				\4 Cualitativa
				\4[] Condiciona fuertemente las decisiones privadas
				\4[] $\to$ Poder coactivo
				\4[] $\to$ Superioridad de medios en países desarrollados
				\4 Cuantitativa
				\4[] Gasto público es 40\% de PIB en OCDE
			\3 Justificación de la intervención pública
				\4 Marco básico de funcionamiento
				\4[] Marco legal de actuación
				\4[] Reducir incertidumbre de agentes económicos
				\4[] Garantizar derechos de propiedad
				\4 Eficiencia
				\4[] Presencia de fallos de mercado
				\4[] $\to$ Asignaciones ineficientes en sentido de Pareto
				\4 Equidad
				\4[] Sociedad realiza juicios de valor
				\4[] sobre deseable de asignaciones
				\4[] $\to$ Actúa para cambiarlas
				\4 Estabilización
				\4[] Suavizar fluctuaciones cíclicas
				\4[] Reducir impacto de shocks sobre bienestar
			\3 Instrumentos de actuación
				\4 Regulación
				\4[] Disposiciones legales y reglamentarias
				\4[] Cumplimiento mediante poder coactivo
				\4 Empresas públicas
				\4[] Ordenación de factores productivos
				\4[] directamente por el Estado
				\4[] $\to$ Proveer bienes y servicios
				\4 Presupuesto público
				\4[] Recaudar fondos mediante ingresos públicos
				\4[] Distribuirlos mediante gasto público
			\3 Función de estabilización
				\4 Presupuesto público
				\4[] Programa de ingresos y de gastos
				\4[] $\to$ Principal herramienta de estabilización
				\4 Keynes
				\4[] Generaliza aceptación de necesidad de estabilizar
				\4[] Sector público puede:
				\4[] $\to$ Estimular demanda
				\4[] $\to$ Aumentar aprovechamiento de factores productivos
				\4[] $\to$ Reducir impacto de fluctuaciones cíclicas
			\3 Vías de estabilización del presupuesto
				\4 Estabilizadores automáticos
				\4 Actuaciones discrecionales
		\2 Objeto
			\3 ¿Cómo se justifica la función estabilizadora del presupuesto?
			\3 ¿Cómo funcionan las medidas estabilizadores discrecionales?
			\3 ¿Cómo funcionan los estabilizadores automáticos del presupuesto?
			\3 ¿Cómo se cuantifican los efectos macroeconómico del presupuesto?
		\2 Estructura
			\3 Justificación de la PM estabilizadora
			\3 Los estabilizadores automáticos
			\3 Política fiscal discreccional
			\3 Medición del efecto macroeconómico del presupuesto
	\1 \marcar{Medición del efecto macroeconómico del presupuesto}
		\2 Idea clave
			\3 Contexto
				\4 Dos herramientas de política fiscal
				\4[] $\to$ Discrecional
				\4[] $\to$ Estabilizadores automáticos
				\4 Valorar efectividad de política fiscal
			\3 Objetivo
				\4 Cuantificar efectos de herramientas
				\4 Valorar importancia relativa sobre saldo público
				\4 Relacionar saldo público con posición cíclica
			\3 Resultados
				\4 Descomponer saldo público en dos partes
				\4[] Cíclica
				\4[] $\to$ Captura efecto de estabilizadores
				\4[] Estructural
				\4[] $\to$ Independiente del ciclo
				\4[] $\to$ Afectada por PF discrecional
				\4 Metodología de estimación del ciclo es relevante
		\2 Formulación
			\3 Saldo público
				\4 $\text{SP}_t = I_t - G_t = \text{SP}_t^E + \text{SP}_t^C$
				\4[] $\text{SP}_t^E$: Saldo estructural
				\4[] $\text{SP}_t^E$: Saldo cíclico
			\3 Output gap
				\4 Idea clave
				\4[] Output alcanzado sin rigideces nominales
				\4[] $\to$ Derivado de modelos de NEK
				\4[] Inobservable
				\4[] Sujeto a cambios en definición concreta
				\4[] $\to$ ¿Qué horizonte temporal?
				\4[] $\to$ ¿Qué son rigideces nominales?
				\4 Métodos univariantes de estimación
				\4[] Estimación puramente estadística
				\4[] Estimar tendencia de output
				\4[] $\to$ Asumir tendencia a desaparición de rigideces
				\4[] $\to$ Filtro de Hodrick-Prescott
				\4 Métodos multivariantes
				\4[] Estimación económetrica postulando modelo subyacente
				\4[] $\to$ Múltiples variables utilizables
				\4[] $\to$ Mayor complejidad
			\3 Saldo cíclico
				\4 Estimado en relación al output gap
				\4 Procedimiento
				\4[] 1. Estimar output gap $x_t$
				\4[] 2. Estimar elasticidad I y G públicos al output gap
				\4[] 3. Estimar saldo cíclico resultante
				\4[] $\then$ $\text{SP}^c_t = \beta_t \cdot x_t$
			\3 Saldo estructural
				\4 Estimado en relación a saldo cíclico
				\4 Procedimiento
				\4[] 1. Estimar saldo cíclico
				\4[] 2. Restar saldo cíclico a saldo público
				\4[] $\then$ $\text{SP}_t^E = \text{SP}_t - \beta_t \cdot x_t$
				\4 Sensibilidad del presupuesto al ciclo económico
				\4[] Estimaciones de 2019 de la CE\footnote{Ver página 74 de Actualización del Programa de Estabilidad 2019--2021.}
				\4[] $\to$ 0,6 en 2019
				\4[] $\to$ 0,54 en 2013
				\4[] $\then$ Ha aumentado sensibilidad del presupuesto al ciclo
		\2 Valoración del tono de la política fiscal
			\3 Idea clave
				\4 Saldo estructural es variable relevante
				\4 Saldo cíclico no depende de decisiones discrecionales
				\4[] Aunque sí depende de política económica
				\4[] $\to$ Vía estabilizadores
			\3 Política fiscal expansiva
				\4 Aumento del saldo estructural
			\3 Política fiscal contractiva
				\4 Reducción del saldo estructural
			\3 Representación gráfica
				\4 Espacio output gap--saldo público
				\4 Política A
				\4[] Muy sensible al ciclo
				\4[] Estabilizadores automáticos
				\4[] $\to$ Gasto cae fuertemente si output gap > 0
				\4[] $\to$ Ingresos caen fuertemente si output gap < 0
				\4[] Saldo estructural fuertemente negativo
				\4[] $\then$ SP muy superavitario en expansión
				\4 Política B
				\4[] Estabilizadores automáticos
				\4[] $\to$ Gasto cae débilmente si output gap > 0
				\4[] $\to$ Ingresos caen poco si output gap < 0
				\4[] Poco sensible a estabilizadores automáticos
				\4[] Saldo estructural débilmente negativo
				\4[] $\then$ SP poco superavitario en expansión
				\4 \grafica{politicasfiscales}
				\4 Fase de expansión con output gap positivo
				\4[] Política A muestra mayor superávit
				\4[] $\to$ Pero realmente es más expansiva
				\4 Fase de recesión con output gap negativo
				\4[] Política A muestra mayor déficit
				\4[] $\to$ Sin relación directa con tono expansivo
		\2 Indicador utilizado por España
			\3 Marco normativo
				\4 Constitución Española
				\4 Ley Orgánica 2/2012 de Estabilidad Presupuestaria y Sostenibilidad Financiera
				\4 Orden Ministerial 2741/2012
			\3 Metodología estilizada
				\4 Output gap
				\4 Elasticidades de ingresos y gastos respecto al ciclo
				\4 Saldo estructural
	\1 \marcar{Justificación de la PF estabilizadora}
		\2 Teoría neoclásica
			\3 Idea clave
				\4 Contexto
				\4[] Economía como sistema estable
				\4[] Tendente a equilibrio eficiente
				\4[] $\to$ Pleno aprovechamiento de recursos
				\4[] Economía clásica
				\4[] Liberalismo
				\4[] $\to$ Limitar poder del estado
				\4 Objetivo
				\4[] ¿Qué efecto tiene la política fiscal?
				\4[] ¿Puede producir desviaciones persistentes del equilibrio?
				\4[] Si no, ¿por qué mecanismo?
				\4 Resultados
				\4[] Economía inherentemente estable
				\4[] Política fiscal sólo afecta a interés
				\4[] Crowding-out en mercado de fondos prestables
			\3 Formulación
				\4 Ecuación de demanda
				\4[] $Y=C+I(r)+G$, $I'(r) <0$
				\4 Ecuación de gasto
				\4[] $Y=C+S(r)+T$, $S'(r) >0$
				\4 Equilibrio demanda-gasto
				\4[] $C+I(r)+G=C+S(r)+T$ $\to$ $S(r) = I(r) + G - T$
			\3 Implicaciones
				\4 Interés como variable de ajuste
				\4[] Aumento del gasto público $\uparrow G$
				\4[] $\to$ Aumento del interés $\uparrow r$
				\4[] $\then$ Caída del consumo $\downarrow C$
				\4[] $\then$ Aumento del ahorro $\uparrow S(r)$
				\4[] $\then$ Caída de la inversión privada $I(r)$
				\4[] Disminución de la presión fiscal $\uparrow T$
				\4 Representación gráfica
				\4[] \grafica{neoclasico}
			\3 Valoración
				\4 Influencia determinante sobre PEconómica
				\4[] Hasta años 30
				\4 Regla de oro de la Teoría Clásica de HPública
				\4[] Presupuesto anual equilibrado
				\4[] Reducir presión sobre interés real
				\4[] Reducir crowding-out de inversión privada
				\4 Estabilización vía presupuesto
				\4[] Indeseable e innecesaria
		\2 Teoría keynesiana
			\3 Idea clave
				\4 Contexto
				\4[] Años 30
				\4[] Desempleo generalizado y persistente
				\4[] Ajuste a pleno empleo parece no funcionar
				\4[] New Deal en EEUU
				\4[] Aumento del gasto militar en Alemania y otros
				\4[] Patrón oro y abandono
				\4[] Liquidacionismo y modelo clásico predominan
				\4[] Demanda determina output
				\4[] $\to$ A diferencia de modelo clásico
				\4 Objetivo
				\4[] Caracterizar mecanismo de persistencia de desempleo
				\4[] Explicar efectividad de presupuesto para estimular output
				\4 Resultados
				\4[] Presupuesto puede estabilizar output
				\4[] PFiscal efectiva para alcanzar pleno empleo
				\4[] Economía no tiende a pleno empleo
				\4[] Múltiples teorías posteriores para fundamentar
				\4[] $\to$ Fallos de coordinación
				\4[] $\to$ Transición lenta hacia equilibrio
				\4[] $\to$ Demandas duales: efectiva y nocional
				\4[] $\to$ Rigideces nominales y reales
			\3 Formulación
				\4 Demanda agregada
				\4[] $DA \equiv C+I+G\equiv C_0 + cY(1-t) + I_0 - I(r) + G_0$
				\4[] DA depende de:
				\4[] $\to$ Output Y
				\4[] $\to$ Interés real r
				\4[] $\to$ Impuestos $t$
				\4[] $\to$ Demanda autónoma $C_0$, $G_0$, $I_0$
				\4 Equilibrio demanda agregada y oferta
				\4[] $Y = \frac{1}{1-c(1-t)}\left[ C_0 + I_0 + G_0 \right]$
				\4 Output de pleno empleo $\bar{Y}$
				\4[] Ningún mecanismo induce convergencia a $Y=\bar{Y}$
			\3 Implicaciones
				\4 Multiplicador del gasto
				\4[] En contexto de desempleo y capacidad sin utilizar
				\4[] Variación de la demanda autónoma
				\4[] $\to$ Tiene efecto mayor que la propia variación
				\4 $\uparrow$ del gasto puede $\uparrow$ output más que proporcionalmente
				\4[] Vía multiplicador del gasto
				\4 Animal spirits
				\4[] Alteraciones exógenas de expectativas
				\4[] $\to$ Pueden deprimir demanda autónoma
				\4 Política fiscal puede estabilizar economía
				\4[] Estimular demanda agregada
				\4[] $\to$ Aumento del gasto
				\4[] $\to$ Reducción de los impuestos
				\4 Política fiscal anticíclica
				\4[] Equilibrio presupuestario no es deseable
				\4[] Principio de eq. presupuestario es desestabilizador
				\4[] $\to$ Caída de $t$ obliga a caída de $G$
				\4[] $\then$ Caída de la demanda agregada
				\4[] $\then$ Proceso inverso en periodos de auge
				\4[] Política fiscal debe ser anticíclica
				\4[] $\to$ Aumento de $G-T$ en recesión para estimular DA
				\4[] $\then$ Déficits públicos en recesión
				\4[] $\to$ Caída de $G-T$ en expansión para estimular DA
				\4[] $\then$ Superávits en fases de expansión
				\4 Representación gráfica
				\4[] \grafica{multiplicador}
				\4 PM interacciona con PF
				\4[] Con exceso de liquidez
				\4[] $\to$ No es posible reducir tipos más allá de 0
				\4[] $\then$ Sólo PF es efectiva para estimular output
			\3 Valoración
				\4 Influencia sobre todas los análisis posteriores
				\4 Política económica puede estabilizar el ciclo
				\4 Economías no tienden a pleno empleo
				\4 Política monetaria subordinada
				\4 Análisis macro del presupuesto
		\2 Síntesis neoclásica
			\3 Idea clave
				\4 Contexto
				\4[] Análisis keynesiano paradigma dominante
				\4[] Éxito de políticas keynesianas en años 30, 40, 50s
				\4[] Formulación matemática y diagramática
				\4[] Marco IS-LM
				\4[] Tensión entre modelo clásico y keynesiano
				\4 Objetivo
				\4[] Reconciliar modelo clásico y keynesiano
				\4[] Efectos de política fiscal en corto y largo plazo
				\4 Resultados
				\4[] Análisis de PF diferenciado
				\4[] $\to$ Entre c/p y l/p
				\4[] Modelo matemático simple de economía
				\4[] $\to$ Ligero cambio en supuestos cambia resultados
			\3 Formulación
				\4 IS
				\4[] Equilibrio en mercado de bienes
				\4[] $Y= \text{DA} \equiv C(Y(1-t)) + I(r) + G = C_0 + cY(1-t) + I_0 - I(r) + G_0$
				\4[] $Y = \frac{1}{1-c(1-t)} \left[ C_0 + I_0 - I(r) + G_0 \right]$
				\4 LM
				\4[] Equilibrio en mercado de dinero
				\4[] $\frac{M_S}{P} = L(r,Y)$
				\4 Representación gráfica
				\4[] Espacio $Y$-$r$
			\3 Implicaciones
				\4 Corto plazo
				\4[] Supuestos keynesianos
				\4[] $\to$ Inversión poco sensible al interés
				\4[] $\to$ Capacidad productiva no utilizada
				\4[] $\then$ Presupuesto sirve para estabilizar output
				\4 Largo plazo
				\4[] Supuestos neoclásicos
				\4[] Ajuste a pleno empleo vía precios
				\4[] Efecto estímulo de PF se disipa
				\4 Curva de Phillips
				\4[] Ocasionalmente se interpreta como menú de política
				\4[] $\to$ Se puede elegir combinación empleo-inflación
				\4[] $\then$ PF como instrumento para elegir
			\3 Valoración
				\4 Marco de análisis flexible
				\4[] Cambio parsimonioso en supuestos
				\4[] $\to$ Permite modelizar diferentes efectos
				\4 PF como herramienta principal de PE
		\2 Monetarismo
			\3 Idea clave
				\4 Contexto
				\4[] Curva de Phillips interpretada como menú
				\4[] $\to$ Posible estimular output vía DA persistente
				\4[] Renta permanente
				\4[] $\to$ Agentes demanda en función de renta total
				\4[] PM utilizada pasivamente
				\4[] $\to$ Dinero considerado poco importante
				\4[] Fine-tuning
				\4[] $\to$ Actuaciones precisas de PE para estabilizar output
				\4 Objetivo
				\4[] Valorar importancia del dinero en PF
				\4[] Mecanismos que hacen inefectiva PF
				\4[] Reformular demanda de dinero
				\4[] $\to$ Más factores que interés de bonos
				\4 Resultados
				\4[] Curva de Phillips que incorpora expectativas
				\4[] Inversión sensible a interés
				\4[] Demanda de dinero más a estable a cambios en interés
			\3 Formulación
				\4 IS-LM+Curva de Phillips
				\4 Curva de Phillips
				\4[] Aumentada por las expectativas
				\4[] Hipótesis de expectativas adaptativas sobre precios
				\4[] $\to$ $\to$ $E_t(P_{t+1}) = E_{t-1}(P_t) + \lambda \left( E_{t-1}(P_t) - P_t \right)$
				\4[] Agentes estiman salario real con HEA
				\4[] $\then$ $u_t = u(\pi_t - \pi_t^e) + u^*$, $u(0) = 0$
				\4[] $\then$ Vertical a largo plazo
				\4 Demanda de dinero poco sensible a interés
				\4[] LM con pendiente elevada
				\4[] Depende de muchos otros factores
				\4[] $\to$ Incluida renta permanente
				\4 Inversión muy sensible a interés
				\4[] IS con pendiente poco elevada
			\3 Implicaciones
				\4 Economía tiende a tasa natural de desempleo
				\4 Estímulos de PF son muy poco efectivos
				\4[] $\to$ Agentes consideran renta permanente
				\4[] $\to$ Crowding-out de la inversión
				\4[] $\to$ Curva de Phillips vertical en largo plazo
				\4[] $\to$ Efectos sujetos a lags y poco previsibles
			\3 Valoración
				\4 Conclusiones influencian modelos posteriores
				\4[] PF poco efectiva
				\4[] Mecanismos ajustan a pleno empleo
				\4 Poca influencia metodológica
				\4[] Microfundamentación apenas incipiente
		\2 Nueva Macroeconomía Clásica
			\3 Idea clave
				\4 Contexto
				\4[] Microfundamentación introducida en 60s
				\4[] Eficiencia de MH y HER
				\4[] Agentes plenamente racionales
				\4[] $\to$ Sólo información imperfecta
				\4[] Análisis dinámico
				\4[] $\to$ Agentes maximizan secuencias temporales
				\4 Objetivo
				\4[] ¿PEconómica tiene efectos sobre output?
				\4[] ¿Agentes perciben deuda como riqueza neta?
				\4[] ¿Puede presuesto estabilizar output si agentes racionales?
				\4[] ¿Consistencia de política económica es importante?
				\4 Resultados
				\4[] Curva de Phillips vertical
				\4[] $\to$ Sólo posible desviarse mediante ``sorpresas''
				\4[] Análisis inicial centrado en política monetaria
				\4[] Equivalencia ricardiana
				\4[] $\to$ Deuda no es riqueza neta
			\3 Formulación
				\4 Microfundamentación
				\4[] Agentes maximizan utilidad
				\4[] $\to$ Decidiendo sendas de consumo y ocio
				\4[] Sujetos a restricciones
				\4 Gobierno como agente maximizador
				\4[] Funciones de pérdida
				\4[] Sujetas a restricciones presupuestarias
			\3 Implicaciones
				\4 Equivalencia ricardiana
				\4 Aritmética monetarista desagradable
				\4[] Existen múltiples equilibrios de precios
				\4[] $\to$ Múltiples sendas posibles
				\4[] si PF es dominante
				\4[] $\to$ PM tendrá que ajustarse
				\4[] $\then$ Posible baja inflación hoy y alta mañana
				\4[] $\then$ Posible alta inflación hoy y baja mañana
				\4 Teoría fiscal del nivel de precios
				\4[] Senda de precios futuros depende de PF
				\4[] Precios se ajustan para mantener valor real de deuda
				\4 Discrecionalidad inefectiva
				\4[] Agentes prevén y descuentan estímulo discrecional
				\4[] PF discrecional no es efectiva
			\3 Valoración
		\2 Modelo del ciclo real
			\3 Idea clave
				\4 Contexto
				\4 Objetivo
				\4 Resultados
				\4[] Gasto público induce crowding-out
				\4[] Aumenta tipo de interés
				\4[] Cae ahorro presente
				\4[] Aumenta oferta de trabajo
			\3 Formulación
			\3 Implicaciones
				\4 Escaso papel estabilizador de PF
				\4 Dis
			\3 Valoración
				\4 Enorme impacto metodológico
				\4 Carencias empíricas
				\4[] Aunque relativo éxito replicando series
		\2 Nueva Economía Keynesiana
			\3 Idea clave
				\4 Contexto
				\4[] Microfundamentación generalizada
				\4[] RBC predominante
				\4[] Ciclo resultado de shocks de productividad
				\4[] Escaso papel estabilizador de PF
				\4 Objetivo
				\4[] Representar fenómenos keynesianos
				\4[] $\to$ En contexto de microfund. y agentes racionales
				\4 Resultados
				\4[] Papel predominante de PM frente a PF
				\4[] Rigideces nom+reales hacen posible efectos reales
				\4[] $\to$ De estímulos de política económica
			\3 Formulación
			\3 Implicaciones
			\3 Valoración
		\2 Implicaciones globales
			\3 Presupuesto tiene efectos reales
			\3 Agentes consideran futuro
			\3 Presupuesto interacciona con política monetaria
			\3 Rigideces nominales y reales modulan efectividad de PF
	\1 \marcar{Política fiscal discreccional en marco IS-LM}
		\2 Idea clave
			\3 Contexto
				\4 Aumento del tamaño del presupuesto público
				\4 Output persistentemente por debajo de potencial
			\3 Objetivos
				\4 Reducir fluctuaciones de la renta
				\4 Valorar efecto de diferentes instrumentos estabilizadores
			\3 Resultados
				\4 Diferentes escuelas arrojan diferentes resultados
				\4 Diferentes efectos de estabilizadores discrecionales
		\2 Formulación
			\3 Marco IS-LM
				\4 IS
				\4[] $Y = C_0 + c(y(1-t)+\text{TR})+I(r) + G$
				\4 LM
				\4[] $\frac{M}{P} = L(y,r)$
				\4 Multiplicador del gasto público
				\4[] $\frac{d Y}{d G} = \frac{1}{1-c(1-t) + I_i \frac{L_Y}{L_i}}$
				\4 Multiplicador de las transferencias
				\4[] $\frac{d Y}{d \text{TR}} = \frac{c}{1-c(1-t) + I_i \frac{L_Y}{L_i}}$
				\4 Multiplicador de los impuestos
				\4[] Tomando $T=t\cdot y$
				\4[] $\frac{d Y}{d T} = \frac{-c}{1-c(1-t) + I_i \frac{L_Y}{L_i}}$
		\2 Implicaciones
			\3 Interacción con demanda de dinero
				\4 Cuanto mayor sensibilidad dda. dinero a Y
				\4[] $\to$ Más debe aumentar interés para eq. en mercado dinero
				\4[] $\to$ Más caída de inversión
				\4[] $\then$ Menor efecto multiplicador
				\4 Cuanto mayor sensibilidad dda. dinero a r
				\4[] $\to$ Mayor efecto multiplicador
			\3 Interacción con interés
				\4 Cuanto mayor sensibilidad de inversión a interés
				\4[] $\to$ Menor efecto multiplicador
				\4[] $\then$ Crowding-out de inversión
			\3 Aumento del gasto manteniendo equilibrio presupuestario
				\4 Suma de efectos de aumento de gasto e impuestos
				\4[] $\frac{d Y}{d G} \cdot d G + \frac{d Y}{d T} \cdot d T = \frac{1-c}{1-c(1-t) + I_i \frac{L_Y}{L_i}}$
				\4[$\then$] Efecto positivo dados supuestos
				\4[$\then$] No se ha considerado exceso de gravamen
			\3 Economía abierta
				\4 Estímulo de DA parcialmente a importaciones
				\4 Parte del efecto se expansivo se ``filtra'' a exterior
			\3 Equivalencia ricardiana
				\4 Modelo anterior no valora efectos dinámicos
				\4[] Aumento del déficit público
				\4[] $\to$ Puede inducir caída de la demanda privada
				\4[] $\then$ Agentes estiman subida futura de impuestos
				\4[] $\then$ Agentes ahorran ahora para pagar impuestos futuros
			\3 Consolidaciones expansivas
				\4 Determinados contextos de crisis de deuda
				\4[] Coste de financiación muy elevado
				\4[] Posible desaparición de financiación
				\4 Mecanismos expansivos de la consolidación
				\4[] Señala compromiso frente a impago
				\4[] $\to$ Reduce posibilidad de sequía de financiación
				\4[] Reducción del coste financiero de la deuda
				\4[] $\to$ Facilita convergencia de senda de deuda
				\4[] Reducción de incertidumbre
				\4[] $\to$ Aumenta inversión
			\3 Retardos
				\4 PF opera siempre con cierto retraso
				\4 Efecto expansivo puede ser procíclico
				\4[] Aunque la intención sea anticíclica
				\4[] $\to$ Efecto acontece cuando recuperación ya en marcha
			\3 Gastos comprometidos
				\4 Gran parte del presupuesto ligada a compromisos
				\4[] Habitual en casi todas economías
				\4[] $\to$ Más cuanto mayores estados del bienestar
				\4 Margen muy reducido para políticas discrecionales
	\1 \marcar{Política fiscal discrecional en marco RBC}
		\2 Idea clave
			\3 Contexto
				\4 Modelos de NMC
				\4[] $\to$ Macroeconomía es equilibrio general walrasiano
				\4[] Crítica de Lucas
				\4[] $\to$ Microfundamentación para tratar de evitar
				\4[] Modelo neoclásico de crecimiento
				\4[] $\to$ Referencia básica
				\4 Autores
				\4[] Kydland y Prescott (1982)
				\4[] Long y Plosser (1983)
				\4[] Otros nombres:
				\4[] $\to$  King, Rebelo, Benhabib, ...
			\3 Objetivos
				\4 Formular modelo cuantitativo de efecto de shocks
				\4 Shocks exclusivamente reales
				\4 Replicar momentos de macromagnitudes principales
				\4[] Varianza
				\4[] Correlaciones
				\4[] $\then$ Con modelo robusto a crítica de Lucas
			\3 Resultados
				\4 Modelo de eg. walrasiano
				\4 Dicotomía clásica
				\4[] Curva de Phillips perfectamente vertical
				\4 Impulso
				\4[] Shocks estocásticos sobre variables reales
				\4[] $\to$ Tecnología
				\4[] $\to$ Gasto público
				\4[] Variables nominales no son tenidas en cuenta
				\4[] Dicotomía clásica perfecta
				\4 Persistencia
				\4[] Autocorrelación de shocks
				\4[] $\to$ Persistencia por definición
				\4[] Inversión en capital
				\4[] $\to$ Persistencia indirecta
				\4 Capital
				\4[] Los agentes acumulan capital
				\4[] Acumulación de capital afecta a producción
				\4[] $\to$ Permite autocorrelación del output
				\4[] $\to$ Permite representar efecto acelerador
				\4 Equilibrio
				\4[] Resultado de optimización estocástica con HER
				\4[] Optimización consumo-ocio intratemporal
				\4[] Optimización consumo-ocio intertemporal
				\4 Eficiencia
				\4[] Desviaciones respecto de la tendencia
				\4[] Son también equilibrios dinámicos
				\4[] No son trayectorias de ajuste hacia eq. eficiente
				\4[] $\to$ Los mercados están en equilibrio en todos los periodos
				\4[] $\to$ Ajuste perfectamente flexible de precios
				\4[] $\to$ Trayectorias de equilibrio son óptimos de Pareto
		\2 Formulación
			\3 Empresas
				\4 Maximización de los beneficios de las empresas
				\4[] Decidiendo sobre:
				\4[] $\to$ Capital
				\4[] $\to$ Trabajo
				\4[] $\underset{N_t, K_t}{\max} \quad \Pi_t = \underbrace{A_t K_t^\alpha N_t^{1-\alpha}}_{Y_t} - w_t N_t - R_t K_t$
				\4[] CPO: \quad $w_t = (1-\alpha) A_t K_t^\alpha N_t^\alpha$
				\4[] \quad \quad $R_t = \alpha A_t K_t^{\alpha-1} N_t^{1-\alpha}$
				\4[] Donde:
				\4[] $\to$ $A_t = (1-\rho)A + \rho_A A_{t-1} + \epsilon_t$
			\3 Consumidores
				\4 Maximización de la utilidad de los consumidores
				\4[] Decidiendo sobre:
				\4[] $\to$ Consumo en periodo
				\4[] $\to$ Trabajo en periodo
				\4[] $\to$ Capital en periodo
				\4[] $\to$ Inversión en activo del gobierno en periodo
				\4[] $\underset{C_t, N_t,K_t, B_t}{\max} \quad \sum_{t=0}^\infty \beta^t u(C_t, N_t)$
				\4[] $\text{s.a}: \quad C_t+\underbrace{K_t - (1-\delta)K_{t-1}}_{I_t} + B_t - (1+r_{t-1})B_{t-1} \leq$
				\4[] \quad \quad \quad $\leq \underbrace{wN_t + R_t K_t + \Pi_t}_{Y_t} - T_t$
				\4[] \quad \quad \quad $\lim_{T \to \infty}  K_t \geq 0$
				\4[] \quad \quad \quad $\then$ $\sum_{t=0}^\infty \frac{C_t + I_t}{(1+r)^t} = \sum_{t=0}^\infty \frac{ Y_t}{(1+r)^{-t}} - \sum_{t=0}^\infty \frac{T_t}{(1+r)^{-t}}$
				\4[] Donde:
				\4[] $\to$ $u(C_t, N_t) = \left( \ln C_t  - v(N_t) \right)$
				\4[] CPO: \quad $u'(C_t) = \beta u'(C_{t+1})$
				\4[] \quad \quad $w_t = \frac{u_{N_T}}{u_{C_t}}$
			\3 Gobierno
				\4 Senda exógena de gasto público sujeta a restricción
				\4[] $G_t + r_{t-1} D_t \leq T_t + D_{t+1} - D_t$
				\4[] $\to$ $T_t = G_t - \left( D_t - (1+r_{t-1} D_{t-1}) \right)$
				\4[] Condición de No-Ponzi + Transversalidad
				\4[] $\then$ $\sum_{t=0}^\infty G_t \cdot \frac{1}{(1+r)^t} = \sum_{t=0}^\infty T_t \cdot \frac{1}{(1+r)^t}$
			\3 Resolución
				\4 Resolución por método de Lagrange
				\4[] Si secuencia de shocks es conocida
				\4[] $\to$ $\epsilon_t$ a productividad
				\4[] $\to$ $G_t$ a gasto público
				\4 Resolución por programación dinámica
				\4[] Si shocks aleatorios
			\3 Dinámica del equilibrio
				\4[] $u'(C_t) = \beta u'(C_{t+1})$
				\4[] $w_t = \frac{u_{N_T}}{u_{C_t}}$
				\4[] $C_t + I_t + G_t = Y_t$
				\4[] $I_t = K_{t+1} - (1-\delta) K_t$
				\4[$\then$] Estado estacionario: secuencias de vars. exógenas
				\4[] $C_t = C(K_t, \left\lbrace A_t \right\rbrace^\infty_0, \left\lbrace G_t \right\rbrace )$
				\4[] $N_t = N(K_t, \left\lbrace A_t \right\rbrace^\infty_0, \left\lbrace G_t \right\rbrace )$
				\4[] $K_t = K (K_t, \left\lbrace A_t \right\rbrace^\infty_0, \left\lbrace G_t \right\rbrace )$
				\4 Aproximación y log-linearización
				\4[] Solución suele tomar forma de
				\4[] sistema de eqs. parciales diferenciales
				\4[] $\to$ Sin solución analítica en forma cerrada
				\4[] $\to$ Aproximación de la solución y linearización
				\4 Ecuaciones de dinámica aproximada
				\4[] Tras linearización del estado estacionario
				\4[] $\tilde{C}_{t+1} = a_{CK}\tilde{K}_{t+1} + a_{CA}\tilde{A}_{t+1} + a_{CG} \tilde{G}_{t+1}$
				\4[] $\tilde{L}_{t+1} = a_{LK}\tilde{K}_{t+1} + a_{LA}\tilde{A}_{t+1} + a_{LG} \tilde{G}_{t+1}$
				\4[] $\tilde{K}_{t+1} = b_{KK}\tilde{K}_{t} + b_{KA}\tilde{A}_{t} + b_{KG} \tilde{G}_{t}$
				\4 Parámetros de las ecuaciones de dinámica
				\4[] Derivados de parámetros estructurales exógenos
				\4[] Entre ellos:
				\4[] $\alpha$: elasticidad-capital del output
				\4[] $g$: tasa de crecimiento tendencial
				\4[] $\rho$: tasa de descuento de la utilidad
				\4[] $\rho_\theta$: persistencia del shock de productividad
				\4[] tipo de interés de equilibrio
				\4[] etc...
			\3 Estimación de shocks tecnológicos
				\4 Shocks tecnológicos
				\4[] Pueden representar perturbaciones sobre:
				\4[] $\to$  Productividad
				\4[] $\to$ Liberalización y desregulación
				\4[] $\to$ Desastres naturales o guerras
				\4 Filtros de tendencias
				\4[] Métodos matemáticos para extraer
				\4[] $\to$ Componente cíclico
				\4[] $\then$ Shocks de productividad
				\4 Estimados mediante diferentes filtros
				\4[] Descomponer tendencia+ciclo
				\4[] Univariables
				\4[] $\to$ A partir de una variable
				\4[] $\to$ Generalmente, PIB
				\4[] Multivariables
				\4 Filtro de Hodrick-Prescott
				\4[] Hallar secuencia de output tendencia
				\4[] $\to$ Que minimiza función de pérdida
				\4[] Función de pérdida penaliza de:
				\4[] $\to$ Diferencia entre output y tendencia
				\4[] $\to$ Variaciones entre periodos de tendencia
				\4[] $\then$ Parametrizable para variar peso de uno y otro
				\4[] Dibujar gráfica $y$--$t$ y tendencia superpuesta
				\4[] $\tilde{C}_t$, $\tilde{K}_{t+1}$, $\tilde{N}_t$
				\4[] $\to$ Expresan diferencias frente a tendencia
		\2 Implicaciones
			\3 Shock transitorio de gasto público
				\4 Supuestos:
				\4[] gasto público improductivo
				\4[] Impuesto de suma fija no distorsionante
				\4 Output aumenta
				\4[] $\to$ Aunque mucho menos que gasto público
				\4 Consumo cae
				\4[] $\to$ Muy ligeramente
				\4 Inversión cae
				\4[] Caída muy pronunciada y recuperación rápida
				\4 Trabajo aumenta
				\4[] Muy ligeramente
				\4[] $\to$ Sin efecto sustitución ocio-consumo
				\4[] $\to$ Pequeño efecto renta
				\4 Salarios caen
				\4[] Muy ligeramente
				\4[] $\to$ Aumento de oferta de trabajo
				\4[] $\to$ Menos capital
				\4[] $\to$ Igual productividad
				\4 Tipo de interés
				\4[] Aumenta muy ligeramente
				\4 Representación gráfica
				\4[] \grafica{rbcefectodynaregastotransitorio}
			\3 Shock permanente del gasto público
				\4 Supuestos:
				\4[] gasto público improductivo
				\4[] Impuesto de suma fija no distorsionante
				\4 Output aumenta
				\4[] Más que con shock transitorio
				\4 Consumo cae
				\4[] Más que con shock transitorio
				\4 Inversión cae
				\4[] Más que con shock transitorio
				\4[] De manera más persistente
				\4 Trabajo aumenta
				\4[] Más que con shock transitorio
				\4[] $\to$ Efecto renta mucho mayor ahora
				\4[] $\to$ Sin efecto sustitución ocio-consumo
				\4 Salarios caen
				\4[] Más que en transitorio
				\4[] $\to$ Aumento de la oferta de trabajo
				\4[] $\to$ Menos capital
				\4[] $\to$ Igual productividad
				\4 Tipo de interés
				\4 Representación gráfica
				\4[] \grafica{rbcefectodynaregastopermanente}
			\3 Comparación transitorio-permanente en gasto público
				\4 Cuanto más persistente sea el shock:
				\4[Consumo] + $\uparrow$ sobre consumo
				\4[Trabajo] + aumento del trabajo
				\4[Salarios] + $\downarrow$ los salarios
				\4[Output] + aumenta el output
				\4[Inversión] + persistente caída de la inversión
				\4[Tipo de interés] + $\uparrow$ el tipo de interés
		\2 Extensiones
			\3 Estimación de shocks
				\4[] Residuo de Solow como estimación de shocks de prod.
				\4[] $\to$ Recibe críticas: necesarios shocks muy grandes
				\4[] $\to$ Crisis implican shocks negativos muy grandes
			\3 Mercado de trabajo
				\4[] Rogerson (1984), Hansen (1985)
				\4[] Trabajo indivisible
				\4[] $\to$ Cambios en trabajo no son sólo cambios en horas
				\4[] $\then$ Sobre todo, cambios en número de empleados
				\4[] Incorporar respuesta de trabajo a shocks
				\4[] $\to$ Shocks implica variación más fuerte de trabajo
				\4[] $\Rightarrow$ Baja respuesta de horas trabajadas a shock
				\4[] $\Rightarrow$ Mejoran replicación de series reales
			\3 Impuestos distorsionantes\footnote{Pág. 230 de Romer.}
				\4[] Puede introducir efectos de sustitución temporal
				\4[] Puede distorsionar rendimientos relativos trabajo y capital
				\4[] $\to$ Equivalencia ricardiana
			\3 Sectores múltiples
				\4[] Efectos de transmisión de shocks entre sectores
				\4[] Estructura de mercados determina respuesta a shocks
			\3 Dinero
				\4[] En RBC, perturbaciones monetarias
				\4[] $\to$ Afectan sólo a variables nominales
				\4[] RBC se abstrae de perturbaciones monetario-real
				\4[] En la práctica
				\4[] $\to$ Perturbaciones monetarias afectan reales
				\4[] Modelos de ciclo nominal o monetario
			\3 Ciclos reales endógenos
				\4[] Benhabib y Nishimura (1985)
				\4[] Conectados con crecimiento endógeno
				\4[] No hay convergencia a un estado estacionario
				\4[] El propio modelo genera fluctuaciones endógenas
		\2 Valoración
			\3 Relación con otros modelos
				\4 Basado en Ramsey (1926), y Cass y Koopmans
				\4 Culmina programa de investigación de Lucas:
				\4[] Formular ciclo económico en marco de equilibrio
				\4[] Ciclos como óptimos de Pareto
				\4[] Sin supuestos no walrasianos ad-hoc
				\4[] $\to$ Mercados deben vaciarse
				\4 Consolidación de marco DSGE
				\4[] iniciado por Lucas 1972
				\4[] optimización Dinámica de los agentes
				\4[] sujetos a impulsos eStocásticos
				\4[] en contexto de Equilibrio General
			\3 Cómo valorar capacidad de replicación
				\4[1] Calibración del modelo
				\4[] Elegir valores de parámetros en base a:
				\4[] $\to$ Teoría microecómica
				\4[] $\to$ Estimaciones microeconómicas
				\4[] $\to$ Teoría macro
				\4[] ...
				\4[2] Estimar estado estacionario
				\4[3] Introducir shocks
				\4[] Shocks de productividad: residuos de Solow
				\4[] Gasto público: ajustes estructurales
				\4[4] Comparar con series reales
				\4[] Los momentos de las series son similares?
			\3 Resultados habituales
				\4 Con modelos básicos de RBC
				\4[] Modelos más complejos mejoran resultados
				\4[] Introducen mayor complejidad y sup. ad-hoc
				\4 Buena replicación de:
				\4[] Primer momento de Y, C, I, N, w
				\4[] Volatilidades relativas del consumo e I
				\4[] $\to$ C mucho menos volátil que Y
				\4[] $\to$ I mucho más volátil que Y
				\4 Replicación relativamente pobre
				\4[] Evolución del tipo de interés
				\4[] Evolución de la inversión
				\4 Mala replicación
				\4[] Correlación entre trabajo y productividad
				\4[] RBC predice alta correlación
				\4[] $\uparrow$ $\frac{Y}{L}$ aumenta mucho horas trabajadas
				\4[] Pero en realidad, $W$ muy débilmente procíclico
				\4[] Reacción sobre todo en margen extensivo
				\4[] $\to$ Más que en horas de trabajo (intensivo)
			\3 Capacidad de predicción
				\4 Replicación de momentos de distribuciones reales
				\4[] Notable acercamiento en algunas variables
			\3 Simplificación general ampliable
				\4 Introducción de ajustes
				\4[] Numerosísimos modelos
				\4 Aumento de capacidad predictiva
	\1 \marcar{Moderna política fiscal estabilizadora}
		\2 Modelos neo-keynesianos de segunda generación
			\3 Idea clave
				\4 Contexto
				\4[] RBC consolidado como marco de modelización
				\4[] $\to$ Microfundamentación+EGW+HER
				\4[] Consenso sobre:
				\4[] $\to$ importancia de dinero
				\4[] $\to$ vars. nominales
				\4[] $\to$ efecto de PF sobre demanda
				\4[] $\then$ RBC no representa
				\4[] Microfundamentación de rigideces nominales y reales
				\4[] Modelos de competencia monopolística bien formalizados
				\4[] $\to$ Dixit y Stiglitz (1977) y derivados
				\4 Objetivo
				\4[] Caracterizar macroeconomía con rigideces
				\4[] Representar efecto de PM y PF sobre macroeconomía
				\4[] Mantener ventajas de marco RBC
				\4[] $\to$ Microfundamentación
				\4[] $\to$ Análisis normativo explícito
				\4[] $\to$ Resistencia a crítica de Lucas
				\4 Resultados
				\4[] Política fiscal puede estimular output
				\4[] Política fiscal interacciona con monetaria
				\4[] En ZLB, PF puede ser más útil
			\3 Formulación\footnote{Basado en De Paoli (2008) en carpeta del tema}
				\4 Consumidores
				\4[] Demanda à la Dixit-Stiglitz sobre bien de consumo
				\4[] Demanda de ocio--oferta de trabajo
				\4 Empresas
				\4[] Producen bien diferenciado con misma tecnología
				\4[] $\to$ Costes marginales crecientes
				\4[] Poder de mercado en su variedad
				\4[] $\to$ Venden con mark-up sobre coste
				\4 Rigideces de precios
				\4[] Diferentes formulaciones
				\4[] Generalmente, à la Calvo
				\4[] $\to$ Porcentaje de empresas puede cambiar en cada periodo
				\4[] $\then$ Menor porcentaje, mayor rigidez
				\4[] Empresas tratan de fijar mark-up óptimo
				\4[] $\to$ En cuanto tienen oportunidad
				\4[] $\to$ Considerando dinámica futura de inflación y demanda
				\4 Dinámica de inflación
				\4[] Demanda por encima de equilibrio
				\4[] $\to$ Menos ventas a mismo precio por rigidez
				\4[] $\then$ Mark-up por debajo de lo deseado
				\4[] $\then$ Presión inflacionaria
				\4[] Demanda por debajo de equilibrio
				\4[] $\to$ Mayores ventas a mismo precio por rigidez
				\4[] $\then$ Mark-up por debajo de lo deseado
				\4[] $\then$ Presión hacia bajada de precios
				\4 Gobierno
				\4[] Decide senda exógena de gasto público
				\4 Ecuaciones de equilibrio
				\4[] IS: $\bar{y}_t - \tilde{g}_t = E_t \left( \hat{y}_{t+1} -\tilde{g}_{t+1} \right) - \sigma^{-1} \left( \hat{i}_t - E_t \left(    \pi_{t+1}\right) - \hat{r}_t^n  \right)$
				\4[] NKPC: $\pi_t = \beta E_t \left( \pi_{t+1} \right) + \textsc{k} \tilde{y}_t$
				\4[] Donde:
				\4[] $\to$ $\hat{r}_t^n$: crece con tipo de interés
			\3 Implicaciones
				\4 Aumento del gasto sin subida de impuestos
				\4[] Aumento de $g_t$
				\4[] $\to$ Aumento de demanda agregada
				\4[] Caída del consumo para acomodar aumento del gasto
				\4[] $\to$ Aumento de la utilidad marginal del consumo
				\4[] Aumento de oferta de trabajo
				\4[] $\to$ Caída del coste marginal de producción
				\4[] $\then$ Aumento del output potencial
				\4[] $\then$ Aumento del tipo de interés natural
				\4[] Efecto sobre inflación depende de política monetaria
				\4[] $\to$ $\uparrow$ DA vs aumento output potencial
				\4[] $\then$ ¿Cual prevalece?
				\4[] Política monetaria que evita inflación
				\4[] $\to$ Implica aumentar tipo de interés real efectivo
				\4[] $\then$ Para contener aumento en demanda y precios
				\4 Aumento del gasto financiado con impuestos al trabajo
				\4[] Impuestos reducen oferta
				\4[] Gasto público aumenta oferta y demanda
				\4[] $\to$ Efecto ambiguo sobre oferta
				\4[] $\to$ Aumento de demanda
				\4[] $\then$ Menor efecto sobre output
				\4 Reglas de política fiscal óptima
				\4[] Modelo permite caracterizar PF como feedback
				\4 Política fiscal en la ZLB
				\4[] Posible herramienta para escapar de ZLB
				\4[] Cuando interés natural + inflación muy bajos
				\4[] $\to$ Imposible bajar nominal bajo cero
				\4[] $\then$ Real natural más bajo que real efectivo
				\4[] $\then$ Economía en trampa recesiva
				\4[] $\then$ Política fiscal como escape
				\4 Consolidaciones expansivas
				\4[] Determinados contextos de crisis de deuda
				\4[] $\to$ Coste de financiación muy elevado
				\4[] $\to$ Posible desaparición de financiación
				\4 Mecanismos expansivos de la consolidación
				\4[] Señala compromiso frente a impago
				\4[] $\to$ Reduce posibilidad de sequía de financiación
				\4[] Reducción del coste financiero de la deuda
				\4[] $\to$ Facilita convergencia de senda de deuda
				\4[] Reducción de incertidumbre
				\4[] $\to$ Aumenta inversión
			\3 Retardos
				\4 PF opera siempre con cierto retraso
				\4 Efecto expansivo puede ser procíclico
				\4[] Aunque la intención sea anticíclica
				\4[] $\to$ Efecto acontece cuando recuperación ya en marcha
			\3 Gastos comprometidos
				\4 Gran parte del presupuesto ligada a compromisos
				\4[] Habitual en casi todas economías
				\4[] $\to$ Más cuanto mayores estados del bienestar
				\4 Margen muy reducido para políticas discrecionales
				\4 Modelo muy versátil
				\4[] Posible incorporar muchos otros mecanismos y contextos
				\4[] $\to$ Economía abierta
				\4[] $\to$ Rigideces reales en mercado laboral
				\4[] $\to$ Fricciones financieras
				\4[] $\to$ Múltiples regímenes de política monetaria
		\2 Evidencia empírica
			\3 Métodos de estimación\footnote{Ver pág. 20 de Batini et al (2014).}
				\4 Modelos VAR y SVAR basados en series temporales
				\4[] Estimación basada en información pasada
				\4[] $\to$ conjunto relativamente reducido de variables
				\4[] Problemas:
				\4[] $\to$ Aislar ``shocks'' de otras fluctuaciones
				\4[] $\to$ Cambios estructurales profundos alteran resultados
				\4 Modelos DSGE calibrados para economías
				\4[] Representación del conjunto de la economía
				\4[] $\to$ Postular modelo subyacente de economía
				\4[] $\to$ Calibrar parámetros
				\4[] $\to$ Estimar efecto de shocks
				\4[] Problemas:
				\4[] $\to$ Poco robusto a cambios en parámetros o modelo
				\4[] $\to$ Poco consenso sobre verdadero modelo de economía
			\3 Multiplicadores
				\4 Ramey (2019)
				\4[] Survey de estudios más recientes
				\4 Realmente, tres tipos de multiplicador a considerar
				\4[] Gasto público/compras del gobierno
				\4[] Presión fiscal
				\4[] Déficit público
				\4 Gasto público
				\4[] Mayoría de estimaciones entre $0,6$ y $1$
				\4 Presión fiscal
				\4[] Entre -2 y -3 para subidas de impuestos
				\4[] Entre 0.5 y 0 ante bajadas de impuestos
				\4[] $\to$ Generalmente inferiores a multiplicador del gasto
				\4 Déficit público
				\4[] Evidencia contradictoria
				\4[] Aparentes diferencias entre recesión y recuperación
				\4 Determinantes estructurales de los multiplicadores
				\4[] Apertura al comercio
				\4[] $\to$ Multiplicadores más altos en economías cerradas
				\4[] $\to$ Economías abiertas sufren desbordamiento exterior
				\4[] $\then$ Parte de $\Delta$ G a importaciones
				\4[] Rigideces en el mercado de trabajo
				\4[] $\to$ Multiplicadores más altos si salarios poco flexibles
				\4[] Estabilizadores automáticos
				\4[] $\to$ Estabilizadores más potentes reducen multiplicador
				\4[] Régimen cambiario
				\4[] $\to$ TCFlexible reduce tamaño del multiplicador
				\4[] $\then$ TCN tiende a apreciarse con estímulo
				\4[] Calidad de la gestión pública
				\4[] $\to$ Mala gestión de gasto y recaudatoria
				\4[] $\then$ Reduce multiplicador
				\4 Persistencia del multiplicador
				\4[] Persistencia no lineal
				\4[] $\to$ Generalmente, curva U invertida
				\4[] $\then$ Poco efecto al principio
				\4[] $\then$ Aumento progresivo
				\4[] $\then$ Efecto expansivo se disipa progresivamente
				\4[] Efectos muy heterogéneos según instrumento de PF
				\4[] $\to$ Impuestos indirectos, consumo y transferencias
				\4[] $\then$ Efecto tiende a disiparse antes de 5 años
				\4[] $\to$ Impuesto de sociedades, inversión pública
				\4[] $\then$ Efecto más duradero o incluso permanente
			\3 Factores coyunturales
				\4 Diferencias expansión vs recesión
				\4[] Evidencia favorable a impacto diferente
				\4[] $\to$ Multiplicador más grande en recesión
				\4[] Generalmente, preferible estímulo en recesión
				\4[] $\to$ Evidencia no totalmente concluyente
				\4 Política monetaria
				\4[] PM expansiva ante contracción fiscal
				\4[] $\to$ Reduce efecto negativo sobre demanda
				\4[] PM con problemas de funcionamiento
				\4[] $\to$ P.ej.: en ZLB
				\4[] $\to$ Aumenta efectividad de PF
				\4[] $\then$ ¿Economía sigue en ZLB cuando efecto PF tiene lugar?
			\3 Importancia de la previsión
				\4 Importantes diferencias si hay lag hasta aparición
				\4[] Evidencia consistentemente favorable
				\4 Shocks inmediatos e imprevistos
				\4[] Efectos inmediatos sobre output
				\4 Impulsos fiscales con retardo
				\4[] Ejemplo: bajada anunciada de impuestos
				\4[] $\to$ Caída de actividad desde anuncio a implementación
				\4[] $\then$ Aumento de actividad tras implementación
				\4 Problema empírico
				\4[] ¿Cómo cuantificar periodo entre anuncio e implementación?
				\4[] ¿Cuándo los agentes son conscientes de shock?
				\4[] ¿Cómo anticipan los agentes el momento de implementación?
			\3 Consolidaciones expansivas
				\4 Cierta literatura en 90s y post-crisis
				\4[] Consolidación fiscal puede tener efectos expansivos
				\4[] $\to$ Vía aumento de la confianza de agentes
				\4 Resultado controvertido
				\4[] Problemas para definir consolidación
				\4[] Problemas para aislar contracción fiscal de otros
				\4[] $\to$ Evidencia apunta a que recuperación vía dda. externa
				\4[] $\then$ No vía aumento de confianza $\to$ $\uparrow$ dda. interna
			\3 Estabilizadores automáticos\footnote{Ver Dolls et al. (2012).}
				\4 Diferencias desarrollados vs PEDs
				\4[] Efecto muy reducido en PEDs
				\4[] Presencia relevante en desarrollados
				\4 Diferencias EEUU vs UE
				\4[] En términos medios, más estabilización en Europa
				\4[] $\to$ Superior al 36\% de los shocks
				\4[] Importante heterogeneidad en Europa
				\4[] $\to$ Muy alta absorción de shocks en norte
				\4[] $\to$ Menor que USA en sur de Europa
	\1 \marcar{Los estabilizadores automáticos}
		\2 Idea clave
			\3 Contexto
				\4 Discrecionalidad sujeta a:
				\4[] $\to$ reacción de agentes
				\4[] $\to$ presiones y ciclos políticos
				\4 Regímenes de política económica
				\4[] Conjunto de reglas y convenciones
				\4[] $\to$ Determinan política económica
				\4[] $\to$ Definidas en relación a indicadores externos
			\3 Objetivo
				\4 Valorar efectos de sistemas automáticos de estabilización
			\3 Resultados
				\4 Variedad de instrumentos estabilizadores
		\2 Estabilizadores automáticos por el lado de los ingresos
			\3 Idea clave
				\4 Impuestos y otros tributos
				\4 Efecto variable con recaudación
				\4 Sentido del efecto contrario a ciclo
			\3 Factores determinantes del efecto estabilizador
				\4 Tamaño de las bases imponibles (+)
				\4[] Más efecto estabilizador cuantas más bases
				\4 Mayor tipo medio de gravamen (+)
				\4[] Mayor suavización del ciclo
				\4 Mayor progresividad (+)
				\4[] Más aumenta recaudación al aumentar renta
				\4[] $\to$ Mayor efecto estabilizador
			\3 Medición del efecto estabilizador
				\4 Elasticidad-renta de recaudación impositiva
				\4[] $\epsilon_{T-Y} = \underbrace{\epsilon_{T-\text{BI}}}_{>0} \cdot \underbrace{\epsilon_{\text{BI}-Y}}_{>0} > 0$
				\4[] $\to$ $\epsilon_{T-\text{BI}}$: progresividad de los impuestos
				\4[] $\to$ $\epsilon_{\text{BI}-Y}$: sensibilidad de base imponible
				\4 Signo de la elasticidad-renta de recaudación
				\4[] Positivo siempre
				\4[] Ambos componentes son > 0
				\4[] $\to$ Recaudación aumenta con renta
				\4[] $\then$ Aunque no necesariamente elástica a renta (i.e. > 1)
			\3 Impuestos según efecto estabilizador
				\4 IRPF e IS
				\4[] Principal papel de estabilización
				\4[] $\to$ Elasticidad-renta de bases > 1
				\4[] IS
				\4[] $\to$ Bases imponibles negativas no tributan
				\4[] IRPF
				\4[] $\to$ Mínimos exentos se superan cuando aumentan rentas
				\4 Impuestos indirectos y cotizaciones
				\4[] Elasticidad-renta de bases < 1
				\4[] Propensión media al consumo decrece con renta
				\4[] Cotizaciones sociales sujetas a tope
				\4[] $\to$ Salvo destope de cotización
		\2 Estabilizadores automáticos por el lado de los gastos
			\3 Idea clave
				\4 Gasto público sensible a ciclo
				\4[] Si anticíclico:
				\4[] $\to$ Efecto estabilizador del ciclo
				\4 Gasto asociado a estabilizadores automáticos
				\4[] Prestaciones por desempleo
				\4[] Prestaciones no contributivas para pobreza
			\3 Seguros de desempleo
			\3 Programas de sustitución de rentas
			\3 Programas de reducción de la pobreza
			\3 Reglas de gasto
			\3 Medición del efecto estabilizador
				\4[] $\epsilon_{G-Y} = \underbrace{\epsilon_{G-u}}_{>0} \cdot \underbrace{\epsilon_{u-Y}}_{<0} < 0$
				\4[] $\to$ $\epsilon_{G-u} > 0$ sensibilidad de prestaciones a desempleo
				\4[] $\to$ $\epsilon_{u-Y} > 0$ sensibilidad de desempleo al ciclo
		\2 Conclusiones
			\3 Efecto estabilizador doble
				\4 Por ingreso
				\4 Por gastos
			\3 Desempleo afecta a estabilizadores por ambos lados
				\4 Por ingresos
				\4[] $\to$ Cotizaciones sociales
				\4[] $\to$ Impuestos sobre la renta
				\4 Por gastos
				\4[] $\to$ Gasto en prestaciones por desempleo y similares
			\3 Importancia relativa de gasto e ingresos
				\4 En España, ingresos más importantes
				\4 Especialmente en 2009
				\4[] $\to$ Muy fuerte caída de ingresos > 4\%
				\4[] $\to$ Gastos aumentaron ligeramente
			\3 Balance de los estabilizadores automáticos
				\4 Ventajas
				\4[] Inmediatez
				\4[] Predecibles
				\4[] Menos sensible a ciclo político
				\4 Inconvenientes
				\4[] A menudo insuficientes
				\4[] Requiere instituciones eficientes
				\4[] Menos flexibles ante imprevistos
	\1[] \marcar{Conclusión}
		\2 Recapitulación
			\3 Justificación de la PM estabilizadora
			\3 Los estabilizadores automáticos
			\3 Política fiscal discreccional
			\3 Medición del efecto macroeconómico del presupuesto
		\2 Idea final
			\3 Gasto público en España
			\3 Medidas de estabilidad presupuestaria y sostenibilidad
			\3 Federalismo fiscal
\end{esquemal}


\graficas


\begin{axis}{4}{Equilibrio en el mercado de fondos prestables: interés real como variable de ajuste ante un aumento del gasto público.}{}{$r$}{neoclasico}
	% Extensión del eje de abscisas
	\draw[-] (4,0) -- (6,0);
	\node[below] at (6,0){$S(r)$, $I(r) + G -T$};
	
	% S -- Ahorro
	\draw[-] (0.5,0.5) -- (3.5,3.5);
	\node[right] at (3.5,3.5){\small $S(r)$};
	
	% I -- Inversión antes del aumento del gasto
	\draw[-] (0.5,3.5) -- (3.5,0.5);
	\node[left] at (3.3,0.5){\small $I(r)+G-T$};
	
	% I -- Inversión después del aumento del gasto
	\draw[dashed] (2,3.5) -- (5,0.5);
	\node[right] at (5,0.5){\small $I(r)+G'-T$};
	
\end{axis}

\begin{axis}{4}{Ajuste dinámico de demanda y renta tras un aumento de la demanda de inversión}{Y}{Demanda}{multiplicador}
	% equilibrio
	\draw[-] (0,0) -- (4,4);
	\node[right] at (4,4){Y=DA};
	
	% demanda agregada 0
	\draw[-] (0,1) -- (4,2.5);
	\node[right] at (4,2.5){$\text{DA}_0$};
	
	% demanda agregada 1
	\draw[-] (0,2) -- (4,3.5);
	\node[right] at (4,3.5){$\text{DA}_1$};
	
	% senda de ajuste
	\draw[-{Latex}] (1.6,1.6) -- (1.6,2.6);
	
	\draw[decorate,decoration={brace,amplitude=3pt},xshift=-2pt,yshift=0pt] (1.6,1.63) -- (1.6,2.57) node[black,midway,xshift=-0.4cm] {\footnotesize $\Delta I$};
	
	\draw[-{latex}] (1.6,2.6) -- (2.6,2.6);
	\draw[-{Latex}] (2.6,2.6) -- (2.6,2.97);
	\draw[-{Latex}] (2.6,2.97) -- (2.97, 2.97);
	\draw[-{Latex}] (2.97,2.97) -- (2.97,3.12);
	
	% Output de equilibrio inicial
	\draw[dashed] (1.6,1.6) -- (1.6,0);
	\node[below] at (1.6,0){\small $Y_0$};
	
	% Output de equilibrio final
	\draw[dashed] (3.2,3.2) -- (3.2,0);
	\node[below] at (3.2,0){\small $Y_1$};
\end{axis}

\begin{axis}{4}{Relación entre output gap y saldo público en dos regímenes de política fiscal con diferente tono expansivo.}{output gap en \%}{saldo público en \%}{politicasfiscales}
	% Expansión de ejes
	% abscisas
	\draw[-] (-4,0) -- (0,0);
	% ordenadas
	\draw[-] (0,0) -- (0,-4);
	
	% Política A -- Tono más expansivo pero más sensible a ciclo
	\draw[-] (-2,-3) -- (4,2);
	\node[right] at (4,2){A};
	
	% Política B -- Tono menos expansivo pero menos sensible al ciclo
	\draw[-] (-4,-2.5) -- (4,1);
	\node[right] at (4,1){B};
\end{axis}

Se aprecia como el saldo estructural (es decir, el saldo público cuando el output gap es nulo) es mayor en valor absoluto para la política A que para la B, siendo deficitario en ambos casos. Por ello, cabe afirmar que la política A tiene un tono más expansivo que la B.

\conceptos

\preguntas

\notas

\bibliografia

Mirar en Palgrave:
\begin{itemize}
	\item 
\end{itemize}

\end{document}
