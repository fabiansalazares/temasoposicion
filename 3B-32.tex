\documentclass{nuevotema}

\tema{3B-32}
\titulo{Economía y medio ambiente. Bienes públicos globales.}

\begin{document}

\ideaclave

Añadir apartado sobre regulación financiera, política monetaria, bancos centrales y medio ambiente. Ver \href{https://voxeu.org/article/central-banks-and-climate-change}{VOXEU (2020)}.

\seccion{Preguntas clave}
\begin{itemize}
	\item ¿En qué consiste el análisis económico del medio ambiente?
	\item ¿Qué problemas y qué fenómenos examina?
	\item ¿Qué modelos teóricos tratan de explicarlos?
	\item ¿Qué implicaciones se derivan?
	\item ¿Qué son los bienes públicos globales?
	\item ¿Qué modelos teóricos explican las ineficiencias en su provisión?
	\item ¿En qué marco institucional se regula su provisión?
\end{itemize}

En este tema hay que meter Barret (1994) sobre política comercial estratégica y medio ambiente. Coordinar con tema 3B-8.

\esquemacorto

\begin{esquema}[enumerate]
	\1[] \marcar{Introducción}
		\2 Contextualización
			\3 Economía
			\3 Medio ambiente
			\3 Interacción entre economía y medio ambiente
		\2 Objeto
			\3 ¿En qué consiste el análisis económico del medio ambiente?
			\3 ¿Qué problemas y fenómenos examina?
			\3 ¿Qué modelos teóricos tratan de explicarlos?
			\3 ¿Qué implicaciones se derivan?
			\3 ¿Qué son los bienes públicos globales?
			\3 ¿En qué marco institucional se regula su provisión?
		\2 Estructura
			\3 Análisis económico del medio ambiente
			\3 Bienes públicos globales
	\1 \marcar{Análisis económico del medio ambiente}
		\2 Recursos naturales
			\3 Idea clave
			\3 Evolución del análisis teórico
			\3 Modelo de Hotelling (1931)
			\3 Producción con recurso natural agotable y  finito
			\3 Modelo de Solow con recurso natural constante
			\3 Implicaciones
			\3 Evidencia empírica
		\2 Energía
			\3 Idea clave
			\3 Intensidad energética
			\3 Causalidad entre energía y output
			\3 Energía como factor de producción
			\3 Transición energética
			\3 Modelos de crecimiento económico con energía
		\2 Contaminación
			\3 Idea clave
			\3 Enfoque pigouviano
			\3 Enfoque de derechos de propiedad
			\3 Mercados de derechos de contaminación
			\3 Estándares de emisión y regulaciones tecnológicas
			\3 Paraísos de la contaminación
			\3 Curva de Kuznets medioambiental
			\3 Implicaciones
		\2 Cambio climático
			\3 Idea clave
			\3 Estrategias de respuesta
			\3 Modelos IAM: DICE y RICE
			\3 Implicaciones
		\2 Comercio internacional
			\3 Idea clave
			\3 Hipótesis de Porter
			\3 Efectos del comercio sobre el medio ambiente
			\3 Acuerdos comerciales
			\3 Política comercial estratégica
			\3 Implicaciones
		\2 Elección colectiva
			\3 Idea clave
			\3 Teorema de Arrow
			\3 Escapes a la imposibilidad
			\3 Criterio de Pareto
			\3 Criterios de compensación
			\3 Análisis coste-beneficio
			\3 Implicaciones
	\1 \marcar{Bienes públicos globales}
		\2 Idea clave
			\3 Contexto
			\3 Objetivos
			\3 Resultados
		\2 Análisis teórico
			\3 Características fundamentales
			\3 Bienes públicos globales finales e intermedios
			\3 Ejemplos
			\3 Tecnologías de provisión
			\3 Reglas de provisión óptima
			\3 Mecanismos de provisión
		\2 Marco institucional
			\3 Antecedentes
			\3 Internacional
			\3 Unión Europea
			\3 España
		\2 Valoración
			\3 Implicaciones de política económica
			\3 Retos
	\1[] \marcar{Conclusión}
		\2 Recapitulación
			\3 Análisis económico del medio ambiente
			\3 Bienes públicos globales
		\2 Idea final
			\3 Contabilidad nacional
			\3 Desarrollo económico
			\3 Interacción entre economía y ciencia climática
			\3 Papel fundamental del mercado

\end{esquema}

\esquemalargo

\begin{esquemal}
	\1[] \marcar{Introducción}
		\2 Contextualización
			\3 Economía
				\4 Definición de Robbins
				\4[] Economía es estudio de comportamiento humano
				\4[] $\to$ Gestionando recursos escasos con usos alternativos
				\4[] $\to$ Para satisfacer una serie de necesidades humanas
			\3 Medio ambiente
				\4 Entorno en el que actividad humana se lleva a cabo
				\4[] Atmósfera, tierra, agua...
				\4 Condiciones óptimas para el ser humano
				\4[] Variables físicas y estados biológicos
				\4[] Temperatura, radiación, presión
				\4[] Fertilidad del suelo, minerales disponibles
				\4[] Diversidad biológica, ausencia de organismos dañinos...
			\3 Interacción entre economía y medio ambiente
				\4 Actividad económica influye en medio ambiente
				\4[] Transformación de energía
				\4[] Liberación de gases atmosféricos
				\4[] Extracción de recursos naturales
				\4[] Deslocalización de actividad económica
				\4[] Elección colectiva sobre medio ambiente
				\4 Medio ambiente influye en actividad económica
				\4[] Productividad total de los factores
				\4[] Rendimientos agrícolas
				\4[] Destrucción de capital físico y humano
				\4[] Crecimiento económico
				\4 Interacción recíproca entre ambas
				\4[$\then$] Necesario estudio conjunto
		\2 Objeto
			\3 ¿En qué consiste el análisis económico del medio ambiente?
			\3 ¿Qué problemas y fenómenos examina?
			\3 ¿Qué modelos teóricos tratan de explicarlos?
			\3 ¿Qué implicaciones se derivan?
			\3 ¿Qué son los bienes públicos globales?
			\3 ¿En qué marco institucional se regula su provisión?
		\2 Estructura
			\3 Análisis económico del medio ambiente
				\4 Recursos naturales
				\4 Energía
				\4 Contaminación
				\4 Cambio climático
				\4 Política comercial
				\4 Elección colectiva
			\3 Bienes públicos globales
				\4 Análisis teórico
				\4 Marco institucional
				\4 Valoración
	\1 \marcar{Análisis económico del medio ambiente}
		\2 Recursos naturales
			\3 Idea clave
				\4 Contexto
				\4[] Recursos naturales
				\4[] No reproducibles mediante actividad económica
				\4[] $\to$ Distintos a capital
				\4[] $\to$ Recursos económicamente significativos
				\4[] $\to$ Presencia en la naturaleza
				\4[] Importancia económica de los recursos naturales
				\4[] $\to$ Inputs en casi todos los procesos productivos
				\4[] $\to$ Cantidades finitas
				\4[] Análisis económico relativamente escaso
				\4[] $\to$ Recursos generalmente no agotados
				\4 Objetivos
				\4[] Valorar importancia de stocks de recursos
				\4[] $\to$ Sobre actividad económica
				\4[] Caracterizar uso óptimo de recursos naturales
				\4[] Explicar efectos económicos de sustitución
				\4 Resultado
				\4[] Análisis dinámico de la extracción de recursos
				\4[] Incorporación de control óptimo
				\4[] Debate económico sobre agotamiento de recursos
			\3 Evolución del análisis teórico
				\4 Debate de largo alcance temporal
				\4 Malthus (1798)
				\4[] Población vs recursos
				\4[] Población es sistema estable
				\4[] $\to$ sistema de frenos al crecimiento
				\4[] $\then$ Población no puede crecer indefinidamente
				\4[] $\then$ Recursos disponibles son límite
				\4 Ricardo (1817)
				\4[] Calidad de los recursos y rentas
				\4[] Mayor demanda de recursos
				\4[] $\to$ Utilización de recursos de menor calidad
				\4[] $\then$ Rendimientos marginales decrecientes
				\4[] $\then$ Rentas de escasez
				\4 Jevons (1865)
				\4[] Efectos de escasez sobre precios
				\4[] $\to$ Escasez aumenta precios
				\4[] $\then$ Encarecimiento induce sustitución
				\4[] $\then$ Sustitutos no necesariamente eficientes
				\4[] $\then$ Frenos al crecimiento
				\4[] Paradoja de Jevons
				\4[] $\to$ Más capacidad induce más demanda
				\4 Hotelling (1931)
				\4[] Primera caracterización formal del problema
				\4 Hipótesis del progreso técnico
				\4[] Barnett y Morse (1963) y otros
				\4[] $\to$ Índices de coste real de extracción de recursos
				\4[] $\then$ Tienden a decrecer con el tiempo
				\4[] Explicación principal
				\4[] $\to$ Progreso técnico
				\4[] $\to$ Descubrimiento de yacimientos
				\4[] $\to$ Sustitución por otras materias primas
				\4[] Progreso técnico
				\4[] $\to$ Cada vez más eficiente uso de materias primas
				\4[] $\then$ Menores cantidades necesarias
				\4[] $\then$ Agotamiento cada vez más lejano
				\4 Hipótesis del agotamiento de recursos
				\4[] Combinación de sucesos
				\4[] $\to$ Meadows et al (1972): Limits to Growth
				\4[] $\to$ Crisis del petróleo de los 70
				\4[] $\to$ Estanflación
				\4[] Costes realmente tienen forma de U
				\4[] $\to$ Decrecen fuertemente inicialmente
				\4[] $\to$ Estancamiento posterior
				\4[] $\to$ Crecimiento final y agotamiento
				\4[] Predicción principal
				\4[] $\to$ Crecimiento econ. limitado por agotamiento
			\3 Modelo de Hotelling (1931)\footnote{Esta sección y la siguiente por ``\textit{exhaustible resources}'' y ``\textit{natural resources}'' de Palgrave.}
				\4 Idea clave
				\4[] Caracterización formal en términos matemáticos
				\4[] Cálculo de variaciones
				\4[] $\to$ Hallar trayectoria óptima de extracción
				\4[] $\to$ Sujeta a restricción de extracción total
				\4 Formulación
				\4[] $\underset{c_t}{\max} \quad \int_0^\infty u(c_t) \cdot e^{-\rho t} \, dt$
				\4[] $\text{s.a:} \quad \int_0^\infty c_t \, dt \leq C$
				\4[] $\then$ $\text{s.a:} \quad \frac{d S_t}{d t} = -c_t$, $S_t \geq 0$
				\4 Implicaciones
				\4[] Resolución mediante uso de Hamiltoniano
				\4[] Utilidad marginal debe igualar precio sombra
				\4[] $\to$ U. de consumo hoy = menor utilidad futura
				\4[] En términos intuitivos
				\4[] $\to$ ¿Cómo dividir consumo de una tarta?
				\4[] $\to$ ¿Cuánto consumir hoy y cuanto mañana?
				\4[] $\then$ Utilidades marginales deben igualarse
				\4[] Si tasa de descuento positiva
				\4[] $\to$ Cada vez menor consumo
				\4[] Si tasa de descuento cero o negativa
				\4[] $\to$ Todas las generaciones igual de importantes
				\4[] $\to$ Consumo óptimo tiende a cero en cada periodo
				\4[] $\then$ Todas deben poder consumir
				\4 Costes de extracción
				\4[] Extracción implica coste adicional
				\4[] Incorporable en función $u(c_t)$
				\4 Stock con valor intrínseco
				\4[] Agentes valoran también stock
				\4[] Ejemplo:
				\4[] $\to$ Biodiversidad
				\4[] $\to$ Bosque de crecimiento lento
				\4 Recurso renovable
				\4[] Incorporable en ecuación de dinámica
				\4[] $\then$ $\text{s.a:} \quad \frac{d S_t}{d t} = f(S_t) -c_t$, $S_t \geq 0$
				\4[] $\to$ $f(S_t)$: renovación del recurso en función de stock
			\3 Producción con recurso natural agotable y  finito
				\4 Dasgupta y Heal (1974) y (1979)
				\4 Dos factores de producción
				\4[] $\to$ Factor de producción acumulable
				\4[] $\to$ Factor de producción agotable $\then$ Recurso natural
				\4[] $\underset{I_t, R_t}{\max} \quad \int_0^\infty \left( F(K_t, R_t) - I_t \right) e^{-r t} \, dt$
				\4[] $\text{s.a:} \quad \frac{d K_t}{d t} = I_t$
				\4[] \quad \quad $\frac{d S_t}{d t} = - R_t$, $S_t \geq 0$
				\4 Asumiendo $F(K,0) = 0$
				\4[] $\to$ Recurso natural es relevante
				\4 Determinantes de dinámica óptima
				\4[--] ESustitución entre factor y recurso natural
				\4[] $\to$ Con ES < 1: agotamiento eventual
				\4[] $\to$ Con ES > 1: agotamiento no es un problema
				\4[] $\to$ Con ES = 1, elast F a K > ES: agotamiento no es problema
				\4[--] Progreso técnico
				\4[] $\to$ Puede aumentar productividad de factor escaso
				\4[] $\to$ Requerir cada vez menor recurso asintóticamente
				\4[--] Aparición de tecnología de reemplazo
				\4[] $\to$ Horizonte $\infty$ pasa a serlo finito
				\4[] $\then$ Ej.: aparición de fusión nuclear
			\3 Modelo de Solow con recurso natural constante\footnote{Ver \href{http://www.artsrn.ualberta.ca/econweb/hryshko/econ403fall2010/CHAPTER9.pdf}{Hryshko (2010)}.}
				\4 Idea clave
				\4[] Recursos naturales no pueden ser producidos
				\4[] Algunos son constantes
				\4[] $\to$ Especialmente, tierra
				\4[] Necesarios en casi todo proceso productivo
				\4[] $\to$ Pero imposibles de replicar
				\4[] $\then$ Inducen rendimientos decrecientes a escala
				\4 Formulación
				\4[] $Y = A K^\alpha T^\beta L^{1-\alpha-\beta}$
				\4[] $\frac{\dot{A}}{A} = g$
				\4[] $\frac{\dot{L}}{L} = n$
				\4[] Retornos decrecientes a escala
				\4[] $\to$ En trabajo y en capital
				\4 Implicaciones
				\4[] Importancia de recurso natural en producción
				\4[] $\to$ Reduce output progresivamente
				\4[] $\then$ Cuanto mayor $\beta$, mayor R$\downarrow$E
				\4[] Output per-cápita tiende a cero
				\4[] $\to$ A medida que se realizan R$\downarrow$E
				\4[] $\then$ Salvo que progreso tecnológico compense
				\4[] Necesario progreso tecnológico
				\4[] $\to$ Para compensar caída vía $R\downarrow E$
				\4 Valoración
				\4[] Peso de factores no renovables en actividad
				\4[] $\to$ Aumenta R$\downarrow$E
				\4[] Sustitución de actividades intensivas en no renovables
				\4[] $\to$ Permite mayor crecimiento en largo plazo
			\3 Implicaciones
				\4 Agotamiento no necesariamente se produce
				\4[] Si suficiente sustitución por otro factor
				\4[] Si avances tecnológicos reducen uso
				\4 Condiciones necesarias para no agotamiento
				\4[] Caracterizadas de diferentes formas
				\4[] Necesario énfasis de PEconómica sobre:
				\4[] $\to$ Búsqueda de alternativas
				\4[] $\to$ Buen funcionamiento de sistema de precios
				\4[] $\to$ Incentivos  a tecnología
				\4 Teoría caracteriza importancia de recursos naturales
				\4[] No necesariamente relevante que se agoten
				\4[] Progreso técnico es posible escape a agotamiento
				\4[] Dadas condiciones óptimas
				\4[] $\to$ Agotamiento es posible
				\4 Aplicación del análisis en marco de Hotelling
				\4[] Agotamiento cuando
				\4[] $\to$ Progreso técnico insuficiente
				\4[] $\to$ Falta de sustitutivos
			\3 Evidencia empírica
				\4 Sin agotamiento de ningún rec. mineral relevante
				\4[] Grandes stocks
				\4[] Sustitutos relativamente fáciles de encontrar
				\4 Agotamiento de recursos biológicos habitual
				\4[] Dificultad para encontrar sustitutos
				\4 Economías planificadas tienen más problemas
				\4[] Peor gestión de los recursos
				\4[] Aparición de sustitutos más tardía
				\4 Aumento periódico del interés en el problema
				\4 Energía y calidad del medio ambiente
				\4[] Principales objetos de análisis/preocupación
		\2 Energía
			\3 Idea clave
				\4 Contexto
				\4[] Energía es:
				\4[] $\to$ input en todo proceso productivo
				\4[] $\to$ input extraído de recursos naturales
				\4[] Recursos naturales energéticos
				\4[] $\to$ No todos renovables o inagotables
				\4[] $\to$ Energía de extracción barata agotable
				\4[] $\to$ Energías alternativas más caras
				\4[] Progreso tecnológico afecta
				\4[] $\to$ Coste de extracción de energía
				\4[] $\to$ Uso de energía para producir output
				\4[] Energías baratas causan externalidades negativas
				\4[] $\to$ Contaminación atmosférica
				\4[] $\to$ Otros tipos de contaminación
				\4[] $\to$ Cambio climático
				\4[] $\then$ Transición a otras energías es costosa
				\4[] $\then$ Necesario valorar trade-offs
				\4 Objetivos
				\4[] Caracterizar relación PIB y consumo de energía
				\4[] Explicar efecto de precios sobre demanda de energía
				\4[] Valorar trade-offs de transición energética
				\4 Resultado
				\4[] Eficiencia energética aumenta en largo plazo
				\4[] Elasticidad-precio muy reducida
				\4[] Transición energética dependiente de progreso
				\4[] Sustituibilidad capital-energía heterogénea
			\3 Intensidad energética
				\4 Concepto
				\4[] Cantidad de energía necesaria
				\4[] $\to$ Para producir una unidad de output
				\4 Elasticidad PIB-consumo de energía
				\4[] Empíricamente, ligeramente inelástica
				\4[] $\to$ $\Delta 1 \%$ PIB $\to$ $\Delta 0,7\%$ energía
				\4 Uso total de energía
				\4[] Creciente en el tiempo
				\4 Energía per-capita
				\4[] Uso total de energía aumenta en el tiempo
				\4[] $\to$ Más que población
				\4[] $\then$ Aumento de energía per-cápita
				\4 Causas de la caída de l/p de intensidad energética
				\4[] Crecimiento tecnológico endógeno
				\4[] $\to$ Precios de la energía inducen sustitución
				\4[] $\to$ Etiquetado mejorado de productos
				\4[] Preferencias ideológicas
				\4[] $\to$ Ecologismo
				\4[] Crecimiento tecnológico exógeno
				\4[] $\to$ Cambio estructural de mix energético
				\4 Asimetría entre consumo e industria
				\4[] Innovaciones técnicas
				\4[] $\to$ Aumentan consumo energético de hogares
				\4[] $\to$ Reduce consumo energético en industria
			\3 Causalidad entre energía y output
				\4 Relación consistente entre energía y output
				\4[] Elasticidad-output de energía entre 0 1
				\4[] $\to$ ¿Pero qué dirección tiene la causalidad?
				\4[] $\then$ ¿Más output causa más uso de energía?
				\4[] $\then$ ¿Más uso de energía causa más output?
				\4 Controlando por precios constantes
				\4[] Crecimiento causa aumento de consumo energético
				\4[] $\to$ Resultado relativamente robusto
				\4 Controlando por stock de capital
				\4[] Resultados contradictorios y poco robustos
				\4[] En c/p, precios energéticos...
				\4[] $\to$ Parece causar PIB
				\4[] $\to$ Parece causar consumo energético
				\4[] En l/p, output...
				\4[] $\to$ Parece causar precios de energía
				\4[] $\to$ Parece causar consumo de energía
			\3 Energía como factor de producción
				\4 Analizable en marco de Hotelling
				\4 Efecto rebote
				\4[] Aumento de eficiencia energética
				\4[] $\to$ Induce mayor consumo de energía
				\4[] $\to$ También de complementarios a energía
				\4[] $\then$ No necesariamente crecimiento neto
				\4[] Explicación
				\4[] $\to$ Más eficiencia reduce demanda
				\4[] $\then$ Caen precios y aumenta demanda
				\4[] $\to$ Mejora técnica induce acumulación de capital
				\4[] $\then$ Más demanda de energía
				\4[] Empíricamente recurrente
				\4[] Aparentemente mayor en PEDs
			\3 Transición energética
				\4 Dos tipos de sustitución
				\4[] i. Energía barata poco abundante
				\4[] $\to$ Por energía más abundante pero más cara
				\4[] ii. Energía por capital
				\4 Sustitución energía por capital
				\4[] Cuando se encarece precio de energía
				\4[] Malos sustitutos en corto plazo
				\4[] $\to$ Cercanos a 0.2
				\4[] Buenos sustitutos en largo plazo
				\4[] $\to$ Cercano a 1
				\4[] $\then$ Posible explicación de caída de intensidad energética
			\3 Modelos de crecimiento económico con energía
				\4 Energía en análisis de crecimiento
				\4[] Tema poco examinado
				\4[] Relativamente poca literatura
				\4 Solow (1974)
				\4 Tres factores de producción
				\4[] $\to$ Capital
				\4[] $\to$ Trabajo
				\4[] $\to$ Energía
				\4 Mostrar condiciones bajo las cuales
				\4[] Mantenimiento de producción constante es posible
				\4[] Asumiendo s.p.g.
				\4[] $\to$ Sin crecimiento técnico
				\4[] $\to$ Sin crecimiento de la población
				\4[] Condición fundamental
				\4[] $\to$ Tecnología suficientemente sensible a $\Delta$ en precios
				\4[] $\then$ Elasticidad de sustitución no inferior a 1
				\4[] Condición necesaria institucional
				\4[] $\to$ Sin descuento de generaciones futuras
				\4[] $\to$ En caso contrario
				\4[] $\then$ Eventual agotamiento de recurso no renovable
				\4 Implicaciones
				\4[] Crecimiento técnico es esencial para no agotar recursos
				\4[] Descuento de generaciones futuras
				\4[] $\to$ Fuerza a favor de agotamiento de recursos
		\2 Contaminación
			\3 Idea clave
				\4 Contexto
				\4[] Externalidades
				\4[] $\to$ Efectos de una actividad sobre terceros
				\4[] Externalidades tecnológicas inducen efecto directo
				\4[] $\to$ Sobre utilidad de terceros
				\4[] $\to$ Sobre tecnología de producción de tercero
				\4[] Contaminación
				\4[] $\to$ Introducción de sustancias en medio ambiente
				\4[] $\then$ con efectos indeseables sobre producción y consumo
				\4[] Mal público
				\4[] $\to$ Similar a bien público
				\4[] $\to$ Mal no rival y no excluible
				\4[] $\then$ Afecta a todos los agentes en determinada área
				\4[] $\then$ Efecto sobre agente no precluye efecto sobre tercero
				\4[] Análisis económico de la contaminación
				\4[] $\to$ Habitualmente ligado a análisis de externalidades
				\4[] $\to$ Enfoque de derechos de propiedad
				\4 Objetivos
				\4[] Caracterizar formalmente efecto económico de contaminación
				\4[] Diseñar instituciones que mitiguen efecto negativo
				\4[] Implementar incentivos correcto para reducir contaminación
				\4 Resultado
				\4[] Dos grandes enfoques
				\4[] $\to$ Contaminación como distorsión
				\4[] $\to$ Contaminación como mercado incompleto
				\4[] Curva de Kuznets
				\4[] $\to$ Resultado empírico
				\4[] $\to$ Contaminación-output: curva de U invertida
				\4[] $\then$ Contaminación crece y luego decrece
				\4[] Outsourcing de la contaminación
				\4[] $\to$ Soluciones para mitigar inducen deslocalización
				\4[] $\then$ Actividades contaminantes se trasladan
				\4[] $\then$ Necesario considerar eq. general
			\3 Enfoque pigouviano
				\4 Contaminación como externalidad negativa
				\4 Externalidad negativa impide cumplimiento de PTFB
				\4[] Agentes igualan:
				\4[] $\to$ Utilidad marginal individual
				\4[] $\to$ Coste marginal individual de consumo
				\4[] Utilidades y costes individuales
				\4[] $\to$ Son distintos a sociales
				\4[] $\then$ Igualación de UMg y CMg indiv. es ineficiente
				\4[] Consumo/producción de un bien
				\4[] $\to$ Impone coste marginal sobre un tercero
				\4[] $\then$ Agente no tiene en cuenta efecto sobre tercero
				\4 Solución pigouviana
				\4[] Transformar condición de equilibrio competitivo
				\4[] $\to$ En condición de óptimo social
				\4 Imponer impuesto en problema de máx.
				\4[] $\to$ Qué iguale condiciones de primer orden
				\4[] $\then$ Agente A internaliza max. bienestar social
				\4 Formulación (MWG)
				\4[] Imponer impuesto T a máx. de A tal que:
				\4[] $\underset{h}{\max} \quad u_A = \phi_A(h) - t_h \cdot h $
				\4[] \fbox{$\text{CPO:} \quad \quad \phi_A'(h) = t_h$}
				\4[] Elegir $t_h$ de óptimo tal que:
				\4[] $\to$ Se igualen CPOs competitiva + impuesto y óptimo
				\4[] $\to$ $t_h = \phi_B'(h^0)$
				\4[] $\Rightarrow$ Impuesto igual a UMg de B en óptimo social
				\4[] \grafica{impuestopigouviano}
				\4 Ejemplo:
				\4[] Fábrica situada río arriba
				\4[] Acuicultores río abajo
				\4[] Producción de fábrica libera contaminantes en río
				\4[] $\to$ Productividad de acuicultor se ve reducida
				\4[] Fábrica no tiene en cuenta efecto sobre acuicultores
				\4[] $\to$ Iguala beneficios y costes marginales privados
				\4[] Impuesto pigouviano sobre producción de fábrica
				\4[] $\to$ Cuantía igual a efecto marginal sobre acuicultores
				\4[] $\to$ Aumenta coste marginal de producción
				\4[] $\then$ Iguala beneficio marginal con óptimo social
				\4 Caso particular de teoría del second-best
				\4[] Dada una restricción que induce distorsión
				\4[] $\to$ Efecto externo tecnológico de consmo/producción
				\4[] No introducir restricciones adicionales
				\4[] $\to$ No necesariamente mejora el equilibrio
			\3 Enfoque de derechos de propiedad
				\4 Coase (1960)
				\4[] Externalidades son realmente mercados incompletos
				\4[] $\to$ No hay mercado para bien/mal externo
				\4 Definición de derechos de propiedad
				\4[] Sobre contaminación
				\4[] $\to$ Permite alcanzar soluciones óptimas
				\4 Ejemplo
				\4[] Fábrica situada río arriba
				\4[] Acuicultores río abajo
				\4[] Producción de fábrica libera contaminantes en río
				\4[] $\to$ Productividad de acuicultor se ve reducida
				\4[] Derecho a utilizar agua sin contaminar
				\4[] $\to$ No está bien definido
				\4[] $\then$ Ni fábrica ni acuicultores poseen
				\4[] Gobierno define propiedad sobre limpieza de aguas
				\4[] $\to$ Se la concede a fábrica o acuicultores
				\4[] Si se la concede a fábrica
				\4[] $\to$ Tiene derecho a contaminar río
				\4[] $\then$ Acuicultores perjudicados
				\4[] Si se la concede a acuicultores
				\4[] $\to$ Tiene derecho a agua sin contaminación
				\4[] $\then$ Fábrica no puede operar
				\4[] Derechos de propiedad permiten acuerdo entre partes
				\4[] $\to$ Fábrica paga acuicultores por contaminar
				\4[] $\to$ Acuicultores pagan por mantener agua limpia
				\4[] $\then$ PTFB vuelve a cumplirse
			\3 Mercados de derechos de contaminación
				\4 ``Cap-and-trade''
				\4 Aplicación de enfoque de derechos de propiedad
				\4 Asignar derechos limitados de contaminación
				\4 Permitir intercambio posterior en mercado organizado
				\4 Determinantes de buen funcionamiento
				\4[] $\to$ Costes de transacción reducidos
				\4[] $\to$ focos de emisión bien delimitados
				\4 Implicaciones
				\4[] Agentes emisores con mayores costes de reducción
				\4[] $\to$ Comprarán derechos a aquellos con menores costes
				\4[] Agentes emisores con menores costes
				\4[] $\to$ Reducirán emisiones y venderán derechos de emisión
				\4[] Innecesario determinar
				\4[] $\to$ Costes de reducción
				\4[] $\to$ Efecto sobre procesos productivos
				\4[] $\then$ Agentes incentivados a revelar costes
				\4[] Limitación de emisiones a menor coste posible
				\4[] $\to$ Si se cumple los determinantes de buen funcionamiento
				\4[] $\to$ Independientemente de nivel óptimo de contaminación
				\4 Ejemplos
				\4[] Programa de reducción de dióxido de sulfuro
				\4[] $\to$ Principios de los 90, Estados Unidos
				\4[] $\to$ Mitigación de efectos de lluvia ácida
				\4[] Emission Trading System
				\4[] $\to$ Unión Europea
				\4[] $\to$ Reducción de emisiones de gases de EInvernadero
				\4 Contextos de difícil aplicación
				\4[] -- Daño marginal depende de localización
				\4[] En algunas áreas, contaminación induce grandes daños
				\4[] En otras, apenas tiene efectos
				\4[] Esquema cap-and-trade puede inducir
				\4[] $\to$ Desviación de polución a lugares con más daño
				\4[] $\then$ Esquema ineficiente.
				\4[] -- Emisiones muy difusas
				\4[] Muchos emisores que emiten pequeñas cuantías
				\4[] Costes de transacción y asignación aumentan fuertemente
			\3 Estándares de emisión y regulaciones tecnológicas
				\4 Complemento técnico a soluciones anteriores
				\4 Límites legales a emisión de contaminantes
				\4 Implementación obligatoria de tecnologías de producción
				\4[] $\to$ Consideradas/probadas menos contaminantes
				\4 Tienen costes adicionales
				\4[] Coste regulatorio
				\4[] Coste de adquisición de información
				\4[] Distorsión de procesos productivos
			\3 Paraísos de la contaminación
				\4 Hipótesis teórica
				\4[] Resultados relativamente poco robustos
				\4[] Confirmación y rechazo con diferentes teorías
				\4 Premisa
				\4[] Regulación medioambiental en todas su formas
				\4[] $\to$ Aumenta coste de producir en jurisdicción
				\4[] Productos intensivos en contaminación
				\4[] $\to$ En desventaja comparativa
				\4 Resultados empíricos
				\4[] Estudios de sección cruzada
				\4[] $\to$ Regresar bien contaminante vs regulación
				\4[] $\then$ Rechazan hipótesis de regulación
				\4[] Estudios de panel recientes (2000 y 2010s)
				\4[] $\to$ Sí hay relación causal regulación-deslocalización
				\4[] $\to$ Sin pruebas para afirmar suboptimalidad
				\4[] Problemas
				\4[] $\to$ Dificil cuantificar regulación fuerte o débil
				\4[] $\to$ Difícil cuantificación de barreras comerciales
				\4[] $\to$ Regulación puede ser endógena a contaminación
				\4 Implicaciones
				\4[] Libre comercio amplifica fenómeno
				\4[] $\to$ Más fácil desplazar a jurisdicciones laxas
				\4[] $\then$ Libre comercio puede aumentar contaminación
				\4[] Posible race-to-the-bottom
			\3 Curva de Kuznets medioambiental
				\4 Basado en curva de Kuznets original
				\4[] Ante crecimiento ecónomico
				\4[] $\to$ Desigualdad primero aumenta y luego cae
				\4 Grossman y Krueger (1995), Banco Mundial (1992)
				\4 Dasgupta, Laplante, Wang, Wheeler (2002)
				\4 Resultado empírico
				\4[] Relación entre:
				\4[] $\to$ PIBpc
				\4[] $\to$ contaminación del medio ambiente (agua, aire..)
				\4[] Inicialmente creciente
				\4[] Decreciente a partir de cierto máximo
				\4[] $\then$ ¿Calidad medioambiental es bien normal?
				\4 Dinámica heterogénea
				\4[] Picos de contaminación dependen de sustancia
				\4[] Generalmente, antes de alcanzar $\$8000$ PIBpc
				\4 Tendencia a desplazarse
				\4[] Aparentemente, pico cada vez con menor PIBpc
				\4 Interpretación causal
				\4[] Resultado empírico depende de formas reducidas
				\4[] Errónea interpretación causal
				\4[] $\to$ Crecimiento no reduce contaminación
			\3 Implicaciones
				\4 Capacidad coactiva del estado
				\4[] Elemento esencial de soluciones
				\4[] $\to$ Introducir restricciones adicionales
				\4[] $\to$ Definir derechos de propiedad
				\4[] $\then$ Implica poder coactivo
				\4[] $\then$ Sin poder coactivo, solución mucho más dificil
				\4 Instrumento óptimo
				\4[] Goulder y Parry (2008)
				\4[] $\to$ Valoración de múltiples instrumentos
				\4[] $\then$ No hay instrumento óptimo general
				\4[] $\then$ Instrumento óptimo depende de circunstancias
				\4 Requisitos informacionales
				\4[] Información asumida disponible en ambos enfoques
				\4[] Diferentes informaciones necesarias
				\4[] $\to$ Bfcios. y costes marginales en pigouviano
				\4[] $\to$ Monitorización de actuaciones en derechos de prop.
				\4[] En la práctica
				\4[] $\to$ Obstáculo principal a implementación efectiva
				\4 Soluciones tienen efectos de equilibrio general
				\4[] Globalización permite movimiento de ff.pp.
				\4 Mercados de derechos
				\4[] Requieren experiencia y adaptación de empresas
				\4 Efectos redistributivos
				\4[] Ambos enfoques tienen efectos redistributivos
				\4[] Necesario considerar impacto
				\4[] $\to$ Economía política
				\4[] $\to$ Inestabilidad social
		\2 Cambio climático
			\3 Idea clave
				\4 Contexto
				\4[] Clima atmosférico
				\4[] $\to$ Temperatura
				\4[] $\to$ Presión
				\4[] $\to$ Precipitaciones
				\4[] $\then$ Influyen en actividad económica
				\4[] Actividad económica y procesos productivos
				\4[] $\to$ Emisión de gases contaminantes
				\4[] $\then$ Evidencia científica relevante
				\4[] Posible existencia de relación causal
				\4[] $\to$ Emisión de gases
				\4[] $\then$ Cambios en clima
				\4[] $\then$ Efectos globales y regionales
				\4[] $\then$ Evidencia empírica relevante
				\4 Objetivos
				\4[] Caracterizar relación entre;
				\4[] $\to$ Actividad económica
				\4[] $\to$ Emisión de gases de efecto climático
				\4[] Diseñar mecanismos de provisión eficiente
				\4[] $\to$
				\4 Resultado
				\4[] Estabilidad climática como bien público
				\4[] Ausencia de gobierno global con capacidad coactiva
				\4[] Problemas de free-riding
				\4[] Incertidumbre dificulta determinación
			\3 Estrategias de respuesta
				\4 Ingeniería climática
				\4[] Soluciones tecnológicas de efecto global
				\4[] Dificultades
				\4[] $\to$ Costes elevados
				\4[] $\to$ Efectividad incierta
				\4 Adaptación
				\4[] Implementación de:
				\4[] $\to$ Soluciones tecnológicas
				\4[] $\to$ Mecanismos institucionales
				\4[] $\then$ Relocalización de actividades
				\4[] $\then$ Sustitución de inputs y bienes de consumo
				\4 Mitigación
				\4[] Sistemas de incentivos
				\4[] $\to$ Reducir emisión de gases de efecto de CClimático
				\4[] $\to$ Sustitución por tecnologías que emiten menos
				\4[] $\to$ Construcción de sumideros de carbono
				\4[] Instrumentos anteriores para contaminación
				\4[] $\to$ Impuestos a emisiones de carbono
				\4[] $\to$ Sistemas cap-and-trade
				\4[] Implementación mucho más compleja
				\4[] $\to$ ¿Acuerdos voluntarios entre países soberanos?
			\3 Modelos IAM: DICE y RICE
				\4 Integrated Assessment Models
				\4 Nordhaus (Nobel 2018), Koopmans (Nobel), otros
				\4 Dynamic Integrated Climate-Economy
				\4 Regional Integrated Climate-Economy
				\4 Idea clave
				\4[] Familia de modelos matemáticos
				\4[] $\to$ Integran crecimiento económico y clima
				\4[] Caracterizar efectos sobre clima a partir de:
				\4[] $\to$ Inversión en capital físico y humano
				\4[] $\to$ Tecnologías de producción
				\4[] $\to$ Demanda de consumo
				\4[] $\to$ Stock de gases de efecto invernadero
				\4[] $\to$ Preferencias de consumidores
				\4[] Equilibrio computable
				\4[] $\to$ Aproximaciones
				\4[] $\to$ Método de Monte-Carlo
				\4[] Herramienta habitual para valorar políticas climáticas
				\4 Representación de escenarios
				\4[] Escenario base
				\4[] $\to$ Sin políticas de mitigación
				\4[] Óptimo
				\4[] $\to$ Maximización de utilidad
				\4[] Temperatura limitada
				\4[] $\to$ Fijación de límite de $\Delta$ temperatura
				\4[] $\then$ ¿Qué restricciones económicas hay que aplicar?
				\4 Críticas
				\4[] Demasiado sensible a condiciones iniciales
				\4[] Simplificación excesiva de dinámicas climáticas
			\3 Implicaciones
				\4 Cambio climático como bien público global
				\4 Modelos teóricos combinan economía+climatología
				\4 Sensibilidad a condiciones iniciales
				\4[] Gran debilidad de modelos
				\4 Necesario ponderar costes y beneficios
				\4[] Limitar efectos climáticos
				\4[] $\to$ Debe ponderarse con coste económico
		\2 Comercio internacional
			\3 Idea clave
				\4 Contexto
				\4[] Comercio internacional permite
				\4[] $\to$ División del trabajo/producción
				\4[] $\then$ Aprovechamiento de ventajas comparativas
				\4[] $\then$ Maximizar utilización de capacidad productiva
				\4[] $\then$ Deslocalización de producción
				\4[] Debate sobre consecuencias medioambientales de LComercio
				\4[] $\to$ Especialmente a partir de NAFTA y WTO
				\4[] Regulación medioambiental
				\4[] $\to$ Altera incentivos de empresas sobre localización
				\4[]
				\4 Objetivos
				\4[] Caracterizar efectos de reg. medioambiental
				\4[] $\to$ Sobre comercio internacional
				\4[] $\to$ Sobre competitividad
				\4[] Considerar incentivos a comportamiento estratégico
				\4[] $\to$ En materia de política comercial
				\4 Resultado
				\4[] Argumentos teóricos a f
			\3 Hipótesis de Porter
				\4 Porter (1991), Porter y van den Linde (1995)
				\4 Contaminación es resultado de ineficiencia
				\4[] Ineficiencia X y otros fenómenos
				\4[] $\to$ Mantienen producción ineficiente
				\4 Regulación medioambiental
				\4[] En la medida en que reduzca contaminación
				\4[] $\to$ En todas sus formas
				\4[] $\to$ Especialmente instrumentos de mercado\footnote{Esto es, cap-and-trade e impuestos al carbono.}
				\4[] $\then$ Aumenta innovación
				\4[] $\then$ Induce mejoras de eficiencia
				\4[] $\then$ Reduce pérdidas de competitividad en terceros
				\4 Efectos de regulación en largo plazo
				\4[] Presión para aumentar eficiencia
				\4[] $\to$ Meora productividad
				\4[] En medio y largo plazo
				\4[] $\to$ Reg. medioambiental aumenta competitividad
				\4 Evidencia empírica\footnote{ver Jaffe, Peterson, Portney, Stavins (1995)}
				\4[] Generalmente:
				\4[] $\to$ Efectos pequeños o no significativos
				\4[] $\to$ Segundo orden respecto a otros determinantes
			\3 Efectos del comercio sobre el medio ambiente
				\4 Coppeland y Taylor (2004)
				\4[] Survey de artículos
				\4[] $\to$ CI y medioambiente
				\4[] Valorar efectos de reg. MAmbiental sobre CI
				\4[] $\to$ Indirectamente, reg. MAmbiental sobre crecimiento
				\4 Primera fase de estudios
				\4[] Desde 70 hasta primeros 90
				\4[] Libre comercio puede reducir bienestar
				\4[] Si costes medioambientales no están internalizados
				\4[] $\to$ Aumento de output por CInternacional
				\4[] $\then$ Aumento de contaminación
				\4[] Política comercial puede sustituir reg. medioambiental
				\4[] $\to$ Posible reducir producción en países contaminantes
				\4[] $\then$ Imponiendo restricciones a política comercial
				\4[] En general, necesario considerar second-best
				\4[] $\to$ Libre comercio puede no ser óptimo
				\4 Segunda fase
				\4[] A partir de años 90
				\4[] Aumento del output induce menor contaminación
				\4[] $\to$ Comercio internacional es vía de reducción
			\3 Acuerdos comerciales
				\4 Regulación medioambiental en ALCs
				\4[] Sujeto de negociación habitual
				\4[] Presencia cada vez mayor
				\4 Diferentes niveles de tratamiento
				\4[] Algunos separan general de medio ambiente
				\4[] $\to$ Dentro del articulado
				\4[] Otros introducen previsiones específicas
				\4[] Acuerdos que no incluyen MAmbiente
				\4[] $\to$ Reformados posteriormente
				\4 Interés defensivo de PEDs
				\4[] PEDs reticentes a integrar medio ambiente
				\4[] $\to$ Sistemas nacionales de protección sin desarrollar
				\4[] $\to$ Poco conocimiento de asuntos medioambientales
				\4[] $\to$ Miedo a perder competitividad
				\4[] Acuerdos entre PEDs
				\4[] $\to$ Poca protección medioambiental
				\4 Ejemplos
				\4[] CETA
				\4[] $\to$ Capítulo 24 dedicado a medio ambiente
				\4[] $\to$ Solución de diferencias MAmbiente expresa
				\4[] USMCA
				\4[] $\to$ Prohibido incentivar comercio o inversión
				\4[] $\then$ Por medio de reducción de protección MAmbiente
				\4[] UE--Mercosur
				\4[] $\to$ Prohibición de derogación de leyes MAmbientales
				\4[] $\to$ Prohibe cumplimiento laxo para $\uparrow$ competitividad
				\4[] $\to$ Gestión sostenible de recursos pesqueros y otros
				\4[] $\to$ Cumplimiento de Acuerdo de París
			\3 Política comercial estratégica
				\4 Barrett (1994)
				\4 Idea clave
				\4[] Reg. medioambiental
				\4[] $\to$ Puede ser instrumento de PComercial
				\4 Incentivos de PComercial a regulación débil/fuerte
				\4[] Regulación débil entendida como:
				\4[] $\to$ Coste de mitigación menor que daño marginal
				\4[] Depende de:
				\4[] $\to$ Estructura competitiva nacional e internacional
				\4 Competencia à la Cournot
				\4[] Incentivos a establecer reg. medioambiental débil
				\4[] Mantener amenaza de cantidad producida elevada
				\4 Competencia à la Bertrand
				\4[] Incentivos a establecer reg. medioambiental fuerte
				\4[] Reg. medioambiental es amenaza creíble de $\uparrow$ precio
			\3 Implicaciones
				\4 Curva de Kuznets medioambiental
				\4[] Si:
				\4[] $\to$ Se cumple realmente
				\4[] $\to$ Comercio tiene efectos positivos sobre crecimiento
				\4[] $\then$ Comercio afecta positivamente a medioambiente
				\4 Regulación medioambiental es instrumento de PolCom
				\4 Ac. comerciales modernos incorporan reg. medioambiental
				\4[] Interés:
				\4[] $\to$ Ofensivo de desarrollados
				\4[] $\to$ Defensivo de PEDs
		\2 Elección colectiva
			\3 Idea clave
				\4 Contexto
				\4[] Medidas de política económica con impacto MAmbiente
				\4[] $\to$ Diferente impacto sobre diferentes agentes
				\4[] Preferencias heterogéneas sobre políticas
				\4[] Medio ambiente tiene efectos generales pero heterogéneos
				\4[] $\to$ Muy amplio número de ciudadanos
				\4[] $\to$ Diferente intensidad para unos y otros
				\4[] Sociedades modernas muy complejas
				\4[] $\to$ Necesario simplificar valoración
				\4 Objetivos
				\4[] Agregar preferencias individuales
				\4[] $\to$ En preferencias sociales
				\4[] Aplicar agregación
				\4[] $\to$ A determinación de políticas medioambientales
				\4 Resultado
				\4[] Resultado de imposibilidad
				\4[] $\to$ Imposible agregar cumpliendo determinados requisitos
				\4[] Posibles agregaciones alternativas
				\4[] $\to$ Preferencias cardinales
				\4[] Análisis coste-beneficio
				\4[] $\to$ Implementación práctica de agregación de preferencias
			\3 Teorema de Arrow
				\4 Contexto de preferencias individuales ordinales
				\4 Perfiles de preferencias
				\4[] Conjuntos de preferencias individuales
				\4[] $\to$ Representan prefs. individuales de una sociedad
				\4[] $\to$ Sobre posibles estados sociales
				\4 Ordenaciones sociales
				\4[] Relación binaria de estados sociales
				\4 Funcional de elección social
				\4[] Aplicación del
				\4[] $\to$ Espacio de perfiles de preferencias
				\4[] Al
				\4[] $\to$ Espacio de ordenaciones sociales
				\4 Axiomas exigibles a funcional de elección social
				\4[] U -- Dominio del funcional no admite restricciones
				\4[] WP -- Criterio débil de Pareto
				\4[] $\to$ Si todos los agentes prefieren un estado a otro
				\4[] $\then$ Ordenación social también debe preferirlo
				\4[] IIA -- Independencia de las alternativas irrelevantes
				\4[] $\to$ Sólo prefs. entre $x$ e $y$ importan para orden social entre x e y
				\4[] $\then$ Cambios en preferencias indiv. sobre otras alternativa
				\4[] $\then$ No alteran preferencia social entre $x$ e $y$
				\4[] D -- Ningún agente individual es ``dictador''
				\4[] $\to$ No hay ord. social que dependa perfectamente de nadie
				\4[] $\then$ Ordenación social no coincide perfectamente con nadie
				\4[] RP -- Ordenación social es un orden de preferencia
				\4[] $\to$ Completo y transitivo (y reflexivo)
				\4[U] -- Dominio completo (Unrestricted domain)
				\4 Resultado
				\4[] Imposible encontrar funcional social
				\4[] $\to$ Que cumpla los cinco axiomas
				\4 Críticas
				\4[] -- Hylland
				\4[] Axioma D realmente es prohibición de ``conformista''
				\4[] $\to$ Término ``dictador'' induce sesgo
				\4[] $\to$ Axioma D parece más relevante de lo que es
				\4[] -- Little
				\4[] Lo importante es ordenar estados sociales
				\4[] $\to$ No tanto relacionar perfiles con ord. social
				\4[] $\to$ Imposibilidad no es muy relevante
			\3 Escapes a la imposibilidad
				\4 Restricción del dominio de perfiles de preferencia
				\4[] P.ej.: preferencias deben ser unimodales
				\4 Comparaciones interpersonales de utilidad cardinal
				\4[] Teorema de imposibilidad ya no se cumple
			\3 Criterio de Pareto
				\4 Criterio más simple de valoración
				\4[] ¿Posible mejorar a todos los agentes?
				\4 Criterio débil
				\4[] Regulación deseable si:
				\4[] $\to$ Todos los agentes prefieren el cambio
				\4 Criterio fuerte
				\4[] Regulación deseable si:
				\4[] $\to$ Al menos un agente prefiere cambio
				\4[] $\to$ Resto es indiferente
				\4 Problema de incompletitud
				\4[] Imposible comparar en mayoría de casos
				\4[] $\to$ Casi siempre hay alguien perjudicado
				\4[] Muy difícil toma de decisiones
				\4[] $\to$ Sesgo hacia mantenimiento de status quo
				\4[] Necesarios otros criterios de comparación
			\3 Criterios de compensación
				\4 Kaldor
				\4[] Regulación deseable si los que ganan
				\4[] $\to$ Puede compensar a los que pierden
				\4[] $\then$ Para que acepten el cambio
				\4 Hicks
				\4[] Regulación deseable si los que pierden
				\4[] $\to$ No pueden pagar a los que ganan
				\4[] $\then$ Para que no acepten el cambio
				\4 Scitovsky
				\4[] Regulación deseable si:
				\4[] $\to$ Deseable en sentido de Kaldor y de Hicks
				\4 Problema
				\4[] Inconsistencia
				\4[] $\to$ Cambio regulatorio deseable
				\4[] $\to$ Vuelta a estado inicial también deseable
				\4[] $\then$ Criterio inconsistente
				\4[] Intransitividades
				\4[] $\to$ Con criterio de Scitovsky
				\4[] Posibilidad de compensar
				\4[] $\to$ Elemento central de comparación
				\4[] ¿Pero realmente se compensa?
				\4[] $\to$ Compensación tiene costes adicionales
				\4[] $\then$ En la práctica, no se lleva a cabo
			\3 Análisis coste-beneficio\footnote{Ver ``\textit{environmental economics}''.}
				\4 Idea clave
				\4[] Marco conceptual de cuantificación
				\4[] $\to$ Beneficios y costes de regulación
				\4[] Basado en criterios de Kaldor y Hicks
				\4[] $\to$ Mejoras potenciales en sentido de Pareto
				\4[] Regulación deseable cuando
				\4[] $\to$ Beneficios son mayores que costes
				\4[] Instrumento esencial de valoración de pol. medioambientales
				\4[] Utilizado ampliamente en agencias medioambientales
				\4 Beneficios
				\4[] Raramente protección de MAmbiente tiene bfcio. explícito
				\4[] $\to$ No existen mercados para comprar ``beneficios''
				\4[] Necesarios métodos de estimación indirecta
				\4[] $\to$ Preferencias reveladas
				\4[] $\to$ Declaraciones respecto preferencias
				\4[] Medidas habituales
				\4[] -- WTA\footnote{Willingness to accept}--Disposición a aceptar
				\4[] Cuánto exige recibir para aceptar pérdida
				\4[] -- WTP--Disposición a pagar
				\4[] Cuánto acepta pagar un agente por mejorar
				\4[] -- VEV--Valor Estadístico de la Vida
				\4[] Vida humana no tiene valoración de mercado
				\4[] Pol. MAmbiental tiene efectos sobre riesgo de muerte
				\4[] Disposición a pagar/aceptar marginal
				\4[] $\to$ Por variación marginal de riesgo
				\4[] $\to$ $\text{VEV} = \frac{\text{WTP o WTA marginal}}{\text{Cambio marginal en riesgo}}$
				\4 Costes
				\4[] Análisis de equilibrio parcial
				\4[] $\to$ Políticas de efectos limitados
				\4[] Análisis de equilibrio general
				\4[] $\to$ Pol. MAmbiental con efectos generales
				\4[] $\to$ Incluir reutilización de ingresos (p.ej. impuestos)
				\4 Tasa de descuento
				\4[] En qué medida bienestar/costes futuros son relevantes
				\4[] Dos grandes enfoques
				\4[] -- TSPT - Tasa Social de Preferencia Temporal
				\4[] Cómo la sociedad descuenta bienestar futuro/presente
				\4[] $\to$ Necesario agregar tasas de descuento individuales
				\4[] -- TSRI - Tasa Social de Rendimiento de la Inversión
				\4[] Cómo pueden trasladarse intertemporal recursos
				\4 Riesgo
				\4[] Beneficios y costes a menudo inciertos
				\4[] Valorable mediante
				\4[] $\to$ Ponderación de beneficios y costes
				\4[] $\to$ Prima de riesgo en tasa de descuentos
				\4 Consideraciones distributivas
				\4[] Pol. MAmbiental tiene efectos desiguales
				\4[] Introducir ponderaciones en bfcios. y costes
			\3 Implicaciones
				\4 Impacto desigual de política medioambiental
				\4[] Diferente impacto según:
				\4[] $\to$ Renta
				\4[] $\to$ Localización geográfica
				\4[] $\to$ Ubicación temporal
				\4 Economía política
				\4[] Grupos más o menos afectados
				\4[] $\to$ Tratan de maximizar bienestar
				\4[] Aparición de coaliciones
				\4[] $\to$ Influyen en políticas medioambientales
				\4[] $\to$ Afectan proceso político
	\1 \marcar{Bienes públicos globales}
		\2 Idea clave
			\3 Contexto
				\4 Globalización
				\4[] Conjunto de fenómenos
				\4[] Sentido habitual en economía
				\4 Sentido en ciencia económica
				\4[] Libre circulación mundial de
				\4[] $\to$ Bienes y servicios
				\4[] $\to$ Factores de producción
				\4[] $\to$ Tecnologías e ideas
				\4 Determinadas actividades económicas
				\4[] Tienen efectos externos a nivel mundial
				\4[] $\to$ Externalidades a nivel global
				\4 Sistema westfaliano
				\4[] Estados soberanos
				\4[] No puede imponerse comportamiento a otros estados
				\4[] Sólo vía del consentimiento
				\4[] $\to$ Plantea dificultades para proveer bien público
				\4[] $\then$ Análisis económico de provisión del BPúblico
				\4[] $\then$ Incentivos a provisión son punto de partida
				\4[] $\then$ Aparición de marco institucional específico
			\3 Objetivos
				\4 Definir bienes públicos globales
				\4 Caracterizar provisión óptima
				\4 Diseñar mecanismos de provisión óptima
				\4 Implementar mecanismos
			\3 Resultados
				\4 Contexto institucional específico
				\4 Condicionado por ausencia de gobierno supranacional
				\4 Métodos heterogéneos de provisión
				\4 Problemas distributivos relevantes
				\4[] Intra- e inter-generacionalmente
		\2 Análisis teórico
			\3 Características fundamentales
				\4 Externalidades de alcance global
				\4[] Generan efectos sobre conjunto de humanidad
				\4[] $\to$ No necesariamente en igual medida
				\4 Externalidades de stock
				\4[] Impacto depende generalmente de flujos acumulados
				\4[] $\to$ Depreciación relevante
				\4[] $\to$ Tasa de generación relevante
				\4 Provisión ineficiente por mercados privados
				\4[] Free-riding
				\4[] $\to$ Imposible restringir beneficios a quién financia
				\4[] $\then$ Incentivos a no financiar provisión
				\4[] Dilema del prisionero
				\4[] $\to$ Óptimo requiere de cooperación
				\4[] $\to$ Pero incentivos a desviarse unilateralmente
				\4[] $\to$ Estado puede usar coerción para inducir óptimo
				\4[] $\then$ En contexto internacional, coerción no disponible
				\4 Dilema westfaliano
				\4[] Origen en Paz de Westphalia de 1648
				\4[] Principio de derecho internacional
				\4[] $\to$ Estados no se inmiscuyen en asuntos ajenos
				\4[] Intervención coactiva del estado
				\4[] $\to$ Muy difícil en contexto internacional
				\4[] $\then$ Difícil implementación de soluciones
			\3 Bienes públicos globales finales e intermedios
				\4 Finales
				\4[] Son fin en sí mismo
				\4[] Ejemplos:
				\4[] $\to$ Estabilidad climática
				\4[] $\to$ Paz internacional
				\4 Intermedios
				\4[] Permiten provisión de otros BPGlobales finales
				\4[] Ejemplo:
				\4[] $\to$ Estabilidad financiera
				\4[] $\to$ Naciones Unidas
				\4[] $\to$ Acuerdos marco para provisión de bienes
			\3 Ejemplos
				\4 Cambio climático
				\4 Estabilidad financiera
				\4 No proliferación armamentística
				\4 Posicionamiento global
				\4 Epidemias
				\4 Investigación básica
				\4 Espacio como espacio económico
				\4[] Satélites
				\4[] Comunicaciones
				\4[] Investigación
				\4[] ...
			\3 Tecnologías de provisión
				\4 Aditivas
				\4[] BPG depende de suma de contribuciones
				\4[] $Q = \sum_{1,...,n} q_i$
				\4[] Generalización de suma ponderada
				\4[] $Q = \sum_{1,...,n} a_j q_j$
				\4[] Ejemplos:
				\4[] $\to$ Emisiones de gases
				\4[] $\to$ Vertidos de plásticos al océano
				\4 Best-shot
				\4[] BPG depende de mejor provisor
				\4[] $Q = \max \left\lbrace q_1, ..., q_n \right\rbrace$
				\4[] Ejemplos:
				\4[] $\to$ Vacunas contra epidemias
				\4[] $\to$ Sistemas de posicionamiento global
				\4[] $\to$ Investigación básica
				\4 Eslabón más débil
				\4[] BPG depende de peor provisor
				\4[] $Q = \min \left\lbrace q_1, ..., q_n \right\rbrace$
				\4[] Ejemplos:
				\4[] $\to$ Prevención de enfermedades contagiosas
				\4[] $\to$ Sistemas financieros inestables
			\3 Reglas de provisión óptima
				\4 Aditivas
				\4[] Equilibrio competitivo
				\4[] $\to$ Free-riding
				\4[] $\to$ Infraprovisión
				\4[] Necesarios sistemas de incentivo
				\4[] $\to$ Coacción mutua
				\4[] $\to$ Sanciones y similares a free-riders
				\4 Best-shot
				\4[] Contexto más difícilmente solucionable
				\4[] Grandes economías
				\4[] $\to$ Probablemente las que pueden contribuir a menor coste
				\4[] Beneficios muy dispersos o poco relevantes
				\4[] $\to$ Pocos incentivos a dedicar recursos a BPG
				\4[] $\then$ Infraprovisión
				\4[] Más probable provisión eficiente
				\4[] $\to$ Si existe potencia hegemónica
				\4[] $\to$ Multipolaridad dificulta
				\4 Eslabón más débil
				\4[] Incentivos naturales a cooperar
				\4[] $\to$ Partes se ven afectadas si no contribuyen
				\4[] $\then$ Si pueden, cooperarán
				\4[] Equilibrio competitivo puede ser óptimo
				\4[] $\to$ Si partes cuentan con recursos suficientes
			\3 Mecanismos de provisión
				\4 Fondos vinculados al desarrollo
				\4 Contribuciones voluntarias
				\4 Tratados y regímenes internacionales
		\2 Marco institucional
			\3 Antecedentes
				\4 Protocolo de Montreal de 1987
				\4[] Reducción de gases
				\4[] $\to$ Destructores de capa de ozono
				\4 Protocolo de Kyoto de 1997
			\3 Internacional
				\4 Naciones Unidas
				\4[] Asamblea General y Consejo de Seguridad
				\4[] $\to$ Paz mundial
				\4[] Programa de NU para el Medio Ambiente
				\4[] $\to$ 1972
				\4[] $\to$ Coordinación de actividades medioambientales
				\4 IPCC -- Grupo Intergubernamental del Cambio Climático
				\4[] Creado en 1988
				\4[] Marco programa ONU MAmbiente
				\4[] Proveer consenso científico sobre cambio climático
				\4[] Coordinar investigación
				\4[] Estímulo a investigación sobre cambio climático
				\4 UNFCCC\footnote{United Nations Framework Convention on Climate Change.}/CMNUCC -- Convención Marco de NU sobre CClimático
				\4[]  Entrada en vigor en 1994
				\4[] 195 países
				\4[] Protocolo de Kyoto en marco de CM
				\4[] Estabilizar concentraciones de gases
				\4[] Conferencias de las Partes anuales
				\4[] Otras agencias especializadas
				\4[] $\to$ OMS
				\4[] $\to$ ONUSIDA
				\4 Acuerdo de París (2015)
				\4[] Adoptado en COP 21 (2015)
				\4[] Firmasa partir de 2016
				\4[] Dentro de CMNUCC
				\4[] $\to$ XXI Conferencia Internacional/COP21
				\4[] Sustituye a Protocolo de Kyoto desde 2020
				\4[] Jurídicamente vinculante si:
				\4[] $\to$ 55 países ratifican
				\4[] $\to$ Representando 55\% de emisiones globales
				\4[] $\then$ Efectivo a partir de 4 de noviembre de 2016
				\4[] Objetivo
				\4[] $\to$ Limitar calentamiento global < 2º en 2100
				\4[] Actuaciones
				\4[] $\to$ Obligatorio presentar planes de reducción
				\4[] $\to$ Sin mecanismo coactivo
				\4[] $\to$ Objetivos de reducción deben ser incrementales
				\4[] EEUU puede retirarse a partir de noviembre de 2020
				\4[] $\to$ Inmediatamente después de elección presidencial
				\4 Fondo Multilateral
				\4[] Implementación del Acuerdo de Montreal
				\4 Banco Mundial
				\4[] Global Environment Facility
				\4[] Fondo Global contra SIDA, tuberculosis y malaria
				\4 FMI
				\4[] Estabilidad financiera
				\4 FSB -- Consejo de Estabilidad Financiera
				\4 BIS -- Banco Internacional de Pagos
				\4[] Servicios de pagos entre bancos centrales
			\3 Unión Europea
				\4 Paquete de 2020
				\4[] $-20\%$ emisiones de gases de efecto invernadero
				\4[] $20\%$ energías renovables
				\4[] $+20\%$ eficiencia energética
				\4 Objetivos de 2030
				\4[] $-40\%$ emisiones de gases de efecto invernadero
				\4[] $32\%$ de energías renovables
				\4[] $+32.5\%$ de eficiencia energética
				\4 Plan de largo plazo de 2050
				\4[] Neutralidad de emisiones
				\4 ETS -- Emissions Trading System
				\4[] Desde 2005
				\4[] Varios gases de efecto invernadero
				\4[] Sistema cap-and-trade a nivel UE
				\4[] Conectado con otros sistemas cap-and-trade
				\4 Programa SET
				\4[] Investigación básica y energías alternativas
				\4[] $\to$ Bien público de tipo best-shot
			\3 España
				\4 Ministerio para la Transición Ecológica
				\4[$\to$] Secretaría de Estado de Medio Ambiente
				\4[] Subsecretaría para la Transición Ecológica
				\4[] Oficina Española de Cambio Climático
				\4[] Dirección General del Agua
				\4 PNIEC -- Plan Nacional Integrado de Energía y Clima
				\4 PNAEE -- Plan Nacional de Acción de Eficiencia Energética
		\2 Valoración
			\3 Implicaciones de política económica
				\4 Incentivos son importantes
				\4[] Para mitigar y adaptar
				\4 Dilema del prisionero a escala global
				\4[] Economías nacionales
				\4[] $\to$ se benefician de inacción local
				\4[] $\to$ se benefician de acción glboal
				\4[] $\then$ Incentivo a no cooperar
				\4[] $\then$ Incentivo a no proveer BPG
				\4 Sistema westfaliano de soberanía
				\4[] Puede dificultar provisión de BPGlobales
				\4 Economía política
				\4[] Dinámicas de economía política
				\4[] $\to$ Fundamentales a la hora de proveer BPG
				\4[] Sectores económicos
				\4[] $\to$ Muy diversos costes de provisión de BPG
				\4[] Economías importadoras de energías fósiles
				\4[] $\to$ Fuerte presión a favor de transición energética
				\4[] $\to$ Exportadores al contrario
				\4 Necesario ponderar beneficios y costes
				\4[] Especialmente en BPGs con costes
				\4 PEDs sufren mayores costes relativos
			\3 Retos
				\4 Emisiones en PEDs
				\4 Outsourcing de emisiones
				\4 Impacto desigual de fenómenos climáticos
				\4[] Huracanes y similares
				\4[] $\to$ Aumento de efectos muy heterogéneo
				\4[] $\to$ Algunas regiones pueden sufrir menos
				\4[] $\to$ Otras regiones mucho más
	\1[] \marcar{Conclusión}
		\2 Recapitulación
			\3 Análisis económico del medio ambiente
			\3 Bienes públicos globales
		\2 Idea final
			\3 Contabilidad nacional
				\4 PIB y medidas relacionadas
				\4[] Creación de valor añadido
				\4[] $\to$ Sin valoración de efectos sobre stocks
				\4 Sistema de Contabilidad Económica Medioambiental
				\4[] Naciones Unidas (2014)
				\4[] Medir:
				\4[] $\to$ flujos de materiales y energía
				\4[] $\to$ Stocks de activos medioambientales
				\4[] $\to$ Actividades relacionadas con medio ambiente
				\4 PIB neto de depreciación y stocks naturales
				\4[] Medida alternativa a PIN
				\4[] Descontar también
				\4[] $\to$ Impacto sobre medio ambiente
			\3 Desarrollo económico
				\4 Ligado íntimamente a consumo de energía
				\4 Países desarrollados han incurrido en coste
				\4[] PEDs aún no
				\4 Debate sobre reparto justo de emisiones
				\4[] Emisiones pasadas deben tenerse en cuenta?
			\3 Interacción entre economía y ciencia climática
				\4 Previsiones de largo plazo de ciencia climática
				\4[] Dependen en gran medida de predicciones económicas
				\4[] $\to$ ¿Cuánto crecerá PIBpc?
				\4[] $\to$ ¿Cómo evolucionará población?
				\4 Incertidumbre muy elevada en ambos
				\4[] No-linealidad
				\4[] Crítica de Lucas
				\4[] ...
			\3 Papel fundamental del mercado
				\4 Infraprovisión de bienes públicos
				\4[] Resultado habitual
				\4[] Fallo de mercado
				\4 Pero mercado y precios son herramienta esencial
				\4[] Innovación
				\4[] Incentivos a provisión y mitigación
				\4[$\then$] Mercado e intervención fuertemente complementarios
\end{esquemal}























\graficas 

\begin{axis}{4}{Efecto de la introducción de un impuesto pigouviano para corregir el efecto de una externalidad negativa.}{$h$}{$\phi_i(h)$}{impuestopigouviano}
	% utilidad marginal de agente A que induce externalidad
	\draw[-] (0,3.5) -- (4,-1);
	\node[right] at (4,3.5){$-\phi_B'(h)$};
	
	% utilidad marginal de agente B que sufre externalidad
	\draw[dashed] (0,0) to [out=20, in=260](4,4);
	\node[above] at (1.2,2.8){$\phi_A'(h)$};
	
	% equilibrio óptimo
	\draw[dashed] (2.15,0) -- (2.15,1.08);
	\node[below] at (2.14,0){$h^0$};
	
	% impuesto pigouviano
	\draw[-] (0,1.08) -- (4,1.08);
	\node[left] at (0,1.08){$t_h$};
\end{axis}

El gráfico muestra como la introducción del impuesto específico permite inducir al agente que genera la externalidad a consumir el nivel óptimo $h^0$. En este punto, el impuesto reduce la utilidad de A tanto como una unidad de $h$ la aumenta y por ello el agente prefiere no consumir más.


\preguntas

\seccion{Test 2018}

\textbf{38.} De las siguientes afirmaciones, ¿cuál responde mejor al concepto de bien público global?

\begin{itemize}
	\item[a] Son bienes no excluibles, pero son rivales en el consumo; pudiendo tener éste distinto alcance geográfico.
	\item[b] Son bienes excluibles, pero no rivales en el consumo; siendo éste de alcance transnacional.
	\item[c] Son bienes públicos puros (no excluibles y no rivales), cuyo consumo es de alcance universal, intergeneracional y extensible a todos los grupos de población.
	\item[d] Son bienes públicos puros (no excluibles y no rivales), cuyo consumo puede tener distintos alcances geográficos.
\end{itemize}

\seccion{Test 2016}

\textbf{39.} Existe una fuerte interrelación entre el crecimiento económico, el bienestar y el medio ambiente. Cuál de las siguientes afirmaciones es falsa:
\begin{itemize}
	\item[a] Ante el problema de contaminación existente en la ciudad de Madrid, la Alcaldía ha decidido que los martes solo podrán circular en la ciudad coches con un número de matrícula par. Este es un ejemplo de aplicación práctica del Teorema de Coase.
	\item[b] Si introducimos recursos naturales limitados como variables en la función de crecimiento del Modelo de Solow (tierra, recursos energéticos...), el agotamiento de dichos recursos puede suponer un freno al crecimiento económico.
	\item[c] En el caso de los recursos de propiedad común, una posible solución a la sobreexplotación de dichos recursos es entregar la propiedad de los mismos a un único individuo.
	\item[d] El fuego, el número cero y El Quijote son tres ejemplos de bienes públicos globales.
\end{itemize}

\seccion{Test 2011}
\textbf{32.} En el ámbito de la lucha contra el cambio climático, señale la afirmación falsa:
\begin{itemize}
	\item[a] El Panel Intergubernamental de Expertos sobre el Cambio Climático (IPCC) es un órgano científico que no forma parte de la estructura legal de la Convención marco y cuya principal actividad es elaborar informes de evaluación sobre las causas, efectos y opciones de respuesta al cambio climático.
	\item[b] Además de órganos subsidiarios, la Convención Marco puede crear órganos adicionales según sus necesidades, los denominados Grupos de Trabajo \textit{ad hoc}.
	\item[c] El Protocolo de Kyoto de 1997 tiene los mismos principios e instituciones que la Convención Marco, si bien refuerza ésta al establecer también objetivos individuales y jurídicamente vinculantes para todos los países participantes en la Convención.
	\item[d] La máxima autoridad de la Convención Marco es la Conferencia de las Partes, que reúne a representantes de todos los países participantes una vez al año. Las decisiones se toman en base al criterio 1 país = 1 voto.
\end{itemize}

\notas

\textbf{2018:} \textbf{38.} C

\textbf{2016:} \textbf{39.} A

\textbf{2011:} \textbf{32.} C

\bibliografia

Mirar en Palgrave:
\begin{itemize}
	\item bioeconomics
	\item climate change, economics of
	\item common property resources
	\item consumption externalities
	\item contingent valuation
	\item depletion
	\item ecological economics
	\item economic development and the environment
	\item energy economics
	\item energy-GDP relationship
	\item energy services
	\item energy transitions
	\item environmental economics
	\item environmental Kuznets curve
	\item exhaustible resources
	\item experimental methods in environmental economics
	\item fisheries
	\item forests
	\item hedonic prices
	\item natural resources
	\item oil and the macroeconomy
	\item pollution haven hypothesis
	\item pollution permits
	\item public goods
	\item rebound effects
	\item renovable resources
	\item social discount rate
	\item trade and environmental regulations
	\item urban environment and quality of life
	\item value of life
	\item voluntary contribution model of public goods
	\item water resources
\end{itemize}

Barrett, S. (1994) \textit{Strategic environmental policy and international trade} Journal of Public Economics -- En carpeta del tema

Callan, S. J.; Thomas, J. M. (2013) \textit{Environmental Economics \& Management. Theory, Policy, and Applications} 6th Edition. South Western -- En carpeta Medio Ambiente

Copeland, B. R.; Tyalor, M. S. (2004) \textit{Trade, Growth and the Environment} Journal of Economic Literature

Dasgupta, S.; Laplante, B.; Wang, H.; Wheeler, D. (2002) \textit{Confronting the Environmental Kuznets Curve} Journal of Economic Perspectives. Vol. 16. No. 1. Winter 2002 -- En carpeta del tema

Field, B. C.; Field, M. K. (2016) \textit{Environmental Economics. An Introduction} McGraw Hill Education -- En carpeta del tema

Grossman, G. M.; Krueger, A. B. (1991) \textit{Environmental Impacts of a North American Free Trade Agreement} NBER Working Papers -- En carpeta del tema

Grossman, G. M.; Krueger, A. B. (1994) \textit{Economic Growth and the Environment} NBER Working Paper Series -- En carpeta del tema

Harris, J.; Roach, B. (2018) \textit{Environmental and Natural Resource Economics. A Contemporary Approach} 4th Edition. Routeldge -- En carpeta Medio ambiente

Kaul, I.; Grunberg, I.; Stern, M. A. (1999) \textit{Global Public Goods. International Cooperation in the 21st Century} United Nations Development Programme -- En carpeta del tema 

Palmer, K.; Oates, W. E.; Portney, P. R. (1995) \textit{Tightening Environmental Standards: The Benefit-Cost or the No-Cost Paradigm} Journal Economic Perspectives. Vol. 9. Number 4. Fall 1995 -- En carpeta del tema

Perman, R.; Ma, Y.; McGilvray, J.; Common, M. (2003) \textit{Natural Resource and Environmental Economics} 3rd Edition. Pearson -- En carpeta Medio Ambiente

Porter, M. E.; van der Linde, C. (1995) \textit{Toward a New Conception of the Environment-Competitiveness Relationship} Journal of Economic Perspectives. Vol. 9. Number 4. Fall 1995 -- En carpeta del tema

Stern, D. I. (2004) \textit{The Rise and Fall of the Environmental Kuznets Curve} World Development -- En carpeta del tema

Solow, R. M. (1974) \textit{The Economics of Resources or the Resources of Economics} American Economic Review. Vol. 64. No. 2. 

Solow, R. M. (1974) \textit{Intergenerational Equity and Exhaustible Resources} The Review of Economic Studies. Vol. 41.

\end{document}
