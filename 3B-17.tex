\documentclass{nuevotema}

\tema{3B-17}
\titulo{Teoría de la integración monetaria.}

\begin{document}

\ideaclave

PARA Revisión: De Grauwe (2006) -- What have we learnt about Monetary Integration since the Maastricht Treat? -- En carpeta del tema

\seccion{Preguntas clave}
\begin{itemize}
	\item ¿Qué es la integración monetaria?
	\item ¿Qué modelos teóricos modelizan la integración monetaria?
	\item ¿Qué predicen estos modelos?
	\item ¿Qué ventajas teóricas presenta la integración monetaria?
	\item ¿Qué desventajas?
	\item ¿Qué recomendaciones de política económica implica la teoría?
	\item ¿Qué evidencia empírica existe al respecto?
\end{itemize}

\esquemacorto

\begin{esquema}[enumerate]
	\1[] \marcar{Introducción}
		\2 Contextualización
			\3 Macroeconomía
			\3 Economías abiertas
			\3 Integración monetaria
		\2 Objeto
			\3 ¿Qué es la integración monetaria?
			\3 ¿Qué modelos teóricos modelizan la integración monetaria?
			\3 ¿Qué predicen estos modelos?
			\3 ¿Qué ventajas presenta la integración monetaria?
			\3 ¿Qué desventajas?
			\3 ¿Qué implicaciones de política económica se derivan?
			\3 ¿Qué evidencia empírica existe al respecto?
		\2 Estructura
			\3 Teoría de la integración monetaria
			\3 Análisis empírico de la integración
			\3 Implicaciones de política económica
	\1 \marcar{Teoría de la integración monetaria}
		\2 Áreas monetarias óptimas
			\3 Contexto
			\3 Objetivo
			\3 Resultados
		\2 Enfoque de requisitos mínimos
			\3 Idea clave
			\3 Integración financiera -- Ingram (1959)
			\3 Movilidad de factores -- Mundell (1961)
			\3 Apertura de la economía -- McKinnon (1963)
			\3 Diversificación -- Kenen (1969)
			\3 Tendencias inflacionarias -- Magnifico (1971)
			\3 Tipo de cambio real -- Vaubel (1976)
		\2 Enfoque coste-beneficio
			\3 Idea clave
			\3 Beneficios
			\3 Costes
		\2 Enfoque moderno: reglas, credibilidad y shocks
			\3 Idea clave
			\3 Reglas vs discrecionalidad
			\3 Modelos DSGE
			\3 Relaciones ancla-cliente
			\3 Nuevo trilema del sector financiero en área monetaria
		\2 Factores que determinan optimalidad de integración
			\3 Idea clave
			\3[\textsc{i}] Tamaño y apertura
			\3[\textsc{ii}] Socio comercial
			\3[\textsc{iii}] Simetría de los shocks
			\3[\textsc{iv}] Movilidad del trabajo
			\3[\textsc{v}] Transferencias fiscales contracíclicas
			\3[\textsc{vi}] Transferencias corrientes contracíclicas
			\3[\textsc{vii}] Voluntad política
		\2 Conclusiones generales del análisis teórico
			\3 Idea clave
			\3 Simetría y flexibilidad
			\3 Representación gráfica
	\1 \marcar{Análisis empírico de la integración monetaria}
		\2 Efectos sobre el comercio
			\3 Idea clave
			\3 Rose (2000)
			\3 Glick y Rose (2001)
			\3 Glick y Rose (2016)
		\2 Efectos sobre el ciclo económico
			\3 Bayoumi y Eichengreen (1993)
			\3 Krugman (1993) y (2001)
			\3 Frankel y Rose (1998)
			\3 Rose (2008)
	\1 \marcar{Implicaciones de política económica}
		\2 Política fiscal
			\3 Idea clave
			\3 Federalismo fiscal
			\3 Sachs y Sala-i-Martin (1992)
		\2 Criterios de convergencia
			\3 Idea clave
			\3 Eichengreen y von Hagen (1996)
			\3 Justificaciones de criterios de convergencia
			\3 Críticas a criterios de convergencia
			\3 Trilema de la Unión Monetaria de Pisany-Ferry
		\2 Economía política
			\3 Idea clave
			\3 Trilema de Rodrik
			\3 Implicaciones
	\1[] \marcar{Conclusión}
		\2 Recapitulación
			\3 Teoría de la integración
			\3 Análisis empírico
			\3 Implicaciones de política económica
		\2 Idea final
			\3 Efecto sobre terceros
			\3 Soberanía política y monetaria
			\3 Tendencia de l/p hacia integración monetaria

\end{esquema}

\esquemalargo

\begin{esquemal}
	\1[] \marcar{Introducción}
		\2 Contextualización
			\3 Macroeconomía
				\4 Análisis de fenómenos económicos a gran escala
				\4 Énfasis sobre variables agregadas
			\3 Economías abiertas
				\4 Comercio internacional
				\4[] Intercambian ByS con otras economías
				\4[] $\to$ Precios relativos son importantes
				\4[] $\then$ Tipo de cambio es importante
				\4[] $\then$ DAgregada depende de exterior
				\4 Flujos financieros internacionales
				\4[] Intercambio de activos y pasivos
				\4[] Suavización intertemporal de rentas
				\4[] Dinámicas de deuda exterior
				\4 Tipo de cambio
				\4[] Precio más importante en una economía abierta
				\4[] $\to$ Relación entre bienes locales y extranjeros
			\3 Integración monetaria
				\4 Fenómeno empíricamente observado
				\4 Economías abandonan moneda propia
				\4[] $\to$ Política monetaria única
				\4[] $\to$ Instituciones monetarias comunes
				\4[] $\to$ Adoptan moneda común
				\4 Diferentes grados de integración
				\4[] Unión monetaria
				\4[] $\to$ Caso extremo de integración monetaria
				\4[] Dolarización
				\4[] $\to$ Implica mantener institución monetaria propia
				\4[] $\then$ Provee divisa a IFs nacionales
				\4[] Junta de conversión
				\4[] $\to$ Se mantiene moneda propia con pleno respaldo
				\4[] Coordinación de políticas monetarias
				\4[] $\to$ Forma más débil de integración monetaria
				\4 Ejemplo más importante:
				\4[] Euro
				\4 Otros:
				\4[] Franco CFA
				\4[] Dólar américano
				\4[] $\to$ USA, ECU, SALVADOR, PANAMA, P. RICO...
				\4[] Dólar del Este del Caribe
				\4[] Proyectos de UM de bloques comerciales
		\2 Objeto
			\3 ¿Qué es la integración monetaria?
			\3 ¿Qué modelos teóricos modelizan la integración monetaria?
			\3 ¿Qué predicen estos modelos?
			\3 ¿Qué ventajas presenta la integración monetaria?
			\3 ¿Qué desventajas?
			\3 ¿Qué implicaciones de política económica se derivan?
			\3 ¿Qué evidencia empírica existe al respecto?
		\2 Estructura
			\3 Teoría de la integración monetaria
			\3 Análisis empírico de la integración
			\3 Implicaciones de política económica
	\1 \marcar{Teoría de la integración monetaria}
		\2 Áreas monetarias óptimas
			\3 Contexto
				\4 Debate sobre régimen TCFijo o Flexible
				\4[] Nurske (1944):
				\4[] $\to$ TCFlexible culpable de años 30
				\4[] Meade (1951):
				\4[] $\to$ TCFijos ajustables son mejor opción
				\4[] Friedman (1953):
				\4[] $\to$ TCFlexible permite ajuste+regla $\Delta M$
				\4[] Comunidad Económica Europea
				\4[] $\to$ En proceso de construcción de UAduanera
				\4[] Debate sobre optimalidad de moneda única europea
				\4[] $\to$ Meade vs Scitovsky\footnote{Meade defendía que no se daban las condiciones para que una moneda única en el mercado europeo fuese óptima. Scitovsky defendía la optimalidad con el argumento de que incentivaría los flujos de capital.}
				\4 Territorios coloniales y descolonización
				\4[] $\to$ Áreas monetarias desaparecen
				\4[] $\to$ Fijación de nuevas monedas respecto a otras
				\4 Bretton Woods
				\4[] $\to$ Funcionamiento adecuado por el momento
				\4[] $\to$ Algunas señales de inquietud
			\3 Objetivo
				\4 Caracterizar optimalidad de áreas monetarias
				\4[] Concepto de área monetaria:
				\4[] $\to$ ``territorio en el que el tipo de cambio es fijo''
				\4[] Formular marco de análisis para responder
				\4[] $\to$ Qué requisitos necesarios para optimalidad
				\4[] $\to$ Qué áreas para cada moneda
			\3 Resultados
				\4 Enmarca debate en décadas posteriores
				\4 Varios enfoques dentro de Área Monetaria Óptima
				\4[] $\to$ Requisitos mínimos
				\4[] $\to$ Coste beneficio
				\4[] $\to$ Enfoque moderno (shocks, credibilidad, otros)
		\2 Enfoque de requisitos mínimos
			\3 Idea clave
				\4 Debe cumplirse un factor en área propuesta
				\4[] $\to$ Para que sea óptima
			\3 Integración financiera -- Ingram (1959)
				\4 Movilidad de capital entre regiones
				\4[] $\to$ Flujos financieros equilibran balanza de pagos
				\4[] $\to$ Reducen necesidad de $\Delta$ TC
				\4[] $\then$ Hacen óptima el área monetaria
			\3 Movilidad de factores -- Mundell (1961)
				\4 Dado shock asimétrico de demanda que provoca:
				\4[] $\to$ Inflación en una región, superávit CC
				\4[] $\to$ Desempleo en otra, déficit CC
				\4[] $\then$ PM expansiva aumenta inflación
				\4[] $\then$ PM contractiva aumenta desempleo
				\4[] $\then$ Una misma PM no puede estabilizar ambas
				\4 Movilidad de trabajo
				\4[] $\to$ De región que crece a región en crisis
				\4[] $\then$ Cae desempleo en región con menos demanda
				\4[] $\then$ Cae inflación en región con más demanda
				\4[] $\then$ TC flexible y PM diferenciada no necesarias
			\3 Apertura de la economía -- McKinnon (1963)
				\4 Economía cerrada:
				\4[] $\to$ Pocos comerciables en relación a no comerciables
				\4 Volatilidad de TC en economía poco abierta
				\4[] $\to$ Aumenta competitividad del sector exportables
				\4[] $\then$ Reasignación de recursos a sector exportador
				\4[] $\then$ Pocos costes de reasignación porque export. es pequeño
				\4[] $\then$ Aceptable que no haya integración
				\4 Economías abiertas sí aprovechan integración monetaria
				\4[] $\to$ Ratio de comerciables/no comerciables alto
				\4 Volatilidad de TC en economía muy abierta
				\4[] $\to$ Reasignación de sector comerciable a no comerciable
				\4[] $\then$ Costes elevados de reasignación
				\4[] $\then$ Costes de transacción elevados
				\4[] $\then$ Preferible mantener TC constante
				\4 Economía abierta con déficit de CC
				\4[] $\to$ Políticas de ESwitching y EChanging menos costosas que $\Delta$ TC
			\3 Diversificación -- Kenen (1969)
				\4 Diversificación hace óptima unión monetaria
				\4 Ecs. muy diversificadas ante shock de dda. sectorial
				\4[] $\to$ Otros sectores compensan
				\4[] $\to$ Demanda agregada constante
				\4[] $\then$ Ajuste vía TCN menos necesario
				\4[] $\then$ AMonetaria más óptima
			\3 Tendencias inflacionarias -- Magnifico (1971)
				\4[] Países tienen tendencias estructurales de inflación
				\4[] $\to$ Resultado de distintas prefs. inflación-empleo
				\4 Misma PM para distintas preferencias
				\4[] $\to$ No se adapta a ninguno
				\4[] $\then$ Ambos países pagan coste
				\4 No se trata de mantener misma tasa de inflación
				\4[] $\to$ Sino de homogeneizar estructuras de países
				\4[] $\then$ Tasas de inflación coherentes a l/p
			\3 Tipo de cambio real -- Vaubel (1976)
				\4 Capacidad para variar TCReal
				\4[] $\to$ Determina optimalidad
				\4 Relación entre:
				\4[] $\to$ Precios de bienes exportación e importación
				\4[] $\to$ Activos financieros domésticos y extranjeros
				\4[] $\then$ Determinan equilibrio de balanza de pagos
				\4 TCN es factor en ambos
				\4[] $\to$ Puede suplirse con movimientos de precios
				\4[] $\then$ Lo cual es difícil si implica deflación en un área
		\2 Enfoque coste-beneficio
			\3 Idea clave
				\4[] Una característica determinada no es suficiente
				\4[] $\to$ Existen beneficios y costes
				\4[] $\then$ Necesario ponderar importancia de ambos
				\4[] Requisitos anteriores se cumplen en grados variables
				\4[] $\to$ Ninguno nada ni completamente
			\3 Beneficios
				\4[\textsc{i}] Desaparición de flujos especulativos
				\4[] Flujos basados en variaciones esperadas de TCN
				\4[] $\to$ No tienen sentido si TCFijo irrevocable
				\4[] Depende de credibilidad de integración monetaria
				\4[] $\to$ Aparecen flujos si se espera desintegración
				\4[\textsc{ii}] Mejoras de eficiencia
				\4[] Desaparece necesidad de cubrirse ante $\Delta$ esperadas
				\4[] Se evitan costes:
				\4[] $\to$ Incertidumbre y volatilidad
				\4[] $\to$ Compra de forwards y otros derivados
				\4[] $\then$ Mejor asignación de recursos
				\4[\textsc{iii}] Eliminación de costes de transacción
				\4[] Spreads en pares diferentes a principales
				\4[] $\to$ Pueden llegar a ser muy altos
				\4[\textsc{iv}] Estabilidad de precios
				\4[] Si integración con principales socios
				\4[] $\to$ Comerciables no afectados por $\Delta$ TCN
				\4[] $\then$ Una fuente menos de volatilidad de precios
				\4[\textsc{v}] Estímulo a integración política y económica
				\4[] Necesidad de coordinar y sincronizar ciclos
				\4[] $\to$ Estímulo a armonización regulatoria
				\4[] $\to$ Incentivos a políticas comunes
				\4[\textsc{vi}] Moneda única como reserva internacional
				\4[] Si:
				\4[] $\to$ Miembros de UM tienen suficiente tamaño
				\4[] $\to$ Credibilidad suficiente de moneda única
				\4[] $\then$ Moneda única aceptada entre terceros
				\4[] $\then$ Miembros UM emiten pasivos en moneda propia
				\4[] $\then$ Cae coste de financiarse en resto del mundo
				\4[] $\then$ Aumenta poder de negociación internacional
				\4[\textsc{vii}] Reducción de reservas de divisas
				\4[] Países que se integran
				\4[] $\to$ No necesitan reservas de divisas respectivas
				\4[] $\then$ Costes de oportunidad de reservas caen
				\4[] $\then$ Menores costes de gestión
				\4[] $\then$ Ahorro puede mantenerse en área monetaria
			\3 Costes
				\4[\textsc{i}] Pérdida de autonomía de política monetaria
				\4[] Libre movilidad de capital
				\4[] $\to$ PM supeditada a mantener TC
				\4[] $\then$ PM autónoma impotente
				\4[] Integración completa
				\4[] $\to$ PM autónoma imposible
				\4[] Si salarios, prod., precios
				\4[] $\to$ Evolucionan de manera divergente
				\4[] $\then$ Desequilibrio externo e interno
				\4[\textsc{ii}] Sin uso discrecional de señoreaje
				\4[] Poco relevante en economías desarrolladas
				\4[] $\to$ Menor tendencia a uso de señoreaje
				\4[] Puede permitir salida de endeudamiento excesivo
				\4[\textsc{iii}] Divergencia en preferencias inflación-desempleo
				\4[] Una sola política monetaria para todos
				\4[] Posiblemente, diferentes preferencias $\pi$-$u$
				\4[] $\then$ Ambos pagan coste de bienestar
				\4[\textsc{iv}] Desestabilización de desequilibrios regionales
				\4[] Capital más móvil que trabajo
				\4[] Fluye hacia regiones más productivas
				\4[] $\to$ Cambio en dotación de capital por trabajador
				\4[] $\then$ Dinámicas de desempleo y migración
				\4[\textsc{v}] Restricciones de política fiscal
				\4[] Si:
				\4[] -- No hay transferencias fiscales incondicionales
				\4[] -- Capital fluye donde interés más alto
				\4[] $\to$ Necesario pagar interés y principal
				\4[] $\to$ Aumenta interés en economía receptora de K
				\4[] $\to$ Cae margen de política fiscal
				\4[] $\to$ Cae margen de actuación de política monetaria
				\4[] $\then$ Posible aparición de endeudamiento excesivo
				\4[] $\then$ Externalidades negativas entre países vía interés
				\4[] $\then$ Necesaria implementación de reglas fiscales
		\2 Enfoque moderno: reglas, credibilidad y shocks
			\3 Idea clave
				\4 Aplicación de avances metodológicos
				\4[] $\to$ A análisis de coste-beneficio anterior
			\3 Reglas vs discrecionalidad
				\4 Kydland y Prescott (1977), Barro y Gordon (1983)...
				\4[] PM discrecional no es efectiva ni deseable
				\4[] $\to$ Pérdida de autonomía PM es menos relevante
				\4 Expectativas y credibilidad
				\4[] Reputación de BC manteniendo inflación baja
				\4[] $\to$ Permite reducir coste de mantener inflación baja
				\4 Asumiendo que inflación baja es deseable
				\4[] $\to$ TCFijo más fácil de entender que inflación
				\4[] $\to$ TCFijo operativamente más sencillo que inflación
				\4[] $\then$ Integración monetaria mejora reputación BC
				\4[] $\then$ Menor coste de utilizar PM en crisis
				\4 Credibilidad de política fiscal
				\4[] En contexto de unión monetaria
				\4[] $\to$ ¿Es creíble BCentral no prestamista último recurso?
				\4[] $\to$ Reglas de no bailout son creíbles?
				\4[] Si no son creíbles:
				\4[] $\to$ Implementar pactos de responsabilidad fiscal
				\4[] $\then$ Evitar sobreendeudamiento esperando rescate
				\4[] $\then$ Problema de zona euro
			\3 Modelos DSGE
				\4[] Cuantificación de efectos de integración
				\4[] $\to$ Respuesta a shocks de distintos tipos
				\4[] $\to$ Efectos de integración sobre comercio
			\3 Relaciones ancla-cliente
				\4 Alesina y Barro (2002)
				\4 Análisis de relaciones clientelares entre monedas
				\4 Países pequeños y abiertos
				\4[] $\to$ Con un socio comercial grande y principal
				\4[] $\to$ Historial de inflación elevada
				\4[] $\to$ Ciclo correlacionado con socio comercial
				\4[] $\then$ Se beneficia de relación ancla-cliente
				\4 Dado:
				\4[] $\to$ Aumento de países en s. XX
				\4[] $\to$ Menor tamaño medio
				\4[] $\to$ Más motivos para buscar anclas
				\4[] $\then$ Esperable mayor número de UM
			\3 Nuevo trilema del sector financiero en área monetaria
				\4 Planteado por Pisany-Ferry (2012)
				\4 Asumiendo:
				\4[] Régimen de tipo fijo factible
				\4[] $\to$ En marco Mundell-Fleming
				\4[] Integración monetaria avanzada
				\4 Sólo son posibles dos:
				\4[] \textsc{I} Interdependencia banca-sector público nacional
				\4[] \textsc{II} Financiación monetaria del déficit prohibida
				\4[] \textsc{III} Sin corresponsabilidad respecto a deuda pública\footnote{Es decir, sin posibilidad de que miembros del área monetaria rescaten el sector público de otro miembro.}
				\4 Con I y II
				\4[] Necesaria unión fiscal
				\4 Con I y III
				\4[] Necesario prestamista de último recurso a nivel de AMonetaria
				\4 Con II y III
				\4[] Necesaria unión financiera efectiva
		\2 Factores que determinan optimalidad de integración\footnote{Basado en apartado 28.5, Handbook of Fixed Exchange Rates, capítulo por Frankel, J.}
			\3 Idea clave
				\4 No existe régimen perfecto en general
				\4 Contexto determina régimen óptimo
				\4[] Coyuntural
				\4[] Idiosincrático de la economía en cuestión
			\3[\textsc{i}] Tamaño y apertura
				\4 Países pequeños y abiertos
				\4[] Más beneficios del comercio
				\4[] $\to$ Mayores ventajas de integración monetaria
				\4 Países grandes y cerrados
				\4[] Menos beneficios del comercio
				\4[] Más margen de actuación de PM
				\4[] $\to$ Más ventajas de independencia monetaria
			\3[\textsc{ii}] Socio comercial
				\4 Existencia de un socio muy grande
				\4[] Con el que lleva a cabo elevado \%:
				\4[] $\to$ Comercio
				\4[] $\to$ Inversión
				\4[] $\to$ Perspectivas de relaciones futuras
				\4[] $\then$ TCFijo ofrece más ventajas
				\4 Fijación respecto a cesta
				\4[] Cuando hay varios socios dominantes
				\4 Países con economías muy diversificadas
				\4[] TCFlexible puede ser más estable que fijo
				\4[] $\to$ Sin inconvenientes de tipo fijo
			\3[\textsc{iii}] Simetría de los shocks
				\4 Correlación alta
				\4[] Shocks que afectan a país emisor de moneda ancla
				\4[] $\to$ Muy correlacionados con shocks domésticos
				\4[] $\to$ Movimiento de interés tenderá a ser similar
				\4[] $\then$ TCFijo es ventajoso
			\3[\textsc{iv}] Movilidad del trabajo
				\4 Mayor movilidad del trabajo
				\4[] Más flexibilidad ante shocks asimétricos
				\4[] $\to$ TCFijo menos vulnerable ante shocks
				\4[] $\then$ Menos razones para TCFlexible
			\3[\textsc{v}] Transferencias fiscales contracíclicas
				\4 Transferencias fiscales
				\4[] Especialmente relevante en UM
				\4[] También receptores de transferencias de K
				\4[] Mayores transferencias:
				\4[] $\to$ Más resistencia a shocks
				\4[] $\then$ TCFlexible menos ventajoso que sin trans.
			\3[\textsc{vi}] Transferencias corrientes contracíclicas
				\4 Especialmente, remesas
				\4 Países con poblaciones emigrantes en desarrollados
				\4[] Remesas contracíclicas a PEDs
				\4[] Responden a diferencial de posiciones cíclicas
				\4[] $\to$ Suavizan shocks domésticos en PEDs
				\4[] $\then$ TCFlexible menos ventajoso que sin trans.
			\3[\textsc{vii}] Voluntad política
				\4 Importancia de soberanía monetaria
				\4[] Sujeto de debate político
				\4[] Soberanía nacional asociada a moneda propia
				\4[] $\to$ Puede ser difícil abandonar moneda propia/TCFijo
		\2 Conclusiones generales del análisis teórico
			\3 Idea clave
				\4 Factores anteriores reducibles a tres
				\4[I] Simetría
				\4[II] Integración
				\4[III] Flexibilidad
			\3 Simetría y flexibilidad
				\4 Shocks muy asimétricos + ecs. muy flexibles
				\4[] Posible unión monetaria
				\4[] Precios y ff.pp. se ajustan a shocks
				\4 Shocks simétricos y ecs. poco flexibles
				\4[] Una misma PM es suficiente para todas
			\3 Representación gráfica
				\4 \grafica{simetriaintegracionflexibilidad}
	\1 \marcar{Análisis empírico de la integración monetaria}
		\2 Efectos sobre el comercio
			\3 Idea clave
				\4 Generalmente, modelos de gravedad
				\4[] Comercio bilateral depende de:
				\4[] $\to$ Tamaño respectivo
				\4[] $\to$ Distancia
				\4[] $\to$ Variable dummy: integración monetaria o no
				\4 Resultados
				\4[] Generalmente, sorprendentemente largos para UM
				\4[] $\to$ TCFijo muestra poco efecto
				\4[] $\to$ Relación débil entre volatilidad TC y comercio
				\4[] Pero discontinuidad si misma moneda
				\4[] $\to$ Relación con efectos frontera
				\4[] $\then$ Unión monetaria actúa vía $\downarrow$ efecto frontera
			\3 Rose (2000)
				\4 Trabajo seminal y enorme impacto
				\4 Contexto
				\4[] Debates relacionados sobre:
				\4[] $\to$ Introducción de moneda única
				\4[] $\to$ Mercado único
				\4[] $\to$ Regímenes cambiarios tras crisis de los 90s
				\4[] Modelo de gravedad
				\4 Objetivo
				\4[] Cuantificar efectos de:
				\4[] $\to$ Volatilidad del tipo de cambio
				\4[] $\to$ Introducción de moneda única
				\4[] ...sobre comercio entre países integrados
				\4[] ...con datos de sección cruzada
				\4 Resultados
				\4[] Efecto relativo de estabilidad cambiaria
				\4[] $\to$ Tres veces más alto si comparten moneda
				\4[] Posibles explicaciones
				\4[] $\to$ Integración financiera
				\4[] $\to$ Cobertura de riesgo cambiario cuesta más que esperado
			\3 Glick y Rose (2001)
				\4 Análisis de datos de panel
				\4[] Shocks: entrada o salida de UM
				\4 Contexto
				\4[] Impacto causado por Rose (2000)
				\4[] Euro ya en vigor
				\4[] $\to$ A punto de entrar en circulación
				\4 Objetivo
				\4[] Cuantificar efectos de unión monetaria
				\4[] $\to$ Asumiendo simetría de entradas y salidas
				\4[] ...sobre comercio entre países integrados
				\4[] ...con datos de panel
				\4[] ...utilizando 50 años de datos pre-1998
				\4 Resultados
				\4[] Entrada/salida implica comercio se dobla
				\4[] Aparentemente robusto y significativo
			\3 Glick y Rose (2016)
				\4 Revisión de paper de 15 años antes
				\4 Contexto
				\4[] UEM pleno rendimiento
				\4[] Nuevos datos
				\4[] Debate sobre efectos de EMU
				\4[] $\to$ ¿Ha aumentado comercio?
				\4 Objetivo
				\4[] Revisar predicciones de paper Glick y Rose (2001)
				\4[] Estimar efecto de UM sobre comercio y exportaciones
				\4[] $\to$ Utilizando datos de UEM
				\4 Resultados
				\4[] i. Simetría de entrada/salida parece razonable
				\4[] ii. EMU ha aumentado exportaciones en 50\%
				\4[] iii. Diferentes UM tienen diferentes efectos
		\2 Efectos sobre el ciclo económico
			\3 Bayoumi y Eichengreen (1993)
				\4 Contexto
				\4[] Mercado único en vigor
				\4[] Debate sobre futura integración monetaria
				\4[] $\to$ ¿Funcionará bien?
				\4[] $\to$ ¿Cómo se compara con otras UM?
				\4 Objetivo
				\4[] Descomponer shocks regionales en EU y USA
				\4[] $\to$ Shocks de oferta y demanda
				\4[] Valorar relación entre shocks
				\4[] $\to$ ¿Más o menos correlados?
				\4[] $\to$ ¿Más o menos idiosincráticos?
				\4[] Datos de 11 EEMM europeos
				\4 Resultados
				\4[] Shocks más idiosincráticos en UE que USA
				\4[] En UE hay centro y periferia
				\4[] $\to$ Magnitud y cohesión similar UE en centro y USA
				\4[] Respuesta a shocks
				\4[] $\to$ Menor en UE que USA
				\4[] $\to$ Posiblemente por menor movilidad de ff.pp.
			\3 Krugman (1993) y (2001)
				\4 Integración y dinámicas de especialización
				\4 Contexto
				\4[] Similar a Rose (2000) y Glick y Rose (2001)
				\4[] Incertidumbre ante efectos de UEM
				\4[] Optimismo generalizado
				\4[] Poco análisis de especialización post-integración
				\4 Objetivo
				\4[] Valorar efectos de integración europea sobre:
				\4[] $\to$ Especialización regional
				\4[] $\to$ Correlación de shocks
				\4[] ...utilizando estados EEUU como referencia
				\4 Resultado
				\4[] UM aumenta:
				\4[] i. Comercio en general
				\4[] ii. Especialización regional
				\4[] Si i>ii y comercio intraindustrial aumenta mucho
				\4[] $\to$ Shocks más correlacionados
				\4[] Si especialización regional más importante
				\4[] $\to$ Menos correlación de shocks
				\4[] $\to$ UM elimina PM
				\4[] $\to$ Shocks regionales más idiosincráticos en UE que USA
				\4[] $\to$ Movimiento de ff.pp. menor en UE que USA
				\4[] $\to$ Transferencias fiscales menor en UE que USA
				\4[] $\then$ Política fiscal regional impotente tras integración
				\4[] $\then$ Integración puede complicar policy-making
				\4[] $\then$ Dinámicas inestables tras integración
				\4[] $\then$ Posible aumento desempleo en regiones afectadas
			\3 Frankel y Rose (1998)
				\4 Contexto
				\4[] Mercado único en vigor
				\4[] Debate sobre futura integración monetaria
				\4[] $\to$ ¿Funcionará bien?
				\4[] $\to$ ¿Cómo se compara con otras uniones monetarias?
				\4[] Criterios de OCAs ex-ante a integración
				\4[] $\to$ Pueden inducir conclusiones erróneas
				\4[] $\then$ Criterios pueden ser endógenos
				\4 Objetivo
				\4[] Verificar hipótesis:
				\4[] $\to$ Sincronización más fuerte que especialización
				\4[] $\to$ Ciclos similares cuanto más comercio
				\4[] $\then$ Optimalidad de integración es endógena
				\4[] Datos de panel 30 años y 20 países
				\4 Resultados
				\4[] Relación positiva entre:
				\4[] $\to$ Comercio bilateral
				\4[] $\to$ Correlación de ciclo económico
				\4[] $\then$ Entrada en UEM aumenta prob. de optimalidad
				\4[] $\then$ Más probable satisfacción ex-post que ex-ante
			\3 Rose (2008)
				\4 Meta-análisis sobre efectos EMU
				\4[] Comercio y sincronización de ciclo
				\4 Contexto
				\4[] Primeros años de EMU
				\4[] Pre-crisis financiera
				\4[] Endogeneidad de las condiciones para OCA
				\4[] $\to$ Planteada por algunos autores
				\4[] Modelos del ciclo más desarrollados
				\4 Objetivo
				\4[] Recopilar y valorar estudios sobre efectos en:
				\4[] $\to$ Comercio
				\4[] $\to$ Ciclo de negocios
				\4 Resultados
				\4[] Aumento del comercio tras integración
				\4[] Ciclos más sincronizados con más comercio
				\4[] $\then$ UEM como OCA es proceso endógeno
				\4[] $\then$ UEM como ciclo virtuoso
				\4[] $\then$ Más sincronización, menos necesidad de PM autónoma
				\4[] $\then$ Frankel y Rose (1997) y (1998) parecen confirmarse
	\1 \marcar{Implicaciones de política económica}
		\2 Política fiscal
			\3 Idea clave
				\4 Integración monetaria implica renunciar a PM
				\4[] Salvo país central, si lo hay
				\4 Política fiscal aumenta importancia
				\4[] En ausencia de criterios para OCA
				\4[] $\to$ Política fiscal es elemento estabilizador
			\3 Federalismo fiscal
				\4 Concepto
				\4[] Análisis de la estructura vertical
				\4[] $\to$ Del sistema fiscal del estado
				\4 Importancia en integración monetaria
				\4[] Sin PM como herramienta estabilizadora
				\4[] $\to$ ¿Cómo organizar estructura fiscal?
				\4[] $\to$ ¿Qué nivel de gobierno grava qué?
				\4[] $\to$ ¿Cuánto puede gastar cada nivel de gobierno?
				\4[] $\to$ ¿Qué función debe tener el gobierno central?
				\4 Implicación general
				\4[] Papel estabilizador de UM debe aumentar
				\4[] Sustituye a gobierno central
			\3 Sachs y Sala-i-Martin (1992)
				\4 Contexto
				\4[] Mercado único en vigor
				\4[] Debate sobre futura integración monetaria
				\4[] $\to$ ¿Funcionará bien?
				\4[] $\to$ ¿Cómo se compara con otras uniones monetarias?
				\4[] Estados Unidos y Unión Europea
				\4[] $\to$ Muy diferentes grados de federalismo fiscal
				\4 Objetivo
				\4[] Comparar absorción fiscal de shocks
				\4[] $\to$ En USA y UE
				\4[] $\then$ Estimación cuantitativa de ambas
				\4 Resultados
				\4[] Federalismo fiscal en USA reduce 30\% de shocks
				\4[] Federalismo fiscal en UE reduce 0.5\% de shocks
				\4[] $\to$ Sin tener en cuenta transferencias one-off
				\4[] Si reducción de impacto de shocks
				\4[] $\to$ es relevante para funcionamiento de UM
				\4[] $\then$ UEM tendrá graves problemas de estabilidad
				\4[] $\then$ Necesario aumentar federalismo fiscal en UE
		\2 Criterios de convergencia
			\3 Idea clave
				\4 Restricciones política fiscal en UM
				\4[] $\to$ Déficit
				\4[] $\to$ Deuda pública
				\4 Objetivos nominales
				\4[] $\to$ Precios
				\4[] $\to$ Inflación
				\4 Enorme literatura ha analizado criterios
				\4 Debate:
				\4[] ¿Deben existir en toda UM?
				\4[] ¿Deben ser flexibles?
				\4[] ¿Deben cumplirse estrictamente?
				\4[] ¿En qué niveles deben situarse?
			\3 Eichengreen y von Hagen (1996)
				\4 Muchas UM no tienen criterios a nivel UM
				\4[] Euro es caso poco frecuente
				\4[] En muchas UM, restricciones a nivel nacional
				\4[] $\to$ Fáciles de superar
				\4[] En EEUU, todos menos Vermont
				\4[] $\to$ Exigen presupuesto equilibrado a nivel estatal
			\3 Justificaciones de criterios de convergencia
				\4 Estabilidad monetaria
				\4[] Reducir sesgo inflacionario
				\4[] Reducir diferenciales de TCReal
				\4[] Mantener competitividad relativa
				\4[] Mitigar riesgo moral
				\4 Estabilidad fiscal
				\4[] Evitar crecimiento excesivo de deuda
				\4[] $\to$ Aumento explosivo de carga financiera
				\4[] $\to$ Presión sobre necesidad de financiación
				\4[] Evitar déficit excesivo
				\4[] $\to$ Reducir presión sobre banco central
				\4[] $\to$ Reducir externalidades negativas sobre otros miembros
				\4[] $\to$ Menor probabilidad de default
				\4[] $\to$ Reducir probabilidad de desintegración
			\3 Críticas a criterios de convergencia
				\4 Pre-crisis
				\4[] Deuda en senda descendente
				\4[] Criterios mucho más flexibles
				\4 Actualidad
				\4[] Crecimiento bajo
				\4[] Deuda estancada o creciente
				\4[] Criterios más rígidos
				\4[$\then$] Criterios de convergencia no cumplen objetivo
				\4[$\then$] Causa de crisis no fue indisciplina fiscal
				\4[$\then$] Sector privado es importante en estabilidad de UM
				\4[$\then$] Sistema financiero es importante en UEM
			\3 Trilema de la Unión Monetaria de Pisany-Ferry
				\4 Planteado por Pisany-Ferry (2012)
				\4 Asumiendo:
				\4[] Régimen de tipo fijo factible
				\4[] $\to$ En marco Mundell-Fleming
				\4[] Integración monetaria avanzada
				\4 Sólo son posibles dos:
				\4[] \textsc{I} Interdependencia banca-sector público
				\4[] \textsc{II} Financiación monetaria del déficit prohibida
				\4[] \textsc{III} Sin corresponsabilidad respecto a deuda pública\footnote{Es decir, sin posibilidad de que miembros del área monetaria rescaten el sector público de otro miembro.}
				\4 Con I y II
				\4[] Necesaria unión fiscal
				\4[] $\to$ En ausencia, necesarios criterios de convergencia
				\4[] Introducción de III
				\4[] Crisis bancaria induce respuesta nacional
				\4[] $\to$ Tensión sobre cuentas públicas
				\4[] $\to$ Sin acceso a prestamista de último recurso
				\4[] $\to$ Sin transferencias fiscales
				\4[] $\then$ Quiebra de sistema bancario y hacienda pública
				\4[] $\then$ Necesaria unión fiscal y transferencias
				\4 Con I y III
				\4[] Necesario prestamista de último recurso a nivel de AMonetaria
				\4[] $\to$ Fin. mon. de déficit compatible con convergencia?
				\4[] Introducción de II
				\4[] $\to$ Imposible provisión de liquidez vía BC
				\4[] $\then$ Quiebra de sistema bancario y hacienda pública
				\4[] $\then$ Necesario prestamista de último recurso a nivel de AMonetaria
				\4 Con II y III
				\4[] Necesaria unión financiera efectiva
				\4[] Criterios de convergencia pueden sustituir parcialmente
				\4[] $\to$ Ayudas fiscales a sector financiero en crisis
				\4[] $\then$ Posible sostener situación parcial/temporalmente
				\4[] Unión financiera
				\4[] $\to$ Capitales fluyen en toda la unión
				\4[] $\to$ No hay vínculo bancos y determinados estados
				\4[] $\to$ Sector financiero tiene dimensión comunitaria
				\4[] Introducción de I
				\4[] $\to$ Rescates públicos de EEMM a banca nacional
				\4[] $\then$ Aparición de tensiones financieras
		\2 Economía política
			\3 Idea clave
				\4 Análisis de efectos de políticas
				\4[] Teniendo en cuenta quién gana y pierde
				\4[] $\to$ Considerando efectos políticos
				\4 Integración monetaria
				\4[] Genera ganadores y perdedores
			\3 Trilema de Rodrik
				\4 Rodrik (2000)\footnote{JEP Winter 2000.}
				\4 Restricción empírica postulada
				\4 Economías abiertas deben elegir 2 de 3:
				\4[I] Integración económica
				\4[II] Democracia
				\4[III] Soberanía nacional
				\4 Tres alternativas:
				\4[A] Camisa de fuerza de oro
				\4[] Integración económica+Estado nación soberano
				\4[] Sin transferencias fiscales entre estados
				\4[] Flujos de capital y comerciales libres
				\4[] Mercados internacionales limitan PEconómica nacional
				\4[] Sólo se proveen BPúblicos compatibles con MFinancieros
				\4[] Necesarias políticas autoritarias/represivas
				\4[] $\to$ Ante crisis de deuda/balanza de pagos
				\4[B] Federalismo supranacional
				\4[] Integración económica+democracia
				\4[] Apertura comercial y financiera plena
				\4[] Estados nación pierden soberanía
				\4[] $\to$ Entidad supranacional asume soberanía
				\4[] Transferencias fiscales entre estados
				\4[] $\to$ Posibles déficits exteriores y fiscales
				\4[] Democracia a nivel supranacional
				\4[] $\to$ Entidad supranacional se convierte en nación
				\4[C] Compromiso à la Bretton Woods
				\4[] Democracia+soberanía nacional
				\4[] Sin plena integración comercial+financiera
				\4[] Barreras a movimiento de capital generalizados
				\4[] Estados pueden evitar endeudamiento exterior
				\4[] Posible provisión democrática de bienes públicos
				\4[] $\to$ En la medida en que permita cap. productiva nacional
				\4[] $\to$ Como lo decidan votantes/responsable soberano
				\4[] Sin transmisión de soberanía a ent. supranacional
			\3 Implicaciones
				\4 Policy-makers deben considerar ec. política
				\4[] Oposición interna y externa a integración
				\4 Integración con mayor número de miembros
				\4[] Poder más diluido
				\4[] Más difícil implementar reformas necesarias
				\4[] $\to$ Para acercarse a OCA
	\1[] \marcar{Conclusión}
		\2 Recapitulación
			\3 Teoría de la integración
			\3 Análisis empírico
			\3 Implicaciones de política económica
		\2 Idea final
			\3 Efecto sobre terceros
				\4 Efectos de integración sobre integrados
				\4[] Objeto principal de análisis
				\4 Integración monetaria afecta a terceros
				\4[] No integrados difícilmente obtienen ventajas
				\4[] Sí pueden sufrir costes potenciales
				\4[$\then$] Modelos de eq. general deben tener en cuenta
				\4[] También posibles reacciones de no integrados
			\3 Soberanía política y monetaria
				\4 Moneda propia es elemento básico de soberanía
				\4 Renuncia a moneda propia
				\4[] Equilibrio dificil de alcanzar
				\4[] Coyuntura muy especial
				\4[] Tensiones políticas entre soberanos
			\3 Tendencia de l/p hacia integración monetaria
				\4 Rogoff a principios de 2001
				\4[] Vamos hacia 3 o 4 monedas en todo el mundo
				\4 Actualidad
				\4[] Muchos proyectos pero muy poca integración efectiva
				\4[] Integración comercial pierde fuelle
\end{esquemal}

\graficas

\begin{axis}{4}{Resumen del enfoque de requisitos mínimos en relación a la flexibilidad e integración de una economía.}{Flexibilidad}{Simetría}{flexibilidadintegracionsimetria}

	% Ejes del país B
	\draw[-] (6,4) -- (6,0) -- (10,0);
	\node[below] at (10,0){Integración};
	\node[left] at (6,4){Simetría};
	
	%%%%%%%%% Simetría y flexibilidad (izquierda)
	% Subtítulo del país
	\node[] at (2,-1){Simetría y flexibilidad};
	
	\draw[-] (0.5,3.5) -- (3.5,0.5);
	
	\node[right] at (8.5,2.5){Optimalidad};
	
	%%%%%%%%% Simetría e integración (derecha)
	% Nombre del país B
	\node[] at (8,-1){Simetría e integración};
	
	\draw[-] (6.5,3.5) -- (9.5,0.5);
	
	\node[right] at (2.5,2.5){Optimalidad};
	
\end{axis}

\preguntas

\seccion{22 de marzo de 2017}
\begin{itemize}
    \item Usted ha dicho que las curvas de isopérdidas son cóncavas. Siempre?
    \item Qué hay detrás de la inconsistencia dinámica de la PM?
    \item Cómo es posible una PM común en una Europa con dos velocidades?
    \item Qué opinión le merece el papel de Draghi al frente del BCE?
    \item Comente la funcion aseguradora del federalismo fiscal. Qué recursos utilizaría en caso de shocks?
\end{itemize}

\seccion{Test 2014}
\textbf{14.} Señale cuál de las siguientes afirmaciones es correcta en un escenario de deflación:
\begin{itemize}
	\item[a] Las ganancias de competitividad en una unión monetaria son más sencillas.
	\item[b] Aumenta el margen de actuación de la política monetaria.
	\item[c] Se incrementa el valor real de la deuda.
	\item[d] Todas las afirmaciones anteriores son falsas.
\end{itemize}

\seccion{Test 2011}
\textbf{28.} Según la teoría de las áreas monetarias óptimas:
\begin{itemize}
	\item[a] Los beneficios de la unión monetaria entre dos países son siempre mayores que sus costes.
	\item[b] Una unión monetaria es tanto más beneficiosa cuanto mayor sea la movilidad de los factores.
	\item[c] La diversificación de una economía es perjudicial para su integración en una unión monetaria.
	\item[d] Una unión monetaria es tanto más beneficiosa cuanto menor es el grado de integración financiera.
\end{itemize}

\seccion{Test 2007}
\textbf{33.} Según la teoría de las áreas monetarias óptimas, la incorporación a una unión monetaria sería tanto más ventajosa para un país cuando:
\begin{itemize}
	\item[a] Si la economía es bastante abierta y su estructura productiva está muy diversificada, su inflación es tradicionalmente alta y sus políticas antiinflacionistas poco creíbles.
	\item[b] Su economía es bastante cerrada y su estructura productiva está muy especializada; su inflación es tradicionalmente baja y sus políticas antiinflacionistas muy creíbles.
	\item[c] Su economía es bastante cerrada y su estructura productiva está muy especializaa; su inflación es tradicionalmente alta y sus políticas antiinflacionistas poco creíbles.
	\item[d] Su economía es bastante abierta y su estructura productiva está muy diversificada, su inflación es tradicionalmente baja y sus políticas antiinflacionistas muy creíbles. 
\end{itemize}

\seccion{Test 2005}
\textbf{30.} De acuerdo con la teoría de las áreas monetarias óptimas, una economía encontraría ventajoso formar una unión monetaria si, con respecto a sus potenciales socios:
\begin{itemize}
	\item[a] Está poco integrada desde el punto de vista comercial, tiene un historial de baja inflación y su ciclo económico está muy correlacionado.
	\item[b] Está muy integrada desde el punto de vista comercial, tiene un historial de elevada inflación, y su ciclo económico está muy correlacionado.
	\item[c] Está poco integrada desde el punto de vista comercial, tiene un historial de elevada inflación, y su ciclo económico está muy correlacionado.
	\item[d] Está muy integrada desde el punto de vista comercial, tiene un historial de elevada inflación y su ciclo económico está poco correlacionado.
\end{itemize}

\seccion{Test 2004}
\textbf{30.} De acuerdo con la teoría de las áreas monetarias óptimas, un conjunto de países serían buenos candidatos para formar una unión monetaria si:
\begin{itemize}
	\item[a] La movilidad de los factores productivos entre dichos países, así como su apertura exterior, son elevadas; y sus estructuras productivas son suficientemente especializadas.
	\item[b] La movilidad de los factores productivos entre dichos países, así como su apertura exterior, son reducidas; y sus estructuras productivas son suficientemente diversificadas. 
	\item[c] La movilidad de los factores productivos entre dichos países, así como su apertura exterior, son elevadas; y sus estructuras productivas son suficientemente diversificadas.
	\item[d] La movilidad de los factores productivos entre dichos países es reducida, aunque su apertura exterior sea elevada; y sus estructuras productivas son suficientemente especializadas.
\end{itemize}

\notas

\textbf{2014:} \textbf{14.} C

\textbf{2011:} \textbf{28.} B

\textbf{2007:} \textbf{33.} A

\textbf{2005:} \textbf{30.} B

\textbf{2004:} \textbf{30.} C

\bibliografia

Mirar en Palgrave:
\begin{itemize}
	\item currency crises models
	\item \textbf{currency unions}
	\item fiscal federalism
	\item fixed exchange rates
	\item international finance
	\item international monetary policy
	\item international real business cyles
	\item \textbf{Mundell, Robert (Born 1932)}
	\item new economic geography
	\item \textbf{optimum currency areas}
	\item real exchange rates 
\end{itemize}

Alesina, A. Barro, R. \textit{Currency Unions} (2002) The  Quarterly Journal of Economics -- En carpeta del tema

Alesina, A.; Barro, R.; Tenreyro, S. \textit{Optimal Currency Areas} (2002) Harvard Institute of Economic Research. Discussion Paper Number 1958 -- En carpeta del tema

Anderson, J.; van Wincoop, E. \textit{Gravity with Gravitas: A Solution to the Border Puzzle} (2001) NBER Working Paper Series -- En carpeta del tema

Arestis, P. Sawyer, M. \textit{Economic and Monetary Union Macroeconomic Policies. Current Practices and Alternatives} (2013) Palgrave MacMillan -- En carpeta Economía internacional

Eichengreen, B.; von Hagen, J. \textit{Federalism, Fiscal Restraints, and European Monetary Union} (1996) American Economic Review -- En carpeta del tema

Bayoumi, T.; Eichengreen, B. \textit{Shocking aspects of European Monetary Unification} (1992) NBER Working Paper Series -- En carpeta del tema

Carlberg, M. \textit{Macroeconomics of Monetary Union} (2007)

De Grauwe, P. (2006) \textit{What Have we Learnt about Monetary Integration since the Maastricht Treaty?} Journal of Common Market Studies -- En carpeta del tema

Dornbusch, R. \textit{Expectations and Exchange Rate Dynamics} (1976) Journal of Political Economy -- En carpeta del tema

Eichengreen, B. \textit{Is Europe an Optimum Currency Area?} (1991) NBER Working Paper Series -- En carpeta del tema

Frankel, J. A. \textit{No Single Currency Regime is Right for all Countries or at all Times} (1999) NBER Working Paper Series -- En carpeta del tema

Frankel, J. Rose, A. \textit{Is EMU more justifiable ex post than ex ante?} (1997) European Economic Review -- En carpeta del tema

Frankel, J. Rose, A. \textit{The Endogeneity of the Optimum Currency Area Criteria} (1998) The Economic Journal -- En carpeta del tema

Gali, J.; Monacelli. \textit{Optimal Monetary and Fiscal Policy in a Currency Union} (2008) Journal of International Economics -- En carpeta del tema

Friedman, M. Mundell, R. \textit{One World, One Money?} (2001) Options Politiques: Mai 2001 -- En carpeta del tema

Gali, J. Perotti, R. \textit{Fiscal Policy and Monetary Integration in Europe} (2003) NBER Working Paper Series -- En carpeta del tema

Gandolfo, G. \textit{International Finance and Open-Economy Macroeconomics} (2016) Springer Verlag. Ch. 20 -- En carpeta Economía internacional

Glick, R.; Rose, A. K. \textit{Does a Currency Union Affect Trade? The Time Series Evidence} (2001) NBER Working Paper Series -- En carpeta del tema

Glick, R.; Rose, A. K. \textit{Currency Unions and Trade: A Post-EMU Mea Culpa} (2015) NBER Working Paper Series -- En carpeta del tema

de Haan, J.; Inklaar, R. \textit{Will Business Cycles in the Euro Area Converge? A Critical Survey of Empirical Research} (2008) Journal of Economic Surveys -- En carpeta del tema

Johnston, A.; Regan, A. \textit{European Monetary Integration and the Incompatibility of National Varietis of Capitalism} (2016) Journal of Common Market Studies -- En carpeta del tema

Krugman, P. \textit{Lessons of Massachusetts for EMU} (2001) International Library of Critical Writings in Economics -- En carpeta del tema

Lane, P. R. \textit{The Real Effects of European Monetary Union} (2006) Journal of Economic Perspectives: Fall 2006 -- En carpeta del tema

Meade, E. \textit{Monetary Integration} (2009) Harvard International Review -- \url{http://hir.harvard.edu/article/?a=1844}

Mundell, R. \textit{A Theory of Optimum Currency Areas} (1961) American Economic Review -- En carpeta del tema

Obstfeld, M.; Shambaugh, J.; Taylor, A. M. \textit{The Trilemma in History: Tradeoffs among Exchange Rates, Monetary Policies, and Capital Mobility} (2004) NBER Working Paper Series -- En carpeta del tema

Obstfeld, M.; Shambaugh, J.; Taylor, A. \textit{Financial Stability, the Trilemma, and International Reserves} (2008) NBER Working Paper Series -- En carpeta del tema

Pilbeam, K. \textit{International Finance} (2006) 3rd Edition -- En carpeta Economía Internacional

Rodrik, D. (2000) \textit{How Far Will International Economoic Integration Go} Journal of Economic Perspectives. Winter 2000. -- En carpeta del tema

Rose, A. K. \textit{One money, one market: the effect of common currencies on trade} (2000) Economic Policy -- En carpeta del tema

Rose, A. K. \textit{Is EMU Becoming an Optimum Currency Area? The Evidence on Trade and Business Cycle Synchronization} (2008) Unpublished draft -- En carpeta del tema

Rose, A. K. \textit{Why Do Estimates of the EMU Effect on Trade Vary so Much?} (2016) NBER Working Paper Series -- En carpeta del tema

Sarno, L.; Taylor, M. \textit{The economics of exchange rates} (2002) Cambridge University Press -- En carpeta Economía internacional

Taylor, M. P. (1995) \textit{The Economics of Exchange Rates} Journal of Economic Literature Vol. XXXIII -- En carpeta del tema

Tomann, H. \textit{Monetary Integration in Europe. The European Monetary Union after the Financial Crisis} (2017) Palgrave MacMillan -- -- En carpeta Economía Internacional

Vicquéry, R. \textit{Optimum Currency Areas and European Monetary Integration. Evidence from the Italian and German Unifications} (2017) Preliminary draft 

\end{document}
