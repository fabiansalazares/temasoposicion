\documentclass{nuevotema}

\tema{3A-44}
\titulo{Evidencia empírica sobre el crecimiento económico: análisis de contabilidad del crecimiento, impacto de los distintos factores que influyen sobre el crecimiento económico y evidencia empírica sobre convergencia.}

\begin{document}

\ideaclave

Ver \url{https://voxeu.org/article/it-s-too-soon-optimism-about-convergence} sobre hallazgos recientes en convergencia y crítica a Barro y Sala-i-Martín (1990)

\seccion{Preguntas clave}
\begin{itemize}
	\item ¿Qué es la contabilidad del crecimiento?
	\item ¿Cómo relacionar acumulación de factores con crecimiento?
	\item ¿Qué otros factores determinan el crecimiento?
	\item ¿Qué es la convergencia económica?
	\item ¿Se produce la convergencia?
	\item ¿Qué factores determinan el grado de convergencia?
	\item ¿Qué evidencia empírica existe al respecto?
\end{itemize}

\esquemacorto

\begin{esquema}[enumerate]
	\1[] \marcar{Introducción}
		\2 Contextualización
			\3 Evolución histórica de la renta per cápita
			\3 Evidencia empírica sobre crecimiento
			\3 Hechos estilizados de Kaldor
		\2 Objeto
			\3 ¿Qué hechos estilizados muestra la evidencia sobre crecimiento?
			\3 ¿Qué es la contabilidad de crecimiento?
			\3 ¿Para qué sirve?
			\3 ¿Qué factores determinan el crecimiento?
			\3 ¿Qué evidencia empírica existe?
			\3 ¿Qué es la convergencia económica?
			\3 ¿Qué evidencia empírica existe al respecto?
			\3 ¿Las economías nacionales y regionales convergen?
		\2 Estructura
			\3 Contabilidad del crecimiento
			\3 Convergencia
			\3 Impacto de los distintos factores
	\1 \marcar{Contabilidad del crecimiento}
		\2 Idea clave
			\3 Autores
			\3 Concepto
			\3 Aplicaciones
		\2 Enfoque convencional
			\3 Idea clave
			\3 Formulación
			\3 Valoración
		\2 Estimación dual basada en precios
			\3 Idea clave
			\3 Formulación
		\2 Estimación econométrica
			\3 Idea clave
			\3 Formulación
			\3 Valoración
		\2 Problemas de la contabilidad del crecimiento
			\3 Heterogeneidad de los factores
			\3 Sobreestimación e infraestimación del capital
			\3 Utilización de los factores
			\3 Producción total o valor añadido
			\3 Contabilidad frente a causalidad
		\2 Evidencia empírica
			\3 Participaciones de los ff.pp. en la renta
			\3 Rasgos generales del crecimiento
			\3 Contribución del capital humano
			\3 Productivity slowdown
			\3 Tigres asiáticos
	\1 \marcar{Convergencia}
		\2 Idea clave
			\3 Concepto
			\3 Objetos de análisis de convergencia
		\2 Tipos de convergencia
			\3 beta-convergencia
			\3 sigma-convergencia
			\3 Convergencia de series temporales
		\2 Predicciones de modelos teóricos
			\3 Modelo neoclásico
			\3 Modelos de crecimiento endógeno
		\2 Evidencia empírica
			\3 beta-convergencia
			\3 sigma-convergencia
	\1 \marcar{Impacto de los distintos factores}
		\2 Idea clave
			\3 Causas próximas y fundamentales
			\3 Análisis empírico y teórico
			\3 Relación con otros programas de investigación
		\2 Problemas econométricos
			\3 Análisis de robustez
			\3 Datos de panel vs sección cruzada
			\3 Multicolinealidad
		\2 Evidencia empírica
			\3 Artículos importantes
			\3 Geografía
			\3 Capital humano
			\3 Consumo público
			\3 Inversión pública
			\3 Demografía
			\3 Instituciones
			\3 Desigualdad
			\3 Democracia o dictadura
			\3 Crecimiento endógeno
		\2 Efectos del comercio sobre el crecimiento
			\3 Dirección de la causalidad
			\3 Efectos escala
			\3 Transferencia vía exportaciones e importaciones
			\3 Spillovers tecnológicos
			\3 Convergencia
			\3 Otros factores
	\1[] \marcar{Conclusión}
		\2 Recapitulación
			\3 Contabilidad del crecimiento
			\3 Convergencia
			\3 Impacto de los distintos factores
		\2 Idea final
			\3 Fenómenos recientes
			\3 Difícil pensar en otra cosa

\end{esquema}

\esquemalargo

















\begin{esquemal}
	\1[] \marcar{Introducción}
		\2 Contextualización
			\3 Evolución histórica de la renta per cápita
				\4 A lo largo de historia humana
				\4[] PIBpc prácticamente estable
				\4[] Muy similar en todo el mundo
				\4 Divergencia global
				\4[] A partir del año 1000 d.C
				\4[] $\to$ Según algunos autores
				\4[] A partir de 1800 d.C.
				\4[] $\to$ Según toda la literatura
				\4[] Europa occidental + satélites
				\4[] $\to$ Comienzan a divergir
				\4[] $\then$ Crecimiento económico sostenido
				\4[] $\then$ Enormes diferencias de renta actuales
			\3 Evidencia empírica sobre crecimiento
				\4 Modelos teóricos proponen explicaciones
				\4[] $\to$ De fenómenos observados
				\4[] Modelo neoclásico
				\4[] $\to$ Acumulación de factores producibles en c/p
				\4[] $\to$ Acumulación de factores no producibles en l/p
				\4[] $\to$ Desviación respecto a estado estacionario
				\4[] $\to$ Progreso tecnológico externo
				\4[] Modelo de crecimiento endógeno
				\4[] $\to$ Rendimientos crecientes a escala
				\4[] $\to$ Externalidades de capital
				\4[] $\to$ Innovación tecnológica
				\4 Análisis empírico
				\4[] Punto de partida
				\4[] $\to$ Para saber qué hay que explicar
				\4[] Herramienta de contrastación de teorías
				\4[] $\to$ ¿Predicciones concuerdan con realidad?
				\4[] $\to$ ¿Realidad presenta anomalías?
				\4 Principales objetivos
				\4[] Relacionar crecimiento con otras variables
				\4[] Contrastar veracidad de modelos teóricos
			\3 Hechos estilizados de Kaldor
				\4 Kaldor (1957), (1961)
				\4 Importancia
				\4[] Se mantienen relevantes en general
				\4[] Algunos ya no se cumplen
				\4 Punto de partida de muchos estudios
				\4[I] Crecimiento PIBpc positivo en l/p
				\4[] Tasa de crecimiento no tiende a disminuir
				\4[II] Crecimiento de K/L positivo
				\4[] Independiente de medida de K elegida
				\4[III] Retorno a K aprox. constante
				\4[] Especialmente en países desarrollados
				\4[] Tendencia reciente a caer
				\4[IV] Ratio K/Y aprox. constante
				\4[] Sin tendencias claras a l/p
				\4[] Capital y producción
				\4[] $\to$ Crecen a tasas similares
				\4[V] Share de ff.pp. en renta varía poco
				\4[] Inversión y beneficio
				\4[] $\to$ Altamente correlacionados
				\4[VI] Diferente $\Delta$ \% de produc. de trabajo
				\4[] Fuertes variaciones en crec. de productividad
		\2 Objeto
			\3 ¿Qué hechos estilizados muestra la evidencia sobre crecimiento?
			\3 ¿Qué es la contabilidad de crecimiento?
			\3 ¿Para qué sirve?
			\3 ¿Qué factores determinan el crecimiento?
			\3 ¿Qué evidencia empírica existe?
			\3 ¿Qué es la convergencia económica?
			\3 ¿Qué evidencia empírica existe al respecto?
			\3 ¿Las economías nacionales y regionales convergen?
		\2 Estructura
			\3 Contabilidad del crecimiento
			\3 Convergencia
			\3 Impacto de los distintos factores
	\1 \marcar{Contabilidad del crecimiento}
		\2 Idea clave
			\3 Autores
				\4[] (Ver bibliografía de \textit{growth accounting})
				\4 Tinbergen (1942)
				\4 Solow (1957, 1960)
				\4 Jorgenson (1966)
				\4 Grilliches y Jorgenson (1966)
			\3 Concepto
				\4 Conjunto de cálculos y supuestos
				\4[] $\to$ Función de producción agregada
				\4[] $\to$ Competencia perfecta
				\4 Establecen relación entre:
				\4[] $\Delta$ de inputs, $\Delta$ de PTF y $\Delta$ de output
				\4[] $\to$ No explicar por qué aumenta input
				\4[] $\to$ No relacionan causas profundas y output
				\4[] $\to$ Causalidad input y output no es directa\footnote{Ver Aghion pág. 112.}
				\4 Representar crecimiento como
				\4[] Como suma ponderada de:
				\4[] $\to$ Tasas de crecimiento de los inputs
				\4[] $\to$ Variación de la PTF
				\4[] Pesos de variación de los inputs respectivos
				\4[] $\to$ Participación en remuneración total de factores
			\3 Aplicaciones
				\4 Atribuir crecimiento a diferentes factores
				\4[] Crecimiento debido a:
				\4[] $\to$ Acumulación de factores de producción
				\4[] $\to$ Mejoras en la PTF
				\4 Contrastar predicciones de modelos
				\4[] Modelo neoclásico:
				\4[] $\to$ Capital y PTF+Población convergen a = tasa
				\4[] $\to$ Convergencia de tasas de crecimiento
				\4[] Crecimiento endógeno:
				\4[] $\to$ No se produce convergencia
		\2 Enfoque convencional
			\3 Idea clave
				\4 $\Delta$ de producto como resultado de:
				\4[] $\to$ $\Delta$ de ff.pp.\footnote{Cuantificado con índice de unidades.}
				\4[] $\to$ $\Delta$ de PTF
				\4 Supuesto clave
				\4[] Competencia perfecta en mercados de factores
				\4[] $\to$ Factores remunerados a productividad marginal
				\4 Tasas de crecimiento conocidas o estimables
				\4[] Crecimiento de la población
				\4[] $\to$ Con alto grado de precisión
				\4[] Crecimiento del capital
				\4[] $\to$ Sujeto a interpretación pero estimable
				\4[] Crecimiento del producto
				\4[] $\to$ Asimilable a producción total o PIB
				\4 PTF como residuo
				\4[] Definido como diferencia entre:
				\4[] $\to$ Crecimiento de la producción
				\4[] $\to$ Contribución de capital y trabajo al crecimiento
				\4[] $\then$ ``\textit{Medida de nuestra ignorancia}''
				\4[] $\then$ ``\textit{Residuo de Solow}''
			\3 Formulación
				\4 Producción agregada
				\4[] $Y(t) = A(t) F(K(t), L(t))$
				\4 Tasas de crecimiento: descomposición
				\4[] 1. Derivar respecto a $t$
				\4[] $\dot{Y}(t) = \dot{A} F(K,L) + A F_K \dot{K} + A F_L \dot{L}$
				\4[] 2. Dividir entre Y
				\4[] $\frac{\dot{Y}}{Y} =  \frac{\dot{A} F}{Y} + \frac{A F_K \dot{K}}{Y} + \frac{A F_L \dot{L}}{Y} $
				\4[] 3. Dividir y multiplicar por $A$, $L$ y $K$
				\4[] $\frac{\dot{Y}}{Y} = \underbrace{\frac{F A}{Y}}_{=1} \frac{\dot{A}}{A} + \underbrace{\frac{A F_K K}{Y}}_{\alpha_K} \frac{\dot{K}}{K} + \underbrace{\frac{A F_L L}{Y}}_{\alpha_L} \frac{\dot{L}}{L}$
				\4[] $\then$ \fbox{$\frac{\dot{Y}}{Y} = \frac{\dot{A}}{A} + \alpha_K \frac{\dot{K}}{K} + \alpha_L \frac{\dot{L}}{L}$}
				\4 Residuo
				\4[] \fbox{$g \equiv \frac{\dot{A}}{A} = \frac{\dot{Y}}{Y} - \alpha_K \frac{\dot{K}}{K} - \alpha_L \frac{\dot{L}}{L}$}
				\4 Valores de $\alpha_L$ y $\alpha_K$
				\4[] $\alpha_K =\frac{F_K K}{Y}$, $\alpha_L = \frac{F_L L}{Y}$
				\4[] $F_K$ y $F_L$ difícilmente estimables
				\4[] Solución:
				\4[] $\to$ Asumir ff.pp. remunerados a PMg
				\4[] $\to$ $A F_K = r$, $A F_L = w$
				\4[] $\then$ $\alpha_K = \frac{w L}{Y}$, $\alpha_L = \frac{rK}{Y}$
				\4[] $\then$ $\alpha$ son partipaciones de ff.pp. en renta
				\4[] $\then$ $\alpha$ son estimables
				\4 Tiempo discreto
				\4[] \fbox{$\frac{\Delta A_t }{A_t} = \frac{\Delta Y_t}{Y_t} - \bar{\alpha}_K \frac{\Delta K_t}{K_t} - \bar{\alpha}_L \frac{\Delta L_t}{L_t}$}
				\4[] $\to$ $\Delta X_t = X_{t+1} - X_t$
				\4[] $\to$ $\bar{\alpha} = \frac{\alpha_t + \alpha_{t+1}}{2}$
			\3 Valoración
				\4 Atribución de ff.pp. a contribución a crecimiento
				\4[] Tiene sentido si:
				\4[] $\to$ Forma funcional asumida aproxima bien
				\4[] $\to$ $\alpha$ son correctos
				\4[] ¿Qué valores asignar a $\alpha$?
				\4[] $\to$ Verdaderos valores no son estimables
				\4[] $\to$ $\alpha$ como participación en renta es supuesto
				\4 Remuneración factorial a PMg de factores
				\4[] Supuesto fuerte
				\4[] $\to$ Requiere competencia perfecta en factores
				\4[] $\to$ Asume estabilidad de participaciones
		\2 Estimación dual basada en precios
			\3 Idea clave
				\4 Hsieh (2002)
				\4[] Explicar explosión del crecimiento en Asia
				\4[] Relajando supuesto de $r=F_K$ y $w=F_L$
				\4 Reducir supuestos necesarios
				\4[] Único supuesto del modelo
				\4[] $\to$ Producto total iguala remuneración de factores
				\4[] No es necesario asumir
				\4[] $\to$ Forma funcional de f. de prod.
				\4[] $\to$ Tipo de cambio tecnológico
				\4[] $\to$ Relación entre PMg y remuneración de factores
			\3 Formulación
				\4 Reparto de la renta
				\4[] $Y(t) = r(t)\cdot K(t) + w(t) \cdot L(t)$
				\4 Transformación a tasas de crecimiento
				\4[] 1. Derivar respecto a $t$
				\4[] $\to$ $\dot{Y} = \dot{r} K + r \cdot \dot{K} + \dot{w} L + w \dot{L}$
				\4[] 2. Dividir entre Y a ambos lados
				\4[] $\to$ $ \frac{\dot{Y}}{Y} = \dot{r} \frac{K}{Y} + \dot{K} \frac{r}{Y} + \dot{w} \frac{L}{Y} + \dot{L} \frac{w}{Y}$
				\4[] 3. Dividir y multiplicar entre $r$, $K$, $w$ y $L$
				\4[] $\to$ $\frac{\dot{Y}}{Y} = \frac{\dot{r}}{r} \cdot \frac{rK}{Y} + \frac{\dot{K}}{K} \cdot \frac{rK}{Y} + \frac{\dot{w}}{w} \cdot \frac{wL}{Y} + \frac{\dot{L}}{L} \cdot \frac{wL}{Y}$
				\4[] $\to$ $\frac{\dot{Y}}{Y} = \alpha_K \cdot \left( \frac{\dot{r}}{r} + \frac{\dot{K}}{K} \right) + \alpha_L \cdot \left( \frac{\dot{w}}{w} + \frac{\dot{L}}{L} \right)$
				\4[] $\to$ $\hat{y} = \alpha_K \left( \hat{r} + \hat{K} \right) + \alpha_L \left( \tilde{w} + \hat{L} \right)$
				\4 Residuo
				\4[] \fbox{$g = \underbrace{\hat{y} - \alpha_K \hat{K} - \alpha_L \hat{L}}_\text{primal} = \underbrace{\alpha_K \hat{r} + \alpha_L \hat{w}}_\text{dual}$}
		\2 Estimación econométrica
			\3 Idea clave
				\4 Estimar residuo y $\alpha_L$, $\alpha_K$
				\4[] Sin asumir remuneración a prod. marginal
				\4[] Ponderaciones $\alpha_K$ y $\alpha_L$
				\4 Regresión econométrica
				\4[] Crecimiento del output
				\4[] $\to$ Variable dependiente
				\4[] Crecimiento de ff.pp
				\4[] $\to$ Variables independientes
				\4[] Residuo de Solow
				\4[] $\to$ Constante/ordenada en origen
			\3 Formulación
				\4 Regresión
				\4[] $\hat{y} = g + \alpha_L \hat{L} + \alpha_K \hat{K} + \epsilon_i$
			\3 Valoración
				\4 Datos temporales utilizados como sección cruzada
				\4[] Plantea problemas
				\4[] $\to$ Se estiman $\alpha$ que en realidad evolucionan
				\4[] $\to$ Variables no estimadas pueden sesgar
				\4[] $\to$ Medición errónea de K y L introduce sesgo
		\2 Problemas de la contabilidad del crecimiento
			\3 Heterogeneidad de los factores
				\4 Considerar sólo L y K
				\4[] Agregación de muchas variedades de L y K
				\4[] $\to$ Problemas habituales de agregación
				\4 Medición de clases de capital
				\4[] Equivalencia entre tipos de capital
				\4[] $\to$ ¿Cuántos tractores hacen un portátil?
				\4[] Equivalencia entre calidades de capital
				\4[] $\to$ ¿Equivalencia entre K de 2018 y 1998?
			\3 Sobreestimación e infraestimación del capital
				\4 Mejoras en calidad de capital
				\4[] Difícil estimación cuantitativa
				\4[] Aplicables métodos hedónicos
				\4[] Generalmente, aplican deflactores
				\4[] $\to$ Tendencia a subestimar aumento
				\4 Corrupción y malas inversiones
				\4[] $\to$ Tendencia a sobreestimar
				\4[$\then$] ¿Sobre. e infra. se compensan?
				\4[] En general, se suele sobreestimar
			\3 Utilización de los factores
				\4 Conversión de stocks a flujos
				\4[] Asumiendo medición de stock de K posible
				\4[] $\to$ ¿Qué flujo de servicios en cada momento?
				\4 Utilización del trabajo
				\4[] Menos problemática
				\4[] $\to$ Medidas como horas de trabajo
				\4[] $\to$ Pero ¿esfuerzo es constante?
				\4 Utilización del capital
				\4[] Proxies como consumo eléctrico
				\4[] ¿Depreciación endógena?
			\3 Producción total o valor añadido
				\4 Utilizar producción total tiene ventajas
				\4[] Estimación mucho más simple
				\4[] $\to$ ``Todo'' lo que sale de los factores
				\4[] F.prod. bien definida para prod. total
				\4[] $\to$ Supuestos menos restrictivos que VA
				\4 Desventajas de producción total
				\4[] Muy sensible a integración vertical
				\4[] $\to$ Menos integración aumenta prod. total
			\3 Contabilidad frente a causalidad
				\4 CCrecimiento expresa relación directa
				\4[] Entre ff.pp. y producto total
				\4[] $\Delta$ de ff.pp. sólo depende de tiempo
				\4 Causalidad es relación más compleja
				\4[] $\Delta$ de ff.pp. puede ser endógeno
				\4 Ejemplo:
				\4[] En contexto de modelo de Solow
				\4[] EE implica $\frac{Y}{K}$, $\frac{K}{AL}$ constantes
				\4[] En EE, K crece a tasa $\hat{A} + \hat{L}$
				\4[] $\to$ K depende de L y A
				\4[] $\then$ Y depende de L y A
				\4[] $\then$ A causa $y \equiv \frac{Y}{L}$
				\4[] Pero en contabilidad de crecimiento:
				\4[] $\hat{Y}$ depende de K, L y A
		\2 Evidencia empírica
			\3 Participaciones de los ff.pp. en la renta
				\4 Participación del capital entre 1960 y 2000
				\4[] Entre 0,2 y 0,5 en desarrollados
				\4 Evolución temporal
				\4[] Pequeña tendencia $\downarrow$ de $\alpha_L$ en últimos lustros
			\3 Rasgos generales del crecimiento
				\4 Últimas 4 décadas de siglo XX
				\4 Crecimiento promedio mundial
				\4[] $\to$ 4\%
				\4 Crecimiento PIBpc
				\4[] $\to$ 2,3\%
				\4 Acumulación de K físico
				\4[] $\to$ 1\% por trabajador o 40\% total
				\4 Crecimiento de PTF
				\4[] $\to$ 1\% por trabajador o 40\% total
				\4 Crecimiento de capital humano
				\4[] $\to$ 3 décimas restantes o 10\%
				\4 Regiones
				\4[] Occidente, Japón
				\4[] $\to$ Fuerte crecimiento hasta 70s
				\4[] $\to$ Sobre todo, PTF
				\4[] Tigres asiáticos
				\4[] $\to$ Crecimiento fortísimo
				\4[] $\to$ Atribuido a ff.pp. pero controversia
				\4[] Latinoamérica, África
				\4[] $\to$ Muy pobre crecimiento
				\4[] $\to$ Aceleración reciente
				\4[] $\to$ Bajísimo crecimiento de PTF
			\3 Contribución del capital humano
				\4 Jorgenson (1995)
				\4[] Países OCDE:
				\4[] $\to$ Contribuciones K físico y humano
				\4[] PTF contribuye menos
				\4[] Capital físico y humano contribuyen más
				\4[] Estudios que no estiman K humano
				\4[] $\to$ Sobrevaloran PTF
			\3 Productivity slowdown
				\4 Desde primeros 70s
				\4[] $\to$ Menor crecimiento de PTF
				\4[] Denison (1985)
				\4[] $\to$ Primero en documentar
				\4[] $\to$ Para EEUU
				\4[] Confirmación posterior para otros países
				\4 Posibles explicaciones
				\4[] Alza del petróleo
				\4[] $\to$ Poco satisfactoria
				\4[] $\to$ PTF no crece cuando petróleo cae
				\4[] Cambio estructural hacia servicios
				\4[] $\to$ Servicios tienen menor $\Delta$ de PTF
				\4[] $\to$ Aumenta enormemente el peso de servicios
				\4[] $\to$ Mejoras en TIC todavía no se habían producido
				\4[] $\to$ En los 90 repunta en EEUU
				\4[] Capital humano
				\4[] $\to$ Hasta 70s, fuerte acumulación K humano
				\4[] $\to$ Estudios no controlan por K humano
				\4[] $\to$ Atribuyen contribución K humano a PTF
				\4[] $\to$ 70s estabilizan contribución
			\3 Tigres asiáticos
				\4 Récord mundial de crecimiento
				\4 Convergencia con Occidente
				\4[] Hong Kong, Singapur, Corea del Sur, Taiwan
				\4 Contribución de factores
				\4[] Contribución débil de PTF
				\4[] Mayor parte, crecimiento del capital
				\4[] Evidencia favorable a Solow
				\4[] Hsieh (2002) contradice
				\4[] $\to$ PMg de K debería haber bajado en SING
				\4[] $\to$ Encuentra que ha aumentado
				\4[] $\to$ Encuentra también que salarios aumentan
				\4[] $\then$ PTF debe haber aumentado
				\4[] $\to$ Primeros estudios concluyen
	\1 \marcar{Convergencia}
		\2 Idea clave
			\3 Concepto
				\4 Tendencia hacia la reducción de diferencias
				\4[] Entre unidades económicas
				\4[] $\to$ Países
				\4[] $\to$ Regiones
				\4[] $\to$ Estados
				\4 Connotación años 50 y 60
				\4[] Tendencia hacia igualdad entre
				\4[] $\to$ Occidente capitalista y países comunistas
				\4 Sentido moderno
				\4[] Persistencia/desaparición de diferencias en PIBpc
				\4[] Ocasionalmente, también otras variables
				\4[] $\to$ Desempleo
				\4[] $\to$ Paro
				\4[] $\to$ Productividad por ocupad
			\3 Objetos de análisis de convergencia
				\4 Definir convergencia de forma precisa
				\4[] Reducción de ¿qué diferencias?
				\4[] $\to$ Diferencia media
				\4[] $\to$ Varianza
				\4[] ...
				\4 Describir hechos empíricos
				\4[] Extraer hechos estilizados relevantes
				\4 Contrastar con modelos teóricos
				\4[] ¿Concuerdan predicciones con hechos empíricos?
		\2 Tipos de convergencia
			\3 beta-convergencia
				\4 Convergencia es relación negativa entre:
				\4[] Renta inicial
				\4[] Tasa de crecimiento
				\4[$\then$] $\uparrow$ Renta, $\downarrow$ Crecimiento
				\4[$\then$] Países más pobres crecen más que ricos
				\4 En términos formales
				\4[] Regresión crecimiento contra PIBpc
				\4[] $\to$ Relación negativa significativa
				\4[] \fbox{$g_i = k + \beta \ln y_{i,0} + \epsilon_i$, $\beta <0$}
				\4 $\beta$-convergencia condicional
				\4[] $\beta$-conv. entre países similares
				\4[] $\then$ Necesario controlar por otras características
				\4[] Añadir otros factores a regresión
				\4[] \fbox{$g_i = k + \beta \ln y_{i,0} + \vec{\gamma} \vec{Z}_i + \epsilon_i$, $\beta <0$}
				\4[] Problema:
				\4[] $\to$ ¿Qué factores son relevantes?
				\4[] $\to$ ¿Qué criterio para elegir factores?
			\3 sigma-convergencia
				\4 Convergencia es tendencia a reducción de
				\4[] Varianza de una sección cruzada de países
				\4 En términos formales
				\4[] $\sigma^2_{\ln y,t} > \sigma^2_{\ln y, t+T} $
				\4 Relación entre $\beta$ y $\sigma$-convergencia
				\4[] $\beta$-convergencia es condición necesaria
				\4[] $\to$ NO es condición suficiente
				\4[] Es decir:
				\4[] $\sigma$-convergencia $\then$ $\beta$-convergencia
				\4[] $\beta$-convergencia $\nRightarrow$ $\sigma$-convergencia
				\4[] Falacia de Galton (FALSO):
				\4[] $\to$ Reversión a la media en series temporales
				\4[] $\then$ Reducción de la varianza
				\4[] \grafica{betasigmaconvergencia}
				\4 $\sigma$-convergencia condicional
				\4[] Similar a $\beta$-convergencia condicional
				\4[] Reducción de $\sigma$ entre países similares
			\3 Convergencia de series temporales
				\4 Convergencia es tendencia a reducción de
				\4[] diferencias en PIBpc en el infinito
				\4[] $\to$ Dado historial de crecimiento pasado
				\4 En términos formales
				\4[] $\lim_{T\to \infty} E\left( \ln y_{i, t+T} - \ln y_{j, t+T} \, | \, F_t \right) = 0$
		\2 Predicciones de modelos teóricos
			\3 Modelo neoclásico
				\4 Predice $\beta$-convergencia condicional
				\4[] En EE: $\frac{Y}{AL} \equiv y^* = \left( \frac{s}{n+g+\delta} \right)^{\alpha/(1-\alpha)}$
				\4[] $\then$ $k^* = k(g,\delta,n)$, $y^* = y(g,\delta,n)$
				\4[] Dos países con mismo $g$, $\delta$, $n$
				\4[] $\then$ Mismo estado estacionario
				\4[] $\then$ ``Club convergence''
				\4[] ¿A qué velocidad convergen hacia EE?
				\4[] $\to$ Depende de $y(k(0))$ inicial
				\4[] Menor $k(0)$ $\then$ Mayor crecimiento
				\4[] $\then$ Convergencia $\beta$ y $\sigma$
				\4 Velocidad de convergencia
				\4[] $\beta \equiv \dv{\left( \dot{k}/k \right)}{\ln k} = \dv{\left( \dot{y}/y \right) }{\ln y} = -(1-\alpha) \cdot (\delta + n + g)$
			\3 Modelos de crecimiento endógeno
				\4 Modelos AK
				\4[] No predicen convergencia
				\4[] Misma tasa de crecimiento con + capital
				\4 Modelos de innovación
				\4[] Algunas variantes predicen convergencia
				\4[] Convergencia vía innovación tecnológica
				\4[] $\to$ Salto a la frontera
		\2 Evidencia empírica
			\3 beta-convergencia
				\4 No hay convergencia absoluta
				\4 Convergencia condicional sí tiene lugar
				\4[] Barro y Sala-i-Martín (1990)
				\4[] $\to$ Internacional
				\4[] $\to$ Interregional
				\4[] Argumento contra CEndógeno
				\4[] Respuesta de CEndógeno
				\4[] $\to$ CEndógeno con convergencia
				\4[] $\to$ Convergencia vía trans. tecnológica
				\4[] $\then$ Club convergence
				\4 Velocidad de convergencia
				\4[] Modelo de Solow con $\alpha$ estándar
				\4[] $\to$ Predice convergencia rápida
				\4[] Series empíricas
				\4[] $\to$ Convergencia más lenta
				\4[] $\to$ Tasa de convergencia del 2\% anual
				\4[] Mankiw, Romer y Weil (1992)
				\4[] $\to$ Considerar capital humano como K
				\4[] $\to$ Aumentar $\alpha$
				\4[] $\then$ Convergencia lenta acorde con datos
				\4 Johnson y Papageorgiu (2018)\footnote{https://voxeu.org/article/it-s-too-soon-optimism-about-convergence.}
				\4[] $\beta$-convergencia no está teniendo lugar
				\4[] Existen clubes de convergencia
				\4[] $\to$ En sentido de equilibrios múltiples
				\4[] $\then$ Similares a ``trampas de crecimiento''
				\4[] Países con ingresos altos (HIC)
				\4[] $\to$ Tasas relativamente elevadas
				\4[] Países con ingresos medios (MIC)
				\4[] $\to$ Crecimiento relativamente elevado
				\4[] $\to$ Generalmente, menor que (HIC)
				\4[] $\to$ Sólo última década mayor, en algunos
				\4[] Países con ingresos bajos (LIC)
				\4[] $\to$ Tasas muy bajas de crecimiento
				\4[] Convergencia sólo a nivel de clubes
				\4[] Marco de Solow no es apropiado
				\4[] $\to$ Convergencia resultado de muy pocos parámetros
				\4[] Implicaciones de política económica
				\4[] $\to$ Necesarias actuaciones profundas en LIC
				\4[] $\to$ PEconómica a pequeña escala es insuficiente en LICs
			\3 sigma-convergencia
				\4 Varianzas muy elevadas
				\4[] Modelos neoclásicos predicen $\sigma$ inferior
				\4[] Diferencias de PIBpc mayores que K per cápita
				\4[] $\to$ Lucas (1990)
				\4 Evolución de la varianza
				\4[] Ligera convergencia interregional
	\1 \marcar{Impacto de los distintos factores}
		\2 Idea clave
			\3 Causas próximas y fundamentales
				\4 Causas próximas:
				\4[] Factores con influencia directa en crecimiento
				\4[] $\to$ Acumulación de capital físico
				\4[] $\to$ Crecimiento demográfico
				\4[] $\to$ Avances tecnológicos
				\4 Causas fundamentales:
				\4[] Causan causas próximas
				\4[] $\to$ Instituciones
				\4[] $\to$ Preferencias
				\4[] $\to$ Cultura
				\4[] $\to$ Geografía física
				\4[] $\to$ Azar
				\4 Asumiendo causas próximas conocidas
				\4[] ¿Qué factores determinan causas próximas?
			\3 Análisis empírico y teórico
				\4 Modelos téoricos son punto de partida
				\4[] $\to$ Apuntar relaciones entre variables
				\4 Análisis econométrico
				\4[] Encontrar correlaciones
				\4[] Tratar de demostrar causalidad
			\3 Relación con otros programas de investigación
				\4 Series temporales
				\4 Experimentos aleatorizados
				\4 Economía del desarrollo
		\2 Problemas econométricos
			\3 Análisis de robustez
				\4 Especificación de regresiones es relevante
				\4 Misma variable puede ser significativa o no
				\4[] En función de:
				\4[] $\to$ Forma de la regresión
				\4[] $\to$ Variables introducidas en la regresión
				\4 Si variable es no siempre significativa
				\4[] $\to$ Dudas acerca de su relevancia
				\4 Método ``extreme-bounds'' o límites extremos
				\4[] Estimar regresiones con variable
				\4[] Calcular intervalos de confianza para todas
				\4[] Si límite inferior < 0  y superior > 0
				\4[] $\then$ No es un determinante robusto
				\4 Sala-i-Martin (1997)
				\4[] Método extreme-bounds
				\4[] $\to$ demasiado restrictivo
				\4[] Propone método alternativo
				\4[] Comparar intervalos positivo y negativo
				\4[] $\to$ Valorar en función de cantidad dentro y fuera
				\4[] Estima varios millones de regresiones
				\4[] $\to$ Combinando varias decenas de variables
				\4[] Encuentra 22 variables con efectos robustos\footnote{Robustos a la formulación de la regresión.}
				\4[] $\to$ Regionales
				\4[] $\to$ Políticas
				\4[] $\to$ Religiosas
				\4[] $\to$ Distorsiones de mercado
				\4[] $\to$ Inversión
				\4[] $\to$ Producción del sector primario
				\4[] $\to$ Apertura al comercio internacional
				\4[] $\to$ Grado de capitalismo
				\4[] $\to$ Origen colonial
			\3 Datos de panel vs sección cruzada
				\4 Sección cruzada:
				\4[] observaciones limitadas
				\4[] Evolución temporal no se tiene en cuenta
				\4 Datos de panel
				\4[] Periodos temporales son relevantes
				\4[] $\to$ Pueden controlarse
				\4[] Aumenta el número de observacioens
				\4[] Modelos de efectos fijos
				\4[] $\to$ Controlar idiosincrasias del país
				\4[] $\to$ Diferencias en diferencias
			\3 Multicolinealidad
				\4 Subconjunto de variables explicativas
				\4[] Muy altamente correlacionadas
				\4 Difícil separar efectos individuales
		\2 Evidencia empírica
			\3 Artículos importantes
				\4 Barro (1991), (1997)
				\4 Barro y Lee (1994)
				\4 Sala-i-Martin (1997)
				\4 Acemoglu, Johnson y Robinson (2000)
			\3 Geografía
				\4 Salida al mar
				\4[] Positivamente con crecimiento
				\4 Mortalidad, enfermedades infecciosas
				\4[] Negativamente con crecimiento
			\3 Capital humano
				\4 Barro (1991)
				\4 Factor importante de convergencia
				\4[] Convergencia condicional
				\4[] $\to$ Si se controla por capital humano
				\4 Relación positiva con crecimiento
				\4[] Capital humano elevado
				\4[] $\to$ Convergencia mucho más rápida
			\3 Consumo público
				\4 Generalmente, relación negativa
				\4[] Crecimiento del PIB
				\4[] Inversión privada\footnote{¿Crowding out?.}
				\4[] $\then$ Introducción de distorsiones
			\3 Inversión pública
				\4 Poco relacionada con crecimiento
			\3 Demografía
				\4 Esperanza de vida
				\4[] Positivamente con crecimiento
				\4 Fertilidad
				\4[] Negativamente con crecimiento
			\3 Instituciones
				\4 Derechos de propiedad protegidos
				\4[] Positivamente con crecimiento
				\4 Desarrollo financiero
				\4[] Positivamente con crecimiento
				\4 Inestabilidad política
				\4[] Medida con:
				\4[] $\to$ Golpes de estado
				\4[] $\to$ Asesinatos políticos
				\4[] $\to$ Revoluciones
				\4[] Relacionada negativamente con crecimiento
				\4[] Muy difícil establecer causalidad
				\4[] ¿Economía causa inestabilidad?
				\4[] ¿Estabilidad causa economía?
				\4 Acemoglu, Johnson y Robinson (2001)
				\4[] Mortalidad de colonos como instrumento
				\4[] $\to$ De buenas instituciones
				\4[] $\then$ Tratar de evitar causalidad inversa
				\4[] Alta mortalidad
				\4[] $\to$ Énfasis en extracción de recursos
				\4[] Resultado: causalidad fuerte
				\4[] $\to$ Buenas instituciones a crecimiento
			\3 Desigualdad
				\4 Curva de Kuznets
				\4[] Evidencia empírica relativamente robusta
				\4[] Crecimiento correlacionado con $\uparrow$ desigualdad
				\4[] $\to$ Posterior reducción
				\4[] $\then$ Forma de U inversa
				\4 Desigualdad sobre crecimiento
				\4[] Generalmente, negativa
				\4 Crecimiento sobre desigualdad
				\4[] Poca evidencia de relación
				\4[] ``high-tide-lifts-all boats''
				\4[] $\to$ Todos se benefician
				\4[] $\to$ Posición relativa se mantiene
			\3 Democracia o dictadura
				\4 Resultados mixtos
				\4 Giavazzi y Tabellini (2005)
				\4[] Liberalización y democracia
				\4[] $\to$ Relación compleja con crecimiento
				\4[] Democracia después que liberalización
				\4[] $\to$ Poco positiva
				\4[] Liberalización después que democracia
				\4[] $\to$ Muy negativa
				\4[] Liberalización en países que democratizan
				\4[] $\to$ Muy positiva
				\4[] Liberalización en países que no democratizan
				\4[] $\to$ Positiva
				\4[] Democratización en países que nunca liberalizan
				\4[] $\to$ Poco positiva
				\4[] Democratización en países que liberalizan
				\4[] $\to$ Positiva
				\4[] Conclusión:
				\4[] Mejor liberalizar primero
				\4[] $\to$ Democratizar después
				\4 Persson y Tabellini (2006)
				\4[] Tipo de democracia y crecimiento
				\4[] Regímenes presidencialistas
				\4[] $\to$ Más $\Delta$ \% que DParlamentaria
				\4 Acemoglu, Naidu, Restrepo y Robinson (2019)
				\4[] Democracia sí causa crecimiento
				\4[] En el largo plazo, hasta $+20\%$
				\4[] Aumento del crecimiento vía:
				\4[] $\to$ Incentivos a la inversión
				\4[] $\to$ Más escolarización
				\4[] $\to$ Reformas económicas
				\4[] $\to$ Provisión de bienes públicos
				\4[] $\to$ Menos conflictividad social
			\3 Crecimiento endógeno
				\4 Mostrar no convergencia
				\4[] Estrategia básica de contrastación
				\4 Decisión de política económica
				\4[] ¿Influye crecimiento a l/p?
				\4[] Respuesta favorable:
				\4[] $\to$ Interpretable como CEndógeno
				\4 Resultados mixtos
				\4[] Evans (1992)
				\4[] $\to$ Sin efecto de PPúblicas a l/p
				\4[] Kocherlakota y Yi (1997)
				\4[] $\to$ Rechazan crec. sea exógeno a PPúblicas
		\2 Efectos del comercio sobre el crecimiento
			\3 Dirección de la causalidad
				\4 Frankel y Romer (1999)
				\4 Dirección de la causalidad difícil de distinguir
				\4[] Comercio causa crecimiento
				\4[] $\to$ Por alguna de las vías mencionadas anteriormente
				\4[] Crecimiento causa comercio
				\4[] $\to$ Porque comercio tiene = determinantes que crecimiento
				\4 Ejemplo:
				\4[] Países que liberalizan comercio interno
				\4[] $\to$ Liberalizan también comercio exterior
				\4[] Liberalización de comercio interior y exterior
				\4[] $\to$ Afecta crecimiento y comercio a la vez
				\4[] $\then$ aparece correlación comercio-crecimiento
				\4 Necesario estimar instrumento alternativo
				\4[] Determinante de comercio
				\4[] $\to$ Que no dependa de decisiones de PE
				\4[] $\then$ Relacionar instrumento con crecimiento
				\4 Modelos de gravedad
				\4[] Explicar comercio como resultado de:
				\4[] $\to$ Tamaño relativo
				\4[] $\to$ Distancia
				\4 Regresión
				\4[] Crecimiento contra instrumento de comercio
				\4[] $\to$ Estimado mediante modelo de gravedad
				\4 Resultados
				\4[] Comercio aumenta crecimiento
				\4[] $\to$ No por causas comunes de crecimiento y comercio
				\4[] Comercio interno aumenta crecimiento
				\4[] Resultados robustos a cambios en formulación
			\3 Efectos escala
				\4 Modelos de crecimiento endógeno
				\4[] A menudo predicen relación entre
				\4[] $\to$ Tamaño de la economía
				\4[] $\then$ Comercio ``integra'' economías
				\4[] $\to$ Tasa de crecimiento
				\4 Muy largo plazo
				\4[] Indicios favorables
				\4[] Kremer (1993)
				\4[] $\to$ Un millón de años hasta hoy
				\4 Corto plazo
				\4[] Pocos indicios favorables
			\3 Transferencia vía exportaciones e importaciones
				\4 Proveedores aprenden de clientes
				\4[] Demandas de clientes transfieren tecnología
				\4[] $\to$ Evidencia favorable
				\4 Importaciones de productos con tec. más avanzada
				\4[] Poca evidencia de que aumenten crecimiento
			\3 Spillovers tecnológicos
				\4 Instrumento de estimación
				\4[] Estimar medidas de gasto en I+D
				\4[] $\to$ De importadores y exportadores
				\4[] Ponderar medidas de gasto en I+D
				\4[] $\to$ Por volumen de mportaciones y exportaciones
				\4 Objetivo
				\4[] Relacionar crecimiento de TFP con I+D ponderando:
				\4[] $\to$ Volumen de comercio sobre total
				\4[] $\to$ Cercanía geográfica
				\4 Ponderando por volumen de comercio
				\4[] Relación $\Delta$ TFP con i+D de importación
				\4[] $\to$ Relación pequeña o poco significativa
				\4[] Relación $\Delta$ TFP con i+D de exportación
				\4[] $\to$ Relación significativa
				\4[] $\then$ Exportadores aprenden de sus clientes
				\4[] $\then$ Clientes no aprenden mucho de sus proveedores
				\4 Ponderando por distancia geográfica
				\4[] Relación significativa
				\4[] $\to$ Debilita conclusión respecto volumen de comercio
				\4[] $\then$ ¿Comercian más porque están más cerca?
				\4 Dirección de los spillovers
				\4[] ¿Son simétricos entre PEDs y desarrollados?
				\4[] Evidencia apunta a asimetría
				\4[] $\to$ De países más avanzados hacia menos
				\4[] IDE también juega papel importante
			\3 Convergencia
				\4 Apertura al comercio influye en convergencia
				\4[] Evidencia favorable
				\4[] Países que comercian entre sí
				\4[] $\to$ Más velocidad de convergencia entre sí
				\4 Países cerrados al comercio
				\4[] Evidencia contraria a convergencia
			\3 Otros factores
				\4 Rent-seeking
				\4[] $\to$ Estudiantes de derecho vs ingeniería
	\1[] \marcar{Conclusión}
		\2 Recapitulación
			\3 Contabilidad del crecimiento
			\3 Convergencia
			\3 Impacto de los distintos factores
		\2 Idea final
			\3 Fenómenos recientes
				\4 Aceleración del crecimiento
				\4[] BRICS
				\4[] Algunos africanos
				\4[] Otros emergentes
				\4 Trampas de ingreso medio
				\4 Cambio climático y crecimiento
			\3 Difícil pensar en otra cosa
				\4 Lucas (1988)
				\4[] Cuando se empieza a pensar en crecimiento...
				\4[] ...difícil pensar en otra cosa
				\4 Cientos de estudios y regresiones
				\4[] Contabilizar efecto de factores
				\4[] Encontrar causas profundas del crecimiento
				\4 Necesario mantener:
				\4[] Simplicidad de los modelos
				\4[] Parsimonia
				\4[] Tratabilidad
				\4[] $\then$ Si no, imposible formular PEconómica
\end{esquemal}

































\graficas

\begin{dibujo}{4}{Representación gráfica de la insuficiencia de la $\beta$-convergencia para que tenga también lugar $\sigma$-convergencia.}{x}{y}{betasigmaconvergencia}
	% GRÁFICA IZQUIERDA: BETA Y SIGMA CONVERGENCIA
	\node[below] at (-4,-1){$\beta$-convergencia y $\sigma$-convergencia};
	
	
	\draw[-] (-6,4) -- (-6,0) -- (-2,0);
	\node[below] at (-2,0){$t$};
	\node[left] at (-6,4){$\ln y$};
	
	% País rico
	\draw[-] (-5.5,3) -- (-2.5,3.5);
	
	% País pobre
	\draw[-] (-5.5,0.5) -- (-2.5, 3.2);
	
	% GRÁFICA DERECHA: BETA CONVERGENCIA SIN SIGMA CONVERGENCIA
	\node[below] at (4,-1){$\beta$-convergencia sin $\sigma$-convergencia};
	
	
	\draw[-] (2,4) -- (2,0) -- (6,0);
	\node[below] at (2,0){$t$};
	\node[left] at (2,4){$\ln y$};
	
	% País rico
	\draw[-] (2.5,2.5) -- (5.5,1);
	
	% País pobre
	\draw[-] (2.5,1.5) -- (5.5,3);
	
\end{dibujo}

\preguntas

\seccion{Test 2018}

\textbf{22.} ¿Cuál de las siguientes afirmaciones es \underline{\textbf{CORRECTA}}?

\begin{itemize}
	\item[a] La $\beta$-convergencia es una condición necesaria y suficiente para la existencia de $\sigma$-convergencia.
	\item[b] La $\sigma$-convergencia es una condición necesaria pero no suficiente para la existencia de $\beta$-convergencia.
	\item[c] La $\beta$-convergencia no es una condición necesaria ni suficiente para la existencia de $\sigma$-convergencia.
	\item[d] Es posible que se dé la $\beta$-convergencia a la vez que aumenta la desigualdad entre países. 
\end{itemize}



\seccion{Test 2007}

\textbf{15.} En una función de producción Cobb-Douglas, la tasa de crecimiento de la productividad total de los factores puede medirse como:

\begin{itemize}
	\item[a] La tasa de crecimiento de la producción por la diferencia entre la tasa de crecimiento del PIB y la suma de las tasas de crecimiento del stock de capital y de la fuerza de trabajo.
	\item[b] la diferencia entre la tasa de crecimiento del PIB y la suma de la tasa de crecimiento del stock de capital multiplicada por la participación del capital en la producción y de la tasa de crecimiento de la fuerza de trabajo multiplicada por la participación del trabajo en el PIB.
	\item[c] la suma de las tasas de crecimiento del stock de capital y de la fuerza de trabajo.
	\item[d] la suma de la tasa de crecimiento del stock de capital multiplicada por la participación del capital en la producción y de la tasa de crecimiento de la fuerza de trabajo multiplicada por la participación del trabajo en el PIB.
\end{itemize}

\notas

Ver Kremer (1990) sobre crecimiento en el muy largo plazo

\textbf{2018:} \textbf{22.} D

\textbf{2007:} \textbf{15.} B

\bibliografia

Mirar en Palgrave:
\begin{itemize}
    \item convergence *
    \item convergence hypothesis *
    \item economic growth *
    \item economic growth in the very long run *
    \item economic growth non-linearities
    \item economic growth, empirical regularities in *
    \item endogenous growth
    \item growth accounting *
    \item growth and civil war 
    \item growth and cycles *
    \item growth and inequality 
    \item growth and institutions *
    \item growth and international trade *
    \item growth models, multisector
    \item growth take-offs *
    \item immiserizing growth
    \item import substitution and export-led growth
    \item human capital, fertility and growth
    \item infrastructure and growth *
    \item inflation and growth
    \item level accounting
    \item limits to growth *
    \item long swings in economic growth *
    \item measurement of economic growth *
    \item multisector growth models 
    \item national leadership and economic growth
    \item population and agricultural growth
    \item Solow residual *
    \item total factor productivity *
    \item trade, technology diffusion and growth
    \item urban growth
\end{itemize}

Acemoglu, D. \textit{Introduction to Modern Economic Growth} (2009) -- En carpeta de crecimiento económico

Aghion, P.; Howitt, P. \textit{A Model of Growth Through Creative Destruction} (1992) Econometrica -- En carpeta del tema

Acemoglu, D.; Naidu, S.; Restrepo, P.; Robinson, J. A. (2019)

Barro, R. J. \textit{Economic Growth in a Cross Section of Countries} (1991) The Quarterly Journal of Economics -- En carpeta del tema

Barro, R. J.; Sala-i-Martin, X. \textit{Economic Growth} (2004) 2nd Edition -- En carpeta de crecimiento económico

Duprey, J. N. \textit{The Search for a Stable Money Demand Equation} (1980) Quarterly Review Federal Reserve Bank of Minneapolis -- En carpeta del tema

Johnson, P.; Papageorgiou (2010) \textit{What remains of Cross-Country Convergence?} Journal of Economic Literature -- En carpeta del tema

\textbf{Jones, C. I. (2016) \textit{The Facts of Economic Growth} Ch. 1 in Handbook of Macroeconomics II -- En carpeta Libros/Macro}

Hsieh, C. T. \textit{What Explains the Industrial Revolution in East Asia?} (2002) American Economic Review -- En carpeta del tema

Huggett, M. \textit{Growth Accounting} (2018) Georgetown University -- En carpeta del tema. \url{http://faculty.georgetown.edu/mh5/class/econ102/lecture/growthaccounting-lecture.pdf}

ILO, OCDE \textit{The Labour Share in G20 Economies} (2015) Report for the G20 Employment Working Group -- En carpeta del tema

Kador, N. \textit{Capital Accumulation and Economic Growth} (1963) Seminar on the Programming of Economic Development -- En carpeta del tema

Quah, D. T. \textit{Empirics for economic growth and converge} (1996) European Economic Review -- En carpeta del tema

Romer, D. \textit{Advanced Macroeconomics (4th ed)}. Ch. 1, 3, 4

Sala-i-Martin, X. \textit{I Just Ran Four Million Regressions} (1997) NBER Working Papers -- En carpeta del tema


\end{document}
