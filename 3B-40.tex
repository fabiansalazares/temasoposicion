\documentclass{nuevotema}

\tema{3B-40}
\titulo{La Unión Europea: El Mercado Interior. El principio de libre circulación de mercancías, servicios, personas y capitales. La política de competencia.}

\begin{document}

\ideaclave

La UE tiene como objetivo garantizar la paz y el bienestar de sus estados miembros. Para ello, se fija una serie de objetivos intermedios. Entre ellos se encuentra la integración económica, cuya herramienta básica es el mercado común. Es decir, la eliminación de todas las barreras al comercio de bienes y servicios, así como de factores de producción. O de otra manera, que no existan mercados fragmentados sino que la Unión Europea sea un sólo gran mercado. En la literatura económica y la doctrina jurídica europea se habla de las \textit{cuatro libertades} para hacer referencia a los parámetros de actuación de la UE para lograr este objetivo de unificación del mercado

El objeto del tema es exponer las políticas que la Unión lleva a cabo en este sentido. Se comienza por mostrar brevemente los beneficios del mercado común que sirven de justificación para su puesta en práctica. Posteriormente, se explican las políticas de la unión por campo de actuación. Para entender éstas es necesario entender su evolución, que va ligada a la composición de la Unión Europa y el contexto económico internacional. A partir de esta evolución, y dividiendo la presentación por áreas de actuación (bienes, servicios, personas, capitales, fiscalidad, competencia) se plantea la situación actual. Para concluir, se recapitula lo examinado y se realiza un breve balance y una breve exposición de las perspectivas de futuro.

\seccion{Preguntas clave}

\begin{itemize}
	\item ¿Qué es el mercado interior?
	\item ¿Por qué es necesario?
	\item ¿Qué políticas se han llevado a cabo hasta la fecha?
	\item ¿En qué situación se encuentra?
\end{itemize}

\esquemacorto

\begin{esquema}[enumerate]
	\1[] \marcar{Introducción}
		\2 Contextualización
			\3 Unión Europea
			\3 Competencias de la UE
			\3 Mercado interior de la UE
			\3 Obstáculos a la integración
			\3 Efectos del mercado único
		\2 Objeto
			\3 ¿En qué consiste el mercado interior de la UE?
			\3 ¿Por qué es necesario?
			\3 ¿Qué políticas se llevan a cabo para hacer efectivo el mercado interior?
			\3 ¿En qué situación se encuentra el mercado interior actualmente?
			\3 ¿Qué políticas de competencia implementa la UE?
		\2 Estructura
			\3 Mercancías
			\3 Servicios
			\3 Personas
			\3 Capitales
			\3 Otros componentes
	\1 \marcar{Mercancías}
		\2 Justificación
			\3 Especialización
			\3 Aprovechamiento de economías de escala
			\3 Variedad de inputs
			\3 Persistencia de barreras
			\3 Distorsiones del mercado único
		\2 Objetivos
			\3 Eliminar barreras físicas
			\3 Eliminar barreras técnicas
			\3 Armonización de regulaciones
			\3 Conectar mercados
		\2 Antecedentes
			\3 Tratado de Roma 1957
			\3 Culminación de UA en 1968
			\3 Años 70: barreras no arancelarias
			\3 Acta Única Europea (1986)
			\3 Informe Cecchini (1988)
			\3 Informe de la Comisión Europea (1996)
			\3 Comisión Europea (2002)
			\3 Barreras técnicas y legales persistentes
		\2 Marco jurídico
			\3 TFUE
			\3 Código Aduanero Europeo
			\3 Jurisprudencia del TJUE
			\3 Reglamento de Reconocimiento Mutuo de 2008
			\3 Directiva 1535/2015 de Notificación de Regulación Técnicas
			\3 Reglamento de vigilancia de mercado 2019/1020
			\3 Reglamento de Reconocimiento Mutuo 2019/515
		\2 Actuaciones
			\3 Eliminación de barreras físicas
			\3 Excepciones a libre circulación
			\3 Armonización regulatoria positiva
			\3 Enfoque tradicional de armonización positiva
			\3 Enfoque moderno de armonización positiva
			\3 Armonización regulatoria negativa
		\2 Valoración
			\3 Comercio extra-UE
			\3 Aumento del comercio intra-UE
		\2 Retos
			\3 Interacción con políticas de transporte
			\3 Dinámicas centro-periferia
			\3 Diversificación intra-UE
			\3 Efectos frontera persistentes
			\3 Comercio extra-europeo
			\3 Inflación regulatoria
	\1 \marcar{Servicios}
		\2 Justificación
			\3 Importancia cuantitativa
			\3 Especialización
			\3 Persistencia de barreras
		\2 Objetivos
			\3 Libre prestación de servicios
			\3 Libre establecimiento
			\3 Igualdad de trato
		\2 Antecedentes
			\3 Tratado de Roma de 1957
			\3 Libro Blanco sobre el Mercado Interior de 1985
			\3 Acta Única Europea de 1987
			\3 Años 90
			\3 Años 2000: Estrategia para el Mercado de Servicios
			\3 Directiva Bolkenstein (2004)
			\3 Directiva 2006/123 de prestación de servicios
		\2 Marco jurídico
			\3 tue 3.3
			\3 tfue
			\3 directiva 123/2006 de servicios
			\3 ámbitos excluidos de directiva de servicios
			\3 directiva 2011/83 de protección del consumidor
			\3 propuesta de 2017: paquete de servicios
			\3 regulaciones sectoriales respectivas
		\2 Actuaciones
			\3 Restricciones de regímenes de autorización
			\3 Ventanillas únicas electrónicas
			\3 Prohibición de requisitos discriminatorios
			\3 Efectos transfronterizos de documentos
			\3 Derechos de los consumidores
			\3 Cooperación administrativa
			\3 Evaluación mutua entre EEMM
			\3 Notificación de borradores legislativos
			\3 DSM -- Mercado único digital
			\3 Directiva de servicios audivisuales de 2018
		\2 Valoración
			\3 Reservas de actividad
			\3 Escasa integración
			\3 Oposición opinión pública
		\2 Retos
			\3 Barreras no regulatorias
			\3 Transferencias de soberanía
	\1 \marcar{Personas}
		\2 Justificación
			\3 Especialización
			\3 Aprovechamiento de recursos humanos
			\3 Existencia de muy diferentes regímenes
			\3 Shocks asimétricos en EEMM
		\2 Objetivos
			\3 Permitir ajuste ante shocks adversos
			\3 Facilitar movilidad intra-europea
			\3 Reducir efectos adversos de migraciones
			\3 Mantener garantías de seguridad
		\2 Antecedentes
			\3 T. de Roma (1957)
			\3 Acuerdo de Schengen de 1985
			\3 Convención de Schengen de 1990
			\3 Tratado de Maastricht (1992):
			\3 Tratado de Amsterdam (1997)
		\2 Marco jurídico
			\3 TUE
			\3 TFUE
			\3 Directiva 38/2004 sobre libre circulación de personas
			\3 Acuerdo y Convención Schengen
			\3 Reglamento de Dublin 2013/604
		\2 Actuaciones
			\3 Igualdad de trato
			\3 Libre circulación y residencia
			\3 Reconocimiento mutuo de profesiones
			\3 Espacio Schengen
			\3 Restricciones a la libertad de movimientos
		\2 Valoración
			\3 Este--oeste
			\3 Sur--norte
		\2 Retos
			\3 Interacción con otras políticas
			\3 Economía política de la migración
			\3 Propuestas de profundización
			\3 Seguridad
			\3 Pandemia 2020
	\1 \marcar{Capitales}
		\2 Justificación
			\3 Asignación eficiente del capital
			\3 Capital escaso en algunos países
			\3 Suavizar shocks asimétricos
		\2 Objetivos
			\3 Eliminar obstáculos a movilidad de K
			\3 Integrar mercados financieros europeos
			\3 Aumentar diversificación de inversiones
			\3 Canalizar ahorro a proyectos rentables
		\2 Antecedentes
			\3 Tratado de Roma de 1957
			\3 Directiva de liberalización de 1960
			\3 Directivas de liberalización de los 80
			\3 Primera fase de la UEM: directiva de 1988
			\3 Tratado de Maastricht: prohibición de restricciones
		\2 Marco jurídico
			\3 TUE y TFUE
			\3 Acuerdos de Basilea
			\3 Single Rulebook
			\3 Directivas de seguros
			\3 MiFID II y MiFIR
			\3 PSD2 (2015) -- Payment Services Directive
		\2 Actuaciones
			\3 Prohibición de obstáculos a la mov. de K
			\3 Supervisión y regulación financiera
			\3 Unión Bancaria
			\3 SEPA
			\3 CMU -- Unión del Mercado de Capitales
			\3 Reglamento sobre titulizaciones y STS de 2017
			\3 Next CMU High-Level Expert Group
			\3 PSD2 -- Directiva Europea de Servicios de Pago
			\3 EMIR -- European Market Infrastructure Regulation
			\3 TARGET 2-Securities
		\2 Valoración
			\3 Excesiva intermediación
			\3 Predominio financiación bancaria
			\3 Reformas lentas
			\3 Aumento de bancarización en últimos años
			\3 Brexit
		\2 Retos
			\3 Diversificación de inversiones
			\3 Regulación macroprudencial
			\3 Transparencia
			\3 Deuda pública
	\1 \marcar{Competencia}
		\2 Justificación
			\3 Economías de escala y poder de mercado
			\3 Regulación nacional de competencia
		\2 Objetivos
			\3 Evitar abusos de poder de mercado
			\3 Incentivar competencia entre empresas
			\3 Evitar distorsiones en el mercado único
			\3 Reducir race-to-the-bottom regulatorio
		\2 Marco jurídico
			\3 TFUE.3
			\3 Reglamento de control de concentraciones de 2004
		\2 Actuaciones
			\3 Acuerdos entre empresas
			\3 Abuso de posición de dominio
			\3 Control de concentraciones
			\3 Ayudas públicas
	\1 \marcar{Otros componentes del mercado interior}
		\2 Propiedad intelectual
			\3 Justificación
			\3 Objetivos
			\3 Actuaciones
			\3 Valoración
			\3 Retos
		\2 Contratación pública
			\3 Justificación
			\3 Objetivos
			\3 Antecedentes
			\3 Marco jurídico
			\3 Actuaciones
			\3 Valoración
			\3 Retos
		\2 Fiscalidad indirecta
			\3 Justificación
			\3 Objetivos
			\3 Actuaciones
		\2 Fiscalidad directa
			\3 Justificación
			\3 Objetivos
			\3 Actuaciones
		\2 Programa Fiscalis 2020
			\3 Idea clave
			\3 Actuaciones
			\3 Valoración
	\1[] \marcar{Conclusión}
		\2 Recapitulación
			\3 Mercancías
			\3 Servicios
			\3 Personas
			\3 Capitales
			\3 Otros componentes
		\2 Idea final
			\3 El mercado interior hoy
			\3 Grado de integración
			\3 Brexit
			\3 WTO y comercio mundial

\end{esquema}

\esquemalargo








































































\begin{esquemal}
	\1[] \marcar{Introducción}
		\2 Contextualización
			\3 Unión Europea
				\4 Institución supranacional ad-hoc
				\4[] Diferente de otras instituciones internacionales
				\4[] Medio camino entre:
				\4[] $\to$ Federación
				\4[] $\to$ Confederación
				\4[] $\to$ Alianza de estados-nación
				\4 Origen de la UE
				\4[] Tras dos guerras mundiales en tres décadas
				\4[] $\to$ Cientos de millones de muertos
				\4[] $\to$ Destrucción económica
				\4[] Marco de integración entre naciones y pueblos
				\4[] $\to$ Evitar nuevas guerras
				\4[] $\to$ Maximizar prosperidad económica
				\4[] $\to$ Frenar expansión soviética
				\4 Objetivos de la UE
				\4[] TUE -- Tratado de la Unión Europea
				\4[] $\to$ Primera versión: Maastricht 91 $\to$ 93
				\4[] $\to$ Última reforma: Lisboa 2007 $\to$ 2009
				\4[] Artículo 3
				\4[] $\to$ Promover la paz y el bienestar
				\4[] $\to$ Área de seguridad, paz y justicia s/ fronteras internas
				\4[] $\to$ Mercado interior
				\4[] $\to$ Crecimiento económico y estabilidad de precios
				\4[] $\to$ Economía social de mercado
				\4[] $\to$ Pleno empleo
				\4[] $\to$ Protección del medio ambiente
				\4[] $\to$ Diversidad cultural y lingüistica
				\4[] $\to$ Unión Económica y Monetaria con €
				\4[] $\to$ Promoción de valores europeos
				\4[$\to$] Objetivos de la UE
				\4[] Paz y bienestar a pueblos de Europa
			\3 Competencias de la UE
				\4 Tratado de la Unión Europea
				\4[] Atribución
				\4[] $\to$ Sólo las que estén atribuidas a la UE
				\4[] Subsidiariedad
				\4[] $\to$ Si no puede hacerse mejor por EEMM y regiones
				\4[] Proporcionalidad
				\4[] $\to$ Sólo en la medida de lo necesario para objetivos
				\4 Exclusivas
				\4[] i. Política comercial común
				\4[] ii. Política monetaria de la UEM
				\4[] iii. Unión Aduanera
				\4[] iv. Competencia para el mercado interior
				\4[] v. Conservación recursos biológicos en PPC
				\4 Compartidas
				\4[] i. Mercado interior
				\4[] ii. Política social
				\4[] iii. Cohesión económica, social y territorial
				\4[] iv. Agricultura y pesca \footnote{Salvo en lo relativo a la conservación de recursos biológicos marinos, que se trata de una competencia exclusiva de la UE}
				\4[] v. Medio ambiente
				\4[] vi. Protección del consumidor
				\4[] vii. Transporte
				\4[] viii. Redes Trans-Europeas
				\4[] ix. Energía
				\4[] x. Área de libertad, seguridad y justicia
				\4[] xi. Salud pública común en lo definido en TFUE
				\4 De apoyo
				\4[] i. Protección y mejora de la salud humana
				\4[] ii. Industria
				\4[] iii. Cultura
				\4[] iv. Turismo
				\4[] v. Educación, formación profesional y juventud
				\4[] vi. Protección civil
				\4[] vii. Cooperación administrativa
				\4 Coordinación de políticas y competencias
				\4[] Política económica
				\4[] Políticas de empleo
				\4[] Política social
			\3 Mercado interior de la UE
				\4 ``Joya de la corona de la UE''
				\4[] Política europea con mayores efectos
				\4[] $\to$ Junto con moneda única
				\4 Principal elemento de cohesión europea
				\4[] Enormes costes de separación potencial
				\4[] Aumento del crecimiento
				\4 Dimensiones esenciales
				\4[] $\to$ Mercados de ByS
				\4[] $\to$ Factores de producción
				\4[] $\to$ Competencia
				\4[] $\to$ Fiscalidad, contratación pública
				\4[] $\then$ 4 libertades + aspectos accesorios
			\3 Obstáculos a la integración
				\4 Diferencias regulatorias
				\4 Economía política
				\4 Desestabilización mercados
				\4 Diferencias culturales
				\4 Costes de transporte
				\4 Dinámicas de aglomeración centro-periferia
			\3 Efectos del mercado único
				\4 Creación y desviación de comercio
				\4 Aumento de renta disponible
				\4 Aumento de variedades disponibles
				\4 Fenómenos de economía política
				\4 Diversificación y especialización
				\4 Aglomeración y dispersión
				\4 Difusión de tecnología
				\4 Dinámicas de especialización
		\2 Objeto
			\3 ¿En qué consiste el mercado interior de la UE?
			\3 ¿Por qué es necesario?
			\3 ¿Qué políticas se llevan a cabo para hacer efectivo el mercado interior?
			\3 ¿En qué situación se encuentra el mercado interior actualmente?
			\3 ¿Qué políticas de competencia implementa la UE?
		\2 Estructura
			\3 Mercancías
			\3 Servicios
			\3 Personas
			\3 Capitales
			\3 Otros componentes
	\1 \marcar{Mercancías}
		\2 Justificación
			\3 Especialización
				\4 Reducción de costes
				\4 Aumento de calidad
			\3 Aprovechamiento de economías de escala
				\4 Acceso a mercados más grandes
				\4[] Mayores demandas
				\4[] $\to$ Reducción de costes
				\4[] $\to$ Mejoras de competitividad ex-UE
			\3 Variedad de inputs
				\4 Diversificación de inputs
				\4 Especialización
				\4 Innovación y transferencia tecnológica
				\4[$\then$] Más variedad aumenta crecimiento
			\3 Persistencia de barreras
				\4 Eliminación de aranceles no es suficiente
				\4 Otros tipos de barreras
				\4[] Técnicas
				\4[] Legales
				\4[] Fiscales
				\4[] Licitación pública
				\4 Impacto enorme de barreras técnicas
				\4[] Pueden ser más gravosas que aranceles
			\3 Distorsiones del mercado único
				\4 Distorsiones a la movilidad de bienes
				\4[] Pondrían en peligro mercado único
				\4[] Incentivos unilaterales a preteger mercados
		\2 Objetivos
			\3 Eliminar barreras físicas
				\4 Todavía existían a finales de los 80
				\4 Prácticamente eliminadas
			\3 Eliminar barreras técnicas
				\4 Persisten entre estados e incluso regiones
			\3 Armonización de regulaciones
				\4 Facilitar cumplimiento de requisitos
				\4[] Cumplir en un EM implique cumplir en resto
			\3 Conectar mercados
				\4 Facilitar acceso a tráfico de mercancías
				\4 Relación con redes de transporte a nivel europeo
		\2 Antecedentes
			\3 Tratado de Roma 1957
				\4 Unión Aduanera como objetivo en 12 años
			\3 Culminación de UA en 1968
				\4 Antes de plazo
				\4 Arancel exterior común
				\4[] $\to$ Media de aranceles ITA, FRA, GER, BENELUX
				\4[] $\to$ Obligación de no introducir nuevas restricciones aduaneras
			\3 Años 70: barreras no arancelarias
				\4 Escasos avances hacia integración
				\4 Persistencia de barreras no arancelarias
				\4 Políticas públicas intervencionistas
			\3 Acta Única Europea (1986)
				\4 En vigor en 1987
				\4 En 80s, toma de conciencia
				\4[] Contexto de ``eurosclerosis''
				\4[] $\to$ Insuficiente integración
				\4[] $\then$ Impulso a integración económica
				\4 Libro Blanco sobre el Mercado Interior de 1985
				\4[] Detalle de restricciones existentes
				\4[] Propone armonización y liberalización adicional
				\4 Acta Única entra en vigor en 1987
				\4 Mercado Interior completo para principio de  1993
			\3 Informe Cecchini (1988)
				\4 Análisis de costes de no integración
				\4[] Incluye potencial
				\4 Conclusiones
				\4[] Integración potencialmente muy beneficiosa
				\4[] 4\% al 7\% del PIB
				\4[] 2 a 5 millones de trabajos
			\3 Informe de la Comisión Europea (1996)
				\4 1\% PIB
				\4 0.5 millones de trabajos
			\3 Comisión Europea (2002)
				\4 1.8\% del PIB
				\4 2.5 millones de trabajos (1.46\%)
			\3 Barreras técnicas y legales persistentes
				\4 Generalizadas a pesar de esfuerzos
		\2 Marco jurídico
			\3 TFUE
				\4 Artículo 4
				\4[] Mercado único es competencia compartida
				\4 Art. 18
				\4[] Discriminación nacional--extranjero prohibida
				\4 Art. 28, 34, 35
				\4[] Normas comerciales de EEMM no pueden obstaculizar
				\4 Art. 36
				\4[] Excepciones a prohibiciones anteriores
				\4[] $\to$ Razones de orden público
				\4[] $\to$ Moralidad y seguridad pública
				\4[] $\to$ Protección de salud y vida
				\4[] $\to$ Protección prop. intelectual, patrimonio
			\3 Código Aduanero Europeo
				\4 TARIC -- Tarif Integré de la Communauté
				\4[] Nomenclatura arancelaria
				\4[] $\to$ Designación de mercancías según producto
				\4[] Tipos arancelarios aplicables
				\4[] $\to$ Arancel Aduanero Común en sí
				\4[] $\to$ Media de tipos vigentes de aranceles
			\3 Jurisprudencia del TJUE
				\4 Dassonvile (1974)
				\4[] Prohibición de exacciones y equivalentes
				\4 Cassis de Dijon (1979)
				\4[] Principio de reconocimiento mutuo
				\4[] $\to$ Aunque no exista armonización expresa
			\3 Reglamento de Reconocimiento Mutuo de 2008
				\4 Propuesta de reforma aprobada en 2019
			\3 Directiva 1535/2015 de Notificación de Regulación Técnicas
			\3 Reglamento de vigilancia de mercado 2019/1020\footnote{\url{https://tem.fi/en/new-eu-regulation-on-market-surveillance-and-compliance-of-products}}
				\4 Aplicable a sectores con armonización específica
				\4 Designa operador responsable en UE
				\4[] Para mercancías importadas fuera UE
				\4 Poderes de EEMM para verificar cumplimiento
			\3 Reglamento de Reconocimiento Mutuo 2019/515
				\4 Deroga reglamento de 2008
				\4 Aplicable si no hay armonización a nivel europeo
		\2 Actuaciones
			\3 Eliminación de barreras físicas
				\4 Ajustes fiscales en frontera
				\4[] Nuevo sistema de declaración de IVA necesario
%				\4 Montantes compensatorios de la PAC
%%				\4[] Tasas y subvenciones en frontera
%				\4[] $\to$ Nivelar los precios agrícolas
				\4 Controles veterinarios y sanitarios
				\4[] Armonización de normas sobre controles sanitarios
				\4 Supervisión de licencias de transporte
				\4 Cuotas bilaterales a terceros países
				\4[] Totalmente administradas por PComercial Común
				\4 Requisitos burocráticos en frontera
				\4[] Nuevos sistemas de recogida de datos
				\4[] $\to$ INTRASTAT
			\3 Excepciones a libre circulación
				\4 TFUE. Art. 36
				\4[] Razones de orden público
				\4[] Moralidad y seguridad pública
				\4[] Protección de salud y vida
				\4[] Protección prop. intelectual, patrimonio
			\3 Armonización regulatoria positiva
				\4 Homogeneizar requisitos técnicos
				\4 Adaptar regulación nacional
				\4[] A regulación común europea
				\4[] $\to$ Requiere nueva regulación nacional
				\4 Enfoque tradicional de armonización
				\4[] Armonización detallada de normas técnicas
				\4[] $\to$ Igualar regulación con directivas
			\3 Enfoque tradicional de armonización positiva
				\4 Todos los aspectos a armonizar
				\4[] Se recogen en regulación
				\4 Reglamentos con todas las especificaciones
				\4[] Determinan regulación general
				\4 Sectores en UE
				\4[] Químicos
				\4[] Fertilizantes
				\4[] Cosméticos
				\4[] Vehículos a motor
				\4[] Productos farmacéuticos
				\4[] Otros
			\3 Enfoque moderno de armonización positiva
				\4 Nuevo enfoque de la armonización regulatoria
				\4[] Tras AUE y Libro Blanco Mercado Interior
				\4[] Objetivo
				\4[] $\to$ Reducir recurso a armonización
				\4 Sólo aspectos esenciales en reglamento
				\4[] Seguridad, medioambiente, salud
				\4 Resto de aspectos a regular
				\4[] Determinados en varios estándares
				\4[] Fabricante eligen estándar con el que cumplir
				\4 Certificación por organismo independiente
				\4[] Certifica cumplimiento de estándar
				\4[] $\to$ ``CE'' en dispositivo
				\4 Sectores en UE
				\4[] Construcción
				\4[] Explosivos
				\4[] Maquinaria
				\4[] Dispositivos médicos
				\4[] Juguetes
				\4[] Otros
				\4 Existencia de regulación nacional
				\4[] No impide reconocimiento de regla técnica de otro EM
				\4[] EM de destino puede no extender reconocimiento
				\4[] $\to$ Si demuestra que cumplir su norma es esencial
				\4[] $\to$ Principio de proporcionalidad
				\4 Establecimiento de puntos de contacto
				\4[] Para comprobar autorizaciones de entrada
			\3 Armonización regulatoria negativa
				\4 Especialmente desde años 80s y 90s
				\4 Eliminar regulación que:
				\4[] Restrictiva de comercio interior
				\4[] Introduce barreras al comercio
				\4[] $\to$ Requiere derogación de normas nacionales
				\4 Directiva 1535/2015
				\4[] Obligación de notificar regulaciones técnicas
				\4[] $\to$ A Comisión Europea
				\4[] Periodo mínimo de 3 meses
				\4[] $\to$ Antes de implementar regulaciones
		\2 Valoración
			\3 Comercio extra-UE
				\4 Fuerte aumento en últimas décadas
			\3 Aumento del comercio intra-UE
				\4 En las últimas décadas
				\4[] EEMM comercian más entre sí
				\4 Relación con mercado único
				\4[] Difícilmente cuantificable
				\4[] Múltiples factores
				\4[] $\to$ Euro ha favorecido
				\4[] $\to$ Política Comercial Común favorable
				\4[] $\to$ Cambios tecnológicos y cadenas de valor globales
		\2 Retos
			\3 Interacción con políticas de transporte
				\4 Medios de transporte esenciales para MInterior de bienes
				\4 Programas de inversión en todos los MFP
			\3 Dinámicas centro-periferia
				\4 Krugman (1991) y otros
				\4 Reducción de costes de transporte y otros
				\4[] Puede aumentar incentivos a aglomeración
				\4 Mercado de bienes de alto valor añadido
				\4[] Tendencia a concentrarse en grandes áreas urbanas
				\4[$\then$] Efectos de política regional
				\4[$\then$] Interacción con políticas de cohesión
			\3 Diversificación intra-UE
				\4 Comercio muy concentrado en determinados socios
				\4 Aumentar conectividad periferia
			\3 Efectos frontera persistentes
				\4 Estudios muestran de forma recurrente
				\4[] Incluso sin barreras legales
				\4[] $\to$ Menor comercio transfronterizo
				\4[] Programa de investigación
				\4[] $\to$ ¿En qué consisten exactamente?
			\3 Comercio extra-europeo
				\4 Pros y contras de aumentar
				\4[] Más diversificación
				\4[] $\to$ UE más resistente a shocks europeos
				\4[] Menor impulso a integración europea y mercado interior
				\4[] $\to$ EEMM menos dependientes de UE
			\3 Inflación regulatoria
				\4 Aumento de reglas y normas indiscriminado
				\4 Regulación demasiado compleja
				\4[$\then$] Armonización puede quedar sin efecto
	\1 \marcar{Servicios}
		\2 Justificación
			\3 Importancia cuantitativa
				\4 $\sim$ $70\%$ PIB y empleo
				\4 Enorme peso en empleo
				\4[] En todos los países miembros
				\4[] Especialmente en España
			\3 Especialización
				\4 Similares a bienes
				\4 Servicios es categoría muy heterogénea
				\4[] Algunos muy fácilmente comerciables
				\4[] Otros muy difíciles
			\3 Persistencia de barreras
				\4 Heterogeneidad impide armonización general
				\4 Algunos sectores muy difíciles de liberalizar
				\4[] Oposición política
		\2 Objetivos
			\3 Libre prestación de servicios
				\4 Ofrecer servicios en cualquier EEMM
				\4 Contratar servicios de cualquier EEMM
			\3 Libre establecimiento
				\4 Libertad de establecimiento permanente
			\3 Igualdad de trato
				\4 Sin trato fiscal diferenciado
				\4 Sin restricciones por razón de nacionalidad
		\2 Antecedentes
			\3 Tratado de Roma de 1957
				\4 Libertad de prestación y circulación de servicios
				\4 Libertad de establecimiento
				\4 Persisten limitaciones en muchos sectores
				\4[] Transportes
				\4[] Servicios financieros
				\4[] Monopolios públicos en industrias de red
				\4 Sin verdadera eliminación de obstáculos jurídicos
			\3 Libro Blanco sobre el Mercado Interior de 1985
				\4 Libro Blanco propone:
				\4[] Reconocimiento mutuo de normas nacionales
				\4[] Armonización de principios básicos
			\3 Acta Única Europea de 1987
			\3 Años 90
				\4 Liberalización por sectores
				\4[] Servicios financieros
				\4[] Telecomunicaciones
				\4[] Transporte aéreo
				\4[] Servicios postales
				\4[] Distribución de gas y energía
				\4 Avance gradual
			\3 Años 2000: Estrategia para el Mercado de Servicios
				\4 Enfoque transversal
				\4[] Todos sectores
				\4 Dos fases
				\4[] 1. Inventario de obstáculos
				\4[] 2. Soluciones mediante legislación transversal
			\3 Directiva Bolkenstein (2004)
				\4 Propuesta de directiva
				\4 Eliminación de todas las barreras legales
				\4 Principio de país de origen
				\4[] Normativa del país donde radica la sede
				\4 Armonización selectiva
				\4 Oposición fuerte en EEMM
				\4[] Miedo a deslocalización
				\4[] $\then$ Que genere competencia regulatoria a la baja
				\4[] Admón. pública nacional pierden poder
				\4 EEMM solicitaron sectores quedaran fuera de la norma
			\3 Directiva 2006/123 de prestación de servicios
		\2 Marco jurídico
			\3 tue 3.3
				\4 establecimiento de mercado único
			\3 tfue
				\4 artículo 4
				\4[] competencia compartida sobre mercado interior
				\4 art. 49
				\4[] libre establecimiento de empresas
				\4 arts. 56 a 62
				\4[]prohibición general de restricciones a libre prestación
				\4[] para nacionales de todos eemm
				\4[] $\to$ establecidos en otro em
			\3 directiva 123/2006 de servicios
				\4 entrada en vigor en 2006
				\4 inspirada en bolkenstein
				\4 cubre 46\% de actividades de servicios
			\3 ámbitos excluidos de directiva de servicios
				\4 básicamente:
				\4[] todos los que tienen regulación específica
				\4[] servicios no económicos de interés general
				\4[] $\to$ seguridad, educación, seguridad social...
				\4 servicios financieros
				\4 teleco
				\4 transportes
				\4 trabajo temporal
				\4 servicios sanitarios
				\4 audiovisuales
				\4 juego
				\4 servicios sociales
				\4 seguridad privada
				\4 notaría
			\3 directiva 2011/83 de protección del consumidor
				\4 propuesta de reforma en 2019
			\3 propuesta de 2017: paquete de servicios
				\4 notificación de borradores legislativos nacionales
				\4 análisis de proporcionalidad en regs. nacionales
				\4[] especialmente en servicios profesionales
				\4[] $\to$ necesario test previo para demostrar necesidad
				\4 guía para reformas nacionales de profs. reguladas
				\4 incluye propuesta de e-card
				\4 propuesta rechazada en 2018
			\3 regulaciones sectoriales respectivas
				\4 telecomunicaciones
				\4[] $\to$ neutralidad de la red (2015)
				\4[] $\to$ roaming móvil
				\4 viajes y transportes
				\4[] liberalización tráfico aéreo
				\4[] liberalización ferroviario
				\4[] liberalización carretera
				\4[] protección consumidor vuelos
				\4 energía
				\4[] directivas de liberalización
				\4[] mantenimiento de stocks mínimos
				\4[] intercambio de electricidad
		\2 Actuaciones
			\3 Restricciones de regímenes de autorización
				\4 Autorización sólo permitida si:
				\4[] No discriminatoria
				\4[] Justificada por interés general
				\4[] Objetivo no pueda conseguirse con control ex-post
				\4[$\then$] No discriminación, proporcionalidad, necesidad
				\4 Elimina autorización para receptores de servicios
			\3 Ventanillas únicas electrónicas
				\4 Simplificación de trámites
				\4 Información y tramitación centralizada
				\4 Fuente única de información
				\4[] Sobre trámites administrativos en otros EEMM
			\3 Prohibición de requisitos discriminatorios
				\4 Prohibido discriminar por origen de receptor
				\4 Trato diferencial sólo permitido con excepciones
			\3 Efectos transfronterizos de documentos
				\4 Siempre que tengan función equivalente
			\3 Derechos de los consumidores
				\4 Evitar desprotección ante prestadores extranjeros
				\4 Requisitos mínimos de información
				\4 Vías de reclamación
				\4 En proceso de reforma
				\4 Propuesta de directiva de 2019
			\3 Cooperación administrativa
			\3 Evaluación mutua entre EEMM
				\4 Evaluación periódica mutua
			\3 Notificación de borradores legislativos
				\4 EEMM deben informar a CE de borradores
				\4 CE emite recomendación sobre legislación
				\4[] ¿Contraria a derecho UE sobre mercado único?
				\4 Evitar implementación de normativa obstaculizadora
			\3 DSM -- Mercado único digital
				\4 Estrategia anunciada en 2015
				\4 Propuesta de directiva en 2016
				\4 Directiva sobre copyright en DSM de 2019
				\4 Tres pilares
				\4[] i. Mejor acceso a bienes en línea
				\4[] $\to$ Eliminación del geo-blocking
				\4[] $\to$ Prohibida discriminación geográfica de acceso digital
				\4[] $\to$ Igual tratamiento de consumidores sin distinción geográfica
				\4[] ii. Redes de acceso
				\4[] $\to$ Medidas de fomento de competitividad
				\4[] $\to$ Código Europeo de Comunicaciones Electrónicas
				\4[] $\to$ Apoyo a acceso wifi gratuito
				\4[] $\to$ Fondos de Plan Juncker/InvestUE
				\4[] iii. Maximizar potencial de crecimiento de economía europea
				\4[] $\to$ Garantizar flujo de información no personal
				\4[] $\to$ Fomento de interoperabilidad
				\4[] $\to$ Fijación de estándares
				\4 Actuaciones
				\4[] Directiva de neutralidad de redes
				\4[] Eliminación de costes de roaming
				\4[] Prohibición de geoblocking
				\4[] Costes de construcción de redes
				\4[] $\to$ Medidas para reducir coste
				\4[] Protección de datos personales
				\4[] $\to$ GDPR
				\4[] $\to$ Garantizar acceso a los propios datos
				\4[] $\to$ Derecho a portabilidad de datos
				\4[] $\to$ Derecho al olvido
				\4[] $\to$ Derecho a saber cuando datos han sido hackeados
			\3 Directiva de servicios audivisuales de 2018
				\4 Libertad de recepción de transmisiones audivisuales
				\4[] Prohibido discriminar emisiones de otros países
				\4 Protección de menores y discursos de odio
				\4 Restricciones a determinada publicidad
				\4 Énfasis en auto-regulación del sector
				\4[] Códigos de conducta nacionales
				\4 Obligación de contribuir a producción cultural europea
				\4[] Principio de país de destino
				\4[] $\to$ Pagar a país en el que se emite
				\4[] $\then$ Cambio respecto principio de país de origen
		\2 Valoración
			\3 Reservas de actividad
				\4 Persisten en muchos ámbitos
				\4 Algunas son necesarias
				\4 Otras son resultado de problemas ec. política
			\3 Escasa integración
				\4 Sectores no comerciables
				\4[] Dificultades intrínsecas a mercado interior
				\4[] Barreras tecnológicas
			\3 Oposición opinión pública
				\4 Economía política
				\4 Sectores potencialmente perjudicados
		\2 Retos
			\3 Barreras no regulatorias
				\4 Idiomáticas
				\4 Culturales
				\4 Desconfianza de los consumidores
			\3 Transferencias de soberanía
				\4 Ámbitos sin liberalizar requieren trans. adicionales
	\1 \marcar{Personas}
		\2 Justificación
			\3 Especialización
				\4 Atracción de trabajo cualificado
				\4 Externalidades positivas
				\4[$\then$] Acumulación de capital humano cualificado
			\3 Aprovechamiento de recursos humanos
				\4 Input esencial
				\4 Efectos directos sobre bienestar humano
			\3 Existencia de muy diferentes regímenes
				\4 Seguridad social
				\4 Desempleo
				\4 Regulación laboral
				\4[$\then$] Dificultan movilidad intra-europea
				\4[] Dificultan conocimiento de regulación
				\4[] Dificultan traslado de derechos sociales
			\3 Shocks asimétricos en EEMM
				\4 Aumento de desempleo
				\4 Aumento de presión sobre seguridad social
				\4 Deterioro cuentas públicas
				\4 Inexistencia de estabilización fiscal nivel UE
		\2 Objetivos
			\3 Permitir ajuste ante shocks adversos
				\4 Migración interna para reducir desempleo
			\3 Facilitar movilidad intra-europea
				\4 Reducir barreras legales
			\3 Reducir efectos adversos de migraciones
				\4 Efectos sobre seg. social, economía política
			\3 Mantener garantías de seguridad
				\4 Riesgo terrorista, crimen organizado, tráfico
		\2 Antecedentes
			\3 T. de Roma (1957)
				\4 Libre circulación de trabajadores
			\3 Acuerdo de Schengen de 1985
				\4 Acuerdo político para eliminar controles
			\3 Convención de Schengen de 1990
				\4 Implementación del acuerdo
				\4 Entrada en vigor en 1995
			\3 Tratado de Maastricht (1992):
				\4 Ciudadanía europea
				\4 Introduce concepto
				\4 Fundamento de libre movilidad de personas
			\3 Tratado de Amsterdam (1997)
				\4 Comunitarización de Schengen
				\4 Hasta entonces, sólo era acuerdo intergubernamental
		\2 Marco jurídico
			\3 TUE
				\4 Artículo 3.2
				\4[] $\to$ Libre movimiento de personas
			\3 TFUE
				\4 Artículo 45
				\4[] Asegura libre movilidad
				\4[] Abolición de discriminación por nacionalidad
				\4[] Establece excepciones
			\3 Directiva 38/2004 sobre libre circulación de personas
			\3 Acuerdo y Convención Schengen
				\4 Acuerdo de 1985
				\4[] Declaración política
				\4 Convención de 1990
				\4[] Implementación del acuerdo
				\4 Inicialmente fuera de UE
				\4[] Acuerdo intergubernamental
			\3 Reglamento de Dublin 2013/604
				\4 Criterios y mecanismos de determinación
				\4[] Solicitudes de protección internaconal
				\4[] $\to$ Refugiados
				\4[] $\to$ Otra protección
				\4[] $\then$ Refugiados y apátridas
		\2 Actuaciones
			\3 Igualdad de trato
				\4 Acceso al empleo
				\4[] Excepciones en empleo público
				\4 Salario
				\4[] Sin discriminación posible por nacionalidad
			\3 Libre circulación y residencia
				\4 Reconocidos derechos de:
				\4[] Estancia corta
				\4[] $\to$ Sin condiciones
				\4[] Estancias > 3 meses
				\4[] $\to$ Sin permiso de residencia
				\4[] $\to$ Necesarios recursos residentes
				\4[] Residencia permanente
				\4[] $\to$ Tras 5 años de residencia
				\4 Limitación al derecho de entrada
				\4[] Motivos de:
				\4[] $\to$ Orden público
				\4[] $\to$ Seguridad pública
				\4[] $\to$ Salud pública
				\4[] $\to$ Abuso o fraude
				\4[] Prohibido limitar derecho por razones económicas
				\4[] Obligación de proporcionalidad
			\3 Reconocimiento mutuo de profesiones
				\4 Profesiones reguladas
				\4[] $\to$ ¿Pueden ejercer en cualquier país?
				\4 Actualmente, sólo 8 profesiones a nivel UE
				\4[] Profesiones médicas:
				\4[] $\to$ Médicos, enfermeros, dentistas, matronas, fisio
				\4[] $\to$ Farmacéuticos
				\4[] $\to$ Veterinarios
				\4[] $\to$ Arquitectos
				\4[] Avances lentos
				\4[] Acuerdos bilaterales
				\4 Proceso de Bolonia: EEES
				\4[] Espacio Europeo de Educación Superior
				\4[] Primer paso hacia reconocimiento generalizado
			\3 Espacio Schengen
				\4 Entrada en vigor en 1995
				\4 Eliminación de controles fronterizos para personas
				\4 Armonización de control en fronteras exteriores
				\4[] Acceso a Schengen desde cualquier frontera
				\4 Política común de visados de corta duración
				\4 Cooperación policial y judicial
				\4 Inicialmente:
				\4[] GER, BEL, FRA, LUX, NED
				\4 Actualmente:
				\4[] Casi todos países de la Unión
				\4[] $\to$ Salvo UK e IRL
				\4[] $\to$ ROM, BUL posible entrada
				\4[] También algunos países de fuera de la Unión
			\3 Restricciones a la libertad de movimientos
				\4 EEMM pueden negar entrada o residencia
				\4 Motivos tasados
				\4[] Orden público
				\4[] Seguridad
				\4[] Salud
				\4 Debe presentar una amenaza seria
				\4[] Intereses fundamentales del estado
				\4 Sujeto a garantías de procedimiento
				\4 Libre movimiento de trabajadores en sector público
				\4[] Limitado
				\4[] Interpretación restrictiva de limitación
				\4[] $\to$ Sólo ejercicio de autoridad pública
				\4[] $\to$ Salvaguardas del interés general del estado
				\4[] Sólo posible restringir a favor de nacionales propios
		\2 Valoración
			\3 Este--oeste
				\4 Ciertos países del este a UK, GER
			\3 Sur--norte
				\4 Algunos países del sur al norte
				\4[] POR, ITA a UK, GER, LUX, NED
				\4 Otros países muestran baja movilidad
				\4[] ESP especialmente
		\2 Retos
			\3 Interacción con otras políticas
				\4 Alquiler vs compra de vivienda
				\4[] Tipos de interés bajos abaratan endeudamiento
				\4[] Endeudamiento barato induce compra de vivienda
				\4 Vivienda en propiedad y movilidad
				\4[] Vivienda en propiedad dificulta movilidad
				\4[] Alquiler más favorable a movilidad
				\4[] Mercados de alquiler poco desarrollados en algunos países
			\3 Economía política de la migración
				\4 Ventajas de la migración intraeuropea
				\4[] Ya señaladas
				\4 Competencia entre mano de obra
				\4 Competencia por protección social
				\4 Despoblación y envejecimiento
				\4[$\then$] Puede alterar equilibrios políticos
				\4[] Inestabilidad política y económica
			\3 Propuestas de profundización
				\4 Reconocimiento automático de más profesiones
				\4 Coordinación de sistemas de Seguridad Social
				\4 Número de Seguridad Social a nivel Europeo
				\4[] $\to$ Más allá de tarjeta sanitaria europea
			\3 Seguridad
				\4 Terrorismo aprovecha Schengen
				\4 Schengen suspendido en varias ocasiones
				\4 Impacto económico de restablecimiento de controles
			\3 Pandemia 2020
				\4 UE fracasa coordinando cierre de fronteras
				\4 Cierres unilaterales relativamente descoordinados
				\4 Incertidumbre sobre futuro
	\1 \marcar{Capitales}
		\2 Justificación
			\3 Asignación eficiente del capital
				\4 Fluir a proyectos más rentables
				\4 Aprovechar oportunidades de inversión
			\3 Capital escaso en algunos países
				\4 Dificulta desarrollo
				\4 Aumenta coste de financiación
			\3 Suavizar shocks asimétricos
				\4 Cosechas en países agrícolas
				\4 Shocks de oferta negativos
				\4 Iliquidez temporal
		\2 Objetivos
			\3 Eliminar obstáculos a movilidad de K
				\4 Eliminar barreras legales a inversión extranjera
			\3 Integrar mercados financieros europeos
				\4 Títulos y empresas europeas
				\4 Mercado financiero interior operativo
				\4[] Banca
				\4[] Seguros
				\4[] Fondos de inversión
			\3 Aumentar diversificación de inversiones
				\4 Reducir riesgo idiosincrático
				\4 Aumentar resistencia a shocks
			\3 Canalizar ahorro a proyectos rentables
				\4 Aumentar eficiencia del capital
				\4 Reducir inconvenientes de flujos masivos de K
		\2 Antecedentes
			\3 Tratado de Roma de 1957
				\4 No prescribe liberalización plena
				\4 Circulación de K sólo en medida necesaria
				\4[] Para implementación del mercado común
			\3 Directiva de liberalización de 1960
				\4 Liberalización incondicional de:
				\4[] IDE
				\4[] Créditos relativos a operaciones comerciales
				\4[] $\to$ Corto o medio plazo
				\4[] Títulos negociados en bolsa
				\4 Desencadenó liberalizaciones unilaterales
				\4[] Prácticamente todas las restricciones
				\4[] RFA, RU, BENELUX
			\3 Directivas de liberalización de los 80
				\4 Extienden ámbito de directivas anteriores
				\4[] Créditos a largo plazo
				\4[] Títulos no negociados en bolsa
			\3 Primera fase de la UEM: directiva de 1988
				\4 Inestabilidad cambiaria y financiera
				\4 Se propone UEM
				\4 Necesaria libre circulación definitiva
				\4 Supresión de todas restricciones
				\4[] Entre todos los residentes de EEMM
			\3 Tratado de Maastricht: prohibición de restricciones
				\4 TUE consagra prohibición de restricciones
				\4 También entre EEMM y terceros países
				\4 Algunas excepciones
		\2 Marco jurídico
			\3 TUE y TFUE
			\3 Acuerdos de Basilea
			\3 Single Rulebook
				\4 CRD IV/CRR I
				\4 CRD V/CRR II
				\4 BRRD/SRMR I y II
				\4 DSGD
				\4[] Deposit Guarantee Scheme Directive
			\3 Directivas de seguros
			\3 MiFID II y MiFIR
			\3 PSD2 (2015) -- Payment Services Directive
		\2 Actuaciones
			\3 Prohibición de obstáculos a la mov. de K
				\4 Entre residentes EEMM y no residentes
				\4 Armonización de normas
			\3 Supervisión y regulación financiera
				\4 ESFS -- Sistema Europeo de Supervisión Financiera
				\4[] EBA+ESMA+EIOPA+ESRB+Nacionales
				\4 Bancaria
				\4[] Competencia del BCE y EBA
				\4[] Delegación en supervisores nacionales
				\4[] Transposición de Basilea
				\4[] Single Rulebook
				\4[] $\to$ CRR, CRD, BRRD, normas de la EBA
				\4 Supervisión seguros
				\4[] EIOPA
				\4[] Competencia nacional de supervisores respectivos
				\4 Supervisión de mercados de valores
				\4[] ESMA
				\4[] MiFID II y MiFIR
			\3 Unión Bancaria
				\4 Mecanismo Único de Supervisión
				\4 Mecanismo Único de Resolución
				\4[] Junta Única de Resolución
				\4[] Fondo Único de Resolución
				\4 Seguro de Depósitos
				\4 Fondos de resolución a nivel nacional
				\4[] De creación obligatoria
			\3 SEPA
				\4 Single Euro Payments Area
				\4 Plataforma de pagos en zona euro
				\4[] Mismas condiciones que pagos nacionales
			\3 CMU -- Unión del Mercado de Capitales
				\4 Idea clave
				\4[] Fuerte bancarización a nivel UE
				\4[] Mercados de capital muy segmentados
				\4[] $\to$ Especialmente en renta fija
				\4[] $\to$ Poco aumento de flujos trans-EM en última década
				\4[] $\to$ Diferenciales de interés elevados
				\4[] $\then$ Concentración del riesgo
				\4[] $\then$ Mala asignación de capital
				\4 Superar tres barreras
				\4[] i. Transparencia
				\4[] $\to$ Centralizar y estandarizar información sobre emisores
				\4[] $\to$ Reducir coste de información de pequeños emisores
				\4[] $\to$ Permitir no cotizadas acceder a mercado
				\4[] ii. Regulación
				\4[] $\to$ Centralizar supervisión de intermed. sistétmicos
				\4[] $\to$ Aumentar convergencia supervisoria
				\4[] $\to$ Armonizar regulación de pensiones para portabilidad
				\4[] iii. Insolvencia
				\4[] $\to$ Estándares mínimos de control
				\4[] $\to$ Sistematizar control de proceso regulatorio
				\4[] Necesario
				\4[] $\to$ Armonizar regulación
				\4[] $\to$ Aumentar tenencia transfronteriza de valores
				\4[] $\to$ Aumentar acceso a capital entre EEMM
				\4[] Lanzamiento de programa en 2015
				\4[] $\to$ CdUE pide a CE plan de acción y hoja de ruta
				\4 Formulación
				\4[] Crear entorno de mercado favorable
				\4[] $\to$ Emisión de obligaciones negociables
				\4[] $\to$ Negociación a nivel UE
				\4[] Armonizar regulación
				\4[] $\to$ Reducir problemas de información
				\4[] $\to$ Aumentar tenencia transfronteriza de títulos
				\4 Actuaciones
				\4[] Reglamento sobre fondos de capital riesgo (2013)
				\4[] $\to$ Armoniza requisitos y condiciones
				\4[] $\to$ Mercado único de fondos
				\4[] Reglamento de titulizaciones de 2017
				\4[] Reglamento de folletos de 2017
				\4[] $\to$ Reduce carga administrativa
				\4[] $\to$ Facilitar acceso a empresas más pequeñas
				\4 Valoración
				\4[] Unión Bancaria sigue siendo objetivo principal
				\4[] UMCapitales más compleja y de largo plazo
				\4[] Doble objetivo de UMCapitales
				\4[] $\to$ Desintermediar sistema financiero
				\4[] $\to$ Desconcentrar financiación en mercados nacionales
			\3 Reglamento sobre titulizaciones y STS de 2017
				\4 Fomentar titulización tras crisis financiera
				\4[] Mercado plenamente recuperado en EEUU
				\4[] Crecimiento en Asia
				\4[] Casi desaparecido en Europa
				\4 Criterio para identificar titulizaciones STS
				\4[] Simple, Transparent and Standardized
				\4[] $\to$ Titulizaciones de alta calidad
				\4[] Puede recibir mejor tratamiento preferente
				\4[] $\to$ A nivel de ratios de capital
				\4[] $\to$ Reforma de CRR en 2017 para introducir
			\3 Next CMU High-Level Expert Group\footnote{Ver FT (2019) en carpeta del tema}
				\4 Otoño de 2019
				\4 Grupo expertos para reimpulsar CMU
				\4 Francia, Alemania, Italia, España, Polonia, Holanda
				\4 Rebranding:
				\4[] Savings and Sustainable Investment Union
				\4 Relajar Solvency 2 y MiFID 2
				\4[] Han empeorado acceso a bolsa de PYMES
				\4[] Han aumentado incentivos a inversión de c/p
				\4 Fomento de covered bonds
				\4[] Reducir peso sobre balances bancarios
			\3 PSD2 -- Directiva Europea de Servicios de Pago
				\4 Aprobada en 2015
				\4 Entrada en vigor en 2019
				\4[] Prórrogas parciales
				\4 Marco regulatorio común
				\4[] Permitir prestación a nivel europea
				\4[] $\to$ Servicios de pago físico
				\4[] $\to$ Comercio electrónico
				\4[] $\to$ ...
				\4 Designación de autoridad competente a nivel nacional
				\4 Fomentar competencia en servicios de pago
				\4 Protección de consumidores en pagos electrónicos
				\4 Servicios transfronterizos de pago más seguros
			\3 EMIR -- European Market Infrastructure Regulation
				\4 Aprobada en 2012
				\4 Transparencia
				\4[] Obligación de proveer información a repositorios
				\4[] $\to$ Registros de derivados
				\4[] Publicación de posiciones agregadas de derivados
				\4[] $\to$ OTC
				\4[] $\to$ Cotizados en mercados oficiales
				\4 Reducción de riesgo de crédito
				\4[] Contratos OTC estandarizados
				\4[] $\to$ Deben compensarse y liquidarse en CCP
				\4[] Contratos fuera de CCP
				\4[] $\to$ Técnicas de mitigación del riesgo
				\4 Regulación de CCP
				\4[] Riesgo operativo
				\4[] Capital
				\4[] Transparencia
				\4 Acreditación de equivalencia
				\4[] Reconocimiento de CCP y repositorios fuera de UE
				\4[] Confirmar requisitos mínimos de fuera-UE
				\4[] Reconocimiento permite:
				\4[] $\to$ Utilización de CCP por residentes en UE
			\3 TARGET 2-Securities
				\4 Interconexión de depósitos centrales de valores
				\4 Liquidación de títulos valor
				\4[] Negociados en mercados oficiales
		\2 Valoración
			\3 Excesiva intermediación
				\4 Mercados de capital no bancarios poco desarrollados
				\4[] Salvo UK, IRL
			\3 Predominio financiación bancaria
				\4 Dificulta liberalización
				\4 Aumenta necesidad de liberalizar
			\3 Reformas lentas
			\3 Aumento de bancarización en últimos años
				\4 En 2018-2019, deterioro respecto 2013-2017
				\4[] Financiación bancaria: 88\%
				\4[] Financiación vía mercados de capital: 12\%
				\4[] $\then$ 2\% menos de mercados de capital respecto 13-17
			\3 Brexit
				\4 Centro de mercados de capital dentro de UE
				\4 Salida de RU, posible empeora situación
				\4[] Nuevas barreras a capita
				\4 ¿Aparición de centro financiero europeo?
		\2 Retos
			\3 Diversificación de inversiones
			\3 Regulación macroprudencial
				\4 ¿Flujos de capital crean problemas?
			\3 Transparencia
				\4 Información insuficiente
			\3 Deuda pública
				\4 Dinámicas de deuda dificultan liberalización
	\1 \marcar{Competencia}
		\2 Justificación
			\3 Economías de escala y poder de mercado
				\4 Mercado único aumenta potenciales ec. de escala
				\4[] $\to$ Aumenta poder de mercado
			\3 Regulación nacional de competencia
				\4 EEMM regulan competencia en ámbito nacional
				\4 Posible colisión con normativa europea
				\4[] $\to$ Incentivos a favorecer empresas nacionales
		\2 Objetivos
			\3 Evitar abusos de poder de mercado
				\4 Aumento de rentas extraídas por monopolista
				\4 Coste de eficiencia
			\3 Incentivar competencia entre empresas
			\3 Evitar distorsiones en el mercado único
			\3 Reducir race-to-the-bottom regulatorio
		\2 Marco jurídico
			\3 TFUE.3
				\4 Competencia exclusiva de UE
				\4[] Normas sobre competencia para Mercado Único
			\3 Reglamento de control de concentraciones de 2004
		\2 Actuaciones\footnote{Ver \url{https://www.europarl.europa.eu/factsheets/es/sheet/82/la-politica-de-competencia}.}
			\3 Acuerdos entre empresas
				\4 TFUE.101
				\4 En general, prohibidos:
				\4[] $\to$ Acuerdos entre empresas
				\4[] $\to$ Decisiones de asociaciones empresariales
				\4[] $\to$ Prácticas concertadas
				\4 Particularmente, si:
				\4[] $\to$ Fijan precios de compraventa
				\4[] $\to$ Limitan producción, inversiones, I+D
				\4[] $\to$ Reparten mercados
				\4[] $\to$ Imponen condiciones desiguales a terceros
				\4[] $\to$ Obligan a terceros a contratar
				\4 Excluidos de regulación sobre acuerdos
				\4[] Si:
				\4[] $\to$ Contribuyen a mejorar producción/distribución
				\4[] $\to$ Fomentan desarrollo técnico o económico
				\4[] Y permiten a usuarios participación equitativa
				\4[] $\to$ En beneficios derivados
				\4 Obligación de notificación a Comisión
				\4[] De acuerdos entre empresas
				\4[] Inicialmente obligatoria para todos
				\4[] Posteriormente, reglamentos de exención
				\4[] $\to$ Franquicias
				\4[] $\to$ Concesionarios de automóviles
				\4[] $\to$ Medioambiente
				\4[] $\to$ Cuota de mercado menor a umbral (10\%)\footnote{Si no compiten, pueden ser inferiores al 15\%.}
				\4[] $\then$ Aunque acuerdos de restricción de comp. son siempre relevantes
				\4 Programa de clemencia
				\4[] Inmunidad total o reducción de multas
				\4[] $\to$ A empresa que presente pruebas
				\4[] $\to$ Antes o después de investigación
				\4[] $\then$ Incentivar ``chivatos''
			\3 Abuso de posición de dominio
				\4 TFUE.102
				\4 Imposición de condiciones desfavorables
				\4[] $\to$ Aprovechando posición dominante en mercado
				\4[] $\then$ Sin excepciones generales
				\4 Ejemplos
				\4[] Precios por debajo de coste
				\4[] $\to$ Arruinar competidores con menos capital disponible
				\4[] Precios excesivos a consumidores
				\4[] $\to$ Aprovechando posición dominante
				\4[] Tying y bundling
				\4[] $\to$ Compra obligatoria de productos aprovechando elast. baja
				\4[] Negativa a negociar con competidores
				\4 Imposiciones particularmente prohibidas
				\4[] Imposición de precios de compra, venta no equitativas
				\4[] Limitación de producción, mercado o desarrollo técnico
				\4[] Condiciones desiguales obligatorias para prestaciones equivalentes
				\4[] Subordinar contratos aceptación de prestaciones suplementarias
				\4 Posición dominante en sí no está prohibida
				\4[] $\to$ Si adquirida con medios legítimos
				\4[] $\then$ Problema de delimitación de mercado
			\3 Control de concentraciones
				\4 Reglamento 139/2004
				\4 Vigilancia de la estructura del mercado
				\4[] $\to$ Control ex-ante
				\4[] $\then$ Operaciones de concentración deben notificarse a CE
				\4[] $\then$ Sin investigaciones sistemáticas posteriores
				\4 Concentraciones proclives a
				\4[] Obstaculizar competencia efectiva
				\4[] Crear posiciones dominantes
				\4[] Potencial falseamiento de la competencia
				\4 Umbrales mínimos de dimensión comunitaria
				\4[] $\to$ Cifra de negocios conjunta > 5000 millones
				\4[] $\to$ Cifra individual > 250 millones
				\4[] $\to$ Firmas facturan menos de 2/3 total en un estado
				\4[] $\to$ Otros requisitos modulan anteriores
				\4[] $\then$ Por debajo, competencia nacional
				\4 Aprobación por CE
				\4[] $\to$ Alrededor de 80\% se aprueba
			\3 Ayudas públicas
				\4 TFUE.107 y ss.
				\4 Comprenden:
				\4[] $\to$ Subvenciones
				\4[] $\to$ Préstamos favorables
				\4[] $\to$ Exoneraciones fiscales
				\4[] $\to$ Garantías de préstamo
				\4[] $\to$ Participaciones públicas en accionariado
				\4[] $\then$ Interpretación amplia de ayudas
				\4 Prohibidas las ayudas que:
				\4[] $\to$ Falseen competencia
				\4[] $\to$ Tengan repercusiones comunitarias
				\4 Toleradas ayudas:
				\4[] $\to$ Genéricas y horizontales
				\4[] $\to$ Carácter social a consumidores sin discriminar producto
				\4[] $\to$ Reparar perjuicios por desastres naturales
				\4[] $\to$ Favorecer economía de determinadas regiones
				\4[] Pueden considerarse compatibles las ayudas que:
				\4[] $\to$ Desarrollo económico regiones pobres
				\4[] $\to$ Favorecer antigua RDA\footnote{Derogable 5 años después p}
				\4[] $\to$ Fomentar realización de un proyecto europeo
				\4[] $\to$ Remediar grave perturbación economía nacional
				\4[] $\to$ Promover cultura o patrimonio
				\4[] $\to$ Otras categorías determinadas por Consejo
				\4 Régimen de vigilancia
				\4[] Comisión examina permanente
				\4[] Notificación obligatoria a Comisión
				\4[] Reglamentos de exención de notificación
				\4 Ayudas públicas en la crisis
				\4[] Especialmente a sector bancario
				\4[] Comisión ha debido supervisar
				\4[] Garantizar compatibilidad entre:
				\4[] $\to$ Estabilidad macroeconómica
				\4[] $\to$ Favorecer bancos nacionales
				\4[] Paquetes de comunicaciones bancarias
	\1 \marcar{Otros componentes del mercado interior}
		\2 Propiedad intelectual
			\3 Justificación
				\4 Elemento central de innovación tecnológica
				\4[] Permite rentas temporales a innovación exitosa
				\4 Creciente importancia de propiedad intelectual
				\4[] Competencia efectiva con emergentes
				\4[] $\to$ Necesario liderazgo en innovación
				\4 Tamaño de mercado
				\4[] Aumenta beneficios para innovadores potenciales
				\4[] Protección efectiva aumenta incentivos a invertir I+D
			\3 Objetivos
				\4 Asegurar protección mínima a protección intelectual
				\4 Reducir coste de protección a nivel europeo
			\3 Actuaciones
				\4 Paquete de 2017
				\4 Oficina de Propiedad Intelectual de la Unión Europea
				\4[] Creado en 1994
				\4[] Protección de:
				\4[] $\to$ Marcas
				\4[] $\to$ Modelos
				\4[] $\to$ Patentes
				\4[] Convergencia de prácticas europeas
			\3 Valoración
			\3 Retos
		\2 Contratación pública
			\3 Justificación
				\4 Importancia cuantitativa de las compras públicas
				\4[] Más del 15\% del PIB
				\4 Reducido valor de adjudicaciones a empresas internacionales
				\4[] Únicamente 2\% de contratos públicos
				\4[] Preferencia por proveedores nacionales
				\4[] $\to$ Normas legislativas o administrativas
				\4 Efectos positivos de competencia en licitaciones
				\4[] Competencia aumentada
				\4[] $\to$ Reducción de precios
				\4[] Economías de escala
				\4[] $\to$ Reducción de costes
				\4[] Reestructuración del mercado
				\4[] $\to$ Empresas más eficientes proveen al público
			\3 Objetivos
				\4 Mercado europeo de contratación pública
				\4 Libre competencia entre proveedores
				\4 Mecanismos de control para que empresas puedan recurrir
			\3 Antecedentes
			\3 Marco jurídico
				\4 Acuerdo de Contratación Pública WTO
				\4[] Government Procurement Agreement
				\4[] $\to$ Reformado en 2014
				\4[] Plurilateral
				\4 Paquete sobre contratación pública de 2014
				\4[] Directiva 24/2014 sobre contratación pública
				\4[] Directiva 25/2014 sobre contratación por entidades de sectores de agua, energía, transportes y servicios postales
				\4 Directiva 55/2014 sobre facturación electrónica
				\4[] Implantación de contratación pública
			\3 Actuaciones
				\4 Estrategia de Contratación Pública Electrónica de 2012
				\4[] Plazo hasta 2016 de contratación pública electrónica
				\4 Armonización de reglas de contratación
				\4 Digitalización de contratación pública
				\4[] Publicación digital de licitaciones
				\4 Contratación pública internacional
				\4 Reducción de carga administrativa
			\3 Valoración\footnote{Ver Resolución del PE de 2018 sobre paquete de medidas de contratación pública \url{https://www.europarl.europa.eu/doceo/document/TA-8-2018-0378_ES.html}.}
				\4 Transparencia y disponibilidad de información
				\4 Niveles reducidos de contratación internacional
				\4[] Persisten barreras más allá de UE
				\4[] En grandes contratos sí hay mercado europeo
				\4[] En pequeños volúmenes
				\4[] $\to$ Empresas locales
				\4[] $\to$ Costes de transporte elevados
				\4[] $\to$ Barreras culturales, legales, idiomáticas
				\4 Trasposición insuficiente y lenta
				\4[] Comunicación sobre medidas de contratación pública
				\4[] Procedimientos de infracción incoados
			\3 Retos
				\4 Aumentar volúmenes en pequeños contratos
				\4 Reducir opacidad y obstáculos persistentes
				\4 Centrales de compras a nivel comunitario
				\4[] Opción muy poco explorada
		\2 Fiscalidad indirecta
			\3 Justificación
				\4 Distorsión a comercio de ByS
				\4 Incentivos a competencia fiscal
				\4 Potencial barreras implícitas
			\3 Objetivos
				\4 Reducir distorsiones de ByS
				\4 Competencia exclusivamente en términos económicos
				\4 Eliminar aranceles encubiertos
			\3 Actuaciones
				\4 Principios de actuación
				\4[] Subsidiariedad
				\4[] Proporcionalidad
				\4[] Unanimidad
				\4[] $\to$ Porque fiscalidad es copmetencia exclusiva de EEMM
				\4 IVA
				\4[] Imposición en destino
				\4[] Hecho imponible: adq. intra-EU de bienes
				\4[] $\to$ Deducción de IVA soportado en origen
				\4[] $\to$ Importadores pagan IVA nacional
				\4[] $\to$ Sin controles en frontera
				\4[] $\to$ Control vía contabilidad empresarial
				\4[] $\to$ Cruce de datos vía INTRASTAT
				\4[] Proyecto de imposición en origen
				\4[] $\to$ Sin avances al respecto
				\4[] $\to$ Muy compleja implementación
				\4[] $\to$ Armonización total de tipos necesaria
				\4[] $\to$ Cámara de compensación de recaudación
				\4[] $\then$ Sólo aplicado en venta minorista
				\4[] Tipos mínimos
				\4[] $\to$ Tipo normal > 15\%
				\4[] $\to$ Tipos reducidos > 5\% con excepciones
				\4[] $\to$ Abolidos tipos de lujo
				\4 IIEE
				\4[] Paquete Cockfield de 1985
				\4[] $\to$ Propuso tipo único por impuesto
				\4[] Acuerdo de 1992
				\4[] $\to$ Armonización alcohol, hidrocarburos, tabacos, carbón, gas natural, electricidad
				\4[] $\to$ Permitidos nuevos impuestos no armonizados
				\4[] $\to$ Tipos armonizados al mínimo indispensable
				\4[] $\to$ Consagración de imposición en destino
		\2 Fiscalidad directa
			\3 Justificación
				\4 Mercados de factores
				\4[] Enorme impacto económico
				\4[] Pilar de mercado interior
				\4 Competencia fiscal perniciosa
				\4[] Race-to-the-bottom fiscal
				\4[] Presión sobre financiación gasto público
				\4 Doble imposición
				\4[] Reduce incentivos a movilidad
			\3 Objetivos
				\4 Eliminar doble imposición
				\4 Evitar distorsiones en mercado interior
				\4 Reducir fraude fiscal
			\3 Actuaciones
				\4 Directiva de 2003 sobre fiscalidad del ahorro
				\4[] Intercambio de información fiscal
				\4[] $\to$ Reducir evasión fiscal
				\4 Directiva de 2011 sobre cooperación administrativa
				\4 Procedimientos ante TJUE
				\4[] CE actúa contra obstaculización de mercado interior
				\4 CCCTB
				\4[] Common Consolidated Corporate Tax Base System
				\4[] Base imponible única a nivel europeo
				\4[] Reparto de base imponible a cada EEMM
				\4[] $\to$ En función de criterio de determinación
				\4[] $\then$ Ingresos en país
				\4[] $\then$ Empleados
				\4[] $\then$ Activos fijos
				\4[] Cuota íntegra
				\4[] $\to$ Determinada en función de tipo nacional
				\4[] Objetivo
				\4[] $\to$ Evitar fuga de bases imponibles
				\4[] $\to$ Eliminar deslocalización por competencia fiscal
				\4[] Imposición sobre empresas
				\4[] Reducir doble tributación
				\4[] Evitar deslocalización por competencia fiscal
				\4[] Propuesta de directiva de 2011
				\4[] $\to$ Actualmente paralizada
				\4[] $\to$ FRA, GER, SPA apoyan
				\4[] $\to$ NED, HUN, IRL rechazan
				\4[] Base imponible armonizada para empresas
				\4[] Métodos comunes en toda UE
				\4[] Compensación ganancias-pérdidas
				\4[] Liquidación única del impuesto
				\4[] Ingreso en hacienda de país de origen
				\4[] $\to$ Remisión de parte a EEMM participante
				\4[] $\to$ Reglas de determinación de reparto
				\4 Impuesto digital
				\4[] Gravar empresas tecnológicas
				\4[] $\to$ Muy alta movilidad de base fiscal
				\4[] Presión EEUU para no implementar
				\4[] $\to$ Posibles aranceles en respuesta
				\4 Propuesta cooperación reforzada para implementar
		\2 Programa Fiscalis 2020
			\3 Idea clave
			\3 Actuaciones
			\3 Valoración
	\1[] \marcar{Conclusión}
		\2 Recapitulación
			\3 Mercancías
			\3 Servicios
			\3 Personas
			\3 Capitales
			\3 Otros componentes
		\2 Idea final
			\3 El mercado interior hoy
				\4 Profundización
				\4 Ampliación territorial
				\4 Ampliación sectorial
			\3 Grado de integración
				\4 Ley de un sólo precio
				\4 Volumen de comercio interior
			\3 Brexit
				\4 UK gran impulsor de liberalización
				\4 Salida implica desequilibrio
				\4[] Entre pro y contra liberalización
				\4[] Aumenta peso de opuestos a liberalización
			\3 WTO y comercio mundial
				\4 Avances en ámbito multilateral
				\4[] Alteran incentivos a liberalización interna
				\4 Paralización proceso multilateral
				\4[] Difícil valorar efecto europeo
\end{esquemal}















































































































































\preguntas

\seccion{Test 2017}
\textbf{42.} Las prácticas concertadas entre empresas (artículo 101 del Tratado de Funcionamiento de la Unión Europea):

\begin{itemize}
	\item[a] Fueron parcialmente modificadas por el Libro Blanco 1985 sobre mercado interior.
	\item[b] Se entienden como incompatibles con el mercado interior, pero existen algunas excepciones.
	\item[c] Conducen indefectiblemente, según la Comisión, hacia la imposición de precios superiores a los del mercado.
	\item[d] Todas las anteriores opciones son ciertas.
\end{itemize}


\seccion{Test 2014}
\textbf{40.} Son competencia exclusiva de la Unión Europea las decisiones adoptadas en el ámbito:

\begin{itemize}
	\item[a] Normas de competencia
	\item[b] Mercado interior
	\item[c] Medio ambiente
	\item[d] Todas las anteriores
\end{itemize}




\seccion{21 de marzo de 2017}
Sahuquillo

\begin{itemize}
    \item El libre comercio de servicios dentro de la UE es la libertad cuyo desarrollo es más tardío y difícil. La Directiva Bolkenstein supuso un impulso al respecto. Sin embargo, ¿Qué sentido tiene ponerla en práctica en España cuando las comunidades autónomas son un obstáculo para que España sea un mercado único en sí misma?
    \item España es uno de los pocos o el único de los pocos países grandes en la UE que no tiene un paraíso fiscal a su servicio. ¿Por qué? ¿Es esto un problema?
    \item ¿Cree usted razonable que PSA recibiese un aval de 7.000 millones de euros del gobierno francés para reflotar el grupo, y que una vez ha sido reflotado gracias a la excelente labor del consejero delegado actual compre Opel?
    \item En la Unión Europea existe un sector con costes de transporte elevados y escasa diferenciación que ha sido objeto de numerosas sanciones por colusión, dados los incentivos inducidos por las propias características del producto. ¿Se le ocurre de cual se trata? \textit{Respuesta correcta: el sector del cemento.}
    \item Comente el hecho de que la prohibición de ayudas estatales a sectores específicos haya acabado por generar gigantes industriales del sector naval en países fuera de la Unión Europea, especialmente Corea del Sur.
    \item ¿Por qué es importante la política de competencia para el buen funcionamiento de un mercado único?
    \item ¿Qué entiende la UE por posición dominante?
\end{itemize}

\notas

\textbf{2017:} \textbf{40.} B

\textbf{2014:} \textbf{40.} A

La introducción del tema de CECO es excelente y se puede adaptar al resto de temas de la Unión Europea.

\begin{enumerate}
	\item Objetivo de primer nivel de la Unión Europea: art. 3 TUE, \textit{promover la paz, sus valores y el bienestar de los pueblos}.
	\item Objetivos intermedios para lograr el objetivo de primer nivel: ejercicio de competencias propias (protección del mercado interior), compartidas y complementarias.
	\item Objetivo de creación de un mercado común o integración económica. Para la consecución del mismo, se trata de garantizar 4 libertades (mercancías, servicios, capitales, personas).
	\item Desde el punto de vista político: aumento de la cooperación entre países, aumento de la interdependencia y reducción del riesgo de conflicto.
	\item Desde el punto de vista económico: efectos que aumentan producción y renta de los individuos. Estáticos (creación vs desviación de comercio) y dinámicos (aumento PTF, estímulo dotaciones, competencia, difusión tecnología, economías de escala).
\end{enumerate}

El nuevo de CECO tiene también unos apéndices interesantes sobre restricciones al movimiento de capitales en el contexto de las crisis.


\bibliografia

Mirar en Palgrave:
\begin{itemize}
	\item European Cohesion Policy
	\item European Monetary Integration
	\item European Union Single Market: Design and Development
	\item European Union Single Market: Economic Impact
\end{itemize}

Ver Revista ICE Mayo--Junio 2018 sobre mercado común europeo -- Carpeta boletines


\end{document}
