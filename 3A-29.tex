\documentclass{nuevotema}

\tema{3A-29}
\titulo{La Nueva Macroeconomía Clásica}

\begin{document}

\ideaclave

La macroeconomía es la rama de la economía que estudia los fenómenos económicos de gran escala. Es decir, aquellos que involucran a decenas de miles, cientos de miles o millones de agentes que toman decisiones con contenido económico que se repercuten mutuamente y que a su vez inducen nuevos cambios respecto las decisiones iniciales. Para estudiar estos fenómenos, la macroeconomía hace uso de variables agregadas. Diferentes escuelas macroeconómicas tratan de dar explicación a los fenómenos que se observan en relación a estas variables agregadas. Éstos fenómenos a explicar conciernen fundamentalmente los cambios en esas variables agregadas, tanto en el largo plazo como en el corto plazo. Para ello, los macroeconomistas utilizan diferentes herramientas que han sido objeto de transformación gradual y que, junto con conjuntos de modelos y sus implicaciones, han dado lugar a paradigmas en el sentido de Kuhn. El \textbf{objeto} de esta exposición es presentar una escuela macroeconómica surgida en los años 70 que revolucionó el método y las conclusiones de la macroeconomía predominante hasta ese momento, fuertemente influenciada por Keynes y los autores posteriores que desarrollaron su obra. Ello se concreta en dar respuesta a una serie de preguntas básicas: ¿qué es la Nueva Macroeconomía Clásica? ¿a qué autores se asocia? ¿qué herramientas teóricas utiliza e introduce? ¿qué aporta al estudio de la macroeconomía? ¿qué implicaciones de política económica se derivan? La \textbf{estructura} de la exposición se divide en tres partes. En primer lugar, se introduce la visión general de la Nueva Macroeconomía a través del contexto económico y teórico de aparición, los autores y universidades principales a los que se asocia, y las ideas centrales que articulan el programa de investigación. A continuación se presentan los modelos principales divididos por tema. Por último, se exponen las principales implicaciones de la Nueva Macroeconomía Clásica sobre la política económica y la teoría macroeconómica.

La \marcar{Nueva Macroeconomía Clásica} aparece en un \textbf{contexto económico} de inflación creciente a nivel mundial como resultado de la crisis del petróleo y de inestabilidad creciente en los mercados financieros internacionales tras la descomposición del Sistema de Bretton Woods. Este contexto económico alimenta un \textbf{contexto teórico} en el que las críticas a las políticas monetaria y fiscal de corte keynesiano llevadas a cabo en Estados Unidos en los 60 aumentan de la mano de numerosos autores. En especial, Milton Friedman y otras voces asociadas a la corriente monetarista afirmaban la inestabilidad de la Curva de Phillips y rechazaban la posibilidad de utilizar la política económica como herramienta de estabilización de las fluctuaciones macroeconómicas. Los intentos de racionalización de la curva de Phillips de Friedman y Phelps inspiran una senda que abría de inspirar a los autores de la Nueva Macroeconomía Clásica: es posible explicar fenómenos macroeconómicos a partir de construcciones teóricas basadas en el comportamiento de agentes racionales. Los principales \textbf{autores} asociados a esta nueva escuela son Robert Lucas (Premio Nobel en 1995), Tom Sargent (Premio Nobel en 2011 junto a Sims), Neil Wallace, Ed Prescott,  Finn Kydland (Premio Nobel en 2004 junto a Prescott) y otros autores como Sims, Llungqvist o Robert Barro. Las principales universidades asociadas a esta escuela son Chicago, Carnegie Mellon, Minnesota, Pennsilvania o Rochester. Como resultado de la situación geográfica de estas universidades, Robert Hall acuñó el término ``\textit{freshwater economics}'' o economía de agua dulce, que se solapa de forma parcial con el concepto de Nueva Macroeconomía Clásica. 

Como paradigma coherente de pensamiento económico, la Nueva Macroeconomía Clásica se sustenta en un conjunto compacto de \textbf{ideas centrales} o principios fundamentales que orientan la generalidad de los modelos. La \underline{microfundamentación} es el primero de ellos. Los modelos keynesianos predominantes hasta el momento se caracterizaban por la imposición de supuestos ad-hoc a partir de los cuales se formulaban sistemas de ecuaciones que trataban de representar el comportamiento de la macroeconomía. Estos modelos no resultaban a priori de ningún conjunto de principios subyacentes derivados de comportamiento maximizador de agentes. A pesar de los esfuerzos en los años 60 por microfundamentar diferentes mercados de forma separada tales como la demanda de consumo o de dinero, la presentación global de los modelos seguía teniendo carácter ad-hoc y no era capaz de integrar de forma satisfactoria las interacciones entre mercados y a menudo no era compatible con comportamiento optimizador generalizado. Como resultado de estas formulaciones teóricas, se formulaban grandes modelos econométricos que estimaban formas reducidas que expresaban la relación entre variables sin atender en modo alguno a los procesos subyacentes que determinaban el valor de los parámetros estimados. La Nueva Macroeconomía Clásica rompe radicalmente con esta forma de representar la macroeconomía. En el contexto de este nuevo paradigma, los modelos macroeconómicos deben derivar sus conclusiones del comportamiento de agentes optimizadores de acuerdo con unas preferencias, una tecnología de producción y unas dotaciones postuladas de forma axiomática. Los modelos se concretan con la imposición de un marco institucional y unas condiciones iniciales. Aunque los autores de la Nueva Macroeconomía Clásica son plenamente conscientes de que una macroeconomía no es un agente representativo optimizador, ni siquiera un conjunto reducido de tipos de agentes, la representación de macroeconomías como si de agentes microeconómicos se tratase impone coherencia interna a las conclusiones derivadas de los modelos, y elimina resultados que difícilmente pueden sostenerse en el largo plazo si la economía está compuesta por agentes que buscan maximizar su bienestar y no incurren en errores sistemáticos o tratan activamente de aprender de sus errores pasados. Otra de las características de los modelos keynesianos de los años 60 era su carácter eminentemente estático. A pesar de algunos intentos por modelizar el ajuste dinámico de unos equilibrios a otros, la metodología keynesiana no era adecuada para introducir la \underline{dimensión temporal} de las decisiones económicas precisamente por la falta de microfundamentación referida anteriormente. La Nueva Macroeconomía Clásica, en el contexto de decisiones adoptadas por agentes racionales, introduce de forma plena la dimensión temporal al incorporar al proceso de decisión las expectativas acerca del futuro y sus efectos sobre el presente. La aparición de nuevos modelos microeconómicos tales como Arrow-Debreu (1954) o el redescubrimiento de Ramsey (1926) por Cass y Koopmans en los años 60, y la introducción en círculos económicos académicos de nuevas técnicas matemáticas como los métodos numéricos, la programación dinámica y la causalidad de Granger hacen posible el desarrollo de modelos en los que los agentes optimizan sus decisiones y construyen secuencias óptimas de decisión que abarcan todo un intervalo temporal. Así, los equilibrios dejan de ser puntos fijos de un sistema, sino secuencias completas de valores en los que los sistemas se encuentran efectivamente en equilibrio. En este contexto de decisión a lo largo de varios periodos temporales y más concretamente el valor futuro de determinadas variables, cuando se introduce la posibilidad de que algunas de ellas sufran perturbaciones estocásticas, es necesario postular la capacidad de los agentes para estimar el valor de éstas. Se trata, en definitiva, de caracterizar el proceso de formación de expectativas de los agentes. En modelos keynesianos que trataban de incorporar la dimensión temporal, era relativamente habitual postular expectativas miopes, de modo que los agentes asumían que las variables de estado tomarían en el futuro el mismo valor que en el último momento conocido. Milton Friedman y otros autores popularizaron la hipótesis de expectativas adaptativas: los agentes ajustan las estimaciones teniendo en cuenta la desviación entre los valores esperados y observados de periodos anteriores. Muth (1961) introduce formalmente la \underline{hipótesis de expectativas racionales} que habría de convertirse en el método de formación de expectativas predominante en la ciencia económica y particularmente en la macroeconomía. Según esta hipótesis, los agentes formulan expectativas de manera tal que el error de predicción tiende al mínimo a medida que aumenta el número de predicciones. O de forma equivalente, los agentes formulan predicciones acerca de variables estocásticas utilizando toda la información disponible al respecto, como si conociesen el proceso subyacente generador de las realizaciones de la variable estocástica. Cuando los agentes optimizan su comportamiento a partir de predicciones generadas bajo el supuesto de expectativas racionales, no es posible aprovechar de forma sistemática los errores de predicción. Si bajo el supuesto de expectativas adaptativas los agentes se equivocaban sistemáticamente porque siempre trataban de estimar a partir de realizaciones pasadas, bajo el supuesto de expectativas racionales los agentes se equivocan en sus predicciones tan sólo en la medida en que las variables que tratan de estimar sean volátiles. Por ello, los agentes aprovechan bajo HER todas las oportunidades para maximizar su comportamiento, y se adaptan en todo momento a cambios en el proceso generador subyacente para reducir el error de estimación. Todos los elementos anteriores se integran en modelos de \underline{equilibrio general}. Aunque algunos autores como Patinkin habían tratado de entender el modelo keynesiano como un modelo de equilibrio general e incluso intentaron caracterizarlo en un contexto walrasiano, los economistas adscritos a la tradición keynesiana tendían a preferir el equilibrio parcial, formulando modelos que consideraban el efecto de un número reducido de variables y aplicaban el supuesto de ceteris paribus al resto. En la Nueva Macroeconomía Clásica el enfoque de equilibrio parcial se reemplaza por el equilibrio general. La interrelación de todas las variables consideradas relevantes se tiene en cuenta, y se recupera plenamente el enfoque walrasiano que había sido ignorado en su mayor parte por el keynesianismo. La aparición de importantes resultados como Arrow y Debreu (1954) y el impulso inicial de Robert Lucas convierten al equilibrio general no sólo en el marco en el que representar economías agregadas a nivel teórico en los modelos de la NMC, sino en la teoría macroeconómica en general. Los resultados de los modelos de la Nueva Macroeconomía Clásica toman la forma de equilibrios walrasianos en los que los agentes efectivamente llevan a cabo sus planes óptimos dada una determinada estructura institucional, y se desechan en general las características no walrasianas que caracterizaban a los modelos del neokeynesianismo del desequilibrio que inspira a determinados autores por su intento de formular fenómenos macroeconómicos en términos microeconómicos. Aunque la Nueva Macroeconomía Clásica revoluciona la modelización teórica, no se limita a éste plano. Aunque el impacto de esta escuela macroeconómica es enorme en el plano teórico, no se limita a éste y ataca el consenso existente respecto a la validez externa de los modelos y los métodos de \underline{constrastación empírica}. Influidos por Friedman y liderados en lo metodológico por Robert Lucas, la bondad de un modelo se afirma en relación a su capacidad para predecir de forma satisfactoria fenómenos no observados antes de formular el modelo, y no por el realismo o falta de él de los supuestos que subyacen al modelo. El estándar de contrastación de un modelo teórico es su capacidad para predecir fenómenos off-sample al tiempo que se mantiene la consistencia interna y se abstiene de introducir supuestos ad-hoc que rompen la consistencia del modelo para tratar de mejorar la predicción. Así, Robert Lucas admite que el enfoque de equilibrio walrasiano no puede considerarse realista como tal, con la famosa frase ``\textit{si usted mira por la ventana y observa Nueva Orleans, es estúpido decir que la ciudad está en equilibrio}''. Los autores de la Nueva Macroeconomía Clásica ponen su énfasis en la contrastación empírica o estadística de sus resultados, y la mayoría de sus trabajos terminan con una sección en la que se proponen métodos de contraste. Como veremos a continuación, algunas de las aportaciones fundamentales de la NMC son críticas frontales a los modelos macroeconométricos propios de la síntesis neoclásica y se adentran de pleno en la teoría econométrica aunque sin alejarse de los grandes ejes del programa de investigación. 

Los postulados y \marcar{modelos principales} se presentan generalmente en forma de artículos académicos, agrupables según sus temas principales. El primer gran tema objeto de análisis fue la relación entre \textbf{salarios nominales, oferta de trabajo, producción e información imperfecta}. \underline{Lucas y Rapping (1969)} se encuentra a medio camino entre la microfundamentación incipiente a finales de los 60 y la Nueva Macroeconomía Clásica como tal. El objeto es presentar un modelo agregado del mercado de trabajo que explique la aparente discordancia entre la oferta de trabajo en el corto y en el largo plazo. En el corto plazo, los estudios muestran una elevada elasticidad de la oferta al salario real, mientras que en el largo plazo sucede al contrario: la oferta de trabajo apenas varía a pesar del crecimiento sostenido del salario real. Lucas y Rapping formulan un modelo de equilibrio general walrasiano en el que los agentes optimizan la utilidad en función del consumo y el ocio presentes y futuro y toman expectativas adaptativas para estimar los precios futuros. Del modelo se desprende que la oferta de trabajo responde a tres factores: \textit{i)} el salario real ``normal'' en el sentido de permanente o tendencial, \textit{ii)} la desviación del salario presente respecto de ese salario normal, \textit{iii)} la desviación de los precios respecto de su valor normal o tendencial. El salario responde muy poco a \textit{i)}, pero mucho a \textit{ii)} y \textit{iii)}, lo que explica que en el largo plazo la elasticidad sea muy baja pero alta en el corto plazo. En este contexto, el desempleo no tiene origen en factores no-walrasianos como es habitual en el modelo keynesiano, sino en el hecho de que los agentes prefieren no trabajar porque consideran que el salario que recibirían es demasiado bajo. 

Siguiendo la estela de Lucas y Rapping (1969), \underline{Lucas (1972)} introduce el modelo que habría de utilizar ya todas las herramientas propias de la NMC, iniciar propiamente el programa de investigación de Lucas y servir de inspiración a toda la modelización macroeconómica de equilibrio general. El objeto central del modelo es explicar la relación positiva pero inestable entre output y nivel de precios que venían observando los trabajos empíricos desde Phillips (1958), en un marco en el que las decisiones de los agentes son óptimas y no existe ningún tipo de restricción que les impida realizar sus planes de consumo y trabajo. En este contexto, el elemento determinante de la curva de Phillips a la que el modelo da lugar es la presencia de información imperfecta. Los agentes que consumen y ofertan trabajo no sufren ningún tipo de ilusión monetaria, optimizan su utilidad intertemporal y minimizan el error de sus predicciones, pero dada la construcción institucional del modelo, no pueden observar el salario real al que venden su trabajo. Pueden, sin embargo, observar el salario nominal recibido por su trabajo. El salario nominal observado está determinado por dos perturbaciones. La primera de ellas es real, y depende de la demanda por el producto que los consumidores-trabajadores ofrecen. La segunda de ellas es nominal: depende de la oferta agregada de dinero y afecta al nivel de precios general pero no implica mayor salario relativo. Partiendo de estas dos perturbaciones, los agentes se enfrentan a un problema de extracción de señales: dada una variación observada en una variable, deben estimar en qué medida esa variación se debe a un aumento del precio relativo de su producto y en qué medida a un aumento general del nivel de precios. Aunque los agentes no pueden distinguir con total certeza un impulso de otro, la hipótesis de expectativas racionales les permite minimizar el error de predicción en el que incurren. Si la varianza de las perturbaciones nominales es muy elevada y la correspondiente a las perturbaciones reales es relativamente baja, los agentes estimarán que una variación en el salario nominal percibido es debida principalmente a una perturbación nominal. Si la relación entre las varianzas de los shocks nominales y reales es inversa, los agentes estimarán que un aumento del salario nominal percibido se debe a una perturbación real y por tanto aumentarán su oferta de trabajo y ello aumentará el producto. Para derivar estos resultados, Lucas se apoya en el modelo de las islas de Phelps y el modelo de generaciones solapadas de Samuelson, así como el marco de optimización intertemporal ocio-consumo ya utilizado en Lucas y Rapping (1969). Las expectativas racionales y el conocimiento del nivel de precios pasado actúan como freno a la posibilidad de explotar sistemáticamente el resultado por parte de la autoridad monetaria. Aunque puede efectivamente ``engañar'' a los agentes para que produzcan más en un periodo determinado, una explotación sistemática de expansiones monetarias para aumentar el producto resultará en una estimación del nivel de precios que tendrá en cuenta este hecho y neutralizará el efecto expansivo sobre la oferta de trabajo y el producto. En \underline{Lucas (1973)}, el autor simplifica el modelo de Lucas (1972) y contrasta sus implicaciones con datos relativos a la varianza de la oferta monetaria, los precios y el output en diferentes países. Las regresiones presentadas muestran como una varianza muy elevada de los shocks nominales induce varianzas elevadas de los precios pero no del output. Este resultado es compatible con Lucas (1972), porque sugiere que la utilización sistemática de shocks monetarios para estimular el output acaba resultando en precios volátiles pero no mayores niveles de producción.

Uno de los objetivos principales objetivos de la macroeconomía es entender esas fluctuaciones persistentes y repetidas aunque no periódicas de las principales magnitudes. Los esfuerzos por entender estas fluctuaciones constituyen el análisis del \textbf{ciclo económico}. A grandes rasgos, las principales familias de modelos teóricos postulan tres causas: shocks nominales, shocks reales y dinámicas endógenas. La Nueva Macroeconomía Clásica inicia una nueva forma de analizar el ciclo económico a partir de modelos en los que la economía se encuentra en todo momento en equilibrio walrasiano. \underline{Lucas (1975)} formula un modelo del ciclo económico que modifica a Lucas (1972) para que las fluctuaciones del output muestren correlación serial y que muestra que es posible modelizar fluctuaciones cíclicas en ausencia de fallos de mercado. Para ello se vale de la introducción de lags de información --de modo que los shocks monetarios pasados tardan en conocerse y los agentes tienen información imperfecta respecto de su duración- y de capital físico --que los agentes utilizan para suavizar intertemporalmente su perfil de consumo-. El modelo muestra que es posible modelizar ciclos económicos en un marco de equilibrio general competitivo y que la suavización del ciclo implica, en este contexto, reducir la varianza de las fluctuaciones monetarias. \underline{Lucas (1977)} simplifica y expone en términos verbales el modelo de Lucas (1975) y carga contra las teorías keynesianas del ciclo. Es un trabajo con muy fuerte carga metodológica: defiende las virtudes del equilibrio general y las expectativas racionales frente a modelos que no tienen en cuenta la reacción de los agentes y que defienden la inestabilidad inherente y la capacidad del sector público para alejar de forma sistemática el output de su valor natural o de equilibrio. 

Entre estos Lucas (1975) y Lucas (1977), Lucas (1976) expresa una crítica con muy fuerte y duradero impacto en relación a los modelos macroeconométricos de la síntesis neoclásica y más concretamente, de su utilización como herramienta para valorar actuaciones de política económica. Esta crítica habría de conocerse posteriormente como \textbf{``crítica de Lucas''}. Se basa en un silogismo simple: si la estructura y los parámetros de los modelos econométricos es resultado de una regla de decisión óptima de los agentes involucrados, y esas reglas de decisión varían cuando cambian las reglas de política económica relevantes, los cambios en las políticas económicas alterarán la estructura y los parámetros de los modelos econométricos. Esto invalida la utilización de modelos econométricos estimados para políticas diferentes a las que se pretenden valorar. El impacto de este trabajo fue enorme. Cualquier modelo macroeconométrico posterior habría de justificarse en términos de la crítica de Lucas, y justificarse en caso de ser vulnerable a ésta. 

La formulación de la \textbf{política económica} es una de las preocupaciones centrales de toda escuela macroeconómica, y la Nueva Macroeconomía Clásica. Formular políticas como reglas generales o como decisiones discreccionales adaptadas al contexto específico en que se llevan a cabo es uno de los temas que los nuevos clásicos examinaron, así como el examen de los instrumentos de política monetaria: ¿el banco central debe regular el interés o la política monetaria? \underline{Sargent y Wallace (1975)} comparan, en un trabajo pionero, las dos alternativas básicas de política monetaria: reglas de oferta monetaria o reglas de tipo de interés. Para ello formulan un modelo ad-hoc que relaciona precio, output y tipo de interés y una función de pérdida a minimizar que depende de la fluctuación del output y de la inflación. El modelo implica que reglas de oferta monetaria no afectan a la distribución de probabilidad del output, y que reglas sobre el tipo de interés dejan indeterminado el precio. Ello concuerda con la proposición general de irrelevancia de la política monetaria a la hora de mantener sistemáticamente el producto por encima de su valor natural, y abre camino a modelos explícitos de política monetaria de que postulen la hipótesis de expectativas racionales. \underline{Kydland y Prescott (1977)} critican las políticas basadas en decisiones discreccionales que tratan de maximizar el bienestar social en un periodo determinado sin tener en cuenta la reacción de los agentes a esa decisión discrecional. Una de las implicaciones del artículo es que el control óptimo no es la herramienta adecuada para formular políticas económicas, y debe reemplazarse por la programación dinámica fundamentada en el principio de optimalidad de Bellman. \underline{Sargent y Wallace (1981)} examinan la interacción entre la política fiscal y la política monetaria, tratando de delimitar los supuestos bajo los cuales determinadas políticas fiscales resultan en inflación futura y bajo qué supuestos la política monetaria puede mantener bajo control la inflación. Llegan a la conclusión de que si la política fiscal es dominante, en el sentido de que ``juega primero'' en terminología de teoría de juegos, y el banco central no es absolutamente independiente --algo casi imposible de garantizar en la práctica-, políticas fiscales expansivas financiadas por deuda acabarán por generar inflación si los ingresos tributarios se mantienen constantes. Este artículo sería el germen de la llamada \textit{teoría fiscal del nivel de precios} posterior. 

Aunque Lucas había tratado de formular un modelo del ciclo que replicase fluctuaciones de series temporales observadas en base a shocks nominales, habría de ser un modelo basado en shocks sobre variable reales el que estabilizase el programa de investigación de la macroeconomía clásica y sentase las bases de la modelización DSGE. Así, el \textbf{modelo del ciclo real} supuso la culminación del programa de investigación de Lucas, aún transformándolo en gran medida. \underline{Kydland y Prescott (1982)} y \underline{Long y Plosser (1983)} formulan un modelo del ciclo económico basado en shocks reales que generan persistencia y amplificación en un marco de equilibrio general y agentes optimizadores de una función de utilidad intertemporal respecto del consumo y el ocio. El modelo del ciclo real consigue una replicación notable de los primeros momentos de series temporales observadas. Aunque adolece de graves problemas y no tiene en cuenta el efecto de shocks nominales, los modelos son el punto de partida de una enorme literatura posterior hasta la actualidad que incorpora toda clase de shocks y variantes institucionales para simular y estudiar los efectos de políticas macroeconómicas con un cierto grado de protección frente a los problemas de los modelos keynesianos. 

Las \marcar{implicaciones} derivadas de la Nueva Macroeconomía Clásica transformaron la teoría macroeconómica pero también el diseño de la \textbf{política económica}. De los modelos de los principales autores, un número creciente de policy-makers derivó un conjunto coherente de recomendaciones de política económica que se aplicaron a partir de los años 80 en la práctica totalidad de las economías de mercado. Las \underline{políticas basadas en reglas} cobraron mayor predicamento por ser más fáciles de evaluar al no desestabilizar las expectativas de los agentes y permitir una formulación más trasparente y menos sujeta a distorsiones de carácter político. El resultado de \underline{inefectividad de la política monetaria} de algunos modelos de la Nueva Macroeconomía Clásica se interpretó en ocasiones como un argumento en contra de la utilización de la política monetaria como herramienta estabilizadora, aunque también dio lugar a poderosos argumentos en contra. La regla de política monetaria basada en un aumento constante de la oferta monetaria asociada a Milton Friedman quedó prácticamente desacreditada en los 80 en buena medida como resultado de los modelos de la NMC. Las políticas de demanda pasaron a verse con creciente desconfianza, y las \underline{políticas basadas en factores de oferta} tomaron fuerza como resultado del supuesto de estabilidad de la macroeconomía. En cuanto al impacto de la NMC sobre la \textbf{teoría económica}, sigue plenamente presente con la preponderancia de los \underline{modelos DSGE} en contextos académicos y la relevancia de la \underline{crítica de Lucas} a la hora de formular modelos que deban pasar el escrutinio de las revistas de mayor impacto. Las \underline{expectativas racionales} se han consolidado como la hipótesis básica respecto al tratamiento de la información de los agentes microeconómico. A pesar de las críticas que ha recibido, resulta muy complicado encontrar otras hipótesis de expectativas que no tengan carácter ad-hoc.

El tema ha presentado una visión general de la macroeconomía clásica, sus modelos centrales y finalmente, sus implicaciones sobre el policy-making y la teoría macroeconómica. Las controversias a las que ha dado lugar la Nueva Macroeconomía Clásica son múltiples y perennes. Grandes nombres de la economía han criticado referentes de la escuela tales como el modelo del ciclo real y la hipótesis de expectativas racionales. Además, se le ha atribuido una agenda ideológica al programa de investigación de la Nueva Macroeconomía Clásica y la imposición de supuestos poco realistas. En cualquier caso, la historia de la NMC es la historia de una victoria metodológica sin apenas matices. La macroeconomía mainstream ha adoptado la práctica totalidad de las herramientas de la NMC. Los resultados que contradicen a la NMC parten de su marco metodológico e introducen modificaciones que mantienen intactos los principios básicos de modelización. El paradigma DSGE se abre poco a poco paso en el policy-making, a pesar de su complejidad analítica, y se ha consolidado como la herramienta central de modelización teórica en el contexto académico a pesar de las numerosas críticas. En definitiva, la Nueva Macroeconomía Clásica es un paradigma de pensamiento económico sin el cual no es posible comprender la macroeconomía actual. 

\begin{itemize}
	\item ¿Qué es la Nueva Macroeconomía Clásica?
	\item ¿A qué autores se asocia?
	\item ¿Qué innovaciones metodológicas utiliza?
	\item ¿Cuáles son sus aportaciones principales?
	\item ¿Qué implicaciones tiene sobre las políticas económicas?
\end{itemize}

\esquemacorto

\begin{esquema}[enumerate]
	\1[] \marcar{Introducción}
		\2 Contextualización
			\3 Macroeconomía
			\3 Escuelas macroeconómicas
			\3 Macroeconomía clásica
		\2 Objeto
			\3 ¿Qué es la Nueva Macroeconomía Clásica?
			\3 ¿A qué autores se asocia?
			\3 ¿Qué herramientas utiliza?
			\3 ¿Qué aporta a la macroeconomía?
			\3 ¿Qué implicaciones de política económica?
		\2 Estructura
			\3 Visión general
			\3 Modelos principales
			\3 Implicaciones
	\1 \marcar{Visión general}
		\2 Contexto económico (finales 60, principios 70)
			\3 Inflación creciente
			\3 Fin de Bretton Woods
			\3 Crisis del petróleo
		\2 Contexto teórico
			\3 Modelos macroeconométricos
			\3 Curva de Phillips
		\2 Autores
			\3 Nombres
			\3 Universidades
		\2 Ideas centrales
			\3 Microfundamentación
			\3 Optimización intertemporal
			\3 Hipótesis de expectativas racionales
			\3 Equilibrio general walrasiano
			\3 Contrastación empírica
	\1 \marcar{Modelos}
		\2 Oferta agregada e información imperfecta
			\3 Lucas y Rapping (1969)
			\3 Lucas (1972)
			\3 Lucas (1973)
		\2 Ciclo nominal basado en información imperfecta
			\3 Lucas (1975)
			\3 Lucas (1977)
		\2 Crítica de Lucas
			\3 Lucas (1976)
		\2 Política económica
			\3 Sargent y Wallace (1973)
			\3 Sargent y Wallace (1975)
			\3 Kydland y Prescott (1977)
			\3 Sargent y Wallace (1981)
			\3 Barro y Gordon (1983a) y (1983b)
		\2 Modelo del ciclo real
			\3 Kydland y Prescott (1982)
			\3 Long y Plosser (1983)
	\1 \marcar{Implicaciones}
		\2 Política económica
			\3 Idea clave
			\3 Reglas vs discrecionalidad
			\3 Inefectividad de la política monetaria
			\3 Políticas de oferta
		\2 Teoría económica
			\3 Modelos DSGE
			\3 Crítica de Lucas
			\3 Expectativas racionales
	\1[] \marcar{Conclusión}
		\2 Recapitulación
			\3 Visión general de la Nueva Macroeconomía Clásica
			\3 Modelos más relevantes
			\3 Implicaciones
		\2 Idea final
			\3 Robert Solow sobre modelos macro y economistas
			\3 Controversias
			\3 Victoria metodológica

\end{esquema}

\esquemalargo












\begin{esquemal}
	\1[] \marcar{Introducción}
		\2 Contextualización
			\3 Macroeconomía
				\4 Análisis de fenómenos económicos a gran escala
				\4 Énfasis sobre variables agregadas
			\3 Escuelas macroeconómicas
				\4 Postulan diferentes explicaciones de la realidad
				\4[] Causas de fluctuaciones
				\4[] Causas de comportamiento de largo plazo
				\4[] Efectos de políticas económicas
				\4 Utilizan diferentes herramientas
				\4[] Métodos matemáticos
				\4[] Métodos verbales
				\4 Revoluciones científicas
				\4[] Sinónimo con cambio de paradigma
				\4[] Cambio radical en
				\4[] $\to$ Herramientas
				\4[] $\to$ Objeto de estudio
				\4[] $\to$ Conclusiones
			\3 Macroeconomía clásica
				\4 Segunda revolución macroeconomía
				\4 Inicio años 70
				\4 Transforma métodos, herramientas, resultados, implicaciones
		\2 Objeto
			\3 ¿Qué es la Nueva Macroeconomía Clásica?
			\3 ¿A qué autores se asocia?
			\3 ¿Qué herramientas utiliza?
			\3 ¿Qué aporta a la macroeconomía?
			\3 ¿Qué implicaciones de política económica?
		\2 Estructura
			\3 Visión general
			\3 Modelos principales
			\3 Implicaciones
	\1 \marcar{Visión general}
		\2 Contexto económico (finales 60, principios 70)
			\3 Inflación creciente
				\4 Paro se mantiene elevado
			\3 Fin de Bretton Woods
				\4 Fluctuaciones en tipo de cambio
			\3 Crisis del petróleo
				\4 Shock de oferta global
				\4 Apoyo teorías cost-push de la inflación
		\2 Contexto teórico
			\3 Modelos macroeconométricos
				\4 Estiman formas reducidas
				\4[] Diferentes sectores de la economía
				\4 No tienen en cuenta reacción de agentes
			\3 Curva de Phillips
				\4 Friedman (1968) resulta muy influyente
				\4[] $\to$ Introduce idea de ``desempleo natural''
				\4 Intentos de racionalización
				\4[] $\to$ Friedman
				\4[] $\to$ Phelps
		\2 Autores
			\3 Nombres
				\4 Robert Lucas
				\4[] Nobel 1995
				\4 Tom Sargent
				\4[] Premio Nobel 2011 junto a Sims
				\4 Neil Wallace
				\4 Prescott
				\4[] Premio Nobel 2004 junto a Kydland
				\4 Kydland
				\4[] Premio Nobel 2004 Junto a Prescott
				\4 Otros
				\4[] Sims
				\4[] Ljungqvist
				\4[] Barro
			\3 Universidades
				\4 Chicago
				\4 Carnegie Mellon
				\4 Minnesota
				\4 Pennsilvania
				\4 Rochester
				\4[$\to$] ``freshwater economics''
		\2 Ideas centrales
			\3 Microfundamentación
				\4 Modelos Keynesianos
				\4[] Sistemas de ecuaciones con supuestos ad-hoc
				\4[] Intentos de microfundamentar partes de sistemas
				\4[] $\to$ Demanda de consumo
				\4[] $\to$ Demanda de dinero
				\4[] $\Rightarrow$ Estructura de modelos sigue siendo ad-hoc
				\4 Basar modelos en supuestos sobre:
				\4[] Preferencias
				\4[] Tecnología de producción
				\4[] Dotaciones
				\4 Concretar modelos en base a:
				\4[] Marco institucional
				\4[] Condiciones iniciales
				\4 Eliminar dicotomía macro vs. micro
			\3 Optimización intertemporal
				\4 Modelos keynesianos
				\4[] Generalmente estáticos
				\4[] $\to$ Conclusiones de estática comparativa
				\4 Agentes tienen en cuenta el futuro
				\4[] Optimización intertemporal de la utilidad
				\4[] $\to$ Agentes optimizar en t entre consumo y trabajo
				\4[] $\to$ Pero también consumo-trabajo entre $t$ y $t+1$
				\4[] Nuevas técnicas matemáticas
				\4[] $\to$ Programación dinámica
				\4[] $\to$ Métodos numéricos de resolución
				\4[] $\to$ Causalidad de Granger
				\4[] $\to$ Modelos VAR
				\4 Equilibrio intertemporal
				\4[] Equilibrio ya no es una asignación estática
				\4[] $\to$ Equilibrio es una senda completa de valores
			\3 Hipótesis de expectativas racionales
				\4 Modelos keynesianos y NeoK del deseq.
				\4[] Hipótesis ad-hoc sobre las expectativas
				\4[] Expectativas miopes
				\4[] $\to$ Los agentes esperan valor anterior
				\4[] Expectativas adaptativas
				\4[] $\to$ Agentes estiman a partir de ``sorpresa'' anterior
				\4 Definición estadística
				\4[] Los agentes minimizan el error de sus predicciones
				\4 Definición económica
				\4[] Debida a Muth (1961)
				\4[] Los agentes conocen el modelo subyacente de la economía
				\4[] Lo utilizan para predecir valores
				\4 Justificación de las expectativas racionales
				\4[] En caso contrario es posible engañar sistemáticamente
				\4[] Agentes no aprenden de sus errores
				\4[] Sin HER, agentes desaprovechan oportunidades de mejora
			\3 Equilibrio general walrasiano
				\4 Modelos keynesianos
				\4[] Excesos de demanda y oferta
				\4[] Ley de Say/Walras no siempre se cumple
				\4[] Equilibrios no son walrasianos
				\4 Equilibrio walrasiano
				\4[] Modelos NMC:
				\4[] $\to$ equilibrio siempre sin excesos demanda/oferta
				\4 Equilibrio general
				\4[] Interacción de todas vars. relevantes se tiene en cuenta
				\4[$\Rightarrow$] Escuela pionera del enfoque DSGE
				\4[] Todos los modelos tienen componentes:
				\4[] $\to$ Dinámicos: optimización intertemporal
				\4[] $\to$ E(S)tocásticos: sujetos a shocks aleatorios
				\4[] $\to$ General: tienen en cuenta todas interacciones
				\4[] $\to$ Equilibrio: los mercados se vacían
			\3 Contrastación empírica
				\4 El realismo de los supuestos no es lo más importante
				\4[] Todos los modelos toman supuestos poco realistas
				\4[] Enfoque de equilibrio es propiedad del modelo
				\4[] $\to$ No de la realidad\footnote{``\textit{You can't look out of this window and ask whether New Orleans is in equilibrium.}-- Snowdown y Vane (1997)}
				\4[] $\to$ Pero es útil para formular
				\4[] $\to$ Induce consistencia interna
				\4[] $\to$ Ausencia de equilibrio es supuesto ad-hoc
				\4 Relevante:
				\4[] Capacidad de predicción off-sample
				\4[] Manteniendo consistencia interna del modelo
	\1 \marcar{Modelos}
		\2 Oferta agregada e información imperfecta
			\3 Lucas y Rapping (1969)
				\4 Idea clave
				\4[] Construir modelo agregado del mercado de trabajo
				\4[] $\to$ En términos intertemporales
				\4[] Influencia de Friedman, Modigliani, Baumol, Tobin..
				\4[] $\to$ Microfundamentar componente de modelo SNC
				\4[] $\to$ Consumo, demanda de dinero ya habían sido explicados
				\4[] $\then$ Explicar oferta de trabajo
				\4[] Explicar factores que determinan la oferta de trabajo
				\4[] Estudios muestran:
				\4[] $\to$ A l/p: oferta de trabajo inelástica a salario real
				\4[] $\to$ A c/p: oferta de trabajo elástica a salario real
				\4[] Plantear modelo teórico de mercado agregado
				\4[] $\to$ Coherente con decisiones micro
				\4[] Contrastar con series temporales
				\4[] Oferta de trabajo como resultado de optimización intertemp.
				\4 Formulación
				\4[] Marco de equilibrio walrasiano
				\4[] Agentes racionales optimizan utilidad
				\4[] Maximización intertemporal U: C y $L_S$ presentes y futuros
				\4[] Modelo de dos periodos de Fisher
				\4[] Expectativas adaptativas sobre precios
				\4[] $\underset{c_1,c_2,n_1,n_2}{\max} \quad U(c_1, c_2, n_1, n_2)$
				\4[] $\quad \text{s.a:} \quad \quad p_1 c_1 + \frac{cp_2}{1+r} c_2 \leq A + w_1 n_1 + \frac{w_2}{1+r} n_2$
				\4[] $\quad \quad \quad \quad \quad n_1 = f(w_1, \frac{w_2}{1+r}, p_1, \frac{p_2}{1+r},A)$
				\4[] Derivadas de $f(\cdot)$ caracterizan respuesta de oferta
				\4[] Asume que:
				\4[] $\pdv{f}{w_1/p_1} > 0$,
				\4[] $\then$ Trabaja en presente cuanto mayor salario real presente
				\4[] $\pdv{f}{\frac{w_2}{p_1(1+r)}} < 0$
				\4[] $\to$ Trabaja menos en presente cuanto mayor salario real futuro
				\4[] $\pdv{f}{\frac{p_2}{p_1 (1+r)} }<0$,
				\4[] $\to$ Trabaja más ahora si precios altos respecto ``normal''
				\4[] $\then$ Curva de Phillips
				\4[] $\pdv{f}{ \frac{A}{p_1}}<0$
				\4[] $\to$ Trabaja menos en presente cuanta mayor riqueza
				\4 Implicaciones
				\4[] Oferta de trabajo responde a tres factores
				\4[] $\to$ (i) Salario real ``normal'' o permanente\footnote{``Normal'' entendido como acorde con la tendencia de largo plazo.}
				\4[] $\to$ (ii) Desviación de salario real presente vs. normal
				\4[] $\then$ Respuesta positiva
				\4[] $\then$ Aprovechan salario alto ahora para ahorrar más
				\4[] $\to$ (iii) Nivel de precios mayor que nivel de precios ``normal''
				\4[] $\then$ Respuesta positiva
				\4[] $\then$ Precios presentes por encima de ``normales''
				\4[] $\then$ El tipo de interés real será más alto
				\4[] $\then$ Merecerá más la pena ahorrar lo ganado ahora
				\4[] Responde poco a (i), mucho a (ii) y (iii).
				\4[] $\to$ Resultado de expectativas adaptativas
				\4[] $\to$ Sustitución intertemporal de $c$ y $l$
				\4[] $\to$ $\frac{l_2}{l_1} = f(\frac{w_2}{w_1})$, $f(\cdot) > 0$
				\4[] Desempleo puede entenderse como voluntario
				\4[] $\to$ Agentes estiman salario que recibirían si trabajan
				\4[] $\to$ Agentes consideran salario real demasiado bajo
				\4[$\then$] Adelanta temas de la NMC
				\4[$\then$] Inicia programa de investigación
				\4[$\then$] Rompe con marco Keynesiano de modelización
			\3 Lucas (1972)
				\4 ``Expectations and the neutrality of money''
				\4 Idea clave
				\4[] Explicar relación sistemática $Y-\Delta P$
				\4[] A partir de:
				\4[] $\to$ Agentes que no tienen ilusión monetaria
				\4[] $\to$ Agentes que optimizan utilidad intertemporal
				\4[] $\to$ Agentes que min. error de predicción $\to$ HER
				\4[] $\to$ Equilibrio walrasiano
				\4[] $\to$ Información imperfecta
				\4[] Modelo seminal de la NMC
				\4[] $\to$ Introduce todas las herramientas habituales
				\4[] Basado en:
				\4[] $\to$ Lucas Rapping 1969
				\4[] $\to$ Modelo de las islas de Phelps
				\4[] $\to$ Generaciones solapadas de Samuelson
				\4 Formulación
				\4[] Dos mercados (basado en ``islas'' de Phelps)
				\4[] Dos generaciones solapadas: viejos y jóvenes
				\4[] \underline{Jóvenes}
				\4[] $\to$ distribuidos de forma aleatoria entre mercados
				\4[] $\to$ producen consumo a cambio de dinero
				\4[] \underline{Viejos}
				\4[] $\to$ Distribuidos de modo que dda. igual en ambos mercados
				\4[] $\to$ compran consumo a cambio de dinero
				\4[] $\to$ reciben transferencia a inicio de periodo
				\4[] $\to$ transferencia proporcional a dinero ahorrado
				\4[] \underline{Oferta de trabajo}
				\4[] $\to$ Depende de valor real de dinero que obtienen
				\4[] $\to$ V. real: (dinero ahorrado + trans. futura) / precios futuros
				\4[] $\to$ Jóvenes no conocen valor real de dinero
				\4[] $\to$ Observan precio en su mercado
				\4[] $\to$ Dinero ahorrado es $\text{trabajo} \cdot \text{precio}$
				\4[] $\to$ Pero ignoran nivel de precios futuro
				\4[] $\to$ Al que podrán vender su dinero ahorrado por consumo
				\4[] $\to$ Tratan de estimar nivel de precios futuros
				\4[] $\to$ A partir de precios observados y modelo de economía
				\4[] \underline{Shocks que afectan a precio}
				\4[] Cantidad de jóvenes en cada mercado:
				\4[] $\to$ Variable aleatoria con $\tau^2$ y media conocida
				\4[] $\to$ Determina desviación frente a nivel de precios
				\4[] $\to$ A - jóvenes, + precio del bien dada = dda. nominal
				\4[] Transferencia a viejos
				\4[] $\to$ Variable
				\4[] $\to$ A + transferencias a viejos, mayor dda. nominal
				\4[] Estimación de precio real
				\4[] $\to$ Problema de ``extracción de señales''
				\4[] $\to$ Jóvenes utilizan toda la información disponible = HER
				\4[] $\to$ Estimación de precios que minimiza error
				\4[] Estimación de nivel de precios
				\4[] $\to$ conocido el precio en el mercado
				\4[] $\to$ Suma ponderada de:
				\4[] $\to$ Precio observado en el mercado
				\4[] $\to$ Nivel de precios estimado sin conocer precio en mercado
				\4[] $E(P_t | I_{t-1}, P_t^z) = \theta  P_t^z + (1- \theta) E (P_t | I_{t-1})$
				\4[] $\to$ Ponderacion: $\theta$
				\4[] $\to$ $\theta = \frac{\sigma^2}{\sigma^2 + \tau^2}$
				\4[] $\to$ $\tau^2$: var. del shock real (población joven)
				\4[] $\to$ $\sigma^2$: var. del shock nominal (transferencia a viejos)
				\4[] Si shock nominales muy variables frente a reales
				\4[] $\theta \to 1$
				\4[] $\then$ Estiman $P_t$ con precio observado $P_t^z$
				\4[] $\then$ Asumen $P_t^z$ se desvía poco de $P_t$
				\4[] $\then$ P observados resultan de estímulo a demanda
				\4[] $\then$ No reaccionan a desviaciones de $P_t^z$ observados
				\4[] $\then$ No ``toman en serio'' cambios en precios
				\4[] Si shock reales muy variables frente a nominales
				\4[] $\theta \to 0$
				\4[] $\then$ Estiman $P_t$ con información disponible
				\4[] $\then$ Asumen $P_t^z$ se desvía de $P_t$
				\4[] $\then$ P observados resultan de condiciones de oferta
				\4[] $\then$ Reaccionan a desviaciones observadas de $P_t^z$
				\4[] $\then$ Se ``toman en serio'' cambios en precios
				\4 Implicaciones
				\4[] Función de oferta agregada
				\4[] $\to$ \fbox{$ \tilde{y}_t = y_t - y_t^n= \lambda (P_t - E(P_t ))$}
				\4[] $\to$ Shocks nominales tienen efectos reales
				\4[] $\to$ No es posible engañar sistemáticamente
				\4[] $\then$ Sólo posible si sorpresa nominal
				\4[] $\to$ Parámetros de shocks determinan oferta
				\4[] $\to$ Sorpresas nominales no son posibles siempre
				\4[] $\then$ \fbox{Sólo posibles si var. de shocks nominales es baja}
				\4[] $\then$ \fbox{A l/p, desempleo igual a desempleo natural}
				\4[] $\then$ \fbox{Curva de Phillips decreciente si sorpresa}
			\3 Lucas (1973)
				\4 Idea clave
				\4[] Contrastar implicaciones de Lucas (1972)
				\4[] Relacionar varianza M con variabilidad Y y $\Delta P$
				\4[] Regresión con de conjunto de datos internacionales
				\4 Formulación
				\4[] Modelo simplificado muy similar a Lucas (72)
				\4[] Comparaciones varianzas
				\4[] Regresiones:
				\4[] $\to$ Varianza de output frente a shocks monetarios
				\4[] $\to$ Variación de precios frente a shocks monetarios
				\4 Implicaciones
				\4[] Si varianza de shocks nominales muy elevada
				\4[] $\to$ Shocks nominales se transmiten a $\Delta P$ no a Y
				\4[] Si shocks monetarios se utilizan mucho
				\4[] $\to$ se transmiten más a P y menos a Y
				\4[] Resultados empíricos:
				\4[] $\to$ Shocks monetarios fuertes inducen:
				\4[] $\then$ Fuerte volatilidad de precios
				\4[] $\then$ Poca variación del output
				\4[] $\then$ Resultados compatibles con Lucas (1972)
		\2 Ciclo nominal basado en información imperfecta
			\3 Lucas (1975)
				\4 ``An equilibrium model of the business cycle''
				\4 Idea clave
				\4[] Lucas (1972) no permite inducir persistencia
				\4[] $\to$ Correlación serial $Y_t$ y $Y_{t+1}$ es 0
				\4 Formulación
				\4[] Parte de la base de Lucas (1972) y Lucas (1973)
				\4[] Introduce lags de información
				\4[] $\to$ Shocks monetarios pasados tardan en conocerse totalmente
				\4[] Introduce capital físico
				\4[] $\to$ Permite un efecto acelerador típico del ciclo\footnote{Es decir, un estímulo a la demanda agregada con origen en el aumento de la inversión que se deriva de un aumento del producto.}
				\4 Implicaciones
				\4[] Es posible modelizar ciclos económicos en EGC
				\4[] Shocks $\Delta M$ no sistemáticos pueden dar lugar a ciclos
				\4[] En el contexto del modelo
				\4[] $\to$ Domar ciclo implica reducir varianza de $\Delta M$
			\3 Lucas (1977)
				\4 ``Understanding Business Cycles''
				\4 Idea clave
				\4[] Resumir Lucas (1975) en términos verbales
				\4[] Criticar modelos keynesianos del ciclo
				\4[] Exponer ventajas de modelo de eq. general y HER
				\4[] $\to$ No necesita oportunidades no aprovechadas sistemáticamente
				\4[] $\to$ Explicación en términos de mercados imperfectamente conectados
				\4[] $\to$ No son necesarias rigideces ad-hoc
				\4[] $\to$ No es necesario decidir qué rigidez introducir
				\4[] $\then$ Modelos de ciclos no implican fallos de mercado
		\2 Crítica de Lucas
			\3 Lucas (1976)
				\4 ``Econometric policy evaluation: a critique''
				\4 Idea clave
				\4[] Examen de los modelos macroeconométricos de la SNC
				\4[] $\to$ ¿Sirven para analizar cambios de política económica?
				\4[] $\to$ ¿Cómo pueden mejorarse los modelos?
				\4 Formulación
				\4[] Objetivo consiste en averiguar movimiento de $y$:
				\4[] $y_{t+1} = f(y_t, x_t, \epsilon_t)$
				\4[] $\to$ $x_t$ es un shock arbitrario no estocástico
				\4[] $\to$ $\epsilon_t$ es un shock estocástico
				\4[] En la práctica, se estima $F(y,x, \theta, \epsilon)$
				\4[] Donde $\theta$ es un vector de parámetros a estimar
				\4[] $\to$ Se estiman a partir de observaciones $y$, $x$, $\epsilon$
				\4[] A partir de $F(\cdot)$ y $\theta$, se puede:
				\4[] $\to$ Estimar efectos de políticas variando $x_t$
				\4[] Pero hay un problema:
				\4[] $\to$ $\theta$ estimado a partir de $x_t$, $y_t$ del pasado
				\4[] $\to$ ¿$\theta$ es estable ante cambios en $x$?
				\4[] $\to$ ¿Agentes reaccionan = ante políticas diferentes?
				\4[] En el c/p, $\theta$ a veces se ajusta lentamente
				\4[] $\to$ Falsa sensación de estabilidad de parámetros estimados
				\4[] $\to$ Predicciones erróneas del efecto de políticas
				\4 Implicaciones
				\4[] Estructura de modelos econométricos depende de:
				\4[] $\to$ Reglas de decisión óptima de agentes
				\4[] Reglas de decisión óptima cambian si:
				\4[] $\to$ Cambian reglas de política económica relevantes
				\4[] $\then$ Cambios en política cambian estructura de modelos
				\4[] Estimación de ecuaciones en forma reducida
				\4[] $\to$ No tienen en cuenta reacciones de agentes
				\4[] Macroeconometría debe tener en cuenta:
				\4[] $\to$ Parámetros profundos y estables que inducen $\theta$
				\4[] $\then$ Necesario estimar sistemas estructurales
				\4[] $\then$ Necesaria coherencia con decisión óptima
		\2 Política económica
			\3 Sargent y Wallace (1973)
				\4 ``Hyperinflation and rational expectations''
				\4 Idea clave
				\4[] Influencia de Cagan (1956) sobre hiperinflación
				\4[] $\to$ Agentes esperan inflación dada inflación pasada
				\4[] $\to$ Gobierno necesita más M para financiar $g$
				\4[] $\then$ Hiperinflación posible sin déficit púb. desbocado
				\4[] $\then$ Por inercia de expectativas adaptativas
				\4[] Incipiente generalización de expectativas irracionales
				\4[] ¿Es posible hiperinflación con HER?
				\4 Formulación
				\4[] Gobierno financia déficit creando M
				\4[] Agentes demanda dinero en función de inflación
				\4[] $\to$ Asumiendo ec. de Fisher
				\4[] $\to$ Asumiendo dda. dinero en fción. de $y$ e $i$
				\4[] Estimación de inflación con HER
				\4[] $\to$ Teniendo en cuenta senda de déficit del gobierno
				\4[] $\to$ Teniendo en cuenta aumento de M e inflación pasadas
				\4[] Espiral hiperinflacionaria
				\4[] $\to$ Agentes estiman inflación
				\4[] $\to$ Gobiernos necesitan mas M para financiar dada $\pi$
				\4[] $\then$ Hiperinflación
				\4 Implicaciones
				\4[] Expectativas adaptativas pueden resultar de HER
				\4[] Hiperinflación posible sin supuesto ad-hoc
				\4[] $\to$ HEA resulta de HER
				\4[] Mejor estimación racional de aumento de M
				\4[] $\to$ Resulta de optimización, no supuestos ad-hoc
				\4 Valoración
				\4[] Hiperinflaciones pueden ser racionales
				\4[] $\to$ No es necesario postular expectativas miopes
				\4[] Aplicación de HER más allá de oferta de trabajo
			\3 Sargent y Wallace (1975)
				\4 ``Rational'' expectations, the Optimal Monetary Instrument, and the Optimal Money Supply Rule''
				\4 Idea clave
				\4[] Influencia de Poole (1970) y Wicksell
				\4[] Poole (1970)
				\4[] $\to$ Comparar PM sobre interés o sobre M
				\4[] Wicksell
				\4[] $\to$ Si estabilidad hacia pleno empleo
				\4[] $\then$ Curva de Phillips vertical
				\4[] $\to$ Con regla de interés
				\4[] $\then$ Cualquier senda de precios es posible
				\4[] $\then$ Ninguna regla de interés determina nivel de precios
				\4[] Lucas (1972)
				\4[] $\to$ Curva de Phillips decreciente compatible con HER
				\4[] ¿Qué efectos tiene regla de M o $i$ sobre precios?
				\4[] Analizar variantes de políticas monetarias
				\4[] Cuando el público tiene expectativas racionales
				\4[] $\to$ Neutralidad de largo plazo
				\4[] Alternativas:
				\4[] $\to$ Regla de oferta monetaria
				\4[] $\to$ Regla de tipo de interés
				\4 Modelo de la economía
				\4[] Modelos ad-hoc que relacionan:
				\4[] $\to$ Precio
				\4[] $\to$ Output
				\4[] $\to$ Oferta monetaria
				\4[] $\to$ Tipo de interés
				\4[] $\then$ Similar a IS-LM
				\4[] $\then$ Dinámico
				\4[] $\then$ Curva de oferta de Phillips
				\4[] Similar a IS-LM pero;
				\4[] $\to$ dinámico
				\4[] $\to$ Curva de oferta de Phillips
				\4[] Minimizar función de pérdida sobre Y e inflación
				\4[] Oferta agregada: output producido
				\4[] $\to$ Stock de capital: aumenta output
				\4[] $\to$ Sorpresa de inflación: crece
				\4[] Demanda agregada:
				\4[] $\to$ Stock de capital: aumenta dda
				\4[] $\to$ Interés real: reduce dda.
				\4[] $\to$ Variables exógenas
				\4[] Mercado monetario:
				\4[] $\to$ Oferta monetaria iguala demanda
				\4[] $\to$ Dda. crece con output
				\4[] $\to$ Dda. crece con interés
				\4[] $\to$ Dda. sufre shocks aleatorios
				\4 Función de pérdida
				\4[] Cuadrática en desviaciones de $y$ y $p$
				\4[] $\to$ Respecto valores de equilibrio
				\4 Decisión de autoridad monetaria
				\4[] Dos estrategias disponibles
				\4[] i. Fijar tipo de interés
				\4[] $\to$ Mediante regla de feedback
				\4[] $\to$ Fija oferta monetaria para mantener cte. r.
				\4[] $\then$ Fija $r$ que minimiza pérdida
				\4 Implicaciones
				\4[] Dado modelo ad-hoc utilizado:
				\4[] $\to$ Regla de M no afecta volatilidad de output
				\4[] $\to$ Regla de interés deja indeterminado el precio
				\4[] Es compatible que:
				\4[] $\to$ $\Delta M$ esperados no tienen efectos reales a l/p
				\4[] $\to$ Curvas de Phillips crecientes a largo plazo
				\4[] Abre camino modelos explícitos de PM con HER
				\4 Valoración
				\4[] Aplicación de HER a reglas monetarias óptimas
				\4[] Formulación incipiente de modelos posteriores
				\4[] Sin análisis de bienestar explícito
				\4[] $\to$ Necesaria función ad-hoc de pérdida
				\4[] Plantea PM como basada en reglas
				\4[] $\to$ Frente a discrecionalidad que prevalecía
			\3 Kydland y Prescott (1977)\footnote{Mejorar con KVA (2004).}
				\4 ``Rules Rather than Discretion: The Inconsistency of Optimal Plans''
				\4 Idea clave
				\4[] Mostrar condiciones para que:
				\4[] $\to$ Reglas $\succ$ discreccionalidad
				\4[] Carácter general
				\4[] Aplicando ejemplos
				\4 Formulación
				\4[] Ejemplo de dos periodos
				\4[] Función de bienestar social $S(x_1, x_2, \pi_1, \pi_2)$
				\4[] $\to$ $x_1, x_2$: reacciones de los agentes
				\4[] $\to$ $\pi_1, \pi_2$: políticas implementadas
				\4[] $\to$ $x_1 = X_1(\pi_2, \pi_1)$
				\4[] $\to$ $x_2 = X_2(\pi_2, x_1(\pi_2))$
				\4[] Política consistente implica:
				\4[] $\to$ No se puede cambiar pasado
				\4[] $\then$ Decisiones pasadas se toman como dadas
				\4[] $\underset{\pi_2}{\max} \quad S(x_1, x_2(\pi_2), \pi_1, \pi_2)$
				\4[(i)] $\text{CPO:}$ $\pdv{S}{\pi_2} + \pdv{S}{X_2}\pdv{X_2}{\pi_2}=0$
				\4[] $\to$ Óptimo no depende de $x_1$
				\4[] Pero maximizar bienestar implica:
				\4[] $\to$ Considerar secuencia completa $\pi_1$, $\pi_2$
				\4[] $\to$ Considerar todas las reacciones $x_1(\pi_1, \pi_2)$, $x_2(x_1, \pi_1, \pi_2)$
				\4[] Problema se convierte en:
				\4[] $\underset{\pi_2}{\max} \quad S\left(\pi_1, \pi_2, x_1(\pi_2), x_2 \left( \pi_2, x_1 (\pi_2) \right) \right)$
				\4[(ii)] $\text{CPO:}$ $\pdv{S}{\pi_2} + \pdv{S}{X_1} \pdv{X_1}{\pi_2} + \pdv{S}{X_2} \left( \pdv{X_2}{X_1} \pdv{X_1}{\pi_2} + \pdv{X_2}{\pi_2} \right) =0$
				\4 Implicaciones
				\4[] Política consistente sólo es óptima si:
				\4[] $\to$ (i) y (ii) son idénticas
				\4[] $\then$ $\pdv{X_1}{\pi_2} = 0$
				\4[] $\then$ Reacción presente no considera política futura
				\4[] La modelización de las expectativas
				\4[] $\to$ Cambia las implicaciones de los modelos
				\4[] $\to$ Cambia resultados de política económica
				\4[] Crítica del control óptimo
				\4[] $\to$ No apropiada para planificación dinámica de una economía
				\4[] $\to$ No basta con ir optimizando dado pasado
				\4[] $\to$ Porque agentes tienen en cuenta políticas futura
				\4[] $\then$ No se alcanzarán óptimos first-best
				\4 Programación dinámica
				\4[] Herramienta adecuada para hallar políticas óptimas
				\4[] Tiene en cuenta expectativas de agentes sobre futuro
				\4[] Resistente a inconsistencia temporal
				\4[] Necesita posibilidad de commitment
				\4 Política monetaria
				\4[] Autoridad monetaria minimiza función de pérdida
				\4[] $\to$ Desempleo por encima de equilibrio
				\4[] $\to$ Inflación diferente de cero
				\4[] Con HER:
				\4[] $\to$ Agentes conocen incentivos de autoridad monetaria
				\4[] $\to$ Estiman inflación sin errores sistemáticos
				\4[] Primero, agentes forman expectativas
				\4[] Después, autoridad monetaria decide inflación
				\4[] $\to$ Incentivos a inflación por encima de esperado
				\4[] $\to$ Agentes conocen incentivos
				\4[] $\then$ Esperan inflación más elevada
				\4[] $\then$ ENPS con inflación mayor a 0 y output natural
				\4[] $\then$ Subóptimo por política óptima inconsistente
				\4[] \grafica{kydlandprescott1977}
				\4 Valoración
				\4[] Enorme influencia sobre diseño de políticas
				\4[] Familia de modelos basados en Kydland y Prescott (1977)
				\4[] $\to$ Barro y Gordon (1983) sobre PM óptima
				\4[] $\to$ Rogoff (1985) sobre independencia de BC
				\4[] $\to$ Lucas y Stokey (1983) sobre deuda e impuestos
			\3 Sargent y Wallace (1981)
				\4 ``Some unpleasant monetarist arithmetic''
				\4 Idea clave
				\4[] ¿Cómo interaccionan política fiscal y monetaria?
				\4[] ¿Bajo qué supuestos PF determina inflación?
				\4[] ¿Cuando PM puede controlar la inflación?
				\4[] $\then$ Dadas ciertas condiciones, PM depende de PF
				\4[] $\then$ Inflación depende de PF, no de PM
				\4[] PM contractiva en presente y dominancia fiscal
				\4[] $\to$ Puede provocar inflación más alta hoy y mañana
				\4 Formulación
				\4[] Senda de déficit fiscal exógena y arbitraria
				\4[] Supuesto de no-default
				\4[] Deuda crece más que economía
				\4[] Cuatro fuentes de financiación del déficit\footnote{Primario y general.}
				\4[] $\to$ Más impuestos ($\uparrow T$)
				\4[] $\to$ Menor gasto ($\downarrow G$)
				\4[] $\to$ Emisión de deuda ($\uparrow B$)
				\4[] $\to$ Señoreaje ($\uparrow M$)
				\4[] Ecuación cuantitativa del dinero
				\4[] Autoridad fiscal determina unilateralmente
				\4[] $\to$ Impuestos
				\4[] $\to$ Gasto
				\4[] Emisión de deuda tiene límites
				\4[] $\to$ Deuda se vende a menor precio o no se vende
				\4[] Si el gasto aumenta:
				\4[] $\to$ Señoreaje debe cubrir lo que $\uparrow T$, $\downarrow G$, $\uparrow B$ no cubre
				\4[] Senda de precios de equilibrio no es única
				\4[] Pueden existir múltiples sendas de precios de equilibrio
				\4[] $\to$ Más inflación al principio, menos después
				\4[] $\to$ PM restrictiva al principio $\to$ expansiva después
				\4 Implicaciones
				\4[] Germen de ``teoría fiscal del nivel de precios''
				\4[] Si política fiscal domina y agentes lo saben
				\4[] $\to$ PM deberá adaptarse tarde o temprano
				\4[] $\to$ Agentes estiman inflación más elevada
				\4[] $\then$ Sesgo inflacionario ya en presente
				\4[] $\then$ PM contractiva insostenible genera --Y, + $\pi$
				\4[$\then$] Si B. Central no es independiente
				\4[]$\to$ inflación antes o después
			\3 Barro y Gordon (1983a) y (1983b)
				\4 Idea clave
				\4[] Desarrollan Kydland y Prescott (1977)
				\4[] Se centran en trade-off empleo-inflación
				\4[] Análisis positivo
				\4[] $\to$ resultados de política discrecional
				\4[] Comparar:
				\4[] $\to$ Reglas
				\4[] $\to$ Política discrecional
				\4[] Valorar mecanismos de reputación
				\4 Formulación
				\4[] BCentral minimizador de inflación--desempleo
				\4[] $S(u_t,\pi_t) = -\frac{1}{2} \left( u_t - k u^* \right)^2 - \frac{1}{2} \gamma \pi_t^2$
				\4[] $u^*$ es desempleo de equilibrio
				\4[] $\to$ Imposibles desviaciones consistentes
				\4[] Expectativas racionales:
				\4[] $E(\pi_t) = \pi_t$
				\4[] $\to$ Desempleo siempre iguala desempleo de equilibrio
				\4[] $\then$ Salvo sorpresas no sistemáticas
				\4[] Proceso de decisión
				\4[] 1. Agentes privados forman expectativas sobre precios
				\4[] 2. Autoridad monetaria decide inflación
				\4[] $\to$ Puede crear sorpresas de inflación
				\4[] $\then$ Sorpresas reducen desempleo
				\4[] $\to$ Si poca inflación esperada, fuerte efecto
				\4[] $\then$ Incentivo para aumentar inflación
				\4[] Agentes con HER conocen incentivos
				\4[] $\to$ Saben que plan óptimo no es consistente sin commitment
				\4[] $\then$ Esperan inflación alta en el futuro
				\4[] $\then$ Resultado subóptimo
				\4[] i. Crecimiento excesivo de inflación y M
				\4[] ii. Pendiente de curva de Phillips es importante
				\4[] iii. Autoridad monetaria actúa contracíclicamente
				\4[] iv. Desempleo acaba siendo independiente de PM
				\4[] Compromiso con regla de PM
				\4[] $\to$ Afecta positivamente a resultados
				\4[] Prestigio de BCentral
				\4[] $\to$ Potencial para saltarse commitment
				\4[] Descuento subjetivo alto
				\4[] $\to$ Dificulta uso de prestigio
				\4 Implicaciones
				\4[] Reglas son preferibles a discrecionalidad
				\4[] $\to$ Cumplimiento de reglas debe ser creíble
				\4[] $\to$ Deseables incentivos que induzcan cumplimiento
				\4[] En situaciones excepcionales
				\4[] $\to$ Aceptables decisiones discrecionales
				\4[] Descuento subjetivo determina adherencia a reglas
				\4[] Prestigio de instituciones es muy importante para que:
				\4[] $\to$ Discrecionalidad excepcional sea efectiva
				\4[] $\to$ Efectividad de PM se mantenga en futuro
				\4[] $\to$ Abaratar la señalización del commitment
		\2 Modelo del ciclo real
			\3 Kydland y Prescott (1982)
			\3 Long y Plosser (1983)
				\4 Idea clave
				\4[] Fluctuaciones cíclicas a partir de:
				\4[] $\to$ Shocks reales de todo tipo
				\4[] $\to$ Productividad, gasto público, preferencias...
				\4[] Shocks nominales no son necesarios
				\4[] $\to$ Para replicar momentos de dist. reales
				\4 Formulación
				\4[] Basado en modelo RCK con ajustes
				\4[] Oferta de trabajo endógena
				\4[] $\to$ Optimización inter e intra temporal
				\4[] $\to$ Consumo y trabajo
				\4[] Calibración
				\4[] $\to$ Elección de valor de parámetros
				\4[] $\to$ A partir de estudios micro y macro
				\4[] $\to$ Generación estado estacionario
				\4[] $\to$ Introducción de shocks aleatorios
				\4[] $\to$ Comparación de momentos con series reales
				\4 Implicaciones
				\4[] Estabiliza enfoque NMC
				\4[] Posible estudiar ciclo separando real y monetario
				\4[] Estabiliza programa DSGE
				\4[] Abre camino a modelos monetarios del ciclo
				\4[] $\to$ sin información imperfecta como Lucas (1975)
	\1 \marcar{Implicaciones}
		\2 Política económica
			\3 Idea clave
				\4 No necesariamente propuestos por autores de NMC
				\4 NMC influye sobre:
				\4[] $\to$ Economía teórica
				\4[] $\to$ Policy-makers
				\4 Policy-makers influenciados por NMC
				\4[] $\to$ Derivan conjunto de implicaciones
				\4[] $\to$ Aplican políticas basadas en implicaciones
			\3 Reglas vs discrecionalidad
				\4 Expectativas son elemento central de efectos de políticas
				\4 Discrecionalidad es muy difícil de evaluar
				\4[] Expectativas de agentes pueden volverse inestables
				\4[] Incentivos de policy-maker cambian con tiempo
				\4[] Aparición de inconsistencia intertemporal
				\4 Reglas sí pueden evaluarse
				\4[] Es posible modelizar expectativas
				\4[] Expectativas racionales tienen sentido
				\4[] Reglas creíbles permiten mejoras de Pareto
				\4[$\then$] Política económica debe basarse en reglas
				\4[$\then$] Reglas no creíbles inducen subóptimos
				\4[$\then$] Posibilidad de commitment es determinante
				\4[$\then$] Deseables incentivos a cumplir reglas
			\3 Inefectividad de la política monetaria
				\4 Estímulos esperados no tienen efectos reales
				\4[] Resultado de HER
				\4[] Agentes toman decisiones respecto variables reales
				\4[] $\to$ Anticipan cambios en variables nominales
				\4[] $\then$ Se anticipa ausencia de efecto real de cambio nominal
				\4 Evaluación de reglas de PM
				\4[] Políticas discreccionales muy dificil evaluación
				\4[] $\to$ Expectativas inestables si discreccionalidad
				\4[$\then$] PM tiene capacidad limitada para est. ciclo
				\4[$\then$] Críticas a regla de $\Delta M$ de Friedman
			\3 Políticas de oferta
				\4 Argumentos en contra de políticas de demanda
				\4[] $\to$ Economías estables tienden a valor natural
				\4[] $\to$ No existe trade-off sostenible inflación-output
				\4 Objetivo de la política económica
				\4[] Centrarse en aumentar output de equilibrio
				\4[] $\to$ Inversión
				\4[] $\to$ Productividad
				\4[] $\to$ Oferta de trabajo
		\2 Teoría económica
			\3 Modelos DSGE
				\4 Programa de investigación se consolida
				\4 Elemento central de doctorados en macroeconomía
				\4 Introducción paulatina en bancos centrales
				\4 Débil introducción en policy-making
				\4 Elemento básico de NEK de 2a generación
			\3 Crítica de Lucas
				\4 Práctica totalidad de modelos tienen en cuenta
				\4[] Al menos, hacen referencia a ella
				\4[] O justifican si falta de inmunidad
			\3 Expectativas racionales
				\4 Omnipresente en macroeconomía mainstream
				\4 Recibe críticas
				\4[] Pero difícil encontrar hipótesis alternativas
				\4[] Difícil justificar otra hipótesis
				\4[] $\to$ Alternativas tiene carácter ad-hoc
	\1[] \marcar{Conclusión}
		\2 Recapitulación
			\3 Visión general de la Nueva Macroeconomía Clásica
			\3 Modelos más relevantes
			\3 Implicaciones
		\2 Idea final
			\3 Robert Solow sobre modelos macro y economistas
				\4 Existen dos tipos de macroeconomistas
				\4 Macroeconomistas que formulan modelo canónico
				\4[] Y tratan de resolver todas las preguntas con el
				\4[] $\to$ Aplicando ligeros cambios
				\4 Macroeconomistas que utilizan un conjunto de modelos
				\4[] Cada uno para resolver diferentes cuestiones
				\4 Nueva Macroeconomía Clásica
				\4[] Ambición de modelo general desde primer momento
				\4[] Aunque primeros modelos, realmente parciales
				\4[] $\to$ Mercado de trabajo
				\4[] $\to$ Consistencia temporal
				\4[] $\to$ Elección de instrumentos
				\4 Programa de investigación del RBC
				\4[] Modelo general de la macroeconomía
				\4[] Ambición integradora
			\3 Controversias
				\4 Múltiples y perennes
				\4[] NMC es objeto de críticas habituales
				\4[] Grandes nombres critican enfoque
				\4[] Especialmente, RBC y expectativas racionales
				\4 NMC tiene una agenda ideológica
				\4[] Afirma un postulado
				\4[] $\to$ Intervención pública es inefectiva o perjudicial
				\4[] Trata de justificarlo
				\4 Supuestos poco realistas
				\4[] Supuestos de la NMC fuera de realidad
				\4[] Lucas lleva al extremo Friedman sobre supuestos
			\3 Victoria metodológica
				\4 Expectativas racionales y equilibrio
				\4[] Alternativas implican supuestos ad-hoc
				\4[] $\to$ Pueden ser igual de buenos que cualquier otro
				\4[] $\to$ No están basados en axiomas subyacentes
				\4[] $\to$ Dependen de resultados empíricos variables
				\4[] $\to$ Difícil caracterizar irracionalidad
				\4 Sin alternativas a paradigma NMC
				\4[] Modelizaciones alternativas son marginales
				\4[] Nueva Economía Keynesiana actual heredera de NMC
				\4[] $\to$ Marco NMC con modificaciones
				\4 Modelización DSGE
				\4[] Predominante en literatura académica
				\4[] Eje central de programas de doctorado
				\4[] Se abre camino en policy-making
\end{esquemal}























%\seccion{V1}
%\begin{esquemal}
%	\1 Introducción
%		\2 Contextualización
%			\3 Macroeconomía
%				\4 Análisis de las economías a nivel agregado
%				\4 Entender y predecir evolución
%			\3 Política económica
%				\4 Qué cambios aplicar a variables que estado controla
%				\4 Para afectar afectar a variables consideradas importantes
%			\3 Modelo de la realidad
%				\4 Elemento básico de la política económica
%				\4[$\to$] ¿Cómo afectarán cambias en unas variables 
%			\3 Diferentes escuelas
%		\2 Objeto
%		\2 Estructura
%	\1 Aspectos teóricos de la NMC
%		\2 Idea clave
%		\2 Antecedentes
%			\3 Microeconomía
%			\3 Expectativas racionales
%			\3 Desarrollos matemáticos
%			\3 Tinbergen
%			\3 Macroeconomía tradicional
%			\3 Teoría de juegos
%		\2 Características comunes
%		\2 Crítica de Lucas
%		\2 Enfoque monetario
%		\2 Teoría del Ciclo Real
%		\2 Influencia en otras escuelas
%			\3 Nueva Economía Keynesiana
%	\1 Implicaciones de política económica
%		\2 Discreccionalidad y credibilidad
%		\2 Bancos Centrales
%	\1 Conclusión
%		\2 Recapitulación
%			\3 Aspectos teóricos
%				\4 Puntos clave
%				\4 Antecedentes
%				\4 Influencia en otras escuelas
%				\4 Críticas
%			\3 Implicaciones de política económica
%		\2 Idea final
%			\3 Revolución de la teoría macroeconómica
%			\3 Límites
%\end{esquemal}


\graficas

\begin{axis}{4}{Kydland y Prescott (1977): inconsistencia de la política monetaria óptima y sesgo inflacionario resultante.}{$u_t - u^*$}{$\pi_t$}{kydlandprescott1977}
	% Extensión de ejes
	\draw[-] (-3,0) -- (0,0); % abscisas
	\draw[-] (0,0) -- (0,-3); % ordenadas
	
	% Curvas de Phillips
	\draw[-] (-3,4) -- (3,-4);
	\draw[-] (-3,5.7) -- (3,-2.3);
	
	% Curvas de indiferencia de función de pérdida
	\draw[-] (-3,3) to [out=-20, in=90](0,0) to [out=270, in=20](-3,-3);
	\draw[-] (-3,3.53) to [out=-20, in=90](0.53,0) to [out=270, in=20](-3,-3.53);
	
	% Óptimo
	\node[circle,fill=black,inner sep=0pt,minimum size=4pt] (a) at (0,0) {};	
	\node[above] at (-0.45,0){O};
	
	% Equilibrio
	\node[circle,fill=black,inner sep=0pt,minimum size=4pt] (a) at (0,1.8) {};
	\node[right] at (0,1.8){B};
	
\end{axis}

El punto A muestra el óptimo alcanzable en presencia de commitment. En ausencia de commitment, el equilibrio es el punto B, en el que el desempleo es el mismo que en el óptimo pero la inflación es mayor. Una vez que se forman las expectativas, la autoridad monetaria tiene incentivos a tratar de situarse en una curva más a la derecha para reducir la función de pérdida aumentando la inflación. Sin embargo, los agentes estiman este comportamiento y la curva de Phillips se desplaza hacia arriba. En el equilibrio, el output es el natural pero la inflación es más alta que si hubiese existido la posibilidad de comprometerse a mantener la inflación baja.

\preguntas

\seccion{Test 2018}

\textbf{14.} ¿Cuál de los siguientes \textbf{NO} es uno de los supuestos básicos de la Nueva Macroeconomía Clásica? 

\begin{itemize}
	\item[a] La existencia de rigideces nominales.
	\item[b] La hipótesis de las expectativas racionales.
	\item[c] El vaciado continuo de los mercados.
	\item[d] Todas las opciones anteriores son supuestos básicos de la Nueva Macroeconomía Clásica.
\end{itemize}


\seccion{Test 2014}

\textbf{21}. En el modelo de Lucas (1972), caso de información imperfecta, ocurrirá que:

\begin{enumerate}
	\item[a] La desviación de la producción de su nivel normal será función creciente de la diferencia del nivel de precios esperado y el nivel actual de los precios.
	\item[b] Una demanda agregada superior a la esperada elevará tanto la producción como los precios.
	\item[c] Una demanda agregada superior a la esperada elevará sólo los precios, ya que el modelo de Lucas es monetarista.
	\item[d] Una demanda agregada superior a la esperada elevará sólo la producción, mientras los precios se mantendrán constantes.
\end{enumerate}

\seccion{Test 2013}
\textbf{21}. Si los agente formulan sus expectativas racionalmente en el sentido de Muth, entonces:

\begin{enumerate}
	\item[a] Los agentes no cometen errores de predicción sistemáticos.
	\item[b] El error de predicción es nulo en un mundo bajo predicción perfecta.
	\item[c] Los agentes emplean toda la información disponible.
	\item[d] Cualquiera de las respuestas anteriores es correcta.
\end{enumerate}

\seccion{Test 2011} 

\textbf{13}. La expresión $E(Y_{t+1} - Y^e_{t+1}) = 0$, donde E es el operador de esperanzas y $Y^e_{t+1}$ es la expectativa para la variable Y.

\begin{enumerate}
	\item[a] Es una propiedad de las expectativas racionales, según la cual no se cometen errores sistemáticos por parte de los agentes.
	\item[b] Es un equilibrio del modelo Y.
	\item[c] Es una propiedad de las expectativas adaptativas.
	\item[d] Significa que los errores son nulos cuando los agentes son racionales.
\end{enumerate}

\seccion{Test 2009}

\textbf{22}. Bajo la hipótesis de expectativas racionales:

\begin{enumerate}
\item[a] La crítica de Lucas es menos relevante.
\item[b] Los agentes privados conocen las reglas de política económica futuras.
\item[c] Los agentes privados analizan el efecto de las políticas económicas que prevén.
\item[d] Los agentes privados no cometen errores de previsión.
\end{enumerate}

\seccion{Test 2008}

\textbf{17}. De acuerdo con la Nueva Macroeconomía Clásica y suponiendo que los individuos forma y revisan racionalmente expectativas, si aquellos esperan que ante la situación económica imperante en el país, se va a aumentar la oferta monetaria:

\begin{enumerate}
	\item[a] Habrá aumento de salarios y de precios, al existir expectativas inflacionistas.
	\item[b] Habrá subida de precios.
	\item[c] No sucederá nada, ya que de acuerdo con esta escuela, la política económica es neutral.
	\item[d] Ninguna de las anteriores.
\end{enumerate}

\seccion{Test 2006}

\textbf{2}. Indique cual de las siguientes proposiciones acerca de la Crítica de Lucas es CORRECTA:

\begin{enumerate}
	\item[a] Cuestionaba la existencia de proposiciones falsables en la teoría económica neoclásica.
	\item[b] Cuestionaba que el modelo keynesiano constituyese un paradigma en el sentido de Kuhn.
	\item[c] Cuestionaba que la política económica pudiera afectar a través de las expectativas de los agentes el valor estimado de los parámetros en los modelos macroeconométricos.
	\item[d] Ninguna de las anteriores.
\end{enumerate}

\seccion{Test 2005}
\textbf{17}. Suponga el modelo que representa una economía cerrada bajo los supuestos: a) curva de oferta de Lucas, b) precios flexibles, c) expectativas racionales, d) información completa. Se sabe que, en equilibrio, la producción fluctúa alrededor de la producción natural.

\begin{enumerate}
	\item[a] El origen de las fluctuaciones está en los cambios en las políticas fiscales que realiza la autoridad política.
	\item[b] El origen de las fluctuaciones está en los cambios en la política monetaria que realiza la autoridad política.
	\item[c] El origen de las fluctuaciones está en las realizaciones de los componentes no esperados de las políticas.
	\item[d] El origen de las fluctuaciones no está en las perturbaciones de oferta.
\end{enumerate}

\seccion{Test 2004}
\textbf{1}. Indique cuál de las siguientes opciones recoge tres supuestos de partida de la conocida como curva de oferta de Lucas:
\begin{enumerate}
	\item[a] Hipótesis de expectativas racionales, mercados perfectamente competitivos y falta de \comillas{superneutralidad} del dinero.
	\item[b] Hipótesis de expectativas racionales, mercados perfectamente competitivos e información perfecta.
	\item[c] Hipótesis de expectativas racionales, mercados perfectamente competitivos y rigideces de precios.
	\item[d] Hipótesis de expectativas racionales, mercados perfectamente competitivos e información imperfecta.
\end{enumerate}

\textbf{22}. En el contexto de la Nueva Macroeconomía Clásica, entre las siguientes afirmaciones: 

\begin{enumerate}
	\item[i] Un aumento en el grado de competencia en los mercados de bienes provoca un deterioro en el nivel de empleo en la economía.
	\item[ii] Un incremento en la demanda de bienes va asociado a un aumento en el nivel de empleo en la economía.
\end{enumerate}

\begin{enumerate}
	\item[a] La afirmación i) es cierta, pero la afirmación ii) es falsa.
	\item[b] La afirmación i) es falsa, pero la afirmación ii) es cierta.
	\item[c] Ambas afirmaciones son falsas.
	\item[d] Ambas afirmaciones son ciertas.
\end{enumerate}


\seccion{28 de marzo de 2017}
\begin{itemize}
    \item ¿Podría decir el cuál es el papel de las TIC para el desarrollo de las economías de los PED? ¿Conoce alguna recomendación de algún organismo internacional al respecto?

    \item ¿Cuáles han sido principales contribuciones de la OMC respecto a los PED?

    \item ¿Son sinónimos crecimiento económico y desarrollo económico? ¿Qué variables considera el IDH?

    \item ¿Qué son los PMA? ¿Qué ventajas comerciales tienen en el caso de la UE?

    \item ¿Puede un PD pasar a ser un PED?
\end{itemize}

\notas

\textbf{2018}: \textbf{14.} A

\textbf{2014}: \textbf{21}. B

\textbf{2013}: \textbf{21}. D

\textbf{2011}: \textbf{13}. A

\textbf{2009}: \textbf{22}. C

\textbf{2008}: \textbf{17}. A

\textbf{2006}: \textbf{2}. D

\textbf{2005}: \textbf{17}. C

\textbf{2004}: \textbf{1}. D \textbf{22}. C

\bibliografia

Mirar en Palgrave:
\begin{itemize}
	\item business cycle measurement
	\item fiscal theory of the price level
	\item Lucas, Robert (born 1937)
	\item Lucas critique
	\item natural rate of unemployment
	\item new classical macroeconomics
	\item optimum quantity of money
	\item Ramsey model
	\item rational expectations
	\item rational expectations models, estimation of
	\item real business cycles
	\item Ricardian equivalence theorem
	
\end{itemize}

Chari, V. V. \textit{Nobel Laureate Robert E. Lucas, Jr.: Architect of Modern Macroeconomics} (1998) Journal of Economic Perspectives -- En carpeta del tema

De Vroey, M. \textit{A History of Macroeconomics from Keynes to Lucas and Beyond} (2016)

Febrero Devesa, R. \textit{La moderna macroeconomía neoclásica y sus consecuencias para la formulación de la política económica}. (Carpeta del tema)

Heijdra, B. J. \textit{Foundations of Modern Macroeconomics} (2017) 3rd ed. -- En carpeta Macro

KVA (2004) \textit{Finn Kydland and Edward Prescott's Contribution to Dynamic Macroeconomics: The Time Consistency of Economic Policy and the Driving Forces Behind Business Cycles} Advanced information on the Bank of Sweden Prize in Economic Sciences in Memory of Alfred Nobel: 11 October 2004 -- En carpeta del tema

Kydland, F. E.; Prescott, E. C. \textit{Rules Rather than Discretion: The Inconsistency of Optimal Plans} (1977) JOurnal of Political Economy -- En carpeta del tema

Lucas, R. E. \textit{Expectations and the Neutrality of Money} (1972) -- En carpeta del tema

Lucas, R. E. \textit{Econometric Policy Evaluation: A Critique} (1976) -- En carpeta del tema

Lucas, R. E. \textit{Understanding Business Cycles} (1976) -- En carpeta del tema

Lucas, R. E. \textit{Monetary Neutrality} (1995) Nobel Prize Lecture -- En carpeta del tema

Muth, J. \textit{Rational Expectations and the Theory of Price Movements}. (1961) Econometrica 29 

Plosser, C. I. \textit{Understanding Real Business Cycles} (1989) Journal of Economic Perspectives -- En carpeta del tema

Romer, D. \textit{Advanced Macroeconomics} 4a, 3a, 2a eds. -- En carpeta Macro

Sargent, T. J. \textit{Nobel Lecture: United States Then, Europe Now} (2012) Nobel Prize Lecture -- En carpeta del tema

Sargent, T. J.; Wallace, N. \textit{Rational expectations and the Dynamics of Hyperinflation} (1973) International Economic Review -- En carpeta del tema

Sargent, T. J.; Wallace, N. \textit{``Rational'' Expectations, the Optimal Monetary Instrument, and the Optimal Money Supply Rule} (1975) Journal of Political Economy -- En carpeta del tema

Sargent, T. J.; Wallace, N. \textit{Rational expectations and the theory of economic policy} (1976) Journal of Monetary Economics -- En carpeta del tema

Sargent, T. J.; Wallace, N. \textit{Some Unpleasant Monetarist Arithmetic} (1981) Federal Reserve Bank of Minneapolis. Quarterly Review: Fall -- En carpeta del tema

Snowdon, B.; Vane, H. R. \textit{Transforming macroeconomics: an interview with Robert E. Lucas Jr.} (1998) Journal of Economic Methodology -- En carpeta del tema

Snowdon, B.; Vane, H. R. \textit{Modern Macroeconomics. Its Origins, Development and Current State} (2005) Edward Elgar Publishing --  En carpeta Macroeconomía

Wallace, N. \textit{An Interview with Neil Wallace} (2013) Federal Reserve Bank of Chicago -- En carpeta del tema

Wallace, N. \textit{Interview with Neil Wallace} (2013) The Region. December issue. \url{https://minneapolisfed.org/publications/the-region/interview-with-neil-wallace}

\end{document}
