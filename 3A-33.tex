\documentclass{nuevotema}

\tema{3A-33}
\titulo{Teorías de la demanda de inversión. Implicaciones de política económica.}

\begin{document}

\ideaclave

La inversión bruta (en términos de contabilidad nacional \textit{formación bruta de capital fijo}) es uno de los principales componentes del producto interior bruto. En 2015, representó un 20,1\% del producto nacional. Además de este importante peso relativo, la inversión tiene una importancia capital en la aparición de ciclos económicos y en el crecimiento a largo plazo. Por ello, es necesario estimar los determinantes de la demanda de inversión.

La relación entre producto e inversión era ya evidente para los economistas ya en el siglo XIX. Considerar el producto como principal determinante de la demanda de inversión es el elemento clave de la teoría del acelerador. Este modelo aparece en el marco de la revolución keynesiana y se consolida hasta los años 60 como la explicación principal de la inversión en una economía. En conjunción con las teorías del multiplicador keynesiano, se aplica también al estudio de los ciclos económicos. Sin embargo, el modelo sufre de una serie de debilidades: no tiene en cuenta la oferta y realiza una serie de predicciones poco satisfactorias a nivel empírico en relación a los ciclos.

En los años 60 aparece el llamado modelo neoclásico de la demanda de inversión. Formulado por Jorgenson en 1963, el modelo plantea la decisión de inversión como fruto de la comparación entre el ingreso marginal derivado de una unidad adicional de capital, y el coste que representa utilizar una unidad adicional de capital. Para utilizar ese capital adicional, las empresas pueden alquilar ese capital, en cuyo caso igualarían producto marginal con coste del alquiler. Pero más habitualmente lo comprarán y lo mantendrán varios periodos. Así, el coste de uso del capital en un periodo determinado es igual al que sufrirían si compraran el capital, lo utilizasen y lo revendiesen inmediatamente. De esta manera, el coste de uso del capital se deriva de tres costes: el de oportunidad por no obtener rendimiento del precio de compra, el de depreciación porque el capital poseído reduce su cuantía con el uso, y un coste derivado de la pérdida de valor en el mercado del capital en caso de intentar revenderlo. Aunque las variables están indexadas a periodos temporales, las empresas alcanzan el óptimo maximizando periodo a periodo. De este modo, shocks en variables exógenas inducen variaciones discretos en variables endógenas como la demanda de capital. Esta predicción del modelo contrasta fuertemente con la realidad empírica, ya que ajustes discreto del stock de capital implican una tasa de variación infinita en el momento del shock y esto es difícilmente realizable. Por esta razón, es necesario introducir dos elementos en el modelo: costes de ajuste que impidan los ajustes discretos, y consideración del ingreso futuro para que el stock de capita siga una trayectoria óptima habida cuenta del efecto negativo de los costes de ajuste. Aparece así el modelo de la q, en el cual la variable que resume todo el efecto positivo y negativo del stock del capital es la llamada q, que no es sino equivalente a la variable de coestado del Hamiltoniano utilizado para caracterizar la función continua que describe la trayectoria óptima del stock de capital. Se trata de un modelo de horizonte infinito en tiempo continuo cuyas conclusiones pueden resumirse en dos ecuaciones que expresan su dinámica y que representadas en un diagrama de fase permiten hallar una solución gráfica. De esta manera extraer predicciones cualitativas respecto a cambios en el tipo de interés, el producto marginal o los costes de ajuste.

Por último, cabe plantearse el éxito de estos modelos realizando un breve recorrido por su contrastación empírica y sus resultados, así como notar el hecho de que los mercados financieros, como canalizadores del ahorro hacia la inversión, juega un papel importante en la demanda de inversión.

\seccion{Preguntas clave}
\begin{itemize}
    \item ¿Qué es la demanda de inversión?
    \item ¿Por qué se demanda inversión?
    \item ¿Qué factores afectan a la demanda de inversión?
    \item ¿Qué modelos existen para explicar la demanda?
\end{itemize}

\esquemacorto

\begin{esquema}[enumerate]
	\1[] \marcar{Introducción} 2'-2'
		\2 Contextualización
			\3 Datos sobre inversión
			\3 Impacto a corto plazo
			\3 Impacto a largo plazo
		\2 Objeto
			\3 Qué es la demanda de inversión
			\3 Por qué se demanda inversión
			\3 Qué factores afectan a la demanda
			\3 Qué modelos existen para modelizar la demanda
			\3 Qué valoración empírica de los modelos
		\2 Estructura
			\3 Modelo del acelerador
			\3 Modelo neoclásico de Jorgenson
			\3 Modelo de la q de Tobin
	\1 \marcar{Precursores}
		\2 Smith
			\3 Idea clave
			\3 Capital
			\3 Capital fijo
			\3 Capital circulante
			\3 Teoría normativa de la inversión
		\2 Teoría de la renta diferencial
			\3 Idea clave
			\3 Formulación
			\3 Implicaciones
			\3 Valoración
		\2 Marx
			\3 Idea clave
			\3 Formulación
			\3 Valoración
		\2 Wicksell
			\3 Idea clave
			\3 Formulación
			\3 Implicaciones
		\2 Modelo austríaco
			\3 Idea clave
			\3 Formulación
			\3 Valoración
		\2 Modelo neoclásico marshalliano
			\3 Idea clave
			\3 Formulación
			\3 Valoración
		\2 Teoría de la inversión de Fisher
			\3 Teoría de la inversión de Fisher
			\3 Teorema de la separación de Fisher
		\2 Keynes
			\3 Idea clave
			\3 Formulación
			\3 Valoración
	\1 \marcar{Modelo del acelerador} 5'-7'
		\2 Idea clave
			\3 Demanda derivada
			\3 Ratio capital producto
			\3 Interación con el multiplicador
			\3 Enfásis en demanda esperada
		\2 Formulación
			\3 Ratio capital-producto
			\3 Sin depreciación
			\3 Acelerador simple
			\3 Acelerador con expectativas
		\2 Implicaciones
			\3 Interacción con el multiplicador
		\2 Valoración
			\3 Influencia otros modelos
			\3 No tiene en cuenta oferta
			\3 Ciclos
	\1 \marcar{Modelo neoclásico } 5'-12'
		\2 Jorgenson
			\3 Idea clave
			\3 Formulación
			\3 Implicaciones
			\3 Valoración
		\2 RCK
			\3 Idea clave
			\3 Formulación
			\3 Implicaciones
			\3 Valoración
	\1 \marcar{Modelo de la q de Tobin} 12'-25'
		\2 Idea clave
			\3 Optimización función intertemporal de Bºs
			\3 Sin depreciación
			\3 Tiempo continuo
			\3 Modelización
		\2 Formulación
			\3 Función objetivo
			\3 Ley de transformación
			\3 Programa de maximización
			\3 Hamiltoniano
			\3 Dinámica
		\2 Implicaciones
			\3 q de Tobin
			\3 Variación productividad
			\3 Variación tipo de interés
			\3 Incertidumbre respecto al output
			\3 Diferentes costes de ajuste
		\2 Valoración
			\3 Influencia
			\3 Microfundamentado
			\3 Problemas de contrastación empírica
	\1 \marcar{Incertidumbre}
		\2 Idea clave
			\3 Contexto
			\3 Objetivos
			\3 Resultados
		\2 Bernanke (1983)
			\3 Idea clave
			\3 Formulación
			\3 Implicaciones
			\3 Valoración
		\2 Dixit (1992)
			\3 Idea clave
			\3 Formulación
			\3 Implicaciones
			\3 Valoración
		\2 Bloom (2009)
			\3 Idea clave
			\3 Formulación
			\3 Implicaciones
			\3 Valoración
		\2 Valoración
			\3 Canal de la incertidumbre de las políticas públicas
			\3 Volatilidad de precios y tipo de cambio
			\3 Aplicación a otras áreas
	\1 \marcar{Aspectos empíricos} 25'-28'
		\2 Idea clave
			\3 Problemas de estimación
		\2 Teoría del acelerador
			\3 Datos
			\3 Introducción de retardos
			\3 Relativo éxito empírico
		\2 Modelo neoclásico
			\3 Hall and Jorgenson (1967)
		\2 Modelo de la q
			\3 Dificil estimación de q marginal
			\3 Resultados
		\2 Impuestos: experimentos naturales
			\3 Idea clave
			\3 Resultados
		\2 Mercados imperfectos
			\3 Información asimétrica
			\3 Correlación empírica: inversión y cash-flow
			\3 Acelerador financiero
		\2 Flujos de caja empresariales
			\3 Idea clave
			\3 Correlación flujos de caja-inversión
	\1[] \marcar{Conclusión} 2'-30'
		\2 Recapitulación
			\3 Teorías de demanda de inversión
			\3 Aspectos empíricos de las diferentes teorías
		\2 Idea final
			\3 Crédito bancario e inversión
			\3 Crisis e inversión
			\3 Sector financiero
			\3 Decisiones de consumo
			\3 Visión de conjunto

\end{esquema}

\esquemalargo





















\begin{esquemal}
	\1[] \marcar{Introducción} 2'-2'
		\2 Contextualización
			\3 Datos sobre inversión
				\4 $\sim$ 20\% del PIB español en 2018
				\4[] 233.000 M de € en 2018
				\4[] $\to$ 250.000 M de € en 2018
				\4 Componente muy volátil
				\4 Crecimiento del consumo entre 2016 y 2018
				\4[] Entre $2,8\%$ y $2\%$
				\4 Crecimiento de la inversión entre 2016 y 2018
				\4[] Entre $2,9\%$ y $7,5\%$
				\4 Existencias
				\4[] Parte de la inversión en capital circulante
				\4[] Parte muy pequeña del PIB
				\4[] $\to$ Pero fluctuaciones enormes en $\%$\footnote{Sin embargo, si expresamos las variaciones de existencias en términos de contribución al crecimiento del PIB, las cantidades son muy pequeñas, porque apenas representan unas décimas del PIB anual.}
			\3 Impacto a corto plazo
				\4 Recesiones
				\4 Ciclos
			\3 Impacto a largo plazo
				\4 Factor de producción acumulable
				\4 Determinante del crecimiento de l/p
				\4[] Por diferentes vías y modelos
				\4[] $\to$ Spill-overs de conocimiento
				\4[] $\to$ Aprendizaje
				\4[] $\to$ Capital humano
				\4[] $\to$ Variedades
				\4[] $\to$ Mejoras de calidad
		\2 Objeto
			\3 Qué es la demanda de inversión
			\3 Por qué se demanda inversión
			\3 Qué factores afectan a la demanda
			\3 Qué modelos existen para modelizar la demanda
			\3 Qué valoración empírica de los modelos
		\2 Estructura
			\3 Modelo del acelerador
			\3 Modelo neoclásico de Jorgenson
			\3 Modelo de la q de Tobin
	\1 \marcar{Precursores}
		\2 Smith
			\3 Idea clave
				\4 Inversión sirve para aumentar output futuro
				\4 inversión en la medida en que sea útil
			\3 Capital
			\3 Capital fijo
				\4[] Activos físicos no consumidos en producción
			\3 Capital circulante
				\4[] Activos físicos consumidos en producción
			\3 Teoría normativa de la inversión
				\4 No todos los usos de la inversión son igual de buenos
				\4 Debe priorizarse la inversión
				\4[] 1. Inversión para aumentar producción
				\4[] 2. Inversión en bienes necesarios para subsistencia
				\4[] 3. Inversión para bienes de consumo duradero
				\4[] 4. Inversión en bienes no duraderos
				\4[] 5. Inversión para aumentar exportaciones
		\2 Teoría de la renta diferencial
			\3 Idea clave
				\4 Contexto
				\4[] Ricardo, Malthus
				\4[] Economía eminentemente agrícola y manufacturera
				\4[] Debate sobre función social de la renta
				\4[] Inversión es sobre todo fondo de salarios
				\4[] $\to$ Pagar trabajo antes de vender producto
				\4 Objetivo
				\4[] Caracterizar máxima inversión rentable
				\4[] Papel de la renta en proceso
				\4 Resultados
				\4[] Inversión hasta eliminación de beneficios
				\4[] Rentas capturan todo el beneficio
				\4[] $\to$ Renta culpable de fin de inversión
			\3 Formulación
				\4 Inversión en capital en proporción a trabajo
				\4[] $\to$ Unidades de ``trabajo--capital'' $(K-L)$
				\4 $\text{Renta} = \text{PMe} \cdot (K-L) - \text{PMg} \cdot (K-L)$
				\4 Rendimientos marginales decrecientes
				\4[] $\to$ $\text{PMg}$ decreciente
				\4[] $\then$ $\text{PMe} > \text{PMg}$ decreciente
				\4 Inversión aumenta hasta agotar beneficios
				\4[] $\to$ Renta captura todo el beneficio
				\4 Representación gráfica
				\4[] \grafica{modelodericardo}
			\3 Implicaciones
				\4 Inversión depende de rentabilidad
				\4[] Beneficio obtenido por unidad invertida
				\4 Inversión acaba frenándose
				\4[] Cuando no es posible extraer beneficio
				\4[] $\to$ Estado estacionario
				\4 Comercio exterior y progreso técnico
				\4[] Vías para seguir aumentando inversión
			\3 Valoración
				\4 Precede análisis marginal
				\4 Enmarca decisión de inversión
				\4[] Como resultado de beneficio y coste
				\4[] Proceso dinámico hasta agotar producto
		\2 Marx\footnote{Ver \href{https://www.socialist.net/marx-s-capital-chapters-23-25-accumulation.htm}{Socialist.net sobre acumulación de capital en Marx}.}
			\3 Idea clave
				\4 Contexto
				\4[] Economías en transformación industrial
				\4[] Tensión trabajo-capital
				\4[] Exceso de trabajo en algunos mercados
				\4[] Acumulación creciente del capital
				\4[] Teoría ricardiana del valor-trabajo
				\4[] $\to$ Trabajo es fuente de valor
				\4[] $\to$ Precio debe aproximarse a valor relativo
				\4[] Teoría de la explotación marxista
				\4[] $\to$ Salarios son trabajo necesario para reproducir trabajo
				\4[] $\to$ Trabajadores alienados de mercancía que se produce con su L
				\4[] $\to$ Empresarios extraen diferencia entre salario y precio venta
				\4[] $\then$ Explotación del trabajador
				\4[] $\then$ Extracción de la plusvalía
				\4[]
				\4 Objetivo
				\4[] Aplicar análisis de condiciones materiales
				\4[] $\to$ A evolución histórica
				\4[] $\then$ Acumulación de K/inversión elemento clave
				\4[] Explicar rol de la acumulación del capital
				\4[] $\to$ En teoría de la historia
				\4[] $\to$ En teoría de la explotación
				\4 Resultado
				\4[] Acumulación de capital/inversión
				\4[] $\to$ Resultado de explotación del trabajo
				\4[] Inversión hace posible explotación
				\4[] Remuneración de la inversión cae
				\4[] $\to$ Competencia cada vez mayor
				\4[] $\to$ Avances tecnológicos reducen riesgo
				\4[] Inversión como motor de crecimiento
			\3 Formulación
				\4 TVTrabajo
				\4[] Valor de uso de un bien
				\4[] $\then$ Capacidad para satisfacer necesidades humanas
				\4[] Valor de intercambio
				\4[] $\to$ Depende de cantidad relativa de trabajo
				\4 Teoría de la explotación
				\4[] Valor de uso del trabajo
				\4[] $\to$ Capacidad para fabricar otras mercancías
				\4[] Valor de intercambio del trabajo en sí mismo
				\4[] $\to$ Valor de medios de subsistencia para reproducir
				\4[] Precio de los bienes
				\4[] $\to$ Relación de intercambio de un bien por otro
				\4 Teoría de la explotación
				\4[] Precio del trabajo (salario)
				\4[] $\to$ Es menor que valor de uso
				\4[] $\then$ Menor que valor de bienes producidos con trabajo
				\4 Capital
				\4[] Dinero utilizado para comprar algo
				\4[] $\to$ Y revenderlo después para obtener beneficio
				\4[] Permite:
				\4[] $\to$ Pagar salarios
				\4[] $\then$ Capital variable
				\4[] $\to$ Adquirir inputs necesarios
				\4[] $\then$ Capital constante
				\4 Inversión es acumulación del capital
				\4[] A partir de beneficios acumulados
				\4[] Se invierte más capital variable
				\4[] $\to$ Para aumentar trabajo que explotar
				\4[] Se invierte más capital fijo
				\4[] $\to$ Para ser + competitivo frente a innovaciones
				\4 Acumulación de capital como motor de crecimiento
				\4[] Aumenta trabajo puesto en funcionamiento
				\4[] Aumenta productividad del trabajo
				\4 Exceso de inversión presiona salarios
				\4[] Demanda agregada supera a oferta
				\4 Crisis del capitalismo
				\4[] Reducen inversión cuando aumentan salarios
				\4[] $\to$ Cae beneficio
				\4[] $\then$ Empresas quiebran
				\4[] Salarios vuelven a caer
				\4[] $\to$ Inversión vuelve a aumentar
				\4[] $\then$ Inversión como mecanismo clave de ciclo
			\3 Valoración
				\4 Gran impacto teórico e histórico
				\4 Erróneo en muchos aspectos
				\4[] Problema de la transformación
				\4[] $\to$ Precios relativos depende de trabajo relativo
				\4[] $\to$ Capitalistas obtienen plusvalía de capital variable
				\4[] $\then$ Beneficio debe ser mayor cuanto menos capital fijo
				\4[] $\then$ Pero empíricamente, beneficio tiende a igualarse
				\4[] $\then$ ¿Por qué valor no se transforma en precios?
				\4[] Ignora el papel de los empresarios
				\4[] $\to$ Incorporan un input de producción propio
				\4[] $\then$ Asunción de riesgos
				\4[] $\then$ Innovación
				\4[] $\then$ Organización de factores
				\4[] $\then$ Plusvalía es remuneración
				\4[] Progreso tecnológico
				\4[] $\to$ Estimula beneficio
				\4[] $\to$ Evita colapso de tasa de beneficio
				\4[] $\to$ No se cumple final de estímulo a inversión
				\4 Problemas de falsabilidad
				\4[] No niega estados de la naturaleza futuros
				\4[] Sólo plantea que algunos se producirán
				\4[] $\to$ Si no se han producido aún
				\4[] $\then$ Ya se producirán
				\4[] $\then$ Si beneficio no se ha anulado, ya pasará
				\4 Énfasis excesivo en capital físico
				\4[] Capital humano
				\4[] capital social
				\4[] Capital ecológico
				\4[] Capital financiero
				\4[] Capital organizativo
				\4[] ...
		\2 Wicksell
			\3 Idea clave
			\3 Formulación
				\4 Interés natural
				\4[] Productividad marginal del capital
				\4 Interés monetario
				\4[] Determinado en mercados financieros
			\3 Implicaciones
				\4 Inversión cuando i nominal menos que natural
				\4 Inversión como origen de inflación
				\4[] Inversión y ahorro tienden a igualarse
				\4[] Expansión monetaria provoca $i < r$
				\4[] $\to$ Presión sobre bienes de capital dado ahorro
				\4[] $\then$ Aumento de precios de bienes de capital
				\4[] $\then$ Repercusión a bienes de consumo
		\2 Modelo austríaco
			\3 Idea clave
				\4 Contexto
				\4[] Marxismo
				\4[] $\to$ Capital como instrumento de explotación
				\4[] $\to$ Énfasis en relación capital y trabajo
				\4[] Emprendedor/empresario como factor de producción
				\4[] $\to$ Aporta
			\3 Formulación
				\4 Inversión depende de interés
				\4 Interés existe porque:
				\4[] i. Utilidad marginal de renta es decreciente
				\4[] $\to$ Y agentes esperan mayor renta mañana
				\4[] ii. Agentes son impacientes y prefieren consumo presente
				\4[] iii. Abstinencia es factor de producción
				\4 Interés es precio de equilibrio
				\4[] $\to$ En mercado de fondos prestables
				\4 Política monetaria acomodaticia
				\4[] Reduce artificialmente el tipo de interés
				\4[] $\to$ Exceso de inversión
				\4[] $\to$ Inversiones improductivas
				\4[] $\to$ Aumento de desequilibrios
				\4 Caída de precio de activos
				\4[] Liquidación de activos
				\4[] Caída de demanda de inversión
			\3 Valoración
				\4 Racionaliza ex-post muchos fenómenos de inversión
				\4 Débil capacidad predictiva ex-ante
				\4 Predicciones no falsables
		\2 Modelo neoclásico marshalliano
			\3 Idea clave
				\4 Contexto
				\4[] Formulación matemática
				\4[] Agentes optimizadores
				\4[] $\to$ Especialmente empresas
				\4[] $\then$ Búsqueda de máximo beneficio
				\4 Objetivo
				\4[] Definir inversión que maximiza beneficios
				\4 Resultado
				\4[] Modelo básico de la decisión empresarial
				\4[] Empresa como caja negra
				\4[] Inversión sirve sólo para generar beneficios
				\4[] $\to$ Implicando también costes
			\3 Formulación
				\4 Costes en forma de U
				\4 Oferta en brazo creciente de curva CMe
				\4[] $\to$ Costes marginales crecientes
				\4 Precio iguala coste marginal
				\4[] $\to$ Precio mayor que coste medio
				\4[] $\then$ Posible extraer beneficios positivos
				\4 Entrada de nuevas empresas
				\4[] Cada empresa demanda inversión óptima
				\4[] $\to$ Consistente con precio y función de producción
				\4[] Entrada mientras que haya beneficios positivos
				\4[] $\then$ Inversión crece hasta beneficios se anulan
			\3 Valoración
				\4 Modelo muy simple y tratable de inversión
				\4 Empíricamente pobre
		\2 Teoría de la inversión de Fisher
			\3 Teoría de la inversión de Fisher
				\4 La Teoría del Interés (1930)
				\4 ¿Cómo distribuir intertemporalmente dotación?
				\4[] $\to$ Para maximizar utilidad
				\4 Modelo dinámico formal
				\4[] Dos periodos
				\4[] Términos matemáticos
				\4 Precursor de modelos dinámicos formales
				\4 Dotación exógena
				\4[] Cantidad $w_1$ en el periodo 1
				\4[] Cantidad 0 en e periodo 2
				\4 Inversión
				\4[] Cantidad de $w_1$ no consumida en 1
				\4[] $\to$ Destinada a capital
				\4 Rentas en periodo 2
				\4[] Inversión aplicado a f. de prod
				\4[] Ahorro a tipo de interés $r$
				\4 Problema de optimización
				\4[] Maximizar utilidad
				\4[] $\to$ Eligiendo consumo en periodo 1 y 2
				\4 Realmente, dos problemas separados
				\4[I] Problema de optimización de inversión
				\4[II] Problema de optimización de consumo
				\4 Problema de optimización de inversión
				\4[] Maximizar valor presente de flujos netos
				\4[] $\underset{\max}{k} \quad w-k + \frac{f(k)}{1+r} \quad \text{s.a:} k \leq w$
				\4[] $\text{CPO:} \quad f'(k) = 1+r$
				\4[] $\then$ Invertir dotación hasta que $f'(k) = 1+r$
				\4[] $\then$ Invertir hasta eficiencia marginal de K sea interés
				\4[] Representación gráfica
				\4[] \grafica{fisherproblemainversion}
				\4 Problema de optimización del consumo
				\4[] Maximizar utilidad dada restricción intertemporal
				\4[] $\underset{\max}{c_1, c_2} \quad u(c_1) + v(c_2)$
				\4[] $\text{s.a:} \quad c_1 + \frac{c_2}{1+r} = w -k^* + \frac{f(k^*)}{1+r}$
				\4[] Representación gráfica
				\4[] \grafica{fisherproblemaconsumidor}
				\4 Optimización intertemporal de empresa y consumidor
				\4[$\then$] Teorema de la separación de Fisher
			\3 Teorema de la separación de Fisher
				\4 Objetivo de la empresa es maximizar valor presente
				\4[] $\to$ Independientemente de preferencias de accionistas
				\4 Si mercados de capital perfectos:
				\4[] $\to$ Financiación independiente de inversión
				\4 Aplicación microeconómica
				\4[] $\to$ Punto en FPP independiente de dda. de consumo óptimo
				\4 Precursor de Modigliani-Miller
				\4 Dos agentes
				\4[] Disponen de las mismas funciones de producción
				\4[] $\to$ Rendimiento de la inversión equivalente
				\4[] Tienen igual acceso al mercado financiero
				\4[] $\to$ Pueden prestar y tomar prestado al mismo tipo
				\4[] Preferencias no tienen por qué ser iguales
				\4 Decisiones de inversión y consumo óptimas
				\4[] Ambos tomarán misma decisión de inversión
				\4[] Decisiones de consumo dependen de prefs. respectivas
		\2 Keynes
			\3 Idea clave
				\4 Contexto
				\4[] Modelo keynesiano
				\4[] Énfasis en lado de la demanda
				\4[] Economía no es necesariamente estable
				\4[] $\to$ Posible exceso de capacidad persistente
				\4[] Expectativas de los agentes son inestables
				\4[] $\to$ Animal spirits
				\4 Objetivo
				\4[] Dda. de inversión sin restricciones de oferta
				\4[] Representar efecto de animal spirits sobre inversión
				\4[] Integrar demanda de inversión en modelo macroconómico
				\4 Resultado
				\4[] Dda. de inversión sujeta a irracionalidad
				\4[] Inversión es motor de ciclo económico
			\3 Formulación
				\4 Empresas ordenan proyectos según:
				\4[] Eficiencia marginal del capital (EMK)
				\4[] EMK = interés que iguala:
				\4[] -- precio\footnote{Según Keynes, el precio no es el precio de mercado sino el ``supply price'', que corresponde al precio que induce a productor a producir una nueva unidad. Sólo en un contexto de competencia se igualan supply price y precio de mercado.}
				\4[] -- flujos de caja descontados
				\4[] $\to$ Equivalente a TIR
				\4[] Si EMK > tipo de interés
				\4[] $\to$ Proyecto se lleva a cabo
				\4 Formulación
				\4[] Generalmente, caracterizada como $I = I_0 + I(r)$
				\4[] $\to$ $\frac{d \, I(r)}{d \, r} < 0$
				\4[] Pero $I(r)$ es inestable
				\4 ¿Por qué inestable?
				\4[] ¿Cómo estiman inversores los flujos esperados?
				\4[] Estimaciones dependen de ``animal spirits''
				\4[] Animal spirits: volatilidad de flujos esperados
				\4[] $\to$ Demanda de inversión es inestable
				\4[] $\to$ Efecto limitado de $\Delta r$ sobre I
				\4[$\then$] Sector público puede estabilizar dda. de inversión
				\4[$\then$] Estímulo público estimula también expectativas
				\4[$\then$] Feedback positivo estímulo-inversión-renta
			\3 Valoración
				\4 Introduce marco conceptual para acelerador
				\4 Énfasis en lado de la demanda
				\4 Economía no tiene porqué ser estable
	\1 \marcar{Modelo del acelerador} 5'-7'
		\2 Idea clave
			\3 Demanda derivada
				\4 Demanda de K depende de demanda de Y
			\3 Ratio capital producto
				\4 Aproximadamente constante
			\3 Interación con el multiplicador
				\4 Más renta $\rightarrow$ más demanda de consumo
				\4 Más demanda de consumo $\rightarrow$ más demanda de inversión
			\3 Enfásis en demanda esperada
				\4 Precios de inputs y tipo de interés no relevantes
		\2 Formulación
			\3 Ratio capital-producto
				\4 Ratio deseado: $K_t^* = v \cdot Y_t$
			\3 Sin depreciación
				\4 Simplificar s.p.g.
			\3 Acelerador simple
				\4 Inversión en t para alcanzar $K_{t+1}^*$ en $t+1$
				\4 $I_t = K_{t+1}^* - K_t* = v \cdot (Y_{t+1} - Y_t)$
				\4 Inversión proporcional a incremento de producción
				\4[] Mantener constante $v$
			\3 Acelerador con expectativas
				\4 $I_t = v \cdot E\left( Y_{t+1} - Y_t \right)$
				\4 Exp. miopes: $I_t = v \cdot (Y_t - Y_{t-1})$
				\4[] Expectativa de crecimiento del output
				\4[] $\to$ Es igual a crecimiento pasado del output
		\2 Implicaciones
			\3 Interacción con el multiplicador
				\4 $\uparrow I \rt \uparrow Y \rt \uparrow C = cY \rt \uparrow Y \rt \varDelta Y = \frac{1}{1-c}$
				\4 $\uparrow Y \rt \uparrow I$
				\4 Variables son desviación de las medias
				\4 Inversión: $I_t = v (Y_{t-1} - Y_{t-2})$
				\4 Consumo: $C_t = (1-s)Y_{t-1}$
				\4 $Y_t = C_t + I_t$
				\4 $Y_t = (1-s + v )Y_{t-1} - v \cdot Y_{t-2}$
				\4 Posibles resultados en función de $v$ y $s$:
				\4[I] Desv. cada vez más positiva de ratio $\frac{K}{Y}$ deseado
				\4[II] Desv. cada vez más negativa de ratio $\frac{K}{Y}$ deseado
				\4[III] Ciclo cada vez menos amplio
				\4[IV] Ciclo cada vez más amplio
		\2 Valoración
			\3 Influencia otros modelos
				\4 Inversión como demanda derivada
				\4 Harrod-Domar
				\4 Modelos neokeynesianos
			\3 No tiene en cuenta oferta
				\4 Sólo demanda como determinante
			\3 Ciclos
				\4 Mala predicción de ciclos\footnote{Parker (pág. 12) \comillas{Moreover, the great variation in lengths and severity of business cycles over time and across countries argues against an <<endogenous>> explanation of the cycle such as that provided by the multiplier-accelerator model. In such a model, each business cycle should be the same length and, depending on the formulation, perhaps of the same magnitude as well.}}
				\4 Ciclos deberían tener misma duración e incluso magnitud
				\4 Teorías modernas: ciclos origen en perturbaciones aleatorias
	\1 \marcar{Modelo neoclásico } 5'-12'
		\2 Jorgenson
			\3 Idea clave
				\4 Optimización periodo a periodo
				\4[] Empresas maximizan beneficio en relación a K
				\4[] Inversión hasta que PMgK = coste de uso
				\4[] Equivalente a maximizar el flujo total
			\3 Formulación
				\4 Función objetivo
				\4[] 		 $\Pi = \pi(K, \vec{x}) - r_K \cdot K$
				\4[]  $\pi_K > 0$, $\pi_{KK}<0$
				\4 Problema de maximización
				\4[] $\underset{K}{\max} \quad \Pi = \pi(K, \vec{x}) - r_K \cdot K$
				\4[] $\Pi_K = 0 \then \pi_K = r_k$
				\4 $\frac{d \pi_K (r_K)}{d r_K} = \frac{d r_K}{d r_K} \rightarrow \pi_{KK} \pdv{K(r_K)}{r_K} = 1 \rightarrow$
				\4[] $\pdv{K}{r_K} = \frac{1}{\pi_{KK}} < 0$
				\4[] $\then$ Porque hemos asumido que $\pi_{KK}$ es negativo.
				\4[] $\uparrow r \rt \downarrow K$
				\4[] $\to$ Menos inversión cuanto mayor coste de uso de capital
				\4[] $\then$ Resultado compatible con teoría keynesiana de EMK
				\4 Coste de uso del capital
				\4[] $r_K$ no es coste de compra, sino de uso
				\4[]$r_K = $ suma de\footnote{Se pueden introducir impuestos en este análisis. Por ejemplo, una deducción por inversión de $f$ por cada unidad de inversión podría introducirse definiendo $r_K = \left[ r(t) + \delta - \frac{\dot{p_K}(t)}{p_K(t)} \right] ( 1- f \tau) p_K(t)$, donde $\tau$ es el tipo del impuesto de sociedades.}:
				\4[] Coste de oportunidad: $r(t) p_K(t)$
				\4[] Depreciación: $\delta p_K$
				\4[] Ganancia de capital: $-\dot{p_K}$
				\4 $r_K = r_t p_K + \delta p_K - \dot{p_K} = \left[ r_t + \delta - \frac{\dot{p_K}}{p_K} \right] p_K$
			\3 Implicaciones
				\4 Interés reduce inversión
				\4 Inversión para ajustar instantáneamente el capital
			\3 Valoración
				\4 Variaciones discretas del capital
				\4[] Ejemplo: variación tipo de interés $r_t$
				\4[] Ajuste instantáneo de k en t
				\4[] En t, demanda es $\infty$
				\4[] Inversión agregada no puede ser infinita\footnote{Podría considerarse que existe capital ocioso y que en un momento determinado pueda satisfacerse un aumento discreto de $k$. Pero esto no resulta factible en general, ni a largo plazo. Aumentos discretos de la inversión provocarían más bien tensiones inflacionarias.}
				\4[] $\then$ $\dot{k} = \infty$
				\4 Expectativas no consideradas
				\4[] Previsión de ingreso marginal futuro
				\4[] Inexistente
				\4[] En la práctica, clave
				\4 Necesarios costes de ajuste
				\4[] Eliminación cambios discretos
				\4[] Necesarias expectativas para no aumentar costes
		\2 RCK
			\3 Idea clave
				\4 Contexto
				\4[] Solow-Swan
				\4[] $\to$ Crecimiento económico
				\4[] $\to$ Acumulación de factores y progreso técnico
				\4[] $\then$ Límite a inversión como motor de crecimiento
				\4[] $\then$ Sin fundamentar inversión
				\4[] Jorgenson
				\4[] $\to$ Demanda de inversión para maximizar beneficios
				\4[] Integrar demanda de inversión en modelo general
				\4[] $\to$ Consumidor que optimiza beneficios
			\3 Formulación
				\4 Inversión es igual ahorro
				\4 Ahorro depende de optimización temporal de consumo
				\4[] Elasticidad de sustitución intertemporal
				\4[] Productividad del trabajo y del capital
				\4 Ecuaciones de dinámica
				\4[] Consumo
				\4[] \fbox{$\frac{\dot{c}_t}{c_t} = \frac{i_t - \rho - \theta g}{\theta}$}
				\4[] \grafica{consumorck}
				\4[] Capital
				\4[] \fbox{$\dot{k}_t = f(k_t) - c_t - (n+g) k_t$}
				\4[] \grafica{capitalrck}
				\4 Interés es PMg de K
				\4[] $i(t) = f'(k_t)$
			\3 Implicaciones
				\4 Shock de productividad aumenta interés
				\4[] Aumento de productividad marginal de K
				\4 Aumento de interés
				\4[] Desplaza a la izquierda $\dot{c}=0$
				\4[] Desplaza hacia fuera $\dot{k}=0$
				\4[] $\to$ Más
				\4 Aumento de preferencia por presente
				\4[] Desplaza $\dot{c}=0$ a la izquierda
				\4[] $\to$ Aumento de consumo instantáneo
				\4[] $\then$ Caída de la inversión
				\4[] $\then$ Caída del capital
				\4[] $\then$ Nuevo EEstacionario con menor c y k
			\3 Valoración
				\4 Fundamento de modelos RBC
				\4 Base de módulos de demanda de inversión en modelos DSGE
	\1 \marcar{Modelo de la q de Tobin} 12'-25'
		\2 Idea clave
			\3 Optimización función intertemporal de Bºs
				\4 Costes de ajuste impiden optimización periodo a periodo
			\3 Sin depreciación
				\4 Simplicidad
			\3 Tiempo continuo
				\4 Adaptable a tiempo discreto
			\3 Modelización
				\4 Variación tipo de interés
				\4 Variaciones de productividad
				\4 Impuestos sobre inversión
				\4 Incertidumbre
				\4 Impacto de diferentes costes de ajuste
				\4 Imperfecciones en mercados financieros
		\2 Formulación
			\3 Función objetivo
				\4 $\Pi = \int_{t=0}^\infty e^{-rt} \left[ \pi \left(  K_t \right) k_t - I_t- C \left( I_t \right) \right] \, dt$
				\4 Costes de ajuste\footnote{Los costes de ajuste pueden ser externos si aparecen como resultado del aumento del precio relativo de los bienes de capital cuando aumenta su demanda, o internos cuando surgen como resultado de procesos necesarios a nivel interno cuando se invierte en nuevo capital tales como la formación de nuevos empleados, acomodamiento de maquinaria, etc... En los modelos presentados la distinción no es relevante. Además, es posible considerar que el coste no aparece cuando varía el stock de capital, sino la inversión. En estos modelos el coste se supone fruto de la variación del stock de capital, pero la distinción no es demasiado relevante y permite una modelización más sencilla.}
				\4[] $C(0)=0$, $C'(0)=0$, $C''(\cdot) > 0$
			\3 Ley de transformación
				\4 $\dot{k}_t = I_t$\footnote{Asumimos por simplicidad que no hay depreciación.}
			\3 Programa de maximización
				\4 $\underset{I_t}{\max} \quad \Pi = \int_{t=0}^\infty e^{-rt} \left[ \pi \left(  K_t \right) k_t - I_t - C \left( I_t \right) \right] \, dt$
				\4[] $\text{s.a:} \quad \dot{k}_t = I_t$
			\3 Hamiltoniano
				\4 $H\left( k_t, I_t \right) = \pi \left( K_t \right) k_t - I_t - C\left( I_t \right) + q_t I_t$
				\4 CPOs:
				\4 $\pdv{H}{I_t} = 0 \rightarrow 1 + C'\left( I_t \right) = q_t$
				\4 $\pdv{H}{k_t} = \pi \left( K_t \right) = rq_t - \dot{q}_t$
				\4 $\lim_{t \to \infty} \quad q_t k_t e^{-rt} = 0$
			\3 Dinámica
				\4 Producción agregada: espacio $q$ -- $K$
				\4 Stock de capital\footnote{La primera CPO establece que: $1 + C'\left( I_t \right) = q_t$. Reordenando tenemos que $C'\left( I_t \right) = q_t - 1$. $I_t = \dot{k}_t$, luego $\dot{K}_t = N \cdot C'^{-1}\left(q_t - 1 \right)$. Dado que todas las empresas son iguales, multiplicamos por $N=\text{Número de empresas}$.}
				\4[] \fbox{$\dot{K}_t = f\left( q_t \right) = NC'^{-1}\left( q_t - 1 \right)$}
				\4 $q_t$ de Tobin
				\4[] \fbox{$\dot{q}_t = rq_t - \pi\left( K_t \right)$}
				\4 \grafica{fasebasico}
		\2 Implicaciones
			\3 q de Tobin
				\4 q es variable de coestado
				\4 Valor presente suma de flujos generados
				\4[] Por unidad adicional de k
				\4[] $\to$ Incorporada en el periodo $t$
				\4 $q(t) = \int_{\tau=t}^{\infty} \pi(K(\tau)) e^{-r(\tau -t)} d \tau$
				\4 q media: $\frac{\text{Valor de mercado}}{\text{Valor de remplazo}}$
				\4 q marginal = q media si costes de ajuste lineales.
			\3 Variación productividad
				\4 \underline{Incremento permanente}
				\4 $\dot{q} = 0$ desplaza hacia arriba, derecha
				\4[] Salto a nueva senda de equilibrio
				\4[] $\dot{K} >0$ hasta equilibrio
				\4[] Nuevo equilibrio con más capital
				\4[] \grafica{permanenteproducto}
				\4 \underline{Incremento temporal}
				\4 Nueva senda de equilibrio arriba a la derecha
				\4 Debe volver a senda inicial
				\4[] Desplazamiento hacia arriba de q
				\4[] Vuelta a senda inicial siguiendo nueva dinámica
				\4[] \grafica{temporalproducto}
			\3 Variación tipo de interés
				\4 \underline{Bajada permanente}
				\4[] $\dot{q} = 0$ se desplaza hacia arriba
				\4[] $\dot{q} = 0$ mayor pendiente \footnote{Porque $q(t) = \frac{\pi \left( K(t) \right)}{r}$ y una bajada de r aumenta $q(t)$ para todo $\pi \left( K(t) \right)$.}
				\4[] $q(t)$ se sitúa en nueva senda de equilibrio
				\4[] $K$ aumenta hasta nuevo equilibrio $\rt$ Inversión positiva
				\4[] Equilibrio con más capital
				\4 Tipos de interés a largo plazo son importantes
				\4[] Vía tipos a corto\footnote{La tasa $r$ introducida en todas las ecuaciones corresponde a la tasa de retorno instantáneo, es decir a corto plazo. Si se cumple la teoría de las expectativas puras de la teoría de la curva de tipos de interés, tenemos que un aumento permanente de r induce un aumento de los tipos a largo plazo. Por ello, los tipos a largo plazo y la inversión están ligadas en este contexto si se cumple la teoría de las expectativas puras.}
			\3 Incertidumbre respecto al output
				\4 Dos sendas posibles
				\4 q inicial tras conocimento de la incertidumbre:\footnote{Es decir, una vez el agente toma conocimiento de que el valor del output futuro es incierto.}
				\4[] A instancia intermedia, ponderada por probabilidad
				\4 Si costes de ajuste asimétricos
				\4[] Situado más cerca del coste menor
				\4[] Generalmente, desinversión más costosa
				\4[] $\to$ más cerca de equilibrio inicial que de nueva senda
			\3 Diferentes costes de ajuste
				\4 \underline{Kinked}
				\4 Si $C'\left( 0 \right) \neq  0$
				\4 $c^+$ para inversión:
				\4 $c^-$ para desinversión
				\4 Región de equilibrios: $1-c^- \leq q(t) \leq 1+c^+$
				\4 Gráfica IV
				\4 \underline{Discontinuo/fijo}
				\4[] A nivel agregado, impacto similar a kinked.
		\2 Valoración
			\3 Influencia
				\4 Nueva Macroeconomía Clásica
				\4 Nueva Economía Keynesiana
			\3 Microfundamentado
				\4 Teoría micro justifica
			\3 Problemas de contrastación empírica
				\4 q marginal difícilmente estimable
	\1 \marcar{Incertidumbre}
		\2 Idea clave
			\3 Contexto
				\4 Inversión implica costes hundidos
				\4 Resultados de inversión son inciertos
				\4 Incertidumbre depende de información
				\4 Información se adquiere con el tiempo
			\3 Objetivos
				\4 Efecto de incertidumbre sobre inversión
				\4 Adquisición de información sobre inversión
				\4 Opciones reales e inversión
				\4 Efectos agregados de incertidumbre
			\3 Resultados
				\4 Incertidumbre afecta dda. de inversión
				\4 Efectos de estímulos afectan inversión
		\2 Bernanke (1983)
			\3 Idea clave
				\4 Contexto
				\4[] Inversión irreversible
				\4[] $\to$ Físicamente difícil de revertir
				\4[] $\to$ Aunque puede transferirse
				\4[] $\then$ Irreversible a nivel agregado
				\4[] Incertidumbre sobre beneficios
				\4[] Inversiones pueden posponerse
				\4[] $\to$ Oportunidades tardan en desaparecer
				\4[] $\to$ Posponer aumenta información disponible
				\4 Objetivo
				\4[] Efecto de incertidumbre sobre:
				\4[] $\to$ Qué proyectos llevar a cabo
				\4[] $\to$ Cuándo llevarlos a cabo
				\4[] $\then$ Cuánto invertir
				\4 Resultados
				\4[] Momento de inversión implica trade-off
				\4[] $\to$ Ahora: posibles más beneficios
				\4[] $\to$ Después: más información sobre beneficios
				\4[] Bienes duraderos más sensibles a incertidumbre
				\4[] $\to$ Porque inversión irreversible en términos físicos
				\4[] $\then$ Mayor volatilidad de dda. de inversión
			\3 Formulación
				\4 Valor de opción real
				\4[] Similar a opción financiera call
				\4[] $\to$ Posibilidad de gastar sólo si hay beneficio
				\4[] $\then$ Valor de no perder dinero
				\4[] Posibilidad de no comprometerse
				\4[] $\to$ Tiene valor económico
				\4 Comparación para decidir inversión
				\4[] Valor de opción real
				\4[] Valor esperado de proyecto de inversión
				\4 Shocks macro afectan decisiones micro
				\4[] Vía incertidumbre sobre beneficios
			\3 Implicaciones
				\4 Variaciones de incertidumbre inducen ciclos
				\4[] $\uparrow$ de incertidumbre $\then$ $\uparrow$ valor de opciones reales
				\4[] $\to$ Empresas demandan menos inversión
				\4[] $\then$ Demanda agregada cae
				\4[] $\then$ Proceso se retroalimenta
				\4[] Economía toca fondo
				\4[] $\to$ Incertidumbre se reduce
				\4[] $\then$ Aumenta inversión
				\4 Recesiones afectan especialmente a duraderos
			\3 Valoración
				\4 Modelo general de inversión irreversible
				\4 Explica mayor volatilidad
				\4 Inicia programa de investigación
				\4[] $\to$ Incertidumbre e inversión
		\2 Dixit (1992)
			\3 Idea clave
				\4 Contexto
				\4[] Histéresis de inversión
				\4[] Concepto de histéresis
				\4[] $\to$ Efectos que permanecen
				\4[] $\then$ Aunque las causas hayan desaparecido
				\4[] Años 80
				\4[] Apreciación del dólar
				\4[] $\to$ Poco aumento de las importaciones
				\4[] $\then$ Finalmente aumentan
				\4[] Depreciación del dólar
				\4[] $\to$ Déficit por cuenta corriente se mantiene
				\4 Objetivo
				\4[] Caracterizar papel de incertidumbre
				\4[] $\to$ En fenómenos observados de histéresis de inversión
				\4 Resultado
				\4[] Incertidumbre puede explicar histéresis de inversión
				\4[] Empresas posponen decisión de inversión
				\4[] $\to$ Para mantener opciones abiertas
				\4[] Empresas realizan inversiones irreversibles
				\4[] $\to$ Y las mantienen aunque causas ya no existan
				\4[] Ejemplo:
				\4[] 1. Mantener mina abierta aunque pérdidas operativas
				\4[] $\to$ Cerrarla implica coste hundido
				\4[] $\to$ Posible aumento de precio de materia prima
				\4[] 2. Retrasar compra de coches eléctricos
				\4[] $\to$ Compra implica costes hundidos
				\4[] $\to$ Aún hay pocas inversiones en eléctricos
				\4[] $\to$ Posible cambio en tecnología futura
				\4[] 3. Mantener producción agrícola aun con exceso de oferta
				\4[] $\to$ Precios altos iniciales indujeron inversión
				\4[] $\to$ Abandonar producción puede tener coste irreversible
				\4[] $\to$ Mantener producción por si precios vuelven a subir
				\4[] $\then$ Causas iniciales de inversión ya no existen
				\4[] $\then$ Inversión se mantiene y sigue generando efectos
			\3 Formulación
			\3 Implicaciones
			\3 Valoración
		\2 Bloom (2009)
			\3 Idea clave
			\3 Formulación
				\4 Modelo estructural
				\4 Parámetros calibrados para ec. americana
				\4 Segundo momento de output sufre shocks
				\4 Costes de ajuste de capital y trabajo
				\4 Incertidumbre aumenta valor de opciones reales
				\4[] Evitar costes hundidos no rentables
			\3 Implicaciones
				\4 Incertidumbre modula shocks de demanda
				\4[] Más incertidumbre reduce efectos estímulo a dda.
				\4 Overshooting de la incertidumbre
				\4[] Output cae más que $\Delta$ de incertidumbre
				\4 Incertidumbre reduce sensibilidad
				\4[] Inversión responde menos a otros shocks
				\4[] Empresas se vuelven cautas
				\4 Incertidumbre produce shocks de prod. negativos
				\4[] Difíciles de fundamentar
				\4[] Generalmente, asumidos a partir de series
				\4[] $\to$ Sin explicar por qué se producen
				\4[] Incertidumbre puede explicar
			\3 Valoración
				\4 Marco de modelización estable
				\4 Combina modelos previos
		\2 Valoración
			\3 Canal de la incertidumbre de las políticas públicas
				\4 Incertidumbre política y legislativa
				\4[] Impacto sobre inversión
				\4[] Importante explicar cuánto y cómo
			\3 Volatilidad de precios y tipo de cambio
				\4 Afectan a inversión
				\4 Canal añadido a nivel
				\4[] P.ej.: TCN apreciado
				\4[] $\to$ Volatilidad de nivel también afecta
			\3 Aplicación a otras áreas
				\4 Mercado de trabajo
				\4[] Costes hundidos de contratar
				\4[] Incertidumbre sobre:
				\4[] $\to$ Productividad
				\4[] $\to$ Demanda de output
				\4[] $\to$ Regulación
				\4[] $\then$ Pueden reducir empleo
				\4 Recursos naturales
				\4[] Costes hundidos de extraer/utilizar
				\4[] Incertidumbre sobre:
				\4[] $\to$ Descubrimiento de nuevas reservas
				\4[] $\to$ Valoración futura
				\4[] $\then$ Pueden reducir extracción
	\1 \marcar{Aspectos empíricos} 25'-28'
		\2 Idea clave
			\3 Problemas de estimación
				\4 Principal determinante de inversión
				\4[] Flujos de caja futuros
				\4[] $\to$ Sujeto de estimación
				\4 ¿Qué flujos esperan los agentes?
				\4[] ¿Cómo estimar esta variable?
		\2 Teoría del acelerador
			\3 Datos
				\4 Contabilidad nacional
				\4 Relativamente inexactos
				\4 Tiempo de Tinbergen\footnote{Pionero en la valoración empírica de la teoría del acelerador.}: escasos datos
			\3 Introducción de retardos
				\4 Hecho empírico: inversión toma tiempo
				\4 Modelo del acelerador flexible
				\4[] Inversión depende de crecimientos del output pasados
			\3 Relativo éxito empírico
				\4 Elasticidad de I a Y
				\4[] En el corto plazo, muy baja
				\4[] $\to$ Compatible con acelerador con lags muy grandes
				\4[] En el largo plazo, elevada
				\4[] $\to$ Compatible con ajuste hacia $v=\frac{K}{Y}$ óptimo
				\4 Pero supuestos restrictivos
				\4[] Nula elasticidad de sustitución entre ff.pp.
				\4 Poco relevante en la actualidad
		\2 Modelo neoclásico
			\3 Hall and Jorgenson (1967)
				\4 Relativa confirmación
				\4 Anomalías:
				\4[] Elasticidad Y a K muy pequeña
				\4[] Share of capital
				\4[] Lags estimados demasiado largos
		\2 Modelo de la q
			\3 Dificil estimación de q marginal
				\4 Supuesto de q media = q marginal
			\3 Resultados
				\4 Poco satisfactorios
				\4 En modelo, todos factores tiene efecto a través de q
				\4 En estimaciones, tienen efectos por sí mismos
				\4 Efecto de q media muy pequeño
				\4[] $\to$ Aunque significativo
				\4 Reformas fiscales como experimento natural
		\2 Impuestos: experimentos naturales
			\3 Idea clave
				\4 Razonable asumir que $\Delta$ impuestos son exógenos
				\4 ¿Qué efectos sobre $q$ media y dda. inversión?
			\3 Resultados
				\4 Estudios muestran efectos fuertes sobre Q
				\4[] $\then$ Efectos sobre inversión
		\2 Mercados imperfectos
			\3 Información asimétrica
				\4 Aumento de r
				\4 Restricciones de crédito
				\4[] Stiglitz y Weiss (1991)
			\3 Correlación empírica: inversión y cash-flow
				\4 ¿Jerarquía financiera/costes de información?
				\4 ¿Cash flow es proxy de la demanda esperada?
				\4 ¿Empresas utilizan reglas mecánicas y no optimización racional?
			\3 Acelerador financiero
				\4 Bernanke (1995) y otros
				\4 Aumento de interés en crisis financiera
				\4[] Provoca contracción adicional del crédito
				\4[] $\to$ Provoca caída de inversión
				\4[] $\then$ Caída ulterior del output
				\4[] $\then$ Crisis se acelera
				\4 Integración en modelos DSGE
		\2 Flujos de caja empresariales
			\3 Idea clave
				\4 Información relativamente fácil de extraer
				\4 Posible estimar relación con inversión
			\3 Correlación flujos de caja-inversión
				\4 Muy fuerte y robusta
				\4 Empresas invierten más cuanto más cash-flow
				\4 Modigliani-Miller
				\4[] Fuentes de financiación no crean valor
				\4[] $\to$ Debería dar igual cash-flow, que deuda que equity
				\4[] $\then$ Restricciones financieras invalidan M-M?
	\1[] \marcar{Conclusión} 2'-30'
		\2 Recapitulación
			\3 Teorías de demanda de inversión
				\4 Acelerador
				\4 Modelo de Jorgenson
				\4 Modelo de la q de Tobin
			\3 Aspectos empíricos de las diferentes teorías
		\2 Idea final
			\3 Crédito bancario e inversión
				\4 Fuente de financiación de empresas afecta inversión
				\4 Empresas muy dependientes de bancos
				\4[] Especialmente en Europa continental
				\4[] $\to$ Disponen de información sobre PYMES
				\4[] $\to$ Departamentos especializados en préstamo a PYMES
				\4[] $\then$ Principal fuente de financiación
				\4 Política monetaria
				\4[] Afecta a condiciones de liquidez y financiación
				\4[] $\to$ Disponibilidad de reservas del BC
				\4[] Bancos deben ajustar oferta de crédito
				\4[] $\then$ Fuerte efecto sobre inversión
			\3 Crisis e inversión
				\4 Inversión combustible de la economía
				\4 Causa de ciclos
				\4 Crecimiento a largo plazo
			\3 Sector financiero
				\4 Mecanismo de transmisión ahorro-inversión
				\4 Retroalimentación con demanda de inversión
			\3 Decisiones de consumo
				\4 Determinan ahorro
			\3 Visión de conjunto
				\4 Necesaria
\end{esquemal}











































\graficas

\begin{axis}{4}{Modelo ricardiano de la renta diferencial y la distribución del ingreso.}{Capital-y-trabajo}{Grano}{modelodericardo}
	
	% Producto medio
	
	\draw[thick] (0,3.9) -- (3.9,1.4);	
	\node[right] at (3.9,1.4){\small Producto medio};	
	\node[left] at (0,2.55){\small PMe};
	
	% Producto marginal
	
	\draw[thick] (0,3.9) -- (3.3,0.4);	
	\node[right] at (3.3,0.4){\small Producto marginal};	
	\node[left] at (0,1.67){\small PMg};
	
	% Salario
	
	\draw[thick] (0,1) -- (4,1);	
	\node[right] at (4,1){\small Salario};
	
	% Distribución del ingreso
	
	% Línea de trabajo utilizado
	
	\draw[-] (2.1,0) -- (2.1,2.55);	
	\node[below] at (2.1,0){\small $N$};
	
	% Trabajo de estado estacionario
	
	\draw[dashed] (2.73,0) -- (2.73,1);	
	\node[below] at (2.73,0){\small $N^*$};
	
	% Separación entre beneficio y renta
	% -> Es la línea horizontal entre el eje de ordenadas y la intersección entre producto marginal y la línea de trabajo utilizado
	
	\draw[-] (0,1.67) -- (2.1,1.67);
	
	% RENTA
	
	\draw[-] (0,2.55) -- (2.1,2.55);	
	\draw[blue, fill=red, opacity=0.2] (0,3.9) -- (0,1.67) -- (2.1,1.67);	
	\draw[pattern=north east lines, pattern color=blue, opacity=0.2] (0,1.67) -- (2.1,1.67) -- (2.1,2.55) -- (0,2.55);
	\node[] at (1,2.1){\small Renta};
	\node[above] at (2.1,2.58){\small R};
	\node[right] at (2.1,1.73){A};
	
	% BENEFICIOS
	
	\draw[blue, fill=green, opacity=0.2] (2.1,1) -- (2.1,1.67) -- (0,1.67) -- (0,1);	
	\node[] at (1,1.3){\small Beneficios};	
	\node[] at (2.25,0.82){K};
	
	% SALARIOS
	
	\draw[blue, fill=yellow, opacity=0.2] (0,0) -- (2.1,0) -- (2.1,1) -- (0,1);	
	\node[] at (1,0.5){\small Salarios};	
\end{axis}

\begin{axis}{4}{Representación gráfica del problema de la inversión en el modelo de inversión de dos periodos de Fisher.}{$Y_1$}{$Y_2$}{fisherproblemainversion}
	% Dotación inicial
	\node[below] at (3,0){$E_1$};
	
	% FPP
	\draw[-] (3,0) to [out=95, in=-5](0,3);
	
	% Recta presupuestaria
	\draw[-] (0.5,4) -- (3.8,0);
	
	% Inversión y producción de óptimo
	\draw[dashed] (0,1.9) -- (2.2,1.9) -- (2.2,0);
	\node[left] at (0,1.9){$Y_2^*$};
	\node[below] at (2.2,0){$Y_1^*$};
\end{axis}


\begin{axis}{4}{Representación gráfica del problema de la optimización del consumo en el modelo de inversión de dos periodos de Fisher.}{x}{y}{fisherproblemaconsumidor}
	% Dotación inicial
	\node[below] at (3,0){$E_1$};
	
	% FPP
	\draw[-] (3,0) to [out=95, in=-5](0,3);
	
	% Recta presupuestaria
	\draw[-] (0.5,4) -- (3.8,0);
	
	% Inversión y producción de óptimo
	\draw[dashed] (0,1.9) -- (2.2,1.9) -- (2.2,0);
	\node[left] at (0,1.9){$Y_2^*$};
	\node[below] at (2.2,0){$Y_1^*$};
	
	% CI 1
	\draw[-] (0.5, 4.52) to [out=275, in=175](3.2,2.52);
	
	% CI 2
	\draw[-] (2.5,2.1) to [out=275, in=175](5.2,0.1);
\end{axis}



\begin{axis}{4}{Diagrama de fase del modelo básico de la q de Tobin}{K}{q}{fasebasico}
    \draw[-] (0,2) -- (4,2);
    \node[left] at (0,2){$1$};
    \node[right] at (4,2){$\dot{K} = 0$};
    
    \draw[-] (0.5,4) -- (4,0.5);
    \node[right] at (4,0.5){$\dot{q}=0$};
    
    \draw[-{Latex}] (0.5, 3) -- (0.9,3);
    \draw[-{Latex}] (0.5,3) -- (0.5, 2.6);
    
    \draw[-{Latex}] (3, 3) -- (3,3.4);
    \draw[-{Latex}] (3,3) -- (3.4,3);
    
    \draw[-{Latex}] (1,1) -- (1,0.6);
    \draw[-{Latex}] (1,1) -- (0.6, 1);
    
    \draw[-{Latex}] (4, 1.2) -- (4,1.6);
    \draw[-{Latex}] (4,1.2) -- (3.6, 1.2);
    
    \draw[red,thick,-{Latex}] (0,2.5) -- (2.5,2);
    \draw[red, thick, -{Latex}] (4,1.75) -- (2.5,2);
    
\end{axis}

\begin{axis}{4}{Efecto de un aumento permanente del producto}{K}{q}{permanenteproducto}[1][8]
    \draw[-] (0,2) -- (8,2);
    \node[left] at (0,2){$1$};
    \node[right] at (8,2){$\dot{K} = 0$};
    
    \draw[-] (0.5,4) -- (4,0.5);
    \node[right] at (4,0.5){$\dot{q}=0$};
    
    \draw[-{Latex}] (0.5, 3) -- (0.9,3);
    \draw[-{Latex}] (0.5,3) -- (0.5, 2.6);
    
    \draw[-{Latex}] (2,3) -- (2,3.4);
    \draw[-{Latex}] (2,3) -- (2.4,3);
    
    \draw[-{Latex}] (1,1) -- (1,0.6);
    \draw[-{Latex}] (1,1) -- (0.6, 1);
    
    \draw[-{Latex}] (4, 1.2) -- (4,1.6);
    \draw[-{Latex}] (4,1.2) -- (3.6, 1.2);
    
    \draw[red,thick,-{Latex}] (0,2.5) -- (2.5,2);
    \draw[red, thick, -{Latex}] (4,1.75) -- (2.5,2);
    
    
    \draw[dashed] (4.5, 4) -- (8,0.5);
    \node[right] at (8,0.5){$\dot{q}=0$};
    
    \draw[dashed, red,thick,-{Latex}] (2.5,2.8) -- (6.5,2);
    \draw[dashed, red, thick, -{Latex}] (8,1.75) -- (6.5,2);
    
\end{axis}

\begin{axis}{4}{Efecto de un aumento temporal del producto}{K}{q}{temporalproducto}[1][8]
    \draw[-] (0,2) -- (8,2);
    \node[left] at (0,2){$1$};
    \node[right] at (8,2){$\dot{K} = 0$};
    
    \draw[-] (0.5,4) -- (4,0.5);
    \node[right] at (4,0.5){$\dot{q}=0$};
    
    \draw[-{Latex}] (0.5, 3) -- (0.9,3);
    \draw[-{Latex}] (0.5,3) -- (0.5, 2.6);
    
    \draw[-{Latex}] (2,3) -- (2,3.4);
    \draw[-{Latex}] (2,3) -- (2.4,3);
    
    \draw[-{Latex}] (1,1) -- (1,0.6);
    \draw[-{Latex}] (1,1) -- (0.6, 1);
    
    \draw[-{Latex}] (4, 1.2) -- (4,1.6);
    \draw[-{Latex}] (4,1.2) -- (3.6, 1.2);
    
    \draw[red,thick,-{Latex}] (0,2.5) -- (2.5,2);
    \draw[red, thick, -{Latex}] (4,1.75) -- (2.5,2);
    
    \draw[dashed] (4.5, 4) -- (8,0.5);
    \node[right] at (8,0.5){$\dot{q}=0$};
    
    \draw[dashed, red,thick,-{Latex}] (2.5,2.8) -- (6.5,2);
    \draw[dashed, red, thick, -{Latex}] (8,1.75) -- (6.5,2);
    
    % Vuelta a saddle-path original
    \draw[dashed, red, -{Latex}] (2.5,2) to [out=100,in=260](2.5,2.5);
    \draw[-,red, -{Latex}](2.5,2.5) to[out=350, in=80](3.5,1.8);
    
\end{axis}

\begin{axis}{4}{Región de equilibrios con costes de ajuste con pico (\textit{kinked})}{K}{q}{kinked}
    \draw[-] (0,2) -- (4,2);
    \node[left] at (0,2){$1-c^-$};
    \node[right] at (4,3){$\dot{K} = 0$};
    
    
    \draw[-] (0,3) -- (4,3);
    \node[left] at (0,3){$1+c^+$};
    \node[right] at (4,2){$\dot{K} = 0$};
    
    \draw [draw=black, fill=yellow, opacity=0.2] (0,2) -- (0,3) -- (4,3) -- (4,2);
    
    \draw[-] (0.5,4) -- (4,0.5);
    \node[right] at (4,0.5){$\dot{q}=0$};
    
    \draw[-{Latex}] (0.5, 3.5) -- (0.9,3.5);
    \draw[-{Latex}] (0.5,3.5) -- (0.5, 3.1);
    
    \draw[-{Latex}] (3, 3.5) -- (3,3.9);
    \draw[-{Latex}] (3,3.5) -- (3.4,3.5);
    
    \draw[-{Latex}] (1,1) -- (1,0.6);
    \draw[-{Latex}] (1,1) -- (0.6, 1);
    
    \draw[-{Latex}] (4, 1.2) -- (4,1.6);
    \draw[-{Latex}] (4,1.2) -- (3.6, 1.2);
    
    \draw[red,thick,-{Latex}] (0,3.3) -- (1.5,3);
    \draw[red, thick, -{Latex}] (4,1.75) -- (2.5,2);
    
\end{axis}

\preguntas

\textbf{Test 2011}

19. En el contexto del modelo de inversión dinámico con costes de ajuste (el modelo de la Q de Tobin), cuáles son los efectos de una disminución en el impuesto sobre beneficios:

\begin{enumerate}
    \item[a] Provoca un aumento, tanto a corto plazo como a largo plazo, en el valor marginal del capital (la ratio q), aumentando la inversión y el stock de capital.
    \item[b] Provoca una disminución a corto plazo, pero un aumento en el largo plazo, en el valor marginal del capital (la ratio q), aumentando la inversión y el stock de capital.
    \item[c] No tiene ningún efecto, ni a corto ni a largo plazo, sobre el valor marginal del capital (la ratio q), si bien aumenta la inversión y el stock de capital.
    \item[d] Provoca un aumento a corto plazo en el valor marginal del capital (la ratio q), mientras que a largo plazo permanece constante, aumentando la inversión y el stock de capital.
\end{enumerate}

\textbf{Test 2008}
20. Considérese el modelo dinámico con costes de ajuste y que el gobierno en una situación en la que la economía presenta problemas, decide intentar mejorar la situación concediendo subvenciones a la inversión. Entonces el valor marginal del capital (la q):

\begin{enumerate}
    \item[a] Aumenta transitoriamente.
    \item[b] Es mayor si las subvenciones son transitorias que si son permanentes.
    \item[c] Se mantiene si las subvenciones son permanentes.
    \item[d] Es mayor si las subvenciones son permanentes que si son transitorias.
\end{enumerate}

\seccion{Test 2004}

\textbf{13.} Considere una empresa cuya tecnología es $Y_t = K^\theta_{t-1} H^{1-\theta}_t$ que opera en competencia perfecta. Si $\theta=1/3$, el tipo de interés nominal $=3\%$, la inflación $=2\%$, la tasa de depreciación $=2\%$, y la productividad media del capital es $2/3$. 

\begin{itemize}
	\item[a] La empresa llevará a cabo una inversión en capital positiva.
	\item[b] Si los costes de ajuste de la inversión son bajos podría llevar a cabo un proceso de desinversión.
	\item[c] La empresa no realizará una inversión en capital positiva.
	\item[d] Si los costes de ajuste de la inversión son altos no aumentará el capital hasta su nivel óptimo.
\end{itemize}

\notas

2011: \textbf{19. D}

2008: \textbf{20. B} Página 424 de Romer.

2004: \textbf{13. A} Tenemos que $\text{PMgK} = \theta PMeK$. Dado que $\text{PMeK} = \frac{2}{3}$ y $\theta = \frac{1}{3}$,  $\text{PMgK}=\frac{2}{9}$, que es superior al tipo de interés real $r \approx i - \pi$. La empresa aumentará su inversión en capital hasta que la productividad se igual al tipo de interés real más la depreciación del capital.

\bibliografia
Mirar en Palgrave:
\begin{itemize}
    \item acceleration principle
    \item accumulation of capital
    \item adjustment costs
    \item classical growth models
    \item Hamiltonians
    \item investment (neoclassical)
    \item inventory investment
    \item irreversible investment
    \item multiplier-accelerator interaction
    \item neoclassical growth theory
    \item Tobin's q
\end{itemize}

Bernanke, B. S. (1983) \textit{Irreversibility, Uncertainty, and Cyclical Investment} Quarterly Journal of Economics -- En carpeta del tema

Bloom, Bond y Van Reenen (2006) \textit{Uncertainty and Investment Dynamics} NBER Working Paper Series -- En carpeta del tema

Dixit, A. (1992) \textit{Investment and Hysteresis} Journal of Economic Perspectives: Winter 1992 -- En carpeta del tema

Dixit, A. K.; Pindyck, R. S. (1994) \textit{Investment under uncertainty} Princeton University Press -- En carpeta del tema

Eklund, J. \textit{Theories of Investment: A Theoretical Review with Empirical Applications} (2013) -- En carpeta del tema LEER PARA REVISAR TEMA

Parker, Jeffrey. \textit{Economics 314 Coursebook} Ch. 15 Theories of Investment Exxpenditures

Pindyck, R. (1991) \textit{Irreversibility, Uncertaint and Investment} Journal of Economic Literature -- En carpeta del tema

Romer, D. \text{Advanced Macroeconomics}. Ch. 9

Sims, E. \textit{Graduate Macro Theory II}. Notes on Investment. http://www3.nd.edu/~esims1/investment\_notes.pdf

Weber, E. J. \textit{Optimal control theory for undergraduates} (2005) -- En carpeta del tema


\end{document}
