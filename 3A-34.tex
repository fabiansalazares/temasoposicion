\documentclass{nuevotema}

\tema{3A-34}
\titulo{Teorías de la demanda de dinero. Implicaciones de política económica.}

\begin{document}

\ideaclave

\begin{itemize}
	\item ¿Qué es el dinero?
	\item ¿Por qué existe?
	\item ¿Por qué los agentes económicos demandan dinero?
	\item ¿Qué modelos teóricos de la demanda de dinero son relevantes?
	\item ¿Qué evidencia empírica existe sobre la demanda de dinero?
	\item ¿Qué recomendaciones de política económica se derivan de los modelos teóricos y la evidencia empírica?
\end{itemize}

\esquemacorto

\begin{esquema}[enumerate]
	\1[] \marcar{Introducción}
		\2 Contextualización
			\3 Concepto de dinero
			\3 Demanda de dinero en economía
			\3 Importancia de la demanda de dinero
		\2 Objeto
			\3 ¿Por qué los agentes económicos demandan dinero?
			\3 ¿Qué teorías de la demanda de dinero son relevantes?
			\3 ¿Qué evidencia empírica existe al respecto?
			\3 ¿Qué recomendaciones de política económica se derivan de las teorías?
		\2 Estructura
			\3 Análisis agregado
			\3 Microfundamentación de la demanda de dinero
			\3 Evidencia empírica
	\1 \marcar{Análisis agregado de la demanda de dinero}
		\2 Idea clave
			\3 Contexto
			\3 Objetivos
			\3 Resultados
		\2 Teoría cuantitativa del dinero
			\3 Idea clave
			\3 Formulación
			\3 Implicaciones
		\2 Demanda de dinero keynesiana
			\3 Idea clave
			\3 Formulación
			\3 Implicaciones
		\2 Monetarismo
			\3 Idea clave
			\3 Formulación
			\3 Implicaciones
		\2 NMC y otros
			\3 Idea clave
			\3 Formulación
			\3 Implicaciones
			\3 Valoración
		\2 Modelos DSGE
			\3 Idea clave
			\3 Formulación
			\3 Implicaciones
	\1 \marcar{Microfundamentación de la demanda de dinero}
		\2 Idea clave
			\3 Contexto
			\3 Objetivos
			\3 Resultados
		\2[I] Dinero en la función de utilidad (MIU)
			\3 Idea clave
			\3 Formulación
			\3 Implicaciones
		\2 Modelos RBC y DSGE
			\3 Idea clave
			\3 Formulación
			\3 Implicaciones
		\2[II] Dinero por motivo de transacción
			\3 Inventario
			\3 Precaución
			\3 Cash-in-advance (CIA)
			\3 Shopping-time -- McCallum (1989)
			\3 Búsqueda -- Kiyotaki y Wright (1989)
		\2[III] Dinero como depósito de valor
			\3 Idea clave
			\3 Formulación
			\3 Implicaciones
	\1 \marcar{Análisis empírico de la demanda de dinero}
		\2 Idea clave
			\3 Objetivo
			\3 Proceso de estimación
		\2 Problemas de medición
			\3 Dinero
			\3 Transacciones
			\3 Coste de oportunidad
			\3 Retorno nominal del dinero
			\3 Inflación
			\3 Costes fijos de transacción
			\3 Otros
		\2 Enfoques de estimación
			\3 Tradicional
			\3 Ajuste parcial
			\3 Enfoque moderno
		\2 Elasticidades estimadas
			\3 Interés
			\3 Renta
			\3 Corto y largo plazo
			\3 Evolución temporal
	\1[] \marcar{Conclusión}
		\2 Recapitulación
			\3 Análisis agregado
			\3 Microfundamentación
			\3 Evidencia empírica
		\2 Idea final
			\3 Insuficiente conocimiento
			\3 Política monetaria actual
			\3 Relación con otras áreas

\end{esquema}

\esquemalargo



















\begin{esquemal}
	\1[] \marcar{Introducción}
		\2 Contextualización
			\3 Concepto de dinero
				\4 Diferentes conceptos
				\4[] Institución social
				\4[] $\to$ Presente en casi todas sociedades
				\4[] $\to$ Bien o ficha aceptable en transacciones
				\4[] $\to$ Facilita comercio
				\4[] $\to$ Elimina ``double coincidence of wants'', Jevons (1875)
				\4[] $\to$ Aceptabilidad se retroalimenta
				\4[] Bien público
				\4[] $\to$ Registro contable descentralizado
				\4[] $\to$ Subastador walrasiano descentralizado
				\4[] $\to$ Quién tiene derecho a qué bienes
				\4[] $\to$ Incentivos a manipular
				\4[] $\to$ Amplio margen para intervención pública
				\4 Formas de dinero
				\4[] Todo tipo de materiales
				\4[] Metales preciosos
				\4[] $\to$ Inicialmente cobre
				\4[] $\to$ Cobre demasiado abundante
				\4[] $\then$ Oro y plata
				\4[] Propiedades de metales preciosos
				\4[] $\to$ Resistentes
				\4[] $\to$ Moldeables
				\4[] $\to$ Escasos
				\4[] $\to$ Propiedades estéticas
				\4[] $\to$ Demanda no sólo monetaria
				\4[] Dinero fiduciario
				\4[] $\to$ Objetos sin valor intrínseco
				\4[] $\to$ Demanda puramente monetaria
				\4[] Dinero derivado
				\4[] $\to$ Representa cantidades de dinero primario
				\4[] $\to$ Pagarés
				\4[] $\to$ Cheques al portador
				\4[] $\to$ Originalmente, billetes
				\4 Funciones del dinero
				\4[] Habituales en libros de texto
				\4[] A. Unidad de cuenta
				\4[] $\to$ Precios se expresan en uds. monetarias
				\4[] $\to$ No necesaria
				\4[] B. Medio de pago
				\4[] $\to$ Principal medio de pago
				\4[] $\to$ Función principal del dinero
				\4[] C. Depósito de valor
				\4[] $\to$ Dinero no es único depósito de valor
				\4[] $\to$ No es función principal
				\4[] $\to$ Mantiene aceptabilidad en el tiempo
				\4[] $\then$ Sin esta función, es token o ficha
			\3 Demanda de dinero en economía
				\4 No economistas
				\4[] Demanda de dinero es concepto absurdo
				\4[] ¿Quién no quiere tener más?
				\4 Economistas
				\4[] Demanda de dinero es concepto importante
				\4[] Dada una riqueza total
				\4[] $\to$ ¿Cómo distribuirla en diferentes activos?
				\4[] $\then$ ¿Cuánta riqueza mantener como dinero?
				\4[] $\then$ ¿Por qué mantener riqueza como dinero?
			\3 Importancia de la demanda de dinero
				\4 Política monetaria
				\4[] Asumiendo:
				\4[] $\to$ Liquidez disponible afecta comportamiento
				\4[] $\to$ Dinero no es neutral sobre actividad real
				\4[] Dinero afecta a actividad productiva
				\4[] $\to$ A través de diferentes canales
				\4[] $\then$ Demanda de dinero puede ser relevante
				\4 Política fiscal
				\4[] Límites a financiación no monetaria del déficit
				\4[] $\to$ Dda. es relevante para financiar vía monetaria
				\4[] Efectos de PF sobre interés y mercados financieros
				\4[] $\to$ Demanda de dinero es factor clave
				\4[] Déficit público e inflación
				\4[] $\to$ Relación compleja y estrecha
				\4[] Demanda de dinero e inflación
				\4[] $\to$ Fuertemente relacionadas
				\4[$\then$] Necesario entender demanda de dinero
				\4[] $\to$ Señoreaje como fuente de financiación
				\4[] $\to$ Hiperinflaciones
				\4[] $\to$ Impacto de inflación sobre bienestar
		\2 Objeto
			\3 ¿Por qué los agentes económicos demandan dinero?
			\3 ¿Qué teorías de la demanda de dinero son relevantes?
			\3 ¿Qué evidencia empírica existe al respecto?
			\3 ¿Qué recomendaciones de política económica se derivan de las teorías?
		\2 Estructura
			\3 Análisis agregado
			\3 Microfundamentación de la demanda de dinero
			\3 Evidencia empírica
	\1 \marcar{Análisis agregado de la demanda de dinero}
		\2 Idea clave
			\3 Contexto
				\4 Modelos neoclásicos sin dinero
				\4[] Múltiples bienes
				\4[] Unidad de cuenta:
				\4[] $\to$ Bien determinado como numéraire
				\4[] $\to$ Fichas/tokens con EDemanda=0
				\4[] Subastador
				\4[] $\to$ Equilibra demanda y oferta
				\4[] $\to$ Todos saturan RPresupuestaria
				\4 Neutralidad del dinero y dicotomía clásica\footnote{Ver ``\textit{money}'' en Palgrave, apartado The Neutrality of Money.}
				\4[] Conceptos relacionados pero distintos
				\4 Neutralidad del dinero
				\4[] Variables reales
				\4[] $\to$ Output
				\4[] $\to$ Trabajo
				\4[] $\to$ Consumo
				\4[] ¿Serían iguales si no existe dinero?
				\4[] $\to$ Sí: dinero es neutral
				\4[] $\to$ No: dinero no es neutral
				\4 Dicotomía clásica
				\4[] Don Patinkin (1956)
				\4[] Variables reales
				\4[] $\to$ Independientes de nominales
				\4[] $\then$ Dicotomía clásica
				\4[] Shocks monetarios
				\4[] Afectan sólo variables nominales
				\4[] $\Delta$ en precios relativos y vars. reales
				\4[] $\to$ Sólo debido a shock reales
				\4 Precio positivo del dinero
				\4[] Bien con o sin valor intrínseco
				\4[] $\to$ Existe demanda más allá de consumo
				\4[] $\then$ Precio no negativo
				\4[] ¿Por qué?
				\4[] $\to$ Porque cumple tres funciones del dinero
				\4[] $\then$ Es dinero
				\4[] ¿Qué factores determinan demanda?
				\4[] $\to$ Qué relación con precio de dinero?
			\3 Objetivos
				\4 Explicar demanda agregada de dinero
				\4 Caracterizar efectos de $\Delta$ dda. agregada de dinero
				\4 Postular reglas heurísticas que definan $\Delta$ dda. dinero
			\3 Resultados
				\4 Modelos que explican variables agregadas
				\4[] Precios
				\4[] Output
				\4[] $\to$ A partir de reglas heurísticas sobre dda. de dinero
		\2 Teoría cuantitativa del dinero
			\3 Idea clave
				\4 Contexto
				\4[] Martin de Azpilcueta
				\4[] Jean Bodino
				\4[] Hume
				\4[] Relación empírica entre:
				\4[] $\to$ Precios
				\4[] $\to$ Cantidad de metal precioso
				\4 Objetivos
				\4[] Explicar relación entre precios y metal precioso
				\4[] Caracterizar demanda de dinero
				\4[] $\to$ Que relaciona precios y cantidad de dinero
				\4 Resultados
				\4[] Relación directa entre:
				\4[] $\to$ Cantidad de dinero
				\4[] $\to$ Nivel de precios
				\4[] Varías teorías en una
				\4[] $\to$ Demanda de dinero
				\4[] $\to$ Nivel de precios
				\4[] $\to$ Inflación
				\4[] Implícitamente asume
				\4[] $\to$ Demanda de dinero es estable
				\4[] $\to$ PIB real exógeno a factores monetarios
				\4[] $\to$ Oferta monetaria es exógena
			\3 Formulación
				\4 Identidad cuantitativa
				\4[] $MV \equiv PT$
				\4[] $V$ iguala $MV$ con $PT$
				\4[] $\to$ $V \equiv \frac{PT}{M}$
				\4[] $\then$ $V$ como variable de ajuste
				\4 Ecuación cuantitativa $\to$ Eje de TCD
				\4[] $M_s V = PT$
				\4[] $\to$ $V$ es exógena
				\4[] $\then$ No se ajusta automáticamente
				\4[] $\then$ Otras variables deben ajustarse
				\4 Formulación de Cambridge
				\4[] $k = \frac{1}{V} = \frac{PT}{M}$
				\4[] $M_d = \frac{1}{v} PT = k PT$
				\4[] $\to$ DDinero depende de transacciones
				\4[] $k$ puede ser función de $i$, $Y$
				\4[] $\to$ Relajación de neutralidad del dinero
				\4[] $\to$ Posibilidad de PolMon
				\4 Output como proxy de nivel de transacciones
				\4[] Cantidad de transacciones
				\4[] $\to$ Prácticamente imposible de medir
				\4[] $\then$ Sujeto a integración vertical
				\4[] $\then$ Problemas de contabilidad
				\4[] Valor añadido total/PIB
				\4[] $\to$ Existen medidas aproximadas
			\3 Implicaciones
				\4 Nivel de precios
				\4[] $\to$ Si $M$ y $T$ son exógenos:
				\4[] $\then$ Dicotomía clásica
				\4[] $\then$ Precios dependen de M y V
				\4[] $\then$ M es instrumento de política monetaria
				\4[] $\then$ V es demanda de dinero y determina efectos de PM
				\4 Renta como proxy de transacciones
				\4[] $\to$ Transacciones difícilmente medibles
				\4[] $\to$ Sujetas a distorsiones por IVertical
				\4[] $\to$ Relación T--Y aprox. lineal
				\4[] $\then$ Renta total como proxy
				\4 Oferta monetaria determina inflación
				\4[] Dados:
				\4[] $\to$ Demanda de dinero exógena
				\4[] $\to$ Output exógeno
				\4[] Cantidad de dinero determina inflación
				\4 Ecuación de Fisher
				\4[] Tipo de interés nominal
				\4[] $\to$ Se ajusta para igualar real + inflación
				\4[] $\then$ $i = r + \pi^e$
				\4[] Crecimiento elevado de la oferta monetaria
				\4[] $\to$ Más inflación
				\4[] $\then$ Tipos de interés elevados
				\4[] Contracción de la oferta monetaria
				\4[] $\to$ Menos inflación
				\4[] $\then$ Tipos de interés más bajos
				\4 Política monetaria
				\4[] $M^S$ asumida determinable por autoridad monetaria
				\4[] Demanda $k$/Velocidad $V$ estables
				\4[] En el c/p, precios más o menos rígidos
				\4[] Posible ajustar $M$ para
				\4[] $\to$ Variar precios
				\4[] $\to$ Variar output
				\4[] En qué medida $\Delta M$ se transmite a $P$ o $Y$
				\4[] $\to$ Cuestión generalmente empírica
				\4[] $\to$ Generalmente, se asume que a P
				\4[] $\to$ Efectos a $Y$ muy leves/transitorios
		\2 Demanda de dinero keynesiana
			\3 Idea clave
				\4 Contexto
				\4[] Críticas a modelo (neo)clásico
				\4[] $\to$ Ajuste a equilibrio de pleno empleo no se produce
				\4[] $\to$ Dinero es activo de depósito de valor
				\4[] Dinero puede ser depósito de valor
				\4[] $\to$ Además de token
				\4[] $\to$ Compite con otros activos financieros: bonos
				\4[] Necesidad de estimular demanda agregada
				\4[] $\to$ Explorar usos de la oferta monetaria
				\4 Objetivos
				\4[] Considerar efecto de demanda por especulación
				\4[] Caracterizar trampa de liquidez
				\4 Resultados
				\4[] Demanda de dinero es menos estable
				\4[] $\to$ Depende también de interés de bonos
				\4[] Dinero afecta a tipo de interés vía bonos
				\4[] $\to$ Afecta a demanda de inversión
				\4[] $\then$ Estímulo a demanda agregada
				\4[] Trampa de liquidez
				\4[] $\to$ Elasticidad infinita a interés cuando $i \to 0$
			\3 Formulación
				\4 Tres motivos para demandar dinero
				\4[i.] Transacción
				\4[] $\to$ Necesario para intercambiar bienes
				\4[ii.] Precaución
				\4[] $\to$ Reducir costes de transacción
				\4[iii.] Especulación
				\4[] $\to$ Alternativa a dinero (bonos) pueden $\downarrow$ precio
				\4[] $\to$ Dinero permite ganancia si bonos $\downarrow$
				\4[] \fbox{$M_D \equiv P \cdot L(i,Y)$}
				\4[] $L_i < 0$, $L_Y > 0$
				\4[] $Y$: motivos de transacción y precaución
				\4[] $i$: motivo de especulación
			\3 Implicaciones
				\4 Interés afecta a demanda de dinero
				\4[] $i$ muy bajo implica bonos muy altos
				\4[] $\to$ Sólo pueden bajar de precio
				\4[] $\then$ Agentes prefieren dinero
				\4[] $\then$ Aceptan cualquier cantidad de dinero
				\4[] $\then$ Posible trampa de liquidez
				\4 Demanda altamente inestable
				\4[] $\to$ Sujeta a fluctuaciones cíclicas
				\4 Política monetaria
				\4[] Puede causar:
				\4[] $\to$ Contracción de output y expansión
				\4[] $\to$ Inflación si límite de capacidad
				\4[] Inestabilidad de demanda de dinero
				\4[] $\to$ Dificulta implementación práctica
				\4[] Lags de implementación y expectativas
				\4[] $\to$ Oferta monetaria es instrumento problemático
				\4[] $\to$ Preferible política fiscal
				\4 Trampa de liquidez
				\4[] Demanda de dinero infinita cuando interés $\to$ 0
				\4[] Demanda puede absorber cualquier $\Delta M$
				\4[] $\to$ Sin efecto de PM sobre interés
				\4[] $\then$ Sin efecto de PM sobre output
		\2 Monetarismo
			\3 Idea clave
				\4 Contexto
				\4[] Keynesianismo
				\4[] $\to$ Demanda de dinero inestable
				\4[] $\then$ Política monetaria pasiva a fiscal
				\4[] $\then$ Mantener déficit bajo
				\4[] $\then$ Buen funcionamiento en 50s y 60s
				\4[] $\then$ Progresivo deterioro efectividad PEconómica
				\4[] Años 60 y primeros 70
				\4[] $\to$ Inflación creciente
				\4[] $\to$ PEconómica no reduce desempleo
				\4[] $\to$ Tensión sobre dólar
				\4 Objetivo
				\4[] Fundamentar importancia de política monetaria
				\4[] $\to$ Explicar por qué demanda monetaria estable
				\4[] Contrastar con evidencia empírica
				\4[] $\to$ ¿Causa exceso de oferta monetaria inflación?
				\4[] $\to$ ¿Compatible evidencia con dda. de dinero estable?
				\4 Resultados
				\4[] Reformulación de la teoría cuantitativa
				\4[] Demanda de dinero principal instrumento de PEconómica
				\4[] $\to$ Justificación más elaborada
				\4[] Ecuación perdida
				\4[] $\to$ Sin teoría sobre transmisión a P y a Y
			\3 Formulación
				\4 \fbox{$M_D = P \cdot f(Y_P,W,R_M^*, R_B^*, R_E^*, u)$}
				\4[] $Y_P$: renta permanente
				\4[] $W$: riqueza no humana
				\4[] $R_M^*$: rendimiento esperado del dinero
				\4[] $R_M^*$: rendimiento esperado de los bonos
				\4[] $R_E^*$: rendimiento esperado de activos físicos
				\4[] $u$: otros factores (volatilidad...)
			\3 Implicaciones
				\4 Velocidad del dinero no es constante
				\4[] $\to$ Pero es relativamente estable y predecible
				\4[] $\to$ No depende sólo de un tipo de interés
				\4[] $\to$ Política monetaria puede basarse en dinero
				\4 LM monetarista
				\4[] $\to$ Depende menos del tipo de interés
				\4[] $\to$ Depende menos de la renta presente
				\4[] $\then$ Pendiente más elevada que LM keynesiana
		\2 NMC y otros
			\3 Idea clave
				\4 Contexto
				\4[] Inicio de microfundamentación
				\4[] Énfasis en oferta
				\4[] $\to$ No tanto en demanda
				\4 Objetivos
				
				\4 Resultados
				\4[] Reglas ad-hoc sobre demanda de dineroº
				\4[] Basadas en formas de TCD o Keynes
			\3 Formulación
			\3 Implicaciones
			\3 Valoración
		\2 Modelos DSGE
			\3 Idea clave
				\4 Contexto
				\4 Objetivos
				\4 Resultados
			\3 Formulación
				\4[] Generalmente, MIU
				\4[] Saldo monetario real
				\4[] $\to$ Argumento de función de u.
				\4[] Ejemplo:
				\4[] $\underset{C_t, M_t, L_t}{\max} \quad \mathcal{U} = \sum_{t=0}^{\infty} \beta^t \left[ U(C_t) + \Gamma \left( \frac{M_t}{P_t} \right) -V(L_t) \right]$
				\4[] $0 < \beta < 1$
				\4[] $\text{s.a:} \quad A_{t+1} - A_t = i_t A_t + M_t + (1+i_t)(W_t L_t - P_t C_t - M_t)$
			\3 Implicaciones
				\4[] \fbox{$\left. \frac{M_t}{P_t} \right|_D = L(Y_t,i_t) = Y_t^{\beta/v} \left( \frac{1+i_t}{i_t} \right)^{1/v}$}
				\4[] $\to$ Asumiendo funciones CEIS $U(C_t)$ y $\Gamma \left( \frac{M_T}{P_t} \right)$
				\4[] Permite estudio de:
				\4[] $\to$ Impacto del dinero sobre output
				\4[] $\to$ Impacto de dinero sobre precios
				\4[] $\to$ Inflación óptima
	\1 \marcar{Microfundamentación de la demanda de dinero}
		\2 Idea clave
			\3 Contexto
				\4 Agentes (representativos) individuales
				\4 Más allá de efectso agregados
				\4[] $\to$ ¿Por qué un agente quiere dinero?
			\3 Objetivos
				\4 Explicar demanda de dinero de agentes individuales
				\4 Racionalizar decisiones de demanda de dinero
				\4 Explicar dda. de dinero evitando crítica de Lucas
			\3 Resultados
				\4 Tres estrategias de explicación
				\4[] I. Dinero en la función de utilidad
				\4[] $\to$ Realmente no explica nada
				\4[] $\to$ Supuesto ad-hoc
				\4[] $\to$ Dinero en sí mismo ``hace felices'' a agentes
				\4[] II. Dinero para transaccionar
				\4[] $\to$ Dinero reduce costes de transacción
				\4[] $\to$ ¿Por qué reduce costes de transacción?
				\4[] III. Dinero como depósito de valor
				\4[] $\to$ Dinero es convención social
				\4[] $\to$ Permite transferencia intertemporal
		\2[I] Dinero en la función de utilidad (MIU)
			\3 Idea clave
				\4 Sidrauski (1967)network industries
			\3 Formulación
				\4[] Consumidor representativo
%				\4[] $\underset{c_t, m_t}{\max} \quad \int_0^\infty e^{-\rho t} u(c_t, m_t) \, \text{d} t$
%				\4[] $\text{s.a.}: \quad \dot{a}_t = f(k_t) - \delta k_t - c_t - \pi_t m_t + v_t  = \dot{k} - \pi_t m_t + v_t$
%				\4[] \quad \quad \quad $\lim_{t\to \infty} \, e^{-\rho t} a_t \geq 0$
				\4[] $\underset{c,m,k,b}{\max} \quad U=\sum_{t=0}^{\infty} \beta^t u(c_t,m_t)$
				\4[] $\text{s.a}: \quad \; c_t + \Delta k_t + \Delta m_t + b_t -(1+i_t) b_{t-1} = f(k_t) + v_t$
				\4[] $\quad \quad \quad v_t = \pi_t m_t $
				\4[] $\quad \quad \quad m_t = \frac{M_t}{P_t}$
				\4[] $\to$ Decide $c_t$, $m_t$, inversión en $k_t$ y bonos
				\4[] $\to$ $m_t = \frac{M_t}{P_t}$
				\4[] $\to$ $\Delta m_t = \frac{M_t - M_{t-1}}{P_t}$
				\4[] $\to$ $v_t$: transferencias monetarias del gobierno
				\4[] Gobierno
				\4[] $\to$ Elige senda de oferta monetaria $M_t$
				\4[] $\to$ Presupuesto equilibrado: $v_t = \pi_t m_t$
				\4[] Precios
				\4[] $\to$ Resultan de oferta y demanda de dinero
				\4[] $\then$ $M_t$ depende de gobierno
				\4[] $\then$ $m_t$ depende de agente y gobierno
				\4[] Condición de óptimo
				\4[] \fbox{$\frac{\partial u/\partial m_t}{\partial u/\partial c_t} = 1 - \frac{1}{1+i_t} $}
				\4[] $\then$ $m_t^d = f(i_t, y_t) = f(r_t, \pi_t, y_t)$
				\4[] $\then$ $u_m=0$ $\then$ $i_t=0$ $\then$ $\pi = -r_t$
				\4[] $\then$ Regla de Friedman
			\3 Implicaciones
				\4[] Óptimo dependiente de formas de:
				\4[] $\to$ Función de utilidad
				\4[] $\to$ Función de producción
				\4[] $\to$ Política monetaria
				\4[] Demanda de dinero depende de:
				\4[] $\to$ Negativamente del tipo de interés
				\4[] Generalmente
				\4[] $\to$ Existe estado estacionario
				\4[] $\to$ Existe política monetaria óptima
				\4[] $\to$ Regla de Friedman\footnote{Ver conceptos.}
				\4[] $\then$ Inflación óptima la que implica INominal=0
				\4[] Superneutralidad
				\4[] $\to$ $\Delta$ \% de M no afecta variables reales
				\4[] $\to$ Resultado muy poco robusto
				\4[] $\to$ Con oferta de trabajo elástica, no se cumple
				\4[] Análisis de bienestar
				\4[] $\to$ Explicita preferencias del agente
				\4[] $\to$ Permite análisis de bienestar
				\4[] $\to$ Permite valorar inflación óptima
				%\4[] VER REED, CECO viejo, BRZOZA SLIDES (en carpeta del tema)
		\2 Modelos RBC y DSGE
			\3 Idea clave
			\3 Formulación
				\4[] Generalmente, MIU
				\4[] También, CIA
				\4[] Saldo monetario real
				\4[] $\to$ Argumento de función de u.
				\4[] Ejemplo:
				\4[] $\underset{C_t, M_t, L_t}{\max} \quad \mathcal{U} = \sum_{t=0}^{\infty} \beta^t \left[ U(C_t) + \Gamma \left( \frac{M_t}{P_t} \right) -V(L_t) \right]$
				\4[] $0 < \beta < 1$
				\4[] $\text{s.a:} \quad A_{t+1} - A_t = i_t A_t + M_{t+1} + (1+i_t)(W_t L_t - P_t C_t - M_t)$
			\3 Implicaciones
				\4[] \fbox{$\frac{M_t}{P_t} = L(Y_t,i_t) = Y_t^{\beta/v} \left( \frac{1+i_t}{i_t} \right)^{1/v}$}
				\4[] $\to$ Asumiendo funciones CEIS $U(C_t)$ y $\Gamma \left( \frac{M_T}{P_t} \right)$
				\4[] Permite estudio de:
				\4[] $\to$ Impacto del dinero sobre output
				\4[] $\to$ Impacto de dinero sobre precios
				\4[] $\to$ Inflación óptima
		\2[II] Dinero por motivo de transacción
			\3 Inventario
				\4[] Baumol (1952) y Tobin (1956)
				\4[] Dos activos posibles:
				\4[] $\to$ Dinero
				\4[] $\to$ Bonos
				\4[] Dinero implica dos costes:
				\4[] i. Por tener
				\4[] $\to$ Interés perdido
				\4[] $\to$ Coste de oportunidad
				\4[] ii. Por no tener
				\4[] $\to$ Transacciones requieren dinero
				\4[] $\to$ Convertir bonos a dinero
				\4[] $\to$ Tomar prestado dinero
				\4[] Agentes demandan dinero
				\4[] $\to$ Para minimizar ambos costes
				\4[] Asumiendo separabilidad
				\4[] $\to$ Costes con resto de utilidad
				\4[] Problema de minimización:
				\4[] $\to$ Decidir saldo medio $m$
				\4[] Dados:
				\4[] $\to$ $T$: transacciones necesarias
				\4[] $\to$ $m$: saldo monetario
				\4[] $\to$ $i$: interés nominal $\to$ coste de oportunidad
				\4[] $\to$ $F$: coste fijo de obtener liquidez
				\4[] $\underset{m}{\min}\quad \frac{V}{m} \cdot t + \frac{m}{2} \cdot i$
				\4[] CPO: $- \frac{V}{m^2} t + \frac{i}{2} = 0$
				\4[] $\then$ \fbox{$m^* = f(V,t,i) = \sqrt{\frac{2Vt}{i}}$}
				\4[] i. Caso particular: $t=0$
				\4[] Costes fijos de conversión nulos
				\4[] $\to$ Conversión ``al vuelo''
				\4[] $\to$ Se tiene dinero en instante de pago
				\4[] ii. Caso particular: $i=0$
				\4[] Sin coste de oportunidad por tener dinero
				\4[] $\then$ Demanda tiende a infinito
				\4[] $\then$ Puede tenerse toda la riqueza en dinero
				\4[] iii. Casos particulares intermedios: $t>0$, $i>0$
				\4[] $\to$ Demanda tiende a $m^*$
				\4[] $\then$ No toda la riqueza se mantiene a dinero
			\3 Precaución
				\4[] Laidler (1993)
				\4[] Similar a Baumol-Tobin
				\4[] $\to$ Modelizando riesgo explícitamente
				\4[] Demanda de dinero $m$ para minimizar
				\4[] $\to$ Riesgo de incurrir en coste
				\4[] Gasto que afrontar $S$
				\4[] $\to$ No se conoce con certeza
				\4[] $\to$ Distribución de prob. no degenerada
				\4[] $m$: Demanda precautoria de dinero
				\4[] Probabilidad de gastar menos que $s$
				\4[] $P \left[ S \leq m \right] = \int_{-\infty}^{m} p(S) \text{d} \, S$
				\4[] Si gasto $S > m$:
				\4[] $\to$ Incurre en coste fijo $b$
				\4[] $i$: coste de oportunidad de mantener $m$
				\4[] Coste total
				\4[] $\underset{m}{\min}\quad \text{TC} = (1-P[S \leq m])b + im$
				\4[] CPO: $-b \dv{P[ S \leq m]}{S} + i = 0$
				\4[] $\then$ \fbox{$bp(S) = i$}
				\4[] Igualdad entre:
				\4[] $\to$ Coste marginal de tenencia $i$
				\4[] $\to$ Coste marginal esperado de no tenencia $bf(s)$
			\3 Cash-in-advance (CIA)
				\4[] Clower (1967), Lucas (1980a), Svensson (1985)
				\4[] Muy utilizado en RBC monetarios + nuevo monetarismo
				\4[] Contexto de crecimiento neoclásico
				\4[] $\to$ Restricción adicional
				\4[] Consumo debe ser $\leq$ saldo monetario
				\4[] \fbox{ $P_t c_t \leq M_t$ $\then$ $c_t \leq \frac{M_t}{P_t}$ }
				\4[] $\then$ Necesario mantener saldos monetarios
				\4[] Problema de maximización
				\4[] $\underset{C_t, L_t, M_t, b_t}{\max} \quad \sum_{t=0}^\infty \beta_t u(c_t, L_t)$
				\4[] $\quad \text{s.a:}$
				\4[] $\quad \quad P_t c_t + M_t + B_t \leq P_t w_t (1-L_t) + M_{t-1} + (1+i_{t-1}) B_{t-1} + T_t$
				\4[] $\quad \quad c_t \leq \frac{M_t}{P_t}$
				\4[] Dinero tiene coste de oportunidad
				\4[] $\to$ Alternativa a dinero genera mayor retorno
				\4[] $\to$ Retorno de otros activos $R > 0$
				\4[] $\to$ Retorno del dinero $\frac{P_{t-1}}{P_t}$
				\4[] $\then$ Inflación induce retorno negativo
				\4[] Gobierno decide política monetaria
				\4[] $\to$ Trayectoria de crecimiento de $M_t$
				\4[] \underline{Senda de crecimiento de $M_t$ conocida}
				\4[] Restricción vinculante $P_t c_t = M_t$
				\4[] $\to$ Demandan mínimo dinero para consumir $P_t c_t$
				\4[] $\then$ Velocidad constante del dinero
				\4[] $\then$ No depende del interés
				\4[] \underline{Senda de crecimiento desconocida}
				\4[] Restricción puede no vincular
				\4[] Agentes no conocen precios con certeza
				\4[] $\to$ Demandan saldos más altos por precaución
				\4[] $\to$ Interés pasa a ser factor relevante
				\4[] $\then$ Demanda depende de renta e interés
				\4[] $\then$ Velocidad del dinero varían el tiempo
				\4[] \underline{Modelos cash and credit}
				\4[] Dos bienes generan utilidad
				\4[] Bien 1 necesita cash, bien 2 no
				\4[] $\to$ Interés alto reduce consumo de bien 1
				\4[] $\then$ Velocidad del dinero variable
				\4[] Implicaciones generales
				\4[] $\to$ Dicotomía clásica puede no satisfacerse
				\4[] $\to$ Variables nominales pueden variar consumo
				\4[] $\to$ Posible mostrar que inflación óptima es 0
				\4[] $\to$ Inflación esperada es relevante, no inesperada
				\4[] $\to$ Posible mostrar regla de Friedman\footnote{El tipo de interés nominal óptimo es 0, ya que el coste marginal de crear dinero para el dinero es 0 y así el rendimiento del dinero iguala el del capital.}
				\4[] $\to$ Restricción CIA es arbitraria
				\4[] $\to$ Sin justificar razones para tener dinero
				\4[] $\to$ Velocidad del dinero constante
			\3 Shopping-time -- McCallum (1989)
				\4[] Basado en ciclo real
				\4[] Explicita motivo para demandar dinero
				\4[] Dinero reduce costes de transacción
				\4[] $\to$ Porque reduce tiempo de compra
				\4[] $\then$ Ocio aumenta con saldo monetario
				\4[] $\underset{c_t, l_t}{\max} \quad \sum_{t=0}^\infty \beta^t u(c_t,l_t)$
				\4[] $\text{s.a:} \quad \sum_{t=1}^\infty \frac{c_t}{(1+r)^t} + \sum_{t=1}^\infty \frac{M_t-M_{t-1}}{(1+i)^t} = \sum_{t=1}^\infty \frac{y_t}{(1+r)^t}$
				\4[] \quad \quad \quad $l_t \leq 1 - s_t - n_t$
				\4[] \quad \quad \quad $s_t = \phi(c_t, m_t)$, \quad $\phi_c > 0$, $\phi_m \leq 0$
				\4[] Bajo funciones $u(\cdot)$ y $\phi(\cdot)$ razonables
				\4[] $\to$ Demanda de saldos reales \fbox{$\frac{M_t}{P_t} = L(c_t, R_t) $}
				\4[] Implicaciones:
				\4[] $\to$ Pueden aparecer problemas de agregación
				\4[] $\to$ Demanda de dinero de empresas requiere cambios
				\4[] $\to$ Relativamente poca aplicación
				\4[] REED, PAG 33 PALGRAVE: DEMAND THEORY
			\3 Búsqueda -- Kiyotaki y Wright (1989)
				\4[] Modelos explicitan poco el papel del dinero
				\4[] $\to$ ¿Por qué necesario para consumir?
				\4[] $\to$ ¿Por qué reduce tiempo de compra'
				\4[] $\to$ ¿Por qué dinero fiduciario puede tener valor?
				\4[] $\then$ Asumen que genera utilidad/reduce coste
				\4[] Jevons: ``Double coincidence of wants''
				\4[] Asumiendo:
				\4[] $\to$ Cada agente dispone de un bien único $Y_i$
				\4[] $\to$ Desea intercambiar por otro bien
				\4[] $\then$ Necesaria ``doble coincidencia de deseos''
				\4[] Probabilidad de que demandas se ``encuentren''
				\4[] $\to$ Más baja cuantos más productos y agentes
				\4[] Aparición de bien X aceptado más que el resto
				\4[] $\to$ Más probable vender X que bien propio $Y_i$
				\4[] $\to$ Agentes aceptan cambiar por bien X
				\4[] $\then$ Bien X se convierte en dinero
				\4[] $\then$ Hay demanda de dinero
		\2[III] Dinero como depósito de valor
			\3 Idea clave
				\4 Samuelson (1958)
				\4 Wallace (1980), Sargent y Wallace (1981)
				\4 Dinero como institución social
				\4[] Convención entre agentes anónimos
				\4 Modelos de generaciones solapadas
				\4[] Principal herramienta en 80s
				\4[] Agentes heterogéneos
				\4[] $\to$ Diferentes horizontes temporales
				\4[] $\to$ Conviven en mismos periodos
				\4[] Generalmente, suboptimalidad paretiana
				\4[] Herramienta principal de modelización
				\4[] $\to$ De dinero como depósito de valor
			\3 Formulación
				\4 Modelo simple
				\4[] Jóvenes
				\4[] $\to$ Viven dos periodos
				\4[] $\to$ Reciben dotación en primer periodo
				\4[] $\to$ Cada generación, $n$\% jóvenes más
				\4[] Viejos
				\4[] $\to$ No reciben nada
				\4[] $\to$ Sólo consumen
				\4[] Viejos en primer periodo
				\4[] $\to$ Nunca son jóvenes
				\4[] $\to$ Nunca reciben nada
				\4[] Equilibrio competitivo
				\4[] $\to$ Jóvenes no prestan a viejos
				\4[] $\to$ Viejos no consumen nada
				\4[] $\to$ Jóvenes consumen todo
				\4[] $\then$ No es óptimo de Pareto
			\3 Implicaciones
				\4 Dinero como óptimo descentralizado
				\4[] Dinero es papeles de colores
				\4[] Viejos en $t=0$ reciben papeles
				\4[] Existen múltiples equilibrios
				\4[] Equilibrio monetario
				\4[] $\to$ Viejos en $t=0$ venden papeles a jóvenes de $t=0$
				\4[] $\to$ Jóvenes en $t=1$ compran a jóvenes de $t=0$\footnote{Jóvenes de $t=0$ que se han convertido en viejos en $t=1$}
				\4[] $\to$ Venta intergeneracional hasta infinito
				\4[] $\to$ Cada vez más jóvenes demandan mismo dinero
				\4[] $\then$ Deflación
				\4[] $\then$ Dinero permite ahorro
				\4[] $\then$ Equilibrio monetario no es único

				\4 Política monetaria
				\4[] Wallace (1980) y otros\footnote{Ver exposición simple en Heijdra pags. 339 y ss.}
				\4[] Generalizan modelo anterior
				\4[] Dinero crece a tasa determinada
				\4[] $\to$ Política monetaria
				\4[] Viejos reciben transferencia monetaria
				\4[] $\to$ Extraída de aumento de $M_t$
				\4[] Tecnología de almacenamiento disponible
				\4[] $\to$ Capital reversible
				\4[] Capital se deprecia a tasa $\delta$
				\4[] Equilibrio monetario aparece si:
				\4[] $\to$ Inflación menor que depreciación
				\4[] En caso contrario:
				\4[] $\to$ Nadie demanda dinero
				\4 Introducción de otras funciones
				\4[] Considerar únicamente depósito de valor
				\4[] $\to$ Conclusiones poco realistas
				\4[] $\to$ Equilibrios no monetarios
				\4[] Modelos mixtos introducen:
				\4[] $\to$ Restricciones cash-in-advance
				\4[] $\then$ Resultados más realistas
				\4[] $\then$ Equilibrios monetarios en todos casos
				\4[] $\then$ Mejor ajuste empírico
	\1 \marcar{Análisis empírico de la demanda de dinero}
		\2 Idea clave
			\3 Objetivo
				\4 Encontrar relaciones causales
				\4 Determinantes que expliquen demanda de dinero
			\3 Proceso de estimación
				\4[1] Definir de forma precisa
				\4[] i) Variable a explicar
				\4[] ii) Factores determinantes
				\4[2] Medir variables consideradas
				\4[] Aplicando definición a partir de teoría
				\4[] Encontrando proxies si necesarios
				\4[] Valorando sesgos y desviaciones
		\2 Problemas de medición
			\3 Dinero
				\4 Motivo de uso de dinero
				\4[] Factor clave definición
				\4[] Muchas formas de dinero posibles
				\4 Dinero para transacción y precaución
				\4[] Liquidez de la forma de dinero
				\4[] $\to$ Característica relevante
				\4[] Clases fundamentales de dinero para transacción
				\4[] $\to$ Dinero en efectivo
				\4[] $\to$ Depósitos a la vista
				\4[] Innovaciones financieras
				\4[] $\to$ Difuminan la línea de lo incluible
				\4[] $\to$ Depósitos a la vista con comisiones
				\4[] $\to$ Billetes de 500€ sospechosos por hacienda
				\4[] $\to$ Depósitos a plazo de alta liquidez
				\4[] $\to$ Cuentas corrientes no convertibles
				\4[] $\to$ Cuentas eurodólar
				\4[] $\to$ ...
				\4 Enfoque de servicios del dinero
				\4[] Diferentes formas de dinero
				\4[] $\to$ Diferentes grados y servicios
				\4[] ``Moneyness''
				\4[] $\to$ Cantidad de servicios de una forma de dinero
			\3 Transacciones
				\4 Definición simple
				\4[] Suma de transacciones mediadas con dinero
				\4 Medición imposible
				\4[] Transacciones no registradas
				\4[] Trueques
				\4[] Transacción vía crédito
				\4[] $\then$ Necesaria variable proxy
				\4 PIB o producción como proxy habitual
				\4[] Sujeto a distorsiones
				\4[] $\to$ Integración vertical
				\4[] $\to$ Bienes intermedios no incluidos
				\4[] $\to$ Servicios financieros difíciles de medir
				\4[] $\to$ Distribución del PIB relevante
			\3 Coste de oportunidad
				\4 Retorno del dinero vs alternativas
				\4 Retorno de alternativas
				\4[] ¿Cuáles?
				\4[] $\to$ Depósitos a la vista
				\4[] $\to$ Letras a c/p
				\4[] $\to$ Papel comercial
				\4[] $\to$ Bonos del gobierno a l/p
				\4[] $\to$ ...
				\4 En la práctica
				\4[] Utilizadas una o varias
				\4[] Contrastar robustez de estimación
			\3 Retorno nominal del dinero
				\4 Dinero en efectivo
				\4[] $\to$ Retorno nominal nulo
				\4 Cuentas corrientes a la vista
				\4[] Pequeños retornos nominales
				\4[] $\then$ ¿Tener en cuenta?
			\3 Inflación
				\4 ¿Qué medida de inflación utilizar?
			\3 Costes fijos de transacción
				\4 Contexto de Baumol-Tobin
				\4[] Coste de convertir a efectivo
				\4 ¿Qué medida utilizar?
				\4[] $\to$ ``shoe-leather'' cost
				\4[] $\to$ Coste de intermediación/brokerage
				\4[] ...
				\4[$\then$] Coste existe pero muy difícil medida
			\3 Otros
				\4[] ...
		\2 Enfoques de estimación
			\3 Tradicional
				\4 $\ln \frac{M_t}{P_t} = \beta_0 + \beta_1 \ln y_t + \beta_2  r_t + \vec{\beta_3} \vec{x}_t + \epsilon_t$
			\3 Ajuste parcial
				\4 $\ln \frac{M_t}{P_t} - \ln \frac{M_{t-1}}{P_{t-1}} = \gamma \left[ \ln \frac{M^*}{P_t} - \ln \frac{M_t}{P_t} \right]$
				\4[] Donde $M^*$ es la demanda estimada $\uparrow$
			\3 Enfoque moderno
				\4 Especificación logarítmica
				\4[] \fbox{$\ln \frac{M_t}{P_t N_t} = \frac{1}{b} \ln \left( \frac{1-a}{a} \right) + \ln c - \frac{1}{b} \ln \frac{i}{1+i}$}
		\2 Elasticidades estimadas
			\3 Interés
				\4 Estimaciones
				\4[] Siempre negativa
				\4[] $\to$ Más interés, menos demanda
				\4[] Generalmente, inferior a -0,5
				\4[] $\to$ Diferente a predicción de Baumol-Tobin
				\4[] En algunos casos se acerca
				\4 Margen intensivo y extensivo
				\4[] Tipos de interés bajos
				\4[] $\to$ No se demandan activos que pagan interés
				\4[] Tipos de interés altos
				\4[] $\to$ Aparece dda. positiva de activos con interés
				\4[] $\then$ Margen extensivo
				\4[] $\then$ Hallazgo robusto
				\4 Variación de la elasticidad al interés
				\4[] Mulligan y Sala-i-Martin (2000)
				\4[] Aumenta con nivel de interés nominal
				\4[] Elast. muy baja con tipos bajos
			\3 Renta
				\4 Generalmente
				\4[] Mayor a 0,5 (de Baumol-Tobin)
				\4[] Mayor en el largo plazo
				\4[] $\then$ Favorable a hipótesis renta permanente
			\3 Corto y largo plazo
				\4 Largo plazo
				\4[] Variables explicativas son determinantes dda.
				\4 Corto plazo
				\4[] Se añaden retardos de demanda
				\4[] $\to$ Caracterizar ajuste lento
			\3 Evolución temporal
				\4 Pre-1974
				\4[] Regresiones econométricas con
				\4[] $\to$ Interés
				\4[] $\to$ Renta
				\4[] $\to$ Retardos recientes
				\4[] $\then$ Buena predicción
				\4[] $\then$ Posible extrapolación
				\4 A partir de 1974
				\4[] Errores cada vez mayores
				\4[] $\to$ Con coeficientes pre-1974
				\4[] $\to$ Con nuevos coeficientes estimados
				\4[] $\then$ Estimación sistemáticamente inferior
				\4[] $\then$ Menos demanda que esperada
				\4[] $\then$ Inestabilidad
				\4 Post-1981
				\4[] Caída brusca en velocidad de dinero
				\4[] $\to$ Aumenta M1 fuertemente
				\4[] Cambios en regulación financiera
				\4[] Innovaciones financieras
				\4[] $\to$ M1 menos aceptado como relevante
				\4 Inestabilidad a partir de 80s
				\4[] Reducción drástica de saldos monetarios
				\4[] Ausencia de explicaciones
				\4[] Comportamiento errático de la demanda
				\4[] $\then$ Abandono en 80s de M como instrumento
				\4[] $\then$ Interés como instrumento de PMonetaria
				\4 Money Zero Maturity (MZM)
				\4[] Agregado construido por FRBSaint Louis
				\4[] Balances disponibles inmediatamente
				\4[] $\to$ Para transferencias a coste nulo
	\1[] \marcar{Conclusión}
		\2 Recapitulación
			\3 Análisis agregado
			\3 Microfundamentación
			\3 Evidencia empírica
		\2 Idea final
			\3 Insuficiente conocimiento
				\4 Consenso en ámbito de investigación
				\4 Dda. monetaria poco comprendida
				\4 Sujeta a:
				\4[] Innovaciones financieras
				\4[] Feedback con política monetaria
				\4 Episodios de inestabilidad
				\4 Debate sobre agregado relevante
				\4[] ¿M1? ¿M2? ¿Otros?
			\3 Política monetaria actual
				\4 Interés e inflación objetivos principales
				\4 Demanda de dinero menos relevante
				\4 Relación inflación--oferta monetaria
				\4[] Renuncia generalizada a relacionar
				\4[] $\to$ Inflación y dinero
				\4[] Quantitative easing
			\3 Relación con otras áreas
				\4 Debate sobre dinero físico
				\4[] ¿Debe seguir existiendo?
				\4 Criptomonedas
				\4[] ¿Son dinero o sólo activos financieros?
				\4[] ¿Pueden sustituir al dinero?
				\4[] ¿Cómo afecta a dda. de dinero?
				\4 Política monetaria
				\4[] ¿Qué instrumentos son relevantes?
				\4[] ¿Agregados monetarios son relevantes?
				\4 Desarrollo económico
				\4[] ¿Es señoreaje útil?
				\4[] ¿Es señoreaje perjudicial?
				\4[] ¿Cuánto?
				\4 Economía internacional
				\4[] ¿Existe el privilegio exorbitante?
				\4[] ¿Qué efecto de PolMon sobre flujos de K?
				\4[] ¿Qué efecto tienen fluctuaciones de TC?
				\4[...]
\end{esquemal}





































\conceptos

\concepto{Demanda de dinero de Cagan}

En Cagan (1956), trata de caracterizar las condiciones necesarias para que aparezca hiperinflación aunque los fundamentales (déficit, tasa de crecimiento de la oferta de dinero) no experimenten un crecimiento explosivo. Para ello, el autor asume la hipótesis de expectativas adaptativas y una función ad-hoc de demanda de dinero con la forma $m = p - \beta \pi$

\concepto{Regla de Friedman}

La regla de Friedman se basa en el supuesto de que el coste marginal de producción del dinero es 0. En el óptimo de Pareto, el coste social de crear dinero adicional debe igualar el coste de mantener saldos monetarios por parte de los agentes privados. Dado que se asume un coste marginal nulo en la producción de dinero, el coste de oportunidad privado de mantener dinero debe ser nulo, luego el tipo de interés nominal debe también ser 0. Asumiendo que se cumple la ecuación de Fisher $i=r+\pi$, la inflación habrá de ser igual al tipo de interés real con signo contrario. La regla de Friedman es robusta a un gran número de modelos y supuestos. La idea clave detrás del concepto es el hecho de que si tratamos de reducir las ineficiencias por la existencia de dinero, es necesario que mantener dinero no suponga un coste para los agentes y ello se consigue eliminando el coste de oportunidad por su tenencia.

\preguntas

\seccion{Test 2018}
\textbf{18.} De acuerdo con Milton Friedman, la demanda de dinero dependerá:

\begin{itemize}
	\item[a] Negativamente de la renta permanente.
	\item[b] Positivamente de la rentabilidad de otros activos.
	\item[c] Negativamente de la inflación esperada.
	\item[d] Todas las opciones anteriores son correctas.
\end{itemize}

\seccion{Test 2015}

\textbf{18.} Suponga una economía cerrada con una única moneda de curso legal. Aunque en este marco de análisis es común interpretar el dinero fiduciario como un activo no rentable en términos nominales, su tasa de retorno real no tiene por qué ser nula. Si se denotase por $\pi_t$ la tasa neta de inflación entre $t$ y $t+1$, la expresión exacta de la tasa neta de retorno real del dinero a lo largo de dicho período sería:

\begin{itemize}
	\item[a] $-\pi_t$
	\item[b] $(1+\pi_t)^{-1}$
	\item[c] $-\pi_t(1+\pi_t)^{-1}$
	\item[d] Ninguna de las respuestas anteriores es correcta.
\end{itemize}

\textbf{19.} Bajo los supuestos de previsión perfecta e inexistencia de rigideces, el banco central de una economía anuncia en el periodo $t$ un incremento de una vez para siempre en la cantidad de dinero del período $t+1$. Señale la respuesta correcta:

\begin{itemize}
	\item[a] En el periodo $t$ se elevará tanto el nivel general de precios como el tipo de interés nominal.
	\item[b] Llegado el momento, el nivel de precios del periodo $t+1$ se revisará al alza, mientras que la tasa de inflación correspondiente al periodo $t$ se reducirá.
	\item[c] Puesto que tanto el nivel de precios en $t$ como en $t+1$ se elevarán, la tasa de inflación correspondiente al periodo $t$ no sufrirá alteración alguna. 
	\item[d] Ninguna de las opciones anteriores.
\end{itemize}

\seccion{Test 2009}

\textbf{18.} Considere una versión del modelo de demanda de dinero especificado por Cagan (1956). Bajo este modelo sabemos que la determinación de los precios viene dada por la expresión

$P_t = \frac{M_t}{\alpha_1} + b( _t P_{t+1} ), \, 0 < b < 1$, siendo $(_t P_{t+1})$ el valor esperado en $t$ de $P_{t+1}$.

Suponga que los agentes forman expectativas de modo racional y que los agentes creen en las políticas que pone en marcha la autoridad monetaria. Suponga que la autoridad monetaria anuncia la siguiente política monetaria $M_{t+1} = \bar{M} + \epsilon_{t+1} + \theta \epsilon_t$, $\forall \, t$, donde $E(\epsilon_t)=0$, $\forall \, t$, $E_t(\epsilon_{t+j}) = 0$, $\forall \, j > 0$. Entonces,

\begin{itemize}
	\item[a] los precios de equilibrio serán $P_t = \frac{1}{\alpha_1} \left[ \frac{\bar{M}}{1-b} + (1+b\theta) \epsilon_t + \theta \epsilon_{t-1} \right]$
	\item[b] si $\theta = 0$, la inflación en $t+1$ es $\pi_{t+1} = \frac{\epsilon_{t+1}}{ \frac{\bar{M}}{1-b} + \epsilon_t}$
	\item[c] si $\theta = 0$, la inflación $t+1$ es: $\pi_{t+1} = \frac{-\epsilon_t}{\frac{\bar{M}}{1-b} + \epsilon_t} $
	\item[d] los precios de equilibrio serán $P_t = \frac{1}{\alpha_1} \left[  \frac{\bar{M}}{1-b} + \epsilon_t + \theta \epsilon_{t-1} \right]$
\end{itemize}

\seccion{Test 2007}

\textbf{23.} Si el modelo de demanda de dinero de Baumol-Tobin es correcto puede concluirse que, a medida que aumenta el tipo de interés:

\begin{itemize}
	\item[a] La velocidad de circulación del dinero permanecerá constante.
	\item[b] La velocidad de circulación del dinero disminuirá.
	\item[c] La velocidad de circulación del dinero aumentará.
	\item[d] No puede determinarse que sucederá con la velocidad de circulación del dinero.
\end{itemize}

\notas

\textbf{2018}: \textbf{18.} C

\textbf{2015:} \textbf{18.} C. La inflación entendida como aumento porcentual del nivel de precios $\pi_t$ entre $t$ y $t+1$ implica que los valores de los saldos monetarios reales evolucionan de forma que: (1) $m_{t+1} = m_t \cdot \frac{1}{1+\pi_t}$. Reordenando, tenemos: (2) $m_{t+1} - m_t = -\pi_t m_{t+1}$. La tasa de retorno real del dinero entre $t$ y $t+1$ corresponde a la variación porcentual del saldo monetario real: $\frac{m_{t+1}-m_t}{m_t}$. Dividiendo la expresión (2) por $m_t$ tenemos que el retorno real del dinero es: (3) $\frac{m_{t+1} - m_t}{m_t} = - \frac{\pi_t}{(1+\pi_t)}$.  \textbf{19.} A

\textbf{2009:} \textbf{18.} A

\textbf{2007:} \textbf{23.} C

\bibliografia

Mirar en Palgrave:
\begin{itemize}
	\item capital, credit and money markets
	\item cheap money *
	\item circular flow
	\item commodity money
	\item dear money
	\item demand for money: empirical studies *
	\item demand for money: theoretical studies *
	\item endogenous and exogenous money *
	\item fiat money
	\item financial intermediaries *
	\item financial intermediation *
	\item Friedman, Milton (1912-2006)
	\item gold standard
	\item high-powered money and the monetary base *
	\item hot money
	\item inside and outside money *
	\item liquidity preference
	\item marginal utility of money *
	\item monetarism
	\item monetary and fiscal policy overview
	\item monetary policy, history of
	\item monetary transmission mechanism
	\item money * 
	\item money and general equilibrium *
	\item money and general equilibrium theory *
	\item money illusion *
	\item money in economic activity *
	\item money supply
	\item money, classical theory of *
	\item moneylenders
	\item neutrality of money
	\item optimum quantity of money *
	\item quantity theory of money *
	\item real balances
	\item sound money *
	\item tight money *
	\item wages, real and money
\end{itemize}

Ellison, M. \textit{Monetary Economics - Ch. 4 Cash in advance model}  Notas de clase. \url{http://users.ox.ac.uk/~exet2581/msc/ec924cia.pdf} -- En carpeta el tema 

Enzler, J. Johnson, L. Paulus, J. \textit{Some Problems of Money Demand} (1976) 

Heijdra, B. J. \textit{Foundations of Modern Macroeconomics} (2017) 3rd ed. -- En carpeta Macro

Laidler, D.; Roberts, Russ. \textit{David Laidler} (2013) Econtalk podcast -- \url{http://www.econtalk.org/david-laidler-on-money/}

McCallum, B. T. \textit{The role of overlapping-generations models in monetary economics} (1982) NBER Working Paper no. 989

McCallum, B. T. \textit{Monetary Economics} (1989) -- En carpeta macro

Poole, W. \textit{Monetary Policy Lessons of Recent Inflation and Disinflation} (1988) Journal of Economic Perspectives -- En carpeta del tema

Romer, D. \text{Advanced Macroeconomics}. (2012) 4ed. -- En carpeta macro

Samuelson, P. A. \textit{An Exact Consumption-Loan Model of Interest with or without the Social Contrivance of Money} (1958) Journal of Political Economy -- En carpeta del tema

Serletis, A. \textit{The Demand for Money} (2007) -- En carpeta del tema

Walsh, C. E. \textit{Monetary Theory and Policy} (2010) 3a edición, (2017) 4a edición -- En carpeta macro

Varios autores. \textit{Handbook in Monetary Economics} (2011) Vol. III -- En carpeta macro


\end{document}
