\documentclass{nuevotema}

\tema{3A-42}
\titulo{Teorías del crecimiento económico (I). El modelo de Harrod-Domar. El modelo de Solow. El modelo de Ramsey-Cass-Koopmans. Sus implicaciones y limitaciones.}

\begin{document}

\ideaclave

Uno de los objetivos de la macroeconomía es entender el crecimiento del producto a largo plazo. Es decir, entender qué factores determinan el nivel de producción en el largo plazo. Aunque la distinción entre largo plazo y corto plazo es relativamente arbitraria, es habitual entenderlo como el horizonte temporal en el que todos los factores de producción pueden ajustar sus valores. Robert Lucas escribió que \comillas{una vez que uno comienza a pensar en el crecimiento económico, es difícil pensar en otra cosa}. La frase se fundamenta en el hecho de que cambios presentes en los factores que determinan este crecimiento tienen enormes efectos futuros por el hecho de que sus consecuencias se van acumulando entre periodos.

Para explicar el fenómeno del crecimiento y sus determinantes existen diferentes familias de teorías. En todas ellas, sin embargo, el crecimiento es resultado de la acumulación de factores productivos. Las dinámicas de acumulación de esos factores, y la consideración de unos u otros conceptos como factores de producción definen en gran medida las diferencias entre unas teorías u otras. El presente tema tiene por objeto exponer tres teorías que se caracterizan por no contemplar el crecimiento del producto per cápita en estado estacionario como un proceso endógeno. Así, el crecimiento per cápita en estado estacionario se modeliza como un proceso exógeno sobre el que los agentes no tienen poder decisión. Partiendo de este rasgo común a los tres modelos presentados en este tema, cabe establecer una diferencia adicional. En los dos primeros modelos, de Harrod-Domar y de Solow-Swann no existe optimización, sino tan sólo un proceso dinámico cuya evolución a largo plazo se describe mediante una ecuación diferencial. En estos dos modelos, el estado estacionario no es el resultado de ninguna decisión optimizadora, sino simplemente de una tasa de ahorro determinada exógena y arbitrariamente. 

El modelo de Ramsey-Cass-Koopmans endogeneiza esa tasa de ahorro de tal manera que es en éste último el resultado de un proceso de optimización con el consumo como variable de decisión. Así, el ahorro en el modelo RCK es aquel que optimiza la utilidad intertemporal, que no es sino una suma o integral de utilidades en cada periodo. Las conclusiones del modelo RCK son similares a las del modelo de Solow en cuanto a la dinámica de la economía, pero muestran algunas diferencias importantes. Por ejemplo, en el modelo RCK el consumo de estado estacionario es inferior al que corresponde en el modelo de Solow al nivel de capital de la \comillas{regla de oro}. 

\seccion{Preguntas clave}

\begin{itemize}
	\item ¿Qué tratan de entender y predecir los modelos de crecimiento ecónomico?
	\item ¿Qué caracteriza a los modelos de crecimiento exógeno?
	\item ¿Cuáles son las implicaciones del modelo de Harrod-Domar, de Solow y de Ramsey-Cass-Koopmans?
	\item ¿Cuáles son sus limitaciones a la hora de explicar el crecimiento?
\end{itemize}

\begin{itemize}
    \item ¿Qué es el crecimiento económico?
    \item ¿De qué depende?
    \item ¿Qué dinámica sigue a largo plazo?
    \item ¿Qué modelos teóricos tratan de caracterizarlo?
    \item ¿Qué factores tienen en cuenta?
    \item ¿Qué conclusiones se extraen de ellos?
    \item ¿Cómo se explican las diferencias de renta per cápita entre países?
\end{itemize}

\esquemacorto

\begin{esquema}[enumerate]
	\1[] \marcar{Introducción} 2'-2'
		\2 Contextualización
			\3 Evolución histórica de la renta per cápita
			\3 Causas próximas y fundamentales
			\3 Enfoques de estudio
		\2 Objeto
			\3 Cómo afecta la acumulación de ff.pp al crecimiento
			\3 Qué PIB per cápita se alcanza a l/p
			\3 Qué modelos neoclásicos relevantes
			\3 Qué implicaciones de política económica
		\2 Estructura
			\3 Harrod-Domar
			\3 Solow-Swan
			\3 Ramsey-Cass-Koopmans
	\1 \marcar{Modelo de Harrod-Domar}
		\2 Idea clave
			\3 Contexto
			\3 Objetivo
			\3 Resultados
		\2 Formulación (Barro y Sala-i-Martín)
			\3 Supuestos
			\3 Producción per cápita
			\3 Dinámica
			\3 Paro
			\3 Sobreinversión
			\3 Utilización plena de factores
		\2 Implicaciones
			\3 Crecimiento equilibrado
			\3 Intervención pública
		\2 Formulación original
			\3 Proporción óptima capital-trabajo
			\3 Crecimiento del output
			\3 Crecimiento con intensidad de capital constante
			\3 Crecimiento del empleo
			\3 Equilibrio de largo plazo
			\3 Factibilidad del crecimiento de largo plazo
			\3 Inestabilidad del equilibrio de largo plazo
		\2 Valoración
			\3 Supuestos inasumibles
			\3 Actualidad
	\1 \marcar{Modelo de Solow-Swann 12'-18'}
		\2 Idea clave
			\3 Contexto
			\3 Objetivo
			\3 Resultados
		\2 Formulación
			\3 Supuestos
			\3 Dinámica
		\2 Implicaciones
			\3 Aumento de tasa de ahorro
			\3 Convergencia
			\3 Diferencias PIBpc
		\2 Extensiones
			\3 Trampas de pobreza
			\3 Crecimiento de la natalidad
			\3 Recursos naturales
			\3 Modelo de Solow con recurso natural constante
		\2 Valoración
			\3 Análisis mecánico
			\3 Acumulación de inputs
			\3 Progreso tecnológico
			\3 Nuevo marco de análisis
			\3 Bienestar
	\1 \marcar{Modelo de Ramsey-Cass-Koopmans 10-28'}
		\2 Idea clave
			\3 Contexto
			\3 Caracterizar evolución de:
			\3 A partir de una regla de decisión
			\3 Resultado
		\2 Formulación (planificador)
			\3 Programa de maximización
			\3 Sujeto a:
			\3 Donde:
			\3 Dinámica del óptimo
			\3 Estado estacionario
		\2 Implicaciones
			\3 Consumo
			\3 Bienestar
			\3 Estática comparativa
		\2 Valoración
			\3 Comparación con modelo de Solow
			\3 Limitaciones
			\3 Modelos relacionados
	\1[] \marcar{Conclusión 2'-30'}
		\2 Recapitulación
			\3 Modelos crecimiento exógeno
			\3 Harrod-Domar
			\3 Solow
			\3 Ramsey-Cass-Koopmans
		\2 Idea final
			\3 Diferencias de riqueza
			\3 Consecuencias futuras del presente
			\3 Convergencia entre países

\end{esquema}

\esquemalargo

\begin{esquemal}
	\1[] \marcar{Introducción} 2'-2'
		\2 Contextualización
			\3 Evolución histórica de la renta per cápita
				\4 A lo largo de historia humana
				\4[] PIBpc prácticamente estable
				\4[] Muy similar en todo el mundo
				\4 Divergencia global
				\4[] A partir del año 1000 d.C
				\4[] $\to$ Según algunos autores
				\4[] A partir de 1800 d.C.
				\4[] $\to$ Según toda la literatura
				\4[] Europa occidental + satélites
				\4[] $\to$ Comienzan a divergir
				\4[] $\then$ Crecimiento económico sostenido
				\4[] $\then$ Diferencias de renta actuales
			\3 Causas próximas y fundamentales
				\4 Causas próximas:
				\4[] Factores con influencia directa en crecimiento
				\4[] $\to$ Acumulación de capital físico
				\4[] $\to$ Crecimiento demográfico
				\4[] $\to$ Avances tecnológicos
				\4 Causas fundamentales:
				\4[] Causan causas próximas
				\4[] $\to$ Instituciones
				\4[] $\to$ Preferencias
				\4[] $\to$ Cultura
				\4[] $\to$ Geografía física
				\4[] $\to$ Azar
				\4 Objetivo de teoría del crecimiento
				\4[] Entender relación entre
				\4[] $\to$ Crecimiento y causas próximas
				\4[] $\to$ Causas próximas y fundamentales
				\4[] $\then$ ¿Por qué unos países crecen y otros no?
				\4 Robert Lucas
				\4[] ``Cuando uno estudia el crec. económico
				\4[] ...es difícil pensar en otra cosa.'''
			\3 Enfoques de estudio
				\4 Dos grandes familias
				\4[] Crecimiento pc exógeno a l/p
				\4[] $\to$ Caracterización de estados estacionarios
				\4[] $\to$ Tecnología es proceso no económico
				\4[] Crecimiento endógeno
				\4[] $\to$ Avances tecnológico es proceso económico
				\4[] $\to$ Crecimiento a l/p es endógeno al modelo
		\2 Objeto
			\3 Cómo afecta la acumulación de ff.pp al crecimiento
			\3 Qué PIB per cápita se alcanza a l/p
			\3 Qué modelos neoclásicos relevantes
			\3 Qué implicaciones de política económica
		\2 Estructura
			\3 Harrod-Domar
			\3 Solow-Swan
			\3 Ramsey-Cass-Koopmans
	\1 \marcar{Modelo de Harrod-Domar}\footnote{Págs. 71 a 74 de Sala-i-Martín.} 4-6'
		\2 Idea clave
			\3 Contexto
				\4 Harrod (1939), Domar (1946) y (1947)
				\4 Teoría general de Keynes
				\4[] Desempleo puede ser persistente
				\4[] Macroeconomías no son necesariamente estables
				\4[] $\to$ No caracteriza condiciones de pleno empleo en l/p
				\4[] $\to$ Se centra en inversión para generar ingreso
				\4 Controversia sobre inestabilidad de economía
				\4[] Capitalismo induce desempleo generalizado?
				\4[] Economías capitalistas pueden utilizar bien los recursos?
			\3 Objetivo
				\4 Caracterizar condiciones de pleno empleo en l/p
				\4 Considerar posibilidad de evolución inestable
				\4 Efecto de inversión sobre capacidad productiva
			\3 Resultados
				\4 Tres estados estacionarios alternativos
				\4[] Paro creciente
				\4[] Sobreinversión creciente
				\4[] Pleno empleo sin inflación
				\4 Equilibrios ineficientes
				\4[] Si tasa de ahorro no toma determinado valor
				\4 Requisitos de pleno empleo de largo plazo
				\4[] i. Inversión de pleno empleo cada periodo
				\4[] ii. Crecimiento de output y población iguales
		\2 Formulación (Barro y Sala-i-Martín)
			\3 Supuestos
				\4 $F(K,L)=\min \, \{AK, BL \}$\footnote{Denominada habitualmente de \comillas{coeficientes fijos} o de Leontieff.}
				\4 Depreciación del capital
				\4[] $\delta \cdot k$
				\4 Crecimiento de la población
				\4 $L(t) = L(0) \cdot e^{nt}$
			\3 Producción per cápita
				\4 $\frac{1}{L}F(K,L) = F(\frac{K}{L}, 1) = f(k) = \min \{ Ak, B\}$
				\4 $f'(k) > 0 \qquad\text{si} \; k < B/A$
				\4 $f'(k) = 0 \qquad\text{si} \; k \geq B/A$
				\4 Gráfica I: $f(k) - k$
				\4[] \grafica{produccionhd}
			\3 Dinámica
				\4 Variación del stock de capital
				\4[] $\dot{k}/{k} = s \frac{\min \{ Ak, B \} }{k} - (n+\delta)$
				\4 Representación gráfica
				\4[] \grafica{equilibriohd}
			\3 Paro
				\4 $k$ tiende asintóticamente a 0
				\4 Cada vez menos capital por trabajador
			\3 Sobreinversión
				\4 Exceso de $k$ constante\footnote{Es decir, hay un exceso de capital disponible por cada unida de trabajo disponible. Dado que el trabajo disponible crece a tasa constante, el exceso de trabajo total crece también a tasa constante y por tanto cada vez hay más capital ocioso.}
				\4 Oferta de trabajo crece
				\4[] $\rightarrow$ Exceso de capital total tiende a $\infty$
			\3 Utilización plena de factores
				\4 Si $k^* = B/A$
				\4[] $\rightarrow sA = n + \delta$
				\4 Improbable: s, A, n y $\delta$ son exógenos
		\2 Implicaciones
			\3 Crecimiento equilibrado
				\4 Prácticamente inalcanzable
				\4 Cualquier perturbación desvía
			\3 Intervención pública
				\4 Necesaria para evitar crisis
				\4 Sistema capitalista inestable
		\2 Formulación original
			\3 Proporción óptima capital-trabajo
				\4 Empresarios desean mantener $\frac{K}{Y}=v$
				\4[] Donde $v$ es una constante que induce óptimo
				\4 Para mantener $\frac{K}{Y}=v$
				\4[] Invierten en proporción $v$ a crecimiento
				\4[] $I = \dv{K}{t} = \dv{Y}{t} \cdot v$
				\4[] No conocen $\dv{Y}{t}$ con certeza
				\4[] $\to$ Pero lo estiman a partir de multiplicador
			\3 Crecimiento del output
				\4 Economía keynesiana con multiplicador
				\4 Crecimiento respecto a periodo anterior
				\4[] Aumento de inversión autónoma por multiplicador
				\4[] $\dv{Y}{t} = \dfrac{d \, I}{s}$
			\3 Crecimiento con intensidad de capital constante
				\4 Empresarios invierten basándose en $\dv{Y}{t} = \dfrac{d \, I}{s}$
				\4[] $\then$ $I = \dfrac{d \, I}{s} \cdot v$ $\then$ \fbox{$\frac{d \, I}{I} = \frac{s}{v}$}
				\4[] Capital crece a misma tasa que crece inversión
				\4[] $\then$ $\frac{d \, I}{I} = \frac{d \, K}{K} = \frac{s}{v}$
			\3 Crecimiento del empleo
				\4 Oferta de trabajo crece a tasa exógena $n$
				\4[] $\frac{d \, L}{L} = n$
			\3 Equilibrio de largo plazo
				\4 Requiere mantener constantes:
				\4[] $\to$ Intensidad de capital en producción $\frac{K}{Y}$
				\4[] $\then$ Producción crece a misma tasa que capital
				\4[] $\then$ $\frac{dY}{Y} = \frac{dI}{I} = \frac{dK}{K} = \frac{s}{v}$
				\4[] $\to$ Relación capital-trabajo $\frac{K}{L}$
				\4[] $\then$ Capital y trabajo deben crecer a misma \%
				\4[] $\then$ $\frac{dI}{I} = \frac{s}{v} = \frac{dL}{L}$
				\4[] $\to$ Intensidad de trabajo en producción $\frac{L}{Y}$
				\4[] $\then$ Output y trabajo deben crecer a misma tasa
				\4[] $\then$ $\frac{dY}{Y} = \frac{dL}{L}$
				\4[$\then$] $\frac{d I}{I} = \frac{d Y}{Y} = \frac{d L}{L}$
			\3 Factibilidad del crecimiento de largo plazo
				\4 Crecimiento de población es variable exógena
				\4[$\then$] Inversión debería ajustarse a
			\3 Inestabilidad del equilibrio de largo plazo
				\4 Perturbaciones inducen trayectorias explosivas
				\4 Aumento inesperado del crecimiento del output
				\4[] Cae proporción $\frac{K}{Y}$ por debajo de $v$ deseada
				\4[] $\to$ Aumentan inversión para corregir $\frac{K}{Y}$
				\4[] $\then$ Output vuelve a crecer más rápido y $\frac{K}{Y} < v$
				\4[] $\then$ Crecimiento explosivo del capital y del output
				\4[] $\then$ Más crecimiento de output que de L induce inflación
				\4 Caída inesperada del crecimiento del output
				\4[] Aumenta proporción $\frac{K}{Y}$ por encima de $v$ deseada
				\4[] $\to$ Reducen inversión para corregir $\frac{K}{Y}$
				\4[] $\then$ Output vuelve a caer por menor inversión y $\frac{K}{Y} > y$
				\4[] $\then$ Tendencia hacia depresión
				\4[] $\then$ Más crecimiento de L que de Y induce desempleo
		\2 Valoración
			\3 Supuestos inasumibles\footnote{Según Sala-i-Martín.}
				\4 Ratio óptimo constante $\frac{K}{Y}$
				\4[] Implica función de proporciones constantes/Leontieff
				\4 Producto marginal del capital A
				\4[] Fijo, no depende de k
				\4[] No se ajusta para que: $s \cdot f(k)/k = n + \delta$
				\4 Ahorro fijo
				\4[] No se ajusta para evitar capital inactivo
				\4 Multiplicador constante
				\4 Oferta de trabajo exógena
			\3 Actualidad
				\4 Modelo relativamente superado
				\4 Influyente en neokeynesianismo
				\4[] Énfasis sobre posible inestabilidad
				\4 Crítica a sistema capitalista
				\4 Base para modelo de Solow\footnote{Robert Solow Nobel Prize Lecture (1987).}
				\4[] Por ello, enorme relevancia
	\1 \marcar{Modelo de Solow-Swann 12'-18'}
		\2 Idea clave
			\3 Contexto
				\4 Basado en Harrod-Domar
				\4[] Sin f. de prod. coeficientes fijos\footnote{\comillas{The bulk of this paper is devoted to a model of long-run growth which accepts all the Harrod-Domar assumptions except that of fixed proportions. Instead I suppose that the single composite commodity is produced by labor and capital under the standard neoclassical conditions. [...]} (Solow, 1956)}
				\4 Función de producción
				\4[] Cumple condiciones de Inada
				\4[] $\to$ $\lim_{x_i \to 0} F_i(\vec{x}) \to \infty$
				\4[] $\to$ $\lim_{x_i \to \infty} F_i(\vec{x}) = 0$
			\3 Objetivo
				\4 Caracterizar equilibrio de largo plazo
				\4 Evitar supuestos inasumibles de Harrod-Domar
				\4[] Ahorro exógeno y estable
				\4[] Función de proporciones fijas
				\4 Identificar contribución de factores al crecimiento
				\4[] Capital, trabajo, tecnología
			\3 Resultados
				\4 Estado estacionario
				\4[] Único
				\4[] Estable
				\4 Crecimiento p/c largo plazo
				\4[] Proceso exógeno es causante
		\2 Formulación
			\3 Supuestos
				\4 Función de producción\footnote{Si la función es Cobb-Douglas, $F(K, AL)$ es equivalente a $F(AK, L)$ porque $K^\alpha (AL)^{1-\alpha} = (\tilde{A} K)^\alpha L^{1-\alpha}$.}: $F(K, AL)$  h.d.g. 1
				\4[] Cobb-Douglas:
				\4[] Elasticidad de sustitución constante = 1
				\4[] Homogénea de grado 1 $\then$ Rdtos. constantes a escala
				\4[] Productividad marginal decreciente
				\4[] Cumple condiciones de Inada
				\4[] $\to$ $F(K,AL) = A(t) K(t)^\alpha L(t)^{1-\alpha}$
				\4[] $\frac{1}{AL}F(K, Al) = F(\frac{K}{AL},1) = f(k)$
				\4[] $f'(k) > 0, \; f''(k)<0$
				\4[] Condiciones de Inada en $\frac{df(k)}{dk}$
				\4 Crecimiento del trabajo
				\4[] $L(t) = L(0)\cdot e^{nt} \rightarrow \frac{\dot{L}}{L} = n$
				\4 Efectividad del trabajo
				\4[] $A(t) = A(0)\cdot e^{gt} \rightarrow \frac{\dot{A}}{A} = g$
			\3 Dinámica
				\4 $\dot{K} = S\cdot F(K, AL) - \delta K$
				\4 \fbox{$\dot{k} = \frac{\dot{K}}{AL} = sf(k) - (n+g+\delta)k$}
				\4 Estado estacionario
				\4[] $sf(k^*) = (n+g+\delta)k^*$
				\4[] $\to$ \fbox{$k^* = \left( \frac{s}{n+g+\delta} \right)^{1/(1-\alpha)}$}
				\4[] Relación entre ahorro, depreciación y capital
				\4[] \grafica{solowahorrodepreciacioncapital}
				\4[] Relación entre crecimiento de capital y capital
				\4[] \grafica{solowdinamicacapital}
		\2 Implicaciones
			\3 Aumento de tasa de ahorro
				\4 \underline{Producción}
				\4[] Aumento de $\dot{k}$
				\4[] Aumento de $k^*$
				\4[] Aumento de producción
				\4[] Más peso del capital en el producto ($\alpha$)
				\4[] $\to$ Más productividad marginal del capital dado $k$
				\4[] $\then$ Más impacto $\Delta$ ahorro
				\4 \underline{Consumo}
				\4[] Consumo EE:
				\4[] $c^* = f(k^*) - sf(k^*)=f(k^*)-(n+g+\delta)k^*$
				\4[] Capital que induce consumo óptimo:
				\4[] $\underset{k_{GR}}{\max} \quad f(k^*) - sf(k^*)$
				\4[] $\then$ CPO: \quad $f'(k_{GR}) = (n+g+\delta)$
				\4[] Ahorro que induce consumo óptimo
				\4[] $\underset{s}{\max} \quad y^* - s y^*$
				\4[] Dada función Cobb-Douglas $F(K,AL) = K^\alpha (AL)^{1-\alpha}$
				\4[] $\to$ $k^* = \left( \frac{s}{n+g+\delta} \right)^{1/(1-\alpha)}$
				\4[] $\to$ $y^* = \left( \frac{s}{n+g+\delta} \right)^{\alpha/(1-\alpha)}$
				\4[] $\underset{s}{\max} \quad \left( \frac{s}{n+g+\delta} \right)^{\alpha/(1-\alpha)} - s \left( \frac{s}{n+g+\delta} \right)^{\alpha/(1-\alpha)}$
				\4[] $\then$ $s_{GR} = \alpha$
				\4[] Dado $s_{GR}$ que maximiza consumo
				\4[] $\to$ Si $s<s_{GR}$ impacto positivo sobre consumo
				\4[] $\to$ Si $s>s_{GR}$ impacto negativo
				\4 {Gráficos}\footnote{Página 20 (Romer).}
				\4[] Aumenta $\dot{k}$ hasta nuevo estado estacionario
				\4[] \grafica{solownuevoahorrocapital}
				\4[] Gráfico $s$--$t$
				\4[] \grafica{solowahorrotiempo}
				\4[] Gráfico $\dot{k}$--$t$
				\4[] \grafica{solowcrecimientocapitaltiempo}
				\4[] Gráfico $k$--$t$
				\4[] \grafica{solowcapitaltiempo}
				\4[] Gráfico Crecimiento \% de Y/L--$t$
				\4[] \grafica{solowcrecimientoproductividadtiempo}
				\4[] Gráfico $\ln(Y/L)$--$t$
				\4[] \grafica{solowlogaritmoproductividadtiempo}
				\4[] Gráfico $c$--$t$
				\4[] \grafica{solowconsumotiempo}
			\3 Convergencia
				\4 Crecimiento del producto
				\4[] Depende de crecimiento de $k$
				\4[] Crecimiento de $k$ es más alto cuanto más lejos de $k^*$ de EE
				\4[] $\then$ $\Delta$ de $y$ más alto cuanto más lejos de EE
				\4[] $\then$ Implica beta-convergencia condicional
				\4[] $\then$ Implica sigma-convergencia condicional
				\4 Velocidad de convergencia
				\4[] ¿Cuánto cae la tasa de crecimiento de $k$\ldots
				\4[] \ldots Ante aumento porcentual de $k$?
				\4[] $\beta \equiv \dv{\left( \dot{k}/k \right)}{\ln k} = \dv{\left( \dot{y}/y \right) }{\ln y} = -(1-\alpha) \cdot (\delta + n + g)$
				\4 {Aproximación mediante expansión de Taylor}
				\4[] Más rápido cuanto mayor $(n+g+\delta)$
				\4[] Más rápido cuanto menor $\alpha_K$\footnote{Contribución del capital a la producción (share of capital).}
				\4 {Ricos y pobres}
				\4[] Mismos parámetros $\rightarrow$ pobre crece más que rico
				\4[] Mayor retorno de k en países pobres
				\4[] Incentivos capital fluya de ricos a pobres
				\4 Estudios empíricos
				\4[] Entendiendo K como capital físico
				\4[] Tomando valores de $\alpha$ habituales
				\4[] $\to$ Predice velocidad de convergencia excesiva
				\4 Capital humano como solución
				\4[] Mankiw, Romer y Weil (1992)
				\4[] $\to$ Tener en cuenta retornos al capital humano
				\4[] $Y=A(t)K^\alpha H^\eta L^{1-\alpha - \eta}$
				\4[] Coeficiente de convergencia similar:
				\4[] $\beta^* = (1-\alpha-\eta)\cdot(n + g + \delta)$
				\4[] $\to$ $\eta$ reduce velocidad de convergencia
				\4[] $\to$ Capital amplio: 2/3 de la renta
				\4[] $\then$ Acorde con resultados empíricos
			\3 Diferencias PIBpc
				\4 Origenes temporales y entre países
				\4[I] Capital por trabajador
				\4[II] Efectividad del trabajo
				\4 Capital por trabajador
				\4[] Diferencias de $K/L$ no explican suficientemente
				\4 Efectividad del trabajo
				\4[] Recoge diferencias PIBpc no explicadas por $K/L$
				\4[] Modelizadas como proceso exógeno
				\4[] Determinantes no identificados
		\2 Extensiones
			\3 Trampas de pobreza
				\4 Posibles múltiples EEstacionarios
				\4[] Dadas funciones de producción irregulares
			\3 Crecimiento de la natalidad
				\4 No necesariamente endógeno
			\3 Recursos naturales
				\4 Nordhaus (Nobel 2018) y otros
			\3 Modelo de Solow con recurso natural constante\footnote{Ver \href{http://www.artsrn.ualberta.ca/econweb/hryshko/econ403fall2010/CHAPTER9.pdf}{Hryshko (2010)}.}
				\4 Idea clave
				\4[] Recursos naturales no pueden ser producidos
				\4[] Algunos son constantes
				\4[] $\to$ Especialmente, tierra
				\4[] Necesarios en casi todo proceso productivo
				\4[] $\to$ Pero imposibles de replicar
				\4[] $\then$ Inducen rendimientos decrecientes a escala
				\4 Formulación
				\4[] $Y = A K^\alpha T^\beta L^{1-\alpha-\beta}$
				\4[] $\frac{\dot{A}}{A} = g$
				\4[] $\frac{\dot{L}}{L} = n$
				\4[] Retornos decrecientes a escala
				\4[] $\to$ En trabajo y en capital
				\4 Implicaciones
				\4[] Importancia de recurso natural en producción
				\4[] $\to$ Reduce output progresivamente
				\4[] $\then$ Cuanto mayor $\beta$, mayor R$\downarrow$E
				\4[] Output per-cápita tiende a cero
				\4[] $\to$ A medida que se realizan R$\downarrow$E
				\4[] $\then$ Salvo que progreso tecnológico compense
				\4[] Necesario progreso tecnológico
				\4[] $\to$ Para compensar caída vía $R\downarrow E$
				\4 Valoración
				\4[] Peso de factores no renovables en actividad
				\4[] $\to$ Aumenta R$\downarrow$E
				\4[] Sustitución de actividades intensivas en no renovables
				\4[] $\to$ Permite mayor crecimiento en largo plazo
		\2 Valoración
			\3 Análisis mecánico
				\4 No fundamenta equilibrios en optimización
				\4 No son equilibrios competitivos
				\4 Simple descripción de una dinámica
			\3 Acumulación de inputs
				\4 Comparación de crecimiento por países
				\4[] Partiendo de modelo de Solow
				\4[] Acumulación de capital y trabajo no explica todo el crecimiento
				\4[] $\to$ Debe existir crecimiento tecnológico
			\3 Progreso tecnológico
				\4 Modelo de Solow no explica
				\4 Terreno abierto a modelos de crecimiento endógeno
				\4[] Capaces de explicar crecimiento p.c. en l/p
			\3 Nuevo marco de análisis
				\4 Contabilidad del crecimiento
				\4 Convergencia
				\4 Papel del ahorro
			\3 Bienestar
				\4 Sin optimización del consumidor
				\4[] $\to$ no hay análisis de bienestar
	\1 \marcar{Modelo de Ramsey-Cass-Koopmans 10-28'}
		\2 Idea clave
			\3 Contexto
				\4 Solow
				\4[] Evolución mecánica
				\4[] Supuestos behaviorales
				\4[] Sin basar en optimización
				\4 Ramsey (1926)
				\4[] Primer intento de caracterizar
				\4[] $\to$ Decisión intertemporal
				\4[] $\then$ Distribuir consumo intertemporalmente
				\4 Utilizar para problema de crecimiento
				\4[] Decisión entre consumo e inversión
				\4[] $\to$ ¿Senda de Solow fundamentación microeconómica?
			\3 Caracterizar evolución de:
				\4 Output
				\4 Consumo
				\4 Capital
				\4[$\to$] Ahorro
				\4[$\to$] Inversión
				\4[$\to$] Tipo de interés
			\3 A partir de una regla de decisión
				\4 Microfundamentada
				\4 Basada en comportamiento racional
				\4[] Un consumidor con horizonte vital infinito
				\4[] O infinitos consumidores que viven 1 periodo
				\4[] e internalizan utilidad de generaciones futuras
				\4[] $\to$ Oferta cantidad fija de trabajo
				\4[] $\to$ Maximiza output dado capital
				\4[$\Rightarrow$] Ag. representativo maximiza f. de u. intertemporal
			\3 Resultado
				\4 Ramsey (1926):
				\4[] Planificador social optimiza consumo
				\4[] Sin tasa de descuento de la utilidad
				\4 Posteriormente:
				\4[] Base de modelos DSGE
				\4[] $\to$ RBC es RCK con oferta de L endógena
				\4 Ausencia de externalidades
				\4[] Primer teorema del bienestar:
				\4[] Equilibrio competitivo es óptimo de Pareto
				\4[] Eq. del planificador es óptimo de Pareto
				\4[$\Rightarrow$] Eq. planificador y descentralizado son equivalentes
				\4 Tasa de ahorro resultante
				\4[] Determinada endógenamente
				\4[] $\to$ Resultado de programa de optimización intertemporal
				\4[] $\to$ Depende de decisión de consumo óptima
		\2 Formulación (planificador)
			\3 Programa de maximización
				\4[]\fbox{ $\underset{c_t}{\max} \quad U = \int_0^{\infty}  u\left( c_t \right) \cdot e^{-\rho t} \cdot e^{(n+g)t} dt$ }
			\3 Sujeto a:
				\4[] \fbox{ $\int_0^\infty c_t \cdot e^{-J_t} \cdot e^{(g+n)t} dt \leq k_0 + \int_0^\infty w_t \cdot e^{-J_t} \cdot e^{(g+n)t} dt $ }
				\4[] \fbox{ $\dot{k}_t = f(k_t) - c_t - (n+g)\cdot k_t$ }
				\4[] \fbox{ $\Lim{t \to \infty}  k_t \cdot e^{-J_t}e^{(n+g)t}  \geq 0$ }
				\4[] $\rho > (1-\theta)g + n$
			\3 Donde:
				\4[] $u(c_t) = \frac{c^{1-\theta} - 1}{1-\theta}$
				\4[] ESI: $\sigma = \frac{1}{\theta} = \frac{d \, \ln \frac{c_{t+1}}{c_t} }{ d \, r}:$
				\4[] $f(k_t) \equiv \frac{1}{A_t \cdot L_t} F (K_t, A_t \cdot L_t) = F \left( \frac{K_t}{A_t L_t}, 1 \right) $
				\4[] $i_t = f'(k_t)$
				\4[] $w_t = f(k_t) - k\cdot f'(k_t)$
				\4[] $J_t = \int_{0}^{t} i_s ds$
			\3 Dinámica del óptimo
				\4 Consumo
				\4[] \fbox{$\frac{\dot{c}_t}{c_t} = \frac{f'(k_t) - \rho - \theta g}{\theta}$}
				\4[] \grafica{consumorck}
				\4 Capital
				\4[] \fbox{$\dot{k}_t = f(k_t) - c_t - (n+g) k_t$}
				\4[] \grafica{capitalrck}
			\3 Estado estacionario
				\4 $\frac{\dot{c}_t}{c_t} = 0$
				\4[] $\then$ \fbox{$f'(k^*_t) = \rho + \theta g$}
				\4 $\dot{k}_t = 0$
				\4[] $\then$ \fbox{$c^* = f(k^*) - (n+g) k^*$}
				\4 Representación gráfica
				\4[] \grafica{faserck}
		\2 Implicaciones
			\3 Consumo
				\4 Consumo por unidad de trabajo efectivo $c_t$
				\4[] Crece a tasa constante = 0
				\4 Consumo per cápita
				\4[] Crece a tasa constante $g$
				\4 El consumo de EE es inferior al de regla de oro\footnote{Es decir, al consumo máximo alcanzable en EE.}
				\4[] En Solow, no hay maximización de utilidad
				\4[] Consumo de regla de oro maximiza consumo
				\4[] $\to$ Sin tener en cuenta optimalidad
				\4[] En RCK, utilidad de consumo se descuenta a $e^{-\rho}$
				\4[] $\to$ $c_t$ aporta más utilidad que $c_{t+1}$
				\4[] $\to$ Consumo futuro es menos importante
				\4[] $\to$ Se prefiere consumir más en presente
				\4[] $\then$ Ahorro óptimo menor que ahorro de GR
				\4[] $\then$ Se produce y se consume menos en futuro
				\4[] $k_{GR}$ es tal que $f'(k_{GR}(t)) = (n+g)$
				\4[] $k^*$ es tal que $f'(k^*) = (\rho + \theta g)$
				\4 $f(k_{GR}) < f(k^*) \Rightarrow k_{GR} > k^* \Rightarrow (n+g) < \rho +\theta g$
				\4 Supuesto inicial: $\rho > (1-\theta) g + n \iff n + g < \rho + \theta g$
				\4[$\then$] $k_{GR} > k^*$
				\4[] Dada forma de $\dot{k} = 0$
				\4[] $\to$ Mayor $k$ implica mayor $c$
				\4[] $\then$ $c_{GR} > c^*$
			\3 Bienestar
				\4 EE del planificador hallable como eq. competitivo
				\4[] Porque se cumple 2er Teorema Fundamental del Bienestar
				\4[] $\to$ Preferencias no saturables, convexas
				\4[] $\to$ Sin externalidades
				\4[] $\then$ Existe senda de precios+dotaciones que induce EE
				\4 Agentes idénticos $\to$ E senda óptima\footnote{Si la senda $c(t)$ que conduce a E es Pareto-eficiente, y todos los agentes que optimizan en la economía son idénticos, E no es sólo un equilibrio Pareto-eficiente, sino que es además el mejor equilibrio posible.}
			\3 Estática comparativa
				\4 Caída de la tasa de descuento
				\4[] $f'(k^*) = \rho + \theta g$
				\4[] $\downarrow \rho \to \downarrow f'(k^*) \to \uparrow k^*$
				\4[] $\then$ Curva $\dot{c} = 0$ se desplaza hacia derecha
				\4[] $\then$ Aumenta $c^*$, $k^*$
				\4[] \grafica{caidadescuentorck}
				\4 Aumento del gasto público improductivo
				\4[] $\dot{k}_t =  f(k_t) - c_t - G_t - (n+g)k_t$
				\4[] $\dot{k} = 0 \then c = f(k) - G - (n+g)k$
				\4[] $\uparrow G \to \downarrow c$
				\4[] $\then$ Curva $\dot{k} = 0$ se desplaza hacia abajo
				\4[] $\then$ Igual $k^*$, disminuye $c^*$
				\4[] \grafica{aumentogastopublicorck}
		\2 Valoración
			\3 Comparación con modelo de Solow
				\4 Resultados cualitativamente similares
				\4[] $\to$ Estado estacionario único
				\4[] $\to$ Límite a crec. por acumulación de factores
				\4[] $\to$ Crecimiento PIBpc de EE es exógeno
				\4 Pero microfundamentando regla de decisión
				\4[] Decisión de ahorro basada en max. de utilidad
				\4 Permite valorar bienestar
				\4 Permite incorporar shocks diferentes
				\4 Diferentes velocidades de ajuste a equilibrio
				\4[] Partiendo de $k_0 < k^*$
				\4[] $\to$ Velocidad de convergencia más alta
				\4[] Cuando $k_t<k^*$, ahorro es más alto
				\4[] En modelo de Solow, el ahorro es fijo
			\3 Limitaciones
				\4 Similares a las del modelo de Solow
				\4 No explica crecimiento en EE
				\4[] $\to$ Endógeno a modelo
			\3 Modelos relacionados
				\4 Inversión neoclásica (Jorgenson)
				\4 Modelos DSGE
				\4[] Ciclo Real
				\4[] Modelos basados en Lucas (1972)
	\1[] \marcar{Conclusión 2'-30'}
		\2 Recapitulación
			\3 Modelos crecimiento exógeno
				\4 Todo lo que no es trabajo ni capital
				\4[] $\to$ Proceso exógeno
			\3 Harrod-Domar
				\4 Ratio capital-output fijo
				\4 Crecimiento equilibrado inestable
			\3 Solow
				\4 Crecimiento equilibrado estable
				\4 Marco de contabilidad del crecimiento
				\4 Tasa de ahorro exógena
			\3 Ramsey-Cass-Koopmans
				\4 Tasa de ahorro endógena
				\4[] Resultado de optimización
		\2 Idea final
			\3 Diferencias de riqueza
				\4 Factores aparte de $\varDelta L$ y $\varDelta K$
				\4 Modelizadas como procesos exógenos
				\4 Modelos crecimiento endógeno:
				\4[] explicar ese proceso
			\3 Consecuencias futuras del presente
				\4 Ahorro presente
				\4[] $\to$ Impacto producción futura
				\4[] $\to$ Impacto bienestar vía consumo
			\3 Convergencia entre países
				\4 Resultado empírico fundamental
				\4[] Países similares tienden a converger
				\4[] $\to$ Pobres inicialmente crecen más
				\4[] $\to$ Ricos inicialmente crecen menos
				\4[] País distintos no convergen
				\4[] $\to$ Últimas cinco décadas: África y Europa
\end{esquemal}



























\graficas

\begin{axis}{4}{Producción por unidad de trabajo en el modelo de Harrod-Domar}{k}{$f(k)$}{produccionhd}
	\draw[-] (0,0) -- (3,3);
	\draw[-] (3,3) -- (4,3);
	\draw[dashed] (3,3) -- (3,0);
	\node[below] at (3,0){$k=B/A$};
\end{axis}

\begin{axis}{4}{Equilibrio con capital ocioso en Modelo de Harrod Domar}{$k$}{$(\delta + n), \; s \cdot \frac{\min \; \{ Ak, B \} }{k}$}{equilibriohd}
	\draw[-] (0,1.5) -- (4,1.5);
	\draw[-] (0,3) -- (1,3);
	\draw[-] (1,3) to [out=280, in=170](4,0.3);
	
	\draw[dashed] (1,3) -- (1,0);
	\node[below] at (1,0){\tiny $k=B/A$};
	
	\draw[dashed] (1.7, 1.5) -- (1.7,0);
	\node[below] at (1.7,0){\tiny $k^*$};
\end{axis}

\begin{axis}{4}{Modelo de Solow: Inversión y reducción del capital por unidad de trabajo efectivo por depreciación, natalidad y progreso tecnológico en función del capital}{k}{$sf(k)$\\$(n+g+\delta)k$}{solowahorrodepreciacioncapital}
	% reducción del capital per cápita
	\draw[-] (0,0) -- (4,4.3);
	\node[right] at (4,4.3){$(n+g+\delta)k$};
	
	% producción por unidad de trabajo efectivo
	\draw[-] (0,0) to [out=85, in=182](4,3.7);
	\node[right] at (4,3.7){$f(k)$};
	
	% ahorro por unidad de trabajo efectivo
	\draw[-] (0,0) to [out=85, in=182](4,2.6);
	\node[right] at (4,2.6){$sf(k)$};
	
	% capital de estado estacionario
	\draw[dashed] (2.12,0) -- (2.12,2.3);
	\node[below] at (2.12,0){$k^*$};
	
	% consumo por unidad de trabajo
	\draw[dashed] (2.12,2.3) -- (0,2.3);
	\draw[dashed] (2.12,3.2) -- (0,3.2);

	\draw[decorate,decoration={brace,amplitude=3pt},xshift=-2pt,yshift=0pt] (0,2.3) -- (0,3.2) node[black,midway,xshift=-0.3cm] {\footnotesize $c^*$};
	
\end{axis}

\begin{axis}{4}{Modelo de Solow: diagrama de fase del capital.}{$\dot{k}$}{k}{solowdinamicacapital}
	\draw[-] (0,0) -- (0,-2);
	
	% trayectoria
	\draw[-] (0,0) to [out=60, in=180](1,1);
	\draw[-] (1,1) to [out=0, in=120](4,-1.5);
	
	% k de estado estacionario
	\draw[-] (3.02,-0.2) -- (3.02,0.2);
	\node[below] at (3.02,-0.25){$k^*$};
\end{axis}


\begin{axis}{4}{Modelo de Solow: cambio en el capital de estado estacionario tras un aumento del ahorro }{k}{$sf(k)$\\$(n+g+\delta)k$}{solownuevoahorrocapital}
	% reducción del capital per cápita
	\draw[-] (0,0) -- (4,4.3);
	\node[right] at (4,4.3){$(n+g+\delta)k$};
	
	% producción por unidad de trabajo efectivo
	\draw[-] (0,0) to [out=85, in=182](4,3.7);
	\node[right] at (4,3.7){$f(k)$};
	
	% ahorro por unidad de trabajo efectivo
	\draw[-] (0,0) to [out=85, in=182](4,2.6);
	\node[right] at (4,2.6){$sf(k)$};
	
	% capital de estado estacionario
	\draw[dashed] (2.12,0) -- (2.12,2.3);
	\node[below] at (2.12,0){$k^*$};
	
	
	% % % % % % %
	% CAMBIO en ahorro
	
	% NUEVO ahorro por unidad de trabajo efectivo
	\draw[dashed] (0,0) to [out=85, in=182](4,3.2);
	\node[right] at (4,3.2){$s'f(k)$};
	
	% NUEVO capital de estado estacionario
	\draw[dashed] (2.81,0) -- (2.81,3);
	\node[below] at (2.81,0){$k^*$};
\end{axis}

\begin{axis}{4}{Modelo de Solow: cambio en el ahorro tras un aumento en la tasa de ahorro en $t_0$}{$t$}{$s$}{solowahorrotiempo}
	% ahorro
	\draw[-] (0,1) -- (2,1) -- (2,2) -- (4,2);
	
	% cambio en t_0
	\draw[dashed] (2,0) -- (2,4);
	\node[below] at (2,0){$t_0$};

\end{axis}

\begin{axis}{4}{Modelo de Solow: efecto de un aumento de la tasa de ahorro en el crecimiento del capital}{$t$}{$\dot{k}$}{solowcrecimientocapitaltiempo}
	% cambio en t_0
	\draw[dashed] (2,0) -- (2,4);
	\node[below] at (2,0){$t_0$};
	
	% cambio en el crecimiento del capital
	\draw[-] (0,0) -- (2,0) -- (2,1.5) to [out=280, in= 180](4,0);
	
\end{axis}

\begin{axis}{4}{Modelo de Solow: efecto de un aumento de la tasa de ahorro sobre el capital.}{$t$}{$k$}{solowcapitaltiempo}
	% cambio en t_0
	\draw[dashed] (2,0) -- (2,4);
	\node[below] at (2,0){$t_0$};
	
	% cambio en el capital
	\draw[-] (0,1.5) -- (2,1.5) to [out=80, in=180](3,2.5) -- (4,2.5);
	
\end{axis}

\begin{axis}{4}{Modelo de Solow: efecto de un aumento de la tasa de ahorro sobre el crecimiento porcentual de la productividad.}{$t$}{$\Delta \% \frac{Y}{L}$}{solowcrecimientoproductividadtiempo}
	% cambio en t_0
	\draw[dashed] (2,0) -- (2,4);
	\node[below] at (2,0){$t_0$};	
	
	% cambio en el crecimiento de la productividad por trabajador
	\draw[-] (0,1.5) -- (2,1.5) -- (2,2.5) to [out=280, in=180](4,1.5) -- (5,1.5);
	\node[left] at (0,1.5){$g$};
\end{axis}

\begin{axis}{4}{Modelo de Solow: efecto de un aumento de la tasa de ahorro sobre el logaritmo de la productividad}{$t$}{$\ln \frac{Y}{L}$}{solowlogaritmoproductividadtiempo}
	% cambio en t_0
	\draw[dashed] (2,0) -- (2,4);
	\node[below] at (2,0){$t_0$};
	
	% logaritmo de la productividad por trabajador
	
	% trayectoria hasta cambio
	\draw[-] (0,0.5) -- (2,1.3);
	
	% trayectoria post-cambio
	\draw[-] (2,1.3) to [out=80, in=200](2.8,2) -- (4,2.45);
	
	% trayectoria sin cambio
	\draw[dashed] (2,1.3) -- (4,2);		
\end{axis}

\begin{axis}{4}{Modelo de Solow: efecto de un aumento de la tasa de ahorro sobre el consumo.}{$t$}{$c$}{solowconsumotiempo}
	% cambio en t_0
	\draw[dashed] (2,0) -- (2,4);
	\node[below] at (2,0){$t_0$};
	
	\draw[-] (0,2) -- (2,2) -- (2,1) to [out=60, in=180](4,1.8);
\end{axis}

Dado que el aumento del ahorro implica un nuevo consumo de estado estacionario inferior al del estado estacionario inicial, el gráfico implica que el capital de estado estacionario inicial antes del aumento de ahorro era superior al capital de la regla de oro.


\begin{axis}{4}{Dinámica del consumo en el modelo de Ramsey-Cass-Koopmans}{k}{c}{consumorck}
	
	\draw[-] (1.4,0) -- (1.4,4);
	\node[right] at (1.4,3.9){$\dot{c}=0$};
	\node[below] at (1.4,0){$k^*$};
		
	\draw[-{Latex}] (.8,.8) -- (.8,1.2);

	\draw[-{Latex}] (3,3) -- (3,2.6);
	
	\draw[-{Latex}] (.6,3) -- (.6,3.4);
	
	\draw[-{Latex}] (2.5,1) -- (2.5,.6);
	
\end{axis}

El consumo óptimo de un modelo RCK sencillo sigue la senda: $\frac{\dot{c}(t)}{c(t)} =  \frac{f'\left( k(t) \right) - \rho - \theta g}{\theta}$. Cuando se cumple que $\dot{c} > 0$, $k(t)$ debe ser más pequeño que $k^* (t)$ dado que $f(\cdot) < 0$. Por ello, la región a la izquierda de $k^*$ se caracteriza por un consumo creciente. De forma contraria, la región a la derecha de $k^*$ implica un nivel de $k(t)$ más alto que $k^*$ y por tanto, un $f'(k(t))$ más bajo, implicando un nivel de consumo decreciente. 

\begin{axis}{4}{Dinámica del capital en el modelo de Ramsey-Cass-Koopmans}{k}{c}{capitalrck}
	
	% Capital = 0 
	\draw[-] (0,0) to [out=88, in=180](2,2.3);
	\draw[-] (2,2.3) to [out=-1, in=92](3.8,0);
	
	\draw[-{Latex}] (.8,.8) -- (1.2,.8);
	
	\draw[-{Latex}] (3,3) -- (2.6,3);

	\draw[-{Latex}] (.6,3) -- (.2,3);
	
	\draw[-{Latex}] (2.5,1) -- (2.9,1);
	
	\node[right] at (3,2){$\dot{k}=0$};
	
	% consumo de regla de oro
	\draw[dashed] (0,2.3) -- (4,2.3);
	\node[left] at (0,2.3){$c_{GR}$};
	
\end{axis}

\begin{axis}{4}{Diagrama de fase completo de las ecuaciones fundamentales del modelo de Ramsey-Cass-Koopmans}{k}{c}{faserck}
	% Consumo = 0 
	\draw[-] (1.4,0) -- (1.4,4);
	\node[below] at (1.4,0){$k^*$};
	
	% Capital = 0 
	\draw[-] (0,0) to [out=88, in=180](2,2.3);
	\draw[-] (2,2.3) to [out=-1, in=92](3.8,0);
	
	\draw[-{Latex}] (.8,.8) -- (.8,1.2);
	\draw[-{Latex}] (.8,.8) -- (1.2,.8);
	
	\draw[-{Latex}] (3,3) -- (3,2.6);
	\draw[-{Latex}] (3,3) -- (2.6,3);
	
	\draw[-{Latex}] (.6,3) -- (.6,3.4);
	\draw[-{Latex}] (.6,3) -- (.2,3);
	
	\draw[-{Latex}] (2.5,1) -- (2.9,1);
	\draw[-{Latex}] (2.5,1) -- (2.5,.6);
	
	\draw[red,thick,-{Latex}] (0,0) to [out=70, in=220](1.4,2.20);
	
	\draw[red,thick,-{Latex}] (3.5,3.5) to [out=200, in=40](1.4,2.20);
	
	\node[right] at(1.4,2){\small $E$};
	
	\node[right] at (1.4,3.9){$\dot{c}=0$};
	\node[right] at (3,2){$\dot{k}=0$};
	
	
	% consumo de regla de oro
	\draw[dashed] (0,2.3) -- (4,2.3);
	\node[left] at (0,2.3){$c_{GR}$};
	
\end{axis}

El punto E muestra la intersección entre las rectas $\dot{c} = 0$ y $\dot{k} = 0$. En el punto E, todas las variables crecen a una tasa constante.

\begin{axis}{4}{Efecto de una caída en la tasa de descuento de un modelo Ramsey-Cass-Koopmans.}{k}{c}{caidadescuentorck}
	% Consumo = 0 
	\draw[-] (1.4,0) -- (1.4,4);
	\node[below] at (1.4,0){$k^*$};
	\node[right] at (1.4,3.9){$\dot{c}=0$};
	
	% Capital = 0 
	\draw[-] (0,0) to [out=88, in=180](2,2.3);
	\draw[-] (2,2.3) to [out=-1, in=92](3.8,0);
	\node[right] at (3,2){$\dot{k}=0$};
	
	% Senda óptima
	\draw[red,thick,-{Latex}] (0,0) to [out=70, in=220](1.4,2.20);
	\draw[red,thick,-{Latex}] (3.5,3.5) to [out=200, in=40](1.4,2.20);
	
	% Consumo = 0  tras caída de la tasa de descuento
	\draw[dashed] (2,0) -- (2,4);
\end{axis}

A medida que caiga la tasa de descuento $\rho$, el consumo futuro aportará más y el consumo en estado estacionario se acercará más al consumo de la regla de oro de un modelo de Solow equivalente.

\begin{axis}{4}{Efecto de un aumento del gasto público sobre el estado estacionario de un modelo Ramsey-Cass-Koopmans.}{k}{c}{aumentogastopublicorck}
	% Consumo = 0 
	\draw[-] (1.4,0) -- (1.4,4);
	\node[below] at (1.4,0){$k^*$};
	\node[right] at (1.4,3.9){$\dot{c}=0$};
	
	% Capital = 0 
	\draw[-] (0,0) to [out=88, in=180](2,2.3);
	\draw[-] (2,2.3) to [out=-1, in=92](3.8,0);
	\node[right] at (3,2){$\dot{k}=0$};
	
	% Senda óptima
	\draw[red,thick,-{Latex}] (0,0) to [out=70, in=220](1.4,2.20);
	\draw[red,thick,-{Latex}] (3.5,3.5) to [out=200, in=40](1.4,2.20);
	
	% Estado estacionario
	%\node[right] at(1.4,2){\small $E$};
	
	% Capital = 0 tras aumento del gasto público
	\draw[dashed] (0,0) to [out=88, in=180](2,1.8) to [out=0, in=92](3.4,0);

	% Senda óptima tras aumento del gasto público
	\draw[dashed,red,thick,-{Latex}] (0,0) to [out=70, in=220](1.4,1.7);
	\draw[dashed,red,thick,-{Latex}] (3.5,3.0) to [out=200, in=40](1.4,1.7);
\end{axis}


\conceptos

\concepto{Ahorro que maximiza consumo en un modelo de Solow con función de producción Cobb-Douglas}

En un modelo de Solow estándar, el estado estacionario se caracteriza por una tasa de crecimiento nula del capital por trabajador efectivo. Ello resulta en una cantidad de capital por trabajador efectivo constante, y en un producto por trabajador efectivo constante. Ambos dependen de los parámetros del modelo y no del periodo. Es decir, de la tasa de ahorro ($s$), de la tasa de crecimiento de la población ($n$), del progreso tecnológico ($g$), de la tasa de depreciación ($\delta$) y de la participación del capital en el producto ($\alpha$). El consumo por trabajador efectivo es igual al producto por trabajador efectivo menos la fracción del producto por trabajador efectivo que se dedica al ahorro. Por ende, el consumo en estado estacionario dependerá también de los parámetros y no del tiempo. En términos formales:

\begin{align*}
\frac{\dot{k}}{k} = 0 \then k^* =  f(s,n,g,\delta,\alpha) \then y^* f(s,n,g,\delta,\alpha) \\
c^* = f(k^*) - sf(k^*) = f(s,n,g,\delta,\alpha)
\end{align*}

Dada una función de producción Cobb-Douglas, se puede demostrar que el capital y el producto por trabajador efectivo están relacionados de forma que $y(t) = k(t)^\alpha$. En estado estacionario, se puede demostrar que capital y producto por trabajador efectivo toman en estado estacionario los siguientes valores:

\begin{align*}
k^* = \left( \frac{s}{n+g+\delta} \right)^{\frac{1}{(1-\alpha)}} \\
y^* = \left( \frac{s}{n+g+\delta} \right)^{\frac{\alpha}{(1-\alpha)}} \\
\end{align*}

Asumiendo que el objetivo es maximizar el consumo por trabajador efectivo, es preciso hallar la tasa de ahorro que optimiza el consumo en estado estacionario. Tenemos así un problema de maximización tal que:

\begin{align*}
\underset{s}{\max} \quad y^* - sy^* = \left( \frac{s}{n+g+\delta} \right)^{\frac{\alpha}{(1-\alpha)}} - s \cdot \left( \frac{s}{n+g+\delta} \right)^{\frac{\alpha}{(1-\alpha)}}
\end{align*}

Sustituyendo $\frac{1}{(1-\alpha)}$ por $\beta$ y $\left( \frac{1}{n+g+\delta} \right)^{\frac{\alpha}{(1-\alpha)}}$ por $\gamma$ y extrayendo $s$ de los paréntesis tenemos que:

\begin{equation*}
\underset{s}{\max} \quad y^* - sy^* = s^\beta \cdot \gamma - s \cdot s^\beta \cdot \gamma = \gamma \cdot \left( s^\beta - s^{1+\beta} \right)
\end{equation*}

La condición de primer orden es tal que:

\begin{align*}
\gamma \cdot \left( \beta \cdot s^{\beta-1} - s^\beta \cdot \left( 1+\beta \right) \right) = 0 \\
\then \beta s^{\beta-1} = (1+\beta) s^\beta \\
\then s^{\beta - \beta + 1} = \frac{\beta}{1+\beta} \\
\then s = \frac{\beta}{1+\beta} = \frac{\frac{\alpha}{1-\alpha}}{1+\frac{\alpha}{1+\alpha}} \\
\then s^* = \frac{\alpha}{1-\alpha + \alpha} \\
\then s^* = \alpha
\end{align*}

Luego se cumple que la tasa de ahorro óptima es igual a la participación del capital en el producto o la elasticidad del producto respecto del capital:

\begin{center}
	\fbox{$s^* = \alpha$}
\end{center}

Una resolución alternativa y más rápida sustituye consiste en expresar el ahorro de estado estacionario en términos de la depreciación y la natalidad y hallar el capital de estado estacionario que maximiza el consumo. Se obtiene así un capital de la regla de oro en términos de $n$, $\delta$, $A$ y $\alpha$. Este capital de la regla de oro se iguala al capital de estado estacionario en términos de tasa de ahorro y el resto de parámetros. Resolviendo de forma sencilla se obtiene la tasa de ahorro óptima, que coincide con la elasticidad del producto per cápita al capital.

Expresando formalmente este proceso, hallamos primero el capital de estado estacionario.

\begin{equation*}
\dot{k} = sf(k) - (n+\delta) k = sk^\alpha - (n+\delta)k \then sk^\alpha = (n+\delta)k \then k^* = \left( \frac{s}{n+g+\delta} \right)^\frac{1}{1-\alpha}
\end{equation*}

Invirtiendo el cociente y elevando a $-1$ tenemos:

\begin{equation*}
k^* = \left( \frac{n+g+\delta}{s} \right)^{\frac{1}{\alpha -1}}
\end{equation*}

A continuación, hallamos el capital de estado estacionario que induce un consumo máximo, $k_{GR}$.

\begin{align*}
\underset{k}{\max} \quad f(k) - sf(k) = \underset{k}{\max} \quad {k}^\alpha - (n+g+\delta)k \\
\text{CPO:} \quad \alpha k^{\alpha - 1} - (n+\delta+g) = 0 \\
 \alpha k^{\alpha -1} = n+\delta+g \\
k_{GR} = \left(\frac{n+g+\delta}{\alpha } \right)^{\frac{1}{\alpha-1}}
\end{align*}

Igualando $k_{GR}$ y $k^*$:

\begin{align*}
k^* = k_{GR} \\
\left( \frac{n+\delta}{\alpha }  \right)^{\frac{1}{\alpha-1}} = \left( \frac{n+\delta}{s} \right)^{\frac{1}{\alpha -1}} \\
\then s = \alpha
\end{align*}


\concepto{Harrod-Domar: formulación original}

En el esquema largo se ha presentado una formulación de Harrod-Domar equivalente en los principales resultados pero en distinta en lo fundamental a la formulación original. La formulación original es necesaria para comprender las teorías del multiplicador-acelerador sobre la demanda de inversión y por ello se presenta aquí. En la exposición puede ser preferible sin embargo explicar la otra versión, ya que está expresada en el mismo contexto que el modelo de Solow y por ello, es fácil dar el salto de uno a otro.

En el modelo de Harrod-Domar existen tres supuestos fundamentales. El \underline{primero} hace referencia a la demanda de inversión por parte de los empresarios. Los empresarios deciden cuanta inversión llevar a cabo de manera que el cociente entre la producción y el stock de capital se mantenga constante en una cantidad determinada ( a lo cual subyace implícitamente ). Así, la inversión en un instante $t$ será igual al cociente deseado $\frac{K}{Y}$ --que denominaremos $v$ a partir de ahora- multiplicado por el crecimiento del producto en ese instante, esto es, $\frac{\text{d} \, y}{\text{d} \, t}$. Ese crecimiento del producto es desconocido a priori para la empresa, por lo que éstas formulan una expectativa del crecimiento del producto conocida toda la trayectoria anterior pero sin conocer el valor que ha tomado. Así, en términos formales, tenemos que:

\begin{equation}
I = \frac{K}{Y} \cdot \frac{\text{d} \, y}{\text{d} \, t} = v \cdot \frac{\text{d} \, y}{\text{d} \, t}
\end{equation}

El \underline{segundo} supuesto clave concierne el crecimiento efectivo de la producción en un momento determinado. En concordancia con el momento histórico en el que aparece el modelo, de forma casi paralela a la publicación de la Teoría General de Keynes, es imaginable que el concepto del multiplicador de la demanda juegue un papel destacado. Así, se asume que el aumento del producto en un periodo determinado es igual al aumento de la inversión multiplicado por el multiplicador, que se simplifica como la inversa de la tasa de ahorro $s$. Asumimos que $s$ corresponde con la tasa de ahorro que induce una demanda efectiva que garantiza el pleno empleo (o plena utilización de la capacidad productiva de la economía).Así, tenemos la expresión formal del segundo supuesto tal que:

\begin{equation}
\dv{y}{t} = \frac{\text{d} \, I}{s}
\end{equation}

Partiendo de estos dos supuestos, es ya posible analizar qué dinámica resultará de la interacción del aumento efectivo de la producción con la decisión de inversión respecto del aumento esperado. Supongamos en primer lugar que tiene el crecimiento esperado corresponde efectivamente con el crecimiento efectivo. Sustituyendo el crecimiento de la segunda ecuación en la primera, tenemos:

\begin{equation}
I = v \cdot \frac{\text{d} \, I}{s} \then \frac{\text{d} \, I}{I} = \frac{s}{v}
\end{equation}

Así, cuando la economía seguirá una trayectoria de crecimiento equilibrado en el que el ratio capital-producto se mantiene constante. Para ello, la inversión crece a una tasa constante igual al cociente entre el ahorro y el propio ratio capital-producto. ¿Qué sucede cuando se produce un shock exógeno, el crecimiento efectivo es distinto del esperado por los productores, y para el que han tomado una decisión de inversión determinada? 

Examinemos en primer lugar el caso en el que el crecimiento es inferior al esperado. Cuando esto sucede, la inversión llevada a cabo por los productores supera la necesaria para mantener el ratio capital-producto en un valor constante. Así, el ratio capital-producto aumenta. Para corregir este aumento indeseado, los productores reducirán su demanda de inversión, lo cual provocará una disminución adicional del crecimiento efectivo del producto, y el ciclo se retroalimentará de nuevo introduciendo a la economía en una espiral de desinversión análoga a una depresión. 

El segundo caso posible parte del fenómeno contrario: un aumento inesperado del crecimiento efectivo induce una caída del ratio capital-producto. Los empresarios tratarán de corregir esta desviación aumentando la inversión. El aumento de la inversión producirá a su vez un aumento adicional del crecimiento, que a su vez desviará el crecimiento más de lo esperado e introducirá a la economía en una espiral explosiva de crecimiento del stock de capital.

Así, la economía sólo podrá mantenerse en una senda de crecimiento que mantenga estable el ratio capital-output si no se producen desviaciones del crecimiento del producto que los empresarios no hayan considerado. 

El \underline{tercer} supuesto clave impone una restricción adicional al crecimiento de la inversión y la economía. La capacidad productiva de la economía depende del stock de capital, pero también de la cantidad de trabajo, de forma tal que es necesario mantener una determinada relación entre los stocks de ambos factores. Si la cantidad de trabajo disponible crece a la misma tasa que el stock de capital, la economía estará manteniendo un ratio capital-trabajo constante. Sin embargo, si el trabajo crece a una tasa inferior, la economía verá restringido su crecimiento al de la tasa de trabajo, por más elevado que sea el crecimiento del stock de capital. Este límite corresponde con la asunción anterior de la tasa de ahorro como correspondiente a la de pleno empleo. Cuando este límite se alcanza, la economía entra en una espiral inflacionaria como resultado de una demanda de inversión que no puede satisfacerse plenamente por la restricción de capacidad que impone el insuficiente crecimiento del trabajo. En algún momento, el crecimiento será menor del esperado por los empresarios, y tendrá lugar una depresión. Es posible también que la tasa de crecimiento de la población sea superior a la tasa de crecimiento de la inversión. En este caso, la capacidad productiva adicional que permite el crecimiento del trabajo no será aprovechada, y el desempleo aumentará de forma constante.

De los tres supuestos anteriores se deduce la condición que resume las conclusiones del modelo: sólo existe una senda de crecimiento equilibrado y de pleno empleo, y esta senda no estable. Esa senda corresponde con una tasa de crecimiento de la inversión igual al cociente entre tasa de ahorro y ratio capital-producto, que además deberá ser igual a la tasa de crecimiento del trabajo:

\begin{equation}
	\frac{\text{d} \, I}{I} = \frac{s}{v} = n \equiv \frac{\text{d} \, L}{L}
\end{equation}

Esta senda no es estable: cualquier desviación puntual introduce a la economía en una senda explosiva o de desempleo creciente. Además, es altamente improbable que tres valores exógenos como $s$, $v$ y  Por ello, el modelo de Harrod-Domar fue utilizado como justificación teórica de la inestabilidad inherente de los sistemas capitalistas.

\concepto{Participación del capital y el trabajo en la renta en funciones Cobb-Douglas} la participación del capital y el trabajo en la distribución de la renta es una variable relevante en algunos modelos. La función Cobb-Douglas con rendimientos constantes a escala posee la peculiaridad de inducir unas remuneraciones como fracción del producto total que son constantes para cualesquiera niveles de los factores productivos. En el caso de tener en cuenta dos factores productivos, la participación corresponde a un sólo parámetro, y de forma general corresponde a los coeficientes $\alpha_i$, exponentes de los factores respectivos.

Supongamos $Y=F(K,L) = K^\alpha \cdot L^{1-\alpha}$. Tenemos que $F_K=\alpha \cdot K^{\alpha-1} \cdot L^{1-\alpha} = \alpha \cdot \dfrac{Y}{K}$, $F_L=(1-\alpha) \cdot K^\alpha \cdot L^{-\alpha} = (1-\alpha) \cdot \dfrac{Y}{L}$. Suponiendo rendimientos constantes a escala ($0 \geq \alpha \geq 1$), la remuneración de un factor productivo es igual a su producto marginal, de forma que $R = F_K$ y $W = F_L$. Así:
\begin{equation}
\dfrac{R}{Y} = \dfrac{F_K\cdot K}{Y} = \dfrac{\alpha \cdot \dfrac{Y}{K}\cdot K}{Y} = \alpha
\end{equation}

\begin{equation}
	\dfrac{W}{Y} = \dfrac{F_L\cdot L}{Y} = \dfrac{(1-\alpha) \cdot \dfrac{Y}{L}\cdot L}{Y} = 1-\alpha
\end{equation}


\preguntas

\seccion{Test 2018}
\textbf{21.} Sea el modelo de Solow. Suponga que la función de producción es $F(K_t, A_t N_t) = K_t^\alpha (A_t N_t)^{1-\alpha}$. Suponga que la economía se encuentra en el estado estacionario. Diga qué respuesta es \textbf{\underline{FALSA}}:

\begin{itemize}
	\item[a] Si la tasa de ahorro es mayor que $\alpha$ entonces una disminución marginal en la tasa de ahorro incrementará el consumo por unidad de trabajo eficiente respecto de su nivel actual en todos los periodos hasta alcanzar el estado estacionario.
	\item[b] Si la tasa de ahorro es igual $\alpha$, las rentas del capital por unidad de trabajo eficiente son mayores que el ahorro total por unidad de trabajo eficiente.
	\item[c] Si la tasa de ahorro es menor que $\alpha$, un aumento marginal en la tasa de ahorro generará una caída actual en el consumo por unidad eficiente, convergiendo a un consumo de largo plazo mayor que el que había inicialmente.
	\item[d] Si la tasa de ahorro es mayor que $\alpha$ entonces la economía teórica se encuentra en la región dinámicamente ineficiente. 
\end{itemize}

\seccion{Test 2017}
\textbf{14.} Suponga que una economía tiene un nivel fijo de tecnología y la población es constante. Inicialmente, se encuentra en un estado estacionario. Una organización internacional regala una cantidad de capital a cada habitante de esta economía. Según el modelo neoclásico de crecimiento económico con progreso tecnológico exógeno (Swan-Solow):

\begin{itemize}
	\item[a] Como esta economía tiene una mayor capacidad productiva, tanto el producto como el nivel de capital crecerían permanentemente.
	\item[b] El producto y el capital convergerían a un nuevo nivel estacionario, con más capital y producción que el anterior, ya que la capacidad productiva es mayor.
	\item[c] Tanto el producto como el nivel de capital convergerán gradualmente a los mismos niveles existentes antes del regalo.
	\item[d] Como el producto marginal del capital es menor cuando el nivel de capital es más alto, el producto y el nivel de capital convergerían gradualmente a un nivel estacionario con menor capital y producción que los existentes antes del regalo.
\end{itemize}

\seccion{Test 2016}

\textbf{21.} En una economía de Ramsey-Cass-Koopmans, con tiempo discreto y T periodos, siendo $\delta$ la depreciación del capital, la restricción presupuestaria en un periodo $s$ cualquiera es:

\begin{ecuacion}
    c_s + k_{s+1} \leq f(k_{s+1}) + (1-\delta) k_s
\end{ecuacion}

Si la inversión fuera irreversible se verificaría, bajo los supuestos convencionales de no saciedad e interioridad, que en el óptimo:

\begin{enumerate}
    \item[a] $k_{T+1} = 0$
    \item[b] $k_{T+1} = k_{T}$
    \item[c] $k_{T+1} = (1-\delta) k_T$
    \item[d] Ninguna de las respuestas anteriores es correcta.
\end{enumerate}

\textbf{26.} En el modelo de Solow sobre el crecimiento económico:

\begin{enumerate}
    \item[a] Si las funciones de producción de las economías presentan rendimientos crecientes en determinados tramos, puede no darse la convergencia.
    \item[b] Bajo la hipótesis de convergencia, dos economías con distintas funciones de producción convergerían al mismo estado estacionario dado que el país con menor dotación de capital tendrá una mayor tasa de crecimiento.
    \item[c] El crecimiento de equilibrio a largo plazo no es estable.
    \item[d] La tasa de crecimiento en el estado estacionario depende de la propensión al ahorro.
\end{enumerate}

\seccion{Test 2015}

\textbf{22}. Señale la respuesta correcta respecto al modelo de crecimiento económico a largo plazo de Solow-Swan con tasas de depreciación y de crecimiento de la población positivas y constantes:

\begin{enumerate}
    \item[a] El modelo presenta $\beta$ convergencia condicional y absoluta.
    \item[b] Si el capital per cápita de un país se sitúa por encima del capital per cápita asociado al estado estacionario, la tasa de crecimiento de la renta per cápita de dicho país será negativa.
    \item[c] A medida que una economía se acerca a su estado estacionario partiendo de un stock de capital per cápita inferior al asociado al estado estacionario, la tasa de crecimiento del consumo per cápita se acelera al descontar los consumidores la mayor renta futura.
    \item[d] Los estados estacionarios asociados a stocks de capital per cápita superiores al stock de capital per cápita de la llamada \textit{regla de oro de la acumulación del capital} son Pareto-eficientes. 
\end{enumerate}

\seccion{Test 2014}

\textbf{15}. Suponga una economía en la que la población no crece, existe progreso técnico, pero se encuentra en estado estacionario. Según el modelo de Solow, si se incrementa el número de trabajadores:

\begin{enumerate}
    \item[a] El nivel de producción por unidad de trabajo efectivo aumentará.
    \item[b] El nivel de producción por unidad de trabajo efectivo disminuirá.
    \item[c] El nivel de producción permanecerá inalterado.
    \item[d] La tasa de crecimiento de la población aumentará.
\end{enumerate}

\seccion{Test 2013}

\textbf{14.} Considérese una versión del modelo de Ramsey-Cass-Koopmans en la que las preferencias del agente representativo están dadas por la función:

\begin{equation}
\int_t^\infty u\left( c(s) \right) e^{-0.1(s-t)} \, \text{ds}
\end{equation}

A lo largo de la senda de equilibrio estacionario se verificará que el tipo de interés será igual a:

\begin{enumerate}
    \item[a] $e^{-0.1}$
    \item[b] $e^{0.1}$
    \item[c] 10\%
    \item[d] Ninguna de las respuestas anteriores.
\end{enumerate}

\textbf{22}. Considérese el modelo de crecimiento de Solow con progreso tecnológico y crecimiento de la población. La tasa de crecimiento de la renta per cápita vendrá determinada en el estado estacionario por:

\begin{enumerate}
    \item[a] La tasa de crecimiento de la población y por la tasa de crecimiento del progreso tecnológico.
    \item[b] Únicamente por la tasa de crecimiento de la población.
    \item[c] Únicamente por la tasa de crecimiento del progreso tecnológico.
    \item[d] En el estado estacionario la tasa de crecimiento es nula, y por tanto, son erróneas todas las respuestas anteriores.
\end{enumerate}

\seccion{Test 2011}

\textbf{20}. Según el modelo de crecimiento exógeno de Solow:
\begin{enumerate}
    \item[a] Un aumento de la tasa de ahorro provoca un aumento en el stock de capital per cápita de estado estacionario.
    \item[b] Un aumento en la tasa de crecimiento de la población, aumenta el stock de capital per cápita.
    \item[c] Un aumento en el stock de capital de estado estacionario provoca un aumento en el nivel de consumo per cápita de estado estacionario.
    \item[d] Un aumento en la tasa de depreciación física del capital provoca un aumento en el stock de capital de estado estacionario.
\end{enumerate}

\seccion{Test 2009}

\textbf{19.} Considere el marco del modelo de crecimiento neoclásico. Indique entre las siguientes afirmaciones cuál es la \textbf{CORRECTA}:

\begin{itemize}
	\item[a] Un cambio en la tasa de ahorro cambiaría la tasa de crecimiento de la economía a largo plazo.
	\item[b] El tiempo que se tarda en cubrir una fracción de la distancia al equilibrio de largo plazo es independiente de los valores específicos del capital inicial y del capital de largo plazo.
	\item[c] Si hay progreso técnico que aumenta la eficiencia del factor trabajo, un cambio en la tasa de progreso técnico cambiaría la producción en unidades de eficiencia.
	\item[d] Un aumento de la tasa de depreciación disminuiría la tasa de crecimiento de la economía a largo plazo.
\end{itemize}

\textbf{20.} Considere que $Y = AK^\theta H^{1-\theta}$, donde $Y$ es la producción agregada, $A$ es la productividad total de los factores, $K$ es el capital agregado, y $H$ son las horas trabajadas. Suponga, en el marco del modelo de crecimiento neoclásico de equilibrio general, que la población en edad de trabajar (N) crece a la tasa $\eta$ y que $A$ crece a la tasa $\gamma$. Entonces, 

\begin{itemize}
	\item[a] El producto por individuo en edad de trabajar, el ratio $Y/N$, crece a largo plazo la tasa $\eta+\gamma$.
	\item[b] El ratio $Y/N$, aumenta a medida que disminuye la jornada normal de trabajo, es decir, las horas trabajadas por individuo en edad de trabajar $H/N$.
	\item[c] El ratio $Y/N$, disminuye a medida que disminuye la intensidad del capital, es decir el ratio capital-producto $K/Y$.
	\item[d] El ratio $Y/N$ es una función lineal de las horas trabajadas por individuo en edad de trabajar, es decir, el ratio $H/N$.
\end{itemize}

\seccion{Test 2007}

\textbf{13.} Señale cuál de las afirmaciones es, en el contexto del modelo de Solow con progreso tecnológico, \textbf{VERDADERA}:

\begin{itemize}
	\item[a] Un aumento de la tasa de ahorro se traducirá, a largo plazo, en un aumento de la tasa de crecimiento de la producción por trabajador.
	\item[b] Un aumento de la tasa de crecimiento del progreso tecnológico se traducirá, a largo plazo, en un aumento de la producción por trabajador efectivo.
	\item[c] Un aumento de la tasa de natalidad se traducirá, a largo plazo, en un aumento de la tasa de crecimiento de la producción.
	\item[d] Un aumento de la tasa de depreciación se traducirá, a largo plazo, en una disminución de la tasa de crecimiento de la producción por trabajador efectivo.
\end{itemize}

\textbf{16.} En el contexto del modelo de Solow con progreso tecnológico, si la tasa de crecimiento de la tecnología es de un $2\%$ y la participación del capital en el PIB es de 1/3, podemos concluir que, si la tasa de crecimiento de la ratio de capital por trabajador efectivo es de un 1\%, la economía se encuentra en:

\begin{itemize}
	\item[a] Su situación de estado estacionario.
	\item[b] Por debajo de sus niveles de estado estacionario.
	\item[c] Por encima de sus niveles de estado estacionario.
	\item[d] Se necesita más información para responder a esta pregunta.
\end{itemize}

\seccion{Test 2006}

\textbf{21.} Suponga el modelo de crecimiento neoclásico de Solow-Swan, con crecimiento tecnológico neutral en el sentido de Harrod, por tanto, la tecnología está representada por la siguiente función: $Y_t = (A_t N_t)^{1-\alpha} K_t^\alpha$, $0 < \alpha < 1$, donde $A_t$ representa el progreso técnico exógeno que crece a una tasa constante $g$. Sea $k_t = \frac{K_t}{A_t N_t}$, el stock de capital por unidad de trabajo efectivo. Suponga que la población crece a la tasa $n$. En este contexto, la ley de movimiento de capital por unidad de trabajo efectivo, descrita en tiempo continuo, vendrá dada por $\dot{k}_t = s k_t^\alpha - (n+\delta+g) k_t$, donde $\delta$ denota la tasa de depreciación del capital. Diga qué afirmación de las siguientes es \textbf{VERDADERA}:

\begin{itemize}
	\item[a] El stock de capital por unidad de trabajo  efectivo de la regla de oro es: $k_\text{RO} = \left( \frac{1}{n+\delta+g} \right)^\frac{1}{1-\alpha}$ y la tasa de ahorro de la regla de oro es: $s_\text{RO} = \alpha$.
	\item[b] El stock de capital por unidad de trabajo efectivo de la regla de oro es: $k_\text{RO} = \left( \frac{1-\alpha}{n+\delta+g} \right)^\frac{1}{1-\alpha}$  y la tasa de ahorro de la regla de oro es: $s_\text{RO} = 1-\alpha$.
	\item[c] El stock de capital por unidad de trabajo efectivo de la regla de oro es: $\left(  \frac{1-\alpha}{n+\delta+g} \right)^\frac{1}{1-\alpha}$ y la tasa de ahorro de la regla de oro es: $s_\text{RO} = \alpha$.
	\item[d] El stock de capital por unidad de trabajo efectivo de la regla de oro es: $k_\text{RO} = \left( \frac{\alpha}{n+\delta+g} \right)^\frac{1}{1-\alpha} $ y la tasa de ahorro de la regla de oro es $s_\text{RO} = \alpha$.
\end{itemize}

\seccion{Test 2005}

\textbf{21.} Suponga el modelo de crecimiento neoclásico de Solow-Swan, con crecimiento tecnológico neutral en sentido de Harrod: la tecnología está representada por la siguiente función $Y_t = \left( A_t N_t \right)^{1-\alpha} K_t^\alpha$, $0<\alpha<1$, donde $A_t$ representa el proceso técnico exógeno que crece a una tasa constante $g$. Sea $k_t \equiv \frac{K_t}{A_t N_t}$, el stock de capital por unidad de trabajo efectivo. Un aumento en la tasa de ahorro:

\begin{itemize}
	\item[a] Genera un incremento en el stock de capital por unidad de trabajo efectivo de estado estacionario.
	\item[b] Genera un incremento en la tasa de crecimiento de estado estacionario del output per cápita. 
	\item[c] Genera una caída inicial en el stock de capital por unidad de trabajo efectivo, si bien converge a un nivel de estado estacionario mayor que antes del aumento en la tasa de ahorro.
	\item[d] Genera un aumento en la productividad marginal del capital de estado estacionario.
\end{itemize}

\textbf{22.} Suponga el mismo modelo que el descrito en la pregunta anterior. Suponga que la tasa de ahorro tiene un valor superior a $\alpha$. Sea $c_t = \frac{C_t}{A_t N_t}$, siendo $c_t$ el consumo por unidad de trabajo efectivo. En esta situación, un aumento marginal en la tasa de ahorro:

\begin{itemize}
	\item[a] Tiene como efecto una disminución inicial en el consumo por unidad de trabajo efectivo, si bien su nivel de estado estacionario será mayor.
	\item[b] Tiene como efecto un aumento en el consumo por unidad de trabajo efectivo tanto inicialmente como en el estado estacionario.
	\item[c] Tiene como efecto una disminución en el consumo por unidad de trabajo efectivo tanto inicialmente como en el estado estacionario.
	\item[d] No tiene efectos sobre el consumo por unidad de trabajo efectivo ni inicialmente ni en el largo plazo.
\end{itemize}

\textbf{23.} Suponga el modelo de crecimiento neoclásico de Cass-Koopmans.

\begin{itemize}
	\item[a] El stock de capital y el consumo por unidad de trabajo efectivo óptimos en el estado estacionario son inferiores a los niveles dados por la Regla de Oro en el modelo de crecimiento de Solow-Swan.
	\item[b] El stock de capital y el consumo por unidad de trabajo efectivo óptimos en el estado estacionario son superiores a los niveles dados por la Regla de Oro en el modelo de crecimiento de Solow-Swan.
	\item[c] El stock de capital por unidad de trabajo efectivo de estado estacionario es mayor que el nivel de capital por unidad de trabajo efectivo dado por la Regla de oro en el modelo de Solow-Swan. Lo contrario occurre con el consumo por unidad de trabajo efectivo óptimo de estado estacionario comparado con el nivel dado por la Regla de Oro en el modelo de Solow-Swan.
	\item[d] El consumo por unidad de trabajo efectivo de estado estacionario es mayor que el nivel de consumo por unidad de trabajo efectivo dado por la Regla de ORo en el modelo de Solow-Swan. Lo contrario ocurre con el stock de capital por unidad de trabajo efectivo óptimo de estado estacionario comparado con el nivel dado por la Regla de oro en el modelo de Solow-Swan.
\end{itemize}

\seccion{Test 2004}

\textbf{19.} Considere que $Y=A K^\theta H^{1-\theta}$, donde $Y$ es la producción agregada, $A$ es la productividad total de los factores, $K$ es el capital agregado, y $H$ son las horas trabajadas. Suponga, en el marco del modelo de crecimiento neoclásico de equilibrio general, que la población en edad de trabajar $(N)$ crece a la tasa $\eta$ y que $A$ crece a la tasa $\gamma$. Entonces,

\begin{itemize}
	\item[a] El ratio $Y/N$ disminuye a medida que disminuye la intensidad del capital, es decir el ratio capital-producto $K/Y$.
	\item[b] El producto por individuo en edad de trabajar, el ratio $Y/N$, crece a largo plazo a la tasa $\eta + \gamma$.
	\item[c] El ratio $Y/N$ es una función lineal de las horas trabajadas por individuo en edad de trabajar, es decir, el ratio $H/N$.
	\item[d] El ratio $Y/N$ aumenta a medida que disminuye la jornada normal de trabajo, es decir las horas trabajadas por individuo en edad de trabajar.
\end{itemize}



\notas

\textbf{2018}: \textbf{21}. B 

\textbf{2017}: \textbf{14}. C

\textbf{2016.} \textbf{21. D.} Ver pág. 134 de Barro y Sala-i-Martin, Appendix 2B: Irreversible investment. El capital por trabajador efectivo cae a la tasa $(x+n+\delta)$ cuando no se invierte nada.  \textbf{26. A}.

\textbf{2015}: \textbf{22. B}

\textbf{2014}. \textbf{15. A}. Al aumentar el número de trabajadores, disminuye el capital por trabajador y aumenta la productividad marginal del capital por trabajador efectivo. Dado que la economía se encontraba en estado estacionario, se pasa de $sf(k^*_1) = (\delta +n +g)k^*_1$ a $sf(k) > (\delta +n +g)k$

\textbf{2013}. \textbf{14. C}. En el equilibrio estacionario, se cumple que $\frac{\dot{C}}{C} = \frac{r(t)-\rho}{\theta} = 0$. Para ello, debe cumplirse que $r(t) = \rho$ y en el modelo planteado, $\rho = 0.1 = 10\%$, luego el tipo de interés será también igual a 10\%. \textbf{22.C}. En el estado estacionario, el producto por unidad de trabajo efectivo es constante. Sin embargo, el producto per cápita (por trabajador), crece a la tasa de crecimiento de la tecnología.

\textbf{2011}. \textbf{20. A}. Un aumento de la tasa de ahorro implica mayor acumulación de capital por periodo, por lo que en el estado estacionario el capital por trabajo efectivo será mayor.

\textbf{2009}: \textbf{19.} B \textbf{20.} C

\textbf{2007}: \textbf{13.} C \textbf{16.} B

\textbf{2006}: \textbf{21.} D

\textbf{2005}: \textbf{21.} A \textbf{22.} C Aunque el tal Juan Carlos pone en sus soluciones que es la B, tiene que ser un error. Dada la función de producción respecto del capital por trabajo efectivo $f(k) = k^\alpha$, la tasa óptima de ahorro habrá de ser igual a $\alpha$. Cualquier variación de la tasa de ahorro que se desvíe de $\alpha$ inducirá un estado estacionario con menor consumo. Si la tasa de ahorro es superior a $\alpha$, cualquier aumento adicional del ahorro resultará en una caída inmediata del consumo y una desviación aún mayor respecto del consumo de regla de oro, lo que será en definitiva una caída del consumo de estado estacionario. Por ello, es imposible que un aumento de la tasa de ahorro resulte en un aumento inmediato del consumo y también en un aumento del consumo de estado estacionario. \textbf{23.} A

\textbf{2004}: \textbf{19.} A


Hay que incluir en este tema una formulación clásica de Harrod-Domar, a partir de \textit{Harrod-Domar growth model} en la 3a edición del Palgrave.

Hay también que completar el modelo de Solow-Swan con un apartado sobre convergencia, explicando que con el modelo con K y L y la elasticidad de la función de producción al capital ($\alpha$) generalmente estimada, el modelo predice una tasa de convergencia superior a la que estimada a partir de los datos empíricos. La introducción del capital humano como un tercer factor de producción en la función Cobb-Douglas mantiene los resultados principales pero permite reducir la tasa de convergencia a valores más acordes con los datos. Ver págs. 56-61 de Barro y Sala-i-Martín.

\bibliografia

Mirar en Palgrave:
\begin{itemize}
	\item aggregate demand and supply analysis
	\item balanced growth
	\item convergence
	\item economic growth 
	\item economic growth non-linearities
	\item economic growth, empirical regularities in 
	\item endogenous growth theory 
	\item growth and institutions
	\item growth and international trade
	\item growth and learning-by-doing 
	\item growth models, multisector
    \item Harrod-Domar growth model
	\item human capital
	\item human capital, fertility and growth
	\item infrastructure and growth
    \item models of growth
    \item natural and warranted rate of growth
	\item neoclassical growth theory 
	\item neoclassical growth theory (new perspectives) 
	\item Schumpeterian growth and growth policy design 
	\item total factor productivity 	
\end{itemize}


Solow, R. M. (1974) \textit{The Economics of Resources or the Resources of Economics} The American Economic Review, Vol. 64, No. 2. -- En carpeta del tema

Solow, R. {Chapter 9: Neoclassical Growth Theory} (1999) Handbook of Macroeconomics - Volume 1A -- En carpeta del tema (extraído el capítulo)

Solow, R. \textit{Nobel Prize Lecture} (1987)

Parker, Jeffrey. \textit{Economics 314 Coursebook} Ch 3, 4

Krauth, B. \textit{Economics 808: Macroeconomic Theory. Class notes}. http://www.sfu.ca/~bkrauth/econ808/welcome.htm

Barro, R.; Sala-i-Martín, X. \textit{Economic Growth}

Romer, D. \text{Advanced Macroeconomics}. Caps. 1, 2, 4.


\end{document}
