\documentclass{nuevotema}

\tema{3A-30}
\titulo{La Nueva Economía Keynesiana}

\begin{document}

\ideaclave


REGLAS DE TAYLOR: Ver \href{https://sais.jhu.edu/sites/default/files/Taylor%20Rule_Berry_Marquez.pdf}{Berry y Marquez: Interest Rate Determination \& the Taylor Rule}

\seccion{Preguntas clave}
\begin{itemize}
	\item ¿Qué es la Nueva Economía Keynesiana?
	\item ¿A qué autores se asocia?
	\item ¿En qué contexto aparece?
	\item ¿Qué herramientas utiliza?
	\item ¿Cuáles son sus principales aportaciones?
	\item ¿Qué implicaciones tiene sobre las políticas económicas?
\end{itemize}

\esquemacorto

\begin{esquema}[enumerate]
	\1[] \marcar{Introducción}
		\2 Contextualización
			\3 Macroeconomía
			\3 Escuelas macroeconómicas
			\3 Nueva Economía Keynesiana
		\2 Objeto
			\3 ¿Qué es la Nueva Economía Keynesiana?
			\3 ¿A qué autores se asocia?
			\3 ¿En qué contexto se desarrolla?
			\3 ¿Qué innovaciones metodológicas introduce?
			\3 ¿Cuáles son sus principales aportaciones?
			\3 ¿Qué implicaciones de política económica se derivan?
		\2 Estructura
			\3 Primera generación
			\3 Segunda generación
	\1 \marcar{Primera generación de NEK}
		\2 Visión general
			\3 Contexto económico
			\3 Contexto teórico
			\3 Autores
			\3 Ideas centrales
		\2 Modelos
			\3 Contratos implícitos
			\3 Precios pegajosos y fijación escalonada
			\3 Salarios de eficiencia
			\3 Costes de menú y racionalidad limitada
			\3 Negociación salarial
			\3 Fallos de coordinación
		\2 Implicaciones
			\3 Política económica
			\3 Teoría económica
	\1 \marcar{Segunda generación de NEK}
		\2 Visión general
			\3 Contexto económico
			\3 Contexto teórico
			\3 Autores
			\3 Ideas centrales
		\2 Modelo canónico de la NEK
			\3 Idea clave
			\3 Formulación
			\3 Implicaciones
			\3 Regla de Taylor
			\3 Extensiones
			\3 Valoración
		\2 Implicaciones globales
			\3 Política económica
			\3 Teoría económica
	\1[] \marcar{Conclusión}
		\2 Recapitulación
			\3 Primera generación de NEK
			\3 Segunda generación de NEK
		\2 Idea final
			\3 Robert Solow sobre modelos macro y economistas
			\3 Críticas a la hegemonía actual
			\3 Estado actual de la NEK
			\3 Crisis financieras

\end{esquema}

\esquemalargo























\begin{esquemal}
	\1[] \marcar{Introducción}
		\2 Contextualización
			\3 Macroeconomía
				\4 Análisis de fenómenos económicos a gran escala
				\4 Énfasis sobre variables agregadas
			\3 Escuelas macroeconómicas
				\4 Postulan diferentes explicaciones de la realidad
				\4[] Causas de fluctuaciones
				\4[] Causas de comportamiento de largo plazo
				\4[] Efectos de políticas económicas
				\4 Utilizan diferentes herramientas
				\4[] Métodos matemáticos
				\4[] Métodos verbales
				\4 Revoluciones científicas
				\4[] Sinónimo con cambio de paradigma
				\4[] Cambio radical en
				\4[] $\to$ Herramientas
				\4[] $\to$ Objeto de estudio
				\4[] $\to$ Conclusiones
			\3 Nueva Economía Keynesiana
				\4 Varias fases
				\4[] Elemento común
				\4[] $\to$ Microfundamentar conceptos keynesianos
				\4 Respuesta a NMC
				\4[] Aparece a finales de años 70
				\4[] Relación compleja con NMC
				\4[] $\to$ Rechazo de algunas conclusiones
				\4[] $\to$ Adopción de metodologías
				\4 Presencia actual
				\4[] Enorme importancia
				\4[] Enfoque predominante de modelización
				\4[] $\to$ Bancos Centrales
				\4[] $\to$ Departamentos de análisis
				\4[] $\to$ Estudios de política macroeconómica
		\2 Objeto
			\3 ¿Qué es la Nueva Economía Keynesiana?
			\3 ¿A qué autores se asocia?
			\3 ¿En qué contexto se desarrolla?
			\3 ¿Qué innovaciones metodológicas introduce?
			\3 ¿Cuáles son sus principales aportaciones?
			\3 ¿Qué implicaciones de política económica se derivan?
		\2 Estructura
			\3 Primera generación
			\3 Segunda generación
	\1 \marcar{Primera generación de NEK}
		\2 Visión general
			\3 Contexto económico
			\3 Contexto teórico
				\4 Nueva vuelta de tuerca
				\4[] Diálogo histórico entre
				\4[] $\to$ Dicotomía clásica vs no neutralidad
				\4[] $\to$ Estabilidad inherente vs inestabilidad
				\4[] $\to$ Equilibrio único vs equilibrios múltiples
				\4[] $\to$ Ley de Say vs insuficiencias de demanda
				\4 Paradigma clásico
				\4[] Salarios y precios perfectamente flexibles
				\4[] Mercados competitivos
				\4[] $\then$ Pleno empleo
				\4 Keynesianismo
				\4[] Precios y sobre todo salarios
				\4[] $\to$ Rígidos
				\4[] Economía no alcanza eq. competitivo
				\4[] $\to$ Mercados no están en equilibrio
				\4[] $\to$ Existen múltiples eq. y algunos subóptimos
				\4 Síntesis neoclásica
				\4[] En el largo plazo
				\4[] $\to$ Precios flexibles
				\4[] $\then$ Equilibrio competitivo
				\4[] En el corto plazo
				\4[] $\to$ Precios y salarios exógenos
				\4[] $\then$ IS-LM y AS-AD
				\4[] $\then$ Desempleo posible
				\4 Nueva Macroeconomía Clásica
				\4[] Precios y salarios flexibles
				\4[] Microfundamentación de la macroeconomía
				\4[] Hipótesis de expectativas racionales
				\4[] $\to$ Información se utiliza eficientemente
				\4[] Equilibrio general walrasiano
				\4[] $\to$ Mercados en equilibrio
				\4 Primera generación de NEK
				\4[] Asumiendo:
				\4[] $\to$ Necesidad de microfundamentar (crítica de Lucas)
				\4[] $\to$ Hipótesis de expectativas racionales
				\4[] Explicar:
				\4[] $\to$ ¿Por qué estímulo a demanda agregada puede aumentar output?
				\4[] $\to$ ¿Por qué estímulos monetarios pueden aumentar output?
				\4[] $\to$ ¿Por qué economías no utilizan capacidad totalmente?
				\4[] $\to$ ¿Por qué existe desempleo al salario que predomina?
				\4[] $\to$ ¿Por qué no se alcanza el pleno empleo?
				\4[] $\to$ ¿Por qué aparecen rigideces nominales?
				\4[] $\to$ ¿Por qué los precios/salarios no son flexibles?
				\4[] $\to$ ¿Por qué economía no alcanza pleno empleo?
			\3 Autores
				\4 Azariadis, Costas
				\4[] Asociado a Lucas y Prescott
				\4 Fischer, Stanley
				\4[] Asociado a NMC
				\4[] Fijación escalonada de precios
				\4 Taylor, John
				\4 Stiglitz, Joseph
				\4 Mankiw, Gregory
				\4 Akerloff y Yellen
			\3 Ideas centrales
				\4 Equilibrio parcial
				\4 Competencia imperfecta
				\4[] No todos los agentes son precio aceptantes
				\4[] Alguien tiene capacidad para:
				\4[] $\to$ Fijar unilateralmente salarios o precios
				\4 Precios y/o salarios rígidos
				\4[] Rigidez salarial
				\4[] $\to$ Nominal o real
				\4 Aceptación de la síntesis neoclásica
				\4[] Aceptan en general c/p vs l/p
				\4 Énfasis en mercado de trabajo
				\4[] Fijación de salarios y desempleo
				\4[] $\to$ Objetivo de mayoría de modelos
		\2 Modelos
			\3 Contratos implícitos
				\4 Azariadis (1975)
				\4 Idea clave
				\4[] Equilibrio parcial
				\4[] Un sólo periodo
				\4[] Para reducir riesgo a trabajadores
				\4[] $\to$ Empresas fijan salarios
				\4 Empresa:
				\4[] neutral al riesgo
				\4[] Incentivos a mantener relación contractual
				\4[] $\to$ Inversiones en capital humano
				\4[] $\to$ Costes fijos incurridos
				\4 Trabajadores
				\4[] Aversos al riesgo
				\4[] Quieren mantener flujo constante de consumo
				\4 Salario suma de dos componentes
				\4[] $\to$ Productividad marginal del trabajo
				\4[] $\to$ Prima o indemnización según estado de naturaleza
				\4[$\Rightarrow$] $\bar{w} = \text{PMgL} + \gamma$
				\4[] $\bar{w}$: fijo
				\4[] $\gamma$: positivo si $\bar{w} > \text{PMgL}$
				\4[$\then$] Aseguramiento frente a fluctuaciones
				\4[$\then$] Salario se mantiene rígido
				\4 Estado de la naturaleza adverso
				\4[] Empresa paga:
				\4[] $\to$ Productividad marginal del trabajo
				\4[] $\to$ Indemnización para cubrir diferencia
				\4[] Si coste de indemnizaciones elevado y despido bajo
				\4[] Empleado paga:
				\4[] $\to$ Prima de aseguramiento
				\4[] $\to$ Empresa despide empleados
				\4 Estado de la naturaleza favorable
				\4[] Empresa paga:
				\4[] $\to$ Productividad marginal del trabajo
				\4[] Empleado paga:
				\4[] $\to$ Prima de aseguramiento
				\4[] Salario real
				\4[] $\to$ Se mantiene constante
				\4 Formulación
				\4[] Dos estados de la naturaleza
				\4[] $\to$ Demanda alta $\then$ Valor marginal alto
				\4[] $\to$ Demanda baja $\then$ Valor marginal bajo
				\4[] Trabajadores aversos al riesgo
				\4[] $\to$ Masa fija de trabajadores
				\4[] $\to$ Oferta individual indivisible
				\4[] $\to$ Consideran salario y riesgo de $\Delta L$
				\4[] $\then$ Dispuestos a pagar para asegurarse
				\4[] Empresa neutral al riesgo
				\4[] $\to$ Sólo considera beneficio esperado
				\4[] $\to$ Pueden asegurar salario a cambio de prima
				\4[] $\to$ No pueden asegurar contra desempleo
				\4[] $\then$ Pueden no contratar a algunos
				\4[] $\then$ No pueden dar subsidio de desempleo
				\4[] Contrato es suma de dos contratos
				\4[] $\to$ Contrato de trabajo
				\4[] $\to$ Contrato de aseguramiento de $w$
				\4[] Estado desfavorable de la naturaleza
				\4[] $\to$ Salario más alto que valor marginal
				\4[] $\to$ Algunos empleos no compensan
				\4[] $\then$ Despidos de trabajadores
				\4[] $\then$ Salario constante para trabajadores mantenidos
				\4 Implicaciones
				\4[] Existe desempleo involuntario
				\4[] Despedidos querrían trabajar a salario fijo
				\4[] Empresa y empleados han fijado salario
				\4[] $\to$ Mercado de trabajo no se ajusta
				\4[] $\then$ Rigidez real
				\4[] $\then$ Desempleo involuntario
				\4[] $\then$ Volatilidad salarial asegurada
				\4[] $\then$ Volatilidad de empleo aumentada
				\4[] Salario acíclico
				\4[] $\to$ Shock $\uparrow$ productividad:
				\4[] $\then$ $w$ =,
				\4[] $\then$ $\uparrow$ Trabajo
				\4[] $\then$ Salario real constante aunque aumente output
				\4[] $\then$ Ajuste en trabajo, no en salarios
				\4[] $\to$ Shocks $\downarrow$ productividad:
				\4[] $\then$ $w$ =,
				\4[] $\then$ $\downarrow$ Trabajo o no contratación
				\4[] $\then$ Salario real constante aunque caiga output
				\4[] $\then$ Ajuste en output, no salarios
				\4[] Despidos concentrados
				\4[] $\to$ Consistente con evidencia empírica
				\4[] $\to$ Cuando coste de indemnizar empleados demasiado alto
				\4[] $\then$ Empresas descargan en seguro de desempleo
				\4[] Aumento de volatilidad del empleo
				\4[] $\to$ Con subsidio de desempleo elevado
				\4[] $\to$ Con empleados que exigen mayor seguridad salarial
				\4 Valoración
				\4[] Pleno empleo no tiene por qué ser óptimo
				\4[] $\to$ Despidos óptimos para ambas partes
				\4[] Carácter ad-hoc de supuestos
				\4[] $\to$ ¿Por qué no hay SDesempleo?
				\4[] $\to$ ¿Por qué no hay ajuste de horas?
				\4[] Influencia posterior
				\4[] $\to$ Avance en teoría de contratos
				\4[] $\to$ 1er paso en microfundamentación rigidez
				\4[] $\to$ Muy escasa en modelos NEK
			\3 Precios pegajosos y fijación escalonada
				\4 Idea clave
				\4[] Compatibilizar:
				\4[] $\to$ HER
				\4[] $\to$ No-neutralidad del dinero
				\4[] Mostrar que relevancia de política monetaria
				\4[] $\to$ Requiere rigidez nominal
				\4[] $\to$ No es incompatible con HER
				\4[] Políticas de estabilización
				\4[] $\to$ Tienen efectos
				\4[] $\to$ Pueden ser necesarias
				\4 Gray(1977), Fischer (1977)
				\4[] Sticky wages
				\4[] $\to$ Duran varios periodos
				\4[] Salarios son fijos en futuro
				\4[] $\to$ Número conocido y fijo de periodos
				\4[] Salario que se espera vacíe mercado
				\4[] Efectos de shock nominal
				\4[] $\to$ Sólo hasta fin de periodo futuro
				\4 Taylor (1979), (1980)
				\4[] Staggered wages
				\4[] $\to$ Varias cohortes fijan escalonadamente
				\4[] $\then$ En función de contratos pasados
				\4[] $\then$ En función de condiciones esperadas futuras
				\4[] ``multiplicador contractual''
				\4[] $\to$ Un contrato afecta más allá de su horizonte
				\4[] $\then$ Relevante + periodos que para los que se fijó
				\4 Formulación -- Taylor (1980)
				\4[] En cada periodo T
				\4[] $\to$ Cohorte T negocia salario
				\4[] $\to$ Salario fijado para T y T+1
				\4[] Precios son mark-up constante
				\4[] $\to$ Sobre salarios
				\4[] $\to$ Luego precios y salarios escalonados
				\4[] (1) Salario para cohorte T
				\4[] $x_T = \frac{1}{2} \left( w_T + E_T w_{t+1} \right) + \frac{1}{2}\cdot g\left( y_T + E_T y_{T+1} \right)$
				\4[] Depende de:
				\4[] $\to$ Salario total medio en T ($w_T$)
				\4[] $\to$ Salario total medio esperado en T+1 ($E_T w_{T+1}$)
				\4[] $\to$ Producción en T ($y_T$)
				\4[] $\to$ Producción esperada en T+1 ($y_{T+1}$)
				\4[] $\to$ Sensibilidad a variación en output ($g$)
				\4[] (2) Salario medio en T
				\4[] $w_T = \frac{1}{2} \left( x_T + x_{t-1} \right)$
				\4[] (3) Equilibrio mercado monetario
				\4[] $m_T = y_T + w_T$
				\4[] Saldos reales $\uparrow$ tanto como $\uparrow$:
				\4[] $\to$ Aumente output
				\4[] $\to$ Aumenten salarios
				\4[] Sustituyendo (2) y (3) y resolviendo:
				\4[] \fbox{$x_T = f(g) x_{T-1} + \left( 1-f(g) \right) m_T$}  $\leftarrow$ $\dv{f}{g} < 0$
				%				\4[] Fischer (1977)
				%				\4[] Oferta
				%				\4[] $y_t^S = \left( p_t - w_t^{t-1} \right) + \mu_t$
				%				\4[] $w_t^{t-1} = E_{t-1} (p_t)$
				%				\4[] $\mu_t = \rho_1 \mu_{t-1} + \epsilon_t$, $|\epsilon_t|<1$
				%				\4[] $\then$ \fbox{$y_s = \left( p_t - E_{t-1}(p_t) \right) + \rho_1 \mu_{t-1} + \epsilon_t$}
				%				\4[] Demanda
				%				\4[] $y^D_t = m_t - p_t - v_t$
				%				\4[] $v_t=\rho_2 v_{t-1} + \eta_t$
				%				\4[] $m_t = \sum_{i=1}{\infty} a_i \mu_{t-1} + \sum_{i=1}^\infty b_1 v_{t-1}$
				%				\4[] $\then$ \fbox{$y_t^D = m_t - p_t - \rho_2 v_{t-1} - \eta_t$, $| \eta_t | < 1$}
				%				\4[] Equilibrio
				%				\4[] \fbox{$y_t = f(\underset{+}{\epsilon_t}, \underset{-}{\eta_t}, \underset{+}{\epsilon_{t-1}}, \underset{+}{a_1}, \underset{+}{\rho_1}, \underset{+}{\eta_{t-1}}, \underset{+}{b_1}, \underset{-}{\rho_2}, \underset{+}{\rho_1}, \underset{+}{\mu_{t-2})}$}
				\4[] Si $g$ muy bajo:
				\4[] $\to$ Sensibilidad salario--output muy baja
				\4[] $\to$ Salario constante aunque output aumente
				\4[] $\then$ Persistencia elevada de contrato de L
				\4[] $\then$ Estímulo monetario tiene mucho efecto
				\4[] $\then$ Política monetaria no neutral
				\4[] Si $g$ muy alto:
				\4[] $\to$ Sensibilidad salario--output muy alta
				\4[] $\to$ Salarios $\uparrow$ si $\uparrow$ output
				\4[] $\then$ Persistencia baja de contrato de L
				\4[] $\then$ Estímulo monetario tiene muy poco efecto
				\4[] $\then$ Política monetaria poco efectiva
				\4 Calvo (1983)
				\4[] Precios escalonados
				\4[] Cada empresa puede cambiar precios
				\4[] $\to$ De acuerdo a lotería no degenerada
				\4 Formulación -- Calvo (1983)
				\4[] Demanda más compleja
				\4[] $\to$ No sólo depende de saldos reales
				\4[] Incorporación de demanda à la Sidrauski -- MIU
				\4[] Precios fijos
				\4[] $\to$ No salarios
				\4[] Empresas cambian precios
				\4[] $\to$ Con probabilidad aleatoria
				\4[] Decisión de fijación de precio depende de:
				\4[] $\to$ Precio esperado en futuro
				\4[] $\to$ Probabilidad de poder cambiar
				\4[] $\to$ Demanda agregada
				\4 Implicaciones
				\4[] Puente entre NMC y keynesianismo
				\4[] HER en keynesianismo puede ser útil
				\4 Valoración
				\4[] Basado en hechos empíricos
				\4[] $\to$ Salarios fijados en términos nominales
				\4[] $\to$ Fijación asíncrona entre trabajadores
				\4[] $\to$ Trabajadores comparan sus salarios
				\4[] Allana camino a segunda generación
				\4[] $\to$ Especialmente, Calvo (1983)
				\4[] Razones de rigidez nominal
				\4[] $\to$ Realmente fundamentadas en otros modelos
			\3 Salarios de eficiencia
				\4 Shapiro y Stiglitz (1984)
				\4[] $\to$ Heredero de Salop (1979)
				\4[] Aplicación de modelización formal de riesgo moral
				\4 Idea clave
				\4[] Salario real mayor a salario de equilibrio
				\4[] $\to$ Puede aumentar beneficios de empresa
				\4[] $\then$ Empresas fijan $w_E > w^*$
				\4[] $\then$ Salario no vacía mercado de trabajo
				\4[] $\then$ Exceso de oferta de trabajo persiste
				\4[] $\then$ Desempleo involuntario
				\4[] Justificar en base a información asimétrica
				\4[] $\to$ Empresas no conocen esfuerzo exacto
				\4[] $\to$ Descubren vagueo con prob. < 1
				\4[] Incentivar empleados a trabajar más
				\4[] $\to$ Ofreciendo salario más elevado
				\4[] $\then$ Pérdida mayor si son descubiertos
				\4[] Aumento salarial aumenta oferta
				\4[] $\to$ Trabajo mejor calidad se oferta
				\4[] Efectos directos de mayor salario
				\4[] $\to$ Aumenta bienestar de trabajadores
				\4[] $\then$ Aumenta capacidad de trabajo
				\4[] \quad (países en desarrollo)
				\4 Formulación
				\4[] Estados de la naturaleza
				\4[] i. Empleado sin vaguear (NSC)\footnote{``\textit{non-shirking} condition''.}
				\4[] ii. Empleado vagueando
				\4[] iii. Despedido por vaguear
				\4[] Trabajadores
				\4[] $\to$ Trabajo genera desutilidad
				\4[] $\to$ Trabajo permite ganar salario
				\4[] $\to$ Incentivos a emplearse pero ``vaguear''\footnote{En inglés se utiliza ``shirking''.}
				\4[] $\to$ Riesgo de perder empleo si vaguean
				\4[] $\to$ Consideran salarios futuros descontados
				\4[] $\then$ ¿Vaguear y arriesgarse es rentable?
				\4[] Empresas
				\4[] $\to$ Descubrir vagos es costoso
				\4[] $\to$ Descubren con determinada probabilidad
				\4[] $\to$ Ofrecen salario que incentive no-vagueo
				\4[] $\then$ Inducir que despido sea muy ``doloroso''
				\4[] NSC: salario real $w^E$ que induce
				\4[] $\to$ VActual de no vaguear > vaguear
				\4[] $\then$ Salario de eficiencia
				\4[] Representación gráfica
				\4[] \grafica{salariodeeficiencia}
				\4 Implicaciones
				\4[] Curva de Phillips
				\4[] $\to$ Menos paro, necesario más salario para no vaguear
				\4[] $\then$ Aumentar pérdida si pierde empleo
				\4[] $\then$ $\dot{w}$ decreciente en desempleo
				\4[] Existe desempleo involuntario
				\4[] $\to$ Agentes querrían trabajar por $w^E$ o menos
				\4[] $\to$ Salario real fijo en $w^E$ por empresas
				\4[] Políticas de demanda innecesarias
				\4[] $\to$ Empresas ya contratan trabajo óptimo
				\4[] $\to$ Pleno empleo no es deseable
				\4[] Reducción del desempleo
				\4[] $\to$ Disminuye probabilidad de paro
				\4[] $\to$ Disminuye coste de ser despedido
				\4[] $\then$ Aumenta salario de eficiencia necesario
				\4[] $\then$ Crecimiento $\uparrow$ w, = desempleo
				\4[] $\then$ Salario real ligeramente procíclico
				\4 Valoración
				\4[] Equilibrio parcial
				\4[] $\to$ Difícil integración con eq. general
				\4[] Microfundamentación de desempleo
				\4[] $\to$ Muy sólidamente justificado
				\4[] $\to$ Razones puramente microeconómicas
				\4[] Desempleo involuntario es óptimo
				\4[] $\to$ Contradice enfoque keynesiano
				\4[] $\to$ Desempleo no es problema a remediar
				\4[] Difícil extraer conclusiones cuantitativas
				\4[] $\to$ Coste de esfuerzo difícil de estimar
				\4[] $\to$ Difícil tratamiento econométrico
			\3 Costes de menú y racionalidad limitada
				\4 Mankiw (1985)
				\4[] Costes de menú propiamente dichos
				\4[] Cambiar precios es costoso en sí mismo
				\4 Akerloff y Yellen (1985a), (1985b)
				\4[] Racionalidad limitada
				\4[] Aplicar cambios racionales
				\4[] $\to$ tiene un coste adicional
				\4[] $\then$ Cambiar puede ser subóptimo
				\4 Idea clave
				\4[] Precios flexibles
				\4[] $\to$ Dinero es neutral
				\4[] Precios costosos de cambiar
				\4[] $\to$ Dinero puede no ser neutral
				\4[] $\to$ Ajuste en cantidades y no precios
				\4[] $\then$ Economía no alcanza pleno empleo
				\4 Formulación
				\4[] Beneficios: $\pi = f(P, Y, L, W_N)$
				\4[] Situación inicial: $Y_0$
				\4[] $\to$ Monopolista fija $P_0$ óptimo
				\4[] Cambio en demanda: $Y_1 > Y_0$
				\4[] $\to$ Óptimo implica $P_1 > P_0$
				\4[] $\to$ Cambio a $P_1$ es costoso
				\4[] $\then$ Mantiene precio más bajo
				\4[] $\then$ Aumenta producción más que con $P_1$
				\4[] Representación gráfica
				\4[] \grafica{costesdemenu}
				\4 Implicaciones
				\4[] Política monetaria
				\4[] $\to$ Tiene efectos reales
				\4[] Activación de demanda
				\4[] $\to$ Aumenta empleo
				\4[] $\to$ Efectos generales positivos
				\4[] $\to$ Deseable aumentar DAgregada
				\4 Valoración
				\4[] Equilibrio parcial
				\4[] Modelos estáticos
				\4[] $\to$ Modelos muy simples y tratables
				\4[] $\to$ Buena representación del fenómeno
				\4[] Modelos dinámicos
				\4[] $\to$ Muy difícil tratamiento
				\4[] Interacción con rigidez real\footnote{Por ejemplo, con salarios de eficiencia.}
				\4[] $\to$ Aumenta efectos de costes de menú
				\4[] Relación con Calvo (1983) y escalonados
				\4[] $\to$ Realmente, basados en supuesto ad-hoc
				\4[] $\then$ CdMenú sirve como fundamentación
			\3 Negociación salarial
				\4 Layard y Nickell (1985), (1986)
				\4[] Simplificado por Carlin y Soskice (1990)
				\4 Idea clave
				\4[] Inflación y desempleo resultado de:
				\4[] $\to$ Negociación entre sindicato y empresa
				\4[] Output y tipos de interés resultado de:
				\4[] $\to$  Modelo IS-LM subyacente
				\4[] Sindicatos fijan salario nominal
				\4[] $\to$ Con un salario real como objetivo
				\4[] $\to$ Estimando una tasa de inflación
				\4[] Empresas fijan precios
				\4[] $\to$ Determinan salario real
				\4[] $\to$ Para extraer un mark-up dado
				\4[] $\to$ Para alcanzar un beneficio real determinado
				\4[] NAIRU como compatibilidad entre:
				\4[] $\to$ Demandas de salario real de sindicatos
				\4[] $\to$ Mark-up que desean extraer las empresas
				\4[] Expectativas
				\4[] $\to$ Adaptativas
				\4[] $\to$ Posición discutible dentro de NEK
				\4 Formulación
				\4[] Dos ecuaciones fundamentales
				\4[] Salario Real Negociado (BRW)
				\4[] \fbox{BRW: $w_S = f(U)$, $\dv{f}{U} < 0$}
				\4[] $\to$ Salario real que desean sindicatos
				\4[] $\to$ Decreciente con sindicatos
				\4[] Salario Real Pagado (PRW)
				\4[] \fbox{PRW: $w_F = \frac{W}{P} = \text{PMe}_L (1-m)$}
				\4[] $\to$ Empresas fijan precios P
				\4[] $\then$ Determinan salario real
				\4[] $\to$ Salario real pagado depende de markup $m$
				\4[] $\then$ $\uparrow$ desempleo $\then$ $\downarrow$ $w_S$
				\4[] $\to$ Exceso de capacidad
				\4[] $\then$ PMeL más o menos constante
				\4[] Equilibrio: $w_S = w_F$
				\4[] $\to$ Demandas compatibles
				\4[] $\to$ Sindicatos satisfechos con $\uparrow$ salario
				\4[] $\to$ $\uparrow$ precio satisface empresas
				\4[] $\then$ No hay inflación salarial
				\4[] Desequilibrio $w_S > w_F$
				\4[] $\to$ Demandas sindicato-empresa incompatibles
				\4[] $\to$ Desempleo por debajo de equilibrio
				\4[] $\then$ Sindicatos demandan más salario
				\4[] $\then$ Empresas $\uparrow$ precio para mantener margen
				\4[] $\then$ Aumenta inflación
				\4[] $\to$ Curva LM se desplaza a izquierda
				\4[] $\then$ Cae producción
				\4[] $\then$ Desempleo vuelve a NAIRU
				\4[] $\then$ Necesario $\uparrow$ M cte. para mantener $U < \text{NAIRU}$
				\4[] Representación gráfica
				\4[] \grafica{carlinsoskice}
				\4 Implicaciones
				\4[] Similares a Friedman (1968)
				\4[] $\to$ Aunque distintos motivos
				\4[] $\to$ Sindicatos como pieza central
				\4[] Desempleo involuntario
				\4[] $\to$ Existe
				\4[] $\to$ Dado PRW, oferta de L mayor
				\4[] Curva de Phillips
				\4[] $\to$ Similar a Friedman
				\4[] $\to$ Curva l/p vertical
				\4[] $\to$ Curvas de c/p requieren $\Delta \pi$ crecientes
				\4[] Políticas de reducción de la NAIRU
				\4[] $\to$ $\downarrow$ de poder de sindicatos
				\4[] $\to$ $\uparrow$ competencia para reducir mark-ups
				\4[] $\to$ $\downarrow$ de impuestos para $\uparrow$ beneficio
				\4[] $\to$ Mejorar productividad del trabajo
				\4[] $\to$ Control de rentas
				\4 Valoración
				\4[] Impacto reducido en literatura
				\4[] $\to$ Apenas continuado por autores NEK
				\4[] Equilibro general incompleto
				\4[] Carácter ad-hoc de supuestos
			\3 Fallos de coordinación
				\4 Idea clave
				\4[] Agentes no pueden coordinar decisiones
				\4[] $\to$ Equilibrios subóptimos son posibles
				\4[] $\then$ Desempleo como resultado
				\4 Diamond (1982), Robert (1987), Howitt
				\4[] Inspirado en Leijonhufvud, Clower, Patinkin..
				\4 Modelo de los cocos de Diamond
				\4[] Metáfora con cocos à la islas de Phelps
				\4[] Agentes viven en economía cerrada
				\4[] Pueden producir bienes recogiendo cocos de arboles
				\4[] Tabú impide consumir cocos que uno mismo recoge
				\4[] $\to$ Debe intercambiar con otro agente
				\4[] Para que un agente recoja cocos
				\4[] $\to$ Debe tener expectativa de que otros también
				\4[] $\then$ Debe creer que podrá intercambiarlos con otro
				\4[] Sin expectativa de intercambiar
				\4[] $\to$ Nadie tendrá incentivo a producir
				\4[] $\then$ Posibles múltiples equilibrios
				\4[] $\then$ Posible capacidad sin utilizar
				\4 Cooper y John (1988)
				\4[] Abandonan idea de rigideces nominales
				\4[] $\to$ Inicialmente entendido como alternativa a NEK 1aGEN
				\4[] $\then$ Posteriormente integrada con Ball y Romer (1991)
				\4[] Complementos estratégicos pueden determinar eq. macro
				\4[] Estrategia óptima de un agente
				\4[] $\to$ Depende positivamente de estrategias de otros
				\4[] $\then$ ``Si nadie produce/vende/baja precios, yo tampoco''
				\4[] Economías pueden quedarse atrapadas en desempleo
				\4[] $\to$ Aunque exista un equilibrio mejor
				\4 Ball y Romer (1991)
				\4[] Rigideces nominales son fallos de coordinación
				\4[] Incorpora instrumentos de NEK1G y 2G
				\4[] $\to$ Demanda à la Dixit-Stiglit
				\4[] $\to$ Equilibrio general
				\4[] Introduce agentes heterogéneos en NEK
				\4[] $\to$ Pionero en HANK
				\4[] Economía formada por dos empresas
				\4[] $\to$ Cada una provee inputs a la otra
				\4[] Shock de demanda negativo
				\4[] $\to$ Reducir precios es óptimo
				\4[] Ninguna quiere reducir unilateralmente primero
				\4[] $\to$ Beneficios caerían
				\4[] $\to$ Precios constantes aunque menos demanda
				\4[] $\then$ Ajuste en cantidades
				\4[] $\then$ Caída de empleo y output
				\4[] Si bajada coordinada de precios fuese posible
				\4[] $\to$ Saldos reales aumentan
				\4[] $\to$ Ambas empresas mejoran
				\4 Implicaciones
				\4[] Múltiples equilibrios
				\4[] Desempleo persistente posible
				\4[] $\to$ Resulta de problema de coordinación
				\4[] Equilibrios múltiples subóptimos son posibles
				\4 Valoración
				\4[] Dificil formular modelos tratables
				\4[] RBC aparece contemporáneamente
				\4[] Poca continuidad
				\4[] Enfoque de Lucas predominó
		\2 Implicaciones
			\3 Política económica
				\4 Política monetaria
				\4[] No es irrelevante
				\4[] No tiene por qué ser neutral
				\4 Activación de demanda
				\4[] Posible y a veces deseable
			\3 Teoría económica
				\4 HER y no-neutralidad del dinero
				\4[] Son compatibles
				\4[] Incumplimiento de dicotomía clásica
				\4[] $\to$ Sin ilusión monetaria
				\4[] $\to$ Rigideces nominales
				\4 Equilibrio parcial
				\4[] Utilizado en mayoría
				\4 Análisis cuantitativo
				\4[] Poco posible
				\4[] Modelos fundamentalmente teóricos y estáticos
				\4 Bases de segunda generación
				\4[] Necesario integrar en modelos de eq. general
	\1 \marcar{Segunda generación de NEK}
		\2 Visión general
			\3 Contexto económico
				\4 Fin de comunismo
				\4 Hegemonía de capitalismo
				\4 Mejora capacidad de procesamiento
				\4 Gran moderación
				\4[] Reducción de volatilidad del ciclo
				\4[] $\to$ Desde 80s hasta Gran Recesión
				\4 Independencia de Bancos Centrales
				\4[] Consolidada en países desarrollados
				\4 Tipos flexibles y movilidad de K creciente
			\3 Contexto teórico
				\4 Enfoque de equilibrio general
				\4[] Plenamente consolidado
				\4 Modelo del ciclo real
				\4[] Éxito metodológico generalizado
				\4[] $\to$ Artículos académicos
				\4[] $\to$ Programas de doctorado
				\4[] Pero sujeto a críticas en la práctica
				\4[] $\to$ Mala replicación de algunas correlaciones\footnote{Fundamentalmente, la correlación entre horas trabajadas y la productividad media del trabajo. Mientras que la realidad muestra una correlación ligeramente negativa (más trabajo al tiempo que menos productividad, luego productividad ligeramente contracíclica), el modelo RBC básico con calibraciones estándar muestra una correlación claramente positiva entre trabajo y productividad media, es decir, una evolución fuertemente pro-cíclica de la productividad. Esta correlación en el modelo RBC tiene sentido como resultado de la necesidad de relacionar shocks reales (que afectan a la productividad) con aumento del output.}
				\4[] $\to$ Dicotomía clásica
				\4[] $\to$ Economía siempre en óptimo de Pareto
				\4[] $\to$ Necesaria elasticidad-salario muy alta
				\4 Modelos keynesianos de primera generación
				\4[] Impacto limitado en análisis cuantitativo
				\4[] Sientan bases de de justificación de rigideces
				\4 Análisis de series temporales
				\4[] Sims (Nobel en 2011)
				\4[] Avances en análisis de series temporales
				\4[] $\to$ Identificar efectos de shocks exógenos
				\4[] $\to$ Avances en identificación de causalidad
				\4[] Aplicación a shocks nominales
				\4[] $\to$ ¿Qué efectos sobre reales?
			\3 Autores
				\4 Galí, Clarida, Gertler, Evans, Rogoff
				\4 Mankiw, Christiano, Eichenbaum, Smets
				\4[] ...
			\3 Ideas centrales
				\4 Aceptación plena de metodología NMC/RBC
				\4[] Microfundamentación
				\4[] Marco walrasiano de eq. general
				\4[] Crítica de Lucas
				\4 Dicotomía clásica no se cumple
				\4[] Vars. nominales afectan vars. reales
				\4 Política monetaria
				\4[] Instrumento más importante de PEconómica
				\4[] Necesario y posible analizar regímenes de PM
				\4 Competencia imperfecta
				\4[] Empresas tienen poder de mercado
				\4 Rigideces reales y nominales
				\4[] Precios no son plenamente flexibles
				\4[] $\to$ Ni absolutos ni relativos
		\2 Modelo canónico de la NEK
			\3 Idea clave
				\4 Contexto
				\4[] Predecesores:
				\4[] $\to$ Dixit y Stiglitz (1977)
				\4[] $\to$ Blanchard y Kiyotaki (1987)
				\4[] $\to$ Rogoff y Obstfeld (1995)
				\4[] $\to$ Hairault y Portier
				\4[] RBC como modelo dominante
				\4[] Mantener simplicidad y capacidad predictiva
				\4[] Debate sobre efectos reales de shocks nominales
				\4[] $\to$ Existen, pero ¿caracterizables?
				\4[] $\to$ ¿Posible formular modelos que representen?
				\4 Objetivo
				\4[] Caracterizar efecto de diferentes shocks
				\4[] $\to$ Oferta/Productividad
				\4[] $\to$ Nominales/Monetarios
				\4[] $\to$ Demanda/Preferencias
				\4[] Análisis cuantitativo
				\4[] $\to$ No sólo cuantitativo
				\4[] $\then$ De forma similar a RBC
				\4 Resultados
				\4[] Equilibrio es secuencia de:
				\4[] $\to$ Demanda y oferta de bienes
				\4[] $\to$ Nivel de precios e inflación
				\4[] $\to$ Demanda y oferta de trabajo
				\4[] $\to$ Salario real
				\4[] $\to$ Interés nominal
				\4[] $\to$ Oferta
				\4[] Análisis positivo: respuestas a shock
				\4[] $\to$ Nominales
				\4[] $\to$ Oferta
				\4[] $\to$ Demanda
				\4[] $\then$ Fluctuaciones en torno a EE
				\4[] Análisis normativo
				\4[] $\to$ Comparar secuencia con óptimo de Pareto
			\3 Formulación
				\4 Consumidores
				\4[] Deciden secuencias de consumo y trabajo
				\4[] $\to$ Consumo y trabajo separables
				\4[] Demandan variedades $i \, \in \, [0,1]$
				\4[] $\to$ Demanda à la Dixit-Stiglitz
				\4[] $\then$ Preferencia por variedad simétricas
				\4[] $\then$ Sustituibilidad imperfecta entre variedades
				\4[] $\then$ Empresas tienen poder de mercado
				\4[] Ofrecen trabajo a cambio de salario real
				\4[] $\to$ Trabajo reduce utilidad
				\4[] Shocks aleatorios de preferencias
				\4[] $\to$ Modulan preferencia por consumo presente
				\4[] $\to$ Capturan ``animal spirits'' hasta cierto punto
				\4[] Bono disponible
				\4[] $\to$ Permite transferencia intertemporal
				\4 Empresas
				\4[] Propiedad de consumidores
				\4[] Producen variedades $i \, \in \, [0,1]$
				\4[] Emplean trabajo con R $\downarrow$ E
				\4[] $\to$ Costes marginales crecientes
				\4[] Shocks aleatorios de productividad
				\4[] $\to$ Generales a todas las empresas
				\4[] $\to$ Generalmente, proceso AR(1)
				\4[] $\then$ Afecta a producto agregado
				\4[] Cada empresa enfrenta demanda decreciente
				\4[] $\to$ Poder de mercado
				\4[] $\then$ Fija precio de monopolista
				\4[] $\then$ Mark-up dependiente de elasticidad demanda
				\4[] $\then$ Más elasticidad implica menor mark-up
				\4 Precios à la Calvo
				\4[] En cada periodo:
				\4[] $\to$ $\theta$ no pueden cambiar precio
				\4[] $\to$ $1-\theta$ pueden cambiar precio
				\4[] Cuando pueden, fijan precios considerando:
				\4[] $\to$ Mark-up deseado
				\4[] $\to$ Coste marginal futuro que esperan
				\4[] $\to$ Nivel de precios futuro que esperan
				\4[] $\to$ Probabilidad de poder cambiar en futuro
				\4[] $\to$ Descuento subjetivo de accionistas
				\4[] $\then$ Ajustan mark-up al deseado
				\4[] $\then$ Output se ajusta a output natural
				\4[] Cuando no pueden ajustar precios y hay shocks:
				\4[] $\to$ Demandas cambian
				\4[] $\to$ Coste marginal cambia
				\4[] $\then$ Mark-up se desvía de mark-up deseado
				\4[] $\then$ Output se desvía de output deseado
				\4 Estado estacionario resumible en 4 ecuaciones
				\4[DIS] IS dinámica
				\4[] \fbox{$\tilde{y}_t = \textrm{E}_t \left\lbrace \tilde{y}_{t+1} \right\rbrace - \frac{1}{\sigma} \left( \underbrace{i_t - \textrm{E}_t \left\lbrace \pi_{t+1} \right\rbrace}_{r_t} - r^n_t \right) $}
				\4[NKPC] Curva de Phillips Neo-Keynesiana
				\4[] \fbox{$\pi_t = \text{E}_t \left\lbrace \pi_{t+1} \right\rbrace + \textsc{k} \tilde{y}_t $}
				\4[WS] Mercado de trabajo
				\4[] \fbox{$w_t - p_t = \sigma c_t + \phi n_t$}
				\4[MP] Mercado de dinero
				\4[] \fbox{$m_t - p_t = y_t - \eta i_t$}
				\4[TR] Regla de Taylor simple
				\4[] \fbox{$i_t = \rho + \phi_\pi \pi_t + \phi_y \tilde{y}_t + v_t $}
			\3 Implicaciones
				\4 Optimalidad
				\4[] Fuentes de desviación posible
				\4[] $\to$ Desviación de precios+rigideces nominales
				\4[] $\to$ Rigideces reales en mercado de trabajo
				\4[] $\to$ Fricciones en mercado de trabajo
				\4 Curva de Phillips $\pi_t$ -- $y_t$ creciente
				\4[] Con HER y equilibrio general
				\4[] No es resultado de inf. imperfecta
				\4[] $\to$ A diferencia de Lucas (1972)
				\4[] Rigidez nominal es factor clave
				\4 Dicotomía clásica se rompe
				\4[] Variables nominales determinan reales
				\4[] $\to$ No se determinan por separado
				\4 Ajuste en cantidades
				\4[] Consecuencia de:
				\4[] $\to$ Competencia monopolística
				\4[] $\to$ Rigidez de precios
				\4 Equilibrios subóptimos de Pareto
				\4[] Consecuencia de:
				\4[] $\to$ Poder de mercado
				\4[] $\to$ Rigidez de precios
				\4 Análisis normativo es posible
				\4[] Bienestar de consumidores explicitado
				\4 Regímenes de política monetaria
				\4[] Determinan senda de equilibrio
				\4[] Alternativas básicas:
				\4[] $\to$ Regla de Taylor
				\4[] $\to$ oferta monetaria exógena
				\4[] Análisis positivo y normativo de reglas de PM
				\4[] $\to$ Ambos son posibles
			\3 Regla de Taylor
				\4 Idea clave
				\4[] Contexto
				\4[] $\to$ Modelos NEK de segunda generación
				\4[] $\to$ Rigideces nominales y reales
				\4[] $\to$ HER
				\4[] Objetivo
				\4[] $\to$ Caracterizar PM que estabiliza inflación
				\4[] $\to$ Evitar análisis normativo de inflación óptima
				\4[] $\to$ Comparar con PM llevadas a cabo por bancos centrales
				\4 Formulación
				\4[] $i = r^* + \pi^* + \phi_t (\pi - \pi^*) + \phi_y (y_t - y_t^n)$
				\4[] Donde:
				\4[] $\to$ $i$: tipo de interés nominal de intervención
				\4[] $\to$ $r^*$: tipo de interés real de equilibrio l/p
				\4[] $\to$ $\pi$: inflación del periodo
				\4[] $\to$ $\pi^*$: inflación objetivo del banco central
				\4[] $\to$ $y$: output gap
				\4[] Regla original de Taylor (1993)
				\4[] $\to$ $i = r^* + \pi^* + 1.5(\pi - \pi^*) + 0.5 y$
				\4 Implicaciones
				\4[] Estabilización implica sobrerreacción a inflación
				\4[] $\to$ Cuando no tiene lugar, inflación y volatilidad macro
				\4[] $\then$ Años 70
				\4[] $\to$ En gran moderación entre 80s y primeros 2000s
				\4[] $\then$ Se cumple aproximadamente
				\4[] $\then$ Baja volatilidad
				\4[] $\to$ Entre 2000s y GCF
				\4[] $\then$ no se cumple
				\4[] $\then$ Aumento de volatilidad
				\4[] $\then$ Conclusión discutida empíricamente
				\4[] Ecuación de Fisher
				\4[] $\to$ Cuando inflación no se desvía de objetivo
				\4[] $\to$ Cuando output no se desvía de natural
				\4[] $\then$ $i=r^* + \pi^*$
				\4[] $\then$ Interés nominal y real, inflación estacionarias
				\4[] Diferentes etapas de política monetaria
				\4[] $\to$ Años 70
				\4[] $\then$ Coeficiente por debajo de 1.5
				\4 Valoración
				\4[] Interés real natural inobservable
				\4[] $\to$ ¿Cómo estimar?
				\4[] $\then$ ¿Política acomodaticia o $r_n^t$ más bajo?
				\4[] Evidencia empírica mixta
				\4[] $\to$ Estabilidad con regla de Taylor
				\4[] $\to$ Estabilidad sin regla de Taylor
				\4[] $\to$ ...
			\3 Extensiones
				\4 Rigideces en salarios
				\4 Desepleo
				\4 Economía abierta: NOEM
				\4 HANK -- Heterogeneous Agents
				\4 Política monetaria óptima
				\4[] Commitment
				\4[] $\to$ Puede comprometerse a implementar regla óptima
				\4[] Sin commitment
				\4[] $\to$ No puede comprometerse a implementar regla óptima
				\4[] $\to$ Implementa en cada periodo
			\3 Valoración
				\4 Buena replicación de primeros momentos
				\4 Carencias graves en modelo básico
				\4[] Sin inercia inflacionaria
				\4[] $\to$ Inflación
				\4[] $\then$ Contradice evidencia empírica
				\4[] $\then$ Necesarias rigideces reales
				\4[] Empleo y salario contracíclicos
				\4[] $\to$ En calibraciones estándar
				\4[] $\to$ Shock tec. reduce empleo y salario
				\4[] $\to$ Contrario evidencia generalizada
				\4[] Crisis financieras
				\4[] $\to$ No explicadas en modelo básico
				\4 Extensiones necesarias
				\4[] Rigideces reales endógenas
				\4[] $\to$ Mejor representación de trabajo y desempleo
				\4[] Economía abierta
				\4[] $\to$ Efecto de tipo de cambio
				\4[] $\to$ Precios internacionales dados
				\4[] $\to$ Activos financieros internacionales
				\4[] Agentes heterogéneos
				\4[] $\to$ Agregación introduce problemas relevantes
				\4[] Fricciones financieras
				\4[] $\to$ ¿Mercados financieros son completos?
				\4[] $\to$ ¿Transmisión de PM es perfecta?
		\2 Implicaciones globales
			\3 Política económica
				\4 Política monetaria
				\4[] Reglas preferibles a discrecionalidad
				\4[] Justificación explícita
				\4[] $\to$ No requiere de f. de pérdida ad-hoc
				\4[] $\to$ Optimización de utilidad de agente
				\4[] Tipo de interés no es único instrumento
				\4[] $\to$ Expectativas son fundamentales
				\4 Políticas de demanda
				\4[] Pueden ser necesarias
				\4[] $\to$ ZLB
				\4[] $\then$ No es posible utilizar PM
				\4[] $\then$ Tipos de interés se acercan a cero
				\4[] $\then$ Pero shock de oferta negativo
				\4[] Economía no utiliza plena capacidad
				\4[] Política monetaria es instrumento principal
				\4[] $\to$ Estabilizar expectativas
				\4[] $\to$ Estabilizar frente a shocks reales
				\4[] $\then$ Ocasionalmente denominado "Nuevo monetarismo"
				\4 Modelo de Smets y Wouters (2002)
				\4[] Modelo DSGE para zona euro
				\4[] Uso pionero de DSGE en bancos centrales
				\4[] Consolida uso de enfoque NEK-DSGE
			\3 Teoría económica
				\4 Nueva Síntesis Neoclásica
				\4[] Combina aspectos de NMC y NEK 1ª generación
				\4[] Mantiene de RBC:
				\4[] $\to$ Marco general walrasiano
				\4[] $\to$ Equilibrio general, dinámico, estocástico
				\4[] $\to$ Principios lucasianos
				\4[] Abandona de RBC:
				\4[] $\to$ Competencia perfecta
				\4[] $\to$ Primer y segundo teorema del bienestar
				\4[] $\to$ Flexibilidad perfecta
				\4[] $\to$ Neutralidad del dinero
				\4[] Mantiene de primera generación de NEK
				\4[] $\to$ Competencia imperfecta
				\4[] $\to$ Rigidez nominal y real
				\4[] $\to$ Abandono de dicotomía clásica
				\4[] Rechaza de primera generación de NEK
				\4[] $\to$ Dicotomía l/p vs. c/p
				\4[] $\to$ Concepto de desempleo involuntario
				\4 Política monetaria es objetivo principal
				\4[] Tras relativo abandono con RBC
				\4 Normalización de competencia imperfecta
				\4[] Pieza clave de rigidez de precios
				\4[] Uso limitado de competencia perfecta
	\1[] \marcar{Conclusión}
		\2 Recapitulación
			\3 Primera generación de NEK
			\3 Segunda generación de NEK
		\2 Idea final
			\3 Robert Solow sobre modelos macro y economistas
				\4 Existen dos tipos de macroeconomistas
				\4 Macroeconomistas que formulan modelo canónico
				\4[] Y tratan de resolver todas las preguntas con el
				\4[] $\to$ Aplicando ligeros cambios
				\4 Macroeconomistas que utilizan un conjunto de modelos
				\4[] Cada uno para resolver diferentes cuestiones
				\4 Primera Generación
				\4[] Múltiples modelos de equilibrio parcial
				\4[] Fundamentar existencia de rigideces
				\4[] Sin modelo general
				\4 Segunda generación
				\4[] Construir modelo general de la macroeconomía
				\4[] Vuelta a enfoque integrador general
			\3 Críticas a la hegemonía actual
				\4 Hegemonía de DSGE de NEK
				\4[] En programas de doctorado
				\4[] En modelos de bancos centrales
				\4[] $\to$ A partir de Smets y Wouters (2002)
				\4 Principales objetos de crítica
				\4[] Complejidad excesiva
				\4[] Difícil aplicación a diseño de políticas
				\4[] No explica crisis financiera
			\3 Estado actual de la NEK
				\4 Galí (2018)
				\4 Múltiples oportunidades de mejora
				\4 Análisis de la ZLB
				\4[] ¿Cuando pol. fiscal es más efectiva?
				\4[] ¿Es posible sólo por shocks de expectativas?
				\4 Desarrollar modelos HANK tratables
				\4[] Frontera de investigación más activa
				\4 Generaciones solapadas
				\4[] Análisis de burbujas racionales
				\4[] Sendas de interés real negativo
			\3 Crisis financieras
				\4 Punto relativamente débil
				\4 ¿Por qué se producen?
				\4[] Acumulación de desequilibrios
				\4[] $\to$ Modelos DSGE actuales no explican
				\4[] $\to$ Representar con shocks muy fuertes
				\4[] $\then$ Poco convicente
				\4 Incorporación de métodos y resultados
				\4[] $\to$ Múltiples equilibrios
				\4[] $\to$ Procesos no lineales
				\4[] $\to$ Fricciones financieras
				\4[] $\to$ Mecanismos endógenos de propagación
\end{esquemal}




\graficas

\begin{axis}{4}{Representación gráfica de la Non-Shirking Condition y su efecto sobre el desempleo}{}{$w$}{salariodeeficiencia}
	% Extensión del eje de abscisas
	\draw[-] (4,0) -- (5,0);
	
	% NSC
	\draw[-] (0,1) to [out=0, in=260](3,2.5) -- (3.2,4);
	\node[left] at (3.2,4){\small NSC};
	
	% Oferta de trabajo
	\draw[-] (3.8,0) -- (3.8,4);
	\node[below] at (3.8,0){$\bar{L}$};
	
	% Demanda de trabajo
	\draw[-] (0,3) -- (5,1);
	\node[right] at (5,1){$D^L$};
	
	% Salario de eficiencia y demanda de trabajo correspondiente
	\draw[dashed] (0,1.9) -- (2.75,1.9) -- (2.75,0);
	\node[below] at (2.75,0){$L^E$};
	\node[left] at (0,1.9){$w^E$};
	
	% Salario y trabajo de equilibrio si salario vacía mercado
	\draw[dashed] (0,1.46) -- (3.8,1.46);
	\node[left] at (0,1.46){$w^*$};
\end{axis}

\begin{axis}{4}{Representación gráfica de los beneficios de segundo orden en modelo con competencia monopolística y costes de menú}{$P$}{$\pi$}{costesdemenu}
	% Isobeneficio inicial
	\draw[-] (1,0.3) to [out=70,in=180](2,2) to [out=0,in=110](3,0.3);
	\node[below] at (3,0.3){\tiny $\pi_0$};
	
	% Beneficio óptimo inicial
	\draw[dashed] (0,2) -- (2,2) -- (2,0);
	\node[below] at (2,0){\tiny $p_0$};

	% Isobeneficio final
	\draw[-] (0.7,0.3) to[out=70,in=180](2.3,3.5) to [out=0, in=110](3.7,0.3);
	\node[below] at (3.7,0.3){\tiny $\pi_1$};
	
	% Beneficio óptimo final
	\draw[dashed] (0, 3.5) -- (2.3,3.5) -- (2.3,0);
	\draw[-] (-0.2,3.5) -- (2.3,3.5);
	\node[below] at (2.3,0){\tiny $p_1$};
	
	% Beneficio subóptimo final
	\draw[dotted] (2,2) -- (2,3.45);
	\draw[-] (-0.2,3.47) -- (1.97,3.46);
\end{axis}

El gráfico muestra como un desplazamiento de la demanda puede requerir una variación del precio óptimo que apenas aumenta el beneficio. Por ello, si el cambio en el precio es costoso puede que la empresa no considere rentable el cambio y mantenga el inicial. 


\begin{dibujo}{4}{Representación gráfica del modelo de Carlin y Soskice (1990) basado en Layard y Nickell (1985)}{}{}{carlinsoskice}
	
	
	%%%%%%%%%%%%%%%%%%%%%%%%%%%%%%%%%%%%%%
	%%%%%%%%%%%%%%%%%%%%%%%%%%%%%%%%%%%%%%
	%%% EJES SUPERIORES: BRW Y PRW
	\draw[-] (0,4) -- (0,0) --(6,0);
	\node[left] at (0,4){$\frac{W}{P}$};
	\node[below] at (6,0){N};
	
	% BRW
	\draw[-] (0.5,0.5) -- (6,3);
	\node[right] at (6,3){\small BRW};
	
	% PRW
	\draw[-] (0,1.5) -- (6,1.5);
	\node[right] at (6,1.5){\small PRW};
	
	% Equilibrio sin aceleración de la inflación
	\draw[dashed] (2.75,1.5) -- (2.75,-4);
	\node[below] at (2.5,0){$N^*$};
	
	% Desviación de salario de equilibrio
	\draw[decorate,decoration={brace,amplitude=5pt},xshift=-2pt,yshift=0pt] (4,2.05) -- (4,1.5)  node[black,midway,xshift=0.6cm, yshift=0pt] {\footnotesize y \%};

	% Desviación de empleo de equilibrio	
	\draw[decorate,decoration={brace,amplitude=5pt},xshift=-2pt,yshift=0pt] (4,0) -- (2.85,0)   node[black,midway,xshift=0cm, yshift=-10pt] {\footnotesize x \%};
	
	%%%%%%%%%%%%%%%%%%%%%%%%%%%%%%%%%%%%%%
	%%%%%%%%%%%%%%%%%%%%%%%%%%%%%%%%%%%%%%
	%%% EJES INFERIORES: CURVA DE PHILLIPS
	\draw[-] (5,4) -- (5,-4) -- (0,-4);
	\node[left] at (0,-4){\small U};
	
	% Curva de Phillips de l/p
	\draw[-] (2.75,-4) -- (2.75,-1);
	\node[below] at (2.72,-4.1){\small $U^*$};
	\node[right] at (5,-1){$\pi$};
	
	% Curva de Phillips 1 de c/p
	\draw[-] (0.1,-5.5) -- (4.5,-3);
	
	% Curva de Phillips 2 de c/p
	\draw[-] (0.1,-4.5) -- (4.5,-2);
	
	% Curva de Phillips 3 de c/p
	\draw[-] (0.1,-3.5) -- (4.5,-1);

	% NAIRU
	\draw[decorate,decoration={brace,amplitude=5pt},xshift=-2pt,yshift=0pt] (5,-4.52) -- (2.72,-4.52)  node[black,midway,xshift=-0cm, yshift=-10pt] {\footnotesize NAIRU};	

	% Aumento de la inflación
	\draw[decorate,decoration={brace,amplitude=5pt},xshift=-2pt,yshift=0pt] (4,-3.32) -- (4,-4)  node[black,midway,xshift=0.4cm, yshift=1pt] {\footnotesize z \%};		
	
\end{dibujo}

La gráfica muestra como una desviación $x \%$ respecto del empleo de equilibrio $N^* \%$ provoca una divergencia entre las exigencias de los sindicatos y el salario que desean pagar las empresas del $y \%$. Esto provoca a su vez una aceleración de la inflación de $z \%$.

\preguntas

\seccion{13 de marzo 2017}

\begin{itemize}
    \item La mayoría de modelos que usted ha cantado son de equilibrio parcial, y microeconómicos. Tal y como usted ha cantado el tema, no parece un tema de macroeconomía. ¿Por qué no ha contado usted un modelo macroeconómico?
    \item ¿Hay algún modelo de equilibrio general relevante en el marco de la NEK que nos permita entender e interpretar la economía a nivel macro?
    \item Según Bernanke, los mercados financieros pueden provocar crisis y amplificarlas. Trump acaba de derogar parcialmente la ley Dodd-Frank, porque afirma que hemos superado la crisis. ¿Qué opinión le merece esta derogación?
    \item Ha hablado usted de situaciones de racionamiento de crédito. ¿Conoce usted la Teoría de la Cuerda?
\end{itemize}

\seccion{Test 2015}
\textbf{14.} Señale la respuesta \textbf{\underline{falsa}} con respecto a la críticas realizadas a la hipótesis de las expectativas racionales:
\begin{itemize}
	\item[a] Los agentes económicos no son racionales en algunas ocasiones ya que pueden juzgar la probabilidad de un suceso futuro por su similitud con eventos recientes y no teniendo en cuenta toda la información disponible.
	\item[b] No es aplicable a situaciones en las que no existe una distribución objetiva de probabilidades a la que referirse.
	\item[c] No es verificable independientemente del modelo.
	\item[d] Los agentes cometen errores sistemáticos en la formación de expectativas.
\end{itemize}

\textbf{15.} Señale la respuesta correcta respecto al modelo la nueva economía keynesiana sobre la tasa de desempleo NAIRU:
\begin{itemize}
	\item[a] Es la tasa de desempleo para la cual la inflación es 0.
	\item[b] Una adecuada política monetaria puede reducir a largo plazo la NAIRU de una economía.
	\item[c] Cuando la tasa de desempleo de la economía se sitúa en el nivel de la NAIRU, las expectativas de inflación de los trabajadores son correctas.
	\item[d] El modelo asume mercados de competencia perfecta.
\end{itemize}

\seccion{Test 2008}
\textbf{18.} Sea la curva de Phillips de una economía:

\begin{equation}
\pi_t - \pi_t^e = 0,24 - 4u_t
\end{equation}

Siendo $\pi_t$ la inflación real, $\pi_t^e$ la inflación esperada y $u_t$ el paro. En esta situación, la tasa natural de paro del país sería:

\begin{itemize}
	\item[a] 8\%
	\item[b] 2\%
	\item[c] 6\%
	\item[d] Ninguna de las anteriores.
\end{itemize}

\seccion{Test 2004}
\textbf{14.} La rigidez de precios puede ser naturalmente el resultado del comportamiento óptimo de empresas que operan bajo competencia imperfecta en el mercado de bienes cuando:

\begin{itemize}
	\item[a] El valor esperado del beneficio obtenido al ajustar únicamente la cantidad producida sea mayor que el valor del beneficio esperado en dicha circunstancia para empresarios aversos al riesgo.
	\item[b] Las empresas se enfrentan a costes de menú y los cambios en la demanda tienen carácter transitorio.
	\item[c] El valor esperado del beneficio sea menor que el valor del beneficio esperado al ajustar precios para empresarios aversos al riesgo.
	\item[d] Las empresas se enfrenten a costes de menú y sus costes marginales sean crecientes.
\end{itemize}

\notas

\textbf{2015} \textbf{14.} D \textbf{15.} C

\textbf{2008} \textbf{18.} C

\textbf{2004} \textbf{14.} B

\bibliografia

Mirar en Palgrave:
\begin{itemize}
	\item cyclical markups
	\item involuntary unemployment
	\item IS-LM *
	\item IS-LM in modern macro *
	\item liquidity trap
	\item microfoundations
	\item monetary business cycles (imperfect information) 
	\item monetary business cycle models (sticky prices) *
	\item money supply
	\item monetary transmission mechanism
	\item natural rate of unemployment
	\item new keynesian economics *
	\item Phillips curve
	\item Phillips curve (new views)
	\item real business cycles
	\item real rigidities
	\item sticky wages and staggered wage setting *
	\item Taylor rules *
\end{itemize}

Ball, L.; Romer, D. (1991) \textit{Sticky prices and coordination failure} The American Economic Review. Vol. 81, No. 3 -- En carpeta del tema

Blanchard, O. \textit{On the future of macroeconomic models} (2018) Oxford Review of Economic Policy -- En carpeta del tema. Número completo en: \url{https://academic.oup.com/oxrep/issue/34/1-2}

Calvo, G. A. \textit{Staggered Prices in a Utility-Maximizing Framework} (1983) -- En carpeta del tema

Christiano, L. J.; Eichenbaum, M. S.; Trabandt, M. \textit{On DSGE Models} (2018) Journal of Economic Perspectives: Summer 2018 -- En carpeta del tema

De Vroey, M. \textit{A History of Macroeconomics: Keynes to Lucas and Beyond} (2016)

Galí, J. \textit{The State of New Keynesian Economics: A Partial Assessment} (2018) Journal of Economic Perspectives: Summer -- En carpeta del tema

Galí, J.; Gertler, M. \textit{Macroeconomic Modeling for Monetary Policy Evaluation} (2007) Journal of Economic Perspectives: Fall 2007 -- En carpeta del tema

Gordon, R. J. \textit{What is New-Keynesian Economics?} (1990) Journal of Economic Literature -- En carpeta del tema

Heijdra, B. J. \textit{Foundations of Modern 
Macroeconomics} (2017) 3rd ed. -- En carpeta Macro

King, R. G. \textit{Will the New Keynesian Macroeconomics Resurrect the IS-LM Model} (1993) Journal of Economic Perspectives: winter -- En carpeta del tema

Layard, R.; Nickell, R. \textit{The Causes of British Unemployment} (1985) National Institute Economic Review -- En carpeta del tema

Mankiw, G. N. \textit{Real Business Cycles: A New Keynesian Perspective} (1989) Journal of Economic Perspectives: summer -- En carpeta del tema

Smets, F.; Wouters, R. \textit{An Estimated Stochastic Dynamic General Equilibrium Model of the Euro Area} (2002) ECB Working Paper Series -- En carpeta del tema

Taylor, J. B.\textit{Staggered wage setting in a macro model} (1979) 

Vines, D.; Wills, S. \textit{The rebuilding macroeconomic theory project: an analytical assessment} (2018) Oxford Review of Economic Policy -- En carpeta del tema. Número completo en: \url{https://academic.oup.com/oxrep/issue/34/1-2}

\end{document}
