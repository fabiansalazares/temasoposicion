\documentclass{nuevotema}

\tema{3B-45}
\titulo{Funcionamiento y evolución del Sistema Monetario Europeo. El euro: origen y evolución.}

\begin{document}

\ideaclave

IMPORTANTE Leer \href{https://www.bde.es/f/webbde/SES/Secciones/Publicaciones/PublicacionesSeriadas/DocumentosOcasionales/20/Fich/do2021.pdf}{Banco de España (2020): Endeudamiento y necesidades de financiación en la Unión Europea}.

Hacer cronología de la crisis del Euro con \href{https://www.ucm.es/data/cont/docs/518-2015-04-15-TIMELINE%20OCT%202009%20MAYO%202013_ampliado.pdf}{Borrell (2015) Euro Crisis Timeline}.

Ver \href{https://www.consilium.europa.eu/en/press/press-releases/2020/04/09/report-on-the-comprehensive-economic-policy-response-to-the-covid-19-pandemic/}{Comunicado del Eurogrupo del 10 de abril de 2020} sobre medidas a adoptar al respecto de la crisis del COVID-19.

Ver \url{http://blognewdeal.com/andrea-lucai/una-funcion-de-estabilizacion-fiscal-para-la-ue/} sobre función de estabilización fiscal a nivel europeo. 

Ver págs. 23-24 de IMF (2019) ESR sobre ajuste de desequilibrios externos e intra-zona euro tras la Crisis Financiera Global.

La zona euro es, como su propio nombre indica, una unión monetaria formada en la actualidad por 19 de los 28 países miembros de la Unión Europea. Los orígenes de la zona euro se encuentran en la inestabilidad cambiaria y las tensiones que precedieron a la caída de Bretton Woods en 1973. Tras una etapa de relativa estabilidad en los años 80, la llegada de los 90 mostraron la necesidad de bien una mayor integración real y nominal, bien una ruptura del sistema europeo de tipo de cambio fijos. El impulso integrador que inició la firma del Acta Única y que continuó el Tratado de Maastricht recuperaron la idea de la Unión Monetaria y Económica. En 1999 y 2002 se inicia esta etapa, con la consecución plena de la Unión Monetaria y la circulación de la moneda única.

El avance de la integración monetaria ha estado ligado a la evolución de la economía mundial (precios del petróleo, déficits balanza de pagos en EEUU, convergencia periferia europea...) así como a la evolución de la teoría económica. Desde que Mundell publicase en 1961 su artículo seminal sobre las áreas monetarias óptimas -que daba forma a ideas anteriores de Friedman y otros-, el debate ha sido intenso. Existen costes y beneficios de un área monetaria, que se resumen básicamente en las diferencias reales entre economías, la flexibilidad para realizar ajustes nominales y reales, y la posibilidad de realizar transferencias fiscales. La zona euro cumple algunos estos requisitos de forma desigual y por ello tanto el balance de su existencia como su permanencia en el futuro están sujetos a controversias e incertidumbres.

Este tema está relacionado con:
\begin{itemize}
    \item Tema Integración Monetaria
    \item Tema Integración Económica
    \item Tema Crisis monetarias y financieras
    \item Tema Bretton Woods
\end{itemize}

\seccion{Preguntas clave}
\begin{itemize}
	\item ¿Qué razones para la integración monetaria en Europa?
	\item ¿Qué fue el Sistema Monetario Europeo?
	\item ¿Qué evolución ha seguido hasta la actualidad?
	\item ¿Qué balance de la Unión Económica y Monetaria?
	\item ¿Qué propuestas de reforma?
\end{itemize}

\esquemacorto

\begin{esquema}[enumerate]
	\1[] \marcar{Introducción}
		\2 Contextualización
			\3 Unión Europea
			\3 Competencias de la UE
			\3 Integración monetaria
			\3 Proceso hasta el euro
		\2 Objeto
			\3 ¿Qué razones para la integración monetaria en Europa?
			\3 ¿Qué fue el Sistema Monetario Europeo?
			\3 ¿Qué evolución ha seguido hasta la actualidad?
			\3 ¿Qué balance de la Unión Económica y Monetaria?
			\3 ¿Qué propuestas de reforma?
		\2 Estructura
			\3 Antecedentes
			\3 Sistema Monetario Europeo
			\3 Unión Económica y Monetaria
			\3 El Euro
	\1 \marcar{Antecedentes}
		\2 Antes de Bretton Woods
			\3 Bimetalismo
			\3 Unión Monetaria Latina
			\3 Patrón Oro convencional
			\3 Entreguerras
		\2 Bretton Woods y desintegración del sistema
			\3 OEEC -- Organización Europea para la Cooperación Económica
			\3 Unión Europea de Pagos
			\3 Tratado de París 1951: CECA
			\3 Tratado de Roma de 1957: CEE y EURATOM
			\3 Gold Pool (1961)
			\3 Informe Werner (1970)
			\3 Desintegración de Bretton Woods
			\3 Acuerdos de Smithsonian (1971)
			\3 Serpiente en el túnel
			\3 Serpiente en el lago
	\1 \marcar{Sistema Monetario Europeo}
		\2 Contexto
			\3 Caída de Bretton Woods
			\3 Inestabilidad cambiaria
			\3 PAC principal política europea
			\3 Unión Aduanera implementada
			\3 Informe Werner
		\2 Eventos
			\3 Creación del SME (1979)
			\3 ERM -- Exchange Rate Mechanism
			\3 Estabilidad en los 80
			\3 SME ``Duro'' y Acta Única de 1987
			\3 Informe Delors y Tratado de Maastricht
			\3 Implementación del Euro
			\3 Crisis del SME en el 92
		\2 Consecuencias
			\3 Relativo éxito del SME
			\3 Precedente de integración
			\3 Aumento de la flexibilidad
	\1 \marcar{Unión Económica y Monetaria}
		\2 Contexto
			\3 Acta Única en proceso de implementación
			\3 Sin realineación de tipos
		\2 Eventos
			\3 Informe Delors y Tratado de Maastricht
			\3 Implementación del Euro
		\2 Consecuencias
			\3 Éxito técnico
			\3 Convergencia
			\3 Moneda de reserva
	\1 \marcar{El Euro}
		\2 Elementos centrales
			\3 Eurosistema
			\3 Política monetaria
			\3 Gobernanza económica
			\3 ESFS -- Sistema Europeo de Supervisión Financiera
			\3 Unión Bancaria
			\3 Mercado Único
			\3 ESM -- Mecanismo Europeo de Estabilidad
			\3 ERM II
		\2 Balance del euro
			\3 Debates previos
			\3 Actualidad
			\3 Crisis de 2020
		\2 Propuestas de reforma
			\3 Trilema de la Unión Monetaria de Pisany-Ferry
			\3 Trilema de Rodrik
			\3 Informe de los 5 presidentes
			\3 Fondo de Garantía de Depósitos
			\3 Fondo Monetario Europeo
			\3 EISF -- European Investment Stabilisation Function
	\1[] \marcar{Conclusión}
		\2 Recapitulación
			\3 Antecedentes
			\3 Sistema Monetario Europeo
			\3 Unión Económica y Monetaria
			\3 El Euro
		\2 Idea final
			\3 Tensión integración-desintegración
			\3 Innovación financiera
			\3 Sistema monetario internacional
			\3 Interacción con política

\end{esquema}

\esquemalargo

\begin{esquemal}
	\1[] \marcar{Introducción}
		\2 Contextualización
			\3 Unión Europea
				\4 Institución supranacional ad-hoc
				\4[] Diferente de otras instituciones internacionales
				\4[] Medio camino entre:
				\4[] $\to$ Federación
				\4[] $\to$ Confederación
				\4[] $\to$ Alianza de estados-nación
				\4 Origen de la UE
				\4[] Tras dos guerras mundiales en tres décadas
				\4[] $\to$ Cientos de millones de muertos
				\4[] $\to$ Destrucción económica
				\4[] Marco de integración entre naciones y pueblos
				\4[] $\to$ Evitar nuevas guerras
				\4[] $\to$ Maximizar prosperidad económica
				\4[] $\to$ Frenar expansión soviética
				\4 Objetivos de la UE
				\4[] TUE -- Tratado de la Unión Europea
				\4[] $\to$ Primera versión: Maastricht 91 $\to$ 93
				\4[] $\to$ Última reforma: Lisboa 2007 $\to$ 2009
				\4[] Artículo 3
				\4[] $\to$ Promover la paz y el bienestar
				\4[] $\to$ Área de seguridad, paz y justicia s/ fronteras internas
				\4[] $\to$ Mercado interior
				\4[] $\to$ Crecimiento económico y estabilidad de precios
				\4[] $\to$ Economía social de mercado
				\4[] $\to$ Pleno empleo
				\4[] $\to$ Protección del medio ambiente
				\4[] $\to$ Diversidad cultural y lingüistica
				\4[] $\to$ Unión Económica y Monetaria con €
				\4[] $\to$ Promoción de valores europeos
				\4[$\to$] Objetivos de la UE
				\4[] Paz y bienestar a pueblos de Europa
			\3 Competencias de la UE
				\4 Tratado de la Unión Europea
				\4[] Atribución
				\4[] $\to$ Sólo las que estén atribuidas a la UE
				\4[] Subsidiariedad
				\4[] $\to$ Si no puede hacerse mejor por EEMM y regiones
				\4[] Proporcionalidad
				\4[] $\to$ Sólo en la medida de lo necesario para objetivos
				\4 Exclusivas
				\4[] i. Política comercial común
				\4[] ii. Política monetaria de la UEM
				\4[] iii. Unión Aduanera
				\4[] iv. Competencia para el mercado interior
				\4[] v. Conservación recursos biológicos en PPC
				\4 Compartidas
				\4[] i. Mercado interior
				\4[] ii. Política social
				\4[] iii. Cohesión económica, social y territorial
				\4[] iv. Agricultura y pesca \footnote{Salvo en lo relativo a la conservación de recursos biológicos marinos, que se trata de una competencia exclusiva de la UE}
				\4[] v. Medio ambiente
				\4[] vi. Protección del consumidor
				\4[] vii. Transporte
				\4[] viii. Redes Trans-Europeas
				\4[] ix. Energía
				\4[] x. Área de libertad, seguridad y justicia
				\4[] xi. Salud pública común en lo definido en TFUE
				\4 De apoyo
				\4[] Protección y mejora de la salud humana
				\4[] Industria
				\4[] Cultura
				\4[] Turismo
				\4[] Educación, formación profesional y juventud
				\4[] Protección civil
				\4[] Cooperación administrativa
				\4 Coordinación de políticas y competencias
				\4[] Política económica
				\4[] Políticas de empleo
				\4[] Política social
			\3 Integración monetaria
				\4 Proceso complejo
				\4[] Múltiples factores influyen
				\4[] No sólo factores económicos
				\4 Diferentes teorías
				\4 Coyuntura económica mundial es muy importante
				\4 Muchos fracasos a lo largo de historia
			\3 Proceso hasta el euro
				\4 Avances y retrocesos
				\4 Papel de Estados Unidos
				\4 Economía política interna de la UE
				\4 Otros shocks
				\4[] Crisis del petróleo
				\4[] Elecciones en EEMM
		\2 Objeto
			\3 ¿Qué razones para la integración monetaria en Europa?
			\3 ¿Qué fue el Sistema Monetario Europeo?
			\3 ¿Qué evolución ha seguido hasta la actualidad?
			\3 ¿Qué balance de la Unión Económica y Monetaria?
			\3 ¿Qué propuestas de reforma?
		\2 Estructura
			\3 Antecedentes
			\3 Sistema Monetario Europeo
			\3 Unión Económica y Monetaria
			\3 El Euro
	\1 \marcar{Antecedentes}
		\2 Antes de Bretton Woods
			\3 Bimetalismo
				\4 Plata principal metal en Europa
				\4 Varios países monometálicos en plata
				\4[] Imperio Austrohúngaro
				\4[] Alemania
				\4[] Rusia
				\4[] España
				\4 Francia bimetálico
				\4[] Ejerce de mecanismo equilibrador
			\3 Unión Monetaria Latina
				\4 Acuerdo de 1865 para estandarizar unidades
				\4[] Monedas de miembros son equivalentes
				\4[] Misma proporción de oro y plata
				\4[] Relación plata-oro 15,5:1
				\4[] $\then$ Circulación libre en toda la unión
				\4 Miembros
				\4[] Francia, Bélgica, Italia, Suiza
				\4[] $\to$ Posteriormente Grecia y otros
				\4 Problemas
				\4[] Países reducen cantidad de oro en monedas
				\4[] $\to$ Pero intercambiables a ROP fijo
				\4[] $\then$ Otros países se ven obligados a hacer lo mismo
				\4 Relativo éxito
				\4[] Monedas de la UML circulan en toda la Unión
				\4[] Fluctuaciones entre oro y plata
				\4[] Incentivos a freeriding y degradación de la moneda
				\4[] $\to$ Emitir monedas con menos metal de lo acordado
				\4[] $\then$ Financiación de déficits por vía monetaria
				\4 Especificaciones sobre:
				\4[] Oro y plata correspondiente a valor facial de monedas
				\4[] $\then$ ROP\footnote{Relación Oro-Plata}
				\4[$\then$] Causa de UML y desaparición
				\4[] Acuerdan UML para reducir problema de devaluaciones
				\4[] Devaluaciones unilaterales destruyen
				\4[] $\to$ Estados papales y Grecia especialmente
				\4[] Estallido de IGM
				\4[] $\to$ Fin de UML de facto
				\4[] 1927: fin de UML de iure
			\3 Patrón Oro convencional
				\4 Cooperación habitual entre BCentrales europeos
				\4 Francia, Inglaterra, Rusia, Alemania
				\4[] Préstamos para mantener convertibilidad
			\3 Entreguerras
				\4 Restauración de la convertibilidad en oro
				\4[] Progresiva
				\4[] Imperio británico: restablecimiento en 1925
				\4[] Incentiva competencia por TCN más barato
				\4[] $\to$ Mejorar competitividad relativa
				\4 Suspensiones sucesivas de convertibilidad
				\4[] 1. Austria, Alemania tras crisis bancaria (1931)
				\4[] 2. Inglaterra poco después (1931)
				\4[] 3. Estados Unidos (1933)
				\4[] 3. Francia, Bélgica, otros mantiene hasta 1936
				\4 Disrupción de comercio europeo
				\4[] Rearme arancelario
				\4[] Búsqueda de ``espacios vitales''
				\4[] $\to$ Áreas con las que comerciar
				\4[] $\to$ Importar materias primas
		\2 Bretton Woods y desintegración del sistema
			\3 OEEC -- Organización Europea para la Cooperación Económica
				\4 Creada en 1948
				\4 Surge de Plan Marshall
				\4[] Miembros europeos beneficiarios de ayuda
				\4 Conferencia de 16
				\4[] Establecer organización permanente
				\4[] Supervisar reconstrucción y asignación de Plan Marshall
				\4[] EEUU considera Plan Marshall mal utilizado
				\4[] $\to$ Dinero se destina a financiar cuenta corriente
				\4[] $\to$ Consideran insuficiente integración económica
				\4[] $\to$ Exigen más liberalización comercial
				\4 Catalizadora de liberalización
				\4[] Foro de negociación entre beneficiarios del PMarshall
				\4[] Influencia Tratado de París de 1950 (CECA)
			\3 Unión Europea de Pagos
				\4 Introducida en 1950
				\4 Bretton Woods/FMI en miniatura
				\4 Creada en marco de la OEEC
				\4 Objetivos
				\4[] Liberalización de cuenta corriente entre miembros
				\4[] $\to$ Reducción de barreras
				\4[] $\to$ Igualdad de trato entre miembros
				\4[] Créditos de divisas
				\4[] $\to$ Hasta agotar cuotas
				\4[] $\to$ Cuentas saldadas en oro
				\4[$\then$] Reducir inestabilidad cambiaria en Europa
				\4 Pagos canalizados vía Banco Internacional de Pagos
				\4 Sustituido por Acuerdo Monetario Europeo en 1955
				\4[] Administrado por OECE
			\3 Tratado de París 1951: CECA
				\4 Libre comercio de carbón y acero
				\4 Supone reducción adicional de restricciones CC
			\3 Tratado de Roma de 1957: CEE y EURATOM
				\4 Establece liberalización de movimientos de K
				\4[] Pero también muchas salvaguardias
				\4 Comité Monetario de la CEE
				\4[] Para mejorar coordinación de políticas monetarias
				\4[] Créditos a países con problemas de BP
			\3 Gold Pool (1961)
				\4 Acuerdo de cooperación Europa-EEUU
				\4 EEMM de CEE + EEUU + UK
				\4 Creada en 1961
				\4 Intervenciones conjuntas para mantener $\$35$/oz.
				\4 Cada país contribuye cantidad de oro
				\4[] Estados Unidos iguala contribuciones 1:1
				\4[] $\then$ Contribuye 50\%
				\4 Defiende hasta 1968
				\4[] Francia abandona gold pool
			\3 Informe Werner (1970)
				\4 Memorandum Barre (1969)
				\4[] Propone más coordinación económica y monetaria
				\4 Cumbre de la Haya (1969)
				\4[] Unión Económica y Monetaria objetivo oficial
				\4[] $\then$ Grupo de trabajo liderado por Pierre Werner
				\4 Dudas crecientes sobre el dólar
				\4[] $\$35$ por libra cada vez menos viable
				\4 Grupo liderado por Pierre Werner
				\4 Proyecto de unión monetaria en 10 años
				\4[] Sin moneda única
				\4[] $\to$ Pero monedas con TC fijo e irrevocable
				\4 Elementos característicos
				\4[] Convertibilidad total e irreversible
				\4[] Integración bancaria y financiera completa
				\4[] Fijación irrevocable de tipos de cambio
				\4[] $\to$ Sin margen alguno de fluctuación
				\4 Tres etapas
				\4[I] Armonización de política económica
				\4[] Concentración de política fiscal a nivel europeo
				\4[] Tres años
				\4[] Coordinación presupuestaria
				\4[] Supervisión de desequilibrios
				\4[] Mercado único de bienes y servicios
				\4[] Libre competencia
				\4[] Políticas desarrollo regional
				\4[II] Fijación de tipos y liberalización
				\4[] Bandas de fluctuación más estrechas
				\4[] Armonización acrecentada
				\4[III] Implementación plena
				\4[] Tipos irrevocables
				\4[] Plena convertibilidad
				\4[] $\then$ Unión monetaria
				\4 Federación de Bancos Centrales
				\4[] Sin Banco Central Europeo
				\4 Resultado
				\4[] Estados Unidos presiona contra implementación
				\4[] Francia acaba apoyando a EEUU
				\4 Inspiración de informe Delors en 1989
				\4[] Muchos elementos comunes
				\4[] Algunas lecciones no aprendidas
			\3 Desintegración de Bretton Woods
				\4 RFAlemana, Holanda abandonan Bretton Woods
				\4[] Dejan monedas flotar en 1970
				\4 Suiza redime dólares por oro
				\4 Nixon Shock el 15 de agosto de 1971
			\3 Acuerdos de Smithsonian (1971)
				\4 Diciembre de 1971
				\4 Acuerdo para bandas de flotación de principales divisas
				\4[] ``Túnel monetario''
				\4[] $\pm 2,25\%$ respecto a dólar
				\4[] $\then$ $4,5\%$ de diferencia entre monedas
				\4[] $\then$ $9\%$ de movimiento máximo
			\3 Serpiente en el túnel\footnote{Ver \href{https://www.elibrary.imf.org/view/IMF022/12444-9781616353124/12444-9781616353124/12444-9781616353124_A002.xml?language=en&redirect=true}{IMF (1973) Finance and Development: the Snake in the Tunnel}.}
				\4 Economías europeas
				\4[] Recesiones generalizadas
				\4[] Aumento de inflación
				\4[] $\to$ Aumento voces por PM contractiva y disciplina
				\4 Margen de flotación en Smithsonian
				\4[] Juzgado demasiado elevado
				\4[] Alemania rechaza exceso de volatilidad cambiaria
				\4[] Francia apoya
				\4 Acuerdo de 1972 gobernadores BCNs
				\4[] Mecanismo adicional de estabilidad cambiaria
				\4[] $\to$ Reduce fluctuación dentro de túnel monetario
				\4[] $\then$ ``Serpiente en el túnel''
				\4[] Margen de flotación entre monedas europeas
				\4[] Inferior al margen del túnel
				\4[] $\to$ Margen bilateral del $2,25\%$ respecto al dólar\footnote{Es decir, $\pm 1,12\%$}.
				\4[] $\then$ ``Serpiente en el lago''
				\4[] Primeros miembros de UE
				\4 Reino Unido, Irlanda apenas dos meses después
				\4[] Problemas de balanza de pagos
				\4[] Presión sobre tipo de cambio
				\4[] $\to$ Abandonan serpiente
				\4 Italia y Dinamarca
				\4[] Presiones similares poco después
				\4[] Abandonan también
				\4 Intento de negociación de un nuevo acuerdo
				\4[] $\to$ Fracaso
				\4 Flotación conjunta de europeos en el 73
				\4[$\then$] Abandono de Acuerdo de Smithsonian en 1973
			\3 Serpiente en el lago
				\4 Presión creciente sobre dólar desde 1972
				\4[] Elevada volatilidad de flujos financieros
				\4[] Ventas masivas de dólares
				\4[] $\to$ BCNs europeos obligados a comprar
				\4 Principios de 1973
				\4[] Estados Unidos valora devaluación de dólar
				\4[] Miembros del túnel anuncian nuevos TCN centrales
				\4[] $\to$ Apreciados respecto a nivel previo
				\4[] $\to$ Italia y Reino Unido permiten libre flotación
				\4 Abandono del túnel en 1973
				\4[] Flotación conjunta de europeas sin túnel
				\4[] Abandono del túnel monetario con el dólar
				\4 Inestabilidad
				\4[] Salidas y entradas frecuentes
				\4 Reasignaciones de las bandas frecuentes
				\4[] Presión apreciación del marco
				\4[] Presión devaluatoria de franco, lira, libra
				\4 Fondo Europeo de Cooperación Monetaria
				\4[] Creado en 1973
				\4[] Coordinar márgenes de fluctuación
				\4[] Coordinar intervención en mercados de cambio
				\4[] Gestionar \textit{ecu} a partir de 1979
				\4[] $\to$ Préstamos de corto plazo a bancos centrales
				\4[] $\to$ Gestionar reservas de divisas de EEMM
	\1 \marcar{Sistema Monetario Europeo}
		\2 Contexto
			\3 Caída de Bretton Woods
				\4 Incertidumbre respecto a nuevo sistema
				\4 Cooperación reducida con EEUU
				\4 Europa ha convergido con EEUU tras reconstrucción
				\4[] Menor dependencia
				\4[] Alemania habitualmente superavitaria
				\4 Expansión fiscal y monetaria en EEUU
				\4[] Debilidad del dólar
			\3 Inestabilidad cambiaria
				\4 Varios marcos de intervención en poco tiempo
				\4 Smithsonian + serpiente en el túnel
				\4 Serpiente en el lago inestable
				\4[] Monedas débiles muy afectadas por crisis petróleo
				\4[] Francia entra y sale de la serpiente
				\4[] Alemania adopta objetivo de oferta monetaria
				\4[] $\to$ No acomoda presiones inflacionistas
				\4[] Francia expande fiscalmente
				\4[] $\to$ Vuelve a salir de la serpiente
				\4 Volatilidad muy elevada
				\4 Marco se consolida como moneda líder
				\4[] Contrario a pretensiones francesas
			\3 PAC principal política europea
				\4 Cantidades crecientes
				\4 TCN volátiles obstaculizan
			\3 Unión Aduanera implementada
				\4 Completada desde 1968
			\3 Informe Werner
				\4 Diseño completado en 1970
				\4 Estados Unidos presiona para no implementar
		\2 Eventos\footnote{Ver Cronologías en Conceptos.}
			\3 Creación del SME (1979)
				\4 Impulso francés
				\4[] Evitar liderazgo unilateral de Alemania
				\4 Aumentar poderes de Comité Monetario de la CEE
				\4[] Facultades de supervisión y control
				\4[] Hasta ahora, supervisión
				\4 Permitir créditos ilimitados de corto plazo
				\4[] Activables si:
				\4[] $\to$ >75\% superada
				\4[] $\to$ Medidas fiscales y de tipo de interés inefectivas
				\4[] Very-Short-Term Financing Facility
				\4[] $\to$ Monedas fuertes prestan a monedas débiles
				\4[] $\then$ Soporte ilimitado ante presión devaluatoria
				\4 Perspectiva alemana
				\4[] Paso hacia mayor integración
				\4[] Continuación del Informe Werner
				\4[] Mitigar efectos de depreciación del dólar
				\4 Propuesta de Fondo Monetario Europeo
				\4[] Depositario de todas las reservas de miembros
				\4[] Facultad de crear \textit{ecus}
				\4[$\then$] Proyecto fracasa
				\4[$\then$] Intervención continúa en marco del FECOM
				\4[$\then$] Obligación de apoyar sujeta a devaluación\footnote{Lo cual resultó de la presión alemana. Sin este compromiso, una moneda sujeta a presión devaluatoria habría recibido soporte ilimitado del Bundesbank sin presión equivalente para devaluar, lo que habría --en la opinión de Alemania-- agravado el desequilibrio sin implementar ningún tipo de remedio-.}
			\3 ERM -- Exchange Rate Mechanism
				\4 Elemento central de SME
				\4 Marco de fijación de TCN
				\4[] Cada participante define TCN respecto \textit{ecu}
				\4[] $\to$ Tipo central
				\4[] Tipos bilaterales definidos respecto a tipo central
				\4 \textit{ECU} definido respecto a cesta de monedas
				\4[] Revisión cada 5 años
				\4[] Peso en función de:
				\4[] $\to$ PIB
				\4[] $\to$ Población
				\4[] $\to$ Peso en el comercio
				\4 Variación máxima permitida
				\4[] $\pm 2,25\%$ respecto \textit{ecu}
				\4[] $\pm 6\%$ para lira italiana
				\4 Obligación de intervenir
				\4[] Cuando se alcanza $75\%$ de desviación permitida
				\4 Fijación no es irrevocable o permanente
				\4[] Paridades respecto a ECU ajustables
				\4 Sin movilidad de capitales completa
				\4 Participantes
				\4[] 8 de 9 países de la CEE
				\4[] Italia mantiene margen más amplio: $6\%$
			\3 Estabilidad en los 80
				\4 Realineaciones
				\4[] Cada 8 meses en primeros 4 años
				\4[] Anualmente hasta 1987
				\4 Sin salidas precipitadas
				\4[] Francia a punto en 1981
				\4[] $\to$ Expansión fiscal inicial con Mitterrand
				\4[] $\to$ Tensión con Alemania
				\4[] $\then$ Moderación posterior de Francia
				\4 Apreciación del dólar
				\4[] Aumenta competitividad europea
				\4[] $\then$ Hace más fácil mantener ERM
				\4 Convergencia nominal
				\4[] Caída de diferenciales de inflación
				\4[] Pero re-alineaciones no compensan inflación
				\4[] $\then$ Aumentan desequilibrios de competitividad
				\4 Política del franco fuerte a partir de 1983
				\4[] Francia cambia orientación de política económica
				\4[] Seguir trayectoria del marco alemán
				\4[] $\to$ Apreciación del franco
				\4[] $\to$ Reducción de la inflación
				\4[] $\then$ Estabilidad de SME fortalecida
				\4 Eurosclerosis
				\4[] A mediados de los 80
				\4[] Desempleo elevado
				\4[] Crecimiento débil
				\4 Entrada de España en el SME en 1989
				\4[] Banda de fluctuación amplia
			\3 SME ``Duro'' y Acta Única de 1987
				\4 Acta Única de 1987
				\4[] Integrar mercados europeos
				\4[] $\to$ Para impulsar crecimiento
				\4[] $\then$ Impulsa también integración monetaria
				\4 Acta Única restringe controles de capital
				\4[] Flujos financieros ante expectativa de devaluación
				\4[] $\to$ Potencialmente mucho mayores
				\4[] $\then$ Más difícil realinear tipos
				\4[] $\then$ Sin realineaciones a partir de 1987
				\4 Banda amplia
				\4[] Libra, peseta, escudo y lira
				\4[] $\to$ Bandas de fluctuación más amplias
			\3 Informe Delors y Tratado de Maastricht
				\4 Integración monetaria completa es nuevo objetivo
				\4[] Tras integración económica via Acta Única
				\4 Informe Delors de 1989
				\4[] Hoja de ruta para moneda única
				\4 Diferencias con Informe Werner
				\4[] $\to$ Banco Central, no sistema federal de BC
				\4[] $\to$ Moneda única, no monedas con TCFijos irrevocables
				\4[] $\to$ Apertura de CF primer paso, no último
			\3 Implementación del Euro
				\4 De acuerdo con plan de Informe Delors
				\4[] Tres fases:
				\4[I.] Eliminación de controles de capital
				\4[] Desde (1990)
				\4[] $\to$ Independencia de bancos centrales a implementar
				\4[] $\to$ Armonización de legislación cam
				\4[II.] Convergencia de políticas nacionales
				\4[] Desde (1994)
				\4[] $\to$ Definida más claramente en TUE de 91
				\4[] $\to$ Creación del EMI -- European Monetary Institute
				\4[] $\to$ Creación del SEBC
				\4[] $\to$ Criterios de convergencia\footnote{Inflación menor al $+1,5\%$ de la media de los tres miembros con menor inflación, déficit inferior al 3\%, deuda inferior al $60\%$, tipo de interés de largo plazo inferior a $+2\%$ de la media de los tres miembros con menor inflación y participación en el ERM II durante al menos 2 años sin tensiones significativas.}
				\4[] $\to$ Mayoría de EEMM cumplen condiciones en 1998
				\4[] $\to$ 11 países en cumbre del Euro de 1998
				\4[] $\then$ Fase III desde 1999
				\4[] $\to$ Fase III con los que cumplan, a partir de 1999
				\4[] $\to$ Primer intento en 1997\footnote{Pero número insuficiente de países.}
				\4[] III. Fijación irrevocable de tipos
				\4[] $\to$ Desde (1999)
				\4[] $\to$ ECB sustituye al EMI
				\4[] $\to$ Wim Duisenberg primer presidente\footnote{Francia pactó con Alemania que Duisenberg sería el presidente durante cuatro años y después dimitiría para que el candidato francés Jean-Claude Trichet tomase el mando.}
				\4[] $\to$ Euro es unida de cuenta en sistema de pagos
				\4[] $\to$ TARGET funciona en euros
				\4[] $\to$ Fabricación de monedas y billetes
				\4[] $\then$ Circulación del euro a partir de 2002
				\4[] $\then$ Funcionamiento exclusivo en € tras transición
			\3 Crisis del SME en el 92
				\4 Dinamarca rechaza Maastricht en junio del 92
				\4[] Lira, libra, peseta, escudo caen
				\4[] $\to$ Se acercan a límite inferior del ERM
				\4 Salidas del ERM en 1992
				\4[] $\to$ Finlandia se sale de peg unilateral
				\4[] $\to$ Presión sobre Suecia y otros
				\4[] $\to$ Libra sale del ERM (Soros, miércoles negro\footnote{16 de septiembre de 1992.})
				\4[] $\then$ Gobiernos acuerdan devaluación de la lira
				\4 Mercados entienden realineación es posible
				\4[] Incrementan presión sobre UK, ESP, POR, ITA
				\4[] $\to$ Realinación demasiado pequeña
				\4 Defensas del TCN
				\4[] Inicialmente, UK lo intenta
				\4[] $\to$ Subidas del tipo de interés
				\4[] Controles de capital en España, Portugal e Irlanda
				\4 Crisis del verano del 92
				\4[] Defender TCN resulta muy caro
				\4[] $\to$ Economías ya en recesión
				\4[] $\to$ Contracción monetaria agrava situación
				\4[] $\to$ Sin embargo, reservas suficientes en general
				\4[] $\then$ UK e Italia salen del ERM en Septiembre
				\4 Explicaciones de la crisis
				\4[] I. Armonización inadecuada de políticas pasadas
				\4[] $\to$ Inflación pasada demasiado elevada
				\4[] $\to$ Desequilibrio de TCReal
				\4[] II. Armonización inadecuada de políticas futuras
				\4[] $\to$ Compromiso dudoso con Maastricht
				\4[] $\to$ No tiene sentido defender TC sin Maastricht
				\4[] III. Ataques especulativos/múltiples equilibrios
				\4[] $\to$ Modelos de crisis de segunda generación
				\4 Aparente calma
				\4[] Francia aprueba Maastricht en referéndum, sep-1992
				\4[] Dinamarca aprueba en segundo referéndum, may-1993
				\4[] Alemania expande
				\4[] $\to$ Menos presión sobre monedas débiles
				\4 Ampliación de bandas de fluctuación
				\4[] Continúan presiones sobre monedas débiles
				\4[] Septiembre del 93
				\4[] Hasta el 15\%
				\4[] $\to$ Más que en cualquier acuerdo anterior
		\2 Consecuencias
			\3 Relativo éxito del SME
				\4 Mayor estabilidad cambiaria que con serpiente
				\4 Contención de inflación gana prestigio
				\4 Ecu ampliamente utilizado
				\4[] Aparece mercado de bonos en ecu
				\4[] Intervenciones habituales en Ecu
			\3 Precedente de integración
				\4 Algunas lecciones aprendidas para Euro
				\4 Crisis del SME aumenta temores sobre integración
			\3 Aumento de la flexibilidad
				\4 Ampliación de bandas permitió mayor estabilidad
				\4[] $\to$ Aparente paradoja
				\4 Más flexibilidad cambiaria y de interés
				\4[] Permitió recuperación y convergencia nominal
				\4[] $\to$ Política económica favoreció convergencia
	\1 \marcar{Unión Económica y Monetaria}
		\2 Contexto
			\3 Acta Única en proceso de implementación
				\4 Eliminación de barreras bienes y servicios
				\4 Progresiva eliminación de restricciones mov. de K
			\3 Sin realineación de tipos
				\4 SME ``duro'' desde 1987
		\2 Eventos
			\3 Informe Delors y Tratado de Maastricht
				\4 Integración monetaria completa es nuevo objetivo
				\4[] Tras integración económica via Acta Única
				\4 Informe Delors de 1989
				\4[] Hoja de ruta para moneda única
				\4 Diferencias con Informe Werner
				\4[] $\to$ Banco Central, no sistema federal de BC
				\4[] $\to$ Moneda única, no monedas con TCFijos irrevocables
				\4[] $\to$ Apertura de CF primer paso, no último
			\3 Implementación del Euro
				\4 De acuerdo con plan de Informe Delors
				\4[] Tres fases:
				\4[I.] Eliminación de controles de capital (1990)
				\4[] $\to$ Independencia de bancos centrales
				\4[] $\to$ Armonización de legislación
				\4[II.] Convergencia de políticas nacionales (1994)
				\4[] $\to$ Definida más claramente en Maastricht 91
				\4[] $\to$ Creación del EMI -- European Monetary Institute
				\4[] $\to$ Creación del SEBC
				\4[] $\to$ Criterios de convergencia\footnote{Inflación menor al $+1,5\%$ de la media de los tres miembros con menor inflación, déficit inferior al 3\%, deuda inferior al $60\%$, tipo de interés de largo plazo inferior a $+2\%$ de la media de los tres miembros con menor inflación y participación en el ERM II durante al menos 2 años sin tensiones significativas.}
				\4[] $\to$ Mayoría de EEMM cumplen condiciones $\then$ Fase III
				\4[] $\to$ Fase III con los que cumplan, a partir de 1999
				\4[III]. Fijación irrevocable de tipos en 1999
				\4[] Implementación de sistema TARGET
				\4[] Entrada en vigor de ERM-II
				\4[] $\to$ Vinculación de monedas no UEM con euro
				\4[] $\then$ Incorporación prevista de iure salvo UK, DIN
				\4[] $\then$ Tipo central respecto a euro fijado
				\4[] $\then$ Bandas de fluctuación en cada país $\pm 15$ máximo
				\4[] $\then$ BCE defiende monedas sujetas a ataques especulativos
				\4[] ECB sustituye al EMI
				\4[] $\to$ Wim Duisenberg primer presidente\footnote{Francia pactó con Alemania que Duisenberg sería el presidente durante cuatro años y después dimitiría para que el candidato francés Jean-Claude Trichet tomase el mando.}
				\4[] $\to$ Euro es unida de cuenta en sistema de pagos
				\4[] $\to$ TARGET funciona en euros
				\4[] Fabricación de monedas y billetes
				\4[] $\then$ Circulación del euro a partir de 2002
				\4[] $\then$ Funcionamiento exclusivo en € tras transición
				\4[] $\then$ Euro única moneda legal desde 1 julio 2002
		\2 Consecuencias
			\3 Éxito técnico
				\4 Implementación sin apenas problemas técnicos
				\4 Transición sencilla a nueva unidad de cuenta
				\4 Monedas y billetes rápidamente en circulación
			\3 Convergencia
				\4 Nominal
				\4[] Relativo éxito
				\4[] Inicialmente, convergencia elevada
				\4 Real
				\4[] Difícil atribución
				\4[] Convergencia en los 90
				\4[] Ralentización posterior entre EEMM del €
				\4[] Convergencia de EEMM que acceden en 2000s
			\3 Moneda de reserva
				\4 Rival del dólar
				\4[] Aunque papel aún subordinado
				\4 Aproximadamente 1/3 de divisas mundiales
				\4[] Varía en función de TCNs respectivos
	\1 \marcar{El Euro}
		\2 Elementos centrales
			\3 Eurosistema
				\4 Sistema de BCNs+BCE
				\4 Consejo de Gobierno
				\4[] Asisten:
				\4[] $\to$ Miembros de Consejo Ejecutivo de BCE
				\4[] $\to$ Gobernadores nacionales
				\4[] $\to$ Presidente del Consejo
				\4[] $\to$ Miembro de Comisión Europea
				\4[] Deciden:
				\4[] $\to$ Miembros del Consejo Ejecutivo
				\4[] $\to$ 15 gobernadores de EEMM
				\4 BCE
				\4[] Lidera política monetaria
				\4[] Supervisión sistema bancario
				\4[] Coordinar resto de bancos centrales
			\3 Política monetaria
				\4 Facilidad permanente
				\4[] Depósito
				\4[] $\to$ -0,5\% actualmente
				\4[] Préstamo
				\4[] $\to$ Pasillo interbancario
				\4[] $\to$ Controlar fluctuaciones de EONIA
				\4 Requisitos mínimos de reservas
				\4[] Alterar necesidades estructurales de liquidez
				\4 Operaciones de mercado abierto
				\4[] Diferentes operaciones de compra/venta
				\4[] Repos
				\4[] $\to$ BCE entrega liquidez a cambio de activo
				\4[] $\to$ Compromiso de recomprar activo a BCE
				\4[] $\then$ Aumentar liquidez disponible
				\4[] Repos inversos
				\4[] $\to$ BCE entrega activo a cambio de liquidez
				\4[] $\to$ Operación inversa a repo
				\4[] $\then$ Drenar liquidez disponible
				\4[] Calendario más o menos fijo
				\4[] Vencimiento más o menos largo
				\4[] Diferentes programas
				\4[] $\to$ MRO: vencimientos hasta 1 semana
				\4[] $\to$ LTRO: vencimientos hasta 3 meses
				\4[] $\to$ TLTRO: vencimiento 1-4 años, condicionalidad
				\4[] $\to$ VLTRO: Vencimiento a tres años
				\4[] $\to$ PELTRO
				\4[] $\to$ Fine-tuning
				\4 No convencional: programas de compras
				\4[] SMP -- Securities Market Programme
				\4[] CBPP1 -- Covered Bonds Purchase Programme
				\4[] CBPP2
				\4[] \underline{APP}\footnote{https://www.ecb.europa.eu/mopo/implement/omt/html/index.en.html}
				\4[] CBPP3
				\4[] CSPP
				\4[] ABSPP
				\4[] PSPP $\to$ Mayor parte de las compras
				\4[] PEPP
				\4[] OMT
				\4 Convencional: forward guidance
				\4[] Comunicación de senda de interés nominal futuro
				\4[] $\then$ Fijar expectativas de agentes
				\4[] $\then$ Reducir incertidumbre
				\4[] $\then$ Aumentar commitment de autoridad monetaria
			\3 Gobernanza económica
				\4 PEC: Brazo Preventivo
				\4[] Establecer sendas de reducción DEstructural
				\4[] Procesos sancionadores si desviación de objetivos
				\4[] $\to$ MTO-- Medium-term budgetary objectives
				\4[] $\then$ Saldo estructural público
				\4[] $\then$ Actualizados cada tres años
				\4[] $\then$ Ajuste hacia objetivo de 0.5\% PIB anual
				\4 PEC: Brazo Corrector
				\4[] Sanciones si déficit+deuda superan umbrales
				\4 PDM -- Procedimiento de Desequilibrios Macroeconómics
				\4 Semestre Europeo
				\4[] Calendario de coordinación de política económica
				\4[] Presentación de programas de reforma
				\4[] Borradores de presupuesto y presupuesto...
				\4 TECG + Fiscal Compact
				\4[] Obligación de incorporar estabilidad presupuestaria
				\4[] $\to$ En legislación nacional
				\4[] MTO mínimo del -0.5\%
				\4[] $\to$ Frente a -1\% por defecto
			\3 ESFS -- Sistema Europeo de Supervisión Financiera
				\4 Red de autoridades supervisoras
				\4[] Asegurar regulación efectiva y consistente
				\4[] $\to$ En conjunto de EEMM
				\4 Doble vertiente
				\4[] Microprudencial
				\4[] Macroprudencial
				\4 Autoridades de regulación financiera sectorial
				\4[] EBA -- European Banking Authority
				\4[] EIOPA -- European Insurance and Occupation Pensions Agency
				\4[] ESMA -- European Securities Market Agency
				\4 ESRB -- European System Risk Board
				\4 Supervisores nacionales
			\3 Unión Bancaria
				\4 Mecanismo Único de Supervisión
				\4 Mecanismo Único de Resolución
				\4 Fondo Único de Resolución
				\4 Single Rulebook
				\4 Garantía de Depósitos
				\4[] Incompleto
				\4[] Garantías nacionales armonizadas
				\4[] Sin mecanismo para compartir riesgos
				\4[] $\to$ Sin Fondo Europeo de Garantía de Depósitos
			\3 Mercado Único
				\4 Pilar fundamental
				\4 Cuatro libertades
				\4[] Mercancías
				\4[] Servicios
				\4[] Capitales
				\4[] Personas
				\4 Regulación de competencia
				\4[] Competencia europea si relevancia europea
				\4 Armonización fiscal
				\4[] Lenta
				\4[] Sujeta a unanimidad
			\3 ESM -- Mecanismo Europeo de Estabilidad
				\4 Fondo intergubernamental
				\4[] Propiamente, fuera de UE
				\4[] $\to$ Propuestas para incluir en acervo UE
				\4 Heredero de EFSM y EFSM
				\4 Comprar deuda de países sin acceso a mercados
				\4 Creado en octubre 2012
				\4[] Respuesta a crisis de deuda
				\4[] $\to$ Tras ``whatever it takes'' de Draghi
			\3 ERM II
				\4 Creado en 1999
				\4 Sustituye ERM
				\4 Marco de intervención previo a accesión al €
				\4[] Intervenciones coordinadas BCNacional--BCE
				\4[] Consejo General monitoriza
				\4 Acuerdo entre ECB y BCN de EMiembro
				\4[] Acuerdan fluctuación máxima permitida
				\4[] $\to$ Límite del $\pm 15\%$
				\4[] Intervenciones coordinadas entre BCE y BCN
				\4[] $\to$ Para mantener margen de fluctuación
				\4[] EMiembro puede decidir unilaterlamente banda + estrecha
				\4[] Consejo General de ECB
				\4[] $\to$ Supervisa ERM II
				\4[] $\to$ Coordina política monetaria y cambiaria
				\4 Dinamarca
				\4[] Fluctuación del $2,25\%$
				\4 Bulgaria y Croacia
				\4[] Se unen a ERM II en 2020
				\4[] Paso previo a entrada en el euro\footnote{Ver \href{https://ec.europa.eu/info/business-economy-euro/euro-area/introducing-euro/adoption-fixed-euro-conversion-rate/erm-ii-eus-exchange-rate-mechanism_en}{CE (2020) sobre ERM II}.}
		\2 Balance del euro
			\3 Debates previos
				\4 Meade vs Scitovsky en 50s
				\4[] Meade:
				\4[] $\to$ No se cumplen requisitos para AMO
				\4[] Scitovsky:
				\4[] $\to$ Flujos de capital convertirán área en óptima
				\4 Alemania y Francia
				\4[] Francia:
				\4[] $\to$ Rechazo de liderazgo monetario único
				\4[] $\to$ Rechazo a reunificación alemana
				\4[] $\then$ Rechazo del privilegio exorbitante del dólar
				\4[] $\then$ Rechazo de liderazgo del marco alemán
				\4[] $\then$ Integración monetaria europea como solución
				\4[] Alemania:
				\4[] $\to$ Fuerte aversión a inflación
				\4[] $\to$ Miedo a integración monetaria inflacionista
				\4[] $\to$ Temor a perder liderazgo y prestigio de Bundesbank
				\4[] $\to$ IMonetaria como herramienta de integración política
				\4[] $\then$ Reunificación como moneda de cambio con Francia
				\4[] $\then$ Amortiguar apreciación excesiva de marco
				\4 Sachs y Sala-i-Martín (1992)
				\4[] Comparación del federalismo fiscal UE-USA
				\4[] Valorar absorción respectiva de shocks asimétricos
				\4[] USA: absorción federal del 30\% de shocks
				\4[] UE: absorción cercana al 0.5\% de shocks
				\4[] $\then$ Shocks asimétricos insuficientemente absorbidos
				\4 Krugman (1993) y (2001)
				\4[] Integración monetaria tiene dos efectos:
				\4[] i. Especialización
				\4[] ii. Aumento del comercio intraindustrial
				\4[] Predominio de especialización
				\4[] $\to$ Aumenta asimetría de shocks
				\4[] $\then$ Integración monetaria no es óptima
				\4[] Predominio de comercio intraindustrial
				\4[] $\to$ Shocks menos idiosincráticos al país
				\4[] $\then$ Integración monetaria es óptima
				\4[] Estudios posteriores apuntan a ii
				\4[] $\to$ Más aumento del comercio intraindustrial
				\4 Bayoumi y Eichengreen (1993)
				\4[] Comparar asimetría de shocks regionales USA-UE
				\4[] $\to$ Valorar efectos de mayor idiosincrasia
				\4[] Shocks regionales más idiosincráticos en UE
				\4[] En UE hay centro y periferia
				\4[] $\to$ Simetría y magnitud similar en UE-centro y USA
				\4[] Respuesta shocks
				\4[] $\to$ Menor en UE que USA
				\4[] $\to$ Posiblemente por menor movilidad de ff.pp.
				\4 Mongrelli y Wyplosz (2009)
				\4[] Crecimiento de X+M mercancías sobre PIB en Z€
				\4[] $\to$ Del 26\% al 33\% del PIB
				\4[] Muy ligero crecimiento de servicios 1999-2007
				\4 Rose (2000) y Glick y Rose (2001)
				\4[] Valorar efecto de integración monetaria sobre comercio
				\4[] Artículos pioneros de gran influencia
				\4[] Modelos de gravedad con dummy para integración monetaria
				\4[] Uniones Monetarias inducen discontinuidad
				\4[] $\to$ Mucho más efecto que otras formas de int. monetaria
				\4[] $\to$ Efecto aparentemente simétrico entrada/salida
				\4 Glick y Rose (2016)
				\4[] Revisión de paper de 15 años antes
				\4[] Contexto
				\4[] $\to$ UEM pleno rendimiento
				\4[] $\to$ Nuevos datos
				\4[] $\to$ Debate sobre efectos de EMU
				\4[] $\to$ ¿Ha aumentado comercio?
				\4[] Objetivo
				\4[]$\to$  Revisar predicciones de paper Glick y Rose (2001)
				\4[] Estimar efecto de UM sobre comercio y exportaciones
				\4[] $\to$ Utilizando datos de UEM
				\4[] Resultados
				\4[] i. Simetría de entrada/salida parece razonable
				\4[] ii. EMU ha aumentado exportaciones en 50\%
				\4[] iii. Diferentes UM tienen diferentes efectos
				\4 Comercio de fuera-UE con UE
				\4[] Aumento significativo tras EMU
			\3 Actualidad\footnote{Ver EPRS (2015) \textit{A History of European monetary integration}.}
				\4 19 miembros
				\4[] Adhesiones progresivas en el este
				\4[] $\to$ También en fase post crisis
				\4[] Eslovenia (2007), Chipre y Malta (2008)
				\4[] Eslovaquia (2009), Estonia (2011),
				\4[] Letonia y Lituania (2015)
				\4[] $\then$ 9 países sin euro
				\4 Relevancia económica de la zona euro
				\4[] 335.4 millones de habitantes
				\4[] $12,1\%$ del PIB mundial
				\4[] PIB per cápita de $28600$ €
				\4 Austeridad vs expansión
				\4[] Diferentes modelos del mundo
				\4 Contrafactuales respecto a no-integración
				\4[] Generalmente favorables al euro
				\4[] Necesario considerar tendencia pre-euro
				\4[] $\to$ Flotación libre muy poco habitual en Europa
				\4[] $\then$ Formas más o menos efectivas de integración monetaria
				\4 Críticas a PEC
				\4[] No es suficientemente flexible
				\4[] Tiene efectos procíclicos
				\4[] Difícil de implementar a nivel político
				\4 Integración comercial
				\4[] Ha aumentado fuertemente tras euro
				\4[] Economías también se han abierto a exterior no-euro
				\4[] Difícil contrastación de efecto
			\3 Crisis de 2020
				\4 PEPP
				\4 Fondo Europeo de Reconstrucción
				\4 Tensiones norte-sur
				\4 Asimetría geográfica del shock
				\4[] Dependencia del turismo
				\4[] Dependencia del sector exportador
				\4 Asimetría temporal
				\4[] Shocks temporalmente asimétricos
				\4[] $\to$ Shock demanda cuando uno recupera
				\4[] $\then$ Empeora al otro
				\4[] $\then$ Retroalimentación
				\4 Eurobonos/coronabonos
				\4[] Propuesta inicial
				\4[] Progresivamente descartada
				\4 Fondo de recuperación
				\4[] Debate a julio de 2020
				\4[] Aumento temporal de cred. comp. y ORD en MFP 2020
				\4[] Comisión Europea puede endeudarse en mercados
				\4[] Cuantía cercana a 750.000 M de €
				\4[] $\to$ Cuánto en forma de préstamo
				\4[] $\to$ Cuánto en forma de transferencias
		\2 Propuestas de reforma
			\3 Trilema de la Unión Monetaria de Pisany-Ferry
				\4 Planteado por Pisany-Ferry (2012)
				\4 Asumiendo:
				\4[] Régimen de tipo fijo factible
				\4[] $\to$ En marco Mundell-Fleming
				\4[] Integración monetaria avanzada
				\4 Sólo son posibles dos:
				\4[] \textsc{I} Interdependencia banca-sector público
				\4[] \textsc{II} Financiación monetaria del déficit prohibida
				\4[] \textsc{III} Sin corresponsabilidad respecto a deuda pública\footnote{Es decir, sin posibilidad de que miembros del área monetaria rescaten el sector público de otro miembro.}
				\4 Con I y II
				\4[] Introducción de III
				\4[] Crisis bancaria induce respuesta nacional
				\4[] $\to$ Tensión sobre cuentas públicas
				\4[] $\to$ Sin acceso a prestamista de último recurso
				\4[] $\to$ Sin transferencias fiscales
				\4[] $\then$ Quiebra de sistema bancario y hacienda pública
				\4[] $\then$ Necesaria unión fiscal y transferencias
				\4 Con I y III
				\4[] Introducción de II
				\4[] $\to$ Imposible provisión de liquidez vía BC
				\4[] $\then$ Quiebra de sistema bancario y hacienda pública
				\4[] $\then$ Necesario prestamista de último recurso a nivel de AMonetaria
				\4 Con II y III
				\4[] Unión financiera
				\4[] $\to$ Capitales fluyen en toda la unión
				\4[] $\to$ No hay vínculo bancos y determinados estados
				\4[] $\to$ Sector financiero tiene dimensión comunitaria
				\4[] Introducción de I
				\4[] $\to$ Rescates públicos de EEMM a banca nacional
				\4[] $\then$ Aparición de tensiones financieras
			\3 Trilema de Rodrik
				\4 Rodrik (2000)\footnote{JEP Winter 2000.}
				\4 Restricción empírica postulada
				\4 Economías abiertas deben elegir 2 de 3:
				\4[I] Integración económica
				\4[II] Democracia
				\4[III] Soberanía nacional
				\4 Tres alternativas:
				\4[A] Camisa de fuerza de oro
				\4[] Integración económica+Estado nación soberano
				\4[] Sin transferencias fiscales entre estados
				\4[] Flujos de capital y comerciales libres
				\4[] Mercados internacionales limitan PEconómica nacional
				\4[] Sólo se proveen BPúblicos compatibles con MFinancieros
				\4[] Necesarias políticas autoritarias/represivas
				\4[] $\to$ Ante crisis de deuda/balanza de pagos
				\4[B] Federalismo supranacional
				\4[] Integración económica+democracia
				\4[] Apertura comercial y financiera plena
				\4[] Estados nación pierden soberanía
				\4[] $\to$ Entidad supranacional asume soberanía
				\4[] Transferencias fiscales entre estados
				\4[] $\to$ Posibles déficits exteriores y fiscales
				\4[] Democracia a nivel supranacional
				\4[] $\to$ Entidad supranacional se convierte en nación
				\4[C] Compromiso à la Bretton Woods
				\4[] Democracia+soberanía nacional
				\4[] Sin plena integración comercial+financiera
				\4[] Barreras a movimiento de capital generalizados
				\4[] Estados pueden evitar endeudamiento exterior
				\4[] Posible provisión democrática de bienes públicos
				\4[] $\to$ En la medida en que permita cap. productiva nacional
				\4[] $\to$ Como lo decidan votantes/responsable soberano
				\4[] Sin transmisión de soberanía a ent. supranacional
			\3 Informe de los 5 presidentes
				\4 Presenta grandes bloques de reformas en 2015-2025
				\4 Superar riesgos para permanencia de EMU
				\4 Fase I:
				\4[] Unión Económica
				\4[] $\to$ Sistema europeo de Autoridades de Competitividad
				\4[] $\to$ Reforma del Semestre Europeo
				\4[] Unión Bancaria
				\4[] $\to$ Mecanismo puente de financiación para SRF
				\4[] $\to$ Marco común sobre seguros de depósitos
				\4[] $\to$ Mejorar instrumento de recapitalización directa en ESM
				\4[] $\to$ Unión del Mercado de Capitales
				\4[] $\to$ Reforzar ESRB
				\4[] Unión Fiscal
				\4[] $\to$ Autoridad Fiscal Europea
				\4[] Instituciones europeas y legitimidad
				\4[] $\to$ Mejorar conocimiento del semestre europeo
				\4[] $\to$ Reforzar gobernanza en Eurogrupo
				\4[] $\to$ Cooperación parlamentos nacionales y europeo
				\4[] $\to$ Integrar TECG en acervo europeo
				\4 Fase II
				\4[] Formalizar proceso de convergencia
				\4[] Fondo de estabilización macro para la zona euro
				\4[] Integrar ESM en acervo de la UE
				\4[] Crear tesoro de la Zona Euro
			\3 Fondo de Garantía de Depósitos
				\4 Completar Unión Bancaria
				\4 Evitar eventuales fugas de depósitos
				\4 Romper vínculo sectores público y bancarios
				\4[] Si sector bancario sufre pánico
				\4[] $\to$ Sector público acude al rescate de depositantes
				\4[] $\then$ Sector público también en peligro
				\4 Propuestas plantean fases
				\4[] Primero reasegurar fondos nacionales
				\4[] Posteriormente, puesta en común de riesgo
			\3 Fondo Monetario Europeo
				\4 Propuesta de la CE en 2017
				\4 Integrar ESM en estructura comunitaria
				\4[] Actualmente, es institución intergubernamental
				\4 Backstop para el FUR
				\4 Papel directo en negociación de MoUs
				\4 QMV en programas de asistencia
				\4[] Actualmente, unanimidad
				\4 Sin derecho de veto de parlamentos nacionales
				\4[] Actualmente, papel determinante
			\3 EISF -- European Investment Stabilisation Function\footnote{Ver EPRS (2019).}
				\4 Estabilización de ciclo vía presupuesto UE
				\4 Comisión tomaría prestado en mercados internacionales
				\4[] Prestaría a EM con problemas
				\4[] Garantía del resto de EEMM
				\4[] $\to$ Inversión en infraestructuras
				\4[] $\then$ Devolución cuando crisis acabe
				\4 Actuación semi-automática
				\4[] Basada en indicadores de desempleo
				\4[] Si situación no resulta de política irresponsable
				\4 Fuerte oposición en el Consejo Europeo
				\4[] Países del Norte
	\1[] \marcar{Conclusión}
		\2 Recapitulación
			\3 Antecedentes
			\3 Sistema Monetario Europeo
			\3 Unión Económica y Monetaria
			\3 El Euro
		\2 Idea final
			\3 Tensión integración-desintegración
				\4 Constante en áreas monetarias
				\4 UEM y Euro no son excepciones
				\4[] Francia en los 80 amenaza salida
				\4[] Crisis del ERM
				\4[] Grecia a punto de salir en 2015
				\4[] Italia amenaza salida actualmente
				\4[] ...
				\4 Existen fuerzas integradoras e integradoras
				\4[] Resultado de:
				\4[] $\to$ Flujos financieros y de información
				\4[] $\to$ Expectativas
				\4[] $\to$ Intereses nacionales
				\4[] $\to$ Fricciones financieras
			\3 Innovación financiera
				\4 Factor de incertidumbre
				\4 Nuevas tecnologías alteran política monetaria
			\3 Sistema monetario internacional
				\4 UEM es economía grande y abierta
				\4[] $\to$ Influye en economía mundial
				\4[] $\to$ Afectada por shocks globales
			\3 Interacción con política
				\4 Ciclos electorales en EEMM
				\4[] Factores importantes de integración
				\4[] $\to$ Numerosos ejemplos a lo largo de historia
\end{esquemal}


\conceptos

\concepto{Cronología hasta el SME}

\begin{itemize}
	\item[1962] Informe Marjolin. Comisión Europea hace primera propuesta sobre una futura Unión Económica y Monetaria.
	\item[1964] Se constituye un Comité de Gobernadores de la Comunidad Económica Europea para institucionalizar la cooperación.
	\item[1969] Cumbre Europea de la Haya. Emitido Memorandum Barre. El Consejo de Ministros encarga a Pierre Werner explorar la posible implementación de una unión monetaria y económica.
	\item[1970] Informe Werner. Plan para implementar una unión económica y monetaria en 1980, en tres fases. La primera implica coordinar la política económica general, los presupuestos y la política fiscal, en tres años. Los intercambios comerciales deben liberalizarse aún más. Fluctuaciones cambiarias deben limitarse a bandas relativamente estables. En la segunda, liberalizan progresivamente los flujos de capital y se restringe la banda de fluctuación. En la tercera, se fijan irrevocablemente los tipos de cambio y se liberalizan totalmente los flujos de capital.
	\item[1971] Nixon Shock el 15 de agosto de 1971. Estados Unidos suspende la convertibilidad del dólar en oro. 
	\item[1971] En el Acuerdo de Smithsonian de 1971, las monedas principales se aprecian frente al dolar, que se devalúa un 8,6\% respecto del oro en relación al nivel previo. El túnel monetario fija una banda de fluctuación respecto al dólar de $\pm 2.25\%$, lo que implica que el tipo de cambio de dos monedas pueden desviarse un 4,5\% entre ellas y llegar a fluctuar entre sí un 9\% si una se aprecia un 4,5\% y otra se deprecia un 4,5\%. 
	\item[1972] Creación de la ``serpiente en el túnel'' para restringir a la mitad la amplitud de la banda de fluctuación que incorporaba el Acuerdo de Smithsonian de 1971 tras el Nixon Shock de 1971. Países miembros de la CEE más Dinamarca, Reino Unido y Noruega. 
	\item[1973] Creación del Fondo Europeo de Cooperación Monetaria para permitir las operaciones de mantenimiento de la serpiente monetaria.
	\item[1973] La crisis del petróleo provocan la salida de varios países de la serpiente del túnel. Dinamarca, Francia, Irlanda, Italia y el Reino Unido abandonan la serpiente varias veces entre 1973 y 1974.
	\item[1974] Implementación del Informe Werner queda en suspenso.
	\item[1977] Roy Jenkins reactiva esfuerzos para introducir una Unión Monetaria.
	\item[1979] Introducción del Sistema Monetario Europeo en marzo de 1979.
\end{itemize}

\concepto{Cronología del SME hasta el Euro}\footnote{Ver \href{https://www.europarl.europa.eu/RegData/etudes/BRIE/2015/551325/EPRS_BRI(2015)551325_EN.pdf}{EPRS (2015).}}

\begin{itemize}
	\item[1977] El Presidente de la Comisión Roy Jenkins reintroduce la posibilidad de implementar una Unión Monetaria del Euro. 
	\item[1979] Alemania y Francia apoyan una versión más limitada y se introduce el Sistema Monetario Europeo en marzo de 1979. Se define el ECU como cesta de monedas nacionales, y el ERM (Exchange Rate Mechanism) que fija un tipo de cambio de cada moneda nacional respecto del ECU. Los tipos bilaterales quedan fijados en función del tipo de cambio con el ECU. El acuerdo incorpora un mecanismo de seguridad: cuando se alcance el 75\% del máximo de fluctuación autorizado, el país tiene que implementar medidas de ajuste de política fiscal y de tipo de interés. Si no tienen efecto, los bancos centrales pueden intervenir comprando y vendiendo moneda. 
	\item[1983] Tras cuatro años de inestabilidad y amenazas de ruptura, Francia introduce la política del ``Franco fuerte''. El Banco de Francia prioriza mantener la paridad con el ECU y alcanzar la convergencia nominal con Alemania. Diferenciales de inflación e interés se reducen.
	\item[1987] Entrada en vigor del Acta Única Europea. Prevista creación de mercado común en cinco años. Los costes de tipos de cambio variables aumentan. 
	\item[1988] El Consejo Europeo establece un Comité presidido por el presidente de la Comisión Jacques Delors para que diseñe un plan de creación de una UEM. 
	\item[1989] El Informe Delors presenta los objetivos y las etapas de una UEM. en la primera, de 1990 a 1994, se completará el mercado único y se eliminarán las restricciones al movimiento de capital. En la segunda etapa, de 1994 a 1999 (inicialmente, hasta 1997), el Instituto Monetario Europeo coordinará la cooperación entre bancos centrales y la introducción del SEBC. En la tercera etapa, de 1999 en adelante, se fijarán los tipos de cambio y se prepará la transición física al Euro. Se establecerán reglas presupuestarias vinculantes. 
	\item[1989] El Consejo Europeo acuerda en junio y en Madrid proceder a la primera etapa. 
	\item[1991] Aprobación del Tratado de Maastricht en diciembre. El tratado incorpora la creación de una UEM y prevé la creación de un banco central independiente. Se introducen los criterios de convergencia que todos los EEMM deberán cumplir. 
	\item[1992] Inestabilidad cambiaria debida a políticas monetarias y fiscales divergentes, unidas a la incertidumbre sobre la ratificación del Tratado de Maastricht provocan una crisis cambiaria entre 1992 y 1993. Reino Unido e Italia abandonan el ERM y España y Protugal deben devaluar.
	\item[1993] Los EEMM deciden en agosto ampliar la banda de fluctuación del ERM hasta el $\pm 15\%$. 
	\item[1993] Dinamarca aprueba el Tratado de Maastricht tras acordarse un opt-out respecto de la UEM.
	\item[1994] Sólo Irlanda y Luxemburgo cumplen los criterios de convergencia.
	\item[1995] El Consejo Europeo confirma que 1999 será el año de introducción de la UEM.
	\item[1995] Los EEMM acuerdan que el nombre de la moneda única será el ``euro''.
	\item[1996] Los EEMM introducen el Pacto de Estabilidad y Crecimiento. Compromiso entre posición alemana (mantener obligaciones de convergencia tras integración) y francesa, italiana y española (excesiva disciplina presupuestaria perjudique al crecimiento).
	\item[1997] Introducción del ERM II para reemplazar al viejo ERM del SME. Fija tipos de cambio de EEMM que no estén en el euro y admite una fluctuación máxima, para evitar el impacto sobre la estabilidad económica del mercado único. 
	\item[1998] El Consejo determina que 11 EEMM satisfacen los requisitos mínimos para entrar en el Euro.
	\item[1999] El 1 de enero entra en vigor la fijación irrevocable de tipos de cambio de 11 monedas nacionales respecto al euro. 
	\item[2000] El Consejo acuerda que Grecia también cumple los criterios de convergencia, sujeto a la reformas adicionales, y puede por tanto adoptar la moneda única. 
	\item[2002] Reemplazo físico de monedas nacionales por euros.
	\item[2002-2020] Entradas sucesivas de nuevos Estados en el Área del Euro, hasta los 19 de mayo de 2020.
\end{itemize}


\concepto{Cronología de la crisis del Euro 2009-2013}
\begin{itemize}
	\item[2009: Otoño] Cambio de gobierno en Grecia. Aflora déficit mucho mayor de lo declarado inicialmente. Bajada de rating de deuda griega. Programa de ajuste.
	\item[2009: Otoño] Crisis de deuda en Dubai.
	\item[2010: Invierno] Aumenta presión sobre deuda griega. Dudas sobre programa de asistencia o no.
	\item[2010: Primavera] Rescate a Grecia I: 110.000 M de €, de los cuales 80 de UE y 30 de FMI.
	\item[2010: Primavera] Anunciada reforma del marco de gobernanza económico y fiscal de la UE.
	\item[2010: Primavera] Anunciada creación de EFSF y de programa SMP para comprar deuda griega (inicialmente) y esterilizar compras.
	\item[2010: Primavera] Programa de ajuste presupuestario en España tras presión de Ecofin. 
	\item[2010: Verano] Six-pack presentado para reformar marco de gobernanza. 5 reglamentos y 1 directiva. No se aprobaría hasta finales de 2011. Incluye Semestre Europeo y PDM.
	\item[2010: Verano] Test de estrés del sistema bancario europeo. Resultados favorables aparentemente
	\item[2010: Otoño] Irlanda anuncia coste de rescate de Anglo-Irish. Bajadas de calificación de deuda irlandesa.

	\item[2010: Otoño] Alemania y Francia acuerdan reforma de tratados futura para crear mecanismo permanente de estabilidad y reestructuraciones de deuda. 
	\item[2010: Otoño] Acuerdo para rescatar Irlanda: 85.000 M de € entre UE y FMI. 
	\item[2011: Invierno] Dimisión de Axel Weber del Bundesbank por desacuerdo con la política crediticia del BCE.
	\item[2011: Invierno] Acuerdo para reestructurar deuda griega. Menos interés y mayor plazo. Privatización por 50.000 M de €. 
	\item[2011: Invierno] Capacidad de préstamo del EFSF efectiva en totalidad de 440.000 M de €. Permitido actuación excepcional en mercados primarios.
	\item[2011: Invierno] Portugal: bajadas generalizadas de calificación crediticia.
	\item[2011: Primavera] Portugal solicita rescate. 78.000 M de € en tres años de UE y FMI.
	\item[2011: Primavera] Especulación sobre posible haircut a deuda griega para inversores privados. Reducción de deuda portuguesa a bono basura.
	\item[2011: Verano] Paquete de austeridad en Italia. Aumenta presión sobre deuda italiana por ambigüedad sobre condiciones del paquete de ayuda.
	\item[2011: Verano] Avances para constituir ESM. 
	\item[2011: Verano] Segundo test de estrés de sistema bancario. Resultados más detallados.
	\item[2011: Verano] Presión sobre Italia para extender programa de recortes.
	\item[2011: Verano] Bajadas de calificación de EEUU y rumores sobre Francia. Bancos sufren empeoramiento de condiciones de financiación.
	\item[2011: Otoño] Reforma de la Constitución Española: artículo 135. Prohibición del déficit estructural. Priorización de pago de la deuda.
	\item[2011: Otoño] Suiza expande oferta monetaria para evitar apreciación ulterior.
	\item[2011: Otoño] Tensiones políticas en Alemania sobre programas de rescate. Karsruhe dictamina que se requiere aprobación por Bundestag.
	\item[2011: Otoño] Acuerdo sobre el Six-Pack.
	\item[2011: Otoño] Ecofin reafirma necesidad de consolidación fiscal.
	\item[2011: Otoño] Cambio de gobierno en España e Italia.
	\item[2011: Otoño] Aumento de rentabilidad de deuda española e italiana. Niveles sin precedentes.
	\item[2011: Otoño] Acuerdo entre principales bancos centrales para proveerse liquidez mutuamente: BCE, Fed, RBEngland, BoJ, SwissNB
	\item[2011: Otoño] Nuevo paquete de consolidación fiscal en Italia.
	\item[2011: Otoño] Draghi pide mayor integración y consolidación fiscal.
	\item[2011: Otoño] Acuerdo para fiscal compact $\to$ futuro TCSG aprobado en 2012. Salvo UK y República Checa.
	\item[2011: Otoño] España: déficit para 2011 supera ampliamente los objetivos del gobierno.
	\item[2012: Invierno] Bajada generalizada de calificación crediticia en toda la Unión.
	\item[2012: Invierno] Entrada en vigor de tratado del MEDE/ESM.
	\item[2012: Invierno] Firma del TCSG salvo RU y República Checa. 
	\item[2012: Invierno] España: programa de consolidación presupuestaria pero objetivo de déficit para 2012 mayor que el inicial. 3\% de objetivo para 2013.
	\item[2012: Invierno] Grecia: Segundo programa de rescate aprobado tras programa de austeridad y haircut del 50\% a inversores privados.
	\item[2012: Invierno] España: PGE presentados con fuertes recortes en AGE, pendientes comunidades autónomas
	\item[2012: Primavera] FMI: aumentada capacidad de préstamo vía NAB.
	\item[2012: Primavera] Francia: cambio de Gobierno en Francia. Cambio de Sarkozy por Hollande.
	\item[2012: Primavera] España: crisis en Bankia. Dimisión de Rodrigo Rato tras auditoría fallida.
	\item[2012: Primavera] España: comienzo de restructuración del sistema bancario. Nacionalización de Bankia. Elevadas provisiones obligatorias por sector inmobiliario. Transferencia de activos al SAREB, acceso al FROB y auditorías.
	\item[2012: Primavera] España: estimados 19.000 M de € para recapitalizar Bankia
	\item[2012: Primavera] España: bajada de rating de deuda soberana.
	\item[2012: Primavera] España: solicitado rescate para recapitalizar sistema bancario. 100.000 M de €, de los cuales se utilizarán finalmente la mitad. Mercados reaccionan desfavorablemente, considerando que aumentará la deuda pública y que la ayuda del MEDE será senior respecto a otras obligaciones.
	\item[2012: Verano] Chipre pide ayuda financiera a UE y FMI.
	\item[2012: Verano] MEDE: acuerdo para evitar seniority en ayuda a España.
	\item[2012: Verano] España: nuevo paquete de medidas de austeridad y extensión del límite para bajar del 3\% hasta 2014.
	\item[2012: Verano] España: coste de financiación aumenta hasta extermos insostenibles.
	\item[2012: Verano] 26 de julio: Whatever it takes. Draghi admite posibilidad de comprar deuda en segmento de corto plazo si un gobierno lo solicita y hay acuerdo previo con ESM/EFSF
	\item[2012: Verano] OMT: presentación oficial. Compras de deuda pública de 1 a 3 años en mercados secundarios para países en situación de crisis, sujeto a condicionalidad. Sustituye a SMP. En teoría, las compras deben estabilizarse también como en el caso del SMP, vía subastas de depósitos.
	\item[2012: Otoño] España: descartada utilización de OMT por tratarse de una recapitalización de bancos y no un rescate al país. 
	\item[2012: Otoño] FMI afirma que ha habido errores en la estimación de los multiplicadores fiscales.
	\item[2012: Otoño] Comisión Europea espera una recuperación del crecimiento en 2013, aunque España icumplirá todos los objetivos
	\item[2012: Otoño] Bernanke: forward guidance de los tipos de interés. No serán aumentados hasta que desempleo baje del 6,5\%, al menos.
	\item[2012: Otoño] Grecia: desbloqueados fondos para evitar quiebra y recapitalizar sistema bancario.
	\item[2012: Otoño] Creación del SSM.
	\item[2013: Invierno] España: prima de riesgo alcanza los 300 puntos, la mitad de lo que alcanzó en verano.
	\item[2013: Invierno] Consejo Europeo acuerda MFP para 2014-2020, limitado al 1\% del PIB, y sin aumento respecto al anterior. 
	\item[2013: Invierno] Aprobado Two-Pack en conciliación, tras largo retraso. 
	\item[2013: Invierno] Irlanda se financia en mercados y paga menos que España.
	\item[2013: Primavera] Chipre: crisis bancaria. Dudas sobre cumplimiento de seguro sobre primeros 100.000 € tras bail out. Finalmente, la UE acepta que se paguen los primeros 100.000 €
	\item[2013: Primavera] España, Francia: CEuropea aumenta en dos años el plazo para bajar déficit del 3\% 
	\item[2013: Primavera] Comienza a atisbarse un cambio de políticas hacia un énfasis mayor en el crecimiento respecto a la disciplina presupuestaria.
	\item[2013: Verano] Schäuble: Grecia necesitará un tercer paquete de ayuda. El presidente del MEDE, Regling, afirma poco después lo mismo que Schäuble. 
	\item[2014: Invierno] Grecia: Troika considera tercer programa de ayuda+reestructuración adicional de deuda con EFSF. Requiere cumplimiento de última fase de memorandum previo y superávit primario.
	\item[2015: Invierno] Grecia: deterioro de la situación fiscal y tensiones de liquidez. Protestas políticas. Syriza toma el gobierno con un programa expansivo y contrario a la consolidación.
	\item[2015: Verano] Grecia: fuertes tensiones políticas sobre nuevo rescate. Tsipras convoca referendum, que gana. Poco después, acepta condiciones de tercer programa de ayuda vía MEDE: 86.000 M de €, que incluyen buffer para el sector bancario. Incorpora programa de ajustes de sistema fiscal.
\end{itemize}



\preguntas

\seccion{Test 2009}

\textbf{42.} ¿Cuáles de las siguientes afirmaciones respecto al Sistema Monetario Europeo (SME) son ciertas?

\begin{itemize}
	\item[I] Se fijó con las bandas de fluctuaciones entre las monedas europeas y el dólar en el 2,5\% y el 9\% para las europeas entre sí.
	\item[II] El Fondo Europeo de Cooperación Monetaria (FECOM) asumió, entre otras funciones, la supervisión del funcionamiento del SME y la gestión de las facilidades crediticias asociadas al SME.
	\item[III] La denominada serpiente monetaria europea fue un componente fundamental del funcionamiento del SME.
	\item[IV] Se trató de un sistema cambiario de zonas objetivo estables pero ajustables.
\end{itemize}

\begin{itemize}
	\item[a] I, II, IV.
	\item[b] I, II, III, IV.
	\item[c] II, IV.
	\item[d] I, III.
\end{itemize}

\seccion{Test 2008}

\textbf{43.} ¿Cuando se adoptó por primera vez el margen de fluctuación entre las monedas del $\pm 2,25\%$?
\begin{enumerate}
	\item[a] Durante la segunda fase de la Serpiente Monetaria Europea.
	\item[b] En los Acuerdos de Jamaica.
	\item[c] Con el establecimiento del Sistema Monetario Europeo.
	\item[d] En los Acuerdos de Washington de diciembre de 1971.
\end{enumerate}

\seccion{Test 2006}

\textbf{45.} Los tres pilares del Sistema Monetario Eurepeo eran:
\begin{enumerate}
	\item[a] La unidad de Cuenta Europea (ECU), el mecanismo de Cambio Europeo (MCE) y el Fondo Europeo de Cooperación Monetaria (FECOM).
	\item[b] La unidad de Cuenta Europea (ECU), el mecanismo de fluctuación de la serpiente en el túnel y el Fondo Europeo de Cooperación Monetaria (FECOM).
	\item[c] El ECU, el MCE y el Instituto Monetario Europeo (IME).
	\item[d] La moneda europea (euro), el MCE) y el Sistema Europeo de Bancos Centrales (SEBC).
\end{enumerate}

\notas

\textbf{2009:} \textbf{42.} C

\textbf{2008}: \textbf{43}. D

\textbf{2006}: \textbf{45}. A

\bibliografia

\begin{itemize}
	\item euro
	\item European Banking Union
	\item European Central Bank
    \item Debt mutualisation in the ongoing Eurozone Crisis -- A tale of the 'North and the 'South
	\item \textbf{European Monetary Integration}
	\item European Monetary Union
	\item European Union (EU) European Semester
	\item Euro Zone Crisis 2010
	\item fiscal federalism
	\item Stability and Growth Pact
	\item Stability and Growth Pact of the European Union
\end{itemize}

Arestis, P. Sawyer, M. \textit{Economic and Monetary Union Macroeconomic Policies. Current Practices and Alternatives} (2013) Palgrave MacMillan -- En carpeta Economía internacional

Baldwin, R. (2016) \textit{The World Trade Organization and the Future of Multilateralism} Journal of Economic Perspectives -- En carpeta del tema

Bordo, M. D.; Cochrane, J. H.; Seru, A. \textit{The Structural Foundations of Monetary Policy} \url{https://www.hoover.org/research/structural-foundations-of-monetary-policy} -- En carpeta macro

Borrel Fontelles, J. (2013) \textit{Euro Crisis Timeline} \href{https://www.ucm.es/data/cont/docs/518-2015-04-15-TIMELINE%20OCT%202009%20MAYO%202013_ampliado.pdf}{Disponible aquí} -- En carpeta del tema

Brunnermeir, M. \textit{The Euro Crisis} (2018) Ch. 7 of The Structural Foundations of Monetary Policy -- En carpeta Macro

Buti, M. Jollès, M. Salto, M. \textit{The euro -- a tale of 20 years: The priorities going forward} (2018) VOX CEPR Policy Portal -- \url{https://voxeu.org/article/euro-tale-20-years-priorities-going-forward}

De Grauwe, P. (2006) \textit{What Have we Learnt about Monetary Integration since the Maastricht Treaty?} Journal of Common Market Studies -- En carpeta del tema

Delgado-Téllez, M.; Kataryniuk, I.; López-Vicente, F.; Pérez, J. J. (2020) \textit{Endeudamiento supranacional y necesidades de financiación en la Unión Europea} Banco de España. Documentos ocasionales. Nº 2021 -- En carpeta del tema.


European Parliamentary Research Service (2015) \textit{A history of European monetary integration} European Parliament -- En carpeta del tema

European Parliamentary Research Service (2019) \textit{Establishment of a European monetary fund (EMF)} Briefing: EU Legislation in Progress -- En carpeta del tema

European Parliamentary Research Service (2019) \textit{European Investment Stabilisation Function (EISF)} European Parliament -- En carpeta del tema

Gali, J. Perotti, R. \textit{Fiscal Policy and Monetary Integration in Europe} (2003) NBER Working Paper Series -- En carpeta del tema

de Haan, J.; Inklaar, R. \textit{Will Business Cycles in the Euro Area Converge? A Critical Survey of Empirical Research} (2008) Journal of Economic Surveys -- En carpeta del tema

IMF (2019) \textit{External Sector Report. The Dynamics of External Adjustment} Julio de 2019 \url{https://www.imf.org/en/Publications/SPROLLs/External-Sector-Reports} -- En carpeta del tema

Lane, P. R. \textit{The Real Effects of European Monetary Union} (2006) Journal of Economic Perspectives: Fall 2006 -- En carpeta del tema

Obstfeld, M.; Shambaugh, J.; Taylor, A. M. \textit{The Trilemma in History: Tradeoffs among Exchange Rates, Monetary Policies, and Capital Mobility} (2004) NBER Working Paper Series -- En carpeta del tema

Obstfeld, M.; Shambaugh, J.; Taylor, A. \textit{Financial Stability, the Trilemma, and International Reserves} (2008) NBER Working Paper Series -- En carpeta del tema

Pilbeam, K. \textit{International Finance} (2006) 3rd Edition -- En carpeta Economía Internacional

Pisany-Ferry, J. \textit{The Euro crisis and the new impossible trinity} (2012) Bruegel Policy Contribution -- En carpeta del tema

Rodrik, D. (2000) \textit{How Far Will International Economoic Integration Go} Journal of Economic Perspectives. Winter 2000. -- En carpeta del tema

Tomann, H. \textit{Monetary Integration in Europe. The European Monetary Union after the Financial Crisis} (2017) Palgrave MacMillan -- En carpeta Economía Internacional

\end{document}
