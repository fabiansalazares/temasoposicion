\documentclass{nuevotema}

\tema{3B-5}
\titulo{Teoría del comercio internacional (I): la teoría ricardiana de la ventaja comparativa. El modelo Heckscher-Ohlin-Samuelson y sus extensiones.}

\begin{document}

\ideaclave

El comercio internacional ha conocido un desarrollo espectacular en los últimos siglos. El descubrimiento de nuevas técnicas de navegación, el ferrocarril, el avión o el teléfono han permitido una reducción de los costes de transporte y transmisión de la información que permiten el comercio de bienes y servicios a lo largo del planeta, entre economías de distintas y similares características. El estudio del comercio internacional es un sujeto de estudio de muy larga trayectoria, pero son los trabajos de Smith, Mill o Ricardo los que inician el estudio en términos económicos modernos. Así, el análisis del comercio internacional ha transcurrido en paralelo al avance de la teoría económica y a los hechos empíricos relativos al comercio internacional. En el siglo XX y gracias a las aportaciones de Marshall, Walras, J. S. Mill y especialmente Gottfried Haberler, se consolida un marco de análisis denominado neoclásico o tradicional que se basa en un modelo walrasiano genérico cuyas realizaciones particulares se diferencian unas de otras por tres aspectos básicos: la tecnología de producción de las empresas, las preferencias de los consumidores y las dotaciones de los factores de producción de que dispongan los agentes. 


\seccion{Preguntas clave}
\begin{itemize}
	\item ¿Qué exportan y que importan los países?
	\item ¿En qué proporciones? / ¿A qué precios relativos?
	\item ¿Qué ganancias aporta el comercio internacional?
	\item ¿A quién perjudica?
	\item ¿Qué modelos tratan de explicarlo?
	\item ¿Qué es el modelo neoclásico de comercio internacional?
	\item ¿Qué es la ventaja comparativa?
	\item ¿Qué es el modelo de Heckscher-Ohlin-Samuelson?
	\item ¿Qué extensiones de estos modelos existen?
\end{itemize}

\esquemacorto

\begin{esquema}[enumerate]
	\1[] \marcar{Introducción}
		\2 Contextualización
			\3 Evolución del comercio internacional
			\3 Modelo neoclásico o tradicional
			\3 Modelo clásico y H-O como casos particulares
		\2 Objeto
			\3 ¿Por qué existe el comercio internacional?
			\3 ¿Qué importan y qué exportan los países?
			\3 ¿En qué proporciones y qué precios relativos?
			\3 ¿Qué modelos tratan de explicarlo?
		\2 Estructura
			\3 Modelo de la ventaja comparativa
			\3 Modelo de Heckscher-Ohlin-Samuelson
	\1 \marcar{El modelo clásico de la ventaja comparativa}
		\2 Idea clave
			\3 Contexto
			\3 Objetivo
			\3 Resultado
		\2 Formulación
			\3 Ejemplo tradicional
			\3 Portugal
			\3 Inglaterra
		\2 Implicaciones
			\3 Ventajas comparativas
			\3 Especialización completa
			\3 Especialización parcial
			\3 Mejora del ingreso
			\3 Representación gráfica
			\3 Beneficios en grandes y pequeños
		\2 Valoración
			\3 Generalización a $n$ bienes, $m$ países
			\3 Falta de determinación de la relación relativa de intercambio
			\3 Capacidad predictiva
			\3 Sin valoraciones de economía política
	\1 \marcar{El modelo de Heckscher-Ohlin-Samuelson}
		\2 Idea clave
			\3 Contexto
			\3 Objetivo
			\3 Resultados
		\2 Formulación
			\3 Supuestos básicos
			\3 Equilibrio de autarquía
			\3 Equilibrio de economía abierta
		\2 Implicaciones
			\3 Teorema de Heckscher-Ohlin
			\3 Teorema de la igualación del precio de los factores
			\3 Teorema de Stolper-Samuelson
			\3 Teorema de Rybczynski
			\3 Teorema de Heckscher-Ohlin-Vanek
			\3 Ganancias del comercio
		\2 Extensiones
			\3 Modelo de factores específicos Ricardo-Viner
			\3 Economía política
			\3 Reversiones de la intensidad factorial
			\3 Costes de transporte
			\3 Generalización de Heckscher-Ohlin-Vanek
			\3 Bienes no comerciables
			\3 Ofertas elásticas de ff.pp.
		\2 Valoración
			\3 Paradoja de Leontieff
			\3 Influencia
	\1[] \marcar{Conclusión}
		\2 Recapitulación
			\3 Modelo clásico de la ventaja comparativa
			\3 Modelo de Heckscher-Ohlin-Samuelson
			\3[$\then$] Casos particulares modelo neoclásico
		\2 Idea final
			\3 Comercio internacional en la actualidad
			\3 Fenómenos no explicados

\end{esquema}

\esquemalargo

\begin{esquemal}
	\1[] \marcar{Introducción}
		\2 Contextualización
			\3 Evolución del comercio internacional
				\4 Explosión en últimos siglos
				\4[] $\to$ Y más aún desde post 2GM
				\4 Avance tecnológico:
				\4[] $\downarrow$ de costes de transporte
				\4[] $\downarrow$ de costees informacionales
				\4 Sujeto de estudio relativamente antiguo:
				\4[] $\to$ Smith, Ricardo, Mill
				\4[] Ligado a la evolución de:
				\4[] $\to$ teoría económica
				\4[] $\to$ hallazgos empíricos
			\3 Modelo neoclásico o tradicional
				\4 Teoría neoclásica $\to$ modelo general
				\4[] Basado en EGWalrasiano 2x2x2x2
				\4[] Dos economías 2x2x2x2
				\4[] Comparar:
				\4[] $\to$ Eq. por separado
				\4[] $\to$ Eq. conjunto
				\4[$\then$] Comercio internacional depende de primitivas:
				\4[] -- Tecnologías
				\4[] -- Preferencias
				\4[] -- Dotaciones
			\3 Modelo clásico y H-O como casos particulares
				\4[] Gottfried Haberler (1935)
				\4[] $\to$ ``\textit{International Trade}''
				\4[] Generaliza mod. clásico, H-O, ff.pp. específicos
				\4[] $\to$ Casos particulares
				\4[] $\to$ Restringiendo alguna primitiva
				\4[] Formula en términos modernos
				\4[] $\to$ Costes de oportunidad y FPP
				\4[] $\to$ Equilibrio general walrasiano
		\2 Objeto
			\3 ¿Por qué existe el comercio internacional?
			\3 ¿Qué importan y qué exportan los países?
				\4[$\to$] Patrón de comercio
			\3 ¿En qué proporciones y qué precios relativos?
				\4[$\to$] Relación relativa de intercambio
			\3 ¿Qué modelos tratan de explicarlo?
				\4[$\to$] ¿Qué caracteriza a los modelos neoclásicos?
				\4[$\to$] ¿Y al modelo de la ventaja comparativa?
				\4[$\to$] ¿En qué consiste el modelo de Heckscher-Ohlin?
				\4[$\to$] ¿Qué extensiones?
		\2 Estructura
			\3 Modelo de la ventaja comparativa
			\3 Modelo de Heckscher-Ohlin-Samuelson
	\1 \marcar{El modelo clásico de la ventaja comparativa}
		\2 Idea clave
			\3 Contexto
				\4 Modelo de la ventaja absoluta (Smith)
				\4[] País de producción de cada bien:
				\4[] $\to$ Donde tienen menor coste absoluto
				\4[] Ejemplo (uds. de trabajo/unidad):
				\4[] Inglaterra:
				\4[] $\to$ Tela: 4, Vino: 8
				\4[] Portugal:
				\4[] $\to$ Tela: 6, Vino: 3
				\4[] $\then$ Producir tela: más barato en Inglaterra
				\4[] $\then$ Producir vino: más barato en Portugal
				\4 ¿Y si uno produce ambos a menor coste?
				\4[] ¿No hay mejoras posibles?
				\4[] Modelo no examina
				\4[] Extrapolando conclusiones:
				\4[] $\to$ Un país produciría todo
			\3 Objetivo
				\4 Explicar patrón de comercio
				\4 Valorar optimalidad del comercio internacional
				\4 Generalizar para todas ventajas
				\4[] Absoluta
				\4[] $\to$ Cada país produce un bien a menor coste
				\4[] Comparativa
				\4[] $\to$ Un país produce todo a menor coste
				\4[] $\to$ Pero diferentes costes de oportunidad
			\3 Resultado
				\4 Teoría clásica del comercio internacional
				\4[] Robert Torrens, James Mill y David Ricardo
				\4[] Primeras décadas de s. XIX
				\4[] Generaliza modelo de Adam Smith
				\4[] Conocido también como modelo clásico
				\4 Ventajas del CI
				\4[] Permite aumentar división del trabajo
				\4[] Induce más especialización
				\4 Costes relativos determinan comercio
				\4[] $\to$ No los costes absolutos
%				\4 Coste relativo:
%				\4[] $\downarrow$ de bien 1 por cada $\uparrow$ de bien 2
%				\4 Precio internacional:
%				\4[] $\downarrow$ de bien 1 por cada $\uparrow$ de bien 2
				\4[] $\text{Coste relativo} > \text{precio internacional}$
				\4[] $\then$ mejor importar que producir
				\4[] $\text{Coste relativo} < \text{precio internacional}$
				\4[] $\then$ mejor producir y exportar
				\4[] Tecnologías respectivas + demanda mundial
				\4[] $\to$ Costes relativos nacionales + precio internacional
				\4[] $\then$ Determinan patrón de comercio
				\4 Modelo no define demanda
				\4[] $\to$ Modelo no determina explícitamente PRelativos
				\4 Tecnología de producción
				\4[] Elemento determinante
				\4[] $\to$ Distinta tecnología en cada país
				\4[] $\then$ Determina costes relativos de los bienes
				\4 Diferencias de costes relativos
				\4[$\Rightarrow$] Dan lugar comercio internacional
		\2 Formulación
			\3 Ejemplo tradicional
				\4 Dos países: Portugal, Inglaterra
				\4 Dos bienes: vino, tela
			\3 Portugal
				\4 Tela: 10 unidades de trabajo
				\4 Vino: 5  unidades de trabajo
				\4 $P_\textrm{vino}^\textrm{P} = \frac{5}{10}= 0.5 \, \textrm{tela} / \textrm{vino}$
				\4 $P_\textrm{tela}^\text{P} = \frac{10}{5} = 2 \, \textrm{vino} / \textrm{tela}$
				\4 Dotación $L_P$ de trabajo
			\3 Inglaterra
				\4 Tela: 4 unidades de trabajo
				\4 Vino: 4 unidades de trabajo
				\4 $P_\textrm{vino}^\textrm{I} = \frac{4}{4}= 1 \, \textrm{tela} / \textrm{vino}$
				\4 $P_\textrm{tela} = \frac{4}{4} = 1 \, \textrm{vino} / \textrm{tela}$
				\4 Dotación $L_I$ de trabajo
		\2 Implicaciones
			\3 Ventajas comparativas
				\4 Portugal en vino porque $P_\textrm{vino}^P < P_\textrm{vino}^I$
				\4 Inglaterra en tela porque $P_\textrm{tela}^I < P_\textrm{tela}^P$
				\4[$\then$] Inglaterra producirá tela
				\4[$\then$] Portugal producirá vino
			\3 Especialización completa
				\4 Si:
				\4[] Mercado de vino: $P_\textrm{vino}^\textrm{I} > P_\textrm{vino} > P_\textrm{vino}^\textrm{P}$ \4[] Mercado de tela: $P_\textrm{tela}^\textrm{P} > P_\textrm{tela} > P_\textrm{tela}^I$
				\4 ING prefiere comprar vino internacional
				\4 POR prefiere comprar tela internacional
				\4[$\then$] Hay comercio internacional
				\4[$\then$] Especialización completa de cada país
			\3 Especialización parcial
				\4 Si:
				\4[] Mercado de vino: $P_\textrm{vino}^I = P_\textrm{vino} > P_\textrm{vino}^\textrm{P} $
				\4[] Mercado de tela:  $P_\textrm{tela}^\textrm{P} > P_\textrm{tela} = P_\textrm{tela}^I$
				\4 ING indiferente entre tela nacional e internacional
				\4 POR prefiere comprar tela internacional
				\4[$\then$] Especialización incompleta
				\4[$\then$] Inglaterra produce vino y tela sin especializarse
				\4[$\then$] Portugal se especializa en vino
				\4 Realmente, especialización completa siempre
				\4[] Apertura afecta precios nacionales
				\4[] $\to$ Muy difícil que precios nacionales constantes
				\4[] $\then$ Un país debe ser mucho mayor que otro
			\3 Mejora del ingreso
				\4 Ambos países mejoran su ingreso total
				\4[] $\to$ Tras apertura comercial
				\4 País con ventaja comparativa en vino:
				\4[] $\to$ Aumenta precio de V y especializa en V
				\4[] $\then$ Aumenta ingreso total
				\4 De forma similar con T en otro país
				\4 Modelo no caracteriza distribución de mejora
				\4[] $\to$ No distingue factores productivos
				\4[] $\then$ No es posible valorar perdedores y ganadores
			\3 Representación gráfica
				\4 Oferta y demandas relativas
				\4[] \grafica{ricardoofertasrelativas}
				\4 Maximización de beneficios
				\4[] FPP: $\frac{\bar{L}}{a_y} - P_x \cdot x = \frac{\bar{L}}{a_y} - \frac{a_x}{a_y} \cdot x = y$
				\4[] Isobeneficio (en términos de y):  $\pi = P_x \cdot x + y$
				\4[] Individual: \grafica{ricardofppind}
				\4[] Mundial: \grafica{ricardofppmundial}
			\3 Beneficios en grandes y pequeños
				\4 Países pequeños
				\4[] Apertura al comercio
				\4[] $\to$ No tiene efecto alguno sobre precios internacionales
				\4 Apertura al comercio de país pequeño
				\4[] Demanda de bien más barato en internacional
				\4[] $\to$ Precio se mantiene constante
				\4[] Oferta de bien con ventaja comparativa
				\4[] $\to$ Precio internacional se mantiene constante
				\4[] Representación gráfica
				\4[] $\to$ \grafica{vcbeneficiospequeno}
				\4 Países grandes
				\4[] Su demanda y oferta
				\4[] $\to$ Capaces de afectar precios internacionales
				\4 Apertura al comercio de país grande
				\4[] Demanda bien más barato en el extranjero
				\4[] $\to$ Demanda encarece precio del bien
				\4[] Oferta de bien con ventaja comparativa
				\4[] $\to$ Abarata precio internacional previo a apertura
				\4[] Representación gráfica
				\4[] $\to$ \grafica{vcbeneficiosgrande}
		\2 Valoración
			\3 Generalización a $n$ bienes, $m$ países
				\4 Posible
				\4 Resultados similares
				\4 País $i$ exporta bien $x$ si $P_x^I < P_x$
			\3 Falta de determinación de la relación relativa de intercambio
				\4 Necesario introducir demandas
				\4 Valor normativo de modelo de Ricardo
				\4[] $\to$ Mostrar ventajas del comercio
				\4[] $\to$ Modelar cualitativamente\footnote{Es decir, tratar de describir el patrón de comercio (quién exporta y quién importa qué) pero no la relación relativa de intercambio (quién exporta cuánto a cambio de importar cuánto).}
				\4 Mill
				\4[] Asumiendo teoría de la ventaja comparativa
				\4[] $\to$ Para determinar especialización
				\4[] Determinación de precios relativos a partir de demandas
				\4[] Asumiendo cuenta financiera cerrada
				\4[] $\to$ Equilibrio de BP implica equilibrio CC
				\4[] $\then$ Precios relativos/RRI resulta de equilibrio CC
				\4[] Precios relativos dependen de:
				\4[] $\to$ Demanda de bienes
				\4[] $\to$ Elasticidad de demanda de bienes
				\4[] Si país A tiene:
				\4[] $\to$ Demanda muy elevada de bien A
				\4[] $\to$ Exceso de demanda de A respecto oferta
				\4[] $\then$ Encarecimiento de A
				\4[] $\to$ Elasticidad de demanda de bien A elevada
				\4[] $\to$ Exceso de demanda de A respecto oferta
				\4[] $\then$ Encarecimiento de B
				\4 Marshall
				\4[] Curvas de oferta recíproca
				\4[] Ilustración gráfica de mecanismo de Mill
			\3 Capacidad predictiva
				\4 Muy difícil de estimar
				\4 Evidencia favorable:
				\4[] Diferencias en productividad y patrón de comercio
				\4 Tendencia reciente hacia IDE vertical
				\4[] Procesos de producción fragmentados
				\4[] $\to$ Procesos según ventajas comparativas
			\3 Sin valoraciones de economía política
				\4 Muestra que com. int. puede mejorar ingreso
				\4 No predice nada sobre reparto beneficio
				\4[] $\to$ Dentro de cada país
				\4 Origen de diferencia en productividad
				\4[] Tecnología
				\4[] $\to$ No dotación de factores
				\4[] $\then$ Sin valoraciones sobre distribución de ingreso
	\1 \marcar{El modelo de Heckscher-Ohlin-Samuelson}
		\2 Idea clave
			\3 Contexto
				\4 Historia del modelo
				\4[] Heckscher (1919) y Ohlin (1933)
				\4[] Samuelson
				\4[] Marco basado en eq. general walrasiano
				\4 Relación con modelo neoclásico general
				\4[] Caso particular del modelo neoclásico general
				\4[] Mismas tecnologías de producción en cada país
				\4[] Diferentes dotaciones de factores
				\4[] $\to$ Móviles entre sectores
				\4[] $\to$ Inmóviles entre países
				\4[] $\then$ Explican patrón y precios relativos
			\3 Objetivo
				\4 Explicar patrón de comercio
				\4 Valorar optimalidad del comercio internacional
				\4 Caracterizar efecto de diferentes dotaciones sobre:
				\4[] $\to$ Patrón de comercio
				\4[] $\to$ Remuneración de factores
			\3 Resultados
				\4 Especialización\footnote{Total o parcial, el modelo admite ambas posibilidades.} en bien intensivo en FP abundante
				\4 Apertura a comercio internacional
				\4[] $\to$ Afecta remuneración de factores
				\4[] $\to$ Perjudica a un factor y beneficia a otro
				\4[] $\then$ Ganadores y perdedores del comercio
				\4 Enorme literatura derivada
				\4[] Conocido como ``modelo tradicional''
				\4[] Múltiples extensiones
				\4[] Numerosos intentos de contrastación
				\4[] $\to$ Da lugar a paradoja de Leontieff
				\4 Informa numerosas políticas
				\4[] Punto de partida de análisis de ec. política
		\2 Formulación
			\3 Supuestos básicos
				\4 Modelo 2x2x2x2
				\4[] 2 factores de producción
				\4[] 2 bienes
				\4[] 2 tecnologías
				\4[] 2 economías
				\4 Factores de producción
				\4[] L y K
				\4[] País 1: L relativamente abundante
				\4[] País 2: K relativamente abundante
				\4 2 bienes de consumo
				\4[] X e Y
				\4 2 tecnologías de producción para X e Y
				\4[] Disponibles en ambas economías
				\4[] Diferentes entre sí
				\4[] Idénticas entre economías
				\4[] Intensidad factorial relativa constante\footnote{En términos cualitativos, no cuantitativos. Es decir, qué bien necesita mayor proporción de $\frac{K}{L}$ que otro. No cuánto mayor es la proporción, que sí varía en función de $\frac{w}{r}$.}:
				\4[] $\to$ ¿Producción de X con mayor $K/L$ que Y siempre?
				\4[] $\then$ Siempre
				\4[] $\then$ Isocuantas se cruzan una sola vez
				\4[] Buen comportamiento
				\4[] $\to$ Rendimientos constantes a escala
				\4[] $\to$ Prod. marginal decreciente de cada factor
				\4[] $\to$ Homotéticas
				\4[] $\to$ Sin reversiones de la intensidad factorial
				\4[] $\then$ Sin efectos de escala sobre intensidad factorial
				\4 Economías inicialmente autárquicas
				\4[] A y B
				\4[] Diferente abundancia relativa de factores
				\4 Tecnología de bien X:
				\4[] $\to$ Relativamente intensiva en factor K
				\4 Tecnología de bien Y:
				\4[] $\to$ Relativamente intensiva en factor L
				\4[] $\then$ Producción sesgada en cada país
				\4 En país A, abundante en K
				\4[] $\to$ Producción sesgada hacia X
				\4 En país B, abundante en L
				\4[] $\to$ Producción sesgada hacia Y
				\4 Apertura post-autarquía implica:
				\4[] $\to$ Libre comercio de bienes
				\4[] $\to$ Un sólo precio relativo de bienes
				\4[] $\to$ Ff.pp. inmóviles entre economías
				\4[] $\to$ Ff.pp. móviles entre sectores
				\4[] $\then$ Cambio respecto a precios de autarquía
				\4 Demandas homotéticas
				\4[] Aumento de renta no altera $\text{RMST}_{LK}$
				\4[] $\then$ Agregables en consumidor representativo
			\3 Equilibrio de autarquía
				\4 Equilibrio en mercado de factores
				\4[] $\left| \text{RMST}^X_{L,K} \right| = \left| \text{RMST}^Y_{L,K} \right| = \frac{w}{r}$
				\4[] $\to$ Precios relativos de ff.pp. de autarquía
				\4[] $\to$ Punto en curva de contrato de empresas
				\4[] $\to$ Distintos equilibrios dibujan FPP
				\4 Equilibrio en mercado de bienes
				\4[] $\text{RMT}_{X,Y} = \frac{P_X^*}{P_Y^*}$
				\4[] \grafica{optimoglobal}
				\4[] $\to$ Precios relativos de autarquía
				\4[] Precio más alto si bien más escaso
				\4[] País A: X más abundante
				\4[] $\to$ Bien X relativamente barato
				\4[] País 2: X menos abundante
				\4[] $\to$ Bien X relativamente caro
				\4[] $\then$ $\frac{p_x^2}{p_y^2} > \frac{p_x^1}{p_y^1} $
				\4[] \grafica{hosautarquia}
			\3 Equilibrio de economía abierta
				\4 Un sólo precio relativo de bienes
				\4[] $\to$ Se iguala para ambos países
				\4 FPP se mantienen distintas
				\4[] Puntos sobre FPPs respectivas
				\4[] $\to$ Deben tener misma pendiente
				\4[] $\then$ Cambios en producción autarquía vs apertura
				\4 Precios relativo post-apertura: $\frac{p_x^*}{p_y^*}$
				\4[] $\frac{p_x^2}{p_y^2} > \frac{p_x^*}{p_y^*} > \frac{p_x^1}{p_y^1} $
				\4 Representación gráfica
				\4[] \grafica{hosaperturageneral}
		\2 Implicaciones
			\3 Teorema de Heckscher-Ohlin
				\4 \fbox{Especialización en intensivo en f.p. abundante}
				\4[] Mostrar gráficamente con FPPs
				\4[] \grafica{teoremahos}
				\4[] Cada país se especializa en el bien cuya producción
				\4[] $\to$ Es intensiva en en el factor más abundante
				\4 Consecuencia de:
				\4[] Igualación de precios relativos
				\4[] Nuevo precio relativo común
				\4[] $\to$ Aumenta precio de X donde era bajo
				\4[] $\to$ Reduce precio de X donde era alto
			\3 Teorema de la igualación del precio de los factores
				\4 \fbox{Tras apertura, mismo precio relativo de ff.pp}
				\4[] Mostrar gráficamente con diagrama de Lerner
				\4[] \grafica{lernerigualacion}
				\4[] Curvas representan isoingreso
				\4[] $\to$ Factores para ingresar misma cantidad
				\4[] Rectas representan isocoste
				\4[] $\to$ Factores para sumar mismo coste
				\4[] Puntos de tangencia
				\4[] $\to$ Igualación entre ingreso y coste
				\4[] Línea que une puntos de tangencia
				\4[] $\to$ Coste relativo de los factores de producción
				\4[] Cono de diversificación
				\4[] $\to$ Área entre origen e intensidades relativas óptimas
				\4[] $\to$ Representa área de no especialización completa
				\4[] $\to$ Puntos dentro de cono son combinación de técnicas
				\4[] Igualación de precios de bienes
				\4[] $\then$ Igualación de precios de factores de producción
				\4[] $\to$ A pesar de inmovilidad de factores
				\4 Consecuencia de:
				\4[] Mismo coste de ff.pp. independiente de uso
				\4[] $\to$ No depende del bien para el que se utilice
				\4[] Asumiendo competencia perfecta
				\4[] $\to$ Beneficios nulos
				\4[] $\to$ Coste total iguala precio
				\4[] $\to$ Precio de 1 unidad igual a coste de producción
				\4[] $\then$ Dependencia entre PRelativos y coste relativo
				\4[] Si precios relativos de bienes se igualan
				\4[] $\to$ Precio de factores se iguala también
			\3 Teorema de Stolper-Samuelson
				\4 \fbox{$\uparrow$ Precio relativo de X $\then$ $\uparrow$ coste f.p. intensivo de X}
				\4[] Mostrar gráficamente en diagrama de Lerner
				\4[] Misma gráfica anterior
				\4 Consecuencia de:
				\4[] Ejemplo: aumenta precio de X (intensivo en L)
				\4[] Menos ff.pp. para producir unidad de ingreso de X
				\4[] $\to$ Isoingreso se desplaza a izquierda
				\4[] Nueva isocoste que une curvas isoingreso
				\4[] $\to$ Necesariamente $\uparrow$ pendiente en valor absoluto
				\4[] $\to$ Aumenta coste relativo de L
				\4[] $\then$ Aumenta coste de factor en que X es intensivo
				\4 Especialización completa
				\4[] No se cumple si la especialización es completa
				\4[] Isocoste ya no tienen porque ser tangentes a dos isoingreso
				\4[] $\to$ Sólo al isoingreso de especialización completa
				\4[] $\then$ $\nexists$ vínculo costes relativos-precios relativos
			\3 Teorema de Rybczynski
				\4 \fbox{$\uparrow$ dot. de f.p. $\to$ $\uparrow$ bien intensivo en f.p, $\downarrow$ otro bien}
				\4[] Mostrar en diagrama de Lerner
				\4[] (Gráfico anterior simplificado y con cambio de dotación)
				\4[] \grafica{lernerrybczynski}
				\4[] Vectores deben alcanzar nuevo punto $(\bar{L'},\bar{K'})$
				\4[] $\to$ Más de vector de bien intensivo cuyo factor $\uparrow$
				\4[] $\to$ Menos de vector de bien intensivo en factor =
				\4[] Aumento de dotación de trabajo implica:
				\4[] $\to$ Aumento de producción de X (intensivo en L)
				\4[] $\to$ Disminución de producción de Y (intensivo en K)
				\4 Consecuencia de:
				\4[] Aumento de dotación de L permite
				\4[] $\to$ Aumentar producción
				\4[] Asumiendo costes relativos y precios constantes
				\4[] $\to$ Preferible aumentar producción de bien intensivo en L
				\4[] $\to$ Aumentar intensivo en K con más L sería ineficiente
				\4[] $\then$ Aumenta producción de bien X intensivo en L
				\4[] Producción de bien intensivo en L requiere también
				\4[] $\to$ Utilizar factor K cuya dotación es constante
				\4[] $\then$ Disminuye producción de bien Y intensivo en K
			\3 Teorema de Heckscher-Ohlin-Vanek
				\4 \fbox{Países exportan servicios netos del f.p abundante}
				\4[] Cantidad de factor utilizado en exportación
				\4[] -- cantidad de factor utilizado en importación
				\4[] = contenido factorial del comercio/exp. de servicios netos
				\4[] Exportación de servicios netos
				\4[] $\to$ Corresponde a abundancia relativa de ff.pp.
				\4 Ejemplo:
				\4[] País con abundancia relativa de K
				\4[] Exportaciones:
				\4[] $\to$ Contenido factorial positivo en K
				\4[] Importaciones:
				\4[] $\to$ Contenido factorial positivo en L
			\3 Ganancias del comercio
				\4 Ambos países aumentan ingreso total
				\4[] Acceden a combinaciones de bienes
				\4[] $\to$ Más caras
				\4[] $\to$ Preferibles a autarquía con dda. homotética
				\4 Pero diferentes ff.pp. reciben distinto beneficio
				\4[] Ff.pp. intensivo en bien de especialización
				\4[] $\to$ Aumenta ingreso
				\4[] Ff.pp. no intensivo en bien de especialización
				\4[] $\to$ Disminuye ingreso
				\4 Ejemplo habitual:
				\4[] Países desarrollados abundante K
				\4[] $\to$ Produce más bien intensivo en K
				\4[] Países en desarrollo abundante L
				\4[] $\to$ Produce más bien intensivo en L
				\4[] $\then$ $\frac{w}{r}$ relativamente alto en desarrollados
				\4[] Apertura comercial
				\4[] En desarrollado:
				\4[] $\to$ Aumenta precio de intensivo en K
				\4[] $\to$ Se reduce precio de intensivo en L
				\4[] $\to$ Desarrollado se especializa en intensivo en K
				\4[] $\to$ $\uparrow$ $r$, $\downarrow$ $w$
				\4[] $\then$ Trabajadores desarrollados pierden
				\4[] $\to$ En desarrollo se especializan en intensivo en L
				\4[] $\to$ $\uparrow$ $w$, $\downarrow$ $k$
				\4[] $\to$ Trabajadores en desarrollo ganan
				\4[] $\then$ L en PEDs prefiere apertura
				\4[] $\then$ K en desarrollados prefiere apertura
				\4[] $\then$ Tensión antiglobalización en desarrollados
		\2 Extensiones
			\3 Modelo de factores específicos Ricardo-Viner
				\4 Viner (1971)
				\4 Caso particular del modelo neoclásico general
				\4[] Entendible como variación de Ricardo
				\4[] Entendible como H-O-S de c/p
				\4[] $\to$ Porque K no es móvil de un sector a otro
				\4 Factores de producción
				\4[] Trabajo
				\4[] $\to$ Móvil entre sectores
				\4[] Capital en sector X
				\4[] $\to$ Inmóvil
				\4[] Capital en sector Y
				\4[] $\to$ Inmóvil
				\4 Condición de óptimo
				\4[] Valor de producto marginal de trabajo
				\4[] $\to$ Igual a salario común a ambos sectores
				\4[] $\then$ Igual valor de producto marginal en ambos
				\4[] $P_x \cdot \text{PMgL}_x = w$
				\4[] $P_y \cdot \text{PMgL}_y = w$
				\4[] $\then$ $\frac{P_x}{P_y} \cdot \text{PMgL}_x = \text{PMgL}_y$
				\4[] L se distribuye entre X e Y para cumplir condición
				\4[] \grafica{factoresespecificos}
				\4 Apertura al comercio
				\4[] Cambio en precios relativos
				\4[] P.ej.: aumento de $P_y$
				\4[] $\then$ Aumento del salario de equilibrio
				\4[] $\then$ Aumenta L dedicado a Y
				\4[] $\then$ Disminuye L dedicado a X
				\4[] Cambio en productividad marginal de K
				\4[] $\to$ PMgK de X baja
				\4[] $\to$ PmgK de Y aumenta
				\4[] $\then$ Propietarios de K de Y ganan y de X pierden
				\4[] Diferente impacto de apertura
				\4[] $\then$ Un sector gana y otro pierde
				\4[] $\then$ Perdedores y ganadores no sólo entre ``clases''
			\3 Economía política
				\4 Concepto de economía política (sentido moderno)
				\4[] Análisis económico de actuaciones de agentes
				\4[] $\to$ Que tienen como objetivo redistribución de la renta
				\4 Contexto previo sin apertura
				\4[] Teorema de S-S
				\4[] $\to$ F.p. intensivo en bien de especialización
				\4[] $\then$ Mejora con apertura
				\4[] $\to$ F.p. intensivo en bien de importación
				\4[] $\then$ Pierde con comercio
				\4[] Países donde trabajo es relativamente escaso
				\4[] $\to$ Trabajadores pierden con apertura
				\4[] $\to$ Capitalistas ganan
				\4 Decisión sobre liberalización mediante voto
				\4[] Votación apertura vs protección
				\4[] $\to$ Votante mediano decide
				\4[] $\then$ Probablemente protección gana
				\4 Capitalistas pueden proponer redistribución
				\4[] Aceptan redistribución si gana apertura
				\4[] $\to$ Alcanzar mejora de Pareto
				\4 Si redistribución o proceso de voto costoso
				\4[] Apertura puede no compensar ganancias
				\4[] $\to$ Capitalistas aceptan no redistribuir
			\3 Reversiones de la intensidad factorial\footnote{Gandolfo pág. 84.}
				\4 Intensidades relativas de producción
				\4[] Cambian en función de precios
				\4[$\to$] ¿Se siguen cumpliendo teoremas?
				\4 Ejemplo:
				\4[] Con $w/r$ bajo
				\4[] $\to$ X es relativamente intensivo en L
				\4[] Con $w/r$ alto
				\4[] $\to$ Y es relativamente intensivo en trabajo
				\4 Requisito gráfico
				\4[] Isocuantas se corten más de una vez
				\4 Cumplimiento de Teorema de H-O
				\4[] Si $\nexists$ reversión entre precios respectivos de autarquía
				\4[] $\to$ H-O se cumple
				\4[] Si $\exists$ reversión entre p. respectivos de autarquía
				\4[] $\to$ H-O sólo se cumple en un país
			\3 Costes de transporte
				\4 Precios relativos no se igualan totalmente
				\4[] Más alto en país importador
				\4 Oportunidades de comercio perdidas
				\4[] Precio de importación suficientemente elevado
				\4[] $\to$ Desincentiva el comercio
				\4 Costes de transporte pueden inducir CI
				\4[] Aparentemente contraintuitivo
				\4[] Sucede con paíss muy grandes
				\4[] Ejemplo: Canada y EEUU
				\4[] $\to$ Seattle y Vancouver comercian más que Seattle y Nueva York
			\3 Generalización de Heckscher-Ohlin-Vanek
				\4 Relación entre número de bienes y ff.pp
				\4[] Factores específicos
				\4[] $\to$ Más factores que bienes
				\4[] H-O-S
				\4[] $\to$ Igual número de bienes que factores
				\4[] H-O-V
				\4[] $\to$ Generalización para bienes > ff.pp.
				\4 Efectos de marco H-O-V
				\4[] Igualación de precios de ff.pp.
				\4[] $\to$ Se sigue cumpliendo
				\4[] Heckscher-Ohlin
				\4[] $\to$ No se cumple
				\4[] $\to$ Producción indeterminada de cada bien
				\4[] Heckscher-Ohlin-Vanek
				\4[] $\to$ Se sigue cumpliendo
			\3 Bienes no comerciables
			\3 Ofertas elásticas de ff.pp.
				\4 Puede inducir curvas de oferta relativa anómalas
				\4 Ejemplo:
				\4[] Aumento de precio de bien X intensivo en L
				\4[] $\to$ Aumenta precio de L o salario
				\4[] $\to$ Aumento de salario reduce oferta de trabajo (ER)
				\4[] $\then$ Por t. de Rybczynski, cae oferta de X
		\2 Valoración
			\3 Paradoja de Leontieff
				\4 Análisis del comercio de EEUU años 40
				\4[] ¿Se cumplen predicciones de H-O?
				\4[] $\to$ Asumiendo en EEUU el K es relativamente abundante
				\4[] Teniendo en cuenta:
				\4[] $\to$ Exportaciones
				\4[] $\to$ Importaciones que pueden ser sustituidas
				\4 Resultados de Leontieff
				\4[] Exporta bienes intensivos en trabajo
				\4[] Importa bienes intensivos en capital
				\4[] $\then$ Contrario a predicción de H-O
				\4 Explicaciones a paradoja
				\4[] (I): fallos estimación empírica de Leontieff
				\4[] (II): modelo H-O no describe la realidad
				\4[I] Fallos en estimación empírica
				\4[] --Trabajo americano es más eficiente
				\4[] $\to$ Equivale a más unidades
				\4[] --Hay otros ff.pp. relevantes
				\4[] $\to$ Recursos naturales
				\4[] $\to$ Capital humano
				\4[] $\then$ Sesgan resultados
				\4[] --Balanza comercial no está en equilibrio
				\4[] $\to$ Valor de trabajo importado es menor que capital
				\4[] $\then$ Sí se cumple H-O pero intertemporalmente
				\4[II] Supuestos de H-O no se cumplen
				\4[] -- Demandas diferentes
				\4[] $\to$ USA demanda bienes intensivos en K
				\4[] -- Ff.pp. relevantes son inmóviles entre países
				\4[] $\to$ Capital es móvil
				\4[] $\to$ L skilled y unskilled son ff.pp. relevantes
			\3 Influencia
				\4 Modelo central de teoría del CI
				\4[] Muchos modelos derivados
				\4[] Relevante en la actualidad
				\4 Globalización y apertura comercial
				\4[] H-O utilizado como explicación
				\4[] L en desarrollados menos salario
				\4[] $\to$ Movimientos antiglobalización en desarrollados
	\1[] \marcar{Conclusión}
		\2 Recapitulación
			\3 Modelo clásico de la ventaja comparativa
			\3 Modelo de Heckscher-Ohlin-Samuelson
			\3[$\then$] Casos particulares modelo neoclásico
				\4 Generalizables en modelo walrasiano
				\4[] Casos particulares
				\4[] $\to$ Idénticos ff.pp., distinta tecnología
				\4[] $\to$ Más ff.pp. que bienes
				\4[] $\to$ Idéntica tecnología, distintos ff.pp.
				\4[] $\to$ Más bienes que ff.pp.
				\4[] ...
		\2 Idea final
			\3 Comercio internacional en la actualidad
				\4 Voces críticas respecto a globalización
				\4 Amenazas proteccionistas recientes
				\4 Modelos anteriores
				\4[] Fundamento teórico de apertura reciente
				\4[] Economía mundial entre-guerras
				\4[] $\to$ Obstáculos al comercio
			\3 Fenómenos no explicados
				\4 Comercio intraindustrial
				\4 Relación tamaño y distancia con comercio
				\4 Efectos de competencia imperfecta
				\4[] $\to$ Competencia monopolística
				\4[] $\to$ Oligopolios
				\4 Crecimiento y comercio internacional
				\4[] Efecto sobre acumulación de ff.pp.
				\4 Movilidad de factores productivos
				\4[] Flujos de capital
				\4[] Movimientos migratorios
\end{esquemal}





























\graficas



\begin{axis}{4}{Equilibrio de oferta y demandas relativas en el modelo de Ricardo-Torrens.}{$\frac{\textrm{vino}}{\textrm{tela}}$}{$p_\textrm{vino}$}{ricardoofertasrelativas}

	% oferta relativa
	
	\draw[-,color=blue,line width=2pt] (0,0) -- (0,1.5) -- (2,1.5) -- (2,3.1) -- (4,3.1);
	
	\node[left] at (0,1.5){$p_\textrm{vino}^\textrm{P}$};
	
	\draw[dashed, opacity=0.7] (0,3.1) -- (2,3.1);
	\node[left] at (0,3.1){$p_\textrm{vino}^\textrm{I}$};
	
	\draw[dashed, opacity=0.7] (2,1.5) -- (2,0);
	\node[below] at (2,0){$ \frac{L_P / a_\textrm{vino}^P }{ L_I / a_\textrm{tela}^I} $};
	
	% demanda relativa
	
	\draw[-, color=red] (0.75, 4) to [out=280, in=181](4,2);
		
\end{axis}

\begin{axis}{4}{Representación del problema de maximización del ingreso \textrm{individual} en el modelo de la ventaja comparativa de Ricardo.}{Tela}{Vino}{ricardofppind}

\draw[-] (0,3) -- (3.9,0);
\draw[dashed] (1.5,4) -- (3.9,0);

\node[right] at (3.3,1){RPresupuestaria};
\node[left] at (1.3,1.8){FPP};

\end{axis}

\begin{axis}{4}{Representación del problema de maximización del ingreso \textrm{mundial} en el modelo de la ventaja comparativa de Ricardo.}{Tela}{Vino}{ricardofppmundial}
	\draw[-] (0,3.5) -- (2.1,2.2);
	\draw[-] (2.1,2.2) -- (3.8,0);

	\draw[dashed] (.2,3) -- (2.9,1);
	\draw[dashed] (.55,3.35) -- (3.25, 1.35);
	\draw[dashed] (.9, 3.7) -- (3.6, 1.7);
	
	\node[right] at (3.6,1.7){Isobeneficio};
	\node[right] at (3.7,0.4){FPP};
	
	\node[circle, fill=black, inner sep=0pt, minimum size=5pt] (a) at (2.1,2.2) {};
\end{axis}

\begin{axis}{4}{Modelo de la ventaja comparativa. Beneficios del comercio en un país pequeño sin capacidad de afectar a los precios internacionales.}{Tela}{Vino}{vcbeneficiospequeno}
	% Precios de autarquía
	\draw[-] (0,2.5) -- (2,0);
	\draw[-{Latex}] (1.7,0.1) -- (1.7,-1);
	\node[below] at (1.8,-1){$\frac{P_P^T}{P_P^V}$};

	%  Precios internacionales
	\draw[-] (0,2.5) -- (3.5,0);
	\draw[-{Latex}] (3.2,0.1) -- (3.2,-0.5);
	\node[below] at (3.2,-0.5){$\frac{P_W^T}{P_W^V}$};

	
	% Curva de demanda
	\draw[-] (1,2.42) to [out=270, in=180](4,0.92);

\end{axis}

\begin{axis}{4}{Modelo de la ventaja comparativa. Beneficios del comercio en un país grande con capacidad de afectar a los precios internacionales.}{Tela}{Vino}{vcbeneficiosgrande}
	% Precios de autarquía
	\draw[-] (0,2.5) -- (2,0);
	\draw[-{Latex}] (1.7,0.1) -- (1.7,-1);
	\node[below] at (1.8,-1){$\frac{P_P^T}{P_P^V}$};

	%  Precios internacionales
	\draw[-] (0,2.5) -- (3.5,0);
	\draw[-{Latex}] (3.2,0.1) -- (3.2,-0.5);
	\node[below] at (3.2,-0.5){$\frac{P_W^T}{P_W^V}$};

	%  Precios internacionales tras apertura
	\draw[dashed] (0,2.5) -- (2.5,0);
	\draw[dashed,-{Latex}] (2.3,0.1) -- (2.3,-0.5);
	\node[below] at (2.3,-0.5){$\frac{{P'}_W^T}{{P'}_W^V}$};
	
	% Curva de demanda pre-apertura
	\draw[-] (1,2.42) to [out=270, in=180](4,0.92);

	% Curva de demanda post-apertura
	\draw[dashed] (0.8,2.22) to [out=270, in=180](3.8,0.72);
\end{axis}

La restricción presupuestaria discontinua muestra cómo, una vez tiene lugar la apertura, el precio internacional de la tela aumenta como resultado de la mayor demanda del país grande, que se ha especializado en vino de manera completa. Ello resulta en una curva de indiferencia más hacia el origen y por tanto representantiva de menor bienestar en relación a la que podría haber alcanzado de haberse mantenido constantes los precios internacionales (como sí sucede para el caso de un país pequeño mostrado más arriba).

\begin{axis}{4}{Óptimo global de Pareto: producción y consumo.}{$X$}{$Y$}{optimoglobal}
	% FPP
	\draw[-] (0,3.5) to [out=340, in=100](3.5,0);
	
	% Recta tangente a FPP que muestra RMT
	\draw[-] (1.8,2.77) -- (3.2,1.334);
	
	% Punto de origen de la caja de Edgeworth sobre FPP
	\node[circle, fill=black, inner sep=0pt, minimum size=5pt] (a) at (2.45,2.1) {};
	
	% Caja de Edgeworth
	\draw[-] (0,2.1) -- (2.45,2.1) -- (2.45,0);
	\draw[-{Latex}] (0,0) -- (2.45,0);
	\draw[-{Latex}] (0,0) -- (0,2.1);
	\draw[-{Latex}] (2.45,2.1) -- (0,2.1);
	\draw[-{Latex}] (2.45,2.1) -- (2.45,0);
	
	\node[below] at (2.45,0){\tiny $X_A$};
	\node[left] at (0,2.1){\tiny $Y_A$};
	
	\node[above] at (0.2,2.1){\tiny $X_B$};
	\node[right] at (2.45,0.15){\tiny $Y_B$};
	
	
	% Recta dentro de caja de Edgeworth con misma pendiente que RMT
	\draw[-] (0.8,1.77) -- (2.2, 0.334);
	
	% Curva de indiferencia 
	\draw[-] (1.08,1.85) to [out=280, in=170](2.28,0.68);
	
	% Curva de indiferencia inversa
	\draw[-] (0.54,1.6) to [out=350, in=100](1.84,0.24);
	
	\node[circle, fill=black, inner sep=0pt, minimum size=5pt] (a) at (1.4,1.14) {};
\end{axis}

\begin{axis}{4}{Equilibrio de autarquía en modelo de Heckscher-Ohlin}{$X$}{$Y$}{hosautarquia}

	% 1: FPP de país que produce más de X
	\draw[-] (0,2.1) to [out=-2,in=95](3.7,0);
	\node[left] at (3.6,0.3){{\footnotesize 1}};
	
	% Tangente a FPP de 1
	\draw[-] (2,2) -- (3.6,0.9);
	\node[right] at (3.6,0.9){{\tiny $\frac{p_x^1}{p_y^1}$ }};
	\node[circle, fill=black, inner sep=0pt, minimum size=3pt] (a) at (2.9,1.37) {};

	% 2: FPP de país que produce más de Y
	\draw[-] (0,3.7) to [out=-2, in=95](2.3,0);
	\node[below] at (0.5,3.6){{\footnotesize 2}};

	% Tangente a FPP de 2
	\draw[-] (1.37,3.2) -- (2.07,1.8);
	\node[right] at (1.37,3.2){{\tiny $\frac{p_x^2}{p_y^2}$}};
	\node[circle, fill=black, inner sep=0pt, minimum size=3pt] (a) at (1.73,2.5) {};
	
\end{axis}

En el gráfico se muestran los precios relativos en situación de autarquía dada una demanda correspondiente a un consumidor representativo (no dibujada). En el país 1 la FPP está sesgada hacia la producción del bien X. En el país 2 sucede al contrario: la producción está sesgada hacia la producción de bien Y. Los precios relativos en cada país reflejan este hecho. En el país 2, donde el bien Y es más abundante que el bien X, el precio de X en términos de Y es relativamente elevado respecto al país 1 (pendiente alta en valor absoluto), donde el precio de X en términos de Y es mucho menor (pendiente baja en valor absoluto). Es decir, el bien X es caro allí donde es relativamente escaso, y de forma opuesta sucede con el bien Y.


\begin{dibujo}{4}{Modelo Heckscher-Ohlin-Samuelson. Efectos de la apertura sobre precios relativos de bienes, costes de factores de producción e intensidades factoriales relativas.}{}{}{hosaperturageneral}
	% Ejes
	\draw[-] (-6,0) -- (0,0) -- (0,4) -- (0,0) -- (8,0);
	\node[below] at (8,0){$K/L$};
	\node[above] at (0,4){$w/r$};
	\node[below] at (-6,0){$P_x/P_y$};

	% Bien X intensivo en L
	\draw[-] (0,0) -- (2,4);
	\node[above] at (2,4){X};

	% Bien Y intensivo en K
	\draw[-] (0,0) -- (8,4);
	\node[above] at (8,4){Y};

	% Relación entre precios relativos de X e Y y precios relativos de L y K
	\draw[-] (0,0) to [out=170, in=280](-4,4);

	% Pre-apertura en H
	\draw[dashed] (-1.5, 0) -- (-1.5,0.5);
	\node[below] at (-1.5,0){\tiny $\frac{P_x^H}{P_y^H}$};

	\draw[dashed] (-1.5,0.5) -- (1,0.5) -- (1,0);
	\node[below] at (1,0){\tiny $\frac{K^H_Y}{L^H_Y}$};

	\draw[dashed] (0.26, 0.5) -- (0.26,0);
	\node[below] at (0.26,0){\tiny $\frac{K^H_X}{L^H_X}$};

	% Pre-apertura en F
	\draw[dashed] (-3.8, 0) -- (-3.8,3.25) -- (6.45,3.25) -- (6.45,0);
	\draw[dashed] (1.6,3.25) -- (1.6,0);
	\node[below] at (-3.8,0){\tiny $\frac{P_X^F}{P_Y^F}$};
	\node[below] at (1.6,0){\tiny $\frac{K^F_X}{L^F_X}$};
	\node[below] at (6.45,0){\tiny $\frac{K^F_Y}{L^F_Y}$};
	
	% Tras apertura
	\draw[dotted](-3,0) --  (-3,1.8) -- (3.5,1.8) -- (3.5,0);
	\node[below] at (-3,0){\tiny $\frac{P_X^*}{P_Y^*}$};
\end{dibujo}

El gráfico representa los efectos de la apertura comercial sobre los costes relativos de los factores, los bienes producidos y la intensidad relativa del capital. Asumimos diferentes dotaciones relativas de capital y trabajo en dos países H y F, de tal manera que en el país F el capital es relativamente abundante, y en el país H el trabajo es relativamente abundante. Ello resulta en que coste relativo del trabajo respecto al capital es más elevado en el país F que en el H. Por otro lado, asumimos que el bien Y es relativamente intensivo en capital en relación al bien X, relativamente intensivo en trabajo. 

En la situación previa a la apertura, los trabajadores disfrutan en el país F de un salario más elevado que en el país H. Además, el precio relativo del bien X es más elevado en el país F que en el H. De forma inversa, los trabajadores tienen un salario más bajo en el país H, y el bien X, intensivo en trabajo, es más barato que el bien Y, intensivo en capital. La apertura al comercio induce diferentes efectos en cada país. En el país F, con un salario inicial relativamente alto, la apertura supone una bajada del salario y un abaratamiento de los bienes intensivos en trabajo, que ahora pueden importarse a menor precio del país H. Así, los trabajadores de F, que eran anteriormente propietarios de un factor relativamente escaso, pierden renta relativa por el menor precio del bien intensivo en trabajo y la consiguiente caída del salario. En el país H, sucede al contrario. La dotación de capital es relativamente inferior a la del país F en relación a la dotación de trabajo. Ello resulta en un elevado coste del capital en relación al coste del trabajo, y un elevado precio relativo del bien Y intensivo en capital. Con la apertura, el precio relativo del bien Y cae en relación al precio de X por la posibilidad de importarlo de F, abundante en capital. Ello resulta en una caída de la remuneración del factor capital en H.

Así, podemos resumir las condiciones previas a la apertura y los efectos de ésta de la siguiente forma: 

\begin{itemize}
	\item Bien X -- Intensivo en trabajo.
	\item Bien Y -- Intensivo en capital. 
	\item País F -- Relativamente abundante en capital. 
	\item País H -- Relativamente abundante en trabajo.
	\item Coste del trabajo -- Relativamente alto en F.
	\item Coste del capital -- Relativamente alto en H.
	\item Perjudicado por apertura en F -- Factor trabajo que era relativamente escaso. Apertura reduce remuneración relativa del trabajo. Apertura reduce remuneración relativa por Stolper-Samuelson.
	\item Perjudicado por apertura en H -- Factor capital que era relativamente escaso. Apertura reduce remuneración relativa. Apertura reduce remuneración relativa por Stolper-Samuelson.
\end{itemize}


\begin{axis}{4}{Equilibrio tras apertura al comercio de bienes en modelo de Heckscher-Ohlin: mejoras inequívocas del bienestar en ambos países.}{$X$}{$Y$}{teoremahos}
	% 1: FPP de país que produce más de X
	\draw[-] (0,2.1) to [out=-2,in=95](3.7,0);
	\node[left] at (3.6,0.3){{\footnotesize 1}};
	
	% 1: Tangente a FPP de 1 en autarquía
	\draw[-] (2,2) -- (3.6,0.9);
	\node[circle, fill=black, inner sep=0pt, minimum size=3pt] (a) at (2.9,1.37) {};
	\node[right] at (2.93,1.37){\tiny $P_1$};

	% 1: Tangente a FPP de 1 con libre comercio
	\draw[-] (2.86,1.5) -- (3.86,0.4);
	\node[circle, fill=black, inner sep=0pt, minimum size=3pt] (a) at (3.25,1.08) {};
	\node[right] at (3.25,1.12){\tiny $P_1$'};
	
	% 1: Transición desde autarquía a libre comercio
	\draw[-{Latex}] (2.8,1.27) -- (3.15,0.98);
	
	% 2: FPP de país que produce más de Y
	\draw[-] (0,3.7) to [out=-2, in=95](2.3,0);
	\node[below] at (0.5,3.6){{\footnotesize 2}};
	
	% 2: Tangente a FPP de 2 en autarquía
	\draw[-] (1.37,3.2) -- (2.07,1.8);
	\node[circle, fill=black, inner sep=0pt, minimum size=3pt] (a) at (1.73,2.5) {};
	\node[right] at (1.73,2.5){\tiny $P_2$};

	% 2: Tangente a FPP de 2 con libre comercio
	\draw[-] (0.91,3.6) -- (1.81,2.5);
	\node[circle, fill=black, inner sep=0pt, minimum size=3pt] (a) at (1.4,3) {};
	\node[left] at (1.4,3){\tiny $P_2$'};

	% 2: Transición desde autarquía a libre comercio
	\draw[-{Latex}]	(1.63,2.4) -- (1.3,2.9);
	
\end{axis}

Tanto $P_1$ como $P_2$ eran el resultado de la tangencia entre la curva de indiferencia del consumidor representativo (no representada) y la curva de la frontera de posibilidades de producción. La apertura al comercio induce en 1 un aumento del precio del bien X en relación al precio de Y, de manera tal que es posible acceder a combinaciones de X e Y pareto-superiores al punto de equilibrio de autarquía $P_1$. De manera similar, en 2, la apertura al comercio induce una abaratamiento del precio de Y respecto a la situación de autarquía, lo que permite acceder a combinaciones de X e Y pareto-superiores al punto de equilibrio de autarquía $P_1$.

El gráfico aporta también conclusiones respecto al efecto de la apertura sobre la especialización relativa. En el país 1, la frontera de posibilidades de producción está sesgada hacia la producción del bien X. La apertura provoca un encarecimiento del bien X respecto del bien en relación a la situación previa a la apertura, lo que induce un aumento de la especialización relativa en la producción de X en el país 1. De forma inversa, en el país 2, la apertura supone un abaratamiento del bien X respecto del precio previo a la apertura. Ello equivale a un encarecimiento relativo del bien Y, a favor del cual está sesgada la FPP en el país 2. La apertura induce así una mayor especialización del país 2 en la producción del bien Y. 



\begin{axis}{4}{Teorema de la igualación del coste relativo de los factores de producción representado sobre el diagrama de Lerner}{L}{K}{lernerigualacion}
	% extender eje de abscisas
	\draw[-] (4,0) -- (6,0);
	
	% extender eje de ordenadas
	\draw[-] (0,4) -- (0,6);
	
	% Y: AUTARQUÍA y LIBRE COMERCIO - isocuanta de tecnología intensiva en K
	\draw[-] (1,5) to [out=271, in=179](3,3);
	\node[above] at (1,5){\tiny $Y=1$};
	
	% X: AUTARQUÍA - isocuanta de tecnología intensiva en L de autarquía
	\draw[-] (3,3.5) to [out=271, in=179](5,1.5);
	\node[right] at (5,1.5){\tiny $X=1/P_x$};
	
	% X: LIBRE COMERCIO -isocuanta de tecnología intensiva en L 
	\draw[-] (2.4,3.1) to [out=271, in=179](4.4,1.1);
	\node[below] at (4.4,1.1){\tiny $X=1/P_x^*$};
	
	% isocoste de autarquía
	\draw[-] (0,4.75) -- (6.33,0);
	\node[right] at (6.05,0.3){\tiny $-\frac{w}{r}$};
	% punto de tangencia con bien Y
	\node[circle, fill=black, inner sep=0pt, minimum size=2pt] (a) at (1.8,3.4) {};
	% punto de tangencia con bien X
	\node[circle, fill=black, inner sep=0pt, minimum size=2pt] (a) at (3.8,1.9) {};
	
	% isocoste de libre comercio
	\draw[-] (0,5.7) -- (4.2,0);
	\node[left] at (3.95,0.3){\tiny $-\frac{w^*}{r^*}$};
	% punto de tangencia con bien Y
	\node[circle, fill=black, inner sep=0pt, minimum size=2pt] (a) at (1.4,3.8) {};
	% punto de tangencia con bien X
	\node[circle, fill=black, inner sep=0pt, minimum size=2pt] (a) at (2.8,1.9) {};
	
	% movimiento de la isocuanta al aumentar el precio de X
	\draw[-{Latex}] (3.7,1.9) -- (3.3,1.5);
	
	% movimiento de la recta isocoste que caracteriza precios relativos
	\draw[-{Latex}, line width=0.1mm] (5.4,0.5) -- (4,0.5);
	
	% cono de diversificación de autarquía
	% con bien x
	\draw[dashed] (0,0) -- (6,3);
	% con bien y
	\draw[dashed] (0,0) -- (3.2,6);
	
	% cono de diversificación de libre comercio
	% con bien x
	\draw[dotted] (0,0) -- (6,4.05);
	% con bien y
	\draw[dotted] (0,0) -- (2.2,6);
	
	% transición de conos de diversificación
	\draw[-{Latex}] (5.5,2.9) to [out=70, in=290] (5.5,3.5); 
	
	% dotación de capital y trabajo
	\node[circle, fill=black, inner sep=0pt, minimum size=3pt] (a) at (5.5,4.4) {};
	\node[right] at (5.5,4.6){$(\bar{L}, \bar{K})$};
	% uso de dotación desde situación de autarquía
	\draw[-{Latex},dashed] (4.3,2.15) -- (5.5,4.4););
	% uso de dotación desde situación de libre comercio
	\draw[-{Latex},dotted] (5.18,3.53) -- (5.5,4.4);
\end{axis}

\begin{axis}{4}{Teorema de Rybczynski: aumento de la dotación de un factor de producción y subsecuente aumento de la producción del bien cuya producción es intensiva en ese factor y disminución de la producción del otro bien.}{}{}{lernerrybczynski}
	% extender eje de abscisas
	\draw[-] (4,0) -- (6,0);
	\node[below] at (5,0){L};
	
	% extender eje de ordenadas
	\draw[-] (0,4) -- (0,6);
	\node[left] at (0,6){K};
	
	% Y: AUTARQUÍA y LIBRE COMERCIO - isocuanta de tecnología intensiva en K
	\draw[-] (1,5) to [out=271, in=179](3,3);
	\node[above] at (1,5){\tiny $Y=1$};
	
	% X: AUTARQUÍA - isocuanta de tecnología intensiva en L de autarquía
	\draw[-] (3,3.5) to [out=271, in=179](5,1.5);
	\node[right] at (5,1.5){\tiny $X=1/P_x$};
	
	% isocoste de autarquía
	\draw[-] (0,4.75) -- (6.33,0);
	\node[right] at (6.05,0.3){\tiny $-\frac{w}{r}$};
	% punto de tangencia con bien Y
	\node[circle, fill=black, inner sep=0pt, minimum size=3pt] (a) at (1.8,3.4) {};
	% punto de tangencia con bien X
	\node[circle, fill=black, inner sep=0pt, minimum size=3pt] (a) at (3.8,1.9) {};
	
	% cono de diversificación de autarquía
	% frontera con bien X
	\draw[dashed] (0,0) -- (8,4);
	% con bien y
	\draw[dashed] (0,0) -- (3.2,6);
	
	% DOTACIÓN INICIAL de capital y trabajo
	\node[circle, fill=black, inner sep=0pt, minimum size=3pt] (a) at (5.5,4.4) {};
	\node[left] at (5.5,4.6){$(\bar{L}, \bar{K})$};
	% uso de dotación desde situación de autarquía
	\draw[-{Latex},dashed] (4.3,2.15) -- (5.5,4.4););

	
	% DOTACIÓN POST-AUMENTO de trabajo manteniendo capital constante
	\node[circle, fill=black, inner sep=0pt, minimum size=3pt] (a) at (7.5,4.4) {};
	\node[right] at (7.5,4.6){$(\bar{L'}, \bar{K'})$};
	% transición de una dotación a otra (para mostrar que es línea recta horizontal y sólo aumenta el trabajo)
	\draw[-{Latex}] (5.5,4.4) -- (7.5,4.4);
	\draw[-{Latex},dashed] (7.03,3.51) -- (7.5,4.4);
	
\end{axis}

\begin{axis}{4}{Modelo de factores específicos: equilibrio en mercado de trabajo y efecto de un aumento en el precio de un bien sobre el salario y la asignación del trabajo}{}{}{factoresespecificos}
	% ejes de ordenadas
	% segundo eje de ordenadas
	\draw[-] (4,0) -- (6,0) -- (6,4);
	
	% Origen de bien X
	\node[below] at (0,0){$O_X$};
	% Origen del bien Y
	\node[below] at (6,0){$O_Y$};
	
	% Productividad marginal de X
	\draw[-] (0,3.5) to [out=300,in=170](5,0.5);
	\node[below] at (5,0.6){\tiny $ \text{PMgL}_x$};
	
	% Productividad marginal de Y
	\draw[-] (6,3.5) to [out=240, in=10](1,0.5);
	\node[below] at (5,1.6){\tiny $\frac{P_y}{P_x} \cdot \text{PMgL}_y$};
	
	% salario de equilibrio pre-apertura
	\draw[dashed] (0,1) -- (6,1);
	\node[left] at (0,1){\tiny $w$};
	\node[right] at (6,1){\tiny $w$};
	% trabajo de equilibrio pre-apertura
	\draw[dashed] (3,1) -- (3,0);
	\node[below] at (3,-0.05){\tiny $L$};
	
	% productividad marginal de Y post-apertura
	\draw[-] (6,4.5) to [out=240, in=10](1,1.5);
	\node[below] at (3,2.8){\tiny $\frac{P_y'}{P_x} \cdot\text{PMgL}_y$};
	\draw[-{Latex}] (3.4,1.4) -- (3,1.9);
	
	% trabajo y salario de Y post-apertura
	\draw[dashed]  (0,1.64) -- (1.71,1.64) -- (1.71,0);
	\node[below] at (1.71,0){\tiny $L'$};
	\draw[-{Latex}] (2.9,-0.25) -- (1.81,-0.25);
	\node[left] at (0,1.64){\tiny $w'$};
	\draw[-{Latex}] (-0.5,1) -- (-0.5,1.64);
\end{axis}


\conceptos

\concepto{Inversión de la intensidad factorial}

(Gandolfo, págs. 64-66)

La condición para que no se produzcan inversiones de la intensidad factorial es que las curvas de indiferencia se crucen una sola vez. Si las curvas de indiferencia se cruzan dos veces, o una es tangente a la otra, la reversión de la intensidad factorial se producirá a ambos lados del eje que parte del origen y une los puntos en los que la pendiente de las isocuantas de los dos bienes es la misma.

En ambos casos, y bajo supuestos muy generales, sucede que un aumento del precio de un factor implica una disminución de la intensidad factorial relativa de ese factor. Así, un aumento de $\frac{P_L}{P_K}$ implica un aumento del ratio $\frac{K}{L}$ correspondiente al punto en el que se minimizan los costes de producción para los precios de los factores dados.

La aparición de inversiones de la intensidad factorial está fuertemente ligada a las elasticidades de sustitución. \textit{Grosso modo}, las isocuantas se cruzan dos o más veces cuando una de ellas es más curva que otra. A su vez, el mayor grado de curvatura refleja una mayor dificultad para sustituir un input por otro y por tanto, una menor elasticidad de sustitución.

\concepto{Relación relativa de intercambio\footnote{Palgrave, \textit{terms of trade}.}}

La relación relativa de intercambio o \textit{terms o trade} es una variable que trata de reflejar la proporción entre los bienes exportados y los importados. Esta proporción puede expresarse en diferentes términos:

\begin{description}
	\item[Términos de intercambio neto] Se trata de la formulación más habitual de la relación de intercambio, y la habitual en series estadísticas. Se trata del cociente entre dos índices de precios: uno de precios de exportación y otro de precios de importación:
	
	\begin{equation*}
		\textrm{RRI} = \frac{p_x}{e \cdot p_m}
	\end{equation*}
	
	Los índices $p_x$ y $p_m$ están sujetos a los problemas habituales de los números índices.
	
	\item[Términos de intercambio brutos] En esta formulación, los términos de intercambio son iguales al cociente entre el volumen de importaciones y el volumen de exportaciones. Si la balanza comercial está equilibrada (asumiendo cuenta de capital inexistente o con saldo nulo) de tal manera que no hay transferencias internacionales ni préstamos, el valor de este cociente será (\textit{debería} ser) igual al de la relación relativa de intercambio calculada en términos de intercambio neto. En caso contrario, estaríamos ante una situación de déficit por cuenta corriente si el valor del presente cociente es mayor al de términos de intercambio neto, y una situación de superávit si el presente cociente es mayor al de términos de intercambio neto. 
\end{description}

\preguntas

\seccion{Test 2018}

\textbf{25.} Sean dos países A y B que producen cereal y tela en el marco del Modelo Ricardiano. Si para producir una unidad de cereal el país A necesita 10 horas, y el país B necesita 8 horas, y para producir una unidad de tela el país A necesita 5 horas, y el país B necesita 4 horas, responda:

\begin{itemize}
	\item[a] No se puede obtener ventaja por comerciar.
	\item[b] Si se encuentra una relación internacional de precios de los bienes intermedia entre las de autarquía de los dos países, el país B se especializará en la producción de los dos bienes.
	\item[c] El país B se especializará totalmente en la producción de cereal, pues es relativamente barato.
	\item[d] El país A se especializará en la producción de tela, pues su precio es menor que el del cereal. 
\end{itemize}

\bigskip
\textbf{26.} Según el Teorema de Rybczynski suponga que, ceteris paribus, en un país aumenta la dotación de factor capital. En dicho país:

\begin{itemize}
	\item[a] Aumentará la producción del bien capital intensivo y la relación $w/r$ bajará.
	\item[b] Ambos bienes se producirán con una relación $K/L$ mayor que antes.
	\item[c] Aumentará la producción de los dos bienes, y la relación $w/r$ permanecerá constante.
	\item[d] La proporción $K/L$ utilizada para producir ambos bienes no variará. 
\end{itemize}

\seccion{Test 2017}
\textbf{26.} En un mundo Heckscher-Ohlin:

\begin{itemize}
	\item[a] La libre circulación de capitales no afecta al volumen de comercio.
	\item[b] La libre circulación de capitales incrementa el volumen de comercio.
	\item[c] La libre circulación de capitales reduce el volumen de comercio.
	\item[d] La libre circulación de capitales aumenta el comercio en algunos países y lo reduce en otros.
\end{itemize}

\seccion{Test 2016}

\textbf{29.} Señale la respuesta correcta en relación a los siguientes modelos de comercio internacional y su contrastación empírica:

\begin{enumerate}
	\item[a] El modelo Heckscher-Ohlin nunca ha sido contrastado por falta de datos.
	\item[b] La contrastación empírica del Modelo HO da mejores resultados cuando se tiene en cuenta las diferencias entre los factores efectivamente incorporados en cada economía.
	\item[c] La \comillas{paradoja de Leontief} muestra que las Tablas Input-Output son poco apropiadas para medir contenidos factoriales.
	\item[d] El \comillas{modelo gravitatorio} es el método más utilizado para la contrastación empírica del Modelo HO.
\end{enumerate}

\textbf{33.} En relación a los modelos clásicos y neoclásicos de comercio internacional señale cuál de las siguientes afirmaciones es la correcta:

\begin{enumerate}
	\item[a] En el \comillas{modelo ricardiano} se asume la presencia de dos factores productivos, el trabajo y el capital, que son móviles entre sectores.
	\item[b] En el \comillas{modelo de factores específicos}, se asume la presencia de tres factores productivos, trabajo, $K_x$ y $K_y$. Sólo el trabajo es móvil entre sectores.
	\item[c] En el modelo H-O, se asume la presencia de dos factores, el trabajo y el capital. Ambos son fijos entre sectores, pues se trata de un modelo de corto plazo.
	\item[d] En el modelo H-O, se asume la presencia de dos factores, el trabajo y el capital. Ambos son móviles entre sectores y entre países.
\end{enumerate}

\textbf{34.} En relación a los modelos clásicos de comercio internacional:
\begin{enumerate}
	\item[a] Según el \comillas{modelo de Ricardo} no hay ganancias del comercio porque no se puede saber cómo queda la relación entre los salarios reales de los dos países tras la apertura.
	\item[b] En el \comillas{modelo de factor específico}, el factor específico del bien que no se exporta es el que presenta las mayores ganancias tras la apertura comercial.
	\item[c] En el modelo H-O, el Teorema Stolper-Samuelson, predice que cuando todos los factores son móviles, un incremento en el precio relativo de los bienes incrementará los ingresos reales del factor utilizado intensamente en la producción de tal bien, y decrecerán los ingresos reales del otro factor.
	\item[d] En el modelo de factor específico, el trabajo es el que obtiene las mayores ganancias tras la apertura comercial.
\end{enumerate}



\seccion{Test 2014}

\textbf{28.} Para dos países, dos productos y dos factores, el modelo de H-O-Samuelson establece que:

\begin{enumerate}
	\item[a] Cada país se especializará en aquel producto que se produzca con el factor más abundante en el país suponiendo movilidad perfecta de los factores.
	\item[b] En ausencia de una especialización completa el precio de los factores se igualará en ambos países.
	\item[c] El comercio intraindustrial entre países se produce en el supuesto de que la demanda sea igual en ambos países.
	\item[d] El incremento en la dotación de un factor, sin que varíe el otro, dará lugar a un cambio en los precios relativos, pero no en las cantidades producidas.
\end{enumerate}

\seccion{Test 2013}

\textbf{29.} En el contexto de la teoría de H-O, sean dos países A y B que producen dos bienes 1 y 2. En los dos países el bien 1 es intensivo en capital respecto al bien 2 y en el país A la cantidad total del capital respecto al trabajo es menor que en el país B. En los dos países la preferencia por los dos bienes y las tecnologías disponibles son las mismas. Siendo $r$ la tasa de remuneración del capital y $w$ los salarios, es verdad que:

\begin{enumerate}
	\item[a] El país B tiene ventaja comparativa en la producción del bien 2.
	\item[b] El $P_1/P_2$ de equilibrio sin comercio es mayor en el país B.
	\item[c] $r/w$ es menor en el país B que en el país A, en la situación de autarquía.
	\item[d] Ninguna de las anteriores. 
\end{enumerate}

\seccion{Test 2011} 

\textbf{25.} A la vista de la siguiente tabla, se puede afirmar que:

\medskip

\begin{tabular}{l | c | c}
& EEUU & Reino Unido \\ \hline
Trigo (arrobas/horas-hombre) & 6 & 1 \\ \hline
Tela (yardas/horas-hombre) & 4 & 2 \\ \hline
\end{tabular}

\medskip

\begin{enumerate}
	\item[a] Reino Unido tiene ventaja absoluta en ambos bienes.
	\item[b] Estados Unidos tiene ventaja comparativa en la producción de trigo.
	\item[c] Reino Unido tiene ventaja comparativa en la producción de trigo.
	\item[d] No existen ganancias derivadas del comercio para ninguno de los dos países.
\end{enumerate}

\textbf{26.} El teorema de Rybczynski:

\begin{enumerate}
	\item[a] Afirma que cuando aumenta la dotación de capital y el precio relativo de los factores no cambia, la producción del bien que es intensivo en capital aumenta, en tanto que la del bien intensivo en trabajo disminuye.
	\item[b] Afirma que cuando aumenta la dotación de capital y el precio relativo de los factores no cambia, la producción del bien que es intensivo en capital aumenta tanto como la del bien intensivo en trabajo.
	\item[c] Afirma que cuando aumenta la dotación en capital y el precio relativo de los factores no cambia, la producción del bien que es intensivo en capital disminuye tanto como la del bien intensivo en trabajo.
	\item[d] Generaliza el teorema de Stolper-Samuelson.
\end{enumerate}

\seccion{Test 2009}

\textbf{25.} El modelo de H-O supone que:

\begin{enumerate}
	\item[a] Existe diferente dotación absoluta de factores entre los países y que las intensidades factoriales relativas entre los productos son diferentes.
	\item[b] Existe diferente dotación relativa de factores entre los dos países y que las intensidades factoriales relativas entre los productos son diferentes.
	\item[c] Existe diferente dotación absoluta de factores entre los países y que las intensidades factoriales relativas entre los productos son las mismas.
	\item[d] Existe diferente dotación relativa de factores entre los países y que las intensidades factoriales relativas entre los productos son las mismas.
\end{enumerate}

\seccion{Test 2008}

\textbf{26.} En términos de la teoría neoclásica de comercio internacional, una vez iniciado el comercio la relación de intercambio está determinada por:

\begin{enumerate}
	\item[a] Las condiciones de oferta y demanda del país importador.
	\item[b] Las condiciones de oferta de los países en régimen de autarquía.
	\item[c] Las condiciones de oferta y demanda de ambos países.
	\item[d] Todas las respuestas son falsas.
\end{enumerate}

\textbf{27.} Dados dos países: país A y país B. Ambos producen los mismos bienes: seda y vino.

\medskip

Coste de producción en horas de trabajo:

\begin{tabular}{c | c | c}
\textbf{país} & \textbf{seda} & \textbf{vino} \\ \hline
A & 70 & 20 \\ \hline
B & 35 & 20 \\ \hline	
\end{tabular}
	
\medskip

Señale la respuesta correcta:
	
\begin{enumerate}
	\item[a] El país A tiene ventaja absoluta en la producción de seda.
	\item[b] El país A tiene ventaja relativa en la producción de seda.
	\item[c] El país B tiene ventaja absoluta en la producción de seda.
	\item[d] El país B no tiene ventaja relativa en la producción de seda.
\end{enumerate}

\seccion{Test 2007}

\textbf{26.} Según la teoría de la ventaja comparativa, sería ventajoso para un país:

\begin{enumerate}
	\item[a] Especializarse en aquellos bienes que produce utilizando una menor cantidad de factores productivos, en comparación con otros países.
	\item[b] Especializarse en aquellos bienes que produce con un menor coste absoluto, en comparación con otros países.
	\item[c] Especializarse en aquellos bienes que produce con un menor coste en términos de los otros bienes, en comparación con otros países.
	\item[d] Especializarse en aquellos bienes que produce en condiciones de economías de escala, en comparación con otros países.
\end{enumerate}

\notas

\textbf{2018}: \textbf{25.} A \textbf{26.} D

\textbf{2017}: \textbf{26}. C

\textbf{2016}: \textbf{29}. B \textbf{33}. B \textbf{34}. C

\textbf{2014}: \textbf{28}. B La pregunta A es falsa porque la especialización no depende del bien más abundante, sino del bien \textit{relativamente} más abundante.

\textbf{2013}: \textbf{29}. C

\textbf{2011}: \textbf{25}. B \textbf{26}. A

\textbf{2009}: \textbf{25}. B

\textbf{2008}: \textbf{26}. C \textbf{27}. C

\textbf{2007}: \textbf{26}. C

\bibliografia

Mirar en Palgrave:
\begin{itemize}
	\item comparative advantage
	\item factor prices in general equilibrium
	\item Heckscher-Ohlin trade theory
	\item terms of trade
\end{itemize}

Deardoff, A. V. \textit{Introduction to the Lerner Diagram} (2002) \url{http://www-personal.umich.edu/~alandear/writings/Lerner.pdf} -- En carpeta del tema

Deardoff, A. V. \textit{Lerner Diagram. Glossary of International Economics} (2002) \url{http://www-personal.umich.edu/~alandear/glossary/figs/Lerner/ld.html#} -- Muy buena explicación interactiva del diagrama y su estática comparativa.

Feenstra, R. C. \textit{Advanced International Trade: Theory and Evidence} (2002) -- En carpeta economía internacional

Gandolfo, G. \textit{International Trade and Policy} (2014) 2ed -- En carpeta \textit{Economía Internacional}


\end{document}
