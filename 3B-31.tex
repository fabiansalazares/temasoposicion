\documentclass{nuevotema}

\tema{3B-31}
\titulo{Instituciones multilaterales de financiación y ayuda al desarrollo: el grupo del Banco Mundial y los Bancos Regionales de Desarrollo.}

\begin{document}

\ideaclave

Hay que añadir un apartado sobre el STDF -- Fondo para la Aplicación de Estándares y el Fomento del Comercio de la OMC. Ver \url{http://www.standardsfacility.org/es/qui\%C3\%A9nes-somos}

Las instituciones multilaterales de financiación y ayuda al desarrollo aparecen tras la Segunda Guerra Mundial y han evolucionado hasta convertirse en un importante agente en el sistema económico internacional. El número de instituciones así como sus objetivos y organización han evolucionado con el tiempo, adaptándose al contexto económico global y a las necesidades de los países en desarrollo. Dejando de lado por el momento las especificidades de cada institución concreta, la mayoría guarda una serie de \marcar{elementos en común} que el tema presenta en primer lugar. El problema fundamental que tratan de solventar las instituciones multilaterales de desarrollo es la dificultad de los países menos avanzados para financiar proyectos a largo plazo. Estas economías no gozan de la confianza de los inversores, y por ello les resulta muy dificil captar préstamos a largo plazo o a costes no prohibitivos. Las instituciones multilaterales de desarrollo captan fondos en los mercados financieros internacionales gracias a la solvencia que confiere la garantía presupuestaria de un conjunto muy amplio de países miembros, y canalizan estos fondos hacia proyectos con potencial para generar tasas de retorno suficientes. Además, los márgenes de beneficio de estos préstamos y otras donaciones de los países desarrollados son utilizados para conceder préstamos a interés por debajo de mercado o incluso sin devolución del principal, en la llamada financiación concesional. Los países que se benefician de estos préstamos deben situarse por debajo de determinados niveles de renta o tener problemas de solvencia. Un objetivo adicional de estas instituciones es la generación, consolidación y centralización del conocimiento científico y económico que permite a las economías desarrollarse y aumentar el nivel de vida de sus habitantes. Dado que los problemas de las economías en desarrollo son a menudo muy diferentes a las de los países desarrollados, y que las razones que impiden el desarrollo son diferentes en cada país, resulta muy útil concentrar y centralizar los esfuerzos de investigación para entender mejor las causas y las posibles soluciones, así como la transmisión de este conocimiento a los agentes que actúan sobre el terreno como empleados públicos locales, empresarios o trabajadores. 

A medida que el contexto económico mundial ha evolucionado, las instituciones multilaterales de financiación también lo han hecho. El germen de la primera institución, el \marcar{Banco Mundial}, se encuentra en el final de la Segunda Guerra Mundial. Los países europeos cuyas industrias e infraestructuras habían sido destruidas necesitaban enormes inyecciones de capital para financiar proyectos de reconstrucción. Estados Unidos y el Reino Unido --en gran medida por temor a que la Unión Soviética aumentase su área de influencia a todo el continente- aceptan crear el \textbf{Banco Internacional de Reconstrucción y Desarrollo} en Bretton Woods como entidad complementaria del FMI y la OMC. El Banco Mundial habría de encargarse de proveer la financiación necesaria para proyectos a largo plazo, mientras que el FMI inyectaría la liquidez para las necesidades a corto plazo que ocasionasen eventuales desequilibrios de la balanza de pagos. A lo largo de los años 40 y los años 50, el éxito de la reconstrucción europea y el Plan Marshall liberaron recursos financieros y abrieron la puerta a la utilización de éstos para cubrir las necesidades de capital de países que no habían sufrido la Guerra Mundial en la misma medida pero que se habían quedado rezagados en la gran ola de desarrollo de los 50 y los 60. Así, el Banco Mundial pasó a centrar sus esfuerzos en otro tipo de economías, y se aumentaron las demandas para prestar financiación a proyectos que si bien no serían capaces de devolver lo recibido por sí mismos, sí que mejorarían potencialmente la calidad de vida de los implicados. Esta tendencia dio lugar a la creación de nuevas instituciones en el seno del Banco Mundial tales como la \textbf{Agencia Internacional de Fomento}, o a la creación de bancos de desarrollo de carácter regional como el Banco Interamericano de Desarrollo. El papel del sector privado en el desarrollo comenzó a tomarse en cuenta y se crearon la \textbf{Corporación Financiera Internacional} --para proporcionar financiación directamente a empresas privadas-, el \textbf{Centro Internacional para el Arreglo de Disputas relativas a Inversiones} --para servir de sede donde arbitrar conflictos entre inversores privados y estados receptores de ayuda- o la \textbf{Agencia Multilateral de Garantía de Inversiones} --para asegurar inversiones privadas en países receptores de préstamos del Banco Mundial-. 

Paralelamente a estas adiciones al Banco Mundial, surgieron instituciones similares pero de carácter regional. Por razones estratégicas, determinados países impulsaron bancos regionales de desarrollo para establecer un área de influencia económica y política. Así, los Estados Unidos impulsaron el \textbf{Banco Interamericano de Desarrollo}, Japón el \textbf{Banco Asiático de Desarrollo} y Nigeria el \textbf{Banco Africano de Desarrollo}. La caída de la Unión Soviética y la necesidad de asistir a los antiguos miembros en la transición hacia economías de mercado dio lugar a la creación del \textbf{Banco Europeo de Reconstrucción y Desarrollo} en 1991. Aunque la organización de cada uno de estas entidades tiene algunas especificidades, en general siguen la estructura de gobernanza y los objetivos del Banco Mundial y sus instituciones --en especial el BIRD, la AFI y la CFI. Aparte de estos cuatro grandes bancos regionales, han aparecido en las últimas décadas una gran variedad de instituciones plurilaterales que con mayor o menor éxito han tratado de financiar el desarrollo de regiones concretas. Más recientemente, la pujanza de las llamadas economías emergentes ha dado lugar a la creación del \textbf{Banco Asiático de Inversión en Infrastructuras} y al \textbf{Nuevo Banco de Desarrollo}, impulsados por China y por los llamados BRICS, respectivamente, en un intento por ofrecer una alternativa a la preponderancia americana, japonesa y europea en las instituciones existentes y especialmente en las que habían sido creadas en el marco de Bretton Woods. El \textbf{Fondo Verde del Clima}, por su parte, ha sido creado para promover inversiones que mitiguen el impacto del cambio climático y ayuden a gestionar sus efectos a los países más vulnerables.

La \marcar{valoración} de las actividades de las instituciones de desarrollo está sujeta a controversias, y en la actualidad las críticas que reciben giran en torno a aspectos como la eficiencia del gasto, la gobernanza y la falta de rendición de cuentas en determinadas áreas. La reacción ha estas críticas ha tenido respuesta en la formulación de agendas de objetivos a largo plazo tales como los Objetivos del Milenio formulados en 2001, la Agenda 2030 o el programa from Billions to Trillions, así como programas de salvaguardia para mitigar los aspectos negativos de la ayuda al desarrollo y propuestas de reforma de la gobernanza de las instituciones que enfrentan en todo caso, fuertes dificultades.

La evolución de las instituciones de financiación del desarrollo en el futuro dependerá de la resolución de conflictos de economía política en el seno de la economía mundial, así como a la evolución de ésta. ¿Quién debe aportar ayuda?, ¿quién debe recibirla? y ¿quién debe gestionarla? son las preguntas que definen los conflictos entre grupos de interés y cuya respuesta determinará en gran medida las reformas de estas instituciones. La evolución de la economía mundial y del cambio climático influirán fuertemente en la disponibilidad de recursos, y obligan ya a plantear el problema de la obtención de recursos en contextos de crisis o de necesidad a corto plazo de recursos.

\seccion{Preguntas clave}

\begin{itemize}
	\item ¿Qué son las instituciones multilaterales de financiación y ayuda al desarrollo?
	\item ¿Cuáles son?
	\item ¿Para qué sirven?
	\item ¿Cómo se organizan?
	\item ¿Qué valoración de su actividad?
\end{itemize}

\esquemacorto

\begin{esquema}[enumerate]
	\1[] \marcar{Introducción}
		\2 Contextualización
			\3 Desigualdades de renta per cápita
			\3 Transferencia de ahorro y tecnología
			\3 Instituciones multilaterales de financiación
		\2 Objeto
			\3 Qué son las instituciones de financiación del desarrollo
			\3 Qué objetivos tienen
			\3 Para qué sirven
			\3 Cuáles son
			\3 Qué resultados obtienen
		\2 Estructura
			\3 Elementos comunes de los bancos de desarrollo
			\3 Grupo del Banco Mundial
			\3 Bancos Regionales
			\3 Valoración
	\1 \marcar{Elementos comunes de los Bancos de desarrollo}
		\2 Funciones
			\3 Canalizar ahorro global hacia proyectos rentables
			\3 Promover desarrollo mediante ayudas
			\3 Proveer conocimiento técnico y económico
		\2 Antecedentes
			\3 Creación Bretton Woods (1944)
			\3 Años 50
			\3 Años 60
			\3 Años 70
			\3 Años 80
			\3 Años 90 y 2000
		\2 Actuaciones -- FAPIC
			\3 \marcar{F}inanciación de proyectos
			\3 \marcar{A}poyo a políticas
			\3 \marcar{P}réstamos al sector privado
			\3 \marcar{I}nvestigación
			\3 \marcar{C}ooperación técnica y asesoramiento
	\1 \marcar{El Grupo del Banco Mundial}
		\2 Banco Internacional de Reconstrucción y Desarrollo (1944)
			\3 Función
			\3 Antecedentes
			\3 Organización
			\3 Actuaciones
		\2 Corporación Financiera Internacional (1956)
			\3 Función
			\3 Antecedentes
			\3 Organización
			\3 Actuaciones
		\2 Agencia Internacional de Fomento / Agencia Internacional de Desarrollo (1960)
			\3 Función
			\3 Antecedentes
			\3 Organización
			\3 Actuaciones
		\2 Centro Internacional de Arreglo de Diferencias relativas a Inversiones (1966)
			\3 Función
			\3 Antecedentes
			\3 Organización
			\3 Actuaciones
		\2 Agencia Multilateral de Garantía de Inversiones (1988)
			\3 Función
			\3 Antecedentes
			\3 Organización
			\3 Actuaciones
	\1 \marcar{Bancos chinos de desarrollo}
		\2 Agentes
			\3 China Development Bank
			\3 Export-Import Bank of China
		\2 Actuaciones
			\3 Financiación no concesional
			\3 AOD
			\3 Infraestructuras
			\3 Belt and Road Initiative
			\3 Regiones principales
		\2 Valoración
			\3 Ayuda ligada
			\3 Tendencia hacia colaboración con IFinDesarrollo regionales
			\3 Influencia política y militar
			\3 Énfasis en materias primas y comercio
	\1 \marcar{Bancos regionales de Desarrollo}
		\2 Grupo del Banco Interamericano de Desarrollo (1959)
			\3 Idea clave
			\3 Organización
			\3 Banco Interamericano de Desarrollo
			\3 IDB Invest/Corporación Interamericana de Inversiones (1984)
			\3 FOMIN-- Fondo Multilateral de Inversiones
		\2 Banco Africano de Desarrollo (1964)
			\3 Idea clave
			\3 Organización
			\3 Actuaciones
		\2 Banco Asiático de Desarrollo (1966)
			\3 Idea clave
			\3 Organización
			\3 Actuaciones
		\2 Banco Europeo de Reconstrucción y Desarrollo (1991)
			\3 Idea clave
			\3 Organización
		\2 Banco Asiático de Inversión en Infraestructuras (2014)
			\3 Idea clave
			\3 Organización
			\3 Actuaciones
			\3 Valoración
		\2 Fondo Verde del Clima (2010)
			\3 Idea clave
			\3 Organización
			\3 Actuaciones
		\2 CAF/Banco de Desarrollo de América Latina
			\3 Idea clave
			\3 Organización
			\3 Actuaciones
		\2 Otras Instituciones
			\3 New Development Bank/Banco de Desarrollo de los BRICS
			\3 Banco Centroamericano de Integración Económica
			\3 Banco Islámico de Desarrollo
			\3 Banco de Desarrollo de África Occidental
			\3 Banco de Desarrollo del Mar Negro
	\1 \marcar{Valoración}
		\2 Rendición de cuentas
			\3 Objetivo
			\3 Resultados
		\2 Políticas de salvaguardias
			\3 Objetivo
			\3 Resultados
		\2 Objetivos de Desarrollo del Milenio (2001)
			\3 Objetivos
			\3 Resultados
		\2 Agenda 2030: Sustainable Development Goals (ONU, 2015)
			\3 Sustainable Development Goals
			\3 Agenda de Acción
			\3 Papel de los bancos de desarrollo
		\2 From Billions to Trillions
			\3 Idea clave
		\2 Reforma de la gobernanza
			\3 Críticas
			\3 Propuestas
	\1[] \marcar{Conclusión}
		\2 Recapitulación
			\3 Elementos comunes
			\3 Grupo Banco Mundial
			\3 Bancos regionales
			\3 Valoración
		\2 Idea final
			\3 Conflictos economía política
			\3 Evolución economía mundial

\end{esquema}

\esquemalargo













\begin{esquemal}
	\1[] \marcar{Introducción}
		\2 Contextualización
			\3 Desigualdades de renta per cápita
				\4 Enormes diferencias permanentes
				\4 Inversión, tecnología, como determinantes
			\3 Transferencia de ahorro y tecnología
				\4 Países desarrollados $\to$ PEDs
				\4 Fin del colonialismo
				\4 Motivos altruistas
				\4 Motivos interés nacional
			\3 Instituciones multilaterales de financiación
		\2 Objeto
			\3 Qué son las instituciones de financiación del desarrollo
			\3 Qué objetivos tienen
			\3 Para qué sirven
			\3 Cuáles son
			\3 Qué resultados obtienen
		\2 Estructura
			\3 Elementos comunes de los bancos de desarrollo
				\4 Objetivos
				\4 Trayectoria
				\4 Actividades
			\3 Grupo del Banco Mundial
			\3 Bancos Regionales
			\3 Valoración
	\1 \marcar{Elementos comunes de los Bancos de desarrollo}
		\2 Funciones
			\3 Canalizar ahorro global hacia proyectos rentables
				\4[] Aprovechar solvencia institución multilateral
				\4[] Financiar a coste inferior a mercado
				\4[] Inversiones de largo plazo
				\4[] Manteniendo sostenibilidad del banco
			\3 Promover desarrollo mediante ayudas
				\4[] Países rentas intermedias y/o solventes
				\4[] Vía interés prestamos no concesionales
			\3 Proveer conocimiento técnico y económico
				\4[] Investigación economía del desarrollo
				\4[] Centralización innovación tecnológica, ayuda al desarrollo
		\2 Antecedentes
			\3 Creación Bretton Woods (1944)
				\4[] Financiar reconstrucción Europa y Asia
				\4[] Énfasis en infraestructuras
				\4[] Temor a influencia soviética
			\3 Años 50
				\4[] Liberación de recursos Europa
				\4[] Plan Marshall despeja necesidades europeas
				\4[] Inicia proyectos resto del mundo
				\4[] Brasil, India, México, Yugoslavia...
			\3 Años 60
				\4[] Énfasis países en desarrollo
				\4[] Industria en detrimento de agricultura
				\4[] Informe Pearson (1969): crítica modelo
				\4[] Implantación 0.7\% Ayuda Oficial al Desarrollo como objetivo
			\3 Años 70
				\4[] Calidad institucional determina efectividad ayudas
				\4[] Enfocar mejor uso de recursos
				\4[] Fracaso en África, problemas América Latina
				\4[] Éxitos en China, Asia
				\4[] Aumento espectacular flujos de financiación
				\4[] Robert McNamara
				\4[] Transformación agri. India, Indonesia, Pakistán
			\3 Años 80
				\4[] Aumento de la condicionalidad
				\4[] Predominancia asesoramiento neoliberal
				\4[] Ajustes sector público en América Latina
				\4[] Medio ambiente empieza a ser objetivo
			\3 Años 90 y 2000
				\4[] Relativa pérdida de peso préstamos del Banco
				\4[] Flujos financieros internacionales explotan
				\4[] Aumento desigualdades
				\4[] Aumento colaboración con FMI
				\4[] Lucha contra pandemias
		\2 Actuaciones -- FAPIC
			\3 \marcar{F}inanciación de proyectos
				\4 Concesional
				\4[] Sin interés o sin devolución de principal
				\4 No concesional
				\4[] Interés de mercado
				\4 Generalmente, cofinanciación
				\4 En ocasiones, sindicación BM/Regionales
				\4 \underline{Fases de los proyectos}
				\4[1] Evaluación de propuestas
				\4[] Directorio ejecutivo suele aprobar/rechazar
				\4[2] Licitación
				\4[] Quién ejecuta el proyecto
				\4[] Normativa del financiador
				\4[] Homogeneización creciente de normativas
				\4[] Concesionales: sin limitación de nacionalidad
				\4[3] Ejecución
				\4[] Supervisada por financiador
				\4[] En ocasiones, transferencias directas a empresa
			\3 \marcar{A}poyo a políticas
				\4 Muy habitual ayuda de países nórdicos
				\4 Inyección de dinero sin fin concreto
				\4 Condicionalidad legislativa y ecónomica
				\4 Incentivos perversos si sustituyen impuestos
			\3 \marcar{P}réstamos al sector privado
				\4 Financiación de proyectos comercialmente viables
				\4 Transportes, industrias...
				\4 Concesión de garantías
				\4 Concesionalidad social, medioambiental
			\3 \marcar{I}nvestigación
				\4 Complementaria a academia
				\4 Programas de investigación específicos
				\4 Foros de intercambio de ideas
				\4 Economista jefe moldea
			\3 \marcar{C}ooperación técnica y asesoramiento
				\4 Asesoramiento para mejorar efectividad proyectos
				\4 Formación de capital humano
				\4 Transferencias tecnológicas
				\4 Mejorar capacidad de instituciones
				\4 A través de fondos fiduciarios
				\4[] Pequeños fondos con un objetivo concreto
				\4[] Enorme variedad de fondos
				\4[] Donantes tratan de ganar visibilidad
				\4[] Tendencia de fondos bilaterales a multilaterales
				\4[] Bancos aumentan eficiencia de fondos
	\1 \marcar{El Grupo del Banco Mundial}
		\2 Banco Internacional de Reconstrucción y Desarrollo (1944)
			\3 Función
				\4 Institución central del Grupo Banco Mundial
				\4 Préstamos no concesionales
				\4 Coordinación con FMI
				\4 Asistencia técnica
				\4 Investigación políticas, desarrollo, tecnología
			\3 Antecedentes
				\4 Bretton Woods
				\4[] Debate sobre creación junto a FMI y Org. Intern. de Comercio
				\4[$\Rightarrow$] Creación del BIRD en 1944
				\4 Evolución ligada a:
				\4[] Apertura economía
				\4[] Evolución teoría económica
				\4[] Economistas jefe
				\4[] Opinión pública
			\3 Organización
				\4 Junta de gobernadores
				\4[] Gobernador por estado miembro
				\4[] Gobernador suplente por miembro
				\4[] Numerosas funciones delegadas en DEjecutivo
				\4[] Miembros de oficio de AIF, CFI, CIADI
				\4[] AMGI elige gobernadores por separado.
				\4[] Reunión anual ordinaria
				\4[] Reuniones extraordinarias
				\4[] Voto de gobernador: capital suscrito
				\4 Directorio ejecutivo
				\4[] 25 directores elegidos por gobernadores
				\4[] Sillas propias (8)\footnote{Ver \url{http://pubdocs.worldbank.org/en/241041541103873167/BankExecutiveDirectors.pdf}}
				\4[] $\to$ USA, CHI, JAP, GER, FRA, UK, KSA, RUS\footnote{Comparte silla con Siria pero tanto el director ejecutivo principal como el alterno son siempre rusos y designados por Rusia.}
				\4[] Presidente elegido por directores, 5 años renovables
				\4[] Presidente dirige el Banco Mundial
				\4[] Presidente anterior: Jim Yong Kim (americano)
				\4[] $\to$ Siempre ha sido un americano
				\4[] Presidente actual
				\4[] $\to$ David Malpass (2019)
				\4[] Voto del presidente sólo divide empates
				\4[] Actúan de oficio en Banco, AIF, CFI
				\4 Derechos de voto
				\4[] Depende de:
				\4[] $\to$ Aportaciones de capital
				\4[] $\to$ Derechos básicos (250 votos)
				\4[] Mayores accionistas:
				\4[] $\to$ US, JAP, CHI, GER, FRA, UK
				\4 Comité para el Desarrollo
				\4[] Foro gobernadores GBM y FMI
				\4[] Coordinar actuaciones
				\4[] Debate estratégico
				\4[] 2 veces al año, coincide con asambleas FMI y GBM
				\4 Gerencia
				\4[] Personal compartido con AIF
				\4 Balance\footnote{\url{http://pubdocs.worldbank.org/en/474791538065340369/211296v2.pdf}}
				\4[] Activo total consolidado del GBM
				\4[] $\to$ $\sim 550.000 M \text{USD}$
				\4[] $\to$ Enero 2020\footnote{Ver \href{https://finances.worldbank.org/Financial-Reporting/Condensed-Balance-Sheets/54xn-mzza/data}{Condensed Balance Sheets del GBM}.}
				\4[] Patrimonio neto disponible del BIRD
				\4[] $\to$ $\sim42.000 M \text{USD}$
				\4[] $\to$ + Capital exigible no suscrito
				\4[] USA mayor accionista
				\4[] China es tercer accionista
				\4[] $\to$ Infrarepresentación
				\4[] Aprobación de ampliaciones de capital
				\4[] $\to$ $>75\%$ de derechos de voto en BIRD
				\4[] $\to$ $>80\%$ de derechos de voto en CFI
				\4[] Propuesta de ampliación de 2018
				\4[] $\to$ Estados Unidos exige reformar para ampliar
				\4[] $\to$ Exige veto en CFI
				\4[] $\then$ Aprobada\footnote{https://www.reuters.com/article/us-imf-g20-wbank/world-bank-shareholders-back-13-billion-capital-increase-idUSKBN1HS0QS}
				\4[] Ampliación de 2018
				\4[] $\to$ 13.000 M USD desembolsado
				\4[] $\then$ 7.500 para BIRD
				\4[] $\then$ 5.500 para CFI
				\4[] $\to$ 52.600 M callable
			\3 Actuaciones
				\4 2017: 4000 millones de \$ comprometidos proyectos activos
				\4 Ligero aumento respecto a 2016
				\4 Sobre todo Asia-Pacífico y Asia-Sur y Latam
				\4 191.000 M USD en préstamos netos\footnote{Ver \href{https://finances.worldbank.org/Financial-Reporting/Condensed-Balance-Sheets/54xn-mzza/data}{Condensed Balance Sheets del GBM}.}
		\2 Corporación Financiera Internacional (1956)
			\3 Función
				\4 Financiar sector privado de PEDs
				\4[] Préstamos a empresas o intermediarios financieros
				\4[] Inversión en empresas (5 a 20\%)
				\4[] Participación en fondos de capital
				\4[] Capital riesgo
				\4[] Financiación comercio
				\4[] Sindicación de créditos a empresas privadas
				\4 Soluciones de tesorería
				\4[] Riesgos de tipo de cambio, tipos de interés, materias primas
				\4 Asesoramiento empresas
			\3 Antecedentes
				\4 Creación en 1956
				\4 Primeros 50: necesario aumentar inversión privada
				\4 Robert L. Garner
				\4 \$100 millones de capital inicial
			\3 Organización
				\4 Junta de gobernadores
				\4[] Mismos gobernadores que el BIRD
				\4 Directorio ejecutivo
				\4[] Mismos directores que BIRD
			\3 Actuaciones
				\4 Emisión de bonos para financiar proyectos
				\4 \textit{Asset Management Company}
				\4[] Inversión en 13 fondos temáticos
				\4[] 10.100 M invertidos en PEDs,
				\4 Ámbitos de actuación
				\4[] Infraestructuras, manufacturas, agroalimentario
				\4[] Servicios, mercados financieros
				\4 \textit{Global Trade Programs}
				\4[] Complementar financiación privada del comercio
				\4[] Proveer garantías, \$, mantenimiento comercio
				\4[] Financiación de comercio intercional
				\4[] Liquidez a exportadores
				\4 \textit{Global Risk Management}
				\4[] Formación banqueros de países emergentes
				\4[] Gestión de riesgos sistémicos
				\4 Otros programas
		\2 Agencia Internacional de Fomento / Agencia Internacional de Desarrollo (1960)
			\3 Función
				\4 Brazo concesional del Banco Mundial
				\4[] Canalizar donaciones
				\4[] Canalizar excedentes BIRD
				\4[] Canalizar repagos anteriores
				\4 Desarrollo de países más pobres
				\4[] PIBpc < \$1165 en 2018
				\4[] $\to$ En 2015 era de $\$1215$
				\4[] $\to$ Revisado periódicamente
				\4[] No solventes
				\4[] Excepciones: insularidad, sostenibilidad de la deuda
				\4[] India hasta 2017\footnote{India se graduó en 2014 y el debate sigue vivo en relación a la conveniencia de permitir o no a países graduados el acceso a este tipo de financiación. En el caso de la India, se esgrime la existencia de regiones de extrema pobreza en un país que es en realidad un continente.}
				\4 No auto-sostenible
				\4 Condonación de deuda para HIPC\footnote{\textit{Highly Indebted Poor Countries}.}
				\4 Orientada a sectores más básicos
				\4 Principales proyectos activos
				\4[] África, Sur de Asia
				\4 Récords de asistencia en 2014 y 2017
			\3 Antecedentes
				\4 Creación en 1960
				\4 Tras Plan Marshall y éxitos años 50
				\4 Ayuda países periféricos
			\3 Organización
				\4 Junta de gobernadores
				\4[] Gobernadores del BIRD
				\4[] Votos según recursos aportados acumulados\footnote{Lo cual plantea problemas de incentivos, porque las economías emergentes con capacidad para aportar ven reducidos sus incentivos dado que seguirán representando un porcentaje pequeño respecto a los países que llevan décadas aportando fondos.}
				\4 Directorio ejecutivo
				\4[] Mismos directores que BIRD
				\4 Balance
				\4[] 160.000 M de USD de equity
				\4[] 150.000 M de USD en préstamos
				\4[] 32.000 M de USD en inversiones
			\3 Actuaciones
				\4 Préstamos netos
				\4[] 150.000 M de USD\footnote{Ver \href{https://finances.worldbank.org/Financial-Reporting/Condensed-Balance-Sheets/54xn-mzza/data}{Condensed Balance Sheets del GBM}.}
				\4 Procesos de reposición de fondos
				\4[] Cada 3 o 4 años
				\4[] Reposición 521.000 millones USD en 2017
				\4 Inversiones de largo plazo
				\4[] 32.000 M de USD
				\4 \$ 18.000+ millones en 2017
				\4 Sobre todo Asia y Sur de Asia
		\2 Centro Internacional de Arreglo de Diferencias relativas a Inversiones (1966)
			\3 Función
				\4 Resolución disputas mediante arbitraje
				\4[] Inversiones internacionales
				\4[] Recurso voluntario al arbitraje/conciliación
				\4 Facilitar instalaciones
				\4 Prestar servicios
				\4[] Listado de árbitros y conciliadores
				\4 Objetivo último:
				\4[] Aumentar confianza de inversores
				\4[] Generar corrientes de inversión
			\3 Antecedentes
				\4 Creado en 1966
				\4 Intentos previos OCDE
				\4[] Resolver disputas sobre expropiaciones
				\4[] Desacuerdo sobre reglas a imponer
			\3 Organización
				\4 Consejo administrativo
				\4[] Mismos que BIRD salvo cambio expreso
				\4[] Sin poder decisión casos concretos
				\4 Secretariado
				\4[] Gestión corriente
			\3 Actuaciones
				\4 Convención CIADI
				\4[] Procedimiento básico
				\4[] Jueces nacionales no intervienen
				\4[] EMiembro vs. entidad privada de otro EMiembro
				\4[] Disputa legal referente a inversión
				\4[] Ambas partes se someten voluntariamente
				\4 Mecanismo Complementario
				\4[] Una de las partes no es miembro del CIADI
				\4[] Disputas no relativas a una inversión
				\4[] Procedimientos de ``\textit{fact-finding}''
				\4 Arbitraje sujeto a diferentes reglas
				\4[] UNCITRAL
				\4[] Otros procedimientos ad-hoc
		\2 Agencia Multilateral de Garantía de Inversiones (1988)
			\3 Función
				\4 Asegurar IED en países en desarrollo
				\4 Mitigar riesgos de:
				\4[] Inconvertibilidad de divisas
				\4[] Restricciones convertibilidad
				\4[] Expropiaciones
				\4[] Guerras, terrorismo, disturbios
				\4[] Infracciones de contratos por gobiernos
				\4[] Impagos por estados
			\3 Antecedentes
				\4 Creado en 1988
			\3 Organización
				\4 Junta de gobernadores
				\4[] Gobernadores elegidos por separado a BIRD
				\4[] Por defecto, gobernadores del BIRD
				\4 Directorio ejecutivo
				\4[] 25 directores elegidos por gobernadores
				\4[] Elegidos por separado a BIRD
			\3 Actuaciones
				\4 Asegurar
				\4[] Precios $\sim$1\% de cantidad
				\4[] Duración máxima 15 años, 20 excepcional
				\4[] 250 millones máx. por operación
				\4[] Importes mayores: sindicación
				\4 Asesorar e investigar
				\4[] IED en mercados en desarrollo
				\4[] Gestión del riesgo
				\4 28000 millones USD en garantías
	\1 \marcar{Bancos chinos de desarrollo}\footnote{Ver \href{https://chinapower.csis.org/china-development-finance/}{China Power}.}
		\2 Agentes
			\3 China Development Bank
			\3 Export-Import Bank of China
		\2 Actuaciones
			\3 Financiación no concesional
				\4 Parte principal de ayuda de China
				\4 No cumple con estándares OCDE
			\3 AOD
				\4 Principalmente transporte y energía
				\4[] Muy diferente a ayuda americana
				\4[] $\to$ Centrada en salud y humanitaria
			\3 Infraestructuras
			\3 Belt and Road Initiative
				\4 Desde 2013
				\4 Inversión en 70 países
				\4 Fondos:
				\4[] Silk Road Fund
				\4[] Banco Asiático de Inversión en Infraestructuras
				\4 Infraestructura de transporte
				\4[] Entre Asia, Europa y África
				\4 Financiación de proyectos
				\4[] Puertos
				\4[] Ferrocarril
				\4[] Aeropuertos
				\4[] Plataformas logísticas
				\4 China como centro del comercio internacional
				\4 A completar en 2049
			\3 Regiones principales
				\4 África
				\4 Latinoamérica
				\4 Países sin vínculos formales con Taiwán
		\2 Valoración
			\3 Ayuda ligada
				\4 Generalmente vinculada a empresas chinas
			\3 Tendencia hacia colaboración con IFinDesarrollo regionales
				\4 Tendencia creciente a cofinanciación
				\4[] BERD
				\4[] Agencia Francesa de Desarrollo
				\4[] Bancos de Europa del Este
				\4[] Otros
				\4 Implica fin de regla de empresas chinas
				\4[] Necesario permitir contratistas de otro origen
				\4[] $\to$ Especialmente si colaboración con Bancos Regionales
				\4[] Posible restricción a dos países
				\4[] $\to$ Si colaboración bilateral
			\3 Influencia política y militar
				\4 Principal objetivo de China
				\4 Votos en ONU
				\4 Apoyo iniciativas chinas internacionales
			\3 Énfasis en materias primas y comercio
				\4 Salidas a exportaciones chinas
				\4 Aumentar vínculos comerciales
				\4 A menudo, MPrimas por inversión
				\4[] Garantizar acceso a materias primas
	\1 \marcar{Bancos regionales de Desarrollo}
		\2 Grupo del Banco Interamericano de Desarrollo (1959)
			\3 Idea clave
				\4 Primer banco regional: 1959
				\4 Oposición histórica de EEUU
				\4 Miedo a influencia soviética
				\4 Aumentos de capital periódicos
				\4 Estrategia Institucional 2010-2020
				\4 Sede en Washington DC
			\3 Organización
				\4 A partir de 1974: miembros de todo el mundo
				\4 48 miembros
				\4[] 26 en LatAm y Caribe
				\4 3 instituciones:
				\4[] Banco Interamericano de Desarrollo
				\4[] IDB Invest
				\4[] Fondo Multilateral de Inversiones
				\4 Mismos órganos de gobierno que GBMundial
				\4[] Junta de Gobernadores
				\4[] $\to$ Ministros de finanzas/presidente banco central
				\4[] $\to$ Reunión anual
				\4[] Directorio ejecutivo
				\4[] $\to$ 14 directores ejecutivos
				\4[] $\to$ 5 comités
				\4[] $\to$ Aprobar préstamos y garantías
				\4[] $\to$ Definición de estrategias-país
				\4[] $\to$ Presupuesto administrativo
				\4[] $\to$ Fijar tipos de interés de préstamos
				\4[] Presidente
				\4[] $\to$ Representante legal
				\4[] $\to$ Día a día de la institución
			\3 Banco Interamericano de Desarrollo
				\4 Primera entidad del grupo
				\4 Gama de productos
				\4[] Préstamos flexibles a 25 años, LIBOR+Margen
				\4[] Coberturas individuales, moneda local
				\4[] Garantías parciales, riesgo político
				\4[] Financiación concesional y blended
				\4 Financiación concesional
				\4[] Fondo para Operaciones Especiales (reemplazada)
				\4[] Ventanilla blanda hasta 2017
				\4[] $\to$ Integrada en capital ordinario
				\4[] Reposiciones mediante ampliaciones de capital
				\4[] Elegibilidad según renta per cápita
				\4[] Préstamos al 0,25\%, 40 años
				\4[] Amortización al final del préstamo
				\4 Financiación blended
				\4[] Combinación de concesional y ordinaria
				\4[] Dos tranches
				\4[] Combinación depende de sostenibilidad
				\4[] IIF -- Intermediate Financing Facility (reemplazado)
				\4[] $\to$ Antiguo fondo para subsidio de intereses
				\4[] $\to$ Reemplazado por financiación blended
			\3 IDB Invest/Corporación Interamericana de Inversiones (1984)
				\4 Creado en 1984, jurídicamente independiente
				\4 Fondo multilateral de inversiones
				\4 Afiliado al Grupo del BIAmD pero independiente
				\4 Membresía no es idéntica a BIAmD
				\4 Actuaciones
				\4[] Préstamos a sector privado
				\4[] Inversiones de capital y garantías
				\4[] Asesoramiento y capacitación
				\4[] Inversión en proyectos privados
				\4 Cartera cercana a 12.000 Millones de USD
				\4 Papel especialmente importante en Covid
				\4[] Líneas de liquidez
				\4[] Capital para continuidad del negocio
			\3 FOMIN-- Fondo Multilateral de Inversiones\footnote{Ver \url{https://www.iadb.org/es/resources-businesses/fondo-multilateral-de-inversiones}}
				\4 Asistencia técnica al sector privado
				\4[] Principal proveedor en LatAm y Caribe
				\4 Microfinanzas
				\4[] Programas pilotos de reducción de pobreza
				\4[] Aunque vía intermediarios
				\4[] $\to$ Nunca financia directamente a empresas
				\4 Financiación de proyectos
				\4[] Pueden alcanzar 2 millones por proyecto
				\4[] Subsidios
				\4[] Préstamos
				\4[] Garantías
				\4[] Capital semilla
				\4[] Cuasi invesiones de capital
				\4 Servicios de consultoría
				\4[] ONG
				\4[] Fundaciones
				\4[] Agencias del sector público
				\4[] Instituciones financieras
				\4[] Empresas del sector privado
				\4 Inversión en fondos de equity y microfinanzas
				\4[] Que a su vez invierten en otros proyectos
				\4[] Creación de red estable de intermediarios
				\4 Sectores prioritarios
				\4 Donaciones (2/3)
				\4 Resto: préstamos, garantías, capital, asesoramiento
		\2 Banco Africano de Desarrollo (1964)
			\3 Idea clave
				\4 Fundado en 1964
				\4 Objetivo:
				\4[] reafirmar independencia de África
				\4 Sede:
				\4[] Costa de Marfil
				\4 Relación con la Unión Africana
				\4 Apertura a capital fuera de África en 1982
				\4 Préstamos flexibles 20 años, 5 de gracia
				\4 Interés LIBOR+Margen
				\4 Inversión en capital
				\4 Garantías parciales
				\4 Gestión de riesgo
				\4 Financiación del comercio
				\4 Donantes principales: RU, USA, GER
				\4 Reposición en 2017
			\3 Organización
				\4 Nigeria principal accionista
				\4 MAR, TUN, EGI principales prestatarios
				\4 Banco Africano de Desarrollo
				\4[] Financiación de mercado
				\4 Fondo Africano de Desarrollo
				\4[] Ventanilla blanda
				%\4[] \textit{Facilidad de Apoyo a la Transición}
				\4 Fondo Especial de Nigeria
				\4[] 1976: ingresos petrolíferos
				\4[] revolving fund
				\4[] $\to$ Reinvierte a partir de repagos
				\4[] Actividad muy limitada
			\3 Actuaciones
				\4 Movilizar recursos para inversión en socios regionales
				\4 Proporcionar
				\4 Última ampliación de capital en 2019\footnote{https://www.egypttoday.com/Article/3/77396/AfDB-shareholders-approve-landmark-115-billion-capital-increase}
				\4[] 115.000 M USD
				\4[] $\to$ Capital se dobla desde 93.000 a 208.000 M USD
		\2 Banco Asiático de Desarrollo (1966)
			\3 Idea clave
				\4 Fundado en 1966
				\4 Japón y EEUU miembros principales
				\4 Sede en Manila
			\3 Organización
				\4 Gobernanza similar a Banco Mundial
				\4 Sin personalidad jurídica separada
				\4 Banco Asiático de Desarrollo
				\4[] Financiación no concesional, blend, y a empresas
				\4 Fondo Asiático de Desarrollo
				\4[] Financiación concesional
				\4[] FAsD sin personalidad separada
			\3 Actuaciones
				\4 Financiación a sector público y privado
				\4 Tres categorías de renta fijadas por AID
				\4[] Ventanilla no concesional
				\4[] Ventanilla blend
				\4[] Ventanilla concesional $\to$ Fondo de Desarrollo Asiático
				\4 Préstamos público: LIBOR + margen
				\4[] Maturity medio < 19 años
				\4[] Posible moneda local
				\4 Préstamos ligados a resultados (desde 2013)
				\4 Préstamos a sector privado
				\4[] Catalizador de recursos adicionales
				\4 Inversión en capital
				\4 Garantías
		\2 Banco Europeo de Reconstrucción y Desarrollo (1991)
			\3 Idea clave
				\4 Fundado en 1991
				\4 Reconversión bloque soviético a mercado
				\4 Promoción iniciativa privada
				\4 Atracción de capital hacia ex-soviéticos
				\4 Expansión progresiva a otras áreas
				\4[] África del Norte, Grecia, Túnez, Mongolia...
				\4 Condicionalidad política
				\4 UE y Banco Europeo de Inversiones: accionistas
				\4 Autosostenible
			\3 Organización
				\4 65 países miembros
				\4 EEUU principal accionistas
				\4 Después, JAP, ALE, FRA, RU, ITA
				\4 Una sola entidad
				\4[] Financiación a sector privado mayoría
				\4[] Financiación pública
				\4[] Sin préstamos concesionales
				\4[] Inversión en capital
				\4[] Inversión fondos de capital
				\4[] Garantías
		\2 Banco Asiático de Inversión en Infraestructuras (2014)
			\3 Idea clave
				\4 Fundado en 2014
				\4 Afirmación de nuevo poder chino
				\4[] Superar instituciones Bretton-Woods
				\4 Déficit de infraestructuras en Asia
				\4 Reticencia inicial potencias tradicionales
				\4 Salvo EEUU, todas se han unido
				\4 Abierto a todos los miembros de BIRD o BAsD
				\4 Actuaciones habituales salvo concesional
			\3 Organización
				\4 Sede en Pekín
				\4 103 miembros actuales
				\4[] 21 candidatos
				\4 Miembros no soberanos admitidos
				\4 Empresas de cualquier país puede recibir fondos
				\4 Departamentos no se solapan con existentes
				\4 Directorio Ejecutivo no residente
				\4 China puede vetar Presidente
				\4 Máxima calificación crediticia
				\4 Estados Unidos trata de evitar adhesiones
				\4 Japón no se une
				\4 Taiwán: entrada no permitida
				\4 100.000 M de USD de capital
				\4[] 20\% desembolsado
				\4 Cuotas dependen de:
				\4[] Localización en en Asia o no
				\4[] GDP nominal y PPP
				\4 Junta de Gobernadores
				\4 Directorio Ejecutivo
				\4[] 12 gobernadores
				\4[] Representan más de un país
				\4[] $\to$ Salvo China, silla exclusiva
				\4[] 9 sillas para Asia-Pacífico
				\4[] Tres representan miembros de otras regiones
				\4 Entrada de nuevos miembros
				\4[] Considerada anualmente
				\4 Nuevos miembros valorados anualmente
			\3 Actuaciones
				\4 Préstamos One Belt One Road Initiative
				\4 Énfasis en infraestructuras de transporte
				\4 Europa también entra en accionariado
				\4[] UK, Alemania, Francia, España, Italia
				\4[] $\to$ Socios fundadores
			\3 Valoración
				\4 Alternativa a FMI y BM en Asia
				\4
		\2 Fondo Verde del Clima (2010)\footnote{Ver \url{https://www.greenclimate.fund/home}}
			\3 Idea clave
				\4 Establecido en la COP 16 (2010) Cancún
				\4 Países miembros de UNFCCC\footnote{Convención de Naciones Unidas sobre Cambio Climático}
				\4 Papel aumentado en COP21 de París 2015
				\4[] Instrumento central
				\4[] $\to$ Mantener aumento temperatura < 2ºC
				\4 Limitar gases efecto invernadero
				\4 Adaptar países vulnerables a cambio climático
			\3 Organización
				\4 Entidades acreditadas presentan proyectos
				\4 8 áreas prefentes de actuación
				\4[] 4 en mitigación
				\4[] 4 en adaptación
				\4 Estructura financiera
				\4[] Movilización inicial de recursos en 2014
				\4[] 10.300 M de USD iniciales
			\3 Actuaciones
				\4 5.600 M de USD comprometidos
				\4 3.380 M de USD en proyectos en implementación
				\4 50\% mitigación, 50\% adaptación
				\4 Énfasis en áreas muy vulnerables a cambio climático
				\4[] PEDs, PMAs, África, pequeños estados insulares
				\4 Préstamos
				\4 Ayudas concesionales
				\4 Equity
				\4 Garantías
		\2 CAF/Banco de Desarrollo de América Latina
			\3 Idea clave
				\4 Creado en 1968
				\4 Ámbito de países latinoamericanos
				\4 España participa en creación
				\4 Integración económica de habla hispana
				\4 Sede en Caracas
			\3 Organización
				\4 19 miembros públicos
				\4[] 19 países
				\4[] Latinoamericanos
				\4[] Caribe
				\4[] España, Portugal
				\4 Bancos privados también participan
				\4 Ligado a Cumbres Iberoamericanas
			\3 Actuaciones
				\4 Financiación a PYMES
				\4 Inversión de largo plazo
				\4 Créditos a exportación
				\4 Gestión de tesorería y carteras
		\2 Otras Instituciones
			\3 New Development Bank/Banco de Desarrollo de los BRICS
				\4 Idea clave
				\4[] Creado en 2012
				\4[] Cumbre de los BRICS
				\4 Organización
				\4[] Miembros
				\4[] $\to$ Brasil, India, Rusia, China, Sudáfrica
				\4[] Capital
				\4[] $\to$ 100.000 M de € autorizado
				\4[] $\to$ 50.000 M de € suscrito
				\4[] Junta de Gobernadores
				\4[] Directorio Ejecutivo
				\4[] Nuevos miembros admitidos
				\4[] $\to$ Pero BRICS capital > 55\%
				\4 Actuaciones
				\4[] Relativamente poca actividad
				\4[] Emisión de bonos verdes en Renminbi
				\4[] Infraestructuras sobre todo
				\4 Valoración
				\4[] Calificación crediticia:
				\4[] $\to$ AA+
			\3 Banco Centroamericano de Integración Económica
				\4 Países centroamericanos
				\4 Costa Rica, El Salvador, Guatemala, Honduras, Nicaragua
				\4 España también participa
				\4 Infraestructura y proyectos agrícolas
			\3 Banco Islámico de Desarrollo
				\4 Fundado en 1973
				\4 Arabia Saudita principal accionista
				\4[] Argelia, Irán, Egipto, Turquía, Emiratos, Kuwait después
				\4 Inversiones en países islámicos
				\4 Cinco subentidades
				\4[] IDB -- Islamic Development Bank
				\4[] IRTI -- Islamic Research and Training Institute
				\4[] ICD -- Islamic Corporation for Development of the Private Sector
				\4[] ICIEC -- Islamic Corporation for Insurance of Investment and Export Credit
				\4 Financiación de proyectos públicos y privados
				\4 Financiación del comercio
				\4 Financiación de PYMES
			\3 Banco de Desarrollo de África Occidental
			\3 Banco de Desarrollo del Mar Negro
				\4 New Development Bank
	\1 \marcar{Valoración}
		\2 Rendición de cuentas
			\3 Objetivo
				\4 Evaluar eficacia
				\4 Alinear incentivos bancos-gobiernos locales
			\3 Resultados
				\4 Presión creciente para $\varDelta$ eficiencia
				\4 Difícil transmitir impacto a opinión pública
				\4 Necesarias reflexiones cualitativas vs cuantitativas
				\4 Consolidación de marcos de evaluación
		\2 Políticas de salvaguardias
			\3 Objetivo
				\4 Minimizar efectos adversos financiación desarrollo
				\4 Mejorar impacto determinadas áreas
				\4[] Medioambiental
				\4[] Derechos humanos
				\4[] Biodiversidad
				\4 Establecer requisitos y directrices
			\3 Resultados
				\4 Debates sobre objetivos prioritarios
				\4 Excesivos requisitos aumentan proyecto
				\4 Requisitos poco definidos $\to$ externalidades negativas
		\2 Objetivos de Desarrollo del Milenio (2001)
			\3 Objetivos
				\4 Erradicación pobreza extrema
				\4 Acceso Universal a la educación primaria
				\4 Igualdad de género
				\4 Reducción de la mortalidad infantil
				\4 Mejora salud de embarazadas
				\4 Combatir SIDA, malaria y otras epidemias
				\4 Asegurar sostenibilidad medioambiental
				\4 Desarrollar herramienta desarrollo global
				\4[$\then$] Para el año 2015
			\3 Resultados
				\4 Relativamente satisfactorios
				\4 Muchas áreas han mejorado espectacularmente
				\4 Especialmente, pobreza extrema e igualdad género en educación
		\2 Agenda 2030: Sustainable Development Goals (ONU, 2015)
			\3 Sustainable Development Goals
				\4 17 objetivos de desarrollo sostenible
				\4 Objetivo 2030
				\4 Papel clave de los BMD
			\3 Agenda de Acción
				\4 Conferencia Internacional sobre la Financiación para el Desarrollo \footnote{Addis Abeba, 2015.}
				\4 Señala ámbitos de actuación de los BMD
				\4 Directrices de mejora de la gobernanza
			\3 Papel de los bancos de desarrollo
				\4 Referencias explícitas
				\4 Reclamación de mejora representación PEDs en Grupo BM y FMI
				\4 En regionales, relativa mejor representación
		\2 From Billions to Trillions
			\3 Idea clave
				\4 Documento conjunto instituciones internacionales:
				\4[] FMI
				\4[] Grupo del Banco Mundial
				\4 GBM, BAfD, BAsD, BERD, BEI, BID, FMI
				\4 Reafirmación objetivos:
				\4[] canalizar recursos
				\4[] aumentar efecto catalizador\footnote{El apoyo de un banco de desarrollo a un país o a un proyecto determinado provoca en numerosas ocasiones un aumento de la inversión en ese país por el aumento de la confianza y las expectativas mejoradas de desarrollo que conlleva la ayuda recibida.}
				\4[] Mejorar asesoramiento
		\2 Reforma de la gobernanza
			\3 Críticas
				\4 Presidentes siempre mismo país en GBM, BAsD
				\4 Composición sesgada del staff
				\4 Derechos de voto: desproporción
			\3 Propuestas
				\4 Desbloqueo nacionalidad de presidentes
				\4 Desbloqueo derechos de voto por aportaciones pasadas
				\4 Nuevas instituciones
				\4 Mayor poder de decisión de receptores
	\1[] \marcar{Conclusión}
		\2 Recapitulación
			\3 Elementos comunes
			\3 Grupo Banco Mundial
			\3 Bancos regionales
			\3 Valoración
		\2 Idea final
			\3 Conflictos economía política
				\4 Quién debe ayudar
				\4 Quién debe recibir ayuda
			\3 Evolución economía mundial
				\4 Países se gradúan
				\4 ¿Cómo ampliar la capacidad de los bancos?
				\4 ¿Cómo mantener capacidad bajo crisis?
\end{esquemal}

























\conceptos

\concepto{Blended finance}

\concepto{Logical Framework Approach}

Similar al Financial Programming Framework utilizado en el FMI, el LFA es una herramiento de gestión y planificación de proyectos desarrollada en los años 70 en el seno del Banco Mundial con el objetivo de mejorar 


\concepto{Países graduados}

Se denominan graduados aquellos países receptores de financiación concesional que alcanzan un nivel de renta determinado y un grado de solvencia tal que dejan de ser susceptibles de recibir tal financiación, pasando a ser candidatos a recibir financiación no concesional. Los procesos de graduación ``inversa'' son también relativamente frecuentes. El proceso de graduación suele llevarse a cabo de manera gradual y transitoria. 


\preguntas

\seccion{Test 2018}

\textbf{37.} La Corporación Financiera Internacional (CFI) del grupo del Banco Mundial:

\begin{itemize}
	\item[a] Concede préstamos no concesionales a países de renta media y a países de renta baja, con capacidad de pago.
	\item[b] Concede préstamos concesionales a los países más pobres.
	\item[c] Ofrece seguros a inversiones ante riesgos no comerciales, en los países en desarrollo.
	\item[d] Concede préstamos no concesionales al sector privado de los países en desarrollo.
\end{itemize}


\seccion{Test 2017}
\textbf{37.} En el marco de las instituciones multilaterlaes de financiación y ayuda al desarrollo:

\begin{itemize}
	\item[a] El Banco Europeo de Reconstrucción y Desarrollo (BERD) cubre un área de operatividad muy amplia que abarca países del Mediterráneo, países de Asia y todos los países del Centro y Este de Europa, excluidos los países pertenecientes a la Unión Europea.
	\item[b] El Grupo del Banco Mundial ha introducido, en épocas recientes, ciertos cambios muy positivos en su gobernanza, pero sin alterar los porcentajes tradicionales de derechos de voto.
	\item[c] Un gran número de bancos regionales ha venido efectuando recientemente ampliaciones significativas en su capital social.
	\item[d] El Banco Asiático de Inversión e Infrastructuras (BAII) fue creado en 1966 y es uno de los grandes bancos regionales en términos de movilización de recursos.
\end{itemize}

\seccion{Test 2015}

\textbf{41.} Seleccione la respuesta correcta relativa al Grupo del Banco Mundial:

\begin{itemize}
	\item[a] La Corporación Financiera Internacional es la mayor institución internacional de desarrollo dedicada exclusivamente al sector público y a las microempresas.
	\item[b] El Organismo Multilateral de Garantía de Inversiones ofrece seguros contra riesgos políticos a inversores para promover la inversión extranjera directa en los países en desarrollo, el crecimiento económico, reducir la pobreza y mejorar la vida de las personas.
	\item[c] La Asociación Internacional de Fomento, creada en 1966, financia a los países más pobres exclusivamente a través de préstamos concesionales para formentar el crecimiento económico y mejorar las condiciones de vida de la población.
	\item[d] El Fondo Multilateral de Inversiones es el principal proveedor de asistencia técnica para el sector privado en América Latina y Caribe.
\end{itemize}

\seccion{Test 2014}

\textbf{45.} Señale cuál de las instituciones del Grupo del Banco Mundial actúa sólo con el sector privado:

\begin{itemize}
	\item[a] La Asociación Internacional de Fomento.
	\item[b] El Banco Internacional de Reconstrucción y Fomento.
	\item[c] La Corporación Financiera Internacional.
	\item[d] Ninguna de las anteriores.
\end{itemize}

\seccion{Test 2011}

\textbf{38.} El Banco Mundial

\begin{itemize}
	\item[a] Está organizado como una empresa por acciones con unos derechos de voto proporcionales a las participaciones de capital.
	\item[b] Es una institución financiera convencional.
	\item[c] Actúa como depositario para las monedas de los países miembros y parte de sus reservas de divisas.
	\item[d] Presta recursos financieros a todos los países miembros.
\end{itemize}

\seccion{Test 2009}

\textbf{38.} El Banco Internacional de Reconstrucción y Desarrollo (BIRD)

\begin{itemize}
	\item[a] Proporciona ayuda en términos concesionales a los países menos avanzados (LICs).
	\item[b] Impulsa el desarrollo económico, incentivando el sector privado en los países miembros.
	\item[c] Realiza funciones de intermediación financiera canalizando fondos obtenidos a través de empréstitos en los mercados financieros internacionales hacia países en desarrollo.
	\item[d] Tiene como objetivo fundamental la emisión de garantías a los inversores contra los riesgos no comerciales que acechan a la inversión extranjera.
\end{itemize}

\seccion{Test 2008}

\textbf{36.} En el marco del grupo del Banco Mundial, la Corporación Financiera Internacional:

\begin{itemize}
	\item[a] Da créditos y proporciona asesoramiento a los países más pobres.
	\item[b] Estimula el establecimiento de un sector privado eficiente y competitivo en sus miembros.
	\item[c] Trata de garantizar a los inversores que se cumplen ciertos requisitos contra las pérdidas resultantes de riesgos no comerciales.
	\item[d] Proporciona facilidades para la conciliación y arbitraje de los acuerdos o disputas entre los gobiernos y los inversores extranjeros.
\end{itemize}

\textbf{37.} Los Bancos de Desarrollo:

\begin{itemize}
	\item[a] Tienen una estructura accionarial en la que participan países donantes, países beneficiarios y las principales instituciones bancarias privadas mundiales.
	\item[b] Realizan operaciones tanto con el sector público de los países en desarrollo como con el sector privado.
	\item[c] Revisan anualmente el poder de voto de sus miembros.
	\item[d] No pueden financiar operaciones de asistencia técnica en los países más pobres.
\end{itemize}

\textbf{38.} En la ``iniciativa HIPC'' (Heavily Indebted Poor Countries), en el denominado ``\textbf{punto de decisíon}'':

\begin{itemize}
	\item[a] Se evalúa la sostenibilidad de la deuda del país candidato a beneficiarse de la iniciativa.
	\item[b] Se decide si un país puede o no recibir financiación ligada.
	\item[c] Se decide si un país puede beneficiarse de una condonación inicial del 67\% de su deuda.
	\item[d] Se reevalúan las condiciones fijadas inicialmente en el punto de culminación.
\end{itemize}

\seccion{Test 2007}

\textbf{39.} Señale la afirmación \textbf{FALSA} referida a la Asociación Internacional de Fomento (AIF):

\begin{itemize}
	\item[a] Es la entidad del Banco Mundial que brinda ayuda a los países más pobres del mundo.
	\item[b] La elección de los países susceptibles de recibir financiación de la AIF se basa en tres criterios esenciales: nivel de renta per cápita, falta de solvencia para tomar préstamos en condiciones de mercado y buen desempeño en materia de políticas.
	\item[c] Los recursos de la AIF provienen, en su mayor parte, de las contribuciones de los gobiernos miembros más ricos.
	\item[d] Los países receptores de fondos AIF no pueden recibir asistencia financiera de otras entidades del grupo banco mundial.
\end{itemize}

\textbf{38.} De las siguientes afirmaciones:

\begin{itemize}
	\item[I] En los proyectos financiados por el Banco Africano de Desarrollo sólo pueden participar empresas de los 53 países africanos miembros.
	\item[II] Los préstamos del Banco Asiático de Desarrollo son tanto más concesionales cuanto menor es el grado de desarrollo de los prestatarios.
	\item[III] El Banco Europeo de Inversiones se creó con la finalidad de contribuir al progreso y la reconstrucción económica de los países de Europa Central y Oriental y de los países miembros de la Comunidad de Estados Independientes.
	\item[IV] La distribución de votos en el Banco Interamericano de Desarrollo es directamente proporcional al peso económico del país miembro.
\end{itemize}

Indique la respuesta correcta:

\begin{itemize}
	\item[a] Todas son verdaderas.
	\item[b] Solo II) y IV) son verdaderas.
	\item[c] Solo II) es verdadera.
	\item[d] Todas son falsas.
\end{itemize}

\notas


\textbf{2018:} \textbf{37.} D

\textbf{2017:} \textbf{37.} C La A es falsa porque el BERD no excluye a los países miembros de la UE. La B es falsa porque sí ha aumentado el \% de voto correspondiente a países en desarrollo. Ver \url{https://www.google.com/url?sa=t&rct=j&q=&esrc=s&source=web&cd=2&cad=rja&uact=8&ved=2ahUKEwi2v4il0IrnAhUQuRoKHVroAI0QFjABegQICxAF&url=https\%3A\%2F\%2Fblogs.worldbank.org\%2Fvoices\%2Fworld-bank-gets-capital-increase-and-reforms-voting-power&usg=AOvVaw1VJar1B7qN6Sg-8gG6e7-p}

\textbf{2015:} \textbf{41.} B

\textbf{2014:} \textbf{45.} C

\textbf{2011:} \textbf{38.} A

\textbf{2009:} \textbf{38.} C

\textbf{2008:} \textbf{36.} B \textbf{37.} B \textbf{38.} A

\textbf{2007:} \textbf{39.} D

\textbf{2006:} \textbf{38.} B

\bibliografia

Mirar en Palgrave:
\begin{itemize}
	\item foreign aid
	\item World Bank
\end{itemize}


Delacampagne Crespo, M. \textit{EL GRUPO DEL BANCO AFRICANO DE DESARROLLO} Revista ICE, Junio de 2013. En carpeta del tema. \url{http://www.revistasice.com/CachePDF/BICE\_3043\_33-42\_\_E15E23FF57B9FBC4F25F99D538D5CE74.pdf}

World Bank. \textit{Project database} \url{http://projects.worldbank.org/search?lang=en} Estadísticas de proyectos IDA y IBRD

Ottenhof, J. \textit{Regional Development Banks} Center for Global Development (EN CARPETA DEL TEMA)

Bibliografía de Tema CECO Nuevo

\end{document}
