\documentclass{nuevotema}

\tema{4A-8}
\titulo{Análisis de los sectores de bienes de equipo y de alta tecnología}

\begin{document}

\ideaclave

\input{/home/alibey/Oposicion/Resumenes_4o/Importancia_cuantitativa_Sectores.tex}

\seccion{Preguntas clave}

\begin{itemize}
	\item 
\end{itemize}

\esquemacorto

\begin{esquema}[enumerate]
	\1[] \marcar{Introducción}
		\2 Contextualización
			\3 Sectores de la economía española
			\3 Factores en común
			\3 Globalización del sector
			\3 Economías de escala y aglomeración
			\3 Capital humano
		\2 Objeto
			\3 Delimitación de sectores
			\3 Importancia
			\3 Modelos teóricos
			\3 Oferta
			\3 Demanda
			\3 Sector exterior
			\3 Políticas
		\2 Estructura
			\3 Análisis estático
			\3 Análisis dinámico
			\3 Política económica
	\1 \marcar{Maquinaria industrial}
		\2 Maquinaria y equipos mecánicos
			\3 Delimitación
			\3 Máquina textil
			\3 Maquinaria agrícola y forestal
			\3 Maquinaria de construcción
			\3 Maquinaria cerámica
		\2 Máquina herramienta
			\3 Delimitación
			\3 Importancia
			\3 Modelos teóricos
			\3 Oferta
			\3 Demanda interna
			\3 Sector exterior
			\3 Perspectivas
			\3 Políticas públicas
		\2 Maquinaria y equipos eléctricos
			\3 Concepto
			\3 Importancia
			\3 Concentración geográfica
			\3 Electrodomésticos
	\1 \marcar{Equipos de transporte}
		\2 Idea clave
			\3 Delimitación
			\3 Importancia
			\3 Rasgos principales
			\3 Evolución
		\2 Construcción naval
			\3 Delimitación
			\3 Importancia
			\3 Modelos teóricos
			\3 Oferta
			\3 Demanda interna
			\3 Sector exterior
			\3 Evolución
			\3 Políticas públicas
		\2 Equipos ferroviarios
			\3 Delimitación
			\3 Importancia
			\3 Modelos teóricos
			\3 Oferta
			\3 Demanda interna
			\3 Sector exterior
			\3 Evolución
			\3 Políticas públicas
		\2 Aeronáutico
			\3 Delimitación
			\3 Importancia
			\3 Modelos teóricos
			\3 Oferta
			\3 Airbus
			\3 Demanda interna
			\3 Sector exterior
			\3 Evolución
			\3 Perspectivas
			\3 Debilidades
			\3 Fortalezas
			\3 Oportunidades
			\3 Defensa
			\3 Espacial
			\3 Políticas públicas
	\1 \marcar{Otros sectores de alta tecnología}
		\2 Idea clave
			\3 Capital humano
			\3 Distribución geográfica
			\3 Saldo exterior
		\2 Farmacéutico
			\3 Delimitación
			\3 Importancia
			\3 Modelos teóricos
			\3 Oferta
			\3 Demanda interna
			\3 Sector exterior
			\3 Distribución de productos farmacéuticos
			\3 Farmacias
			\3 Evolución
			\3 Perspectivas
			\3 Políticas públicas
		\2 Electrónica y ópticos
			\3 Delimitación
			\3 Importancia
			\3 Modelos teóricos
			\3 Oferta
			\3 Demanda interna
			\3 Demanda externa
			\3 Evolución
			\3 Políticas públicas
		\2 Software e informática
			\3 Delimitación
			\3 Importancia
			\3 Modelos teóricos
			\3 Oferta
			\3 Blockchain
			\3 Demanda interna
			\3 Sector exterior
			\3 Evolución
			\3 Diagnóstico
			\3 Políticas públicas
		\2 Biotecnología
			\3 Delimitación
			\3 Importancia
			\3 Modelos teóricos
			\3 Capital
			\3 Trabajo
			\3 Demanda interna
			\3 Sector exterior
			\3 Evolución
			\3 Políticas públicas
		\2 Energías renovables
			\3 Eólica
			\3 Solar fotovoltaica
			\3 Solar termoeléctrica
			\3 Solar térmica
		\2 Instrumentos de precisión
			\3 Delimitación
			\3 Importancia
			\3 Modelos teóricos
			\3 Oferta
			\3 Demanda interna
			\3 Demanda externa
			\3 Evolución
			\3 Políticas públicas
	\1[] \marcar{Conclusión}
		\2 Recapitulación
		\2 Idea final

\end{esquema}

\esquemalargo

\begin{esquemal}
	\1[] \marcar{Introducción}
		\2 Contextualización
			\3 Sectores de la economía española\footnote{Presentación Kingdom of Spain del Tesoro Público, diciembre de 2019 (fuente: INE).}
				\4 Porcentaje sobre VAB
				\4 Servicios: 74,7\%
				\4 Industria: 15,4\%
				\4 Construcción: 6,5\%
				\4 Sector primario: 3\%
				\4 Industria manufacturera
				\4[] Bienes de consumo tradicional: 35\%
				\4[] Alta tecnología: 15\%
				\4[] Bienes de equipo: 25\%
				\4[] Intermedios: 20\%
			\3 Factores en común
				\4 Número de factores en común
				\4[] Globalización del sector
				\4[] $\to$ Competencia global
				\4[] $\to$ Comercio relativamente fácil
				\4[] Economías de escala y aglomeración
				\4[] Capital humano
			\3 Globalización del sector
				\4 Bienes con buena relación peso/valor
				\4 Competencia global
				\4 Economías de escala elevadas
			\3 Economías de escala y aglomeración
				\4 Efecto sobre productividad
				\4 Ya descrito por Marshal
				\4 Tendencia a concentración
				\4[] Clústers de empresas
				\4[] Grandes áreas urbanas
			\3 Capital humano
				\4 Sectores con muy elevada capitalización
				\4 Productividad del trabajo superior a la media
				\4 Salarios por encima de la media
		\2 Objeto
			\3 Delimitación de sectores
			\3 Importancia
			\3 Modelos teóricos
			\3 Oferta
			\3 Demanda
			\3 Sector exterior
			\3 Políticas
		\2 Estructura
			\3 Análisis estático
			\3 Análisis dinámico
			\3 Política económica
	\1 \marcar{Maquinaria industrial}\footnote{\href{https://www.sepe.es/contenidos/observatorio/mercado_trabajo/2731-2.pdf}{Ver SEPE (2016) Estudio prospectivo del Sector de Bienes de Equipo en España.}}
		\2 Maquinaria y equipos mecánicos
			\3 Delimitación
				\4 CNAE División 28
				\4[] Fabricación de maquinaria y equipo n.c.o.p\footnote{No comprendido en otras partes.}
				\4 Subsectores
				\4[] Agroalimentaria
				\4[] Maquinaria textil
				\4[] Máquina herramienta
				\4[] Maquinaria agrícola y forestal
				\4[] Maquinaria de construcción
				\4[] $\to$ Elevación
				\4[] Maquinaria para la cerámica
			\3 Máquina textil
				\4 Fundamentalmente en Cataluña
				\4 Principales clientes en Asia
			\3 Maquinaria agrícola y forestal
				\4 Equipamientos para explotaciones agrícolas
				\4[] Tractores
				\4[] Cosechadoras
				\4[] Invernaderos
				\4[] Equipos de riego
				\4 Tractores
				\4[] Getafe
				\4[] $\to$ John Deere
				\4[] Leganés
				\4[] $\to$ Kubota
				\4 Otro equipo
				\4[] Cataluña
				\4[] Aragón
				\4 Equipos de riego e invernaderos
				\4[] Murcia
				\4[] Almería
				\4[] Valencia
				\4 Competidores
				\4[] EEUU
				\4[] Japón
				\4[] Italia
				\4[] Israel
				\4 Conducción autónoma
				\4[] Tecnología de futuro en tractores
				\4[] $\to$ Mayor potencia a c/p que vehículos
			\3 Maquinaria de construcción
				\4 Elevado atomización
				\4[] PYMES
				\4 Andamiaje
				\4[] PYMES
				\4 Movimiento de tierras
				\4[] Empresas americanas y japonesas
				\4[] $\to$ Caterpillar
				\4[] $\to$ Komatsu
				\4[] $\to$ Hitachi
				\4[] $\to$ JVC
				\4[] $\to$ Volvo
				\4[] Déficit exterior
				\4[] $\to$ Principales proveedores EEUU, Japón
				\4[] $\to$ Sin fábricas de relevancia en España
			\3 Maquinaria cerámica
				\4 Muy potente industria azulejera y cerámica
				\4 España tercer proveedor mundial
				\4 Tracción de demanda de maquinaria
				\4 Cataluña, Valencia, Castellón
		\2 Máquina herramienta\footnote{Ver \href{http://www.interempresas.net/Flipbooks/IM/2018/pdf/IM3.pdf}{Informe Interempresas -- Sector Máquina Herramienta}}
			\3 Delimitación
				\4 Instrumentos de manipulación y fabricación de:
				\4[] Bienes intermedios
				\4[] Otras herramientas
				\4[] Bienes finales
			\3 Importancia
				\4 Importante predictor de actividad futura
				\4 Eslabonamientos hacia atrás
				\4[] Software
				\4[] Bienes intermedios
				\4[] Energía
				\4 Eslabonamientos hacia delante
				\4[] Otros sectores industriales
				\4[] Automóvil
			\3 Modelos teóricos
			\3 Oferta
				\4 Especialmente en País Vasco
				\4[] Casi 90\% del sector
				\4 Feria de Máquina Herramienta de Bilbao
				\4 Organización en clústeres
			\3 Demanda interna
			\3 Sector exterior
				\4 3er productor y exportador de la UE
				\4 9 Productor y exportador del mundo
				\4 Superávit estructural
				\4[] Cercano a 400 M de €
			\3 Perspectivas
				\4 Impresión 3D
				\4 Fabricación aditiva
				\4 Startups
			\3 Políticas públicas
		\2 Maquinaria y equipos eléctricos
			\3 Concepto
				\4 CNAE División 27
				\4[] Fabricación de material y equipo eléctrico
				\4 Motores eléctricos
				\4 Generadores
				\4 Transformadores
			\3 Importancia
				\4 Cercano a 0,5\% del PIB
				\4[] 5000 M de VAB
				\4 0,35\% del empleo
				\4[] 71.000 empleos
				\4 Más productiva que en la UE
			\3 Concentración geográfica
				\4 Cataluña y País Vasco especialmente
			\3 Electrodomésticos
				\4 Posible interpretar como consumo duradero
				\4[] $\to$ Electrodomésticos de consumo
				\4 También como sector de bienes de equipo
				\4[] Input para otros sectores
				\4 Sector muy maduro
				\4 Demanda ligada a:
				\4[] Áreas urbanas
				\4[] Fase expansiva de ciclo
				\4[] Confianza de consumidores
				\4[] Políticas públicas de renovación de eficiencia energética
	\1 \marcar{Equipos de transporte}
		\2 Idea clave
			\3 Delimitación
				\4 División 30 de CNAE
				\4 Diseño, construcción, reparación y mantenimiento
				\4[] $\to$ Aeronáutica
				\4[] $\to$ Ferroviaria
				\4[] $\to$ Vehículos militares de combate
				\4[] $\to$ Espacial
			\3 Importancia
				\4 Cercano a 0,7\% de VAB
				\4 0,3\% de empleo
				\4 Superávit exterior
			\3 Rasgos principales
				\4 Superávit exterior estructural
				\4 Más productivo que en UE
				\4 España capaz de cadena completa
				\4[] Ferroviaria
				\4[] Aeronáutica
				\4[] Algunos equipamientos de defensa
			\3 Evolución
				\4 Años 50 y 60
				\4[] Ferroviaria y naval mucho apoyo público
				\4[] Problema de competitividad en 70s y 80s
				\4 80s, 90s, 2000
				\4[] Integración plena en CVG
		\2 Construcción naval\footnote{Ver \href{https://pymar.com/sites/default/files/pymar_informe_anual_2019.pdf}{Informe PYMAR (2019)}}
			\3 Delimitación
				\4 CNAE Grupo 301: construcción naval
				\4 Astilleros
				\4 Fabricación de componentes
				\4 Reparación y mantenimiento
				\4 Fuertemente procíclicos
			\3 Importancia
				\4 Cuantitativa
				\4[] Importancia decreciente en industrial
				\4[] España tercer productor a nivel europeo
				\4[] Fuerte pérdida de peso a nivel mundial
				\4 Cualitativa
				\4[] Arrastre hacia atrás
				\4[] $\to$ Máquinas herramienta
				\4[] $\to$ Metales
				\4[] $\to$ Materiales
				\4[] $\to$ Energía
				\4[] $\to$ Minería
				\4[] $\to$ Software
				\4[] Arrastre hacia delante
				\4[] $\to$ Sector pesquero
				\4[] $\to$ Sector público: defensa
			\3 Modelos teóricos
			\3 Oferta
				\4 Distribución geográfica astilleros privados
				\4[] Galicia
				\4[] $\to$ Principal en astilleros privados
				\4[] Asturias y País Vasco siguientes en importancia
				\4 PYMAR
				\4[] Patronal de pequeños y medianos astilleros
				\4 Navantia
				\4[] Ferrol
				\4[] Cádiz
				\4[] Cartagena
			\3 Demanda interna
			\3 Sector exterior
				\4 Demanda exterior
				\4[] >90\% cartera de pedidos
				\4[] Buques pesqueros principal componente
				\4[] Acuicultura aumenta peso
				\4[] 25 nuevos contratos en 2019
				\4[] Demanda resiste relativamente en 2020
				\4[] Muy sensible a ciclo global
				\4 Competidores
				\4[] China
				\4[] Corea
				\4[] Japón
				\4[] India
				\4[] Francia
				\4[] Alemania
				\4[] Finlandia
				\4[] Rusia
				\4[] India
				\4 Sector deficitario dentro de mat. transporte
			\3 Evolución
				\4 Años 60
				\4[] Uno de principales sectores industriales
				\4[] Redescuento especial de créditos
				\4 Años 70
				\4[] Problemas de competitividad
				\4[] Necesaria reconversión
				\4 Años 80 y 90
				\4[] Exceso de capacidad
				\4 Años 2000
				\4[] Consolidación
				\4[] AESA + Bazán
				\4[] $\to$ Ambos astilleros públicos
				\4[] $\then$ IZAR
				\4[] $\then$ Uno de principales grupos mundiales
				\4[] Conversión posterior en Navantia
				\4[] $\to$ Astilleros militares
				\4[] Astilleros privados resistentes
				\4[] $\to$ Aunque presión costes Asia
				\4 Actualidad
				\4[] Especialización creciente
				\4[] Pesca principal demandante
				\4[] Defensa sigue ciclo propio
				\4 Perspectivas
				\4[] Caída en offshore (plataformas gas y petróleo)
				\4[] Aumento demanda renovación flota pesquera
				\4[] Acuicultura motor de demanda
			\3 Políticas públicas
				\4 Sector público
				\4[] Navantia
				\4[] $\to$ Buques militares
				\4 CDTI
				\4[] Transferencia tecnológica naval
				\4 IDAE
				\4[] Programas de eficiencia energética en buques
				\4 Industria Conectada 4.0
				\4[] Conexión empresas navales
				\4[] Manufactura aditiva
				\4[] Tratamiento masivo de datos
				\4 Programa de apoyo a I+d+I industrial
				\4 CESCE y PYMAR
				\4[] Convenios de colaboración
				\4[] Seguros de exportación de buques
		\2 Equipos ferroviarios
			\3 Delimitación
				\4 CNAE 302
				\4 Diseño y construcción de material rodante
				\4 Actividades de reparación y mantenimiento
				\4 No incluye vías
				\4 Elevada ciclicidad
				\4[] Por peso de inversión pública
			\3 Importancia
				\4 Cuantitativa
				\4 Cualitativa
			\3 Modelos teóricos
			\3 Oferta
				\4 Empresas españolas competitivas a nivel mundial
				\4[] AVE a la Meca con principales empresas
				\4[] $\to$ ADIF
				\4[] $\to$ Talgo
				\4[] $\to$ Renfe
				\4[] $\to$ Indra
				\4[] $\to$ OHL
				\4[] $\to$ ...
				\4 Concentración geográfica
				\4[] País Vasco
				\4[] $\to$ CAF
				\4[] Aragón
				\4[] $\to$ CAF
				\4[] Alstom
				\4[] $\to$ Barcelona
				\4[] Madrid
				\4[] $\to$ Talgo en Las Matas
				\4[] Siemens (motores para trenes)
				\4[] $\to$ Barcelona
			\3 Demanda interna
				\4 ADIF, ADIF alta velocidad, Renfe
				\4[] Principales demandantes
				\4[] Práctico monopsonio
				\4 Metros de grandes ciudades
			\3 Sector exterior
				\4 Superávit comercial
				\4 Actuación en consorcios de empresas
				\4[] P.ej. AVE a la Meca
				\4[] $\to$ Fase de construcción y fabricación material
				\4[] $\to$ Fase de operación
				\4 Demanda global en expansión
			\3 Evolución
			\3 Políticas públicas
				\4 AVE
				\4[] Debates sobre rentabilidad
				\4[] Enorme efecto tracción sobre demanda
		\2 Aeronáutico
			\3 Delimitación
				\4 CNAE Grupo 303: Construcción aeronáutica y espacial
				\4 Diseño, construcción, reparación y mantenimiento
				\4[] $\to$ Naves espaciales
				\4[] $\to$ Satélites
				\4 Cadena de valor
				\4[] OEM
				\4[] $\to$ Final de la cadena
				\4[] $\to$ Montaje final de aparatos
				\4[] $\to$ Entrega
				\4[] $\to$ Elevada concentración industrial
				\4[] Tier 1
				\4[] $\to$ Manufactura de componentes para montaje
				\4[] $\to$ Ensamblaje final
				\4[] $\to$ Tendencia a integración con OEM
				\4[] Tier 2
				\4[] $\to$ Subsistemas
				\4[] $\to$ Rango de productos más restringido
				\4[] $\to$ Compañías más pequeñas a otras tier
				\4[] Tier 3
				\4[] $\to$ Componentes y partes
			\3 Importancia
				\4 Cuantitativa
				\4[] Tasas de retorno del I+D muy elevadas
				\4 Cualitativa
				\4[] Largos periodos de maduración
				\4[] $\to$ Necesidades de financiación
			\3 Modelos teóricos
			\3 Oferta
				\4 Empresas
				\4[] Alestis
				\4[] Aciturri
				\4[] Aernova
				\4[] Airbus
				\4[] $\to$ Getafe
				\4[] $\to$ Illescas
				\4[] $\to$ Puerto Real
				\4[] $\to$ Sevilla
				\4[] $\to$ Albacete
				\4[] Indra
				\4[] $\to$ Sistemas de control aéreo
				\4[] $\to$ Sistemas de aviones militares
				\4[] $\to$ Simuladoras
				\4[] ITP Aero
				\4[] $\to$ Motores
				\4 Trabajo
				\4[] Cercano a 40.000 empleos
				\4[] Altamente cualificado
				\4[] Elevada demanda de graduados universitarios
			\3 Airbus
				\4 Planta de Getafe
				\4[] Diseño, manufactura de componentes
				\4[] Estabilizador horizontal para A380
				\4[] Helicópteros Tiger
				\4[] Otros helicópteros militares
				\4 Albacete
				\4[] Ensamblado helicópteros militares
				\4[] Mantenimiento de flota
				\4 Illescas
				\4[] Componentes para todos aviones y Eurofighter
				\4 Puerto Real (Cádiz)
				\4[] Ensamblado final de estabilizadores
				\4[] Test de estabilizadores
				\4 Sevilla
				\4[] Transporte militar
				\4[] A400 y otros
				\4[] Centro de formación pilotos aviones militares de carga
				\4 Trabajo
				\4[] Airbus:
				\4[] $\to$ 12.700 empleados
				\4[] $\to$ Estabilizadores horizontales en Getafe, Illescas, Puerto Real
			\3 Demanda interna
			\3 Sector exterior
				\4 Muy orientada a la exportación
				\4 Superavitario persistentemente
				\4 Relativamente poca demanda nacional de aeronaves
				\4 Competidores emergentes
				\4[] Brasil
				\4[] India
				\4[] China
				\4 Contenciosos comerciales
			\3 Evolución
				\4 Hispano Suiza fabricación de motores
				\4[] Primera guerra mundial
				\4 CASA en 1923
				\4[] Pionero de aviación europea
				\4[] José Ortiz-Echagüe
				\4 INTA
				\4[] Instituto Técnica Aeroespacial
				\4[] $\to$ Fundado después de la guerra
				\4[] $\to$ Ministerio del Aire
				\4 Airbus absorbe CASA en 2000
			\3 Perspectivas
				\4 A380
				\4[] Avión con casi 600 pasajeros posibles
				\4[] Concepción de mercado ha cambiado
				\4[] Objeto comercial de A380
				\4[] $\to$ Alimentar grandes rutas intercontinentales
				\4[] $\then$ Vuelos regionales hacia hub continental
				\4[] $\then$ Vuelos intercontinentales desde hubs
				\4[] Finalmente
				\4[] $\to$ Pasajeros prefieren vuelos sin escala
				\4[] $\to$ Vuelos directos intercontinentales desde muchos puntos
				\4[] $\then$ Preferibles aviones más pequeños
				\4[] Último avión producido en 2020
				\4[] Sustituidos por A330 y A350
				\4 Estimaciones pre-covid
				\4[] Fuerte crecimiento en 2020
				\4[] Transporte aéreo y también defensa
				\4[] Airbus gana peso frente a Boeing
				\4[] $\to$ Problemas de seguridad en nuevos modelos
				\4 Tras covid
				\4[] Cambio radical en expectativas
				\4[] $\to$ Paralización casi completa en algunos meses
				\4 Turismo y viajes fortísima reducción
				\4 Ruptura de cadena de suministro
				\4 Consolidación de empresas
				\4[] Alestis + Aciturri
				\4 Cancelaciones masivas de pedidos
			\3 Debilidades
				\4 Excesiva dependencia de Airbus
				\4 Elevada fragmentación de proveedores
				\4 Bajo nivel de inversión nacional
				\4 Excesivo énfasis en aeroestructuras
				\4[] Poco en motores
				\4[] Poco en sistemas de avión
				\4[] $\to$ Poco en sectores sensibles a modernización de aviones
				\4 Expansión de capacidad previa a covid
				\4[] Covid frena en seco planes
			\3 Fortalezas
				\4 Participación en programas espaciales europeos
				\4 Fuerza de trabajo muy formada
				\4[] Sólo unos pocos países comparables
				\4 Puede cubrir ciclo de vida completo
				\4 Experiencia en comercialización
				\4 Elevada internacionalización
			\3 Oportunidades
				\4
			\3 Defensa
				\4 España muy competitiva en turbohélices
				\4 AIRBUS y Tier 1 tienen segmento de defensa
				\4 Estrategia de Seguridad Nacional
				\4 SENER en misiles
			\3 Espacial
				\4 Segmentos
				\4[] Upstream
				\4[] $\to$ Fabricación de satélites
				\4[] $\to$ Instalaciones en tierra
				\4[] Downstream
				\4[] $\to$ Aplicaciones de tecnologías espaciales
				\4[] $\to$ Volúmenes de facturación mucho mayores
				\4 Fábricas
				\4[] Airbus
				\4[] $\to$ Barajas
				\4[] $\to$ Tres cantos
				\4[] Proyectos para ESA
				\4[] Satélites españoles PAZ e INgenio
				\4 Cluster aeroespacial de Sevilla
				\4[] Proveedores tier 1
				\4[] PYMES
				\4[] Universidades
				\4[] Centros tecnológicos
				\4 Perspectivas
				\4[] Sensible a gasto militar
				\4[] Menos afectado que aeronáutica
				\4[] Muy dependiente de inversión pública
				\4 Políticas públicas
			\3 Políticas públicas
				\4 Sector fuertemente intervenido
				\4[] Consideraciones estratégicas
				\4[] Componente militar
				\4 Espacial
				\4[] Agencia Espacial Europea
				\4[] Otras agencias regionales
				\4[] España sin agencia espacial
				\4[] $\to$ Políticas directamente desde ministerios
				\4[] Empresas pública
				\4[] $\to$ Hispasat
				\4 FASEE
				\4 Ayudas turismo
				\4[] Recuperar actividad tras crisis
				\4[] $\to$ Estimular demanda indirectamente en aeronáutica
	\1 \marcar{Otros sectores de alta tecnología}
		\2 Idea clave
			\3 Capital humano
				\4 España relativa debilidad
				\4 Menos profesiones STEM
				\4 Patentes relativamente bajas
			\3 Distribución geográfica
				\4 Grandes áreas urbanas
				\4 Madrid, Cataluña, Valencia, PV
				\4 Algunos núcleos secundarios
				\4[] Aragón, Galicia, CYL, Andalucía
			\3 Saldo exterior
				\4 España fuertemente deficitario
				\4 Déficit persistente
				\4 Heterogeneidad dentro de déficit agregado
				\4[] Algunos componentes superavitarios
		\2 Farmacéutico
			\3 Delimitación
				\4 CNAE Divisiones 21: fabricación productos farmacéuticos
				\4 Divisiones en:
				\4[] $\to$ Productos de base
				\4[] $\to$ Especialidades farmacéuticas
				\4 Canales de distribución
				\4[] Hospitalario
				\4[] Farmacias
			\3 Importancia
				\4 0,7\% VAB
				\4 0,5\% empleo
				\4 Menor peso relativo que UE
				\4 Elevada peso en I+D privado
				\4 Eslabonamientos hacia atrás
				\4[] Biotecnología
				\4[] Sector químico
				\4[] Materias primas
				\4[] Máquina herramienta
				\4 Eslabonamientos hacia delante
				\4[] Servicios sanitarios
				\4[] Distribución de consumo
			\3 Modelos teóricos
			\3 Oferta
				\4 0,7\% VAB
				\4[] Creciente
				\4 Menor peso que otros países UE
				\4 Intensivo en empleo cualificado
				\4 300 empresas
				\4 160 plantas de productos farmacéuticos
				\4 Principalmente medicamentos uso humano
				\4 Distribución geográfica
				\4[] 1. Cataluña ($\sim 40\%$)
				\4[] 2. Madrid ($\sim 76\%$)
				\4[] 3. Andalucía, Valencia, Galicia
				\4 Principales empresas
				\4[] Gilead
				\4[] Cinfa
				\4[] Novartis
				\4[] Abbvie
				\4[] Glaxosmith-Kline
				\4[] Astrazeneca
				\4[] Bayer
				\4[] Roche
				\4[] ...
				\4 Elevadas economías de escala
				\4 Fuerte gasto publicitario
			\3 Demanda interna
				\4 Sistema público de salud
				\4[] Elevado peso en demanda
				\4[] Cercano a monopsonio
				\4[] Elevada capacidad de negociación
				\4 Genéricos vs marca
				\4[] Creciente demanda de genéricos
			\3 Sector exterior
				\4 Déficit estructural
				\4 Saldo deficitario cercano a 2.000 M de €
				\4[] Relativamente estable
			\3 Distribución de productos farmacéuticos
				\4 Segmentos
				\4[] Especialidades: 45\%
				\4[] Prescripción: 45\%
				\4[] OTC: 2\%
				\4[] Parafarmacia: 8\%
				\4 Capital
				\4[] Sobre todo capital nacional
				\4[] Cooperativas predominantes
				\4[] $\to$ Socios son oficinas de farmacia
				\4 Principales
				\4[] Cofares
				\4[] Bidafarma
				\4[] Unnefar
				\4[] Otros
				\4 Grupo Cofares principal distribuidor
			\3 Farmacias
				\4 Sector fuertemente regulado
				\4[] Necesaria concesión administrativa
				\4[] $\to$ Mercado secundario
				\4 Baja ratio de habitantes/farmacia respecto UE
				\4[] Muy heterogéneo por CCAA
				\4[] $\to$ Pocas farmacias en Andalucía, Cat, Madrid
				\4[] $\to$ Muchas en RIO, CNT, EXT, NAV...
				\4 Medicamentos dispensados por SNS
				\4[] Casi 60\% de facturación
				\4 Aumento de fármacos sin receta
				\4 Perspectivas
				\4[] Envejecimiento de población
				\4[] $\to$ Aumento de gasto
				\4[] Genéricos
				\4[] $\to$ Aumento de consumo
				\4[] $\to$ Heterogéneo por CCAA
				\4 Problemas de viabilidad en áreas rurales
				\4[] CCAA con baja densidad
			\3 Evolución
				\4 Elevado gasto farmacéutico
				\4[] Problema de SS
				\4 PVP de medicamentos bajo
				\4[] Incentiva gasto y consumo excesivo
				\4[] Intentos de racionalización
				\4[] $\to$ Ya en Pactos de la Moncloa de 1977
				\4[] $\to$ Reforma 2012 racionalización coberturas
				\4 Crecimiento número de empresas
				\4[] Desde crisis
				\4[] Interacción con sector biotecnológico
			\3 Perspectivas
				\4 Vacunas
				\4 Nanotecnología
				\4 Cáncer
				\4[] Variedad de producto muy elevada
				\4[] Interacción con otros sectores:
				\4[] $\to$ Bienes de equipo: maquinaria de precisión
				\4[] Políticas públicas
				\4[] $\to$ Investigación básica
				\4[] $\to$ Investigación de nuevos medicamentos
				\4[] $\then$ CNIO
				\4[] $\then$ Tecnologías de base para sector privado
			\3 Políticas públicas
		\2 Electrónica y ópticos
			\3 Delimitación
				\4 CNAE División 26: fabricación prods. informáticos, electrónicos y ópticos
			\3 Importancia
				\4 Cuantitativa
				\4[] Reducida en términos de producción nacional
				\4 Cualitativa
				\4[] Eslabonamientos hacia delante con todos sectores
				\4[] $\to$ A pesar de ser sector de alta tecnología
				\4[] Digitalización casi omnipresente
				\4[] Dependencia exterior elevada
			\3 Modelos teóricos
				\4 Shocks de oferta
				\4 IDE horizontal y multinacionales
				\4 IDE vertical y CVG
			\3 Oferta
				\4 Elevadas economías de escala
				\4[] Especialmente en semiconductores
				\4[] $\to$ Más economías de escala upstream que downstream
				\4 Trabajo
				\4[] Apenas 15.000 personas
				\4 Capital
				\4[] Empleo muy estable
				\4[] Caída progresiva de producción en últimas décadas
				\4 Empresas
				\4 Bienes intermedios
				\4[] España importadora neta de tierras raras
				\4[] $\to$ Producción nacional escasa
				\4[] $\to$ Dda. nacional pequeña por escasa producción nacional
				\4[] $\to$ Potencial productora nacional
				\4[] $\to$ Elevado coste medioambiental
			\3 Demanda interna
				\4 Creciente anualmente
				\4 Tamaño de mercado español relativamente pequeño
				\4[] Insuficiente para economías de escala
			\3 Demanda externa
				\4 Sector claramente deficitario
				\4[] A diferencia de servicios informáticos
				\4[] $\to$ Que sí son superavitarios
				\4 Exportaciones
				\4[] 3.500 M de €
				\4 Importaciones
				\4[] 15.000 M de €
				\4[$\then$] Saldo cercano a -12.000 M de €
				\4 Proveedores exteriores
				\4[] China
				\4 Exportaciones de bienes TIC son mínimas
				\4[] Ordenadores
				\4[] Equipos periféricos
				\4 Marruecos principal destino de bienes TIC
			\3 Evolución
				\4 Primeros departamentos de informática años 60
				\4[] Grandes empresas extranjeras
				\4[] Procesar datos de clientes
				\4[] Cálculos de ingenieria
				\4 Departamentos universitarios años 70 y 80
				\4 Industria fuertemente protegida en años 70
				\4[] Incentivo a producción nacional
				\4[] Contrabando
				\4[] Producción nacional de:
				\4[] $\to$ Televisiones
				\4[] $\to$ Radios
				\4[] $\to$ Otra electrónica de consumo
				\4[] A menudo licencias de multinacionales extranjeras
				\4 Años 80 y 90
				\4[] Desaparece industria televisores nacionales
				\4[] Asia pacífico entra en mercado
				\4 Actualidad
				\4[] Dependencia casi total en componentes electrónicos
				\4[] Software y gestión de TIC sí relevante en España
				\4[] $\to$ Indra
				\4[] $\to$ Sistemas de control de vuelo
				\4[] $\to$ Sistemas de gestión de reservas turísticas
			\3 Políticas públicas
				\4 Industria conectada 4.0
				\4 Reciclaje componentes electrónicos
				\4 REACH
				\4 Tierras raras
				\4[] Programas de inventariado y prospección
				\4[] Difícil explotación
				\4[] $\to$ Elevado coste medioambiental
				\4[] Elevada dependencia respecto a China
				\4[] $\to$ Perspectivas de intentos de reducción
				\4[] Trade-off con políticas medioambientales:
		\2 Software e informática
			\3 Delimitación
				\4 Concepto
				\4[] Actividades informáticas
				\4 Subsectores
				\4[] $\to$ Programación
				\4[] $\to$ Consultoría
				\4[] $\to$ Gestión de programas informáticos
				\4[] $\to$ Videojuegos
				\4[] $\to$ Diseño web
				\4[] $\to$ Reparación de ordenadores
				\4[] $\to$ Hosting y proceso de datos
				\4[] $\to$ ...
				\4 Diferenciación
				\4[] Muy dependiente del segmento
				\4[] Alto grado de personalización
				\4 Ciclicidad
				\4[] Relativamente acíclica
				\4[] Tendencia de crecimiento muy sostenido
			\3 Importancia
				\4 Cualitativa
				\4[] Input:
				\4[] $\to$ en casi todo sistema industrial
				\4[] $\to$ distribución comercial de manera creciente
				\4[] $\to$ telecomunicaciones
				\4[] $\to$ sector servicios
				\4 Cuantitativa
				\4[] Actividades informáticas: 1,5\% del PIB\footnote{Ver \href{https://www.ontsi.red.es/es/estudios-e-informes/Sector-TIC/Informe-Anual-del-Sector-TIC-y-de-los-Contenidos-en-Espana-2019}{ONTSI Informe 2019}}
				\4[] 300.000 en servicios informáticos
			\3 Modelos teóricos
				\4 Ley de Moore
				\4[] Duplicación de número de transistores
				\4[] $\to$ Aproximadamente cada dos años
				\4[] $\then$ Bajada de coste de fabricación
			\3 Oferta
				\4 Empresas
				\4[] Cercanas a 17.000 servicios informáticos
				\4[] Distribución muy heterogénea
				\4[] $\to$ Muchas pequeñas empresas en gestión
				\4[] $\to$ Tamaño mayor en producción de software
				\4[] $\to$ Algunas con tamaño muy superior a resto
				\4[] Barreras de entrada dependen de segmento
				\4 Trabajo
				\4[] Mujeres cercano a 35\%
				\4[] Apenas 15.000 en fabricación
				\4[] 300.000 en informática
				\4[] $\to$ Crecimiento sostenido
				\4[] Muy bajo desempleo
				\4 Cifra de negocios
				\4[] Sostenido crecimiento hasta 2008
				\4[] Estabilización durante crisis
				\4[] Fuerte crecimiento post 2013
				\4[] Caída brusca en Covid
			\3 Blockchain\footnote{Ver Informe sobre el estado del arte de Blockchain en la empresa española.}
				\4 1/10 empresas españolas utilizan blockchain
				\4 Especial importancia en fintech actualmente
				\4 Previsto fuerte crecimiento en sector industrial
				\4 Mayoría de empresas españolas ignoran impacto
			\3 Demanda interna
			\3 Sector exterior\footnote{Ver Informe ONTSI (2019), páginas 149, 150, 152 y 153}
				\4 Exportaciones
				\4[] 10.000 M de € en 2018
				\4 Importaciones
				\4[] 4.000 M de €  en 2018
				\4 Saldo exterior
				\4[] $\then$ 6.000 M de € en 2018
			\3 Evolución
				\4 Antecedentes
				\4 Actualidad
				\4 Perspectivas
				\4[] Inteligencia artificial
				\4[] Blockchain
				\4[] Centralización vs descentralización
				\4[] Digitalización de PYMES
			\3 Diagnóstico
				\4 Debilidades
				\4[] Capital humano en informática
				\4[] $\to$ Falta de capital en España
				\4[] $\to$ Faltan graduados
				\4 Amenazas
				\4[] Capital humano en emergentes
				\4[] $\to$ Menores costes laborales
				\4 Fortalezas
				\4[] Economía muy orientada a servicios
				\4[] Infraestructura de telecomunicaciones avanzada
				\4 Oportunidades
				\4[] Bajas barreras de entrada
				\4[] Sector muy globalizado
			\3 Políticas públicas
				\4 Red.es
				\4[] Entidad pública adscrita al mineco
				\4[] Despliegue de planes de digitalización
				\4[] Gestión de dominios.es
				\4[] Gestión de RedIRIS y ONTSI
				\4 RedIRIS
				\4[] Red académica y de investigación
				\4[] Provisión de servicios de telecomunicaciones avanzados
				\4[] Instituciones afiliadas
				\4[] $\to$ Universidades y centros públicos de investigación
				\4[] Certificados digitales
				\4[] Servicios de seguridad
				\4[] Miembro de GÉANT
				\4[] $\to$ Red académica europea
				\4 ONTSI
				\4[] Observatorio Nacional Tecnologías de las Telecomunicaciones y de la SI
				\4 GDPR\footnote{Ver \href{https://www.europarl.europa.eu/thinktank/en/document.html?reference=EPRS_STU\%282020\%29641530}{EPRS sobre GDPR e IA}}
				\4[] Reglamento General de Protección de Datos
				\4[] Tensión entre:
				\4[] $\to$ Aplicaciones de datasets
				\4[] $\to$ Protección de datos personales
				\4[] Aspectos positivos
				\4[] $\to$ Puede generar confianza en consumidores
				\4[] $\then$ Más disposición a ceder datos personales
				\4 EBSI
				\4[] European Blockchain Services Infrastructure
				\4[] Red europea de prestación de servicios públicos
				\4[] $\to$ A través de blockchain
				\4 Estrategia Nacional de Inteligencia Artificial
				\4[] Registrar entidades con capacidades de investigación
				\4[] $\to$ Universidades
				\4[] $\to$ Institutos
				\4[] $\then$ 154 entidades
				\4[] $\then$ Fomentar sinergias
				\4[] Estrategia Española de I+D+I en IA
				\4[] $\to$ Aprobado en 2019
				\4[] $\to$ Planificación de inversiones futuras
				\4 Oficina del Dato y CDO
				\4[] Gestión de datos generados por AAPP
				\4[] Creación de repositorios sectoriales
				\4 Plan España Digital 2025\footnote{Ver \href{https://www.lamoncloa.gob.es/presidente/actividades/Documents/2020/230720-Espa\%C3\%B1aDigital_2025.pdf}{Plan España Digital 2025}}
				\4[] Actualización digital de AAPP
				\4[] $\to$ Mitad de servicios públicos por medios digitales
				\4[] Sistemas de identificación basados en blockchain
				\4[] $\to$ Prohibición por decreto RD 2019
				\4 Patentes de software
				\4 Neutralidad de red
				\4[] Concepto
				\4[] $\to$ Todo tráfico reciba igual trato
				\4[] $\to$ Operadores no alteran tráfico
				\4[] $\then$ Salvo supuestos legalmente contemplados
				\4[] Basado en reglamento TSM
		\2 Biotecnología
			\3 Delimitación
				\4 Concepto
				\4[] Aplicación tecnológica de:
				\4[] $\to$ Sistemas biológicos
				\4[] $\to$ Organismos vivos o derivados
				\4 Subsectores
				\4[] Biotecnología
				\4[] $\to$ Desarrollo de nuevos productos
				\4[] Biotecnología como herramienta productiva
				\4[] $\to$ Aplicación de biotech a otros procesos
				\4[] Biotecnología como actividad secundaria
				\4 Diferenciación
				\4[] Salud humana
				\4[] Alimentación
				\4[] Otras
				\4 Ciclicidad
				\4[] Relativamente acíclica
				\4[] Similar a otros sectores de alta tecnología
				\4 Fuentes de información
				\4[] Informe AseBio Anual
			\3 Importancia
				\4 Cualitativa
				\4[] Importancia creciente
				\4[] Eslabonamientos hacia abajo con muchos sectores
				\4[] $\to$ Agroalimentario
				\4[] $\to$ Farmacéutico
				\4[] $\to$ Industrial
				\4[] $\to$ Servicios sanitarios
				\4[] Muy elevada inversión en I+D
				\4[] $\to$ Tercer sector con mayor inversión/producción
				\4 Cuantitativa
				\4[] Relativamente pequeña en VAB directo
				\4[] \4[] Cercano a 2.600 M de €
				\4[] 105.000 empleos totales inducidos
				\4[] $\to$ 27.000 trabajadores directos
				\4[] Crecimiento más rápido que otros sectores
				\4[] Potencial impacto cuantitativo elevado
				\4[] Muy elevada importancia en I+D
				\4[] $\to$ Elevado crecimiento de I+D
				\4[] $\to$ Uno de primeros sectores en I+D en 2018
			\3 Modelos teóricos
			\3 Capital
				\4 Organización de empresas
				\4[] Asociación Española de Bioempresas
				\4 Financiación
				\4[] Creciente financiación privada
				\4[] $\to$ Especialmente capital riesgo
				\4[] Crowdfunding mayor que otros sectores
				\4[] Autofinanciación principal fuente
				\4[] Financiación por empresas matrices
				\4[] Programas de apoyo sector público
				\4[] $\to$ CDTI
				\4[] $\to$ AXIS
				\4[] $\to$ ENISA
				\4[] $\to$ ...
				\4[] Aumento de financiación en mercados de capital
				\4 Distribución geográfica
				\4[] Cataluña, Madrid, Andalucía, País Vasco
			\3 Trabajo
				\4 27.000 empleos directos
				\4 105.000 empleos inducidos
				\4 Productividad por trabajador
				\4[] Mucho más elevada que la media
				\4 Elevada cualificación
				\4 Salarios por encima de media
				\4 Elevado peso salarios sobre \% costes operativos
				\4[] $\to$ Intensivo en capital humano
				\4 Aumento de matriculados universitarios en biotecnología
				\4 Elevado peso de mujeres
				\4[] 60\% de nuevos titulados
				\4[] Muy por encima de otros sectores STEM
				\4 Elevada contratación de investigadores
				\4[] Muy por encima de otros sectores
				\4 Elevada tasa de mujeres en mano de obra
			\3 Demanda interna
				\4 Sector agroalimentario
				\4 Sector sanitario
				\4 Sector fuertemente orientado a internacional
			\3 Sector exterior
				\4 España 9a potencia mundial en biotecnología
				\4 Aumento de patentes y citas de artículos
				\4 Sector muy abierto a IDE
				\4 Especialmente vía capital riesgo
				\4 Aumento reciente de adquisiciones empresas españolas
			\3 Evolución
				\4 Rápido crecimiento en últimos años
				\4 Formación de K humano crecimiento rápido
				\4[] Matriculación en estudios universitarios
				\4 Desarrollo del sector a nivel mundial
				\4[] Tasas de crecimiento mucho más que otros
				\4 Otros sectores fuerte tracción
				\4[] Agroalimentario
				\4[] Sanitario
				\4[] $\to$ Envejecimiento de población
			\3 Políticas públicas
				\4 Actividades principales de apoyo
				\4[] Financiación
				\4[] Asesoramiento
				\4[] Transferencia de tecnología
				\4 AEI -- Agencia Estatal de Innovación
				\4 CDTI
				\4[] Innvierte
				\4 Ayudas Neotec
				\4[] Financiación de empresas de base tecnológica
				\4[] Biotecnología uno de mayores receptores
		\2 Energías renovables
			\3 Eólica
				\4 Delimitación
				\4[] Fabricación e instalación de aerogeneradores
				\4 Diferenciación
				\4[] Eólica terrestre
				\4[] Eólica marina
				\4 Importancia
				\4[] Cualitativa
				\4[] $\to$ España productor de patentes a nivel mundial
				\4[] $\to$ Elevado gasto en I+D
				\4[] $\to$ Cumplimiento de objetivos climáticos
				\4[] $\to$ Reducción de precios de electricidad
				\4[] Cuantitativa
				\4 Empleo
				\4[] 23.000 trabajadores
				\4[] Mayoría de trabajo es cualificado
				\4 Empresas
				\4[] 207 centros de fabricación en España
				\4[] Casi todas las CCAA cuentan con algún centro
				\4[] Promotores e instalaciones eólicas
				\4[] $\to$ Iberdrola
				\4[] $\to$ Acciona
				\4[] $\to$ EDPR
				\4[] Principales fabricantes
				\4[] $\to$ Siemens GAmesa
				\4[] $\to$ Vesta
				\4[] $\to$ GE
				\4[] $\to$ Nordex Acciona
				\4 Sector exterior
				\4[] Exportaciones: 2.400 M de € en exportaciones
				\4[] $\to$ Cuarto exportador de aerogeneradores
				\4[] Gran mayoría de generadores españoles para exportación
				\4[] Casi todos los instalados en España\footnote{Ver \href{https://reoltec.net/la-industria-eolica-espanola/}{REOLTEC.NET}}
				\4[] $\to$ Fabricados en España > 85\%
				\4[] Elevada tasa de cobertura > 100\%
				\4[] Componentes de aerogeneradores
				\4[] $\to$ >90\% fabricados en España
				\4 PNIEC
			\3 Solar fotovoltaica
				\4 Concepto
				\4[] Fabricación e instalación de generadores
				\4[] $\to$ Electricidad a partir de energía solar
				\4 Diferenciación
				\4[] Fabricación
				\4[] Instalación
				\4[] Autoconsumo
				\4[] Proyectos comerciales
				\4 Importancia
				\4[] Cualitativa
				\4[] $\to$ Motor de I+D
				\4[] $\to$ Cumplimiento PNIEC
				\4 Empresas
				\4[] Fuerte crecimiento en 2000s de fábricas
				\4[] $\to$ Régimen de remuneración favorable
				\4[] $\then$ Aumento demanda nacional
				\4[] $\then$ Efecto home-market
				\4[] Caída posterior
				\4[] $\to$ Competencia con fabricantes chinos
				\4[] Creciente desde 2016
				\4[] $\to$ Nuevos fabricantes
				\4[] $\to$ Onyx Solar
				\4[] $\to$ DHV Solar (satélites)
				\4[] $\to$ Fotobull
				\4[] Solaria
				\4[] $\to$ Ibex 35 en octubre 2020
				\4[] $\to$ Generación y fabricación de fotovoltaica
				\4[] $\to$ Producción externalizada fuera de España
				\4[] $\to$ Plantas de generación España, Italia, Uruguay
				\4 I+D
				\4[] Media de otros sectores renovables
				\4 Sector exterior
				\4[] Deficitario desde 2008
				\4[] Medidas antidumping en la UE
				\4[] $\to$ Paneles chinos
				\4 Regulación del autoconsumo
				\4[] Sin excedentes
				\4[] $\to$ No pueden inyectar energía excedentaria al exterior
				\4[] $\to$ Sin gravamen especial
				\4[] Con excedentes
				\4[] $\to$ Pueden inyectar a redes de transporte
				\4[] $\then$ Deben abonar peajes de acceso
				\4[] $\then$ Deben abonar IVPEE\footnote{Impuesto sobre el Valor de la Producción de la Energía Eléctrica.}
			\3 Solar termoeléctrica
				\4 Concepto
				\4[] Máquinas de generación de electricidad
				\4[] $\to$ A partir de calor derivado de energía solar
				\4 Diferenciación
				\4[] Diferentes tipos de tecnología
				\4[] $\to$ Canales parabólicos
				\4[] $\to$ Campo de heliostatos
				\4[] $\to$ Reflectores lineales
				\4[] $\to$ Discos parabólicos
				\4[] $\to$ ...
				\4 Importancia
				\4[] Relativamente pequeño en VAB
				\4[] Fuerte demanda de empleo en construcción y fabricación
				\4[] Baja en mantenimiento de plantas
				\4[] Pequeño \% sobre capacidad instalada y generada
				\4 Trabajo
				\4[] 10.000 puestos de trabajo/anuales por planta construida
				\4[] $\to$ Desde inicio hasta finalización
				\4[] $\to$ Directos e indirectos
				\4[] Poco intensivas en trabajo una vez construidas
				\4 Empresas
				\4[] Acciona
				\4[] Abengoa
				\4[] Cobra
				\4[] TSK
				\4[] Sener
				\4 Sector exterior
				\4[] España líder mundial
				\4[] $\to$ Capacidad instalada
				\4[] $\to$ Capacidad tecnológica
				\4[] Exportación de tecnología
			\3 Solar térmica
				\4 Concepto
				\4 Diferenciación
				\4[] Plantas de ciclo combinado
				\4[] Captadores domésticos de agua sanitaria
				\4[] Precalentamiento procesos industriales
				\4 Importancia
				\4[] Cualitativa
				\4[] $\to$ Arrastre sector industrial bienes intermedios
				\4[] Cuantitativa
				\4 Empresas
				\4[] Acciona
				\4[] Elevado número de PYMES
				\4[] $\to$ Instaladoras
				\4[] $\to$ Estudios
				\4[] Productividad por encima de media general
				\4[] $\to$ Pero relat. menor a otros sectores MA
				\4 Demanda interna
				\4[] Previsto aumento progresivo
				\4[] Incentivización PNIEC y otros
				\4 Sector exterior
				\4[] Superávit exterior
				\4[] Exportaciones principalmente a:
				\4[] $\to$ Alemania
				\4[] $\to$ Bélgica
				\4[] $\to$ Chile
				\4[] $\to$ Francia
				\4[] $\to$ Jordania
				\4[] $\to$ Italia
				\4[] $\to$ Marruecos
				\4[] $\then$ Sobre todo pais horas de sol
				\4[] Diseño de proyectos comerciales
		\2 Instrumentos de precisión
			\3 Delimitación
				\4 Numerosos apartados
				\4 Relativos a:
				\4[] Instrumentos de navegación
				\4[] Instrumentos verificación
				\4[] Relojes
				\4[] Instrumental médico y odontológico
				\4[] Óptica y equipos fotográficos
			\3 Importancia
			\3 Modelos teóricos
			\3 Oferta
			\3 Demanda interna
			\3 Demanda externa
			\3 Evolución
			\3 Políticas públicas
	\1[] \marcar{Conclusión}
		\2 Recapitulación
		\2 Idea final
\end{esquemal}


\graficas

\conceptos

\preguntas

\notas

\bibliografia

Mirar en Palgrave:
\begin{itemize}
	\item 
\end{itemize}

SEPE (2016) \textit{Estudio prospectivo del Sector de Bienes de Equipo en España.} Observatorio de las Ocupaciones. -- \href{https://www.sepe.es/contenidos/observatorio/mercado_trabajo/2731-2.pdf}{Disponible aquí} -- En carpeta del tema.


\end{document}
