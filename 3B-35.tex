\documentclass{nuevotema}

\tema{3B-35}
\titulo{La OMC. Los acuerdos distintos de los de mercancías.}

\begin{document}

\ideaclave

Hay que añadir un apartado sobre el STDF -- Fondo para la Aplicación de Estándares y el Fomento del Comercio de la OMC. Ver \url{http://www.standardsfacility.org/es/qui\%C3\%A9nes-somos}

Aunque las ventajas del comercio internacional en términos de bienestar fueron enunciadas por Adam Smith y David Ricardo hace ya más de dos siglos, la apertura comercial al extranjero sigue siendo una materia controvertida a nivel nacional en casi todas las economías del mundo. La consolidación del GATT de 1947 en forma de la Organización Mundial de Comercio tras la conclusión de la Ronda de Uruguay en 1994 supuso un enorme paso adelante en este proceso de liberalización y multilateralización. La OMC le otorga rango institucional pleno al GATT, introduce un mecanismo de solución de diferencias comerciales y amplia el alcance de la apertura multilateral más allá del comercio de mercancías. Actualmente, el comercio internacional es mucho más que el tráfico de mercancías. La propiedad intelectual, los servicios, las denominaciones de origen, las armonizaciones regulatorias... son hoy todos ellos elementos clave del comercio internacional que se sitúan en el foco de las negociaciones comerciales. Aunque el margen de liberalización en esas materias son aún muy importantes, la creación de la OMC trajo consigo dos acuerdos de especial relevancia por ser los primeros acuerdos multilaterales es sus respectivas materias: el GATS sobre el comercio de servicios y el TRIPS (ADPIC) sobre la protección de la propiedad intelectual.

El \textbf{objeto} de la exposición consiste en dar respuesta a una serie de preguntas en relación a estos acuerdos multilaterales distintos de los de mercancías: ¿en qué consisten estos acuerdos? ¿cuáles son sus antecedentes? ¿qué aportan? ¿qué sentido económico tienen? ¿qué implicaciones tienen? ¿qué otros acuerdos plurilaterales existen? ¿cuáles son las perspectivas de las negociaciones? ¿qué intereses tienen los distintos bloques comerciales en las negociaciones de nuevos acuerdos? La \textbf{estructura} de la exposición se divide en tres partes: el GATS, el TRIPS y otros acuerdos multilaterales y plurilaterales relevantes en el contexto del comercio distinto de mercancías.

El \textbf{General Agreement on Trade in Services} (GATS) (o Acuerdo General sobre el Comercio de Servicios --AGCS-) es el acuerdo multilateral concluido en Marraqués en 1994 que establece el marco de liberalización del comercio internacional de servicios. La importancia de la liberalización del comercio de servicios es tanto cuantitativa como cualitativa. El crecimiento en volumen ha sido enorme en la última mitad del siglo, hasta el punto de que aproximadamente dos tercios de la producción y el empleo mundiales corresponden a los servicios. Su importancia en el comercio internacional es también muy relevante: hasta el 20\% del valor del comercio mundial corresponde a los servicios y en términos de valor añadido la cifra asciende hasta el 50\%. En el caso de la cuenta corriente española, el peso de los servicios es aún mayor. La importancia cualitativa del acuerdo es también fundamental: es un interés ofensivo de vital importancia para los países desarrollados y un interés defensivo para los países en desarrollo en la mayoría de los sectores, aunque ofensivo en otros como la prestación de servicios mediante desplazamiento de trabajadores o el acceso a inputs esenciales. En este juego de intereses, la liberalización de los servicios ha sido utilizado por los países en desarrollo como moneda de cambio respecto de concesiones en el ámbito agrícola. El objetivo central del acuerdo es doble: promover la expansión del comercio de servicios y definir una serie de normas comunes para procesos de apertura ulterior. Las diferencias con el GATT son notables y van más allá del ámbito de aplicación. El tráfico de servicios es mucho más complejo que el comercio de mercancías: no basta con negociar un arancel más bajo sino que a menudo es necesario modificar la legislación nacional para hacer posible la prestación de servicios por parte de agentes internacionales. Cambios legales son siempre actuaciones sensibles y complejas que dificultan las negociaciones y el grado de ambición de las partes. Otra diferencia fundamental con el GATT es la mayor flexibilidad del principio de trato nacional. En el GATS, el trato nacional es negociable y los estados miembros pueden decidir si conceden o no trato nacional, en qué grado lo conceden, en qué sector y en qué modo de suministro. El GATS es gran medida un acuerdo sin \textit{antecedentes}. Las liberalizaciones del tráfico de servicios nunca antes se habían llevado a cabo en el ámbito multilateral. Es por ello que era necesario, en un contexto de integración creciente de la economía mundial y desarrollo de tecnologías que hacen posible un tráfico de servicios más fluido, la implementacion de un marco legal de liberalización que aportase seguridad jurídica.

La \textit{estructura del GATS} comienza por definir el ámbito de aplicación. Todos los servicios y medidas del sector público forman parte del acuerdo, lo cual no implica que todos estos sectores sean liberalizados. Existen dos excepciones al ámbito de aplicación. La primera concierne los servicios públicos prestados en ejercicio de la autoridad gubernativa de forma no comercial o sin competencia con otros proveedores. La segunda corresponde a los servicios de tráfico aéreo y relacionados. La clasificación de los sectores de servicios es especialmente relevante para concretar los sectores liberalizados en las negociaciones comerciales. La mayoría de países utilizan la \textit{Central Product Classification} de la ONU de 1991, que define 12 sectores de servicios y 150 subsectores. Un reto al que se enfrenta el GATS es la actualización de la clasificación. El Comité de Compromisos Específicos (o de clasificación) se encarga de esta tarea en la actualidad. Se definen también \textit{cuatro modos de suministro}, reflejo de la mayor complejidad del tráfico de servicios frente al tráfico de mercancías. El modo 1 hace referencia al comercio transfronterizo de servicios prestados a distancia, de tal manera que ni prestador ni cliente se desplazan. Ejemplos de este modo de prestación son los servicios digitales o el asesoramiento a distancia. El modo 2 concierne el consumo de servicios en el extranjero cuando el consumidor se desplaza al país de prestación. Ejemplos de este modo son el turismo, los servicios médicos en los que el paciente se desplaza, los desplazamientos por motivo de estudios o las reparaciones de embarcaciones o aeronaves en el extranjero. El modo 3 de prestación corresponde a aquellas situaciones en las que el prestador se instala en el país de prestación a través de una representación comercial, una filial o una inversión en una empresa extranjera, como pueden ser un hotel en el extranjero o una sucursal bancaria. El cuarto modo de prestación hace referencia a aquellas situaciones en las que el prestador de servicios desplaza a trabajadores al lugar de prestador. Ejemplos de este último modo son la invitación de profesores, consultores, instaladores de maquinaria, gerentes desplazados sobre el terreno, obreros, mano de obra agrícola desplazada... 

El acuerdo establece a continuación una serie de \textit{obligaciones generales} a los estados miembros que como su nombre indica, deben extenderse a todos los miembros y todo los sectores. El \textit{principio de nación más favorecida} es la primera de ellas, de tal manera que una ventaja a un estado miembro debe extenderse también a los demás estados miembros. Las excepciones a este principio son conciernen los acuerdos regionales tales como acuerdos de libre comercio y uniones aduaneras que no impongan condiciones más restrictivas a terceros países y que tengan carácter general. Además, en el momento de la entrada en vigor del GATS se establecieron algunas excepciones limitadas de 10 años de duración. El principio de \textit{transparencia} impone la obligación de publicar los cambios regulatorios e informar de ellos a la OMC. La \textit{normativa nacional} relativa a la prestación de servicios debe ser objetiva, transparente y proporcionada, y ofrecer la posibilidad de un recurso imparcial ante los tribunales. Los \textit{monopolios} están permitidos pero se debe respetar el principio de nación más favorecida y evitar las prácticas anticompetitivas a favor de proveedores concretos en los sectores liberalizados. Otras obligaciones generales son el compromiso general de liberalización, evitar prácticas restrictivas del comercio de servicios, otorgar un trato favorable a países en desarrollo. En el futuro, se plantea la posibilidad de negociar cláusulas de salvaguardia al semejanza del GATT, pero de momento no se ha llevado a cabo esta posibilidad. Las obligaciones generales admiten restricciones cuando en situaciones de crisis de la balanza de pagos, cuestiones de seguridad, moral, orden público, saludo medio ambiente, y existan amenazas a la estabilidad del sistema financiero.

Las \textit{obligaciones específicas} son particulares a los estados miembros y a determinados sectores y compromisos específicos de acceso al mercado. Los estados miembros presentan en el momento de su accesión a la OMC una lista de compromisos específicos de liberalización en los que definen qué sectores liberalizarán, aunque tienen libertad para decidir cuáles. Los compromisos pueden ser de dos tipos: acceso a mercado y trato nacional. El compromiso de acceso a mercado implica evitar las limitaciones al número de proveedores extranjeros, al valor de los activos o las transacciones, al número de operaciones de servicios prestadas, al número de personas físicas empleadas, a los tipos de persona jurídica y a la limitación del capital extranjero en la estructura de capital de las empresas. Por otra parte, el compromiso de trato nacional implica dar igual trato a proveedores nacionales y extranjeros, sea cual sea éste. Los compromisos específicos son en todo caso listas positivas, de modo que sólo se liberaliza lo que está efectivamente en la lista. En la Ronda de Uruguay y de forma previa a la presentación de los compromisos de liberalización, los estados miembros presentaron demandas de liberalización a los países en los cuales tenían intereses especiales, solicitando la liberalización de determinados sectores.

La \textit{valoración del GATS} en la actualidad es mixta. La evolución ha sido en cierto modo decepcionante. Tras un comienzo prometedor con la propia conclusión del GATS y el Acuerdo sobre Tecnologías de la Información en 1996, la apertura al comercio de servicios tendió a estancarse bajo el peso de la controversia agrícola. La Conferencia Ministerial de Ginebra, en 2011 supuso un ligero avance con la aprobación del waiver para PMAs que permite la liberalización selectiva a países menos desarrollados. En Nairobi 2015 el waiver comenzó a ponerse en práctica y se notificaron listas de liberalización a los PMAs. Aunque éstos habían demandado la liberalización de una amplia gama de sectores y modos de provisión, el temor a un free-riding generalizado resultó en compromisos de liberalización tímidos que se limitaron básicamente a consolidar la liberalización ya existente. Los países desarrollados han liberalizado en general mucho más que los países en desarrollo. Especialmente importantes han sido las liberalizaciones en el turismo, los servicios financieros y las telecomunicaciones y en el modo 2, el más fácilmente liberalizable. El modo 4 se mantiene poco liberalizado, a pesar de ser el que mayor peso tiene en el comercio de servicios junto con el modo 3. El modo 1 gana peso creciente gracias al desarrollo de las nuevas tecnologías y la generalización de internet. En términos globales, la ambición del GATS es una de las causas de su resultado incompleto. La gran flexibilidad que permite el acuerdo incentiva el free-riding y acaba por frenar una mayor liberalización. De cara al futuro, la complejidad del comercio de servicios seguirá siendo un obstáculo a la liberalización. Asimismo, la negociación agrícola seguirá siendo utilizada como moneda de cambio en la negociación de servicios. La necesidad de modificar legislación y reducir barreras no arancelarias que es inherente a liberalización comercial provoca a menudo difíciles debates de política interna que tienden a dificultar la liberalización. Otro de los aspectos que reducen la efectividad del GATS es el hecho de que algunas de las liberalizaciones no son sino consolidaciones de procesos de liberalización unilaterales concluidos previamente. La UE ha mostrado un liderazgo importante en los avances que se han producido. En la actualidad trata de mantener vivo el impulso liberalizador a pesar de los reveses que ha sufrido el comercio internacional en los últimos dos años. En la actualidad se debate el \textit{Trade in Service Agreements} (TiSA) que comentaremos posteriormente. Este tratado plurilateral agruparía al 70\% del comercio mundial de servicios, estaría abierto a nuevos miembros y es susceptible de multilateralización en el futuro. 

El \textbf{Acuerdo sobre Derechos de Propiedad Intelectual Relacionados con el Comercio} (TRIPS) es el otro gran acuerdo multilateral de la organización mundial de comercio. Los sectores generadores de propiedad intelectual tienen un peso cada vez mayor en la generación de valor añadido y empleo. En la Unión Europea, casi el 40\% del PIB se genera en sectores intensivos en propiedad intelectual. La generación de propiedad intelectual implica generalmente grandes costes fijos. Por ello, para incentivar esta creación de PI son a menudo necesarias los monopolios temporales. Estos costes fijos dan lugar a la existencia de grandes economías de escala. Su aprovechamiento pleno se ve dificultado cuando el mercado es demasiado pequeño. Por ello, la expansión internacional de las empresas que han generado capital en forma de propiedad intelectual es necesaria y deseable. Tanto si toma forma de exportación como de inversión extranjera directa, la expansión internacional requiere la protección de los derechos de protección internacional. Así, el objetivo del TRIPS es doble: establecer unas normas de protección mínima de la propiedad intelectual y poner a disposición de los miembros de la OMC el mecanismo de solución de diferencias también para la propiedad intelectual.

La \textit{evolución} de los acuerdos de PI tiene origen en las Convenciones de París y Berna de 1880. En 1960 se crea la World Intellectual Property Organization para fomentar la protección de la PI en todo el mundo. En los países desarrollados, los regímenes de protección pre-TRIPS no tienen apenas diferencias significativas con el propio TRIPS, y en muchos casos la protección era y es superior. En los países en desarrollo, sin embargo, es donde el TRIPS ha supuesto un avance real. La protección de la PI era en estas economías muy débil y suponía un freno a la inversión extranjera directa. Además, desincentivaba la apertura comercial hacia estos países. 

La \textit{estructura} del acuerdo comienza por enumerar los principios habituales de trato nacional y nación más favorecida en términos similares al GATT y el GATS y las obligaciones generales. Incorpora sin embargo un principio adicional denominado de \textit{protección equilibrada}. Este principio postula la necesidad de equilibrar los incentivos a la innovación a través de la protección de la PI con los costes de corto plazo para los usuarios potenciales que no tienen acceso a la creacióon de propiedad intelectual. Es decir, la protección de la PI debe tener como objetivo mejorar el bienestar social y no sólo proporcionar beneficios a los creadores. Esto se traduce en la obligación de buscar periodos de protección equilibrados suficientemente largos como para incentivar la innovación pero no como para evitar transferencia de tecnología, e implementar grados de protección especial para países en vías de desarrollo. El instrumento para lograr estos objetivos son las llamadas \textit{normas mínimas de protección} que examinaremos posteriormente. Otras obligaciones generales son la prohibición de implementar prácticas anti-competitivas para restringir la competencia o la transferencia de tecnología, la obligación de establecer medidas nacionales para asegurar implementación y los periodos transitorios de aplicación. Las medidas nacionales son un elemento esencial del acuerdo: no basta con declarar la existencia de normas para proteger la PI, sino que es necesario también que los agentes involucrados sean también capaces de hacer valer sus derechos en el país en el que consideran violada su propiedad intelectual. El TRIPS establece la obligación de implementar medidas ex-ante en forma de acceso a los tribunales por vías civiles y penales, y medidas ex-post en forma de sanciones lo suficientemente elevadas como para disuadir infracciones ulteriores. Los periodos de transición varían desde 1 año para los países desarrollados hasta 11 años para los países menos avanzados, con un periodo especial para las patentes de medicamentos hasta 2016. 

Las \textit{normas mínimas de protección} se han mencionado anteriormente como el instrumento básico de protección de la PI en el marco del TRIPS. Como su nombre indica, consisten en la obligación de implementar unos mínimos de protección en términos de años mínimos de protección de la PI, en cuantías diferentes según su tipo. Las \underline{patentes} son objeto de una protección mínima de 20 años, están sujetas a la obligación de que sean nuevas y tengan aplicación industrial, así como a la obligación de divulgar información técnica al respecto y a la posibilidad de obligar al titular a emitir licencias de uso en condiciones razonables. Algunos ámbitos están exentos de esta obligación mínima de protección tales como las invenciones contrarias a la salud, el orden público, la vida o la moral, así como los métodos de diagnóstico médicos y las plantas y animales (aunque sí pueden patentarse los microorganismos tales como las vacunas). Los \underline{derechos de autor} disfrutan una protección mínima de 50 años. Los \underline{diseños industriales} y los \underline{circuitos integrados} gozan ambos de 10 años de protección y las \underline{marcas} de 7 años anulables por falta de uso. La protección de las \underline{indicaciones geográficas} concierne la implementación de medidas para impedir el uso de IG falsas que puedan inducir a error y aprovechar la reputación que tienen tales denominaciones. Es especialmente relevante en las bebidas alcohólicas. Algunos países productores de bienes especialmente asociados con regiones geográficas e importante reputación a nivel mundial presionan a los países importadores para aumentar la protección de estas denominaciones. 

La \textit{valoración} del TRIPS está sujeta a un grado importante de controversia. Mientras que los países desarrollados siguen presionando por su implementación por ser la PI uno de sus grandes intereses ofensivos, los países en desarrollo se oponen a la implementación de determinadas medidas e incluso a debatir el grado de observancia del TRIPS en los órganos de la OMC. Alegan, además, que los periodos de transición son demasiado cortos y que sirven sólo a la consolidación de los beneficios de las empresas de países desarrollados. 

En la actualidad, las \textit{áreas contenciosas principales} son las patentes sobre medicamentos, la protección de las indicaciones geográficas y los convenios sobre biodiversidad. La declaración de Doha de 2001 supuso una importante cesión a los países en desarrollo: se acepta la posibilidad de que cada estado miembro conceda licencias obligatorias para producir medicamentos, con la condición de que la producción se destine sólo al mercado interno. Además, se extiende el plazo de la exención a los países menos avanzados hasta 2016. Sin embargo, muchos de los países menos avanzados no tienen capacidad para desarrollar nuevos medicamentos. En 2005, los miembros de la OMC acordaron ignorar temporalmente la condición de que los medicamentos producidos bajo licencia sean destinados exclusivamente al consumo interno, de tal manera que los PMAs puedan importar medicamentos genéricos producidos en el extranjero. Además, en 2015 se extendió la exención de los PMAs y las patentes farmacéuticas hasta 2033. 

La protección de las IGs es uno de los principales intereses de la Unión Europea y en la actualidad propone la creación de un registro multilateral de vinos y bebidas alcohólicas y la extensión de la protección a otros productos tales como el arroz, el té o el queso. Este debate enfrente a países que exportan productos con muy antigua tradición y muy buena imagen de marca en todo el mundo. Se pretende evitar el aprovechamiento de la imagen de marca y el deterioro subsecuente que resulta de la venta de productos de peor calidad.

El Convenio de Diversidad Biológica trata de regular la utilización sostenible de la biodiversidad y el material genético que en ella se encuentra, a través de la implementación de mecanismos de cooperación, acceso a recursos genéticos y transferencia de tecnología necesarias para una explotación más respetuosa de los recursos naturales. Algunos países piden la incorporación de este convenio al TRIPS, aunque no la UE ni los Estados Unidos (que no han ratificado el acuerdo). Si se llegase a acordar la integración, los solicitantes de patentes estarían obligados a divulgar el origen genético o los conocimientos tradicionales que han utilizado para su desarrollo. Es por ello que países en desarrollo especialmente ricos en biodiversidad presionen para su introducción.

Las \textit{perspectivas de futuro} parecen apuntar a avances en el plano plurilateral o bilateral. En el marco del TRIPS resulta muy difícil avanzar por la utilización de estos temas como moneda de cambio en la negociación agrícola. Así, la Unión Europea incluye capítulos sobre PI en sus acuerdos bilaterales. Los TPP y TTIP introducen también capítulos específicos pero la probabilidad de su consecución ha disminuido mucho en los últimos años. En el plano pluriteral, el ACTA (Anticounterfeiting Trade Agreement) fue un intento por desarrollar el TRIPS y concretar las medidas de control de la observancia que no llegó a término y que se encuentra en suspenso tras el voto en contra del Parlamento Europeo en 2012.

Más alla del GATS y el TRIPS, son varios los \marcar{otros acuerdos} que se encuentran en proceso de debate preliminar o negociación. En el plano \textbf{multilateral} son relevantes los 4 temas de Singapur: regulación de la competencia, regulación y protección de la inversión directa internacional, contratación pública y facilitación del comercio. Se plantean también recientemente discusiones en relación al comercio electrónico y la transferencia de tecnología. Sin embargo, en un contexto de paralización o abandono de la Ronda de Doha, todo parece apuntar a la negociación plurilateral o bilateral.

Así, en el contexto \textbf{plurilateral} encontramos en la actualidad tres acuerdos en vigor relativos al comercio distinto de mercancías: el acuerdo de comercio de aeronaves civiles, el acuerdo de contratación pública y el acuerdo sobre tecnologías de la información (ATI). Además, tres acuerdos se encuentran en negociación: una ampliación del ATI, el acuerdo TiSA sobre servicios y el EGA sobre bienes medioambientales. El \textbf{EGA} trata de eliminar los aranceles de bienes que pueden acercar el cumplimiento de objetivos medioambientales tales como los relacionados con producción de energías renovables, control de residuos, gestión de residuos... y reducir las barreras no arancelarias en una segunda fase. Los 27 países involucrados son casi todos países desarrollados. Por el momento, las negociaciones avanzan con lentitud. La posible multilateralización del acuerdo una vez se alcance una masa crítica supone realmente un obstáculo a las negociaciones porque los países que negocian temen el free-riding. El acuerdo TiSa actualmente se encuentra en negociación por 23 miembros de la OMC, no todos ellos desarrollados, que tratan de liberalizar los servicios a gran escala. Cubriría todos los servicios sin exclusiones a priori salvo las del GATS. Además, tiene por objetivo concretar las obligaciones generales y definir las que conciernen a sectores concretos que requieren regulación específica como las telecomunicaciones, los servicios financieros, el comercio electrónico o las compras públicas. 

El \textit{Acuerdo de Contratación Pública} o GPA tiene como objetivo la apertura de los mercados de contratación pública y la mejora de la transparencia y la competencia en este tipo de procedimiento. Fue firmado en 1994, junto con la creación de la OMC. En 2012 se aprobó una mejora que no entró en vigor hasta 2014. El peso de la contratación pública en el PIB de las economías desarrolladas oscila entre el 10\% y el 20\% y su apertura al exterior tiene el potencial de contribuir a un aumento de la eficiencia. Como es habitual en los acuerdos de apertura distintos de mercancías, el acuerdo se divide en una parte normativa y las listas de compromisos de apertura de las partes. El acuerdo establece en su parte normativa dos dos principios clave: transparencia y trato nacional. Las listas de cobertura determinan las actividades que están sujetas a estos principios en cada país, así como los umbrales de contratación a partir de los cuales se aplica. Se incorporan dos mecanismos para solucionar disputas: a nivel interno del país de contratación, y en el seno del mecanismo de solución de diferencias de la OMC. Como valoración, el ACP ejerce una presión sobre la eficiencia del sector público muy positiva. Dificulta el proteccionismo en el sector público y dificulta en cierta medida prácticas corruptas. Sin embargo, necesita de adaptación a la legislación nacional y choca a menudo con tendencias proteccionistas del tipo ``buy American''. 

Como \marcar{conclusión}, cabe señalar que España tiene claros objetivos ofensivos y defensivos en las negociaciones comerciales. En el plano ofensivo se encuentran los servicios, la PI y las IG, el acceso a mercado para productos no agrícolas, los bienes y servicios medioambientales, la regulación de la IDE y la transparencia en las compras públicas, entre otros. Entre los intereses defensivos se encuentran las subvenciones para la pesca y el algodón así como el mantenimiento de la PAC. En cualquier caso, es necesario tener presente que las turbulencias políticas mundiales y la aparición de tensiones comerciales entre China y Estados Unidos, y potencialmente entre el Reino Unido y la UE, además de la creciente influencia china en Asia tiene el potencial de cambiar el panorama de apertura comercial. La OMC se encuentra también sumida en una crisis tras los problemas para renovar los cargos en el órgano de apelación y el fracaso de la Ronda de Doha. Todos estos factores distraen en cierta medida la atención respecto del proceso liberalizador y pueden continuar siendo obstáculos en los próximos años.


\seccion{Preguntas clave}

\begin{itemize}
	\item ¿Qué otros acuerdos aparte del GATT se aplican a los miembros de la OMC?
	\item ¿Cuáles son sus antecedentes?
	\item ¿Qué aportan?
	\item ¿Qué sentido económico tienen?
	\item ¿Qué estipulan?
	\item ¿Qué valoración de los tratados?
	\item ¿En qué situación se encuentran las negociaciones?
\end{itemize}

\esquemacorto

\begin{esquema}[enumerate]
	\1[] \marcar{Introducción}
		\2 Contextualización
			\3 Beneficios del comercio
			\3 Organización Mundial de Comercio
			\3 Apertura comercial más allá de las mercancías
			\3 Situación actual
		\2 Objeto
			\3 ¿En qué consisten el GATS y el TRIPS?
			\3 ¿Qué antecedentes?
			\3 ¿Qué aportan?
			\3 ¿Qué sentido económico?
			\3 ¿Que implican?
			\3 ¿Qué acuerdos plurilaterales?
			\3 ¿Cuáles son las perspectivas de las negociaciones?
			\3 ¿Qué intereses tienen los distintos bloques?
		\2 Estructura
			\3 GATS
			\3 TRIPS
			\3 Otros acuerdos
	\1 \marcar{GATS} \textit{Acuerdo General Sobre el Comercio de Servicios}
		\2 Idea clave
			\3 Importancia del comercio de servicios
			\3 Objetivo del GATS
			\3 Diferencias con GATT
		\2 Antecedentes
			\3 Sin precedentes
			\3 Ronda de Uruguay
		\2 Estructura
			\3 Ámbito de aplicación
			\3 Modos de suministro
			\3 Obligaciones generales
			\3 Obligaciones específicas
		\2 Valoración
			\3 Evolución
			\3 Situación actual
			\3 Perspectivas futuras
	\1 \marcar{TRIPS} \textit{Acuerdo Sobre Derechos de Propiedad Intelectual Relacionados con el Comercio (ADPIC)}
		\2 Idea clave
			\3 Importancia del TRIPS
			\3 Objetivos
			\3 Intereses
		\2 Antecedentes
			\3 Acuerdos pre-TRIPS
			\3 Países desarrollados
			\3 Países en desarrollo
		\2 Estructura
			\3 Principios básicos y obligaciones generales
			\3 Normas mínimas de protección para 7 tipos de DPI
			\3 Resolución de disputas
		\2 Valoración
			\3 Situación actual
			\3 Países desarrollados
			\3 Países en desarrollo
			\3 Valoración
			\3 Perspectivas
	\1 \marcar{Otros acuerdos}
		\2 Multilaterales
			\3 Medio ambiente y comercio
			\3 Comercio electrónico
			\3 Transferencia de tecnología
			\3 Comercio y desarrollo
		\2 Plurilaterales
			\3 TISA - Trade in Services Agreement
			\3 EGA
			\3 GPA Acuerdo de Contratación Pública
	\1[] \marcar{Conclusión}
		\2 Recapitulación
			\3 GATS
			\3 TRIPS
			\3 Otros acuerdos
		\2 Idea final
			\3 Ofensivos de España
			\3 Defensivos de España
			\3 Postura negociadora de la UE
			\3 Crisis de la OMC

\end{esquema}

\esquemalargo












\begin{esquemal}
	\1[] \marcar{Introducción}
		\2 Contextualización
			\3 Beneficios del comercio
				\4 Reconocidos generalmente
				\4[] Teoría
				\4[] Evidencia empírica
				\4 Ganadores y perdedores del comercio
				\4[] Dificulta liberalización
				\4[] $\to$ Problemas de economía política
			\3 Organización Mundial de Comercio
				\4 Uruguay 1994
				\4 Consolidación del GATT de 1947
				\4 Mejoras múltiples:
				\4[] Rango institucional
				\4[] Solución de diferencias
				\4[] Ampliación de temas
			\3 Apertura comercial más allá de las mercancías
				\4 Actualmente, CI es mucho más que mercancías
				\4[] PI, servicios, denominaciones de origen...
				\4 Frontera de las negociaciones comerciales
				\4[] Mucho que negociar y acordar
				\4[] Mayor dificultad que liberalización de mercancías
			\3 Situación actual
				\4 Dos acuerdos multilaterales aparte del GATT
				\4 Acuerdos pluri. y multi. en negociación
				\4 Bloqueo y abandono de la Ronda de Doha
				\4 Tendencias proteccionistas
		\2 Objeto
			\3 ¿En qué consisten el GATS y el TRIPS?
			\3 ¿Qué antecedentes?
			\3 ¿Qué aportan?
			\3 ¿Qué sentido económico?
			\3 ¿Que implican?
			\3 ¿Qué acuerdos plurilaterales?
			\3 ¿Cuáles son las perspectivas de las negociaciones?
			\3 ¿Qué intereses tienen los distintos bloques?
		\2 Estructura
			\3 GATS
			\3 TRIPS
			\3 Otros acuerdos
	\1 \marcar{GATS} \textit{Acuerdo General Sobre el Comercio de Servicios}
		\2 Idea clave
			\3 Importancia del comercio de servicios
				\4 Cuantitativa
				\4[] Crecimiento enorme en las últimas décadas
				\4[] 2/3 de la producción y empleo mundiales
				\4[] 20\% del comercio mundial y 50\% VA
				\4[] En com. int. de España, peso aún mayor
				\4 Cualitativa
				\4[] Primer y única regulación del comercio de servicios
				\4[] Interés ofensivo de los países desarrollados
				\4[] Interés defensivo en general de los PEDs
				\4[] $\to$ pero ofensivo en desplazamiento de trabajadores
				\4[] $\to$ y permite acceder a inputs esenciales
				\4[] Moneda de cambio para los PEDs
				\4[] $\to$ concesiones agrícolas
			\3 Objetivo del GATS
				\4 Expansión del comercio de servicios
				\4 Definir normas comunes para CI de servicios
			\3 Diferencias con GATT
				\4 Ámbito de aplicación
				\4[] Servicios son actividades más complejas
				\4[] No basta con negociar un arancel más bajo
				\4[] Necesario modificar legislación
				\4 Mayor flexibilidad
				\4[] El Trato Nacional se puede negociar
		\2 Antecedentes
			\3 Sin precedentes
				\4 Sin liberalizaciones generalizadas de servicios
				\4 Acuerdos bilaterales o a nivel regional
			\3 Ronda de Uruguay
				\4 Primer marco de liberalización generalizada
				\4 Necesario marco legal de liberalización
				\4[] $\to$ Aportar seguridad jurídica
				\4[] $\to$ Tanto o + beneficios que con mercancías
		\2 Estructura
			\3 Ámbito de aplicación
				\4 Sujetos todos servicios y medidas del sector público
				\4 Excepciones:
				\4[] Servicios públicos en ejercicio de autoridad
				\4[] $\to$ No comercialmente
				\4[] $\to$ Sin competencia con otros proveedores
				\4[] $\to$ Ej.: sanidad, Seguridad Social, bomberos..
				\4[] Servicios de tráfico aéreo y relacionados
				\4 Clasificación de sectores
				\4[] Diferentes sistemas de clasificación
				\4[] $\to$ CPC -- Central Product Classification
				\4[] $\to$ Standard International Trade Classification
				\4 GATS usa SSCS del 1991\footnote{Ver \href{http://i-tip.wto.org/services/default.aspx}{I-TIP de OMC y Banco Mundial}.}
				\4[] Services Sectoral Classification List
				\4[] 12 sectores de servicios y 160 subsectores
				\4[$\to$] Basada en Central Product Clasification de ONU
				\4[] Clasificación de servicios de ONU (1991)
				\4[] $\to$ Última versión 2.1 de 2015\footnote{Ver \href{https://unstats.un.org/unsd/classifications/unsdclassifications/cpcv21.pdf}{ONU (2015)}.}
				\4[] Mayoría de miembros de OMC la utilizan
				\4[] Obsoleta con aparición de nuevos servicios
				\4 Comité de Compromisos Específicos
				\4[] (o \textit{de clasificación})
				\4[] Trata de actualizar clasificación de productos
				\4 Excepciones
				\4[] Servicios públicos no comerciales o en monopolio
				\4[] Derechos de tráfico aéreo y relacionados
			\3 Modos de suministro
				\4 Servicios más complejos que mercancías
				\4[$\to$] En mercancías, sólo transfronterizo
				\4 Necesario definir modos de suministro
				\4[1] Comercio transfronterizo
				\4[] Servicio prestado a distancia
				\4[] Ej.: servicios digitales, asesoramiento
				\4[2] Consumo en el extranjero
				\4[] El consumidor se desplaza para consumir
				\4[] Ej.: turismo, pacientes médicos, estudiantes
				\4[3] Presencia comercial
				\4[] Prestador de servicios se establece en país de prestación\footnote{Por esta razón, este tipo de servicios no se computan en la balanza de pagos por el criterio de residencia. Ha sido necesario crear una estadística explícita para este tipo de servicios, la \textit{``Foreign Affiliates Trade in Services''}.}
				\4[] Ej.: servicios de sucursales de bancos, hoteles
				\4[4] Movimiento temporal de trabajadores
				\4[] Una persona se traslada para prestar el servicio
				\4[] Ej.: profesores invitados, consultores, instaladores de maquinaria
				\4[] gerentes desplazados, obreros temporalmente desplazados...
			\3 Obligaciones generales
				\4 \marcar{N}ación Más Favorecida
				\4[] $\Rightarrow$ cualquier ventaja se extiende a otros miembros
				\4[] $\Rightarrow$ problema del free-riding
				\4[] Excepciones:
				\4[] $\to$ acuerdos regionales (UAs y FTAs)
				\4[] $\to$ reconocimiento de títulos académicos (bilateral)
				\4[] $\to$ excepciones limitadas al firmar GATS (10 años)
				\4 \marcar{T}ransparencia
				\4[] Publicar e informar a OMC de cambios regulatorios
				\4 \marcar{N}ormativa nacional
				\4[] $\to$ Objetiva
				\4[] $\to$ Transparente
				\4[] $\to$ Proporcionada
				\4[] Posibilidad de recurso imparcial ante tribunales
				\4 \marcar{M}onopolios y prácticas anticompetitivas
				\4[] Monopolios están permitidos
				\4[] Pero se debe respetar NMF
				\4[] No deben abusar de posición dominante
				\4[] Prohibidas prácticas anticompetitivas
				\4[] $\to$ En sectores liberalizados
				\4 Otras
				\4[] Compromiso de liberalización
				\4[] Evitar prácticas restrictivas
				\4[] Trato favorable a PEDs y PMA
				\4[] Posibilidad de negociar cláusulas de salvaguardia
				\4 Excepciones
				\4[] Protección de balanza de pagos
				\4[] Protección seguridad, moral, orden público, salud, medio ambiente
				\4[] Estabilidad del sistema financiero
				\4[] $\to$ Aplicable a liberalización servicios financieros
			\3 Obligaciones específicas
				\4 Cada miembro presenta lista de compromisos
				\4[$\to$] Sectores y grado en que liberalizarán
				\4[$\to$] Libertad de decisión
				\4 Compromisos de dos tipos
				\4[] Acceso a mercado
				\4[] Trato nacional
				\4 Acceso a mercado
				\4[] Países pueden limitar múltiples aspectos:
				\4[] $\to$ número de proveedores extranjeros
				\4[] $\to$ valor de activos o transacciones
				\4[] $\to$ número de operaciones de servicios
				\4[] $\to$ número de personas físicas empleadas
				\4[] $\to$ tipos de persona jurídica
				\4[] $\to$ participación de capital extranjero
				\4 Trato Nacional
				\4[] Obligación de no modular condiciones
				\4[] $\to$ Para favorecer proveedores locales
				\4[] Pero modulable en muchos otros aspectos
				\4[] $\to$ Con efectos prácticos similares
				\4 Lista positiva de compromisos de liberalización (``\textit{schedules}'')
				\4[$\to$] Lo que está en la lista, se liberaliza
				\4[] Incluye grado de liberalización
				\4[] Sectores de liberalizacion
		\2 Valoración
			\3 Evolución
				\4 Singapur 1996:
				\4[] Acuerdo sobre Tecnologías de la Información
				\4 Ginebra 2011:
				\4[] \textit{waiver} de servicios\footnote{\url{https://www.wto.org/english/news_e/news11_e/serv_17dec11_e.htm}.}
				\4[] Trato especial a PMAs en servicios
				\4[] $\to$ Liberalización parcial
				\4[] $\to$ No PMAs no obtienen acceso
				\4[] Válido durante 15 años
				\4 Bali 2013:
				\4[] Acuerdo de Facilitación de Comercio
				\4[] $\to$ No se aplica a servicios
				\4[] PMAs demandan colectivamente apertura modo 4
				\4 Nairobi 2015:
				\4[] Waiver se pone en práctica
				\4[] Notificación de listas a los PMAs
				\4[] $\to$ 23 miembros notifican
				\4[] PMAs valoran negativamente las ofertas
			\3 Situación actual
				\4 Compromisos escasos de liberalización
				\4 Miedo al free-riding
				\4 GATS consolidó liberalización ya existente
				\4 PD han liberalizado mucho más que PEDs
				\4 Sectores más liberalizados:
				\4[] Turismo
				\4[] Servicios financieros
				\4[] Telecomunicaciones
				\4 Por modos de provisión
				\4[] Modo 2 el más liberalizado
				\4[] Modo 4 el menos liberalizado
				\4[] Modos 3 y 4 mayor peso en el comercio
				\4[] Modo 1 gana peso poco a poco por nuevas tecnologías
				\4 Acuerdo ambicioso pero incompleto
				\4[] Muchos temas pendientes por negociar
				\4[] $\to$ incertidumbre por Ronda de Doha
				\4 Acuerdo muy flexible
				\4 Incentiva el free-riding
				\4[] $\to$ dificulta liberalización
				\4 Complejidad alta del GATS
				\4[$\to$] Negociaciones muy complejas para PEDs
				\4 Liberalización de servicios es más compleja
				\4[] Implica cambiar legislación
				\4[] Barreras no arancelarias siempre más complejas
				\4 Moneda de cambio en negociaciones agrícolas
			\3 Perspectivas futuras
				\4 Desbloqueo de negociación agrícola
				\4 Grado de liberalización muy diverso
				\4[] Dada libertad de elección de EEMM
				\4[] Consolidación o liberalización adicional
				\4 Proliferación de bilateralismo y plurilateralismo
				\4 TISA -- Trade in Services Agreement
	\1 \marcar{TRIPS} \textit{Acuerdo Sobre Derechos de Propiedad Intelectual Relacionados con el Comercio (ADPIC)}
		\2 Idea clave
			\3 Importancia del TRIPS
				\4 Primer acuerdo multilateral sobre PI
				\4 Importancia de sectores generadores de PI
				\4[] $\to$ Cada vez mayores en VA y empleo
				\4[] $\to$ 39\% del PIB de UE en sectores intensivos en PI
				\4 Incentivo a la innovación
				\4[] $\to$ Generación de PI implica grandes costes
				\4[] $\to$ Necesario proteger monopolios temporales
				\4[] $\to$ Permite expansión internacional
				\4[] $\to$ Realización de economías de escala
			\3 Objetivos
				\4 Fijar normas mínimas de protección de PI
				\4 Establecer mecanismo de solución de diferencias
			\3 Intereses
				\4 Países desarrollados:
				\4[] Interés ofensivo
				\4[] Principales productores de PI
				\4[] $\to$ Pierden ingresos y competitividad si no respeto
				\4 Países en desarrollo:
				\4[] Interés defensivo
				\4[] Reduce competitividad de PEDs
				\4[] No tienen ventajas comparativas en PI
		\2 Antecedentes
			\3 Acuerdos pre-TRIPS
				\4 Convenciones de París y Berna en 1880s
				\4 WIPO - World Intellectual Property Organization en 1960
			\3 Países desarrollados
				\4 Sin diferencias significativas pre-TRIPS
				\4 Regímenes PI similares a TRIPS o más
			\3 Países en desarrollo
				\4 Protección muy débil de PI
				\4 Obstáculo a IDE extranjera
				\4 Obstáculo a apertura comercial hacia PEDs
		\2 Estructura
			\3 Principios básicos y obligaciones generales
				\4 No discriminación
				\4[] $\to$ Trato Nacional
				\4[] $\to$ Nación Más Favorecida
				\4[] Mismos términos que GATT y GATS
				\4 Protección equilibrada
				\4[] Equilibrio entre beneficio de incentivo a innovación
				\4[] y costes a corto plazo para los que no tienen acceso
				\4[] $\to$ Periodo de protección equilibrado
				\4[] $\to$ Grado de protección especial para PEDs
				\4[] $\to$ Desarrollado mediante normas mínimas de protección
				\4 Restricción de prácticas anticompetitivas
				\4[] Evitar licencias en términos abusivos
				\4 Medidas nacionales para asegurar aplicación
				\4[] \textit{Ex-ante}:
				\4[] $\to$ Posible exigir derechos ante tribunal
				\4[] $\to$ Vías de reclamación accesibles
				\4[] $\to$ Normativa efectiva que proteja
				\4[] \textit{Ex-post}:
				\4[] $\to$ Sanciones que realmente desincentiven
				\4[] Procedimientos rápidos a coste razonable
				\4[] Evitar entrada en circuito comercial
				\4[] Destruir mercancía
				\4 Periodos transitorios
				\4[] 1 año para desarrollados
				\4[] 5 años para PEDs
				\4[] 11 años para PMA
				\4[] Nuevas adhesiones: sin periodo transitorio
			\3 Normas mínimas de protección para 7 tipos de DPI\footnote{Se trata de unos estándares mínimos, de manera que un país puede facultativamente aplicar reglas más estrictas.}
				\4 \underline{Patentes}
				\4[] mínimo de 20 años
				\4[] Nueva y con aplicación industrial
				\4[] Excepciones:
				\4[] $\to$ Contrarias a orden público, moral, salud, vida, medio ambiente
				\4[] $\to$ Métodos de diagnóstico terapéutico y quirúrgico
				\4[] $\to$ Plantas y animales (sí microorganismos)
				\4[] Licencias obligatorias de patentes
				\4[] Obligación de divulgar información técnica
				\4 \underline{Derechos de autor}
				\4[] Mínimo de 50 años
				\4[] Derechos de reproducción, interpretación, grabación, cine
				\4[] Radio, traducción y adaptación
				\4 \underline{Marcas}
				\4[] Primer registro: duración mínima de 7 años.
				\4[] Derecho exclusivo de uso
				\4[] Puede anularse tras 3 años sin uso
				\4[] Licencias permitidas, pero sin licencia obligatoria
				\4[] Posible transferir empresas sin transferir marcas
				\4 \underline{Diseños industriales}
				\4[] 10 años de protección como mínimo
				\4[] Nuevos u originales
				\4[] Especial protección en productos de consumo
				\4[] Titular tiene derecho exclusivo de uso
				\4 \underline{Esquemas de trazado de circuitos integrados}\footnote{También denominados topografías.}
				\4[] Protección mínima de 10 años
				\4 \underline{Información no divulgada}
				\4[] Secretos comerciales protegidos
				\4[] $\to$ Si se tomaron las precauciones adecuadas
				\4 \underline{Indicaciones geográficas}
				\4[] Prohibición indicaciones falsas
				\4[] $\to$ si inducen a error o compiten deslealmente
				\4[] Protección aumentada en licores
			\3 Resolución de disputas
				\4 Mecanismo general de resolución de OMC
				\4 Mismo procedimiento que mercancías
		\2 Valoración
			\3 Situación actual
				\4 Patentes sobre medicamentos
				\4[] Importante Declaración de Doha 2001
				\4[] Acepta facultad de proteger salud pública
				\4[] $\to$ conceder licencias obligatorias de medicamentos
				\4[] $\to$ para mercado interno
				\4[] PMAs exentos de patentar medicamentos hasta 2016
				\4[] $\to$ Extendida en 2015 hasta 2033\footnote{\url{https://www.wto.org/english/news_e/news15_e/trip_06nov15_e.htm}}
				\4[] PMAs no productores pueden importar genéricos
				\4 Protección de Indicaciones Geográficas
				\4[] Objetivos de las negociaciones:
				\4[] $\to$ Creación registro multilateral de vinos
				\4[] $\to$ Extensión de protección más allá de bebidas
				\4[] Sin acuerdo
				\4[] Países con tradición productora $\to$ ofensivos
				\4 Biodiversidad
				\4[] Obligación de divulgar material genético
				\4[] O conocimientos tradicionales utilizados
				\4[] Convenio de Diversidad Biológica (CBD)
				\4[] $\to$ Ratificado por UE pero no por EEUU
				\4[] $\to$ W-52 pide incluirlo en TRIPS
			\3 Países desarrollados
				\4 Relativamente poco efecto
				\4[$\to$] Niveles elevados de protección antes
			\3 Países en desarrollo
				\4 Difícil implementación en países muy pobres
				\4 Puede contribuir a mantener dependencia importaciones
				\4 En general, han mejorado protección a PI
				\4 Países que exportan mucho a países que apoyaron TRIPS
				\4[] $\to$ Han aumentado sus niveles de protección
				\4[] $\to$ Amenaza de medidas de retorsión
			\3 Valoración
				\4 Fuertes controversias
				\4[] Interés ofensivo muy claro de desarrollados
				\4[] $\to$ Principales generadores de PI
				\4[] Interés defensivo de PEDs
				\4[] $\to$ Desventaja comparativa en creación de PI
				\4[] Cortos periodos de transición
				\4[] $\to$ Permiten afianzar beneficios de desarrollados
			\3 Perspectivas
				\4 Acuerdo de mínimos
				\4 Grades discrepancias
				\4 Moneda de cambio con temas agrícolas
				\4[$\to$] Muy difícil avanzar en marco TRIPS
				\4 Se avanza plurilateralmente
				\4[$\to$] ACTA - Anticounterfeiting Trade Agreement 2006
				\4[] Complemento de TRIPS
				\4[] Controversias temas de PI en internet
				\4[] Parlamento Europeo rechazó en 2012
				\4[$\to$] Bilateral
				\4[] UE incluye capítulos sobre PI en sus acuerdos bilaterales
				\4[] TPP y TTIP incluyen capítulos pero están bloqueados
	\1 \marcar{Otros acuerdos}
		\2 Multilaterales
			\3 Medio ambiente y comercio
				\4 Exenciones en GATS y GATT
				\4 Medidas sanitarias y fitosanitarias (SPS)
				\4 AGricultura: exenciones en ámbito de medio ambiente
				\4 ADPIC: posible rechazar patentes perjudiciales para la vida
			\3 Comercio electrónico
				\4 Discusiones desde Ginebra 2011
				\4 PEDs poco interesados
				\4 EEUU, EU y Japón desarrollando su propia regulación
			\3 Transferencia de tecnología
				\4 Fuerte relación con TRIPS
				\4 Por el momento, mismo marco salvo excepciones
				\4 Especialmente contencioso con China
				\4 Empresas extranjeras obligadas a transferir tecnología
				\4[] $\to$ Si quieren invertir en China
				\4 UE ha abierto contencioso en OMC
				\4[] Respecto a transferencias forzosas en China
			\3 Comercio y desarrollo
				\4 Objetivo de Doha 2001
				\4[] Trato preferencial a PEDs
				\4 Bali 2013:
				\4[] Mecanismo de Vigilancia del Trato Especial y Diferenciado
		\2 Plurilaterales
			\3 TISA - Trade in Services Agreement
				\4 23 miembros
				\4[] UE
				\4[] no todos desarrollados
				\4[] Miembros suponen 70\% comercio mundial de servicios
				\4 Objetivos
				\4[] Liberalización de gran escala
				\4[] Concretar obligaciones generales
				\4[] Definir obligaciones sectores específicos
				\4[] $\to$ telecomunicaciones
				\4[] $\to$ comercio electrónico
				\4[] $\to$ informática..
				\4 En negociaciones
				\4[] Abierto a nuevos miembros
				\4[] Multilateralizable
			\3 EGA
				\4 Environmental Goods Agreement
				\4 Objetivo
				\4[] Eliminar aranceles
				\4[] barreras no arancelarias
				\4[] $\to$ Bienes y servicios medioambientales
				\4 Negociaciones lentas
				\4 Posible multilateralización
				\4[] Miedo a free-riding vía NMF
				\4[] $\to$ Desincentiva concesiones
			\3 GPA Acuerdo de Contratación Pública
				\4 Objetivos
				\4[] Apertura de mercados de contratación pública
				\4[] Mejora de transparencia y competencia
				\4 Antecedentes
				\4[] Ronda de Tokio 1979 - Código del Sector Público
				\4[] ACP 1994 incorporado junto a Ronda Uruguay
				\4[] Reforma en 2012
				\4[] $\to$ entra en vigor en 2014
				\4 Contenido
				\4[] Transparencia y trato nacional
				\4[] No se aplican a todas las actividades
				\4[] Listas de cobertura de las partes
				\4[] $\to$ Determinar umbrales de aplicación del acuerdo
				\4[] Administrado por Comité de Contratación Pública
				\4[] Resolución de disputas
				\4[] $\to$ Recursos nacionales y mecanismo de solución de diferencias
				\4 Valoración
				\4[] Ejerce presión sobre eficiencia del sector público
				\4[] Contratación pública representa 10-15\% del PIB según OMC
				\4[] 17 partes incluida UE
				\4[] Dificulta proteccionismo en contratación pública
				\4[] Necesidad de adaptar legislación nacional
				\4[] Tendencias proteccionistas en ciertos países
				\4[] Mucho por avanzar
	\1[] \marcar{Conclusión}
		\2 Recapitulación
			\3 GATS
			\3 TRIPS
			\3 Otros acuerdos
				\4 Multilaterales
				\4 Plurilaterales
		\2 Idea final
			\3 Ofensivos de España
				\4 Servicios
				\4 Propiedad intelectual - indicaciones geográficas
				\4 Acceso a los mercados para los productos no agrícolas
				\4 Liberalización de bienes y servicios medioambientales
				\4 Comercio e inversiones
				\4 Transparencia en las compras públicas
				\4 Restricciones y tasas a la exportación
			\3 Defensivos de España
				\4 Normas OMC - Subvenciones a la pesca
				\4 Algodón
				\4 Frutas y hortalizas
			\3 Postura negociadora de la UE
				\4 Se extiende a la postura española
				\4 Priorizar acuerdos multilaterales
				\4[] Si no es posible, acuerdos multilateralizables
				\4 Acceso al mercado ya no es prioridad
				\4[] Ahora, reducir barreras no arancelarias
				\4[] Evitar que normativas nacionales sean obstáculo
			\3 Crisis de la OMC
				\4 Determinará evolución futura
				\4 Turbulencias graves en sistema multilateral
				\4[] $\to$ Distraen atención de liberalización
\end{esquemal}
























\conceptos

\textbf{Requisitos de las zonas de libre comercio y uniones aduaneras para estar excluidas del principio de la Nación Más Favorecida}

Para que un FTA o una UA no impliquen extensión del trato favorable al resto de miembros del GATS, deben cumplir dos condiciones:
\begin{itemize}
	\item \textbf{Cobertura amplia:} número suficiente de sectores. La justificación es que el acuerdo debe implicar unos efectos positivos del aumento de exportaciones entre miembros que compensen los efectos negativos de la desviación de comercio.
	\item \textbf{No puede ser más restrictivo frente a terceros}. De esta forma, el acuerdo limita la desviación de comercio.
\end{itemize}

\preguntas

\seccion{Test 2019}

\textbf{41.} En relación al comercio de Propiedad Intelectual (PI), señale la afirmación \textbf{\underline{falsa}}:

\begin{itemize}
	\item[a] En el trascurso de las negociaciones durante la Ronda de Uruguay (1986-1994) de la Organización Mundial de Comercio (OMC), se firmó el acuerdo sobre los Aspectos de los Derechos de Propiedad Intelectual relacionados con el Comercio (ADPIC).
	\item[b] En el acuerdo ADPIC no se contempla el principio de trato de nación más favorecida (NMF).
	\item[c] Para las consultas y la solución de diferencias en el marco del acuerdo ADPIC se aplican las disposiciones previstas en el sistema de solución de diferencias de la OMC.
	\item[d] El acuerdo ADPIC es parte integrante de la OMC.
\end{itemize}


\seccion{Test 2016}

\textbf{41.} El Acuerdo General de la OMC sobre el Comercio de Servicios es:

\begin{itemize}
	\item[a] Un acuerdo multilateral independiente sobre el comercio de servicios que los Miembros de la OMC pueden elegir para firmar.
	\item[b] Una parte del Acuerdo de Marrakech obligatoria para todos los miembros de la OMC.
	\item[c] Una parte del Acuerdo de Marrakech, pero voluntaria para todos los Miembros de la OMC.
	\item[d] Un acuerdo multilateral independiente sobre el comercio de servicios obligatorio para todos los miembros de la OMC.
\end{itemize}

\seccion{Test 2014}

\textbf{37.} El Fondo para la aplicación de Normas y el Fomento del comercio de la Organización Mundial del Comercio (OMC):

\begin{itemize}
	\item[a] Es una iniciativa mundial que se creó para los países desarrollados y los países en desarrollo a fin de reforzar su capacidad para aplicar las normas, directrices y recomendaciones sanitarias y fitosanitarias internacionales (MSF), a fin de mejorar su situación en lo referente a la salud de las personas y los animales y la conservación de las plantas, y poder así acceder a los mercados y mantenerse en ellos.
	\item[b] Tiene como mandato proporcionar ayuda y fondos a los países en desarrollo para la formulación y la aplicación de los proyectos orientados al cumplimiento de los requisitos sanitarios y fitosanitarios internacionales.
	\item[c] Se rige por las disposiciones administrativas establecidas por la Organización Mundial del Comercio (OMC), teniendo personalidad jurídica propia.
	\item[d] Cuenta entre sus miembros con un total de siete expertos de países desarrollados y seis experots de países en desarrollo, a saber, dos de las Américas, dos de África y dos de Asia (excluido el Pacífico).
\end{itemize}


\notas

\textbf{2019.} \textbf{41.} B

\textbf{2016.} \textbf{41:} B

\textbf{2014.} \textbf{37:} B

Este tema se puede concluir con los intereses de España o con la conclusión que se incluye, en función del tiempo trascurrido. Los intereses de España no deberían llevar más de dos minutos, incluyendo una brevísima descripción de 'ofensivo' y 'defensivo' en términos de negociaciones comerciales.

\bibliografia

Délégation permanente de la France auprès de l’Organisation Mondiale du Commerce. \textit{Brèves de l’OMC – Décembre 2017} (2017) En carpeta del tema. \url{https://www.tresor.economie.gouv.fr/Articles/2017/12/15/breves-de-l-omc-de-decembre-2017-edition-speciale-buenos-aires-don-t-cry-for-me-argentina}

Cardwell, R.; Ghazalian, P. L; \textit{The Effects of the TRIPS Agreement on International Protection of Intellectual Property Rights} The International Trade Journal (2012) -- En carpeta del tema



\end{document}
