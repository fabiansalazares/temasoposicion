\documentclass{nuevotema}

\tema{3B-37}
\titulo{La Unión Europea: Instituciones y Orden jurídico. Los tratados de la Unión Europea.}

\begin{document}

\ideaclave

COMPLETAR CON FACTSHEET EN CARPETA DEL TEMA SOBRE MECANISMO DE COOPERACIÓN REFORZADA

La Unión Europea es la institución supranacional que ha logrado hasta la fecha y en la historia contemporánea, un mayor grado de integración política y económica. El germen de la Unión Europea es una compleja interacción de factores a mediados de los años 40, con una Europa devastada por la guerra, una Unión Soviética deseosa de ampliar su esfera de influencia hacia Europa Occidental, y unos Estados Unidos que temían tal expansión soviética pero que al mismo tiempo eran reticentes a mantener la presencia física y el gasto económico que suponía la tutela directa o al menos cercana de las grandes potencias europeas. A estos factores se añadía la existencia de un estado de opinión favorable a una integración que pudiese frenar en el futuro las pulsiones expansionistas o nacionalistas de los estados centrales que se habían enfrentado en las guerras anteriores. La exposición recorre la evolución de la UE desde sus orígenes como tres comunidades de naciones diferenciadas y con diferentes objetivos pero mismos integrantes, hasta el ente supranacional con personalidad jurídica propia e instituciones singularizadas y 28 estados miembros de la actualidad.

La exposición comienza recorriendo la secuencia de tratados que han configurado la Unión Europea, señalando los avances que cada uno de ellos introdujo, así como aquellos que permitieron la adhesión de nuevos estados miembros. Posteriormente, se examina el ordenamiento jurídico que la Unión Europea ha construido, atendiendo a los principios fundamentales que lo fundamentan, a los instrumentos de derecho primario y derivado, a la prelación jerárquica entre fuentes y al procedimiento legislativo por el que se aprueban en el seno de la Unión los actos jurídicos.

Por último, y como elemento central de la exposición, se examinan las siete principales instituciones de la Unión Europea consagradas en el Tratado de Lisboa: la Comisión, el Parlamento, el Consejo de la UE, el Consejo Europeo, el Tribunal de Justicia de la Unión Europea, el Tribunal de Cuentas y el Banco Central Europeo, atendiendo en los sietes casos a su función, antecedentes, composición, actuaciones y valoración de su actividad hasta el momento. Se mencionan además brevemente cuatro órganos adicionales de gran relevancia: el Comité Económico y Social, el Comité de las Regiones, el Eurogrupo y el Servicio Europeo de Acción Exterior. 

\seccion{Preguntas clave}
\begin{itemize}
    \item ¿Qué es la UE?
    \item ¿Cómo se organiza internamente?
    \item ¿Cuál es su funcionamiento?
    \item ¿En qué normas y principios jurídicos se basa?
    \item ¿Cómo funcionan sus instituciones?
\end{itemize}

\esquemacorto

\begin{esquema}[enumerate]
	\1[] \marcar{Introducción}
		\2 Contextualización
			\3 Unión Europea
			\3 Competencias de la UE
		\2 Objeto
			\3 Qué es la UE?
			\3 Cómo se organiza internamente?
			\3 Qué instituciones la componen?
			\3 Qué tratados la articulan?
			\3 Qué trayectoria histórica?
		\2 Estructura
			\3 Evolución histórica
			\3 Ordenamiento jurídico
			\3 Instituciones
	\1 \marcar{Evolución histórica}
		\2 Post-guerra
			\3 Amenaza soviética
			\3 Influencia de EEUU
			\3 Instituciones predecesoras
		\2 Tratado de la CECA (París, 1951)
			\3 Objetivo
			\3 Actuaciones
		\2 Tratado de la CEE (Roma, 1957)
			\3 Objetivo
			\3 Actuaciones
			\3 Consecuencias
		\2 Tratado Euroatom (Roma, 1957)
			\3 Objetivo
			\3 Actuaciones
			\3 Estatus actual
		\2 Tratado de Fusión (Bruselas, 1965)
			\3 Objetivo
			\3 Actuaciones
		\2 Tratado de Luxemburgo (Luxemburgo, 1970)
			\3 Objetivo
			\3 Actuaciones
		\2 Adhesión de 1973
			\3 Idea clave
			\3 Países
			\3 Contexto
			\3 Consecuencias
		\2 Adhesión de 1981: GRE
			\3 Idea clave
			\3 Países
		\2 Adhesión de 1986: ESP, POR
			\3 Idea clave
			\3 Países
			\3 Contexto
		\2 Acuerdo de Fontainebleu (1984)
			\3 Objetivos
			\3 Contexto
			\3 Actuaciones
		\2 Acta Única Europea (1985) $\to$ (1987)
			\3 Objetivos
			\3 Contexto
			\3 Actuaciones
		\2 Tratado de la Unión Europea (Maastricht, 1992) $\to$ (1993)
			\3 Entrada en vigor
			\3 Contexto
			\3 Actuaciones
		\2 Adhesión de 1995
			\3 Idea clave
			\3 Países
		\2 Tratado de Amsterdam (1997)
			\3 Idea clave
			\3 Actuaciones
		\2 Tratado de Niza (2001)
			\3 Contexto
			\3 Actuaciones
		\2 Adhesión de 2003
			\3 Idea clave
			\3 Países
		\2 Constitución Europea (2005)
			\3 Contexto
			\3 Objetivos
		\2 Adhesión de 2007
			\3 Idea clave
			\3 Países
		\2 Tratado de Lisboa (2007)
			\3 Contexto
			\3 Actuaciones
		\2 Adhesión de 2013
			\3 Idea clave
			\3 Países
		\2 Brexit (2016) -- (2020)
			\3 Objetivo
			\3 Contexto
			\3 Acuerdo de salida de noviembre de 2019
			\3 Acuerdo para la relación posterior
			\3 Consecuencias
	\1 \marcar{Ordenamiento jurídico}
		\2 Principios del Tratado de la Unión Europea
			\3 Carta de las Naciones Unidas
			\3 Convención Europea de los Derechos Humanos (1950)
			\3 Carta de los Derechos Fundamentales de la UE
			\3 Atribución
			\3 Subsidiariedad
			\3 Proporcionalidad
		\2 Derecho primario de la Unión
			\3 Tratados fundacionales
			\3 Tratados modificativos
			\3 Tratados de adhesión
		\2 Derecho derivado
			\3 Reglamentos
			\3 Directivas
			\3 Decisiones
			\3 Recomendaciones y dictámenes
		\2 Jerarquía de fuentes
			\3[1] Derecho primario de la UE
			\3[2] Actos legislativos
			\3[3] Actos delegados
			\3[4] Actos de ejecución
		\2 Procedimiento legislativo
			\3 Ordinario
			\3 Aprobación del presupuesto anual
			\3 Cooperación reforzada
			\3 Especiales
	\1 \marcar{Instituciones}
		\2 Comisión Europea
			\3 Función
			\3 Antecedentes
			\3 Organización
			\3 Actuaciones
			\3 Valoración
		\2 Parlamento Europeo
			\3 Función
			\3 Antecedentes
			\3 Organización
			\3 Actuaciones
			\3 Valoración
		\2 Consejo de la Unión Europea
			\3 Función
			\3 Antecedentes
			\3 Organización
			\3 Actuaciones
			\3 Valoración
		\2 Consejo Europeo
			\3 Función
			\3 Antecedentes
			\3 Organización
			\3 Actuaciones
			\3 Valoración
		\2 Tribunal de Justicia de la Unión Europea
			\3 Función
			\3 Antecedentes
			\3 Organización
			\3 Actuaciones
			\3 Valoración
		\2 Tribunal de Cuentas
			\3 Función
			\3 Antecedentes
			\3 Organización
			\3 Actuaciones
			\3 Valoración
		\2 Banco Central Europeo
			\3 Función
			\3 Antecedentes
			\3 Organización
			\3 Actuaciones
			\3 Valoración
		\2 Otras
			\3 Comité Económico y Social Europeo (CESE)
			\3 Comité de las Regiones
			\3 Eurogrupo
			\3 Servicio Europeo de Acción Exterior
	\1[] \marcar{Conclusiones}
		\2 Recapitulación
			\3 Evolución histórica
			\3 Ordenamiento jurídico
		\2 Idea final
			\3 Sistema complejo
			\3 Sistema en evolución

\end{esquema}

\esquemalargo

\begin{esquemal}
	\1[] \marcar{Introducción}
		\2 Contextualización
			\3 Unión Europea
				\4 Institución supranacional ad-hoc
				\4[] Diferente de otras instituciones internacionales
				\4[] Medio camino entre:
				\4[] $\to$ Federación
				\4[] $\to$ Confederación
				\4[] $\to$ Alianza de estados-nación
				\4[] Creación de un orden jurídico propio
				\4 Origen de la UE
				\4[] Siglos con muy frecuentes guerras
				\4[] $\to$ Equilibrio de fuerzas para frenar conflicto
				\4[] $\to$ Aparición de imperios
				\4[] $\then$ Soluciones inestables
				\4[] $\then$ Tendencia a desintegración o guerra
				\4[] Tras dos guerras mundiales en tres décadas
				\4[] $\to$ Cientos de millones de muertos
				\4[] $\to$ Destrucción económica
				\4[] Marco de integración entre naciones y pueblos
				\4[] $\to$ Evitar nuevas guerras
				\4[] $\to$ Maximizar prosperidad económica
				\4[] $\to$ Frenar expansión soviética
				\4 Objetivos de la UE
				\4[] TUE -- Tratado de la Unión Europea
				\4[] $\to$ Primera versión: Maastricht 91 $\to$ 93
				\4[] $\to$ Última reforma: Lisboa 2007 $\to$ 2009
				\4[] Artículo 3
				\4[] $\to$ Promover la paz y el bienestar
				\4[] $\to$ Área de seguridad, paz y justicia s/ fronteras internas
				\4[] $\to$ Mercado interior
				\4[] $\to$ Crecimiento económico y estabilidad de precios
				\4[] $\to$ Economía social de mercado
				\4[] $\to$ Pleno empleo
				\4[] $\to$ Protección del medio ambiente
				\4[] $\to$ Diversidad cultural y lingüistica
				\4[] $\to$ Unión Económica y Monetaria con €
				\4[] $\to$ Promoción de valores europeos
				\4[$\to$] Objetivos de la UE
				\4[] Paz y bienestar a pueblos de Europa
			\3 Competencias de la UE
				\4 Tratado de la Unión Europea
				\4[] Atribución
				\4[] $\to$ Sólo las que estén atribuidas a la UE
				\4[] Subsidiariedad
				\4[] $\to$ Si no puede hacerse mejor por EEMM y regiones
				\4[] Proporcionalidad
				\4[] $\to$ Sólo en la medida de lo necesario para objetivos
				\4 Exclusivas
				\4[] i. Política comercial común
				\4[] ii. Política monetaria de la UEM
				\4[] iii. Unión Aduanera
				\4[] iv. Competencia para el mercado interior
				\4[] v. Conservación recursos biológicos en PPC
				\4 Compartidas
				\4[] i. Mercado interior
				\4[] ii. Política social
				\4[] iii. Cohesión económica, social y territorial
				\4[] iv. Agricultura y pesca \footnote{Salvo en lo relativo a la conservación de recursos biológicos marinos, que se trata de una competencia exclusiva de la UE}
				\4[] v. Medio ambiente
				\4[] vi. Protección del consumidor
				\4[] vii. Transporte
				\4[] viii. Redes Trans-Europeas
				\4[] ix. Energía
				\4[] x. Área de libertad, seguridad y justicia
				\4[] xi. Salud pública común en lo definido en TFUE
				\4 De apoyo
				\4[] Protección y mejora de la salud humana
				\4[] Industria
				\4[] Cultura
				\4[] Turismo
				\4[] Educación, formación profesional y juventud
				\4[] Protección civil
				\4[] Cooperación administrativa
				\4 Coordinación de políticas y competencias
				\4[] Política económica
				\4[] Políticas de empleo
				\4[] Política social
		\2 Objeto
			\3 Qué es la UE?
			\3 Cómo se organiza internamente?
			\3 Qué instituciones la componen?
			\3 Qué tratados la articulan?
			\3 Qué trayectoria histórica?
		\2 Estructura
			\3 Evolución histórica
			\3 Ordenamiento jurídico
			\3 Instituciones
	\1 \marcar{Evolución histórica}
		\2 Post-guerra
			\3 Amenaza soviética
				\4 Europa del Este bajo control soviético
				\4 Vacío de poder en Europa occidental
			\3 Influencia de EEUU
				\4 Presión para evitar caída bajo URSS
				\4 Plan Marshall
				\4 Intereses compañías americanas:
				\4[] acceso mercado europeo
			\3 Instituciones predecesoras
				\4 Consejo de Europa
				\4 Comisión Económica Europea
				\4[] Marco de Naciones Unidas
				\4 OECE
				\4[] Organización para la Cooperación Económica Europea
				\4 Cooperación militar creciente
				\4[] Intento de creación de Comunidad Europea de Defensa
				\4[] Fracasa por oposición de Francia en 1954
				\4[] Considerado requisito para desocupar Alemania
				\4 Unión Europea Occidental -- Western European Union
				\4[] Menos ambiciosa que CEDefensa
				\4[] Acuerdo para cooperación militar
				\4[] $\to$ FRA, RU, ITA, LUX, NED, BEL
				\4[] $\to$ ALE, GRE, POR, ESP
		\2 Tratado de la CECA (París, 1951)
			\3 Objetivo
				\4 Integración económica
				\4[] $\to$ Desincentivar futuros conflictos
				\4 Primer paso:
				\4[] $\to$ Integración industrias clave para guerra
				\4 Solución atractiva para FRA y GER
			\3 Actuaciones
				\4 Liberalización comercio carbón y acero
				\4 Prohibición discriminación por nación
				\4[] Principio de Nación Más Favorecida entre miembros
				\4 6 fundadores: FRA, GER, BEL, NED, LUX, ITA
		\2 Tratado de la CEE (Roma, 1957)
			\3 Objetivo
				\4 Mercado común general
				\4[] Mercancías
				\4[] Servicios
				\4[] Personas
				\4[] Capitales
				\4 Establecer unión aduanera general
				\4 Principio de unión cada vez más estrecha
			\3 Actuaciones
				\4 Unión aduanera
				\4 Periodo 12 años de adaptación:
				\4[] Arancel común, eliminación barreras
				\4[] Política comercio común...
				\4 Creación de instituciones comunitarias
				\4[] Consejo de Ministros
				\4[] Comisión de la CEE
				\4[] Asamblea parlamentaria
				\4[] Tribunal de Justicia
				\4[] Comité Económico y Social
				\4 Regular políticas comunes
				\4[] Política agrícola común
				\4[] Política comercial común
				\4[] Política común de transportes
				\4[] Prevé creación de otras políticas comunes
			\3 Consecuencias
				\4 Tratado más importante hasta Maastricht
				\4 Reformado en múltiples ocasiones
				\4 Motor principal de integración económica
				\4[] Base para aumentar integración en 4 libertades
				\4 Convertido en TFUE actual
				\4[] Plenamente en vigor
		\2 Tratado Euroatom (Roma, 1957)
			\3 Objetivo
				\4 Creación de un mercado común de energía atómica
				\4[] Materias primas necesarias para energía atómica
				\4 Venta de excedentes a países fuera de UE
				\4 Garantizar seguridad de reservas
				\4 Mitigar déficit energético europeo
				\4 Peso de energía atómica en producción eléctrica
				\4[] Alrededor del 30\% total
			\3 Actuaciones
				\4 Promoción y regulación común energía atómica
				\4 Desarrollo de energía atómica en EEMM
				\4 Puesta en común de tecnología e investigación
				\4 Investigación sobre fusión nuclear
				\4 Regulación del uso civil
			\3 Estatus actual
				\4 Existe aún separado de la UE
				\4 Plenamente en vigor
				\4 Pocas reformas
				\4 Permite enfoque comunitario armonizado en ENuclear
		\2 Tratado de Fusión (Bruselas, 1965)
			\3 Objetivo
				\4 Introducir idea de Comunidades Europeas
				\4 Simplificar esquema institucional
			\3 Actuaciones
				\4 Reducir instituciones duplicadas
				\4 Comunidades Europeas ya compartían algunas
				\4[] Tribunal de Justicia
				\4[] Asamblea parlamentaria
				\4 Fusión de Consejos de ministros
				\4[] Previamente, uno por Comunidad
				\4 Fusión de Comisión Europea
				\4 Comunidades permanecen separadas
				\4[] CECA, CEE, Euratom
		\2 Tratado de Luxemburgo (Luxemburgo, 1970)
			\3 Objetivo
				\4 Establecer sistema de financiación
				\4 Proveer a UE con recursos financieros propios
				\4 Acabar con sistema de contribuciones nacionales
				\4 Dotar a PE de compentencias presupuestarias
			\3 Actuaciones
				\4 Tratado menor
				\4 Primer tratado presupuestario
				\4 Recursos propios de la UE
				\4 Antes de Tratado:
				\4[] UE financiada con:
				\4[] $\to$ contribuciones de miembros
				\4[] $\to$ tasas de aduanas desde 1962\footnote{La PAC se introduce en ese año pero las tasas agrícolas no eran suficientes para financiar la comunidad.}
				\4 Tratado instaura recursos propios de la Unión:\footnote{Ver \url{https://www.europarl.europa.eu/news/es/headlines/eu-affairs/20130723STO17551/pasado-presente-y-futuro-del-sistema-europeo-de-recursos-propios}}
				\4[] $\to$ Sustituye sistema de tasas agrícolas
				\4[] Aduanas
				\4[] IVA
				\4[] Exacciones agrícolas
				\4 Parlamento Europeo adquiere competencias presupuestarias
		\2 Adhesión de 1973
			\3 Idea clave
				\4 Primera expansión de la UE
				\4 Países que habían fundado la EFTA en 1959
				\4[] $\to$ Y que había entrado en vigor en 1960
				\4 Firmado en 1971
				\4 Entrada en vigor en 1973
			\3 Países
				\4 Reino Unido
				\4 Irlanda
				\4 Dinamarca
			\3 Contexto
				\4 RU: intentos previos de acceder a CECA
				\4[] Veto francés en los años 60
				\4 Creación de EFTA en años 60
				\4[] Países norte y oeste de europa no-CEuropeas
				\4[] $\to$ DIN, IRL, UK, AUS, POR, SWE, SWI
			\3 Consecuencias
				\4 Controversia en RU sobre términos de adhesión
				\4[] Origen histórico de controversia RU-UE
				\4[] $\to$ Debate l/p en política británica
				\4 RU: referendum de salida en 1975
				\4[] Gana permanencia en UE
		\2 Adhesión de 1981: GRE
			\3 Idea clave
				\4 Segunda expansión
				\4 Primer país que sale de dictadura
				\4 Primera fase de expansión mediterránea de la UE
			\3 Países
				\4 Grecia
		\2 Adhesión de 1986: ESP, POR\footnote{Ver \href{https://www.cvce.eu/en/education/unit-content/-/unit/02bb76df-d066-4c08-a58a-d4686a3e68ff/d4c04734-67dc-4e67-8168-1f996b10672f}{CVCE.eu sobre tercera expansión de UE.}}
			\3 Idea clave
				\4 Tercera expansión
				\4 Fuerte impacto sobre PAC, PPC y cohesión
				\4 Segunda fase de expansión mediterránea
				\4 Segunda entrada de ex-dictaduras
			\3 Países
				\4 España
				\4 Portugal
			\3 Contexto
				\4 Reservas francesas iniciales
				\4 España proyecta accesión desde 1959
				\4[] Estabilización, liberalización, apertura
				\4 Acuerdo Preferencial de 1970
				\4[] Notablemente favorable a intereses españoles
				\4[] Desfigurado por accesión de 1971-1973
				\4 Solicitudes previas de adhesión de España
				\4[] En 1962 sin respuesta
				\4[] En 1977, respuesta favorable
				\4[] $\to$ Apertura de muy largas negociaciones
				\4 Italia y Francia temen competencia agrícola
				\4[] Especialmente respecto España
				\4[] Mercados ya saturados
				\4[] $\to$ PAC y otros
				\4[] Se temen flujos migratorios hacia el norte
		\2 Acuerdo de Fontainebleu (1984)
			\3 Objetivos
				\4 Mitigar problemas presupuestarios
				\4 Reducir tensiones entre miembros
				\4[] ALE-FRA por política de demanda y monetaria
				\4[] RU por contribuciones para PAC
			\3 Contexto
				\4 Ruptura entre Francia y Alemania
				\4[] Tras cambio de gobierno a Mitterrand en 1981
				\4[] $\to$ Énfasis en estimular demanda
				\4[] $\to$ Frente a estabilidad política de Alemania
				\4 Crisis abierta entre los 10 miembros
				\4 Gasto agrícola
				\4[] Enorme crecimiento en últimos años
				\4[] Fuerte presión sobre presupuesto comunitario
				\4 Reino Unido
				\4[] Rechaza enorme aumento del gasto agrícola
				\4[] Exige reducir contribuciones
			\3 Actuaciones
				\4 Recurso del IVA: aumento
				\4[] Aumento de \% de IVA para CEuropeas
				\4 Medidas de limitación del gasto agrícola
				\4[] Comienzan a adelantar reformas posteriores
				\4 Reconocido derecho a evitar contribuciones excesivas
				\4[] Da lugar en el futuro a numerosas correcciones
				\4 ``Cheque Británico''
				\4[] Mecanismo concebido como provisional
				\4[] Corrección de contribución británica
				\4[] 66\% de reducción de contribución IVA
				\4[] $\to$ Coste repartido entre resto de miembros
				\4[] $\to$ Cuota de ALE solo aumenta 2/3
				\4 Negociaciones ESP y POR deben concluir en 1984
				\4[] Impulso final a negociaciaciones
				\4[] Pesca contencioso con España
		\2 Acta Única Europea (1985) $\to$ (1987)\footnote{\href{https://www.cvce.eu/en/education/unit-content/-/unit/02bb76df-d066-4c08-a58a-d4686a3e68ff/c5c70c4a-898e-4eeb-96d2-94c7091dd293}{CVCE.eu sobre Acta Única}}
			\3 Objetivos
				\4 Recuperar dinamismo de economía europea
				\4 Completar objetivos iniciales de Roma 1957
				\4 Aumentar comercio intra-europeo
				\4 Aprovechamiento de EEscala de mercado europeo
			\3 Contexto
				\4 Crisis años 80
				\4[] Contracción monetaria en Estados Unidos
				\4[] Volatilidad cambiaria
				\4[] Crisis de deuda en Latinoamérica
				\4 Libro Blanco de Delors
				\4[] Propuesta de acuerdo para completar mercado único
				\4[] 310 medidas
				\4[] 1992 como objetivo
				\4[] $\to$ Completar mercado único para entonces
				\4[] $\then$ Creación de mayor mercado mundial
				\4[] Base de Acta Única
				\4 Tensiones Alemania Francia
				\4 Falta de dinamismo economías europeas
				\4[] Desempleo elevado
				\4[] Crecimiento del output cae relación USA, JAP
				\4 Adhesiones ESP, POR, GRE
				\4[] Reducen coherencia de área económica
				\4[] $\Rightarrow$ menor coherencia
				\4 Negociaciones muy complicadas
				\4 Rechazo inicial de Grecia y Dinamarca
				\4[] Dinamarca: parlamento rechaza en 1986
				\4[] $\to$ Pide reabrir negociaciones
				\4[] $\to$ Resto de EEMM rechaza
				\4[] $\then$ Referendum en Dinamarca
				\4[] $\then$ Sí a Acta Única gana referéndum
			\3 Actuaciones
				\4 Conclusión de la redacción final
				\4[] 1985
				\4 Ratificaciones sucesivas en 1986
				\4 Entrada en vigor
				\4[] Julio de 1987
				\4 ESP y POR invitadas a negociaciones
				\4 4 Libertades: bienes, servicios, capital, trabajo
				\4 Nuevas políticas:
				\4[] Medioambiente, salud, i+D, refuerzo cohesión,
				\4[] cooperación exterior, pol. monetaria, económica
				\4 Ampliación del voto por mayoría
				\4[] $\to$  QMV en asuntos del mercado único
				\4 Control del gasto de la PAC
				\4 Reformas institucionales
				\4[] Consejo Europeo: base legal explícita
				\4[] Parlamento Europeo: primera aparición formal
				\4 Eliminación de fronteras
				\4[] 300 medidas a adoptar
				\4[] $\to$ Plazo hasta 1992 para adoptar
				\4[] Eliminar controles a bienes
				\4[] Medidas de simplificación aduanera
				\4[] Armonización de extranjería
				\4[] $\to$ Derecho de asilo
				\4[] $\to$ Estatus de residencai de extranjeros
				\4[] Liberalización en algunos servicios
				\4 Cooperación política europea
				\4[] Formalización de prácticas informales
				\4[] Introduce concepto de seguridad a nivel eurpeo
				\4[] Mecanismos de cooperación a nivel político
				\4 Europa social
				\4[] Presión francesa
				\4[] Nuevas políticas para reducir desigualdades
				\4[] $\to$ Mercado único podía potencialmente ampliar
				\4 Estandarización
				\4[] Múltiples medidas
				\4[] Positiva pero también negativa
				\4 Investigación científica
				\4[] Programa multianual de cooperación
				\4[] $\to$ Germen de MFP
		\2 Tratado de la Unión Europea (Maastricht, 1992) $\to$ (1993)
			\3 Entrada en vigor
				\4 1993
			\3 Contexto
				\4 Caída del muro $\to$ Alemania reunificada
				\4 Caída Unión Soviética, pacto de Varsovia
				\4 Yugoslavia proceso de desintegración
				\4 Guerra del Golfo
			\3 Actuaciones
				\4 Preparar Unión Monetaria y Económica
				\4 Ciudadanía europea
				\4[] Sin definición expresa
				\4 Aumento competencias políticas
				\4 Creación de UE que englobe a comunidades europeas
				\4[] CEE, CECA, Euratom
				\4 Otros dos pilares: Política Exterior y de Defensa, Justicia e Interior
				\4 Tratado de la Unión Europea
		\2 Adhesión de 1995
			\3 Idea clave
				\4 Referéndum en 1991 en países EFTA
				\4[] ¿Unirse a la CEE y salir de EFTA?
				\4 Referendums posteriores de adhesión a UE
				\4[] SWE, FIN, AUS
				\4[] $\to$ Gana el SÍ a la adhesión
				\4[] NOR
				\4[] $\to$ Gana el NO
			\3 Países
				\4 Austria
				\4 Finlandia
				\4 Suecia
		\2 Tratado de Amsterdam (1997)
			\3 Idea clave
				\4 Entrada en vigor 1999
			\3 Actuaciones
				\4 Modificaciones TUE, TFUE
				\4[$\to$] Clarificaciones, correcciones
				\4 Comunitarización de Acuerdo Schengen
				\4 Inmigración para armonizar libre circulación en UE
				\4 Creación de figura del Alto Representante
				\4 Extiende ciudadanía europea
				\4[] Suplementaria, no sustituye a nacional
				\4 Avances políticas existentes
				\4 Energía, turismo
				\4 Procedimiento de Cooperación Reforzada
				\4 Consolidación principio de subsidiariedad
				\4 Objetivo de transparencia
		\2 Tratado de Niza (2001)
			\3 Contexto
				\4 Prevista ampliación al este
				\4 Negociación difícil
				\4 Aprobación difícil: rechazada en referendum por IRL
			\3 Actuaciones
				\4 Aumento áreas con QMV
				\4 Preparación ampliación:
				\4[] Limitación de número de comisarios
				\4 Carta de Derechos, no vinculante
		\2 Adhesión de 2003
			\3 Idea clave
				\4 Mayor adhesión hasta la fecha
				\4[] 10 países
				\4 Países de Europa del Este/antiguo bloque de Varsovia
			\3 Países
				\4 Estonia
				\4 Letonia
				\4 Lituania
				\4 Hungría
				\4 Polonia
				\4 República Checa
				\4 Eslovaquia
				\4 Eslovenia
				\4 Malta
				\4 Chipre
		\2 Constitución Europea (2005)
			\3 Contexto
				\4 Cumbre de Niza: futuro de la UE?
			\3 Objetivos
				\4 Preparar UE para Unión Política
				\4 Consolidar avances
				\4 Preparar para ampliaciones
				\4 Fracaso: rechazo FRA, NED
		\2 Adhesión de 2007
			\3 Idea clave
				\4 Países más pobres de Europa del Este
				\4 Última adhesión de peso
				\4 Controversia sobre efecto migratorio
			\3 Países
				\4 Bulgaria
				\4 Rumanía
		\2 Tratado de Lisboa (2007)
			\3 Contexto
				\4 Tras rechazo Constitución
			\3 Actuaciones
				\4 Firmado 2007
				\4 Ratificado 2008
				\4 Vigor 2009 tras rechazo y renegociación IRL, CZK
				\4 Simplifica y unifica TUE
				\4 Creación Presidente del Consejo Europeo
				\4 Crea Alto Representante PES
				\4 Abre puertas cooperación defensa
				\4[] Vía procedimiento de cooperación reforzado
				\4 Nuevos poderes Parlamento Europeo
				\4 Elimina veto nacional en:
				\4[] inmigración, asilo, transporte marítimo y aéreo
				\4 Mantiene veto en:
				\4[] impuestos, defensa, exteriores, financiación
				\4 Reforma sistema de voto en consejo
				\4 Art. 50: salida de la UE
				\4 Aumenta poderes Eurozona
				\4 Incorpora Carta de Derechos Fundamentales
				\4[] Introduce en acervo europeo como vinculante
				\4[] $\to$ Estrictamente, no incluido en TLisboa
				\4 Intento de restringir número miembros de CE
				\4 Cambio número parlamentarios por país:
				\4[] entre 6 y 96
				\4 Artículo 50 sobre abandono de la UE
				\4 BCE oficialmente institución de la UE
				\4[] Presidente nombrado por QMV por CEuropeo
				\4 Formalización del Eurogrupo
		\2 Adhesión de 2013
			\3 Idea clave
				\4 Última adhesión hasta la fecha
				\4 Perspectivas de adhesión de otros países balcánicos
				\4[] Albania
				\4[] Montenegro
				\4[] Serbia
				\4 Potenciales
				\4[] Bosnia y Herzegovina
				\4[] Kosovo
				\4[] Turquía
			\3 Países
				\4 Croacia
		\2 Brexit (2016) -- (2020)\footnote{\href{https://ec.europa.eu/commission/sites/beta-political/files/slides_the_wa_explained.pdf}{European Commission slides on the Withdrawal Agreement (2020).}}
			\3 Objetivo
				\4 Implementar decisión de referéndum británico
				\4 Definir relación futura UK-UE
				\4 Regular compromisos adquiridos previos
				\4 Evitar incertidumbre jurídica
				\4 Mantener coherencia del mercado único
			\3 Contexto
				\4 Independencia de Argelia y Groenlandia
				\4[] Precedentes de abandono de UE por territorio
				\4[] $\to$ No de EEMM pleno derecho
				\4 Tratado de Lisboa: artículo 50
				\4[] Primera previsión formal de salida de UE
				\4 Tensiones UK-UE
				\4[] UK
				\4[] $\to$ Favorable a entrada nuevos miembros
				\4[] $\to$ Profundización mercado único
				\4[] $\to$ Liberalización adicional
				\4[] $\to$ Flujos de capital añadidos
				\4[] $\to$ Restricciones a migración hacia UK
				\4[] $\to$ Contención presupuestaria
				\4[] UE
				\4[] $\to$ Algunos miembros vetan accesiones
				\4[] $\to$ Migración intra-europea considerada esencial
				\4[] $\to$ Debate sobre aumento de presupuesto
				\4[] $\to$ Zona Euro cada vez más relevante
				\4 Referendum sobre permanencia en 2016
				\4[] Victoria inesperada de la salida
				\4 Notificación de decisión de salida
				\4[] En 2017
				\4[] Comienzo de dos años de negociaciones
				\4 Acuerdo de 2018
				\4[] No aprobado por parlamento británico
				\4[] Varios rechazos
				\4[] Necesario extensión del artículo 50
				\4 Acuerdo sobre salida de UE: noviembre 2019
				\4[] Periodo de transición extendido hasta final de 2020
				\4 Salida de la UE 2020
				\4[] 31 de enero de 2020
				\4[] Inicio de periodo transitorio
				\4[] $\to$ Potencialmente extensible 1 o dos años
				\4 Protocolos sobre Gibraltar, Chipre, Irlanda del Norte
				\4 Negociaciones sobre relación posterior
				\4[] Actualmente en marcha (junio 2020)
			\3 Acuerdo de salida de noviembre de 2019
				\4 Movimiento de personas
				\4[] Trabajadores pueden mantener residencia en UK y UE
				\4[] Trato nacional para trabajadores extranjeros
				\4[] Acuerdos de reconocimiento profesional se mantienen
				\4[] Posible cobrar prestaciones sociales
				\4[] $\to$ Aunque desplacen residencia a/fuera de UK o UE
				\4 Propiedad intelectual y denominaciones de origen
				\4[] Protección recíproca se mantiene
				\4 Cooperación judicial
				\4[] Se mantiene para procedimientos ya en marcha
				\4 Protección de datos
				\4[] UK mantiene reglas europeas hasta aprobación nuevas
				\4[] $\to$ UE deberá valorar adecuación posteriormente
				\4 Contratación pública
				\4[] Se mantienen procedimientos previos a salida
				\4 Euratom
				\4[] UK sale del tratado
				\4[] Compensaciones mutuas por transferencia propiedades
				\4 TJUE: competencia
				\4[] Mantiene competencia para litigios previos a salida
				\4[] 4 años de periodo transitorio
				\4[] $\to$ Mantiene jurisdicción casos contra UK
				\4 MFP
				\4[] En 2020, UK plenamenente a MFP 14-20
				\4[] Sin participación posterior
				\4[] $\to$ Aunque se acuerde extensión de periodo transitorio
				\4[] $\then$ Se decidirá contribución ad-hoc
				\4 PESC
				\4[] UK se mantiene hasta final transición
				\4[] Puede participar en misiones futuras militares
				\4[] $\to$ Pero sin liderazgo
				\4[] Puede participar en proyectos de defensa
				\4[] $\to$ Pero sin liderazgo
				\4 Interior y justicia
				\4[] Posible rechazar entrega a UK por euroorden
				\4[] Posible mantenimiento de cooperación
				\4 PPC -- Política Común de la Pesca
				\4[] Se mantienen acuerdos de acceso a pesquerías
				\4[] $\to$ Hasta final del periodo de transición
				\4 Acuerdo financiero
				\4[] Tres principios
				\4[] i. EEMM debe pagar + o recibir menos por salida
				\4[] $\to$ En relación a lo que estaba ya acordado
				\4[] ii. RU debe pagar lo ya comprometido
				\4[] iii. RU no pagará más o más pronto por salir de UE
				\4[] Cálculo de contribución de RU
				\4[] $\to$ Correspondiente a 2019 y 2020 como si fuese EM
				\4[] Obligaciones de la UE correspondientes a UK
				\4[] $\to$ Reducidas en proporción a activos de UK
				\4[] UK pagará parte correspondiente a pensiones
				\4[] $\to$ Funcionarios de la UE hasta desaparición
				\4 Irlanda e Irlanda del Norte
				\4[] Objetivos
				\4[] $\to$ Evitar frontera dura en la isla de Irlanda
				\4[] Mantener acuerdo de viernes santo
				\4[] Mantener integridad de Mercado Interior
				\4[] $\to$ IdNorte mantiene reglas básicas de UA de UE
				\4[] $\then$ Legislación IVA
				\4[] $\then$ Normativa SPS
				\4[] $\then$ Normativa agrícola
				\4[] $\then$ Normas sobre ayudas estatales

				\4[] Mantener Irlanda del Norte en UA británica
				\4[] $\to$ Permitir aprovechamiento PComercial independiente
				\4[] $\to$ Permanencia plena dentro de UA británica
				\4[] $\to$ Sujección a política comercial exterior británica
				\4[] Potestades legislativas especiales a Asambla de IdNorte
			\3 Acuerdo para la relación posterior
				\4 Actualmente en marcha (junio 2020)
			\3 Consecuencias
				\4 Contribuciones presupuestarias
				\4 Instituciones europeas en suelo británico
				\4[] Agencia Europea del Medicamento y EBA
				\4[] $\to$ Deben abandonar Reino Unido
					
	\1 \marcar{Ordenamiento jurídico}
		\2 Principios del Tratado de la Unión Europea
			\3 Carta de las Naciones Unidas
				\4 Respeto a la paz
				\4 Intervenciones internacionales basadas en reglas
				\4 Respeto asuntos internos
			\3 Convención Europea de los Derechos Humanos (1950)
				\4 Países del Consejo de Europa
			\3 Carta de los Derechos Fundamentales de la UE
				\4 Niza, 2000
				\4 Revisión, 2007
				\4 Fuera de Tratado de Lisboa
				\4[] Pero vinculante a partir de Lisboa
				\4[] $\to$ Mismo valor que resto de Tratados
				\4 No aplicable a POL, UK
				\4 Reunión de todos los dchos. reconocidos
				\4[] $\to$ En un sólo tratado
				\4[] $\to$ Dispersión previa
			\3 Atribución
				\4 Actuación sólo si lo permite un tratado
			\3 Subsidiariedad
				\4 Cuando UE no tiene competencia exclusiva
				\4[] $\to$ Actuación sólo si necesario
				\4 Si objetivo más fácil de alcanzar por EEMM
				\4[] $\to$ Preferible actuación lo más cercana al ciudadano
				\4 Sólo actúa
				\4[] Si realmente necesaria para alcanzar objetivo
				\4[] Si de forma realista no puede alcanzarse por EEMM
			\3 Proporcionalidad
				\4 Actuación de acuerdo con fines
				\4 Intervención limitada a lo necesario
				\4[] Para alcanzar fines de los tratados
		\2 Derecho primario de la Unión
			\3 Tratados fundacionales
				\4 Tratado de Roma de 1957 (TFUE)
				\4 Tratado de Maastricht 1992 (TUE)
			\3 Tratados modificativos
				\4 Todos los anteriores
				\4 Especialmente importante: Lisboa 07
				\4[] Reforma general de tratados
				\4[] Nombre actual: TUE+TFUE
				\4[] Muchas otras provisiones
			\3 Tratados de adhesión
				\4 Todos los relatados anteriormente
				\4 Contienen también provisiones accesorias
				\4[] Periodos de transición
				\4[] Ajustes presupuestarios
				\4[] ...
			
		\2 Derecho derivado
			\3 Reglamentos
				\4 Directamente aplicables
				\4 Vinculantes en todos los EEMM
			\3 Directivas
				\4 Aplicables en toda la UE
				\4 EEMM obligados a transponer
				\4[] En ordenamiento nacional
				\4 Doctrina del efecto directo:
				\4[] Si no implementada $\Rightarrow$ efecto directo + daños
			\3 Decisiones
				\4 Vinculante para sujeto(s) determinados
				\4[] EEMM, Estado Miembro, empresas, individuo(s), empresa(s)...
			\3 Recomendaciones y dictámenes
				\4 Sin carácter vinculante
				\4 Expresan postura UE O instituciones UE
		\2 Jerarquía de fuentes
			\3[1] Derecho primario de la UE
				\4 Tratados de la Unión Europea
			\3[2] Actos legislativos
				\4 Aprobados por acto legislativo ordinario o especial
			\3[3] Actos delegados
				\4 Actos de naturaleza legislativa
				\4[] Añaden elementos a normas o modifican
				\4 PE y Consejo delegan en Comisión
				\4[] Partes no fundamentales de una norma
				\4[] Anexos
				\4 Posible veto de PE y Consejo
			\3[4] Actos de ejecución
				\4 Normalmente, ejecución concierne a EEMM
				\4 Condiciones uniformes de aplicación
				\4[] Necesitan a veces ejecución uniforme
				\4[] $\Rightarrow$ Comisión ejecuta directamente
				\4[] $\Rightarrow$ Sustituye a EEMM legislación
				\4 Comitología
				\4[] Procedimiento aplicable a actos de ejecución
				\4[] $\to$ No a todos
				\4[] Comité asiste a Comisión en ejecución
				\4[] Representantes de todos los EEMM
				\4[] + funcionario de la Comisión
				\4[] Aplicables uno de dos procedimientos
				\4[] $\to$ Examen
				\4[] $\to$ Consultivo
				\4[] Procedimiento de examen
				\4[] $\to$ Comisión plantea propuesta de ejecución
				\4[] $\to$ Si QMV vota a favor, se ejecuta
				\4[] $\to$ Si QMV vota en contra, no se ejecuta
				\4[] $\to$ Si ninguno de ambos, Comisión propone de nuevo o ejecuta
				\4[] Procedimiento consultivo
				\4[] $\to$ Necesario consultar pero no vinculante
		\2 Procedimiento legislativo
			\3 Ordinario\footnote{Ver \url{http://www.europarl.europa.eu/ordinary-legislative-procedure/en/ordinary-legislative-procedure.html}}
				\4[0] Comisión propone legislación
				\4[1] Primera lectura
				\4[] Primera lectura PE
				\4[] Aprueba
				\4[] Enmienda
				\4[$\to$] Remisión al Consejo
				\4[] Primera lectura Consejo
				\4[] Aprueba $\to$ legislación aprobada
				\4[] Enmienda $\to$ remisión al PE
				\4[2] Segunda lectura
				\4[] Segunda lectura PE
				\4[] Aprueba o no se pronuncia $\to$ legislación aprobada
				\4[] Rechaza $\to$ legislación rechazada
				\4[] Enmienda $\to$ remisión a la Comisión
				\4[] Comisión opina
				\4[$\to$] Remisión al Consejo
				\4[] Segunda lectura Consejo
				\4[] Si Comisión favorable:
				\4[] $\to$ Aprobación por QMV
				\4[] Si Comisión desfavorable:
				\4[] $\to$ Aprobación por unanimidad
				\4[] Rechazo del Consejo $\to$ Conciliación
				\4[3] Conciliación y tercera lectura
				\4[] Conciliación
				\4[] Comité PE + Consejo, igual n. de miembros + Comisión
				\4[] (diálogo tripartito/triálogo/trilogue)
				\4[] Plazo de 6 semanas
				\4[] Si no hay acuerdo $\to$ legislación rechazada
				\4[] Si acuerdo $\to$ 3a lectura
				\4[] 3a lectura PE y Consejo
				\4[] Sin enmiendas posibles
				\4[] Si ambos aprueban $\to$ legislación aprobada
				\4[] Si PE y/o Consejo rechazan $\to$ legislación rechazada
				\4 \textit{Emergency Break}
				\4[] En determinadas áreas, un EEMM puede solicitar suspensión proceso
				\4[] $\to$ Envío directo al Consejo
			\3 Aprobación del presupuesto anual
				\4 Duración del proceso
				\4[] 1 de julio a 31 de diciembre
				\4[1.] Proyecto de presupuesto (Comisión):
				\4[] tras previsiones gasto instituciones
				\4[] 1 de septiembre como tarde
				\4[2.] Consejo toma posición y envía a PE
				\4[] 1 de octubre como tarde
				\4[3.] Parlamento aprueba/enmienda
				\4[] Si aprueba, presupuesto aprobado
				\4[] Si no se pronuncia, aprobado
				\4[] Si enmienda, a Conciliación
				\4[4.] Conciliación (Comité de Conciliación)
				\4[] Compuesto por miembros de PE y CdUE
				\4[] 21 días para acordar texto conjunto
				\4[] Si no hay acuerdo
				\4[] $\to$ Comisión debe proponer nuevo proyecto
				\4[] Si hay acuerdo:
				\4[] $\to$ Sometido a aprobación en CdUE y PE
				\4[] $\to$ 14 días para aprobar
				\4[5.] Aprobación o extensión de 1/12 de anterior
				\4[] CdUE
				\4[] $\to$ Aprueba o rechaza
				\4[] Si CdUE rechaza PE puede aprobar con:
				\4[] $\to$ Mayoría de los miembros de PE
				\4[] $\to$ 3/5 favorables de votos emitidos
			\3 Cooperación reforzada
				\4 Concepto
				\4[] Al menos 9 EEMM
				\4[] Implementar medidas cuando fallan otros acuerdos
				\4[] $\to$ Superar situaciones de parálisis
				\4 Introducido en Tratado de Amsterdam
				\4[] En vigor a partir de 1999
				\4 Tratado de Lisboa (2009)
				\4[] Formaliza
				\4[] Introduce posibilidad de PESC
				\4 Autorización requiere:
				\4[] $\to$  QMV en CdUE
				\4[] $\to$ Consentimiento del Parlamento
				\4 Cambios en reglas de cooperación reforzada
				\4[] Unanimidad de estados participantes
				\4[] Consulta al PE
				\4 Accesión a la cooperación reforzada
				\4[] En cualquier momento por nomiembros
				\4 Procedimiento especial para defensa
				\4 Actualmente en vigor
				\4[] Patente Única Europea\footnote{Todos participan salvo España y Croacia. España se opone a que los idiomas oficiales del sistema sean el inglés, el francés y el alemán. La postura española defiende que o sólo ingleś, o también italiano y español.}
				\4[] Ley de divorcio
				\4[] Fiscal Europeo
				\4[] $\to$ Casos de fraude contra la UE
				\4[] Marco legal de propiedad común en divorcios
				\4[] Estructura de cooperación permanente en defensa y seguridad
			\3 Especiales
				\4 Consejo tiene papel predominante
				\4 Dos tipos de procedimiento
				\4[] $\to$ Consentimiento
				\4[] $\to$ Consulta
				\4 Procedimiento de consentimiento
				\4[] Comisión propone a CdUE
				\4[] $\to$ Raramente, otras instituciones o EEMM proponen
				\4[] CdUE puede adoptar con consentimiento de PE
				\4[] PE consiente por mayoría absoluta
				\4[] PE puede rechazar pero no enmendar
				\4[] CdUE no puede superar rechazo de PE
				\4[] Aplicable a:
				\4[] $\to$ Cláusula de flexibilidad horizontal
				\4[] $\to$ Combatir discriminación
				\4[] $\to$ Adhesión a la UE
				\4[] $\to$ Acuerdos de salida de la UE
				\4[] $\to$ Acuerdos de asociación
				\4[] $\to$ Acuerdos con implicaciones presupuestarias
				\4[] $\to$ ...
				\4[] También aplicable actos no legislativos
				\4 Procedimiento de consulta
				\4[] PE aprueba, enmienda o rechaza
				\4[] $\to$ Pero CdUE no está vinculado por PE
				\4[] $\to$ Mero fin consultivo
				\4[] Paso por PE es necesario para aprobación
				\4[] Aplicable a:
				\4[] $\to$ Excepciones al mercado común
				\4[] $\to$ Competencia
				\4 Comisión puede adoptar unilateralmente
				\4[] Actos no legislativos
				\4[] Competencia
				\4[] Telecomunicaciones
				\4[] Arancel exterior común
	\1 \marcar{Instituciones}\footnote{Del Tratado de Lisboa, artículo 9.}
		\2 Comisión Europea
			\3 Función
				\4 Iniciar procedimiento legislativo
				\4 Ejecutar políticas UE
				\4 Velar por cumplimiento de Tratados junto con TJUE
				\4 Representar a la Unión Europea y negociar tratados
			\3 Antecedentes
				\4 Predecesor: Alto Comisionado de la CECA
				\4 Comisiones separadas de la CEE y Euratom
				\4 Tratado de Fusión 1965: una sola Comisión en 1967
				\4 Crecimiento n. comisarios paralelo a adhesiones
			\3 Organización
				\4 Sede en Bruselas
				\4 ~30.000 empleados
				\4 Presidente
				\4[] Propuesto por Consejo Europeo por QMV hasta 2014
				\4[] En 2014, PE propuso y sentó precedente
				\4[] $\to$ ``Spitzenkandidat''
				\4[] Confirmado por PE
				\4 Comisarios
				\4[] 28/27 comisarios, 1 por país
				\4[] Posible reducción futura\footnote{El Tratado de Lisboa establecía inicialmente que a partir de 2014 el número total de comisarios habría de ser no superior a dos tercios del total de EEMM. Sin embargo, los EEMM pequeños se opusieron frontalmente a la idea, y una de las condiciones ofrecidas a Irlanda tras el voto negativo en referendum fue la confirmación de la regla de un comisario por país. Sin embargo, existe una cláusula de revisión al respecto.}
				\4[] 27 comisarios en legislatura 2019-2024
				\4[] Nominados por Presidente de la Comisión
				\4[] Aprobados por Parlamento de forma conjunta\footnote{Un caso notable fue el de Rocco Buttiglione en 2004, nominado para Comisario de Justicia. El PE acabó forzando su reemplazo por Franco Frattini.}
				\4[] Posible remoción por PE a mitad de mandato
				\4[] Elegidos por 5 años
				\4[] Direcciones Generales
				\4[] 32 DG solapadas con Comisarios
				\4 Secretaría General
				\4[] Servicios técnicos
				\4[] Aconsejar Comisión
				\4[] Coordinar
				\4 Comités
				\4[] Sin poderes formales
				\4[] Funcionarios de Comisión + expertos nacionales
				\4[] "\textit{Comitología}"
			\3 Actuaciones
				\4 Relaciones exteriores
				\4[] Negociación en nombre de EEMM acuerdos comerciales
				\4[] Representación UE
				\4[] Servicio Europeo de Acción Exterior ha sustituido en parte
				\4 Iniciación del proceso legislativo
				\4[] Leyes necesarias para cumplimiento de Tratados y sus principios
				\4[] 1er paso: dirección general presenta borrador
				\4[] 2o paso: Comisión aprueba/rechaza (reuniones semanales)
				\4[] 3er paso: presentación PE y C. de UE
				\4 Poderes de implementación
				\4[] Monitorizar progreso de EEMM en implementación
				\4[] Llevar EEMM ante TJUE
				\4[] Imponer sanciones por incumplimiento
				\4 Green papers / Libros verdes
				\4[] Destinados a interesados, para recabar opinión
				\4 White papers / Libros blancos
				\4[] Propuestas de la Comisión tras \textit{green papers}
			\3 Valoración
				\4 Objeto de debate:
				\4[] Federalismo vs. funcionalismo
				\4 Comisarios:
				\4[] ¿Representantes de la UE o intereses nacionales?
				\4 Crítica:
				\4[] Exceso de burocracia, excesivo tamaño
				\4[] En realidad, tamaño pequeño dada población UE
		\2 Parlamento Europeo
			\3 Función
				\4 Similares a poder legislativo
				\4 Aprobar legislación junto con Consejo
				\4 Controlar Comisión (poder ejecutivo)
			\3 Antecedentes
				\4 Institución inicialmente consultiva
				\4 Único poder inicial: hacer dimitir a la Comisión
				\4 Atribución gradual de poderes
				\4 Tratado de Lisboa: verdadero cuerpo legislativo
			\3 Organización
				\4 Estrasburgo: sesiones plenarias
				\4 Bruselas: comités parlamentarios
				\4[] $\to$ Preparan contenido de plenos
				\4 Luxemburgo: secretariado
				\4 Pre-Brexit
				\4[] 751 diputados
				\4[] No. diputados según población
				\4[] Límites máx-min: 96-6 diputados
				\4[] Por 5 años
				\4[] Sufragio universal directo
				\4[] Regional o nacional: depende de país
				\4[] Organizados por grupo político europeo
				\4[] España: 54 diputados
				\4 Tras Brexit
				\4[] Reducción a 705 diputados
				\4[] $\to$ 46 mantenidos para futuras ampliaciones
				\4[] $\to$ 27 redistribuidos a diferentes países\footnote{España ganará 5 diputados.}
				\4[] Ningún país pierde
				\4[] Reducción sólo parcial de 73 diputados británicos
				\4[] Se reservan algunos para futuras adhesiones
			\3 Actuaciones
				\4 Procedimiento legislativo ordinario
				\4[] $\to$ Junto con Consejo
				\4 Presupuesto de la UE
				\4[] Aprobación presupuesto final junto con Consejo de la UE
				\4[] Firma del presidente del Parlamento
				\4 Controlar ejecución del presupuesto
				\4 Aprobar nombramiento de Comisión
				\4[] Cuestiona cada Comisario propuesto
				\4[] Vota a favor o en contra
				\4[] $\to$ Respecto de conjunto de Comisión
				\4 Moción de censura a la Comisión
				\4[] Forzar la renuncia
				\4[] Necesarios 2/3 de los votos emitidos
				\4 Nombrar al Ombudsman
			\3 Valoración
				\4 Debate continúa sobre papel del PE:
				\4[] $\to$ Rol consultivo
				\4[] $\to$ Creación segunda cámara representantes parlamentos nacionales
				\4[] $\to$ Aumento poderes
				\4 Institución dinámica, no estabilizada
				\4 Elección parlamentarios en clave nacional
				\4 Votantes no deciden en términos europeos
				\4 Posibles cambios tras Brexit
				\4[] ¿Qué hacer con parlamentarios salientes?
		\2 Consejo de la Unión Europea
			\3 Función
				\4 Representar intereses nacionales en la UE
				\4 Contrapeso a Comisión (intereses UE), Parlamento
				\4 Coordinar políticas de EEMM
				\4 Aprobar legislación junto a EP
			\3 Antecedentes
				\4 Tres Consejos de Ministros hasta Tratado de Fusión (1965)
				\4 Consejo de Ministros hasta TUE (1992)
				\4 Cierto traspaso de poderes en favor PE
			\3 Organización
				\4 Ministros de EEMM
				\4 Representantes de la Comisión, sin voto
				\4 10 configuraciones posibles
				\4 Más importantes: reunión mensual
				\4[] ECOFIN, GAC\footnote{General Affairs Council}, FAC\footnote{Foreign Affairs Council}, Agricultura y Pesca
				\4 Bruselas y Luxemburgo
				\4 COREPER\footnote{Comité de Representantes Permanentes.}
				\4[] Enorme importancia, decide mayoría de asuntos
				\4[] COREPER I: asuntos económicos y sociales
				\4[] COREPER II: embajadores, asuntos políticos, financieros y exteriores
				\4[] Preparan asuntos del Consejo
				\4[] Salvo agricultura: comité especial
			\3 Actuaciones
				\4 Aprobación anual orientación política económica
				\4 Presupuesto
				\4[] Comisión propone
				\4[] Consejo negocia con PE
				\4[] Potestad para aprobar gastos obligatorios\footnote{En lo que respecta a los gastos no obligatorios, el Parlamento Europeo tiene la última palabra.}
				\4 Adopción de decisiones
				\4[] Tres procedimientos
				\4[1] Unanimidad
				\4[2] Mayoría cualificada / QMV
				\4[] Procedimiento legislativo ordinario
				\4[] $\geq 65\%$ población, $\geq 55\%$ miembros del Consejo = 15 miembros
				\4[] Minoría de bloqueo:
				\4[] $\to$ 35\% población, 45\% EEMM y al menos 4 EEMM
				\4[] Método previo: ponderación por EEMM
				\4[3] Mayoría simple
				\4[] Mayoría de 15 miembros
				\4[] Asuntos procesales del Consejo
				\4[] Requerir a Comisión propuestas o estudios
			\3 Valoración
				\4 Consejo decide asuntos clave o sensibles
				\4[] Resto: COREPER
				\4 Propósito inicial:
				\4[] Fomentar cooperación y trabajo conjunto entre EEMM
				\4 Actualidad:
				\4[] Escenifica conflictos entre EEMM
				\4[] Objeto de batallas opinión pública
		\2 Consejo Europeo
			\3 Función
				\4 Dirección estratégica y política\footnote{Tratado de Lisboa: \comillas{shall provide the Union with the necessary impetus for its development}.}
				\4 Lineas generales actuación instituciones
			\3 Antecedentes
				\4 De Gaulle años 60
				\4[] Intento de menoscabar instituciones europeas
				\4 Reuniones alternativas París-Bonn bianuales
				\4 Dublín, 1975: primer Consejo formal
				\4 Acta Única 1987: primera mención en tratados
				\4 Maastricht 1992:
				\4[] primera atribución formal de contenido
				\4 Tratado de Lisboa:
				\4[] Institución de la Unión Europea
			\3 Organización
				\4 Jefes de Estado/Primeros Ministros\footnote{Depende del país.}
				\4 Presidente del Consejo Europeo
				\4[] Sin voto
				\4[] En la actualidad Charles Michel
				\4[] Elegido por mayoría cualificada
				\4[] 2 años y medio, renovable una vez
				\4 Asistencia a cumbres: presidente PE, BCE...
			\3 Actuaciones
				\4 Mínimo de 4 encuentros al año
				\4[] Extraordinarios, previstos e informales
				\4 Informales: declaraciones
				\4 Previstos, extraordinarios: conclusiones
				\4 Decisión:
				\4[] En general, por consenso
				\4[] Donde tratados lo prevean, por QMV\footnote{Ver artículo 15.4 de TUE.}
				\4 Definición posición internacional crisis
				\4 Nombramientos:
				\4[] Alto Representante Asuntos Exteriores\footnote{Aunque el CEuropeo lo nombra, debe someterse a la aprobación posterior del Parlamento como miembro de la Comisión.}
				\4[] Presidente del Banco Central Europeo
				\4 Propone:
				\4[] Presidente de la Comisión, al Parlamento
				\4 Poder de influencia
				\4[] Composición de la Comisión
				\4[] Interior y Justicia
				\4[] Suspensión de membresía
				\4[] Cambio mecanismos de voto
			\3 Valoración
				\4 Necesario para desbloquear temas perfil más alto
				\4 Relativo éxito: acuerdos adhesión, Lisboa, EFSB...
				\4 En ocasiones referido como \comillas{órgano supremo}
		\2 Tribunal de Justicia de la Unión Europea
			\3 Función
				\4 Asegurar cumplimiento acervo
				\4 Guiar tribunales nacionales aplicación de acervo
			\3 Antecedentes
				\4 Tribunal General:
				\4[] Tribunal de Primera Instancia (creado en 1989)
				\4 Tribunal de la Función Pública:
				\4[] derogado en 2016
			\3 Organización
				\4 Tribunal de Justicia
				\4[] Un juez por estado miembro
				\4[] $\to$ 6 años renovables
				\4[] Presidente elegido por jueces, de entre jueces
				\4[] $\to$ por 3 años renovables
				\4[] 11 abogados generales\footnote{6 de los cuales corresponden a Alemania, Francia, Italia, España, Reino Unido y Polonia, y los 5 restantes a los otros 22 países.}
				\4[] 1 Secretario
				\4[] Formaciones
				\4[] $\to$ Pleno
				\4[] $\to$ Gran Sala (15 jueces)
				\4[] $\to$ Salas de 5 y 3 jueces
				\4 Tribunal General
				\4[] 45 jueces en 2017
				\4[] En 2019, previsto 2 jueces por estado
			\3 Actuaciones
				\4 Tribunal de Justicia
				\4[] Dictámenes Derecho UE bajo petición de tribunales nacionales
				\4[] Acciones contra EEMM por inaplicación de Derecho de la UE
				\4[] Acciones contra legislación europea por violar tratados
				\4[] Acciones contra decisiones, acciones u omisiones de UE
				\4 Tribunal General
				\4[] Particulares contra instituciones UE, agencias, directorados...
				\4[] Contra actos regulatorios, acciones, omisiones que les conciernen directamente
				\4[] EEMM contra Comisión
				\4[] EEMM contra Consejo sobre ayudas, dumping, competencias
				\4[] Compensación de daños causados por UE
				\4[] Sobre contratos con UE que dan jurisdicción a tribunal
				\4[] Marcas registradas
				\4[] Empleados de la UE
			\3 Valoración
				\4 Gran contribución construcción europea
				\4 Consolidación ordenamiento europeo
		\2 Tribunal de Cuentas
			\3 Función
				\4 Controlar ejecución del presupuesto
				\4 Opinar sobre legislación con impacto en finanzas
			\3 Antecedentes
				\4 Creado en 1975
				\4 Maastricht institucionaliza en Tratado
				\4 Amsterdam: competencia fiscalización todas las instituciones
			\3 Organización
				\4 Un juez por país
				\4 Elegidos por 6 años renovables
				\4 Consejo de la UE elige por unanimidad
				\4[] Tras consulta con PE
				\4 Presidente: renovable por tres años
			\3 Actuaciones
				\4 Informe anual a PE y Consejo
				\4 Encontrar irregularidades: informar Oficina Antifraude
				\4 Sin competencias de ejecución
			\3 Valoración
				\4 Poca atención medios
				\4 Importantes casos:
				\4[] Regiones inglesas
				\4[] Influencia PE sobre reprobación de ejecución del presupuesto
				\4[] En 1984, 1999 PE rechazó aprobar
				\4[] Dimisión Comisión Santer en 1999
		\2 Banco Central Europeo
			\3 Función
				\4 Mantenimiento estabilidad de precios\footnote{\textit{<<The ECB aims at maintaining inflation rates below, but close to, 2\% over the medium term.>>}}
				\4 Gestión reservas de divisas extranjeras
				\4 Mantener funcionamiento sistema de pagos
			\3 Antecedentes
				\4 Comité de Gobernadores (1964)
				\4 Fondo Europeo de Cooperación Monetaria (1973)
				\4[] Miembros de la CEE
				\4[] Coordinar acciones para paridades en túnel
				\4[] Asume funciones del SME desde 1979
				\4 Informe Delors (1989): creación de la EMU
				\4 Instituto Monetario Europeo (1994)
				\4 Creación BCE (1998)
			\3 Organización
				\4 Comité ejecutivo
				\4[] Presidente
				\4[] Vicepresidente
				\4[] 4 miembros
				\4[] Todos elegidos por acuerdo Consejo Europeo
				\4[] Mandato 8 años no renovables
				\4 Consejo de gobierno
				\4[] 6 miembros consejo ejecutivo
				\4[] 19 gobernadores de la Eurozona
				\4[] Votan los 6 del CEjecutivo + 15 gob de BCNs
				\4[] Reunión dos veces al mes
				\4 Consejo General
				\4[] Presidente y vicepresidente BCE, 28 gobernadores
				\4 Sistema Europeo de Bancos Centrales (SEBC)
				\4[] BCE+BCNs de UE
				\4[] Provisional hasta integración todos EEMM en Z€
				\4 Eurosistema
				\4[] BCE + 19 bancos centrales zona euro
				\4 Consejo de Supervisión
				\4[] Tareas de supervisión
				\4[] Dos reuniones al mes
				\4[] Presidente propio
				\4[] Vicepresidente de entre miembros Comité Ejecutivo
				\4[] Bancos centrales de los 27/28
			\3 Actuaciones
				\4 Implementación política monetaria
				\4 Emisión moneda
				\4 Intervención mercado divisas, deuda
				\4 Seguimiento evolución macroeconómica
				\4 Emisión recomendaciones
			\3 Valoración
				\4 Papel clave durante la crisis
				\4 Elemento central integración monetaria
				\4 Eje de política económica europea
				\4 Críticas cruzadas
				\4 Conflictos de intereses diferentes países
		\2 Otras
			\3 Comité Económico y Social Europeo (CESE)
				\4 Dividido en tres grupos:
				\4[] Patronal, sindicatos, grupos de interés
				\4 Labor consultiva de Comisión, Consejo, Parlamento
				\4 También opiniones propias
			\3 Comité de las Regiones
				\4 Similar papel a CESE
				\4 Compuesto por cargos nivel regional
			\3 Eurogrupo
				\4 Ministros de economía/finanzas de la zona euro
				\4 Subconjunto del ECOFIN
				\4 Presidente propio
				\4[] Elegido por dos años y medio
				\4[] Debe ser ministro de finanzas/economía
				\4 Decide asuntos económicos de la zona €
			\3 Servicio Europeo de Acción Exterior
				\4 Apoyo a Alta Representante
				\4 Relaciones con países no UE
	\1[] \marcar{Conclusiones}
		\2 Recapitulación
			\3 Evolución histórica
				\4 Tratados
				\4 Adhesiones
			\3 Ordenamiento jurídico
				\4 Principios del TUE
				\4 Derecho primario
				\4 Derecho derivado
				\4 Jerarquía de fuentes
				\4 Procedimiento legislativo
		\2 Idea final
			\3 Sistema complejo
				\4 Ordenamiento jurídico propio
				\4 Instituciones independientes
				\4 Iniciativa propia
			\3 Sistema en evolución
				\4 ¿Qué depara futuro a la Unión?
				\4 Frecuentes propuestas de reforma
				\4 Tensión federalistas-funcionalistas
				\4[] i.e.: EEMM vs. UE en sí misma
\end{esquemal}































%\seccion{Primera versión}
%\begin{multicols}{2}
%\begin{esquema}[enumerate]
%    \1[] Introducción
%        \2 Objetivo UE: art. 3 TUE
%        \2 Objetivos intermedios
%        \2 Estructura estatal-interestatal
%            \3 Poderes
%            \3 Instituciones
%            \3 Ordenamiento jurídico propio
%        \2 Objeto de la exposición
%        \2 Estructura
%    \1 Evolución histórica
%        \2 Factores
%            \3 Coyuntura económica
%            \3 Exigencias prácticas del proceso
%            \3 Acontecimientos políticos
%            \3 Vocación europea
%                \4 Escuela realista
%                \4 Escuela federalista
%        \2 Origen y expansión (1951-1973)
%            \3 Robert Schumann
%            \3 Tratado de la CECA (París, 1951)
%            \3 Tratados de Roma: CEE y Euratom (1958)
%                \4 Firmantes
%                \4 Política comercial común
%                \4 PAC
%                \4 PPC
%                \4 Transportes
%                \4 Cometencia
%                \4 Fondo Social Europeo
%                \4 Banco Europeo de Inversiones
%                \4 Comité Económico y Social
%            \3 Primera ampliación: RU, IRL, DIN
%        \2 Crisis (1973-1985)
%            \3 Silla vacía
%            \3 Creación del Consejo Europeo
%            \3 Atribución de competencias al PE
%            \3 Segunda y tercera ampliación (GRE, ESP, POR)
%        \2 Relanzamiento (1985-1995)
%            \3 Acta Única Europea (1986)
%                \4 Política Monetaria
%                \4 Política social
%                \4 Cohesión
%                \4 Investigación y medio ambiente
%                \4 Política exterior
%            \3 Logros
%                \4 Mejora cap. decisión Consejo de la UE
%                \4 Fortalecimiento PE
%            \3 Plan Delors 1989: integración monetaria
%            \3 Tratado de Maastricht 1992 (TUE)
%                \4 Entrada en vigor: 1 de noviembre 1993
%                \4 Primer pilar: Comunidad Europea (CEE, CECA, Euratom)
%                \4 Segundo pilar: Política Exterior y de Seguridad Común
%                \4 Tercer pilar: Cooperación Justicia y Asuntos de Interior
%                \4 Ciudadanía Europea
%                \4 Sistema Europeo de Bancos Centrales
%                \4 Aumento poder del consejo: mayoría cualificada
%            \3 Cuarta ampliación: AUS, FIN, SUE (1995)
%        \2 Proceso de reforma institucional (1995-2004)
%            \3 Solicitudes de ampliación
%            \3 Tratado de Amsterdam (1997)
%                \4 Incorporación Schengen
%                \4 Comunitarización materias tercer pilar
%                \4 Procedimiento de cooperación reforzada
%                \4 Fortalecimiento PE: codecisión
%            \3 Tratado de Niza (2001)
%                \4 Límite un solo comisario (2014)
%                \4 Ponderación votos Consejo de la UE
%                \4 Ampliación competencias PE
%                \4 Mayoría cualificada: ampliación materias
%                \4 Cooperación reforzada
%                \4 Proyecto de Constitución Europea
%        \2 Situación actual: ampliación a 28 y Tratado de Lisboa (2007)
%            \3 Quinta y sexta ampliación
%                \4 Quinta (2004): MAL, CHI, EST, LET, LIT, POL, CHE, EVQ, HUN, ESL
%                \4 Sexta (2007) BUL, RUM
%                \4 Croacia (2013)
%            \3 Tratado de Lisboa (2007)
%                \4 Abandono estructura cuasi-estatal
%                \4 Entrada en vigor: 2009
%                \4 Reparto de competencias: exclusivas, compartidas, de apoyo
%                \4 Principio de atribución
%                \4 Principio de subsidiariedad
%                \4 Principio de proporcionalidad
%                \4 Alto Representante de Asuntos Exteriores
%                \4 Servicio Europeo de Acción Exterior
%                \4 Parlamento Europeo: legislador pleno
%                \4 Fin ponderación votos Consejo de la UE
%                \4 Mayor flexibilidad integración
%    \1 Ordenamiento jurídico
%        \2 Principios
%            \3 Autonomía
%            \3 Primacía
%            \3 Efecto directo
%            \3 Unidad
%            \3 Complejidad
%        \2 Fuentes y jerarquía
%            \3 Derecho primario/originario/constitutivo
%                \4 TUE
%                \4 TFUE
%                \4 Euratom
%                \4 Carta de los Derechos Fundamentales de la UE
%            \3 Acuerdos de Derecho Internacional de la Unión
%            \3 Derecho derivado o secundario
%            \3 Principios generales del Derecho
%        \2 Instrumentos de Derecho derivado
%            \3 Reglamento
%                \4 Alcance general
%                \4 Obligatorio en todos los elementos
%                \4 Aplicabilidad directa
%                \4 Efecto directo
%            \3 Directiva
%                \4 Sin alcance general: solo estados miembros destinatarios
%                \4 Obligación de resultado
%                \4 Sin aplicabilidad directa: transposición
%                \4 Sin efecto directo salvo excepciones
%            \3 Decisión
%                \4 Sin alcance general: destinatarios específicos (EEMM, personas físicas y jurídicas)
%                \4 Obligatoria en todos elementos
%                \4 Notificación a destinatarios
%                \4 Efecto directo
%            \3 Recomendación y dictamen
%                \3 Recomendación: directiva no obligatoria a destinatarios
%                \3 Dictamen (avis): punto de vista de una institución
%        \2 Aprobación de la legislación (procedimiento ordinario)
%            \3 Propuesta de la Comisión
%                \4 Excepcionalmente, actores solicitan a Comisión
%                \4 Presentación PE y Consejo de la UE
%                \4 Publicación DOUE
%            \3 Primera lectura en PE
%                \4 Solicitar retirada
%                \4 Aceptar sin cambios
%                \4 Enmendar
%            \3 Primera lectura en Consejo de la UE
%                \4 Rechazar en totalidad
%                \4 Aceptar sin cambios
%                \4 Enmendar
%            \3 Segunda lectura en el PE
%                \4 Aprobar posición
%                \4 Agotar plazo: aprobación
%                \4 Rechazar
%                \4 Enmendar
%            \3 Segunda lectura en el Consejo de la UE
%                \4 Aprobar enmiendas
%                \4 No aprobar todas las enmiendas: comité de conciliación
%            \3 Conciliación
%                \4 No aprobar
%                \4 Aprobar texto Conjunto
%            \3 Tercera lectura PE y Consejo de la UE
%                \4 Aprobar en ambos órganos
%                \4 Rechazo en uno o dos: conclusión
%    \1 Instituciones del Tratado de Lisboa
%        \2 Comisión Europea
%            \3 Composición
%                \4 28 miembros
%                \4 Consejo propone a PE candidato presidente
%                \4 Aprobación comisarios por PE
%            \3 Duración 5 años
%            \3 Comisión Actual
%                \4 Jean-Claude Juncker
%                \4 Federica Mogherini
%                \4 Un vicepresidente primero
%                \4 Vicepresidentes
%            \3 Papel destacado del presidente
%            \3 Independencia EEMM
%            \3 Moción de censura PE
%            \3 Funciones
%                \4 Iniciativa legislativa
%                \4 Cumplimiento de los tratados
%                \4 Poder ejecutivo: actos delegados y de ejecución
%            \3 Poder de negociación internacional
%            \3 Funcionamiento
%                \4 Reuniones del Colegio de Comisarios: semanalmente (miércoles)
%                \4 Estructura administrativa
%        \2 Parlamento Europeo
%            \3 Composición
%                \4 Máximo 750 miembros + presidente
%                \4 Consejo determina composición exacta
%                \4 Proporcional a población: máx. 90, mín. 6
%                \4 Grupos políticos
%                \4 Presidente: 2 años y medio
%                \4 Mesa: presidente + mesa del parlamento y mesa+
%            \3 Funciones
%                \4 Legislativa: ordinario y consulta
%                \4 Consultiva: emisión dictámenes conformes
%                \4 Presupuestaria: aprobación, ejecución y gestión
%                \4 Control democrático instituciones
%            \3 Organización
%                \4 Comisiones
%                \4 Sesiones plenarias
%                \4 Secretaría General PE
%            \3 Sede
%                \4 Oficial: Estrasburgo
%                \4 Sesiones plenarias: Estrasburgo, Bruselas
%                \4 Secretaría General: Luxemburgo
%            \3 Reglas decisión
%                \4 Mayoría absoluta votos emitidos o miembros
%                \4 Relaciones Parlamentos nacionales
%        \2 Consejo de la Unión Europea
%            \3 Composición
%                \4 Miembros: 28
%                \4 10 formaciones del Consejo
%                \4 Presidencia rotativa salvo Consejo de Asuntos Exteriores
%            \3 Funciones
%                \4 Adopción legislación
%                \4 Posición política
%                \4 Aprobación del presupuesto
%                \4 Acuerdos internacionales
%                \4 Desarrollo de la política exterior y de seguridad común
%                \4 Coordinación de las políticas de los EEMM
%                \4 Preparación de los trabajos del Consejo: COREPER, Comités y Secretaría General
%            \3 Reglas de votación
%                \4 Mayoría simple
%                \4 Mayoría cualificada
%                \4 Unanimidad
%            \3 Eurogrupo
%                \4 Ministros finanzas Zona Euro + Comisión + BCE
%                \4 Presidente: dos años y medio
%                \4 Reunión mensual
%                \4 Carácter informal
%        \2 Consejo Europeo
%            \3 Miembros
%                \4 28 Estados + Presidente Consejo + Presidente Comisión
%                \4 Presidente: dos años y medio (Donald Tusk)
%                \4 Posibilidad de fusión presidente ConsejoE y Comisión
%            \3 Funciones
%                \4 Carácter político
%                \4 Objetivos estratégicos política exterior
%                \4 Marco financiero plurianual
%                \4 Gobernanza económica
%            \3 Organización cumbres
%                \4 Dos veces por semestre
%                \4 Preparación: Consejo de Asuntos Generales
%                \4 Conclusiones
%            \3 Relación con instituciones
%                \4 Total independencia
%            \3 Reglas de decisión
%                \4 Consenso
%                \4 Mayoría cualificada (excepcional)
%            \3 Cumbre del Euro
%                \4 Presidentes Z€ + Presidente Comisión + Presidente BCE
%        \2 Tribunal de Justicia de la Unión Europea
%            \3 Composición
%                \4 Un juez por estado miembro
%                \4 Nueve abogados generales: dictámenes
%                \4 Designación por 6 años : comité idoneidad
%                \4 Renovación parcial cada 3 años
%                \4 Presidente por 3 años
%            \3 Funciones
%                \4 Control legalidad de todos los actos normativos y ejecutivos
%                \4 Control a los Estados miembros cumplimiento obligciones Tratados y DDerivado
%                \4 Interpretación Derecho Comunitario
%            \3 Funcionamiento
%                \4 Pleno 
%                \4 Gran sala (13 jueces)
%                \4 Salas de 3 o 5 jueces
%                \4 Tribunal General: primera instancia
%                \4 Tribunal de la función Pública
%        \2 Tribunal de Cuentas
%            \3 Composición
%                \4 Un nacional de cada estado
%                \4 Por 6 años, mayoría cualificada
%                \4 Presidente
%            \3 Funciones
%                \4 Control ejecución presupuesto comunitario
%                \4 Control de la legalidad y regularidad de los ingresos y gastos de la UE
%                \4 Función consultiva
%            \3 Funcionamiento
%                \4 División en cinco salas
%                \4 Elementos de independencia: libertad de selección de temas
%        \2 Banco Central Europeo
%            \3 Composición
%                \4 Consejo general del SEBC
%                \4 Comité Ejecutivo del BCE
%                \4 Consejo de Gobierno del BCE
%                
%    \1[] Conclusión
%\end{esquema}
%\end{multicols}

\conceptos

\concepto{Instituciones del Tratado de Lisboa:} El artículo 13.1 del TUE consolidado (\url{http://eur-lex.europa.eu/resource.html?uri=cellar:2bf140bf-a3f8-4ab2-b506-fd71826e6da6.0023.02/DOC\_1\&format=PDF}) establece las siguientes instituciones como \comillas{Instituciones de la Unión}:
\begin{itemize}
	\item El Parlamento Europeo.
	\item El Consejo Europeo.
	\item El Consejo.
	\item La Comisión Europea.
	\item La Corte de Justicia de la Unión Europea.
	\item El Banco Central Europeo.
	\item El Tribunal de Cuentas de la Unión Europea.
\end{itemize}


\preguntas

\seccion{Test 2018}

\textbf{40.} Respecto a la aplicación de las directivas comunitarias en un país de la UE, señale la respuesta \underline{\textbf{CORRECTA}}:

\begin{itemize}
	\item[a] Para ser aplicable, la directiva requiere siempre de un acto de transposición en la legislación nacional por parte de las autoridades nacionales; en ningún caso puede tener aplicación directa.
	\item[b] En algunos casos, la directiva podría tener efecto directo al margen del acto de transposición, si su contenido confiere derechos a los ciudadanos y las autoridades nacionales han realizado la transposición fuera de plazo.
	\item[c] Las directivas son directamente aplicables en cada Estado miembro de la UE a partir de su publicación en el Diario Oficial de la UE, pues dispone de la máxima eficacia y fuerza vinculante a nivel comunitario.
	\item[d] La directiva no es un acto jurídico vinculante para los Estados miembro.
\end{itemize}

\bigskip
\textbf{42.} La Carta de los Derechos Fundamentales de la Unión Europea es vinculante desde la ratificación:

\begin{itemize}
	\item[a] Del Tratado de Maastricht o Tratado de la Unión Europea.
	\item[b] Del Tratado de Ámsterdam.
	\item[c] Del Tratado de Niza.
	\item[d] Del Tratado de Lisboa.
\end{itemize}

\seccion{Test 2014}

\textbf{41.} El Consejo Europeo adopta sus decisiones:

\begin{enumerate}
	\item[a] Por consenso.
	\item[b] Por consenso o mayoría simple.
	\item[c] Por consenso o mayoría cualificada.
	\item[d] Por mayoría simple o mayoría cualificada.
\end{enumerate}

\seccion{Test 2013}

\textbf{38.} En la Unión Europea el principio de iniciativa legislativa corresponde:

\begin{enumerate}
	\item[a] El Parlamento Europeo.
	\item[b] El Consejo de la Unión Europea.
	\item[c] La Comisión Europea.
	\item[d] El Consejo Europeo.
\end{enumerate}

\seccion{Test 2009}

\textbf{40.} La Comisión Europea de la Unión Europea:

\begin{enumerate}
	\item[a] Es la institución supranacional que, en el seno de la Unión Europea, defiende los intereses de los Estados Miembros.
	\item[b] Es el órgano ejecutivo de la Unión Europea y como tal, y entre otras funciones, gestiona los fondos estructurales y elabora y aplica el presupuesto.
	\item[c] Es la institución supranacional de la Unión Europea encargada de controlar la gestión y la legalidad financiera del presupuesto comunitario.
	\item[d] Es el órgano supranacional encargado de hacer respetar el ordenamiento jurídico comunitario mediante la interpretación y aplicación de las normas comunitarias.
\end{enumerate}

\seccion{Test 2008}

\textbf{42.} ¿Cuál constituye una clara innovación del Acta Única Europea?
\begin{enumerate}
	\item[a] Establecer un procedimiento transparente y democrático para la toma de decisiones en el Tribunal de Justicia Europeo.
	\item[b] La introducción del concepto de ciudadanía de la UE.
	\item[c] Ampliar el voto por mayoría en lugar de la unanimidad.
	\item[d] Ninguna respuesta es correcta.
\end{enumerate}

\textbf{44.} Si comparamos jurídicamente las diferencias entre un reglamento y una directiva, surgen de inmediato:
\begin{enumerate}
	\item[a] Mientras que el reglamento tiene aplicación sobre un sujeto comunitario particular (Estados Miembros y personas físicas o jurídicas), la directiva se aplica uniformemente sobre todos los sujetos comunitarios.
	\item[b] Ambas tienen aplicación directa, pero el reglamento es inmediato, mientras que la directiva exige el visto bueno de los parlamentos nacionales para su aprobación.
	\item[c] El reglamento exige transposición, mientras que la directiva es de aplicación inmediata.
	\item[d] Ninguna de las anteriores.
\end{enumerate}

\seccion{Test 2006}

\textbf{41.} [Pregunta previa a Tratado de Lisboa] Para que pueda adoptarse una decisión, en el procedimiento de decisión por mayoría cualificada del Consejo de Ministros de la Unión Europea actualmente vigente, además de superarse el umbral definido para la mayoría cualificada:
\begin{enumerate}
	\item[a] El voto a favor debe incluir al menos 55\% de los miembros del Consejo, incluyendo al menos a quince de ellos y representando a Estados miembros que tengan al menos el 65\% de la población de la Unión (además una minoría de bloque deberá incluir al menos cuatro miembros del Consejo).
	\item[b] El voto a favor debe incluir a la mayoría de los Estados miembros, incluyendo al menos a trece de ellos y representando a Estados miembros que tengan al menos el 60\% de la población de la Unión.
	\item[c] El voto a favor debe incluir a la mayoría de Estados Miembros y cualquier Estado puede solicitar que se compruebe que la mayoría cualificada representa como mínimo al 62\% de la población.
	\item[d] Ninguna de las anteriores.
\end{enumerate}

\textbf{44.} El Consejo Europeo está compuesto por:
\begin{enumerate}
	\item[a] Los Jefes de Estado o de Gobierno de los estados miembros.
	\item[b] Los Ministros de asuntos exteriores de los Estados Miembros asistidos por un secretariado de apoyo.
	\item[c] Los Jefes de Estado o de Gobierno de los Estados Miembros y el Presidente de la Comisión, asistidos por los Ministros de Asuntos Exteriores y por un miembro de la Comisión.
	\item[d] Ninguna de las Anteriores. 
\end{enumerate}

\seccion{Test 2005}

\textbf{41.} Según el Artículo 133 del Tratado Constitutivo de la Comunidad Europea, los acuerdos comerciales:
\begin{enumerate}
	\item[a] Celebrados por la Comunidad son competencia exclusiva de la Comisión.
	\item[b] Los concluye la Comisión en nombre del Consejo.
	\item[c] Los concluye el Consejo y los Estados Miembros conjuntamente.
	\item[d] Los negocia la Comisión y los concluye el Consejo.
\end{enumerate}

\seccion{Test 2004}
\textbf{44.} ¿Cuál de las siguientes afirmaciones relativas a las Politicas Exterior y de Seguridad Común (PESC) es \textbf{FALSA}?
\begin{enumerate}
	\item[a] La PESC tiene su origen en las formas de cooperación política establecidas por los Estados miembros que condujeron a las disposiciones sobre cooperación en materia de política exterior incluidas en el Acta Única Europea.
	\item[b] la PESC es un ejemplo claro de cooperación intergubernamental. Sus principios de funcionamiento fueron claramente establecidos en el Tratado de Roma, pero han sido modificados de manera sustantiva en el Tratado de Niza.
	\item[c] El principal órgano institucional de la PESC es el Alto Representante de la UE en temas de política exterior y de seguridad común, si bien sus atribuciones no están suficientemente separadas de las ejercidas por otras instituciones comunitarias competentes en esas mismas materias.
	\item[d] Aunque la PESC es un elemento fundamental de las políticas exteriores de la UE, sus objetivos finales están aún muy lejos de ser alcanzados. 
\end{enumerate}

\seccion{Cante 15 de marzo de 2017}
\begin{itemize}
    \item ¿Cuál es la principal característica que diferencia a la UE de otros organismos internacionales?
    \item A menudo se afirma que la UE es un gigante económico y un enano político, ¿cuál es su opinión respecto a esta afirmación?     
    \item ¿Puede aclarar las diferencias entre el Consejo Europeo, el Consejo de Europa y el Consejo de la UE?
\end{itemize}

\seccion{Cante 16 de marzo de 2017}
\begin{itemize}
    \item ¿Qué país que presidía la comisión y comisario español en momento de lanzamiento del euro?
    \item ¿Por qué sin éxitos en actividades de tipo bélico o seguridad y defensa?
    \item ¿Qué problema hay en las instituciones? ¿qué papel tienen en la actual crisis política de nacionalismos, Brexit?
    \item ¿Tiene sentido diferentes sistemas de votación, no sería mejor un sistema de mayorías simples?
    \item ¿Qué es en la UE una minoría de bloqueo?
    \item ¿Cuál es el papel del Parlamento Europeo respecto a los acuerdos negociados con otros países?
    \item ¿Qué características tiene la Unión Europea respecto a otros organismos como el FMI?
    \item ¿Qué es el Consejo de Europa?
    \item No ha mencionado la Constitución Europea.
    \item ¿Que es el ECU?
\end{itemize}

\seccion{Cante 28 de marzo de 2017}
\begin{itemize}
    \item Comente los principios de subsidiariedad y proporcionalidad.
    \item ¿Cuales son los pilares, la finalidad y la estructura del Tratado de la Unión Europea?
    \item ¿Cómo están recogidas las políticas de cohesión en los Tratados?
    \item ¿Quién elabora, ejecuta y verifica el presupuesto?
    \item ¿Qué diferencia la UE del resto de organizaciones internacionales?
\end{itemize}

\notas

\textbf{2018}: \textbf{40.} B \textbf{42.} D

\textbf{2014}: \textbf{41.} C

\textbf{2013}: \textbf{38.} C

\textbf{2009}: \textbf{40.} B

\textbf{2008}: \textbf{42.} C, \textbf{44.} D

\textbf{2006}: \textbf{41.} C, \textbf{44.} C

\textbf{2005}: \textbf{41.} D

\textbf{2004}: \textbf{44}. B



El tema de Juan le da demasiada importancia al recorrido histórico, ocupa más de la mitad del tema. Pero tiene un apartado interesante sobre la estrategia Europa 2020, aunque quizás no sea el tema adecuado para ese apartado.

En la medida de lo posible, tratar de evitar demasiado recorrido histórico. Aburre y quita tiempo para meter contenido de más interés en relación a la situación actual.


\bibliografia

El Agraa, A. \textit{The European Union Illuminated. Its nature, importance and future}. (2015) Ch. 3 Decision-making in the EU

Mirar en Palgrave:
\begin{itemize}
	\item European Central Bank
	\item European Monetary Union
\end{itemize}

\end{document}
