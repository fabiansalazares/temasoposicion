\documentclass{nuevotema}

\tema{3B-34}
\titulo{La OMC. Antecedentes y Organización actual. El GATT y los Acuerdos sobre el comercio de mercancías. Situación actual.}

\begin{document}

\ideaclave

Hay que añadir un apartado sobre el STDF -- Fondo para la Aplicación de Estándares y el Fomento del Comercio de la OMC. Ver \url{http://www.standardsfacility.org/es/qui\%C3\%A9nes-somos}

Ver WTO (2019) Annual Report en carpeta del tema

HABLAR DE Aranceles de Trump en 2018, guerra comercial con China, EU, Canadá, México

El comercio internacional es sujeto de controversias políticas y sesudos análisis económicos desde hace siglos. ¿Debe permitirse la libre circulación de mercancías entre países? ¿Quiénes se benefician y quiénes se ven perjudicados ante una liberalización del comercio? Las respuestas a estas preguntas y sus respectivos debates han fluctuado a lo largo de la historia, pero el desarrollo de la teoría económica en los dos últimos siglos y especialmente desde la obra de Adam Smith tiende a sobreponderar las ventajas del comercio internacional frente a sus perjuicios. Se trata sin embargo de una valoración global de los efectos del comercio, y cuando los países deben decidir de forma individual su postura frente a la apertura comercial, los incentivos no son siempre favorables a la ésta. Así, motivos de economía política interna de los países, incentivos al free-riding y aspectos estratégicos y geopolíticos han hecho del comercio internacional un área fuertemente contenciosa de las relaciones económicas y políticas del mundo moderno. Con el objetivo de suavizar las tensiones que genera la apertura comercial y su ausencia aparece el GATT en 1947 y posteriormente la Organización Mundial de Comercio, en cuyo seno se negocian y se administran un grupo de acuerdos comerciales relativos al comercio de mercancías y servicios. 

La \marcar{OMC} tiene sus \textbf{antecedentes} en el periodo inmediatamente posterior a la Segunda Guerra Mundial. Las tensiones políticas y económicas relacionadas con el comercio internacional provocaron, junto a otros muchos factores, las dos guerras mundiales. Por ello, se plantea en 1944 en el seno de Bretton Woods la creación de una Organización Internacional del Comercio que regule el comercio internacional y que complemente al Fondo Monetario Internacional y al Banco Mundial. En 1947 se aprueba en La Habana su carta fundacional y un tratado provisional para la ordenación de la apertura comercial denominado General Agreement on Trade and Tariffs. El Congreso de los Estados Unidos rechaza ratificar la carta fundacional de la OIC, y el GATT se consolida como una cuasi-institución internacional a la cual se adherirán en las décadas posteriores una amplia mayoría de países de todo el mundo. Esta entidad impulsará un largo y profundo proceso de liberalización comercial del tráfico de mercancías. A lo largo del periodo entre la aprobación del GATT y la cumbre de Marraqués de 1994 se organizaron en el marco del GATT numerosas rondas de negociación. En un primer momento estuvieron centradas en la reducción generalizada de aranceles, pero posteriormente se extendieron también a otras materias referentes al tráfico de mercancías tales como los efectos sobre los países en desarrollo, los trámites aduaneros, sectores concretos que liberalizar, introducción de excepciones, etc... Es habitual denominar estas rondas sucesivas de negociación de acuerdo con el lugar donde se produjeron o donde se inició el proceso. En 1987, la Ronda de Uruguay se inicia con el objetivo de establecer la Organización Mundial de Comercio. El objetivo de esta nueva institución internacional que iniciaría su andadura en 1995 (tras la conclusión de la negociación en Marraqués en 1994) es múltiple: servir de marco de negociación de nuevos acuerdos comerciales, supervisar el cumplimiento de los acuerdos y promover la solución de diferencias respecto a prácticas comerciales. 
 
La \textbf{organización} de la OMC se articula en torno a dos órganos principales: la Conferencia Ministerial formada por los ministros de comercio de todos los miembros y que se reúne cada dos años, y el Consejo General, con poderes similares pero constituida por representantes permanentes en la sede de la OMC en Ginebra. Éste último órgano se reúne también en tanto que Órgano de Examen de las Políticas Comerciales y Órgano de Solución de Diferencias. Del Consejo General penden asimismo una serie de comités adscritos a determinadas materias de negociación. En la actualidad, casi todos los países forman parte de la OMC, salvo un pequeño grupo de países productores de petróleo con pocos incentivos a liberalizar sus mercados y que apenas obtienen beneficio de la liberalización comercial puesto que su mayor exportación --el crudo- no está gravada en casi ninguna economía. 

Las \textbf{actuaciones} de la OMC giran en torno en la organización de Conferencias Ministeriales cada dos años, la provisión de un marco de negociación de nuevos acuerdos, la solución de diferencias comerciales, la supervisión de políticas comerciales llevadas a cabo por los países miembros y la realización de estudios e investigaciones sobre el comercio internacional. Es importante tener en cuenta que algunas materias como la gestión de visados, la regulación laboral y las excepciones culturales quedan fuera del ámbito de la OMC a pesar de haber estado en la lista de materias reguladas por la fallida Organización Internacional del Comercio. 

Una vez presentadas las características de la OMC, la exposición se centra en los \marcar{acuerdos multilaterales} existentes en el seno de la OMC. El GATT, acuerdo fundamental, tiene por objeto regular el comercio de mercancías en general y disminuir de forma general los aranceles que pesan sobre las exportaciones. De su aplicación se derivan una serie de importantes implicaciones respecto a las tasas de arancel y el tratamiento de mercancías extranjeras que entran en un territorio aduanero, así como una serie de excepciones a la aplicación de estas reglas. Posteriormente se examinan otros acuerdos aplicables a todos los miembros de la OMC como el Acuerdo sobre Normas de Origen, el Acuerdo sobre Licencias de Importación, el reciente Acuerdo de Facilitación de Comercio, los acuerdos SPS y TBT sobre medidas sanitarias y fitosanitarias y obstáculos al comercio, respectivamente, el Acuerdo Antidumping y el Acuerdo Antisubvención, el Acuerdo de Salvaguardias, el TRIMS sobre inversiones relacionadas con el comercio (en general, lo que incluye las mercancías) y el Acuerdo Agricola.

Por último, se explican la razón de ser y los problemas de los \marcar{acuerdos plurilaterales}, que vinculan a un subconjunto reducido de los miembros de la OMC. Entre estos acuerdos examinados se cuenta el Acuerdo de Aeronaves Civiles, el de Productos Farmacéuticos, el de Bienes Medioambientales, el de productos químicos o el de Tecnologías de la Información.

La exposición \marcar{concluye} con una recapitulación de lo expuesto y una reflexión final en torno a las críticas que recibe la OMC, sus éxitos y sus perspectivas de evolución futura. El fracaso de la Ronda de Doha, los problemas de gobernanza dado el enorme número de miembros y el sesgo en contra de los países en desarrollo y menos avanzados en determinadas áreas se contraponen al éxito a veces subestimado que supone el hecho de coordinar en una institución la solución de diferencias comerciales y la negociación de acuerdos en una materia tan difícil y sensible como el comercio internacional. En estos momentos, el futuro de la OMC y más generalmente del comercio internacional se encuentra pendiente de los efectos del giro que Estados Unidos ha dado a su postura ante el comercio tras la elección de Donald Trump, el fracaso de los megatratados comerciales y su sustitución por otros menos ambiciosos o con otros impulsores, los acuerdos o falta de ellos respecto al Brexit y la incertidumbre general respecto a la tensión en determinadas áreas del planeta.


\seccion{Preguntas clave}
\begin{itemize}
	\item ¿Qué es la OMC?
	\item ¿Cuál ha sido su evolución?
	\item ¿Cómo se organiza?
	\item ¿Qué hace?
	\item ¿Qué es el GATT?
	\item ¿Qué otros acuerdos rigen el comercio de mercancías?
\end{itemize}

\esquemacorto

\begin{esquema}[enumerate]
	\1[] \marcar{Introducción}
		\2 Contextualización
			\3 Post Segunda Guerra Mundial
			\3 Enfoque multilateralista
			\3 Comercio internacional de mercancías actual
		\2 Objeto
			\3 Qué es la OMC
			\3 Cómo ha evolucionado
			\3 Cómo se organiza
			\3 Qué hace
			\3 Qué es el GATT
			\3 Qué otros acuerdos rigen el comercio de mercancías
		\2 Estructura
			\3 La OMC
			\3 Acuerdos multilaterales sobre mercancías
			\3 Acuerdos plurilaterales
	\1 \marcar{La OMC}
		\2 Función
			\3 Negociación
			\3 Administración
			\3 Solución de diferencias
		\2 Antecedentes
			\3 Negociación del GATT
			\3 Ronda de Ginebra -- 1947
			\3 Ronda de Annecy -- 1949
			\3 Ronda de Torquay -- 1951
			\3 Ronda de Ginebra -- 1956
			\3 Ronda de Dillon -- 1960-61
			\3 Ronda Kennedy -- 1964-1967
			\3 Ronda de Tokio -- 1972-1979
			\3 Ronda de Uruguay -- 1986-1994
			\3 Ronda de Doha
		\2 Organización
			\3 Miembros
			\3 Financiación
			\3 Órganos
			\3[] \underline{Primer Nivel}
			\3[I] Conferencia Ministerial
			\3 \underline{Segundo Nivel}
			\3[II] Consejo General
			\3[] Órgano de Examen de las Políticas Comerciales
			\3[] Órgano de Solución de Diferencias
			\3[] \underline{Tercer Nivel}
			\3[] Dirección General
		\2 Actuaciones
			\3 Conferencias ministeriales
			\3 Negociación de acuerdos
			\3 Solución de diferencias
			\3 Supervisión políticas comerciales
			\3 Asistencia técnica
			\3 Estudios e investigación
			\3 Áreas no gestionadas por OMC
	\1 \marcar{Acuerdos multilaterales sobre mercancías}
		\2 \marcar{G}ATT
			\3 Idea clave
			\3 Disposiciones
			\3 Excepciones a disposiciones generales
		\2 Acuerdo \marcar{A}ntisubvención y de Medidas Compensatorias
			\3 Idea clave
			\3 Subvenciones prohibidas
			\3 Subvenciones recurribles
			\3 No recurribles
		\2 Acuerdo sobre \marcar{L}icencias de importación
			\3 Idea clave
		\2 Acuerdo \marcar{S}PS / Sanitarias y Fitosanitarias
			\3 Idea clave
		\2 Acuerdo\marcar{T}BT / Technical Barriers to Trade
			\3 Idea clave
		\2 Acuerdo \marcar{A}ntidumping
			\3 Idea clave
			\3 Determinación del dumping
			\3 Procedimiento de investigación
			\3 Medidas antidumping
		\2 Acuerdo de \marcar{S}alvaguardias
			\3 Idea clave
			\3 Procedimiento
			\3 Medidas de salvaguardia
			\3 Compensación a terceros
		\2 Acuerdos sobre Normas de \marcar{O}rigen
			\3 Idea clave
			\3 Uso
			\3 Valoración
		\2 Acuerdo sobre Medidas en materia de Inversiones relacionadas con el Comercio (\marcar{T}RIMS)
			\3 Idea clave
		\2 Acuerdo \marcar{A}grícola
			\3 Idea clave
			\3 Acceso a mercado
			\3 Ayuda interna a la agricultura
			\3 Apoyo a la exportación
			\3 Salvaguardas agrícolas
		\2 Acuerdo de \marcar{F}acilitación de comercio
			\3 Idea clave
			\3 Valoración
		\2 Otros acuerdos y decisiones
			\3 Empresas estatales
			\3 Preferencias Generalizadas
	\1 \marcar{Acuerdos plurilaterales}
		\2 Idea clave
			\3 Membresía restringida
			\3 Cobertura sectorial limitada
			\3 Desventajas
		\2 Acuerdos $\left\lbrace \text{QAFIE} \right\rbrace$
			\3 Productos \marcar{Q}uímicos
			\3 \marcar{A}eronaves civiles
			\3 Productos \marcar{F}armacéuticos
			\3 \marcar{I}TA (\textit{Information Technologies Agreement})
			\3 \marcar{E}GA (\textit{Environmental Goods Agreement})
			\3 \marcar{G}PA (\textit{Government Procurement Agreement}
			\3 Carne y lácteos (hasta 1997)
	\1[] \marcar{Conclusión}
		\2 Recapitulación
			\3 La OMC
			\3 Acuerdos multilaterales de mercancías
			\3 Acuerdos plurilaterales
		\2 Idea final
			\3 Críticas a la OMC
			\3 Éxitos
			\3 Evolución futura

\end{esquema}

\esquemalargo
















\begin{esquemal}
	\1[] \marcar{Introducción}
		\2 Contextualización
			\3 Post Segunda Guerra Mundial
				\4 Evitar futuras guerras
				\4 Integración mediante comercio
				\4 Ventana de oportunidad para nuevas instituciones
			\3 Enfoque multilateralista
				\4 Acuerdos entre multitud de estados
				\4[$\Rightarrow$] Sistemas basados en reglas
				\4[$\Rightarrow$] Efectos globales
				\4 Aplicación al comercio internacional
			\3 Comercio internacional de mercancías actual
				\4 OMC y GATT: pilares del sistema
				\4 Multitud de acuerdos accesorios
				\4 Proceso dinámico sujeto a transformaciones
				\4 1947: arancel medio > 30\%
				\4 Actualidad: arancel medio < 5\%
			
		\2 Objeto
			\3 Qué es la OMC
			\3 Cómo ha evolucionado
			\3 Cómo se organiza
			\3 Qué hace
			\3 Qué es el GATT
			\3 Qué otros acuerdos rigen el comercio de mercancías
		\2 Estructura
			\3 La OMC
			\3 Acuerdos multilaterales sobre mercancías
			\3 Acuerdos plurilaterales
	\1 \marcar{La OMC}
		\2 Función
			\3 Negociación
				\4 Formulación de reglas de negociación
				\4 Coordinación agenda de negociación
				\4 Foro de negociaciones
				\4 Asistencia negociación PEDs
			\3 Administración
				\4 Registro de acuerdos comerciales
				\4 Vigilancia de cumplimiento de reglas
				\4 Supervisión general
			\3 Solución de diferencias
				\4 Resolución disputas comerciales
				\4 Búsqueda de soluciones satisfactorias
				\4[] Tratando de evitar represalias, bloqueos, etc...
		\2 Antecedentes\footnote{Ver \url{https://www.wto.org/english/thewto_e/whatis_e/tif_e/fact4_e.htm}.}
			\3 Negociación del GATT
				\4 Carta de la Habana - OIC\footnote{Organización Internacional de Comercio}
				\4[] Adoptada en 1948
				\4[] Sobre comercio, empleo, productos básicos, servicios,
				\4[] Prácticas comerciales, inversiones internacionales...
				\4 General Agreement on Trade and Tariffs
				\4[] Acelerar reducción de aranceles hasta OIC
				\4[] Inicialmente, carácter provisional
				\4[] 1947, semanas antes de Carta de la Habana
				\4[] Entra en vigor 1 de enero de 1948
				\4 Congreso EEUU rechaza Carta de la Habana en 1950
				\4[] $\Rightarrow$ GATT se consolida como permanente
			\3 Ronda de Ginebra -- 1947
				\4 Reducción de aranceles
			\3 Ronda de Annecy -- 1949
				\4 Reducción de aranceles
			\3 Ronda de Torquay -- 1951
				\4 Reducción de aranceles
			\3 Ronda de Ginebra -- 1956
				\4 Reducción de aranceles
			\3 Ronda de Dillon -- 1960-61
				\4 Reducción de aranceles
			\3 Ronda Kennedy -- 1964-1967
				\4 60 países
				\4 Acuerdo Antidumping desarrolla artículo VI de GATT
				\4 Sección sobre desarrollo añadida a GATT
				\4 España entra en GATT en 1963
			\3 Ronda de Tokio -- 1972-1979
				\4 Más de 100 países
				\4 Introducción de la fórmula suiza
				\4[] \fbox{$t_1 = A \cdot \frac{t_0}{A+t_0}$}
				\4[] $A:$ coeficiente de reducción deseado
				\4[] $t_1$: arancel post reducción
				\4[] $t_0$:  arancel antes de reducción
				\4[] $A \to$ también es límite superior arancel aplicable
				\4[] Aplicable a todos los productos
				\4[] Progresiva: aranceles más altos se reducen más
				\4 Códigos SPS\footnote{Acuerdo sobre medidas Sanitarias y Fitosanitarias / \textit{Sanitary and Phytosanitary Measures}.} y TBT\footnote{Acuerdo sobre Obstáculos al Comercio / \textit{Technical Barriers to Trade}.}
				\4[] con carácter plurilateral
				\4[] Uruguay los multilateraliza
				\4 Enabling clause
				\4[] Excepción al principio de NMF
				\4[] Permite preferencias generalizadas a favor de PEDs
			\3 Ronda de Uruguay -- 1986-1994
				\4 Inicio en Punta del Este, Uruguay
				\4[] Fin en Marraqués.
				\4 Creación de la OMC
				\4[] Institución internacional
				\4 Acuerdo GATS\footnote{General Agreement on Trade in Services.}
				\4 Acuerdo TRIPs\footnote{Trade-Related Aspects of Intellectual Property Rights.}
				\4 Acuerdo TRIMs\footnote{Trade-Related Investment Measures}
				\4 Acuerdo sobre la Agricultura
			\3 Ronda de Doha
				\4 Inicio en 2001
				\4 Compromiso países desarrollados:
				\4[] Permitir acceso exportaciones agrícolas PEDs
				\4[] No exigir reducciones adicionales a PEDs
				\4 Objetivos:
				\4[] Mejorar acceso a mercados
				\4[] Apertura comercio agrícola, menor intervención
				\4[] Reducción aranceles
				\4[] Reducción barreras no arancelarias
				\4[] Facilitación de comercio $\to$ Aprobado en Bali
				\4[] $\to$ Ratificado en 2017
				\4 Mejora solución de disputas
				\4[] $\to$ Fuera del single-undertaking
				\4 Fracaso de la negociación
		\2 Organización\footnote{\href{https://www.wto.org/english/thewto_e/whatis_e/tif_e/org1_e.htm}{WTO: Understanding the WTO: The Organization}}
			\3 Miembros
				\4 Artículo XII de GATT-94
				\4[] Artículo XXXIII de GATT-47
				\4 >160 miembros actualmente
				\4 98\% del comercio mundial
				\4 Organización \textit{member-driven}
				\4[] $\to$ Los miembros impulsan
				\4[] $\to$ Director y staff son catalizadores de negociación
				\4[] $\to$ No presentan iniciativas propias
				\4 Muy pocos países fuera:
				\4[] Irak, Irán, Guinea Ecuatorial, Libia, Argelia...
				\4[] Líbano, Siria, Cuba, Serbia, Bosnia,
				\4[] Sudán, Somalia, Etiopía, Bielorrusia...
				\4[] Sobre todo exportadores de petróleo
				\4[] $\to$ Sin incentivos a liberalizar
				\4 Adhesión\footnote{Ver Bratanov sobre adhesión a OMC (en carpeta del tema).}
				\4[] Territorio aduanero diferenciado
				\4[] Procesos largos
				\4[] Requiere voto 2/3 miembros de la WTO
				\4[] $\to$ En la práctica, consenso
				\4 Formación de grupos y bloques\footnote{Ver \url{https://www.wto.org/english/tratop_e/agric_e/negoti_groups_e.htm}}
				\4[] G90: PEDs -- África, CAP y PMAs\footnote{Países Menos Avanzados. Es una clasificación aceptada por todos los miembros y elaborada por Naciones Unidas}
				\4[] ACPs: África, Caribe, Pacífico
				\4[] Africanos
				\4[] Grupo de Cairns: exportadores agrícolas\footnote{Desarrollados y no desarrollados: Australia, Brasil, Argentina, Canadá, Indonesia, Nueva Zelanda...}
				\4[] Economías pequeñas y vulnerables
				\4[] Reciente adhesión: Rusia, China, Kazajistán
				\4[] Cotton-4: exportadores de algodón
				\4[] G-5: China, UE, USA, Brasil, India
				\4 Países en desarrollo
				\4[] Autodefinición
				\4[] Permite trato especial y diferenciado
				\4[] Da lugar a conflicto
				\4[] Sólo ``\textit{país menos avanzado}'' es aceptada en general
			\3 Financiación
				\4 Cuotas anuales de los miembros
				\4 200 millones de francos suizos anuales
				\4 Aportación según:
				\4[] $\to$ peso en comercio internacional
				\4[] $\to$ corregido por desarrollo
			\3 Órganos
			\3[] \underline{Primer Nivel}
			\3[I] Conferencia Ministerial
				\4 Órgano máximo de decisión
				\4 Reunido cada dos años
				\4 Ministros de miembros
			\3 \underline{Segundo Nivel}
			\3[II] Consejo General
				\4 Mismo poder de decisión que Conf. Ministerial
				\4 Representantes de nivel embajador
				\4 Se reúne bajo diferentes reglas:
				\4[] Órgano de Solución de Diferencias
				\4[] Órgano de revisión de Políticas comerciales
				\4 Depende de él:
				\4[] Consejo de Mercancías (goods council)
				\4[] $\to$ Comités sectoriales por debajo
				\4[] Consejo de Servicios
				\4[] $\to$ Comités sectoriales por debajo
				\4[] Consejo TRIPS
				\4[] $\to$ Comités sectoriales por debajo
				\4[$\then$] Gestionan decisiones unilaterales respectivos ámbitos
			\3[] Órgano de Examen de las Políticas Comerciales
				\4 Misma composición que Consejo General
				\4 Revisiones periódicas de políticas comerciales
			\3[] Órgano de Solución de Diferencias
				\4 Misma composición que Consejo General
				\4 Procedimientos enfocados a contener conflictos
			\3[] \underline{Tercer Nivel}
				\4 Dependientes de Consejo General
				\4[] Consejo de Mercancías (goods council)
				\4[] $\to$ Comités sectoriales por debajo
				\4[] $\then$ Acceso a mercado
				\4[] $\then$ Agricultura
				\4[] $\then$ SPS
				\4[] $\then$ Textiles
				\4[] $\then$ TBT
				\4[] $\then$ Subsididios y compensatorias
				\4[] $\then$ Valoración en aduana
				\4[] $\then$ Reglas de origen
				\4[] $\then$ Licencias de importación
				\4[] $\then$ TRIMS
				\4[] $\then$ Salvaguardias
				\4[] $\then$ Comercio de estado
				\4[] Consejo de Servicios
				\4[] $\to$ Comités sectoriales por debajo
				\4[] Consejo TRIPS
				\4[] $\to$ Comités sectoriales por debajo
				\4[] Comité de Negociaciones comerciales
				\4[] $\to$ Centraliza negociaciones en áreas temáticas
				\4[] $\to$ Utilidad incierta tras fracaso de Doha
				\4[$\then$] Gestionan decisiones unilaterales respectivos ámbitos
				\4 Dependientes de Órgano de Solución de Disputas
				\4[] Panel de expertos
				\4[] Órgano de apelación
			\3[] Dirección General
				\4 Director General: Ricardo Azevedo
				\4 Gabinete del Director General
				\4 Subdirector general: Alejandro Jara
				\4 Sin poder ejecutivo propio real
				\4[] Por eso es organización member-driven
				\4 Decisiones en manos de miembros
				\4[] Dirección-General organiza logística neg.
				\4[] Cierto poder de control de agenda
		\2 Actuaciones
			\3 Conferencias ministeriales
				\4 Idea clave
				\4[] Reunión de todos los miembros
				\4[] Negociar orientaciones generales de acuerdos
				\4[] Concluir negociaciones comerciales
				\4[] Decisión sobre cualquier acuerdo
				\4[] Decisiones por consenso hasta ahora
				\4[] $\to$ Pero de iure, posible decisión por mayoría
				\4[] $\then$ Excepciones para miembros: 3/4
				\4[] $\then$ Reformas de acuerdos multi\footnote{Que sin embargo, sólo tienen vigor para aquellos miembros que voten a favor de las reformas.}: 2\3

				\4 Singapur -- 1996
				\4[] Proposición de temas para Ronda de Doha
				\4[] $\to$ Agricultura
				\4[] $\to$ Competencia
				\4[] $\to$ Inversiones
				\4[] $\to$ Contratación pública
				\4[] $\to$ Facilitación de comercio
				\4 Doha -- 2001
				\4[] Adhesión de China
				\4[] Lanzamiento de la Ronda de Doha:
				\4[] -- Agricultura
				\4[] $\to$ Más acceso a mercado
				\4[] $\to$ Eliminación de subsidios a exportación
				\4[] $\to$ Reducción ayudas nacionales
				\4[] $\to$ Seguridad alimentaria
				\4[] $\to$ Desarrollo rural
				\4[] -- NAMA\footnote{Non-agricultural market access.}
				\4[] $\to$ Reducción ulterior de aranceles
				\4[] $\to$ Suavización de perfiles arancelarios
				\4[] $\to$ Eliminación de barreras no arancelarias
				\4[] -- Servicios
				\4[] $\to$ Mejorar acceso a mercado
				\4[] -- Facilitación de comercio
				\4[] $\to$ Procedimientos simplificados
				\4[] $\to$ Menor coste de transacción
				\4[] -- Medio ambiente
				\4[] $\to$ Libre comercio bienes medioambientales
				\4[] $\to$ Acuerdos de protección mínimia
				\4[] -- Denominaciones protegidas en vinos y alcoholes
				\4[] $\to$ Única parte de PIntelectual en Doha
				\4 Cancún -- 2003
				\4[] UE retira algunos temas de la negociación
				\4[] Intento de marco de negociación para concluir Doha
				\4[] Fracaso general de negociaciones
				\4[] $\to$ Sin acuerdos
				\4 Hong Kong -- 2005
				\4[] Avances Ronda de Doha
				\4[] $\to$ Facilitación de comercio
				\4[] $\to$ Contratación pública
				\4[] $\to$ Posible abandonar inversiones y competencia
				\4[] UE aceptó eliminar subsidios exportación
				\4[] $\to$ En fecha futura cierta
				\4 Paquete de julio de 2008
				\4[] A punto de cerrar Ronda de Doha
				\4[] Desacuerdos India--EEUU descarrilan acuerdo
				\4 Ginebra -- 2011
				\4[] Adhesión de Rusia
				\4[] Estancamiento Ronda de Doha
				\4[] Waiver para servicios de PMAs.
				\4 Bali -- 2013
				\4[] Aprobación Acuerdo de Facilitación de Comercio
				\4[] $\to$ Ratificado en 2017
				\4[] Reducción ayudas al algodón
				\4[] $\to$ Algunos países PMA muy dependientes
				\4[] Acuerdos sobre Seguridad Alimentaria. \footnote{Solución temporal a exigencias de la India sobre existencias públicas de bienes agrícolas con objeto de garantizar la seguridad alimentaria de la población. Bajo determinadas condiciones y por lo acordado en esta Ronda, los países en desarrollo no serán denunciados cuando subsidien su agricultura de forma distorsionante aunque excedan los límites permitidos si lo hacen en un contexto de seguridad alimentaria y bajo determinados límites.}
				\4[] Algunos PEDs compran y mantienen stocks de alimentos
				\4[] $\to$ Garantizar alimentación frente a crisis
				\4[] $\to$ Asegurar rentas agrícolas a pesar de crisis
				\4[] $\then$ Introducen distorsiones en el comercio
				\4[] Acuerdo provisional para incluir en caja verde
				\4 Nairobi -- 2015
				\4[] Acuerdo para limitar apoyo a exportación agrícola
				\4 Buenos Aires -- 2017 (11 conferencia)
				\4[] Sin avances
				\4 Astaná -- 2020
			\3 Negociación de acuerdos
				\4 Procesos relativamente largos
				\4 Un país, un voto
				\4 Posible decisión por mayoría simple
				\4[] $\to$ Favorece PEDs
				\4 Hasta ahora, decisión por consenso
				\4[] $\to$ Ningún país expresa oposición
				\4 Desequilibrios poder de negociación
				\4[] Negociaciones informales en privado
				\4[] PEDs sin representaciones permanentes
				\4[] PEDs no pueden asistir a todas
				\4[] PEDs sin staff cualificado
				\4 Acuerdos de adhesión
				\4[] Miembros exigen concesiones
				\4[] Aprobación por 2/3 de la membresía
				\4[] Hasta ahora, por consenso
			\3 Solución de diferencias
				\4 Entre miembros de la OMC, sin empresas
				\4 No se aplica a:
				\4[] Defensa de la competencia
				\4[] Expropiaciones a inversores
				\4 Éxito de la OMC
				\4[] 500 disputas en OMC
				\4[] Tan sólo 200 en GATT-47
				\4[] Mera existencia de OSD disuade
				\4[1] Celebración de consultas
				\4[] Intentar solución amistosa
				\4[] Si no es posible: $\then$
				\4[2] Composición de panel de expertos: \textit{grupo especial}
				\4[] 3/5 expertos de prestigio en ámbito comercial
				\4[] Partes presentan alegaciones, pruebas
				\4[] Encargado de redactar informe
				\4[3] Emisión de informe
				\4[] ¿Las medidas denunciadas son conforme a OMC?
				\4[] Informe dictamina ruptura/cumplimiento de reglas
				\4[4] Apelación ante órgano de apelación
				\4[] $\to$ 7 miembros
				\4[] $\to$ Actualmente sólo 3 miembros
				\4[] $\then$ Estados Unidos bloquea nombramientos
				\4[] $\then$ Situación crítica a partir de 2019
				\4[] Órg. de apelación confirma, desestima, recomienda otras medidas
				\4[5] Adopción por OSD
				\4[] Admite informe de panel o órgano de apelación
				\4[] Sólo puede rechazarse por consenso negativo
				\4[6] Adopción de recomendaciones por acusado
				\4[] $\to$ Concesiones comerciales
				\4[] $\to$ Retirada de legislación no conforme
				\4[] Si el informe así lo preveía
				\4[] Tiempo para cumplir
				\4[] Si no actúa, debe negociar compensación a denunciante
				\4[] Si no hay acuerdo, o no actúa
				\4[7] Permiso para medidas de retorsión
				\4[] Órgano de Solución de Disputas decide
				\4[8] Medidas de retorsión
				\4[] Proporcionales y destinadas a aliviar daño
				\4[] $\to$ Retirada de preferencias previamente concedidas
				\4[] Preferentemente, en sectores similares
				\4[] Retorsiones cruzadas excepcionalmente\footnote{Es decir, retorsiones en sectores diferentes del que provocó originalmente la disputa.}
				\4[] OSD supervisa
			\3 Supervisión políticas comerciales
				\4 Organo de Exámen de las Políticas Comerciales
				\4 A propuesta de Secretaría de la OMC:
				\4[] $\to$ cada 2 años, miembros más grandes
				\4[] Cada 4 o 6 años resto
				\4 Sin implicaciones formales
				\4[] Obliga a explicar cuestiones
				\4[] Aumenta transparencia
			\3 Asistencia técnica
				\4 Asistencia técnica en negociaciones
				\4 Marco Integrado Mejorado
				\4[] Programa mejora capacidad comercial de PMA
				\4[] 47 PMA y 4 graduados
				\4[] Fondos de donantes
				\4 Administración de fondos fiduciarios
				\4 Ayuda técnica aplicación AFC\footnote{Acuerdo para la Facilitación de Comercio.}
				\4 Centro de Comercio Internacional
				\4[] Conjunto con UNCTAD\footnote{United Nations Commission on Trade and Development.}
				\4[] Integrar pymes de PEDs en cadenas de valor
			\3 Estudios e investigación
				\4 Análisis sobre el comercio internacional
				\4 Informe Anual del comercio Mundial
				\4 Informes específicos
			\3 Áreas no gestionadas por OMC
				\4 Visados
				\4 Regulación laboral o medioambiental
				\4 Excepción cultural
				\4 Regulación de las inversiones internacionales
				\4 Apoyo al crédito a la exportación\footnote{Se regula en la OCDE, en el llamado \textit{Consenso de la OCDE}.}
				\4 Defensa de la competencia
				\4 Apenas sin regulación:
				\4[] comercio electrónico, tasas a la exportación
	\1 \marcar{Acuerdos multilaterales sobre mercancías}\footnote{Mnemotécnico: GALSTASOTAF}
		\2 \marcar{G}ATT
			\3 Idea clave
				\4 Contexto
				\4[] GATT de 1947
				\4[] $\to$ Acuerdo pionero
				\4[] $\to$ Marco provisional hasta creación WCO
				\4[] $\to$ Ya incluye no discriminación
				\4[] Consolidación de liberalización multilateral
				\4[] $\to$ Sucesivas rondas de negociación
				\4[] $\to$ Reducción de aranceles
				\4[] Años 80
				\4[] $\to$ Restricciones voluntarias a exportación
				\4[] $\to$ Tendencia resolver disputas bilateralemente
				\4 Objetivos
				\4 Resultados
				\4[] Principio de no discriminación
				\4[] $\to$ NMF
				\4[] $\to$ Trato nacional
				\4[] Arancelización
				\4[] $\to$ Prohibición de cuotas salvo excepciones
				\4[] GATT de 1994
				\4[] $\to$ Integra GATT de 1947
				\4[] $\to$ Añade disposiciones
				\4[] $\to$ Integra acuerdos post-Ronda Kennedy
				\4[] Enorme bajada de arancel medio desde 1947
			\3 Disposiciones
				\4 Principios de No Discriminación
				\4[] I. Principio de Nación Más Favorecida
				\4[] II. Principio de Trato Nacional
				\4 Prohibición de las cuotas
				\4[] Prohibidos límites cuantitativos a exportaciones
				\4[] $\to$ Salvo excepciones en defensa comercial
				\4[] Permitidos contigentes arancelarias
				\4 Aranceles consolidados
				\4[] En el GATT se negocia un arancel máximo
				\4[] Si arancel supera a consolidado
				\4[] $\to$ compensaciones
				\4[] $\to$ retirada de preferencias
				\4 Sistema Armonizado
				\4[] Lista de aranceles consolidados, 6 dígitos
				\4 Libertad de tránsito
				\4 Cálculo de arancel en aduana
				\4[] Permitido:
				\4[] $\to$ Específico
				\4[] $\to$ Ad-valorem
				\4[] $\to$ Mezcla de ambos
				\4[] Valor debe ser cierto
				\4[] $\to$ No permitido inflar base para $\uparrow$ arancel
				\4[] Diferentes métodos de valoración
				\4[] $\to$ A solicitud del importador
				\4[] $\to$ No a discreción de agente de aduanas
				\4[] 1. Valor pagado a vendedor en origen
				\4[] 2. Valor pagado por otros bienes idénticos
				\4[] 3. Valor de bienes similares
				\4[] 4. Valor agregado de bienes similares\footnote{Agregados de acuerdo con métodos estadísticos a nivel de territorio aduanero de importación, divididos por la cantidad en cuestión.}
				\4[] 5. Coste de producción computado\footnote{Menos utilizado y más complicado método de todoslos disponibles. Implica computar el valor de los materiales y la mano de obra, la tasa de beneficio y coste de las ventas, así como otros costes tales como el transporte o el seguro. }
				\4[] Generalmente, sobre precio CIF
				\4[] Algunos países sobre FOB
				\4[] $\to$ P.ej: Estados Unidos
				\4 Prohibición de cuotas
				\4[] Aranceles tan altos como país quiera negociar
				\4 Contingentes arancelarios\footnote{Los contigentes arancelarios son aranceles aplicados a una cantidad determinada de mercancía. O equivalentemente, aranceles por tramos según volumen.} están permitidos
				\4[] Permitidos
			\3 Excepciones a disposiciones generales
				\4 Crisis de balanza de pagos
				\4 Crisis alimentarias o situaciones de carestía
				\4 Situaciones de conflicto
				\4 \textit{Waivers} en Ronda Kennedy\footnote{Establecidos en la parte IV del GATT}
				\4[] Sistema de preferencias generalizadas de la UE
				\4 Industrias nacientes en países en desarrollo
				\4 Uniones Aduaneras y ALC
				\4[] Artículo XXIV\footnote{Ver \url{https://commonslibrary.parliament.uk/brexit/no-deal-brexit-and-wto-article-24-explained/}}
				\4[] $\to$ Cubrir >90\% del comercio entre partes
				\4[] $\to$ Fijar arancel promedio inferior a inicial\footnote{Algunos productos pueden tener arancel superior.}
				\4[] $\to$ Países perjudicados por UA pueden exigir compensaciones
		\2 Acuerdo \marcar{A}ntisubvención y de Medidas Compensatorias\footnote{\url{https://www.wto.org/english/tratop_e/scm_e/subs_e.htm}}
			\3 Idea clave
				\4 Prohibir subvenciones discriminatorias
				\4 Obligación de notificar subvenciones
				\4 Similar antidumping
				\4[] Países y no empresas infractoras en este caso.
				\4 Dos vías de respuesta:
				\4[] -- Denuncia ante OSD
				\4[] -- Imposición de medidas compensatorias
				\4 Tres tipos de subvenciones
			\3 Subvenciones prohibidas
				\4 Vinculadas a objetivos concretos de exportación
				\4 Vinculadas a uso de productos nacionales
				\4 Invocables frente a OSD
			\3 Subvenciones recurribles
				\4 No están prohibidos
				\4[] $\to$ Pero pueden ser recurridos
				\4[] $\to$ Pueden imponerse medidas compensatorias
				\4 Deben causar efectos adversos a país que toma medidas
				\4[] Tres tipos de efectos adversos:
				\4[] i) Hay daño a industria doméstica
				\4[] ii) Hay perjuicio grave a exportaciones
				\4[] $\to$ Hacia el país que implementa subsidio
				\4[] iii) Perjudica acceso al mercado
				\4[] $\to$ AaMercado por reducción de tarifas se anula
				\4 Perjuicio grave a producción de otros miembros
			\3 No recurribles
				\4 Ligadas a I+D, regiones desfavorecidas
				\4 Medioambiente
				\4 No invocables ante OSD
				\4 Posible recomendar abandono
		\2 Acuerdo sobre \marcar{L}icencias de importación
			\3 Idea clave
				\4 Licencias de importación permitidas
				\4[] Automáticas si permitidas
				\4[] $\to$ Simple, transparente y predecible
				\4[] $\then$ Fines de monitorización
				\4[] No automáticas no permitidas
				\4[] $\to$ Equivalen a cuotas
				\4 Discriminación no permitida entre miembros
				\4 Plazos y pautas de concesión
		\2 Acuerdo \marcar{S}PS / Sanitarias y Fitosanitarias
			\3 Idea clave
				\4 Ronda de Tokio (1979)
				\4[] Multilateralizado
				\4 Reglas sobre seguridad de alimentos, animales
				\4[] preservación de vegetales
				\4 Evitar discriminación o arbitrariedad
		\2 Acuerdo\marcar{T}BT / Technical Barriers to Trade
			\3 Idea clave
				\4 Ronda de Tokio (1979)
				\4[] Multilateralizado
				\4 Similar a SPS
				\4 Elminar obstáculos de normas técnicas
		\2 Acuerdo \marcar{A}ntidumping
			\3 Idea clave
				\4 Evitar dumping que perjudique industria nacional
				\4 Caracterizar casos de aplicación posible
				\4 Armonizar regulación antidumping nacional
			\3 Determinación del dumping
				\4 Vendido a menos de su valor normal
				\4[] $\to$  Vendido a menos de su precio en país exportador
			\3 Procedimiento de investigación
				\4 Industria afectada solicita investigación
				\4 País afectado investiga
				\4 Reglas de transparencia y debido proceso
			\3 Medidas antidumping
				\4 Permite medidas antidumping
				\4[] $\to$ Dumping >3\% volumen total importaciones
				\4[] $\to$ O >1\% del mercado interno
				\4 Industria afectada inicia procedimiento
				\4 Permitidas medidas provisionales
				\4 Medidas interpuestas recurribles
		\2 Acuerdo de \marcar{S}alvaguardias\footnote{\href{https://www.wto.org/english/tratop_e/safeg_e/safeg_info_e.htm}{WTO: Safeguard Measures}}
			\3 Idea clave
				\4 Permitir imposición de protección unilateral
				\4[] Medidas del artículo XIX de GATT 94
				\4 Contexto de grave perjuicio a rama nacional
				\4 Establecer procedimientos mínimos
				\4 Reducir uso de medidas de ``zona gris'':
			\3 Procedimiento
				\4 Industria afectada denuncia ante autoridad nacional
				\4 Autoridad nacional investiga
				\4[] Comisión Europea responsable para miembros UE
				\4[] $\to$ Estados Miembros notifican a CEuropea
				\4 Determinación de perjuicio grave
				\4[] Procedimiento transparente
				\4[] Publicidad de actuaciones
				\4 Comité de Salvaguardia
				\4[] Notificar medidas adoptadas
				\4[] Informar de cada fase de investigación y decisiones
				\4[] Notificar medidas adoptadas
				\4[] Foro de discusión de medidas
			\3 Medidas de salvaguardia
				\4 Aplicadas a todos las importaciones del sector
				\4[] Sin atender a su origen
				\4 Retirada de preferencias
				\4 Suspensión de obligaciones de tratados
				\4 Límites temporales
				\4[] Medidas provisionales
				\4[] $\to$ 200 días
				\4[] Tres años sin:
				\4[] $\to$  requerir compensación
				\4[] $\to$ sin posibilidad de recurrir ante OSD
				\4 Restricciones cada vez menores con el tiempo
				\4 Prohibidas medidas de zona gris
				\4[] $\to$ Restricciones voluntarias de exportación
				\4[] $\to$ Acuerdos de compra ordenada
			\3 Compensación a terceros
				\4 Imposición de salvaguardias tiene coste
				\4[] Para estado que impone
				\4[] $\to$ Debe compensar al que sufre
				\4[] $\to$ No se impone por ilegalidad de nadie
				\4 Acuerdo sobre compensación
				\4[] Debe alcanzarse entre miembros afectados
				\4 Ausencia de acuerdo en 30 días
				\4[] Miembros afectados pueden adoptar represalias
				\4[] $\to$ Suspensión de concesiones al que impone salvag.
				\4[] $\to$ Salvo que Consejo de Mercancías rechace represalias
		\2 Acuerdos sobre Normas de \marcar{O}rigen
			\3 Idea clave
				\4 Criterios técnicos de determinación de origen
				\4 Armonización normas de origen
				\4 Pymes soportan coste de certificación
				\4 Armonizar reglas de determinación
				\4[] Eliminar obstáculos innecesarios
			\3 Uso
				\4 Implementar medidas de política comercial
				\4 Determinar quién recibe trato NMF
				\4 Estadísticas de comercio
				\4 Requisitos de etiquetado y marketing
			\3 Valoración
				\4 Relativo fracaso
				\4 ALCs imponen reglas particulares
		\2 Acuerdo sobre Medidas en materia de Inversiones relacionadas con el Comercio (\marcar{T}RIMS)\footnote{\textit{Trade Related Investment Measures}.}
			\3 Idea clave
				\4 Evitar requisitos comerciales discriminatorios
				\4 Prohibición de objetivos mínimos de exportación
				\4 Prohibición de porcentajes mínimos de suministro local
		\2 Acuerdo \marcar{A}grícola
			\3 Idea clave
				\4 Previamente al acuerdo
				\4[$\to$] bienes agri.\footnote{Ver en \textit{Conceptos}.} excluidos del desarme arancelario
				\4 Tres pilares
				\4[] Acceso a mercado
				\4[] Ayuda interna a la agricultura
				\4[] Ayudas a la exportación
				\4 Marco de negociación de otros temas
				\4[] Algodón: PDEs eliminan aranceles post CM Nairobi
			\3 Acceso a mercado
				\4 Arancelización de protección no arancelaria
				\4 Reducción media: 36\% en 6 años para desarrollados
				\4 Reducción media del 24\% para PEDs
			\3 Ayuda interna a la agricultura
				\4 Ayudas \underline{ámbar}
				\4[] Distorsionan comercio internacional
				\4[] Reducción 20\% entre 1995-2001
				\4[] 13\% para PEDs
				\4 Ayudas de caja \underline{azul}
				\4[] Pagos directos para limitar producción
				\4[] Ayudas < 5\% valor total (10\% en PEDs)
				\4 Ayudas de caja \underline{verde}
				\4[] Efectos mínimos en el comercio
				\4[] Fomento investigación
				\4[] Pagos directos a agricultores
				\4[] Programas medioambientales
				\4[] Ayudas disociadas UE son caja verde
				\4 India e industria alimentaria
				\4[] CM Bali 2013:
				\4[] Permitir PEDs fondos de seguridad alimentaria
				\4[] $\to$ Subvenciones a la producción
				\4[] Compromiso temporal de no denunciar a OSD
			\3 Apoyo a la exportación
				\4 Prohibición nuevas ayudas
				\4 Existentes:
				\4[] Notificadas a OMC
				\4[] Porcentaje a reducir: mismo que aranceles agrícolas
				\4 Otras formas de apoyo no reguladas
				\4[] Crédito a la exportación
				\4[] Empresas públicas
				\4[] Ayuda alimentaria como colocación de excedentes
				\4[] $\to$ Limitadas en CM Nairobi 2016
			\3 Salvaguardas agrícolas
				\4 Ver \href{https://www.wto.org/english/tratop_e/agric_e/negs_bkgrnd11_ssg_e.htm}{WTO sobre salvaguardas agrícolas.}
		\2 Acuerdo de \marcar{F}acilitación de comercio
			\3 Idea clave
				\4 Simplificar trámites de aduanas
				\4 Aprobado Bali CM 2013
				\4 Entrada en vigor en 2017
				\4[] Tras ratificación de 2/3 membresía OMC
			\3 Valoración
				\4 Limitación de tasas
				\4 Simplificación envíos
				\4 Permitir \textit{operadores económicos autorizados}
				\4 Algunas reformas sólo voluntarias
				\4 Modelo futuras negociaciones OMC
		\2 Otros acuerdos y decisiones
			\3 Empresas estatales
				\4 Idea clave
				\4[] Empresas públicas pueden distorsionar comercio
				\4[] Utilización para fines prohibidos en acuerdos WTO
				\4[] Ej.: monopolio público distribuidor
				\4[] $\to$ Vende producto extranjero muy caro
				\4[] $\to$ Vende producto nacional más barato
				\4[] $\then$ Equivalente a arancel
				\4[] Obligación de transparencia
				\4 Obligación de información
				\4[] Proveer estadísticas de empresas y productos
				\4[] Medida de CI llevado a cabo por empresas públicas
				\4 Definición de empresa estatal
				\4[] Entendimiento sobre artículo XVIII del GATT
				\4[] Definición de empresa estatal
				\4[] Diferentes tipos de empresa estatal
			\3 Preferencias Generalizadas
				\4 SPG
				\4 SPG+
				\4 EBA
	\1 \marcar{Acuerdos plurilaterales}
		\2 Idea clave
			\3 Membresía restringida
				\4 Negociados en seno de OMC
				\4 Sólo subconjunto de miembros
			\3 Cobertura sectorial limitada
				\4 Por esto, no son uniones aduaneras
				\4[] Por Nación Más Favorecida
				\4[] $\to$ Artículo XXIV de GATT no aplicable
				\4[] $\then$ No firmantes se benefician
				\4[] $\then$ Free-riding
			\3 Desventajas
				\4 Problemas de delimitación de productos
				\4 Free-riding sistemático
				\4[] Argentina, México, Brasil
				\4 Reducción de incentivos a concluir rondas globales
				\4[$\to$] Necesaria masa crítica para nuevos plurilaterales\footnote{De forma que los beneficios de firmar un nuevo acuerdo supere el coste del free-riding que llevan a cabo los no-firmantes.}
		\2 Acuerdos $\left\lbrace \text{QAFIE} \right\rbrace$
			\3 Productos \marcar{Q}uímicos
				\4 No ligado a WTO
				\4 Chemical Tariff Harmonization Agreement
				\4 Sector con enorme importancia en comercio mundial
				\4[] $\to$ 2.2 billones USD
				\4[] $\to$ Incluye farmacéuticos
				\4[] $\to$ UE domina
				\4 Aprobado en Ronda de Uruguay
				\4 Productos químicos orgánicos e inorgánicos
				\4 30 países
				\4 Consolidación aranceles 6,5\%, 5\%, 0\%
				\4 Reducir distorsiones en productos intermedios
				\4[] Por perfiles arancelarios complejos
				\4[] $\to$ En mercado con muy elevada diferenciación
				\4 Aumentar protección efectiva
				\4[] Reducir protección de inputs intermedios
				\4[] $\to$ Aumentar protección en términos de VA
				\4 Incluido en adhesión a WTO para 14 países
				\4 Algunos países fuera del acuerdo mantienen elevada protección
				\4[] Hasta 60\%
				\4[] Introduce enormes distorsiones
				\4 Periodos transitorios
				\4[] Para ajustar aranceles
				\4 Intentos posteriores por aumentar armonización
				\4[] Especificidades para PEDs
				\4[] Periodos transitorios más largos
			\3 \marcar{A}eronaves civiles
				\4 Ligado a acuerdo WTO
				\4 Aprobado tras Ronda de Tokio en 1980
				\4 28 miembros
				\4 Eliminación de aranceles
				\4[] todas aeronaves civiles
				\4[]  Componentes de aeronaves civiles
				\4[] Otros productos relacionados
				\4[] Simuladores de vuelo
				\4[] ...
			\3 Productos \marcar{F}armacéuticos
				\4 Ronda de Uruguay
				\4 Eliminación de aranceles
				\4 Medicamentos humanos y animales
				\4 Principios activos
			\3 \marcar{I}TA (\textit{Information Technologies Agreement})
				\4 Productos de tecnologías de la información
				\4 Ordenadores, impresoras, teléfonos...
				\4 Eliminación de aranceles
			\3 \marcar{E}GA (\textit{Environmental Goods Agreement})
				\4 Células fotovoltaicas
				\4 Molinos Eólicos
				\4 Reducción de contaminación
				\4 En negociación desde 2014
			\3 \marcar{G}PA (\textit{Government Procurement Agreement}
				\4 Ligado a WTO
				\4 No directamente relacionado con bienes
				\4 Pero contratación pública arrastra importaciones
				\4 Objetivos
				\4[] Apertura de mercados de contratación pública
				\4[] Mejora de transparencia y competencia
				\4 Antecedentes
				\4[] Ronda de Tokio 1979 - Código del Sector Público
				\4[] ACP 1994 incorporado junto a Ronda Uruguay
				\4[] Reforma en 2012
				\4[] $\to$ entra en vigor en 2014
				\4 Contenido
				\4[] Transparencia y trato nacional
				\4[] No se aplican a todas las actividades
				\4[] Listas de cobertura de las partes
				\4[] $\to$ Determinar umbrales de aplicación del acuerdo
				\4[] Administrado por Comité de Contratación Pública
				\4[] Resolución de disputas
				\4[] $\to$ Recursos nacionales y mecanismo de solución de diferencias
				\4 Valoración
				\4[] Ejerce presión sobre eficiencia del sector público
				\4[] Contratación pública representa 10-15\% del PIB según OMC
				\4[] 17 partes incluida UE
				\4[] Dificulta proteccionismo en contratación pública
				\4[] Necesidad de adaptar legislación nacional
				\4[] Tendencias proteccionistas en ciertos países
				\4[] Mucho por avanzar
			\3 Carne y lácteos (hasta 1997)
				\4 Finalizados en 1997
				\4 Considerados inefectivos
	\1[] \marcar{Conclusión}
		\2 Recapitulación
			\3 La OMC
				\4 Función
				\4 Antecedentes
				\4 Organización
				\4 Actuaciones
			\3 Acuerdos multilaterales de mercancías
			\3 Acuerdos plurilaterales
		\2 Idea final
			\3 Críticas a la OMC
				\4 Fracaso de la Ronda de Doha
				\4 Single-undertaking difícil tras $\varDelta$ de miembros
				\4 Parálisis en numerosas áreas
				\4 Desencanto entre PEDs y PMAs
				\4 Avanzados también por lentitud
				\4 Falta de diferenciación entre PEDs
				\4 Saturación en OSD
			\3 Éxitos
				\4 Institucionalización del GATT
				\4 Foro de debate consolidado
				\4 OSD disuade de incumplir
			\3 Evolución futura
				\4 Impasse general
				\4 Abandono de TPP, TTIP por Donald Trump
				\4 Creciente peso y asertividad China
				\4 Brexit
				\4 Incertidumbre general
\end{esquemal}































\conceptos

\concepto{Bienes agrícolas}

Los bienes agrícolas incluyen todos los productos de la agricultura y la ganadería, vinos, bebidas, algodón y fibras naturales, pieles para cueros y algunos bienes agrícolas transformados como azúcar y chocolate pero otros bienes procesados (galletas, preparados alimenticios...) conocidos como PATs (Productos Agrícolas Transformados) que se negocian en el marco del GATT en la categoría denominada NAMA (\textit{non-agricultural market access}) están excluidos del desarme arancelario. Los productos de la pesca están también excluidos del Acuerdo Agrícola.

\concepto{CIF}

El precio CIF (\textit{Cost, Insurance, Freight}) incluye los costes de seguro y flete hasta la frontera, aunque estos no estén incluidos en la factura. Cuando esto suceda, puede reconstruir a través de datos de mercancías idénticas o similares que llegan al mismo país.

\concepto{Dumping en el GATT}

En el contexto del GATT, la práctica del dumping se define como la exportación de un bien por debajo de su valor normal que genere o pueda generar un perjuicio a la industria del país importador. El \textit{margen de dumping} se define como la diferencia entre el precio de venta y el valor normal. Cuando este margen supera el 2\%, se permiten medidas antidumping siempre inferiores al propio margen de dumping. Las medidas antidumping tienen una duración máxima de 5 años (\textit{sunset clause}).

<<\textit{Dumping is, in general, a situation of international price discrimination, where the price of a product when sold in the importing country is less than the price of that product in the market of the exporting country. Thus, in the simplest of cases, one identifies dumping simply by comparing prices in two markets. However, the situation is rarely, if ever, that simple, and in most cases it is necessary to undertake a series of complex analytical steps in order to determine the appropriate price in the market of the exporting country (known as the “normal value”) and the appropriate price in the market of the importing country (known as the “export price”) so as to be able to undertake an appropriate comparison.}>>

\concepto{Fracaso de la Ronda de Doha}

\concepto{Fórmula suiza de reducción de aranceles}

La fórmula suiza se utiliza para cálcular el arancel resultante de una ronda de negociación de reducción de aranceles. Toma la forma siguiente

\begin{equation}
T_1 =  A \cdot \frac{T_0}{A + T_0}
\end{equation}

Donde $A$ es el arancel máximo a aplicar, $T_0$ es el arancel inicial y $T_1$ es el arancel resultante.


\concepto{Órgano de Apelación de la OMC}

En el ámbito del proceso de solución de diferencias comerciales, el Órgano de Apelación se encarga de analizar los recursos respecto del informe emitido por el Grupo Especial. El Órgano de Apelación está compuesto por 7 miembros elegidos por periodos de 4 años y que representan aproximadamente la composición de la OMC. Es necesario que un mínimo de tres miembros traten cada caso. En la actualidad, Estados Unidos bloquea la elección de nuevos miembros del Órgano de Apelación, lo que amenaza con paralizar por completo el mecanismo de solución de disputas.


\concepto{Principio de libertad de tránsito}

De acuerdo con este principio, las mercancías que atraviesan un territorio aduanero con destino a otro territorio no pueden ser objeto de obstaculización a la libre circulación, incluidas cualquier comprobación de tipo comercial. Se trata de un principio muy importante para países sin litoral. Permite el comercio de bienes falsificados, si estos están protegidos en el país de destino final.

\concepto{Principio del \textit{principal supplier}}

En una negociación de reducción de aranceles, el principio del \textit{principal supplier} establece que en una negociación en la que un país ofrece reducir sus barreras a un producto determinado, éste espera que el principal productor del producto en cuestión reduzca sus barreras a la importación del producto para el que el primer país (el que ofrecía la reducción inicial) es en su caso el principal proveedor.

<<\textit{Granting a concession to a small supplier implies giving away the
	concession to the principal supplier, since the latter will benefit from
	it due to the MFN rule. The principal supplier is the trading nation
	which benefits the most from a concession and is thus probably
	prepared to offer more reciprocal trade liberalization than a smaller
	supplier would be prepared or able to do.}>>

\concepto{Principio de reciprocidad}

\concepto{Principio de no discriminación / \textit{most-favored nation}}

De acuerdo con este principio, cuando dos países negocian una reducción de aranceles, la reducción debe aplicarse también al resto de países, sin poder discriminar a países fuera de la negociación. Se trata de una de las claves del éxito de la OMC y ha permitido salvaguardar el multilateralismo frente a la tentación de los tratados bilaterales.

Existen excepciones a este principio. Bajo determinadas situaciones tales como el comercio de servicios o la formación de grupos comerciales, los países pueden excluir selectivamente a otros países de la aplicación de un acuerdo comercial.

\concepto{Principio de tratamiento nacional}

De acuerdo con este principio, una vez una importación ha accedido al mercado local, no puede ser tratada de forma diferente a los productos nacionales.

\concepto{Single Undertaking / Principio del compromiso único}

\textit{Nada se acuerda hasta que no se haya acordado todo}. 

\preguntas

\seccion{Test 2018}

\textbf{39.} Señale la respuesta \textbf{CORRECTA} en relación a las normas de funcionamiento de la Organización Mundial de Comercio:

\begin{itemize}
	\item[a] Por el principio de Trato de la Nación Más Favorecida, un país miembro de la OMC debe otorgar a los demás miembros de la OMC el mejor trato que otorgara a otro socio de la OMC.
	\item[b] Por el principio de Trato de la Nación Más Favorecida, un país socio de la OMC debe otorgar a los demás miembros de la OMC el mejor trato que otorgara a uno de sus socios de la Unión Aduanera o Zona de Libre Comercio, aunque no sea miembro de la OMC.
	\item[c] Por el principio de Trato de la Nación Más Favorecida, un país miembro de la OMC debe otorgar a los demás miembros de la OMC el mejor trato que otorgara a uno de sus socios de Unión Aduanera o Zona de Libre Comercio, pero sólo si dicho socio es también miembro de la OMC. 
	\item[d] Las respuestas a) y c) son correctas.
\end{itemize}

\seccion{Test 2017}
\textbf{39.} Según datos de la Organización Mundial de Comercio (OMC):

\begin{itemize}
	\item[a] La Unión Europea es, en términos de valor, el principal exportador mundial de productos agropecuarios.
	\item[b] Desde 2010, la ratio del crecimiento del comercio mundial con respecto al crecimiento del PIB mundial se ha incrementado en relación a los niveles inmediatamente anteriores a la crisis de 2008.
	\item[c] El valor del comercio Sur-Sur, es decir, el valor de las exportaciones de mercancías de las economías en desarrollo a otras economías en desarrollo ha venido descendiendo en el periodo 2012-2015.
	\item[d] En el periodo 2014-2015 el valor de las exportaciones de mercancías de Estados Unidos se sitúa ligeramente por encima del valor de las exportaciones de mercancías de China.
\end{itemize}

\seccion{Test 2015}

\textbf{42.} Señale la respuesta verdadera con respecto a la 10ª y última Conferencia Ministerial de la Organización Mundial del Comercio:
\begin{itemize}
	\item[a] Se adoptó el llamado Paquete de Nairobi, que consta de 6 decisiones ministeriales entre las que destaca el compromiso para eliminar los subsidios a las exportaciones de productos agrícolas.
	\item[b] Se adoptó el llamado Paquete de Nairobi que consta de 6 decisiones ministeriales entre las que destaca el compromiso para eliminar las barreras arancelarias y no arancelarias a las importaciones de productos agrícolas.
	\item[c] Se adoptó el llamado Paquete de Ginebra de 2015 que consta de 6 decisiones ministeriales entre las que destaca el compromiso para eliminar los subsidios a las exportacoines de productos agrícolas.
	\item[d] Se adoptó el llamado Paquete de Ginebra de 2015 que consta de 6 decisiones ministeriales entre las que destaca el compromiso para eliminar las barreras arancelarias y no arancelarias a las importaciones de productos agrícolas.
\end{itemize}

\seccion{Test 2013}

\textbf{46.} Entre las siguientes afirmaciones, escoja la respuesta verdadera:

\begin{itemize}
	\item[a] Todos los miembros de la Organización Mundial de Comercio (OMC), que no sean Países Menos Adelantados, pueden participar en todos los conejos, comités, etc., con excepción del Órgano de Apelación, los grupos especiales de Solución de Diferencias y los comités establecidos en el marco de los Acuerdos Plurilaterales de la OMC.
	\item[b] Todos los miembros de la Organización Mundial de Comercio (OMC) pueden participar en todos los consejos, comités, etc., con excepción del Órgano de Apelación, los grupos especiales de Solución de Diferencias y los comités establecidos en el marco de los Acuerdos Plurilaterales de la OMC.
	\item[c] Todos los miembros de la Organización Mundial de Comercio (OMC) pueden participar en todos los consejos, comités, etc., incluido el Órgano de Apelación. Sin embargo, no todos pueden particpiar en los grupos especiales de Solución de Diferencias ni en los comités establecidos en el marco de los Acuerdos Plurilaterales de la OMC.
	\item[d] Todos los miembros de la Organización Mundial de Comercio (OMC) pueden participar en todos los consejos, comités, etc., incluidos el Órgano de Apelación y los grupos especiales de solución de diferencias. Sin embargo, no todos pueden participar en los comités establecidos en e marco de los Acuerdos Plurilaterales de la OMC.
\end{itemize}

\seccion{Test 2011}

\textbf{31.} A diferencia de las 5 rondas anteriores del GATT, la Ronda de Kennedy y la de Tokyo no se caracterizaron por:

\begin{itemize}
	\item[a] Negociación de tarifas \textit{across-the-board}.
	\item[b] Una agenda centrada en la reducción de las cuotas y la arancelización.
	\item[c] La cataloguización de nuevas barreras no arancelaria.
	\item[d] Mayor atención a los países en desarrollo.
\end{itemize}

\seccion{Test 2009}

\textbf{39.} El sistema de comercio internacional regulado por la Organización Mundial de Comercio frente al sistema de comercio regulado por el Acuerdo General Sobre Aranceles Aduaneros y Comercio (GATT) significa:

\begin{itemize}
	\item[a] La finalización del carácter provisional del GATT que se refleja en que todos los acuerdos que administra la OMC son obligatorios para todos los países miembros, con la excepción de los Acuerdos Plurilaterales.
	\item[b] La creación de un amparo institucional a la financiación oficial al desarrollo.
	\item[c] La adopción de un rango jurídico inferior para la normativa comercial ya que no ha requerido su ratificación por parte de los legislativos nacionales.
	\item[d] La aplicación de la regla de mayoría en todas las decisiones que se adoptan.
\end{itemize}

\seccion{Test 2008}

\textbf{39.} Anteriormente a la celebración de la Ronda Kennedy, las negociaciones comerciales se organizaban:

\begin{itemize}
	\item[a] A través de negociaciones partida por partida arancelaria.
	\item[b] Se acordaba previamente una reducción lineal de todos los aranceles, negociándose las excepciones por partidas arancelarias.
	\item[c] Se pactaba la fórmula suiza de reducciones arancelarias, negociándose después los coeficientes a aplicar.
	\item[d] Las partes contratantes acordaban bilateralmente las reducciones arancelarias recíprocas, sin que fuese de obligado cumplimiento la cláusula de nación más favorecida.
\end{itemize}

\seccion{Test 2007}

\textbf{40.} Indíquese lo que \textbf{NO} proceda: El mecanismo de solución de diferencias de la OMC prevé:
\begin{itemize}
	\item[a] La retirada de la legislación que no es conforme con los acuerdos de la OMC.
	\item[b] La compensación en forma de concesiones comerciales.
	\item[c] La compensación en términos monetarios.
	\item[d] Las sanciones comerciales en el caso de que no se produzcan compensaciones.
\end{itemize}

\textbf{41.} ¿A qué materias no se aplica el mecanismo de Solución de Diferencias de la OMC?

\begin{itemize}
	\item[1.] Disputas sobre derecho de defensa de la competencia.
	\item[2.] Disputas sobre requisitos sanitarios y fitosanitarios.
	\item[3.] Disputas sobre servicios.
	\item[4.] Disputas sobre expropiación a inversores
\end{itemize}

Respuestas:

\begin{itemize}
	\item[a] 1, 2 y 3 
	\item[b] 2 y 3
	\item[c] 1 y 4
	\item[d] 3 y 4
\end{itemize}

\seccion{Test 2006}

\textbf{39.} Señale la respuesta correcta.

Desde la creación de la OMC:

\begin{itemize}
	\item[a] Los flujos de comercio mundial de mercancías y servicios han venido creciendo a una tasa media superior al crecimiento medio del PIB mundial.
	\item[b] El porcentaje de participación del comercio de servicios en el total del comercio mundial, ha superado al porcentaje del comercio de mercancías en dicho total.
	\item[c] Las exportaciones a China han crecido a gran velocidad, hasta el punto de ser en el año 2005 el primer exportador mundial.
	\item[d] Se ha reducido el peso del comercio intra-regional en el volumen global del comercio mundial.
\end{itemize}

\textbf{40.} Señale la respuesta correcta:
\begin{itemize}
	\item[a] El establecimiento de la OMC en 1994 supuso la creación por primera vez de un foro para la resolución de conflictos sobre la interpretación de las normas comerciales.
	\item[b] Con el establecimiento de la OMC en 1994 se integraron al sistema de comercio multilateral los acuerdos sobre servicios, propiedad intelectual y agricultura.
	\item[c] Hasta el establecimiento de la OMC, las negociaciones comerciales del GATT sólo incluían reducciones arancelarias.
	\item[d] En la Ronda Uruguay se cerraron todos los aspectos de las negociaciones sobre indicaciones geográficas.
\end{itemize}

\seccion{Test 2005}

\textbf{37.} Indique cuál de las siguientes afirmaciones es \textbf{FALSA}:
\begin{itemize}
	\item[a] De acuerdo con el Entendimiento de Solución de Diferencias (ESD) de la OMC, la parte demandante puede solicitar el establecimiento de un Grupo Especial (panel) que dictamine sobre el contencioso en cuestión.
	\item[b] Un inconveniente del actual ESD es que para la parte ``condenada'' es fácil conseguir una minoría de bloqueo en el Órgano de Solución de Diferencias e impedir así la adopción de las recomendaciones del Grupo Especial (panel).
	\item[c] El ESD permite en determinadas circunstancias a la parte demandante la adopción de medidas compensatorias incluso en sectores o actividades diferentes de aquel en que se planteó la reclamación.
	\item[d] Existe un Órgano de Apelación al que pueden presentarse recursos de apelación contra los informes y recomendaciones del Grupo Especial (panel).
\end{itemize}

\seccion{Test 2004}

\textbf{36.} De las siguientes afirmaciones:

\begin{itemize}
	\item[i)] La Conferencia Ministerial se concibe como órgano superior de la OMC, debiendo la misma tener lugar como mínimo cada dos años.
	\item[ii)] El GATT surgió en 1947 como tratado provisional entre partes contratantes de cara a las negociaciones dirigidas a la creación de la OIC (Organización Internacional del Comercio).
	\item[iii)] De forma similar al GATT, el GATS defiende la libertad de pagos y transferencias previéndose excepciones a este principio general.
	\item[iv)] De forma similar al GATT, el GATS construye sobre la cláusula de nación más favorecida la primera y principal obligación general del Acuerdo.
\end{itemize}

\begin{itemize}
	\item[a] Todas son verdaderas.
	\item[b] Sólo son verdaderas i) y iii).
	\item[c] Sólo son verdaderas i), iii) y iv).
	\item[d] Todas son falsas.
\end{itemize}


\seccion{9 de marzo de 2017}

\begin{itemize}
    \item Cuales son otras cuestiones sobre las que no avanzado mucho el GATT.
    \item Sobre las compras públicas realizadas por el estado (Ej: Buy American). ¿Qué medidas llevaría usted a cabo? ¿Y en política de defensa de la competencia? ¿ y en competencia desleal en materia de medioambiente?
    \item ¿Existe algún acuerdo de compras públicas en el marco de la OMC?
    \item ¿Cómo le afecta a China el hecho de que no sea reconocida una economía de mercado?
    \item A parte de las tres mencionadas (UA, PMAs y Waiver), ¿Qué otras excepciones existen a lo no discriminación?
    \item ¿Qué sucede cuando un país no cumple con lo pactado?¿Quien pone las sanciones? ¿Quién hace que realmente se cumplan?
    \item ¿A que se refiere cuando en el artículo III hablan de trato nacional?
\end{itemize}

\notas

\textbf{2018:} \textbf{39.} A

\textbf{2017:} \textbf{39.} A

\textbf{2015:} \textbf{42.} A

\textbf{2013:} \textbf{46.} B

\textbf{2011:} \textbf{31.} B

\textbf{2009:} \textbf{39.} A

\textbf{2008:} \textbf{39.} A

\textbf{2007:} \textbf{40.} C \textbf{41.} C

\textbf{2006:} \textbf{39.} A \textbf{40.} B

\textbf{2005:} \textbf{37.} B

\textbf{2004:} \textbf{36.} C


\bibliografia
Mirar en Palgrave:
\begin{itemize}
	\item tariffs
	\item trade policy, political economy of
	\item World Trade Organization
\end{itemize}

Baldwin, R.; Nakatomi, M. \textit{A world without the WTO: what’s at stake?} (2015) CEPR Policy Insight -- En carpeta del tema \marcar{LEER}

Bown, C. Zhang, E. (2019) \textit{Will a US-China trade deal remove or just restructure the massive 2018 tariffs?} PIIE Trade \& Investment Watch, 24 april 2019 -- \url{https://voxeu.org/content/will-us-china-trade-deal-remove-or-just-restructure-massive-2018-tariffs}

Délégation permanente de la France auprès de l’Organisation Mondiale du Commerce. \textit{Brèves de l’OMC – Décembre 2017} (2017) En carpeta del tema. \url{https://www.tresor.economie.gouv.fr/Articles/2017/12/15/breves-de-l-omc-de-decembre-2017-edition-speciale-buenos-aires-don-t-cry-for-me-argentina}

Financial Times. \textit{Trump attack on WTO sparks backlash from members} 10 de diciembre de 2017. \url{https://www.ft.com/content/3e05f236-dd72-11e7-a8a4-0a1e63a52f9c}

Narlikar, A. \textit{The World Trade Organization. A Very Short Introduction} (2005) Oxford University Press

The New York Times. \textit{Global trade after the failure of the Doha Round} \url{https://www.nytimes.com/2016/01/01/opinion/global-trade-after-the-failure-of-the-doha-round.html}


WTO E-Learning. \textit{Introduction to WTO Basic Principles and Rules} -- En carpeta del tema

WTO (2019) \textit{Overview of developments in the international trading environment} Annual report by the Director-General -- En carpeta del tema


\end{document}
