\documentclass{nuevotema}

\tema{3B-3}
\titulo{La empresa y las decisiones de financiación: financiación propia frente a financiación ajena. Política de dividendos y estructura del capital.}

\begin{document}

\ideaclave

Leer \href{https://guiasjuridicas.wolterskluwer.es/Content/Documento.aspx?params=H4sIAAAAAAAEAMtMSbF1jTAAAUMTM2MztbLUouLM_DxbIwMDCwNzA0uQQGZapUt-ckhlQaptWmJOcSoA9wD6lDUAAAA=WKE}{Wolters Kluwers sobre aspectos legales del dividendo en sociedades de capital en España.} 


Existen múltiples definiciones de empresa, que a su vez se corresponden con diferentes justificaciones de su existencia. Una empresa puede concebirse como un proceso tecnológico que transforma inputs en outputs con el fin de proveer a la sociedad con un producto para el cual existe demanda, como un conjunto de contratos que configuran una unidad de decisión diferenciada, o como un proyecto de inversión que utiliza un capital aportado por accionistas y deudores para obtener rentabilidad, entre muchas otras definiciones. Tal obtención de rentabilidad se concreta en la creación de valor para accionistas y deudores. Correlativamente a la asunción de mayores riesgos, la legislación confiere a los accionistas la gestión de la empresa. Desde el punto de vista financiero, la gestión de la empresa se concreta en la toma de decisiones de inversión y financiación. Si la decisión de inversión consiste en estimar la rentabilidad de diferentes proyectos y elegir aquellos que se estimen capaces de crear más valor, la decisión de financiación implica decidir cómo y de qué fuentes obtener el capital necesario para llevar a cabo las inversiones previstas. Además de la simple obtención de fondos, los gestores financieros deben también decidir cómo remunerar a los accionistas, por la condición discrecional del reparto de beneficios a estos proveedores de fondos. El \textbf{objeto} de la exposición es responder a una serie de preguntas básicas en relación a estos aspectos: ¿cómo pueden financiarse las empresas? ¿cómo se financian? ¿cómo deben financiarse? ¿existe una estructura de financiación que maximice el valor de la empresa? ¿cuánto beneficio debe repartirse a los accionistas? ¿cómo debe repartirse? Para ello, la \textbf{estructura} de la exposición se divide en tres partes. En primer lugar, examinamos las peculiaridades de las diferentes \underline{fuentes de financiación}. A continuación tratamos el problema del \underline{diseño de la estructura del capital} y la existencia de una estructura óptima. Por último, analizamos las \underline{políticas de reparto de dividendos} y otras posibles formas de repartir el beneficio y remunerar a los accionistas.

Para financiarse, las empresas emiten activos financieros que colocan en los mercados de capital a cambio de un precio determinado. Estos activos financieros que constituyen las \marcar{fuentes de financiación} de la empresa se pueden clasificar en dos grandes familias: la deuda y el equity o acciones. Los activos de \textbf{deuda} confieren a sus propietarios el derecho a percibir una serie de pagos de cuantía definida o definible a partir de una regla o un índice de referencia que no depende directamente de la marcha general de la empresa. En general, esta corriente de pagos a la que dan lugar tiene un límite temporal, aunque es posible emitir activos de deuda perpetuos cuya corriente de pagos se extiende indefinidamente. Los pagos a los deudores tienen preferencia sobre los pagos a los accionistas, y en caso de liquidación de la empresa, el valor de venta de los activos se destina en primer lugar al pago de la deuda. Existen sin embargo numerosas formas de deuda que introducen variaciones de la prioridad entre diferentes obligaciones, posibilidad de negociación en mercados secundarios como títulos valores, así como garantías de pago. El concepto de bono referencia de forma general los títulos valores que representan deuda y se negocian en mercados financieros. Otras denominaciones son habituales en función del plazo de redención de la deuda, como letras u obligaciones. Los covered bonds y asset-backed securities son títulos de deuda a los cuales se adscriben una serie de activos como garantía de pago en caso de que la empresa no sea capaz de obtener fondos suficientes para hacer frente a sus obligaciones. Otras formas de deuda como el crédito bancario son muy habituales en la práctica y se caracterizan por no ser directamente negociables en mercados secundarios. El \textbf{equity o emisión de acciones} como forma de financiación hace referencia a títulos-valores que representan la propiedad de una parte del capital social de la empresa y confieren el derecho a la parte alícuota correspondiente del patrimonio de la empresa una vez liquidadas todas las obligaciones con deudores, así como al pago de dividendos y otras formas de reparto de beneficios que aprueben los accionistas o los gestores. Este reparto de beneficios no está sujeto, en principio, a regularidad u obligación alguna, de tal manera que dependen en gran medida de la marcha de la empresa y su capacidad parar generar flujos de caja. Más allá de la deuda y el equity, existen numerosas fuentes de financiación \textbf{híbridas} que integran características de las otras dos. Así, encontramos acciones sin derecho a voto que gozan a cambio de prioridad respecto a las acciones ordinarias en el reparto de beneficios, deuda subordinada cuyo pago de intereses está supeditado al pago de intereses de deuda con mayor preferencia, deuda convertible en acciones que se caracteriza por incluir una opción put a favor del emisor de tal manera que éste puede convertir la deuda en equity en cualquier momento, y muchas otras clases. En general, todas aquellas variedades que aumentan la relación entre los pagos y la marcha de la empresa, introducen mayor riesgo y por ello requieren mayores tasas de remuneración. En lo que sigue, adoptaremos una \textbf{perspectiva puramente financiera} respecto del valor de las fuentes de financiación, de tal manera que consideraremos su valor en términos de precio de mercado y no de valor en libros.

Conocidas las posibles formas de financiación a disposición de las empresas, es posible tratar la cuestión de la \marcar{estructura del capital}, su forma adecuada y sus efectos. La estructura del capital de una empresa es la relación entre diferentes formas de financiación, de acuerdo con la proporción que representan sobre el total del capital utilizado. El problema del diseño de una estructura óptima del capital concierne la existencia de una proporción de equity y deuda que maximice el valor para el accionista, y en caso de existir, qué factores influyen en su valor concreto.

El concepto de \textbf{coste del capital} es central al problema de la estructura óptima. El coste del capital es la rentabilidad exigida por los inversores para comprar o mantener una inversión determinada, dadas su características de riesgo, liquidez, imposición fiscal aplicable e inflación. Entendiendo una empresa en su conjunto como proyecto de inversión, el coste del capital de la empresa es la tasa de descuento utilizada para descontar al presente el valor de los flujos de caja generados y tras sumarlos, obtener el valor total de la empresa. El método directo de estimación del coste del capital consiste en tratar de estimar directamente la rentabilidad exigida por el mercado dado el riesgo de los activos en su conjunto. El método indirecto consiste en estimar por separado el coste del capital de la deuda y el equity, y después obtener el coste del capital de la empresa como suma ponderada de acuerdo con los pesos relativos de la deuda y el equity, dando lugar al llamado WACC (\textit{Weighted Average Cost of Capital}). El método implícito consiste en hallar el coste del capital como la tasa de descuento aplicada a la suma de los flujos que el consenso del mercado estima para dar lugar a un precio de la empresa que se asume conocido. Partiendo de este concepto, y entendiendo así el valor del conjunto de la empresa como la suma de flujos de caja descontados al presente que los activos son capaces de generar, el problema del diseño de la estructura óptima del capital consiste en distribuir el origen de las fuentes de financiación de manera que el coste del capital de la empresa en su conjunto sea lo más bajo posible y por ende, aumente el valor de la empresa. 

La \textbf{existencia de una estructura óptima} es el punto de partida del debate. La concepción \underline{tradicional} del problema afirma que es posible encontrar una estructura de capital que minimice efectivamente el coste del capital. Así, una hipotética empresa financiada íntegramente mediante equity, puede reducir su WACC emitiendo deuda a rentabilidad menor que la exigida al equity. A pesar del aumento del riesgo para los accionistas, los partidarios de este modelo afirman que el WACC no se mantiene constante y que efectivamente disminuye hasta alcanzar cierto óptimo, a partir del cual vuelve a crecer. En 1958, Modigliani y Miller publicaron un artículo que introdujo la concepción moderna del problema, caracterizando las condiciones bajo las cuales la estructura de financiación no afecta al coste del capital. Así, en el modelo de Modigliani y Miller, cuando los mercados financieros no están afectados por fricción alguna, no hay impuestos, no hay costes de insolvencia y todos los agentes disponen de la misma información ya sean outsiders o insiders a la operativa de la empresa, el coste del capital se mantendrá constante sea cual sea el grado de endeudamiento. El mecanismo que mantiene constante el coste del capital es el aumento del coste del capital del equity, de tal manera que se compensa la disminución del coste ponderado que resulta de aumentar la proporción de la deuda que se supone requiere una menor rentabilidad. La eliminación de posibilidades de arbitraje sustenta este mecanismo. Si una empresa endeudada tuviese más valor que una empresa con idénticos activos financiada íntegramente con equity, un inversor podría arbitrar la diferencia de valor y obtener beneficio sin incurrir en sacrificio o riesgo alguno, simplemente vendiendo equity de la empresa endeudada, endeudándose y comprando equity de la empresa sin endeudar, de tal manera que con el mismo riesgo aumentase el valor de su inversión. En el discurso de aceptación del Premio Nobel, Miller resumió su aportación como ``\textit{lo importante es el tamaño de la pizza, no las partes en las que se divida}''. 

El objetivo de Modigliani y Miller (1958) no es en todo caso afirmar que efectivamente la estructura de capital no tiene impacto alguno sobre el coste del capital, sino caracterizar los factores que hacen relevante la estructura de financiación. Por ello, el análisis de la estructura óptima de capital en la práctica requiere de la \textbf{relajación de los supuestos} anteriores. \underline{Modigliani y Miller (1963)} modeliza el impacto de un impuesto sobre los beneficios al estilo del impuesto de sociedades. Asumiendo que los intereses son deducibles, la presencia de deuda da lugar al llamado ``escudo fiscal'', que no es sino la reducción de los beneficios y por ende, el aumento de los fondos disponibles para remunerar a los proveedores de fondos que capitalizados, conforman el valor de la empresa. De esta forma, se llega a la conclusión de que las empresas deben financiarse con la mayor cantidad de deuda posible dados los supuestos de M-M (1958) y la presencia de impuesto sobre los beneficios. En la misma línea, \underline{Miller (1977)} introduce un refinamiento adicional: la presencia de impuestos sobre la renta pueden compensar el efecto de los impuestos sobre los beneficios y hacer irrelevante la estructura óptima del capital. Para ello, el impuesto sobre los intereses habría de compensar el efecto conjunto de los impuestos sobre los dividendos y sobre los beneficios. La introducción de \underline{costes de insolvencia} implica tener en cuenta la posibilidad de que los activos de la empresa no pueden, en caso de liquidación, ser reutilizados a coste nulo. Este factor aumenta las posibles pérdidas en caso de impago, e implica que mayores proporciones de deuda efectivamente pueden aumentar el WACC en relación a una financiación sin deuda. Así, cuando existen costes de insolvencia significativos, la estructura financiera puede ser relevante. En la práctica, este factor tiene una gran relevancia y supone el principal freno al endeudamiento excesivo para financiar crecimiento. Los \underline{problemas de agencia} aparecen cuando la estructura de incentivos de un agente le induce a tomar decisiones tendentes a realizar objetivos diferentes para los cuales ha sido designado. La estructura del capital puede ayudar a reducir los problemas de agencia, creando valor y por ello, haciendo importante su determinación. La presencia de deuda en el capital de una empresa obliga a los managers a elegir proyectos que generen flujos de caja con mayor certeza, y los empujan a desechar proyectos de rentabilidad más dudosa más alineados con sus propios objetivos que con los del accionista. Un caso extremo de este fenómeno son los LBO (\textit{leveraged buy-out}) en los que los managers se convierten en dueños de la empresa tras endeudarse fuertemente. Las \underline{asimetrías de información} pueden interaccionar con la presencia de deuda de tal manera que la estructura de capital sea relevante. Dado que los managers y los accionistas --en menor medida- disponen de información con la que el público no cuenta, cambios en la estructura de capital pueden servir para señalizar la marcha de la empresa. Por ejemplo, los managers pueden endeudarse para señalizar optimismo respecto del comportamiento futuro. Ampliaciones de capital pueden indicar a los inversores externos o accionistas minoritarios que la acción está sobrevalorada. De forma general, si los inversores disponen de menos información exigirán mayores rentabilidades y la empresa disminuirá su valor.

Si la estructura de financiación es relevante a efectos del coste del capital, cabe interrogarse acerca del \textbf{diseño de la estructura de capital}. El factor más importante a tener en cuenta es que el coste del capital depende, en equilibrio, del riesgo de la inversión y no de la forma concreta o del origen de la financiación. Además, es necesario tener en todo momento presente que el coste contable de capital no es el coste verdadero, que debe tener en cuenta necesariamente el coste de oportunidad de la inversión. Debe también considerarse el impacto de la deuda sobre la liquidez que toda empresa necesita para hacer frente a obligaciones de corto plazo o aprovechar oportunidades de inversión inesperadas. De forma general, la estructura óptima depende así de la firma y del momento temporal. En ella influyen factores macroeconómicos, la aversión al riesgo de inversores y managers en momentos determinados, factores que incentivan la sobreinversión como la inflación y de forma inversa o el desapalancamiento como es el caso de la deflación. En la práctica, la búsqueda de flexibilidad y la preservación de la calificación de la deuda son los factores principales en empresas de tamaño medio o grande. El ciclo vital de la compañía caracteriza a menudo las decisiones de financiación. Así, en la fase de start-up las empresas afrontan riesgos operativos muy elevados que hacen difícil el acceso a la deuda y se financian casi totalmente vía equity. En mercados maduros, empresas con perspectivas estables tienden a preferir la financiación vía deuda. La teoría del \underline{pecking order} o del orden jerárquico propuesta por Myers (1984) y Donaldson (1981), caracteriza las decisiones de financiación en relación con el esfuerzo que suponen para los managers como resultado del disclosure necesario y el marketing necesario para vender activos financieros. Según el modelo, los managers prefieren así la autofinanciación a cualquier otra forma de financiación, y subsecuentemente la deuda a las ampliaciones de capital. De esta preferencia por la autofinanciación se deriva el problema la valoración del coste del capital correcto: los managers pueden no estimar correctamente el coste del capital de las inversiones que llevan a cabo, lo que resultará a medio/largo plazo en un valor en libro superior al valor de mercado.

La estructura del capital tiene \textbf{efectos sobre indicadores de la empresa}. Una proporción de deuda más alta eleva el riesgo de liquidez y afecta negativamente a la solvencia por imponer mayores compromisos de pago que requieren, a su vez, volúmenes más elevados de generación de caja. La deuda reduce además el beneficio neto. El beneficio neto por acción, sin embargo, tiende a aumentar con mayores proporciones de deuda siempre que la rentabilidad del capital utilizado sea superior al interés de la deuda. Es necesario tener presente que esto no implica necesariamente creación de valor, ya que el riesgo del equity aumenta también.

De forma similar a la estructura de financiación, la política de \marcar{reparto de beneficios} genera debate en cuanto a su capacidad de creación de valor. El reparto de beneficios no es sino la devolución del beneficio neto a los accionistas de la empresa, en alguna de sus manifestaciones: dividendos, recompra de acciones o reducciones de capital a través de redención de acciones. El \textit{pay-out ratio} es la medida básica de cuantificación del reparto de beneficios y consiste simplemente en el cociente entre beneficio repartido y beneficio neto. En Europa, la media ha oscilado en los últimos años en torno al 50\%. El ciclo económico es en la práctica un factor que afecta de forma notable en este ratio: en situaciones de crisis tiende a aumentar por la bajada del beneficio neto, y a disminuir en fases alcistas del ciclo por razones opuestas. La financiación propia que se refirió anteriormente no es sino la utilización de ese beneficio neto no repartido a los accionistas para financiar nuevos proyectos de inversión. La retención de fondos y su mantenimiento en forma de activo libre de riesgo disminuye el coste del capital de la empresa pero no es en general una inversión preferida por los inversores porque ellos mismos pueden llevarla a cabo. Sin embargo, la retención de beneficios puede efectivamente crear valor si se utiliza en proyectos que generen rentabilidad por encima del coste del capital, si permite mantener la liquidez y la flexibilidad de la empresa o resulta en aprovechamiento de ventajas fiscales.

¿De qué depende así la decisión de \textbf{repartir o no repartir el beneficio}? Modigliani y Miller (1961) caracterizó el problema en términos similares al artículo pionero de 1958. Dada una política de inversión fija y conocida, inexistencia de costes de transacción ni problemas de agencia, misma información de insiders y outsiders y mercados en equilibrio, los autores demuestran que la política de distribución de beneficios no crea ni destruye valor y sólo supone un cambio en la composición de las carteras de los inversores. Sin embargo, la relajación de estos supuestos puede dar lugar a que efectivamente se cree valor repartiendo beneficios de algunas formas o reteniendo beneficios. Miller y Scholes (1978) analizan el efecto de \underline{diferentes tipos impositivos} para dividendos y ganancias de capital. Si los dividendos se gravan más que las ganancias de capital, la política de reparto de beneficios puede efectivamente crear o destruir valor. Sin embargo, la evidencia empírica muestra que aunque efectivamente se produce un impacto sobre el valor, no se trata de un determinante de primer orden a la hora de determinar la política de reparto de beneficios de las empresas. Los \underline{problemas de agencia} pueden convertir en relevante la política de reparto en lo que respecta a la creación de valor. Una empresa que reparta beneficios de forma recurrente implicará recursos más habituales al mercado de capitales para financiar nuevas inversiones, lo que redundará en un alineamiento de los incentivos de directivos y accionistas. Evidencia empírica apunta a que efectivamente se produce este hecho. La \underline{señalización} es también uno de los posibles canales de creación de valor. Dado que los directivos tienen más información que los accionistas, el reparto de dividendos puede utilizarse para enviar información al mercado acerca de las perspectivas futuras. Así, un mantenimiento del dividendo puede servir para señalizar optimismo y capacidad de mantener la generación de caja a pesar de un bache en el presente. Los directivos tratan en ocasiones de aprovechar esto para enviar señales falsas al mercado. La evidencia empírica lo corrobora, mostrando como en general los mercados reaccionan más ante bajadas del dividendos que ante subidas. En cualquier caso, no parece ser un determinante de primer orden. En ciertos contextos se produce el fenómeno contrario. Por ejemplo, la supresión del dividendo por Telefónica en 1998 para financiar su expansión en Latinoamérica provocó un fuerte aumento del precio de las acciones. 

Más allá del debate sobre repartir o no repartir beneficios, y una vez decidido que es adecuado repartirlos, se plantea el problema sobre \textbf{cómo repartirlo}. Repartir dividendos es la primera de las opciones y consiste en abonar a los accionistas una cantidad proporcional al número de acciones de manera periódica si se trata del llamado dividendo ordinario, o de manera extraordinaria cuando la empresa quiere repartir un beneficio derivado de una ganancia que previsiblemente no se repetirá. El reparto de dividendos mantiene la estructura del accionariado, de forma que no se modifican las relaciones de poder en el seno de la junta de accionistas. En general, las empresas tienden a preferir mantener el dividendo para no generar un sentimiento desfavorable en los inversores, aunque este hecho se modula de forma acorde con la ciclicidad del sector considerado. Así, por ejemplo empresas en sectores fuertemente cíclicos muestran dividendos más volátiles. Es deseable que las políticas de dividendos sean lo más transparentes y creíbles para los inversores como sean posible. Existen además distintas variantes del dividendo ordinario tales como dividendos a cuenta que permiten suavizar el perfil de tesorería de la empresa, dividendos en especie pagados en acciones (scrip dividend) o dividendo preferente pagado a accionistas que han mantenido la propiedad durante un periodo mínimo (sólo algunas jurisdicciones lo permiten) o el dividendo preferente pagado a acciones preferentes sin derecho a voto. La \underline{recompra de acciones o share buyback}(o la redención de acciones, con un efecto muy similar) tiene a priori los mismos efectos que el reparto de dividendos sobre la estructura económica de la empresa. Sin embargo, desde el punto de vista financiero, existen claras diferencias. La recompra de acciones puede afectar a la estructura de accionariado, modificando el poder relativo de los diferentes propietarios. Permite más flexibilidad que el reparto de dividendos ya que no genera expectativas de mantenimiento y transmite menos información a accionistas. Además, afecta a determinados ratios y precios tales como el ratio de endeudamiento o el valor de opciones sobre acciones (en algunas jurisdicciones es obligatorio corregir el strike de opciones tras una recompra). 

A lo largo de la exposición hemos examinado las diferentes fuentes de financiación disponibles, analizado la decisión de financiación que corresponde a la utilización de unas formas u otras y por último, tratado la política de dividendos y más generalmente, el reparto de beneficios. Como mensaje esencial, cabe destacar el papel de Modigliani y Miller enmarcando el problema de la relación entre inversión y financiación, y tener en cuenta que numerosos factores que van más allá de su modelo tienen un enorme impacto, tales como la existencia de fricciones, shocks macroeconómicos, así como factores legales y psicológicos.


\seccion{Preguntas clave}
\begin{itemize}
    \item ¿Cómo pueden financiarse las empresas?
    \item ¿Cómo se financian?
    \item ¿Cómo deben financiarse?
    \item ¿Existe una estructura óptima del capital?
    \item ¿Cuánto dividendo deben repartir?
    \item ¿Cómo pueden repartirlo?
    \item ¿Cómo deben repartirlo?
\end{itemize}

\esquemacorto

\begin{esquema}[enumerate]
	\1[] \marcar{Introducción}
		\2 Contextualización
			\3 Definición de empresa
			\3 Papel empresas economía
			\3 Concepto de inversión
			\3 Financiación de la inversión
		\2 Objeto
			\3 Cómo pueden financiarse las empresas
			\3 Cómo se financian
			\3 Cómo deben financiarse
			\3 Existencia de una estructura óptima del capital
			\3 Cuánto beneficio repartir
			\3 Cómo repartirlo
		\2 Estructura
			\3 Fuentes de financiación
			\3 Estructura del capital
			\3 Política de dividendos
	\1 \marcar{Fuentes de financiación}
		\2 Fondos ajenos
			\3 Idea clave
			\3 Características
			\3 Modalidades
		\2 Fondos propios
			\3 Idea clave
			\3 Características
			\3 Modalidades
		\2 Formas híbridas
			\3 Acciones preferentes
			\3 Deuda subordinada
			\3 Deuda convertible
			\3 Deuda perpetua
		\2 Perspectiva financiera
			\3 Valor contable
			\3 Valor de mercado
			\3 Valor relevante
	\1 \marcar{Estructura del capital}
		\2 Idea clave
			\3 Definición de estructura del capital
			\3 Estructura óptima
		\2 Coste del capital
			\3 Idea clave
			\3 Métodos de cálculo
			\3 Determinantes de rentabilidad exigida
		\2 Estructura financiera óptima
			\3 Definición
			\3 Debate
		\2 Modigliani y Miller (1958)
			\3 Idea clave
			\3 Proposición I
			\3 Proposición II
			\3 Mecanismo
			\3 Explicación basada en arbitraje
		\2 Relajación de supuestos
			\3 Impuesto de sociedades
			\3 Impuesto de la renta de las personas físicas (Miller 1977)
			\3 Costes de insolvencia
			\3 Problemas de agencia
			\3 Señalización
			\3 Negociación colectiva
		\2 Diseño de la estructura del capital
			\3 Principios
			\3 Estructura óptima depende de la firma y el momento
			\3 Criterios utilizados en la práctica
			\3 Pecking order
		\2 Efectos estructura sobre indicadores
			\3 Liquidez
			\3 Solvencia
			\3 Beneficios
			\3 Beneficio por acción
	\1 \marcar{Política de dividendos}
		\2 Usos del beneficio
			\3 Reparto de beneficios
			\3 Reinversión de beneficios
			\3 Creación de valor vía política de dividendos
		\2 Repartir o no repartir
			\3 Modigliani-Miller (1961)
			\3 Impuestos
			\3 Costes de agencia
			\3 Señalización
		\2 Cómo repartirlo
			\3 Dividendo
			\3 Recompra de acciones / share buyback
	\1[] \marcar{Conclusión}
		\2 Recapitulación
			\3 Fuentes de financiación
			\3 Análisis decisiones de financiación
			\3 Análisis política de dividendos
		\2 Idea final
			\3 Inversión inseparable de financiación
			\3 Factores relevantes

\end{esquema}

\esquemalargo

\begin{esquemal}
	\1[] \marcar{Introducción}
		\2 Contextualización
			\3 Definición de empresa
				\4 Múltiples formas de concebir
				\4 Caja negra que transforma inputs
				\4[] Visión de teoría microeconómica clásica
				\4[] Elige inputs dentro de planes posibles
				\4[] $\to$ Produce output
				\4[] $\then$ Beneficios
				\4 Nexo común de contratos
				\4[] Alchian y Demsetz
				\4[] Empresa como instrumento de coordinación
				\4[] Múltiples procesos productivos
				\4[] $\to$ Unidos por un agente contractual común
				\4[] $\then$ Empresa como vínculo comón
				\4[] $\then$ Empresa como herramienta de coordinación
				\4 Empresa como estructura jerárquica
				\4[] Coase (1937)
				\4[] Dentro de empresa, reglas de mercado no rigen
				\4[] $\to$ Manager ordenan a trabajadores actuar
				\4[] $\to$ Manager ordena uso de capital
				\4 Williamson: enfoque de costes de transacción
				\4[] Análisis formal de costes de Coase (1937)
				\4[] Empresa sirve para reducir costes de transacción
				\4[] $\to$ Por contratos repetidos en el tiempo
				\4[] $\to$ Por incertidumbre respecto resultado
			\3 Papel empresas economía
				\4 Producción de bienes y servicios
				\4 Ordenación capital y trabajo
				\4 Generación de valor
				\4 Oportunidades de inversión
			\3 Concepto de inversión
				\4 Sacrificio presente para beneficio futuro
				\4 Decisión de inversión sujeta a problemas
				\4[] Cómo elegir entre inversiones alternativas
				\4[] Cómo estimar beneficio futuro
				\4[] Cómo valorar diferentes inversiones posibles
			\3 Financiación de la inversión
				\4 Sacrificio presente requiere financiación
				\4[] ¿De donde obtener los fondos?
				\4 Financiación es inversión en sí misma
				\4[] Desde el punto de vista que provee los fondos
				\4 Financiación implica costes
				\4[] Para el que la necesita
				\4 Decisiones de financiación
				\4 Creación de valor
				\4[] Debate de largo alcance
				\4[] Decisiones de financiación como creación de valor
				\4[] $\to$ ¿Posible o no?
		\2 Objeto
			\3 Cómo pueden financiarse las empresas
			\3 Cómo se financian
			\3 Cómo deben financiarse
			\3 Existencia de una estructura óptima del capital
				\4 Si existe, de qué depende
			\3 Cuánto beneficio repartir
			\3 Cómo repartirlo
		\2 Estructura
			\3 Fuentes de financiación
			\3 Estructura del capital
			\3 Política de dividendos
	\1 \marcar{Fuentes de financiación}
		\2 Fondos ajenos
			\3 Idea clave
				\4 Capital pertenece a terceros
				\4 Necesario remunerar y devolver
				\4 Cuantía y límite temporal de amortización
				\4[] Definidos en momento de provisión de fondos
			\3 Características
				\4 Pagos independientes de marcha de la empresa
				\4 Límite temporal
				\4 Prioridad frente a equity
				\4 Coste contable
				\4[] Intereses a pagar
				\4[] Nominal a redimir
				\4 Coste financiero
				\4[] Coste de refinanciar deuda
				\4 Pequeña diferencia entre coste contable y financiero
			\3 Modalidades
				\4 Deuda bancaria
				\4 Papel comercial
				\4 Pagarés corporativos
				\4 Bonos
				\4 Covered bonds
				\4 Bonos verdes
				\4 Arrendamiento financiero
				\4 Crowdfunding
		\2 Fondos propios
			\3 Idea clave
				\4 Capital pertenece a terceros
				\4 Sin obligación de devolver o remunerar
				\4 Confiere derechos residuales
				\4[] Gestión
				\4[] Remanente tras liquidación
			\3 Características
				\4 Pagos dependen de marcha de la empresa
				\4[] Si compañía insolvente, accionistas pierden todo
				\4[] Reparto de dividendos discrecional
				\4[] Subordinada a pago de deuda
				\4[] $\Rightarrow$ Derechos políticos como consecuencia
				\4 Sin compromiso de amortización
				\4 Coste contable
				\4[] Sin coste explícito
				\4[] $\to$ Dividendos no son obligatorios
				\4[] $\to$ Redenciones no son obligatorias
				\4 Coste financiero
				\4[] Retorno esperado de acciones con características similares
				\4 Diferencia grande entre coste contable y financiero
				\4[] Elevada en emisión de acciones
				\4[] $\to$ Coste contable explicitado en momento de emisión
				\4[] Muy elevada en reinversión de beneficios
				\4[] $\to$ Coste contable sin explicitar
			\3 Modalidades
				\4 Emisión de acciones
				\4[] Venta de participaciones en el capital social
				\4[] Sin obligación de redención
				\4 Autofinanciación
				\4[] Reinversión de beneficios acumulados
				\4[] No requiere autorización de todos accionistas
				\4[] Aumento de capital implícito
		\2 Formas híbridas
			\3 Acciones preferentes
				\4 Acciones sin derecho a voto
				\4 Dividendos prioritarios frente a acciones comunes
				\4 Coste de financiación
				\4[] Ligeramente menor acciones ordinarias
			\3 Deuda subordinada
				\4 Menor preferencia que deuda
				\4 Mayor riesgo para proveedores de capital
				\4 Mayor coste de emisión
				\4 Remuneración ligada a resultados
				\4 Coste de financiación
				\4[] Superior a deuda convencional
			\3 Deuda convertible
				\4 Empresa compra opción put sobre equity
				\4[] Si opción in-the-money, conversión a equity
				\4 Mayor riesgo que deuda
				\4 Contingent Convertibles (cocos)
				\4 Coste de financiación
				\4[] YTM + valor de opción implícita
			\3 Deuda perpetua
				\4 Paga cupón periódico
				\4 Sin fecha de amortización
				\4[] Sin amortización obligatoria en la práctica
				\4 No confiere derechos políticos
		\2 Perspectiva financiera
			\3 Valor contable
				\4 Valor de adquisición
				\4 Reflejado en libros
			\3 Valor de mercado
				\4 Reflejo del valor intrínseco
			\3 Valor relevante
				\4 Valor de mercado es relevante
				\4[] A la hora de diseñar estructura óptima
	\1 \marcar{Estructura del capital}
		\2 Idea clave
			\3 Definición de estructura del capital
				\4 Relación entre fuentes de financiación
				\4[] ¿Cuánta financiación mediante equity?
				\4[] ¿Cuánta financiación mediante deuda?
			\3 Estructura óptima
				\4 ¿Cómo es una estructura óptima?
				\4 ¿Existe una estructura óptima?
				\4 ¿Qué cambia con una u otra estructura?
		\2 Coste del capital
			\3 Idea clave
				\4 Rentabilidad exigida al capital utilizado
				\4 Tasa de descuento rentas generadas por la empresa
			\3 Métodos de cálculo
				\4 Directo
				\4[] Calcular rentabilidad exigida a activos
				\4[] Métodos de valoración de activos
				\4[] $\to$ P. ej.: CAPM
				\4 Indirecto: WACC
				\4[] Calcular rentabilidades exigidas a FPropios y FAjenos
				\4[] Suma ponderada en función de valores relativos
				\4[] $k = k_e \cdot \frac{V_E}{V_E + V_D} + k_d \cdot \frac{V_D}{V_E + V_D}$
				\4 Implícito
				\4[] Conocidos:
				\4[] $\to$ Valor de la empresa
				\4[] $\to$ Flujos de caja libres después de impuestos
				\4[] Calcular tasa de descuento
				\4[] $\to$ Que iguala valor de empresa y FCF descontados
			\3 Determinantes de rentabilidad exigida
				\4 Riesgo
				\4 Liquidez
				\4 Impuestos
				\4 Inflación
		\2 Estructura financiera óptima
			\3 Definición
				\4 Estructura financiera
				\4[] $\to$ que maximiza el valor de empresa?
				\4[] $\to$ que minimiza el coste del capital utilizado?
			\3 Debate
				\4 ¿Existe una estructura financiera óptima?
				\4[$\to$] Variando proporciones de deuda y equity:
				\4[] es posible crear valor?
				\4 Enfoque tradicional
				\4[] Existe estructura óptima del capital
				\4[] $\to$ Maximiza $V = V_E + V_D$
				\4[] Posible reducir $k$ aumentando deuda
				\4[] Aumento de deuda:
				\4[] $\to$  NO conlleva aumento de $k_e$
				\4[] \quad A pesar de aumento del riesgo
				\4[] $\Rightarrow$ $k$ NO se mantiene constante
				\4[] \grafica{waccmodelotradicional}
				\4 Enfoque moderno / Modigliani-Miller / arbitraje:
				\4[] Modigliani y Miller (1958)
				\4[] Estructura de capital no afecta coste del capital
				\4[] Miller, discurso del Nobel
				\4[] $\to$ ``Lo importante es tamaño de la pizza
				\4[] \quad no en cómo se divida''
		\2 Modigliani y Miller (1958)
			\3 Idea clave
				\4 Supuestos
				\4[] Sólo dos instrumentos: deuda y equity
				\4[] Mercados financieros sin fricciones
				\4[] No hay impuestos
				\4[] No hay costes de insolvencia
				\4[] Insiders y outsiders tienen misma información
				\4[$\then$] Coste del capital utilizado es constante
				\4[] WACC no varía aunque más deuda
			\3 Proposición I\footnote{ ``\textit{the average cost of capital, to any firm 'IS completely independent of its capital structure and is equal to the capitalization rate of a pure equity stream of its class}'', Modigliani y Miller (1958).}
				\4 Valor de firma no depende de estructura
				\4[] Dos firmas A y B
				\4[] $\to$ Con idénticos activos y negocio
				\4[] $\to$ Sólo diferente estructura del capital
				\4[] $\then$ \fbox{ $V_A = V_{e,A} + V_{d,A} = V_{e,B} + V_{d,B} = V_B$}
			\3 Proposición II\footnote{ ``\textit{... The expected yield of a share of stock is equal to the appropriate capitalization rate $\rho_k$ for a pure equity stream in the class, plus a premium related to financial risk equal to the debt-to-equity ratio times the spread between $\rho_k$ and $r$.}'', Modigliani y Miller (1958).}
				\4 Rdto. exigido al equity
				\4[] Igual al de empresa no apalancada
				\4[] $\to$ Más prima por riesgo por apalancamiento
				\4[] $\then$ Igual a diferencial coste de $k$ y coste financiero de deuda
				\4[] $\then$ Multiplicado por apalancamiento.
				\4[] Formulación\footnote{Efectivamente, esta fórmula es análoga a la del rendimiento financiero en relación al rendimiento económico: $R_F = R_E + (R_E - i) \frac{D}{\text{PN}}$. Sin embargo, ya que se ha adoptado una perspectiva financiera, no define en términos del valor contable.}:
				\4[] \fbox{$k_e = k + (k-k_d) \frac{V_D}{V_E}$}
			\3 Mecanismo
				\4 Aumento de deuda
				\4[] Aumenta el riesgo financiero
				\4[] $\to$ Aumenta riesgo para accionistas
				\4[] $\to$ aumento de $k_e$
				\4[] $\Rightarrow$ mantiene $k$ constante
				\4[] \grafica{waccmodiglianimiller}
			\3 Explicación basada en arbitraje
				\4 Si aumento de deuda puede reducir $k$
				\4[] $\to$ Existirán posibilidades de arbitraje
				\4 Empresas con deuda y $k$ más bajo
				\4[] $\to$ Tienen más valor de mercado
				\4 Estrategia rentable\footnote{Vernimmen, pág. 594.}
				\4[] $\to$ Vender acciones en empresas con deuda
				\4[] $\to$ Comprar empresa idéntica sin deuda
				\4[] $\Rightarrow$ Misma rentabilidad del capital empleado
				\4[] $\Rightarrow$ Mayor inversión en equity
		\2 Relajación de supuestos
			\3 Impuesto de sociedades
				\4 Modigliani y Miller (1963)
				\4[] Cuando los beneficios están gravados
				\4[] $\to$ La deuda forma un escudo fiscal
				\4 Necesarios beneficios para sacar provecho
				\4 Gráficos tarta: dividendos, intereses, impuestos\footnote{Dibujar tres gráficos: en el primero, la "tarta" se reparte entre dividendos e intereses, con los intereses tomando 1/4 y los dividendos los 3/4 restantes. En el segundo, con la misma estructura del capital pero con impuesto de sociedades al 33\%, los intereses se llevan el mismo porcentaje, los impuestos con 1/4 y los dividendos los 2/4 restantes. En el tercer gráfico, los intereses se llevan 3/4 del beneficio operativo, los dividendos $1/4\cdot2/3=1/6$ y los impuestos $1/4\cdot 1/3=1/12$.}
				\4 El escudo fiscal aumenta valor de empresa
				\4[] Aumento de valor:
				\4[] $\to$ Reducción anual de impuestos capitalizada
				\4[] Si reducción de impuestos cada año:
				\4[] $\to$ $\Delta \text{Valor} = \frac{T_c \cdot k_D \cdot V_D}{k_D} = T_c \cdot V_D$
				\4[] También posible descontar a coste del equity
			\3 Impuesto de la renta de las personas físicas (Miller 1977)
				\4 Otros impuestos pueden compensar escudo fiscal:
				\4[] $\to$ Impuesto sobre intereses percibidos
				\4[] $\to$ Impuesto sobre dividendos
				\4[$\then$] Posible que estructura fin. sea irrelevante
				\4[$\then$] Posible que empresas no maximicen deuda
			\3 Costes de insolvencia
				\4 Activos no pueden ser reutilizados a coste nulo
				\4[] Aumenta riesgo de impago de la deuda
				\4[] Crece prima de riesgo de deuda
				\4[] $\to$  Quiebra costosa en sí misma
				\4[$\then$] Mayor WACC con más deuda
				\4[$\then$] Estructura financiera puede ser relevante
				\4[] $\to$ Si mercados no valoran correctamente
			\3 Problemas de agencia
				\4 Deuda puede ser herramienta de control
				\4[] Managers obligados a generar caja
				\4[] Alineación de incentivos managers-accionistas
				\4 Ejemplo extremo: LBOs
				\4[] $\to$ Managers son dueños
				\4[] $\to$ Apalancamiento muy elevado
				\4[] $\Rightarrow$ Managers se lo juegan todo
				\4[] $\Rightarrow$ Incentivo a generar caja
			\3 Señalización
				\4 Managers tienen información privada
				\4[] Conocen mejor riesgo y rentabilidad de proyectos
				\4 Deben financiar proyectos
				\4 Fuente de financiación es señal que envían a mercado
				\4[] Sobre información privada de que disponen
				\4 Financiación vía deuda
				\4[] Señalizan optimismo
				\4 Financiación vía equity
				\4[] Señalizan incertidumbre
				\4 Señal es tomada en serio sólo si mentir es costoso
				\4[] P.ej.:
				\4[] $\to$ Información negativa sobre inversiones posibles
				\4[] $\to$ Aumentar deuda para financiar
				\4[] $\then$ Aumenta riesgo de quiebra y perder empleo
				\4[] $\then$ Costoso mentir
				\4 Ejecutivos señalizan optimismo con deuda
				\4[] Sujetos a riesgo de pérdida de empleo si quiebra
				\4 Ampliaciones de capital
				\4[] Señalan acción posiblemente sobrevalorada
				\4[] $\to$ Precio equity cae con ampliación de capital
				\4 Menos información
				\4[] $\then$ Mayor riesgo percibido
				\4[] $\then$ Mayor coste
				\4 Evidencia empírica favorable
				\4[] Ampliaciones de capital reducen valor de empresa
				\4[] $\to$ Evidencia robusta
				\4[] $\then$ Interpretable como aumento de riesgo percibido
				\4[] Managers que venden paquetes de acciones
				\4[] $\to$ Señal muy desfavorable de información privada
				\4[] $\then$ Managers deben informar de acciones poseídas
				\4 ``Skin in the game''
			\3 Negociación colectiva
				\4 Matsa (2010)
				\4 EFinanciación puede ser variable estratégica
				\4 Niveles elevados de liquidez y equity
				\4[] Permiten a trabajadores aumentar demandas
				\4[] $\to$ Amenaza de quiebra por empresa es menos creíble
				\4 Niveles elevados de deuda
				\4[] Amenaza de quiebra y despido es creíble
				\4[] $\to$ Aumenta poder de negociación frente a sindicatos
				\4 Moderación de demandas salariales
				\4[] Permite aumentar beneficio operativo y neto
		\2 Diseño de la estructura del capital
			\3 Principios
				\4 Coste de capital de un proyecto de inversión depende
				\4[] $\to$ del riesgo de la inversión
				\4[] no de:
				\4[] $\to$ la forma
				\4[] $\to$ ni el origen de la financiación
				\4 Diferenciar entre:
				\4[] $\to$ coste aparente (contable)
				\4[] $\to$ coste verdadero
				\4 Impacto de la deuda sobre liquidez
				\4[] Tener en cuenta tensiones de tesorería y liquidez
			\3 Estructura óptima depende de la firma y el momento
				\4 Factores macro
				\4 Aversión al riesgo
				\4 Inflación:
				\4[] tendencia a sobreinversión vía deuda
				\4 Deflación:
				\4[] desapalancamiento
				\4 Equity:
				\4[] garantía frente a impago de deuda
				\4[] indica cuanto riesgo accionistas quieren asumir
			\3 Criterios utilizados en la práctica\footnote{Mencionar Graham \& Harvey (2001).}
				\4 Preservar flexibilidad
				\4[] Margen para acometer nuevas inversiones
				\4[] Remunerar accionistas
				\4 Preservar rating
				\4[] Mantener acceso a financiación ajena
				\4 Ciclo vital de la compañía
				\4[] Start-ups:
				\4[] $\to$ Alto riesgo operativo
				\4[] $\to$ Muy difícil acceso a deuda
				\4[] $\to$ Financiación vía equity
				\4[] Empresas estables en mercados maduros
				\4[] $\to$ Bajo riesgo operativo
				\4[] $\to$ Acceso a financiación vía deuda
				\4 Preferencias de los accionistas
				\4 Oportunidades por cotización de la acción
				\4 Aversión al riesgo de managers
			\3 Pecking order
				\4 Myers y Majluf (1984)
				\4 Contexto
				\4[] Mercados financieros eficientes
				\4[] $\to$ Capital mismo coste que VAN de inversiones
				\4[] En la práctica, inversores que proveen capital
				\4[] $\to$ No tienen toda la información sobre inversiones
				\4[] Asimetrías de información y señalización
				\4[] $\to$ Deuda señaliza información privada positiva
				\4[] $\then$ Menor coste de financiación
				\4[] $\then$ Skin in the game
				\4[] $\then$ Money where their mouth is
				\4 Objetivos
				\4[] Caracterizar decisión entre alternativas de financiación
				\4[] $\to$ En función de costes relativos
				\4[] $\to$ Valorando costes de información
				\4 Resultados
				\4[] Si proveedores de capital disponen de menos información
				\4[] $\to$ Exigen mayor coste/menos riesgo
				\4[] Para decidir entre fuentes de financiación empresas consideran
				\4[] $\to$ Coste de proveer información
				\4[] $\to$ Coste de financiación depende de información
				\4 Formulación
				\4[] Managers disponen de información privada
				\4[] $\to$ Conocen mejor rentabilidad y riesgo de proyectos
				\4[] Autofinanciación no requiere provisión de información
				\4[] $\to$ Ya se dispone de fondos
				\4[] $\to$ Basta con no devolver a proveedores como dividendos
				\4[] Managers minimizan coste de financiación
				\4[] $\to$ Valorando coste de información asimétrica
				\4[] 1. Autofinanciación
				\4[] $\to$ Sin coste en términos de información a proveer
				\4[] 2. Deuda
				\4[] $\to$ Proveedores de K asumen menos riesgo
				\4[] $\to$ Necesario proveer menos información
				\4[] $\to$ Implica señalización de información privada
				\4[] 3. Acciones
				\4[] $\to$ Proveedores de K asumen más riesgo
				\4[] $\to$ Necesario proveer más información privada
				\4[] $\to$ Mayor coste de capital
				\4 Valoración
				\4[] Buena contrastación empírica
				\4[] Algunas excepciones: empresas tecnológicas
				\4[] $\to$ Deuda es de hecho muy arriesgada
				\4[] $\then$ Activos muy específicos y poco tangibles
				\4[] $\then$ Activos ilíquidos si insolvencia
				\4[] $\then$ Poco capital de gestión
		\2 Efectos estructura sobre indicadores
			\3 Liquidez
				\4 Deuda aumenta riesgo de liquidez
				\4[] Flujos de caja comprometidos a acreedores
				\4 En crisis general, solventes no pueden financiarse
			\3 Solvencia
				\4 Deuda afecta negativamente
				\4[] Valor de equity más volátil que deuda
				\4[] Más deuda reduce margen de absorción de pérdidas
				\4[] $\to$ Vía caída del PN
				\4[$\to$] Mayores compromisos de pago
				\4[$\to$] Mayor necesidad de generación de caja
			\3 Beneficios
				\4 Apalancamiento aumenta volatilidad de beneficio neto
				\4[] Lo reduce vía coste de financiación
				\4[] Lo aumenta si activos más rentables que coste
			\3 Beneficio por acción
				\4 Aumenta con apalancamiento:
				\4[] si interés es:
				\4[] $\to$ inferior a rentabilidad tras impuestos
				\4 A cambio de mayor riesgo
				\4 No implica creación de valor
				\4[] En la medida en que se cumpla M-M 58
	\1 \marcar{Política de dividendos}
		\2 Usos del beneficio
			\3 Reparto de beneficios
				\4 Beneficio neto devuelto a accionistas
				\4[] Diferentes formas
				\4[] $\to$ Dividendo
				\4[] $\to$ Recompra de acciones
				\4[] $\to$ Redención de acciones / redución de capital
				\4 Payout ratio
				\4[] $\frac{\text{Dividendo}}{\text{Beneficio neto}}$
				\4 Media europea en 2016:
				\4[] Payout ratio del 55\%
				\4 Ciclo económico
				\4[] Empresas tienden a suavizar dividendos
				\4[] Menos fluctuación que beneficio neto
				\4[] $\to$ Payout ratio sube en crisis
				\4[] $\to$ Payout ratio baja en expansiones
			\3 Reinversión de beneficios
				\4 Utilización de beneficio neto
				\4[] Para financiar nuevos proyectos
				\4[] $\to$ Equivale a ampliación de capital obligatoria
				\4 Retener fondos sin invertir
				\4[] Equivale a invertir a activo libre de riesgo
				\4[] $\Rightarrow$ Reduce coste del capital
				\4 ¿Es lo que quieren los inversores?
			\3 Creación de valor vía política de dividendos
				\4 Si empresa retiene a rent. libre de riesgo
				\4[] $\to$ Inversores pueden hacerlo ellos mismos
				\4 Existen razones para retener beneficios
				\4[] $\to$ Aprovechamiento de oportunidades $> k_e$
				\4[] $\to$ Mantener liquidez y flexibilidad
				\4[] $\to$ Razones fiscales
				\4[] $\Rightarrow$ Creación de valor
		\2 Repartir o no repartir
			\3 Modigliani-Miller (1961)
				\4 Supuestos
				\4[] Política de inversión fija y conocida
				\4[] Sin impuestos
				\4[] Sin costes de transacción
				\4[] Sin problemas de agencia
				\4[] Misma información insiders y outsiders
				\4[] Mercados en equilibrio
				\4 Políticas de dividendos no crean valor
				\4[] Accionistas indiferentes a política de dividendos
				\4[] $\to$ Dividendos sólo cambian composición de cartera
				\4[] $\to$ Dividendos y reinvertir: mismo valor
				\4 Ejemplo:
				\4[] Coste del capital del equity: 10\%
				\4[] Valor del equity: 100 €
				\4[] Beneficio neto: 10 €
				\4[] Si reinvierte 10 € a 10\%:
				\4[] $\to$ Aumento de valor de equity de 10€
				\4[] Si distribuye 10 € como dividendos:
				\4[] $\to$ Aumento de 10 € de cash
				\4 Relajación de supuestos
				\4[] Puede implicar creación de valor
				\4[] $\to$ Distintas políticas pueden crear + o -- valor
			\3 Impuestos
				\4 Miller y Scholes (1978)
				\4 Diferente tributación dividendos-ganancias de capital
				\4[] Si dividendos gravados distinto que ganancias de capital
				\4[$\Rightarrow$] Política de dividendos crea/destruye valor
				\4 Evidencia empírica favorable
				\4 Dudoso que sea determinante de primer orden
			\3 Costes de agencia
				\4 Dividendos obligan a recurrir a mercados de capital
				\4 Alinean incentivos directivos y accionistas
				\4[] Más dividendo implica más endeudamiento relativo
				\4[] $\to$ Más riesgo para deudores
				\4[] $\to$ Más riesgo para managers
				\4[$\Rightarrow$] Política de dividendos crea valor
				\4 Evidencia confirma\footnote{Por ejemplo, La Porta et al. (2000) muestra como mayor protección de los accionistas aumenta reparto de dividendos en una muestra de 33 países.}
			\3 Señalización
				\4 Directivos tienen más información que propietarios
				\4[] Dividendos señalizan buenas perspectivas
				\4[] $\to$ Por ejemplo, mantener a pesar de bache
				\4[] Señalan capacidad de empresa para disponer fondos
				\4 Evidencia empírica:
				\4[] $\to$ mayor reacción a bajadas que a subidas
				\4[] $\to$ descarta que sea determinante de primer orden
				\4 Excepciones en ciertos contextos:
				\4[] Telefónica suprime dividendo en 1998
				\4[] para financiar expansión en Latinoamérica
				\4[] $\Rightarrow$ Acciones suben 9\%
		\2 Cómo repartirlo
			\3 Dividendo
				\4 Abonado a accionistas
				\4 Generalmente ordinario
				\4 Raramente, dividendos extraordinarios
				\4 Tendencia a no reducirse
				\4 Menor flexibilidad
				\4 Mantiene estructura del accionariado
				\4 Fiscalidad puede desincentivar
				\4 Tipos de dividendo
				\4[] Efectivo ordinario
				\4[] A cuenta
				\4[] $\to$ Antes de conocer beneficio neto
				\4[] Scrip dividend
				\4[] $\to$ En forma de acciones de la empresa
			\3 Recompra de acciones / share buyback
				\4 Fundamentalmente mismos efectos
				\4 Posible cambio estructura del accionariado
				\4 Mayor flexibilidad
				\4[] Percibido como dividendo extraordinario
				\4[] No genera expectativas de mantenimiento
				\4[] Transmite menos información a accionistas
				\4 Afecta determinados ratios y precios
				\4[] Ratio de endeudamiento
				\4[] Estructura del accionariado
				\4[] Afecta a precios de opciones sobre acciones
				\4[] $\to$ Determinados países necesario recalcular strike
				\4[] Popularidad de stock options
				\4[] $\to$ aumento de popularidad de buybacks
	\1[] \marcar{Conclusión}
		\2 Recapitulación
			\3 Fuentes de financiación
				\4 Deuda
				\4 Fondos propios
			\3 Análisis decisiones de financiación
				\4 Utilizar deuda o equity
				\4 Análisis estructura óptima del capital
				\4 Existencia
				\4 Supuestos necesarios
			\3 Análisis política de dividendos
				\4 Cuánto repartir
				\4 Cuando es importante la cantidad a repartir
				\4 Dividendos o recompra de acciones
		\2 Idea final
			\3 Inversión inseparable de financiación
				\4 Modigliani-Miller: enmarcan problema
				\4 Relajación de supuestos: abordarlo
			\3 Factores relevantes
				\4 Eficiencia de los mercados financieros
				\4 Fricciones
				\4 Macroeconómicos
				\4 Legales
				\4 Psicológicos
\end{esquemal}























\graficas

\begin{axis}{4}{Coste del capital de una empresa en función del ratio de endeudamiento, según el modelo tradicional.}{$\frac{V_D}{V}$}{$k$, $k_e$, $k_d$}{waccmodelotradicional}

% coste de la deuda

\node[below] at (3.7,1.6){$k_d$};
\draw[-] (0,1.2) to [out=0, in=210](4,2);

% coste del equity

\node[above] at (3.2,3.5){$k_e$};
\draw[-] (0,2) to [out=-5, in=220](4,4);

% WACC

\node[above, color=blue] at (2.2,1.8){$k$};
\draw[-, color=blue] (0,2) to [out=-10, in=190](4,2);

% WACC mínimo

\draw[dashed] (2,1.8) -- (2,0);
\node[below] at (2,0){\tiny Estructura óptima};

\end{axis}


\begin{axis}{4}{Coste del capital de una empresa en función del ratio de endeudamiento, según el modelo de Modigliani-Miller.}{$\frac{V_D}{V}$}{$k$, $k_e$, $k_d$}{waccmodiglianimiller}
	
% coste de la deuda

\node[below] at (3.7,1.6){$k_d$};
\draw[-] (0,1.2) to [out=0, in=210](4,2);

% coste del equity

\node[above] at (3.2,3.5){$k_e$};
\draw[-] (0,2) to [out=5, in=180](4,3.5);

% WACC

\node[above, color=blue] at (2.2,2.05){$k$};
\draw[-, color=blue] (0,2) -- (4,2);

\end{axis}

En el modelo de Modigliani-Miller, el coste del equity varía de tal forma que el coste del capital de la empresa es constante aunque varíe el ratio de endeudamiento.


\conceptos

\concepto{Financiación externa-interna y ajena-propia} Existe cierto grado de confusión en relación a la emisión de acciones. La emisión de acciones es sin duda financiación externa, pero no es financiación ajena sino propia porque contribuye a los fondos propios.

\preguntas


\seccion{Test 2018}

\textbf{24.} ¿Cuál de los siguientes supuestos es necesario para basar las proposiciones iniciales de la teoría de Modigliani-Miller sobre la estructura financiera de la empresa?

\begin{itemize}
	\item[a] El coste de los recursos propios es independiente del ratio de endeudamiento.
	\item[b] La empresa mantiene la misma política de dividendos a lo largo del período considerado.
	\item[c] La empresa obtiene toda su financiación del mercado de capitales.
	\item[d] No existen costes asociados al riesgo de insolvencia de la empresa.
\end{itemize}

\seccion{Test 2017}
\textbf{24.} De acuerdo con la teoría de Modigliani-Miller, elija la respuesta correcta:

\begin{itemize}
	\item[a] Debe atenderse a consideraciones de tipo fiscal, comercial, institucional y de movilidad de factores productivos para determinar la estructura financiera óptima de cualquier empresa.
	\item[b] Su proposición I (1958) establece que el valor de la empresa es independiente de la estructura de capital como consecuencia del equilibrio en mercados de capitales perfectos.
	\item[c] Su proposición II establece que el coste del capital ajeno es función lineal creciente del coeficiente de endeudamiento.
	\item[d] Ninguna de las anteriores es correcta.
\end{itemize}

\seccion{Test 2016}
\textbf{38.} Cuál de las siguientes no es una buena razón para que una empresa compre sus acciones propias:

\begin{enumerate}
	\item[a] Que los gestores consideren que un dividendo de caja estable es lo más conveniente para los accionistas.
	\item[b] Que los gestores consideren que las acciones propias están sobrevaloradas.
	\item[c] Que los gestores quieran aumentar la proporción de deuda en su estructura de capital.
	\item[d] Ninguna de las anteriores.
\end{enumerate}

\seccion{Test 2015}
\textbf{25.} Señale la respuesta \textbf{incorrecta} relativa a la teoría de la estructura financiera de la empresa de Modigliani-Miller:

\begin{enumerate}
	\item[a] Bajo el conjunto de supuestos en los que se basan las proposiciones iniciales de Modigliani-Miller, el coste de los recursos propios de la empresa crece a medida que aumenta su ratio de endeudamiento. En este contexto, la política de dividendos de la empresa puede contribuir a maximizar su valor.
	\item[b] Bajo el conjunto de supuestos en los que se basan las proposiciones iniciales de Modigliani-Miller, el valor de la empresa es independiente de su estructura financiera. Este resultado puede entenderse como una condición de ausencia de estrategias de arbitraje de un inversor que deba elegir entre dos empresas con idéntica corriente de flujos de caja que únicamente se diferencien por su estructura financiera.
	\item[c] Si al conjunto de supuestos en los que se basan las proposiciones iniciales de Modigliani-Miller se añade la existencia de un impuesto sobre los beneficios netos empresariales, el valor de la empresa no dependerá únicamente de la capacidad económica de sus activos para generar beneficios.
	\item[d] Si al conjunto de supuestos en los que se basan las proposiciones iniciales de Modigliani-Miller se añade la existencia de un impuesto sobre los beneficios netos empresariales y de costes asociados a la insolvencia de la empresa, cuanto mayores sean los costes asociados a la insolvencia menor será la ratio óptima de endeudamiento. 
	
\end{enumerate}

\seccion{Test 2014}
\textbf{26.} Modigliani y Miller consideran que lo que determina el valor de la firma, y por tanto el valor de las acciones, es:

\begin{enumerate}
	\item[a] El dividendo.
	\item[b] La capacidad de generar dividendos futuros.
	\item[c] La capacidad de generar rentas de sus activos.
	\item[d] El incremento de liquidez de las acciones.
\end{enumerate}

\seccion{Test 2013}
\textbf{26.} Durante el presente ejercicio, la autofinanciación de una empresa cuyo coeficiente de endeudamiento (deudas/pasivo total) es 0,6 ha aumentado en 100 millones. Suponiendo que la empresa va a mismo coeficiente de endeudamiento, el montante de recursos financieros totales inducido por este aumento será:

\begin{enumerate}
	\item[a] 250 millones.
	\item[b] 166,6 millones.
	\item[c] 160 millones.
	\item[d] Ninguna de las anteriores.
\end{enumerate}

\seccion{Test 2008}
\textbf{24.} Una de las proposiciones originales de Mogliani-Miller (MM) es:

\begin{enumerate}
	\item[a] Supone que un incremento del endeudamiento no afecta al tipo de interés de la deuda de la empresa.
	\item[b] Supone que si la empresa no está endeudada la rentabilidad esperada de sus acciones es inferior a la rentabilidad esperada de sus activos.
	\item[c] Implica que el valor de la empresa dependerá únicamente de la rentabilidad de sus proyectos de inversión.
	\item[d] Establece que la rentabilidad esperada para los accionistas es independiente de las decisiones de financiación que tome la empresa.
\end{enumerate}

\seccion{Test 2007}
\textbf{25.} El Modelo de Modigliani-Miller \textbf{AFIRMA} que:

\begin{enumerate}
	\item[a] Versión 2 (1963) (con impuesto de sociedades, sin costes de transacción ni de agencia): El coste de capital medio ponderado después de impuestos decrece al aumentar el grado de endeudamiento.
	\item[b] Versión 2 (1963) (con impuesto de sociedades, sin costes de transacción ni de agencia): El coste de capital ajeno después de impuestos crece al aumentar el grado de endeudamiento.
	\item[c] Versión 1 (1958) (sin impuestos ni costes de transacción ni de agencia): El coste de capital propio es indiferente al grado de endeudamiento de la empresa.
	\item[d] Ninguna de las anteriores es válida. 
\end{enumerate}



\seccion{Test 2004} 
\textbf{24.} El valor de una empresa, de acuerdo con el primer teorema de Modigliani y Miller (1958), es independiente de su estructura de capital. Respecto a este teorema, indique cuál de las siguientes afirmaciones es \textbf{FALSA}:

\begin{enumerate}
	\item[a] Uno de los supuestos del teorema es la neutralidad de los agentes ante el riesgo. Modigliani y Miller (1963) extienden este teorema para agentes aversos al riesgo; en este caso, el valor de una empresa sí varía con el grado de apalancamiento.
	\item[b] El modelo inicial de Modigliani y Miller (1958) no tiene en cuenta el efecto de los impuestos sobre el endeudamiento de las empresas.
	\item[c] Como consecuencia de este teorema, si las empresas optan por incrementar su apalancamiento, se incrementará el beneficio por acción, permaneciendo el precio de la acción constante.
	\item[d] La existencia de información asimétrica invalida las predicciones de Modigliani y Miller (1958) sobre la estructura de capital de las empresas.
\end{enumerate}

\notas

\textbf{2018}: \textbf{24.} D

\textbf{2017}: \textbf{24.} B

\textbf{2016}: \textbf{38.} B

\textbf{2015}: \textbf{25.} A

\textbf{2014}: \textbf{26.} C

\textbf{2013}: \textbf{26.} A

\textbf{2008}: \textbf{24.} C

\textbf{2007}: \textbf{25.} A

\textbf{2004}: \textbf{24.} A

Leer capítulo 7 del libro de Damodaran.

\bibliografia


Mirar en el Palgrave:
\begin{itemize}
    \item assets and liabilities
    \item accounting and economics
    \item dividend policy
    \item retention ratio
\end{itemize}

Vernimmen caps. 32, 33, 34, 35, 36, 37, 38

Damodaran cap. 7

Interesante el libro de Ivo Welch (usado en Harvard), aunque el tema no está basado en este libro (http://book.ivo-welch.info/read/)

Miller, M.; Modigliani, F. (1958) \textit{The cost of capital, corporation finance and the theory of investment} American Economic Review. -- En carpeta del tema

Miller, M.; Modigliani, F. (1961) \textit{Dividend policy, growth, and the valuation of shares}  Journal of Business.

Miller, M.; Modigliani, F. (1963) \textit{Corporate income taxes and the cost of capital}  American Economic Review

Miller, M. (1977) \textit{Debt and taxes} Journal of Finance

Miller, M. (1998) \textit{The M\&M proposition 40 years later} 

Miller, M.; Rock, K. (1985) \textit{Dividend Policy under Asymmetric Information} The Journal of Finance. Vol. 40, No.4 -- En carpeta del tema

Myers, S. C.; Majluf, N. S. (1984) \textit{Corporate financing and investment decisions when firms have information that investors do not have} Journal of Financial Economics: 13 -- En carpeta del tema

Wolters Kluwer. \textit{Guías jurídicas: dividendos} \href{https://guiasjuridicas.wolterskluwer.es/Content/Documento.aspx?params=H4sIAAAAAAAEAMtMSbF1jTAAAUMTM2MztbLUouLM_DxbIwMDCwNzA0uQQGZapUt-ckhlQaptWmJOcSoA9wD6lDUAAAA=WKE}{Disponible aquí}

Stiglitz, J. (1974) \textit{On the irrelevance of corporate financial policy}  American Economic Review.

\end{document}
