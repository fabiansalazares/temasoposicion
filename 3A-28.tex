\documentclass{nuevotema}

\tema{3A-28}
\titulo{Determinación de renta, precios e inflación en una economía abierta. Análisis de las políticas monetaria y fiscal.}

\begin{document}
	
\ideaclave

VER \href{https://www.econstor.eu/bitstream/10419/22359/1/wep66.pdf}{Bofinger, Mayer, Wollmershäuser (2006): Teaching New Keynesian Open Economy Macroeconomics at the Intermediate Level}. Reforma del IV apartado sobre NOEM \marcar{URGENTE}. Utilizar conclusión para comparar MF y modelos NOEM en cuanto a implicaciones de PM y PF. Añadir comparación e idea general. Reformar en términos de cambios que introducen los modelos NOEM:

-- Cumplimiento posible de UIP 
-- Cumplimiento posible de PPA que no se cumple en MF
-- Efecto de LCP (local currency pricing, precios de bienes importados son rígidos en el país importador) y PCP (producer currency pricing, precios de bienes importados)
-- Análisis de bienestar
-- Análisis de política monetaria como regla de interés, que no es posible en M-F, y enfatizar cambio en perspectiva resultado de contexto económico: de oferta monetaria como variable relevante, a inflation targetting y similares. 

Ver también \href{http://repositorio.ipea.gov.br/bitstream/11058/4943/1/DiscussionPaper_164.pdf} Como ejemplo de modelo dos países sencillo. Añadir lista de extensiones relevantes. 

Añadir mención al modelo utilizado en CECO nuevo, y líneas muy generales. Echar un vistazo a Corsetti (2007) en Palgrave (\textit{NOEM})


\seccion{Preguntas clave}
\begin{itemize}
	\item ¿Qué es una economía abierta?
	\item ¿Qué factores influyen en la determinación de la renta en una economía abierta?
	\item ¿Qué modelos explican las relaciones entre renta e inflación en economías abiertas?
	\item ¿Cómo afecta la política fiscal a la renta y los precios?
	\item ¿Cómo afecta la política monetaria a la renta y los precios?
\end{itemize}

\esquemacorto

\begin{esquema}[enumerate]
	\1[] \marcar{Introducción}
		\2 Contextualización
			\3 Macroeconomía
			\3 Economías abiertas
			\3 Importancia de ec. abierta
		\2 Objeto
			\3 ¿Qué factores determinan la renta en economías abiertas?
			\3 ¿De qué dependen los precios en economías abiertas?
			\3 ¿Qué modelos tratan de explicar las economías abiertas?
			\3 ¿Cómo afecta las políticas fiscal y monetaria a la renta y los precios?
		\2 Estructura
			\3 Predecesores
			\3 Mundell-Fleming
			\3 Dornbusch (1976)
			\3 NOEM
	\1 \marcar{Predecesores}
		\2 Modelo de flujo-especie
			\3 Idea clave
			\3 Valoración
		\2 Economía clásica
			\3 Idea clave
			\3 Mill
			\3 Valoración
	\1 \marcar{Mundell-Fleming}
		\2 Idea clave
			\3 Contexto
			\3 Objetivo
			\3 Resultados
		\2 Formulación
			\3 Ecuaciones fundamentales
			\3 Supuestos generales
		\2 Implicaciones
			\3 \underline{Movilidad perfecta de capital}
			\3[] Tipo de cambio flexible
			\3 Tipo de cambio fijo
			\3 \underline{Movilidad nula de capital}
			\3[] Tipo de cambio flexible
			\3 Tipo de cambio fijo
			\3 Efectividad de los estímulos
			\3 País grande o país pequeño
			\3 Trinidad imposible
		\2 Análisis DA-OA
			\3 Idea clave
			\3 Formulación
			\3 Implicaciones
			\3 Valoración
		\2 Valoración
			\3 Influencia teórica
			\3 Ajuste empírico
	\1 \marcar{Modelo de Dornbusch (1976)}
		\2 Idea clave
			\3 Contexto
			\3 Objetivo del modelo
		\2 Formulación
			\3 IS
			\3 LM
			\3 UIP -- Paridad descubierta de interés
			\3 Curva de Phillips
			\3 HER sobre tipo de cambio nominal
			\3 Diagrama de fase
		\2 Implicaciones
			\3 Largo plazo
			\3 Política monetaria a corto plazo
			\3 Política fiscal
		\2 Valoración
			\3 Influencia teórica
			\3 Ajuste empírico
	\1 \marcar{Nueva Macroeconomía de Economía Abierta -- NOEM}
		\2 Idea clave
			\3 Contexto
			\3 Objetivos
			\3 Resultados
		\2 Formulación
			\3 Dos países
			\3 Consumidores
			\3 Productores
			\3 Estado estacionario
			\3 Política monetaria
			\3 Shocks
		\2 Implicaciones
			\3 Precios
			\3 Política monetaria contractiva
			\3 Política fiscal expansiva
			\3 Efecto de estímulo de PM sobre bienestar
		\2 Extensiones
			\3 LCP vs PCP
			\3 Decisión endógena entre LCP y PCP
			\3 Apertura y sustituibilidad endógenas
			\3 Bienes comerciables y no comerciables
			\3 Creación de variedades
			\3 Costes de transacción
			\3 ...
		\2 Valoración
			\3 Teórica
			\3 Empírica
	\1[] \marcar{Conclusión}
		\2 Recapitulación
			\3 Modelos Pre-Keynesianos
			\3 Mundell-Fleming
			\3 NOEM
		\2 Idea final
			\3 Complejidad de la economía abierta
			\3 Integración económica y monetaria
			\3 Regímenes cambiarios
			\3 Historia del sistema monetario

\end{esquema}

\esquemalargo

\begin{esquemal}
	\1[] \marcar{Introducción}
		\2 Contextualización
			\3 Macroeconomía
				\4 Análisis de fenómenos económicos a gran escala
				\4 Énfasis sobre variables agregadas
			\3 Economías abiertas
				\4 Comercio internacional
				\4[] Intercambian ByS con otras economías
				\4[] $\to$ Precios relativos son importantes
				\4[] $\then$ Tipo de cambio es importante
				\4[] $\then$ DAgregada depende de exterior
				\4 Flujos financieros internacionales
				\4[] Intercambio de activos y pasivos
				\4[] Suavización intertemporal de rentas
				\4[] Dinámicas de deuda exterior
				\4 Interacción de sector exterior y ec. doméstica
				\4[] Demanda exterior sobre demanda agregada y output
				\4[] Diferenciales de precios
				\4[] Condiciones de financiación
			\3 Importancia de ec. abierta
				\4 Apertura post-IIGM
				\4[] Economías vuelven a abrirse al comercio
				\4 Fin de Bretton-Woods
				\4[] Flujos financieros crecen enormemente
				\4 Aparecen bloques comerciales y monetarios
				\4[$\then$] Efectos profundos sobre economías nacionales
				\4[$\then$] Economías no son ajenas a exterior
				\4[$\then$] Ec. mundial son muchas economías conectadas
		\2 Objeto
			\3 ¿Qué factores determinan la renta en economías abiertas?
			\3 ¿De qué dependen los precios en economías abiertas?
			\3 ¿Qué modelos tratan de explicar las economías abiertas?
			\3 ¿Cómo afecta las políticas fiscal y monetaria a la renta y los precios?
		\2 Estructura
			\3 Predecesores
			\3 Mundell-Fleming
			\3 Dornbusch (1976)
			\3 NOEM
	\1 \marcar{Predecesores}
		\2 Modelo de flujo-especie\footnote{Ver Bordo y Schwartz (1984).}
			\3 Idea clave
				\4 Contexto
				\4[] Azpilcueta, Hume (1752)
				\4 Objetivo
				\4[] Qué efecto tiene el influjo de metal precioso
				\4[] $\to$ Sobre output y precios
				\4 Resultados
				\4[] Metal precioso determina oferta monetaria
				\4[] Oferta monetaria determina precios
				\4[] $\to$ Por TCD
				\4[] Metal precioso fluye hacia precios bajos
				\4[] $\to$ Por ley de un sólo precio
				\4[] $\then$ En l/p, oro no afecta output
				\4[] $\then$ En c/p, influjo estimula demanda
			\3 Valoración
				\4 Modelo básico de macroeconomía internacional
				\4[] $\to$ Impacto omnipresente hasta M-F
				\4 Enmarca problema
				\4[] ¿Política monetaria puede aumentar output?
				\4[] ¿Precios pueden ser mecanismo de ajuste?
				\4 Limitación
				\4[] Sin análisis del régimen cambiario
				\4[] $\to$ Oro/plata moneda común
				\4[] Sin formalizar
				\4[] $\to$ Herramientas muy limitadas
		\2 Economía clásica\footnote{Ver Bordo y Schwartz (1984).}
			\3 Idea clave
				\4 Debates sobre efectos de PM
				\4 En general, aceptan modelo flujo-especie
				\4 Economías estables que ajustan a pleno empleo
				\4 Análisis detallados sobre:
				\4[] ¿Es oferta monetaria exógena?
				\4[] ¿Qué política de descuento óptima?
				\4[] ¿Cómo regular sistema bancario?
			\3 Mill
				\4 Introduce mecanismo del tipo de interés
				\4[] Divisas comienzan a liquidar transacciones internacionales
				\4[] $\to$ Tipo de cambio más relevante que antes
				\4 No sólo comercio ajusta oferta monetaria
				\4[] Condiciones de descuento de letras
				\4[] $\to$ Determinan tipo de interés
				\4[] $\then$ Provocan flujos de metal precioso/capital
				\4[] $\then$ Bancos centrales pueden acelerar ajuste
				\4[] Precios son menos relevantes
				\4[] $\to$ Capital fluye no sólo por arbitraje de bienes
				\4[] $\then$ Necesarios menores ajustes en nivel de precios
			\3 Valoración
				\4 Pocas contribuciones novedosas
				\4 Análisis básicamente igual a Hume
				\4 Inestabilidad de patrón oro reduce capacidad explicativa
	\1 \marcar{Mundell-Fleming}
		\2 Idea clave
			\3 Contexto
				\4 Teórico
				\4[] Mundell (1963b), Fleming (1962)
				\4[] IS-LM
				\4[] $\to$ Economía no ajusta a pleno empleo
				\4[] $\then$ Margen para políticas de demanda
				\4[] Apertura años 60
				\4[] $\to$ IS-LM ya no aproxima bien economía americana
				\4 Económico
				\4[] Sistema monetario post-BW
				\4[] TCN fijo pero ajustable
				\4[] $\to$ Fijos respecto a dólar
				\4[] Cuentas financieras
				\4[] $\to$ Poco abiertas
				\4[] $\to$ Escasa capacidad para afectar TC
				\4[] Post-Bretton Woods
				\4[] $\to$ Flujos de capital
				\4[] $\then$ Modelo mantiene relevancia
			\3 Objetivo
				\4 Modelo tratable de política económica
				\4[] Valorar efectos de PF y PM
				\4[] Alcanzar equilibrio interno y externo
				\4[] $\to$ ¿Qué políticas son efectivas?
				\4 Efecto de cuenta financiera
				\4[] Determinación de TC
				\4[] $\to$ Depende también de flujos financieros
			\3 Resultados
				\4 Con movilidad de capital
				\4[] Tipo flexible
				\4[] $\to$ PM es efectiva
				\4[] Tipo fijo
				\4[] $\to$ PF es efectiva
				\4 Sin movilidad de capital
				\4[] Tipo flexible
				\4[] $\to$ PM y PF son efectivas
				\4[] Tipo fijo
				\4[] $\to$ Ni PF ni PM son efectivas
		\2 Formulación
			\3 Ecuaciones fundamentales
				\4[IS] $Y = C(Y) + I(r) + NX(Y,E)$
				\4[] $Y = C_0 + c Y + I_0 + I(r) + X_0 + X(E) - (M_0 + m Y + M(E))$
				\4[] $\to$ $c,m < 0$, $I_r < 0$, $X_E > 0$, $M_E < 0$,
				\4[LM] $\frac{M}{P} = L(Y,r)$
				\4[] $\to$ $L_Y > 0$, $L_r > 0$
				\4[BP] $\Delta R = NX(Y,E) + CF(r, r^*)$
				\4[] $\Delta R = NX(Y,E) + K \cdot (r- r^*)$
				\4[] $\to$ $K \to \infty$: mov. perfecta de K
				\4[] $\to$ $K \to 0$: sin movilidad de K
				\4 Tres ecuaciones en tres incógnitas
				\4[] Asumiendo $\Delta R$ exógena = 0 en general
				\4[] $\to$ $Y$, $r$, $E$
			\3 Supuestos generales
				\4 Gasto público autónomo
				\4[] Es variable exógena
				\4[] Incluye tributación
				\4[] Incluido en consumo s.p.g.
				\4 Propensión al consumo
				\4[] Propensión marginal menor a 1
				\4[] $\to$ Para garantizar estabilidad de Y
				\4[] $\then$ Propensión media decreciente
				\4 Marshall-Lerner
				\4[] Depreciaciones $\uparrow$ déficit de BComercial
				\4[] Apreciaciones $\downarrow$ déficit de BComercial
				\4[] $\to$ Elasticidades-TCN de X y M (VAbsoluto) suman más que 1
				\4[] $\to$ $\frac{P_X X}{P_M M E} \eta_X + \left| \eta_M \right| > 1$
				\4 Balanza de pagos en equilibrio
				\4[] Sin pérdida de generalidad
				\4[] Análisis más complejo posible alterando $\frac{P_X X}{P_M M E}$
				\4[] $\to$ Posible inestabilidad de NX respecto TCN
				\4 Movimiento de capital
				\4[] Examinamos casos polares
				\4[] Apertura perfecta a mov. de K
				\4[] $\to$ Elasticidad infinita de K a diferencias de interés
				\4[] $\to$ EDemanda de divisas infinito en VAbsoluto
				\4[] Entrada/salida de divisas por mov. de K determina $\Delta E$
				\4[] $\to$ Entrada infinita de K si $r>r^*$
				\4[] $\then$ Moneda se aprecia si $r > r^*$
				\4[] $\to$ Salida infinita de K si $r < r^*$
				\4[] $\then$ Moneda se deprecia si $r < r^*$
				\4[] Cierre perfecto a mov. de K
				\4[] $\to$ Elasticidad 0 a diferencias de interés
				\4[] $\to$ EDemanda divisas insensible a interés
				\4[] $\then$ Sólo NX generan EDemanda de divisas
				\4[] $\text{NX} \neq 0$
		\2 Implicaciones
			\3 \underline{Movilidad perfecta de capital}
			\3[] Tipo de cambio flexible
				\4 Estímulo de política monetaria
				\4[] 1. Expansión monetaria desplaza LM a derecha
				\4[] $\to$ Estímulo a inversión y aumento DA y output
				\4[] $\to$ Expenditure changing positivo
				\4[] $\to$ Interés nacional cae por debajo de mundial
				\4[] $\then$ Desequilibrio externo
				\4[] Salida de capitales
				\4[] $\to$ Exceso de demanda de divisas
				\4[] $\to$ Presión hacia depreciación de MNacional
				\4[] Depreciación de MNacional
				\4[] $\to$ Aumenta exportaciones netas
				\4[] $\to$ Expenditure switching
				\4[] 2. Aumento de exportaciones netas aumenta DA
				\4[] $\to$ IS se desplaza a la derecha
				\4[] $\to$ Interés nacional se iguala a mundial
				\4[] $\to$ Output aumenta aún más
				\4[] $\then$ Equilibrio interno y externo
				\4[] $\then$ $\uparrow$ Y, $=r$, $\uparrow E$\footnote{Aumentos de E equivalen a depreciaciones cuando se asume que el tipo de cambio es directo (unidad de cuenta es moneda local).}
				\4[] $\then$ Estímulo de PM es efectivo
				\4[] \grafica{mkflexpm}
				\4 Estímulo de política fiscal
				\4[] 1. Expansión fiscal desplaza IS a derecha
				\4[] $\to$ Aumenta DA y output de equilibrio
				\4[] $\to$ Expenditure changing positivo
				\4[] $\to$ Interés nacional por encima de mundial
				\4[] $\then$ Desequilibrio externo
				\4[] Entrada de capitales
				\4[] $\to$ Exceso de oferta de divisas
				\4[] $\to$ Presión hacia apreciación de MNacional
				\4[] Apreciación de MNacional
				\4[] $\to$ Caída de exportaciones netas
				\4[] $\to$ Expenditure switching negativo
				\4[] 2. Caída de exportaciones netas reduce DA
				\4[] $\to$ IS se desplaza hacia izquierda
				\4[] $\to$ Interés nacional se iguala a mundial
				\4[] $\to$ Output vuelve a situación inicial
				\4[] $\then$ Equilibrio interno y externo
				\4[] $\then$ $= Y$, $= r$, $\downarrow E$
				\4[] $\then$ Estímulo de PF es inefectivo
				\4[] \grafica{mkflexpf}
			\3 Tipo de cambio fijo
				\4 Estímulo de política monetaria
				\4[] 1. Expansión monetaria desplaza LM a derecha
				\4[] $\to$ Estímulo a inversión y aumento DA y output
				\4[] $\to$ Expenditure changing positivo
				\4[] $\to$ Interés nacional cae por debajo de mundial
				\4[] $\then$ Desequilibrio externo
				\4[] Salida de capitales
				\4[] $\to$ Exceso de demanda de divisas
				\4[] $\to$ Presión hacia depreciación de MNacional
				\4[] Depreciación de MNacional
				\4[] $\to$ Autoridad monetaria comprometida a evitar
				\4[] $\to$ Necesario frenar salida de capitales
				\4[] $\then$ Necesario igualar interés nacional a mundial
				\4[] $\then$ Necesario $\uparrow$ interés con contracción monetaria
				\4[] 2. Contracción monetaria desplaza LM a izquierda
				\4[] $\to$ Aumenta interés nacional
				\4[] $\to$ Cae inversión, DA y output a pre-estímulo
				\4[] $\to$ Expenditure changing negativo
				\4[] $\then$ Eq. interno y externo
				\4[] $\then$ $= Y$, $= r$, $= E$
				\4[] $\then$ Estímulo de PM es inefectivo
				\4[] \grafica{mkfijopm}
				\4 Estímulo de política fiscal
				\4[] 1. Expansión fiscal desplaza IS a la derecha
				\4[] $\to$ Aumenta DA y output de equilibrio
				\4[] $\to$ Expenditure changing positivo
				\4[] $\to$ Interés nacional por encima de mundial
				\4[] $\then$ Desequilibrio externo
				\4[] Entrada de capitales
				\4[] $\to$ Exceso de oferta de divisas
				\4[] $\to$ Presión hacia apreciación de MNacional
				\4[] Apreciación de MNacional
				\4[] $\to$ Autoridad monetaria comprometida a evitar
				\4[] $\to$ Necesario frenar entrada de capitales
				\4[] $\then$ Necesario igualar interés nacional a mundial
				\4[] $\then$ Necesario $\downarrow$ interés con expansión monetaria
				\4[] 2. Expansión monetaria desplaza LM a derecha
				\4[] $\to$ Cae interés nacional
				\4[] $\to$ Aumenta inversión, DA y output
				\4[] $\to$ Expenditure changing positivo
				\4[] $\then$ Eq. interno y externo
				\4[] $\then$ $\uparrow Y$, $= r$, $=E$
				\4[] $\then$ Estímulo de PF es efectivo
				\4[] \grafica{mkfijopf}
			\3 \underline{Movilidad nula de capital}
			\3[] Tipo de cambio flexible
				\4 Estímulo de política monetaria
				\4[] 1. Expansión monetaria desplaza LM a derecha
				\4[] $\to$ Cae interés, aumenta DA y output
				\4[] $\to$ Expenditure changing positivo
				\4[] $\to$ Aumenta déficit comercial
				\4[] $\to$ No entra capital para financiar déficit
				\4[] $\then$ Exceso de demanda de divisas
				\4[] $\then$ Desequilibrio externo
				\4[] 2. Exceso de dda. de divisas deprecia moneda
				\4[] $\to$ Expenditure switching positivo
				\4[] $\to$ Déficit comercial cae hasta desaparecer
				\4[] $\to$ BP se desplaza a derecha
				\4[] $\then$ Eq. interno y externo
				\4[] $\then$ $\uparrow Y$, $\downarrow r$, $\uparrow E$
				\4[] $\then$ Estímulo de PM es efectivo
				\4[] \grafica{skflexpm}
				\4 Estímulo de política fiscal
				\4[] 1. Expansión fiscal desplaza IS a la derecha
				\4[] $\to$ Aumenta interés, DA y output
				\4[] $\to$ Expenditure changing positivo
				\4[] $\to$ Aumenta déficit comercial
				\4[] $\to$ No entra capital para financiar déficit
				\4[] $\then$ Exceso de demanda de divisas
				\4[] $\then$ Desequilibrio externo
				\4[] 2. Exceso de dda. de divisas deprecia moneda
				\4[] $\to$ Expenditure switching positivo
				\4[] $\to$ Déficit comercial cae hasta desaparecer
				\4[] $\to$ BP se desplaza a derecha
				\4[] $\then$ Eq. interno y externo
				\4[] $\then$ $\uparrow Y$, $\uparrow r$, $\uparrow E$
				\4[] $\then$ Estímulo de PF es inefectivo
				\4[] \grafica{skflexpf}
			\3 Tipo de cambio fijo
				\4 Estímulo de política monetaria
				\4[] 1. Expansión monetaria desplaza LM a derecha
				\4[] $\to$ Cae interés, aumentan DA y output
				\4[] $\to$ Expenditure changing positivo
				\4[] $\to$ Aumenta déficit comercial
				\4[] $\then$ Exceso de demanda de divisas
				\4[] $\then$ Desequilibrio externo
				\4[] 2. Necesario cubrir exceso de demanda de divisas
				\4[] $\to$ BCentral vende reservas por moneda nacional
				\4[] $\to$ Oferta monetaria y reservas se contraen\footnote{La autoridad monetaria puede esterilizar el efecto contractivo de la venta de reservas comprando al mismo tiempo crédito interno. Sin embargo, las reservas de divisas son finitas y esta estrategia no será sostenible a largo plazo. Si las reservas se agotan y la oferta monetaria no se ha contraído como resultado de la esterilización, la balanza de pagos deberá equilibrarse por vía de un ajuste brusco de las exportaciones netas.}
				\4[] $\to$ Aumenta interés y cae demanda agregada
				\4[] $\to$ Déficit comercial se reduce
				\4[] $\then$ Eq. interno y externo
				\4[] $\then$ $=Y$, $=r$,$= E$, $\Delta R <0$
				\4[] $\then$ Estímulo de PM es inefectivo
				\4[] \grafica{skfijopm}
				\4 Estímulo de política fiscal
				\4[] 1. Expansión monetaria desplaza IS a derecha
				\4[] $\to$ Aumenta interés, aumentan DA y output
				\4[] $\to$ Expenditure changing positivo
				\4[] $\to$ Aumenta déficit comercial
				\4[] $\then$ Exceso de demanda de divisas
				\4[] $\then$ Desequilibrio externo
				\4[] 2. Necesario cubrir exceso de demanda de divisas
				\4[] $\to$ BCentral vende reservas por moneda nacional
				\4[] $\to$ Oferta monetaria y reservas se contraen
				\4[] $\to$ Aumenta interés y cae demanda agregada
				\4[] $\to$ Déficit comercial se reduce
				\4[] $\then$ Eq. interno y externo
				\4[] $\then$ $=Y$, $\uparrow r$, $= E$, $\Delta R < 0$
				\4[] $\then$ Estímulo de PF es inefectivo
				\4[] \grafica{skfijopf}
			\3 Efectividad de los estímulos\footnote{Este apartado puede cantarse en vez de los diferentes casos apertura de BP--régimen cambiario--shock de política económica.}
				\4 Política fiscal
				\4[] Estímulo exógeno de la DA
				\4[] $\then$ Expansión del output
				\4[] A oferta monetaria constante
				\4[] $\to$ Crowding-out de liquidez
				\4[] $\then$ Presiona interés al alza
				\4[] Dos efectos contrarios sobre mercado divisas
				\4[] -- $\uparrow$ DA reduce exportaciones netas
				\4[] $\to$ Aumenta demanda de divisas
				\4[] $\then$ Presión hacia depreciación de moneda
				\4[] -- $\uparrow$ de interés nacional respecto a mundial
				\4[] $\to$ Incentiva entrada de capital
				\4[] $\then$ Presión hacia apreciación de moneda
				\4[] ¿Qué fuerza predomina? ¿Qué signo de EDemanda?
				\4[] $\to$ Depende de apertura de cuenta financiera
				\4[] $\then$ Movimiento de K vs exportaciones netas
				\4[] Movimiento de K
				\4[] $\to$ EDemanda negativo, apreciación
				\4[] Exportaciones netas
				\4[] $\to$ Exportaciones netas, depreciación
				\4[] ¿Cómo interacciona con régimen cambiario?
				\4[] $\to$ Depende de entrada de K
				\4[] Tipo flexible
				\4[] $\to$ $\Delta$ de TCN depende de fuerza predominante
				\4[] $\to$ Mov. de K predomina: apreciación de moneda
				\4[] $\then$ Reducción de exportaciones netas y output
				\4[] $\then$ PF inefectiva
				\4[] $\to$ X netas: depreciación de moneda
				\4[] $\then$ Acomoda reducción de NX y $\uparrow$ DA
				\4[] $\then$ PF efectiva
				\4[] Tipo fijo
				\4[] $\to$ Fuerza predominante debe compensarse
				\4[] $\to$ Mov. de K predomina: reducir interés
				\4[] $\then$ Expansión adicional de DA y output
				\4[] $\then$ PF especialmente efectiva
				\4[] $\to$ X netas: $\uparrow$ interés para $\downarrow$ DA
				\4[] $\then$ Caída de DA reduce efecto de estímulo
				\4[] $\then$ PF inefectiva
				\4 Política monetaria
				\4[] Expansión/contracción de oferta monetaria
				\4[] $\then$ Presiona tipos a la baja
				\4[] A precios constantes
				\4[] $\to$ Aumenta demanda de inversión
				\4[] $\then$ Aumenta demanda agregada
				\4[] Dos efectos añadidos sobre mercado de divisas
				\4[] -- $\uparrow$ DA reduce exportaciones netas
				\4[] $\to$ Aumenta demanda de divisas
				\4[] $\then$ Presión hacia depreciación de moneda
				\4[] -- $\downarrow$ de interés nacional respecto a mundial
				\4[] $\to$ Incentiva salida de capital
				\4[] $\then$ Presión hacia depreciación de moneda
				\4[] Siempre provoca exceso de demanda de divisas
				\4[] ¿Cómo interacciona con régimen cambiario?
				\4[] $\to$ Depende
				\4[] Tipo flexible
				\4[] $\to$ Depreciación de moneda nacional
				\4[] $\to$ Mejora exportaciones netas y output
				\4[] $\then$ Nuevo equilibrio con más output y depreciación
				\4[] $\then$ PM especialmente efectiva
				\4[] Tipo fijo
				\4[] $\to$ Depreciación debe compensarse reduciendo M
				\4[] $\to$ Aumento de interés y reducción déficit comercial
				\4[] $\then$ Nuevo equilibrio con mismo output
				\4[] $\then$ PM supeditada a mantener TCN luego inefectiva
			\3 País grande o país pequeño
				\4 Análisis anterior con país pequeño
				\4[] Economía infinitesimal respecto a ec. mundial
				\4[] Elasticidad infinita de movimiento de K
				\4[] $\to$ E/S de K infinitesimal respecto al mundo
				\4[] $\then$ Interés mundial constante
				\4 País grande: interés mundial alterable
				\4[] E/S de K afecta K disponible en RMundo
				\4[] $\to$ Afecta tipo de interés mundial
				\4[] $\then$ Desplazamiento de y a lo largo de BP
				\4[] $\to$ Oferta de $K$ varía con $r^*$
				\4[] $\then$ Pendiente creciente de BP
				\4[] \grafica{mkflexpfgrande}
			\3 Trinidad imposible
				\4 Resultado de modelo Mundell-Fleming
				\4[] Imposible compatibilizar:
				\4[] A. TCN Fijo
				\4[] B. Movilidad de capital
				\4[] C. Política monetaria exógena
				\4[] $\then$ Sólo dos posibles a la vez entre A,B,C
				\4 TCN fijo + movilidad de capital
				\4[] Shock de política monetaria
				\4[] $\to$ Desvía tipo de interés
				\4[] $\to$ Flujos de capital
				\4[] $\then$ Excesos de dda. de divisas no nulos
				\4[] $\then$ TCN se aprecia/deprecia
				\4[] $\then$ Elegir entre abandono TCFijo o PM endógena
				\4 TCN fijo + política monetaria exógena
				\4[] Apertura de cuenta financiera
				\4[] $\to$ Interés nacional distinto de mundial
				\4[] $\to$ Flujos de capital
				\4[] $\then$ TCN se aprecia/deprecia
				\4 Movilidad de capital + PM exógena
				\4[] TCN fijo
				\4[] $\to$ Necesario eliminar EDemanda de divisas
				\4[] $\then$ PM endógena a EDemanda de divisas
		\2 Análisis DA-OA
			\3 Idea clave
				\4 Oferta en Mundell-Fleming
				\4[] Asumida capacidad superior a output
				\4
			\3 Formulación
				\4 Incorporar sector de oferta
				\4 Múltiples supuestos posibles
				\4[] Curvas de Phillips más o menos crecientes
				\4[] Variación de Curvas de Phillips
				\4[] $\to$ En función de expectativas
			\3 Implicaciones
				\4 Sector exterior puede contribuir a inflación
			\3 Valoración
				\4 Análisis simples a nivel didáctico
				\4 Poca influencia en policy-making real
				\4 Gran impacto en debate público
		\2 Valoración
			\3 Influencia teórica
				\4 Gran impacto
				\4[] Inspiración de Dornbusch (1976)
				\4 Diseño de política económica
				\4[] Herramienta básica durante décadas
				\4[] ¿Cómo alcanzar eqs. interno y externo?
				\4[] $\to$ Modelo ofrece respuestas rápidas
				\4[] Apertura de CF y tipos flexibles
				\4[] $\to$ PF deja de ser efectivo
				\4[] $\then$ ``Adelanta'' cambio hacia políticas de oferta
			\3 Ajuste empírico
				\4 Buen ajuste cualitativo
				\4[] Predicciones muy ajustadas
				\4 Ajuste cuantitativo limitado
				\4[] Crítica de Lucas
				\4[] Modelos macroeconométricos de formas reducidas
	\1 \marcar{Modelo de Dornbusch (1976)}
		\2 Idea clave
			\3 Contexto
				\4 Limitaciones de análisis estático
				\4[] Agentes no consideran variación de TC
				\4[] Precios rígidos
				\4[] Ajuste instantáneo de interés
				\4[] $\to$ Aun con movilidad perfecta de K
				\4[] $\then$ Poco realista
				\4[] $\then$ Trayectoria de ajuste omitida
				\4 Caída de Bretton-Woods
				\4[] Flotación de principales divisas
				\4[] $\to$ Depreciaciones más que devaluaciones
				\4 Volatilidad observada muy alta
				\4[] Respuestas exageradas a shocks
				\4[] $\to$ Aumento de volatilidad de TCR
				\4[] $\then$ ¿Por qué?
				\4 Elementos centrales
				\4[] UIP
				\4[] HER
				\4[] Precios rígidos en corto plazo
				\4[] $\to$ Ajuste hacia output natural
			\3 Objetivo del modelo
				\4 Valor efecto de PM cuando:
				\4[] Mercados financieros perfectos
				\4[] $\to$ Arbitran rendimiento de bonos
				\4 Agentes estiman evolución de TC
				\4[] Aplicando expectativas racionales
				\4 Oferta agregada
				\4[] Constante en largo plazo
				\4[] Variable en corto plazo
				\4[] $\to$ Sin microfundamentar explícitamente
		\2 Formulación\footnote{Heijdra (2016) basado en original de Dornbusch (1976) y analizando también política fiscal.}
			\3 IS
				\4[] \fbox{$y = g + \delta(s + p^* - p) - \sigma i$}
			\3 LM
				\4[] \fbox{$m-p = \phi y - \lambda i$}
			\3 UIP -- Paridad descubierta de interés
				\4[] \fbox{$i = i^* + \dot{s}^e$}
			\3 Curva de Phillips
				\4[] \fbox{$\dot{p} = \pi(y-\bar{y}) $}
			\3 HER sobre tipo de cambio nominal
				\4[] \fbox{$\dot{s}^e = \dot{s}$}
			\3 Diagrama de fase
				\4 Espacio $p$--$e$
				\4[] $p$ en abscisas, $e$ en ordenadas
				\4 Dos curvas definen cuatro regiones
				\4[$\dot{p}=0$] -- Curva de precios constantes
				\4[] Precio constante implica output natural
				\4[] $\to$ $0=\pi(y-\bar{y})$ $\then$ $y=\bar{y}$
				\4[] $\then$ Economía produce output natural ($\bar{y}$)
				\4 A la derecha de $\dot{p} = 0$ dado TCN
				\4[] Precios altos
				\4[] $\to$ Tipo real apreciado
				\4[] $\to$ Bienes nacionales más caros
				\4[] $\to$ Menores exportaciones netas
				\4[] $\then$ Output cae por debajo de natural
				\4[] $\then$ Precios caen ($\leftarrow$)
				\4 A la izquierda de $\dot{p} = 0$
				\4[] $\then$ Output sube por encima de natural
				\4[] $\then$ Precios aumentan ($\rightarrow$)
				\4[$\dot{e} = 0$] -- Curva de TCN constante
				\4[] TCN cte. implica interés doméstico igual a mundial
				\4[] $\to$ $\dot{e}=0 \then i = i^*$
				\4 Al norte de $\dot{e}=0$ dado precio
				\4[] Tipo de cambio depreciado
				\4[] $\to$ Tipo de cambio real depreciado
				\4[] $\to$ Bienes nacionales más baratos
				\4[] $\to$ Mayores exportaciones netas
				\4[] $\to$ Mayor renta
				\4[] $\to$ Mayor demanda de dinero
				\4[] $\then$ Aumento de $i$ para eq. mercado de dinero
				\4[] $\then$ $i > i^*$ implica depreciación
				\4[] $\then$ TCN aumenta ($\uparrow$) (depreciación)
				\4 Al sur de $\dot{e}=0$ dado precio
				\4[] $\then$ Caída de $i$ para eq. mercado de dinero
				\4[] $\then$ $i < i^*$ implica apreciación
				\4[] $\then$ TCN cae ($\downarrow$) (apreciación)
				\4 Representación gráfica
				\4[] Norte: $\uparrow \rightarrow$
				\4[] Oeste: $\downarrow \rightarrow$
				\4[] Sur: $\leftarrow \downarrow$
				\4[] Este: $\leftarrow \uparrow$
				\4[] \grafica{dornbusch}
		\2 Implicaciones
			\3 Largo plazo
				\4 Output
				\4[] Output natural $\bar{y}$
				\4 Interés
				\4[] Igual a mundial $i^*$
				\4 Tipo de cambio
				\4[] Constante porque $i=i^*$
				\4 Oferta monetaria
				\4[] Exógenamente determinada
				\4 Precios
				\4[] Constantes porque output es natural
				\4[] Mercado de dinero en equilibrio:
				\4[] $m - \bar{p} = \phi \bar{y} - \lambda i^*$
				\4[] $\then$ $\bar{p} = f(\underset{+}{m}, \underset{-}{\bar{y}}, \underset{+}{i^*}) = m - \phi \bar{y} + \lambda i^*$
				\4[$\then$] Dicotomía clásica en largo plazo
				\4[] Vars. reales y nominales det. por separado
				\4[] Dado output e interés mundial
				\4[] $\to$ Oferta monetaria determina nivel de precios
			\3 Política monetaria a corto plazo
				\4 Expansión permanente de la oferta monetaria
				\4 \grafica{dornbuschpm}
				\4 Overshooting del tipo de cambio
				\4[] Sobrerreacción del TC en corto plazo:
				\4[] $\to$ Depreciación inmediata más allá de nivel de l/p
				\4[] $\to$ Apreciación a largo plazo
				\4[] ¿Por qué sucede?
				\4[] Necesario compatibilizar
				\4[] $\to$ Caída de interés por efecto liquidez
				\4[] $\to$ Apreciación por UIP
				\4[] $\to$ Depreciación por aumento de precios de l/p
				\4[] Puede no ocurrir si PM afecta output
				\4[] $\to$ Dda. dinero muy elástica a output
				\4[] $\then$ Requiere aumento de interés
				\4[] $\then$ Necesaria depreciación, no depreciación
				\4 Ausencia de overshooting
				\4[] TCN no sobrerreacciiona ante $\Delta M$
				\4[] $\to$ No se produce efecto liquidez
				\4[] $\then$ No necesaria apreciación para cumplir UIP
				\4[] Sucede si demanda de dinero
				\4[] $\to$ Reacciona muy poco ante cambios en interés
				\4[] $\to$ Reacciona mucho ante cambios en renta
				\4[] $\to$ Rigidez de precios implica transmisión de $\Delta M$ a $Y$
				\4[] Ejemplo:
				\4[] 1. Shock monetario no reduce $i$ nominal
				\4[] $\to$ Ante $\uparrow m$, $\Delta \phi \bar{y} > \Delta \lambda i$
				\4[] $\then$ $i$ puede incluso aumentar
				\4[] 2. Posible cumplir PPA y UIP
				\4[$\then$] Sensibilidad de dda. dinero a $i$ e $y$ determinan overshooting
				\4 Efectos sobre output
				\4[] $\then$ Aumento temporal de output
				\4[] $\then$ Nuevo equilibrio de l/p con TCN depreciado
				\4[] $\then$ Depreciación instantánea con overshooting
				\4[] $\then$ Apreciación hasta nuevo equilibrio de largo plazo
				\4[] $\then$ Aumento del nivel de precios en l/p
				\4 Comparación con Mundell-Fleming
				\4[] Similares en c/p
				\4[] $\to$ Aumento de output y depreciación
				\4[] En l/p Mundell-Fleming no explicita
				\4[] $\to$ Dornbusch caracteriza proceso de ajuste
				\4[] $\to$ Output vuelve a valor natural
			\3 Política fiscal
				\4 Expansión permanente de la demanda agregada
				\4[] Aumento de $g$ en IS
				\4 \grafica{dornbuschpf}
				\4 Efectos sobre output
				\4[] Curva $\dot{p} = 0$ hacia derecha
				\4[] $\to$ Dado $s$, necesario más $p$ para mantener $y$
				\4[] Curva $\dot{s} = 0$ hacia izquierda
				\4[] $\to$ Dado $p$, necesario menor $s$ para mantener $i=i^*$
				\4[] Precios constantes a corto plazo
				\4[] $\to$ Se mantienen en $p_0$
				\4[] TCN cae instantáneamente
				\4[] $\to$ Se aprecia TCR y cae NX
				\4[] $\to$ Caída demanda agregada
				\4[] $\to$ Apreciación TCN compensa $\uparrow$ DA autónoma
				\4[$\then$] Sin efecto sobre output
				\4[$\then$] Nuevo equilibrio de l/p con TCN apreciado
				\4 Comparación con Mundell-Fleming
				\4[] Idéntico efecto si mov. de K y TC flex.
				\4[] $\to$ Mismo output, TC apreciado
		\2 Valoración
			\3 Influencia teórica
				\4 Enorme impacto académico
				\4[] Continúa en la actualidad
				\4[] Artículo más citado de international finance
				\4 Rigidez nominal en economía abierta
				\4[] Modelo enfatiza importancia
				\4 Nuevo marco de análisis de PM
				\4[] Efectos dinámicos de PM y PF
				\4[] $\to$ Con HER
				\4[] $\to$ Con rigidez nominal
				\4[] $\to$ Con tipos flexibles
				\4 Explica overshooting
				\4[] Sin irracionalidad
				\4[] Sin comportamientos de rebaño
				\4[] $\to$ Sin supuestos ad-hoc adicionales
				\4 Enfatiza interacción ec. real y financiera
				\4[] Mercados de bienes de ajuste lento
				\4[] Mercados financieros volátiles y rápidos
				\4 Limitaciones
				\4[] Oferta agregada ad-hoc
				\4[] $\to$ Necesaria fundamentación consistente
				\4[] Régimen TCN fijo mal representado
				\4 Mantuvo relevancia de Mundell-Fleming
				\4[] Para nuevo contexto económico y teórico
			\3 Ajuste empírico
				\4 Pobre explicación del TCN
				\4[] Meese y Rogoff (1983)
				\4[] $\to$ Sólo era buen in-sample
				\4 Se ajusta bien a relación precios-TCN
				\4[] TCN en general mucho más volátil
				\4 Buenas predicciones de Y, precios y NX
				\4[] En marco muy sencillo
	\1 \marcar{Nueva Macroeconomía de Economía Abierta -- NOEM}\footnote{Ver Corsetti (2007) y Obstfeld y Rogoff (2001) citados abajo.}
		\2 Idea clave
			\3 Contexto
				\4 Modelos DSGE
				\4[] Microfundamentación de macroeconomía
				\4[] $\to$ Shocks estocásticos
				\4[] $\to$ Equilibrio General
				\4[] $\to$ Análisis dinámico
				\4 Nueva Economía Keynesiana
				\4[] Dinero no es neutral
				\4[] $\to$ Resultado de rigideces nominales
				\4[] Agentes optimizadores de utilidad y beneficios
				\4[] $\to$ En contexto dinámico
				\4 Política monetaria
				\4[] Generalmente respecto a interés de intervención
				\4[] $\to$ No tanto oferta monetaria
				\4 Autores y modelos
				\4[] Blanchard y Kiyotaki (1987)
				\4[] Svensson y Wijnbergen (1989)
				\4[] Obstfeld y Rogoff (1995)
				\4[] Corsetti y Pesenti (2001)
				\4[] Galí y Monacelli (2005)
			\3 Objetivos
				\4 Mantener ventajas de modelos tradicionales
				\4[] Tratabilidad
				\4[] Buen ajuste empírico
				\4 Permitir superar limitaciones
				\4[] Análisis de bienestar
				\4[] Crítica de Lucas
				\4[] Múltiples economías
				\4[] Oferta agregada endógena
				\4[] Desviaciones de PPA
				\4[] Bienes comerciables y no comerciables
				\4[$\then$] Modelos DSGE aplicados a diseño PE en ec. abierta
			\3 Resultados
				\4 Mantienen líneas generales de M-F
				\4 Muy amplia gama
				\4[] Parámetros y supuestos cambian resultados
				\4[] $\to$ Calibración/estimación es esencial
				\4 Fijación de precios es importante
				\4[] $\to$ ¿Rígidos en moneda local?
				\4[] $\to$ ¿Rígidos en moneda extranjera?
				\4[] $\to$ ¿Cuánto pass-through?
		\2 Formulación
			\3 Dos países
				\4 Estructura similar en ambos
				\4 Parámetros pueden ser diferentes
			\3 Consumidores
				\4 Demandan:
				\4[] Variedades nacionales
				\4[] $\to$ Sustituibilidad cte. entre sí
				\4[] $\then$ À la Dixit-Stiglitz
				\4[] Variedades extranjeras
				\4[] $\to$ Sustituibilidad cte. con nacionales
				\4[] $\then$ Cobb-Douglas con variedades nacionales
				\4 Ofertan
				\4[] Trabajo a cambio de salario
				\4 Optimizan
				\4[] Secuencia intertemporal de utilidad
				\4[] $\to$ Respecto a consumo y ocio
				\4[] $\to$ Descuento intertemporal de la utilidad
			\3 Productores
				\4 Enfrentan demandas decrecientes
				\4 Aplican mark-up sobre coste marginal
				\4 Fijan precios à la Calvo
				\4[] $\to$ Pueden cambiar precios con prob. aleatoria
				\4[] $\then$ Fijan precios estimando mark-up futuro
			\3 Estado estacionario
				\4[IS] $\tilde{y}_t = E_t \left\lbrace \tilde{y}_{t+1} \right\rbrace - \frac{1}{\sigma(v)} \left( i_t - E_t \left\lbrace \pi_{H,t+1} \right\rbrace - r_t^n \right)$
				\4[NKPC] $\pi_{H,t} = \beta E_t \left\lbrace \pi_{H,t+1} \right\rbrace + k_v \tilde{y}_t$
				\4[NX] $\text{nx}_t = v f( \underset{+}{s_t}, \underset{-}{z})$
				\4[UIP] $i_t = i_t^* + \dot{s}_t^e$
				\4[UIP] $\dot{s}_t^e  = i_t - i_t^*$
				\4[] Donde:
				\4[] $v$: grado de apertura
				\4[] $\sigma(v)$: depende negativamente\footnote{En realidad, que la relación sea negativa depende del grado de sustituibilidad de la economía. Ver págs. 225 y 235 de Galí (2015).} de $v$
				\4[] $\frac{1}{\sigma(v)}$: sensibilidad de output a interés
				\4[] $\then$ + apertura de la economía, + sensibilidad de $y$ a interés\footnote{Esto puede justificarse considerando que un aumento del interés real efectivo respecto al interés natural ($r_t^n$) induce una caída de la demanda. La caída de la demanda tiene efecto sobre la demanda de bienes nacionales y extranjeros. Si la economía es muy abierta al exterior y los bienes nacionales son poco sustituibles por bienes extranjeros, es posible que la caída recaiga con más fuerza sobre la demanda de bienes nacionales que extranjeros y por ello, el output nacional sea más sensible. Todo ello está relacionado con el hecho de que ante un shock de demanda (vía $z_t$), la balanza comercial mejore (o empeore). Sin embargo, el modelo permite parametrizaciones para las que sucede de hecho lo contrario: mayor apertura de la economía implica menor efecto negativo de un shock, porque aunque cae la demanda nacional, la demanda exterior no tiene porque verse afectada y el output gap se ve menos afectado. }
				\4[] $z$: factor de variación de la demanda sujeto a shocks
				\4[] $r_t^n$: tipo de interés natural dependiente de:
				\4[] $\to$ Descuento intertemporal
				\4[] $\to$ Crecimiento del output
				\4[] $\to$ Grado de apertura de la economía (+)
				\4[] $\to$ Output a nivel mundial (+)
			\3 Política monetaria
				\4 Regla de Taylor
				\4[] $i_t = \rho + \phi_\pi \pi_{H,t} + \phi_y \tilde{y}_t+v_t$
				\4 Otras reglas incorporables en modelo
			\3 Shocks
				\4 Tasa de descuento intertemporal
				\4[] Afecta a demanda
				\4 Productividad
				\4[] $\to$ Afectan al coste marginal
				\4[] $\then$ Alteran mark-up aplicado
				\4[] $\then$ Presión hacia cambio de precios
				\4 Política monetaria
				\4[] Implementación imperfecta
		\2 Implicaciones
			\3 Precios
				\4 IPC o deflactor
				\4[] Consumidores acceden a nac. y extranjero
				\4[] $\then$ Aparece nueva distinción de precios
				\4[] Demanda agregada
				\4[] $\to$ Depende de IPC
				\4[] Demanda de producto nacional
				\4[] $\to$ Depende de deflactor relativo
				\4[] $\then$ Efecto más complejo de PM sobre output
			\3 Política monetaria contractiva
				\4 Tipos de interés nominales
				\4[] Aumentan
				\4 Tipos de interés reales
				\4[] Aumentan
				\4 Tipo de cambio nominal
				\4[] Se aprecia fuertemente inicialmente
				\4[] Se deprecia posteriormente
				\4[] $\then$ Similar a Dornbusch (1976)
				\4[] $\then$ Posible overshooting
				\4 Output nacional
				\4[] Cae por:
				\4[] $\to$ caída de demanda
				\4[] $\to$ apreciación del tipo de cambio
				\4 Inflación
				\4[] Cae doblemente
				\4[] $\to$ Caída de demanda
				\4[] $\to$ Apreciación inicial del tipo de cambio
				\4 Comparación con economía cerrada
				\4[] Cualitativamente similar
				\4[] Cuantitativamente distinto
				\4[] $\to$ Apertura amplifica shocks
				\4[$\then$] Política monetaria afecta output
			\3 Política fiscal expansiva
			\3 Efecto de estímulo de PM sobre bienestar
				\4 Dependen de parametrización
				\4[] Dos efectos contrapuestos:
				\4 Aumento del output nacional
				\4 Importaciones más caras
				\4[$\then$] Efecto depende de parámetros
		\2 Extensiones
			\3 LCP vs PCP
				\4 Diferentes efectos
				\4[] En función de como se fijen precios
				\4 Local Currency Pricing
				\4[] Precio en moneda de consumidor
				\4[] $\to$ Sin pass-through del tipo de cambio
				\4[] $\then$ TC no afecta precios
				\4 Producer Currency Pricing
				\4[] Precio en moneda de productor
				\4[] $\to$ Con pass-through del tipo de cambio
				\4[] $\then$ RRI se deteriora con apreciación
				\4[] $\then$ TC afecta precios
			\3 Decisión endógena entre LCP y PCP
				\4 Dependiente de volatilidad de PE nacional
				\4 Países con PM inestable
				\4[] Exportadores tenderán a preferir LCP
				\4[] $\to$ Aseguran ingresos
				\4[] $\then$ Menor sensibilidad de output a PM
				\4 Países con PM estables
				\4[] Al contrario
				\4[] $\then$ Mayor sensibilidad de output a PM
			\3 Apertura y sustituibilidad endógenas
			\3 Bienes comerciables y no comerciables
			\3 Creación de variedades
			\3 Costes de transacción
			\3 ...
		\2 Valoración
			\3 Teórica
				\4 Generalizado en diseño de PM
				\4 Modelos DSGE modernos
				\4[] $\to$ Son aplicaciones de anterior
				\4[] $\to$ BCentrales, Gobiernos, FMI...
			\3 Empírica
				\4 Muy dependiente de calibración
				\4 Buena representación de persistencia
				\4 Algunas correlaciones difíciles de ajustar
				\4[] Problemas con pass-through
	\1[] \marcar{Conclusión}
		\2 Recapitulación
			\3 Modelos Pre-Keynesianos
			\3 Mundell-Fleming
			\3 NOEM
		\2 Idea final
			\3 Complejidad de la economía abierta
				\4 Mayor que economía cerrada
				\4[] Tipo de cambio
				\4[] Precios relativos
				\4[] Grado de apertura
				\4[] Sustituibilidad de bienes
				\4[] Stock de inversión internacional
				\4[] ...
				\4 Ec. mundial como sistema de ecs. abiertas
			\3 Integración económica y monetaria
				\4 Modelos anteriores para valorar efectos
			\3 Regímenes cambiarios
				\4 Modelos anteriores permiten valorar
				\4 Preguntas que responder
				\4[] ¿Qué objetivo se trata de alcanzar?
				\4[] ¿Son sostenibles?
			\3 Historia del sistema monetario
				\4 Modelos anteriores permiten entender
				\4 Evolución histórica de sistema monetario
				\4[] Genera contexto para formular modelos
\end{esquemal}

























\graficas

\begin{axis}{4}{Movilidad perfecta de capital, tipo de cambio flexible: estímulo de política monetaria}{Y}{r}{mkflexpm}
	% IS inicial
	\draw[-] (0.5,3.5) -- (3.5,0.5);
	\node[above] at (0.5,3.5){\tiny $\text{IS}_0$};
	
	% LM inicial
	\draw[-] (0.5,0.5) -- (3.5,3.5);
	\node[above] at (3.5,3.5){\tiny $\text{LM}_0$};
	
	% Equilibrio inicial
	\node[circle,fill=black,inner sep=0pt,minimum size=4pt] (a) at (2,2) {};
	\node[below] at (2,2){\tiny $0$};	
	
	% BP
	\draw[-] (0,2) -- (4,2);
	\node[right] at (4,2){\tiny $\text{BP}$};
	
	% LM después de estímulo monetario
	\draw[dashed] (1.5,0.5) -- (4.5,3.5);
	\node[above] at (4.5,3.5){\tiny $\text{LM}_1$};
	
	% Equilibrio post estímulo
	\node[circle,fill=black,inner sep=0pt,minimum size=4pt] (a) at (2.5,1.5) {};
	\node[below] at (2.5,1.5){\tiny $1$};
	
	% IS después de depreciación del tipo de cambio
	\draw[dashed] (1.5,3.5) -- (4.5,0.5);
	\node[above] at (1.5,3.5){\tiny $\text{IS}_1$};

	% Equilibrio final tras depreciación y desplazamiento de IS a la derecha
	\node[circle,fill=black,inner sep=0pt,minimum size=4pt] (a) at (3,2) {};
	\node[below] at (3,2){\tiny $2$};

\end{axis}

\begin{axis}{4}{Movilidad perfecta de capital, tipo de cambio flexible: estímulo de política fiscal}{Y}{r}{mkflexpf}
	% IS inicial
	\draw[-] (0.5,3.5) -- (3.5,0.5);
	\node[above] at (0.5,3.5){\tiny $\text{IS}_0$};
	
	% LM inicial
	\draw[-] (0.5,0.5) -- (3.5,3.5);
	\node[above] at (3.5,3.5){\tiny $\text{LM}_0$};

	% Equilibrio inicial
	\node[circle,fill=black,inner sep=0pt,minimum size=4pt] (a) at (2,2) {};
	\node[below] at (2,2){\tiny $0$};
		
	% BP
	\draw[-] (0,2) -- (4,2);
	\node[right] at (4,2){\tiny $\text{BP}$};

	% IS después de estímulo de PF fiscal
	\draw[dashed] (1.5,3.5) -- (4.5,0.5);
	\node[above] at (1.5,3.5){\tiny $\text{IS}_1$};
	
	% IS vuelve a posición inicial tras apreciación de tipo de cambio
	\draw[-{Latex}] (1.65,3.25) -- (1.35,2.9);
	\draw[-{Latex}] (3.65,1.25) -- (3.35,0.9);

	% Equilibrio post estímulo
	\node[circle,fill=black,inner sep=0pt,minimum size=4pt] (a) at (2.5,2.5) {};
	\node[below] at (2.5,2.5){\tiny $1$};
	
	% Equilibrio final 
	%\node[circle,fill=black,inner sep=0pt,minimum size=4pt] (a) at (2.5,2.5) {};
	\node[above] at (2,2){\tiny $2$};
\end{axis}

\begin{axis}{4}{Movilidad perfecta de capital, tipo de cambio fijo: estímulo de política monetaria}{Y}{r}{mkfijopm}
	% IS inicial
	\draw[-] (0.5,3.5) -- (3.5,0.5);
	\node[above] at (0.5,3.5){\tiny $\text{IS}_0$};
	
	% LM inicial
	\draw[-] (0.5,0.5) -- (3.5,3.5);
	\node[above] at (3.5,3.5){\tiny $\text{LM}_0$};	

	% Equilibrio inicial
	\node[circle,fill=black,inner sep=0pt,minimum size=4pt] (a) at (2,2) {};
	\node[below] at (2,2){\tiny $0$};
	
	% BP
	\draw[-] (0,2) -- (4,2);
	\node[right] at (4,2){\tiny $\text{BP}$};
	
	% LM después de estímulo monetario
	\draw[dashed] (1.5,0.5) -- (4.5,3.5);
	\node[above] at (4.5,3.5){\tiny $\text{LM}_1$};
	
	% Equilibrio post estímulo
	\node[circle,fill=black,inner sep=0pt,minimum size=4pt] (a) at (2.5,1.5) {};
	\node[below] at (2.5,1.5){\tiny $1$};
	
	% LM vuelve a equilibrio inicial tras salida neta de capital y presión hacia depreciación
	\draw[-{Latex}] (1.8,0.9) -- (1.3,1.3);
	\draw[-{Latex}] (3.8,2.9) -- (3.3,3.3);
	
	% Equilibrio final 
	\node[circle,fill=black,inner sep=0pt,minimum size=4pt] (a) at (2,2) {};
	\node[above] at (2,2){\tiny $2$};
\end{axis}

\begin{axis}{4}{Movilidad perfecta de capital, tipo de cambio fijo: estímulo de política fiscal}{Y}{r}{mkfijopf}
	% IS inicial
	\draw[-] (0.5,3.5) -- (3.5,0.5);
	\node[above] at (0.5,3.5){\tiny $\text{IS}_0$};
	
	% LM inicial
	\draw[-] (0.5,0.5) -- (3.5,3.5);
	\node[above] at (3.5,3.5){\tiny $\text{LM}_0$};

	% Equilibrio inicial
	\node[circle,fill=black,inner sep=0pt,minimum size=4pt] (a) at (2,2) {};
	\node[below] at (2,2){\tiny $0$};
	
	% BP
	\draw[-] (0,2) -- (4,2);
	\node[right] at (4,2){\tiny $\text{BP}$};

	% IS después de estímulo de PF fiscal
	\draw[dashed] (1.5,3.5) -- (4.5,0.5);
	\node[above] at (1.5,3.5){\tiny $\text{IS}_1$};
	
	% Equilibrio tras estímulo fiscal
	\node[circle,fill=black,inner sep=0pt,minimum size=4pt] (a) at (2.5,2.5) {};
	\node[below] at (2.5,2.5){\tiny $1$};

	% LM después de estímulo de PF fiscal para evitar apreciación
	\draw[dashed] (1.5,0.5) -- (4.5,3.5);
	\node[above] at (4.5,3.5){\tiny $\text{LM}_1$};
	
	% Equilibrio final tras estímulo de PF y aumento de base monetaria
	\node[circle,fill=black,inner sep=0pt,minimum size=4pt] (a) at (3,2) {};
	\node[below] at (3,2){\tiny $2$};
\end{axis}

\begin{axis}{4}{Sin movilidad de capital, tipo de cambio flexible: estímulo de política monetaria}{Y}{r}{skflexpm}
	% IS inicial
	\draw[-] (0.5,3.5) -- (3.5,0.5);
	\node[above] at (0.5,3.5){\tiny $\text{IS}_0$};
	
	% LM inicial
	\draw[-] (0.5,0.5) -- (3.5,3.5);
	\node[above] at (3.5,3.5){\tiny $\text{LM}_0$};

	% Equilibrio inicial
	\node[circle,fill=black,inner sep=0pt,minimum size=4pt] (a) at (2,2) {};
	\node[right] at (2,2){\tiny $0$};

	% LM después de estímulo monetario
	\draw[dashed] (1.5,0.5) -- (4.5,3.5);
	\node[above] at (4.5,3.5){\tiny $\text{LM}_1$};
	
	% BP inicial
	\draw[-] (2,0) -- (2,4);
	\node[above] at (2,4){\tiny $\text{BP}_0$};
	
	% BP final
	\draw[dashed] (2.5,0) --(2.5,4);
	\node[above] at (2.5,4){\tiny $\text{BP}_1$};

	% Equilibrio final
	\node[circle,fill=black,inner sep=0pt,minimum size=4pt] (a) at (2.5,1.5) {};
	\node[right] at (2.5,1.5){\tiny $1$};
\end{axis}

\begin{axis}{4}{Sin movilidad de capital, tipo de cambio flexible: estímulo de política fiscal}{Y}{r}{skflexpf}
	% IS inicial
	\draw[-] (0.5,3.5) -- (3.5,0.5);
	\node[above] at (0.5,3.5){\tiny $\text{IS}_0$};
	
	% LM inicial
	\draw[-] (0.5,0.5) -- (3.5,3.5);
	\node[above] at (3.5,3.5){\tiny $\text{LM}_0$};

	% Equilibrio inicial
	\node[circle,fill=black,inner sep=0pt,minimum size=4pt] (a) at (2,2) {};
	\node[right] at (2,2){\tiny $0$};
	
	% IS después de estímulo de PF fiscal
	\draw[dashed] (1.5,3.5) -- (4.5,0.5);
	\node[above] at (1.5,3.5){\tiny $\text{IS}_1$};
	
	% BP inicial
	\draw[-] (2,0) -- (2,4);
	\node[above] at (2,4){\tiny $\text{BP}$};
	
	% BP final
	\draw[dashed] (2.5,0) --(2.5,4);
	\node[above] at (2.5,4){\tiny $\text{BP}_1$};
	
	% Equilibrio final
	\node[circle,fill=black,inner sep=0pt,minimum size=4pt] (a) at (2.5,2.5) {};
	\node[right] at (2.5,2.5){\tiny $1$};
\end{axis}

Bajo tipo de cambio flexible y sin movilidad de capitales, los déficit de cuenta corriente se eliminarán por medio de variaciones en el tipo de cambio. Cualquier estímulo de política fiscal o monetaria que aumente la demanda agregada y el output inducirá un déficit por cuenta corriente que se saldará por medio de una depreciación de la moneda nacional.

\begin{axis}{4}{Sin movilidad de capital, tipo de cambio fijo: estímulo de política monetaria}{Y}{r}{skfijopm}
	% IS inicial
	\draw[-] (0.5,3.5) -- (3.5,0.5);
	\node[above] at (0.5,3.5){\tiny $\text{IS}_0$};
	
	% LM inicial
	\draw[-] (0.5,0.5) -- (3.5,3.5);
	\node[above] at (3.5,3.5){\tiny $\text{LM}_0$};

	% Equilibrio inicial
	\node[circle,fill=black,inner sep=0pt,minimum size=4pt] (a) at (2,2) {};
	\node[right] at (2,2){\tiny $0$};

	% LM después de estímulo monetario
	\draw[dashed] (1.5,0.5) -- (4.5,3.5);
	\node[above] at (4.5,3.5){\tiny $\text{LM}_1$};
	
	% BP inicial
	\draw[-] (2,0) -- (2,4);
	\node[above] at (2,4){\tiny $\text{BP}$};
	
	% Equilibrio después de estímulo monetario
	\node[circle,fill=black,inner sep=0pt,minimum size=4pt] (a) at (2.5,1.5) {};
	\node[right] at (2.5,1.5){\tiny $1$};
	
	% Equilibrio final
	% \node[circle,fill=black,inner sep=0pt,minimum size=4pt] (a) at (2,2) {};
	\node[left] at (2,2){\tiny $2$};
	
	% LM vuelve a situación inicial
	\draw[-{Latex}] (1.8,1) -- (1.55,1.35);
	\draw[-{Latex}] (3.8,3) -- (3.55,3.35);
\end{axis}

\begin{axis}{4}{Sin movilidad de capital, tipo de cambio fijo: estímulo de política fiscal}{Y}{r}{skfijopf}
	% IS inicial
	\draw[-] (0.5,3.5) -- (3.5,0.5);
	\node[above] at (0.5,3.5){\tiny $\text{IS}_0$};
	
	% LM inicial
	\draw[-] (0.5,0.5) -- (3.5,3.5);
	\node[above] at (3.5,3.5){\tiny $\text{LM}_0$};

	% Equilibrio inicial
	\node[circle,fill=black,inner sep=0pt,minimum size=4pt] (a) at (2,2) {};
	\node[right] at (2,2){\tiny $0$};
	
	% IS después de estímulo de PF fiscal
	\draw[dashed] (1.5,3.5) -- (4.5,0.5);
	\node[above] at (1.5,3.5){\tiny $\text{IS}_1$};
	
	% BP inicial
	\draw[-] (2,0) -- (2,4);
	\node[above] at (2,4){\tiny $\text{BP}$};
	
	% Equilibrio después de estímulo fiscal
	\node[circle,fill=black,inner sep=0pt,minimum size=4pt] (a) at (2.5,2.5) {};
	\node[right] at (2.5,2.5){\tiny $1$};
	
	% LM después de estímulo fiscal para cubrir déficit en BP
	\draw[dashed] (-0.5,0.5) -- (2.5,3.5);
	\node[above] at (2.5,3.5){\tiny $\text{LM}_1$};
	
	% Equilibrio final después de estímulo fiscal
	\node[circle,fill=black,inner sep=0pt,minimum size=4pt] (a) at (2,3) {};
	\node[right] at (2,3){\tiny $2$};
\end{axis}

Bajo tipo de cambio fijo y sin movilidad de capital, ningún estímulo es efectivo. 

Si un estímulo de política fiscal aumenta la demanda agregada, el output aumentará pero aparecerá un déficit por cuenta corriente que inducirá también un exceso de demanda de divisas. La política monetaria tiene por objetivo --en este contexto de tipo de cambio fijo- eliminar cualquier variación del tipo de cambio. Para ello, la única opción en este contexto es eliminar la expansión del output por vía de un aumento del interés que reduzca la demanda agregada hasta el nivel inicial que implica déficit nulo en la cuenta corriente. 

Si un estímulo de política monetaria reduce el interés y aumenta la demanda agregada, la situación será análoga a la anterior: aparecerá un déficit por cuenta corriente que presionará el tipo de cambio hacia la depreciación. Para eliminar esa presión depreciadora, la política monetaria deberá pasar a ser contractiva.

\begin{axis}{4}{País grande: influye en tipo de interés mundial. Movilidad perfecta de capital, tipo de cambio flexible: estímulo de política fiscal}{Y}{r}{mkflexpfgrande}
	% IS inicial
	\draw[-] (0.5,3.5) -- (3.5,0.5);
	\node[above] at (0.5,3.5){\tiny $\text{IS}_0$};
	
	% LM inicial
	\draw[-] (0.5,0.5) -- (3.5,3.5);
	\node[above] at (3.5,3.5){\tiny $\text{LM}_0$};
	
	% Equilibrio inicial
	\node[circle,fill=black,inner sep=0pt,minimum size=4pt] (a) at (2,2) {};
	\node[below] at (2,2){\tiny $0$};
	
	% BP inicial
	\draw[-] (0,1.87) -- (4,2.123);
	\node[right] at (4,2){\tiny $\text{BP}_0$};
	\node[left] at (0,2){$r^*$};
	
	% IS después de estímulo de PF fiscal
	\draw[dashed] (1.5,3.5) -- (4.5,0.5);
	\node[above] at (1.5,3.5){\tiny $\text{IS}_1$};
	
	% IS desplaza a izquierda tras apreciación y aumento de tipo mundial
	\draw[dashed] (1,3.5) -- (4,0.5);
	\draw[-{Latex}] (1.65,3.25) -- (1.45,3);
	\draw[-{Latex}] (3.65,1.25) -- (3.45,1);
	\node[above] at (1,3.5){\tiny $\text{IS}_2$};
	
	% Equilibrio post estímulo
	\node[circle,fill=black,inner sep=0pt,minimum size=4pt] (a) at (2.5,2.5) {};
	\node[above] at (2.5,2.5){\tiny $1$};
	
	% Equilibrio final 
	%\node[circle,fill=black,inner sep=0pt,minimum size=4pt] (a) at (2.5,2.5) {};
	\node[circle,fill=black,inner sep=0pt,minimum size=4pt] (a) at (2.25,2.25) {};
	\node[above] at (2.25,2.25){\tiny $2$};
	
	% BP final
	\draw[dashed] (0,2.123) -- (4,2.37);
	\node[left] at (0,2.25){$r^{*'}$};
	\node[right] at (4,2.25){\tiny $\text{BP}_1$};
\end{axis}

\begin{axis}{4}{Modelo de Dornbusch (1976): diagrama de fase}{p}{e}{dornbusch}
	% Línea de precio constante
	\draw[-] (0.5,0.5) -- (4,4);
	\node[above] at (4,4){$\dot{p}=0$};
	
	% Línea de tipo de cambio constante
	\draw[-] (0.5,4) -- (4,0.5);
	\node[above] at (0.5,4){$\dot{e}=0$};
	
	% Senda estable de punto de silla
	%\draw[-] (0.5,3) -- (4,1.5);
	\draw[-{Latex}] (0.5,3) -- (1,2.79);
	\draw[-{Latex}] (1,2.79) -- (1.5,2.57);
	\draw[-{Latex}] (1.5,2.57) -- (2,2.36);
	\draw[-{Latex}] (2,2.36)-- (2.25,2.25);
	\draw[-{Latex}] (4,1.5) -- (3.5,1.71);
	\draw[-{Latex}] (3.5,1.71) -- (3,1.93);
	\draw[-{Latex}] (3,1.93) -- (2.5,2.143);
	\draw[-{Latex}] (2.5,2.143) -- (2.25,2.25);
	
	% NORTE: Precio creciente y tipo de cambio que se deprecia 
	\draw[-{Latex}] (2,3.5) -- (2,4);
	\draw[-{Latex}] (2,3.5) -- (2.5,3.5);
	
	% OESTE: Precio creciente y tipo de cambio que se aprecia
	\draw[-{Latex}] (0.5,2) -- (0.5,1.5);
	\draw[-{Latex}] (0.5,2) -- (1,2);
	
	% SUR: Precio decreciente y tipo de cambio que se aprecia
	\draw[-{Latex}] (2.5,1) -- (2,1);
	\draw[-{Latex}] (2.5,1) -- (2.5,0.5);
	
	% ESTE
	\draw[-{Latex}] (4,2.5) -- (4,3);
	\draw[-{Latex}] (4,2.5) -- (3.5,2.5);
	
	% Tipo de cambio de equilibrio
	\draw[dotted] (2.25,2.25) -- (0,2.25);
	\node[left] at (0,2.25){$\bar{e}_0$};
	
	% Precio de equilibrio
	\draw[dotted] (2.25,2.25) -- (2.25,0);
	\node[below] at (2.25,0){$\bar{p}_0$};
\end{axis}

\begin{axis}{4}{Modelo de Dornbusch (1976): efecto de un estímulo de política monetaria inesperado.}{p}{e}{dornbuschpm}
	% Línea de precio constante
	\draw[-] (0.5,0.5) -- (4,4);
	\node[above] at (4,4){\tiny $\dot{p}_0=0$};
	
	% Línea de tipo de cambio constante
	\draw[-] (0.5,4) -- (4,0.5);
	\node[above] at (0.5,4){\tiny $\dot{e}=0$};
	
	% Senda estable de punto de silla
	%\draw[-] (0.5,3) -- (4,1.5);
	% Hacia abajo y derecha
	\draw[-{Latex}] (0.5,3) -- (1,2.79);
	\draw[-{Latex}] (1,2.79) -- (1.5,2.57);
	\draw[-{Latex}] (1.5,2.57) -- (2,2.36);
	\draw[-{Latex}] (2,2.36)-- (2.25,2.25);
	% Hacia arriba e izquierda
	\draw[-{Latex}] (4,1.5) -- (3.5,1.71);
	\draw[-{Latex}] (3.5,1.71) -- (3,1.93);
	\draw[-{Latex}] (3,1.93) -- (2.5,2.143);
	\draw[-{Latex}] (2.5,2.143) -- (2.25,2.25);
	
	% NORTE: Precio creciente y tipo de cambio que se deprecia 
	\draw[-{Latex}] (2.4,3.5) -- (2.4,4);
	\draw[-{Latex}] (2.4,3.5) -- (2.9,3.5);
	
	% OESTE: Precio creciente y tipo de cambio que se aprecia
	\draw[-{Latex}] (0.5,2) -- (0.5,1.5);
	\draw[-{Latex}] (0.5,2) -- (1,2);
	
	% SUR: Precio decreciente y tipo de cambio que se aprecia
	\draw[-{Latex}] (2.5,1) -- (2,1);
	\draw[-{Latex}] (2.5,1) -- (2.5,0.5);
	
	% ESTE
	\draw[-{Latex}] (4.5,2.5) -- (4.5,3);
	\draw[-{Latex}] (4.5,2.5) -- (4,2.5);
	
	% Tipo de cambio de equilibrio
	\draw[dotted] (2.25,2.25) -- (0,2.25);
	\node[left] at (0,2.25){\tiny $\bar{e}_0$};
	
	% Precio de equilibrio
	\draw[dotted] (2.25,2.25) -- (2.25,0);
	\node[below] at (2.25,0){\tiny $\bar{p}_0$};
	
	% Línea de tipo constante tras estímulo de PM
	\draw[dashed] (1.5,4) -- (5,0.55);
	\node[above] at (1.5,4){\tiny $\dot{e}_1=0$};
	
	% Precio de equilibrio constante en el muy corto plazo tras estímulo
	\draw[dotted] (2.25,2.25) -- (2.25,3);
	
	% Tipo de cambio de overshooting
	\draw[dotted] (2.25,3) -- (0,3);
	\node[left] at (0.05,3.1){\tiny $e^*$};
	
	% Tipo de cambio de equilibrio tras estímulo
	\draw[dotted] (2.78,2.78) -- (0,2.78);
	\node[left] at (0,2.75){\tiny $\bar{e}_1$};
	
	% Precio de equilibrio tras estímulo
	\draw[dotted] (2.78,2.78) -- (2.78,0);
	\node[below] at (2.78,0){\tiny $\bar{p}_1$};
	
	% Nueva senda estable de punto de silla
	% y = (111/28) - (3/7)x
	%\draw[-] (0.5,3.75) -- (4,2.25); RESOLVER ECUACIÓN
	\draw[dashed,-{Latex}] (0.5,3.75) -- (1,3.54);
	\draw[dashed,-{Latex}] (1,3.54) -- (1.5,3.32);
	\draw[dashed,-{Latex}] (1.5,3.32) -- (2,3.11);
	\draw[dashed,-{Latex}] (2,3.11) -- (2.5,2.89);
	\draw[dashed,-{Latex}] (2.5,2.89) -- (2.78,2.78);
	
	\draw[dashed,-{Latex}] (4,2.25) -- (3.5,2.46);
	\draw[dashed,-{Latex}] (3.5,2.46) -- (3,2.68);
	\draw[dashed,-{Latex}] (3,2.68) -- (2.5,2.89);
	\draw[dashed,-{Latex}] (2.5,2.89) -- (2.78,2.78);
	
	%	\draw[dashed,-{Latex}] (4,2.25) -- (3.5,2.46);
	%	\draw[dashed,-{Latex}] (3.5,2.46) -- (3,2.68);
	%	\draw[dashed,-{Latex}] (3,2.68) -- (2.5,2.89);
	%	\draw[dashed,-{Latex}] (2.5,2.89) -- (2.25,3);
\end{axis}

\begin{axis}{4}{Modelo de Dornbusch (1976): efecto de un estímulo permanente e inesperado de política fiscal}{p}{e}{dornbuschpf}
	% Línea de precio constante
	\draw[-] (0.5,0.5) -- (4,4);
	\node[above] at (4,4){\tiny $\dot{p}_0=0$};
	
	% Línea de tipo de cambio constante
	\draw[-] (0.5,4) -- (4,0.5);
	\node[above] at (0.5,4){\tiny $\dot{e}_0=0$};
	
	% Senda estable de punto de silla
	% Línea recta plana: \draw[-] (0.5,3) -- (4,1.5);
	\draw[-{Latex}] (0.5,3) -- (1,2.79);
	\draw[-{Latex}] (1,2.79) -- (1.5,2.57);
	\draw[-{Latex}] (1.5,2.57) -- (2,2.36);
	\draw[-{Latex}] (2,2.36)-- (2.25,2.25);
	\draw[-{Latex}] (4,1.5) -- (3.5,1.71);
	\draw[-{Latex}] (3.5,1.71) -- (3,1.93);
	\draw[-{Latex}] (3,1.93) -- (2.5,2.143);
	\draw[-{Latex}] (2.5,2.143) -- (2.25,2.25);

	% NORTE: Precio creciente y tipo de cambio que se deprecia 
	\draw[-{Latex}] (2,3.5) -- (2,4);
	\draw[-{Latex}] (2,3.5) -- (2.5,3.5);
	
	% OESTE: Precio creciente y tipo de cambio que se aprecia
	\draw[-{Latex}] (0.5,2) -- (0.5,1.5);
	\draw[-{Latex}] (0.5,2) -- (1,2);
	
	% SUR: Precio decreciente y tipo de cambio que se aprecia
	\draw[-{Latex}] (2.5,1) -- (2,1);
	\draw[-{Latex}] (2.5,1) -- (2.5,0.5);
	
	% ESTE
	\draw[-{Latex}] (4,2.5) -- (4,3);
	\draw[-{Latex}] (4,2.5) -- (3.5,2.5);
	
	
	% Tipo de cambio de equilibrio
	\draw[dotted] (2.25,2.25) -- (0,2.25);
	\node[left] at (0,2.25){\tiny $\bar{e}_0$};
	
	% Precio de equilibrio
	\draw[dotted] (2.25,2.25) -- (2.25,0);
	\node[below] at (2.25,0){\tiny $\bar{p}_0$};
	
	% Línea de precio constante tras estímulo de PF	
	\draw[dashed] (1.5,0.5) -- (5,4);
	\node[above] at (5,4){\tiny $\dot{p_1}=0$};

	% Línea de tipo de cambio constante tras estímulo de PF
	\draw[dashed] (0.5,3) -- (4,-0.5);
	\node[above] at (0.5,3){\tiny $\dot{e_1}=0$};
	
	% Nuevo tipo de cambio de equilibrio
	\draw[dotted] (2.25,1.25) -- (0,1.25);
	\node[left] at (0,1.25){\tiny $\bar{e}_1$};
\end{axis}


\preguntas

\seccion{Test 2019}

\textbf{18.} Según el modelo de Mundell-Fleming, la efectividad de las medidas de política económica varían en función del régimen cambiario y la libertad de movimiento de capitales:

\begin{itemize}
	\item[i] La política monetaria con tipo de cambio fijo gana eficacia a medida que aumenta la movilidad del capital.
	\item[ii] La política monetaria con tipo de cambio flexible gana eficacia al aumentar la movilidad del capital.
	\item[iii] La política fiscal con tipo de cambio fijo y perfecta movilidad de capitales es eficaz.
	\item[iv] La política fiscal con tipo de cambio flexible es eficaz, cuando la movilidad de capital es alta o muy alta.
\end{itemize}

Elija la respuesta correcta:

\begin{itemize}
	\item[a] Únicamente la respuesta ii es verdadera.
	\item[b] Las respuestas i y iii son falsas.
	\item[c] Las respuestas ii y iii son verdaderas.
	\item[d] La respuesta iv es verdadera.
\end{itemize}

\seccion{Test 2018}

\textbf{13.} Suponga una economía abierta y pequeña, con movilidad perfecta de capitales y tipo de cambio flexible.

\begin{itemize}
	\item[a] Una política monetaria expansiva no tiene efectos sobre el saldo comercial si los precios y los salarios son flexibles.
	\item[b] Una política fiscal expansiva no tiene efectos sobre el saldo comercial si los precios y los salarios son flexibles.
	\item[c] Un aumento de la renta mundial no tiene efectos sobre el saldo comercial si los precios y los salarios son flexibles, si bien el tipo de cambio real se depreciará.
	\item[d] Un aumento del tipo de interés mundial no tendrá efectos sobre la renta nacional si los precios son rígidos.
\end{itemize}

\seccion{Test 2016}

\textbf{24}. Considere una economía pequeña descrita por el modelo de Mundell-Fleming con tipo de cambio fluctuante y movilidad perfecta de capitales. En el corto plazo, un aumento de los aranceles a la importación provocará:

\begin{enumerate}
	\item[a] Un incremento de la renta y una apreciación de la moneda.
	\item[b] Una apreciación de la moneda y el mantenimiento de la renta constante.
	\item[c] Una disminución de la renta y una depreciación de la moneda.
	\item[d] Una depreciación de la moneda y el mantenimiento de la renta constante.
\end{enumerate}

\textbf{25}. El país A es una economía abierta GRANDE con movilidad perfecta de capitales y tipo de cambio flexible. Dicho país decide aplicar una política fiscal expansiva:
\begin{enumerate}
	\item[a] La renta al final del ajuste se mantendrá igual que al inicio.
	\item[b] La renta al final del ajuste será mayor que al inicio, pero menor que con tipo de cambio fijo.
	\item[c] Se producirá un doble efecto expulsión (crowding-out) dado que la subida del tipo de interés doméstico expulsará inversión [$I(I_0, i, Y)$] y reducirá la competitividad de sus productos vía apreciación de la moneda.
	\item[d] Las respuestas b) y c) son ciertas.
\end{enumerate}

\seccion{Test 2013}

\textbf{28}. En el contexto de una economía abierta con un tipo de cambio flexible, perfecta movilidad de capital, donde los precios están dados y el tipo de interés internacional está dado, un aumento del gasto público tendrá como efecto (señale la correcta):
\begin{enumerate}
	\item[a] Aumentarán el nivel de renta, el tipo de cambio y las exportaciones netas.
	\item[b] Permanecerán constantes el nivel de renta y las exportaciones netas, al desplazar el aumento del gasto público a la inversión privada en la misma cuantía.
	\item[c] Permanecerá constante el nivel de renta y se reducirá el tipo de cambio y las exportaciones netas.
	\item[d] Aumentará el nivel de renta y se reducirán las exportaciones netas puesto que el aumento del gasto público no desplazará a la inversión privada en la misma cuantía.
\end{enumerate}

\seccion{Test 2011} 
\textbf{15}. En un modelo de economía abierta de Mundell-Fleming
\begin{enumerate}
	\item[a] Si el tipo de cambio es flexible, cuanto mayor sea la movilidad de capitales, mayor será el efecto expansivo de la política fiscal.
	\item[b] Si el tipo de cambio es fijo, con movilidad imperfecta de capitales, la política fiscal expansiva genera un aumento final de la renta mayor que el impulso inicial.
	\item[c] Si el tipo de cambio es flexible y movilidad alta de capitales, la política monetaria no lleva a un mayor nivel de renta pero si a un menor tipo de interés.
	\item[d] Si el tipo de cambio es fijo, la política monetaria no tiene efectos expansivos, independientemente del grado de movilidad de capitales.
\end{enumerate}

\seccion{Test 2009}

\textbf{14}. Suponga una economía abierta, con movilidad perfecta de capitales y tipo de cambio flexible. Un aumento en el tipo de interés internacional (r):
\begin{enumerate}
	\item[a] Provocará una depreciación del tipo de cambio real mayor si los precios son flexibles que si son rígidos.
	\item[b] Provocará el mismo efecto sobre el tipo de cambio real tanto si los precios son rígidos como si son flexibles.
	\item[c] No tenemos suficiente información como para saber que ocurrirá con el tipo de cambio real cualquiera que sea la capacidad de ajuste vía precios.
	\item[d] Provocará una depreciación del tipo de cambio real mayor si los precios son rígidos que si son flexibles.
\end{enumerate}

(movida a tema 3B-13 sobre ajuste de la balanza de pagos)

\textbf{30}. Una economía abierta, que mantiene un sistema de tipo de cambio fijo, sufre una perturbación que afecta al mercado monetario de forma contractiva. El consiguiente ajuste de la balanza de pagos se llevaría a cabo mediante:
\begin{enumerate}
	\item[a] Un aumento de las reservas de divisas.
	\item[b] Una disminución de las reservas de divisas.
	\item[c] Una devaluación del tipo de cambio.
	\item[d] Una apreciación del tipo de cambio.
\end{enumerate}

\seccion{Test 2008}

\textbf{16}. Un país pequeño que adopta un sistema de cambios flexibles con perfecta movilidad de capitales decide aumentar el gasto público. Ello supondrá:

\begin{enumerate}
	\item[a] Reducir la renta de equilibrio.
	\item[b] Que la renta de equilibrio no se altere.
	\item[c] Aumentar el tipo de interés nacional.
	\item[d] Una depreciación de la moneda nacional.
\end{enumerate}

\seccion{Test 2007}

\textbf{32}. Supongamos que, en una economía con tipo de cambio flexible, descrita por el modelo Mundell-Fleming, las autoridades deciden llevar a cabo una política monetaria expansiva. Los efectos de dicha medida serían:
\begin{enumerate}
	\item[a] El nivel de actividad aumenta con respecto a su valor inicial, y las reservas de divisas disminuyen.
	\item[b] El nivel de actividad no se altera con respecto a su valor inicial, y el tipo de cambio se deprecia. 
	\item[c] El nivel de actividad aumenta con respecto a su valor inicial, y el tipo de cambio se deprecia.
	\item[d] El nivel de actividad no se altera con respecto a su valor inicial, y las reservas de divisas disminuyen.
\end{enumerate}

\textbf{34}. Supongamos un modelo Mundell-Fleming de dos países, con tipo de cambio fijo y perfecta movilidad de capitales, donde el gobierno del primero de ellos decide llevar a cabo una política fiscal expansiva. Los efectos de esta medida sobre el nivel de actividad del segundo país (que denominaremos \comillas{resto del mundo}) serían:
\begin{enumerate}
	\item[a] Aumenta el nivel de actividad del resto del mundo, pues al efecto expansivo derivado del aumento del nivel de actividad del primer país se añade el déficit de balanza de pagos con el resto del mundo, el cual se traduce en una depreciación de su tipo de cambio, con el consiguiente efecto expansivo adicional.
	\item[b] Se produce un efecto ambiguo sobre el nivel de actividad del resto del mundo, pues al efecto expansivo derivado del aumento del nivel de actividad del primer país se añade el déficit de la balanza de pagos en el resto del mundo, el cual se traduce en una disminución de sus reservas de divisas, con el consiguiente efecto contractivo.
	\item[c] No varía el nivel de actividad del resto del mundo, pues el nivel de actividad del primer país no varía y, por tanto, no se produce ningún tipo de transmisión al resto del mundo.
	\item[d] Disminuye el nivel de actividad del resto del mundo, pues, aunque el nivel de actividad del primer país no varía, el déficit de la balanza de pagos en el resto del mundo se traduce en una disminución de sus reservas de divisas, con el consiguiente efecto contractivo.
\end{enumerate}

\seccion{Test 2006}

\textbf{17}. Suponga una economía abierta \comillas{pequeña} con movilidad perfecta de capitales y tipo de cambio flexible. Una política fiscal expansiva provocada por un aumento en el gasto público:
\begin{enumerate}
	\item[a] No tendrá efectos sobre la renta de equilibrio tanto si los precios del bien final son rígidos como si son flexibles.
	\item[b] Sólo tendrá efectos sobre la renta de equilibrio si los precios del bien final son rígidos.
	\item[c] Sólo tendrá efectos sobre la renta de equilibrio si los precios del bien final son flexibles.
	\item[d] Depreciará el tipo de cambio real si los precios del bien final son rígidos.
\end{enumerate}

\seccion{Test 2005}

\textbf{15}. Suponga una economía abierta y \comillas{pequeña} con movilidad perfecta del capital, tipo de cambio fijo y precios flexibles:

\begin{enumerate}
	\item[a] Una devaluación no tiene efectos sobre el tipo de cambio real.
	\item[b] Una devaluación conduce a una depreciación real.
	\item[c] Un aumento del gasto público conduce a una depreciación real.
	\item[d] Un aumento de la renta mundial no tiene efectos sobre el tipo de cambio real.
\end{enumerate}

\textbf{29}. Supongamos que, en una economía con tipo de cambio flexible, descrita por el modelo Mundell-Fleming, se produce un aumento del tipo de interés del resto del mundo. Los efectos sobre el nivel de actividad de dicha economía serían:
\begin{enumerate}
	\item[a] El nivel de actividad tendería a no alterarse con respecto a su valor inicial, ya que la acumulación de reservas de divisas compensaría el efecto contractivo de la apreciación del tipo de cambio.
	\item[b] El nivel de actividad tendería a reducirse con respecto a su valor inicial, debida a la disminución de las reservas de divisas.
	\item[c] El nivel de actividad tendería a aumentar con respecto a su valor inicial, en respuesta a la depreciación del tipo de cambio.
	\item[d] El nivel de actividad tendería a no alterarse con respecto a su valor inicial, ya que la depreciación del tipo de cambio tendería a compensar el efecto contractivo del incremento del tipo de interés del resto del mundo.
\end{enumerate}

\seccion{Test 2004}

\textbf{31}. Supongamos, en el contexto del modelo Mundell-Fleming, que, en una economía con tipo de cambio flexible, se produce una alteración en los flujos comerciales a nivel internacional, lo que se traduce en una caída de sus exportaciones y el consiguiente déficit comercial. Los efectos sobre el nivel de actividad de dicha economía serían:
\begin{enumerate}
	\item[a] El nivel de actividad tendería a no alterarse con respecto a su valor inicial, ya que la acumulación de reservas de divisas compensaría el efecto contractivo del déficit comercial.
	\item[b] El nivel de actividad tendería a no alterarse con respecto a su valor inicial, ya que la depreciación del tipo de cambio compensaría el efecto contractivo del déficit comercial.
	\item[c] El nivel de actividad tendería a reducirse, ya que la depreciación del tipo de cambio compensaría sólo parcialmente el efecto contractivo del déficit comercial, especialmente si la movilidad de capitales fuera perfecta.
	\item[d] El nivel de actividad tendería a reducirse, ya que la apreciación del tipo de cambio se sumaría al efecto contractivo del déficit comercial.
\end{enumerate}

\textbf{32}. En los círculos de política económica, se ha subrayado en los últimos tiempos la imposibilidad de que funcionen de manera simultánea los siguientes tres elementos:
\begin{enumerate} 
	\item[a] Política monetaria autónoma a nivel nacional, tipo de cambio fijo y movilidad perfecta de capitales.
	\item[b] Política monetaria autónoma a nivel nacional, tipo de cambio flexible y movilidad perfecta de capitales.
	\item[c] Política fiscal autónoma a nivel nacional, y tipo de cambio flexible y movilidad perfecta de capitales.
	\item[d] Políticas monetaria y fiscal autónomas a nivel nacional, tipo de cambio fijo y movilidad perfecta de capitales.
\end{enumerate}


\seccion{16 de marzo de 2017}
\begin{itemize}
    \item En una economía abierta, ¿es más efectiva la política fiscal o la política monetaria?
    
    \item ¿Qué efecto tiene cada una en precios y reservas?
\end{itemize}

\notas

\textbf{2019}: \textbf{18.} C

\textbf{2018}: \textbf{13.} A

\textbf{2016}: \textbf{24.} B \textbf{25.} D

\textbf{2013}: \textbf{28.} C

\textbf{2011}: \textbf{15.} D. La B es errónea, ver págs. 223 y 224 de Gandolfo.

\textbf{2009}: \textbf{14.} D \textbf{30.} A

\textbf{2008}: \textbf{16.} B

\textbf{2007}: \textbf{32.} C \textbf{34.} B

\textbf{2006}: \textbf{17.} A

\textbf{2005}: \textbf{15.} A \textbf{29.} C

\textbf{2004}: \textbf{31.} B \textbf{32.} A

\bibliografia

Mirar en Palgrave:
\begin{itemize}
	\item new open economy macroeconomics
\end{itemize}

Bofinger, P.; Mayer, E.; Wollmershäuser, T. (2001)

Bordo, M. D.; Schwartz, A. J. (1984) \textit{A Retrospective on the Classical Gold Standard, 1821--1931} University of Chicago Press -- En carpeta Historia Económica

Corsetti, G. Eichengreen, B. Hale, G. Tallman, E. \textit{The euro crisis in the mirror of the European Monetary System} (2019) -- \url{https://voxeu.org/article/euro-crisis-mirror-european-monetary-system}

Corsetti, G. \textit{New Open Economy Macroeconomics} (2007) EUI Working Papers -- En carpeta del tema

Dornbusch, R. \textit{Expectations and Exchange Rate Dynamics} (1976) Journal of Political Economy -- En carpeta del tema

Dornbusch, R. \textit{Expectations and Exchange Rate Dynamics} (1976) Journal of Political Economy -- En carpeta del tema

Dutt, A. K. \textit{Aggregate Demand–Aggregate Supply Analysis: A History} (2002) History of Political Economy -- En carpeta del tema

Gandolfo, G. \textit{International Finance and Open-Economy Macroeconomics}. (2016) Springer Verlag -- En carpeta Economía internacional, Ch. 10, 11, 18

Heijdra, B. J. \textit{Foundations of Modern Macroeconomics} (2017) 3rd ed. -- En carpeta Macro

Lane, P. R. (2001) \textit{The new open economy macroeconomics: a survey} Journal of International Economics -- En carpeta del tema

Obstfeld, M.; Rogoff, K. \textit{Exchange Rate Dynamics Redux} (1995) Journal of Political Economy -- En carpeta del tema

Rogoff, K. \textit{Dornbusch's Overshooting Model After Twenty-Five Years} (2002) International Monetary Fund - IMF Staff Papers -- En carpeta del tema

Sarno, L.; Taylor, M. \textit{The Economics of exchange rates} (2002) Cambridge University Press -- En carpeta Economía internacional

Uribe, M.; Schmitt-Grohe, S. \textit{Open Economy Macroeconomics} (2017) Diapositivas -- \url{http://www.columbia.edu/~mu2166/book/} Guardadas en carpeta Macroeconomía

Obstfeld, M. \textit{International Macroeconomics: Beyond the Mundell-Fleming Model} (2000) International Monetary Fund - IMF Staff Papers -- En carpeta del tema

Sarno, L.; Taylor, M. \textit{The economics of exchange rates} (2002) Cambridge University Press -- En carpeta Economía internacional

Wang, P. \textit{The Economics of Foreign Exchange and Global Finance} (2005) 2nd Edition -- En carpeta de economía internacional

\end{document}
